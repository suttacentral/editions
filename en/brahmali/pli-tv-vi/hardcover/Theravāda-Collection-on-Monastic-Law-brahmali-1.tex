\documentclass[12pt,openany]{book}%
\usepackage{lastpage}%
%
\usepackage{ragged2e}
\usepackage{verse}
\usepackage[a-3u]{pdfx}
\usepackage[inner=1in, outer=1in, top=.7in, bottom=1in, papersize={6in,9in}, headheight=13pt]{geometry}
\usepackage{polyglossia}
\usepackage[12pt]{moresize}
\usepackage{soul}%
\usepackage{microtype}
\usepackage{tocbasic}
\usepackage{realscripts}
\usepackage{epigraph}%
\usepackage{setspace}%
\usepackage{sectsty}
\usepackage{fontspec}
\usepackage{marginnote}
\usepackage[bottom]{footmisc}
\usepackage{enumitem}
\usepackage{fancyhdr}
\usepackage{emptypage}
\usepackage{extramarks}
\usepackage{graphicx}
\usepackage{relsize}
\usepackage{etoolbox}

% improve ragged right headings by suppressing hyphenation and orphans. spaceskip plus and minus adjust interword spacing; increase rightskip stretch to make it want to push a word on the first line(s) to the next line; reduce parfillskip stretch to make line length more equal . spacefillskip and xspacefillskip can be deleted to use defaults.
\protected\def\BalancedRagged{
\leftskip     0pt
\rightskip    0pt plus 10em
\spaceskip=1\fontdimen2\font plus .5\fontdimen3\font minus 1.5\fontdimen4\font
\xspaceskip=1\fontdimen2\font plus 1\fontdimen3\font minus 1\fontdimen4\font
\parfillskip  0pt plus 15em
\relax
}

\hypersetup{
colorlinks=true,
urlcolor=black,
linkcolor=black,
citecolor=black,
allcolors=black
}

% use a small amount of tracking on small caps
\SetTracking[ spacing = {25*,166, } ]{ encoding = *, shape = sc }{ 25 }

% add a blank page
\newcommand{\blankpage}{
\newpage
\thispagestyle{empty}
\mbox{}
\newpage
}

% define languages
\setdefaultlanguage[]{english}
\setotherlanguage[script=Latin]{sanskrit}

%\usepackage{pagegrid}
%\pagegridsetup{top-left, step=.25in}

% define fonts
% use if arno sanskrit is unavailable
%\setmainfont{Gentium Plus}
%\newfontfamily\Marginalfont[]{Gentium Plus}
%\newfontfamily\Allsmallcapsfont[RawFeature=+c2sc]{Gentium Plus}
%\newfontfamily\Noligaturefont[Renderer=Basic]{Gentium Plus}
%\newfontfamily\Noligaturecaptionfont[Renderer=Basic]{Gentium Plus}
%\newfontfamily\Fleuronfont[Ornament=1]{Gentium Plus}

% use if arno sanskrit is available. display is applied to \chapter and \part, subhead to \section and \subsection.
\setmainfont[
  FontFace={sb}{n}{Font = {Arno Pro Semibold}},
  FontFace={sb}{it}{Font = {Arno  Pro Semibold Italic}}
]{Arno Pro}

% create commands for using semibold
\DeclareRobustCommand{\sbseries}{\fontseries{sb}\selectfont}
\DeclareTextFontCommand{\textsb}{\sbseries}

\newfontfamily\Marginalfont[RawFeature=+subs]{Arno Pro Regular}
\newfontfamily\Allsmallcapsfont[RawFeature=+c2sc]{Arno Pro}
\newfontfamily\Noligaturefont[Renderer=Basic]{Arno Pro}
\newfontfamily\Noligaturecaptionfont[Renderer=Basic]{Arno Pro Caption}

% chinese fonts
\newfontfamily\cjk{Noto Serif TC}
\newcommand*{\langlzh}[1]{\cjk{#1}\normalfont}%

% logo
\newfontfamily\Logofont{sclogo.ttf}
\newcommand*{\sclogo}[1]{\large\Logofont{#1}}

% use subscript numerals for margin notes
\renewcommand*{\marginfont}{\Marginalfont}

% ensure margin notes have consistent vertical alignment
\renewcommand*{\marginnotevadjust}{-.17em}

% use compact lists
\setitemize{noitemsep,leftmargin=1em}
\setenumerate{noitemsep,leftmargin=1em}
\setdescription{noitemsep, style=unboxed, leftmargin=1em}

% style ToC
\DeclareTOCStyleEntries[
  raggedentrytext,
  linefill=\hfill,
  pagenumberwidth=.5in,
  pagenumberformat=\normalfont,
  entryformat=\normalfont
]{tocline}{chapter,section}


  \setlength\topsep{0pt}%
  \setlength\parskip{0pt}%

% define new \centerpars command for use in ToC. This ensures centering, proper wrapping, and no page break after
\def\startcenter{%
  \par
  \begingroup
  \leftskip=0pt plus 1fil
  \rightskip=\leftskip
  \parindent=0pt
  \parfillskip=0pt
}
\def\stopcenter{%
  \par
  \endgroup
}
\long\def\centerpars#1{\startcenter#1\stopcenter}

% redefine part, so that it adds a toc entry without page number
\let\oldcontentsline\contentsline
\newcommand{\nopagecontentsline}[3]{\oldcontentsline{#1}{#2}{}}

    \makeatletter
\renewcommand*\l@part[2]{%
  \ifnum \c@tocdepth >-2\relax
    \addpenalty{-\@highpenalty}%
    \addvspace{0em \@plus\p@}%
    \setlength\@tempdima{3em}%
    \begingroup
      \parindent \z@ \rightskip \@pnumwidth
      \parfillskip -\@pnumwidth
      {\leavevmode
       \setstretch{.85}\large\scshape\centerpars{#1}\vspace*{-1em}\llap{#2}}\par
       \nobreak
         \global\@nobreaktrue
         \everypar{\global\@nobreakfalse\everypar{}}%
    \endgroup
  \fi}
\makeatother

\makeatletter
\def\@pnumwidth{2em}
\makeatother

% define new sectioning command, which is only used in volumes where the pannasa is found in some parts but not others, especially in an and sn

\newcommand*{\pannasa}[1]{\clearpage\thispagestyle{empty}\begin{center}\vspace*{14em}\setstretch{.85}\huge\itshape\scshape\MakeLowercase{#1}\end{center}}

    \makeatletter
\newcommand*\l@pannasa[2]{%
  \ifnum \c@tocdepth >-2\relax
    \addpenalty{-\@highpenalty}%
    \addvspace{.5em \@plus\p@}%
    \setlength\@tempdima{3em}%
    \begingroup
      \parindent \z@ \rightskip \@pnumwidth
      \parfillskip -\@pnumwidth
      {\leavevmode
       \setstretch{.85}\large\itshape\scshape\lowercase{\centerpars{#1}}\vspace*{-1em}\llap{#2}}\par
       \nobreak
         \global\@nobreaktrue
         \everypar{\global\@nobreakfalse\everypar{}}%
    \endgroup
  \fi}
\makeatother

% don't put page number on first page of toc (relies on etoolbox)
\patchcmd{\chapter}{plain}{empty}{}{}

% global line height
\setstretch{1.05}

% allow linebreak after em-dash
\catcode`\—=13
\protected\def—{\unskip\textemdash\allowbreak}

% style headings with secsty. chapter and section are defined per-edition
\partfont{\setstretch{.85}\normalfont\centering\textsc}
\subsectionfont{\setstretch{.95}\normalfont\BalancedRagged}%
\subsubsectionfont{\setstretch{1}\normalfont\itshape\BalancedRagged}

% style elements of suttatitle
\newcommand*{\suttatitleacronym}[1]{\smaller[2]{#1}\vspace*{.3em}}
\newcommand*{\suttatitletranslation}[1]{\linebreak{#1}}
\newcommand*{\suttatitleroot}[1]{\linebreak\smaller[2]\itshape{#1}}

\DeclareTOCStyleEntries[
  indent=3.3em,
  dynindent,
  beforeskip=.2em plus -2pt minus -1pt,
]{tocline}{section}

\DeclareTOCStyleEntries[
  indent=0em,
  dynindent,
  beforeskip=.4em plus -2pt minus -1pt,
]{tocline}{chapter}

\newcommand*{\tocacronym}[1]{\hspace*{-3.3em}{#1}\quad}
\newcommand*{\toctranslation}[1]{#1}
\newcommand*{\tocroot}[1]{(\textit{#1})}
\newcommand*{\tocchapterline}[1]{\bfseries\itshape{#1}}


% redefine paragraph and subparagraph headings to not be inline
\makeatletter
% Change the style of paragraph headings %
\renewcommand\paragraph{\@startsection{paragraph}{4}{\z@}%
            {-2.5ex\@plus -1ex \@minus -.25ex}%
            {1.25ex \@plus .25ex}%
            {\noindent\normalfont\itshape\small}}

% Change the style of subparagraph headings %
\renewcommand\subparagraph{\@startsection{subparagraph}{5}{\z@}%
            {-2.5ex\@plus -1ex \@minus -.25ex}%
            {1.25ex \@plus .25ex}%
            {\noindent\normalfont\itshape\footnotesize}}
\makeatother

% use etoolbox to suppress page numbers on \part
\patchcmd{\part}{\thispagestyle{plain}}{\thispagestyle{empty}}
  {}{\errmessage{Cannot patch \string\part}}

% and to reduce margins on quotation
\patchcmd{\quotation}{\rightmargin}{\leftmargin 1.2em \rightmargin}{}{}
\AtBeginEnvironment{quotation}{\small}

% titlepage
\newcommand*{\titlepageTranslationTitle}[1]{{\begin{center}\begin{large}{#1}\end{large}\end{center}}}
\newcommand*{\titlepageCreatorName}[1]{{\begin{center}\begin{normalsize}{#1}\end{normalsize}\end{center}}}

% halftitlepage
\newcommand*{\halftitlepageTranslationTitle}[1]{\setstretch{2.5}{\begin{Huge}\uppercase{\so{#1}}\end{Huge}}}
\newcommand*{\halftitlepageTranslationSubtitle}[1]{\setstretch{1.2}{\begin{large}{#1}\end{large}}}
\newcommand*{\halftitlepageFleuron}[1]{{\begin{large}\Fleuronfont{{#1}}\end{large}}}
\newcommand*{\halftitlepageByline}[1]{{\begin{normalsize}\textit{{#1}}\end{normalsize}}}
\newcommand*{\halftitlepageCreatorName}[1]{{\begin{LARGE}{\textsc{#1}}\end{LARGE}}}
\newcommand*{\halftitlepageVolumeNumber}[1]{{\begin{normalsize}{\Allsmallcapsfont{\textsc{#1}}}\end{normalsize}}}
\newcommand*{\halftitlepageVolumeAcronym}[1]{{\begin{normalsize}{#1}\end{normalsize}}}
\newcommand*{\halftitlepageVolumeTranslationTitle}[1]{{\begin{Large}{\textsc{#1}}\end{Large}}}
\newcommand*{\halftitlepageVolumeRootTitle}[1]{{\begin{normalsize}{\Allsmallcapsfont{\textsc{\itshape #1}}}\end{normalsize}}}
\newcommand*{\halftitlepagePublisher}[1]{{\begin{large}{\Noligaturecaptionfont\textsc{#1}}\end{large}}}

% epigraph
\renewcommand{\epigraphflush}{center}
\renewcommand*{\epigraphwidth}{.85\textwidth}
\newcommand*{\epigraphTranslatedTitle}[1]{\vspace*{.5em}\footnotesize\textsc{#1}\\}%
\newcommand*{\epigraphRootTitle}[1]{\footnotesize\textit{#1}\\}%
\newcommand*{\epigraphReference}[1]{\footnotesize{#1}}%

% map
\newsavebox\IBox

% custom commands for html styling classes
\newcommand*{\scnamo}[1]{\begin{Center}\textit{#1}\end{Center}\bigskip}
\newcommand*{\scendsection}[1]{\begin{Center}\begin{small}\textit{#1}\end{small}\end{Center}\addvspace{1em}}
\newcommand*{\scendsutta}[1]{\begin{Center}\textit{#1}\end{Center}\addvspace{1em}}
\newcommand*{\scendbook}[1]{\bigskip\begin{Center}\uppercase{#1}\end{Center}\addvspace{1em}}
\newcommand*{\scendkanda}[1]{\begin{Center}\textbf{#1}\end{Center}\addvspace{1em}} % use for ending vinaya rule sections and also samyuttas %
\newcommand*{\scend}[1]{\begin{Center}\begin{small}\textit{#1}\end{small}\end{Center}\addvspace{1em}}
\newcommand*{\scendvagga}[1]{\begin{Center}\textbf{#1}\end{Center}\addvspace{1em}}
\newcommand*{\scrule}[1]{\textsb{#1}}
\newcommand*{\scadd}[1]{\textit{#1}}
\newcommand*{\scevam}[1]{\textsc{#1}}
\newcommand*{\scspeaker}[1]{\hspace{2em}\textit{#1}}
\newcommand*{\scbyline}[1]{\begin{flushright}\textit{#1}\end{flushright}\bigskip}
\newcommand*{\scexpansioninstructions}[1]{\begin{small}\textit{#1}\end{small}}
\newcommand*{\scuddanaintro}[1]{\medskip\noindent\begin{footnotesize}\textit{#1}\end{footnotesize}\smallskip}

\newenvironment{scuddana}{%
\setlength{\stanzaskip}{.5\baselineskip}%
  \vspace{-1em}\begin{verse}\begin{footnotesize}%
}{%
\end{footnotesize}\end{verse}
}%

% custom command for thematic break = hr
\newcommand*{\thematicbreak}{\begin{center}\rule[.5ex]{6em}{.4pt}\begin{normalsize}\quad\Fleuronfont{•}\quad\end{normalsize}\rule[.5ex]{6em}{.4pt}\end{center}}

% manage and style page header and footer. "fancy" has header and footer, "plain" has footer only

\pagestyle{fancy}
\fancyhf{}
\fancyfoot[RE,LO]{\thepage}
\fancyfoot[LE,RO]{\footnotesize\lastleftxmark}
\fancyhead[CE]{\setstretch{.85}\Noligaturefont\MakeLowercase{\textsc{\firstrightmark}}}
\fancyhead[CO]{\setstretch{.85}\Noligaturefont\MakeLowercase{\textsc{\firstleftmark}}}
\renewcommand{\headrulewidth}{0pt}
\fancypagestyle{plain}{ %
\fancyhf{} % remove everything
\fancyfoot[RE,LO]{\thepage}
\fancyfoot[LE,RO]{\footnotesize\lastleftxmark}
\renewcommand{\headrulewidth}{0pt}
\renewcommand{\footrulewidth}{0pt}}
\fancypagestyle{plainer}{ %
\fancyhf{} % remove everything
\fancyfoot[RE,LO]{\thepage}
\renewcommand{\headrulewidth}{0pt}
\renewcommand{\footrulewidth}{0pt}}

% style footnotes
\setlength{\skip\footins}{1em}

\makeatletter
\newcommand{\@makefntextcustom}[1]{%
    \parindent 0em%
    \thefootnote.\enskip #1%
}
\renewcommand{\@makefntext}[1]{\@makefntextcustom{#1}}
\makeatother

% hang quotes (requires microtype)
\microtypesetup{
  protrusion = true,
  expansion  = true,
  tracking   = true,
  factor     = 1000,
  patch      = all,
  final
}

% Custom protrusion rules to allow hanging punctuation
\SetProtrusion
{ encoding = *}
{
% char   right left
  {-} = {    , 500 },
  % Double Quotes
  \textquotedblleft
      = {1000,     },
  \textquotedblright
      = {    , 1000},
  \quotedblbase
      = {1000,     },
  % Single Quotes
  \textquoteleft
      = {1000,     },
  \textquoteright
      = {    , 1000},
  \quotesinglbase
      = {1000,     }
}

% make latex use actual font em for parindent, not Computer Modern Roman
\AtBeginDocument{\setlength{\parindent}{1em}}%
%

% Default values; a bit sloppier than normal
\tolerance 1414
\hbadness 1414
\emergencystretch 1.5em
\hfuzz 0.3pt
\clubpenalty = 10000
\widowpenalty = 10000
\displaywidowpenalty = 10000
\hfuzz \vfuzz
 \raggedbottom%

\title{Theravāda Collection on Monastic Law}
\author{Bhikkhu Brahmali}
\date{}%
% define a different fleuron for each edition
\newfontfamily\Fleuronfont[Ornament=9]{Arno Pro}

% Define heading styles per edition for chapter and section. Suttatitle can be either of these, depending on the volume. 

\let\oldfrontmatter\frontmatter
\renewcommand{\frontmatter}{%
\chapterfont{\setstretch{.85}\normalfont\centering}%
\sectionfont{\setstretch{.85}\normalfont\BalancedRagged}%
\oldfrontmatter}

\let\oldmainmatter\mainmatter
\renewcommand{\mainmatter}{%
\chapterfont{\thispagestyle{empty}\setstretch{.85}\normalfont\centering}%
\sectionfont{\clearpage\thispagestyle{plain}\setstretch{.85}\normalfont\centering}%
\oldmainmatter}

\let\oldbackmatter\backmatter
\renewcommand{\backmatter}{%
\chapterfont{\setstretch{.85}\normalfont\centering}%
\sectionfont{\setstretch{.85}\normalfont\BalancedRagged}%
\pagestyle{plainer}%
\oldbackmatter}
%
%
\begin{document}%
\normalsize%
\frontmatter%
\setlength{\parindent}{0cm}

\pagestyle{empty}

\maketitle

\blankpage%
\begin{center}

\vspace*{2.2em}

\halftitlepageTranslationTitle{Theravāda Collection on Monastic Law}

\vspace*{1em}

\halftitlepageTranslationSubtitle{A translation of the Pali Vinaya Piṭaka into English}

\vspace*{2em}

\halftitlepageFleuron{•}

\vspace*{2em}

\halftitlepageByline{translated and introduced by}

\vspace*{.5em}

\halftitlepageCreatorName{Bhikkhu Brahmali}

\vspace*{4em}

\halftitlepageVolumeNumber{Volume 1}

\smallskip

\halftitlepageVolumeAcronym{Bu Vb}

\smallskip

\halftitlepageVolumeTranslationTitle{Analysis of Rules for Monks (1)}

\smallskip

\halftitlepageVolumeRootTitle{Bhikkhu Vibhaṅga}

\vspace*{\fill}

\sclogo{0}
 \halftitlepagePublisher{SuttaCentral}

\end{center}

\newpage
%
\setstretch{1.05}

\begin{footnotesize}

\textit{Theravāda Collection on Monastic Law} is a translation of the Theravāda Vinayapiṭaka by Bhikkhu Brahmali.

\medskip

Creative Commons Zero (CC0)

To the extent possible under law, Bhikkhu Brahmali has waived all copyright and related or neighboring rights to \textit{Theravāda Collection on Monastic Law}.

\medskip

This work is published from Australia.

\begin{center}
\textit{This translation is an expression of an ancient spiritual text that has been passed down by the Buddhist tradition for the benefit of all sentient beings. It is dedicated to the public domain via Creative Commons Zero (CC0). You are encouraged to copy, reproduce, adapt, alter, or otherwise make use of this translation. The translator respectfully requests that any use be in accordance with the values and principles of the Buddhist community.}
\end{center}

\medskip

\begin{description}
    \item[Web publication date] 2021
    \item[This edition] 2025-01-13 01:01:43
    \item[Publication type] hardcover
    \item[Edition] ed3
    \item[Number of volumes] 6
    \item[Publication ISBN] 978-1-76132-006-4
    \item[Volume ISBN] 978-1-76132-007-1
    \item[Publication URL] \href{https://suttacentral.net/editions/pli-tv-vi/en/brahmali}{https://suttacentral.net/editions/pli-tv-vi/en/brahmali}
    \item[Source URL] \href{https://github.com/suttacentral/bilara-data/tree/published/translation/en/brahmali/vinaya}{https://github.com/suttacentral/bilara-data/tree/published/translation/en/brahmali/vinaya}
    \item[Publication number] scpub8
\end{description}

\medskip

Map of Jambudīpa is by Jonas David Mitja Lang, and is released by him under Creative Commons Zero (CC0).

\medskip

Published by SuttaCentral

\medskip

\textit{SuttaCentral,\\
c/o Alwis \& Alwis Pty Ltd\\
Kaurna Country,\\
Suite 12,\\
198 Greenhill Road,\\
Eastwood,\\
SA 5063,\\
Australia}

\end{footnotesize}

\newpage

\setlength{\parindent}{1em}%%
\newpage

\vspace*{\fill}

\begin{center}
\epigraph{I will lay down a training rule for the following ten reasons: for the well-being of the Sangha, for the comfort of the Sangha, for the restraint of bad people, for the ease of good monks, for the restraint of corruptions relating to the present life, for the restraint of corruptions relating to future lives, to give rise to confidence in those without it, to increase the confidence of those who have it, for the longevity of the true Teaching, and for supporting the training.}
{
\epigraphTranslatedTitle{\textsanskrit{Theravāda} Monastic Law}
\epigraphRootTitle{Vinaya}
\epigraphReference{Monks’s Expulsion Rule One (\textsanskrit{Pārājika} 1)}
}
\end{center}

\vspace*{2in}

\vspace*{\fill}

\newgeometry{inner=0mm, outer=.5in, top=.6in, bottom=0mm}
\setlength{\parindent}{0em}
\sbox\IBox{\includegraphics{/app/sutta_publisher/images/jambudipa_map.png}}%
\includegraphics[trim=0 0 \dimexpr\wd\IBox-\textwidth{} 0,clip]{/app/sutta_publisher/images/jambudipa_map.png}
\newpage
\includegraphics[trim=\textwidth{} 0 0 0,clip]{/app/sutta_publisher/images/jambudipa_map.png}
\newpage
\restoregeometry

\blankpage%

\setlength{\parindent}{1em}
%
\tableofcontents
\newpage
\pagestyle{fancy}
%
\chapter*{Publisher’s Foreword}
\addcontentsline{toc}{chapter}{Publisher’s Foreword}
\markboth{Publisher’s Foreword}{Publisher’s Foreword}

\scbyline{Bhikkhu Sujato, 25 October 2023}

Ajahn Brahmal’s translation of the Pali Vinaya \textsanskrit{Piṭaka} is the culmination of work that began in 2013 as a revision of the standard translation by I.B. Horner for the Pali Text Society. As these things go, he rapidly found that more revisions were required, and the project became a new translation independent of Horner’s.

I have been honored to support Ven. Brahmali through this process as I was meanwhile developing my Sutta translations. We have discussed points of translation on many occasions, but the reader should be aware that this is a distinct work of his. We have not attempted to make the translations consistent, as there are only a few passages that directly overlap. I have, however, adopted Brahmali’s renderings for most Vinaya terms on the few occasions they appear in the Suttas.

Details aside, one of the great advantages Brahmali brings to his work is the wisdom of experience. He has lived and practiced for many years in a community that lives by the Vinaya. This brings a whole wealth of perspective and clarity to his work, as issues that are debated theoretically in academic circles are a part of daily life in a community. Through this whole process, he has been deeply contemplating the meaning of the Pali text and its expression in English, creating a living document that speaks to both letter and spirit.

Since 2005 SuttaCentral has provided access to the texts, translations, and parallels of early Buddhist texts. In 2018 we started creating and publishing our translations of these seminal spiritual classics. The “Editions” series now makes selected translations available as books in various forms, including print, PDF, and EPUB.

Editions are selected from our most complete, well-crafted, and reliable translations. They aim to bring these texts to a wider audience in forms that reward mindful reading. Care is taken with every detail of the production, and we aim to meet or exceed professional best standards in every way. These are the core scriptures underlying the entire Buddhist tradition, and we believe that they deserve to be preserved and made available in the highest quality without compromise.

SuttaCentral is a charitable organization. Our work is accomplished by volunteers and through the generosity of our donors. Everything we create is offered to all of humanity free of any copyright or licensing restrictions. 

%
\chapter*{Preface and acknowledgments}
\addcontentsline{toc}{chapter}{Preface and acknowledgments}
\markboth{Preface and acknowledgments}{Preface and acknowledgments}

Welcome to the first absolutely complete translation from Pali into English of the Vinaya \textsanskrit{Piṭaka}, the Monastic Law, of the Theravada school of Buddhism. This translation has been over ten years in the making, with the actual beginnings of the process no more than a hazy memory. When I started out, I had no clear sense that this would ever get published, and so it is especially satisfying to have reached this point.

The present translation is divided into six volumes, as follows:

\begin{enumerate}%
\item The Monks’ \textsanskrit{Pātimokkha} rules and their analysis (\textsanskrit{Mahā}-\textsanskrit{vibhaṅga}), part I%
\item The Monks’ \textsanskrit{Pātimokkha} rules and their analysis (\textsanskrit{Mahā}-\textsanskrit{vibhaṅga}), part II%
\item The Nuns’ \textsanskrit{Pātimokkha} rules and their analysis (\textsanskrit{Bhikkhunī}-\textsanskrit{vibhaṅga})%
\item The Chapters (Khandhakas), part I%
\item The Chapters (Khandhakas), part II%
\item The Compendium (\textsanskrit{Parivāra}).\footnote{The meaning of these titles and the contents of each volume will be discussed in the General Introduction and the introductions to each volume. }%
\end{enumerate}

This translation is based on the \textsanskrit{Mahāsaṅgīti} version (MS) of the Pali Canon as found on the website SuttaCentral.net. MS was created by the Dhamma Society of Thailand, and is essentially a corrected version of the \textsanskrit{Chaṭṭha} \textsanskrit{Saṅgāyana} \textsanskrit{Tipiṭaka}, the official Theravada \textsanskrit{Tipiṭaka} as produced at the Sixth Council in Burma. Sometimes the readings of MS were unclear or ambiguous, in which case I have consulted other versions of the \textsanskrit{Tipiṭaka}, specifically the Siamrath \textsanskrit{Tipiṭaka} (SRT) of Thailand, the Buddha \textsanskrit{Jayantī} \textsanskrit{Tipiṭaka} of Sri Lanka, and the Pali Text Society version (PTS). Whenever I depart from the readings of MS, I have recorded this in a footnote. Occasionally I have also consulted parallel texts in other languages, especially in the Āgama literature as preserved in Chinese. As to the commentaries and sub-commentaries, I have relied mostly on the VRI version available at tipitaka.org.

The purpose of the current translation has been to produce an accurate, clear, and accessible rendering of the Monastic Law into English. I have tried, throughout, to have the true users of the Vinaya in mind, that is, the monastics who live their lives according to these scriptures. To this end, I have attempted to make the text both meaningful and easy to read, with the objective of producing an easy-to-use guide that can readily be applied in one’s monastic life. I have endeavored to find a balance between formality and natural spoken language. My aim has been to give the reader a sense that the most important parts of the Vinaya \textsanskrit{Piṭaka} consist of real teachings, often spoken by the Buddha himself.

A secondary purpose has been to improve on I. B. Horner’s incomplete and at times inaccurate translation. Despite her admirable and careful work as a pioneer, Horner’s translation suffers from a number of shortcomings. Especially problematic is her failure to translate the more risqué parts of the Vinaya, of which there are quite a few. It so happens that these parts can be of critical importance to a monastic trying to understand the details of their training rules. Furthermore, Horner’s translation is often inaccurate, and occasionally outright wrong.\footnote{A glaring example is her translation of \textit{\textsanskrit{pariveṇa}} as “cell”, when it should be “yard”, that is, the area surrounding a building. Her rendering was perhaps influenced by the realities of medieval Christian monasticism. } At times it is impossible to understand her renderings, giving the impression that she did not properly grasp the meaning of the underlying Pali.\footnote{For instance, she renders the phrase \textit{\textsanskrit{anatthasaṁhite} \textsanskrit{setughāto} \textsanskrit{tathāgatānaṁ}} as “bridge-breaking for Truth-finders is among what does not belong to the goal”, which is unintelligible. A proper understanding of the Pali leads to a rendering along the following lines: “Buddhas are incapable of doing what is unbeneficial”. That is, \textit{\textsanskrit{setughāta}}, which literally means “breaking the bridge”, needs to be understood metaphorically as conveying the inability to do certain actions. } I have tried to avoid such issues by always translating clearly, even in cases where the meaning was in doubt. In my view, it is better to translate meaningfully, even if sometimes wrongly, than to leave the reader bewildered. At least a text with a clear meaning can be duly criticized.

The Vinaya \textsanskrit{Piṭaka} is a specialist’s corner of the Pali Canonical texts, and the readership will inevitably be limited. In fact, the Vinaya is really a kind of support literature, with the Suttas being at the core of the Buddhist tradition. Never mind, I shall rest content with the tiniest of readership. At least one person has already benefitted. May you enjoy it too!

\thematicbreak
To ensure the quality of my translation, I have often compared my understanding of the Pali with that of my illustrious predecessors. In particular, I have consulted I. B. Horner’s pioneering translation, \textit{The Book of the Discipline} (BD). Despite often disagreeing with her renderings, I have learned much from her philological approach. Another important source has been Ajahn \textsanskrit{Ṭhānissaro}’s work, \textit{The Buddhist Monastic Code I} and \textit{II}, in which he makes a large number of corrections to I. B. Horner’s translation. Ven. \textsanskrit{Ñāṇatusita}’s detailed \textit{Analysis of the Bhikkhu \textsanskrit{Pātimokkha}} has been yet another important reference work. His detailed analysis of the \textsanskrit{Pātimokkha} rules is a goldmine of information. I have also consulted Ven. Bhikkhu Bodhi’s translations “The Numerical Discourses of the Buddha”, “The Connected Discourses of the Buddha”, and, with Ven. \textsanskrit{Ñāṇamoli}, “The Middle Length Discourses of the Buddha”, as well as Ven. \textsanskrit{Ñāṇamoli}’s “The Life of the Buddha”. Other works consulted are referenced in the footnotes.

During the long process of translating the Vinaya \textsanskrit{Piṭaka}, I have received the kind and generous help of a number of people. First and foremost, I have benefitted from the unstinting support of my preceptor Ajahn Brahm, who first taught me both Vinaya and Pali. During a span of more than three decades, he has shown me how to combine an overarching view of the purpose of the Vinaya with an appreciation of minutiae and the legal nature of the text. Moreover, his pragmatic and compassionate approach to understanding the Vinaya makes it eminently relevant to modern monasticism. I never would have dared to undertake this enterprise without this ballast.

Then there is my good friend and sometimes mentor, Bhante Sujato, who is ultimately responsible for this whole project. It is truly astonishing what he has been able to achieve as an extended result of his work on SuttaCentral. Bhante Sujato has been a critical resource for everything from technical discussions to details of layout.

I have had a few indefatigable supporters whose intelligence and eye for detail have made this translation so much better. My main supporter has been Tracy Lau, now Ven. \textsanskrit{Nadī} of Dhammasara Nuns Monastery, who has read through the entire manuscript at least three times, making innumerable corrections, suggestions, and observations. It’s a true blessing to have such dedicated and intelligent support over such a long period. Another important helper has been my co-monastic Ven. \textsanskrit{Mettavihāri} of Australia who read through the present translation on two separate occasions. I am especially grateful for his grammatical corrections and his help in laying out the document in the appropriate way.

In the early stages of this project, a significant amount of work was done to match my translation line by line with the segmented Pali text on SuttaCentral. Four people in particular put a lot of effort into this task: Ven. \textsanskrit{Vimalā} of the Netherlands, Ven. \textsanskrit{Sabbamittā} of Germany, Ven. \textsanskrit{Nadī} of Canada, and Tara Athan of the US. In addition, Ven. \textsanskrit{Vimalā} has kindly shared their detailed research on \textit{\textsanskrit{paṇḍakas}} and \textit{\textsanskrit{ubhatobyañjanakas}} (for which see Appendix I: Technical Terms), whereas Ven. \textsanskrit{Sabbamittā} has assisted with proofreading.

In addition to these major helpers, a number of others have lent their support in various ways. I have had several helpful discussions with Ven. \textsanskrit{Munissarā} of Australia on a variety of technical points, while Ven. Bhikkhu Bodhi helped with a tricky passage in the \textsanskrit{Parivāra}. Ven. \textsanskrit{Dhammānando} of England helped occasionally with difficult readings from the commentaries, and Ven. Sunyo of Australia made important suggestions for the introduction. Ven. \textsanskrit{Vimalañāṇī} of Germany was especially helpful with looking up parallels in the Vinayas in Chinese translation, whereas Ven. \textsanskrit{Suvīrā} of Australia helped me understand the nature of the \textit{\textsanskrit{saṅkacchika}}. Other significant helpers include Ven. Khemaratana and Ven. Khemarato of the US, Ven. \textsanskrit{Karuṇikā} of Australia, Ven. Abhayaratana of Canada, Ven. Vimutti of Australia, and Ven. \textsanskrit{Pāladhammika} of the US. I must also mention Bryan Levman for our occasional discussions of Vinaya terminology. Finally, I am grateful for the help I have received from a large number of others, either indirectly by way of Vinaya discussions or directly as corrections or suggestions for changes to the manuscript.

To all of you, thank you for your invaluable support!

Bhikkhu Brahmali\\

Perth, Western Australia\\

13 November 2024

%
\chapter*{General introduction to the Monastic Law}
\addcontentsline{toc}{chapter}{General introduction to the Monastic Law}
\markboth{General introduction to the Monastic Law}{General introduction to the Monastic Law}

\scbyline{Bhikkhu Brahmali, 2024}

The Vinaya \textsanskrit{Piṭaka}, “the Basket of Monastic Law”, contains the rules that are binding on monastics and the regulations that apply to monastic communities. The Monastic Law is available in more recensions than any other part of the \textsanskrit{Tipiṭaka}. There is a full version in Pali, belonging to the Theravada school of Buddhism. Then there are four complete versions extant in Chinese translation, all belonging to different schools of early Buddhism: \textsanskrit{Mahāsāṅghika}, Dharmaguptaka, \textsanskrit{Mahīśāsaka}, and \textsanskrit{Sarvāstivāda}. The Chinese \textsanskrit{Tipiṭaka} also preserves other Vinaya related texts, such as an independent \textit{bhikkhu \textsanskrit{pātimokkha}} of the \textsanskrit{Kāśyapīya} School and several more or less school-specific Vinaya texts. The Vinaya of the \textsanskrit{Mūlasarvāstivāda} school exists in three versions: a complete text in Tibetan translation, a mostly complete version in Chinese, and substantial portions in Sanskrit. There are also several Vinaya texts, as well as a large number of fragments, in Sanskrit and other Indic languages, mostly of \textsanskrit{Mahāsāṅghika}, \textsanskrit{Sarvāstivāda}, and \textsanskrit{Mūlasarvāstivāda} provenance. The present work is a full translation into English of the Pali version of the Vinaya \textsanskrit{Piṭaka}.

\section*{Origin}

The word \textit{vinaya}, here translated as “Monastic Law,” originally probably meant “training,” as can be seen from its usage in the Sutta \textsanskrit{Piṭaka}, “the Basket of Discourses.” In this sense it complements the Dhamma, the doctrine or teaching, which provides the instructions on how the training is to be achieved. The compound \textit{dhamma-vinaya} is a common one in the earliest literature and might be rendered as “theory and practice.” Gradually the meaning of \textit{vinaya} shifted to denote the rules of conduct instead, thus referring to the training in a narrower sense. Although the former usage is more common in the Suttas, it is this latter usage of \textit{vinaya} which has become the dominant one and which has prevailed to the present day.\footnote{For further discussion of the meanings of \textit{vinaya} and \textit{dhammavinaya}, see Appendix I: Technical Terms. }

The Monastic Law developed over a period of several centuries after the Buddha’s passing away. Yet given the close agreement on some of the most fundamental aspects of the Vinaya across all surviving scriptures, it seems likely that the earliest parts originated in the lifetime of the Buddha. This includes the rules of conduct binding on all monastics, known as the \textsanskrit{Pātimokkha}, and several of the most important legal procedures that regulate the proper functioning of the monastic communities. It is only these parts of the Vinaya that are part of the Early Buddhist Texts in the strictest sense.

Around this kernel the Vinaya gradually expanded. Over time, the \textsanskrit{Pātimokkha} rules gained a canonical commentary that included origin stories, word analyses, detailed permutation series on the applicability of the rules, non-offense clauses, and case studies. For the rest of the Vinaya, known as the Khandhakas, the expansion was less structured, with minor rules, stories, and procedures apparently being added as the need arose. It has been shown by Frauwallner that, despite a significant common core, many of the details of this part of the Vinaya vary between the schools.\footnote{Frauwallner, 1956. For instance, on p. 4 he says: “Many differences are indeed apparent in the arrangement and elaboration of the materials.” }

The exact cut-off-point after which no new material was added to the Canonical Vinaya is impossible to pin down and it would have varied from school to school. On linguistic grounds, it seems likely that the majority of additions to the Pali Vinaya, with the exception of the \textsanskrit{Parivāra}, were done prior to its arrival in Sri Lanka in the third century BCE.\footnote{See Norman 1990, p. 90: “From the point of view of its language, we should have expected anything added in Sri Lanka to show traces of the local Prakrit, but there are few signs of borrowings from Sinhalese Prakrit being inserted into it …” } After this point new material was added to the commentarial literature, which, despite its likely origin in the mainland, was greatly expanded and developed in Sri Lanka.

The Vinaya was not established as part of an overall plan to provide the monastic community with a legal structure, but was laid down rule by rule in response to problems as they arose in the Sangha. It is the Dhamma, the teaching, that guided the laying down of the Vinaya, and the Vinaya is subsidiary to and bound up with the broader concerns of the proper practice of the Buddhist path. A large number of rules were laid down in response to the lay people’s criticism of the Sangha.

\section*{Textual transmission and the schools}

The number of extant Vinaya texts is quite large and the process of transmission and translation into various Indic languages and especially into Chinese and Tibetan is quite complex. In what follows I give an outline of how the main Vinaya texts were transmitted to China and Tibet.

The first split in the Sangha occurred between the \textsanskrit{Mahāsāṅghikas} and the Sthaviras, very roughly around 200 BCE. Each of these branches subsequently split into a number of sub-schools. Of the six complete Vinayas still extant, only one belongs to the \textsanskrit{Mahāsāṅghika} group and the remaining five to sub-schools of the Sthaviras. We should therefore expect to find shared qualities between the Vinayas of the Sthavira schools that are lacking in the \textsanskrit{Mahāsāṅghika} Vinaya. Indeed, the Khandhakas of the \textsanskrit{Mahāsāṅghika} Vinaya is structured differently from that of all the other Vinayas.\footnote{Frauwallner, pp. 11 and 198–207. }

The sub-schools of the Sthavira branch for which we still have complete Vinayas fall into two sub-groups: the \textsanskrit{Sarvāstivāda} and the \textsanskrit{Mūlasarvāstivāda} on the one hand, and the Dharmaguptaka, the \textsanskrit{Mahīśāsaka}, and the Theravada on the other. First the \textsanskrit{Sarvāstivādins} split from the rest of the Sthaviras. Over time the \textsanskrit{Mūlasarvāstivāda} emerged as a sub-school of the \textsanskrit{Sarvāstivāda}, and for this reason the Vinayas of these two schools share certain characteristics.\footnote{Frauwallner, p. 194. } After the \textsanskrit{Sarvāstivādin} split, the remainder of the Sthaviras divided further, including into the Dharmaguptaka, the \textsanskrit{Mahīśāsaka}, and the Theravada. Yet these three schools were probably no more than regional variations of each other and consequently their Vinayas have much in common.\footnote{See respectively Sujato, 2012b, p. 101, and Frauwallner, p. 181. }

Apart from the Theravada Vinaya, the following are the main Canonical Vinayas still extant:

\begin{itemize}%
\item A complete \textsanskrit{Mahāsāṅghika} Vinaya, found in the Chinese \textsanskrit{Tipiṭaka} at T 1425, was translated into Chinese by Faxian and Buddhabhadra in 416–418 CE. Although its section of Khandhakas is structured differently from that of the other schools, the content appears to largely overlap. Further study is required to clarify the degree of divergence. Substantial parts of this Vinaya have also been preserved in Buddhist Hybrid Sanskrit, including the \textsanskrit{Mahāvastu}, a large work mostly concerned with the biography of the Buddha, as well as the Sutta-\textsanskrit{vibhaṅga} for the nuns’ and the monks’ \textsanskrit{Pātimokkhas}.%
\item A complete \textsanskrit{Sarvāstivāda} Vinaya is preserved in Chinese at T 1435, translated by \textsanskrit{Kumārajīva} in 404–409 CE. There are also a number of surviving fragments in Sanskrit.%
\item A full translation of the \textsanskrit{Mūlasarvāstivāda} Vinaya into Tibetan, found in the Kanjur at D 1–7/P 1030–1036, was completed in the first decade of 9th century CE by Jinamitra of Kashmir and various others. There is a version of this Vinaya in Chinese at T 1441–1457, largely translated by Yijing in 703–710 CE. This translation is incomplete and full of gaps (Frauwallner, 1956, p. 195). In addition to this, approximately 80 percent of the Khandhakas exist in Sanskrit (Clarke, 2015, p. 75).%
\item Apart from a few fragments in Sanskrit and \textsanskrit{Gāndhārī}, a full Dharmaguptaka Vinaya is only preserved in Chinese at T 1428, translated by \textsanskrit{Buddhayaśas} and Zhu Fonian in 410–412 CE. Of all the extant Vinayas, this is the one normally regarded as closest to the Theravada Vinaya (Clarke, 2015, p. 69).%
\item The \textsanskrit{Mahīśāsaka} Vinaya is only extant in Chinese at T 1421, translated by \textsanskrit{Buddhajīva} from Kashmir and others in 423–424 CE from a manuscript brought from Sri Lanka by Faxian. According to Frauwallner (1956, pp. 183–84), this Vinaya is full of gaps. It is closely related to the Dharmaguptaka Vinaya (Frauwallner, 1956, p. 181).%
\item Apart from the full Vinayas listed above, there are a variety of Canonical Vinaya texts and fragments in different languages. One significant text is the monks’ \textsanskrit{Pātimokkha} of the \textsanskrit{Kāśyapīya} School, available at T 1460 and translated into Chinese by Gautama \textsanskrit{Prajñāruci} in 543 CE.%
\end{itemize}

\section*{Content}

The Vinaya \textsanskrit{Piṭaka} is divided into two main parts: the Sutta-\textsanskrit{vibhaṅga}, “The Analysis of the \textsanskrit{Pātimokkha} Rules,” and The Khandhakas, “the Chapters.” The individual schools sometimes have additional texts, such as the \textsanskrit{Parivāra}, “The Compendium,” belonging to the Theravada tradition, and the Uttaragrantha belonging to the \textsanskrit{Mūlasarvāstivādins}.

\subsection*{The Sutta-\textsanskrit{vibhaṅga}}

Sutta-\textsanskrit{vibhaṅga} means “Analysis of the Sutta.” Sutta here does not refer to the discourses, but rather to the \textsanskrit{Pātimokkha} rules as a complete set.

The Sutta-\textsanskrit{vibhaṅga} consists of the \textsanskrit{Pātimokkha} rules embedded in a commentary that analyzes each rule in detail. The Sutta-\textsanskrit{vibhaṅga} is divided into two parts, which in the Theravada tradition center on the 227 rules for the monks and the 311 rules for the nuns. The majority of rules are the same for the two Sanghas, but 130 are specific to the nuns and 46 specific to the monks. We will discuss the discrepancy in the number of rules for the two Sanghas in detail in the introduction to volume 3.

The rules are categorized according to the penalty incurred for breaching them. The heaviest penalty, expulsion from the Sangha, is incurred only for conduct that is fundamentally opposed to monastic life, such as sexual intercourse or murder. There are four such rules for the monks and eight for the nuns. The second heaviest penalty consists of a period of suspension and probation during which time one is not a full member of the Sangha. There are thirteen such rules for the monks and seventeen for the nuns. The vast majority of offenses, however, are cleared simply by confession. These rules are subdivided into a number of categories dependent on factors such as the severity of the breach, the sort of confession that is required, and additional requirements such as relinquishment of wrongly acquired requisites. The last seven rules of the Sutta-\textsanskrit{vibhaṅga} are principles for resolving legal issues, that is, any issue the Sangha needs to deal with as a community. Most of the material connected with these principles is now found in the Khandhakas.

Within the Sutta-\textsanskrit{vibhaṅga}, each rule is largely self-contained and forms its own subsection. These sections begin with one or more origin stories that relate the incident that led the Buddha to lay down a particular rule. Many of these are no more than brief accounts of a stereotypical monk or nun who is simply stated to have done something inappropriate. A few are elaborate narratives that may include sub-rules or important procedures for the Sangha, and occasionally even \textit{sutta}-type material or \textsanskrit{Jātaka}-type stories. The majority of origin stories fall somewhere in between these two extremes.

Following the origin story is the actual rule. In a number of cases the original rule is later amended by the Buddha, sometimes several times, before it reaches its final form. The rule is then analyzed in detail in a word commentary, in which each significant word of the rule is defined. These definitions range from merely supplying a synonym to large sections with a detailed exposition. The word commentary is always technical in nature.

After the word commentary, many rules are further analyzed as to their applicability, given a number of general scenarios. These sections normally take the form of a permutation series in which a certain number of factors are varied in all possible combinations with each other. These sections, too, are highly technical.

Next comes a non-offense clause, which sets out important exemptions for each rule. The non-offense clause is sometimes followed by a set of case studies. These concern specific instances where a monastic acts in such a way that it is not clear-cut whether they have committed an offense. The incident is related and the Buddha then decides on the matter. This section is similar in content to the origin stories. Only the first nine rules of the monks’ \textsanskrit{Pātimokkha} have this section.

Comparative study of the various \textsanskrit{Pātimokkhas} makes it clear that these texts in large part go back to the pre-sectarian period of Buddhism.\footnote{Pachow, 2000. } As for the rest of the material in the Sutta-\textsanskrit{vibhaṅga}, academics normally consider this material to be significantly later than the \textsanskrit{Pātimokkha} rules, but it is nevertheless likely that some of it goes back to the earliest period. In the absence of more detailed research, it seems prudent to regard the \textsanskrit{Pātimokkha} as the only part of the Sutta-\textsanskrit{vibhaṅga} that belongs to the Early Buddhist Texts.\footnote{See respectively Hinüber, 2000, pp. 13f, and Pachow, pp. 14ff. }

But even this overstates the case, for it is clear that not even all the \textsanskrit{Pātimokkha} rules belong to the earliest period. This is true of many, perhaps all, of the most minor rules of the monks’ \textsanskrit{Pātimokkha}, the \textit{sekhiyas}, but especially of the rules for the nuns, many of which vary considerably between the different schools, making it likely that they stem from the sectarian period.\footnote{See Pachow, 2000, and Kabilsingh. 1998. }

\subsection*{The Khandhakas}

The other main part of the Vinaya, the Khandhakas, is a group of sections that each discuss a major area of monastic law, such as a section on ordination, several sections on allowable requisites, and a number of sections that deal with technical matters. The Theravada Khandhakas are a set of 22 such sections, all of which are matched by equivalent sections in the other existing Vinaya recensions, with the partial exception of the \textsanskrit{Mahāsāṅghikas}.\footnote{Frauwallner, p. 3. } The Khandhakas of the \textsanskrit{Mahāsāṅghikas}, although containing much of the same material as the other Vinaya recensions, are structured differently. There is as yet no scholarly consensus as to why this is the case and what might be the implications for the historical evolution of the Khandhakas.

The Khandhakas lack the close unifying principle found in the Sutta-\textsanskrit{vibhaṅga}, which, as we have seen, is organized as a commentary and analysis of the \textsanskrit{Pātimokkha} rules. This makes the Khandhakas less integrated and more diverse than the Sutta-\textsanskrit{vibhaṅga}.

In place of the rigid structure of the Sutta-\textsanskrit{vibhaṅga}, the Khandhakas are loosely structured around the life story of the Buddha. After the Buddha’s awakening, he set out to teach others about his discovery. As he started to gain a monastic following, the need for rules and procedures gradually arose. This need continued throughout the Buddha’s life. It is this process that furnishes the framework for the Khandhakas as a whole.

The “biography” of the Buddha is in fact largely considered part of the Vinaya in all Buddhist schools. The Khandhakas show ordinary interactions of the Buddha with monastics and lay people, and we get a glimpse of the Buddha as a real person, not just as the distant teacher and leader of a large religious organization. We see him walking around large parts of the Ganges plain, meeting a variety of people. We see him in close contact with his monastic disciples, criticizing their misdeeds, but also praising them when they get it right. The touching story of the Buddha and Ānanda cleaning up a monk suffering from dysentery is found in the Khandhakas. This close and almost personal view of the Buddha is one factor that makes the Khandhakas a particularly interesting collection.

One of the main functions of the Khandhakas is to present the procedures by which the Sanghas conduct their business. These include not only the ordination procedure and the \textit{uposatha} ceremony, but also a number of other procedures that enable the Sanghas to function properly. These procedures are governed by precise rules, especially regarding democratic participation and decentralized decision making. They allow for effective and harmonious dispatch of monastic business.

The Khandhakas include a large number of minor rules not found in the \textsanskrit{Pātimokkha}. These rules are diverse, but can broadly be summarized as prohibiting luxuries and sensual behavior, both of which are incompatible with the renunciant life.

The Khandhakas also include background stories of some of the Buddha’s most well-known lay disciples, such as \textsanskrit{Anāthapiṇḍika}, \textsanskrit{Visākhā}, and \textsanskrit{Jīvaka}. There are also stories about monastic disciples, such as the remarkable story of Pilindavaccha, the inspiring stories of \textsanskrit{Soṇa} Kolivisa and \textsanskrit{Soṇa} \textsanskrit{Kuṭikaṇṇa}, as well as the downfall of Devadatta. Then there are several \textsanskrit{Jātaka}-type stories, some of which are also found in the \textsanskrit{Jātaka} collection. On top of this, each section often has its own origin story, similar to those found in the Sutta-\textsanskrit{vibhaṅga}. But apart from the origin stories, the Khandhakas lack the detailed exegetical material of the Sutta-\textsanskrit{vibhaṅga}.

The third last chapter of the Khandhakas deals with rules and procedures that are specific to the nuns, including their ordination procedure. Unless otherwise stated or implied, the rest of Khandhakas apply to both Sanghas.

The Khandhakas end with a description of the first Council, \textit{\textsanskrit{saṅgīti}}, a communal recitation of the teachings after the Buddha’s passing away, as well as the famous \textsanskrit{Vesālī} affair, sometimes known as the second Council, where the Sangha with difficulty resolved a disagreement over issues of Vinaya. The \textsanskrit{Vesālī} affair is said to have happened around one hundred years after the Buddha passed away. It is around this time that sectarian tendencies are starting to form in the Sangha, and this is roughly the cut-off point for the common heritage of all Buddhists.

\subsection*{Other texts}

The Theravada tradition includes the \textsanskrit{Parivāra}, “the Compendium”, in its Vinaya \textsanskrit{Piṭaka}. Oskar von Hinüber (2000, p. 22) suggests it was completed no later than the first century AD. The \textsanskrit{Parivāra} is an analytical summary of the first two parts of the Vinaya. In style and method, it is sometimes compared to the Abhidhamma.

Other schools, too, have Vinaya summaries and addenda that may or may not share material with the \textsanskrit{Parivāra}. Because of a lack of research, not much is known about these texts. It seems clear, however, that none of them is part of the Early Buddhist Texts.

\section*{Modern perspectives}

Most of the early schools of Buddhism have long since disappeared, but three Vinaya traditions are still alive: the Dharmaguptaka, practiced in East Asia, including China and Korea; the \textsanskrit{Mūlasarvāstivāda}, practiced in Tibet and Mongolia; and the Theravada, practiced in South and Southeast Asia.

In practice, it is rare for monastics to follow all the stipulations of their chosen Vinaya lineage. For instance, although the use of money is prohibited by the \textsanskrit{Pātimokkha} rules of all schools, it is nevertheless used by the vast majority of monastics. The extent to which the rules are followed varies enormously, but most monastics do at least follow the most important rules, that is, the rules entailing expulsion and those entailing suspension. A similar situation holds for the procedures that govern the Sanghas. Sometimes they are practiced to the letter, such as most ordination ceremonies in the Theravada tradition. At other times the procedures are misinterpreted or simply disregarded, such as the procedures for choosing the officials of the Sangha.

Over the course of Buddhist history, there have been periodic reform movements and irregular attempts at purifying the Sangha. Typically, the Sangha gradually degenerates until a charismatic leader starts a reform movement aimed at the proper practice of the Buddhist path, including the Vinaya. These reform movements sometimes manifest as “forest traditions,” whereby monastics establish forest monasteries in conformity with the ideals of early Buddhism. Since the mid-1990s, one controversial and ongoing reform has been the reestablishment of the Sangha of nuns, \textit{\textsanskrit{bhikkhunīs}}, in the Theravada tradition.

\section*{The commentaries}

Another important component of the monastic Vinaya is the vast commentarial literature that has gradually evolved over the centuries and millennia, and continues to do so to the present day. All three of the living Vinaya traditions have such a commentarial literature.

The commentarial literature begins with the Sutta-\textsanskrit{vibhaṅga}, which, although it is now part of the Canon, is an early commentary on the \textsanskrit{Pātimokkha} rules. Next, we have other Canonical commentaries or summaries, such as the \textsanskrit{Parivāra} of the Theravadins. Beyond these, we come to the commentaries proper, the \textit{\textsanskrit{atthakathās}}, “The Discussion on Meaning.”

The most important non-canonical commentary on the Theravadin Vinaya \textsanskrit{Piṭaka} is the \textsanskrit{Samantapāsādikā}, composed in Sri Lanka by Buddhaghosa in the fifth century CE based on pre-existing commentaries that probably originated in India. There is also another important commentary from this period, the \textsanskrit{Kaṅkhāvitaraṇī}, also composed by Buddhaghosa. The next layer of commentaries are the \textit{\textsanskrit{ṭīkās}}, the sub-commentaries, of which there are over a dozen, including highly specialized literature, such as handbooks on monastery zones (\textit{\textsanskrit{sīmās}}). \textit{\textsanskrit{Ṭīkās}} continue to be composed to the present day. The extent to which the Canonical Vinaya needs to be interpreted in line with this commentarial tradition is typically controversial, and practices vary widely.

To navigate this vast literature, many Theravada monasteries rely on modern summaries for their practice of the Vinaya. Examples include the \textit{Vinayamukha} in Thai and Bhikkhu \textsanskrit{Ṭhānissaro}’s \textit{The Buddhist Monastic Code} in English.

In addition to the above, most Theravada monasteries follow a number of rules that are more informal in nature. These include rules used to distinguish individual sects (\textsanskrit{Nikāyas}), such as rules on the style of robes and on the manner of wearing them. Then there are rules that pertain to particular teacher traditions, such as those that often form around especially charismatic and famous teachers. The final set of rules are those laid down at individual monasteries. These regulate the daily schedule and other aspects of monastic life that are monastery specific. Although all these rules are sometimes called Vinaya and therefore assumed to stem from the Vinaya \textsanskrit{Piṭaka} or at least the commentaries, in reality few of them have any Canonical basis.

%
\chapter*{Introduction to the Monks’ Analysis, part I}
\addcontentsline{toc}{chapter}{Introduction to the Monks’ Analysis, part I}
\markboth{Introduction to the Monks’ Analysis, part I}{Introduction to the Monks’ Analysis, part I}

The present volume is the first of six, the total of which constitutes a complete translation of the Pali Vinaya \textsanskrit{Piṭaka}, the Monastic Law, of the Theravada school of Buddhism. For a general overview of the Monastic Law, see the General Introduction above. In the present introduction, I will survey the contents of volume 1 and make observations of points of particular interest.

The first part of the Vinaya \textsanskrit{Piṭaka} is known as the Sutta-\textsanskrit{vibhaṅga}, which can be rendered as the Analysis of the Sutta. In this context the word \textit{sutta} does not mean a discourse of the Buddha, but refers to the \textsanskrit{Pātimokkha}, the Monastic Code, which consists of the rules of conduct that form the kernel of the Sutta-\textsanskrit{vibhaṅga}.\footnote{See discussion in Bhikkhu Ñā\textsanskrit{ṇatusita}, “Analysis of the Bhikkhu Pātimokkha”, pp. 50–55. } The Sutta-\textsanskrit{vibhaṅga}, then, is the analysis of the rules of the \textsanskrit{Pātimokkha}.

Because there are two \textsanskrit{Pātimokkhas}, the Sutta-\textsanskrit{vibhaṅga} is divided into two principal parts, the Bhikkhu-\textsanskrit{vibhaṅga} and the \textsanskrit{Bhikkhunī}-\textsanskrit{vibhaṅga}, the Monks’ Analysis and the Nuns’ Analysis.\footnote{Modern versions of the Vinaya \textsanskrit{Piṭaka} divide the Sutta-\textsanskrit{vibhaṅga} into two volumes, known as \textsanskrit{Pārājika}-\textsanskrit{pāḷi} and Pacittiya-\textsanskrit{pāḷi}. This division is not mentioned in any of the texts available in the online VRI version, and seems to be an entirely modern division, presumably to fit the text into suitable volumes for printing. } The Monks’ Analysis is also known as the \textsanskrit{Mahā}-\textsanskrit{vibhaṅga}, the Great Analysis. In the present introduction, I will focus on the first part of the monks’ \textsanskrit{Pātimokkha} and its Analysis, specifically the first three classes of rules, known as the \textit{\textsanskrit{pārājikas}}, the \textit{\textsanskrit{saṅghādisesas}}, and the \textit{aniyatas}, the meanings of which I will give shortly. I will also discuss some general themes. The remaining monks’ rules will be treated in a separate introduction to volume 2, while the nuns’ rules will feature in volume 3.

The monks’ \textsanskrit{Pātimokkha} consists of 227 rules that are divided into eight classes, presented in what may be considered as a descending order of importance. In brief, they are as follows:

\begin{enumerate}%
\item The \textit{\textsanskrit{pārājikas}} (Pj), “the offenses entailing expulsion”%
\item The \textit{\textsanskrit{saṅghādisesas}} (Ss), “the offenses entailing suspension”%
\item The \textit{aniyatas} (Ay), “the indeterminate offenses”%
\item The \textit{nissaggiya \textsanskrit{pācittiyas}} (NP), “the offenses entailing relinquishment and confession”%
\item The \textit{\textsanskrit{pācittiyas}} (Pc), “the offenses entailing confession”%
\item The \textit{\textsanskrit{pāṭidesanīyas}} (Pd), “the offenses entailing acknowledgment”%
\item The \textit{sekhiyas} (Sk), “the rules of training”%
\item The \textit{\textsanskrit{adhikaraṇasamathadhammas}}, or just \textit{\textsanskrit{adhikaraṇasamathas}} (As), “the principles for settling legal issues”.%
\end{enumerate}

\section*{The historical development of the \textsanskrit{Pātimokkha}}

In what follows, I will briefly discuss the historical development of these rules. A good place to start is with W. Pachow’s Comparative Study of the \textsanskrit{Prātimokṣa}.\footnote{Pachow, 1955. } In this ground-breaking study, Pachow compares all existing versions of the \textsanskrit{Pātimokkha}, altogether ten recensions coming from seven different schools of early Buddhism. One of his interesting discoveries is that the rules, both in number and in wording, are very closely related to each other, with the exception of the second last class, the \textit{sekhiyas}. If we leave these aside, it is obvious that the \textsanskrit{Pātimokkha} rules hark back to a common ancestor that must have existed before the separate schools of Buddhism started to emerge. And given the conservatism of Buddhism, which seems to be a result of the Buddha’s explicit instructions, it seems plausible that this common core goes back to the Buddha himself.\footnote{For instance at \href{https://suttacentral.net/dn16/en/sujato\#1.6.13}{DN~16:1.6.13} and \href{https://suttacentral.net/dn16/en/sujato\#6.1.5}{DN~16:6.1.5}, and also at \href{https://suttacentral.net/dn29/en/sujato\#17.1}{DN~29:17.1}. }

As to the \textit{sekhiya} rules, there is significant variation between the different schools. The number of \textit{sekhiyas} varies from 66 in the \textsanskrit{Mahāsāṅghika} recension to 113 for the \textsanskrit{Sarvāstivādins}. Further, the order of the rules is often very different in the different schools. Yet, as can be seen from Pachow’s concordance tables, 28 \textit{sekhiyas} are found in almost identical form across the various schools, a number that increases to 45 if we disregard cases where only a single school is missing a rule.\footnote{The counting of rules is not as straightforward as it may seem. Two of the versions used by Pachow are not recensions of the \textsanskrit{Pātimokkha} as such, but other kinds of texts that include the \textsanskrit{Pātimokkha} rules. I have not included these versions. Then there is the problem of multiple \textsanskrit{Pātimokkhas} of the same school. In these cases, I have counted a rule only if it exists in all versions of a specific school. The counting could be done differently, but the results would not change dramatically. } From this we can conclude that even the \textit{sekhiyas}, as a class, must have existed in the earliest period. We are left with a picture of the earliest \textsanskrit{Pātimokkha}, presumably as laid down by the Buddha, as consisting of all the classes we have today, with the only significant difference being the number of \textit{sekhiya} rules.

Let us now take a closer look at the Pali tradition. In the \textsanskrit{Aṅguttara} \textsanskrit{Nikāya}, the Numerical Discourses, we find a few \textit{suttas}, namely, \href{https://suttacentral.net/an3.84/en/sujato}{AN~3.84}, \href{https://suttacentral.net/an3.86/en/sujato}{AN~3.86}, \href{https://suttacentral.net/an3.87/en/sujato}{AN~3.87}, and \href{https://suttacentral.net/an3.88/en/sujato}{AN~3.88}, that speak of “over a hundred and fifty training rules”. Clearly this must refer to a time when the \textsanskrit{Pātimokkha} was shorter than it is now. The fact that all four \textit{suttas} speak of over 150 training rules, \textit{\textsanskrit{sikkhāpadas}}, may suggest that this was the number reached while the Buddha was still alive. It is also possible that the number is deliberately round because the rules were still being added to. Still, given that the text uses the number 150, it seems reasonable to assume that the number was significantly less than 200. So, what might these 150+ rules have been?

To start with, we can exclude the last seven rules, the \textit{\textsanskrit{adhikaraṇasamathas}}. \href{https://suttacentral.net/pli-tv-pvr10/en/brahmali\#38.1}{Pvr~10:38.1}–38.5 states that the monks had 220 training rules, while the nuns had 304, which means the last seven rules are not counted. This makes sense for they are not “training rules”, but rather broader principles for dealing with Sangha “issues”. This, however, does not mean that the seven were not originally part of the \textsanskrit{Pātimokkha}, as suggested by Bhikkhu Nyanatusita and K. R. Norman.\footnote{See Bhikkhu Ñā\textsanskrit{ṇatusita}, “Analysis of the Bhikkhu Pātimokkha”, p. 46, and K. R. Norman, 1983, p. 19. } The \textit{\textsanskrit{adhikaraṇasamathas}} are found in all the early Buddhist schools, and even the order of rules is largely the same. This makes it likely that these rules were there from earliest times. Nevertheless, it is unexpected to find such rules in the \textsanskrit{Pātimokkha}, a state of affairs I will discuss further below.

Second, based on the findings of Pachow discussed above, we can say with a fair amount of certainty that the discrepancy is to be found among the \textit{sekhiyas}. If we remove the class of \textit{sekhiya} offenses as a whole, however, as suggested by some,\footnote{See Bhikkhu Ñā\textsanskrit{ṇatusita}, “Analysis of the Bhikkhu Pātimokkha”, p. 46. } the total number of rules is reduced to 145, which is too low. We must therefore conclude, once again, that some of the \textit{sekhiyas} go back to the earliest period. Nonetheless, it is clearly the other classes of rules that form the core of the \textsanskrit{Pātimokkha}.

We are now in a position to say what may have been the earliest training rules. The early \textsanskrit{Pātimokkha} had in excess of 150 training rules, comprising the \textit{\textsanskrit{pārājikas}}, \textit{\textsanskrit{saṅghādisesas}}, \textit{\textsanskrit{pācittiyas}}, \textit{\textsanskrit{pāṭidesanīyas}}, and a relatively small number of \textit{sekhiyas} compared to what we have now. The total number of rules would have been in the range 150 to 200, but probably closer to 150. Moreover, this number would have been in flux as new rules were added as and when required. The \textit{\textsanskrit{adhikaraṇasamathas}}, and probably also the \textit{aniyatas}, were part of the \textsanskrit{Pātimokkha}, but were not considered training rules. These rules, rather, were part of the broader Vinaya material that was included in the \textsanskrit{Pātimokkha}, for which see below.

The above result is reinforced by another \textit{sutta}, \href{https://suttacentral.net/an4.244/en/sujato}{AN~4.244}, which only mentions four classes of offenses, that is, \textit{\textsanskrit{pārājikas}}, \textit{\textsanskrit{saṅghādisesas}}, \textit{\textsanskrit{pācittiyas}}, and \textit{\textsanskrit{pāṭidesanīyas}}. This is in contrast to the \textsanskrit{Parivāra}, which mentions either five or seven classes, adding \textit{\textsanskrit{dukkaṭas}} in the first instance, and additionally \textit{thullaccayas} and \textit{\textsanskrit{dubbhāsitas}} in the second.\footnote{Respectively, offenses of wrong conduct, serious offenses, and offenses of wrong speech. } The \textit{thullaccayas} and \textit{\textsanskrit{dubbhāsitas}} are not mentioned in the \textsanskrit{Pātimokkha}, but what about the \textit{\textsanskrit{dukkaṭas}}? Are they not found in the \textit{sekhiyas}? To answer this, we first need to note that, unlike all the other classes of rules, the \textit{sekhiyas} do not include any offense in the rule formulation. Only in the \textsanskrit{Vibhaṅga} do we find a \textit{\textsanskrit{dukkaṭa}} offense for breaking these rules out of disrespect. If we assume that the \textsanskrit{Vibhaṅga} material was added some time after the formulation of the rules, for which there is significant evidence,\footnote{See, for instance, Oskar von Hinüber, 2000, §24–26; and K. R. Norman, 1983, pp. 19–21. } then in the earliest period the \textit{sekhiyas} were not strictly offenses, but rather general rules of training. It follows that the main rules of the \textsanskrit{Pātimokkha}, those that result in specific offenses and which are the core rules of the Vinaya \textsanskrit{Piṭaka} as a whole, are the four classes mentioned at \href{https://suttacentral.net/an4.244}{AN~4.244}.

This is perhaps not as surprising as it may at first seem, for it is rather curious that there are four different classes of offenses that are all cleared by simple confession. In addition to the \textit{\textsanskrit{pācittiyas}}, which as we have seen are among the earliest offenses, the \textit{thullaccayas}, \textit{\textsanskrit{dukkaṭas}}, and \textit{\textsanskrit{dubbhāsitas}} are all clearable in this way. It is not at all obvious why the Buddha would have laid down four different classes of offenses that are resolved in the same way. A solution to this conundrum may be that the Buddha never laid down the latter three classes at all, but that they emerged over time. The fact that a text as late as the \textsanskrit{Parivāra} speaks of five \textit{or} seven classes of offenses, as if there was a disagreement or evolution in the number, supports this contention. There is also some direct textual evidence to suggest that these offenses originally were general ways of describing wrong conduct and only later became classes of offenses. The word \textit{dukkata},\footnote{I disregard the spelling difference between \textit{\textsanskrit{dukkaṭa}}, normally used of the offense, and \textit{dukkata}, used when the meaning is bad conduct in general. } for instance, is used quite commonly in the general sense of “bad conduct”, without being referred to as an offense, an \textit{\textsanskrit{āpatti}}. An example of this, found in Kd 9, is the expression \textit{\textsanskrit{dukkaṭa} kamma}, which just means a badly done legal procedure. My suggestion, then, is that these three classes of offenses initially were just general ways of speaking of bad conduct. The \textit{thullaccayas} would have been “serious faults”, the \textit{dukkatas} “bad conduct”, and the \textit{\textsanskrit{dubbhāsitas}} “bad speech”. I conclude that the earliest Vinaya probably only had the four classes of offenses mentioned at \href{https://suttacentral.net/an4.244/en/sujato}{AN~4.244}.

I now wish to return to the unexpected inclusion of the \textit{\textsanskrit{adhikaraṇasamathas}} in the \textsanskrit{Pātimokkha}. As mentioned above, the evidence seems to suggest that they have belonged to the \textsanskrit{Pātimokkha} from the earliest period and thus that they are an integral part of it. This forces us to reconsider the traditional view that the \textsanskrit{Pātimokkha} is no more than a series of training rules. In fact, once we look at the \textsanskrit{Pātimokkha} with this in mind, we discover many other “anomalies” that point in the same direction. Here are a few of them.

\href{https://suttacentral.net/pli-tv-bu-vb-pj1/en/brahmali\#7.1.16.1}{Bu~Pj~1} includes the important stipulation that if one renounces the training in advance, one cannot commit this offense. This is not directly related to the rule at hand, but is rather a general principle of Monastic Law. At the end of the \textit{\textsanskrit{saṅghādisesa}} offenses, we find a section describing the procedure for clearing such offenses. This is another general principle that is not immediately related to the committing of these offenses. Next, we have the two \textit{aniyata} rules, which are principles for deciding the severity of an offense, not offenses as such. At \href{https://suttacentral.net/pli-tv-bu-vb-np10/en/brahmali\#1.3.7}{Bu~NP~10}, we find an extended procedure for how to appoint an attendant to receive and manage funds on behalf of a monastic, none of which relates directly to the committing of the offense in that rule. This is just a quick summary of some obvious cases, but the point is clear: the \textsanskrit{Pātimokkha} includes quite a bit of general Vinaya material. Why is this so?

I have made the point in the General Introduction that the word Vinaya did not refer to the Vinaya \textsanskrit{Piṭaka} in the earliest period for the simple reason that no such \textsanskrit{Piṭaka} existed. The word \textsanskrit{Pātimokkha}, by contrast, is encountered quite frequently in the earliest texts, in total over one hundred times in the four main \textsanskrit{Nikāyas}, including several mentions in the \textsanskrit{Pātimokkha} itself at \href{https://suttacentral.net/pli-tv-bu-vb-pc72/en/brahmali\#1.20.1}{Bu~Pc~72} and \href{https://suttacentral.net/pli-tv-bu-vb-pc73/en/brahmali\#1.15.1}{73}, and also in the introduction, the \textsanskrit{Pātimokkha}-\textsanskrit{nidāna}. As such, it is reasonable to think that the \textsanskrit{Pātimokkha} is older than the Vinaya \textsanskrit{Piṭaka}. Nevertheless, although there was no Vinaya \textsanskrit{Piṭaka} in the earliest period, the accumulating Vinaya material would have had to be somehow organized. Now, since we see quite a bit of Vinaya material in the \textsanskrit{Pātimokkha}, my suggestion is that all this material was included within the \textsanskrit{Pātimokkha}, making the \textsanskrit{Pātimokkha} a kind of proto Vinaya \textsanskrit{Piṭaka}, from which the full \textsanskrit{Piṭaka} later evolved. And so the \textsanskrit{Pātimokkha} may have started as a repository for rules, which would have been needed early on, and then developed to include all material that dealt with the regulations of monastic life.

This way of thinking about the \textsanskrit{Pātimokkha} helps us better understand several of its peculiarities. One of these is the fact that the \textit{\textsanskrit{adhikaraṇasamathas}} do not follow the usual structure of the \textsanskrit{Pātimokkha} rules. We are now in a position to explain why this might be so, for which see the discussion of these rules at the end of the introduction to volume 2. We may also be in a better position to understand why the Buddha used the unusual word \textsanskrit{Pātimokkha}. If the \textsanskrit{Pātimokkha} were no more than a set of training rules, we would probably have seen a different name, perhaps a word connected to the idea of \textit{\textsanskrit{sikkhāpada}}, “a training rule”. Instead, we have the coinage of a new term, which suggests that we are dealing with more than a set of training rules and instead with a new and unique literature. The word itself may well be derived from the idea of liberation, \textit{mokkha}, via the prefix \textit{\textsanskrit{paṭi}/pati}, which can be understood in a number of ways.\footnote{For a detailed discussion of the meaning of the word \textit{\textsanskrit{pātimokkha}}, see Bhikkhu Ñā\textsanskrit{ṇatusita}, “Analysis of the Bhikkhu Pātimokkha”, pp. 46–49. } It makes good sense that the Buddha would have used a word that relates to liberation to name the corpus of rules and regulations that govern the monastic life.

Before we take a more detailed look at the specific content of the Sutta-\textsanskrit{vibhaṅga}, we need to briefly discuss why the Buddha laid down the \textsanskrit{Pātimokkha} in the first place. At \href{https://suttacentral.net/pli-tv-bu-vb-pj1/en/brahmali\#5.11.32}{Bu~Pj~1:5.11.32}, and several other places in the Vinaya \textsanskrit{Piṭaka}, the Buddha gives a list of ten reasons for laying down a training rule. These reasons can be summarized as the well-being of the monastics, the increase of faith in the Dhamma, and the longevity of the Dhamma. It is interesting that the focus is on supporting the spiritual life and not on protecting people from others’ bad conduct. In other words, although many of the rules were laid down on moral grounds, their main purpose is to protect the potential perpetrator, not the potential victim, especially if the victim is a non-monastic. On reflection, this is to be expected since the Dhamma is a personal spiritual path and not the equivalent of secular law. And this is also why there is more emphasis in the Vinaya on protecting monastic victims as opposed to non-monastics. I will note some such instances as I look at the individual rules below.

Yet the rules of the \textsanskrit{Pātimokkha}, and indeed the whole Vinaya \textsanskrit{Piṭaka}, are more than moral injunctions. A common reason for the Buddha to lay down a new rule is the complaints of lay supporters. These complaints are often about monastics indulging in sensuality comparable to lay life. A significant number of rules were therefore laid down to curb such indulgence and to stop the acquisition of goods that were regarded as too luxurious.

\section*{The personalities of the Sutta-\textsanskrit{vibhaṅga}}

This might be the right place to say a few words about the various characters one meets in the Vinaya \textsanskrit{Piṭaka}. It is often assumed in Buddhist circles that ancient India was an especially good time and place to be reborn. Those who were able to meet the Buddha in person would surely only be able to do so because of a vast store of good \textit{kamma} from the past. Their spiritual faculties would have been highly developed and they would have attained awakening with relative ease. The present generation, by contrast, has no such store of good \textit{kamma} from the past, which may make it impossible to make real progress on the path. Our best strategy might be to accumulate the requisite \textit{kamma} and wait for the next Buddha to arise. And indeed, when one reads the Suttas, which tend to be full of highly inspiring monastics and lay people, one might well come away with such a one-sided view.

The Vinaya \textsanskrit{Piṭaka} quickly disabuses us of such a rose-tinted view of the past. When you see the mischief the people of ancient India got up to, you realize that humanity has always been more or less the same. Read in the right way, this is very encouraging for our present generation. It means that we are in all probability neither more nor less spiritually developed than past generations. If it was possible to reach awakening then, it will also be possible now.

An interesting aspect of the origin stories to the \textsanskrit{Pātimokkha} rules—for an explanation of which see the next section—is that certain monastics tend to feature again and again, often in relation to certain kinds of offenses. It is not entirely clear whether this is because some people ended up as caricatures who were employed in the origin stories to illustrate the commission of particular kinds of offenses or whether certain people just committed lots of offenses. The reality is probably a mix of both. An example of the former is the notorious group of six monks who become an empty vessel into which all sorts of wrong conduct was projected. No doubt these monks were historical figures, since they occur so frequently in all sorts of places, both prominent and obscure. Still, when we see that all the \textit{sekhiya} offenses bar three were first committed by this group, we suspect that these origin stories are artificial and the group of six are used in a stereotypical fashion to provide suitable perpetrators. This impression is reinforced by the fact that the vast majority of origin stories in the \textit{sekhiya} chapter are no more than bare bones, with virtually no narrative apart from saying that the group of six monks committed the offense in question. What we are seeing is a group of people used as a literary device, not as historically real, at least in some contexts.

There are other characters, however, who have a greater claim to being proper historical figures, not the least because their personalities are drawn in quite a bit of detail. One such person is the monk \textsanskrit{Udāyī}, who became notorious for his sexual shenanigans. He is the originator of four important \textit{\textsanskrit{saṅghādisesa}} rules, numbers 2–5, all dealing with serious sexual misconduct, and also plays an important role in \textit{\textsanskrit{saṅghādisesa}} 1, which concerns masturbation. As rules were laid down and his avenues for expressing his defilements were gradually cut off, he would venture further afield trying to find satisfaction for his urges through all sorts of dubious conduct. He would meet women in private, either speaking Dhamma or frivolously, as the circumstances would allow, see \href{https://suttacentral.net/pli-tv-bu-vb-ay1/en/brahmali\#1.1}{Ay~1} and \href{https://suttacentral.net/pli-tv-bu-vb-ay2/en/brahmali\#1.1}{Ay~2}; at \href{https://suttacentral.net/pli-tv-bu-vb-np4/en/brahmali\#1.1}{Bu~NP~4} and \href{https://suttacentral.net/pli-tv-bu-vb-pc30/en/brahmali\#1.1}{Bu~Pc~30} he meets up with his ex-wife to indulge in completely unrestrained behavior; at \href{https://suttacentral.net/pli-tv-bu-vb-np5/en/brahmali\#1.1.1}{Bu~NP~5} he asks a nun to give him her lower robe, presumably to see her naked; at \href{https://suttacentral.net/pli-tv-bu-vb-pc7/en/brahmali\#1.1}{Bu~Pc~7} he teaches Dhamma to women by whispering in their ears; at \href{https://suttacentral.net/pli-tv-bu-vb-pc26/en/brahmali\#1.1}{Bu~Pc~26} he sews a robe with an indecent design and gives it to a nun. It seems he was also cruel, as a certain \textsanskrit{Udāyī} killed crows almost as a hobby, thereby becoming the originator of the rule against killing animals at \href{https://suttacentral.net/pli-tv-bu-vb-pc61/en/brahmali\#1.1}{Bu~Pc~61}.

Another character who makes a frequent showing in the Vinaya \textsanskrit{Piṭaka} is Upananda the Sakyan. His main defilement was an excessive greed for robes and cloth. He is the originator of a significant number of \textit{nissaggiya \textsanskrit{pācittiya}} rules, specifically numbers \href{https://suttacentral.net/pli-tv-bu-vb-np6/en/brahmali\#1.1.1}{6}, \href{https://suttacentral.net/pli-tv-bu-vb-np8/en/brahmali\#1.1.1}{8}, \href{https://suttacentral.net/pli-tv-bu-vb-np9/en/brahmali\#1.1}{9}, \href{https://suttacentral.net/pli-tv-bu-vb-np10/en/brahmali\#1.1.1}{10}, \href{https://suttacentral.net/pli-tv-bu-vb-np18/en/brahmali\#1.1}{18}, \href{https://suttacentral.net/pli-tv-bu-vb-np20/en/brahmali\#1.1}{20}, \href{https://suttacentral.net/pli-tv-bu-vb-np25/en/brahmali\#1.1}{25}, and \href{https://suttacentral.net/pli-tv-bu-vb-np27/en/brahmali\#1.1}{27}. Except for Bu NP 18, all of these relate to his greed for robes and robe-cloth. In Bu NP 18 he receives money after himself suggesting that it be given. It is perhaps not too outlandish to speculate that he used the money on cloth! Yet Upananda, too, enjoyed the company of women. \href{https://suttacentral.net/pli-tv-bu-vb-pc42/en/brahmali\#1.1}{Bu~Pc~42}–45 were all laid down as a consequence of Upananda’s misbehavior in this area. Upananda was also the originator of \href{https://suttacentral.net/pli-tv-bu-vb-pc46/en/brahmali\#1.1}{Bu~Pc~46}, concerned with visiting private homes at the wrong time, and \href{https://suttacentral.net/pli-tv-bu-vb-pc58/en/brahmali\#1.1}{Bu~Pc~58} where his greed for robes once again is on display. Finally, he is the originator of \href{https://suttacentral.net/pli-tv-bu-vb-pc87/en/brahmali\#1.1}{Bu~Pc~87}, a rule that concerns the use of luxurious beds. In addition to this, he is indirectly involved in a number of other rules.

There is something realistic about these characters. Strong defilements are difficult to overcome and so it is not unreasonable to think that a single person could be behind such a large number of offenses. The world was much the same then as it is now. Where the Suttas show us the human potential, the Vinaya \textsanskrit{Piṭaka} shows us a darker side of humanity.

Now let us turn to a more detailed consideration of the content of the Sutta-\textsanskrit{vibhaṅga}.

\section*{The introduction to the Sutta-\textsanskrit{vibhaṅga}}

The Monks’ Analysis begins with a story of how the \textsanskrit{Pātimokkha} rules came to be laid down. The story begins at the village of \textsanskrit{Verañjā},\footnote{According to Ven. Shravasti Dhammika, private communication, \textsanskrit{Verañjā} can be identified with modern Ataranji Khera about 13 kilometers north of Etah in Uttar Pradesh. This about 90 kilometers northeast of Agra. } with a brahmin objecting to the Buddha’s revolutionary new teachings. The Buddha then describes his awakening experience, upon which the brahmin accepts his superior insight and becomes a disciple. The purpose of this preamble is presumably to establish the Buddha’s authority in laying down the rules for the monks and the nuns. The story then goes on to tell of two incidents where the Buddha is required to intervene in the lives of his disciples. Yet since the monks involved, Ānanda and \textsanskrit{Mahāmoggallāna}, were pure and easy to correct, gentle guidance from the Buddha was sufficient. One point of these stories is perhaps to show that the laying down of rules was inevitable. Another is no doubt to make it clear that the most eminent monastics, especially the fully awakened ones such as \textsanskrit{Mahāmoggallāna}, accept the Buddha’s guidance without argument.

We now come to the essence of this introductory section. \textsanskrit{Sāriputta} approaches the Buddha and asks him to lay down rules so that the Dhamma may last for a long time. The Buddha responds that he does not lay down rules until certain causes for corruption arise in the Sangha.

The Buddha’s response is interesting for several reasons. First, it shows that the Buddha was a pragmatist. Rules are laid down to counter existing problems, not potential ones. One reason for this is presumably that it is very difficult to tailor solutions to future problems. In fact, it is difficult enough to lay down rules that are appropriate for existing problems. At \href{https://suttacentral.net/pli-tv-bu-vb-pc32/en/brahmali\#1.15.0}{Bu~Pc~32}, for instance, we see that the Buddha has to amend the rule a full six times to make it workable. In total over 40 of the monks’ rules needed to be amended, including all the \textit{\textsanskrit{pārājika}} offenses. In the Khandhakas, we even find cases of the Buddha having to abolish rules because they prove impractical, such as the prohibition against eating mangoes (\href{https://suttacentral.net/pli-tv-kd15/en/brahmali\#5.1.18}{Kd~15:5.1.18}), which was subsequently overturned (\href{https://suttacentral.net/pli-tv-kd15/en/brahmali\#5.2.9}{Kd~15:5.2.9}). What we see, then, is that the Buddha’s approach to dealing with problems is to tailor fairly precise solutions. He does not try to anticipate the future.

Second, the Buddha’s response suggests that he was not omniscient, at least not in the broadest sense of the term. If he could fully foresee the future, the most straightforward and simple solution to avoid future problems would have been to lay down all the monastic rules at the outset of his teaching career. Moreover, he would not have to keep on amending certain rules. And so, the way the monastic rules were laid down is one of the strongest arguments against the traditional Buddhist view that the Buddha was omniscient.

The introduction comes to an end with the Buddha departing \textsanskrit{Verañjā} and walking to \textsanskrit{Vesālī}, a distance of over 900 kilometers following the modern network of roads, according to Google Maps. This sets the stage for the origin story to first offense entailing expulsion, which takes place near \textsanskrit{Vesālī}.

\section*{The Monks’ \textsanskrit{Pātimokkha} rules and their analysis}

Apart from this brief introduction, the bulk of the Monks’ Analysis is devoted to a technical discussion of the \textsanskrit{Pātimokkha} rules, starting with the most serious offenses. We then encounter the various classes of offenses in descending order of importance, ending with a number of rules that are not offenses at all, but procedural in nature. We will discuss each class in turn, and note some of the interesting features and details of each one. Before getting into too many details, however, we will have a look at how the analysis of each rule is structured. There will be some overlap in content with the General Introduction found above.

At the center of the Monks’ Analysis is the \textsanskrit{Pātimokkha}, the Monastic Code, consisting of the most important rules that every \textit{bhikkhu}, or monk, is supposed to keep by virtue of being a monastic. These rules are the oldest part of the Sutta-\textsanskrit{vibhaṅga}, most of them probably originating with the Buddha himself, as I have argued in more detail above. The \textsanskrit{Pātimokkha} rules are by far the most important content of the Sutta-\textsanskrit{vibhaṅga}. As such, they are recited every fortnight in most monastic communities, a tradition that itself stems from the earliest period of Buddhism.

Around these \textsanskrit{Pātimokkha} rules, there arose, over time, a copious explanatory framework, which forms the bulk of the Sutta-\textsanskrit{vibhaṅga}. This framework is divided into several parts, starting with an origin story that describes the circumstances in which each rule was laid down. A rule is then formulated. Often there are further developments, described in subsequent origin stories, that necessitate additions to the rule, either expanding them or setting limitations. This then culminates in the formulation of a final rule, which is the rule as we find it in the \textsanskrit{Pātimokkha}.

Next comes the word analysis, known as the \textit{\textsanskrit{padabhājanīya}}, which defines significant words or phrases in the rule. Sometimes this is no more than the giving of a series of synonyms, but at other times it involves lengthy sections with detailed explanations, for instance, the explanation of disrobal, \textit{\textsanskrit{sikkhaṁ} \textsanskrit{paccakkhāya}}, at \href{https://suttacentral.net/pli-tv-bu-vb-pj1/en/brahmali\#8.2.1}{Bu~Pj~1:8.2.1}–8.4.21.

The word analysis is usually followed by a “permutation series”, which sets out various combinations of factors for which the offense in question is either fulfilled, partially fulfilled, or not fulfilled. A partially fulfilled offense will often result in the incurring of a lesser offense. For the serious offenses, that is, the \textit{\textsanskrit{pārājikas}} and \textit{\textsanskrit{saṅghādisesas}}, these permutation series are particularly long, often amounting to more than half of a rule’s total word count. Sometimes the permutation series adds important details to our understanding of a rule, yet too often it seems to be little more than an exercise in the mechanical listing of all possible combinations of factors that may give rise to a specific offense, with little gained in terms of understanding. These series sometimes employ a specialized terminology, found at the end of sections, to help the reader keep track of how the series evolves. I have provided a separate addendum to explain this terminology, see Appendix II: Specialized Vocabulary.

Then comes the non-offense clause, which lists a number of circumstances in which there is no offense. There are certain universal non-offenses, namely, insanity, being possessed or deranged, being overwhelmed by pain, or being the first offender. This last class refers to the person who gave rise to the rule. In other words, the rules do not apply retroactively.

The final part of the analysis of each rule is a series of case studies where the Buddha adjudicates a specific action and declares whether the monastic has committed an offense or not. These case studies only exist for a small subset of rules, specifically the four \textit{\textsanskrit{pārājikas}} and the first five \textit{\textsanskrit{saṅghādisesas}}.

\subsection*{The \textit{\textsanskrit{pārājikas}} (Pj)}

We are now ready to look more closely at each class of offenses and some of the individual rules within them. We start with the most serious offenses, the \textit{\textsanskrit{pārājikas}}. Any monastic who commits such an offense is by default expelled from the monastic community, whether anyone else knows of the offense or not. They are barred from being a fully ordained monk or nun, a \textit{bhikkhu} or \textit{\textsanskrit{bhikkhunī}}, for the rest of their life. There are four such offenses for the monks. In what follows, whenever a rule is held in common between the monks and the nuns, I speak of “a monastic” rather than “a monk”.

The gravity of a \textit{\textsanskrit{pārājika}} offense is expressed through a simile found at \href{https://suttacentral.net/an4.244/en/sujato\#1.3}{AN~4.244}, which compares committing the offense to having one’s head chopped off. The loss of one’s head is obviously related to the irreversibility of one’s exclusion from the Sangha. The symbolism is a powerful testimony to the importance placed on monastic life in the early \textit{suttas}.

The first \textit{\textsanskrit{pārājika}} offense concerns sexual intercourse. As always, the text starts with an origin story, which in the present case gives an account of the rather tragic events in the life of the young man Sudinna. He starts out being extremely inspired by the Buddha and his teaching, but then succumbs to the demands from his family of producing an heir to the family fortune. It is thus that he ends up having sexual intercourse with his ex-wife, resulting in profound remorse. Both his ex-wife and son end up ordaining, eventually becoming \textit{arahants}, whereas Sudinna fades away into obscurity.

This story is remarkable in that it is almost identical to the story of \textsanskrit{Raṭṭhapāla} in \href{https://suttacentral.net/mn82/en/sujato}{MN~82}. The main difference between the two stories is that \textsanskrit{Raṭṭhapāla} ends up as an \textit{arahant}, thus fulfilling his great potential. Ven. \textsanskrit{Anālayo} argues convincingly that the story at MN 82 is the original from which the current story was taken and adapted.\footnote{\textsanskrit{Anālayo}, 2012, p. 407. } This points to an important principle of interpretation. Although the \textsanskrit{Pātimokkha} rules go back to the earliest period of Buddhism, this is not necessarily true of the \textsanskrit{Vibhaṅga} material. In other words, the \textsanskrit{Vibhaṅga} may not, or may not always, go back to the Buddha himself. This view is supported by a scan of the recent translation by Ven. \textsanskrit{Vimalañāṇī} of the \textsanskrit{Mahāsāṅghika} version of the \textsanskrit{Bhikkhunī}-\textsanskrit{vibhaṅga}.\footnote{\textsanskrit{Vimalañāṇī} \textsanskrit{Bhikkhunī}, 2024. } The origin stories are for the most part different from those found in the Pali. For a further discussion of this, see the separate introduction to the \textsanskrit{Bhikkhunī}-\textsanskrit{vibhaṅga} in volume 3.

The \textsanskrit{Vibhaṅga} continues with two further origin stories, upon which the final rule is laid down. It is through the last of these origin stories that the final version of the rule comes to include the important stipulation that if one renounces the training in advance, one cannot commit this offense.

We then come to the word analysis, which includes a long section on the meaning of renouncing the training, followed by the permutation series. This series focuses on what kind of partner and orifice fulfill the offense. At the end of this long permutation series comes a second and much shorter series. This series is not directly connected to the previous one. Moreover, it uses a non-standard vocabulary, especially the pair \textit{magga} and \textit{amagga} (literally, “path” and “non-path”, but see Appendix I: Technical Terms for an explanation), as well as the verbs \textit{\textsanskrit{vippaṭipajjati}} (“to rape”) and \textit{\textsanskrit{nāseti}} (“to expel”), none of which is otherwise used in the context of the \textit{\textsanskrit{pārājikas}}. This leads to the question of what the relationship is between these two apparently unconnected permutation series. On the assumption that greater detail and a standardized vocabulary are signs of later development, it seems likely that the second part of the permutation series is the earlier part, with the first part added at a later time. We see similar developments in a number of other rules, including \textit{\textsanskrit{pārājika}} 2 and \textit{bhikkhu \textsanskrit{saṅghādisesas}} 1, 2, 5, 6, and 8.

At the end of the two permutation series, we find the non-offense clause, which in addition to the standard non-offenses adds “not knowing” and “not consenting”. Then come the case studies, headed by a series of mnemonic verses. A number of these cases are interesting. To begin with, one of the cases concerns a nun, specifically the rape of the nun \textsanskrit{Uppalavaṇṇā}. Because the nuns do not have a separate \textsanskrit{Vibhaṅga} for the rules they have in common with the monks, it is natural for such cases to be included here. In fact, there are further such cases below (\href{https://suttacentral.net/pli-tv-bu-vb-pj1/en/brahmali\#10.6.6}{Bu~Pj~1:10.6.6}), as well as in the next two \textit{\textsanskrit{pārājika}} offenses (\href{https://suttacentral.net/pli-tv-bu-vb-pj2/en/brahmali\#7.6.20}{Bu~Pj~2:7.6.20}, \href{https://suttacentral.net/pli-tv-bu-vb-pj2/en/brahmali\#7.45.1}{Bu~Pj~2:7.45.1}, \href{https://suttacentral.net/pli-tv-bu-vb-pj2/en/brahmali\#7.45.12}{Bu~Pj~2:7.45.12}, and \href{https://suttacentral.net/pli-tv-bu-vb-pj3/en/brahmali\#5.33.10}{Bu~Pj~3:5.33.10}).

Another interesting case is one where a monastic has a gender change, apparently for natural reasons. According to the story, the Buddha says that the monastic concerned should simply join the Sangha of the opposite gender. This leads to the unexpected situation that we have a precedent in the Vinaya \textsanskrit{Piṭaka} for how to include transgender people within the Sangha. Moreover, this episode suggests that the Buddha tried to find solutions to give everyone a chance to become or remain as a monk or a nun, regardless of their identity. It seems reasonable to infer from this that we should look for ways to make the Sangha as inclusive as possible.

The second \textit{\textsanskrit{pārājika}} starts with the story of the monk Dhaniya who tricks the keeper of King \textsanskrit{Bimbisāra}’s woodyard to give him wood. The Buddha then lays down a \textit{\textsanskrit{pārājika}} for stealing. Observers have pointed out that there is a mismatch here between the action that led to the rule and the actual rule. Dhaniya did not, strictly speaking, steal the wood. Rather, it was given to him after he deceived the caretaker of the woodyard. It follows that Dhaniya’s actions did not amount to a \textit{\textsanskrit{pārājika}} offense under this rule. There are similar discrepancies between the origin stories and the rule also in other cases, for instance at \href{https://suttacentral.net/pli-tv-bu-vb-np4/en/brahmali\#1.1}{Bu~NP~4}. The interesting question then arises of why there are such discrepancies.

It has been suggested by some, such as Oskar von Hinüber,\footnote{Oskar von Hinüber, 2000, §23. } that this mismatch between rule and origin story must be the result of the origin stories having been added later, after the original events were forgotten. This, however, does not seem very plausible, for it implies an unaccountable incompetency on the part of the early Sangha. If the origin stories were fabricated, it is much more likely that they would match the rule very closely. Indeed, this is exactly what we find in the case of the \textit{sekhiyas}. In these rules, in which the origin stories in most cases are no more than short stereotypical formulas, and which involve the group of six monks in 72 out of 75 rules, it seems indisputable that they have been fabricated. Moreover, in most of these 72 cases the origin story is either just a verbatim repetition of the misconduct described in the rule or a short phrase closely related to it. In other words, we have strong evidence to suggest that artificial origin stories tend to be a perfect match for the rule they belong with.

What, then, might be the reason for the observed discrepancies? I have already made the case that the \textsanskrit{Vibhaṅga} material is later than the actual rules. If this is so, then the origin stories would have faded in memory by the time it was decided they should be included as part of the textual heritage. In some cases, such as \textit{bhikkhu \textsanskrit{pārājika}} 1 and 3,\footnote{For \textit{\textsanskrit{pārājika}} 1, see \textsanskrit{Anālayo}, 2012. For \textit{\textsanskrit{pārājika}} 3, see \textsanskrit{Anālayo}, 2014. } this fading would have resulted in details being recalled incorrectly. In other cases, especially for the nuns’ rules,\footnote{This information comes from \textsanskrit{Bhikkhunī} \textsanskrit{Vimalañāṇī} (private communication). } the Sangha would have been unsure of which origin story was historically correct and would probably have used a suitable anecdote to fill the gap. This, then, would have led to a mismatch between origin story and rule. Indeed, it stands as a testimony to the fidelity of the Sangha to the received tradition that they did \emph{not} alter these stories even though they must have been aware of these discrepancies. We should see the mismatch as a sign of the conservatism of the reciters. They preserved the texts as handed down and only made corrections if they had very good reasons to suspect textual corruption.

There might, however, be another cause for the perceived mismatch. I believe there is no good reason why the origin stories should always be a perfect lead-up to the subsequent laying down of a rule. Upon seeing a problem in the Sangha, the Buddha would have laid down whatever rule seemed appropriate to deal with the situation. In \textit{\textsanskrit{pārājika}} 2, although Dhaniya did not technically steal, his actions were tantamount to stealing. The laying down of the rule can be explained without the need for a perfectly fitting origin story.

In the last section of this rule, we again find several interesting case studies. One story concerns a monk who takes a rag from a fresh corpse. The monk must have gotten a shock when the corpse tells him not to steal the cloth. Still, the monk pays no heed, upon which the corpse gets up and follows behind him, before collapsing outside the monk’s hut. This is presumably the first zombie story in the history of literature, with Hollywood being a Johnny-come-lately to the genre and perhaps ultimately inspired by the Vinaya \textsanskrit{Piṭaka} itself! As so often, Buddhism was there first. In any case, the Buddha laid down a rule that a monastic should not take cloth from a fresh corpse. Who said the Vinaya was not entertaining?

The last case study at \href{https://suttacentral.net/pli-tv-bu-vb-pj2/en/brahmali\#7.49.1}{Bu~Pj~2} relates the brief story of a monk who tells his teacher that he has committed a \textit{\textsanskrit{pārājika}} by stealing a turban. The teacher, however, is not content with just accepting his student’s word. He asks the student to bring the turban, and then has it valued. They discover that the value is below the threshold for committing a \textit{\textsanskrit{pārājika}}. And so, it turns out that the student has not committed a \textit{\textsanskrit{pārājika}} after all. This goes to show that it is the duty of a teacher, and presumably any fellow monastic, to go to some length to help a co-monastic get out of trouble, if at all possible.

\textit{\textsanskrit{Pārājika}} 3, which concerns the killing of a human being, begins with the extraordinary story of a large number of monks seeking to die after hearing a teaching from the Buddha on the contemplation of the impurity of the body. When the Buddha emerges from solitary retreat after two weeks, the Sangha is much diminished. The story is found in much the same form in all extant Vinayas, which suggests it is likely to reflect a real historical event. Moreover, because this story puts Buddhism and even the Buddha in a bad light, it is unlikely to have been inserted as a fictional addition by later generations.

Apart from concluding that the story may well be true, it is hard to know what to make of it. Ven. \textsanskrit{Anālayo} gives it a valiant try and concludes that the story “reflects the influence of a prevalent negative attitude towards the body and the tolerance of suicide in ancient Indian ascetic circles”.\footnote{\textsanskrit{Anālayo}, 2014, p. 42. } In any case, this motivates the Buddha to teach mindfulness of breathing, with the standard set of sixteen steps set out at this point. This is by no means the only inclusion of such typical \textit{sutta} material in the Vinaya \textsanskrit{Piṭaka}. We find a significant number of such instances throughout the collection, especially in the Khandhakas, including three full \textit{suttas} at the start of Kd 1. The Suttas and the Vinaya are complementary aspects of the Dhamma that are interwoven to form a complete picture of the Buddha’s teachings.

Coming to the case studies, the very first one tells the story of monks who, out of compassion, spoke to a sick monk in favor of dying. Despite the wholesome motivation, the Buddha says that they have nevertheless committed \textit{\textsanskrit{pārājika}} offenses. Further on, there are cases that treat abortion in a similar fashion. This means that a monastic may never suggest dying for any reason whatsoever, no matter how positively motivated. It is important to realize, however, that this is not the final word on contentious moral issues such as euthanasia or abortion. Monastic Law and \textit{kamma} are two different things. The Monastic Law was laid down for a number of reasons, whereas \textit{kamma} is governed strictly by intention. This is as true in the area of euthanasia or abortion as it is for any other.

\textit{\textsanskrit{Pārājika}} 4 concerns falsely claiming supernormal powers. It may at first seem surprising that this should be regarded as so serious. On reflection, however, one realizes that such claims subvert the very purpose of the spiritual life for selfish purposes. In the origin story, a group of monks claimed such powers so as to get fed by the lay supporters during a famine. The Buddha states that this is the worst possible kind of theft.

The second half of the case studies involves a series of supernormal claims that may seem like boasts, but that turn out to be true. In each case the Buddha steps in to correct the doubters. The point of this series is presumably to make it clear that all sorts of psychic experiences are possible, and that one should be careful with jumping to the conclusion that such claims are false.

The four \textit{\textsanskrit{pārājikas}} end with a short passage stating that anyone who commits any of them is no longer a monastic. However, there is no prohibition in the Vinaya for a such a person to carry on as a novice monk or nun.

\subsection*{The \textit{\textsanskrit{saṅghādisesas}} (Ss)}

After the \textit{\textsanskrit{pārājikas}}, the second most important class of rules are known as the \textit{\textsanskrit{saṅghādisesas}}, “the offenses entailing suspension”. Together with the \textit{\textsanskrit{pārājikas}}, they are known as “the heavy offenses”, \textit{\textsanskrit{garukāpatti}}. The rest of the \textsanskrit{Pātimokkha} offenses are light, \textit{\textsanskrit{lahukāpatti}}.

A monk who commits a \textit{\textsanskrit{saṅghādisesa}} offense must undergo a trial period of six days. If he hides his offense, he must additionally undergo a period of probation equal in length to the number of days he hid his offense. If he behaves properly during this period, that is, according to the rules laid down in \href{https://suttacentral.net/pli-tv-kd12/en/brahmali\#1.1.1}{Kd~12}, he is to be rehabilitated by a \textit{sangha} of at least 20 monks. The severity of a \textit{\textsanskrit{saṅghādisesa}} offense is conveyed through a simile found at \href{https://suttacentral.net/an4.244/en/sujato\#2.1}{AN~4.244}, which compares the offense to being beaten in public with a club. There are thirteen such offenses for the monks.

The first four \textit{\textsanskrit{saṅghādisesa}} offenses are sexual in nature. In terms of how often they are committed, the first three of these—masturbation, groping, and indecent speech—are by far the most important, which is presumably why they are listed first. It is striking that sexual expression is considered to be such a serious fault, especially since sexuality is often celebrated in lay life. Part of this is about the sexual harassment aspect of such behavior. But more broadly, it reflects the danger that sexuality poses for progress on the spiritual path and the ease with which one may get trapped in sensuality. These rules, then, create a barrier to getting stuck in the sensory realm.

\href{https://suttacentral.net/pli-tv-bu-vb-ss5/en/brahmali\#2.2.13.1}{Bu~Ss~5} prohibits matchmaking. It is probably this rule that is at the root of Buddhist monastics not officiating at marriage ceremonies. \href{https://suttacentral.net/pli-tv-bu-vb-ss6/en/brahmali\#1.6.6.1}{Bu~Ss~6} and \href{https://suttacentral.net/pli-tv-bu-vb-ss7/en/brahmali\#1.19.1}{Bu~Ss~7} concern construction, specifically not putting up an oversize hut or building in the wrong location. These rules are in part about monastics not burdening their lay supporters unreasonably. The origin story to \href{https://suttacentral.net/pli-tv-bu-vb-ss6/en/brahmali\#1.3.1}{Bu~Ss~6} includes a \textsanskrit{Jātaka} tale known as the \textsanskrit{Maṇikaṇṭha}-\textsanskrit{jātaka}, number 253 of that collection. There are further \textsanskrit{Jātaka} tales in the Sutta-\textsanskrit{vibhaṅga} and the Khandhakas, which is one among many indicators that this part of the Vinaya is later than the earliest parts of the Sutta \textsanskrit{Piṭaka}.\footnote{The \textsanskrit{Jātaka} tales are considered commentaries on the \textsanskrit{Jātaka} verses, which are regarded as Canonical. } This rule also incorporates a story that seems to appear nowhere else in the Pali corpus, as well as a reference to \textsanskrit{Raṭṭhapāla} and his father, the two main protagonists of \href{https://suttacentral.net/mn82/en/sujato}{MN~82}.

\href{https://suttacentral.net/pli-tv-bu-vb-ss8/en/brahmali\#1.9.32.1}{Bu~Ss~8} and \href{https://suttacentral.net/pli-tv-bu-vb-ss9/en/brahmali\#1.2.14.1}{Bu~Ss~9} are to do with falsely accusing a fellow monk of having committed a \textit{\textsanskrit{pārājika}} offense. Groundlessly accusing a monk of a lesser offense is an offense at \href{https://suttacentral.net/pli-tv-bu-vb-pc76/en/brahmali\#1.11.1}{Bu~Pc~76}. \href{https://suttacentral.net/pli-tv-bu-vb-ss8/en/brahmali\#1.1.1}{Bu~Ss~8} starts with the entertaining story of Dabba the Mallian who is said to have become an \textit{arahant} at the age of seven and then spent his life in service to the Sangha. One of his jobs was to assign dwellings to newly-arrived monks. When monks arrived late at night, he would enter the fire element, make his finger glow, and then take the monks to their huts using his finger as a flashlight!

There is also a darker side to the story of Dabba the Mallian. At some point, the two bad monks Mettiya and \textsanskrit{Bhūmajaka}, who were members of the notorious group of six monks, developed a grudge against Dabba. They then had the nun \textsanskrit{Mettiyā} go to the Buddha and falsely accuse Dabba of raping her. The Buddha asks Dabba whether this is true, which he denies, and the Buddha then exonerates him on the basis of his perfect memory. This is how “resolution through recollection”, \textit{sativinaya}, becomes established as the second of the seven “principles for settling legal issues”, the \textit{\textsanskrit{adhikaraṇasamathadhammas}} (\href{https://suttacentral.net/pli-tv-kd14/en/brahmali\#4.10.10}{Kd~14:4.10.10}). The more troubling aspect of this incident is that the Buddha then expels the nun \textsanskrit{Mettiyā} despite the fact that her conduct was at most grounds for a \textit{\textsanskrit{saṅghādisesa}} offense. A solution to this unexpected expulsion, according to Ven. \textsanskrit{Anālayo}, may be found in the \textsanskrit{Mahāsāṅghika} Vinaya, according to which \textsanskrit{Mettiyā}, or whatever she is called by the \textsanskrit{Mahāsāṅghikas}, was already pregnant by someone else.\footnote{\textsanskrit{Anālayo}, 2012, pp. 425–426. }

\href{https://suttacentral.net/pli-tv-bu-vb-ss10/en/brahmali\#1.3.16.1}{Bu~Ss~10}–13 apply only to a very specific and narrow set of circumstances, and as such are unlikely ever to be committed. \href{https://suttacentral.net/pli-tv-bu-vb-ss10/en/brahmali\#1.1.1}{Bu~Ss~10} and \href{https://suttacentral.net/pli-tv-bu-vb-ss11/en/brahmali\#1.1}{Bu~Ss~11} concern the case of Devadatta and his friends trying, and eventually succeeding, in creating a schism in the Sangha. The full story is found at \href{https://suttacentral.net/pli-tv-kd17/en/brahmali\#2.1.1}{Kd~17:2.1.1}–4.5.15. \href{https://suttacentral.net/pli-tv-bu-vb-ss12/en/brahmali\#1.1}{Bu~Ss~12} is about the recalcitrant monk Channa who refuses to be admonished by anyone. Later on, he is ejected for not recognizing and making amends for his offenses (\href{https://suttacentral.net/pli-tv-kd11/en/brahmali\#25.1.1}{Kd~11:25.1.1}–31.1.218). Eventually, on the Buddha’s instruction, the Sangha imposes the “supreme penalty”, the \textit{\textsanskrit{brahmadaṇḍa}}, on him. He then sees the errors of his ways and even becomes an \textit{arahant} (\href{https://suttacentral.net/pli-tv-kd21/en/brahmali\#1.15.1}{Kd~21:1.15.1}–1.15.16). \href{https://suttacentral.net/pli-tv-bu-vb-ss13/en/brahmali\#1.1.1}{Bu~Ss~13} once again concerns the group of six monks. In this case they corrupt the lay supporters at a place called \textsanskrit{Kīṭāgiri} and draw them away from the true Dhamma. The story is developed further in connection with the legal procedure of banishment (\href{https://suttacentral.net/pli-tv-kd11/en/brahmali\#13.1.1}{Kd~11:13.1.1}–17.2.18).

The last four \textit{\textsanskrit{saṅghādisesa}} offenses are special in that they require the Sangha to perform a legal procedure that in effect admonishes the offending monk and gives him the opportunity to mend his ways. If he refuses, then the offense is committed once the legal procedure has been completed.

The \textit{\textsanskrit{saṅghādisesa}} chapter ends with the procedure for clearing such offenses, which, as noted above, is one among a number of similar passages that make it clear that the \textsanskrit{Pātimokkha} is more than a simple set of training rules. The procedure for clearing offenses is expanded on in \href{https://suttacentral.net/pli-tv-kd12/en/brahmali}{Kd~12} and \href{https://suttacentral.net/pli-tv-kd13/en/brahmali}{Kd~13}, which give a great amount of detail as to how it is to be implemented.

\subsection*{The \textit{aniyatas} (Ay)}

\textit{Aniyata} means indeterminate. There are only two such rules, \href{https://suttacentral.net/pli-tv-bu-vb-ay1/en/brahmali\#1.32.1}{Ay~1} and \href{https://suttacentral.net/pli-tv-bu-vb-ay2/en/brahmali\#1.19.1}{Ay~2}. They are called indeterminate because they are not a class of offenses, but rather procedures for deciding what offense has been committed. Once this is decided, one makes amends according to the relevant class of offense, either \textit{\textsanskrit{pārājika}}, \textit{\textsanskrit{saṅghādisesa}}, or \textit{\textsanskrit{pācittiya}}.

The remaining rules for monks are discussed in the introduction to volume 2.

%
\chapter*{Bibliography}
\addcontentsline{toc}{chapter}{Bibliography}
\markboth{Bibliography}{Bibliography}

\begin{enumerate}%
\item \textsanskrit{Anālayo}, Bhikkhu. 2010. “The Influence of Commentarial Exegesis on the Transmission of Āgama Literature” in Translating Buddhist Chinese: Problems and Prospects, Harrassowitz, Wiesbaden.%
\item Analayo. 2012. “The Case of Sudinna: On the Function of Vinaya Narrative, Based on a Comparative Study of the Background Narration to the First Parajika Rule”, Journal of Buddhist Ethics, Volume 19.%
\item \textsanskrit{Anālayo}. 2014. “The Mass Suicide of Monks in Discourse and \textit{Vinaya} Literature”, Journal of the Oxford Centre for Buddhist Studies, Volume 7.%
\item Analayo. 2015. “Levitation in Early Buddhist Discourse”, Journal of the Oxford Centre for Buddhist Studies, pp. 9–42.%
\item Analayo. 2016. “Levitation in Early Buddhist Discourse”, Journal of the Oxford Centre for Buddhist Studies, pp. 11–26.%
\item Bodhi, Bhikkhu. 1982. “The first \textsanskrit{saṅghādisesa} rule for bhikkhus”, The Open Buddhist University, https://www.suttas.net/english/vinaya/bhikkhu-bodhi/the-first-sanghadisesa-rule-for-bhikkhus--by-ven.bodhi.pdf, accessed 10 February 2024.%
\item Bodhi, Bhikkhu, trans. 2012. “The Numerical Discourses of the Buddha”, Wisdom Publications, U.S.%
\item \textsanskrit{Brahmavaṁso}, Ajahn. 1982. “Vinaya notes, volume 1, \textsanskrit{Pārājika} to Nissaggiya \textsanskrit{pācittiya} 30”, The Open Buddhist University, https://bswa.org/bswp/wp-content/uploads/2019/01/Ajahn\_Brahmavamso\_Vinaya\_Notes.pdf, accessed 10 February 2024.%
\item Clarke, Shayne. 2015. “Vinayas”, in Brill’s Encyclopaedia of Buddhism, Leiden, vol. I, pp. 60–87.%
\item Cone, Margaret. 2001–2020. “A Dictionary of \textsanskrit{Pāli}”, part I-III. Oxford: Pali Text Society.%
\item Dhammika, Bhante S. 2015. “Nature and the Environment in Early Buddhism”, Journal of the Oxford Centre for Buddhist Studies, Volume 6.%
\item Erdosy, George. 1988. “Urbanisation in Early Historic India”. Oxford: BAR International Series 430.%
\item Frauwallner, Erich. 1956. “The Earliest Vinaya and the Beginnings of Buddhist Literature”, Rome: Istituto Italiano per il Medio ed Estremo Oriente.%
\item Gombrich, Richard. 1997. “How Buddhism Began: The Conditioned Genesis of the Early Teachings”, Munshiram Manoharlal Publishers.%
\item Gombrich, Richard. 2014. Addendum in “The Mass Suicide of Monks in Discourse and Vinaya Literature”, Journal of the Oxford Centre for Buddhist Studies, volume 7.%
\item Härtel, Herbert. 1995. “Archaeological Research on Ancient Buddhist Sites”. In: When Did the Buddha Live?: The Controversy on the Dating of the Historical Buddha. New Delhi: Sri Satguru.%
\item Hinüber, Oskar von. 1991. “The Oldest \textsanskrit{Pāli} Manuscript”, Akademie der Wissenschaften und der Literatur, Stuttgart: F. Steiner.%
\item Hinüber, Oskar von. 2000. “A Handbook of \textsanskrit{Pāli} Literature”, de Gruyter, Berlin.%
\item Hinüber, Oskar von. 2006. “Hoary Past and Hazy Memory”, in Journal of the International Association of Buddhist Studies, Volume 29, Number 2, pp. 193–201.%
\item Hinüber, Oskar von, and \textsanskrit{Anālayo}, Bhikkhu. 2016. “The Robes of a \textsanskrit{Bhikkhunī}”, Journal of Buddhist Studies, vol. XIII, pp. 79–90.%
\item Horner, I. B., trans. 1938–66. “The Book of the Discipline” (Vinaya \textsanskrit{Piṭaka}), 6 vols. London: Pali Text Society.%
\item Kabilsingh, C. 1998. “The \textsanskrit{Bhikkhunī} \textsanskrit{Pātimokkha} of the Six Schools”, Sri Satguru Publications.%
\item Kabilsingh, C. 1984. “A Comparative Study of \textsanskrit{Bhikkhunī} \textsanskrit{Pāṭimokkha}”, Chaukhambha Orientalia.%
\item Kieffer-Pülz, Petra. 2014. “\textsanskrit{Pārājika} 1 and \textsanskrit{Saṅghādisesa} 1: Hitherto Untranslated Passages from the \textsanskrit{Vinayapiṭaka} of the \textsanskrit{Theravādins}”, in: The Book of the Discipline I, pp. 349–373.%
\item Levman, Bryan. 2008–2009. “\textsanskrit{Sakāya} \textsanskrit{niruttiyā} revisited”, BEI 26–27, pp. 33–51.%
\item Liyanaratne, J. 1994. “South Asian Flora as reflected in the twelfth-century Pali lexicon \textsanskrit{Abhidhānapadīpikā}”, in the Journal of the Pali Text Society, Vol. XX, pp. 43–161.%
\item Malalasekera, D. P. 1937. “Dictionary of \textsanskrit{Pāli} Proper Names”. London: John Murray.%
\item Monier-Williams, M. 1899. “Sanskrit-English Dictionary”. Reprint, Delhi: Motilal Banarsidass, 2005.%
\item \textsanskrit{Ñāṇamoli}, Bhikkhu, and Bodhi, Bhikkhu, trans. 1995. “The Middle Length Discourses of the Buddha”, Wisdom Publications, U.S.%
\item Nyanatusita, Bhikkhu, ed. 2010. \textit{\textsanskrit{Bhikkhunī} \textsanskrit{Pātimokkha} \textsanskrit{Pāḷi}}. Buddhist Publication Society, Kandy.%
\item \textsanskrit{Ñāṇatusita}, Bhikkhu. 2014. “Analysis of the Bhikkhu Patimokkha”, Kandy: Buddhist Publication Society.%
\item \textsanskrit{Ñāṇatusita}, Bhikkhu. 2014. “Pali Manuscripts of Sri Lanka”. In: “From Birch Bark to Digital Data: Recent Advances in Buddhist Manuscript Research, Papers Presented at the Conference Indic Buddhist Manuscripts: The State of the Field, Stanford, June 15–19 2009”, Paul Harrison and Jens-Uwe Hartmann, eds., Vienna, pp. 351–387.%
\item Norman, K. R. 1983. “\textsanskrit{Pāli} Literature”, Otto Harrassowitz, Wiesbaden.%
\item Norman, K. R. 1992. “\textsanskrit{Pāli} Lexicographical Studies IX”, In Journal of the Pali Text Society XVI, 1992, pp. 77–85.%
\item Norman, K. R. 2006. “A Philological Approach to Buddhism”, PaIi Text Society.%
\item Ohnuma Reiko. 2013. “Bad Nun: \textsanskrit{Thullanandā} in \textsanskrit{Pāli} Canonical and Commentarial Sources”. Journal of Buddhist Ethics, volume 20, pp. 18–66.%
\item Olivelle, Patrick, trans. 2013. “King, Governance, and Law in Ancient India: \textsanskrit{Kauṭilya}’s \textsanskrit{Arthaśāstra}”, Oxford University Press Inc.%
\item Pachow, W. 1955. “A Comparative Study of the \textsanskrit{Prātimokṣa}”, Motilal Banarsidass, Delhi, 2000.%
\item \textsanskrit{Pāḷi} \textsanskrit{Tipiṭaka}. Https://www.tipitaka.org.%
\item Pandanus Database of Plants, “\textsanskrit{Ativiṣā}”. 1998–2009. Http://iu.ff.cuni.cz/pandanus/database/details.php?id=1235, accessed 15 October 2024.%
\item Rhys Davids, T. W. 1877. “On the Ancient Coins and Measures of Ceylon: with a discussion of the Ceylon date of the Buddha’s death”, Trubner \& Co, London. Available at https://archive.org/details/onancientcoinsme00davirich, accessed 10 February 2024.%
\item Rhys Davids, T. W. and Herman Oldenberg. 1881. “Vinaya texts”, part I, in Sacred Books of the East series. The Clarendon Press: Oxford. Available at https://archive.org/details/vinayatexts01davi, accessed 22 November 2024.%
\item Rhys Davids, T. W., and William Stede. 1921–25. “\textsanskrit{Pāli}-English Dictionary”. Reprint, Pali Text Society, Oxford, 1999.%
\item Roser, Max and Appel, Cameron and Ritchie, Hannah. 2021. “Human Height”, Published online at OurWorldinData.org. Retrieved from https://ourworldindata.org/human-height, accessed 10 February 2024.%
\item Sujato, Bhikkhu. 2012a. “Bhikkhuni Vinaya Studies”, Santipada, http://santifm.org/santipada/wp-content/uploads/2012/08/Bhikkhuni\_Vinaya\_Studies\_Bhikkhu\_Sujato.pdf, accessed 31 October 2024.%
\item Sujato, Bhikkhu. 2012b. “Sects and Sectarianism”, Santipada, https://santifm.org/santipada/wp-content/uploads/2012/08/Sects Sectarianism\_Bhikkhu\_Sujato.pdf, accessed 11 October 2024.%
\item Sujato, Bhikkhu. 2023. “The lost Vajjian clan of the \textsanskrit{Ñātikas}”, https://discourse.suttacentral.net/t/the-lost-vajjian-clan-of-the-natikas, accessed 11 October 2024.%
\item Sujato, Bhikkhu. 2024. “On the brahmin who said ‘\textsanskrit{huṁ}’”, https://discourse.suttacentral.net/t/on-the-brahmin-who-said-hu, accessed 14 October 2024.%
\item Sujato, Bhikkhu. 17 June 2024. “\textsanskrit{Sakāya} \textsanskrit{niruttiyā} with my own interpretation”. Https://discourse.suttacentral.net/t/sakaya-niruttiya-with-my-own-interpretation/34466, accessed 15 October 2024.%
\item \textsanskrit{Ṭhānissaro}, Bhikkhu. 2013. “The Buddhist Monastic Code I”. Valley Center, CA, https://www.accesstoinsight.org/lib/authors/thanissaro/bmc1.pdf, accessed 10 February 2024.%
\item \textsanskrit{Ṭhānissaro}, Bhikkhu. 2013. “The Buddhist Monastic Code II”. Valley Center, CA, https://www.accesstoinsight.org/lib/authors/thanissaro/bmc2.pdf, accessed 11 October 2024.%
\item Thiradhammo, Bhikkhu. 1993. “Corrections to the Book of the Discipline”, in the Journal of the Pali Text Society, Vol. XIX, pp. 65–68.%
\item Trenckner, V, begun by; revised, continued and edited by Dines Anderson and Helmer Smith and others. 2001. “A Critical Pali Dictionary”, The Pali Text Society, Oxford.%
\item Tropical Plants Database, Ken Fern. Tropical.theferns.info. 2024–10–15. Http://tropical.theferns.info/viewtropical.php?id=Plectranthus+elegans, accessed 15 October 2024.%
\item University of Tuebingen: Height datahub (2015) – processed by Our World in Data. “Human Height (University of Tuebingen (2015))” [dataset], https://ourworldindata.org/grapher/average-height-of-men-for-selected-countries?country=IND, accessed 10 February 2024.%
\item Vimala, Bhikkhu*\textsanskrit{nī}. 2021. “Through the Yellow Gate: Ordination of Gender-Nonconforming People in the Buddhist Vinaya”, Academia, https://www.academia.edu/45662764/Through\_the\_Yellow\_Gate\_Ordination\_of\_Gender\_Nonconforming\_People\_in\_the\_Buddhist\_Vinaya, accessed 10 February 2024.%
\item Vimalanyani, \textsanskrit{Bhikkhunī}, trans. 5 June 2024 (last updated: 20 September 2024). “\textsanskrit{Mahāsaṅghika} Vinaya, \textsanskrit{Bhikkhunī} \textsanskrit{Pātimokkha}”, https://vimalanyani.github.io/vinaya-lzh/mg/pm/, accessed 13 October 2024.%
\item Wikipedia contributors. “True hermaphrodism.” Wikipedia, The Free Encyclopedia, 20 January 2024, https://en.wikipedia.org/w/index.php?title=True\_hermaphroditism\&oldid=1197351960, accessed 10 February 2024.%
\item Wikipedia contributors. “Wrightia antidysenteri.” Wikipedia, The Free Encyclopedia, 12 August 2023, https://en.wikipedia.org/w/index.php?title=Wrightia\_antidysenterica\&oldid=1169916779, accessed 10 October 2024.%
\item Wisdom Library, “Śatapatha-\textsanskrit{brāhmaṇa}: The \textsanskrit{Darśapūrṇamāsa}-\textsanskrit{iṣṭī} or new and full-moon sacrifices”, Eggeling, Julius, trans. 1882. Https://www.wisdomlib.org/hinduism/book/satapatha-brahmana-english/d/doc63113.html, accessed 15 October 2024.%
\end{enumerate}

%
\chapter*{Abbreviations}
\addcontentsline{toc}{chapter}{Abbreviations}
\markboth{Abbreviations}{Abbreviations}

\begin{description}%
\item[AN] \textsanskrit{Aṅguttara} Nikāya (references are to Nipāta and \textit{sutta} numbers)%
\item[AN-a] \textsanskrit{Aṅguttara} Nikāya \textsanskrit{aṭṭhakathā}, the commentary on the \textsanskrit{Aṅguttara} Nikāya%
\item[As] \textit{\textsanskrit{adhikaraṇasamathadhamma}}%
\item[Ay] \textit{aniyata}%
\item[Bi] \textit{\textsanskrit{bhikkhunī}}%
\item[Bu] \textit{bhikkhu}%
\item[CPD] Critical Pali Dictionary%
\item[DN] \textsanskrit{Dīgha} \textsanskrit{Nikāya} (references are to \textit{sutta} numbers)%
\item[DN-a] \textsanskrit{Dīgha} \textsanskrit{Nikāya} \textsanskrit{aṭṭhakathā}, the commentary on the \textsanskrit{Dīgha} \textsanskrit{Nikāya}%
\item[DOP] Dictionary of Pali%
\item[f, ff] and the following page, pages%
\item[Iti] Itivuttaka (references are to verse numbers)%
\item[Ja] \textsanskrit{Jātaka} and \textsanskrit{Jātaka} \textsanskrit{aṭṭhakathā}%
\item[Kd] Khandhaka%
\item[Khuddas-\textsanskrit{pṭ}] \textsanskrit{Khuddasikkhā}-\textsanskrit{purāṇaṭīkā} (references are to paragraph numbers)%
\item[Khuddas-\textsanskrit{nṭ}] \textsanskrit{Khuddasikkhā}-\textsanskrit{abhinavaṭīkā} (references are to paragraph numbers)%
\item[Kkh] \textsanskrit{Kaṅkha}̄\textsanskrit{vitaraṇi}̄%
\item[Kkh-\textsanskrit{pṭ}] \textsanskrit{Kaṅkhāvitaraṇīpurāṇa}-\textsanskrit{ṭīkā}%
\item[MN] Majjhima \textsanskrit{Nikāya} (references are to \textit{sutta} numbers)%
\item[MN-a] Majjhima \textsanskrit{Nikāya} \textsanskrit{aṭṭhakathā}, the commentary on the Majjhima \textsanskrit{Nikāya}%
\item[MS] \textsanskrit{Mahāsaṅgīti} \textsanskrit{Tipiṭaka} (the version of the \textsanskrit{Tipiṭaka} found on SuttaCentral)%
\item[N\&E] “Nature and the Environment in Early Buddhism”, Bhante Dhammika%
\item[Nidd-a] \textsanskrit{Mahāniddesa} \textsanskrit{aṭṭhakathā} (references are to VRI edition paragraph numbers)%
\item[NP] \textit{nissaggiya \textsanskrit{pācittiya}}%
\item[p., pp.] page, pages%
\item[Pc] \textit{\textsanskrit{pācittiya}}%
\item[Pd] \textit{\textsanskrit{pāṭidesanīya}}%
\item[PED] Pali English Dictionary%
\item[Pj] \textit{\textsanskrit{pārājika}}%
\item[PTS] Pali Text Society%
\item[Pvr] \textsanskrit{Parivāra}%
\item[SAF] “South Asian Flora as reflected in the twelfth-century Pali lexicon \textsanskrit{Abhidhānapadīpikā}”, J. Liyanaratne%
\item[SED] Sanskrit English Dictionary%
\item[Sk] \textit{sekhiya}%
\item[SN] \textsanskrit{Saṁyutta} \textsanskrit{Nikāya} (references are to \textsanskrit{Saṁyutta} and \textit{sutta} numbers)%
\item[SN-a] \textsanskrit{Saṁyutta} \textsanskrit{Nikāya} \textsanskrit{aṭṭhakathā}, the commentary on the \textsanskrit{Saṁyutta} \textsanskrit{Nikāya} (references are to volume number and paragraph numbers of the VRI version)%
\item[Sp] Samantapāsādikā, the commentary on the Vinaya \textsanskrit{Piṭaka} (references are to volume and paragraph numbers of the VRI version)%
\item[Sp‑ṭ] Sāratthadīpanī-ṭīkā (references follow the division into five volumes of the Canonical text and then add the paragraph number of the VRI version of the sub-commentary)%
\item[Sp‑yoj] \textsanskrit{Pācityādiyojanā} (volume numbers match those of Sp of the online VRI version, which, given that Sp‑yoj starts with the \textit{bhikkhu \textsanskrit{pācittiyas}}, means that Sp‑yoj is divided into four volumes, starting at volume 2; paragraph numbers are those of the VRI version)%
\item[SRT] Siamrath \textsanskrit{Tipiṭaka}, official edition of the \textsanskrit{Tipiṭaka} published in Thailand%
\item[Ss] \textit{\textsanskrit{saṅghādisesa}}%
\item[sv.] \textit{sub voce}, see under%
\item[\textsanskrit{Thīg}] \textsanskrit{Therīgāthā}%
\item[Ud-a] \textsanskrit{Udāna} \textsanskrit{aṭṭhakathā}, the commentary on the \textsanskrit{Udāna} (references are to \textit{sutta} number)%
\item[Vb] \textsanskrit{Vibhaṅga}, the second book of the Abhidhamma \textsanskrit{Piṭaka}%
\item[Vin-\textsanskrit{ālaṅ}-\textsanskrit{ṭ}] \textsanskrit{Vinayālaṅkāra}-\textsanskrit{ṭīkā} (references are to chapter number and paragraph numbers of the VRI version)%
\item[Vin-vn-\textsanskrit{ṭ}] \textsanskrit{Vinayavinicchayaṭīkā} (references are to paragraph numbers of the VRI version)%
\item[Vjb] \textsanskrit{Vajirabuddhiṭīkā} (references are to volume and paragraph numbers of the VRI version)%
\item[Vmv] \textsanskrit{Vimativinodanī}-\textsanskrit{ṭīkā} (references are to volume and paragraph numbers of the VRI version)%
\item[VRI] Vipassana Research Institute, the publisher of the online version of the Sixth Council edition of the Pali Canon at https://www.tipitaka.org%
\item[Vv-a] \textsanskrit{Vimānavatthu} \textsanskrit{aṭṭhakathā}, the commentary on the \textsanskrit{Vimānavatthu} (references are to paragraph numbers of the VRI edition).%
\end{description}

%
\mainmatter%
\pagestyle{fancy}%
\addtocontents{toc}{\let\protect\contentsline\protect\nopagecontentsline}
\part*{Analysis of Rules for Monks (1)}
\addcontentsline{toc}{part}{Analysis of Rules for Monks (1)}
\markboth{}{}
\addtocontents{toc}{\let\protect\contentsline\protect\oldcontentsline}

%
\addtocontents{toc}{\let\protect\contentsline\protect\nopagecontentsline}
\chapter*{Expulsion }
\addcontentsline{toc}{chapter}{\tocchapterline{Expulsion }}
\addtocontents{toc}{\let\protect\contentsline\protect\oldcontentsline}

%
%
\section*{{\suttatitleacronym Bu Pj 1}{\suttatitletranslation The first training rule on expulsion }{\suttatitleroot Methunadhamma}}
\addcontentsline{toc}{section}{\tocacronym{Bu Pj 1} \toctranslation{The first training rule on expulsion } \tocroot{Methunadhamma}}
\markboth{The first training rule on expulsion }{Methunadhamma}
\extramarks{Bu Pj 1}{Bu Pj 1}

\scnamo{Homage to the Buddha, the Perfected One, the fully Awakened One }

\subsection*{At \textsanskrit{Verañjā}: the origin of Monastic Law }

At\marginnote{1.1.1} one time the Buddha was staying at \textsanskrit{Verañjā} at the foot of \textsanskrit{Naḷeru}’s Nimba tree with a large Sangha of five hundred monks. A brahmin in \textsanskrit{Verañjā} was told: 

“Sir,\marginnote{1.1.3} the ascetic Gotama, the Sakyan, who has gone forth from the Sakyan clan, is staying at \textsanskrit{Verañjā} at the foot of Naleru’s Nimba tree with a large sangha of five hundred monks. That good Gotama has a fine reputation: ‘He is a Buddha, perfected and fully awakened, complete in insight and conduct, happy, knower of the world, supreme leader of trainable people, teacher of gods and humans, awakened, a Buddha. With his own insight he has seen this world with its gods, its lords of death, and its supreme beings, this society with its monastics and brahmins, its gods and humans, and he makes it known to others. He has a Teaching that’s good in the beginning, good in the middle, and good in the end. It has a true goal and is well articulated. He sets out a perfectly complete and pure spiritual life.’ It’s good to see such perfected ones.” 

That\marginnote{1.2.1} brahmin then went to the Buddha, exchanged pleasantries with him, sat down, and said, 

“I’ve\marginnote{1.2.2} heard, good Gotama, that you don’t bow down to old brahmins, stand up for them, or offer them a seat. I’ve now seen that this is indeed the case. This isn’t right.” 

“Brahmin,\marginnote{1.2.6} in the world with its gods, lords of death, and supreme beings, in this society with its monastics and brahmins, its gods and humans, I don’t see anyone I should bow down to, rise up for, or offer a seat. If I did, their head would explode.” 

“Good\marginnote{1.3.1} Gotama lacks taste.” 

“There’s\marginnote{1.3.2} a way you could rightly say that I lack taste. For I’ve abandoned the taste for forms, sounds, smells, flavors, and touches. I’ve cut it off at the root, made it like a palm stump, eradicated it, and made it incapable of reappearing in the future. But that’s not what you had in mind.” 

“Good\marginnote{1.3.5} Gotama has no enjoyment.” 

“There’s\marginnote{1.3.6} a way you could rightly say that I have no enjoyment. For I’ve abandoned the enjoyment of forms, sounds, smells, flavors, and touches. I’ve cut it off at the root, made it like a palm stump, eradicated it, and made it incapable of reappearing in the future. But that’s not what you had in mind.” 

“Good\marginnote{1.3.9} Gotama teaches inaction.” 

“There’s\marginnote{1.3.10} a way you could rightly say that I teach inaction. For I teach the non-doing of misconduct by body, speech, and mind. I teach the non-doing of the various kinds of bad, unwholesome actions. But that’s not what you had in mind.” 

“Good\marginnote{1.3.14} Gotama is an annihilationist.” 

“There’s\marginnote{1.3.15} a way you could rightly say that I’m an annihilationist. For I teach the annihilation of sensual desire, ill will, and confusion. I teach the annihilation of the various kinds of bad, unwholesome qualities. But that’s not what you had in mind.” 

“Good\marginnote{1.3.19} Gotama is disgusting.”\footnote{The literal meaning is “Good Gotama is disgusted,” but I am taking literary license to make it more meaningful and punchy. Sp 1.7: \textit{Puna \textsanskrit{brāhmaṇo} “jigucchati \textsanskrit{maññe} \textsanskrit{samaṇo} gotamo \textsanskrit{idaṁ} \textsanskrit{vayovuḍḍhānaṁ} \textsanskrit{abhivādanādikulasamudācārakammaṁ}, tena \textsanskrit{taṁ} na \textsanskrit{karotī}”ti \textsanskrit{maññamāno} \textsanskrit{bhagavantaṁ} \textsanskrit{jegucchīti} \textsanskrit{āha}}, “Again, the brahmin says ‘The Buddha is disgusted’ because he thinks, ‘It seems the ascetic Gotama is disgusted with doing the wholesome actions of bowing down, etc., to elders.’” The brahmin clearly didn’t approve of such conduct, perhaps even finding it disgusting. } 

“There’s\marginnote{1.3.20} a way you could rightly say that I’m disgusting. For I am disgusted by misconduct by body, speech, and mind. I am disgusted by the various kinds of bad, unwholesome qualities. But that’s not what you had in mind.” 

“Good\marginnote{1.3.24} Gotama is an exterminator.” 

“There’s\marginnote{1.3.25} a way you could rightly say that I’m an exterminator. For I teach the extermination of sensual desire, ill will, and confusion. I teach the extermination of the various kinds of bad, unwholesome qualities. But that’s not what you had in mind.” 

“Good\marginnote{1.3.29} Gotama is austere.” 

“There’s\marginnote{1.3.30} a way you could rightly say that I’m austere. For I say that bad, unwholesome qualities—misconduct by body, speech, and mind—are to be disciplined. One who has abandoned them, cut them off at the root, made them like a palm stump, eradicated them, and made them incapable of reappearing in the future—such a one I call austere. Now I’ve abandoned the bad, unwholesome qualities that are to be disciplined. I’ve cut them off at the root, made them like a palm stump, eradicated them, and made them incapable of reappearing in the future. But that’s not what you had in mind.” 

“Good\marginnote{1.3.35} Gotama is retiring.”\footnote{“Retiring” renders \textit{apagabbha}, explained in the commentaries, at Sp 1.10, as: \textit{Gabbhato apagatoti apagabbho}, “\textit{Apagabbha} means departed from the womb.” However, there is an alternative, and perhaps more convincing, derivation of this word. According to SED, in Vedic Sanskrit we find the word \textit{apagalbha} in the meaning “wanting in boldness” or “timid”. It seems possible, then, that we here have a play on words, where the brahmin refers to “timid” whereas the Buddha responds according to the meaning “departed from the womb” or “retired from rebirth”. I have used the word “retiring” in an attempt at catching this pun. } 

“There’s\marginnote{1.3.36} a way you could rightly say that I’m retiring. For one who has retired from any future conception in a womb, any rebirth in a future life, who has cut it off at the root, made it like a palm stump, eradicated it, and made it incapable of reappearing in the future—such a one I call retiring. Now I’ve retired from any future conception in a womb, any rebirth in a future life. I’ve cut it off at the root, made it like a palm stump, eradicated it, and made it incapable of reappearing in the future. But that’s not what you had in mind. 

Suppose,\marginnote{1.4.1} brahmin, there was a hen with eight, ten, or twelve eggs, which she had properly covered, warmed, and incubated. The first chick that hatches safely—after piercing through the eggshell with its claw or its beak—is it to be called the eldest or the youngest?” 

“It’s\marginnote{1.4.4} to be called the eldest, for it’s the eldest among them.” 

“Just\marginnote{1.4.6} so, in this deluded society, enveloped like an egg, I alone in the world have cracked the shell of delusion and reached the supreme full awakening. I, brahmin, am the world’s eldest and best. 

I\marginnote{1.5.1} was firmly energetic and had clarity of mindfulness; my body was tranquil and my mind stilled and unified. Fully secluded from the five senses, secluded from unwholesome mental qualities, I entered and remained in the first absorption, which has movement of the mind, as well as the joy and bliss of seclusion. Through the stilling of the movement of the mind, I entered and remained in the second absorption, which has internal confidence and unification of mind, as well as the joy and bliss of stillness. Through the fading away of joy, I remained even-minded, mindful, and fully aware, experiencing bliss directly, and I entered and remained in the third absorption of which the noble ones declare: ‘You are even-minded, mindful, and abide in bliss.’ Through the abandoning of bliss and pain and the earlier ending of joy and aversion, I entered and remained in the fourth absorption, which has neither pain nor bliss, but consists of purity of mindfulness and even-mindedness. 

Then,\marginnote{1.6.1} with my mind stilled, purified, cleansed, flawless, free from defilements, supple, wieldy, steady, and unshakable, I directed it to the knowledge that consists of recollecting past lives. I recollected many past lives, that is, one birth, two births, three births, four births, five births, ten births, twenty births, thirty births, forty births, fifty births, a hundred births, a thousand births, a hundred thousand births; many eons of world dissolution, many eons of world evolution, and many eons of both dissolution and evolution. And I knew: ‘There I had such a name, such a family, such an appearance, such food, such an experience of pleasure and pain, and such a lifespan. Passing away from there, I was reborn elsewhere, and there I had such a name, such a family, such an appearance, such food, such an experience of pleasure and pain, and such a lifespan. Passing away from there, I was reborn here.’ In this way I recollected many past lives with their characteristics and particulars. This was the first true insight, which I attained in the first part of the night. Delusion was dispelled and true insight arose, darkness was dispelled and light arose, as happens to one who is heedful, energetic, and diligent. This, brahmin, was my first breaking out, like a chick from an eggshell. 

Then,\marginnote{1.7.1} with my mind stilled, purified, cleansed, flawless, free from defilements, supple, wieldy, steady, and unshakable, I directed it to the knowledge of the passing away and arising of beings. With superhuman and purified clairvoyance, I saw beings passing away and getting reborn, inferior and superior, beautiful and ugly, gone to good destinations and to bad destinations, and I understood how beings pass on according to their actions: ‘These beings who engaged in misconduct by body, speech, and mind, who abused the noble ones, who had wrong views and acted accordingly, at the breaking up of the body after death, have been reborn in a lower realm, a bad destination, a world of misery, hell. But these beings who engaged in good conduct of body, speech, and mind, who did not abuse the noble ones, who held right view and acted accordingly, at the breaking up of the body after death, have been reborn in a good destination, a heaven world.’ In this way, with superhuman and purified clairvoyance, I saw beings passing away and getting reborn, inferior and superior, beautiful and ugly, gone to good destinations and to bad destinations, and I understood how beings pass on according to their actions. This was the second true insight, which I  attained in the middle part of the night. Delusion was dispelled and true insight arose, darkness was dispelled and light arose, as happens to one who is heedful, energetic, and diligent. This, brahmin, was my second breaking out, like a chick from an eggshell. 

Then,\marginnote{1.8.1} with my mind stilled, purified, cleansed, flawless, free from defilements, supple, wieldy, steady, and unshakable, I directed it to the knowledge of the ending of the corruptions. I knew according to reality: ‘This is suffering;’ ‘This is the origin of suffering;’ ‘This is the end of suffering;’ ‘This is the path leading to the end of suffering.’ I knew according to reality: ‘These are the corruptions;’ ‘This is the origin of the corruptions;’ ‘This is the end of the corruptions;’ ‘This is the path leading to the end of the corruptions.’ When I knew and saw this, my mind was freed from the corruption of worldly desire, from the corruption of desire to exist, and from the corruption of delusion. When it was freed, I knew it was freed. I understood that birth had come to an end, the spiritual life had been fulfilled, the job had been done, there was no further state of existence. This was the third true insight, which I attained in the last part of the night. Delusion was dispelled and true insight arose, darkness was dispelled and light arose, as happens to one who is heedful, energetic, and diligent. This, brahmin, was my third breaking out, like a chick from an eggshell.” 

That\marginnote{1.9.1} brahmin then said to the Buddha, 

“Good\marginnote{1.9.2} Gotama is the eldest! Good Gotama is the best! Wonderful, good Gotama, wonderful! Just as one might set upright what’s overturned, or reveal what’s hidden, or show the way to one who’s lost, or bring a lamp into the dark so that one with eyes might see what’s there—just so has the Buddha made the Teaching clear in many ways. Good Gotama, I go for refuge to the Buddha, the Teaching, and the Sangha of monks. Please accept me as a lay follower who’s gone for refuge for life. And please consent to spend the rainy-season residence at \textsanskrit{Verañjā} together with the Sangha of monks.” The Buddha consented by keeping silent, and the brahmin understood. He then got up from his seat, bowed down, circumambulated the Buddha with his right side toward him, and left. 

At\marginnote{2.1.1} that time \textsanskrit{Verañjā} was short of food and afflicted with hunger, with crops affected by whiteheads and turned to straw. It was not easy to get by on almsfood.\footnote{“Whiteheads” renders \textit{\textsanskrit{setaṭṭ}(h)\textsanskrit{ikā}}, literally, “white bones”. Sp 4.403: \textit{\textsanskrit{Setaṭṭhikā} \textsanskrit{nāma} \textsanskrit{rogajātīti} eko \textsanskrit{pāṇako} \textsanskrit{nāḷimajjhagataṁ} \textsanskrit{kaṇḍaṁ} vijjhati, yena \textsanskrit{viddhattā} nikkhantampi \textsanskrit{sālisīsaṁ} \textsanskrit{khīraṁ} \textsanskrit{gahetuṁ} na sakkoti}, “The disease called \textit{\textsanskrit{setaṭṭhikā}} means: an insect penetrates the stem, goes to the middle of the stalk, from the penetration of which the rice grains are not able to get sap.” This seems to be a description of so-called “whiteheads”, pale panicles without rice grains, caused by stem borers. } Just then some horse dealers from \textsanskrit{Uttarāpatha} had entered the rainy-season residence at \textsanskrit{Verañjā} with five hundred horses. In the horse pen they prepared portion upon portion of steamed grain for the monks. 

Then,\marginnote{2.1.4} after robing up in the morning, the monks took their bowls and robes and entered \textsanskrit{Verañjā} for alms. Not getting anything, they went to the horse pen. They then brought back many portions of steamed grain to the monastery, where they pounded and ate them. Venerable Ānanda crushed a portion on a stone, took it to the Buddha, and the Buddha ate it. 

And\marginnote{2.1.6} the Buddha heard the sound of the mortar. When Buddhas know what is going on, sometimes they ask and sometimes not. They know the right time to ask and when not to ask. Buddhas ask when it is beneficial, otherwise not, for Buddhas are incapable of doing what is unbeneficial.\footnote{“Incapable of doing” renders \textit{\textsanskrit{setughāta}}, literally, “destroyed the bridge”. Sp 1.16: \textit{Setu vuccati maggo, maggeneva \textsanskrit{tādisassa} vacanassa \textsanskrit{ghāto}, samucchedoti \textsanskrit{vuttaṁ} hoti}, “The path is called the bridge. What is said is that there is the destruction and cutting off of such speech by the path.” The commentary seems to take \textit{setu}, “bridge”, as a reference to the eightfold path. According to this understanding, the Buddha has cut off access to bad qualities, including bad speech, by fulfilling the eightfold path. I prefer to understand “bridge” as a metaphor for access, that is, the Buddhas no longer have access to what is unbeneficial. } Buddhas question the monks for two reasons: to give a teaching or to lay down a training rule. 

And\marginnote{2.1.12} so he said to Ānanda, “Ānanda, what’s this sound of a mortar?” Ānanda told him what was happening. 

“Well\marginnote{2.1.14} done, Ānanda. You’re all superior people who have conquered the problems of famine. Later generations will despise even meat and rice.”\footnote{I have supplied “the problems of famine” from the commentary to bring out the meaning. Sp 1.16: \textit{Dubbhikkhe \textsanskrit{dullabhapiṇḍe} \textsanskrit{imāya} \textsanskrit{sallahukavuttitāya} \textsanskrit{iminā} ca sallekhena \textsanskrit{vijitaṁ}}. } 

Then\marginnote{2.2.1} Venerable \textsanskrit{Mahāmoggallāna} went to the Buddha, bowed, sat down, and said, 

“At\marginnote{2.2.2} present, sir, \textsanskrit{Verañjā} is short of food and afflicted with hunger, with crops affected by whiteheads and turned to straw. It’s not easy to get by on almsfood. But the undersurface of this great earth abounds with food, tasting just like pure honey. Would it be good, sir, if I inverted the earth so that the monks may enjoy the nutrition in that ground-fungus?”\footnote{Sp 1.17: \textit{Ettha \textsanskrit{sādhūti} \textsanskrit{āyācanavacanametaṁ}}, “Here this \textit{\textsanskrit{sādhu}} is expressing a question.” And the same below. \textit{\textsanskrit{Pappaṭaka}}, according to \href{https://suttacentral.net/dn27/en/brahmali\#14.1}{DN 27}, is a kind of mushroom. } 

“But\marginnote{2.2.6} what will you do, \textsanskrit{Moggallāna}, with the creatures that live there?” 

“I’ll\marginnote{2.2.7} transform one hand to be like the great earth and make those creatures go there. I’ll then invert the earth with the other hand.” 

“Let\marginnote{2.2.9} it be, \textsanskrit{Moggallāna}, don’t invert the earth. Those creatures might lose their minds.” 

“In\marginnote{2.2.11} that case, sir, would it be good if the whole Sangha of monks could go to Uttarakuru for alms?” 

“Let\marginnote{2.2.12} it be, \textsanskrit{Moggallāna}, don’t pursue this.” 

Soon\marginnote{3.1.1} afterwards, while reflecting in private, Venerable \textsanskrit{Sāriputta} thought, “Which Buddhas had a long-lasting spiritual life, and which not?” 

In\marginnote{3.1.4} the evening, after coming out of seclusion, \textsanskrit{Sāriputta} went to the Buddha, bowed, sat down, and said, “Just now, sir, while I was reflecting in private, I was wondering which Buddhas had a long-lasting spiritual life, and which not?” 

“\textsanskrit{Sāriputta},\marginnote{3.1.7} the spiritual life established by the Buddhas \textsanskrit{Vipassī}, \textsanskrit{Sikhī}, and \textsanskrit{Vessabhū} didn’t last long. But the spiritual life established by the Buddhas Kakusandha, \textsanskrit{Konāgamana}, and Kassapa did.” 

“And\marginnote{3.2.1} why did the spiritual life established by the former three Buddhas not last long?” 

“They\marginnote{3.2.2} made no effort to give detailed teachings to their disciples. They gave few discourses in prose or in mixed prose and verse; few expositions, verses, heartfelt exclamations, quotations, birth stories, amazing accounts, and analyses. Nor did they lay down training rules or recite a monastic code. After the disappearance of those Buddhas and the disciples awakened under them, those who were the last disciples—of various names, families, and castes, who had gone forth from various households—allowed that spiritual life to disappear rapidly. It’s like flowers on a wooden plank. If they’re not held together with a string, they’re scattered about, whirled about, and destroyed by the wind. Why? Because they’re not held together with a string. Just so, after the disappearance of those Buddhas and the disciples awakened under them, those who were the last disciples allowed that spiritual life to disappear rapidly. 

Instead\marginnote{3.2.10} they were untiring at instructing their disciples by reading their minds. At one time, \textsanskrit{Sāriputta}, the Buddha \textsanskrit{Vessabhū}, the Perfected and the fully Awakened One, was staying in a certain frightening forest grove. He instructed a sangha of a thousand monks by reading their minds, saying, ‘Think like this, not like that; pay attention like this, not like that; abandon this and attain that.’ When they had been instructed by Buddha \textsanskrit{Vessabhū}, their minds were freed from the corruptions through letting go. But if anyone with sensual desire entered that frightening forest grove, they usually had goosebumps all over. This is why the spiritual life established by those Buddhas did not last long.” 

“Why\marginnote{3.3.1} then did the spiritual life established by the latter three Buddhas last long?” 

“The\marginnote{3.3.2} Buddhas Kakusandha, \textsanskrit{Konāgamana}, and Kassapa were untiring in giving detailed teachings to their disciples. They gave many discourses in prose and in mixed prose and verse; many expositions, verses, heartfelt exclamations, quotations, birth stories, amazing accounts, and analyses. And they laid down training rules and recited a monastic code. After the disappearance of those Buddhas and the disciples awakened under them, those who were the last disciples—of various names, families, and castes, who had gone forth from various households—made that spiritual life last for a long time. It’s like flowers on a wooden plank. If they’re held together with a string, they’re not scattered about, whirled about, or destroyed by the wind. Why? Because they are held together with a string. Just so, after the disappearance of those Buddhas and the disciples awakened under them, those who were the last disciples made that spiritual life last for a long time. This is why the spiritual life established by those Buddhas lasted long.” 

\textsanskrit{Sāriputta}\marginnote{3.4.1} then got up from his seat, arranged his upper robe over one shoulder, raised his joined palms, and said, “This is the time, venerable sir, for laying down training rules and reciting a monastic code, so that this spiritual life may last for a long time.” 

“Hold\marginnote{3.4.4} on, \textsanskrit{Sāriputta}. The Buddha knows the appropriate time for this. The Teacher doesn’t lay down training rules or recite a monastic code until the causes of corruption appear in the Sangha. 

And\marginnote{3.4.7} they don’t appear until the Sangha has attained long standing, great size, an abundance of the best material support, or great learning. When the causes of corruption appear for any of these reasons, then the Teacher lays down training rules for his disciples and recites a monastic code in order to counteract these causes. 

\textsanskrit{Sāriputta},\marginnote{3.4.15} the Sangha of monks is free from cancer and danger, stainless, pure, and established in the essence. Even the least developed of these five hundred monks is a stream-enterer. They will not be reborn in the lower world, but are fixed in destiny and bound for awakening.” 

Then\marginnote{4.1} the Buddha said to Ānanda, “Ānanda, it’s the custom for Buddhas not to go wandering the country without taking leave of those who invited them to spend the rainy-season residence. Let’s go to the brahmin of \textsanskrit{Verañjā} and take leave.” 

“Yes,\marginnote{4.3} sir.” 

The\marginnote{4.4} Buddha robed up, took his bowl and robe and, with Ānanda as his attendant, went to that brahmin’s house where he sat down on the prepared seat. The brahmin approached the Buddha, bowed, and sat down. 

And\marginnote{4.6} the Buddha said, “Brahmin, we’ve completed the rains residence according to your invitation, and now we take leave of you. We wish to depart to wander the country.” 

“It’s\marginnote{4.8} true, good Gotama, that you’ve completed the rains residence according to my invitation, but I haven’t given anything. That’s not good. It’s not because I didn’t want to, but because household life is so busy. Would you and the Sangha of monks please accept a meal from me tomorrow?” 

The\marginnote{4.12} Buddha consented by keeping silent. Then, after instructing, inspiring, and gladdening that brahmin with a teaching, the Buddha got up from his seat and left. 

The\marginnote{4.14} following morning the brahmin prepared various kinds of fine foods in his own house and then had the Buddha informed that the meal was ready. 

The\marginnote{4.15} Buddha robed up, took his bowl and robe and, together with the Sangha of monks, he went to that brahmin’s house where he sat down on the prepared seat. And that brahmin personally served and satisfied the Sangha of monks headed by the Buddha with various kinds of fine foods. When the Buddha had finished his meal, the brahmin gave him a set of three robes and to each monk two pieces of cloth. The Buddha instructed, inspired, and gladdened him with a teaching, and then got up from his seat and left. 

After\marginnote{4.18} remaining in \textsanskrit{Verañjā} for as long as he liked, the Buddha traveled to \textsanskrit{Payāgapatiṭṭhāna} via Soreyya, \textsanskrit{Saṅkassa}, and \textsanskrit{Kaṇṇakujja}. There he crossed the river Ganges and continued on to Benares.\footnote{Regarding these names, Ven. Shravasti Dhammika of Australia tells me (private communication) the following about the geographical situation of these places: “\textsanskrit{Verañjā} is probably the huge mound at Atranji Khera, about 13 km. north of Etah in Uttar Pradesh; Soreyya is possibly Soron, directly south of \textsanskrit{Verañjā}; \textsanskrit{Saṅkassa} is the modern Sankisa; \textsanskrit{Kaṇṇakujja} is the modern Kannauj; and \textsanskrit{Payāgapatiṭṭhāna} is, as Horner correctly says, Allahabad, recently renamed Prayag, its original name. Except for the uncertain Soreyya, all these places are on a roughly west/east alignment, undoubtedly following the ancient Madhura to \textsanskrit{Payāga} road.” In a subsequent email he clarified that Soron is actually north of Atranjki Khera or \textsanskrit{Verañjā}. Following the modern road network, the total distance from Atranji Khera to Vaishali (\textsanskrit{Vesālī}), according to Google maps, is in excess of 900 kilometers. } After remaining at Benares for as long as he liked, he set out wandering toward \textsanskrit{Vesālī}. When he eventually arrived, he stayed in the hall with the peaked roof in the Great Wood. 

\scend{The section for recitation on \textsanskrit{Verañjā} is finished. }

\subsection*{The first training rule on expulsion }

\subsubsection*{First sub-story: the section for recitation on Sudinna }

At\marginnote{5.1.1} that time Sudinna, the son of a wealthy merchant, lived in a village called Kalanda not far from \textsanskrit{Vesālī}. On one occasion Sudinna went to \textsanskrit{Vesālī} on some business together with a number of friends. Just then the Buddha was seated giving a teaching, surrounded by a large gathering of people. When Sudinna saw this, he thought, “Why don’t I listen to the Teaching?” He then approached that gathering and sat down. 

As\marginnote{5.1.6} he was sitting there, he thought, “The way I understand the Buddha’s Teaching, it’s not easy for one who lives at home to lead the spiritual life perfectly complete and pure as a polished conch shell. Why don’t I cut off my hair and beard, put on the ocher robes, and go forth into homelessness?” 

When\marginnote{5.1.9} those people had been instructed, inspired, and gladdened by the Buddha, they got up from their seats, bowed down, circumambulated him with their right sides toward him, and left. 

Sudinna\marginnote{5.1.10} then approached the Buddha, bowed, sat down, and told him what he had thought, adding, 

“Sir,\marginnote{5.1.11} please give me the going forth.” 

“But,\marginnote{5.1.14} Sudinna, do you have your parents’ permission?” 

“No.”\marginnote{5.1.15} 

“Buddhas\marginnote{5.1.16} don’t give the going forth to anyone who hasn’t gotten their parents’ permission.” 

“I’ll\marginnote{5.1.17} do whatever is necessary, sir, to get my parents’ permission.” 

After\marginnote{5.2.1} finishing his business in \textsanskrit{Vesālī}, Sudinna returned to Kalanda. He then went to his parents and said, “Mom and dad, the way I understand the Buddha’s Teaching, it’s not easy for one who lives at home to lead the spiritual life perfectly complete and pure. I want to cut off my hair and beard, put on the ocher robes, and go forth into homelessness. Please give me permission to go forth.” 

“But,\marginnote{5.2.5} Sudinna, you’re our only child, and we love you dearly. You live in comfort and we care for you. You don’t have any suffering.  Even if you died we wouldn’t want to lose you.  So how can we allow you to go forth into homelessness while you’re still living?” 

Sudinna\marginnote{5.2.9} asked his parents a second   and a third time, but got the same reply. 

\marginnote{5.2.27} He then lay down on the bare ground and said, “I’ll either die right here or go forth!” And he did not eat at the next seven meals. 

His\marginnote{5.2.31} parents repeated what they had said, adding, “Get up, Sudinna, eat, drink, and enjoy yourself! Enjoy the pleasures of the world and do acts of merit. We won’t allow you to go forth.” But Sudinna did not respond. 

His\marginnote{5.2.39} parents said the same thing a second and a third time,   but Sudinna remained silent. 

Then\marginnote{5.3.1} Sudinna’s friends went to him and repeated three times what his parents had said.    When Sudinna still did not respond, 

Sudinna’s\marginnote{5.4.1} friends went to his parents and said, “Sudinna says he’ll either die right there on the bare ground or go forth.  If you don’t allow him to go forth, he’ll die there. But if you allow him to go forth, you’ll see him again afterwards. And if he doesn’t enjoy the going forth, what alternative will he have but to come back here? So please allow him to go forth.” 

“Alright,\marginnote{5.4.8} then.” 

And\marginnote{5.4.9} Sudinna’s friends said to him,  “Get up, Sudinna, your parents have given you permission to go forth.” 

When\marginnote{5.4.11} Sudinna heard this, he was excited and joyful, stroking his limbs with his hands as he got up. After spending a few days to regain his strength, he went to the Buddha, bowed, sat down, and said, “Sir, I’ve gotten my parents’ permission to go forth. Please give me the going forth.” 

He\marginnote{5.4.15} then received the going forth and the full ordination in the presence of the Buddha. Not long afterwards he practiced these kinds of ascetic practices: he stayed in the wilderness, ate only almsfood, was a rag-robe wearer, and went on continuous almsround. And he lived supported by a certain Vajjian village. 

Soon\marginnote{5.5.1} afterwards, the Vajjians were short of food and afflicted with hunger, with crops affected by whiteheads and turned to straw. It was not easy to get by on almsfood. Sudinna considered this and thought, “I have many wealthy relatives in \textsanskrit{Vesālī}. Why don’t I get them to support me? My relatives will be able to make offerings and merit, the monks will get material support, and I’ll have no trouble getting almsfood.” 

He\marginnote{5.5.6} then put his dwelling in order, took his bowl and robe, and set out for \textsanskrit{Vesālī}. When he eventually arrived, he stayed in the hall with the peaked roof in the Great Wood. His relatives heard that he had arrived in \textsanskrit{Vesālī}, and they presented him with an offering of sixty servings of food. Sudinna gave the sixty servings to the monks. He then took his bowl and robe and went to the village of Kalanda for alms. As he was going on continuous almsround, he came to his own father’s house. 

Just\marginnote{5.6.1} then a female slave of Sudinna’s relatives was about to throw away the previous evening’s porridge. Sudinna said to her,  “If that’s to be thrown away, sister, put it here in my almsbowl.” 

As\marginnote{5.6.4} she was putting the porridge into his bowl, she recognized his hands, feet, and voice. She then went to his mother and said,  “Please be aware, ma’am,  that master Sudinna is back.” 

“Gosh,\marginnote{5.6.8} if you’re telling the truth, you’re a free woman!” 

As\marginnote{5.6.9} Sudinna was eating the previous evening’s porridge at the base of a certain wall, his father was coming home from work. When he saw him sitting there, he went up to him and said,  “But, Sudinna, isn’t there … What! You’re eating old porridge!\footnote{Sp 1.32: \textit{\textsanskrit{Idañhi} \textsanskrit{vuttaṁ} hoti – “atthi nu kho, \textsanskrit{tāta} sudinna, \textsanskrit{amhākaṁ} \textsanskrit{dhanaṁ}, na \textsanskrit{mayaṁ} \textsanskrit{niddhanāti}”}; “For this is said: ‘Isn’t there, dear Sudinna, our wealth? We are not poor.’” The point, perhaps, is that the father meant to say much more, but was interrupted at the shock of seeing his son eating old porridge. } Why don’t you come to your own house?” 

“We\marginnote{5.6.14} went to your house, householder. That’s where we received this porridge.” 

Sudinna’s\marginnote{5.6.16} father took him by the arm and said, “Come, let’s go home.” 

Sudinna\marginnote{5.6.17} went to his father’s house and sat down on the prepared seat. His father said to him, “Please eat, Sudinna.” 

“There’s\marginnote{5.6.19} no need. I’m done for today.” 

“Then\marginnote{5.6.20} come back for the meal tomorrow.” 

Sudinna\marginnote{5.6.21} consented by keeping silent, and he got up from his seat and left. 

The\marginnote{5.6.22} next morning Sudinna’s mother had the floor smeared with fresh cow-dung. She then piled up two heaps, one with money, the other with gold.\footnote{For the rendering of \textit{\textsanskrit{hirañña}} as “money”, see Appendix of Technical Terms. } The heaps were so large that a man standing on one side could not see a man standing on the other. She hid the heaps behind screens, prepared a seat between them, and surrounded it all with a curtain. And she said to Sudinna’s ex-wife, “Now please adorn yourself in the way our son Sudinna found you especially attractive.” 

“Yes,\marginnote{5.6.26} ma’am.” 

Then,\marginnote{5.7.1} after robing up, Venerable Sudinna took his bowl and robe and went to his father’s house where he sat down on the prepared seat. His father went up to him, uncovered the heaps, and said, “This dowry, dear Sudinna, is the fortune from your mother. It’s yours. Another is the fortune from your father and another the fortune from your ancestors. Please return to the lower life, enjoy wealth, and make merit.” 

“I\marginnote{5.7.6} can’t, father. I’m enjoying the spiritual life.” 

Sudinna’s\marginnote{5.7.7} father repeated his request and Sudinna replied as before. When Sudinna’s father repeated his request a third time, Sudinna said, “If you wouldn’t get offended, I could tell you what to do.” 

“Let’s\marginnote{5.7.13} hear it.” 

“Well\marginnote{5.7.14} then, make some large hempen sacks, put all the money and gold inside, take it all away in carts, and dump it in the middle of the Ganges. And why? Because that way you will avoid the danger, fear, and terror that it will otherwise cause you, as well as the trouble with protecting it.” 

His\marginnote{5.7.17} father became upset, thinking, “How can our son Sudinna say such things?” 

He\marginnote{5.7.18} then said to Sudinna’s ex-wife, “Well then, since you were so dear to him, perhaps our son Sudinna will listen to you?” 

Sudinna’s\marginnote{5.7.20} ex-wife took hold of his feet and said, “What are these nymphs like, sir, for which you practice the spiritual life?” 

“Sister,\marginnote{5.7.22} I don’t practice the spiritual life for the sake of nymphs.” 

His\marginnote{5.7.23} ex-wife thought, “Sudinna is now calling me ‘sister’,” and she fainted right there. 

Sudinna\marginnote{5.8.1} said to his father,  “If there’s food to be given, householder, give it,  but don’t annoy me.” 

“Eat,\marginnote{5.8.4} Sudinna,” he said. And Sudinna’s mother and father personally served and satisfied him with various kinds of fine foods. 

When\marginnote{5.8.6} Sudinna had finished his meal, his mother said to him: “Sudinna dear, our family is rich. Please return to the lower life, enjoy wealth, and make merit.” 

“Mother,\marginnote{5.8.10} I can’t. I’m enjoying the spiritual life.” 

His\marginnote{5.8.11} mother repeated her request a second time, but got the same reply. She then said, “Our family is wealthy, Sudinna. Please give us an offspring, so that the \textsanskrit{Licchavīs} don’t take over our heirless property.” 

“Yes,\marginnote{5.8.16} mother, I can do that.” 

“But\marginnote{5.8.17} where are you staying?” 

“In\marginnote{5.8.18} the Great Wood.” And he got up from his seat and left. 

Sudinna’s\marginnote{5.9.1} mother then said to his ex-wife, “Well then, as soon as you reach your fertile period, please tell me.” 

“Yes,\marginnote{5.9.3} ma’am.” Not long afterwards Sudinna’s ex-wife reached her fertile period, and she reported it to Sudinna’s mother. 

“Now,\marginnote{5.9.6} please adorn yourself in the way that our son Sudinna found you especially attractive.” 

“Yes.”\marginnote{5.9.7} 

Then\marginnote{5.9.8} Sudinna’s mother, together with his ex-wife, went to Sudinna in the Great Wood, and she said to him: “Our family, dear Sudinna, is rich. Please return to the lower life, enjoy wealth, and make merit.” 

“Mother,\marginnote{5.9.12} I can’t. I’m enjoying the spiritual life.” 

His\marginnote{5.9.14} mother repeated her request a second time, but got the same reply. She then said this: “Well then, please give us an offspring. We don’t want the \textsanskrit{Licchavīs} to take over our heirless property.” 

“Alright,\marginnote{5.9.19} mother.” He then took his ex-wife by the arm, disappeared into the Great Wood and—there being no training rule and he seeing no danger—had sexual intercourse with her three times. As a result she conceived. 

And\marginnote{5.9.22} the earth gods cried out: “Sirs, the Sangha of monks has been free from cancer and danger. But Sudinna of Kalanda has produced a cancer and endangered it.” 

Hearing\marginnote{5.9.25} the earth gods, the gods of the four great kings cried out … the gods of the Thirty-three … the \textsanskrit{Yāma} gods … the contented gods … the gods who delight in creation … the gods who control the creation of others … the gods of the realm of the supreme beings cried out: “Sirs, the Sangha of monks has been free from cancer and danger. But Sudinna of Kalanda has produced a cancer and endangered it.” Thus in that moment, in that instance, the news spread as far as the world of the supreme beings. 

In\marginnote{5.9.35} the meantime, the pregnancy of Sudinna’s ex-wife developed, and she eventually gave birth to a son. Sudinna’s friends called him Offspring, while calling Sudinna’s ex-wife Offspring’s mother, and Venerable Sudinna Offspring’s father. After some time, they both went forth into homelessness and realized perfection. 

But\marginnote{5.10.1} Sudinna was anxious and remorseful, thinking, “This is truly bad for me, that after going forth on such a well-proclaimed spiritual path, I wasn’t able to practice the perfectly complete and pure spiritual life to the end.”\footnote{“Spiritual path” renders \textit{dhammavinaya}. See Appendix of Technical Terms for discussion. } And because of his anxiety and remorse, he became thin, haggard, and pale, with veins protruding all over his body. He became sad, sluggish, miserable, and depressed, weighed down by remorse. 

The\marginnote{5.10.5} monks who were Sudinna’s friends said to him: “In the past, Sudinna, you had a good color, a bright face, clear skin, and sharp senses. But look at you now. Could it be that you are dissatisfied with the spiritual life?” 

“I\marginnote{5.10.9} am not dissatisfied with the spiritual life, but I’ve done something bad. I’ve had sexual intercourse with my ex-wife. I’m anxious and remorseful because I wasn’t able to practice the perfectly complete and pure spiritual life to the end.” 

“No\marginnote{5.10.14} wonder you’re anxious, Sudinna, no wonder you have remorse. Hasn’t the Buddha given many teachings for the sake of dispassion, not for the sake of lust; for the sake of freedom from bondage, not for the sake of bondage; for the sake of non-grasping, not for the sake of grasping? When the Buddha has taught in this way, how could you choose lust, bondage, and grasping? Hasn’t the Buddha given many teachings for the fading away of lust, for the clearing away of intoxication, for the removal of thirst, for the uprooting of attachment, for the cutting off of the round of rebirth, for the stopping of craving, for fading away, for ending, for extinguishment? Hasn’t the Buddha in many ways taught the abandoning of worldly pleasures, the full understanding of the perception of worldly pleasures, the removal of thirst for worldly pleasures, the elimination of thoughts of worldly pleasures, the stilling of the fever of worldly pleasures? This will affect people’s confidence and cause some to lose it.” 

After\marginnote{5.11.1} rebuking Sudinna in many ways, they told the Buddha. The Buddha then had the Sangha of monks gathered and questioned Sudinna: “Is it true, Sudinna, that you had sexual intercourse with your ex-wife?” 

“It’s\marginnote{5.11.4} true, sir.” 

The\marginnote{5.11.5} Buddha rebuked him, “Foolish man, it’s not suitable, it’s not proper, it’s not worthy of a monastic, it’s not allowable, it’s not to be done. How could you go forth on such a well-proclaimed spiritual path and not be able to practice the perfectly complete and pure spiritual life to the end? Haven’t I given many teachings for the sake of dispassion, not for the sake of lust; for the sake of freedom from bondage, not for the sake of bondage; for the sake of non-grasping, not for the sake of grasping? When I have taught in this way, how could you choose lust, bondage, and grasping? Haven’t I given many teachings for the fading away of lust, for the clearing away of intoxication, for the removal of thirst, for the uprooting of attachment, for the cutting off of the round of rebirth, for the stopping of craving, for fading away, for ending, for extinguishment? Haven’t I in many ways taught the abandoning of worldly pleasures, the full understanding of the perceptions of worldly pleasures, the removal of thirst for worldly pleasures, the elimination of thoughts of worldly pleasures, the stilling of the fevers of worldly pleasures? It would be better, foolish man, for your penis to enter the mouth of a highly venomous snake than to enter a woman. It would be better for your penis to enter the mouth of a black snake than to enter a woman. It would be better for your penis to enter a blazing charcoal pit than to enter a woman. Why is that? For although it might cause death or death-like suffering, it would not cause you to be reborn in a bad destination. But \emph{this} might. Foolish man, you have practiced what is contrary to the true Teaching, the common practice, the low practice, the coarse practice, that which ends with a wash, that which is done in private, that which is done wherever there are couples. You are the forerunner, the first performer of many unwholesome things. This will affect people’s confidence, and cause some to lose it.” 

Then\marginnote{5.11.29} the Buddha spoke in many ways in dispraise of being difficult to support and maintain, in dispraise of great desires, discontent, socializing, and laziness; but he spoke in many ways in praise of being easy to support and maintain, of fewness of wishes, contentment, self-effacement, ascetic practices, serenity, reduction in things, and being energetic. After giving a teaching on what is right and proper, he addressed the monks: 

“Well\marginnote{5.11.31} then, monks, I will lay down a training rule for the following ten reasons: for the well-being of the Sangha, for the comfort of the Sangha, for the restraint of bad people, for the ease of good monks, for the restraint of the corruptions relating to the present life, for the restraint of the corruptions relating to future lives, to give rise to confidence in those without it, to increase the confidence of those who have it, for the longevity of the true Teaching, and for supporting the training.\footnote{“Training” renders \textit{vinaya}. See Appendix of Technical Terms for discussion. } And, monks, this training rule should be recited like this: 

\subsubsection*{First preliminary ruling }

\scrule{‘If a monk has sexual intercourse, he is expelled and excluded from the community.’” }

In\marginnote{5.11.35} this way the Buddha laid down this training rule for the monks. 

\scend{The section for recitation on Sudinna is finished. }

\subsubsection*{Second sub-story: the account of the female monkey }

Sometime\marginnote{6.1} later, in the Great Wood near \textsanskrit{Vesālī}, a certain monk befriended a female monkey by giving her food. He then had sexual intercourse with her. Soon afterwards, after robing up in the morning, he took his bowl and robe and entered \textsanskrit{Vesālī} for alms. 

Just\marginnote{6.3} then a number of monks who were walking about the dwellings came to the one belonging to this monk. The female monkey saw those monks coming. She went up to them, shook her buttocks in front of them, wagged her tail, presented her backside, and made a gesture. The monks thought, “This monk must be having sexual intercourse with this monkey,” and they hid to one side. Then, when that monk had finished his almsround in \textsanskrit{Vesālī} and had returned with his almsfood, he ate one part himself and gave the rest to that female monkey. After eating the food, the monkey presented her buttocks to the monk, and he had sexual intercourse with her. 

The\marginnote{6.11} other monks said to him, “Hasn’t a training rule been laid down by the Buddha? Why then do you have sexual intercourse with a monkey?” 

“It’s\marginnote{6.13} true that a training rule has been laid down by the Buddha, but it concerns women, not animals.” 

“But\marginnote{6.14} that’s just the same. It’s not suitable, it’s not proper, it’s not worthy of a monastic, it’s not allowable, it’s not to be done. How could you go forth on such a well-proclaimed spiritual path and not be able to practice the perfectly complete and pure spiritual life to the end? Hasn’t the Buddha given many teachings for the sake of dispassion … the stilling of the fevers of worldly pleasures? This will affect people’s confidence, and cause some to lose it.” 

After\marginnote{6.21} rebuking that monk in many ways, they told the Buddha. 

The\marginnote{6.22} Buddha then had the Sangha of monks gathered and questioned that monk: “Is it true, monk, that you did this?” 

“It’s\marginnote{6.24} true, sir.” 

The\marginnote{6.25} Buddha rebuked him, “Foolish man, it’s not suitable, it’s not proper, it’s not worthy of a monastic, it’s not allowable, it’s not to be done. How could you go forth on such a well-proclaimed spiritual path and not be able to practice the perfectly complete and pure spiritual life to the end? Haven’t I given many teachings for the sake of dispassion … for the stilling of the fevers of worldly desire? It would be better, foolish man, for your penis to enter the mouth of a highly venomous snake than to enter a female monkey. It would be better for your penis to enter the mouth of a black snake than to enter a female monkey. It would be better for your penis to enter a blazing charcoal pit than to enter a female monkey. Why is that? For although it might cause death or death-like suffering, it would not cause you to be reborn in a bad destination. But \emph{this} might. Foolish man, you’ve practiced what is contrary to the true Teaching, the common practice, the low practice, the coarse practice, that which ends with a wash, that which is done in private, that which is done wherever there are couples. This will affect people’s confidence …” … 

“And\marginnote{6.39} so, monks, this training rule should be recited like this: 

\subsubsection*{Second preliminary ruling }

\scrule{‘If a monk has sexual intercourse, even with a female animal, he is expelled and excluded from the community.’” }

In\marginnote{6.41} this way the Buddha laid down this training rule for the monks. 

\scend{The account of the female monkey is finished. }

\subsubsection*{Third sub-story: the section for recitation on covering }

Soon\marginnote{7.1.1} afterwards there were a number of Vajjian monks from \textsanskrit{Vesālī} who ate, slept, and bathed as much as they liked. Then, not reflecting properly and without first renouncing the training and revealing their weakness, they had sexual intercourse. After some time they were affected by loss of relatives, loss of property, and loss of health. They then went to Venerable Ānanda and said, 

“Venerable\marginnote{7.1.4} Ānanda, we don’t blame the Buddha, the Teaching, or the Sangha; we only have ourselves to blame. We were unfortunate and had little merit—after going forth on such a well-proclaimed spiritual path, we were unable to practice the perfectly complete and pure spiritual life to the end. If we were now to get the going forth and the full ordination in the presence of the Buddha, we would have clarity about wholesome qualities and be devoted day in and day out to developing the aids to awakening. Venerable Ānanda, please inform the Buddha.” 

Saying,\marginnote{7.1.10} “Yes,” he went to the Buddha and told him. 

“It’s\marginnote{7.1.11} impossible, Ānanda, that the Buddha should abolish a training rule that entails expulsion because of the Vajjians.” 

The\marginnote{7.1.12} Buddha then gave a teaching and addressed the monks: 

\scrule{“Monks, if someone, without first renouncing the training and revealing their weakness, has sexual intercourse, they may not receive the full ordination once again. But, monks, if someone has sexual intercourse after first renouncing the training and revealing their weakness, they may receive the full ordination once again. }

And\marginnote{7.1.15} so, monks, this training rule should be recited like this: 

\subsection*{Final ruling }

\scrule{‘If a monk, after taking on the monks’ training and way of life, without first renouncing the training and revealing his weakness, has sexual intercourse, even with a female animal, he is expelled and excluded from the community.’”\footnote{\textit{\textsanskrit{Paccakkhāya}} from \textit{\textsanskrit{paccakkhāti}} means to “speak against” or “renounce verbally”. To avoid clunkiness, I normally render this as “renounce”, except where the context requires the more complete formulation. } }

\subsection*{Definitions }

\begin{description}%
\item[A: ] whoever, of such a kind, of such activity, of such caste, of such name, of such family, of such conduct, of such behavior, of such association, who is senior, who is junior, or who is of middle standing—this is called “a”. %
\item[Monk: ] he is a monk because he lives on alms; a monk because he has gone over to living on alms; a monk because he wears a patchwork cloth; a monk by convention; a monk on account of his claim; a “come, monk” monk; a monk given the full ordination by taking the three refuges; a good monk; a monk of substance; a trainee monk; a fully trained monk; a monk who has been given the full ordination by a unanimous Sangha through a legal procedure consisting of one motion and three announcements that is irreversible and fit to stand.\footnote{“Irreversible” renders \textit{akuppa}. See \textit{kuppa} in Appendix of Technical Terms for discussion. } The monk who has been given the full ordination by a unanimous Sangha through a legal procedure consisting of one motion and three announcements that is irreversible and fit to stand—this sort of monk is meant in this case. %
\item[Training: ] the three trainings: the training in the higher morality, the training in the higher mind, the training in the higher wisdom. The training in the higher morality is the training meant in this case. %
\item[Way of life: ] whatever training rule has been laid down by the Buddha—this is called “way of life”. One trains in this; therefore it is called “after taking on the way of life”. %
\item[Without first renouncing the training and revealing his weakness: ] “There\marginnote{8.2.2} is, monks, a revealing of weakness without the training being renounced; and there is a revealing of weakness together with the training being renounced. 

And\marginnote{8.2.4} how is there a revealing of weakness without the training being renounced? 

It\marginnote{8.2.5} may be that a monk is dissatisfied, discontent, desiring to give up the monastic life; troubled, ashamed, and disgusted with the monkhood; longing to be a householder, longing to be a lay follower, longing to be a monastery worker, longing to be a novice, longing to be a monastic of another religion, longing to be a lay follower of another religion, longing to be a non-ascetic, longing to be a non-monastic, and he says and declares: ‘Why don’t I renounce the Buddha?’ In this way, monks, there’s a revealing of weakness without the training being renounced. 

Or\marginnote{8.2.17} again, dissatisfied, discontent, desiring to give up the monastic life; troubled, ashamed, and disgusted with the monkhood; longing to be a householder … longing to be a non-monastic, he says and declares: ‘Why don’t I renounce the Teaching?’ … the Sangha?’ … the practice?’ … the training?’ … the Monastic Code?’ … the recitation?’ … my preceptor?’ … my teacher?’ … my student?’ … my pupil?’ … my co-student?’ … my co-pupil?’ … he says and declares: ‘Why don’t I renounce my fellow monastics?’ …\footnote{Here and in the next segment I have added ellipses points at the end. These ellipses points seem to have been omitted by mistake from the Pali. } ‘Why don’t I become a householder?’ … ‘Why don’t I become a lay follower?’ … ‘Why don’t I become a monastery worker?’ … ‘Why don’t I become a novice?’ … ‘Why don’t I become a monastic of another religion?’ … ‘Why don’t I become a lay follower of another religion?’ … ‘Why don’t I become a non-ascetic?’ … ‘Why don’t I become a non-monastic?’ In this way too, monks, there’s a revealing of weakness without the training being renounced. 

Or\marginnote{8.2.43} again, dissatisfied, discontent, desiring to give up the monastic life; troubled, ashamed, and disgusted with the monkhood; longing to be a householder … longing to be a non-monastic, he says and declares: ‘What if I renounced the Buddha?’ … he says and declares: ‘What if I were a non-monastic?’ … he says and declares: ‘Perhaps I should renounce the Buddha?’ … he says and declares: ‘Perhaps I should be a non-monastic?’ … he says and declares: ‘Well then, I should renounce the Buddha.’ … he says and declares: ‘Well then, I should be a non-monastic.’ … he says and declares: ‘I think I should renounce the Buddha.’ … he says and declares: ‘I think I should be a non-monastic.’ In this way too, monks, there’s a revealing of weakness without the training being renounced. 

Or\marginnote{8.2.56} again, dissatisfied, discontent, desiring to give up the monastic life; troubled, ashamed, and disgusted with the monkhood; longing to be a householder … longing to be a non-monastic, he says and declares: ‘I remember my mother.’ … ‘I remember my father.’ … ‘I remember my brother.’ … ‘I remember my sister.’ … ‘I remember my son.’ … ‘I remember my daughter.’ … ‘I remember my wife.’ … ‘I remember my relations.’ … ‘I remember my friends.’ … ‘I remember my village.’ … ‘I remember my town.’ … ‘I remember my fields.’ … ‘I remember my land.’ … ‘I remember my money.’ … ‘I remember my gold.’ … ‘I remember my profession.’ … he says and declares: ‘I remember my former laughter, chatter, and play.’ In this way too, monks, there’s a revealing of weakness without the training being renounced. 

Or\marginnote{8.2.78} again, dissatisfied, discontent, desiring to give up the monastic life; troubled, ashamed, and disgusted with the monkhood; longing to be a householder … longing to be a non-monastic, he says and declares: ‘I have a mother who should be supported by me.’ … ‘I have a father … ‘I have a brother … ‘I have a sister … ‘I have a son … ‘I have a daughter … ‘I have a wife … ‘I have relations … he says and declares: ‘I have friends who should be supported by me.’ In this way too, monks, there’s a revealing of weakness without the training being renounced. 

Or\marginnote{8.2.92} again, dissatisfied, discontent, desiring to give up the monastic life; troubled, ashamed, and disgusted with the monkhood; longing to be a householder … longing to be a non-monastic, he says and declares: ‘I have a mother; she’ll support me.’ … ‘I have a father; he’ll support me.’ … ‘I have a brother; he’ll support me.’ … ‘I have a sister; she’ll support me.’ … ‘I have a son; he’ll support me.’ … ‘I have a daughter; she’ll support me.’ … ‘I have a wife; she’ll support me.’ … ‘I have relations; they’ll support me.’ … ‘I have friends; they’ll support me.’ … ‘I have a village; I’ll live by means of it.’ … ‘I have a town; I’ll live by means of it.’ … ‘I have fields; I’ll live by means of them.’ … ‘I have land; I’ll live by means of it.’ … ‘I have money; I’ll live by means of it.’ … ‘I have gold; I’ll live by means of it.’ … he says and declares: ‘I have a profession; I’ll live by means of it.’ In this way too, monks, there’s a revealing of weakness without the training being renounced. 

Or\marginnote{8.2.113} again, dissatisfied, discontent, desiring to give up the monastic life; troubled, ashamed, and disgusted with the monkhood; longing to be a householder … longing to be a non-monastic, he says and declares: ‘This is difficult to do.’ … ‘This isn’t easy to do.’ … ‘This is difficult.’ … ‘This isn’t easy.’ … ‘I can’t endure.’ … ‘I’m unable.’ … ‘I don’t enjoy myself.’ … ‘I take no delight.’ In this way too, monks, there’s a revealing of weakness without the training being renounced. 

And\marginnote{8.3.1} how is there a revealing of weakness together with the training being renounced? 

It\marginnote{8.3.2} may be that a monk is dissatisfied, discontent, desiring to give up the monastic life; troubled, ashamed, and disgusted with the monkhood; longing to be a householder … longing to be a non-monastic, and he says and declares: ‘I renounce the Buddha.’ In this way, monks, there’s a revealing of weakness together with the training being renounced. 

Or\marginnote{8.3.8} again, dissatisfied, discontent, desiring to give up the monastic life; troubled, ashamed, and disgusted with the monkhood; longing to be a householder … longing to be a non-monastic, he says and declares: ‘I renounce the Teaching.’ … ‘I renounce the Sangha.’ … ‘I renounce the practice.’ … ‘I renounce the training.’ … ‘I renounce the Monastic Code.’ … ‘I renounce the recitation.’ … ‘I renounce my preceptor.’ … ‘I renounce my teacher.’ … ‘I renounce my student.’ … ‘I renounce my pupil.’ … ‘I renounce my co-student.’ … ‘I renounce my co-pupil.’ … ‘I renounce my fellow monastics.’ … ‘Consider me a householder.’ … ‘Consider me a lay follower.’ … ‘Consider me a monastery worker.’ … ‘Consider me a novice monk.’ … ‘Consider me a monastic of another religion.’ … ‘Consider me a lay follower of another religion.’ … ‘Consider me a non-ascetic.’ … ‘Consider me a non-monastic.’ In this way too, monks, there’s a revealing of weakness together with the training being renounced. 

Or\marginnote{8.3.34} again, dissatisfied, discontent, desiring to give up the monastic life; troubled, ashamed, and disgusted with the monkhood; longing to be a householder … longing to be a non-monastic, he says and declares: ‘I’m done with the Buddha.’ … ‘I’m done with my fellow monastics.’ In this way too, monks, there’s a revealing of weakness together with the training being renounced. 

Or\marginnote{8.3.41} again … he says and declares: ‘No more of the Buddha for me.’ … ‘No more of my fellow monastics for me.’ … 

‘The\marginnote{8.3.44} Buddha is of no use to me.’ … ‘My fellow monastics are of no use to me.’ 

‘I’m\marginnote{8.3.46} well freed from the Buddha.’ … ‘I’m well freed from my fellow monastics.’ In this way too, monks, there’s a revealing of weakness together with the training being renounced. 

Or\marginnote{8.3.49} whatever other synonyms there are for the Buddha, for the Teaching, for the Sangha, for the practice, for the training, for the Monastic Code, for the recitation, for a preceptor, for a teacher, for a student, for a pupil, for a co-student, for a co-pupil, for a fellow monastic, for a householder, for a lay follower, for a monastery worker, for a novice monk, for a monastic of another religion, for a lay follower of another religion, for a non-ascetic, or for a non-monastic—he speaks and declares by way of these indications, by way of these marks, by way of these signs. In this way, monks, there’s a revealing of weakness together with the training being renounced. 

And\marginnote{8.4.1} how is the training not renounced? 

If\marginnote{8.4.2} you renounce the training by way of these indications, by way of these marks, by way of these signs, but you’re insane, then the training isn’t renounced. If you renounce the training to one who’s insane, the training isn’t renounced. If you renounce the training when you’re deranged, the training isn’t renounced. If you renounce the training to one who’s deranged, the training isn’t renounced. If you renounce the training when you’re overwhelmed by pain, the training isn’t renounced. If you renounce the training to one who’s overwhelmed by pain, the training isn’t renounced. If you renounce the training to a god, the training isn’t renounced. If you renounce the training to an animal, the training isn’t renounced. If an Indo-Aryan renounces the training to a non-Indo-Aryan who doesn’t understand, the training isn’t renounced. If a non-Indo-Aryan renounces the training to an Indo-Aryan who doesn’t understand, the training isn’t renounced. If an Indo-Aryan renounces the training to an Indo-Aryan who doesn’t understand, the training isn’t renounced. If a non-Indo-Aryan renounces the training to a non-Indo-Aryan who doesn’t understand, the training isn’t renounced. If you renounce the training as a joke, the training isn’t renounced. If you renounce the training because of speaking too fast, the training isn’t renounced. If you announce what you don’t wish to announce, the training isn’t renounced. If you don’t announce what you wish to announce, the training isn’t renounced. If you announce to one who doesn’t understand, the training isn’t renounced. If you don’t announce to one who understands, the training isn’t renounced. Or if you don’t make a full announcement, the training isn’t renounced. In this way, monks, the training isn’t renounced.” 

%
\item[Sexual intercourse: ] what is contrary to the true Teaching, the common practice, the low practice, the coarse practice, that which ends with a wash, that which is done in private, that which is done wherever there are couples—this is called “sexual intercourse”. %
\item[Has: ] whoever makes an organ enter an organ, a genital enter a genital, even to the depth of a sesame seed—this is called “has”. %
\item[Even with a female animal: ] even having had sexual intercourse with a female animal, he is not an ascetic, not a Sakyan monastic, let alone with a woman—therefore it is called “even with a female animal”. %
\item[He is expelled: ] just as a man with his head cut off is unable to continue living by reconnecting it to the body, so too is a monk who has had sexual intercourse not an ascetic, not a Sakyan monastic. Therefore it is said, “he is expelled.” %
\item[Excluded from the community: ] Community: joint legal procedures, a joint recitation, the same training—this is called “community”. He does not take part in this—therefore it is called “excluded from the community”. %
\end{description}

\subsection*{Permutations }

\subsubsection*{Permutations part 1 }

\paragraph*{Summary }

There\marginnote{9.1.1} are three kinds of females: a female human being, a female spirit, a female animal. There are three kinds of hermaphrodites: a human hermaphrodite, a hermaphrodite spirit, a hermaphrodite animal.\footnote{“Hermaphrodite” renders \textit{\textsanskrit{ubhatobyañjanaka}}. See Appendix of Technical Terms for discussion. } There are three kinds of \textit{\textsanskrit{paṇḍakas}}: a human \textit{\textsanskrit{paṇḍaka}}, a \textit{\textsanskrit{paṇḍaka}} spirit, a \textit{\textsanskrit{paṇḍaka}} animal.\footnote{For the meaning of the term \textit{\textsanskrit{paṇḍaka}}, see Appendix of Technical Terms. } There are three kinds of males: a human male, a male spirit, a male animal. 

\paragraph*{Exposition part 1 }

He\marginnote{9.1.9.1} commits an offense entailing expulsion if he has sexual intercourse with a female human being through three orifices: the anus, the vagina, or the mouth. … with a female spirit … He commits an offense entailing expulsion if he has sexual intercourse with a female animal through three orifices: the anus, the vagina, or the mouth. … with a human hermaphrodite … with a hermaphrodite spirit … He commits an offense entailing expulsion if he has sexual intercourse with a hermaphrodite animal through three orifices: the anus, the vagina, or the mouth. He commits an offense entailing expulsion if he has sexual intercourse with a human \textit{\textsanskrit{paṇḍaka}} through two orifices: the anus or the mouth. … with a \textit{\textsanskrit{paṇḍaka}} spirit … with a \textit{\textsanskrit{paṇḍaka}} animal … with a human male … with a male spirit … He commits an offense entailing expulsion if he has sexual intercourse with a male animal through two orifices: the anus or the mouth. 

\paragraph*{Exposition part 2 }

\subparagraph*{Voluntary sexual intercourse }

If\marginnote{9.2.1} a monk has the intention of sexual relations and he makes his penis enter the anus of a female human being …\footnote{“Intention of sexual relations” renders \textit{sevanacitta}, literally, “intention of association”. The kind of association applicable to this rule, however, is sexual intercourse. } the vagina of a female human being … the mouth of a female human being, he commits an offense entailing expulsion. If a monk has the intention of sexual relations and he makes his penis enter the anus of a female spirit … the anus of a female animal … the anus of a human hermaphrodite … the anus of a hermaphrodite spirit … the anus of a hermaphrodite animal … the vagina of a hermaphrodite animal … the mouth of a hermaphrodite animal, he commits an offense entailing expulsion. If a monk has the intention of sexual relations and he makes his penis enter the anus of a human \textit{\textsanskrit{paṇḍaka}} … the anus of a \textit{\textsanskrit{paṇḍaka}} spirit … the anus of a \textit{\textsanskrit{paṇḍaka}} animal … the anus of a human male … the anus of a male spirit … the anus of a male animal … the mouth of a male animal, he commits an offense entailing expulsion. 

\subparagraph*{Forced sexual intercourse: bringing the partner to the monk }

Enemy\marginnote{9.3.1} monks bring a female human being to a monk and make her sit down so that his penis enters her anus.\footnote{“Enemy monks” renders \textit{\textsanskrit{bhikkhupaccatthikā}}, translated as “opponents of monks” by I. B. Horner, which seems to be incorrect. Below we find parallel compounds with other kinds of people, for instance \textit{\textsanskrit{corapaccatthikā}}, which in the context must mean “enemy criminals”, not “enemies of criminals”. Moreover, all these other people would already be accounted for if \textit{\textsanskrit{bhikkhupaccatthikā}} meant “enemies of monks”. There would be no need to mention them separately. } If he consents to the entry, and he consents to having entered, and he consents to the remaining, and he consents to the taking out, he commits an offense entailing expulsion.\footnote{That is, he consents to the sexual intercourse at each of these points. } Enemy monks bring a female human being to a monk and make her sit down so that his penis enters her anus. If he does not consent to the entry, but he consents to having entered, and he consents to the remaining, and he consents to the taking out, he commits an offense entailing expulsion. Enemy monks bring a female human being to a monk and make her sit down so that his penis enters her anus. If he does not consent to the entry, nor to having entered, but he consents to the remaining, and he consents to the taking out, he commits an offense entailing expulsion. Enemy monks bring a female human being to a monk and make her sit down so that his penis enters her anus. If he does not consent to the entry, nor to having entered, nor to the remaining, but he consents to the taking out, he commits an offense entailing expulsion. Enemy monks bring a female human being to a monk and make her sit down so that his penis enters her anus. If he does not consent to the entry, nor to having entered, nor to the remaining, nor to the taking out, there is no offense. 

Enemy\marginnote{9.3.11} monks bring a female human being to a monk and make her sit down so that his penis enters her vagina … her mouth. If he consents to the entry, and he consents to having entered, and he consents to the remaining, and he consents to the taking out, he commits an offense entailing expulsion. … If he does not consent to the entry, nor to having entered, nor to the remaining, nor to the taking out, there is no offense. 

Enemy\marginnote{9.3.15} monks bring a female human being who is awake … asleep … intoxicated … insane … heedless … dead but undecomposed … dead and mostly undecomposed … he commits an offense entailing expulsion. They bring one who is dead and mostly decomposed to a monk and make her sit down so that his penis enters her anus … her vagina … her mouth. If he consents to the entry, and he consents to having entered, and he consents to the remaining, and he consents to the taking out, he commits a serious offense. … If he does not consent … there is no offense. 

Enemy\marginnote{9.3.28} monks bring a female spirit … a female animal … a human hermaphrodite … a hermaphrodite spirit … a hermaphrodite animal to a monk and make it sit down so that his penis enters its anus … its vagina … its mouth. If he consents to the entry, and he consents to having entered, and he consents to the remaining, and he consents to the taking out, he commits an offense entailing expulsion. … If he does not consent … there is no offense. 

Enemy\marginnote{9.3.37} monks bring a hermaphrodite animal that is awake … asleep … intoxicated … insane … heedless … dead but undecomposed … dead and mostly undecomposed … he commits an offense entailing expulsion. They bring one that is dead and mostly decomposed to a monk and make it sit down so that his penis enters its anus … its vagina … its mouth. If he consents to the entry, and he consents to having entered, and he consents to the remaining, and he consents to the taking out, he commits a serious offense. … If he does not consent … there is no offense. 

Enemy\marginnote{9.3.50} monks bring a human \textit{\textsanskrit{paṇḍaka}} … a \textit{\textsanskrit{paṇḍaka}} spirit … a \textit{\textsanskrit{paṇḍaka}} animal to a monk and make it sit down so that his penis enters its anus … its mouth. If he consents to the entry, and he consents to having entered, and he consents to the remaining, and he consents to the taking out, he commits an offense entailing expulsion. … If he does not consent … there is no offense. 

Enemy\marginnote{9.3.56} monks bring a \textit{\textsanskrit{paṇḍaka}} animal that is awake … asleep … intoxicated … insane … heedless … dead but undecomposed … dead and mostly undecomposed … he commits an offense entailing expulsion. They bring one that is dead and mostly decomposed to a monk and make it sit down so that his penis enters its anus … its mouth. If he consents to the entry, and he consents to having entered, and he consents to the remaining, and he consents to the taking out, he commits a serious offense. … If he does not consent … there is no offense. 

Enemy\marginnote{9.3.68} monks bring a human male … a male spirit … a male animal to a monk and make it sit down so that his penis enters its anus … its mouth. If he consents to the entry, and he consents to having entered, and he consents to the remaining, and he consents to the taking out, he commits an offense entailing expulsion. … If he does not consent … there is no offense. 

Enemy\marginnote{9.3.74} monks bring a male animal that is awake … asleep … intoxicated … insane … heedless … dead but undecomposed … dead and mostly undecomposed … he commits an offense entailing expulsion. They bring one that is dead and mostly decomposed to a monk and make it sit down so that his penis enters its anus … its mouth. If he consents to the entry, and he consents to having entered, and he consents to the remaining, and he consents to the taking out, he commits a serious offense … If he does not consent … there is no offense. 

\subparagraph*{Forced sexual intercourse with cover: bringing the partner to the monk }

Enemy\marginnote{9.4.1} monks bring a female human being to a monk and make her sit down so that his penis enters her anus … her vagina … her mouth, the female covered and the monk uncovered; the female uncovered and the monk covered; the female covered and the monk covered; the female uncovered and the monk uncovered. If he consents to the entry, and he consents to having entered, and he consents to the remaining, and he consents to the taking out, he commits an offense entailing expulsion. … If he does not consent … there is no offense. 

Enemy\marginnote{9.4.10} monks bring a female human being who is awake … asleep … intoxicated … insane … heedless … dead but undecomposed … dead and mostly undecomposed … he commits an offense entailing expulsion. They bring one who is dead and mostly decomposed to a monk and make her sit down so that his penis enters her anus … her vagina … her mouth, the female covered and the monk uncovered; the female uncovered and the monk covered; the female covered and the monk covered; the female uncovered and the monk uncovered. If he consents to the entry, and he consents to having entered, and he consents to the remaining, and he consents to the taking out, he commits a serious offense. … If he does not consent … there is no offense. 

Enemy\marginnote{9.4.27} monks bring a female spirit … a female animal … a human hermaphrodite … a hermaphrodite spirit … a hermaphrodite animal to a monk and make it sit down so that his penis enters its anus … its vagina … its mouth, the animal covered and the monk uncovered; the animal uncovered and the monk covered; the animal covered and the monk covered; the animal uncovered and the monk uncovered. If he consents to the entry, and he consents to having entered, and he consents to the remaining, and he consents to the taking out, he commits an offense entailing expulsion. … If he does not consent … there is no offense. 

Enemy\marginnote{9.4.40} monks bring a hermaphrodite animal that is awake … asleep … intoxicated … insane … heedless … dead but undecomposed … dead and mostly undecomposed … he commits an offense entailing expulsion. They bring one that is dead and mostly decomposed to a monk and make it sit down so that his penis enters its anus … its vagina … its mouth, the animal covered and the monk uncovered; the animal uncovered and the monk covered; the animal covered and the monk covered; the animal uncovered and the monk uncovered. If he consents to the entry, and he consents to having entered, and he consents to the remaining, and he consents to the taking out, he commits a serious offense. … If he does not consent … there is no offense. 

Enemy\marginnote{9.4.57} monks bring a human \textit{\textsanskrit{paṇḍaka}} … a \textit{\textsanskrit{paṇḍaka}} spirit … a \textit{\textsanskrit{paṇḍaka}} animal … a human male … a male spirit … a male animal to a monk and make it sit down so that his penis enters its anus … its mouth, the animal covered and the monk uncovered; the animal uncovered and the monk covered; the animal covered and the monk covered; the animal uncovered and the monk uncovered. If he consents to the entry, and he consents to having entered, and he consents to the remaining, and he consents to the taking out, he commits an offense entailing expulsion. … If he does not consent … there is no offense. 

Enemy\marginnote{9.4.70} monks bring a male animal that is awake … asleep … intoxicated … insane … heedless … dead but undecomposed … dead and mostly undecomposed … he commits an offense entailing expulsion. They bring one that is dead and mostly decomposed to a monk and make it sit down so that his penis enters its anus … its mouth, the animal covered and the monk uncovered; the animal uncovered and the monk covered; the animal covered and the monk covered; the animal uncovered and the monk uncovered. If he consents to the entry, and he consents to having entered, and he consents to the remaining, and he consents to the taking out, he commits a serious offense. … If he does not consent … there is no offense. 

\subparagraph*{Forced sexual intercourse: bringing the monk to the partner }

Enemy\marginnote{9.5.1} monks bring a monk to a female human being and make him sit down so that his penis enters her anus … her vagina … her mouth. If he consents to the entry, and he consents to having entered, and he consents to the remaining, and he consents to the taking out, he commits an offense entailing expulsion. … If he does not consent … there is no offense. 

Enemy\marginnote{9.5.6} monks bring a monk to a female human being who is awake … asleep … intoxicated … insane … heedless … dead but undecomposed … dead and mostly undecomposed … he commits an offense entailing expulsion. They bring a monk to one who is dead and mostly decomposed and make him sit down so that his penis enters her anus … her vagina … her mouth. If he consents to the entry, and he consents to having entered, and he consents to the remaining, and he consents to the taking out, he commits a serious offense. … If he does not consent … there is no offense. 

Enemy\marginnote{9.5.19} monks bring a monk to a female spirit … a female animal … a human hermaphrodite … a hermaphrodite spirit … a hermaphrodite animal … a human \textit{\textsanskrit{paṇḍaka}} … a \textit{\textsanskrit{paṇḍaka}} spirit … a \textit{\textsanskrit{paṇḍaka}} animal … a human male … a male spirit … a male animal and make him sit down so that his penis enters its anus … its mouth. If he consents to the entry, and he consents to having entered, and he consents to the remaining, and he consents to the taking out, he commits an offense entailing expulsion. … If he does not consent … there is no offense. 

Enemy\marginnote{9.5.33} monks bring a monk to a male animal that is awake … asleep … intoxicated … insane … heedless … dead but undecomposed … dead and mostly undecomposed … he commits an offense entailing expulsion. They bring a monk to one that is dead and mostly decomposed and make him sit down so that his penis enters its anus … its mouth. If he consents to the entry, and he consents to having entered, and he consents to the remaining, and he consents to the taking out, he commits a serious offense … If he does not consent … there is no offense. 

\subparagraph*{Forced sexual intercourse with cover: bringing the monk to the partner }

Enemy\marginnote{9.6.1} monks bring a monk to a female human being and make him sit down so that his penis enters her anus …\footnote{The Pali phrase \textit{\textsanskrit{aṅgajātena} \textsanskrit{vaccamaggaṁ} \textsanskrit{abhinisīdenti}} could be rendered, “They make (the monk) sit down with his penis in the anus.” Above we find the inverse expression, \textit{vaccamaggena \textsanskrit{aṅgajātaṁ} \textsanskrit{abhinisīdenti}}, which could be rendered, “They make (the woman) sit on his penis with her anus.” The effect is the same, but the agent in the two cases is different. Since the agent is clear from the context, I have rendered the two expressions in the same way. } her vagina … her mouth, the monk covered and the female uncovered; the monk uncovered and the female covered; the monk covered and the female covered; the monk uncovered and the female uncovered. If he consents to the entry, and he consents to having entered, and he consents to the remaining, and he consents to the taking out, he commits an offense entailing expulsion. … If he does not consent … there is no offense. 

Enemy\marginnote{9.6.10} monks bring a monk to a female human being who is awake … asleep … intoxicated … insane … heedless … dead but undecomposed … dead and mostly undecomposed … he commits an offense entailing expulsion. They bring a monk to a female human being who is dead and mostly decomposed and make him sit down so that his penis enters her anus … her vagina … her mouth, the monk covered and the female uncovered; the monk uncovered and the female covered; the monk covered and the female covered; the monk uncovered and the female uncovered. If he consents to the entry, and he consents to having entered, and he consents to the remaining, and he consents to the taking out, he commits a serious offense. … If he does not consent … there is no offense. 

Enemy\marginnote{9.6.27} monks bring a monk to a female spirit … a female animal … a human hermaphrodite … a hermaphrodite spirit … a hermaphrodite animal … a human \textit{\textsanskrit{paṇḍaka}} … a \textit{\textsanskrit{paṇḍaka}} spirit … a \textit{\textsanskrit{paṇḍaka}} animal … a human male … a male spirit … a male animal and make him sit down so that his penis enters its anus … its mouth, the monk covered and the animal uncovered; the monk uncovered and the animal covered; the monk covered and the animal covered; the monk uncovered and the animal uncovered. If he consents to the entry, and he consents to having entered, and he consents to the remaining, and he consents to the taking out, he commits an offense entailing expulsion. … If he does not consent … there is no offense. 

Enemy\marginnote{9.6.45} monks bring a monk to a male animal that is awake … asleep … intoxicated … insane … heedless … dead but undecomposed … dead and mostly undecomposed … he commits an offense entailing expulsion. They bring a monk to one that is dead and mostly decomposed and make him sit down so that his penis enters its anus … its mouth, the monk covered and the animal uncovered; the monk uncovered and the animal covered; the monk covered and the animal covered; the monk uncovered and the animal uncovered. If he consents to the entry, and he consents to having entered, and he consents to the remaining, and he consents to the taking out, he commits a serious offense. … If he does not consent … there is no offense. 

As\marginnote{9.7.1} “enemy monks” has been explained in detail, so should the following categories be explained: 

Enemy\marginnote{9.7.2} kings … enemy bandits … enemy scoundrels … “lotus-scent” enemies. The section in brief is finished. 

\subsubsection*{Permutations part 2 }

If\marginnote{9.7.7.1} he makes a private part enter a private part, there is an offense entailing expulsion.\footnote{For an explanation of the words \textit{magga} and \textit{amagga} as used here, see \textit{magga} in Appendix of Technical Terms. In this section, as above, the instrumental case signifies the orifice that is entered. This may seem unusual, but it follows the pattern found elsewhere where the gateway through which anything (such as a house, a village, or a cul-de-sac) is entered is in the instrumental case. } If he makes the mouth enter a private part, there is an offense entailing expulsion. If he makes a private part enter the mouth, there is an offense entailing expulsion. If he makes the mouth enter the mouth, there is a serious offense. 

A\marginnote{9.7.11} monk rapes a sleeping monk: if he wakes up and consents, both should be expelled;\footnote{“Should be expelled” renders \textit{\textsanskrit{nāsetabbā}}. For a discussion of the verb \textit{\textsanskrit{nāseti}}, see Appendix of Technical Terms. } if he wakes up but does not consent, the rapist should be expelled.\footnote{“Rapist” renders \textit{\textsanskrit{dūsaka}}. See Appendix of Technical Terms for discussion. } A monk rapes a sleeping novice: if he wakes up and consents, both should be expelled; if he wakes up but does not consent, the rapist should be expelled. A novice rapes a sleeping monk: if he wakes up and consents, both should be expelled; if he wakes up but does not consent, the rapist should be expelled. A novice rapes a sleeping novice: if he wakes up and consents, both should be expelled; if he wakes up but does not consent, the rapist should be expelled. 

\subsection*{Non-offenses }

There\marginnote{9.7.23.1} is no offense: if he does not know; if he does not consent; if he is insane; if he is deranged; if he is overwhelmed by pain; if he is the first offender. 

\scend{The section for recitation on covering is finished. }

\scuddanaintro{Summary verses of case studies }

\begin{scuddana}%
“The\marginnote{9.7.32} female monkey, and the Vajjians, \\
Householder, and a naked one, monastics of other religions; \\
The girl, and \textsanskrit{Uppalavaṇṇā}, \\
Two others with characteristics. 

Mother,\marginnote{9.7.36} daughter, and sister, \\
And wife, supple, with long; \\
Two on wounds, and a picture, \\
And a wooden doll. 

Five\marginnote{9.7.40} with Sundara, \\
Five about charnel grounds, bones; \\
A female dragon, and a female spirit, and a female ghost, \\
A \textit{\textsanskrit{paṇḍaka}}, impaired, should touch. 

The\marginnote{9.7.44} sleeping Perfected One in Bhaddiya, \\
Four others in \textsanskrit{Sāvatthī}; \\
Three in \textsanskrit{Vesālī}, garlands, \\
The one from Bharukaccha in his dream. 

\textsanskrit{Supabbā},\marginnote{9.7.48} \textsanskrit{Saddhā}, a nun, \\
A trainee nun, and a novice nun; \\
A sex worker, a \textit{\textsanskrit{paṇḍaka}}, a female householder, \\
Each other, gone forth in old age, a deer.” 

%
\end{scuddana}

\subsubsection*{Case studies }

At\marginnote{10.1.1} one time a monk had sexual intercourse with a female monkey. He became anxious, thinking, “The Buddha has laid down a training rule. Could it be that I’ve committed an offense entailing expulsion?” He told the Buddha. “You’ve committed an offense entailing expulsion.” 

At\marginnote{10.2.1} one time a number of Vajjian monks from \textsanskrit{Vesālī} had sexual intercourse without first renouncing the training and revealing their weakness. They became anxious, thinking, “The Buddha has laid down a training rule. Could it be that we’ve committed an offense entailing expulsion?” They told the Buddha. “You’ve committed an offense entailing expulsion.” 

At\marginnote{10.3.1} one time a monk had sexual intercourse while dressed like a householder, thinking he would avoid an offense. He became anxious, thinking, “The Buddha has laid down a training rule. Could it be that I’ve committed an offense entailing expulsion?” He told the Buddha. “You’ve committed an offense entailing expulsion.” 

At\marginnote{10.3.9} one time a monk had sexual intercourse while naked, thinking he would avoid an offense. He became anxious … “You’ve committed an offense entailing expulsion.” 

At\marginnote{10.3.14} one time a monk had sexual intercourse while dressed in a grass sarong … while dressed in a bark sarong … while dressed in a sarong made of bits of wood …\footnote{Sp 1.67: \textit{\textsanskrit{Phalakacīraṁ} \textsanskrit{nāma} \textsanskrit{phalakasaṇṭhānāni} \textsanskrit{phalakāni} \textsanskrit{sibbitvā} \textsanskrit{katacīraṁ}}, “\textit{\textsanskrit{Phalakacīra}}: a robe made by sewing together bits of wood or what has the appearance of wood.” } while dressed in a sarong made of human hair … while dressed in a sarong made of horse-hair … while dressed in a sarong of owls’ wings … while dressed in a sarong of antelope hide, thinking he would avoid an offense. He became anxious … “You’ve committed an offense entailing expulsion.” 

At\marginnote{10.4.1} one time a monk who was an alms-collector saw a little girl lying on a bench. Being lustful, he inserted his thumb into her vagina. She died. He became anxious … “There’s no offense entailing expulsion, but there’s an offense entailing suspension.” 

At\marginnote{10.5.1} one time a young brahmin had fallen in love with the nun \textsanskrit{Uppalavaṇṇā}. When \textsanskrit{Uppalavaṇṇā} had gone to the village for alms, he entered her hut and hid himself. When she had eaten her meal and returned from almsround, \textsanskrit{Uppalavaṇṇā} washed her feet, entered her hut, and sat down on the bed. Then that young brahmin grabbed hold of her and raped her. She told the nuns what had happened. The nuns told the monks, who in turn told the Buddha. “There’s no offense for one who doesn’t consent.” 

At\marginnote{10.6.1} one time female characteristics appeared on a monk. They told the Buddha. 

\scrule{“Monks, I allow that discipleship, that ordination, those years as a monk, to be transferred to the nuns. The monks’ offenses that are in common with the nuns are to be cleared with the nuns. For the monks’ offenses that are not in common with the nuns, there’s no offense.” }

At\marginnote{10.6.6} one time male characteristics appeared on a nun. They told the Buddha. 

\scrule{“Monks, I allow that discipleship, that ordination, those years as a nun, to be transferred to the monks. The nuns’ offenses that are in common with the monks are to be cleared with the monks. For the nuns’ offenses that are not in common with the monks, there’s no offense.” }

At\marginnote{10.7.1} one time a monk had sexual intercourse with his mother … had sexual intercourse with his daughter … had sexual intercourse with his sister, thinking he would avoid an offense. … He became anxious … “You’ve committed an offense entailing expulsion.” 

At\marginnote{10.7.8} one time a monk had sexual intercourse with his ex-wife. He became anxious … “You’ve committed an offense entailing expulsion.” 

At\marginnote{10.8.1} one time there was a monk with a supple back who was plagued by lust. He inserted his penis into his own mouth. He became anxious … “You’ve committed an offense entailing expulsion.” 

At\marginnote{10.8.5} one time there was a monk with a long penis who was plagued by lust. He inserted his penis into his own anus. He became anxious … “You’ve committed an offense entailing expulsion.” 

At\marginnote{10.9.1} one time a monk saw a dead body with a wound next to the genitals. Thinking he would avoid an offense, he inserted his penis into the genitals and exited through the wound. He became anxious … “You’ve committed an offense entailing expulsion.” 

At\marginnote{10.9.7} one time a monk saw a dead body with a wound next to the genitals. Thinking he would avoid an offense, he inserted his penis into the wound and exited through the genitals. He became anxious … “You’ve committed an offense entailing expulsion.” 

At\marginnote{10.10.1} one time a lustful monk touched the genitals in a picture with his penis. He became anxious … “There’s no offense entailing expulsion, but there’s an offense of wrong conduct.” 

At\marginnote{10.10.5} one time a lustful monk touched the genitals of a wooden doll with his penis. He became anxious … “There’s no offense entailing expulsion, but there’s an offense of wrong conduct.” 

At\marginnote{10.11.1} one time a monk called Sundara who had gone forth in \textsanskrit{Rājagaha} was walking along a street.\footnote{Sp 1.73 explains the ablative \textsanskrit{Rājagahā} as a locative \textsanskrit{Rājagahe}. } A woman said to him, “Please wait, sir, I’ll pay respect to you.” As she was paying respect, she held up his sarong and inserted his penis into her mouth. He became anxious … “Monk, did you consent?” 

“I\marginnote{10.11.7} didn’t consent, sir.” 

“There’s\marginnote{10.11.8} no offense for one who doesn’t consent.” 

At\marginnote{10.12.1} one time a woman saw a monk and said, “Sir, come and have sexual intercourse.” 

“It’s\marginnote{10.12.3} not allowable.” 

“I’ll\marginnote{10.12.4} make the effort, not you. In this way there won’t be any offense for you.” The monk acted accordingly. He became anxious … “You’ve committed an offense entailing expulsion.” 

At\marginnote{10.12.9} one time a woman saw a monk and said, “Sir, come and have sexual intercourse.” 

“It’s\marginnote{10.12.11} not allowable.” 

“You\marginnote{10.12.12} make the effort, not I. In this way there won’t be any offense for you.” The monk acted accordingly. He became anxious … “You’ve committed an offense entailing expulsion.” 

At\marginnote{10.12.17} one time a woman saw a monk and said, “Sir, come and have sexual intercourse.” 

“It’s\marginnote{10.12.19} not allowable.” 

“Rub\marginnote{10.12.20} inside but discharge outside. … Rub outside but discharge inside. In this way there won’t be any offense for you.” The monk acted accordingly. He became anxious … “You’ve committed an offense entailing expulsion.” 

At\marginnote{10.13.1} one time a monk went to a charnel ground where he saw an undecomposed corpse. He had sexual intercourse with it. He became anxious … “You’ve committed an offense entailing expulsion.” 

At\marginnote{10.13.4} one time a monk went to a charnel ground where he saw a mostly undecomposed corpse. He had sexual intercourse with it. He became anxious … “You’ve committed an offense entailing expulsion.” 

At\marginnote{10.13.7} one time a monk went to a charnel ground where he saw a mostly decomposed corpse. He had sexual intercourse with it. He became anxious … “There’s no offense entailing expulsion, but there’s a serious offense.” 

At\marginnote{10.13.12} one time a monk went to a charnel ground where he saw a decapitated head. He inserted his penis into the open mouth, making contact as he entered.\footnote{Sp-\textsanskrit{ṭ} 1.73: \textit{\textsanskrit{Vaṭṭakateti} imassa \textsanskrit{atthaṁ} dassento “\textsanskrit{vivaṭe}”ti \textsanskrit{āha}}, “\textit{\textsanskrit{Vaṭṭakate}}: to show the meaning of this, they say ‘open’.” } He became anxious … “You’ve committed an offense entailing expulsion.” 

At\marginnote{10.13.15} one time a monk went to a charnel ground where he saw a decapitated head. He inserted his penis into the open mouth, without making contact as he entered. He became anxious … “There’s no offense entailing expulsion, but there’s an offense of wrong conduct.” 

At\marginnote{10.13.19} one time a monk was in love with a certain woman. When she died, the body was dumped on a charnel ground. After some time only scattered bones remained. The monk went to the charnel ground, collected the bones, and brought his penis into the genital area. He became anxious … “There’s no offense entailing expulsion, but there’s an offense of wrong conduct.” 

At\marginnote{10.14.1} one time a monk had sexual intercourse with a female dragon … had sexual intercourse with a female spirit … had sexual intercourse with a female ghost … had sexual intercourse with a \textit{\textsanskrit{paṇḍaka}}. He became anxious … “You’ve committed an offense entailing expulsion.” 

At\marginnote{10.15.1} one time there was a monk with impaired faculties. Thinking he would avoid an offense because he felt neither pleasure nor pain, he had sexual intercourse. … They told the Buddha. “Whether or not that fool felt anything, there’s an offense entailing expulsion.” 

At\marginnote{10.16.1} one time a monk who intended to have sexual intercourse with a woman felt remorse at the mere touch. He became anxious … “There’s no offense entailing expulsion, but there’s an offense entailing suspension.” 

At\marginnote{10.17.1} one time a monk was lying down in the \textsanskrit{Jātiyā} Grove at Bhaddiya, having gone there for the day’s meditation. He had an erection because of wind.\footnote{\textit{\textsanskrit{Aṅgamaṅgāni}} literally means, “various bodily parts”. The point is perhaps that the wind element caused stiffness throughout the body. Sp-\textsanskrit{ṭ} 1.74: \textit{\textsanskrit{Aṅgamaṅgāni} \textsanskrit{vātupatthaddhāni} \textsanskrit{hontīti} \textsanskrit{evaṁ} \textsanskrit{vuttavātupatthambhena}}, “\textit{\textsanskrit{Aṅgamaṅgāni} \textsanskrit{vātupatthaddhāni} honti}: in this way erection due to wind is spoken of.” See \href{https://suttacentral.net/pli-tv-bu-vb-ss1/en/brahmali\#3.2.15}{Bu Ss 1:3.2.15} for the expression \textit{\textsanskrit{vātupatthambha}}. } A certain woman saw him and sat down on his penis. Having taken her pleasure, she left. Seeing the moisture, the monks told the Buddha. “Monks, an erection occurs for five reasons: because of sensual desire, feces, urine, or wind, or because of being stung by caterpillars. It’s impossible that that monk had an erection because of sensual desire. That monk is a perfected one. There’s no offense for that monk.” 

At\marginnote{10.18.1} one time a monk was lying down in the Dark Wood at \textsanskrit{Sāvatthī}, having gone there for the day’s meditation. A woman cowherd saw him and sat down on his penis. The monk consented to the entry, to having entered, to the remaining, and to the taking out. He became anxious … “You’ve committed an offense entailing expulsion.” 

At\marginnote{10.18.6} one time a monk was lying down in the Dark Wood at \textsanskrit{Sāvatthī}, having gone there for the day’s meditation. A woman goatherd saw him … A woman gathering fire-wood saw him … A woman gathering cow-dung saw him and sat down on his penis. The monk consented to the entry, to having entered, to the remaining, and to the taking out. He became anxious … “You’ve committed an offense entailing expulsion.” 

At\marginnote{10.19.1} one time a monk was lying down in the Great Wood at \textsanskrit{Vesālī}, having gone there for the day’s meditation. A woman saw him and sat down on his penis. Having taken her pleasure, she stood laughing nearby. The monk woke up and said, “Did you do this?” 

“Yes.”\marginnote{10.19.5} 

He\marginnote{10.19.6} became anxious … 

“Did\marginnote{10.19.7} you consent?” 

“I\marginnote{10.19.8} didn’t even know, sir.” 

“There’s\marginnote{10.19.9} no offense for one who doesn’t know.” 

At\marginnote{10.20.1} one time a monk went to the Great Wood at \textsanskrit{Vesālī} for the day’s meditation. He lay down, resting his head against a tree. A woman saw him and sat down on his penis. The monk got up quickly. He became anxious … “Did you consent?” 

“I\marginnote{10.20.6} didn’t consent, sir.” 

“There’s\marginnote{10.20.7} no offense for one who doesn’t consent.” 

At\marginnote{10.20.8} one time a monk went to the Great Wood at \textsanskrit{Vesālī} for the day’s meditation. He lay down, resting his head against a tree. A woman saw him and sat down on his penis. The monk kicked her off. He became anxious … “Did you consent?” 

“I\marginnote{10.20.13} didn’t consent, sir.” 

“There’s\marginnote{10.20.14} no offense for one who doesn’t consent.” 

At\marginnote{10.21.1} one time a monk went to the hall with the peaked roof in the Great Wood near \textsanskrit{Vesālī} for the day’s meditation. He opened the door, lay down, and had an erection because of wind. Just then a number of women came to the monastery to look at the dwellings, bringing scents and garlands. They saw that monk and sat down on his penis. Having taken their pleasure, they said, “What a bull of a man.” They then put up their scents and garlands and left. The monks saw the moisture and told the Buddha. 

“Monks,\marginnote{10.21.6} an erection occurs for five reasons: because of sensual desire, feces, urine, or wind, or because of being stung by caterpillars. It’s impossible that that monk had an erection because of sensual desire. That monk is a perfected one. There’s no offense for that monk. 

\scrule{But, monks, you should close the door when you are in seclusion during the day.”\footnote{“Should” renders \textit{\textsanskrit{anujānāmi}}. See Appendix of Technical Terms for a discussion of this word. } }

At\marginnote{10.22.1} one time a monk from Bharukaccha dreamed that he had sexual intercourse with his ex-wife. He thought he was no longer a monastic and that he would have to disrobe.\footnote{“Disrobe” renders \textit{\textsanskrit{vibbhamissāmi}}. According to PED and SED the general meaning of this word is something like “to go astray”. However, the implied meaning throughout the Vinaya \textsanskrit{Piṭaka} is that one leaves the Sangha, that is, one disrobes. I therefore take this word to express the functional equivalence of disrobing. This is supported by the commentaries. Sp 1.435: \textit{\textsanskrit{Vibbhamantīti} ekacce \textsanskrit{gihī} honti}, “\textit{Vibbhamanti}: some became householders.” Sp 4.434: \textit{Yadeva \textsanskrit{sā} \textsanskrit{vibbhantāti} \textsanskrit{yasmā} \textsanskrit{sā} \textsanskrit{vibbhantā} attano \textsanskrit{ruciyā} \textsanskrit{khantiyā} \textsanskrit{odātāni} \textsanskrit{vatthāni} \textsanskrit{nivatthā}, \textsanskrit{tasmāyeva} \textsanskrit{sā} \textsanskrit{abhikkhunī}, na \textsanskrit{sikkhāpaccakkhānenāti} dasseti}, “\textit{Yadeva \textsanskrit{sā} \textsanskrit{vibbhantā}} means she is no longer a nun because, according to her own preference and choice, she dresses in white.” } While on his way to Bharukaccha, he saw Venerable \textsanskrit{Upāli} and told him what had happened. Venerable \textsanskrit{Upāli} said, “There’s no offense when it occurs while dreaming.” 

At\marginnote{10.23.1} one time in \textsanskrit{Rājagaha} there was a female lay follower called \textsanskrit{Supabbā} who had misplaced faith.\footnote{\textit{\textsanskrit{Mudhappasannā}} is not directly defined in the commentaries, but we do find indications of its meaning. Sp-\textsanskrit{ṭ} 1.165: \textit{Balavasaddho hi \textsanskrit{mandapañño} mudhappasanno hoti, \textsanskrit{avatthusmiṁ} \textsanskrit{pasīdati}}, “One who has strong faith and weak wisdom is called \textit{mudhappasanna}. Their confidence is without basis.” } She had the view that a woman who gives sexual intercourse gives the highest gift. She saw a monk and said, “Sir, come and have sexual intercourse.” 

“It’s\marginnote{10.23.6} not allowable.” 

“Then\marginnote{10.23.7} rub between the thighs. In this way there won’t be any offense for you. … Then rub against the navel. … Then rub against the stomach. … Then rub in the armpit. … Then rub against the throat. … Then rub against the ear-hole. … Then rub against a coil of hair. … Then rub between the fingers. … Then I’ll make you discharge with my hand. In this way there won’t be any offense for you.” The monk acted accordingly. He became anxious … “There’s no offense entailing expulsion, but there’s an offense entailing suspension.” 

At\marginnote{10.24.1} one time in \textsanskrit{Sāvatthī} there was a female lay follower called \textsanskrit{Saddhā} who had misplaced faith. She had the view that a woman who gives sexual intercourse gives the highest gift. She saw a monk and said, “Sir, come and have sexual intercourse.” 

“It’s\marginnote{10.24.6} not allowable.” 

“Then\marginnote{10.24.7} rub between the thighs. … Then I’ll make you discharge with my hand. In this way there won’t be any offense for you.” The monk acted accordingly. He became anxious … “There’s no offense entailing expulsion, but there’s an offense entailing suspension.” 

At\marginnote{10.25.1} one time in \textsanskrit{Vesālī} some \textsanskrit{Licchavī} youths grabbed a monk and made him commit misconduct with a nun. … made him commit misconduct with a trainee nun. … made him commit misconduct with a novice nun. Both consented: both should be expelled. Neither consented: there is no offense for either. 

At\marginnote{10.25.8} one time in \textsanskrit{Vesālī} some \textsanskrit{Licchavī} youths grabbed a monk and made him commit misconduct with a sex worker. … made him commit misconduct with a \textit{\textsanskrit{paṇḍaka}}. … made him commit misconduct with a female householder. The monk consented: he should be expelled. The monk did not consent: there is no offense. 

At\marginnote{10.25.15} one time in \textsanskrit{Vesālī} some \textsanskrit{Licchavī} youths grabbed two monks and made them commit misconduct with each other. Both consented: both should be expelled. Neither consented: there is no offense for either. 

At\marginnote{10.26.1} one time a monk who had gone forth in old age went to see his ex-wife. Saying, “Come and disrobe,” she grabbed him. Stepping backward, the monk fell on his back. She pulled up his robe and sat down on his penis. He became anxious … “Did you consent, monk?” 

“I\marginnote{10.26.7} didn’t consent, sir.” 

“There’s\marginnote{10.26.8} no offense for one who doesn’t consent.” 

At\marginnote{10.27.1} one time a certain monk was staying in the wilderness. A young deer came to his place of urination, drank the urine, and took hold of his penis with its mouth. The monk consented. He became anxious … “You’ve committed an offense entailing expulsion.” 

\scendsutta{The first offense entailing expulsion is finished. }

%
\section*{{\suttatitleacronym Bu Pj 2}{\suttatitletranslation The second training rule on expulsion }{\suttatitleroot Adinnādāna}}
\addcontentsline{toc}{section}{\tocacronym{Bu Pj 2} \toctranslation{The second training rule on expulsion } \tocroot{Adinnādāna}}
\markboth{The second training rule on expulsion }{Adinnādāna}
\extramarks{Bu Pj 2}{Bu Pj 2}

\subsection*{Origin story }

\subsubsection*{First sub-story }

At\marginnote{1.1.1} one time the Buddha was staying on the Vulture Peak at \textsanskrit{Rājagaha}. At that time a number of monks who were friends had made grass huts on the slope of Mount Isigili and had entered the rainy-season residence there. Among them was Venerable Dhaniya the potter. When the three months were over and they had completed the rainy-season residence, the monks demolished their grass huts, put away the grass and sticks, and left to wander the country. But Venerable Dhaniya spent the winter and the summer right there. 

Then,\marginnote{1.1.6} on one occasion, while Dhaniya was in the village to collect almsfood, some women gathering grass and firewood demolished his grass hut and took away the grass and sticks. A second time Dhaniya collected grass and sticks and made a grass hut, and again the hut was demolished in the same way. The same thing happened a third time. 

Dhaniya\marginnote{1.1.11} thought, “Three times this has happened. But I’m well-trained and experienced in my own craft of pottery. Why don’t I knead mud myself and make a hut entirely of clay?” 

And\marginnote{1.1.15} he did just that. He then collected grass, sticks, and cow-dung, and he baked his hut. It was a pretty and attractive little hut, red in color like a scarlet rain-mite. And when struck, it sounded just like a bell. 

Soon\marginnote{1.2.1} afterwards the Buddha was descending from the Vulture Peak with a number of monks when he saw that hut. He said to the monks, “What’s this pretty and attractive thing that’s red in color like a scarlet rain-mite?” The monks told the Buddha, who then rebuked Dhaniya: 

“It’s\marginnote{1.2.6} not suitable for that foolish man, it’s not proper, it’s not worthy of a monastic, it’s not allowable, it’s not to be done. How could he make a hut entirely of clay? Doesn’t he have any consideration, compassion, and mercy for living beings? Go, monks, and demolish this hut, so that future generations don’t follow his example. 

\scrule{And, monks, you shouldn’t make a hut entirely of clay. If you do, you commit an offense of wrong conduct.” }

Saying,\marginnote{1.2.13} “Yes, sir,” they went to demolish it. 

And\marginnote{1.2.14} Dhaniya said to them, “Why are you demolishing my hut?” 

“The\marginnote{1.2.16} Buddha has asked us to.” 

“Demolish\marginnote{1.2.17} it then, if the Lord of the Truth has said so.” 

Dhaniya\marginnote{1.3.1} thought, “Three times, while I was in the village to collect almsfood, women gathering grass and firewood demolished my hut and took away the grass and sticks. And now my hut made entirely of clay has been demolished at the Buddha’s request. Now, the caretaker of the woodyard is a friend of mine. Why don’t I ask him for timber and make a hut out of that?” 

Dhaniya\marginnote{1.3.6} then went to the caretaker of the woodyard and told him what had happened, adding, “Please give me some timber, I want to make a wooden hut.” 

“There’s\marginnote{1.3.11} no timber, sir, that I could give you. This timber is held by the king. It’s meant for repairs of the town and put aside in case of an emergency. You can only have it if the king gives it away.” 

“Actually,\marginnote{1.3.14} it’s been given by the king.” 

The\marginnote{1.3.15} caretaker of the woodyard thought, “These Sakyan monastics have integrity. They are celibate and their conduct is good, and they are truthful, moral, and have a good character. Even the king has faith in them. These venerables wouldn’t say something is given if it wasn’t.” And he said to Dhaniya, “You may take it, sir.” Dhaniya then had that timber cut into pieces, took it away on carts, and made a wooden hut. 

Soon\marginnote{1.4.1} afterwards the brahmin \textsanskrit{Vassakāra}, the chief minister of Magadha, was inspecting the public works in \textsanskrit{Rājagaha} when he went to the caretaker of the woodyard and said, “What’s going on? Where’s the timber held by the king that’s meant for repairs of the town and put aside in case of an emergency?” 

“The\marginnote{1.4.4} king has given it to Venerable Dhaniya.” 

\textsanskrit{Vassakāra}\marginnote{1.4.5} was upset and thought, “How could the king give away this timber to Dhaniya the potter?” 

He\marginnote{1.4.7} then went to King Seniya \textsanskrit{Bimbisāra} of Magadha and said, “Is it true, sir, that you have given away to Dhaniya the potter the timber that was held for repairs of the town and put aside in case of an emergency?” 

“Who\marginnote{1.4.9} said that?” 

“The\marginnote{1.4.10} caretaker of the woodyard.” 

“Well\marginnote{1.4.11} then, brahmin, summon the caretaker of the woodyard.” And \textsanskrit{Vassakāra} had the caretaker of the woodyard bound and taken by force. 

Dhaniya\marginnote{1.4.13} saw this and said to him, “Why is this happening to you?” 

“Because\marginnote{1.4.16} of the timber, sir.” 

“Go\marginnote{1.4.17} then, and I’ll come too.” 

“Please\marginnote{1.4.18} come before I’m done for.” 

Dhaniya\marginnote{1.5.1} then went to King \textsanskrit{Bimbisāra}’s house and sat down on the prepared seat. The king approached Dhaniya, bowed, sat down, and said, “Is it true, venerable, that I have given to you the timber held for repairs of the town and put aside in case of an emergency?” 

“Yes,\marginnote{1.5.5} great king.” 

“We\marginnote{1.5.6} kings are very busy—we may give and not remember. Please remind me.” 

“Do\marginnote{1.5.9} you remember, great king, when you were first anointed, speaking these words: ‘I give the grass, sticks, and water for the monastics and brahmins to enjoy’?” 

“I\marginnote{1.5.11} remember. There are monastics and brahmins who have a sense of conscience, who are afraid of wrongdoing and fond of the training. They are afraid of wrongdoing even in regard to small matters. When I spoke, I was referring to them, and it concerned what’s ownerless in the wilderness. Yet you imagine that you can take timber not given to you by means of this pretext? Even so, I cannot beat, imprison, or banish a monastic or brahmin living in my own kingdom. Go, you’re free because of your status, but don’t do such a thing again.” 

But\marginnote{1.6.1} people complained and criticized him: “These Sakyan monastics are shameless and immoral liars. They claim to have integrity, to be celibate and of good conduct, to be truthful, moral, and good. But they don’t have the good character of a monastic or a brahmin. They’ve lost the plot! They even deceive the king, never mind other people.” 

The\marginnote{1.6.9} monks heard the complaints of those people. The monks of few desires, who had a sense of conscience, and who were contented, afraid of wrongdoing, and fond of the training, complained and criticized Venerable Dhaniya, “How could he take the king’s timber without it being given to him?” 

After\marginnote{1.6.12} rebuking Dhaniya in many ways, they told the Buddha. The Buddha then had the Sangha of monks gathered and questioned Venerable Dhaniya: “Is it true, Dhaniya, that you did this?” 

“It’s\marginnote{1.6.15} true, sir.” 

The\marginnote{1.6.16} Buddha rebuked him, “Foolish man, it’s not suitable, it’s not proper, it’s not worthy of a monastic, it’s not allowable, it’s not to be done. How could you do this? This will affect people’s confidence and cause some to lose it.” 

Just\marginnote{1.6.21} then a former judge who had gone forth with the monks was sitting near the Buddha. The Buddha said to him, “For stealing how much does King \textsanskrit{Bimbisāra} beat, imprison, or banish a thief?” 

“For\marginnote{1.6.24} stealing a \textit{\textsanskrit{pāda}} coin, sir, or the value of a \textit{\textsanskrit{pāda}}.” At that time in \textsanskrit{Rājagaha} a \textit{\textsanskrit{pāda}} coin was worth five \textit{\textsanskrit{māsaka}} coins. 

After\marginnote{1.6.26} rebuking Venerable Dhaniya in many ways, the Buddha spoke in dispraise of being difficult to support … “And, monks, this training rule should be recited like this: 

\subsubsection*{Preliminary ruling }

\scrule{‘If a monk, intending to steal, takes what has not been given to him—the sort of stealing for which kings, having caught a thief, would beat, imprison, or banish him, saying, “You’re a bandit, you’re a fool, you’ve gone astray, you’re a thief”—he too is expelled and excluded from the community.’” }

In\marginnote{1.6.31} this way the Buddha laid down this training rule for the monks. 

\subsubsection*{Second sub-story }

At\marginnote{2.1} one time the monks from the group of six went to the dyers, stole their stock of cloth, brought it back to the monastery, and shared it out.\footnote{The text literally says “the dyers’ spread”, \textit{\textsanskrit{rajakattharaṇaṁ}}, but the commentary at Sp 1.90 qualifies that this refers to their spread of “cloth”, \textit{\textsanskrit{vatthāni}}. } The other monks said to them, “You have great merit, seeing that you’ve gotten so much robe-cloth.”\footnote{For the rendering of \textit{\textsanskrit{cīvara}} as “robe-cloth”, see Appendix of Technical Terms. } 

“How\marginnote{2.5} is it that we have merit? Just now we went to the dyers and stole their cloth.” 

“But\marginnote{2.7} hasn’t the Buddha laid down a training rule? Why then do you steal the dyers’ cloth?” 

“It’s\marginnote{2.9} true that the Buddha has laid down a training rule, but it concerns inhabited areas, not the wilderness.” 

“But\marginnote{2.11} that’s just the same. It’s not suitable, it’s not proper, it’s not worthy of a monastic, it’s not allowable, it’s not to be done. How could you steal the dyers’ cloth? This will affect people’s confidence, and cause some to lose it.” 

After\marginnote{2.16} rebuking those monks in many ways, they told the Buddha. 

The\marginnote{2.17} Buddha had the Sangha of monks gathered and questioned those monks: “Is it true, monks, that you did this?” 

“It’s\marginnote{2.19} true, sir.” 

The\marginnote{2.20} Buddha rebuked them, “It’s not suitable, foolish men, it’s not proper, it’s not worthy of a monastic, it’s not allowable, it’s not to be done. How could you do this? This will affect people’s confidence, and cause some to lose it.” Then, after rebuking the monks from the group of six in many ways, the Buddha spoke in dispraise of being difficult to support … but he spoke in praise of … being energetic. Having given a teaching on what is right and proper, he addressed the monks … “And so, monks, this training rule should be recited like this: 

\subsection*{Final ruling }

\scrule{‘If a monk, intending to steal, takes from an inhabited area or from the wilderness what has not been given to him—the sort of stealing for which kings, having caught a thief, would beat, imprison, or banish him, saying, “You’re a bandit, you’re a fool, you’ve gone astray, you’re a thief”—he too is expelled and excluded from the community.’” }

\subsection*{Definitions }

\begin{description}%
\item[A: ] whoever … %
\item[Monk: ] … The monk who has been given the full ordination by a unanimous Sangha through a legal procedure consisting of one motion and three announcements that is irreversible and fit to stand—this sort of monk is meant in this case. %
\item[An inhabited area: ] an inhabited area of one hut, an inhabited area of two huts, an inhabited area of three huts, an inhabited area of four huts, an inhabited area with people, an inhabited area without people, an enclosed inhabited area, an unenclosed inhabited area, a disorganized inhabited area, and even a caravan settled for more than four months is called “an inhabited area”.\footnote{“Inhabited area” renders \textit{\textsanskrit{gāma}}. See Appendix of Technical Terms for discussion. } %
\item[The access to an inhabited area: ] of an enclosed inhabited area: a stone’s throw of a man of average height standing at the threshold of the gateway to the inhabited area; of an unenclosed inhabited area: a stone’s throw of a man of average height standing at the access to a house. %
\item[The wilderness: ] apart from inhabited areas and the access to inhabited areas, the remainder is called “the wilderness”.\footnote{“Access” renders \textit{\textsanskrit{upacāra}}. See Appendix of Technical Terms for discussion. } %
\item[What has not been given: ] what has not been given, what has not been let go of, what has not been relinquished; what is guarded, what is protected, what is regarded as “mine”, what belongs to someone else. This is called “what has not been given”. %
\item[Intending to steal: ] the thought of theft, the thought of stealing. %
\item[Takes: ] takes, carries off, steals, interrupts the movement of, moves from its base, does not stick to an arrangement. %
\item[The sort: ] a \textit{\textsanskrit{pāda}} coin, the value of a \textit{\textsanskrit{pāda}}, or more than a \textit{\textsanskrit{pāda}}. %
\item[Kings: ] kings of the earth, kings of a region, rulers of islands, rulers of border areas, judges, government officials, or whoever metes out physical punishment—these are called “kings”. %
\item[A thief: ] whoever, intending to steal, takes anything that has not been given, having a value of five \textit{\textsanskrit{māsaka}} coins or more—he is called “a thief”. %
\item[Would beat: ] would beat with the hand, the foot, a whip, a cane, a cudgel, or by mutilation. %
\item[Would imprison: ] would imprison by constricting with a rope, by constricting with shackles, by constricting with chains, by constricting to a house, by constricting to a city, by constricting to a village, by constricting to a town, or by guarding. %
\item[Would banish: ] would banish from a village, from a town, from a city, from a country, or from a district. %
\item[You’re a bandit, you’re a fool, you’ve gone astray, you’re a thief: ] this is a rebuke.\footnote{For the following segments, because I have not tried to replicate the structure of the Pali—in this case the subordinate/demonstrative clause structure and the resulting repetitiveness—the next two terms defined in the Pali have no counterpart in my translation. But no information is lost since both terms have already been defined earlier. The two terms in question are \textit{\textsanskrit{tathārūpaṁ}}, which is defined in the same way as \textit{\textsanskrit{yathārūpaṁ}}, “the sort”, and secondly, \textit{\textsanskrit{ādiyamāno}}, which is defined in the same way as \textit{\textsanskrit{ādiyeyya}}, “takes”. } %
\item[He too: ] this is said with reference to the preceding offense entailing expulsion. %
\item[Is expelled: ] just as a fallen, withered leaf is incapable of becoming green again, so too is a monk who, intending to steal, takes an ungiven \textit{\textsanskrit{pāda}} coin, the value of a \textit{\textsanskrit{pāda}}, or more than a \textit{\textsanskrit{pāda}}, not an ascetic, not a Sakyan monastic. Therefore it is said, “he is expelled.” %
\item[Excluded from the community: ] Community: joint legal procedures, a joint recitation, the same training—this is called “community”. He does not take part in this—therefore it is called “excluded from the community”. %
\end{description}

\subsection*{Permutations }

\subsubsection*{Permutations part 1 }

\paragraph*{Summary }

Being\marginnote{4.1.1} underground, being on the ground, being in the air, being above ground, being in water, being in a boat, being in a vehicle, carried as a load, being in a park, being in a monastic dwelling, being in a field, being on a site, being in an inhabited area, being in the wilderness, water, tooth cleaner, forest tree, that which is carried, that which is deposited, customs station, a living being, footless, two-footed, four-footed, many-footed, a spy, a keeper of entrusted property, mutually agreed stealing, acting by arrangement, making a sign. 

\paragraph*{Exposition }

\begin{description}%
\item[Being underground: ] the\marginnote{4.2.2} goods have been placed underground, buried, concealed. If, intending to steal, he thinks, “I’ll steal the underground goods,” and he seeks for a companion, seeks for a spade or a basket, or goes there, he commits an offense of wrong conduct.\footnote{“Goes there” renders \textit{gacchati}. Sp 1.94: \textit{Gacchati \textsanskrit{vā} \textsanskrit{āpatti} \textsanskrit{dukkaṭassāti} \textsanskrit{evaṁ} \textsanskrit{pariyiṭṭhasahāyakudālapiṭako} \textsanskrit{nidhiṭṭhānaṁ} gacchati}, “\textit{Gacchati \textsanskrit{vā} \textsanskrit{āpatti} \textsanskrit{dukkaṭassa}}: in this way he goes with the sought friend, with the spade and basket, to the place of the goods.” } If he breaks a twig or a creeper growing there, he commits an offense of wrong conduct. If he digs the soil or heaps it up or removes it, he commits an offense of wrong conduct. If he touches the container, he commits an offense of wrong conduct.\footnote{“Container” renders \textit{kumbhi}, which actually is a pot or cooking vessel. In the present context, however, the pot is used as a container for goods. } If he makes it stir, he commits a serious offense. If he moves it from its base, he commits an offense entailing expulsion. 

If,\marginnote{4.2.9} intending to steal, he puts his own vessel into the container and touches something worth five \textit{\textsanskrit{māsaka}} coins or more, he commits an offense of wrong conduct. If he makes it stir, he commits a serious offense. If he makes it enter his own vessel or takes it with his fist, there is an offense entailing expulsion. 

If,\marginnote{4.2.12} intending to steal, he touches goods made of string—an ornamental hanging string, a necklace, an ornamental girdle, a wrap garment, or a turban—he commits an offense of wrong conduct. If he makes it stir, he commits a serious offense. If he grasps it at the top and pulls it, he commits a serious offense. If he rubs it while lifting it, he commits a serious offense. If he removes the goods even as much as a hair’s breadth over the rim of the container, he commits an offense entailing expulsion. 

If,\marginnote{4.2.17} intending to steal, he drinks—in a single action—ghee, oil, honey, or syrup having a value of five \textit{\textsanskrit{māsaka}} coins or more, he commits an offense entailing expulsion.\footnote{“Syrup” renders \textit{\textsanskrit{phāṇita}}. I. B. Horner instead translates it as “molasses”, which doesn’t quite hit the mark. SED defines \textit{\textsanskrit{phāṇita}} as “the inspissated juice of the sugar cane or other plants”, in other words, “cane syrup”. According to the commentary at Sp 1.623 it can be either cooked or uncooked, the difference presumably whether it is raw or concentrated. “Syrup” seems closer to the mark than “molasses”. } If he destroys it, throws it away, burns it, or renders it useless, he commits an offense of wrong conduct. 

%
\item[Being on the ground: ] the goods have been placed on the ground. If, intending to steal, he thinks, “I’ll steal the goods on the ground,” and he either searches for a companion or goes there, he commits an offense of wrong conduct. If he touches them, he commits an offense of wrong conduct. If he makes them stir, he commits a serious offense. If he moves them from their base, he commits an offense entailing expulsion. %
\item[Being in the air: ] the goods are in the air—a peacock, a partridge, or a quail; or a wrap garment or a turban; or money or gold that falls after being cut loose.\footnote{For a discussion of \textit{\textsanskrit{hirañña}}, see Appendix of Technical Terms. } If, intending to steal, he thinks, “I’ll steal the goods in the air,” and he either searches for a companion or goes there, he commits an offense of wrong conduct. If he cuts off their course of movement, he commits an offense of wrong conduct. If he touches them, he commits an offense of wrong conduct. If he makes them stir, he commits a serious offense. If he moves them from their base, he commits an offense entailing expulsion. %
\item[Being above ground: ] the goods are above ground—on a bed, on a bench, on a bamboo robe rack, on a clothesline, on a wall peg, in a tree, or even just on a bowl rest.\footnote{\textit{Bhittikhilepi \textsanskrit{nāgadantakepi}}, literally, “wall pegs and elephant tusks”. These are different kinds of pegs and I have not tried to differentiate between them. } If, intending to steal, he thinks, “I’ll steal the goods that are above ground,” and he either searches for a companion or goes there, he commits an offense of wrong conduct. If he touches them, he commits an offense of wrong conduct. If he makes them stir, he commits a serious offense. If he moves them from their base, he commits an offense entailing expulsion. %
\item[Being in the water: ] the goods have been placed in water. If,\marginnote{4.6.2} intending to steal, he thinks, “I’ll steal the goods in the water,” and he either searches for a companion or goes there, he commits an offense of wrong conduct. If he either dives into the water or floats on the surface, he commits an offense of wrong conduct. If he touches the goods, he commits an offense of wrong conduct. If he makes them stir, he commits a serious offense. If he moves them from their base, he commits an offense entailing expulsion. 

If,\marginnote{4.6.8} intending to steal, he touches either a blue, red, or white lotus growing there, or a lotus root, or a fish, or a turtle having a value of five \textit{\textsanskrit{māsaka}} coins or more, he commits an offense of wrong conduct. If he makes them stir, he commits a serious offense. If he moves them from their base, he commits an offense entailing expulsion. 

%
\item[A boat: ] that by means of which one crosses. %
\item[Being in a boat: ] the\marginnote{4.7.4} goods have been placed in a boat. If, intending to steal, he thinks, “I’ll steal the goods in the boat,” and he either searches for a companion or goes there, he commits an offense of wrong conduct. If he touches them, he commits an offense of wrong conduct. If he makes them stir, he commits a serious offense. If he moves them from their base, he commits an offense entailing expulsion. 

If,\marginnote{4.7.9} intending to steal, he thinks, “I’ll steal the boat,” and he either searches for a companion or goes there, he commits an offense of wrong conduct. If he touches it, he commits an offense of wrong conduct. If he makes it stir, he commits a serious offense. If he loosens the moorings, he commits an offense of wrong conduct. If, after loosening the moorings, he touches it, he commits an offense of wrong conduct. If he makes it stir, he commits a serious offense. If he makes it move upstream or downstream or across the water, even as much as a hair’s breadth, he commits an offense entailing expulsion. 

%
\item[A vehicle: ] a wagon, a carriage, a cart, a chariot. %
\item[Being in a vehicle: ] the\marginnote{4.8.4} goods have been placed in a vehicle. If, intending to steal, he thinks, “I’ll steal the goods in the vehicle,” and he either searches for a companion or goes there, he commits an offense of wrong conduct. If he touches them, he commits an offense of wrong conduct. If he makes them stir, he commits a serious offense. If he moves them from their base, he commits an offense entailing expulsion. 

If,\marginnote{4.8.9} intending to steal, he thinks, “I’ll steal the vehicle,” and he either searches for a companion or goes there, he commits an offense of wrong conduct. If he touches it, he commits an offense of wrong conduct. If he makes it stir, he commits a serious offense. If he moves it from its base, he commits an offense entailing expulsion. 

%
\item[A load: ] a load carried on the head, a load carried on the shoulder, a load carried on the hip, one hanging down. If, intending to steal, he touches the load on the head, he commits an offense of wrong conduct. If he makes it stir, he commits a serious offense. If he lowers it to the shoulder, he commits an offense entailing expulsion. If, intending to steal, he touches the load on the shoulder, he commits an offense of wrong conduct. If he makes it stir, he commits a serious offense. If he lowers it to the hip, he commits an offense entailing expulsion. If, intending to steal, he touches the load on the hip, he commits an offense of wrong conduct. If he causes it to stir, he commits a serious offense. If he takes it with the hand, there is an offense entailing expulsion. If, intending to steal a load in the hand, he places it on the ground, he commits an offense entailing expulsion. If, intending to steal, he picks it up from the ground, he commits an offense entailing expulsion. %
\item[A park: ] a garden, an orchard.\footnote{Since \textit{\textsanskrit{ārāma}} is a standard term for a monastery in the Vinaya \textsanskrit{Piṭaka}, monasteries are presumably included under this heading. } %
\item[Being in a park: ] the\marginnote{4.10.4} goods have been placed in a park in four locations: underground, on the ground, in the air, above the ground. If, intending to steal, he thinks, “I’ll steal the goods in the park,” and he either searches for a companion or goes there, he commits an offense of wrong conduct. If he touches them, he commits an offense of wrong conduct. If he makes them stir, he commits a serious offense. If he moves them from their base, he commits an offense entailing expulsion. 

If,\marginnote{4.10.9} intending to steal, he touches something growing there—a root, a piece of bark, a leaf, a flower, or a fruit—having a value of five \textit{\textsanskrit{māsaka}} coins or more, he commits an offense of wrong conduct. If he makes it stir, he commits a serious offense. If he moves it from its base, he commits an offense entailing expulsion. 

If\marginnote{4.10.12} he claims the park, he commits an offense of wrong conduct. If he evokes doubt in the owner as to his ownership, he commits a serious offense. If the owner thinks, “I won’t get it back,” and he gives up the effort of reclaiming it, he commits an offense entailing expulsion. If he resorts to the law and defeats the owner, he commits an offense entailing expulsion.\footnote{Sp 1.102: \textit{\textsanskrit{Sāmikaṁ} \textsanskrit{parājetīti} \textsanskrit{vinicchayikānaṁ} \textsanskrit{ukkocaṁ} \textsanskrit{datvā} \textsanskrit{kūṭasakkhiṁ} \textsanskrit{otāretvā} \textsanskrit{ārāmasāmikaṁ} \textsanskrit{jinātīti} attho}, “‘He defeats the owner’: having given a bribe to those deciding (the legal case), having brought a false witness, he defeats the owner of the park. This is the meaning.” In other words, the lawsuit is illegitimate. } If he resorts to the law but is defeated, he commits a serious offense. 

%
\item[Being in a monastic dwelling: ] the\marginnote{4.11.2} goods have been placed in a monastic dwelling in four locations: underground, on the ground, in the air, above the ground.\footnote{I render \textit{\textsanskrit{vihāra}} as “monastic dwelling”. In later usage, especially in the commentaries, \textit{\textsanskrit{vihāra}} comes to refer to entire monasteries, rather than individual dwellings. The commentaries seem to agree that in its early usage the word refers to a dwelling. Sp 1.493: \textit{\textsanskrit{Vihāro} nivesanasadiso}, “A \textit{\textsanskrit{vihāra}} is like a house.” } If, intending to steal, he thinks, “I’ll steal the goods in the monastic dwelling,” and he either searches for a companion or goes there, he commits an offense of wrong conduct. If he touches them, he commits an offense of wrong conduct. If he makes them stir, he commits a serious offense. If he moves them from their base, he commits an offense entailing expulsion. 

If\marginnote{4.11.7} he claims the monastic dwelling, he commits an offense of wrong conduct. If he evokes doubt in the owner as to his ownership, he commits a serious offense. If the owner thinks, “I won’t get it back,” and he gives up the effort of reclaiming it, he commits an offense entailing expulsion. If he resorts to the law and defeats the owner, he commits an offense entailing expulsion. If he resorts to the law but is defeated, he commits a serious offense. 

%
\item[A field: ] where grain or vegetables grow. %
\item[Being in a field: ] the\marginnote{4.12.4} goods have been placed in a field in four locations: underground, on the ground, in the air, above the ground. If, intending to steal, he thinks, “I’ll steal the goods in the field,” and he either searches for a companion or goes there, he commits an offense of wrong conduct. If he touches them, he commits an offense of wrong conduct. If he makes them stir, he commits a serious offense. If he moves them from their base, he commits an offense entailing expulsion. 

If,\marginnote{4.12.9} intending to steal, he touches grain or vegetables that grow there, having a value of five \textit{\textsanskrit{māsaka}} coins or more, he commits an offense of wrong conduct. If he makes them stir, he commits a serious offense. If he moves them from their base, he commits an offense entailing expulsion. 

If\marginnote{4.12.12} he claims the field, he commits an offense of wrong conduct. If he evokes doubt in the owner as to his ownership, he commits a serious offense. If the owner thinks, “I won’t get it back,” and he gives up the effort of reclaiming it, he commits an offense entailing expulsion. If he resorts to the law and defeats the owner, he commits an offense entailing expulsion. If he resorts to the law but is defeated, he commits a serious offense. 

If\marginnote{4.12.17} he moves a post, a rope, a fence, or a boundary, he commits an offense of wrong conduct. When one action of the moving remains, he commits a serious offense. When the last action of the moving is completed, he commits an offense entailing expulsion. 

%
\item[A site: ] the site of a park or a monastery, the site of a monastic dwelling.\footnote{Again, \textit{\textsanskrit{ārāma}} means both park and monastery. } %
\item[Being on a site: ] the\marginnote{4.13.4} goods have been placed on a site in four locations: underground, on the ground, in the air, above the ground. If, intending to steal, he thinks, “I’ll steal the goods on the site,” and he either searches for a companion or goes there, he commits an offense of wrong conduct. If he touches them, he commits an offense of wrong conduct. If he makes them stir, he commits a serious offense. If he moves them from their base, he commits an offense entailing expulsion. 

If\marginnote{4.13.9} he claims the site, he commits an offense of wrong conduct. If he evokes doubt in the owner as to his ownership, he commits a serious offense. If the owner thinks, “I won’t get it back,” and he gives up the effort of reclaiming it, he commits an offense entailing expulsion. If he resorts to the law and defeats the owner, he commits an offense entailing expulsion. If he resorts to the law but is defeated, he commits a serious offense. 

If\marginnote{4.13.14} he moves a post, a rope, a fence, or a boundary, he commits an offense of wrong conduct. When one action of the moving remains, he commits a serious offense. When the last action of the moving is completed, he commits an offense entailing expulsion. 

%
\item[Being in an inhabited area: ] the goods have been placed in an inhabited area in four locations: underground, on the ground, in the air, above the ground. If, intending to steal, he thinks, “I’ll steal the goods in the inhabited area,” and he either searches for a companion or goes there, he commits an offense of wrong conduct. If he touches them, he commits an offense of wrong conduct. If he makes them stir, he commits a serious offense. If he moves them from their base, he commits an offense entailing expulsion. %
\item[The wilderness: ] any wilderness which is owned by people. %
\item[Being in the wilderness: ] the\marginnote{4.15.4} goods have been placed in the wilderness in four locations: underground, on the ground, in the air, above the ground. If, intending to steal, he thinks, “I’ll steal the goods in the wilderness,” and he either searches for a companion or goes there, he commits an offense of wrong conduct. If he touches them, he commits an offense of wrong conduct. If he makes them stir, he commits a serious offense. If he moves them from their base, he commits an offense entailing expulsion. 

If,\marginnote{4.15.9} intending to steal, he touches something that belongs there—a twig, a creeper, or grass—having a value of five \textit{\textsanskrit{māsaka}} coins or more, he commits an offense of wrong conduct. If he makes it stir, he commits a serious offense. If he moves it from its base, he commits an offense entailing expulsion. 

%
\item[Water: ] in\marginnote{4.16.2} a vessel, in a pond, or in a reservoir. If, intending to steal, he touches it, he commits an offense of wrong conduct. If he makes it stir, he commits a serious offense. If he moves it from its base, he commits an offense entailing expulsion. 

If,\marginnote{4.16.6} intending to steal, he puts his own vessel into the container holding the water, and he touches water having a value of five \textit{\textsanskrit{māsaka}} coins or more, he commits an offense of wrong conduct.\footnote{The Pali text just says “into”, and I have added “the container holding the water” for clarity. } If he makes it stir, he commits a serious offense. If he puts it into his own vessel, he commits an offense entailing expulsion. 

If\marginnote{4.16.9} he breaks the embankment, he commits an offense of wrong conduct. If, after breaking the embankment, he allows water to escape that has a value of five \textit{\textsanskrit{māsaka}} coins or more, he commits an offense entailing expulsion. If he allows water to escape that has a value of more than one \textit{\textsanskrit{māsaka}} but less than five \textit{\textsanskrit{māsakas}}, he commits a serious offense. If he allows water to escape that has a value of one \textit{\textsanskrit{māsaka}} or less, he commits an offense of wrong conduct. 

%
\item[Tooth cleaner: ] either ready for use or not. If, intending to steal, he touches what has a value of five \textit{\textsanskrit{māsaka}} coins or more, he commits an offense of wrong conduct. If he makes it stir, he commits a serious offense. If he moves it from its base, he commits an offense entailing expulsion. %
\item[Forest tree: ] whatever useful tree is owned by people. If, intending to steal, he fells it, then for each blow he commits an offense of wrong conduct. When one blow remains before the tree is felled, he commits a serious offense. When the last blow is completed, he commits an offense entailing expulsion. %
\item[Goods being carried: ] the\marginnote{4.19.2} goods of another are being carried. If, intending to steal, he touches them, he commits an offense of wrong conduct. If he makes them stir, he commits a serious offense. If he moves them from their base, he commits an offense entailing expulsion. 

If\marginnote{4.19.6} he thinks, “Together with the carrier I’ll carry off the goods,” and he makes the carrier move one foot, he commits a serious offense. If he makes him move the second foot, he commits an offense entailing expulsion. 

If\marginnote{4.19.8} he thinks, “I’ll take the fallen goods,” and he makes them fall, he commits an offense of wrong conduct. If, intending to steal, he touches fallen goods having a value of five \textit{\textsanskrit{māsaka}} coins or more, he commits an offense of wrong conduct. If he makes them stir, he commits a serious offense. If he moves them from their base, he commits an offense entailing expulsion. 

%
\item[Deposit: ] goods deposited with a monk. If the monk is told, “Give me my goods,” and he says, “I won’t get them for you,” he commits an offense of wrong conduct. If he evokes doubt in the mind of the owner as to whether he will get them back, he commits a serious offense.\footnote{The Pali text just says that he evokes doubt in the mind of the owner, \textit{\textsanskrit{sāmikassa} \textsanskrit{vimatiṁ} \textsanskrit{uppādeti}}. That the doubt refers to the possibility of getting the goods back is supplied from the commentary. Sp 1.112: \textit{“Dassati nu kho me no”ti \textsanskrit{sāmiko} \textsanskrit{vimatiṁ} \textsanskrit{uppādeti}}, “He evokes doubt in the mind of the owner in this way, ‘Will he give it or not?’” } If the owner thinks, “He won’t give them to me,” and he gives up the effort of getting them back, he commits an offense entailing expulsion. If he resorts to the law and defeats the owner, he commits an offense entailing expulsion. If he resorts to the law but is defeated, he commits a serious offense. %
\item[Customs station: ] it\marginnote{4.21.2} is established by a king in a mountain-pass, at a ford in a river, or at the gateway of a village so that tax can be collected from any person passing through. If, intending to steal and having entered the customs station, he touches goods that have a tax value to the king of five \textit{\textsanskrit{māsaka}} coins or more, he commits an offense of wrong conduct.\footnote{Although not explicitly stated in the Canonical text, the commentary confirms that the value meant is the tax value. Sp 1.113: \textit{\textsanskrit{Rājaggaṁ} \textsanskrit{bhaṇḍanti}: \textsanskrit{rājārahaṁ} \textsanskrit{bhaṇḍaṁ}; yato \textsanskrit{rañño} \textsanskrit{pañcamāsakaṁ} \textsanskrit{vā} \textsanskrit{atirekapañcamāsakaṁ} \textsanskrit{vā} \textsanskrit{agghanakaṁ} \textsanskrit{suṅkaṁ} \textsanskrit{dātabbaṁ} hoti, \textsanskrit{taṁ} \textsanskrit{bhaṇḍanti} attho}, “Goods having a value to the king means: goods having a worth to the king. The meaning is: the goods for which a tax having a value of five \textit{\textsanskrit{māsaka}} coins or more is to be given to the king.” } If he makes them stir, he commits a serious offense. If he goes beyond the customs station with one foot, he commits a serious offense. If he goes beyond the customs station with the second foot, he commits an offense entailing expulsion. 

If,\marginnote{4.21.8} standing within the customs station, he makes the goods fall outside the customs station, he commits an offense entailing expulsion. 

If\marginnote{4.21.9} he avoids the customs station altogether, he commits an offense of wrong conduct. 

%
\item[Creature: ] a\marginnote{4.22.2} human being is what is meant. If, intending to steal, he touches the person, he commits an offense of wrong conduct. If he makes the person stir, he commits a serious offense. If he moves the person from their base, he commits an offense entailing expulsion. 

If\marginnote{4.22.6} he thinks, “I’ll take the person away on foot,” and he makes them move the first foot, he commits a serious offense. If he makes them move the second foot, he commits an offense entailing expulsion. 

%
\item[Footless: ] snakes and fish. If, intending to steal, he touches what has a value of five \textit{\textsanskrit{māsaka}} coins or more, he commits an offense of wrong conduct. If he makes it stir, he commits a serious offense. If he moves it from its base, he commits an offense entailing expulsion. %
\item[Two-footed: ] humans\marginnote{4.24.2} and birds. If, intending to steal, he touches it, he commits an offense of wrong conduct. If he makes it stir, he commits a serious offense. If he moves it from its base, he commits an offense entailing expulsion. 

If\marginnote{4.24.6} he thinks, “I’ll take it away on foot,” and he makes it move the first foot, he commits a serious offense. If he makes it move the second foot, he commits an offense entailing expulsion. 

%
\item[Four-footed: ] elephants,\marginnote{4.25.2} horses, camels, cattle, donkeys, domesticated animals. If, intending to steal, he touches it, he commits an offense of wrong conduct. If he makes it stir, he commits a serious offense. If he moves it from its base, he commits an offense entailing expulsion. 

If\marginnote{4.25.6} he thinks, “I’ll take it away on foot,” and he makes it move the first foot, he commits a serious offense. If he makes it move the second foot, he commits a serious offense. If he makes it move the third foot, he commits a serious offense. If he makes it move the fourth foot, he commits an offense entailing expulsion. 

%
\item[Many-footed: ] scorpions,\marginnote{4.26.2} centipedes, caterpillars. If, intending to steal, he touches what has a value of five \textit{\textsanskrit{māsaka}} coins or more, he commits an offense of wrong conduct. If he makes it stir, he commits a serious offense. If he moves it from its base, he commits an offense entailing expulsion. 

If\marginnote{4.26.6} he thinks, “I’ll take it away on foot,” and he makes it move, he commits a serious offense for each leg that moves. When the last leg moves, he commits an offense entailing expulsion. 

%
\item[A spy: ] having spied out goods. If he describes them, saying, “Steal such-and-such goods,” he commits an offense of wrong conduct. If he steals those goods, there is an offense entailing expulsion for both. %
\item[A protector of goods: ] a monk who guards goods that have been brought to him. If, intending to steal, he touches what has a value of five \textit{\textsanskrit{māsaka}} coins or more, he commits an offense of wrong conduct. If he makes them stir, he commits a serious offense. If he moves them from their base, he commits an offense entailing expulsion. %
\item[Mutually agreed stealing: ] a number have agreed together. If only one steals the goods, there is an offense entailing expulsion for all of them. %
\item[Acting by arrangement: ] one makes an arrangement for before the meal or for after the meal, for the night or for the day. If he says, “Steal those goods according to this arrangement,” he commits an offense of wrong conduct. If the other steals those goods according to that arrangement, there is an offense entailing expulsion for both. If he steals those goods before or after the time of the arrangement, there is no offense for the instigator, but an offense entailing expulsion for the thief. %
\item[Making a sign: ] he makes a sign. If he says, “When I wink, at that sign steal the goods,” or, “When I raise an eyebrow, at that sign steal the goods,” or, “When I nod, at that sign steal the goods,” he commits an offense of wrong conduct. If, at that sign, the other steals the goods, there is an offense entailing expulsion for both. If he steals the goods before or after the sign, there is no offense for the instigator, but an offense entailing expulsion for the thief. %
\end{description}

\subsubsection*{Permutations part 2 }

If\marginnote{5.1.1} a monk tells a monk, “Steal such-and-such goods,” he commits an offense of wrong conduct. If the other monk steals them, thinking they are the ones he was told to steal, there is an offense entailing expulsion for both. 

If\marginnote{5.1.4} a monk tells a monk, “Steal such-and-such goods,” he commits an offense of wrong conduct. If the other monk steals other goods, thinking they are the ones he was told to steal, there is no offense for the instigator, but there is an offense entailing expulsion for the thief. 

If\marginnote{5.1.9} a monk tells a monk, “Steal such-and-such goods,” he commits an offense of wrong conduct. If the other monk steals them, thinking they are other than what he was told to steal, there is an offense entailing expulsion for both. 

If\marginnote{5.1.13} a monk tells a monk, “Steal such-and-such goods,” he commits an offense of wrong conduct. If the other monk steals other goods, thinking they are other than what he was told to steal, there is no offense for the instigator, but there is an offense entailing expulsion for the thief. 

If\marginnote{5.2.1} a monk tells a monk, “Tell so-and-so to tell so-and-so to steal such-and-such goods,” he commits an offense of wrong conduct. In telling the next person, there is an offense of wrong conduct. If the potential thief agrees, there is a serious offense for the instigator. If he steals those goods, there is an offense entailing expulsion for all of them. 

If\marginnote{5.2.7} a monk tells a monk, “Tell so-and-so to tell so-and-so to steal such-and-such goods,” he commits an offense of wrong conduct. If the other monk tells someone else than the one he was told to tell, he commits an offense of wrong conduct. If the potential thief agrees, there is an offense of wrong conduct. If he steals those goods, there is no offense for the instigator, but there is an offense entailing expulsion for the messenger and for the thief. 

If\marginnote{5.3.1} a monk tells a second monk, “Steal such-and-such goods,” he commits an offense of wrong conduct. He goes, but returns, saying, “I’m not able to steal those goods.” If the first monk tells him again, “When you’re able, then steal those goods,” he commits an offense of wrong conduct. If the second monk steals the goods, there is an offense entailing expulsion for both. 

If\marginnote{5.4.1} a monk tells a second monk, “Steal such-and-such goods,” he commits an offense of wrong conduct. He then regrets it, but does not say, “Don’t steal them.” If the second monk then steals those goods, there is an offense entailing expulsion for both. 

If\marginnote{5.4.8} a monk tells a second monk, “Steal such-and-such goods,” he commits an offense of wrong conduct. He then regrets it and says, “Don’t steal them.” If the second monk replies, “I’ve been told by you to do so,” and he then steals those goods, there is no offense for the instigator, but an offense entailing expulsion for the thief. 

If\marginnote{5.4.16} a monk tells a second monk, “Steal such-and-such goods,” he commits an offense of wrong conduct. He then regrets it and says, “Don’t steal them.” If the second monk replies, “Fine,” and desists, there is no offense for either. 

\subsubsection*{Permutations part 3 }

For\marginnote{6.1.1} one who steals there is an offense entailing expulsion when five factors are fulfilled: it is the possession of another; he perceives it as such; it is a valuable possession worth five \textit{\textsanskrit{māsaka}} coins or more; he has the intention to steal it; if he touches it, he commits an offense of wrong conduct; if he makes it stir, he commits a serious offense; if he moves it from its base, he commits an offense entailing expulsion. 

For\marginnote{6.1.9} one who steals there is a serious offense when five factors are fulfilled: it is the possession of another; he perceives it as such; it is an ordinary possession worth more than one \textit{\textsanskrit{māsaka}} coin, but less than five; he has the intention to steal it; if he touches it, he commits an offense of wrong conduct; if he makes it stir, he commits an offense of wrong conduct; if he moves it from its base, he commits a serious offense. 

For\marginnote{6.1.17} one who steals there is an offense of wrong conduct when five factors are fulfilled: it is the possession of another; he perceives it as such; it is an ordinary possession worth one \textit{\textsanskrit{māsaka}} coin or less; he has the intention to steal it; if he touches it, he commits an offense of wrong conduct; if he makes it stir, he commits an offense of wrong conduct; if he moves it from its base, he commits an offense of wrong conduct. 

For\marginnote{6.2.1} one who steals there is an offense entailing expulsion when six factors are fulfilled: he does not perceive it as his own; he does not take it on trust; he does not borrow it; it is a valuable possession worth five \textit{\textsanskrit{māsaka}} coins or more; he has the intention to steal it; if he touches it, he commits an offense of wrong conduct; if he makes it stir, he commits a serious offense; if he moves it from its base, he commits an offense entailing expulsion. 

For\marginnote{6.2.10} one who steals there is a serious offense when six factors are fulfilled: he does not perceive it as his own; he does not take it on trust; he does not borrow it; it is an ordinary possession worth more than one \textit{\textsanskrit{māsaka}} coin, but less than five; he has the intention to steal it; if he touches it, he commits an offense of wrong conduct; if he makes it stir, he commits an offense of wrong conduct; if he moves it from its base, he commits a serious offense. 

For\marginnote{6.2.19} one who steals there is an offense of wrong conduct when six factors are fulfilled: he does not perceive it as his own; he does not take it on trust; he does not borrow it; it is an ordinary possession worth one \textit{\textsanskrit{māsaka}} coin or less; he has the intention to steal it; if he touches it, he commits an offense of wrong conduct; if he makes it stir, he commits an offense of wrong conduct; if he moves it from its base, he commits an offense of wrong conduct. 

For\marginnote{6.3.1} one who steals there is an offense of wrong conduct when five factors are fulfilled: it is not the possession of another; but he perceives it as the possession of another; it is a valuable possession worth five \textit{\textsanskrit{māsaka}} coins or more; he has the intention to steal it; if he touches it, he commits an offense of wrong conduct; if he makes it stir, he commits an offense of wrong conduct; if he moves it from its base, he commits an offense of wrong conduct. 

For\marginnote{6.3.9} one who steals there is an offense of wrong conduct when five factors are fulfilled: it is not the possession of another; but he perceives it as the possession of another; it is an ordinary possession worth more than one \textit{\textsanskrit{māsaka}} coin, but less than five; he has the intention to steal it; if he touches it, he commits an offense of wrong conduct; if he makes it stir, he commits an offense of wrong conduct; if he moves it from its base, he commits an offense of wrong conduct. 

For\marginnote{6.3.17} one who steals there is an offense of wrong conduct when five factors are fulfilled: it is not the possession of another; but he perceives it as the possession of another; it is an ordinary possession worth one \textit{\textsanskrit{māsaka}} coin or less; he has the intention to steal it; if he touches it, he commits an offense of wrong conduct; if he makes it stir, he commits an offense of wrong conduct; if he moves it from its base, he commits an offense of wrong conduct. 

\subsection*{Non-offenses }

There\marginnote{6.4.1} is no offense: if he perceives it as his own; if he takes it on trust;\footnote{This refers to a situation where you have an agreement with a close friend that you may take their belongings on trust. The conditions for taking on trust are set out at \href{https://suttacentral.net/pli-tv-kd8/en/brahmali\#19.1.5}{Kd 8:19.1.5}. } if he borrows it; if it is the possession of a ghost; if it is the possession of an animal; if he perceives it as discarded; if he is insane; if he is deranged; if he is overwhelmed by pain; if he is the first offender. 

\scend{The first section for recitation on stealing is finished. }

\scuddanaintro{Summary verses of case studies }

\begin{scuddana}%
“Five\marginnote{6.4.14} are told with dyers, \\
And four with bedspreads; \\
Five with darkness, \\
And five with a carrier. 

Five\marginnote{6.4.18} are told with ways of speaking, \\
Another two with wind; \\
Fresh, drawing lots, \\
With the sauna it is ten. 

Five\marginnote{6.4.22} are told with animal kills, \\
And five on without proper reason; \\
Boiled rice during a shortage of food, and meat, \\
Cookies, pastries, cakes. 

Six\marginnote{6.4.26} on requisites, and bag, \\
Mattress, bamboo, on not coming out; \\
And taking fresh food on trust, \\
Another two on perceiving as one’s own. 

Seven\marginnote{6.4.30} on ‘We didn’t steal,’ \\
And seven where they did steal; \\
Seven where they stole from the Sangha, \\
Another two with flowers. 

And\marginnote{6.4.34} three on taking messages, \\
Three on taking gems past; \\
And pigs, deer, fish, \\
And he set a vehicle in motion. 

Two\marginnote{6.4.38} on a piece, two on wood, \\
Discarded, two on water; \\
Step by step, by arrangement, \\
Another did not amount to it. 

Four\marginnote{6.4.42} handfuls at \textsanskrit{Sāvatthī}, \\
Two on kills, two about grass; \\
Seven where they divided the belongings of the Sangha, \\
And seven on non-owners. 

Wood,\marginnote{6.4.46} water, clay, two on grass, \\
Seven on stealing the Sangha’s bedding; \\
And one should not take away what has an owner, \\
One may borrow what has an owner. 

\textsanskrit{Campā},\marginnote{6.4.50} and in \textsanskrit{Rājagaha}, \\
And Ajjuka at \textsanskrit{Vesālī}; \\
And Benares, \textsanskrit{Kosambī}, \\
And \textsanskrit{Sāgalā} with Dalhika.” 

%
\end{scuddana}

\subsubsection*{Case studies }

On\marginnote{7.1.1} one occasion the monks from the group of six went to the dyers and stole their collection of cloth. They became anxious, thinking, “The Buddha has laid down a training rule. Could it be that we’ve committed an offense entailing expulsion?” They told the Buddha. “Monks, you have committed an offense entailing expulsion.” 

On\marginnote{7.2.1} one occasion a monk went to the dyers, saw a valuable cloth, and had the intention to steal it. He became anxious … “The Buddha has laid down a training rule. Could it be that I’ve committed an offense entailing expulsion?” He told the Buddha. “There’s no offense for the arising of a thought.” 

On\marginnote{7.2.7} one occasion a monk went to the dyers, saw a valuable cloth, and touched it, intending to steal it. He became anxious … “There’s no offense entailing expulsion, but there’s an offense of wrong conduct.” 

On\marginnote{7.2.11} one occasion a monk went to the dyers, saw a valuable cloth, and made it stir, intending to steal it. He became anxious … “There’s no offense entailing expulsion, but there’s a serious offense.” 

On\marginnote{7.2.15} one occasion a monk went to the dyers, saw a valuable cloth, and moved it from its base, intending to steal it. He became anxious … “You have committed an offense entailing expulsion.” 

On\marginnote{7.3.1} one occasion an alms-collecting monk saw a valuable bedspread and had the intention to steal it. … “There’s no offense for the arising of a thought.” … and touched it, intending to steal it. … “There’s no offense entailing expulsion, but there’s an offense of wrong conduct.” … and made it stir, intending to steal it. … “There’s no offense entailing expulsion, but there’s a serious offense.” … and moved it from its base, intending to steal it. … “You have committed an offense entailing expulsion.” 

On\marginnote{7.4.1} one occasion a monk saw some goods during the day. He took note of them with the thought, “I’ll steal them at night.” And he stole them, thinking they were the ones he had seen. … But he stole other goods, thinking they were the ones he had seen. … And he stole them, thinking they were other than the ones he had seen. … But he stole other goods, thinking they were other than the ones he had seen. He became anxious … “You have committed an offense entailing expulsion.” 

On\marginnote{7.4.9} one occasion a monk saw some goods during the day. He took note of them with the thought, “I’ll steal them at night.” But he stole his own goods, thinking they were the ones he had seen. He became anxious … “There’s no offense entailing expulsion, but there’s an offense of wrong conduct.” 

On\marginnote{7.5.1} one occasion a monk who was carrying the goods of another on his head touched the load, intending to steal it. … “There’s no offense entailing expulsion, but there’s an offense of wrong conduct.” … made it stir, intending to steal it. … “There’s no offense entailing expulsion, but there’s a serious offense.” … lowered it onto his shoulder, intending to steal it. … “You have committed an offense entailing expulsion.” … 

touched\marginnote{7.5.4} the load on the shoulder, intending to steal it. … “There’s no offense entailing expulsion, but there’s an offense of wrong conduct.” … made it stir, intending to steal it. … “There’s no offense entailing expulsion, but there’s a serious offense.” … lowered it onto his hip, intending to steal it. … “You have committed an offense entailing expulsion.” … 

touched\marginnote{7.5.7} the load on the hip, intending to steal it. … “There’s no offense entailing expulsion, but there’s an offense of wrong conduct.” … made it stir, intending to steal it. … “There’s no offense entailing expulsion, but there’s a serious offense.” … took hold of it with his hand, intending to steal it. … “You have committed an offense entailing expulsion.” … 

placed\marginnote{7.5.10} the load in his hand on the ground, intending to steal it. … “You have committed an offense entailing expulsion.” … picked it up from the ground, intending to steal it. … “You have committed an offense entailing expulsion.” 

On\marginnote{7.6.1} one occasion a monk spread out his robe outside and entered his dwelling. A second monk, thinking, “Let me look after it,” put it away. The first monk came out of his dwelling and asked the monks, “Who’s taken my robe?” The second monk said, “I’ve taken it.” The first monk took hold of him and said, “You’re not a monastic anymore!” The second monk became anxious … He told the Buddha. “What were you thinking?” 

“Sir,\marginnote{7.6.9} it was just a way of speaking.” 

“If\marginnote{7.6.10} it was just a way of speaking, there’s no offense.” 

On\marginnote{7.6.11} one occasion a monk placed his robe on a bench …\footnote{I have added one set of ellipses points at the end of the sentence. They seem to have been omitted by mistake from the Pali. } placed his sitting mat on a bench … put his almsbowl under a bench and entered his dwelling. A second monk, thinking, “Let me look after it,” put it away. The first monk came out and asked the monks, “Who’s taken my bowl?” The second monk said, “I’ve taken it.” The first monk took hold of him and said, “You’re not a monastic anymore!” The second monk became anxious … “If it was just a way of speaking, there’s no offense.” 

On\marginnote{7.6.20} one occasion a nun spread out her robe on a fence and entered her dwelling. A second nun, thinking, “Let me look after it,” put it away. The first nun came out and asked the nuns, “Venerables, who’s taken my robe?” The second nun said, “I’ve taken it.” The first nun took hold of her and said, “You’re not a monastic anymore!” The second nun became anxious … She told the nuns, who in turn told the monks, who in turn told the Buddha. … “If it was just a way of speaking, there’s no offense.” 

On\marginnote{7.7.1} one occasion a monk saw a wrap garment blown up by a whirlwind. He took hold of it, thinking, “I’ll give it to the owners.” But the owners accused him, saying, “You’re not a monastic anymore!” He became anxious … “What were you thinking, monk?” 

“I\marginnote{7.7.6} didn’t intend to steal it, sir.” 

“There’s\marginnote{7.7.7} no offense for one who doesn’t intend to steal.” 

On\marginnote{7.7.8} one occasion a monk took hold of a turban that had been blown up by a whirlwind, intending to steal it before the owners found out. The owners accused him, saying, “You’re not a monastic anymore!” He became anxious … “You have committed an offense entailing expulsion.” 

On\marginnote{7.8.1} one occasion a monk went to a charnel ground and took the rags from a fresh corpse. The ghost was still dwelling in that body, and it said to the monk, “Sir, don’t take my wrap.” The monk took no notice and left. Then the corpse got up and followed behind that monk. The monk entered his dwelling and closed the door, and the corpse collapsed right there. He became anxious … “There’s no offense entailing expulsion. 

\scrule{But a monk shouldn’t take rags from a fresh corpse. If he does, he commits an offense of wrong conduct.” }

On\marginnote{7.9.1} one occasion robe-cloth belonging to the Sangha was being distributed. A monk disregarded the draw and took the robe-cloth, intending to steal it.\footnote{“Disregarded the draw” renders \textit{\textsanskrit{kusaṁ} \textsanskrit{saṅkāmetvā}}. \textit{Kusa}-grass was used to draw lots when distributing robe-cloth, see \href{https://suttacentral.net/pli-tv-kd8/en/brahmali\#9.4.4}{Kd 8:9.4.4}. } He became anxious … “You have committed an offense entailing expulsion.” 

On\marginnote{7.10.1} one occasion when Venerable Ānanda was in a sauna, he thought the sarong of another monk was his own and put it on. The other monk said, “Ānanda, why did you put on my sarong?” 

“I\marginnote{7.10.4} thought it was my own.” 

They\marginnote{7.10.5} told the Buddha. “There’s no offense for one who perceives it as his own.” 

On\marginnote{7.11.1} one occasion a number of monks were descending from the Vulture Peak when they saw the remains of a lion’s kill. They had it cooked and ate it. They became anxious … “There’s no offense when it’s the remains of a lion’s kill.” 

On\marginnote{7.11.4} one occasion a number of monks were descending from the Vulture Peak when they saw the remains of a tiger’s kill … saw the remains of a panther’s kill … saw the remains of a hyena’s kill … saw the remains of a wolf’s kill. They had it cooked and ate it. They became anxious … “There’s no offense when it’s the possession of an animal.” 

On\marginnote{7.12.1} one occasion, when rice belonging to the Sangha was being distributed, a monk said without grounds, “Please give me a portion for one more,” and he took it away. He became anxious … “There’s no offense entailing expulsion, but there’s an offense entailing confession for lying in full awareness.” 

On\marginnote{7.12.6} one occasion, when fresh food belonging to the Sangha was being distributed …\footnote{“Fresh food” renders \textit{\textsanskrit{khādanīya}}. See Appendix of Technical Terms for discussion. } when cookies belonging to the Sangha were being distributed … when sugarcane belonging to the Sangha was being distributed … when gaub fruits belonging to the Sangha were being distributed, a monk said without grounds, “Please give me a portion for one more,” and he took it away. He became anxious … “There’s no offense entailing expulsion, but there’s an offense entailing confession for lying in full awareness.” 

On\marginnote{7.13.1} one occasion during a shortage of food, a monk entered a rice kitchen and took a bowlful of boiled rice, intending to steal it. He became anxious … “You have committed an offense entailing expulsion.” 

On\marginnote{7.13.4} one occasion during a shortage of food, a monk entered a slaughterhouse and took a bowlful of meat, intending to steal it. He became anxious … “You have committed an offense entailing expulsion.” 

On\marginnote{7.13.7} one occasion during a shortage of food, a monk entered a bakery and took a bowlful of cookies, intending to steal it. … took a bowlful of pastries, intending to steal it. … took a bowlful of cakes, intending to steal it. He became anxious … “You have committed an offense entailing expulsion.” 

On\marginnote{7.14.1} one occasion a certain monk saw a requisite during the day. He took note of it with the thought, “I’ll steal it at night.” He then stole it, thinking it was what he had seen … He then stole something else, thinking it was what he had seen … He then stole it, thinking it was something other than what he had seen … He then stole something else, thinking it was something other than what he had seen. He became anxious … “You have committed an offense entailing expulsion.” 

On\marginnote{7.14.9} one occasion a certain monk saw a requisite during the day. He took note of it with the thought, “I’ll steal it at night.” But he stole his own requisite, thinking it was what he had seen. He became anxious … “There’s no offense entailing expulsion, but there’s an offense of wrong conduct.” 

On\marginnote{7.15.1} one occasion a monk saw a bag on a bench. He thought, “If I take it from there I shall be expelled,” and so he took it by moving the bench. He became anxious … “You have committed an offense entailing expulsion.” 

On\marginnote{7.16.1} one occasion a monk took a mattress from the Sangha, intending to steal it. He became anxious … “You have committed an offense entailing expulsion.” 

On\marginnote{7.17.1} one occasion a monk took a robe from a bamboo robe rack, intending to steal it. He became anxious … “You have committed an offense entailing expulsion.” 

On\marginnote{7.18.1} one occasion a monk stole a robe in a dwelling. He thought, “If I come out from here, I shall be expelled,” and he remained in that dwelling. They told the Buddha. “Whether that fool comes out or not, he has committed an offense entailing expulsion.” 

At\marginnote{7.19.1} one time there were two monks who were friends. One of them went into the village for almsfood. When fresh food belonging to the Sangha was being distributed, the second monk took his friend’s portion. Taking it on trust, he ate it. When he found out about this, the first monk accused him, saying, “You’re not a monastic anymore!” He became anxious … 

“What\marginnote{7.19.8} were you thinking, monk?” 

“I\marginnote{7.19.9} took it on trust, sir.” 

“There’s\marginnote{7.19.10} no offense for one who takes on trust.” 

On\marginnote{7.20.1} one occasion a number of monks were making robes. When fresh food belonging to the Sangha was being distributed, they took their shares and put them aside. A certain monk ate another monk’s portion, thinking it was his own. When the other monk found out about this, he accused him, saying, “You’re not a monastic anymore!” He became anxious … 

“What\marginnote{7.20.7} were you thinking, monk?” 

“I\marginnote{7.20.8} thought it was my own, sir.” 

“There’s\marginnote{7.20.9} no offense for one who perceives it as his own.” 

On\marginnote{7.20.10} one occasion a number of monks were making robes. When fresh food belonging to the Sangha was being distributed, they brought a certain monk’s share in another monk’s almsbowl and put it aside. The monk who was the owner of the bowl ate the food, thinking it was his own. When he found out about this, the owner of the food accused him … “There’s no offense for one who perceives it as his own.” 

On\marginnote{7.21.1} one occasion mango thieves cut down some mangoes, collected them in a bundle, and left. The owners pursued them. When they saw the owners, the thieves dropped the bundle and ran away. Some monks perceived those mangoes as discarded, had them offered, and ate them. But the owners accused them, saying, “You’re not monastics anymore!” They became anxious … They told the Buddha. 

“What\marginnote{7.21.9} were you thinking, monks?” 

“Sir,\marginnote{7.21.10} we perceived them as discarded.” 

“There’s\marginnote{7.21.11} no offense for one who perceives something as discarded.” 

On\marginnote{7.21.12} one occasion rose-apple thieves …\footnote{“Rose apple” renders \textit{jambu}. The \textit{jambu} is normally identified with the rose apple (tree), but according to PED it is the \textit{Eugenia jambolana} (tree), which is actually the black-plum tree. I have not been able to ascertain which of these interpretations is correct. } bread-fruit thieves … jack-fruit thieves … palm-fruit thieves … sugarcane thieves … gaub fruit thieves picked some fruit, collected them in a bundle, and left. The owners pursued them. When they saw the owners, the thieves dropped the bundle and ran away. Some monks perceived those gaub fruit as discarded, had them offered, and ate them. But the owners accused them, saying, “You’re not monastics anymore!” They became anxious … “There’s no offense for one who perceives something as discarded.” 

On\marginnote{7.22.1} one occasion mango thieves cut down some mangoes, collected them in a bundle, and left. The owners pursued them. When they saw the owners, the thieves dropped the bundle and ran away. Some monks ate them, intending to steal them before the owners found them. The owners accused those monks, saying, “You’re not monastics anymore!” They became anxious … “You have committed an offense entailing expulsion.” 

On\marginnote{7.22.9} one occasion rose-apple thieves … bread-fruit thieves … jack-fruit thieves … palm-fruit thieves … sugarcane thieves … gaub fruit thieves picked some fruit, collected them in a bundle, and left. The owners pursued them. When they saw the owners, the thieves dropped the bundle and ran away. Some monks ate them, intending to steal them before the owners found them. The owners accused those monks, saying, “You’re not monastics anymore!” They became anxious … “You have committed an offense entailing expulsion.” 

On\marginnote{7.23.1} one occasion a monk took a mango from the Sangha, intending to steal it. … a rose apple … a bread-fruit … a jack-fruit … a palm-fruit … a sugarcane … a gaub fruit from the Sangha, intending to steal it. He became anxious … “You have committed an offense entailing expulsion.” 

On\marginnote{7.24.1} one occasion a monk went to a garden and took a cut flower worth five \textit{\textsanskrit{māsaka}} coins, intending to steal it. He became anxious … “You have committed an offense entailing expulsion.” 

On\marginnote{7.24.4} one occasion a monk went to a garden, picked a flower worth five \textit{\textsanskrit{māsaka}} coins, and took it away, intending to steal it. He became anxious … “You have committed an offense entailing expulsion.” 

On\marginnote{7.25.1} one occasion a certain monk who was going to the village said to another monk, “I can take a message to the family that supports you.”\footnote{The unusual expression \textit{vutto vajjemi} recurs in the next two cases. \textit{Vajjemi} is apparently an optative formation from \textit{\textsanskrit{vadāmi}}, “I could say”, whereas \textit{vutto} is the past participle of \textit{vuccati}, “spoken to”, the overall meaning being, “I, (having been) spoken to, could say to the family that supports you.” The idea conveyed seems to be the taking of a message. Sp 1.150: \textit{Tava vacanena \textsanskrit{vadāmīti} attho}, “The meaning is, ‘I will speak your statement.’” } He went there and brought back a wrap garment that he used himself. When the other monk found out about this, he accused him, saying, “You’re not a monastic anymore!” He became anxious … “There’s no offense entailing expulsion. 

\scrule{But you shouldn’t say, ‘I can take a message.’ If you do, you commit an offense of wrong conduct.” }

On\marginnote{7.25.10} one occasion a certain monk was going to the village. Another monk said to him, “Please take a message to the family that supports me.” He went there and brought back a pair of wrap garments. He used one himself and gave the other to the other monk. When the other monk found out about this, he accused him, saying, “You’re not a monastic anymore!” He became anxious … “There’s no offense entailing expulsion. 

\scrule{But you shouldn’t say, ‘Please take a message.’ If you do, you commit an offense of wrong conduct.” }

On\marginnote{7.25.20} one occasion a monk who was going to the village said to another monk, “I can take a message to the family that supports you.” He replied, “Please do.” He went there and brought back an \textit{\textsanskrit{āḷhaka}} measure of ghee, a \textit{\textsanskrit{tulā}} measure of sugar, and a \textit{\textsanskrit{doṇa}} measure of husked rice, which he ate himself.\footnote{According to T. W. Rhys Davids in “On the Ancient Coins and Measures of Ceylon: with a discussion of the Ceylon date of the Buddha's death”, p. 18, one \textit{\textsanskrit{doṇa}} is equivalent to 64 handfuls. } When the other monk found out about this, he accused him, saying, “You’re not a monastic anymore!” He became anxious … “There’s no offense entailing expulsion. 

\scrule{But you shouldn’t say, ‘I can take a message;’ nor should you say, ‘Please do.’ If you do, you commit an offense of wrong conduct.” }

At\marginnote{7.26.1} one time a man who was traveling with a monk was carrying a valuable gem. When the man saw a customs station, he put the gem into the monk’s bag without his knowing. When they had gone past the customs station, he retrieved it. The monk was anxious … 

“What\marginnote{7.26.4} were you thinking, monk?” 

“I\marginnote{7.26.5} didn’t know, sir.” 

“There’s\marginnote{7.26.6} no offense for one who doesn’t know.” 

At\marginnote{7.26.7} one time a man who was traveling with a monk was carrying a valuable gem. When the man saw a customs station, he pretended to be sick, and gave his own bag to the monk. When they had passed the customs station, he said to the monk, “Please give me my bag, sir, I’m not sick.” 

“Then\marginnote{7.26.11} why did you say so?” 

The\marginnote{7.26.12} man told the monk. He became anxious … “What were you thinking, monk?” “I didn’t know, sir.” “There’s no offense for one who doesn’t know.” 

At\marginnote{7.26.17} one time a monk was traveling with a group. A man bribed that monk by giving him food. Seeing a customs station, he gave the monk a valuable gem, saying, “Sir, please take this gem past the customs,” which the monk did. He became anxious … “You have committed an offense entailing expulsion.” 

On\marginnote{7.27.1} one occasion a monk, feeling compassion, released a pig trapped in a snare. He became anxious … “What were you thinking, monk?” 

“I\marginnote{7.27.4} was motivated by compassion, sir.” 

“There’s\marginnote{7.27.5} no offense for one who is motivated by compassion.” 

On\marginnote{7.27.6} one occasion a monk released a pig trapped in a snare, intending to steal it before the owners found it. He became anxious … “You have committed an offense entailing expulsion.” 

On\marginnote{7.27.10} one occasion a monk, feeling compassion, released a deer trapped in a snare. … “There’s no offense for one who is motivated by compassion.” … released a deer trapped in a snare, intending to steal it before the owners found it. … “You have committed an offense entailing expulsion.” … feeling compassion, released fish trapped in a fish-net … “There’s no offense for one who is motivated by compassion.” … released fish trapped in a fish-net, intending to steal them before the owners found them. He became anxious … “You have committed an offense entailing expulsion.” 

On\marginnote{7.28.1} one occasion a monk saw some goods in a vehicle. He thought, “If I take them from there, I’ll be expelled.” So he took them by setting the vehicle in motion. He became anxious … “You have committed an offense entailing expulsion.” 

On\marginnote{7.29.1} one occasion a monk seized a piece of meat picked up by a hawk, intending to give it to the owners. But the owners accused him, saying, “You’re not a monastic anymore!” He became anxious … “There’s no offense for one who doesn’t intend to steal.” 

On\marginnote{7.29.7} one occasion a monk seized a piece of meat picked up by a hawk, intending to steal it before the owners found out. The owners accused him, saying, “You’re not a monastic anymore!” He became anxious … “You have committed an offense entailing expulsion.” 

At\marginnote{7.30.1} one time some men made a raft that they put on the river \textsanskrit{Aciravatī}. Because the binding ropes snapped, the sticks were scattered about. Some monks removed them from the water, perceiving them as discarded. The owners accused those monks, saying, “You’re not monastics anymore!” They became anxious … “There’s no offense for one who perceives something as discarded.” 

At\marginnote{7.30.8} one time some men made a raft, which they put on the river \textsanskrit{Aciravatī}. Because the binding ropes snapped, the sticks were scattered about. Some monks removed them from the water, intending to steal them before the owners found them. The owners accused those monks, saying, “You’re not monastics anymore!” They became anxious … “You have committed an offense entailing expulsion.” 

On\marginnote{7.31.1} one occasion a cowherd hung his wrap garment on a tree and went to relieve himself. A monk thought it had been discarded and took it. The cowherd accused him, saying, “You’re not a monastic anymore!” He became anxious … “There’s no offense for one who perceives something as discarded.” 

On\marginnote{7.32.1} one occasion, a wrap garment that had escaped from the hands of a dyer stuck to a monk’s foot as he was crossing a river. The monk took hold of it, thinking, “I’ll give it to its owners.” But the owners accused him, saying, “You’re not a monastic anymore!” He became anxious … “There’s no offense for one who doesn’t intend to steal.” 

On\marginnote{7.32.7} one occasion, a wrap garment that had escaped from the hands of a dyer stuck to a monk’s foot as he was crossing a river. The monk took hold of it, intending to steal it before the owners found it. The owners accused him, saying, “You’re not a monastic anymore!” He became anxious … “You have committed an offense entailing expulsion.” 

On\marginnote{7.33.1} one occasion a monk saw a pot of ghee and ate it little by little. He became anxious … “There’s no offense entailing expulsion, but there’s an offense of wrong conduct.” 

At\marginnote{7.34.1} one time a number of monks made an arrangement and then left, thinking, “We’ll steal these goods.” One of them stole the goods. The others said, “We’re not expelled. He who stole them is expelled.” They told the Buddha. “You’ve all committed an offense entailing expulsion.” 

At\marginnote{7.34.9} one time a number of monks made an arrangement, stole some goods, and shared them out. Each one of them received a share worth less than five \textit{\textsanskrit{māsaka}} coins. They said, “We’re not expelled.” They told the Buddha. “You have committed an offense entailing expulsion.” 

On\marginnote{7.35.1} one occasion when \textsanskrit{Sāvatthī} was short of food, a monk took a handful of rice from a shopkeeper, intending to steal it. He became anxious … “You have committed an offense entailing expulsion.” 

On\marginnote{7.35.4} one occasion when \textsanskrit{Sāvatthī} was short of food, a monk stole a handful of mung beans from a shopkeeper, intending to steal it. … a handful of black gram … a handful of sesame from a shopkeeper, intending to steal it. He became anxious … “You have committed an offense entailing expulsion.” 

At\marginnote{7.36.1} one time in the Dark Wood near \textsanskrit{Sāvatthī}, thieves killed a cow, ate some of the flesh, put the remainder aside, and went away. Some monks had it offered and ate it, perceiving it as discarded. The thieves accused those monks, saying, “You’re not monastics anymore!” They became anxious … “There’s no offense for one who perceives something as discarded.” 

At\marginnote{7.36.7} one time in the Dark Wood near \textsanskrit{Sāvatthī}, thieves killed a pig, ate some of the flesh, put the remainder aside, and went away. Some monks had it offered and ate it, perceiving it as discarded. The thieves accused those monks, saying, “You’re not monastics anymore!” They became anxious … “There’s no offense for one who perceives something as discarded.” 

On\marginnote{7.37.1} one occasion a monk went to a meadow and took cut grass worth five \textit{\textsanskrit{māsaka}} coins, intending to steal it. He became anxious … “You have committed an offense entailing expulsion.” 

On\marginnote{7.37.4} one occasion a monk went to a meadow, cut grass worth five \textit{\textsanskrit{māsaka}} coins, and took it away, intending to steal it. He became anxious … “You have committed an offense entailing expulsion.” 

On\marginnote{7.38.1} one occasion some newly-arrived monks shared out the mangoes belonging to the Sangha and ate them. The resident monks accused those monks, saying, “You’re not monastics anymore!” They became anxious … They told the Buddha. 

“What\marginnote{7.38.6} were you thinking, monks?” 

“We\marginnote{7.38.7} thought they were meant for eating, sir.” 

“There’s\marginnote{7.38.8} no offense for one who thinks it is meant for eating.” 

On\marginnote{7.38.9} one occasion some newly-arrived monks shared out the rose apples belonging to the Sangha … the bread-fruit belonging to the Sangha … the jack-fruit belonging to the Sangha … the palm fruits belonging to the Sangha … the sugarcane belonging to the Sangha … the gaub fruit belonging to the Sangha and ate them. The resident monks accused those monks, saying, “You’re not monastics anymore!” They became anxious … “There’s no offense for one who thinks it is meant for eating.” 

On\marginnote{7.39.1} one occasion the keepers of a mango grove gave a mango to some monks. The monks, thinking, “They have the authority to guard, but not to give away,” were afraid of wrongdoing and did not accept it. They told the Buddha. “There’s no offense if it’s a gift from a guardian.” 

On\marginnote{7.39.6} one occasion the keepers of a rose-apple grove … the keepers of a bread-fruit grove … the keepers of a jack-fruit grove … the keepers of a palm grove … the keepers of a sugarcane field … the keepers of a gaub fruit grove gave a gaub fruit to some monks. The monks, thinking, “They have the authority to guard, but not to give away,” were afraid of wrongdoing and did not accept it. They told the Buddha. “There’s no offense if it’s a gift from a guardian.” 

On\marginnote{7.40.1} one occasion a monk borrowed a piece of wood belonging to the Sangha and used it to support the wall of his own dwelling. The monks accused him, saying, “You’re not a monastic anymore!” He became anxious and told the Buddha. “What were you thinking, monk?” 

“I\marginnote{7.40.7} was borrowing it, sir.” 

“There’s\marginnote{7.40.8} no offense for one who is borrowing.” 

On\marginnote{7.41.1} one occasion a monk took water from the Sangha, intending to steal it. … took clay from the Sangha, intending to steal it. … took a pile of grass from the Sangha, intending to steal it. … He became anxious … “You have committed an offense entailing expulsion.” 

On\marginnote{7.41.6} one occasion a monk set fire to a pile of grass belonging to the Sangha, intending to steal. He became anxious … “There’s no offense entailing expulsion, but there’s an offense of wrong conduct.” 

On\marginnote{7.42.1} one occasion a monk took a bed from the Sangha, intending to steal it. … He became anxious … “You have committed an offense entailing expulsion.” 

On\marginnote{7.42.4} one occasion a monk took a bench from the Sangha, intending to steal it … a mattress from the Sangha … a pillow from the Sangha … a door from the Sangha … a window from the Sangha … took a rafter from the Sangha, intending to steal it. … He became anxious … “You have committed an offense entailing expulsion.” 

At\marginnote{7.43.1} one time the monks used elsewhere the equipment belonging to a certain lay follower. That lay follower complained and criticized them, “How can the venerables use equipment where it doesn’t belong?” They told the Buddha. 

\scrule{“You shouldn’t use equipment where it doesn’t belong. If you do, you commit an offense of wrong conduct.” }

Soon\marginnote{7.44.1} afterwards, being afraid of wrongdoing, the monks did not take any furniture to the observance-day hall or to meetings, and they sat down on the bare ground. They became dirty, as did their robes. They told the Buddha. 

\scrule{“I allow you to borrow.” }

On\marginnote{7.45.1} one occasion at \textsanskrit{Campā}, a nun who was a pupil of the nun \textsanskrit{Thullanandā} went to a family that supported \textsanskrit{Thullanandā} and said, “The venerable wants to drink the triple pungent congee.” When it was ready, she took it away and ate it herself. When \textsanskrit{Thullanandā} found out about this, she accused her, saying, “You’re not a monastic anymore!” She became anxious … She then told the nuns, who in turn told the monks, who then told the Buddha. “There’s no offense entailing expulsion, but there’s an offense entailing confession for lying in full awareness.” 

On\marginnote{7.45.12} one occasion in \textsanskrit{Rājagaha}, a nun who was a pupil of the nun \textsanskrit{Thullanandā} went to a family that supported \textsanskrit{Thullanandā} and said, “The venerable wants a honey-ball.” When it was ready, she took it away and ate it herself. When \textsanskrit{Thullanandā} found out about this, she accused her, saying, “You’re not a monastic anymore!” She became anxious … “There’s no offense entailing expulsion, but there’s an offense entailing confession for lying in full awareness.” 

At\marginnote{7.46.1} that time there was a householder in \textsanskrit{Vesāli} who was a supporter of Venerable Ajjuka and who had two children living with him, a son and a nephew. He said to Ajjuka,\footnote{Sp 1.158: \textit{\textsanskrit{Etadavocāti} \textsanskrit{gilāno} \textsanskrit{hutvā} avoca}, “‘He said’: he said it because of illness.” } “Sir, please assign my property to the one of these two boys who has faith and confidence.” 

It\marginnote{7.46.4} turned out that the householder’s nephew had faith and confidence, and so Ajjuka assigned the property to him. He then established a household with that wealth and made a gift. 

The\marginnote{7.46.6} householder’s son then said to Venerable Ānanda, “Who is the father’s heir, Venerable Ānanda, the son or the nephew?” 

“The\marginnote{7.46.8} son is the father’s heir.” 

“Sir,\marginnote{7.46.9} Venerable Ajjuka has assigned our wealth to our housemate.” 

“Venerable\marginnote{7.46.10} Ajjuka is not a monastic anymore.” 

Ajjuka\marginnote{7.46.11} then said to Ānanda, “Ānanda, please do a proper investigation.” 

On\marginnote{7.46.13} that occasion Venerable \textsanskrit{Upāli} was siding with Ajjuka, and he said to Ānanda, “Ānanda, when one is asked by the owner to assign a property to so-and-so and one does as asked, what has one committed?” 

“One\marginnote{7.46.16} hasn’t committed anything, sir, not even an act of wrong conduct.” 

“Venerable\marginnote{7.46.17} Ajjuka was asked by the owner to assign his property to so-and-so, which he did. There’s no offense for Venerable Ajjuka.” 

At\marginnote{7.47.1} that time a family in Benares that supported Venerable Pilindavaccha was harassed by criminals. Two of their children were kidnapped. Soon afterwards Pilindavaccha brought those children back by his supernormal powers and put them in a stilt house.\footnote{“Stilt house” renders \textit{\textsanskrit{pāsāda}}. See Appendix of Technical Terms for discussion. } 

When\marginnote{7.47.4} people saw those children, they said, “This is the greatness of Venerable Pilindavaccha’s supernormal powers,” and they gained confidence in him. 

But\marginnote{7.47.7} the monks complained and criticized him, “How could Venerable Pilindavaccha bring back children who had been kidnapped by criminals?” They told the Buddha. 

\scrule{“There’s no offense for someone who uses their supernormal powers.” }

At\marginnote{7.48.1} that time the two monks \textsanskrit{Paṇḍaka} and Kapila were friends. One was staying in a village and one at \textsanskrit{Kosambī}. Then, while one of them was traveling from that village to \textsanskrit{Kosambī}, he had to cross a river. As he did so, a lump of fat that had escaped from the hands of a pig butcher stuck to his foot. He grabbed it, thinking, “I’ll give it to the owners.” But the owners accused him, saying, “You’re not a monastic anymore!” 

Just\marginnote{7.48.7} then a woman cowherd who had seen him crossing said, “Come, sir, have sexual intercourse.” Thinking he was no longer a monastic, he had sexual intercourse with her. 

When\marginnote{7.48.10} he arrived at \textsanskrit{Kosambī}, he told the monks, who in turn told the Buddha. “There’s no offense entailing expulsion for stealing, but there’s an offense entailing expulsion for having sexual intercourse.” 

At\marginnote{7.49.1} that time a monk at \textsanskrit{Sāgalā} who was a student of Venerable \textsanskrit{Daḷhika} was plagued by lust. He stole a turban from a shopkeeper and said to \textsanskrit{Daḷhika}, “Sir, I’m not a monastic anymore. I’ll disrobe.” 

“But\marginnote{7.49.3} what have you done?” He told him. Venerable \textsanskrit{Daḷhika} had the turban brought and valued. It was worth less than five \textit{\textsanskrit{māsaka}} coins. Saying, “There’s no offense entailing expulsion,” he gave a teaching. And that monk was delighted. 

\scendsutta{The second offense entailing expulsion is finished. }

%
\section*{{\suttatitleacronym Bu Pj 3}{\suttatitletranslation The third training rule on expulsion }{\suttatitleroot Manussaviggaha}}
\addcontentsline{toc}{section}{\tocacronym{Bu Pj 3} \toctranslation{The third training rule on expulsion } \tocroot{Manussaviggaha}}
\markboth{The third training rule on expulsion }{Manussaviggaha}
\extramarks{Bu Pj 3}{Bu Pj 3}

\subsection*{Origin story }

\subsubsection*{First sub-story }

At\marginnote{1.1.1} one time the Buddha was staying in the hall with the peaked roof in the Great Wood near \textsanskrit{Vesālī}. At that time the Buddha spoke to the monks in many ways about unattractiveness—he spoke in praise of unattractiveness, of developing the mind in unattractiveness, and of the attainment of unattractiveness. 

The\marginnote{1.1.3} Buddha then addressed the monks: “Monks, I wish to go into solitary retreat for half a month. No one should visit me except the one who brings me almsfood.” 

“Yes,\marginnote{1.1.6} venerable sir.” 

Soon\marginnote{1.1.7} afterwards the monks reflected that the Buddha had praised unattractiveness in many ways, and they devoted themselves to developing the mind in unattractiveness in its many different facets. As a consequence, they became troubled by their own bodies, ashamed of and disgusted with them. Just as a young woman or man—someone fond of adornments, with freshly washed hair—would be ashamed, humiliated, and disgusted if the carcass of a snake, dog, or man was hung around her neck, just so those monks were troubled by their own bodies. They took their own lives, took the lives of one another, and they went to \textsanskrit{Migalaṇḍika}, the monastic lookalike, and said, “Please kill us. You will get our bowl and robes.” And hired for a bowl and robes, \textsanskrit{Migalaṇḍika} killed a number of monks. He then took his blood-stained knife to the river \textsanskrit{Vaggumudā}. 

While\marginnote{1.1.17} washing it, he became anxious and remorseful, thinking, “What the heck have I done? I’ve made so much demerit by killing good monks.” 

Then\marginnote{1.1.20} a god from the realm of the Lord of Death, coming across the water, said to \textsanskrit{Migalaṇḍika}, “Well done, superior man, you’re truly fortunate. You’ve made much merit by helping across those who hadn’t yet crossed.” 

\textsanskrit{Migalaṇḍika}\marginnote{1.1.23} thought, “So it seems that I’m fortunate, that I’ve made much merit!” He then went from dwelling to dwelling, from yard to yard, and said,\footnote{“Yard” renders \textit{\textsanskrit{pariveṇa}}. See Appendix of Technical Terms for discussion. } “Who hasn’t crossed yet? Who can I help across?” The monks who still had worldly attachments became fearful and terrified, with goosebumps all over. Only those who were free from worldly attachments were unaffected. 

Then,\marginnote{1.1.29} on a single day, \textsanskrit{Migalaṇḍika} killed one monk, two monks, three, four, five, ten, twenty, thirty, forty, fifty, even sixty monks. 

At\marginnote{1.2.1} the end of that half-month, when the Buddha came out of seclusion, he said to Venerable Ānanda, “Ānanda, why is the Sangha of monks so reduced?” 

Ānanda\marginnote{1.2.3} told him what had happened,   adding, “Please give another instruction, sir, for the Sangha of monks to become established in perfect insight.” 

“Well\marginnote{1.2.14} then, Ānanda, bring together in the assembly hall all the monks who live supported by \textsanskrit{Vesālī}.” “Yes.” When he had done so, he went to the Buddha and said, “Sir, the Sangha of monks is gathered. Please do as you think appropriate.” 

The\marginnote{1.2.19} Buddha then went to the assembly hall, sat down on the prepared seat, and addressed the monks: 

“Monks,\marginnote{1.3.1} when stillness by mindfulness of breathing is developed and cultivated, it is peaceful and sublime, and a satisfying state of bliss. And it removes bad and unwholesome qualities on the spot, whenever they arise. Just as a great, unseasonal storm in the last month of summer removes the dust and dirt from the air, just so, when stillness by mindfulness of breathing is developed and cultivated, it is peaceful and sublime, and it removes bad and unwholesome qualities on the spot, whenever they arise. 

And\marginnote{1.3.4} how is stillness by mindfulness of breathing developed and cultivated in this way? 

A\marginnote{1.3.5} monk sits down in the wilderness, at the foot of a tree, or in an empty hut. He crosses his legs, straightens his body, and sets up mindfulness in front of him. Simply mindful, he breathes in; mindful, he breathes out. 

When\marginnote{1.3.7} he breathes in long, he knows it; and when he breathes out long, he knows that. When he breathes in short, he knows it; and when he breathes out short, he knows that. When breathing in, he trains in fully experiencing the breath; when breathing out, he trains in fully experiencing the breath.\footnote{I have here translated \textit{\textsanskrit{kāya}} as breath in accordance with the usage in the \textsanskrit{Ānāpānasati} Sutta (\href{https://suttacentral.net/mn118/en/brahmali\#24.6}{MN 118:24.6}) where the breath is specifically said to be a body among bodies. Sp 1.165: \textit{Sakalassa \textsanskrit{assāsakāyassa} \textsanskrit{ādimajjhapariyosānaṁ} \textsanskrit{viditaṁ} karonto \textsanskrit{pākaṭaṁ} karonto “\textsanskrit{assasissāmī}”ti sikkhati}, “He trains, ‘I will breathe in’, producing knowledge of and familiarity with the beginning, the middle, and the end of the whole body of the breath.” } When breathing in, he trains in calming the activity of the body; when breathing out, he trains in calming the activity of the body. 

When\marginnote{1.3.11} breathing in, he trains in experiencing joy; when breathing out, he trains in experiencing joy. When breathing in, he trains in experiencing bliss; when breathing out, he trains in experiencing bliss. When breathing in, he trains in experiencing the activity of the mind; when breathing out, he trains in experiencing the activity of the mind. When breathing in, he trains in calming the activity of the mind; when breathing out, he trains in calming the activity of the mind. 

When\marginnote{1.3.15} breathing in, he trains in experiencing the mind; when breathing out, he trains in experiencing the mind. When breathing in, he trains in gladdening the mind; when breathing out, he trains in gladdening the mind. When breathing in, he trains in stilling the mind; when breathing out, he trains in stilling the mind. When breathing in, he trains in freeing the mind; when breathing out, he trains in freeing the mind. 

When\marginnote{1.3.19} breathing in, he trains in contemplating impermanence; when breathing out, he trains in contemplating impermanence. When breathing in, he trains in contemplating fading away; when breathing out, he trains in contemplating fading away. When breathing in, he trains in contemplating ending; when breathing out, he trains in contemplating ending. When breathing in, he trains in contemplating relinquishment; when breathing out, he trains in contemplating relinquishment. 

Monks,\marginnote{1.3.23} when stillness by mindfulness of breathing is developed and cultivated like this, it is peaceful and sublime, and a satisfying state of bliss. And it removes bad and unwholesome qualities on the spot, whenever they arise.” 

The\marginnote{1.4.1} Buddha then had the Sangha gathered and questioned the monks: “Is it true, monks, that there are monks who have taken their own lives, who have killed one another, and who have said to \textsanskrit{Migalaṇḍika}, ‘Please kill us. You will get our bowl and robes’?” 

“It’s\marginnote{1.4.4} true, sir.” 

The\marginnote{1.4.5} Buddha rebuked them, “Monks, it’s not suitable for these monks, it’s not proper, it’s not worthy of a monastic, it’s not allowable, it’s not to be done. How could those monks do this? This will affect people’s confidence …” … “And, monks, this training rule should be recited like this: 

\subsubsection*{Preliminary ruling }

\scrule{‘If a monk intentionally kills a human being or seeks an instrument of death for them,  he too is expelled and excluded from the community.’” }

In\marginnote{1.4.13} this way the Buddha laid down this training rule for the monks. 

\subsubsection*{Second sub-story }

At\marginnote{2.1} one time a certain lay follower was sick. He had a beautiful and pleasant wife, who the monks from the group of six had fallen in love with. They said to each other, “If this lay follower recovers, we won’t get her. Come, let’s praise death to him.” 

They\marginnote{2.7} then went to that lay follower and said, “You’ve done what’s good and wholesome; you’ve made a shelter against fear. You haven’t done anything bad; you haven’t been greedy or immoral. So why carry on with this miserable and difficult life? Death is better for you. When you’ve passed away, you’ll be reborn in a happy place, in heaven. There you’ll be able to enjoy the pleasures of heaven.” 

That\marginnote{2.14} lay follower thought, “The venerables have spoken the truth, for I’ve done what’s good and avoided what’s bad, and after death I’ll be reborn in a happy place.” 

From\marginnote{2.22} then on he ate various kinds of detrimental food and drank detrimental drinks, and as a consequence, he became very ill and died. 

But\marginnote{2.25} his wife complained and criticized those monks, “These Sakyan monastics are shameless and immoral liars. They claim to have integrity, to be celibate and of good conduct, to be truthful, moral, and good. But they don’t have the good character of a monastic or brahmin. They’ve lost the plot! They praised death to my husband, and as a result my husband is dead.” 

And\marginnote{2.32} other people complained and criticized them in the same way. 

The\marginnote{2.38} monks heard the criticism of those people. Those monks who had few desires and a sense of conscience, who were contented, afraid of wrongdoing, and fond of the training, complained and criticized those monks, “How could they praise death to that lay follower?” 

After\marginnote{2.41} rebuking those monks in many ways, they told the Buddha … 

“Is\marginnote{2.42} it true, monks, that you did this?” 

“It’s\marginnote{2.43} true, sir.” 

The\marginnote{2.44} Buddha rebuked them, “Foolish men, it’s not suitable, it’s not proper, it’s not worthy of a monastic, it’s not allowable, it’s not to be done. How could you do this? This will affect people’s confidence …” … “And so, monks, this training rule should be recited like this: 

\subsection*{Final ruling }

\scrule{‘If a monk intentionally kills a human being or seeks an instrument of death for them or praises death or incites someone to die, saying,\footnote{“An instrument of death” renders \textit{\textsanskrit{satthahāraka}}. I follow Richard Gombrich’s interpretation of this word. See “The Mass Suicide of Monks in Discourse and Vinaya Literature”, by Analayo, Journal of the Oxford Centre for Buddhist Studies, 7: 11–55, 2014. } “My friend, what’s the point of this miserable and difficult life? Death is better for you than life!”—thinking and intending thus, if he praises death in many ways or incites someone to die—he too is expelled and excluded from the community.’” }

\subsection*{Definitions }

\begin{description}%
\item[A: ] whoever … %
\item[Monk: ] … The monk who has been given the full ordination by a unanimous Sangha through a legal procedure consisting of one motion and three announcements that is irreversible and fit to stand—this sort of monk is meant in this case. %
\item[Intentionally: ] knowing, perceiving, having intended, having decided, he transgresses. %
\item[A human being: ] from the mind’s first appearance in the mother’s womb, from the first manifestation of consciousness, until the time of death: in between these—this is called “a human being”. %
\item[Kills: ] cuts off the life faculty, brings it to an end, interrupts its continuation. %
\item[Or seeks an instrument of death for them: ] a sword, a dagger, an arrow, a club, a rock, a knife, poison, or a rope. %
\item[Or praises death: ] he shows the drawbacks of living and speaks in praise of death. %
\item[Or incites someone to die: ] he says, “Kill yourself with a knife,” “Eat poison,” “Die by hanging yourself with a rope.” %
\item[My friend: ] this is a form of address. %
\item[What’s the point of this miserable and difficult life: ] Miserable life: the life of the poor is miserable compared to the life of the rich; the life of the impoverished is miserable compared to the life of the wealthy; the life of humans is miserable compared to the life of the gods. %
\item[Difficult life: ] the life of one whose hands are cut off, whose feet are cut off, whose hands and feet are cut off, whose ears are cut off, whose nose is cut off, whose ears and nose are cut off. Because of this sort of miserableness and this sort of difficult life, one says, “Death is better for you than life!” %
\item[Thinking: ] mind and thought are equivalent. %
\item[Intending: ] perceiving death, intending death, aiming at death. %
\item[In many ways: ] in various manners. %
\item[He praises death: ] he shows the drawbacks of living and speaks in praise of death, saying, “When you’ve passed away, you’ll be reborn in a happy destination, in heaven. There you’ll be able to enjoy the pleasures of heaven.” %
\item[Or incites someone to die: ] he says, “Kill yourself with a knife,” “Eat poison,” “Die by hanging yourself with a rope,” “Jump into a chasm,” “Jump into a pit,” “Jump off a cliff.” %
\item[He too: ] this is said with reference to the preceding offenses entailing expulsion. %
\item[Is expelled: ] just as an ordinary stone that has broken in half cannot be put back together again, so too is a monk who has intentionally killed a human being not an ascetic, not a Sakyan monastic. Therefore it is said, “he is expelled.” %
\item[Excluded from the community: ] Community: joint legal procedures, a joint recitation, the same training—this is called “community”. He does not take part in this—therefore it is called “excluded from the community”. %
\end{description}

\subsection*{Permutations }

\paragraph*{Summary }

Oneself,\marginnote{4.1.1} having made a determination, by messenger, by a series of messengers, by a messenger who does not follow instructions, by a messenger gone and returned again. 

Not\marginnote{4.1.2} in private, but perceiving it as private. In private, but perceiving it as not private. Not in private, and perceiving it as not private. In private, and perceiving it as private. 

He\marginnote{4.1.6} praises by means of the body. He praises by means of speech. He praises by means of both the body and speech. He praises by means of a messenger. He praises by means of writing. 

A\marginnote{4.1.11} pit, a piece of furniture, placing near, tonic, arranging a sight, arranging a sound, arranging a smell, arranging a taste, arranging a touch, arranging a mental quality, information, instruction, acting by arrangement, making a sign. 

\paragraph*{Exposition }

\begin{description}%
\item[Oneself: ] one oneself kills by means of the body or by means of something connected to the body or by means of something released. %
\item[Having made a determination: ] having made a determination, he tells someone, “Hit thus, strike thus, kill thus.” %
\item[By messenger: ] If\marginnote{4.2.5} a monk tells a second monk, “Kill so-and-so,” he commits an offense of wrong conduct. If the second monk kills that person, thinking it is the one he was told to kill, there is an offense entailing expulsion for both. 

If\marginnote{4.2.7} a monk tells a second monk, “Kill so-and-so,” he commits an offense of wrong conduct. If the second monk kills another person, thinking it is the one he was told to kill, there is no offense for the instigator, but there is an offense entailing expulsion for the murderer. 

If\marginnote{4.2.9} a monk tells a second monk, “Kill so-and-so,” he commits an offense of wrong conduct. If the second monk kills that person, thinking it is someone other than the one he was told to kill, there is an offense entailing expulsion for both. 

If\marginnote{4.2.11} a monk tells a second monk, “Kill so-and-so,” he commits an offense of wrong conduct. If the second monk kills another person, thinking it is someone other than the one he was told to kill, there is no offense for the instigator, but there is an offense entailing expulsion for the murderer. 

%
\item[By a series of messengers: ] If a monk tells a second monk, “Tell so-and-so to tell so-and-so to kill so-and-so,” he commits an offense of wrong conduct. In telling the next person, there is an offense of wrong conduct. If the potential murderer agrees, there is a serious offense for the instigator. If he kills that person, there is an offense entailing expulsion for all of them. %
\item[By a messenger who does not follow instructions: ] If a monk tells a second monk, “Tell so-and-so to tell so-and-so to kill so-and-so,” he commits an offense of wrong conduct. If the other monk tells another person than the one he was told to tell, he commits an offense of wrong conduct. If the potential murderer agrees, there is an offense of wrong conduct. If he kills that person, there is no offense for the instigator, but there is an offense entailing expulsion for the messenger and for the murderer. %
\item[By a messenger gone and returned again: ] If\marginnote{4.2.22} a monk tells a second monk, “Kill so-and-so,” he commits an offense of wrong conduct. He goes, but returns, saying, “I wasn’t able to kill them.” If the first monk tells him again, “When you’re able, then kill them,” he commits an offense of wrong conduct. If the second monk kills that person, there is an offense entailing expulsion for both. 

If\marginnote{4.2.26} a monk tells a second monk, “Kill so-and-so,” he commits an offense of wrong conduct. He then regrets it, but does not say, “Don’t kill them.” If the second monk then kills that person, there is an offense entailing expulsion for both. 

If\marginnote{4.2.29} a monk tells a second monk, “Kill so-and-so,” he commits an offense of wrong conduct. He then regrets it and says, “Don’t kill them.” If the second monk replies, “I’ve been told by you to do so,” and then kills that person, there is no offense for the instigator, but there is an offense entailing expulsion for the murderer. 

If\marginnote{4.2.32} a monk tells a second monk, “Kill so-and-so,” he commits an offense of wrong conduct. He then regrets it and says, “Don’t kill them.” If the second monk replies, “Fine,” and desists, there is no offense for either. 

%
\item[Not in private, but perceiving it as private: ] if he says aloud, “I wish so-and-so was killed,” he commits an offense of wrong conduct. %
\item[In private, but perceiving it as not private: ] if he says aloud, “I wish so-and-so was killed,” he commits an offense of wrong conduct. %
\item[Not in private, and perceiving it as not private: ] if he says aloud, “I wish so-and-so was killed,” he commits an offense of wrong conduct. %
\item[In private, and perceiving it as private: ] if he says aloud, “I wish so-and-so was killed,” he commits an offense of wrong conduct. %
\end{description}

\begin{description}%
\item[He praises by means of the body: ] if a monk makes a gesture with the body, indicating, “Whoever dies thus, receives wealth,” or, “Whoever dies thus, becomes famous,” or, “Whoever dies thus, goes to heaven,” he commits an offense of wrong conduct. If, because of that praise, the target person thinks, “I shall die,” and they do something painful, the monk commits a serious offense. If the person dies, the monk commits an offense entailing expulsion. %
\item[He praises by means of speech: ] if a monk says, “Whoever dies thus, receives wealth,” or, “Whoever dies thus, becomes famous,” or, “Whoever dies thus, goes to heaven,” he commits an offense of wrong conduct. If, because of that praise, the target person thinks, “I shall die,” and they do something painful, the monk commits a serious offense. If the person dies, the monk commits an offense entailing expulsion. %
\item[He praises by means of the body and speech: ] if a monk makes a gesture with the body and says, “Whoever dies thus, receives wealth,” or, “Whoever dies thus, becomes famous,” or, “Whoever dies thus, goes to heaven,” he commits an offense of wrong conduct. If, because of that praise, the target person thinks, “I shall die,” and they do something painful, the monk commits a serious offense. If the person dies, the monk commits an offense entailing expulsion. %
\item[He praises by means of a messenger: ] if a monk gives instructions to a messenger, saying, “Whoever dies thus, receives wealth,” or, “Whoever dies thus, becomes famous,” or, “Whoever dies thus, goes to heaven,” he commits an offense of wrong conduct. If, after hearing the messenger’s instruction, the target person thinks, “I shall die,” and they do something painful, the monk commits a serious offense. If the person dies, the monk commits an offense entailing expulsion. %
\item[He praises by means of writing: ] if a monk writes, “Whoever dies thus, receives wealth,” or, “Whoever dies thus, becomes famous,” or, “Whoever dies thus, goes to heaven,” he commits an offense of wrong conduct for each character he writes. If, after seeing the writing, the target person thinks, “I shall die,” and they do something painful, the monk commits a serious offense. If the person dies, the monk commits an offense entailing expulsion. %
\end{description}

\begin{description}%
\item[A pit: ] if\marginnote{4.5.2} a monk digs a pit for a human being, thinking, “Falling into it, they will die,” he commits an offense of wrong conduct. If a person falls into it and experiences pain, the monk commits a serious offense.\footnote{Sp 1.176: \textit{\textsanskrit{Yaṁ} uddissa khanati, tassa \textsanskrit{dukkhuppattiyā} \textsanskrit{thullaccayaṁ}, \textsanskrit{maraṇena} \textsanskrit{pārājikaṁ}. \textsanskrit{Aññasmiṁ} \textsanskrit{patitvā} mate \textsanskrit{anāpatti}}, “If the person he dug the pit for experiences pain, there is a serious offense. If he dies, there is an offense entailing expulsion. If someone else falls in and dies, there is no offense.” The commentary makes a point that goes beyond what is found in the Canonical text. According to the Canonical text, if you dig a pit for any human being to fall into, then the offense occurs when any human being falls into it. The commentary, however, makes the additional point that if you dig a pit for a specific human to fall into, you only commit an offense if that same human being falls into it. } If the person dies, the monk commits an offense entailing expulsion. 

If\marginnote{4.5.5} a monk digs a non-specific pit, thinking, “Whatever falls into it, will die,” he commits an offense of wrong conduct.\footnote{Sp 1.176: \textit{Sace anuddissa “yo koci \textsanskrit{marissatī}”ti khato hoti, \textsanskrit{yattakā} \textsanskrit{patitvā} maranti, \textsanskrit{tattakā} \textsanskrit{pāṇātipātā}}, “If it is not dug for a specific person or being, but with the thought, ‘Whatever dies’, then, to the extent that a being falls in and dies, there is killing.” And therefore an offense entailing expulsion if that being is human. } If a person falls into it, the monk commits an offense of wrong conduct. If they experience pain after falling in, the monk commits a serious offense. If they die, the monk commits an offense entailing expulsion. If a spirit, ghost, or animal in human form falls into it, the monk commits an offense of wrong conduct.\footnote{Kkh-\textsanskrit{pṭ}: \textit{Manussaviggaho \textsanskrit{nāgasupaṇṇādisadiso} \textsanskrit{tiracchānagato} \textsanskrit{tiracchānagatamanussaviggaho}}, “\textit{\textsanskrit{Tiracchānagatamanussaviggaha}}: an animal like a dragon or a \textit{garuda}, etc., in human form.” } If it experiences pain after falling in, the monk commits an offense of wrong conduct. If it dies, the monk commits a serious offense. If an animal falls into it, the monk commits an offense of wrong conduct. If it experiences pain after falling in, the monk commits an offense of wrong conduct. If it dies, the monk commits an offense entailing confession. 

%
\item[A piece of furniture: ] if a monk places a dagger in a piece of furniture, smears the furniture with poison, or makes it weak, or if he places it near a lake, a pit, or a cliff, thinking, “Falling down, they’ll die,” he commits an offense of wrong conduct. If the target person experiences pain because of the dagger, the poison, or the fall, the monk commits a serious offense. If the person dies, the monk commits an offense entailing expulsion. %
\item[Placing near: ] if a monk places a knife, a dagger, an arrow, a club, a rock, a sword, poison, or a rope near a person, thinking, “Using this, they’ll die,” he commits an offense of wrong conduct. If the target person thinks, “Using that, I shall die,” and he does something painful, the monk commits a serious offense. If the person dies, the monk commits an offense entailing expulsion. %
\item[Tonics: ] if a monk gives a person ghee, butter, oil, honey, or syrup, thinking, “After tasting this, they’ll die,” he commits an offense of wrong conduct.\footnote{“Syrup” renders \textit{\textsanskrit{phāṇita}}. I. B. Horner instead translates it as “molasses”, which doesn’t quite hit the mark. SED defines \textit{\textsanskrit{phāṇita}} as “the inspissated juice of the sugar cane or other plants”, in other words, “cane syrup”. According to the commentary at Sp 1.623 it can be either cooked or uncooked, the difference presumably whether it is raw or concentrated. “Syrup” seems closer to the mark than “molasses”. } If the target person tastes it and experiences pain, the monk commits a serious offense. If the person dies, the monk commits an offense entailing expulsion. %
\end{description}

\begin{description}%
\item[Arranging a sight: ] if a monk arranges a dreadful and terrifying sight, thinking, “Seeing this and becoming terrified, they’ll die,” he commits an offense of wrong conduct. If the target person sees it and becomes terrified, the monk commits a serious offense. If the person dies, the monk commits an offense entailing expulsion. If a monk arranges a lovely sight, thinking, “Seeing this and then being unable to get hold of it, they’ll wither and die,” he commits an offense of wrong conduct. If the target person sees it and then withers because of not getting hold of it, the monk commits a serious offense. If the person dies, the monk commits an offense entailing expulsion. %
\item[Arranging a sound: ] if a monk arranges a dreadful and terrifying sound, thinking, “Hearing this and becoming terrified, they’ll die,” he commits an offense of wrong conduct. If the target person hears it and becomes terrified, the monk commits a serious offense. If the person dies, the monk commits an offense entailing expulsion. If a monk arranges a lovely and heart-stirring sound, thinking, “Hearing this and then being unable to get hold of it, they’ll wither and die,” he commits an offense of wrong conduct. If the target person hears it and then withers because of not getting hold of it, the monk commits a serious offense. If the person dies, the monk commits an offense entailing expulsion. %
\item[Arranging a smell: ] if a monk arranges a disgusting and repulsive smell, thinking, “Smelling this, they’ll die from disgust and repulsion,” he commits an offense of wrong conduct. If the target person smells it and experiences suffering because of disgust and revulsion, the monk commits a serious offense. If the person dies, the monk commits an offense entailing expulsion. If a monk arranges a fragrant scent, thinking, “Smelling this and then being unable to get hold of it, they’ll wither and die,” he commits an offense of wrong conduct. If the target person smells it and then withers because of not getting hold of it, the monk commits a serious offense. If the person dies, the monk commits an offense entailing expulsion. %
\item[Arranging a taste: ] if a monk arranges a disgusting and repulsive flavor, thinking, “Tasting this, they’ll die from disgust and repulsion,” he commits an offense of wrong conduct. If the target person tastes it and experiences suffering because of disgust and repulsion, the monk commits a serious offense. If the person dies, the monk commits an offense entailing expulsion. If a monk arranges a delicious flavor, thinking, “Tasting this and then being unable to get hold of it, they’ll wither and die,” he commits an offense of wrong conduct. If the target person tastes it and then withers because of not getting hold of it, the monk commits a serious offense. If the person dies, the monk commits an offense entailing expulsion. %
\item[Arranging a touch: ] if a monk arranges a painful and harsh physical contact, thinking, “Touched by this, they’ll die,” he commits an offense of wrong conduct. If the target person makes contact with it and experiences pain, the monk commits a serious offense. If the person dies, the monk commits an offense entailing expulsion. If a monk arranges a pleasant and soft physical contact, thinking, “Touched by this and then being unable to get hold of it, they’ll wither and die,” he commits an offense of wrong conduct. If the target person is touched by it and then withers because of not getting hold of it, the monk commits a serious offense. If the person dies, the monk commits an offense entailing expulsion. %
\item[Arranging a mental quality: ] if a monk talks about hell to someone bound for hell, thinking, “Hearing this and becoming terrified, they’ll die,” he commits an offense of wrong conduct. If the target person hears it and becomes terrified, the monk commits a serious offense. If the person dies, the monk commits an offense entailing expulsion. If a monk talks about heaven to someone of good behavior, thinking, “Hearing this and being keen on it, they’ll die,” he commits an offense of wrong conduct. If the target person hears it, becomes keen on it, and thinks, “I shall die,” and they do something painful, the monk commits a serious offense. If the person dies, the monk commits an offense entailing expulsion. %
\end{description}

\begin{description}%
\item[Information: ] if, being asked, a monk says, “Die like this. Anyone who does receives wealth,” or, “Die like this. Anyone who does becomes famous,” or, “Die like this. Anyone who does goes to heaven,” he commits an offense of wrong conduct. If, because of that information, the target person thinks, “I shall die,” and they do something painful, the monk commits a serious offense. If the person dies, the monk commits an offense entailing expulsion. %
\item[Instruction: ] if, without being asked, a monk says, “Die like this. Anyone who does receives wealth,” or, “Die like this. Anyone who does becomes famous,” or, “Die like this. Anyone who does goes to heaven,” he commits an offense of wrong conduct. If, because of that instruction, the target person thinks, “I shall die,” and they do something painful, the monk commits a serious offense. If the person dies, the monk commits an offense entailing expulsion. %
\item[Acting by arrangement: ] if a monk makes an arrangement for before the meal or for after the meal, for the night or for the day, telling another person, “Kill that person according to this arrangement,” he commits an offense of wrong conduct. If the other person kills that person according to that arrangement, there is an offense entailing expulsion for both. If he kills him before or after the time of the arrangement, there is no offense for the instigator, but there is an offense entailing expulsion for the murderer. %
\item[Making a sign: ] a monk makes a sign. If he says to another person, “When I wink, at that sign kill that person,” “When I raise an eyebrow, at that sign kill that person,” or, “When I nod, at that sign kill that person,” he commits an offense of wrong conduct. If, at that sign, the other person kills that person, there is an offense entailing expulsion for both. If he kills him before or after the sign, there is no offense for the instigator, but there is an offense entailing expulsion for the murderer. %
\end{description}

\subsection*{Non-offenses }

There\marginnote{4.11.1} is no offense: if it is unintentional; if he does not know; if he is not aiming at death; if he is insane; if he is the first offender. 

\scend{The first section for recitation on expulsion in relation to human beings is finished. }

\scuddanaintro{Summary verses of case studies }

\begin{scuddana}%
“Praising,\marginnote{4.11.9} sitting down, \\
And with pestle, with mortar; \\
Gone forth when old, flowing out, \\
First, experimental poison. 

And\marginnote{4.11.13} three with making sites, \\
Another three with bricks; \\
And also machete, and rafter. \\
An elevated platform, coming down, fell down. 

Sweating,\marginnote{4.11.17} and nose treatment, massage, \\
By bathing, and by rubbing; \\
Making get up, making lie down, \\
Death through food, death through drink. 

Child\marginnote{4.11.21} by a lover, and co-wives; \\
mother, child, he killed both, \\
he killed neither; crushing, \\
Heating, barren, fertile. 

Tickling,\marginnote{4.11.25} in taking hold of, a spirit, \\
And predatory spirits, sending; \\
Thinking it was them, he gave a blow, \\
In talking about heaven, and about hell. 

Three\marginnote{4.11.29} trees at \textsanskrit{Āḷavī}, \\
Three others with forest groves; \\
Don’t torture, no I can’t, \\
Buttermilk, and salty purgative.” 

%
\end{scuddana}

\subsubsection*{Case studies }

On\marginnote{5.1.1} one occasion a certain monk was sick. Out of compassion, the monks praised death to him. He died. They became anxious and said, “The Buddha has laid down a training rule. Could it be that we’ve committed an offense entailing expulsion?” They told the Buddha. “You’ve committed an offense entailing expulsion.” 

On\marginnote{5.2.1} one occasion an alms-collecting monk sat down on a bench, crushing a boy who was concealed by an old cloth. The boy died. The monk became anxious and thought, “The Buddha has laid down a training rule. Could it be that I’ve committed an offense entailing expulsion?” He told the Buddha. “There’s no offense entailing expulsion. 

\scrule{But you shouldn’t sit down on a seat without checking it. If you do, you commit an offense of wrong conduct.” }

On\marginnote{5.3.1} one occasion a monk was preparing a seat in a dining hall in an inhabited area. When he took hold of a pestle high up, a second pestle fell down, hitting a boy, who died. The monk became anxious … “What were you thinking?” 

“I\marginnote{5.3.6} didn’t intend it, sir.” 

“There’s\marginnote{5.3.7} no offense when it’s unintentional.” 

On\marginnote{5.3.8} one occasion a monk was preparing a seat in a dining hall in an inhabited area when he forcefully kicked the implements belonging to a mortar.\footnote{Sp-\textsanskrit{ṭ} 1.180: \textit{\textsanskrit{Udukkhalabhaṇḍikanti} \textsanskrit{udukkhalatthāya} \textsanskrit{ānītaṁ} \textsanskrit{dārubhaṇḍaṁ}}, “\textit{\textsanskrit{Udukkhalabhaṇḍika}}: wooden goods brought for the sake of a mortar.” That \textit{akkamati} can mean “kick” can be seen from \href{https://suttacentral.net/pli-tv-bu-vb-pj1/en/brahmali\#10.20.9}{Bu Pj 1:10.20.9}: \textit{\textsanskrit{Aññatarā} \textsanskrit{itthī} \textsanskrit{passitvā} \textsanskrit{aṅgajāte} \textsanskrit{abhinisīdi}}, “The monk kicked her off.” } They crushed a boy, who died. He became anxious … “There’s no offense when it’s unintentional.” 

At\marginnote{5.4.1} one time a father and son had gone forth with the monks. When the time was announced for a certain event, the son said to his father,\footnote{“For a certain event” is not found in the Pali, but has been added to make the Pali idiom clearer. Sp 2.108: \textit{\textsanskrit{Kāle} \textsanskrit{ārociteti} \textsanskrit{yāgubhattādīsu} yassa kassaci \textsanskrit{kāle} \textsanskrit{ārocite}}, “\textit{\textsanskrit{Kāle} \textsanskrit{ārocite}}: when the time is announced for whatever, such as rice porridge, a meal, etc.” } “Go, sir, the Sangha is waiting for you,” and seizing him by the back, he pushed him. The father fell and died. The son became anxious … “What were you thinking?” 

“I\marginnote{5.4.8} didn’t mean to kill him, sir.” 

“There’s\marginnote{5.4.9} no offense for one who isn’t aiming at death.” 

At\marginnote{5.4.10} one time a father and son had gone forth with the monks. When the time was announced for a certain event, the son said to his father, “Go, sir, the Sangha is waiting for you,” and seizing him by the back, he pushed him, aiming to kill him. The father fell and died. The son became anxious … “You have committed an offense entailing expulsion.” 

At\marginnote{5.4.17} one time a father and son had gone forth with the monks. When the time was announced for a certain event, the son said to his father, “Go, sir, the Sangha is waiting for you,” and seizing him by the back, he pushed him, aiming to kill him. The father fell, but did not die. The son became anxious … “There’s no offense entailing expulsion, but there’s a serious offense.” 

On\marginnote{5.5.1} one occasion a monk got meat stuck in his throat while eating. A second monk hit him on the neck. The meat was expelled together with blood, and the monk died. The second monk became anxious … “There’s no offense for one who isn’t aiming at death.” 

On\marginnote{5.5.7} one occasion a monk got meat stuck in his throat while eating. Another monk hit him on the neck, aiming to kill him. The meat was expelled together with blood, and the monk died. The second monk became anxious … “You have committed an offense entailing expulsion.” 

On\marginnote{5.5.13} one occasion a monk got meat stuck in his throat while eating. Another monk hit him on the neck, aiming to kill him. The meat was expelled together with blood, but the monk did not die. He became anxious … “There’s no offense entailing expulsion, but there’s a serious offense.” 

On\marginnote{5.6.1} one occasion an alms-collecting monk received poisoned almsfood. He brought it back and gave the first portion to other monks. They died. He became anxious … “What were you thinking, monk?” 

“I\marginnote{5.6.5} didn’t know, sir.” 

“There’s\marginnote{5.6.6} no offense for one who doesn’t know.” 

On\marginnote{5.6.7} one occasion a monk gave poison to a second monk with the purpose of investigating it. That monk died. The first monk became anxious … “What were you thinking, monk?” 

“My\marginnote{5.6.11} purpose was to investigate it, sir.” 

“There’s\marginnote{5.6.12} no offense entailing expulsion, but there’s a serious offense.” 

On\marginnote{5.7.1} one occasion the monks of \textsanskrit{Āḷavī} were preparing a site for a dwelling when a monk lifted up a stone to another monk above him. As the second monk did not grasp it properly, it fell on the head of the monk below, who died. The second monk became anxious … “There’s no offense when it’s unintentional.” 

On\marginnote{5.7.7} one occasion the monks of \textsanskrit{Āḷavī} were preparing a site for a dwelling when a monk lifted up a stone to another monk above him. The second monk dropped the stone on his head, aiming to kill him. He died. …\footnote{This is to be expanded as in segments 5.5.7–5.5.19 above. } He did not die. The second monk became anxious … “There’s no offense entailing expulsion, but there’s a serious offense.” 

On\marginnote{5.8.1} one occasion the monks of \textsanskrit{Āḷavī} were building a wall for a dwelling when a monk lifted up a brick to another monk above him. As the second monk did not grasp it properly, it fell on the head of the monk below, who died. The second monk became anxious … “There’s no offense when it’s unintentional.” 

On\marginnote{5.8.7} one occasion the monks of \textsanskrit{Āḷavī} were building a wall for a dwelling when a monk lifted up a brick to another monk above him. The second monk dropped the brick on his head, aiming to kill him. He died. … He did not die. The second monk became anxious … “There’s no offense entailing expulsion, but there’s a serious offense.” 

On\marginnote{5.9.1} one occasion the monks of \textsanskrit{Āḷavī} were doing building work when a monk lifted up a machete to another monk above him. As the second monk did not grasp it properly, it fell on the head of the monk below, who died. The second monk became anxious … “There’s no offense when it’s unintentional.” 

On\marginnote{5.9.7} one occasion the monks of \textsanskrit{Āḷavī} were doing building work when a monk lifted up a machete to another monk above him. The second monk dropped the machete on his head, aiming to kill him. He died. … He did not die. The second monk became anxious … “There’s no offense entailing expulsion, but there’s a serious offense.” 

On\marginnote{5.10.1} one occasion the monks of \textsanskrit{Āḷavī} were doing building work when a monk lifted up a rafter to another monk above him. As the second monk did not grasp it properly, it fell on the head of the monk below, who died. The second monk became anxious … “There’s no offense when it’s unintentional.” 

On\marginnote{5.10.7} one occasion the monks of \textsanskrit{Āḷavī} were doing building work when a monk lifted up a rafter to another monk above him. The second monk dropped the rafter on his head, aiming to kill him. He died. … He did not die. The second monk became anxious … “There’s no offense entailing expulsion, but there’s a serious offense.” 

On\marginnote{5.11.1} one occasion the monks of \textsanskrit{Āḷavī} were putting together an elevated platform while doing building work. A monk said to another monk, “Put it together while standing here.” He did, and he fell down and died. The first monk became anxious … “What were you thinking, monk?” 

“I\marginnote{5.11.7} didn’t mean to kill him, sir.” 

“There’s\marginnote{5.11.8} no offense for one who isn’t aiming at death.” 

On\marginnote{5.11.9} one occasion the monks of \textsanskrit{Āḷavī} were putting together an elevated platform while doing building work. A monk said to another monk, “Put it together while standing here,” aiming to kill him. He did, and he fell down and died. … he fell down, but did not die. The first monk became anxious … “There’s no offense entailing expulsion, but there’s a serious offense.” 

On\marginnote{5.12.1} one occasion a monk was coming down after roofing a dwelling. A second monk said to him, “Come down here.” He did, and he fell down and died. The second monk became anxious … “There’s no offense for one who isn’t aiming at death.” 

On\marginnote{5.12.7} one occasion a monk was coming down after roofing a dwelling. A second monk said to him, “Come down here,” aiming to kill him. He did, and he fell down and died. … he fell down, but did not die. The first monk became anxious … “There’s no offense entailing expulsion, but there’s a serious offense.” 

On\marginnote{5.13.1} one occasion a monk who was plagued by lust climbed the Vulture Peak, jumped off the cliff, and hit a basket-maker. The basket-maker died,\footnote{The Pali seem to say that he simply fell off the cliff, \textit{papatanto}, but the context seems to require something more deliberate. Perhaps it is just an idiom. } and the monk became anxious … “There’s no offense entailing expulsion. 

\scrule{But, monks, you shouldn’t jump off anything. If you do, you commit an offense of wrong conduct.” }

On\marginnote{5.13.5} one occasion the monks from the group of six climbed the Vulture Peak and threw down a stone for fun. It hit a cowherd, who died. They became anxious … “There’s no offense entailing expulsion. 

\scrule{But, monks, you shouldn’t throw down stones for fun. If you do, you commit an offense of wrong conduct.” }

On\marginnote{5.14.1} one occasion a certain monk was sick. The monks made him sweat by heating him. He died. They became anxious … “There’s no offense for one who isn’t aiming at death.” 

On\marginnote{5.14.6} one occasion a certain monk was sick. The monks made him sweat by heating him, aiming to kill him. He died. … He did not die. The monks became anxious … “There’s no offense entailing expulsion, but there’s a serious offense.” 

On\marginnote{5.15.1} one occasion a monk had a severe headache. The monks gave him medical treatment through the nose. He died. They became anxious … “There’s no offense for one who isn’t aiming at death.” 

On\marginnote{5.15.6} one occasion a certain monk had a severe headache. The monks gave him medical treatment through the nose, aiming to kill him. He died. … He did not die. The monks became anxious … “There’s no offense entailing expulsion, but there’s a serious offense.” 

On\marginnote{5.16.1} one occasion a certain monk was sick. The monks massaged him. He died. They became anxious … “There’s no offense for one who isn’t aiming at death.” 

On\marginnote{5.16.6} one occasion a certain monk was sick. The monks massaged him, aiming to kill him. He died. … He did not die. The monks became anxious … “There’s no offense entailing expulsion, but there’s a serious offense.” 

On\marginnote{5.16.13} one occasion a certain monk was sick. The monks bathed him. He died. They became anxious … “There’s no offense for one who isn’t aiming at death.” 

On\marginnote{5.16.18} one occasion a certain monk was sick. The monks bathed him, aiming to kill him. He died. … He did not die. The monks became anxious … “There’s no offense entailing expulsion, but there’s a serious offense.” 

On\marginnote{5.16.25} one occasion a certain monk was sick. The monks rubbed him with oil. He died. They became anxious … “There’s no offense for one who isn’t aiming at death.” 

On\marginnote{5.16.30} one occasion a certain monk was sick. The monks rubbed him with oil, aiming to kill him. He died. … He did not die. The monks became anxious … “There’s no offense entailing expulsion, but there’s a serious offense.” 

On\marginnote{5.16.37} one occasion a certain monk was sick. The monks made him get up. He died. They became anxious … “There’s no offense for one who isn’t aiming at death.” 

On\marginnote{5.16.42} one occasion a certain monk was sick. The monks made him get up, aiming to kill him. He died. … He did not die. The monks became anxious … “There’s no offense entailing expulsion, but there’s a serious offense.” 

On\marginnote{5.16.49} one occasion a certain monk was sick. The monks made him lie down. He died. They became anxious … “There’s no offense for one who isn’t aiming at death.” 

On\marginnote{5.16.54} one occasion a certain monk was sick. The monks made him lie down, aiming to kill him. He died. … He did not die. The monks became anxious … “There’s no offense entailing expulsion, but there’s a serious offense.” 

On\marginnote{5.16.61} one occasion a certain monk was sick. The monks gave him food. He died. They became anxious … “There’s no offense for one who isn’t aiming at death.” 

On\marginnote{5.16.66} one occasion a certain monk was sick. The monks gave him food, aiming to kill him. He died. … He did not die. The monks became anxious … “There’s no offense entailing expulsion, but there’s a serious offense.” 

On\marginnote{5.16.73} one occasion a certain monk was sick. The monks gave him a drink. He died. They became anxious … “There’s no offense for one who isn’t aiming at death.” 

On\marginnote{5.16.78} one occasion a certain monk was sick. The monks gave him a drink, aiming to kill him. He died. … He did not die. The monks became anxious … “There’s no offense entailing expulsion, but there’s a serious offense.” 

At\marginnote{5.17.1} one time a woman whose husband was living away from home became pregnant by a lover. She said to a monk who associated with her family, “Venerable, please help me have an abortion.” “Alright,” he said, and he helped her have an abortion. The child died. The monk became anxious … “You have committed an offense entailing expulsion.” 

At\marginnote{5.18.1} one time a certain man had two wives, one barren and one fertile. The barren one said to a monk who associated with her family, “If the other wife gives birth to a son, venerable, she’ll become the head wife. Please make her have an abortion.” “Alright,” he said, and he did so. The child died, but the mother did not die. The monk became anxious … “You have committed an offense entailing expulsion.” 

At\marginnote{5.18.12} one time a certain man had two wives, one barren and one fertile. The barren one said to a monk who associated with her family, “If the other wife gives birth to a son, venerable, she’ll become the head wife. Please make her have an abortion.” “Alright,” he said, and he did so. The mother died, but the child did not die. The monk became anxious … “There’s no offense entailing expulsion,\footnote{It is not an offense entailing expulsion because he was aiming to kill the child, not the mother. } but there’s a serious offense.” 

At\marginnote{5.18.24} one time a certain man had two wives, one barren and one fertile. The barren one said to a monk who associated with her family, “If the other wife gives birth to a son, venerable, she’ll become the head wife. Please make her have an abortion.” “Alright,” he said, and he did so. Both died. … Neither died. The monk became anxious … “There’s no offense entailing expulsion, but there’s a serious offense.” 

On\marginnote{5.19.1} one occasion a woman who was pregnant said to a monk who associated with her family, “Venerable, please help me have an abortion.” “Well then, crush it,” he said. She crushed it and had an abortion. The monk became anxious … “You have committed an offense entailing expulsion.” 

On\marginnote{5.19.7} one occasion a woman who was pregnant said to a monk who associated with her family, “Venerable, please help me have an abortion.” “Well then, heat yourself,” he said. She heated herself and had an abortion. The monk became anxious … “You have committed an offense entailing expulsion.” 

On\marginnote{5.20.1} one occasion a barren woman said to a monk who associated with her family, “Please find some medicine, venerable, to help me become pregnant.” “Alright,” he said, and he gave her some medicine. She died. He became anxious … “There’s no offense entailing expulsion, but there’s an offense of wrong conduct.” 

On\marginnote{5.21.1} one occasion a fertile woman said to a monk who associated with her family, “Please find some medicine, venerable, to help me not become pregnant.” “Alright,” he said, and he gave her some medicine. She died. He became anxious … “There’s no offense entailing expulsion, but there’s an offense of wrong conduct.” 

On\marginnote{5.22.1} one occasion the monks from the group of six tickled a monk from the group of seventeen to make him laugh. Being unable to catch his breath, he died. They became anxious … “There’s no offense entailing expulsion.” 

On\marginnote{5.23.1} one occasion the monks from the group of seventeen overpowered a monk from the group of six, intending to do a legal procedure against him. He died.\footnote{The text doesn’t say what the legal procedure, the \textit{(\textsanskrit{saṅgha}-)kamma}, was about, but it seems implied that it was against that monk. This is also the position of the sub-commentary at Sp-\textsanskrit{ṭ} 1.187: \textit{\textsanskrit{Kammādhippāyāti} \textsanskrit{tajjanīyādikammakaraṇādhippāyā}}, “Aiming at a legal procedure: aiming at doing a legal procedure of censure, etc.” The monks from the group of seventeen were in regular conflict with the monks from the group of six. } They became anxious … “There’s no offense entailing expulsion.” 

On\marginnote{5.24.1} one occasion an exorcist monk killed a spirit. He became anxious … “There’s no offense entailing expulsion, but there’s a serious offense.” 

On\marginnote{5.25.1} one occasion a monk sent a second monk to a dwelling inhabited by predatory spirits. The spirits killed him. The first monk became anxious … “There’s no offense for one who isn’t aiming at death.” 

On\marginnote{5.25.5} one occasion a monk sent a second monk to a dwelling inhabited by predatory spirits, aiming to kill him. The spirits killed him. … The spirits did not kill him. The first monk became anxious … “There’s no offense entailing expulsion, but there’s a serious offense.” 

On\marginnote{5.26.1} one occasion a monk sent a second monk to a wilderness inhabited by predatory animals. The predatory animals killed him. The first monk became anxious … “There’s no offense for one who isn’t aiming at death.” 

On\marginnote{5.26.5} one occasion a monk sent a second monk to a wilderness inhabited by predatory animals, aiming to kill him. The predatory animals killed him. … The predatory animals did not kill him. The first monk became anxious … “There’s no offense entailing expulsion, but there’s a serious offense.” 

On\marginnote{5.26.11} one occasion a monk sent a second monk to a wilderness inhabited by criminals. The criminals killed him. The first monk became anxious … “There’s no offense for one who isn’t aiming at death.” 

On\marginnote{5.26.15} one occasion a monk sent a second monk to a wilderness inhabited by criminals, aiming to kill him. The criminals killed him. … The criminals did not kill him. The first monk became anxious … “There’s no offense entailing expulsion, but there’s a serious offense.” 

On\marginnote{5.27.1} one occasion a monk killed a person, thinking it was them … killed another person, thinking it was them … killed a person, thinking they were another … killed another person, thinking they were another. That monk became anxious … “You have committed an offense entailing expulsion.” 

At\marginnote{5.28.1} one time a monk was possessed by a spirit. Another monk gave him a blow. He died. The other monk became anxious … “There’s no offense for one who isn’t aiming at death.” 

At\marginnote{5.28.6} one time a monk was possessed by a spirit. A second monk gave him a blow, aiming to kill him. He died. … He did not die. The second monk became anxious … “There’s no offense entailing expulsion, but there’s a serious offense.” 

On\marginnote{5.29.1} one occasion a monk gave a talk about heaven to a man of good behavior. He became keen on it and died. The monk became anxious … “There’s no offense for one who isn’t aiming at death.” 

On\marginnote{5.29.5} one occasion a monk gave a talk about heaven to a man of good behavior, aiming to kill him. He became keen on it and died. … He became keen on it, but did not die. The monk became anxious … “There’s no offense entailing expulsion, but there’s a serious offense.” 

On\marginnote{5.29.11} one occasion a monk gave a talk about hell to a man bound for hell. He became terrified and died. The monk became anxious … “There’s no offense for one who isn’t aiming at death.” 

On\marginnote{5.29.15} one occasion a monk gave a talk about hell to a man bound for hell, aiming to kill him. He became terrified and died. … He became terrified, but did not die. The monk became anxious … “There’s no offense entailing expulsion, but there’s a serious offense.” 

On\marginnote{5.30.1} one occasion the monks of \textsanskrit{Āḷavī} were felling a tree while doing building work. A monk said to a second monk, “Fell it while standing here.” He did. The tree fell on him, and he died. The first monk became anxious … “There’s no offense for one who isn’t aiming at death.” 

On\marginnote{5.30.7} one occasion the monks of \textsanskrit{Āḷavī} were felling a tree while doing building work. A monk said to a second monk, “Fell it while standing here,” aiming to kill him. He did. The tree fell on him, and he died. … The tree fell on him, but he did not die. The first monk became anxious … “There’s no offense entailing expulsion, but there’s a serious offense.” 

On\marginnote{5.31.1} one occasion the monks from the group of six set fire to a forest grove. People were burned and died. The monks became anxious … “There’s no offense for one who isn’t aiming at death.” 

On\marginnote{5.31.5} one occasion the monks from the group of six set fire to a forest grove, aiming to cause death. People were burned and died. … People were burned, but did not die. The monks became anxious … “There’s no offense entailing expulsion, but there’s a serious offense.” 

On\marginnote{5.32.1} one occasion a monk went to a place of execution and said to the executioner, “Don’t torture him. Kill him with a single blow.” “Alright, sir,” he said, and he killed him with a single blow. The monk became anxious … “You have committed an offense entailing expulsion.” 

On\marginnote{5.32.8} one occasion a monk went to a place of execution and said to the executioner, “Don’t torture him. Kill him with a single blow.” Saying, “No, I can’t,” he executed him. The monk became anxious … “There’s no offense entailing expulsion, but there’s an offense of wrong conduct.” 

On\marginnote{5.33.1} one occasion a man whose hands and feet had been cut off was at his relatives’ house, surrounded by his relations. A monk said to those people, “Do you want to euthanize him?” 

“Yes,\marginnote{5.33.4} sir.” 

“Then\marginnote{5.33.5} give him buttermilk.” 

They\marginnote{5.33.6} gave him buttermilk and he died. The monk became anxious … “You have committed an offense entailing expulsion.” 

On\marginnote{5.33.10} one occasion a man whose hands and feet had been cut off was at home, surrounded by his relations. A nun said to those people, “Do you want to euthanize him?” 

“Yes,\marginnote{5.33.13} venerable.” 

“Then\marginnote{5.33.14} give him the salty purgative.” 

They\marginnote{5.33.15} gave him the salty purgative and he died. The nun became anxious … She then told the nuns, who in turn told the monks, who then told the Buddha. “Monks, that nun has committed an offense entailing expulsion.” 

\scendsutta{The third offense entailing expulsion is finished. }

%
\section*{{\suttatitleacronym Bu Pj 4}{\suttatitletranslation The fourth training rule on expulsion }{\suttatitleroot Uttarimanussadhamma}}
\addcontentsline{toc}{section}{\tocacronym{Bu Pj 4} \toctranslation{The fourth training rule on expulsion } \tocroot{Uttarimanussadhamma}}
\markboth{The fourth training rule on expulsion }{Uttarimanussadhamma}
\extramarks{Bu Pj 4}{Bu Pj 4}

\subsection*{Origin story }

\subsubsection*{First sub-story }

At\marginnote{1.1.1} one time when the Buddha was staying in the hall with the peaked roof in the Great Wood near \textsanskrit{Vesālī}, a number of monks who were friends had entered the rainy-season residence on the banks of the river \textsanskrit{Vaggumudā}. At that time \textsanskrit{Vajjī} was short of food and afflicted with hunger, with crops affected by whiteheads and turned to straw.\footnote{“Whiteheads” renders \textit{\textsanskrit{setaṭṭ}(h)\textsanskrit{ikā}}, literally, “white bones”. Sp 4.403: \textit{\textsanskrit{Setaṭṭhikā} \textsanskrit{nāma} \textsanskrit{rogajātīti} eko \textsanskrit{pāṇako} \textsanskrit{nāḷimajjhagataṁ} \textsanskrit{kaṇḍaṁ} vijjhati, yena \textsanskrit{viddhattā} nikkhantampi \textsanskrit{sālisīsaṁ} \textsanskrit{khīraṁ} \textsanskrit{gahetuṁ} na sakkoti}, “The disease called \textit{\textsanskrit{setaṭṭhikā}} means: an insect penetrates the stem, goes to the middle of the stalk, from the penetration of which the rice grains are not able to get sap.” This seems to be a description of so-called “whiteheads”, pale panicles without rice grains, caused by stem borers. } It was not easy to get by on almsfood. 

The\marginnote{1.1.5} monks considered the difficult circumstances, and they thought, “How can we have a comfortable rains, live in peace and harmony, and get almsfood without trouble?” 

Some\marginnote{1.1.8} said, “We could work for the householders, and they’ll support us in return.” 

Others\marginnote{1.1.13} said, “There’s no need to work for the householders. Let’s instead take messages for them, and they’ll support us in return.” 

Still\marginnote{1.1.19} others said, “There’s no need to work or take messages for them. Let’s instead talk up one another’s superhuman qualities to the householders: ‘That monk has the first absorption, that monk the second absorption, that monk the third, that monk the fourth; that monk is a stream-enterer, that monk a once-returner, that a non-returner, that a perfected one; that monk has the three true insights, and that the six direct knowledges.’ Then they’ll support us. In this way we’ll have a comfortable rains, live together in peace and harmony, and get almsfood without trouble. This is the way to go.” 

Then\marginnote{1.1.37} those monks did just that. And the people there thought, “We’re so fortunate that such monks have come to us for the rainy-season residence. Such virtuous and good monks have never before entered the rains residence with us.” And they gave such food and drink to those monks that they did not even eat and drink themselves, or give to their parents, to their wives and children, to their slaves, servants, and workers, to their friends and companions, or to their relatives. Soon those monks had a good color, bright faces, clear skin, and  sharp senses. 

Now\marginnote{1.2.1} it was the custom for monks who had completed the rainy-season residence to go and visit the Buddha. And so, when the three months were over and they had completed the rains residence, those monks put their dwellings in order, took their bowls and robes, and set out for \textsanskrit{Vesālī}. When they eventually arrived, they went to the hall with the peaked roof in the Great Wood. There they approached the Buddha, bowed, and sat down. 

At\marginnote{1.2.4} that time the monks who had completed the rains residence in that region were thin, haggard, and pale, with veins protruding all over their bodies. Yet the monks from the banks of the \textsanskrit{Vaggumudā} had a good color, bright faces, clear skin, and  sharp senses. Since it is the custom for Buddhas to greet newly-arrived monks, the Buddha said to them, “I hope you’re keeping well, monks, I hope you’re getting by? 

I\marginnote{1.2.9} hope you had a comfortable and harmonious rains, and got almsfood without trouble?” 

“We’re\marginnote{1.2.10} keeping well, sir, we’re getting by. We had a comfortable and harmonious rains, and got almsfood without trouble.” When Buddhas know what is going on, sometimes they ask and sometimes not. They know the right time to ask and when not to ask. Buddhas ask when it is beneficial, otherwise they do not, for Buddhas are incapable of doing what is unbeneficial.\footnote{“Incapable of doing” renders \textit{\textsanskrit{setughāta}}, literally, “destroyed the bridge”. Sp 1.16: \textit{Setu vuccati maggo, maggeneva \textsanskrit{tādisassa} vacanassa \textsanskrit{ghāto}, samucchedoti \textsanskrit{vuttaṁ} hoti}, “The path is called the bridge. What is said is that there is the destruction and cutting off of such speech by the path.” The commentary seems to take \textit{setu}, “bridge”, as a reference to the eightfold path. According to this understanding, the Buddha has cut off access to bad qualities, including bad speech, by fulfilling the eightfold path. I prefer to understand “bridge” as a metaphor for access, that is, the Buddhas no longer have access to what is unbeneficial. } Buddhas question the monks for two reasons: to give a teaching or to lay down a training rule. 

And\marginnote{1.2.14} the Buddha said to those monks, “In what way, monks, did you have a comfortable and harmonious rains? And how did you get almsfood without trouble?” 

They\marginnote{1.2.16} then told him. 

“But\marginnote{1.2.17} did you really have those superhuman qualities?” 

“No,\marginnote{1.2.18} sir.” 

The\marginnote{1.2.19} Buddha rebuked them, “It’s not suitable, foolish men, it’s not proper, it’s not worthy of a monastic, it’s not allowable, it’s not to be done. How could you for the sake of your stomachs talk up one another’s superhuman qualities to householders? It would be better for your bellies to be cut open with a sharp butcher’s knife than for you to talk up one another’s superhuman qualities to householders. Why is that? For although it might cause death or death-like suffering, it would not cause you to be reborn in a bad destination. But \emph{this} might. This will affect people’s confidence …” After rebuking them and giving a teaching, he addressed the monks: 

“Monks,\marginnote{1.3.1} there are these five notorious gangsters to be found in the world. What five? There is the notorious gangster who thinks like this: ‘When the heck will I walk about in villages, towns, and royal capitals, with a following of a hundred or a thousand men, killing, destroying, and torturing?’ Then after some time, he does just that. Just so, monks, a bad monk thinks like this: ‘When the heck will I walk about in villages, towns, and royal capitals, with a following of a hundred or a thousand people, being honored, respected, and revered by both lay people and those gone forth, getting robes, almsfood, dwellings, and medicinal supplies?’ Then after some time, he does just that. This is the first notorious gangster to be found in the world. 

Or\marginnote{1.3.12} a bad monk learns the spiritual path proclaimed by the Buddha and takes it as his own. This is the second notorious gangster to be found in the world. 

Or\marginnote{1.3.14} a bad monk groundlessly charges someone living a pure spiritual life with a sexual offense. This is the third notorious gangster to be found in the world. 

Or\marginnote{1.3.16} a bad monk takes valuable goods and requisites from the Sangha—a monastery, the land of a monastery, a dwelling, the site of a dwelling, a bed, a bench, a mattress, a pillow, a metal pot, a metal jar, a metal bucket, a metal bowl, a machete, a hatchet, an ax, a spade, a chisel, a creeper, bamboo, a reed, grass, clay, wooden goods, earthenware goods—\footnote{I have rendered \textit{\textsanskrit{muñja}}-reed and \textit{pabbaja}-reed with the single word “reed”. I am not aware that these two kinds of reed can be distinguished in English. │ I render \textit{\textsanskrit{vihāra}} as “dwelling”, the idea that it is a monastic dwelling being implied. In later usage, especially in the commentaries, \textit{\textsanskrit{vihāra}} comes to refer to entire monasteries, rather than individual dwellings. The commentaries seem to agree that in its early usage the word refers to a dwelling. Sp 1.493: \textit{\textsanskrit{Vihāro} nivesanasadiso}, “A \textit{\textsanskrit{vihāra}} is like a house.” } and uses them to bribe and create a following among householders. This is the fourth notorious gangster to be found in the world. 

But\marginnote{1.3.20} in this world with its gods, lords of death, and supreme beings, in this society with its monastics and brahmins, its gods and humans, this is the most notorious gangster of all: one who claims to have a non-existent superhuman quality. Why is that? Monks, you’ve eaten the country’s almsfood by theft.” 

\begin{verse}%
“Whoever\marginnote{1.3.23} should declare himself\footnote{This verse is also found at \href{https://suttacentral.net/sn1.35/en/brahmali\#2.1}{SN 1.35}. } \\
To be other than he truly is, \\
Has eaten this by theft, \\
Like a cheater who has deceived. 

Many\marginnote{1.3.27} ocher-necks of bad qualities, \\
Uncontrolled and wicked—\\
By their wicked deeds, \\
They are reborn in hell. 

It’s\marginnote{1.3.31} better to eat an iron ball, \\
As hot as a licking flame, \\
Than for the immoral and uncontrolled \\
To eat the country’s alms.” 

%
\end{verse}

After\marginnote{1.3.35} rebuking the monks from the banks of the \textsanskrit{Vaggumudā} in many ways for being difficult to maintain, difficult to support … “And, monks, this training rule should be recited like this: 

\subsubsection*{Preliminary ruling }

\scrule{‘If a monk falsely claims for himself a superhuman quality, a knowledge and vision worthy of the noble ones, saying, “This I know, this I see,” but after some time—whether questioned or not, but having committed the offense and seeking purification—should say: “Not knowing I said that I know, not seeing that I see; what I said was empty and false,” he too is expelled and excluded from the community.’” }

In\marginnote{1.3.40} this way the Buddha laid down this training rule for the monks. 

\subsubsection*{Second sub-story }

Soon\marginnote{2.1} afterwards a number of monks, thinking they had seen and realized what in fact they had not, declared final knowledge because of overestimation. After some time, their minds inclined to sensual desire, ill will, and confusion. They became anxious, thinking, “The Buddha has laid down a training rule, yet we declared final knowledge because of overestimation. Could it be that we’ve committed an offense entailing expulsion?” They told Venerable Ānanda, who told the Buddha. He said, “This is negligible, Ānanda. 

And\marginnote{2.11} so, monks, this training rule should be recited like this: 

\subsection*{Final ruling }

\scrule{‘If a monk falsely claims for himself a superhuman quality, a knowledge and vision worthy of the noble ones, saying, “This I know, this I see,” but after some time—whether he is questioned or not, but having committed the offense and seeking purification—should say: “Not knowing I said that I know, not seeing that I see; what I said was empty and false,” then, except if it is due to overestimation, he too is expelled and excluded from the community.’” }

\subsection*{Definitions }

\begin{description}%
\item[A: ] whoever … %
\item[Monk: ] … The monk who has been given the full ordination by a unanimous Sangha through a legal procedure consisting of one motion and three announcements that is irreversible and fit to stand—this sort of monk is meant in this case. %
\item[Falsely: ] although a certain wholesome quality is non-existent in himself, not real, not to be found, and he does not see it or know it, he says, “I have this wholesome quality.” %
\item[A superhuman quality: ] absorption, release, stillness, attainment, knowledge and vision, development of the path, realization of the fruits, abandoning the defilements, a mind without hindrances, delighting in solitude.\footnote{“Delighting in solitude” renders \textit{\textsanskrit{suññāgāra} abhirati}, literally, “delighting in an empty dwelling”. According to the commentaries, this is often an idiom for solitude. MN-a 1.88: \textit{Tattha ca \textsanskrit{rukkhamūlānīti} \textsanskrit{iminā} \textsanskrit{rukkhamūlasenāsanaṁ} dasseti. \textsanskrit{Suññāgārānīti} \textsanskrit{iminā} \textsanskrit{janavivittaṭṭhānaṁ}}; “And there \textit{\textsanskrit{rukkhamūlāni}}: by this is shown dwellings at the foot of a tree; \textit{\textsanskrit{suññāgārāni}}: by this (is shown) a place free from people.” } %
\item[For himself: ] either he presents those good qualities as in himself, or he presents himself as among those good qualities. %
\item[Knowledge: ] the three true insights. %
\item[Vision: ] knowledge and vision are equivalent. %
\item[Claims: ] announces to a woman or a man, to a lay person or one gone forth. %
\item[This I know, this I see: ] “I know these qualities,” “I see these qualities,” “These qualities are found in me and I conform to them.” %
\item[After some time: ] the moment, the second, the instant after he has made the claim. %
\item[He is questioned: ] he is questioned in regard to what he has claimed: “What did you attain?” “How did you attain it?” “When did you attain it?” “Where did you attain it?” “Which defilements did you abandon?” “Which qualities did you gain?” %
\item[Not: ] he is not spoken to by anyone. %
\item[Having committed the offense: ] having bad desires, overcome by desire, claiming to have a non-existent superhuman quality, he has committed an offense entailing expulsion. %
\item[Seeking purification: ] he desires to be a householder, a lay follower, a monastery worker, or a novice monk. %
\item[Not knowing I said that I know, not seeing that I see: ] “I don’t know these qualities,” “I don’t see these qualities,” “These qualities aren’t found in me and I don’t conform to them.” %
\item[What I said was empty and false: ] “What I said was empty,” “What I said was false,” “What I said was unreal,” “I said it without knowing.” %
\item[Except if it is due to overestimation: ] unless it is due to overestimation. %
\item[He too: ] this is said with reference to the preceding offenses entailing expulsion. %
\item[Is expelled: ] just as a palm tree with its crown cut off is incapable of further growth, so too is a monk with bad desires, overcome by desire, who claims to have a non-existent superhuman quality not an ascetic, not a Sakyan monastic. Therefore it is said, “he is expelled.” %
\item[Excluded from the community: ] Community: joint legal procedures, a joint recitation, the same training—this is called “community”. He does not take part in this—therefore it is called “excluded from the community”. %
\end{description}

\subsection*{Permutations }

\paragraph*{Summary }

A\marginnote{4.1.1} superhuman quality: absorption, release, stillness, attainment, knowledge and vision, development of the path, realization of the fruits, abandoning the defilements, a mind without hindrances, delighting in solitude. 

\paragraph*{Definitions }

\begin{description}%
\item[Absorption: ] the first absorption, the second absorption, the third absorption, the fourth absorption. %
\item[Release: ] emptiness release, signless release, desireless release. %
\item[Stillness: ] emptiness stillness, signless stillness, desireless stillness. %
\item[Attainment: ] emptiness attainment, signless attainment, desireless attainment. %
\item[Knowledge and vision: ] the three true insights. %
\item[Development of the path: ] the four applications of mindfulness, the four right efforts, the four foundations for supernormal power, the five spiritual faculties, the five spiritual powers, the seven factors of awakening, the noble eightfold path. %
\item[Realization of the fruits: ] realization of the fruit of stream-entry, realization of the fruit of once-returning, realization of the fruit of non-returning, realization of perfection. %
\item[Abandoning the defilements: ] the abandoning of sensual desire, the abandoning of ill will, the abandoning of confusion. %
\item[A mind without hindrances: ] a mind without sensual desire, a mind without ill will, a mind without confusion. %
\item[Delighting in solitude: ] because of the first absorption, there is delight in solitude; because of the second absorption, there is delight in solitude; because of the third absorption, there is delight in solitude; because of the fourth absorption, there is delight in solitude. %
\end{description}

\paragraph*{Exposition }

\subparagraph*{First absorption }

If\marginnote{4.2.1} he lies in full awareness, saying, “I attained the first absorption,” he commits an offense entailing expulsion when three conditions are fulfilled: before he has lied, he knows he is going to lie; while lying, he knows he is lying; after he has lied, he knows he has lied. 

If\marginnote{4.2.5} he lies in full awareness, saying, “I attained the first absorption,” he commits an offense entailing expulsion when four conditions are fulfilled: before he has lied, he knows he is going to lie; while lying, he knows he is lying; after he has lied, he knows he has lied; he misrepresents his view of what is true.\footnote{“Of what is true” is not in the Pali, but has been added for clarity. } 

If\marginnote{4.2.10} he lies in full awareness, saying, “I attained the first absorption,” he commits an offense entailing expulsion when five conditions are fulfilled: before he has lied, he knows he is going to lie; while lying, he knows he is lying; after he has lied, he knows he has lied; he misrepresents his view of what is true; he misrepresents his belief of what is true. 

If\marginnote{4.2.15} he lies in full awareness, saying, “I attained the first absorption,” he commits an offense entailing expulsion when six conditions are fulfilled: before he has lied, he knows he is going to lie; while lying, he knows he is lying; after he has lied, he knows he has lied; he misrepresents his view of what is true; he misrepresents his belief of what is true; he misrepresents his acceptance of what is true. 

If\marginnote{4.2.22} he lies in full awareness, saying, “I attained the first absorption,” he commits an offense entailing expulsion when seven conditions are fulfilled: before he has lied, he knows he is going to lie; while lying, he knows he is lying; after he has lied, he knows he has lied; he misrepresents his view of what is true; he misrepresents his belief of what is true; he misrepresents his acceptance of what is true; he misrepresents his sentiment of what is true. 

If\marginnote{4.3.1} he lies in full awareness, saying, “I’m attaining the first absorption,” he commits an offense entailing expulsion when three conditions are fulfilled: before he has lied, he knows he is going to lie; while lying, he knows he is lying; after he has lied, he knows he has lied. 

If\marginnote{4.3.5} he lies in full awareness, saying, “I’m attaining the first absorption,” he commits an offense entailing expulsion when four conditions are fulfilled: before he has lied, he knows he is going to lie; while lying, he knows he is lying; after he has lied, he knows he has lied; he misrepresents his view of what is true. 

If\marginnote{4.3.10} he lies in full awareness, saying, “I’m attaining the first absorption,” he commits an offense entailing expulsion when five conditions are fulfilled: before he has lied, he knows he is going to lie; while lying, he knows he is lying; after he has lied, he knows he has lied; he misrepresents his view of what is true; he misrepresents his belief of what is true. 

If\marginnote{4.3.16} he lies in full awareness, saying, “I’m attaining the first absorption,” he commits an offense entailing expulsion when six conditions are fulfilled: before he has lied, he knows he is going to lie; while lying, he knows he is lying; after he has lied, he knows he has lied; he misrepresents his view of what is true; he misrepresents his belief of what is true; he misrepresents his acceptance of what is true. 

If\marginnote{4.3.23} he lies in full awareness, saying, “I’m attaining the first absorption,” he commits an offense entailing expulsion when seven conditions are fulfilled: before he has lied, he knows he is going to lie; while lying, he knows he is lying; after he has lied, he knows he has lied; he misrepresents his view of what is true; he misrepresents his belief of what is true; he misrepresents his acceptance of what is true; he misrepresents his sentiment of what is true. 

If\marginnote{4.3.31} he lies in full awareness, saying, “I’ve attained the first absorption,” he commits an offense entailing expulsion when three conditions are fulfilled: before he has lied, he knows he is going to lie; while lying, he knows he is lying; after he has lied, he knows he has lied. 

If\marginnote{4.3.35} he lies in full awareness, saying, “I’ve attained the first absorption,” he commits an offense entailing expulsion when four conditions are fulfilled: before he has lied, he knows he is going to lie; while lying, he knows he is lying; after he has lied, he knows he has lied; he misrepresents his view of what is true. 

If\marginnote{4.3.40} he lies in full awareness, saying, “I’ve attained the first absorption,” he commits an offense entailing expulsion when five conditions are fulfilled: before he has lied, he knows he is going to lie; while lying, he knows he is lying; after he has lied, he knows he has lied; he misrepresents his view of what is true; he misrepresents his belief of what is true. 

If\marginnote{4.3.46} he lies in full awareness, saying, “I’ve attained the first absorption,” he commits an offense entailing expulsion when six conditions are fulfilled: before he has lied, he knows he is going to lie; while lying, he knows he is lying; after he has lied, he knows he has lied; he misrepresents his view of what is true; he misrepresents his belief of what is true; he misrepresents his acceptance of what is true. 

If\marginnote{4.3.53} he lies in full awareness, saying, “I’ve attained the first absorption,” he commits an offense entailing expulsion when seven conditions are fulfilled: before he has lied, he knows he is going to lie; while lying, he knows he is lying; after he has lied, he knows he has lied; he misrepresents his view of what is true; he misrepresents his belief of what is true; he misrepresents his acceptance of what is true; he misrepresents his sentiment of what is true. 

If\marginnote{4.3.61} he lies in full awareness, saying, “I obtain the first absorption,” he commits an offense entailing expulsion when three conditions are fulfilled: before he has lied, he knows he is going to lie; while lying, he knows he is lying; after he has lied, he knows he has lied. 

If\marginnote{4.3.65} he lies in full awareness, saying, “I obtain the first absorption,” he commits an offense entailing expulsion when four conditions are fulfilled: before he has lied, he knows he is going to lie; while lying, he knows he is lying; after he has lied, he knows he has lied; he misrepresents his view of what is true. 

If\marginnote{4.3.70} he lies in full awareness, saying, “I obtain the first absorption,” he commits an offense entailing expulsion when five conditions are fulfilled: before he has lied, he knows he is going to lie; while lying, he knows he is lying; after he has lied, he knows he has lied; he misrepresents his view of what is true; he misrepresents his belief of what is true. 

If\marginnote{4.3.76} he lies in full awareness, saying, “I obtain the first absorption,” he commits an offense entailing expulsion when six conditions are fulfilled: before he has lied, he knows he is going to lie; while lying, he knows he is lying; after he has lied, he knows he has lied; he misrepresents his view of what is true; he misrepresents his belief of what is true; he misrepresents his acceptance of what is true. 

If\marginnote{4.3.83} he lies in full awareness, saying, “I obtain the first absorption,” he commits an offense entailing expulsion when seven conditions are fulfilled: before he has lied, he knows he is going to lie; while lying, he knows he is lying; after he has lied, he knows he has lied; he misrepresents his view of what is true; he misrepresents his belief of what is true; he misrepresents his acceptance of what is true; he misrepresents his sentiment of what is true. 

If\marginnote{4.3.91} he lies in full awareness, saying, “I master the first absorption,” he commits an offense entailing expulsion when three conditions are fulfilled: before he has lied, he knows he is going to lie; while lying, he knows he is lying; after he has lied, he knows he has lied. 

If\marginnote{4.3.95} he lies in full awareness, saying, “I master the first absorption,” he commits an offense entailing expulsion when four conditions are fulfilled: before he has lied, he knows he is going to lie; while lying, he knows he is lying; after he has lied, he knows he has lied; he misrepresents his view of what is true. 

If\marginnote{4.3.100} he lies in full awareness, saying, “I master the first absorption,” he commits an offense entailing expulsion when five conditions are fulfilled: before he has lied, he knows he is going to lie; while lying, he knows he is lying; after he has lied, he knows he has lied; he misrepresents his view of what is true; he misrepresents his belief of what is true. 

If\marginnote{4.3.106} he lies in full awareness, saying, “I master the first absorption,” he commits an offense entailing expulsion when six conditions are fulfilled: before he has lied, he knows he is going to lie; while lying, he knows he is lying; after he has lied, he knows he has lied; he misrepresents his view of what is true; he misrepresents his belief of what is true; he misrepresents his acceptance of what is true. 

If\marginnote{4.3.113} he lies in full awareness, saying, “I master the first absorption,” he commits an offense entailing expulsion when seven conditions are fulfilled: before he has lied, he knows he is going to lie; while lying, he knows he is lying; after he has lied, he knows he has lied; he misrepresents his view of what is true; he misrepresents his belief of what is true; he misrepresents his acceptance of what is true; he misrepresents his sentiment of what is true. 

If\marginnote{4.3.121} he lies in full awareness, saying, “I’ve realized the first absorption,” he commits an offense entailing expulsion when three conditions are fulfilled: before he has lied, he knows he is going to lie; while lying, he knows he is lying; after he has lied, he knows he has lied. 

If\marginnote{4.3.125} he lies in full awareness, saying, “I’ve realized the first absorption,” he commits an offense entailing expulsion when four conditions are fulfilled: before he has lied, he knows he is going to lie; while lying, he knows he is lying; after he has lied, he knows he has lied; he misrepresents his view of what is true. 

If\marginnote{4.3.130} he lies in full awareness, saying, “I’ve realized the first absorption,” he commits an offense entailing expulsion when five conditions are fulfilled: before he has lied, he knows he is going to lie; while lying, he knows he is lying; after he has lied, he knows he has lied; he misrepresents his view of what is true; he misrepresents his belief of what is true. 

If\marginnote{4.3.136} he lies in full awareness, saying, “I’ve realized the first absorption,” he commits an offense entailing expulsion when six conditions are fulfilled: before he has lied, he knows he is going to lie; while lying, he knows he is lying; after he has lied, he knows he has lied; he misrepresents his view of what is true; he misrepresents his belief of what is true; he misrepresents his acceptance of what is true. 

If\marginnote{4.3.143} he lies in full awareness, saying, “I’ve realized the first absorption,” he commits an offense entailing expulsion when seven conditions are fulfilled: before he has lied, he knows he is going to lie; while lying, he knows he is lying; after he has lied, he knows he has lied; he misrepresents his view of what is true; he misrepresents his belief of what is true; he misrepresents his acceptance of what is true; he misrepresents his sentiment of what is true. 

\subparagraph*{Other individual attainments }

As\marginnote{4.3.151.1} the first absorption has been expanded in detail, so should all be expanded: 

If\marginnote{4.4.1} he lies in full awareness, saying, “I attained the second absorption,” … “I attained the third absorption,” … “I attained the fourth absorption,” … “I’m attaining … “I’ve attained … “I obtain … “I master … “I’ve realized the fourth absorption,” he commits an offense entailing expulsion when three conditions are fulfilled … when seven conditions are fulfilled: before he has lied, he knows he is going to lie; while lying, he knows he is lying; after he has lied, he knows he has lied; he misrepresents his view of what is true; he misrepresents his belief of what is true; he misrepresents his acceptance of what is true; he misrepresents his sentiment of what is true. 

If\marginnote{4.5.1} he lies in full awareness, saying, “I attained the emptiness release,” … “I attained the signless release,” … “I attained the desireless release,” … “I’m attaining … “I’ve attained … “I obtain … “I master … “I’ve realized the desireless release,” he commits an offense entailing expulsion when three conditions are fulfilled: … 

If\marginnote{4.5.9} he lies in full awareness, saying, “I attained the emptiness stillness,” … “I attained the signless stillness,” … “I attained the desireless stillness,” … “I’m attaining … “I’ve attained … “I obtain … “I master … “I’ve realized the desireless stillness,” he commits an offense entailing expulsion when three conditions are fulfilled. 

If\marginnote{4.5.17} he lies in full awareness, saying, “I attained the emptiness attainment,” … “I attained the signless attainment,” … “I attained the desireless attainment,” … “I’m attaining … “I’ve attained … “I obtain … “I master … “I’ve realized the desireless attainment,” he commits an offense entailing expulsion when three conditions are fulfilled. 

If\marginnote{4.5.25} he lies in full awareness, saying, “I attained the three true insights,” … “I’m attaining … “I’ve attained … “I obtain … “I master … “I’ve realized the three true insights,” he commits an offense entailing expulsion when three conditions are fulfilled. 

If\marginnote{4.5.31} he lies in full awareness, saying, “I attained the four applications of mindfulness,” … “I attained the four right efforts,” … “I attained the four foundations for supernormal power,” … “I’m attaining … “I’ve attained … “I obtain … “I master … “I’ve realized the four foundations for supernormal power,” he commits an offense entailing expulsion when three conditions are fulfilled. 

If\marginnote{4.5.39} he lies in full awareness, saying, “I attained the five spiritual faculties,” … “I attained the five spiritual powers,” … “I’m attaining … “I’ve attained … “I obtain … “I master … “I’ve realized the five spiritual powers,” he commits an offense entailing expulsion when three conditions are fulfilled. 

If\marginnote{4.5.46} he lies in full awareness, saying, “I attained the seven factors of awakening,” … “I’m attaining … “I’ve attained … “I obtain … “I master … “I’ve realized the seven factors of awakening,” he commits an offense entailing expulsion when three conditions are fulfilled. 

If\marginnote{4.5.52} he lies in full awareness, saying, “I attained the noble eightfold path,” … “I’m attaining … “I’ve attained … “I obtain … “I master … “I’ve realized the noble eightfold path,” he commits an offense entailing expulsion when three conditions are fulfilled. 

If\marginnote{4.5.58} he lies in full awareness, saying, “I attained the fruit of stream-entry,” … “I attained the fruit of once-returning,” … “I attained the fruit of non-returning,” … “I attained perfection … “I’m attaining … “I’ve attained … “I obtain …  “I master …  “I’ve realized perfection,” he commits an offense entailing expulsion when three conditions are fulfilled. 

If\marginnote{4.5.65} he lies in full awareness, saying, “I’ve given up sensual desire,  renounced it, let it go, abandoned it, relinquished it, forsaken it, thrown it aside,” he commits an offense entailing expulsion when three conditions are fulfilled. 

If\marginnote{4.5.66} he lies in full awareness, saying, “I’ve given up ill will, renounced it, let it go, abandoned it, relinquished it, forsaken it, thrown it aside,” he commits an offense entailing expulsion when three conditions are fulfilled. 

If\marginnote{4.5.67} he lies in full awareness, saying, “I’ve given up confusion, renounced it, let it go, abandoned it, relinquished it, forsaken it, thrown it aside,” he commits an offense entailing expulsion when three conditions are fulfilled. 

If\marginnote{4.5.68} he lies in full awareness, saying, “My mind is free from the hindrance of sensual desire,” he commits an offense entailing expulsion when three conditions are fulfilled. 

If\marginnote{4.5.69} he lies in full awareness, saying, “My mind is free from the hindrance of ill will,” he commits an offense entailing expulsion when three conditions are fulfilled. 

If\marginnote{4.5.70} he lies in full awareness, saying, “My mind is free from the hindrance of confusion,” he commits an offense entailing expulsion when three conditions are fulfilled … when seven conditions are fulfilled: before he has lied, he knows he is going to lie; while lying, he knows he is lying; after he has lied, he knows he has lied; he misrepresents his view of what is true; he misrepresents his belief of what is true; he misrepresents his acceptance of what is true; he misrepresents his sentiment of what is true. 

\scend{The basic series is finished.\footnote{“Basic series” renders, \textit{suddhika}, which is a technical term used to create sections for long permutation series. See Appendix of Specialized Vocabulary for a detailed explanation. } }

\subparagraph*{Combinations of two attainments }

If\marginnote{4.6.1} he lies in full awareness, saying, “I attained the first absorption and the second absorption,” … “I’m attaining … “I’ve attained … “I obtain … “I master … “I’ve realized the first absorption and the second absorption,” he commits an offense entailing expulsion when three conditions are fulfilled: before he has lied, he knows he is going to lie;  while lying, he knows he is lying;  after he has lied, he knows he has lied. 

If\marginnote{4.6.7} he lies in full awareness, saying, “I attained the first absorption and the third absorption,” … “I’m attaining … “I’ve attained … “I obtain … “I master … “I’ve realized the first absorption and the third absorption,” he commits an offense entailing expulsion when three conditions are fulfilled. 

If\marginnote{4.6.13} he lies in full awareness, saying, “I attained the first absorption and the fourth absorption,” … “I’m attaining … “I’ve attained … “I obtain … “I master … “I’ve realized the first absorption and the fourth absorption,” he commits an offense entailing expulsion when three conditions are fulfilled. 

If\marginnote{4.6.19} he lies in full awareness, saying, “I attained the first absorption and the emptiness release,” … “I attained the first absorption and the signless release,” … “I attained the first absorption and the desireless release,” … “I’m attaining … “I’ve attained … “I obtain … “I master … “I’ve realized the first absorption and the desireless release,” he commits an offense entailing expulsion when three conditions are fulfilled. 

If\marginnote{4.6.27} he lies in full awareness, saying, “I attained the first absorption and the emptiness stillness,” … “I attained the first absorption and the signless stillness,” … “I attained the first absorption and the desireless stillness,” … “I’m attaining … “I’ve attained … “I obtain … “I master … “I’ve realized the first absorption and the desireless stillness,” he commits an offense entailing expulsion when three conditions are fulfilled. 

If\marginnote{4.6.35} he lies in full awareness, saying, “I attained the first absorption and the emptiness attainment,” … “I attained the first absorption and the signless attainment,” … “I attained the first absorption and the desireless attainment,” … “I’m attaining … “I’ve attained … “I obtain … “I master … “I’ve realized the first absorption and the desireless attainment,” he commits an offense entailing expulsion when three conditions are fulfilled. 

If\marginnote{4.6.43} he lies in full awareness, saying, “I attained the first absorption and the three true insights,” … “I’m attaining … “I’ve attained … “I obtain … “I master … “I’ve realized the first absorption and the three true insights,” he commits an offense entailing expulsion when three conditions are fulfilled. 

If\marginnote{4.6.49} he lies in full awareness, saying, “I attained the first absorption and the four applications of mindfulness,” … “I attained the first absorption and the four right efforts,” … “I attained the first absorption and the four foundations for supernormal power,” … “I’m attaining … “I’ve attained … “I obtain … “I master … “I’ve realized the first absorption and the four foundations for supernormal power,” he commits an offense entailing expulsion when three conditions are fulfilled. 

If\marginnote{4.6.57} he lies in full awareness, saying, “I attained the first absorption and the five spiritual faculties,” … “I attained the first absorption and the five spiritual powers,” … “I’m attaining … “I’ve attained … “I obtain … “I master … “I’ve realized the first absorption and the five spiritual powers,” he commits an offense entailing expulsion when three conditions are fulfilled. 

If\marginnote{4.6.64} he lies in full awareness, saying, “I attained the first absorption and the seven factors of awakening,” … “I’m attaining … “I’ve attained … “I obtain … “I master … “I’ve realized the first absorption and the seven factors of awakening,” he commits an offense entailing expulsion when three conditions are fulfilled. 

If\marginnote{4.6.70} he lies in full awareness, saying, “I attained the first absorption and the noble eightfold path,” … “I’m attaining … “I’ve attained … “I obtain … “I master … “I’ve realized the first absorption and the noble eightfold path,” he commits an offense entailing expulsion when three conditions are fulfilled. 

If\marginnote{4.6.76} he lies in full awareness, saying, “I attained the first absorption and the fruit of stream-entry,” … “I attained the first absorption and the fruit of once-returning,” … “I attained the first absorption and the fruit of non-returning,” … “I attained the first absorption and perfection,” … “I’m attaining … “I’ve attained … “I obtain … “I master … “I’ve realized the first absorption and perfection,” he commits an offense entailing expulsion when three conditions are fulfilled. 

If\marginnote{4.6.85} he lies in full awareness, saying, “I attained the first absorption and I’ve given up sensual desire,” … “I’m attaining … “I’ve attained … “I obtain … “I master … “I’ve realized the first absorption and I’ve given up sensual desire … and I’ve given up ill will … and I’ve given up confusion, renounced it, let it go, abandoned it, relinquished it, forsaken it, thrown it aside,” he commits an offense entailing expulsion when three conditions are fulfilled. 

If\marginnote{4.6.93} he lies in full awareness, saying, “I attained the first absorption and my mind is free from the hindrance of sensual desire,” … “I’m attaining … “I’ve attained … “I obtain … “I master … “I’ve realized the first absorption and my mind is free from the hindrance of sensual desire,” … and my mind is free from the hindrance of ill will,” … and my mind is free from the hindrance of confusion,” he commits an offense entailing expulsion when three conditions are fulfilled … when seven conditions are fulfilled: before he has lied, he knows he is going to lie; while lying, he knows he is lying; after he has lied, he knows he has lied; he misrepresents his view of what is true; he misrepresents his belief of what is true; he misrepresents his acceptance of what is true; he misrepresents his sentiment of what is true. 

\scend{The unconnected permutation series is finished.\footnote{“The unconnected permutation series” renders \textit{\textsanskrit{khaṇḍacakka}}, which is another technical term used with the permutation series. See Appendix of Specialized Vocabulary under \textit{\textsanskrit{khaṇḍa}} and \textit{cakka} for further details. } }

If\marginnote{4.7.1} he lies in full awareness, saying, “I attained the second absorption and the third absorption,” … “I’m attaining … “I’ve attained … “I obtain … “I master … “I’ve realized the second absorption and the third absorption,” he commits an offense entailing expulsion when three conditions are fulfilled. 

If\marginnote{4.7.7} he lies in full awareness, saying, “I attained the second absorption and the fourth absorption,” … “I’m attaining … “I’ve attained … “I obtain … “I master … “I’ve realized the second absorption and the fourth absorption,” he commits an offense entailing expulsion when three conditions are fulfilled. 

If\marginnote{4.7.13} he lies in full awareness, saying, “I attained the second absorption and the emptiness release,” … and the signless release,” … and the desireless release,” … and the emptiness stillness,” … and the signless stillness,” … and the desireless stillness,” … and the emptiness attainment,” … and the signless attainment,” … and the desireless attainment,” … and the three true insights,” … and the four applications of mindfulness,” … and the four right efforts,” … and the four foundations for supernormal power,” … and the five spiritual faculties,” … and the five spiritual powers,” … and the seven factors of awakening,” … and the noble eightfold path,” … and the fruit of stream-entry,” … and the fruit of once-returning,” … and the fruit of non-returning,” … and perfection,” … “I’m attaining … “I’ve attained … “I obtain … “I master … “I’ve realized the second absorption and perfection,” he commits an offense entailing expulsion when three conditions are fulfilled. 

If\marginnote{4.7.39} he lies in full awareness, saying, “I attained the second absorption and I’ve given up sensual desire … “I’m attaining … “I’ve attained … “I obtain … “I master … “I’ve realized the second absorption and I’ve given up sensual desire … and I’ve given up ill will … and I’ve given up confusion, renounced it, let it go, abandoned it, relinquished it, forsaken it, thrown it aside,” … and my mind is free from the hindrance of sensual desire,” … and my mind is free from the hindrance of ill will,” … and my mind is free from the hindrance of confusion,” he commits an offense entailing expulsion when three conditions are fulfilled. 

If\marginnote{4.7.50} he lies in full awareness, saying, “I attained the second absorption and the first absorption,” … “I’m attaining … “I’ve attained … “I obtain … “I master … “I’ve realized the second absorption and the first absorption,” he commits an offense entailing expulsion when three conditions are fulfilled … when seven conditions are fulfilled … he misrepresents his sentiment of what is true. 

\scend{The linked permutation series is finished.\footnote{“The linked permutation series” renders \textit{baddhacakka}, which is yet another technical term used with permutation series. See Appendix of Specialized Vocabulary under \textit{baddha} and \textit{cakka}. } }

In\marginnote{4.8.1} this way each section is to be dealt with as in the linked permutation series. 

Here\marginnote{4.8.2} it is in brief: 

If\marginnote{4.8.3} he lies in full awareness, saying, “I attained the third absorption and the fourth absorption,” … the third absorption and perfection,” … “I’m attaining … “I’ve attained … “I obtain … “I master … “I’ve realized the third absorption and perfection,” he commits an offense entailing expulsion when three conditions are fulfilled. 

If\marginnote{4.8.10} he lies in full awareness, saying, “I attained the third absorption and I’ve given up sensual desire … “I’m attaining … “I’ve attained … “I obtain … “I master … “I’ve realized the third absorption and I’ve given up sensual desire … and I’ve given up ill will … and I’ve given up confusion, renounced it, let it go, abandoned it, relinquished it, forsaken it, thrown it aside,” …\footnote{It seems the Pali has left the ellipses points out by mistake. } and my mind is free from the hindrance of sensual desire,” … and my mind is free from the hindrance of ill will,” … and my mind is free from the hindrance of confusion,” he commits an offense entailing expulsion when three conditions are fulfilled. 

If\marginnote{4.8.21} he lies in full awareness, saying, “I attained the third absorption and the first absorption,” … “I attained the third absorption and the second absorption,” … “I’m attaining … “I’ve attained … “I obtain … “I master … “I’ve realized the third absorption and the second absorption,” he commits an offense entailing expulsion when three conditions are fulfilled. …\footnote{It seems ellipses points are missing from the end of the Pali segment. } 

If\marginnote{4.8.28} he lies in full awareness, saying, “My mind is free from the hindrance of confusion and I attained the first absorption,” … the second absorption,” … the third absorption,” … the fourth absorption,” … “I’m attaining … “I’ve attained … “I obtain … “I master … “My mind is free from the hindrance of confusion and I’ve realized the fourth absorption,” he commits an offense entailing expulsion when three conditions are fulfilled. 

If\marginnote{4.8.37} he lies in full awareness, saying, “My mind is free from the hindrance of confusion and I attained the emptiness release,” … and I attained the signless release,” … and I attained the desireless release,” … “I’m attaining … “I’ve attained … “I obtain … “I master … “My mind is free from the hindrance of confusion and I’ve realized the desireless release,” he commits an offense entailing expulsion when three conditions are fulfilled. 

If\marginnote{4.8.45} he lies in full awareness, saying, “My mind is free from the hindrance of confusion and I attained the emptiness stillness,” … and I attained the signless stillness,” … and I attained the desireless stillness,” … “I’m attaining … “I’ve attained … “I obtain … “I master … “My mind is free from the hindrance of confusion and I’ve realized the desireless stillness,” he commits an offense entailing expulsion when three conditions are fulfilled. 

If\marginnote{4.8.53} he lies in full awareness, saying, “My mind is free from the hindrance of confusion and I attained the emptiness attainment,” … and I attained the signless attainment,” … and I attained the desireless attainment,” … “I’m attaining … “I’ve attained … “I obtain … “I master … “My mind is free from the hindrance of confusion and I’ve realized the desireless attainment,” he commits an offense entailing expulsion when three conditions are fulfilled. 

If\marginnote{4.8.61} he lies in full awareness, saying, “My mind is free from the hindrance of confusion and I attained the three true insights,” … “I’m attaining … “I’ve attained … “I obtain … “I master … “My mind is free from the hindrance of confusion and I’ve realized the three true insights,” he commits an offense entailing expulsion when three conditions are fulfilled. 

If\marginnote{4.8.67} he lies in full awareness, saying, “My mind is free from the hindrance of confusion and I attained the four applications of mindfulness,” … and I attained the four right efforts,” … and I attained the four foundations for supernormal power,” … “I’m attaining … “I’ve attained … “I obtain … “I master … “My mind is free from the hindrance of confusion and I’ve realized the four foundations for supernormal power,” he commits an offense entailing expulsion when three conditions are fulfilled. 

If\marginnote{4.8.75} he lies in full awareness, saying, “My mind is free from the hindrance of confusion and I attained the five spiritual faculties,” … and I attained the five spiritual powers,” … “I’m attaining … “I’ve attained … “I obtain … “I master … “My mind is free from the hindrance of confusion and I’ve realized the five spiritual powers,” he commits an offense entailing expulsion when three conditions are fulfilled. 

If\marginnote{4.8.82} he lies in full awareness, saying, “My mind is free from the hindrance of confusion and I attained the seven factors of awakening,” … “I’m attaining … “I’ve attained … “I obtain … “I master … “My mind is free from the hindrance of confusion and I’ve realized the seven factors of awakening,” he commits an offense entailing expulsion when three conditions are fulfilled. 

If\marginnote{4.8.88} he lies in full awareness, saying, “My mind is free from the hindrance of confusion and I attained the noble eightfold path,” … “I’m attaining … “I’ve attained … “I obtain … “I master … “My mind is free from the hindrance of confusion and I’ve realized the noble eightfold path,” he commits an offense entailing expulsion when three conditions are fulfilled. 

If\marginnote{4.8.94} he lies in full awareness, saying, “My mind is free from the hindrance of confusion and I attained the fruit of stream-entry,” … and I attained the fruit of once-returning,” … and I attained the fruit of non-returning,” … and I attained perfection,” … and I’m attaining … and I’ve attained … and I obtain … and I master … “My mind is free from the hindrance of confusion and I’ve realized perfection,” he commits an offense entailing expulsion when three conditions are fulfilled. 

If\marginnote{4.8.103} he lies in full awareness, saying, “My mind is free from the hindrance of confusion and I’ve given up sensual desire … and I’ve given up ill will … and I’ve given up confusion, renounced it, let it go, abandoned it, relinquished it, forsaken it, thrown it aside,” he commits an offense entailing expulsion when three conditions are fulfilled. 

If\marginnote{4.8.106} he lies in full awareness, saying, “My mind is free from the hindrance of confusion and my mind is free from the hindrance of sensual desire,” … and my mind is free from the hindrance of ill will,” he commits an offense entailing expulsion when three conditions are fulfilled … when seven conditions are fulfilled: before he has lied, he knows he is going to lie; while lying, he knows he is lying; after he has lied, he knows he has lied; he misrepresents his view of what is true; he misrepresents his belief of what is true; he misrepresents his acceptance of what is true; he misrepresents his sentiment of what is true. 

\scend{The section based on one item is finished.\footnote{“Based on” renders \textit{\textsanskrit{mūlaka}}. \textit{\textsanskrit{Mūlaka}} is used in repetition series to denote the number of items that form the basis for the series. In the present context, although two attainments are given in each case of the above permutation series, it seems that only the one which remains unchanged within each sub-section is considered the “one item”. } }

\subparagraph*{Combinations of more than two attainments }

The\marginnote{4.9.1} sections based on two items, etc., are to be given in detail in the same way as the section based on one item. 

\subparagraph*{Combination of all attainments }

This\marginnote{4.9.2.1} is the section based on all items: 

If\marginnote{4.9.3} he lies in full awareness, saying, “I attained the first absorption and the second absorption and the third absorption and the fourth absorption and the emptiness release and the signless release and the desireless release and the emptiness stillness and the signless stillness and the desireless stillness and the emptiness attainment and the signless attainment and the desireless attainment and the three true insights and the four applications of mindfulness and the four right efforts and the four foundations for supernormal power and the five spiritual faculties and the five spiritual powers and the seven factors of awakening and the noble eightfold path and the fruit of stream-entry and the fruit of once-returning and the fruit of non-returning and perfection … and I’m attaining … and I’ve attained … etc. … and I’ve given up sensual desire, renounced it, let it go, abandoned it, relinquished it, forsaken it, thrown it aside; and I’ve given up ill will, renounced it, let it go, abandoned it, relinquished it, forsaken it, thrown it aside; and I’ve given up confusion, renounced it, let it go, abandoned it, relinquished it, forsaken it, thrown it aside; and my mind is free from the hindrance of sensual desire and my mind is free from the hindrance of ill will and my mind is free from the hindrance of confusion,” he commits an offense entailing expulsion when three conditions are fulfilled … when seven conditions are fulfilled: before he has lied, he knows he is going to lie; while lying, he knows he is lying; after he has lied, he knows he has lied; he misrepresents his view of what is true; he misrepresents his belief of what is true; he misrepresents his acceptance of what is true; he misrepresents his sentiment of what is true. 

\scend{The section based on all items is finished. }

\scend{The exposition of the section on the basic series is finished. }

\subparagraph*{Meaning to say first absorption, but saying something else }

If\marginnote{5.1.1} he lies in full awareness, meaning to say, “I attained the first absorption,” while actually saying, “I attained the second absorption,” then, if the listener understands, he commits an offense entailing expulsion when three conditions are fulfilled;\footnote{“If the listener understands” renders \textit{\textsanskrit{paṭivijānantassa}}. This is in accordance with Sp 1.219, which has this to say: \textit{Atha pana yassa \textsanskrit{āroceti}, so na \textsanskrit{jānāti} “ki \textsanskrit{ayaṁ} \textsanskrit{bhaṇatī}”ti, \textsanskrit{saṁsayaṁ} \textsanskrit{vā} \textsanskrit{āpajjati}, \textsanskrit{ciraṁ} \textsanskrit{vīmaṁsitvā} \textsanskrit{vā} \textsanskrit{pacchā} \textsanskrit{jānāti}, \textsanskrit{appaṭivijānanto} icceva \textsanskrit{saṅkhyaṁ} gacchati}”, “When he who is informed does not understand, thinking, ‘What does he say?’ or he has doubt, or he understands after reflecting for a long time, then it is considered ‘one who does not understand.’” Grammatically \textit{\textsanskrit{paṭivijānantassa}} could refer to either the speaker or the listener (it can be regarded as a genitive agreeing with \textit{\textsanskrit{bhaṇantassa}}, thus referring to the speaker, or it can be regarded as a dative of the person spoken to, that is, the listener), but logically it seems it must refer to the listener. In accordance with common usage, “understanding” (\textit{\textsanskrit{paṭivijānantassa}}) must refer to understanding the overall meaning of what is said, not to knowing the exact words that have been spoken. Since the speaker knows he is lying, he understands the overall meaning. It follows that the understanding here must refer to the listener. A parallel construction is found at \href{https://suttacentral.net/pli-tv-bu-vb-pj1/en/brahmali\#8.4.10}{Bu Pj 1:8.4.10} where \textit{\textsanskrit{paṭivijānāti}} is used in connection with giving up the monastic training. Here the verb clearly refers to understanding on the part of the listener, that is, one has only succeeded in renouncing the training if the listener understands what one is saying. At \href{https://suttacentral.net/pli-tv-bu-vb-ss3/en/brahmali\#4.1.4}{Bu Ss 3:4.1.4} the same verb, this time in the aorist form \textit{\textsanskrit{paṭivijāni}}, again refers to the listener. } if the listener does not understand, he commits a serious offense when three conditions are fulfilled. 

If\marginnote{5.1.3} he lies in full awareness, meaning to say, “I attained the first absorption,” while actually saying, “I attained the third absorption,” then, if the listener understands, he commits an offense entailing expulsion when three conditions are fulfilled; if the listener does not understand, he commits a serious offense when three conditions are fulfilled. 

If\marginnote{5.1.5} he lies in full awareness, meaning to say, “I attained the first absorption,” while actually saying, “I attained the fourth absorption,” then, if the listener understands, he commits an offense entailing expulsion when three conditions are fulfilled; if the listener does not understand, he commits a serious offense when three conditions are fulfilled. 

If\marginnote{5.1.7} he lies in full awareness, meaning to say, “I attained the first absorption,” while actually saying, “I attained the emptiness release,” … the signless release,” … the desireless release,” … the emptiness stillness,” … the signless stillness,” … the desireless stillness,” … the emptiness attainment,” … the signless attainment,” … the desireless attainment,” … the three true insights,” … the four applications of mindfulness,” … the four right efforts,” … the four foundations for supernormal power,” … the five spiritual faculties,” … the five spiritual powers,” … the seven factors of awakening,” … the noble eightfold path,” … the fruit of stream-entry,” … the fruit of once-returning,” … the fruit of non-returning,” … perfection,” … “I’ve given up sensual desire … “I’ve given up ill will … “I’ve given up confusion, renounced it, let it go, abandoned it, relinquished it, forsaken it, thrown it aside.,” “My mind is free from the hindrance of sensual desire,” … “My mind is free from the hindrance of ill will,” … “My mind is free from the hindrance of confusion,” then, if the listener understands, he commits an offense entailing expulsion when three conditions are fulfilled; if the listener does not understand, he commits a serious offense when three conditions are fulfilled … when seven conditions are fulfilled: before he has lied, he knows he is going to lie; while lying, he knows he is lying; after he has lied, he knows he has lied; he misrepresents his view of what is true; he misrepresents his belief of what is true; he misrepresents his acceptance of what is true; he misrepresents his sentiment of what is true. 

\scend{The unconnected permutation series based on one item with a speech extension is finished.\footnote{The Pali text reads \textit{vatthu}, “basis”, rather than \textit{vattu}, “speech”. However, I have chosen to follow the alternative reading of \textit{vattu}, found in the PTS edition, since this seems more reasonable to me given the context. } }

\subparagraph*{Meaning to say second absorption, but saying something else }

If\marginnote{5.2.1} he lies in full awareness, meaning to say, “I attained the second absorption,” while actually saying, “I attained the third absorption,” then, if the listener understands, he commits an offense entailing expulsion when three conditions are fulfilled; if the listener does not understand, he commits a serious offense when three conditions are fulfilled. 

If\marginnote{5.2.3} he lies in full awareness, meaning to say, “I attained the second absorption,” while actually saying, “I attained the fourth absorption,” … “My mind is free from the hindrance of confusion,” then, if the listener understands, he commits an offense entailing expulsion when three conditions are fulfilled; if the listener does not understand, he commits a serious offense when three conditions are fulfilled. 

If\marginnote{5.2.6} he lies in full awareness, meaning to say, “I attained the second absorption,” while actually saying, “I attained the first absorption,” then, if the listener understands, he commits an offense entailing expulsion when three conditions are fulfilled; if the listener does not understand, he commits a serious offense when three conditions are fulfilled … when seven conditions are fulfilled … he misrepresents his sentiment of what is true. 

\scend{The linked permutation series based on one item with a speech extension is finished. }

\scend{The basis in brief is finished.\footnote{“Basis” renders \textit{\textsanskrit{mūla}}. The \textit{\textsanskrit{mūla}} denotes a basic pattern to be followed in the succeeding permutation series. } }

\subparagraph*{Meaning to say he is free from confusion, but saying something else }

If\marginnote{5.3.1} he lies in full awareness, meaning to say, “My mind is free from the hindrance of confusion,” while actually saying, “I attained the first absorption,” then, if the listener understands, he commits an offense entailing expulsion when three conditions are fulfilled; if the listener does not understand, he commits a serious offense when three conditions are fulfilled. 

If\marginnote{5.3.3} he lies in full awareness, meaning to say, “My mind is free from the hindrance of confusion,” while actually saying, “My mind is free from the hindrance of ill will,” then, if the listener understands, he commits an offense entailing expulsion when three conditions are fulfilled; if the listener does not understand, he commits a serious offense when three conditions are fulfilled … when seven conditions are fulfilled … he misrepresents his sentiment of what is true. 

\scend{The section based on one item with a speech extension is finished. }

\subparagraph*{Meaning to say any particular combination of individual attainments, but saying something else }

The\marginnote{5.4.1} sections based on two items, etc., are to be given in detail in the same way as the section based on one item. 

\subparagraph*{Meaning to say all the attainments but one, and instead saying the remaining one }

This\marginnote{5.4.2.1} is the section based on all items: 

If\marginnote{5.4.3} he lies in full awareness, meaning to say, “I attained the first absorption and the second absorption and the third absorption and the fourth absorption and the emptiness release and the signless release and the desireless release and the emptiness stillness and the signless stillness and the desireless stillness and the emptiness attainment and the signless attainment and the desireless attainment and the three true insights and the four applications of mindfulness and the four right efforts and the four foundations for supernormal power and the five spiritual faculties and the five spiritual powers and the seven factors of awakening and the noble eightfold path and the fruit of stream-entry and the fruit of once-returning and the fruit of non-returning and perfection … and I’ve given up sensual desire … and I’ve given up ill will … and I’ve given up confusion, renounced it, let it go, abandoned it, relinquished it, forsaken it, thrown it aside; and my mind is free from the hindrance of sensual desire and my mind is free from the hindrance of ill will,” while actually saying, “My mind is free from the hindrance of confusion,” then, if the listener understands, he commits an offense entailing expulsion when three conditions are fulfilled; if the listener does not understand, he commits a serious offense when three conditions are fulfilled … when seven conditions are fulfilled. 

If\marginnote{5.4.8} he lies in full awareness, meaning to say, “I attained the second absorption and the third absorption and the fourth absorption and the emptiness release and the signless release and the desireless release and the emptiness stillness and the signless stillness and the desireless stillness and the emptiness attainment and the signless attainment and the desireless attainment and the three true insights and the four applications of mindfulness and the four right efforts and the four foundations for supernormal power and the five spiritual faculties and the five spiritual powers and the seven factors of awakening and the noble eightfold path and the fruit of stream-entry and the fruit of once-returning and the fruit of non-returning and perfection … and I’ve given up sensual desire … and I’ve given up ill will … and I’ve given up confusion, renounced it, let it go, abandoned it, relinquished it, forsaken it, thrown it aside; and my mind is free from the hindrance of sensual desire and my mind is free from the hindrance of ill will and my mind is free from the hindrance of confusion,” while actually saying, “I attained the first absorption,” then, if the listener understands, he commits an offense entailing expulsion when three conditions are fulfilled; if the listener does not understand, there is a serious offense when three conditions are fulfilled. 

If\marginnote{5.4.13} he lies in full awareness, meaning to say, “I attained the third absorption and the fourth absorption … and my mind is free from the hindrance of confusion and I attained the first absorption,” while actually saying, “I attained the second absorption,” then, if the listener understands, he commits an offense entailing expulsion when three conditions are fulfilled; if the listener does not understand, he commits a serious offense when three conditions are fulfilled. 

If\marginnote{5.4.16} he lies in full awareness, meaning to say, “My mind is free from the hindrance of confusion and I attained the first absorption and the second absorption and the third absorption and the fourth absorption … and my mind is free from the hindrance of sensual desire,” while actually saying, “My mind is free from the hindrance of ill will,” then, if the listener understands, he commits an offense entailing expulsion when three conditions are fulfilled; if the listener does not understand, he commits a serious offense when three conditions are fulfilled … when seven conditions are fulfilled: before he has lied, he knows he is going to lie; while lying, he knows he is lying; after he has lied, he knows he has lied; he misrepresents his view of what is true; he misrepresents his belief of what is true; he misrepresents his acceptance of what is true; he misrepresents his sentiment of what is true. 

\scend{The section based on all items with a speech extension is finished. }

\scend{The successive permutation series with a speech extension is finished. }

\scend{The exposition of the section on “meaning to say” is finished. }

\subparagraph*{Gross hinting: in regard to dwellings }

If\marginnote{6.1.1} he lies in full awareness, saying, “The monk who stayed in your dwelling attained the first absorption,” … is attaining … has attained … obtains … masters … has realized the first absorption,” then, if the listener understands, he commits a serious offense when three conditions are fulfilled; if the listener does not understand, he commits an offense of wrong conduct when three conditions are fulfilled: before he has lied, he knows he is going to lie; while lying, he knows he is lying; after he has lied, he knows he has lied. 

If\marginnote{6.1.11} he lies in full awareness, saying, “The monk who stayed in your dwelling attained the first absorption,” … is attaining … has attained … obtains … masters … has realized the first absorption,” then, if the listener understands, he commits a serious offense when four … five … six … seven conditions are fulfilled; if the listener does not understand, he commits an offense of wrong conduct when seven conditions are fulfilled: before he has lied, he knows he is going to lie; while lying, he knows he is lying; after he has lied, he knows he has lied; he misrepresents his view of what is true; he misrepresents his belief of what is true; he misrepresents his acceptance of what is true; he misrepresents his sentiment of what is true. 

If\marginnote{6.1.25} he lies in full awareness, saying, “The monk who stayed in your dwelling attained the second absorption,” … the third absorption,” … the fourth absorption,” … the emptiness release,” … the signless release,” … the desireless release,” … the emptiness stillness,” … the signless stillness,” … the desireless stillness,” … the emptiness attainment,” … the signless attainment,” … the desireless attainment,” … the three true insights,” … the four applications of mindfulness,” … the four right efforts,” … the four foundations for supernormal power,” … the five spiritual faculties,” … the five spiritual powers,” … the seven factors of awakening,” … the noble eightfold path,” … the fruit of stream-entry,” … the fruit of once-returning,” … the fruit of non-returning,” … perfection,” … is attaining … has attained … obtains … masters … has realized perfection,” then, if the listener understands, he commits a serious offense when three conditions are fulfilled; if the listener does not understand, he commits an offense of wrong conduct when three conditions are fulfilled. 

If\marginnote{6.1.55} he lies in full awareness, saying, “The monk who stayed in your dwelling has given up sensual desire … has given up ill will … has given up confusion, renounced it, has let it go, has abandoned it, has relinquished it, has forsaken it, has thrown it aside,” then, if the listener understands, he commits a serious offense when three conditions are fulfilled; if the listener does not understand, he commits an offense of wrong conduct when three conditions are fulfilled. 

If\marginnote{6.1.59} he lies in full awareness, saying, “The monk who stayed in your dwelling has a mind free from the hindrance of sensual desire,” … a mind free from the hindrance of ill will,” … a mind free from the hindrance of confusion,” then, if the listener understands, he commits a serious offense when three conditions are fulfilled; if the listener does not understand, he commits an offense of wrong conduct when three conditions are fulfilled … when seven conditions are fulfilled: before he has lied, he knows he is going to lie; while lying, he knows he is lying; after he has lied, he knows he has lied; he misrepresents his view of what is true; he misrepresents his belief of what is true; he misrepresents his acceptance of what is true; he misrepresents his sentiment of what is true. 

If\marginnote{6.1.70} he lies in full awareness, saying, “The monk who stayed in your dwelling attained the first absorption in solitude,” … the second absorption … the third absorption … the fourth absorption … is attaining … has attained … obtains … masters … has realized the fourth absorption in solitude,” then, if the listener understands, he commits a serious offense when three conditions are fulfilled; if the listener does not understand, he commits an offense of wrong conduct when three conditions are fulfilled … when seven conditions are fulfilled: before he has lied, he knows he is going to lie; while lying, he knows he is lying; after he has lied, he knows he has lied; he misrepresents his view of what is true; he misrepresents his belief of what is true; he misrepresents his acceptance of what is true; he misrepresents his sentiment of what is true. 

\subparagraph*{Gross hinting: in regard to any requisite }

The\marginnote{6.1.87.1} remainder should be given in detail in the same way: 

If\marginnote{6.2.1} he lies in full awareness, saying, “The monk who made use of your robe-cloth … who made use of your almsfood … who made use of your furniture … who made use of your medicinal supplies attained the fourth absorption in solitude,” … is attaining … has attained … obtains … masters … has realized the fourth absorption in solitude,” then, if the listener understands, he commits a serious offense when three conditions are fulfilled; if the listener does not understand, he commits an offense of wrong conduct when three conditions are fulfilled … when seven conditions are fulfilled … he misrepresents his sentiment of what is true. 

If\marginnote{6.2.12} he lies in full awareness, saying, “The monk who has made use of your dwelling … who has made use of your robe-cloth … who has made use of your almsfood … who has made use of your furniture …\footnote{Because \textit{\textsanskrit{vihāra}}, “dwelling”, is mentioned just before, I here render \textit{\textsanskrit{senāsana}} as furniture. } who has made use of your medicinal supplies\footnote{The Pali has a set of ellipses points at the end of this segment. This seems to be a mistake. } attained the fourth absorption in solitude,” … is attaining … has attained … obtains … masters … has realized the fourth absorption in solitude,” then, if the listener understands, he commits a serious offense when three conditions are fulfilled; if the listener does not understand, he commits an offense of wrong conduct when three conditions are fulfilled … when seven conditions are fulfilled … he misrepresents his sentiment of what is true. 

If\marginnote{6.2.25} he lies in full awareness, saying, “The monk you gave a dwelling to … you gave robe-cloth to … you gave almsfood to … you gave furniture to … you gave medicinal supplies to, he attained the fourth absorption in solitude,” … is attaining … has attained … obtains … masters … he has realized the fourth absorption in solitude,” then, if the listener understands, he commits a serious offense when three conditions are fulfilled; if the listener does not understand, he commits an offense of wrong conduct when three conditions are fulfilled … when seven conditions are fulfilled: before he has lied, he knows he is going to lie; while lying, he knows he is lying; after he has lied, he knows he has lied; he misrepresents his view of what is true; he misrepresents his belief of what is true; he misrepresents his acceptance of what is true; he misrepresents his sentiment of what is true. 

\scend{The successive fifteen are finished. }

\scend{The exposition of the section on that connected with requisites is finished. }

\scend{The successive permutation series on superhuman qualities is finished. }

\subsection*{Non-offenses }

There\marginnote{7.1} is no offense: if he overestimates himself; if he does not intend to make a claim; if he is insane; if he is deranged; if he is overwhelmed by pain; if he is the first offender. 

\scuddanaintro{Summary verses of case studies }

\begin{scuddana}%
“About\marginnote{7.9} overestimation, in the wilderness, \\
Almsfood, preceptor, behavior; \\
Fetters, qualities while in solitude, \\
Dwelling, supported. 

Not\marginnote{7.13} difficult, and then energy, fear of death, \\
Remorseful friend, rightly; \\
To be reached by energy, to be reached by exertion, \\
Then two on the enduring of feeling. 

Five\marginnote{7.17} cases of a brahmin, \\
Three on declaring final knowledge; \\
Home, rejected worldly pleasures, \\
And delight, set out. 

Bone,\marginnote{7.21} and lump—both are cattle butchers; \\
A morsel is a poultry butcher, a sheep butcher is flayed; \\
And a pig butcher and sword, a deer hunter and knife, \\
And a torturer and arrow, a horse trainer and needle. 

And\marginnote{7.25} a slanderer is sewn, \\
A corrupt magistrate had testicles as burden; \\
An adulterer submerged in a pit, \\
An eater of feces was a wicked brahmin. 

A\marginnote{7.29} flayed woman was an adulteress, \\
An ugly woman was a fortune-teller; \\
A sweating woman poured coals on a co-wife, \\
A beheaded man was an executioner. 

A\marginnote{7.33} monk, a nun, a trainee nun, \\
A novice monk, then a novice nun—\\
These having gone forth in the training of Kassapa \\
did bad deeds right there. 

The\marginnote{7.37} \textsanskrit{Tapodā}, battle in \textsanskrit{Rājagaha}, \\
And with the plunging in of elephants; \\
The perfected monk Sobhita \\
recalled five hundred eons.” 

%
\end{scuddana}

\subsubsection*{Case studies, part 1 }

At\marginnote{8.1.1} one time a monk declared final knowledge because of overestimation. He became anxious, thinking, “The Buddha has laid down a training rule. Could it be that I’ve committed an offense entailing expulsion?” He told the Buddha. “There’s no offense for overestimates.” 

At\marginnote{8.2.1} one time a monk lived in the wilderness because he wanted people to esteem him. People esteemed him. He became anxious … “There’s no offense entailing expulsion. 

\scrule{But, monks, you should not live in the wilderness because of a wish. If you do, you commit an offense of wrong conduct.” }

At\marginnote{8.2.7} one time a monk was walking for almsfood because he wanted people to esteem him. People esteemed him. He became anxious … “There’s no offense entailing expulsion. 

\scrule{But, monks, you should not walk for almsfood because of a wish. If you do, you commit an offense of wrong conduct.” }

At\marginnote{8.3.1} one time a monk said to another monk, “Those who are pupils of our preceptor are all perfected ones.” He became anxious … “What were you thinking, monk?” 

“I\marginnote{8.3.5} wanted to make a claim, sir.” 

“There’s\marginnote{8.3.6} no offense entailing expulsion, but there’s a serious offense.” 

At\marginnote{8.3.8} one time a monk said to another monk, “Those who are pupils of our preceptor all have great supernormal power.” He became anxious … “What were you thinking, monk?” 

“I\marginnote{8.3.12} wanted to make a claim, sir.” 

“There’s\marginnote{8.3.13} no offense entailing expulsion, but there’s a serious offense.” 

At\marginnote{8.4.1} one time a monk did walking meditation because he wanted people to esteem him … stood because he wanted people to esteem him … sat because he wanted people to esteem him … lay down because he wanted people to esteem him. People esteemed him. He became anxious … “There’s no offense entailing expulsion. 

\scrule{But, monks, you should not lie down because of a wish. If you do, you commit an offense of wrong conduct.” }

At\marginnote{8.5.1} one time a monk claimed a superhuman quality to another monk, saying, “I’ve abandoned the fetters.” He became anxious … “You’ve committed an offense entailing expulsion.” 

At\marginnote{8.6.1} one time a monk claimed a superhuman quality while in solitude. A monk who could read minds rebuked him, saying, “No, you haven’t got it.” He became anxious … “There’s no offense entailing expulsion, but there’s an offense of wrong conduct.” 

At\marginnote{8.6.8} one time a monk claimed a superhuman quality while in solitude. A god rebuked him, saying, “No, sir, you haven’t got it.” He became anxious … “There’s no offense entailing expulsion, but there’s an offense of wrong conduct.” 

At\marginnote{8.7.1} one time a monk said to a lay follower, “The monk living in your dwelling is a perfected one.” He was the one who lived in that dwelling. He became anxious … “What were you thinking, monk?” 

“I\marginnote{8.7.6} wanted to make a claim, sir.” 

“There’s\marginnote{8.7.7} no offense entailing expulsion, but there’s a serious offense.” 

At\marginnote{8.7.9} one time a monk said to a lay follower, “The one you support with with robe-cloth, almsfood, a dwelling, and medicinal supplies, he’s a perfected one.” He was the one who was supported in that way. He became anxious … “What were you thinking, monk?” 

“I\marginnote{8.7.14} wanted to make a claim, sir.” 

“There’s\marginnote{8.7.15} no offense entailing expulsion, but there’s a serious offense.” 

At\marginnote{8.8.1} one time a monk was ill. The monks said to him, “Venerable, do you have any superhuman qualities?” 

“It’s\marginnote{8.8.4} not difficult to declare final knowledge.” 

He\marginnote{8.8.5} became anxious and thought, “Those who are true disciples of the Buddha may say that, but I’m no such disciple. Could it be that I’ve committed an offense entailing expulsion?” He told the Buddha. “What were you thinking, monk?” 

“I\marginnote{8.8.11} didn’t intend to make a claim, sir.” 

“There’s\marginnote{8.8.12} no offense for one who doesn’t intend to make a claim.” 

At\marginnote{8.9.1} one time a monk was ill. The monks said to him, “Venerable, do you have any superhuman qualities?” 

“Superhuman\marginnote{8.9.4} qualities are attained by those who are energetic.” He became anxious … 

“There’s\marginnote{8.9.6} no offense for one who doesn’t intend to make a claim.” 

At\marginnote{8.9.7} one time a monk was ill. The monks said to him, “Don’t be afraid.” 

“I’m\marginnote{8.9.10} not afraid of death.” He became anxious … 

“There’s\marginnote{8.9.12} no offense for one who doesn’t intend to make a claim.” 

At\marginnote{8.9.13} one time a monk was ill. The monks said to him, “Don’t be afraid.” 

“One\marginnote{8.9.16} who’s remorseful might be afraid.” He became anxious … 

“There’s\marginnote{8.9.18} no offense for one who doesn’t intend to make a claim.” 

At\marginnote{8.9.19} one time a monk was ill. The monks said to him, “Venerable, do you have any superhuman qualities?” 

“Superhuman\marginnote{8.9.22} qualities are attained by those who apply themselves rightly.” He became anxious … 

“There’s\marginnote{8.9.24} no offense for one who doesn’t intend to make a claim.” 

At\marginnote{8.9.25} one time a monk was ill. The monks said to him, “Venerable, do you have any superhuman qualities?” 

“Superhuman\marginnote{8.9.28} qualities are attained by those who are energetic.” He became anxious … 

“There’s\marginnote{8.9.30} no offense for one who doesn’t intend to make a claim.” 

At\marginnote{8.9.31} one time a monk was ill. The monks said to him, “Venerable, do you have any superhuman qualities?” 

“Superhuman\marginnote{8.9.34} qualities are attained by those who exert themselves.” He became anxious … 

“There’s\marginnote{8.9.36} no offense for one who doesn’t intend to make a claim.” 

At\marginnote{8.10.1} one time a monk was ill. The monks said to him, “We hope you’re bearing up? We hope you’re comfortable?” 

“It’s\marginnote{8.10.4} not possible for just anyone to endure this.” He became anxious … 

“There’s\marginnote{8.10.6} no offense for one who doesn’t intend to make a claim.” 

At\marginnote{8.10.7} one time a monk was ill. The monks said to him, “We hope you’re bearing up? We hope you’re comfortable?” 

“It’s\marginnote{8.10.10} not possible for an ordinary person to endure this.” He became anxious … 

“What\marginnote{8.10.12} were you thinking, monk?” 

“I\marginnote{8.10.13} intended to make a claim, sir.” 

“There’s\marginnote{8.10.14} no offense entailing expulsion, but there’s a serious offense.” 

At\marginnote{8.11.1} one time a brahmin invited the monks, saying, “Perfected sirs, please come.” 

They\marginnote{8.11.3} became anxious and said, “We’re not perfected ones, and yet this brahmin speaks to us as if we are. What should we do?” They told the Buddha. 

“There’s\marginnote{8.11.8} no offense when something is spoken in faith.” 

At\marginnote{8.11.9} one time a brahmin invited the monks, saying, “Perfected sirs, please be seated.” … “Perfected sirs, please eat.” … “Perfected sirs, please be satisfied.” … “Perfected sirs, please go.” 

They\marginnote{8.11.14} became anxious and said, “We’re not perfected ones, and yet this brahmin speaks to us as if we are. What should we do?” They told the Buddha. 

“There’s\marginnote{8.11.19} no offense when something is spoken in faith.” 

At\marginnote{8.12.1} one time a monk claimed a superhuman quality to another monk, saying, “I’ve abandoned the corruptions.” He became anxious … “You’ve committed an offense entailing expulsion.” 

At\marginnote{8.12.5} one time a monk claimed a superhuman quality to another monk, saying, “I have these qualities.” He became anxious … “You’ve committed an offense entailing expulsion.” 

At\marginnote{8.12.9} one time a monk claimed a superhuman quality to another monk, saying, “I conform to these qualities.” He became anxious … “You’ve committed an offense entailing expulsion.” 

At\marginnote{8.13.1} one time the relatives of a certain monk said to him, “Come, sir, live at home.” 

“One\marginnote{8.13.3} like me is incapable of living at home.” He became anxious … 

“There’s\marginnote{8.13.5} no offense for one who doesn’t intend to make a claim.” 

At\marginnote{8.13.6} one time the relatives of a certain monk said to him, “Come, sir, enjoy worldly pleasures.” 

“The\marginnote{8.13.8} pleasures of the world have been rejected by me.” He became anxious … 

“There’s\marginnote{8.13.10} no offense for one who doesn’t intend to make a claim.” 

At\marginnote{8.13.11} one time the relatives of a certain monk said to him, “Come, sir, enjoy yourself.” 

“I’m\marginnote{8.13.13} enjoying myself with the highest enjoyment.” 

He\marginnote{8.13.14} became anxious, thinking, “Those who are true disciples of the Buddha may say that, but I’m no such disciple. Could it be that I’ve committed an offense entailing expulsion?” He told the Buddha. 

“What\marginnote{8.13.19} were you thinking, monk?” 

“I\marginnote{8.13.20} didn’t intend to make a claim, sir.” 

“There’s\marginnote{8.13.21} no offense for one who doesn’t intend to make a claim.” 

At\marginnote{8.14.1} one time a number of monks entered the rainy-season residence in a certain monastery, making this agreement: “Whoever sets out from this monastery first, we’ll know him as a perfected one.” 

One\marginnote{8.14.3} of them thought, “Let them think I’m a perfected one,” and he set out first from that monastery. He became anxious … 

“You’ve\marginnote{8.14.7} committed an offense entailing expulsion.” 

\subsubsection*{Case studies, part 2 }

At\marginnote{9.1.1} one time when the Buddha was staying at \textsanskrit{Rājagaha} in the Bamboo Grove, the squirrel sanctuary, Venerable \textsanskrit{Lakkhaṇa} and Venerable \textsanskrit{Mahāmoggallāna} were staying on the Vulture Peak. One morning \textsanskrit{Mahāmoggallāna} robed up, took his bowl and robe, went to \textsanskrit{Lakkhaṇa}, and said, “\textsanskrit{Lakkhaṇa}, let’s enter \textsanskrit{Rājagaha} for almsfood.” 

“Yes.”\marginnote{9.1.5} 

As\marginnote{9.1.6} they descended from the Vulture Peak, \textsanskrit{Mahāmoggallāna} smiled at a certain place. \textsanskrit{Lakkhaṇa} asked him why, and \textsanskrit{Māhamoggallāna} replied, 

“This\marginnote{9.1.9} isn’t the right time to ask. Please ask me in the presence of the Buddha.” 

Then,\marginnote{9.2.1} when they had eaten their meal and returned from almsround, \textsanskrit{Lakkhaṇa} and \textsanskrit{Mahāmoggallāna} went to the Buddha, bowed, and sat down. And \textsanskrit{Lakkhaṇa} said to \textsanskrit{Mahāmoggallāna}, 

“Earlier\marginnote{9.2.4} on, as we were descending from the Vulture Peak, you smiled at a certain place. Why was that?” 

“As\marginnote{9.2.6} I was coming down from the Vulture Peak, I saw a skeleton flying through the air. Vultures, crows, and hawks were pursuing it, striking it between the ribs, while it uttered cries of distress. And I thought how amazing and astonishing it is that such a being should exist, such a spirit, such a state of existence.” 

But\marginnote{9.2.13} the monks complained and criticized him, “He’s claiming a superhuman ability!” 

The\marginnote{9.2.15} Buddha then said to them: 

“There\marginnote{9.2.16} are disciples who have vision and knowledge, who can know, see, and witness such things. I too, monks, have seen that being, but I didn’t speak about it. If I had, others wouldn’t have believed me, which would have led to their harm and suffering for a long time. That being was a cattle butcher right here in \textsanskrit{Rājagaha}. As a result of his deeds, he was tormented in hell for many hundreds of thousands of years. And now, because of the remaining result of his actions, he’s experiencing such an existence. \textsanskrit{Moggallāna} spoke truthfully. There’s no offense for \textsanskrit{Moggallāna}.” 

“As\marginnote{9.3.1} I was coming down from the Vulture Peak, I saw a lump of flesh flying through the air. Vultures, crows, and hawks were pursuing it, tearing at it and pulling it to pieces, while it uttered cries of distress. …” … 

“…\marginnote{9.3.4} That being, monks, was a cattle butcher right here in \textsanskrit{Rājagaha}. …” 

“As\marginnote{9.3.5} I was coming down from the Vulture Peak, I saw a morsel of flesh flying through the air. Vultures, crows, and hawks were pursuing it, tearing at it and pulling it to pieces, while it uttered cries of distress. …” … 

“…\marginnote{9.3.8} That being, monks, was a poultry butcher right here in \textsanskrit{Rājagaha}. …” 

“As\marginnote{9.3.9} I was coming down from the Vulture Peak, I saw a flayed man flying through the air. Vultures, crows, and hawks were pursuing him, tearing at him and pulling him to pieces, while he uttered cries of distress. …” … 

“…\marginnote{9.3.12} That being, monks, was a sheep butcher right here in \textsanskrit{Rājagaha}. …” 

“As\marginnote{9.3.13} I was coming down from the Vulture Peak, I saw a man with swords for body hairs flying through the air. Again and again he was stabbed by those swords, while uttering cries of distress. …” … 

“…\marginnote{9.3.16} That being, monks, was a pig butcher right here in \textsanskrit{Rājagaha}. …” 

“As\marginnote{9.3.17} I was coming down from the Vulture Peak, I saw a man with knives for body hairs flying through the air. Again and again he was stabbed by those knives, while uttering cries of distress. …” … 

“…\marginnote{9.3.20} That being, monks, was a deer hunter right here in \textsanskrit{Rājagaha}. …” 

“As\marginnote{9.3.21} I was coming down from the Vulture Peak, I saw a man with arrows for body hairs flying through the air. Again and again he was pierced by those arrows, while uttering cries of distress. …” … 

“…\marginnote{9.3.24} That being, monks, was a torturer right here in \textsanskrit{Rājagaha}. …” 

“As\marginnote{9.3.25} I was coming down from the Vulture Peak, I saw a man with needles for body hairs flying through the air. Again and again he was pierced by those needles, while uttering cries of distress. …” … 

“…\marginnote{9.3.28} That being, monks, was a horse trainer right here in \textsanskrit{Rājagaha}. …” 

“As\marginnote{9.3.29} I was coming down from the Vulture Peak, I saw a man with needles for body hairs flying through the air. Those needles entered his head and came out through his mouth, entered his mouth and came out through his breast, entered his breast and came out through his stomach, entered his stomach and came out through his thighs, entered his thighs and came out through his calves, entered his calves and came out through his feet, as he uttered cries of distress. …” … 

“…\marginnote{9.3.37} That being, monks, was a slanderer right here in \textsanskrit{Rājagaha}. …” 

“As\marginnote{9.3.38} I was coming down from the Vulture Peak, I saw a man flying through the air with testicles like pots. When he moved, he lifted his testicles onto his shoulders; when he sat down, he sat on top of his testicles. Vultures, crows, and hawks were pursuing him, tearing at him and pulling him to pieces, while he uttered cries of distress. …” … 

“…\marginnote{9.3.43} That being, monks, was a corrupt magistrate right here in \textsanskrit{Rājagaha}. …” 

“As\marginnote{9.3.44} I was coming down from the Vulture Peak, I saw a man submerged in a cesspit …” 

“…\marginnote{9.3.45} That being, monks, was an adulterer right here in \textsanskrit{Rājagaha}. …” 

“As\marginnote{9.3.46} I was coming down from the Vulture Peak, I saw a man submerged in a cesspit, eating feces with both hands …” 

“…\marginnote{9.3.47} That being, monks, was a wicked brahmin right here in \textsanskrit{Rājagaha}. At the time of Kassapa, the fully Awakened One, he had invited the Sangha of monks to a meal. He filled a trough with feces, had them informed that the meal was ready, and said, ‘Sirs, eat as much as you like and take the leftovers with you.’ …” 

“As\marginnote{9.3.50} I was coming down from the Vulture Peak, I saw a flayed woman flying through the air. Vultures, crows, and hawks were pursuing her, tearing at her and pulling her to pieces, while she uttered cries of distress. …” … 

“…\marginnote{9.3.53} That woman, monks, was an adulteress right here in \textsanskrit{Rājagaha}. …” 

“As\marginnote{9.3.54} I was coming down from the Vulture Peak, I saw a foul-smelling and ugly woman flying through the air. Vultures, crows, and hawks were pursuing her, tearing at her and pulling her to pieces, while she uttered cries of distress. …” … 

“…\marginnote{9.3.57} That woman, monks, was a fortune-teller right here in \textsanskrit{Rājagaha}. …” 

“As\marginnote{9.3.58} I was coming down from the Vulture Peak, I saw a woman flying through the air, roasting, sweltering, and sooty. Vultures, crows, and hawks were pursuing her, tearing at her and pulling her to pieces, while she uttered cries of distress. …” … 

“…\marginnote{9.3.60} That woman, monks, was the chief queen of the king of \textsanskrit{Kāliṅga}. Overcome by jealousy, she poured a pan of burning coals over a rival. …” 

“As\marginnote{9.3.62} I was coming down from the Vulture Peak, I saw the headless trunk of a body flying through the air, with eyes and mouth on its chest. Vultures, crows, and hawks were pursuing it, tearing at it and pulling it to pieces, while it uttered cries of distress. …” … 

“…\marginnote{9.3.66} That being, monks, was an executioner called \textsanskrit{Hārika} right here in \textsanskrit{Rājagaha}. …” 

“As\marginnote{9.3.67} I was coming down from the Vulture Peak, I saw a monk flying through the air. His outer robe was ablaze and burning, as were his bowl, belt, and body.\footnote{For the rendering of \textit{\textsanskrit{saṅghāṭi}} as “outer robe”, see Appendix of Technical Terms. } He uttered cries of distress. …” … 

“…\marginnote{9.3.70} During the time of the Buddha Kassapa’s dispensation, he was a bad monk.” … 

“As\marginnote{9.3.71} I was coming down from the Vulture Peak, I saw a nun … I saw a trainee nun … I saw a novice monk … I saw a novice nun flying through the air. Her outer robe was ablaze and burning, as were her bowl, belt, and body. She uttered cries of distress. And I thought how amazing and astonishing it is that such a being should exist, such a spirit, such a state of existence.” 

But\marginnote{9.3.81} the monks complained and criticized him, “He’s claiming a superhuman ability!” 

The\marginnote{9.3.83} Buddha then said to them: 

“There\marginnote{9.3.84} are disciples who possess vision and knowledge, who can see, know, and witness such things. I too, monks, have seen that novice nun, but I didn’t speak about it. If I had, others wouldn’t have believed me, which would have led to their harm and suffering for a long time. During the time of Buddha Kassapa’s dispensation, she was a bad novice nun. As a result of her deeds, she was tormented in hell for many hundreds of thousands of years. And now, because of the remaining result of her actions, she’s experiencing such an existence. \textsanskrit{Moggallāna} spoke truthfully. There’s no offense for \textsanskrit{Moggallāna}.” 

Venerable\marginnote{9.4.1} \textsanskrit{Mahāmoggallāna} said to the monks, “This stream, the \textsanskrit{Tapodā}, flows from a lake with clear water—cool, sweet, and pure—with smooth and pleasant banks, with an abundance of fish and turtles, and with blooming lotuses the size of wheels.” 

The\marginnote{9.4.3} monks complained and criticized him, “How can Venerable \textsanskrit{Mahāmoggallāna} talk like this when the \textsanskrit{Tapodā} is actually hot? He’s claiming a superhuman ability!” And they told the Buddha. 

“Monks,\marginnote{9.4.9} the \textsanskrit{Tapodā} does flow from a lake with clear water—cool, sweet, and pure—with smooth and pleasant banks, with an abundance of fish and turtles, and with blooming lotuses the size of wheels. But the \textsanskrit{Tapodā} runs between two great hells. That’s why it’s hot. \textsanskrit{Moggallāna} spoke truthfully. There’s no offense for \textsanskrit{Moggallāna}.” 

At\marginnote{9.5.1} one time King Seniya \textsanskrit{Bimbisāra} of Magadha was defeated in battle by the \textsanskrit{Licchavīs}. The king then rallied his army and beat the \textsanskrit{Licchavīs}. People were delighted and the word spread that the \textsanskrit{Licchavīs} had been defeated by the king. 

But\marginnote{9.5.5} Venerable \textsanskrit{Mahāmoggallāna} said to the monks, “The king was defeated by the \textsanskrit{Licchavīs}.” 

The\marginnote{9.5.7} monks complained and criticized him, “How can Venerable \textsanskrit{Moggallāna} talk like this when people are delighted and the word is spreading that the \textsanskrit{Licchavīs} have been defeated by the king? He’s claiming a superhuman ability!” They told the Buddha. 

“Monks,\marginnote{9.5.14} first the king was defeated by the \textsanskrit{Licchavīs}, but then he rallied his army and beat them. \textsanskrit{Moggallāna} spoke truthfully. There’s no offense for \textsanskrit{Moggallāna}.” 

Venerable\marginnote{9.6.1} \textsanskrit{Mahāmoggallāna} said to the monks, “After attaining an unshakable stillness on the banks of the river \textsanskrit{Sappinikā}, I heard the sound of elephants plunging in and emerging from the water, and trumpeting too.” 

The\marginnote{9.6.3} monks complained and criticized him, “How can Venerable \textsanskrit{Mahāmoggallāna} talk like this? He’s claiming a superhuman ability!” They told the Buddha. 

“Monks,\marginnote{9.6.7} there is such a stillness, but it’s not wholly purified. \textsanskrit{Moggallāna} spoke truthfully. There’s no offense for \textsanskrit{Moggallāna}.” 

Venerable\marginnote{9.7.1} Sobhita said to the monks, “I can recall five hundred eons.” 

The\marginnote{9.7.3} monks complained and criticized him, “How can Venerable Sobhita talk like this? He’s claiming a superhuman ability!” They told the Buddha. 

“Monks,\marginnote{9.7.8} Sobhita has this ability, and that was just one birth. Sobhita spoke truthfully. There’s no offense for Sobhita.” 

\scendsutta{The fourth offense entailing expulsion is finished. }

“Venerables,\marginnote{9.7.13} the four rules on expulsion have been recited. If a monk commits any one of them, he is no longer part of the community of monks. As before, so after: he’s expelled and excluded from the community. In regard to this I ask you, ‘Are you pure in this?’ A second time I ask, ‘Are you pure in this?’ A third time I ask, ‘Are you pure in this?’ You are pure in this and therefore silent. I’ll remember it thus.” 

\scend{The offenses entailing expulsion are finished. }

\scuddanaintro{This is the summary: }

\begin{scuddana}%
“Sexual\marginnote{9.7.26} intercourse, and stealing, \\
Person, super—\\
The four offenses entailing expulsion, \\
Definitive grounds for cutting off.” 

%
\end{scuddana}

\scendkanda{The chapter on offenses entailing expulsion is finished. }

%
\addtocontents{toc}{\let\protect\contentsline\protect\nopagecontentsline}
\chapter*{Suspension }
\addcontentsline{toc}{chapter}{\tocchapterline{Suspension }}
\addtocontents{toc}{\let\protect\contentsline\protect\oldcontentsline}

%
%
\section*{{\suttatitleacronym Bu Ss 1}{\suttatitletranslation 1. The training rule on emission of semen }{\suttatitleroot Sukkavisaṭṭhi}}
\addcontentsline{toc}{section}{\tocacronym{Bu Ss 1} \toctranslation{1. The training rule on emission of semen } \tocroot{Sukkavisaṭṭhi}}
\markboth{1. The training rule on emission of semen }{Sukkavisaṭṭhi}
\extramarks{Bu Ss 1}{Bu Ss 1}

Venerables,\marginnote{0.5} these thirteen rules on suspension come up for recitation. 

\subsection*{Origin story }

\subsubsection*{First sub-story }

At\marginnote{1.1.1} one time the Buddha was staying at \textsanskrit{Sāvatthī} in the Jeta Grove, \textsanskrit{Anāthapiṇḍika}’s Monastery. At that time Venerable Seyyasaka was dissatisfied with the spiritual life. He became thin, haggard, and pale, with veins protruding all over his body. Venerable \textsanskrit{Udāyī} saw him in this condition and said to him, “Seyyasaka, you’re not looking well. You’re not dissatisfied with the spiritual life, are you?” 

“I\marginnote{1.1.7} am.” 

“Well\marginnote{1.1.8} then, eat , sleep, and bathe as much as you like. And whenever you become discontent and lust overwhelms you, just masturbate with your hand.” “But is that allowable?” 

“Yes,\marginnote{1.1.11} I do it too.” 

Then\marginnote{1.1.12} Seyyasaka ate, slept, and bathed as much as he liked, and whenever he became discontent and lust overwhelmed him, he masturbated with his hand. After some time Seyyasaka had a good color, a bright face, clear skin, and sharp senses. The monks who were his friends said to him, “Seyyasaka, you used to be thin, haggard, and pale, with veins protruding all over your body. But now you have a good color, a bright face, clear skin, and sharp senses. Have you been taking medicine?” 

“No.\marginnote{1.1.19} I just eat, sleep, and bathe as much as I like, and whenever I become discontent and lust overwhelms me, I masturbate with my hand.” “Do you eat the food given in faith with the same hand you use to masturbate?” 

“Yes.”\marginnote{1.2.2} 

The\marginnote{1.2.3} monks of few desires complained and criticized him, “How can Venerable Seyyasaka masturbate with his hand?” 

They\marginnote{1.2.5} rebuked Seyyasaka in many ways and then told the Buddha. The Buddha then had the Sangha gathered and questioned Seyyasaka: “Is it true, Seyyasaka, that you do this?” 

“Yes,\marginnote{1.2.8} sir.” 

The\marginnote{1.2.9} Buddha rebuked him, “Foolish man, it’s not suitable, it’s not proper, it’s not worthy of a monastic, it’s not allowable, it’s not to be done. How can you do this? Haven’t I given many teachings for the sake of dispassion, not for the sake of passion; for freedom from bondage, not for the sake of bondage; for the sake of non-grasping, not for the sake of grasping? When I’ve taught like this, how can you choose passion, bondage, and grasping? Haven’t I given many teachings for the fading away of lust, for the clearing away of intoxication, for the removal of thirst, for the uprooting of attachment, for the cutting off of the round of rebirth, for the stopping of craving, for fading away, for ending, for extinguishment? Haven’t I given many teachings for the abandoning of worldly pleasures, for the full understanding of the perceptions of worldly pleasures, for the removal of thirst for worldly pleasures, for the elimination of thoughts of worldly pleasures, for the stilling of the fevers of worldly pleasures? Foolish man, this will affect people’s confidence, and cause some to lose it.” Then, after rebuking Seyyasaka in many ways, the Buddha spoke in dispraise of being difficult to support … “And, monks, this training rule should be recited like this: 

\subsubsection*{Preliminary ruling }

\scrule{‘Intentional emission of semen is an offense entailing suspension.’” }

In\marginnote{1.2.23} this way the Buddha laid down this training rule for the monks. 

\subsubsection*{Second sub-story }

Soon\marginnote{2.1.1} afterwards some monks ate fine foods, fell asleep absentminded and heedless, and emitted semen while dreaming. They became anxious, thinking, “The Buddha has laid down a training rule that intentional emission of semen is an offense entailing suspension. We had an emission while dreaming, which is not without intention. Could it be that we’ve committed an offense entailing suspension?” They told the Buddha. “It’s true, monks, that a dream is not without intention, but it’s negligible. And so, monks, this training rule should be recited like this: 

\subsection*{Final ruling }

\scrule{‘Intentional emission of semen, except while dreaming, is an offense entailing suspension.’” }

\subsection*{Definitions }

\begin{description}%
\item[Intentional: ] knowing, perceiving, having intended, having decided, he transgresses. %
\item[Semen: ] there are ten kinds of semen: blue, yellow, red, white, the color of buttermilk, the color of water, the color of oil, the color of milk, the color of curd, the color of ghee. %
\item[Emission: ] making it move from its base—this is what is meant by “emission”. %
\item[Except while dreaming: ] apart from that which occurs while dreaming. %
\item[An offense entailing suspension: ] only the Sangha gives probation for that offense, sends back to the beginning, gives the trial period, and rehabilitates—not several monks, not an individual. Therefore it is called “an offense entailing suspension”.\footnote{“Gives the trial period” renders \textit{\textsanskrit{mānattaṁ} deti}. For the meaning of \textit{\textsanskrit{mānatta}} see Bhikkhu Ñā\textsanskrit{ṇatusita}, “Analysis of the Bhikkhu Pātimokkha”, p. 123. | The point here, which can only be understood from the Pali, is that the word \textit{\textsanskrit{saṅghādisesa}} (\textit{\textsanskrit{saṅgha}} + \textit{\textsanskrit{ādisesa}}) is derived from the fact that only the \textit{\textsanskrit{saṅgha}} can do the legal procedures required when a monastic commits this offense. | “Several” renders \textit{sambahula}. See Appendix of Technical Terms for a discussion of this rendering. } This is the name and designation of this class of offense. Therefore, too, it is called “an offense entailing suspension”. %
\end{description}

\subsection*{Permutations }

\subsubsection*{Permutations part 1 }

\paragraph*{Summary }

He\marginnote{3.1.1} emits by means of his own body. He emits by means of something external. He emits both by means of his own body and by means of something external. He emits shaking the pelvis in the air. 

He\marginnote{3.1.5} emits because of stiffness due to sensual desire. He emits because of stiffness due to feces. He emits because of stiffness due to urine. He emits because of stiffness due to intestinal gas. He emits because of stiffness due to being stung by caterpillars. 

He\marginnote{3.1.10} emits for the sake of health. He emits for the sake of pleasure. He emits for the sake of medicine. He emits for the sake of a gift. He emits for the sake of merit. He emits for the sake of sacrifice. He emits for the sake of heaven. He emits for the sake of seed. He emits for the sake of investigating. He emits for the sake of fun. 

He\marginnote{3.1.20} emits blue. He emits yellow. He emits red. He emits white. He emits the color of buttermilk. He emits the color of water. He emits the color of oil. He emits the color of milk. He emits the color of curd. He emits the color of ghee. 

\paragraph*{Definitions }

\begin{description}%
\item[By means of his own body: ] by means of his own organic body. %
\item[By means of something external: ] by means of something organic or inorganic, externally. %
\item[Both by means of his own body and by means of something external: ] by means of both. %
\item[Shaking the pelvis in the air: ] for one making an effort in the air, the penis becomes erect. %
\item[Because of stiffness due to sensual desire: ] for one oppressed by sensual desire, the penis becomes erect. %
\item[Because of stiffness due to feces: ] for one oppressed by feces, the penis becomes erect. %
\item[Because of stiffness due to urine: ] for one oppressed by urine, the penis becomes erect. %
\item[Because of stiffness due to intestinal gas: ] for one oppressed by intestinal gas, the penis becomes erect. %
\item[Because of stiffness due to being bitten by caterpillars: ] for one bitten by caterpillars, the penis becomes erect. %
\item[For the sake of health: ] thinking, “I’ll be healthy.” %
\item[For the sake of pleasure: ] thinking, “I’ll produce a pleasant feeling.” %
\item[For the sake of medicine: ] thinking, “There will be medicine.” %
\item[For the sake of a gift: ] thinking, “I’ll give a gift.” %
\item[For the sake of merit: ] thinking, “There will be merit.” %
\item[For the sake of sacrifice: ] thinking, “I’ll offer a sacrifice.” %
\item[For the sake of heaven: ] thinking, “I’ll go to heaven.” %
\item[For the sake of seed: ] thinking, “There will be seed.” %
\item[For the sake of investigating: ] thinking, “Will it be blue?”, “Will it be yellow?”, “Will it be red?”, “Will it be white?”, “Will it be the color of buttermilk?”, “Will it be the color of water?”, “Will it be the color of oil?”, “Will it be the color of milk?”, “Will it be the color of curd?”, “Will it be the color of ghee?” %
\item[For the sake of fun: ] desiring to play. %
\end{description}

\paragraph*{Exposition part 1 }

If,\marginnote{3.3.1} by means of his own body, he intends and makes an effort, and semen is emitted, he commits an offense entailing suspension. 

If,\marginnote{3.3.2} by means of something external, he intends and makes an effort, and semen is emitted, he commits an offense entailing suspension. 

If,\marginnote{3.3.3} both by means of his own body and by means of something external, he intends and makes an effort, and semen is emitted, he commits an offense entailing suspension. 

If,\marginnote{3.3.4} shaking the pelvis in the air, he intends and makes an effort, and semen is emitted, he commits an offense entailing suspension. 

If,\marginnote{3.3.5} when there is stiffness due to sensual desire, he intends and makes an effort, and semen is emitted, he commits an offense entailing suspension. 

If,\marginnote{3.3.6} when there is stiffness due to feces, he intends and makes an effort, and semen is emitted, he commits an offense entailing suspension. 

If,\marginnote{3.3.7} when there is stiffness due to urine, he intends and makes an effort, and semen is emitted, he commits an offense entailing suspension. 

If,\marginnote{3.3.8} when there is stiffness due to intestinal gas, he intends and makes an effort, and semen is emitted, he commits an offense entailing suspension. 

If,\marginnote{3.3.9} when there is stiffness due to being bitten by caterpillars, he intends and makes an effort, and semen is emitted, he commits an offense entailing suspension. 

\paragraph*{Exposition part 2 }

\subparagraph*{A single purpose }

If,\marginnote{3.3.10.1} for the sake of health, he intends and makes an effort, and semen is emitted, he commits an offense entailing suspension. 

If,\marginnote{3.3.11} for the sake of pleasure … If, for the sake of medicine … If, for the sake of a gift … If, for the sake of merit … If, for the sake of sacrifice … If, for the sake of heaven … If, for the sake of seed … If, for the sake of investigating … If, for the sake of fun, he intends and makes an effort, and semen is emitted, he commits an offense entailing suspension. 

\subparagraph*{One kind of semen }

If\marginnote{3.3.20.1} he intends blue, makes an effort, and semen is emitted, he commits an offense entailing suspension. 

If\marginnote{3.3.21} he intends yellow … If he intends red … If he intends white … If he intends the color of buttermilk … If he intends the color of water … If he intends the color of oil … If he intends the color of milk … If he intends the color of curd … If he intends the color of ghee, makes an effort, and semen is emitted, he commits an offense entailing suspension. 

\scend{The basic series is finished.\footnote{“Basic series” renders, \textit{suddhika}, which is a technical term used to create sections for long permutation series. See \textit{suddhika} in Appendix of Specialized Vocabulary for details. } }

\subparagraph*{Two purposes }

If,\marginnote{3.4.1.1} for the sake of health and for the sake of pleasure, he intends and makes an effort, and semen is emitted, he commits an offense entailing suspension. 

If,\marginnote{3.4.2} for the sake of health and for the sake of medicine … If, for the sake of health and for the sake of a gift … If, for the sake of health and for the sake of merit … If, for the sake of health and for the sake of sacrifice … If, for the sake of health and for the sake of heaven … If, for the sake of health and for the sake of seed … If, for the sake of health and for the sake of investigating … If, for the sake of health and for the sake of fun, he intends and makes an effort, and semen is emitted, he commits an offense entailing suspension. 

\scend{The unconnected permutation series based on one item is finished.\footnote{For these section-ending terms, see Appendix of Specialized Vocabulary under \textit{\textsanskrit{mūlaka}}, \textit{\textsanskrit{khaṇḍa}}, and \textit{cakka}. } }

If,\marginnote{3.5.1} for the sake of pleasure and for the sake of medicine, he intends and makes an effort, and semen is emitted, he commits an offense entailing suspension. 

If,\marginnote{3.5.2} for the sake of pleasure and for the sake of a gift … If, for the sake of pleasure and for the sake of merit … If, for the sake of pleasure and for the sake of sacrifice … If, for the sake of pleasure and for the sake of heaven … If, for the sake of pleasure and for the sake of seed … If, for the sake of pleasure and for the sake of investigating … If, for the sake of pleasure and for the sake of fun, he intends and makes an effort, and semen is emitted, he commits an offense entailing suspension. 

If,\marginnote{3.5.9} for the sake of pleasure and for the sake of health, he intends and makes an effort, and semen is emitted, he commits an offense entailing suspension. 

If,\marginnote{3.5.10} for the sake of medicine and for the sake of a gift … If, for the sake of medicine and for the sake of merit … If, for the sake of medicine and for the sake of sacrifice … If, for the sake of medicine and for the sake of heaven … If, for the sake of medicine and for the sake of seed … If, for the sake of medicine and for the sake of investigating … If, for the sake of medicine and for the sake of fun, he intends and makes an effort, and semen is emitted, he commits an offense entailing suspension. 

If,\marginnote{3.5.17} for the sake of medicine and for the sake of health … If, for the sake of medicine and for the sake of pleasure, he intends and makes an effort, and semen is emitted, he commits an offense entailing suspension. 

If,\marginnote{3.5.19} for the sake of a gift and for the sake of merit … If, for the sake of a gift and for the sake of sacrifice … If, for the sake of a gift and for the sake of heaven … If, for the sake of a gift and for the sake of seed … If, for the sake of a gift and for the sake of investigating … If, for the sake of a gift and for the sake of fun, he intends and makes an effort, and semen is emitted, he commits an offense entailing suspension. 

If,\marginnote{3.5.25} for the sake of a gift and for the sake of health … If, for the sake of a gift and for the sake of pleasure … If, for the sake of a gift and for the sake of medicine, he intends and makes an effort, and semen is emitted, he commits an offense entailing suspension. 

If,\marginnote{3.5.28} for the sake of merit and for the sake of sacrifice … If, for the sake of merit and for the sake of heaven … If, for the sake of merit and for the sake of seed … If, for the sake of merit and for the sake of investigating … If, for the sake of merit and for the sake of fun, he intends and makes an effort, and semen is emitted, he commits an offense entailing suspension. 

If,\marginnote{3.5.33} for the sake of merit and for the sake of health … If, for the sake of merit and for the sake of pleasure … If, for the sake of merit and for the sake of medicine … If, for the sake of merit and for the sake of a gift, he intends and makes an effort, and semen is emitted, he commits an offense entailing suspension. 

If,\marginnote{3.5.37} for the sake of sacrifice and for the sake of heaven … If, for the sake of sacrifice and for the sake of seed … If, for the sake of sacrifice and for the sake of investigating … If, for the sake of sacrifice and for the sake of fun, he intends and makes an effort, and semen is emitted, he commits an offense entailing suspension. 

If,\marginnote{3.5.41} for the sake of sacrifice and for the sake of health … If, for the sake of sacrifice and for the sake of pleasure … If, for the sake of sacrifice and for the sake of medicine … If, for the sake of sacrifice and for the sake of a gift … If, for the sake of sacrifice and for the sake of merit, he intends and makes an effort, and semen is emitted, he commits an offense entailing suspension. 

If,\marginnote{3.5.46} for the sake of heaven and for the sake of seed … If, for the sake of heaven and for the sake of investigating … If, for the sake of heaven and for the sake of fun, he intends and makes an effort, and semen is emitted, he commits an offense entailing suspension. 

If,\marginnote{3.5.49} for the sake of heaven and for the sake of health … If, for the sake of heaven and for the sake of pleasure … If, for the sake of heaven and for the sake of medicine … If, for the sake of heaven and for the sake of a gift … If, for the sake of heaven and for the sake of merit … If, for the sake of heaven and for the sake of sacrifice, he intends and makes an effort, and semen is emitted, he commits an offense entailing suspension. 

If,\marginnote{3.5.55} for the sake of seed and for the sake of investigating … If, for the sake of seed and for the sake of fun, he intends and makes an effort, and semen is emitted, he commits an offense entailing suspension. 

If,\marginnote{3.5.57} for the sake of seed and for the sake of health … If, for the sake of seed and for the sake of pleasure … If, for the sake of seed and for the sake of medicine … If, for the sake of seed and for the sake of a gift … If, for the sake of seed and for the sake of merit … If, for the sake of seed and for the sake of sacrifice … If, for the sake of seed and for the sake of heaven, he intends and makes an effort, and semen is emitted, he commits an offense entailing suspension. 

If,\marginnote{3.5.64} for the sake of investigating and for the sake of fun, he intends and makes an effort, and semen is emitted, he commits an offense entailing suspension. 

If,\marginnote{3.5.65} for the sake of investigating and for the sake of health … If, for the sake of investigating and for the sake of pleasure … If, for the sake of investigating and for the sake of medicine … If, for the sake of investigating and for the sake of a gift … If, for the sake of investigating and for the sake of merit … If, for the sake of investigating and for the sake of sacrifice … If, for the sake of investigating and for the sake of heaven … If, for the sake of investigating and for the sake of seed, he intends and makes an effort, and semen is emitted, he commits an offense entailing suspension. 

If,\marginnote{3.5.73} for the sake of fun and for the sake of health … If, for the sake of fun and for the sake of pleasure … If, for the sake of fun and for the sake of medicine … If, for the sake of fun and for the sake of a gift … If, for the sake of fun and for the sake of merit … If, for the sake of fun and for the sake of sacrifice … If, for the sake of fun and for the sake of heaven … If, for the sake of fun and for the sake of seed … If, for the sake of fun and for the sake of investigating, he intends and makes an effort, and semen is emitted, he commits an offense entailing suspension. 

\scend{The linked permutation series based on one item is finished.\footnote{For these section-ending terms, see Appendix of Specialized Vocabulary under \textit{\textsanskrit{mūlaka}}, \textit{baddha}, and \textit{cakka}. } }

\subparagraph*{Three purposes }

If,\marginnote{3.5.83.1} for the sake of health and for the sake of pleasure and for the sake of medicine, he intends and makes an effort, and semen is emitted, he commits an offense entailing suspension. … If, for the sake of health and for the sake of pleasure and for the sake of fun, he intends and makes an effort, and semen is emitted, he commits an offense entailing suspension. 

\scend{The unconnected permutation series based on two items is finished. }

If,\marginnote{3.5.86} for the sake of pleasure and for the sake of medicine and for the sake of a gift, he intends and makes an effort, and semen is emitted, he commits an offense entailing suspension. … If, for the sake of pleasure and for the sake of medicine and for the sake of fun … If, for the sake of pleasure and for the sake of medicine and for the sake of health, he intends and makes an effort, and semen is emitted, he commits an offense entailing suspension. 

\scend{The linked permutation series based on two items in brief is finished. }

If,\marginnote{3.5.90} for the sake of investigating and for the sake of fun and for the sake of health, he intends and makes an effort, and semen is emitted, he commits an offense entailing suspension. … If, for the sake of investigating and for the sake of fun and for the sake of seed, he intends and makes an effort, and semen is emitted, he commits an offense entailing suspension. 

\scend{The section based on two items is finished. }

\subparagraph*{Four to nine purposes }

Three\marginnote{3.5.93.1} items, four items, five items, six items, seven items, eight items, and nine items are to be expanded in the same way. 

\subparagraph*{Ten purposes }

This\marginnote{3.5.94.1} is the section based on all items: 

If,\marginnote{3.5.95} for the sake of health and for the sake of pleasure and for the sake of medicine and for the sake of a gift and for the sake of merit and for the sake of sacrifice and for the sake of heaven and for the sake of seed and for the sake of investigating and for the sake of fun, he intends and makes an effort, and semen is emitted, he commits an offense entailing suspension. 

\scend{The section based on all items is finished. }

\subparagraph*{Two kinds of semen }

If\marginnote{3.6.1} he intends blue and yellow, makes an effort, and semen is emitted, he commits an offense entailing suspension. 

If\marginnote{3.6.2} he intends blue and red … If he intends blue and white … If he intends blue and the color of buttermilk … If he intends blue and the color of water … If he intends blue and the color of oil … If he intends blue and the color of milk … If he intends blue and the color of curd … If he intends blue and the color of ghee, makes an effort, and semen is emitted, he commits an offense entailing suspension. 

\scend{The unconnected permutation series based on one item is finished. }

If\marginnote{3.6.11} he intends yellow and red, makes an effort, and semen is emitted, he commits an offense entailing suspension. 

If\marginnote{3.6.12} he intends yellow and white … If he intends yellow and the color of buttermilk … If he intends yellow and the color of water … If he intends yellow and the color of oil … If he intends yellow and the color of milk … If he intends yellow and the color of curd … If he intends yellow and the color of ghee, makes an effort, and semen is emitted, he commits an offense entailing suspension. 

If\marginnote{3.6.19} he intends yellow and blue, makes an effort, and semen is emitted, he commits an offense entailing suspension. 

\scend{The linked permutation series based on one item is finished. }

If\marginnote{3.6.21} he intends red and white, makes an effort, and semen is emitted, he commits an offense entailing suspension. 

If\marginnote{3.6.22} he intends red and the color of buttermilk … If he intends red and the color of water … If he intends red and the color of oil … If he intends red and the color of milk … If he intends red and the color of curd … If he intends red and the color of ghee, makes an effort, and semen is emitted, he commits an offense entailing suspension. 

If\marginnote{3.6.28} he intends red and blue … If he intends red and yellow, makes an effort, and semen is emitted, he commits an offense entailing suspension. 

If\marginnote{3.6.30} he intends white and the color of buttermilk … If he intends white and the color of water … If he intends white and the color of oil … If he intends white and the color of milk … If he intends white and the color of curd … If he intends white and the color of ghee, makes an effort, and semen is emitted, he commits an offense entailing suspension. 

If\marginnote{3.6.36} he intends white and blue … If he intends white and yellow … If he intends white and red, makes an effort, and semen is emitted, he commits an offense entailing suspension. 

If\marginnote{3.6.39} he intends the color of buttermilk and the color of water … If he intends the color of buttermilk and the color of oil … If he intends the color of buttermilk and the color of milk … If he intends the color of buttermilk and the color of curd … If he intends the color of buttermilk and the color of ghee, makes an effort, and semen is emitted, he commits an offense entailing suspension. 

If\marginnote{3.6.44} he intends the color of buttermilk and blue … If he intends the color of buttermilk and yellow … If he intends the color of buttermilk and red … If he intends the color of buttermilk and white, makes an effort, and semen is emitted, he commits an offense entailing suspension. 

If\marginnote{3.6.48} he intends the color of water and the color of oil … If he intends the color of water and the color of milk … If he intends the color of water and the color of curd … If he intends the color of water and the color of ghee, makes an effort, and semen is emitted, he commits an offense entailing suspension. 

If\marginnote{3.6.52} he intends the color of water and blue … If he intends the color of water and yellow … If he intends the color of water and red … If he intends the color of water and white … If he intends the color of water and the color of buttermilk, makes an effort, and semen is emitted, he commits an offense entailing suspension. 

If\marginnote{3.6.57} he intends the color of oil and the color of milk … If he intends the color of oil and the color of curd … If he intends the color of oil and the color of ghee, makes an effort, and semen is emitted, he commits an offense entailing suspension. 

If\marginnote{3.6.60} he intends the color of oil and blue … If he intends the color of oil and yellow … If he intends the color of oil and red … If he intends the color of oil and white … If he intends the color of oil and the color of buttermilk … If he intends the color of oil and the color of water, makes an effort, and semen is emitted, he commits an offense entailing suspension. 

If\marginnote{3.6.66} he intends the color of milk and the color of curd … If he intends the color of milk and the color of ghee, makes an effort, and semen is emitted, he commits an offense entailing suspension. 

If\marginnote{3.6.68} he intends the color of milk and blue … If he intends the color of milk and yellow … If he intends the color of milk and red … If he intends the color of milk and white … If he intends the color of milk and the color of buttermilk … If he intends the color of milk and the color of water … If he intends the color of milk and the color of oil, makes an effort, and semen is emitted, he commits an offense entailing suspension. 

If\marginnote{3.6.75} he intends the color of curd and the color of ghee, makes an effort, and semen is emitted, he commits an offense entailing suspension. 

If\marginnote{3.6.76} he intends the color of curd and blue … If he intends the color of curd and yellow … If he intends the color of curd and red … If he intends the color of curd and white … If he intends the color of curd and the color of buttermilk … If he intends the color of curd and the color of water … If he intends the color of curd and the color of oil … If he intends the color of curd and the color of milk, makes an effort, and semen is emitted, he commits an offense entailing suspension. 

If\marginnote{3.6.84} he intends the color of ghee and blue, makes an effort, and semen is emitted, he commits an offense entailing suspension. 

If\marginnote{3.6.85} he intends the color of ghee and yellow … If he intends the color of ghee and red … If he intends the color of ghee and white … If he intends the color of ghee and the color of buttermilk … If he intends the color of ghee and the color of water … If he intends the color of ghee and the color of oil … If he intends the color of ghee and the color of milk … If he intends the color of ghee and the color of curd, makes an effort, and semen is emitted, he commits an offense entailing suspension. 

\scend{The linked permutation series based on one item is finished. }

\subparagraph*{Three kinds of semen }

If\marginnote{3.6.94.1} he intends blue and yellow and red, makes an effort, and semen is emitted, he commits an offense entailing suspension. … If he intends blue and yellow and the color of ghee, makes an effort, and semen is emitted, he commits an offense entailing suspension. 

\scend{The unconnected permutation series based on two items is finished. }

If\marginnote{3.6.97} he intends yellow and red and white, makes an effort, and semen is emitted, he commits an offense entailing suspension. … If he intends yellow and red and the color of ghee … If he intends yellow and red and blue, makes an effort, and semen is emitted, he commits an offense entailing suspension. 

\scend{The linked permutation series based on two items in brief is finished. }

If\marginnote{3.6.101} he intends the color of curd and the color of ghee and blue, makes an effort, and semen is emitted, he commits an offense entailing suspension. … If he intends the color of curd and the color of ghee and the color of milk, makes an effort, and semen is emitted, he commits an offense entailing suspension. 

\scend{The section based on two items is finished. }

\subparagraph*{Four to nine kinds of semen }

The\marginnote{3.6.104.1} sections based on three items, four items, five items, six items, seven items, eight items, and nine items are to be expanded in the same way. 

\subparagraph*{Ten kinds of semen }

This\marginnote{3.6.105.1} is the section based on all items: 

If\marginnote{3.6.106} he intends blue and yellow and red and white and the color of buttermilk and the color of water and the color of oil and the color of milk and the color of curd and the color of ghee, makes an effort, and semen is emitted, he commits an offense entailing suspension. 

\scend{The section based on all items is finished. }

\subparagraph*{Purposes combined with kinds of semen }

If\marginnote{3.7.1} he intends for the sake of health and blue, makes an effort, and semen is emitted, he commits an offense entailing suspension. 

If\marginnote{3.7.2} he intends for the sake of health and for the sake of pleasure and blue and yellow, makes an effort, and semen is emitted, he commits an offense entailing suspension. 

If\marginnote{3.7.3} he intends for the sake of health and for the sake of pleasure and for the sake of medicine and blue and yellow and red, makes an effort, and semen is emitted, he commits an offense entailing suspension. 

(In\marginnote{3.7.4} this way both aspects are to be expanded.) 

If\marginnote{3.7.5} he intends for the sake of health and for the sake of pleasure and for the sake of medicine and for the sake of a gift and for the sake of merit and for the sake of sacrifice and for the sake of heaven and for the sake of seed and for the sake of investigating and for the sake of fun and blue and yellow and red and white and the color of buttermilk and the color of water and the color of oil and the color of milk and the color of curd and the color of ghee, makes an effort, and semen is emitted, he commits an offense entailing suspension. 

\scend{The mixed permutation series is finished. }

\subparagraph*{Intending one kind of semen, emitting another kind }

If\marginnote{3.8.1} he intends, “I’ll emit blue,” makes an effort, and yellow is emitted, he commits an offense entailing suspension. 

If\marginnote{3.8.2} he intends, “I’ll emit blue,” makes an effort, and red is emitted … white … the color of buttermilk … the color of water … the color of oil … the color of milk … the color of curd … the color of ghee is emitted, he commits an offense entailing suspension. 

\scend{The unconnected permutation series is finished. }

If\marginnote{3.8.11} he intends, “I’ll emit yellow,” makes an effort, and red is emitted, he commits an offense entailing suspension. 

If\marginnote{3.8.12} he intends, “I’ll emit yellow,” makes an effort, and white is emitted … the color of buttermilk … the color of water … the color of oil … the color of milk … the color of curd … the color of ghee … blue is emitted, he commits an offense entailing suspension. 

\scend{The basis of the linked permutation series in brief is finished. }

…\marginnote{3.8.21} If he intends, “I’ll emit the color of ghee,” makes an effort, and blue is emitted, he commits an offense entailing suspension. 

If\marginnote{3.8.22} he intends, “I’ll emit the color of ghee,” makes an effort, and yellow is emitted … red … white … the color of buttermilk … the color of water … the color of oil … the color of milk … the color of curd is emitted, he commits an offense entailing suspension. 

\scend{The core permutation series is finished.\footnote{For these section-ending terms, see Appendix of Specialized Vocabulary under \textit{kucchi} and \textit{cakka}. } }

If\marginnote{3.9.1} he intends, “I’ll emit yellow,” makes an effort, and blue is emitted, he commits an offense entailing suspension. 

If\marginnote{3.9.2} he intends, “I’ll emit red … white … the color of buttermilk … the color of water … the color of oil … the color of milk … the color of curd … the color of ghee,” makes an effort, and blue is emitted, he commits an offense entailing suspension. 

\scend{The first round of the additional permutation series is finished.\footnote{See Appendix of Specialized Vocabulary under \textit{\textsanskrit{piṭṭhi}}, \textit{cakka}, and \textit{gamana}. } }

If\marginnote{3.9.11} he intends, “I’ll emit red,” makes an effort, and yellow is emitted, he commits an offense entailing suspension. 

If\marginnote{3.9.12} he intends, “I’ll emit white … the color of buttermilk … the color of water … the color of oil … the color of milk … the color of curd … the color of ghee … blue,” makes an effort, and yellow is emitted, he commits an offense entailing suspension. 

\scend{The second round of the additional permutation series is finished. }

If\marginnote{3.9.21} he intends, “I’ll emit white,” makes an effort, and red is emitted, he commits an offense entailing suspension. 

If\marginnote{3.9.22} he intends, “I’ll emit the color of buttermilk … the color of water … the color of oil … the color of milk … the color of curd … the color of ghee … blue … yellow,” makes an effort, and red is emitted, he commits an offense entailing suspension. 

\scend{The third round of the additional permutation series is finished. }

If\marginnote{3.9.31} he intends, “I’ll emit the color of buttermilk,” makes an effort, and white is emitted, he commits an offense entailing suspension. 

If\marginnote{3.9.32} he intends, “I’ll emit the color of water … the color of oil … the color of milk … the color of curd … the color of ghee … blue … yellow … red,” makes an effort, and white is emitted, he commits an offense entailing suspension. 

\scend{The fourth round of the additional permutation series is finished. }

If\marginnote{3.9.41} he intends, “I’ll emit the color of water,” makes an effort, and the color of buttermilk is emitted, he commits an offense entailing suspension. 

If\marginnote{3.9.42} he intends, “I’ll emit the color of oil … the color of milk … the color of curd … the color of ghee … blue … yellow … red … white,” makes an effort, and the color of buttermilk is emitted, he commits an offense entailing suspension. 

\scend{The fifth round of the additional permutation series is finished. }

If\marginnote{3.9.51} he intends, “I’ll emit the color of oil,” makes an effort, and the color of water is emitted, he commits an offense entailing suspension. 

If\marginnote{3.9.52} he intends, “I’ll emit the color of milk … the color of curd … the color of ghee … blue … yellow … red … white … the color of buttermilk,” makes an effort, and the color of water is emitted, he commits an offense entailing suspension. 

\scend{The sixth round of the additional permutation series is finished. }

If\marginnote{3.9.61} he intends, “I’ll emit the color of milk,” makes an effort, and the color of oil is emitted, he commits an offense entailing suspension. 

If\marginnote{3.9.62} he intends, “I’ll emit the color of curd … the color of ghee … blue … yellow … red … white … the color of buttermilk … the color of water,” makes an effort, and the color of oil is emitted, he commits an offense entailing suspension. 

\scend{The seventh round of the additional permutation series is finished. }

If\marginnote{3.9.71} he intends, “I’ll emit the color of curd,” makes an effort, and the color of milk is emitted, he commits an offense entailing suspension. 

If\marginnote{3.9.72} he intends, “I’ll emit the color of ghee … blue … yellow … red … white … the color of buttermilk … the color of water … the color of oil,” makes an effort, and the color of milk is emitted, he commits an offense entailing suspension. 

\scend{The eighth round of the additional permutation series is finished. }

If\marginnote{3.9.81} he intends, “I’ll emit the color of ghee,” makes an effort, and the color of curd is emitted, he commits an offense entailing suspension. 

If\marginnote{3.9.82} he intends, “I’ll emit blue … yellow … red … white … the color of buttermilk … the color of water … the color of oil … the color of milk,” makes an effort, and the color of curd is emitted, he commits an offense entailing suspension. 

\scend{The ninth round of the additional permutation series is finished. }

If\marginnote{3.9.91} he intends, “I’ll emit blue,” makes an effort, and the color of ghee is emitted, he commits an offense entailing suspension. 

If\marginnote{3.9.92} he intends, “I’ll emit yellow … red … white … the color of buttermilk … the color of water … the color of oil … the color of milk … the color of curd,” makes an effort, and the color of ghee is emitted, he commits an offense entailing suspension. 

\scend{The tenth round of the additional permutation series is finished. }

\scend{The additional permutation series is finished. }

\subsubsection*{Permutations part 2 }

If\marginnote{4.1} he intends, makes an effort, and semen is emitted, he commits an offense entailing suspension. 

If\marginnote{4.2} he intends, makes an effort, but semen is not emitted, he commits a serious offense. 

If\marginnote{4.3} he intends, but does not make an effort, yet semen is emitted, there is no offense. 

If\marginnote{4.4} he intends, but does not make an effort, nor is semen emitted, there is no offense. 

If\marginnote{4.5} he does not intend, but makes an effort, and semen is emitted, there is no offense. 

If\marginnote{4.6} he does not intend, but makes an effort, yet semen is not emitted, there is no offense. 

If\marginnote{4.7} he does not intend, nor make an effort, yet semen is emitted, there is no offense. 

If\marginnote{4.8} he does not intend, nor make an effort, nor is semen emitted, there is no offense. 

\subsection*{Non-offenses }

There\marginnote{4.9.1} is no offense: if it is while dreaming; if he is not aiming at emission; if he is insane; if he is deranged; if he is overwhelmed by pain; if he is the first offender. 

\scuddanaintro{Summary verses of case studies }

\begin{scuddana}%
“Dream,\marginnote{4.17} feces, urine, \\
Thought, and with warm water; \\
Medicine, scratching, path, \\
Foreskin, sauna, massage. 

Novice,\marginnote{4.21} and asleep, \\
Thigh, pressed with the fist; \\
In the air, rigid, staring, \\
Keyhole, rubbed with wood. 

Current,\marginnote{4.25} mud, running, \\
Mud play, lotus; \\
Sand, mud, pouring,\footnote{“Pouring” renders \textit{usseko}, which I take to be related to \textit{seka}, perhaps via the prefix \textit{ud}. } \\
Bed, and with the thumb.” 

%
\end{scuddana}

\subsubsection*{Case studies }

On\marginnote{5.1.1} one occasion a monk had an emission of semen while dreaming. He became anxious, thinking, “The Buddha has laid down a training rule. Could it be that I’ve committed an offense entailing suspension?” He told the Buddha, who said, “There’s no offense when it occurs while dreaming.” 

On\marginnote{5.2.1} one occasion a monk was defecating, and semen was emitted. He became anxious … “What were you thinking, monk?” 

“I\marginnote{5.2.4} wasn’t aiming at emission, sir.” 

“There’s\marginnote{5.2.5} no offense if one isn’t aiming at emission.” 

On\marginnote{5.2.6} one occasion a monk was urinating, and semen was emitted. He became anxious … “There’s no offense if one isn’t aiming at emission.” 

On\marginnote{5.3.1} one occasion a monk was thinking a worldly thought, and semen was emitted. He became anxious … “There’s no offense for one thinking a worldly thought.” 

On\marginnote{5.4.1} one occasion a monk was bathing in warm water, and semen was emitted. He became anxious … “What were you thinking, monk?” 

“I\marginnote{5.4.4} wasn’t aiming at emission, sir.” 

“There’s\marginnote{5.4.5} no offense if one isn’t aiming at emission.” 

On\marginnote{5.4.6} one occasion a monk bathed in warm water aiming at emission, and semen was emitted. He became anxious … “You’ve committed an offense entailing suspension.” 

On\marginnote{5.4.9} one occasion a monk bathed in warm water aiming at emission, but semen was not emitted. He became anxious … “There’s no offense entailing suspension, but there’s a serious offense.” 

At\marginnote{5.5.1} one time a monk had a sore on his penis. While he was applying medicine, semen was emitted. He became anxious … “There’s no offense if one isn’t aiming at emission.” 

At\marginnote{5.5.5} one time a monk had a sore on his penis. He applied medicine aiming at emission, and semen was emitted. … semen was not emitted. He became anxious … “There’s no offense entailing suspension, but there’s a serious offense.” 

On\marginnote{5.6.1} one occasion a monk scratched his scrotum, and semen was emitted. He became anxious … “There’s no offense if one isn’t aiming at emission.” 

On\marginnote{5.6.4} one occasion a monk scratched his scrotum aiming at emission, and semen was emitted. … semen was not emitted. He became anxious … “There’s no offense entailing suspension, but there’s a serious offense.” 

On\marginnote{5.7.1} one occasion a monk was walking along a path, and semen was emitted. He became anxious … “There’s no offense if one isn’t aiming at emission.” 

On\marginnote{5.7.4} one occasion a monk walked along a path aiming at emission, and semen was emitted. … semen was not emitted. He became anxious … “There’s no offense entailing suspension, but there’s a serious offense.” 

On\marginnote{5.8.1} one occasion a monk took hold of his foreskin, urinated, and semen was emitted.\footnote{Vmv 1.264: \textit{Vatthinti \textsanskrit{aṅgajātasīsacchādakacammaṁ}}, “\textit{Vatthi}: the skin covering the head of the penis.” } He became anxious … “There’s no offense if one isn’t aiming at emission.” 

On\marginnote{5.8.4} one occasion a monk, aiming at emission, took hold of his foreskin, urinated, and semen was emitted. … semen was not emitted. He became anxious … “There’s no offense entailing suspension, but there’s a serious offense.” 

On\marginnote{5.8.9} one occasion a monk was having his belly heated in the sauna, and semen was emitted. He became anxious … “There’s no offense if one isn’t aiming at emission.” 

On\marginnote{5.8.12} one occasion a monk, aiming at emission, had his belly heated in the sauna, and semen was emitted. … semen was not emitted. He became anxious … “There’s no offense entailing suspension, but there’s a serious offense.” 

On\marginnote{5.8.17} one occasion a monk massaged his preceptor’s back in the sauna, and semen was emitted. He became anxious … “There’s no offense if one isn’t aiming at emission.” 

On\marginnote{5.8.20} one occasion a monk, aiming at emission, massaged his preceptor’s back in the sauna, and semen was emitted. … semen was not emitted. He became anxious … “There’s no offense entailing suspension, but there’s a serious offense.” 

On\marginnote{5.8.25} one occasion a monk was having his thigh massaged, and semen was emitted. He became anxious … “There’s no offense if one isn’t aiming at emission.” 

On\marginnote{5.8.28} one occasion a monk, aiming at emission, had his thigh massaged, and semen was emitted. … semen was not emitted. He became anxious … “There’s no offense entailing suspension, but there’s a serious offense.” 

On\marginnote{5.9.1} one occasion a monk, aiming at emission, said to a novice, “Take hold of my penis.” The novice took hold of his penis, and the monk emitted semen. He became anxious … “There’s an offense entailing suspension.” 

On\marginnote{5.9.7} one occasion a monk took hold of the penis of a sleeping novice,\footnote{Bhikkhu \textsanskrit{Ṭhānissaro} argues in  “The Buddhist Monastic Code I”, page 105, that it was the novice monk who emitted semen. But these rules concern monks, not novices. In the present case the Buddha specifically says that there is no offense of \textit{\textsanskrit{saṅghādisesa}}, which would make no sense if it referred to the novice monk. Bhikkhu \textsanskrit{Ṭhānissaro} argues that his reading is required, otherwise we are forced to conclude that \textit{\textsanskrit{saṅghādisesa}} offenses can be incurred by indirect stimulation. Yet the cases found here are widely diverging. It is not a given that they must all, even potentially, give rise to a \textit{\textsanskrit{saṅghādisesa}}. } and the monk emitted semen. He became anxious … “There’s no offense entailing suspension, but there’s an offense of wrong conduct.” 

On\marginnote{5.10.1} one occasion a monk pressed his penis between his thighs aiming at emission, and semen was emitted. … semen was not emitted. He became anxious … “There’s no offense entailing suspension, but there’s a serious offense.” 

On\marginnote{5.10.6} one occasion a monk pressed his penis with his fist aiming at emission, and semen was emitted. … semen was not emitted. He became anxious … “There’s no offense entailing suspension, but there’s a serious offense.” 

On\marginnote{5.10.11} one occasion a monk shook his pelvis in the air aiming at emission, and semen was emitted. … semen was not emitted. He became anxious … “There’s no offense entailing suspension, but there’s a serious offense.” 

On\marginnote{5.11.1} one occasion a monk made his body rigid, and semen was emitted. He became anxious … “There’s no offense if one isn’t aiming at emission.” 

On\marginnote{5.11.4} one occasion a monk made his body rigid aiming at emission, and semen was emitted. … semen was not emitted. He became anxious … “There’s no offense entailing suspension, but there’s a serious offense.” 

On\marginnote{5.12.1} one occasion a lustful monk stared at a woman’s genitals, and semen was emitted. He became anxious … “There’s no offense entailing suspension. But you should not stare at a woman’s genitals motivated by lust. If you do, you commit an offense of wrong conduct.” 

On\marginnote{5.13.1} one occasion a monk inserted his penis into a keyhole aiming at emission, and semen was emitted. … semen was not emitted. He became anxious … “There’s no offense entailing suspension, but there’s a serious offense.” 

On\marginnote{5.14.1} one occasion a monk rubbed his penis with a piece of wood aiming at emission, and semen was emitted. … semen was not emitted. He became anxious … “There’s no offense entailing suspension, but there’s a serious offense.” 

On\marginnote{5.15.1} one occasion a monk bathed against the current, and semen was emitted. He became anxious … “There’s no offense if one isn’t aiming at emission.” 

On\marginnote{5.15.4} one occasion a monk bathed against the current aiming at emission, and semen was emitted. … semen was not emitted. He became anxious … “There’s no offense entailing suspension, but there’s a serious offense.” 

On\marginnote{5.16.1} one occasion a monk was playing in mud, and semen was emitted. He became anxious … “There’s no offense if one isn’t aiming at emission.” 

On\marginnote{5.16.4} one occasion a monk played in mud aiming at emission, and semen was emitted. … semen was not emitted. He became anxious … “There’s no offense entailing suspension, but there’s a serious offense.” 

On\marginnote{5.16.9} one occasion a monk ran in water, and semen was emitted. He became anxious … “There’s no offense if one isn’t aiming at emission.” 

On\marginnote{5.16.12} one occasion a monk ran in water aiming at emission, and semen was emitted. … semen was not emitted. He became anxious … “There’s no offense entailing suspension, but there’s a serious offense.” 

On\marginnote{5.16.17} one occasion a monk was playing by sliding in the mud, and semen was emitted.\footnote{Sp-\textsanskrit{ṭ} 1.267: \textit{\textsanskrit{Pupphāvalīti} \textsanskrit{kīḷāvisesassādhivacanaṁ}. \textsanskrit{Taṁ} \textsanskrit{kīḷantā} \textsanskrit{nadīādīsu} \textsanskrit{chinnataṭaṁ} udakena \textsanskrit{cikkhallaṁ} \textsanskrit{katvā} tattha ubho \textsanskrit{pāde} \textsanskrit{pasāretvā} \textsanskrit{nisinnā} papatanti}; “\textit{\textsanskrit{Pupphāvalīti}}: it is an expression for a kind of game. Playing it in rivers, etc., having made a muddy, steep slope with water, having extended both feet right there, they sit and slide down.” } He became anxious … “There’s no offense if one isn’t aiming at emission.” 

On\marginnote{5.16.20} one occasion a monk, aiming at emission, was playing by sliding in the mud, and semen was emitted. … semen was not emitted. He became anxious … “There’s no offense entailing suspension, but there’s a serious offense.” 

On\marginnote{5.16.25} one occasion a monk was running in a lotus grove, and semen was emitted. He became anxious … “There’s no offense if one isn’t aiming at emission.” 

On\marginnote{5.16.28} one occasion a monk ran in a lotus grove aiming at emission, and semen was emitted. … semen was not emitted. He became anxious … “There’s no offense entailing suspension, but there’s a serious offense.” 

On\marginnote{5.17.1} one occasion a monk inserted his penis into sand aiming at emission, and semen was emitted. … semen was not emitted. He became anxious … “There’s no offense entailing suspension, but there’s a serious offense.” 

On\marginnote{5.17.6} one occasion a monk inserted his penis into mud aiming at emission, and semen was emitted. … semen was not emitted. He became anxious … “There’s no offense entailing suspension, but there’s a serious offense.” 

On\marginnote{5.17.11} one occasion a monk poured water on his penis, and semen was emitted. He became anxious … “There’s no offense if one isn’t aiming at emission.” 

On\marginnote{5.17.14} one occasion a monk poured water on his penis aiming at emission, and semen was emitted. … semen was not emitted. He became anxious … “There’s no offense entailing suspension, but there’s a serious offense.” 

On\marginnote{5.17.19} one occasion a monk rubbed his penis against his bed aiming at emission, and semen was emitted. … semen was not emitted. He became anxious … “There’s no offense entailing suspension, but there’s a serious offense.” 

On\marginnote{5.17.24} one occasion a monk rubbed his penis with his thumb aiming at emission, and semen was emitted. … semen was not emitted. He became anxious, thinking, “The Buddha has laid down a training rule. Could it be that I’ve committed an offense entailing suspension?” He told the Buddha, who said, “There’s no offense entailing suspension, but there’s a serious offense.” 

\scendsutta{The training rule on emission of semen, the first, is finished. }

%
\section*{{\suttatitleacronym Bu Ss 2}{\suttatitletranslation 2. The training rule on physical contact }{\suttatitleroot Kāyasaṁsagga}}
\addcontentsline{toc}{section}{\tocacronym{Bu Ss 2} \toctranslation{2. The training rule on physical contact } \tocroot{Kāyasaṁsagga}}
\markboth{2. The training rule on physical contact }{Kāyasaṁsagga}
\extramarks{Bu Ss 2}{Bu Ss 2}

\subsection*{Origin story }

At\marginnote{1.1.1} one time the Buddha was staying at \textsanskrit{Sāvatthī} in the Jeta Grove, \textsanskrit{Anāthapiṇḍika}’s Monastery. At that time Venerable \textsanskrit{Udāyī} was staying in the wilderness. He had a beautiful dwelling with a room in the middle and corridors on all sides. The bed and bench were nicely made up, and the water for drinking and the water for washing were ready for use. The yards were well swept.\footnote{For the rendering of \textit{\textsanskrit{pariveṇa}} as “yard”, see Appendix of Technical Terms. } Many people came to see \textsanskrit{Udāyī}’s dwelling, 

among\marginnote{1.1.8} them a certain brahmin and his wife. They approached \textsanskrit{Udāyī} and said, “Venerable, we would like to see your dwelling.” 

“Well\marginnote{1.1.10} then, brahmin, please do.” 

\textsanskrit{Udāyī}\marginnote{1.1.11} took the key, lifted the latch, opened the door, and entered the dwelling. The brahmin entered after him and then the brahmin lady. Opening some windows and closing others, \textsanskrit{Udāyī} walked around the inner room and came up behind the brahmin lady, touching her all over. Then the brahmin thanked \textsanskrit{Udāyī} and left. 

And\marginnote{1.1.16} he expressed his delight, “These Sakyan monastics who live in the wilderness are superb. Venerable \textsanskrit{Udāyī} is superb!” 

But\marginnote{1.1.19} the brahmin lady said, “What’s superb about him? He touched me all over just like you do.” 

The\marginnote{1.1.22} brahmin then complained and criticized him, “These Sakyan monastics are shameless and immoral liars. They claim to have integrity, to be celibate and of good conduct, to be truthful, moral, and good. But they don’t have the good character of a monastic or a brahmin. They’ve lost the plot! How could the ascetic \textsanskrit{Udāyī} touch my wife all over? It’s not possible to go to a monastery or a monk’s dwelling with a wife from a respectable family, or with a daughter, a girl, a daughter-in-law, or a female slave from a respectable family. If you do, the Sakyan monastics might molest them.” 

The\marginnote{1.2.1} monks heard the criticism of that brahmin. The monks of few desires complained and criticized \textsanskrit{Udāyī}, “How could Venerable \textsanskrit{Udāyī} make physical contact with a woman?” 

They\marginnote{1.2.4} told the Buddha. He then had the Sangha gathered and questioned \textsanskrit{Udāyī}: 

“Is\marginnote{1.2.6} it true, \textsanskrit{Udāyī}, that you did this?” 

“It’s\marginnote{1.2.7} true, sir.” 

The\marginnote{1.2.8} Buddha rebuked him, “Foolish man, it’s not suitable, it’s not proper, it’s not worthy of a monastic, it’s not allowable, it’s not to be done. How could you do this? Haven’t I given many teachings for the sake of dispassion, not for the sake of passion … the stilling of the fevers of worldly pleasures? This will affect people’s confidence …” … “And, monks, this training rule should be recited like this: 

\subsection*{Final ruling }

\scrule{‘If a monk, overcome by lust and with a distorted mind, makes physical contact with a woman—holding her hand or hair, or touching any part of her body—he commits an offense entailing suspension.’” }

\subsection*{Definitions }

\begin{description}%
\item[A: ] whoever … %
\item[Monk: ] … The monk who has been given the full ordination by a unanimous Sangha through a legal procedure consisting of one motion and three announcements that is irreversible and fit to stand—this sort of monk is meant in this case. %
\item[Overcome by lust: ] having lust, longing for, in love with. %
\item[Distorted: ] a lustful mind is distorted; an angry mind is distorted; a confused mind is distorted. But in this case “distorted” refers to the lustful mind. %
\item[A woman: ] a female human being, not a female spirit, not a female ghost, not a female animal; even a girl born on that very day, let alone an older one. %
\item[With: ] together. %
\item[Makes physical contact: ] misconduct is what is meant. %
\item[Hand: ] from the elbow to the tip of the nails. %
\item[Hair: ] just the hair; or the hair with strings in it, with a garland, with gold coins, with gold, with pearls, or with gems.\footnote{“Gold coins” renders \textit{\textsanskrit{hirañña}}. See Appendix of Technical Terms for discussion. } %
\item[Any part of her body: ] anything apart from the hand and the hair is called “any part of her body”. %
\end{description}

\paragraph*{Sub-definitions }

Physical\marginnote{2.2.1} contact, touching, stroking downwards, stroking upwards, pulling down, lifting up, pulling, pushing, squeezing, pressing, taking hold of, contacting. 

\begin{description}%
\item[Physical contact: ] mere physical contact. %
\item[Touching: ] touching here and there. %
\item[Stroking downwards: ] lowering down. %
\item[Stroking upwards: ] raising up. %
\item[Pulling down: ] bending down. %
\item[Lifting up: ] raising up. %
\item[Pulling: ] drawing to. %
\item[Pushing: ] sending away. %
\item[Squeezing: ] taking hold of a bodily part and then pressing. %
\item[Pressing: ] pressing with something. %
\item[Take hold of: ] mere taking hold of. %
\item[Contacting: ] mere contact. %
\end{description}

\begin{description}%
\item[He commits an offense entailing suspension: ] … Therefore, too, it is called “an offense entailing suspension”. %
\end{description}

\subsection*{Permutations }

\subsubsection*{Permutations part 1 }

\subparagraph*{Making direct contact with a single person or animal: body to body }

It\marginnote{3.1.1} is a woman, he perceives her as a woman, and he has lust. If the monk makes physical contact with the woman, body to body, if he touches her, strokes her downwards, strokes her upwards, pulls her down, lifts her up, pulls her, pushes her, squeezes her, presses her, takes hold of her, contacts her, he commits an offense entailing suspension. 

It\marginnote{3.1.3} is a woman, but he is unsure of it, and he has lust. If the monk makes physical contact with the woman, body to body, if he touches her … takes hold of her, contacts her, he commits a serious offense. 

It\marginnote{3.1.6} is a woman, but he perceives her as a \textit{\textsanskrit{paṇḍaka}}, and he has lust.\footnote{For the meaning of the term \textit{\textsanskrit{paṇḍaka}}, see Appendix of Technical Terms. } If the monk makes physical contact with the woman, body to body, if he touches her … takes hold of her, contacts her, he commits a serious offense. 

It\marginnote{3.1.9} is a woman, but he perceives her as a man, and he has lust. If the monk makes physical contact with the woman, body to body, if he touches her … takes hold of her, contacts her, he commits a serious offense. 

It\marginnote{3.1.12} is a woman, but he perceives her as an animal, and he has lust. If the monk makes physical contact with the woman, body to body, if he touches her … takes hold of her, contacts her, he commits a serious offense. 

It\marginnote{3.1.15} is a \textit{\textsanskrit{paṇḍaka}}, he perceives him as a \textit{\textsanskrit{paṇḍaka}}, and he has lust. If the monk makes physical contact with the \textit{\textsanskrit{paṇḍaka}}, body to body, if he touches him … takes hold of him, contacts him, he commits a serious offense. 

It\marginnote{3.1.18} is a \textit{\textsanskrit{paṇḍaka}}, but he is unsure of it, and he has lust. If the monk makes physical contact with the \textit{\textsanskrit{paṇḍaka}}, body to body, if he touches him … takes hold of him, contacts him, he commits an offense of wrong conduct. 

It\marginnote{3.1.21} is a \textit{\textsanskrit{paṇḍaka}}, but he perceives him as a man, and he has lust. If the monk makes physical contact with the \textit{\textsanskrit{paṇḍaka}}, body to body, if he touches him … takes hold of him, contacts him, he commits an offense of wrong conduct. 

It\marginnote{3.1.24} is a \textit{\textsanskrit{paṇḍaka}}, but he perceives him as an animal, and he has lust. If the monk makes physical contact with the \textit{\textsanskrit{paṇḍaka}}, body to body, if he touches him … takes hold of him, contacts him, he commits an offense of wrong conduct. 

It\marginnote{3.1.27} is a \textit{\textsanskrit{paṇḍaka}}, but he perceives him as a woman, and he has lust. If the monk makes physical contact with the \textit{\textsanskrit{paṇḍaka}}, body to body, if he touches him … takes hold of him, contacts him, he commits an offense of wrong conduct. 

It\marginnote{3.1.30} is a man, he perceives him as a man, and he has lust. If the monk makes physical contact with the man, body to body, if he touches him … takes hold of him, contacts him, he commits an offense of wrong conduct. 

It\marginnote{3.1.33} is a man, but he is unsure of it … It is a man, but he perceives him as an animal … It is a man, but he perceives him as a woman … It is a man, but he perceives him as a \textit{\textsanskrit{paṇḍaka}}, and he has lust. If the monk makes physical contact with the man, body to body, if he touches him … takes hold of him, contacts him, he commits an offense of wrong conduct. 

It\marginnote{3.1.39} is an animal, he perceives it as an animal, and he has lust. If the monk makes physical contact with the animal, body to body, if he touches it … takes hold of it, contacts it, he commits an offense of wrong conduct. 

It\marginnote{3.1.42} is an animal, but he is unsure of it … It is an animal, but he perceives it as a woman … It is an animal, but he perceives it as a \textit{\textsanskrit{paṇḍaka}} … It is an animal, but he perceives it as a man, and he has lust. If the monk makes physical contact with the animal, body to body, if he touches it … takes hold of it, contacts it, he commits an offense of wrong conduct. 

\scend{The section based on one item is finished.\footnote{See Appendix of Specialized Vocabulary under \textit{\textsanskrit{mūlaka}}. } }

\subparagraph*{Making direct contact with two beings of the same kind: body to body }

It\marginnote{3.2.1} is two women, he perceives both as women, and he has lust. If the monk makes physical contact with the two women, body to body, if he touches them … takes hold of them, contacts them, he commits two offenses entailing suspension. 

It\marginnote{3.2.4} is two women, but he is unsure about both, and he has lust. If the monk makes physical contact with the two women, body to body, if he touches them … takes hold of them, contacts them, he commits two serious offenses. 

It\marginnote{3.2.7} is two women, but he perceives both as \textit{\textsanskrit{paṇḍakas}} … but he perceives both as men … but he perceives both as animals, and he has lust. If the monk makes physical contact with the two women, body to body, if he touches them … takes hold of them, contacts them, he commits two serious offenses. 

It\marginnote{3.2.12} is two \textit{\textsanskrit{paṇḍakas}}, he perceives both as \textit{\textsanskrit{paṇḍakas}}, and he has lust. If the monk makes physical contact with the two \textit{\textsanskrit{paṇḍakas}}, body to body, if he touches them … takes hold of them, contacts them, he commits two serious offenses. 

It\marginnote{3.2.14} is two \textit{\textsanskrit{paṇḍakas}}, but he is unsure about both … but he perceives both as men … but he perceives both as animals … but he perceives both as women, and he has lust. If the monk makes physical contact with the two \textit{\textsanskrit{paṇḍakas}}, body to body, if he touches them … takes hold of them, contacts them, he commits two offenses of wrong conduct. 

It\marginnote{3.2.20} is two men, he perceives both as men, and he has lust. If the monk makes physical contact with the two men, body to body, if he touches them … takes hold of them, contacts them, he commits two offenses of wrong conduct. 

It\marginnote{3.2.22} is two men, but he is unsure about both … but he perceives both as animals … but he perceives both as women … but he perceives both as \textit{\textsanskrit{paṇḍakas}}, and he has lust. If the monk makes physical contact with the two men, body to body, if he touches them … takes hold of them, contacts them, he commits two offenses of wrong conduct. 

It\marginnote{3.2.28} is two animals, he perceives both as animals, and he has lust. If the monk makes physical contact with the two animals, body to body, if he touches them … takes hold of them, contacts them, he commits two offenses of wrong conduct. 

It\marginnote{3.2.31} is two animals, but he is unsure about both … but he perceives both as women … but he perceives both as \textit{\textsanskrit{paṇḍakas}} … but he perceives both as men, and he has lust. If the monk makes physical contact with the two animals, body to body, if he touches them … takes hold of them, contacts them, he commits two offenses of wrong conduct. 

\subparagraph*{Making direct contact with two beings of different kinds: body to body }

It\marginnote{3.3.1} is a woman and a \textit{\textsanskrit{paṇḍaka}}, but he perceives both as women, and he has lust. If the monk makes physical contact with both, body to body, if he touches them … takes hold of them, contacts them, he commits one offense entailing suspension and one offense of wrong conduct. 

It\marginnote{3.3.4} is a woman and a \textit{\textsanskrit{paṇḍaka}}, but he is unsure about both, and he has lust. If the monk makes physical contact with both, body to body, if he touches them … takes hold of them, contacts them, he commits one serious offense and one offense of wrong conduct. 

It\marginnote{3.3.7} is a woman and a \textit{\textsanskrit{paṇḍaka}}, but he perceives both as \textit{\textsanskrit{paṇḍakas}}, and he has lust. If the monk makes physical contact with both, body to body, if he touches them … takes hold of them, contacts them, he commits two serious offenses. 

It\marginnote{3.3.10} is a woman and a \textit{\textsanskrit{paṇḍaka}}, but he perceives both as men, and he has lust. If the monk makes physical contact with both, body to body, if he touches them … takes hold of them, contacts them, he commits one serious offense and one offense of wrong conduct. 

It\marginnote{3.3.13} is a woman and a \textit{\textsanskrit{paṇḍaka}}, but he perceives both as animals, and he has lust. If the monk makes physical contact with both, body to body, if he touches them … takes hold of them, contacts them, he commits one serious offense and one offense of wrong conduct. 

It\marginnote{3.3.16} is a woman and a man, but he perceives both as women and he has lust. If the monk makes physical contact with both, body to body, if he touches them … takes hold of them, contacts them, he commits one offense entailing suspension and one offense of wrong conduct. 

It\marginnote{3.3.19} is a woman and a man, but he is unsure about both … but he perceives both as \textit{\textsanskrit{paṇḍakas}} … but he perceives both as men … but he perceives both as animals, and he has lust. If the monk makes physical contact with both, body to body, if he touches them … takes hold of them, contacts them, he commits one serious offense and one offense of wrong conduct. 

It\marginnote{3.3.25} is a woman and an animal, but he perceives both as women, and he has lust. If the monk makes physical contact with both, body to body, if he touches them … takes hold of them, contacts them, he commits one offense entailing suspension and one offense of wrong conduct. 

It\marginnote{3.3.28} is a woman and an animal, but he is unsure about both … but he perceives both as \textit{\textsanskrit{paṇḍakas}} … but he perceives both as men … but he perceives both as animals, and he has lust. If the monk makes physical contact with both, body to body, if he touches them … takes hold of them, contacts them, he commits one serious offense and one offense of wrong conduct. 

It\marginnote{3.3.34} is a \textit{\textsanskrit{paṇḍaka}} and a man, but he perceives both as \textit{\textsanskrit{paṇḍakas}}, and he has lust. If the monk makes physical contact with both, body to body, if he touches them … takes hold of them, contacts them, he commits one serious offense and one offense of wrong conduct. 

It\marginnote{3.3.37} is a \textit{\textsanskrit{paṇḍaka}} and a man, but he is unsure about both … but he perceives both as men … but he perceives both as animals … but he perceives both as women, and he has lust. If the monk makes physical contact with both, body to body, if he touches them … takes hold of them, contacts them, he commits two offenses of wrong conduct. 

It\marginnote{3.3.43} is a \textit{\textsanskrit{paṇḍaka}} and an animal, but he perceives both as \textit{\textsanskrit{paṇḍakas}}, and he has lust. If the monk makes physical contact with both, body to body, if he touches them … takes hold of them, contacts them, he commits one serious offense and one offense of wrong conduct. 

It\marginnote{3.3.46} is a \textit{\textsanskrit{paṇḍaka}} and an animal, but he is unsure about both … but he perceives both as men … but he perceives both as animals … but he perceives both as women, and he has lust. If the monk makes physical contact with both, body to body, if he touches them … takes hold of them, contacts them, he commits two offenses of wrong conduct. 

It\marginnote{3.3.52} is a man and an animal, but he perceives both as men, and he has lust. If the monk makes physical contact with both, body to body, if he touches them … takes hold of them, contacts them, he commits two offenses of wrong conduct. 

It\marginnote{3.3.55} is a man and an animal, but he is unsure about both … but he perceives both as animals … but he perceives both as women … but he perceives both as \textit{\textsanskrit{paṇḍakas}}, and he has lust. If the monk makes physical contact with both, body to body, if he touches them … takes hold of them, contacts them, he commits two offenses of wrong conduct. 

\scend{The section based on two items is finished. }

\subparagraph*{Making indirect contact: body to what is connected to the body }

It\marginnote{3.4.1} is a woman, he perceives her as a woman, and he has lust. If the monk, with his own body, makes physical contact with something connected to her body, if he touches it … takes hold of it, contacts it, he commits a serious offense. … 

It\marginnote{3.4.4} is two women, he perceives both as women, and he has lust. If the monk, with his own body, makes physical contact with something connected to the body of both, if he touches it … takes hold of it, contacts it, he commits two serious offenses. … 

It\marginnote{3.4.7} is a woman and a \textit{\textsanskrit{paṇḍaka}}, but he perceives both as women, and he has lust. If the monk, with his own body, makes physical contact with something connected to the body of both, if he touches it … takes hold of it, contacts it, he commits one serious offense and one offense of wrong conduct. … 

It\marginnote{3.4.10} is a woman, he perceives her as a woman, and he has lust. If the monk, with something connected to his own body, makes physical contact with her body, if he touches it … takes hold of it, contacts it, he commits a serious offense. … 

It\marginnote{3.4.13} is two women, he perceives both as women, and he has lust. If the monk, with something connected to his own body, makes physical contact with the body of both, if he touches them … takes hold of them, contacts them, he commits two serious offenses. … 

It\marginnote{3.4.16} is a woman and a \textit{\textsanskrit{paṇḍaka}}, but he perceives both as women, and he has lust. If the monk, with something connected to his own body, makes physical contact with the body of both, if he touches them … takes hold of them, contacts them, he commits one serious offense and one offense of wrong conduct. … 

\subparagraph*{Making indirect contact: what is connected to the body to what is connected to the body }

It\marginnote{3.4.19.1} is a woman, he perceives her as a woman, and he has lust. If the monk, with something connected to his own body, makes physical contact with something connected to her body, if he touches it … takes hold of it, contacts it, he commits an offense of wrong conduct. … 

It\marginnote{3.4.22} is two women, he perceives both as women, and he has lust. If the monk, with something connected to his own body, makes physical contact with something connected to the body of both, if he touches those things … takes hold of them, contacts them, he commits two offenses of wrong conduct. … 

It\marginnote{3.4.25} is a woman and a \textit{\textsanskrit{paṇḍaka}}, but he perceives both as women, and he has lust. If the monk, with something connected to his own body, makes physical contact with something connected to the body of both, if he touches those things … takes hold of them, contacts them, he commits two offenses of wrong conduct. … 

\subparagraph*{Making indirect contact: contact by releasing }

It\marginnote{3.4.28.1} is a woman, he perceives her as a woman, and he has lust.\footnote{“Releasing” means throwing, dropping, etc. } If the monk, with something released by him, makes physical contact with her body, he commits an offense of wrong conduct. … 

It\marginnote{3.4.30} is two women, he perceives both as women, and he has lust. If the monk, with something released by him, makes physical contact with the body of both, he commits two offenses of wrong conduct. … 

It\marginnote{3.4.32} is a woman and a \textit{\textsanskrit{paṇḍaka}}, but he perceives both as women, and he has lust. If the monk, with something released by him, makes physical contact with the body of both, he commits two offenses of wrong conduct. … 

It\marginnote{3.4.34} is a woman, he perceives her as a woman, and he has lust. If the monk, with something released by him, makes physical contact with something connected to her body, he commits an offense of wrong conduct. … 

It\marginnote{3.4.36} is two women, he perceives both as women, and he has lust. If the monk, with something released by him, makes physical contact with something connected to the body of both, he commits two offenses of wrong conduct. … 

It\marginnote{3.4.38} is a woman and a \textit{\textsanskrit{paṇḍaka}}, but he perceives both as women, and he has lust. If the monk, with something released by him, makes physical contact with something connected to the body of both, he commits two offenses of wrong conduct. … 

It\marginnote{3.4.40} is a woman, he perceives her as a woman, and he has lust. If the monk, with something released by him, makes physical contact with something released by her, he commits an offense of wrong conduct. … 

It\marginnote{3.4.42} is two women, he perceives both as women, and he has lust. If the monk, with something released by him, makes physical contact with something released by both, he commits two offenses of wrong conduct. … 

It\marginnote{3.4.44} is a woman and a \textit{\textsanskrit{paṇḍaka}}, but he perceives both as women, and he has lust. If the monk, with something released by him, makes physical contact with something released by both, he commits two offenses of wrong conduct. … 

\scend{The successive series on a monk is finished.\footnote{See Appendix of Specialized Vocabulary under \textit{\textsanskrit{peyyāla}}. } }

\subparagraph*{Others making direct contact with a monk: body to body }

It\marginnote{3.5.1} is a woman, he perceives her as a woman, and he has lust. If the woman makes physical contact with the monk, body to body, if she touches him, strokes him downwards, strokes him upwards, pulls him down, lifts him up, pulls him, pushes him, squeezes him, presses him, takes hold of him, contacts him, and he, aiming at connection, makes an effort with the body and experiences contact, he commits an offense entailing suspension. … 

It\marginnote{3.5.3} is two women, he perceives both as women, and he has lust. If the women make physical contact with the monk, body to body, if they touch him, stroke him downwards, stroke him upwards, pull him down, lift him up, pull him, push him, squeeze him, press him, take hold of him, contact him, and he, aiming at connection, makes an effort with the body and experiences contact, he commits two offenses entailing suspension. … 

It\marginnote{3.5.5} is a woman and a \textit{\textsanskrit{paṇḍaka}}, but he perceives both as women, and he has lust. If they both make physical contact with the monk, body to body, if they touch him … take hold of him, contact him, and he, aiming at connection, makes an effort with the body and experiences contact, he commits one offense entailing suspension and one offense of wrong conduct. … 

\subparagraph*{Others making indirect contact with a monk: body to what is connected to the body }

It\marginnote{3.5.8.1} is a woman, he perceives her as a woman, and he has lust. If the woman, with her own body, makes physical contact with something connected to his body, if she touches it … takes hold of it, contacts it, and he, aiming at connection, makes an effort with the body and experiences contact, he commits a serious offense. … 

It\marginnote{3.5.11} is two women, he perceives both as women, and he has lust. If the women, with their own bodies, make physical contact with something connected to his body, if they touch it … take hold of it, contact it, and he, aiming at connection, makes an effort with the body and experiences contact, he commits two serious offenses. … 

It\marginnote{3.5.14} is a woman and a \textit{\textsanskrit{paṇḍaka}}, but he perceives both as women, and he has lust. If they both, with their own bodies, make physical contact with something connected to his body, if they touch it … take hold of it, contact it, and he, aiming at connection, makes an effort with the body and experiences contact, he commits one serious offense and one offense of wrong conduct. … 

It\marginnote{3.5.17} is a woman, he perceives her as a woman, and he has lust. If the woman, with something connected to her own body, makes physical contact with his body, if she touches him … takes hold of him, contacts him, and he, aiming at connection, makes an effort with the body and experiences contact, he commits a serious offense. … 

It\marginnote{3.5.20} is two women, he perceives both as women, and he has lust. If the women, with something connected to their own bodies, make physical contact with his body, if they touch him … take hold of him, contact him, and he, aiming at connection, makes an effort with the body and experiences contact, he commits two serious offenses. … 

It\marginnote{3.5.23} is a woman and a \textit{\textsanskrit{paṇḍaka}}, but he perceives both as women, and he has lust. If they both, with something connected to their own bodies, make physical contact with his body, if they touch him … take hold of him, contact him, and he, aiming at connection, makes an effort with the body and experiences contact, he commits one serious offense and one offense of wrong conduct. … 

\subparagraph*{Others making indirect contact with a monk: what is connected to the body to what is connected to the body }

It\marginnote{3.5.26.1} is a woman, he perceives her as a woman, and he has lust. If the woman, with something connected to her own body, makes physical contact with something connected to his body, if she touches it … takes hold of it, contacts it, and he, aiming at connection, makes an effort with the body and experiences contact, he commits an offense of wrong conduct. … 

It\marginnote{3.5.29} is two women, he perceives both as women, and he has lust. If the women, with something connected to their own bodies, make physical contact with something connected to his body, if they touch it … take hold of it, contact it, and he, aiming at connection, makes an effort with the body and experiences contact, he commits two offenses of wrong conduct. … 

It\marginnote{3.5.32} is a woman and a \textit{\textsanskrit{paṇḍaka}}, but he perceives both as women, and he has lust. If they both, with something connected to their own bodies, make physical contact with something connected to his body, if they touch it … take hold of it, contact it, and he, aiming at connection, makes an effort with the body and experiences contact, he commits two offenses of wrong conduct. … 

\subparagraph*{Others making indirect contact with a monk: contact by releasing }

It\marginnote{3.5.35.1} is a woman, he perceives her as a woman, and he has lust. If the woman, with something released by her, makes physical contact with his body, and he, aiming at connection, makes an effort with the body and experiences contact, he commits an offense of wrong conduct. … 

It\marginnote{3.5.38} is two women, he perceives both as women, and he has lust. If the women, with something released by both, make physical contact with his body, and he, aiming at connection, makes an effort with the body and experiences contact, he commits two offenses of wrong conduct. … 

It\marginnote{3.5.41} is a woman and a \textit{\textsanskrit{paṇḍaka}}, but he perceives both as women, and he has lust. If they both, with something released by both, make physical contact with his body, and he, aiming at connection, makes an effort with the body and experiences contact, he commits two offenses of wrong conduct. … 

It\marginnote{3.5.44} is a woman, he perceives her as a woman, and he has lust. If the woman, with something released by her, makes physical contact with something connected to his body, and he, aiming at connection, makes an effort with the body and experiences contact, he commits an offense of wrong conduct. … 

It\marginnote{3.5.47} is two women, he perceives both as women, and he has lust. If the women, with something released by both, make physical contact with something connected to his body, and he, aiming at connection, makes an effort with the body and experiences contact, he commits two offenses of wrong conduct. … 

It\marginnote{3.5.50} is a woman and a \textit{\textsanskrit{paṇḍaka}}, but he perceives both as women, and he has lust. If they both, with something released by both, make physical contact with something connected to his body, and he, aiming at connection, makes an effort with the body and experiences contact, he commits two offenses of wrong conduct. … 

It\marginnote{3.5.53} is a woman, he perceives her as a woman, and he has lust. If the woman, with something released by her, makes physical contact with something released by him, and he, aiming at connection, makes an effort with the body, but does not experience contact, he commits an offense of wrong conduct. …\footnote{Sp 1.278: \textit{Ettha ca \textsanskrit{kāyena} \textsanskrit{vāyamati} na ca \textsanskrit{phassaṁ} \textsanskrit{paṭivijānātīti} \textsanskrit{attanā} \textsanskrit{nissaṭṭhaṁ} \textsanskrit{pupphaṁ} \textsanskrit{vā} \textsanskrit{phalaṁ} \textsanskrit{vā} \textsanskrit{itthiṁ} attano nissaggiyena pupphena \textsanskrit{vā} phalena \textsanskrit{vā} \textsanskrit{paharantiṁ} \textsanskrit{disvā} \textsanskrit{kāyena} \textsanskrit{vikāraṁ} karoti, \textsanskrit{aṅguliṁ} \textsanskrit{vā} \textsanskrit{cāleti}, \textsanskrit{bhamukaṁ} \textsanskrit{vā} ukkhipati, \textsanskrit{akkhiṁ} \textsanskrit{vā} \textsanskrit{nikhaṇati}, \textsanskrit{aññaṁ} \textsanskrit{vā} \textsanskrit{evarūpaṁ} \textsanskrit{vikāraṁ} karoti, \textsanskrit{ayaṁ} vuccati “\textsanskrit{kāyena} \textsanskrit{vāyamati} na ca \textsanskrit{phassaṁ} \textsanskrit{paṭivijānātī}”ti}; “Here ʻmakes an effort with the body, but does not experience contact’ means: having seen a flower or fruit released from oneself hitting a flower or fruit released from the woman, he makes a gesture with the body, wags a finger, raises an eyebrow, winks, or makes any similar gesture—this is called ʻmakes an effort with the body, but does not experience contact’.” } 

It\marginnote{3.5.56} is two women, he perceives both as women, and he has lust. If the women, with something released by both, make physical contact with something released by him, and he, aiming at connection, makes an effort with the body, but does not experience contact, he commits two offenses of wrong conduct. … 

It\marginnote{3.5.59} is a woman and a \textit{\textsanskrit{paṇḍaka}}, but he perceives both as women, and he has lust. If they both, with something released by both, make physical contact with something released by him, and he, aiming at connection, makes an effort with the body, but does not experience contact, he commits two offenses of wrong conduct. … 

\subsubsection*{Permutations part 2 }

If,\marginnote{3.6.1} aiming at connection, he makes an effort with the body and experiences contact, he commits an offense entailing suspension. 

If,\marginnote{3.6.2} aiming at connection, he makes an effort with the body, but does not experience contact, he commits an offense of wrong conduct. 

If,\marginnote{3.6.3} aiming at connection, he makes no effort with the body, but experiences contact, there is no offense. 

If,\marginnote{3.6.4} aiming at connection, he makes no effort with the body and does not experience contact, there is no offense. 

If,\marginnote{3.6.5} aiming to free himself, he makes an effort with the body and experiences contact, there is no offense. 

If,\marginnote{3.6.6} aiming to free himself, he makes an effort with the body, but does not experience contact, there is no offense. 

If,\marginnote{3.6.7} aiming to free himself, he makes no effort with the body, but experiences contact, there is no offense. 

If,\marginnote{3.6.8} aiming to free himself, he makes no effort with the body and does not experience contact, there is no offense. 

\subsection*{Non-offenses }

There\marginnote{3.7.1} is no offense: if it is unintentional; if he is not mindful; if he does not know; if he does not consent; if he is insane; if he is deranged; if he is overwhelmed by pain; if he is the first offender. 

\scuddanaintro{Summary verses of case studies }

\begin{scuddana}%
“Mother,\marginnote{3.7.11} daughter, and sister, \\
Wife, and female spirit, \textit{\textsanskrit{paṇḍaka}}; \\
Asleep, dead, female animal, \\
And with a wooden doll. 

About\marginnote{3.7.15} harassment, bridge, road, \\
Tree, and boat, and rope; \\
A staff, pushed with a bowl, \\
When paying respect, made an effort but did not touch.” 

%
\end{scuddana}

\subsubsection*{Case studies }

At\marginnote{4.1.1} one time a monk touched his mother out of affection. He became anxious, thinking, “The Buddha has laid down a training rule. Could it be that I’ve committed an offense entailing suspension?” He told the Buddha, who said, “There’s no offense entailing suspension, but there’s an offense of wrong conduct.” 

At\marginnote{4.1.8} one time a monk touched his daughter out of affection … his sister out of affection. He became anxious … “There’s no offense entailing suspension, but there’s an offense of wrong conduct.” 

At\marginnote{4.2.1} one time a monk made physical contact with his ex-wife. He became anxious … “You’ve committed an offense entailing suspension.” 

At\marginnote{4.3.1} one time a monk made physical contact with a female spirit. He became anxious … “There’s no offense entailing suspension, but there’s a serious offense.” 

At\marginnote{4.3.5} one time a monk made physical contact with a \textit{\textsanskrit{paṇḍaka}}. He became anxious … “There’s no offense entailing suspension, but there’s a serious offense.” 

At\marginnote{4.4.1} one time a monk made physical contact with a sleeping woman. He became anxious … “You’ve committed an offense entailing suspension.” 

At\marginnote{4.4.4} one time a monk made physical contact with a dead woman. He became anxious … “There’s no offense entailing suspension, but there’s a serious offense.” 

At\marginnote{4.4.8} one time a monk made physical contact with a female animal. … “There’s no offense entailing suspension, but there’s an offense of wrong conduct.” 

At\marginnote{4.4.12} one time a monk made physical contact with a wooden doll. … “There’s no offense entailing suspension, but there’s an offense of wrong conduct.” 

At\marginnote{4.5.1} one time a number of women harassed a monk by leading him about arm in arm. He became anxious … “Did you consent, monk?” 

“No,\marginnote{4.5.4} sir.” 

“There’s\marginnote{4.5.5} no offense if one doesn’t consent.” 

At\marginnote{4.6.1} one time a lustful monk shook the bridge that a woman was standing on. He became anxious … “There’s no offense entailing suspension, but there’s an offense of wrong conduct.” 

At\marginnote{4.7.1} one time a monk saw a woman coming in the opposite direction, and being lustful, he struck her with his shoulder. He became anxious … “You’ve committed an offense entailing suspension.” 

At\marginnote{4.8.1} one time a lustful monk shook the tree that a woman had climbed. He became anxious … “There’s no offense entailing suspension, but there’s an offense of wrong conduct.” 

At\marginnote{4.8.5} one time a lustful monk shook the boat that a woman had boarded. He became anxious … “There’s no offense entailing suspension, but there’s an offense of wrong conduct.” 

At\marginnote{4.9.1} one time a lustful monk pulled the rope that a woman was holding. He became anxious … “There’s no offense entailing suspension, but there’s a serious offense.” 

At\marginnote{4.9.5} one time a lustful monk pulled the staff that a woman was holding. He became anxious … “There’s no offense entailing suspension, but there’s a serious offense.” 

At\marginnote{4.10.1} one time a lustful monk pushed a woman with his almsbowl. He became anxious … “There’s no offense entailing suspension, but there’s a serious offense.” 

At\marginnote{4.11.1} one time a lustful monk raised his foot as a woman was paying respect to him. He became anxious … “You’ve committed an offense entailing suspension.” 

At\marginnote{4.11.4} one time a monk, thinking, “I’ll take hold of a woman,” exerted himself, but did not make contact. He became anxious … “There’s no offense entailing suspension, but there’s an offense of wrong conduct.” 

\scendsutta{The training rule on physical contact, the second, is finished. }

%
\section*{{\suttatitleacronym Bu Ss 3}{\suttatitletranslation 3. The training rule on indecent speech }{\suttatitleroot Duṭṭhullavācā}}
\addcontentsline{toc}{section}{\tocacronym{Bu Ss 3} \toctranslation{3. The training rule on indecent speech } \tocroot{Duṭṭhullavācā}}
\markboth{3. The training rule on indecent speech }{Duṭṭhullavācā}
\extramarks{Bu Ss 3}{Bu Ss 3}

\subsection*{Origin story }

At\marginnote{1.1.1} one time the Buddha was staying at \textsanskrit{Sāvatthī} in the Jeta Grove, \textsanskrit{Anāthapiṇḍika}’s Monastery. At that time Venerable \textsanskrit{Udāyī} was staying in a beautiful dwelling in the wilderness. On one occasion a number of women came to the monastery to see the dwellings. They approached \textsanskrit{Udāyī} and said, “Venerable, we would like to see your dwelling.” 

Then,\marginnote{1.1.8} after showing them his dwelling, he praised and disparaged their private parts; he requested and implored, asked and enquired, described, instructed, and abused their private parts.\footnote{\textit{\textsanskrit{Vaccamaggaṁ} \textsanskrit{passāvamaggaṁ}} refer respectively to the anus and the genitals. } The shameless and indecent women flirted with \textsanskrit{Udāyī}; they called out to him, laughed with him, and teased him. But after leaving, those of them who had a sense of shame complained to the monks, “Venerables, this isn’t suitable or appropriate. We wouldn’t want to hear such speech from our own husbands, let alone from Venerable \textsanskrit{Udāyī}.” 

The\marginnote{1.2.1} monks of few desires complained and criticized him, “How could Venerable \textsanskrit{Udāyī} speak indecently to women?” 

They\marginnote{1.2.3} rebuked \textsanskrit{Udāyī} in many ways and told the Buddha. The Buddha had the Sangha gathered and questioned \textsanskrit{Udāyī}: “Is it true, \textsanskrit{Udāyī}, that you spoke like this?” 

“It’s\marginnote{1.2.6} true, sir.” 

The\marginnote{1.2.7} Buddha rebuked him, “It’s not suitable, foolish man, it’s not proper, it’s not worthy of a monastic, it’s not allowable, it’s not to be done. How could you speak like this? Haven’t I given many teachings for the sake of dispassion, not for the sake of passion … the stilling of the fevers of worldly pleasures? This will affect people’s confidence …” … “And, monks, this training rule should be recited like this: 

\subsection*{Final ruling }

\scrule{‘If a monk, overcome by lust and with a distorted mind, speaks indecent words to a woman, as a young man might to a young woman and referring to sexual intercourse, he commits an offense entailing suspension.’” }

\subsection*{Definitions }

\begin{description}%
\item[A: ] whoever … %
\item[Monk: ] … The monk who has been given the full ordination by a unanimous Sangha through a legal procedure consisting of one motion and three announcements that is irreversible and fit to stand—this sort of monk is meant in this case. %
\item[Overcome by lust: ] having lust, longing for, in love with. %
\item[Distorted: ] a lustful mind is distorted; an angry mind is distorted; a confused mind is distorted. But in this case “distorted” refers to the lustful mind. %
\item[A woman: ] a female human being, not a female spirit, not a female ghost, not a female animal. She understands and is capable of discerning bad speech and good speech, what is decent and what is indecent. %
\item[Indecent words: ] speech connected with the anus, the vagina, or sexual intercourse. %
\item[Speaks: ] misconduct is what is meant. %
\item[As a young man might to a young woman: ] a male youth to a female youth, a boy to a girl, a male who indulges in worldly pleasures to a female who indulges in worldly pleasures. %
\item[Referring to sexual intercourse: ] connected with the sexual act. %
\item[He commits an offense entailing suspension: ] … Therefore, too, it is called “an offense entailing suspension”. %
\end{description}

\subsection*{Permutations }

\paragraph*{Summary }

Referring\marginnote{3.1.1} to the two private orifices, he praises, disparages, requests, implores, asks, enquires, describes, instructs, abuses. 

\paragraph*{Sub-definitions }

\begin{description}%
\item[Praises: ] extols, praises, commends the two private orifices. %
\item[Disparages: ] despises, reviles, censures the two private orifices. %
\item[Requests: ] saying, “Give to me; you should give to me.” %
\item[Implores: ] saying, “When will you please your mother?” “When will you please your father?” “When will you please the gods?” “When will there be a good opportunity, a good time, a good moment?” “When will I have sexual intercourse with you?” %
\item[Asks: ] saying, “How do you give to your husband?” “How do you give to your lover?” %
\item[Enquires: ] saying, “So you give to your husband like this, and to your lover like this?” %
\item[Describes: ] when asked, he says, “Give like this. If you do, you’ll be dear and pleasing to your husband.” %
\item[Instructs: ] without being asked, he says, “Give like this. If you do, you’ll be dear and pleasing to your husband.” %
\item[Abuses: ] saying, “You lack genitals;” “You have incomplete genitals;” “You don’t menstruate;” “You menstruate continuously;” “You always wear a menstruation pad;” “You’re incontinent;” “You have genital prolapse;” “You lack sexual organs;” “You’re manlike;” “You have fistula;” “You’re a hermaphrodite.” %
\end{description}

\paragraph*{Exposition }

\subparagraph*{Referring to the private parts of a single person or animal }

It\marginnote{3.2.1} is a woman, he perceives her as a woman, and he has lust. If the monk, referring to the anus or the vagina of the woman, praises, disparages, requests, implores, asks, enquires, describes, instructs, or abuses, he commits an offense entailing suspension. … … (To be expanded as in \href{https://suttacentral.net/pli-tv-bu-vb-ss2\#3.1.3}{Bu Ss 2:3.1.3}–Bu Ss 2:3.1.49, with appropriate substitutions.) …\footnote{There are no instructions in the Pali or in the commentaries for how the expansion here or below is to be done. I therefore assume it is to be done in accordance with the previous rule. } 

\subparagraph*{Referring to the private parts of two beings of the same kind }

It\marginnote{3.2.4.1} is two women, he perceives them both as women, and he has lust. If the monk, referring to the anus or the vagina of both women, praises, disparages … or abuses, he commits two offenses entailing suspension. … (To be expanded as in \href{https://suttacentral.net/pli-tv-bu-vb-ss2\#3.2.4}{Bu Ss 2:3.2.4}–Bu Ss 2:3.2.26, with appropriate substitutions.) … 

\subparagraph*{Referring to the private parts of two beings of different kinds }

It\marginnote{3.2.8.1} is a woman and a \textit{\textsanskrit{paṇḍaka}}, but he perceives both as women, and he has lust. If the monk, referring to the anus or the vagina of both of them, praises, disparages … or abuses, he commits one offense entailing suspension and one offense of wrong conduct. … (To be expanded as in \href{https://suttacentral.net/pli-tv-bu-vb-ss2\#3.3.4}{Bu Ss 2:3.3.4}–Bu Ss 2:3.3.61, with appropriate substitutions.) … 

\subparagraph*{Referring to other parts of the body: below the collar bone and above the knees }

It\marginnote{3.3.1} is a woman, he perceives her as a woman, and he has lust. If the monk, referring to any part of the woman below the collar bone but above the knees, apart from the anus or the vagina, praises, disparages … or abuses, he commits a serious offense. … 

It\marginnote{3.3.4} is two women, he perceives them both as women, and he has lust. If the monk, referring to any part of both women below the collar bone but above the knees, apart from the anus or the vagina, praises, disparages … or abuses, he commits two serious offenses. … 

It\marginnote{3.3.7} is a woman and a \textit{\textsanskrit{paṇḍaka}}, but he perceives both as women, and he has lust. If the monk, referring to any part of both of them below the collar bone but above the knees, apart from the anus or the vagina, praises, disparages … or abuses, he commits one serious offense and one offense of wrong conduct. … 

\subparagraph*{Referring to other parts of the body: above the collar bone or below the knees }

It\marginnote{3.4.1} is a woman, he perceives her as a woman, and he has lust. If the monk, referring to any part of the woman above the collar bone or below the knees, praises, disparages … or abuses, he commits an offense of wrong conduct. … 

It\marginnote{3.4.4} is two women, he perceives them both as women, and he has lust. If the monk, referring to any part of both women above the collar bone or below the knees, praises, disparages … or abuses, he commits two offenses of wrong conduct. … 

It\marginnote{3.4.7} is a woman and a \textit{\textsanskrit{paṇḍaka}}, but he perceives both as women, and he has lust. If the monk, referring to any part of both of them above the collar bone or below the knees, praises, disparages … or abuses, he commits two offenses of wrong conduct. … 

\subparagraph*{Referring to anything connected to the body }

It\marginnote{3.5.1} is a woman, he perceives her as a woman, and he has lust. If the monk, referring to anything connected to the woman’s body, praises, disparages … or abuses, he commits an offense of wrong conduct. … 

It\marginnote{3.5.4} is two women, he perceives them both as women, and he has lust. If the monk, referring to anything connected to the body of both women, praises, disparages … or abuses, he commits two offenses of wrong conduct. … 

It\marginnote{3.5.7} is a woman and a \textit{\textsanskrit{paṇḍaka}}, but he perceives both as women, and he has lust. If the monk, referring to anything connected to the body of both of them, praises, disparages … or abuses, he commits two offenses of wrong conduct. … 

\subsection*{Non-offenses }

There\marginnote{3.6.1} is no offense: if he is aiming at something beneficial; if he is aiming at giving a teaching; if he is aiming at giving an instruction; if he is insane; if he is the first offender. 

\scuddanaintro{Summary verses of case studies }

\begin{scuddana}%
“Red,\marginnote{3.6.8} coarse, thick, \\
Rough, long, sown; \\
I hope the path has an end, \\
Faith, with a gift, with work.” 

%
\end{scuddana}

\subsubsection*{Case studies }

At\marginnote{4.1.1} one time a certain woman was wearing a newly dyed woolen cloak. A lustful monk said to her, “Sister, is that red thing yours?” She did not understand and said, “Yes, venerable, it’s a newly dyed woolen cloak.” He became anxious and thought, “The Buddha has laid down a training rule. Could it be that I’ve committed an offense entailing suspension?” He told the Buddha, who said, “There’s no offense entailing suspension, but there’s an offense of wrong conduct.” 

At\marginnote{4.2.1} one time a certain woman was wearing a coarse woolen cloak. A lustful monk said to her, “Sister, is that coarse hair yours?” She did not understand and said, “Yes, venerable, it’s a coarse woolen cloak.” He became anxious … “There’s no offense entailing suspension, but there’s an offense of wrong conduct.” 

At\marginnote{4.3.1} one time a certain woman was wearing a newly woven woolen cloak. A lustful monk said to her, “Sister, is that thick hair yours?” She did not understand and said, “Yes, venerable, it’s a newly woven woolen cloak.” He became anxious … “There’s no offense entailing suspension, but there’s an offense of wrong conduct.” 

At\marginnote{4.4.1} one time a certain woman was wearing a rough woolen cloak. A lustful monk said to her, “Sister, is that rough hair yours?” She did not understand and said, “Yes, venerable, it’s a rough woolen cloak.” He became anxious … “There’s no offense entailing suspension, but there’s an offense of wrong conduct.” 

At\marginnote{4.5.1} one time a certain woman was wearing a fleecy upper robe. A lustful monk said to her, “Sister, is that long hair yours?” She did not understand and said, “Yes, venerable, it’s a fleecy upper robe.” He became anxious … “There’s no offense entailing suspension, but there’s an offense of wrong conduct.” 

At\marginnote{4.6.1} one time a certain woman was returning after having had a field sown. A lustful monk said to her, “Have you sown, Sister?” She did not understand and said, “Yes, venerable, but the furrow isn’t closed yet.” He became anxious … “There’s no offense entailing suspension, but there’s an offense of wrong conduct.” 

At\marginnote{4.7.1} one time a lustful monk saw a female wanderer coming in the opposite direction. He said to her, “Sister, I hope the path has an end?” She did not understand and said, “Yes, just keep going.” He became anxious … “There’s no offense entailing suspension, but there’s a serious offense.” 

At\marginnote{4.8.1} one time a lustful monk said to a woman, “You have faith, Sister, yet you don’t give us what you give your husband.” 

“What’s\marginnote{4.8.4} that, venerable?” 

“Sexual\marginnote{4.8.5} intercourse.” He became anxious … “There’s an offense entailing suspension.” 

At\marginnote{4.9.1} one time a lustful monk said to a woman, “You have faith, Sister, yet you don’t give us the highest gift.” 

“What’s\marginnote{4.9.4} that, venerable?” 

“Sexual\marginnote{4.9.5} intercourse.” He became anxious … “There’s an offense entailing suspension.” 

At\marginnote{4.10.1} one time a certain woman was doing some work. A lustful monk said to her, “Stand, Sister, I’ll work.” … “Sit, Sister, I’ll work.” … “Lie down, Sister, I’ll work.” She did not understand. He became anxious … “There’s no offense entailing suspension, but there’s an offense of wrong conduct.” 

\scendsutta{The training rule on indecent speech, the third, is finished. }

%
\section*{{\suttatitleacronym Bu Ss 4}{\suttatitletranslation 4. The training rule on satisfying one’s own desires }{\suttatitleroot Attakāmpāricariya}}
\addcontentsline{toc}{section}{\tocacronym{Bu Ss 4} \toctranslation{4. The training rule on satisfying one’s own desires } \tocroot{Attakāmpāricariya}}
\markboth{4. The training rule on satisfying one’s own desires }{Attakāmpāricariya}
\extramarks{Bu Ss 4}{Bu Ss 4}

\subsection*{Origin story }

At\marginnote{1.1.1} one time the Buddha was staying at \textsanskrit{Sāvatthī} in the Jeta Grove, \textsanskrit{Anāthapiṇḍika}’s Monastery. At that time Venerable \textsanskrit{Udāyī} was associating with and visiting a number of families in \textsanskrit{Sāvatthī}. One morning \textsanskrit{Udāyī} robed up, took his bowl and robe, and went to the house of a beautiful widow where he sat down on the prepared seat. She then approached \textsanskrit{Udāyī}, bowed, and sat down. And \textsanskrit{Udāyī} instructed, inspired, and gladdened her with a teaching. She then said, “Venerable, please tell me what you need. I can give robe-cloth, almsfood, a dwelling, and medicinal supplies.”\footnote{“Robe-cloth” renders \textit{\textsanskrit{cīvara}}. See Appendix of Technical Terms for discussion. } 

“It’s\marginnote{1.1.10} not hard for us to get those requisites. Give instead what’s hard for us to get.” 

“What’s\marginnote{1.1.12} that, venerable?” 

“Sexual\marginnote{1.1.13} intercourse.” 

“Is\marginnote{1.1.14} it wanted now?” 

“Yes.”\marginnote{1.1.15} 

Saying,\marginnote{1.1.16} “Come,” she entered her bedroom, took off her wrap, and lay back on the bed. \textsanskrit{Udāyī} went up to her and spat out, “Who would touch this stinking wretch!” and he left. 

That\marginnote{1.1.21} woman then complained and criticized him, “These Sakyan monastics are shameless and immoral liars. They claim to have integrity, to be celibate and of good conduct, to be truthful, moral, and good. But they don’t have the good character of a monastic or  brahmin. They’ve lost the plot! How could the ascetic \textsanskrit{Udāyī} ask for sexual intercourse from me, but then spit out, ‘Who would touch this stinking wretch!’ and leave? What’s bad about me? How am I stinking? How am I inferior to anyone?” 

Other\marginnote{1.2.1} women, too, criticized him in the same way. 

The\marginnote{1.2.5} monks heard the criticism of those women, and the monks of few desires complained and criticized him, “How could Venerable \textsanskrit{Udāyī} encourage a woman to satisfy his own desires?” 

Those\marginnote{1.2.8} monks rebuked \textsanskrit{Udāyī} in many ways and then told the Buddha. The Buddha had the Sangha gathered and questioned \textsanskrit{Udāyī}: “Is it true, \textsanskrit{Udāyī}, that you did this?” 

“It’s\marginnote{1.2.11} true, sir.” 

The\marginnote{1.2.12} Buddha rebuked him, “It’s not suitable, foolish man, it’s not proper, it’s not worthy of a monastic, it’s not allowable, it’s not to be done. How could you do such a thing? Haven’t I given many teachings for the sake of dispassion, not for the sake of passion … the stilling of the fevers of worldly pleasures? This will affect people’s confidence …” … “And, monks, this training rule should be recited like this: 

\subsection*{Final ruling }

\scrule{‘If a monk, overcome by lust and with a distorted mind, encourages a woman to satisfy his own desires, saying, “Sister, she provides the highest service who in this way satisfies one like me, who is virtuous, celibate, and of good character,” and if it is a reference to sexual intercourse, he commits an offense entailing suspension.’” }

\subsection*{Definitions }

\begin{description}%
\item[A: ] whoever … %
\item[Monk: ] … The monk who has been given the full ordination by a unanimous Sangha through a legal procedure consisting of one motion and three announcements that is irreversible and fit to stand—this sort of monk is meant in this case. %
\item[Overcome by lust: ] having lust, longing for, in love with. %
\item[Distorted: ] a lustful mind is distorted; an angry mind is distorted; a confused mind is distorted. But in this case “distorted” refers to the lustful mind. %
\item[A woman: ] a female human being, not a female spirit, not a female ghost, not a female animal. She understands and is capable of discerning bad speech and good speech, what is decent and what is indecent. %
\item[A woman:\footnote{Because I have rendered \textit{\textsanskrit{mātugāmassa} santike} (literally “in the presence of a woman”) simply as “woman”, the English rendering of the word to be defined is the same here as in the previous definition. } ] near a woman, close to a woman. %
\item[His own desires: ] his own desires, for the sake of himself, aiming at himself, satisfying himself. %
\item[The highest: ] this is the highest, this is the best, this is the foremost, this is the utmost, this is the most excellent. %
\item[She: ] an aristocrat woman, a brahmin woman, a merchant woman, or a woman worker. %
\item[One like me: ] a male aristocrat, a male brahmin, a male merchant, or a male worker. %
\item[Who is virtuous: ] one who refrains from killing living beings, who refrains from stealing, who refrains from lying. %
\item[Celibate: ] one who refrains from sexual intercourse. %
\item[Of good character: ] he is one of good character because of that virtue and because of being celibate. %
\item[In this way: ] with sexual intercourse. %
\item[Satisfies: ] gives pleasure to. %
\item[If it is a reference to sexual intercourse: ] if it is connected with the sexual act. %
\item[He commits an offense entailing suspension: ] … Therefore, too, it is called “an offense entailing suspension”. %
\end{description}

\subsection*{Permutations }

\subsubsection*{Encouraging sexual intercourse to a single person or animal }

It\marginnote{3.1.1} is a woman, he perceives her as a woman, and he has lust. If the monk encourages the woman to satisfy his own desires, he commits an offense entailing suspension. 

It\marginnote{3.1.3} is a woman, but he is unsure of it … It is a woman, but he perceives her as a \textit{\textsanskrit{paṇḍaka}} … It is a woman, but he perceives her as a man … It is a woman, but he perceives her as an animal, and he has lust. If the monk encourages the woman to satisfy his own desires, he commits a serious offense. 

It\marginnote{3.1.8} is a \textit{\textsanskrit{paṇḍaka}}, he perceives him as a \textit{\textsanskrit{paṇḍaka}}, and he has lust. If the monk encourages the \textit{\textsanskrit{paṇḍaka}} to satisfy his own desires, he commits a serious offense. 

It\marginnote{3.1.10} is a \textit{\textsanskrit{paṇḍaka}}, but he is unsure of it … It is a \textit{\textsanskrit{paṇḍaka}}, but he perceives him as a man … It is a \textit{\textsanskrit{paṇḍaka}}, but he perceives him as an animal … It is a \textit{\textsanskrit{paṇḍaka}}, but he perceives him as a woman, and he has lust. If the monk encourages the \textit{\textsanskrit{paṇḍaka}} to satisfy his own desires, he commits an offense of wrong conduct. 

It\marginnote{3.1.15} is a man … … (To be expanded as above with appropriate adjustments.) …  It is an animal, he perceives it as an animal … It is an animal, but he is unsure of it … It is an animal, but he perceives it as a woman … It is an animal, but he perceives it as a \textit{\textsanskrit{paṇḍaka}} … It is an animal, but he perceives it as a man, and he has lust. If the monk encourages the animal to satisfy his own desires, he commits an offense of wrong conduct. 

\subsubsection*{Encouraging sexual intercourse to two beings of the same kind }

It\marginnote{3.1.23.1} is two women, he perceives them both as women, and he has lust. If the monk encourages both women to satisfy his own desires, he commits two offenses entailing suspension. (To be expanded as in \href{https://suttacentral.net/pli-tv-bu-vb-ss2\#3.2.4}{Bu Ss 2:3.2.4}–Bu Ss 2:3.2.26, with appropriate substitutions.) 

\subsubsection*{Encouraging sexual intercourse to two beings of different kinds }

It\marginnote{3.1.26.1} is a woman and a \textit{\textsanskrit{paṇḍaka}}, but he perceives both as women, and he has lust. If the monk encourages both of them to satisfy his own desires, he commits one offense entailing suspension and one offense of wrong conduct. (To be expanded as in \href{https://suttacentral.net/pli-tv-bu-vb-ss2\#3.3.4}{Bu Ss 2:3.3.4}–Bu Ss 2:3.3.61, with appropriate substitutions.) 

\subsection*{Non-offenses }

There\marginnote{3.2.1} is no offense: if he says, “Support us with robe-cloth, almsfood, dwellings, and medicinal supplies;” if he is insane; if he is the first offender. 

\scuddanaintro{Summary verses of case studies }

\begin{scuddana}%
“How\marginnote{3.2.6} a barren woman, may she have a child, \\
And dear to, may I be fortunate; \\
What may I give, how may I support, \\
How may I get a good rebirth.” 

%
\end{scuddana}

\subsubsection*{Case studies }

On\marginnote{4.1.1} one occasion a barren woman said to a monk who associated with her family, “Sir, how can I get pregnant?” 

“Well,\marginnote{4.1.3} Sister, give the highest gift.” 

“What’s\marginnote{4.1.4} that?” 

“Sexual\marginnote{4.1.5} intercourse.” He became anxious … “You’ve committed an offense entailing suspension.” 

On\marginnote{4.2.1} one occasion a fertile woman said to a monk who associated with her family, “Sir, how can I get a child?” 

“Well,\marginnote{4.2.3} Sister, give the highest gift.” 

“What’s\marginnote{4.2.4} that?” 

“Sexual\marginnote{4.2.5} intercourse.” He became anxious … “You’ve committed an offense entailing suspension.” 

On\marginnote{4.3.1} one occasion a woman said to a monk who associated with her family, “Sir, how can I get my husband to love me?” 

“Well,\marginnote{4.3.3} Sister, give the highest gift.” 

“What’s\marginnote{4.3.4} that?” 

“Sexual\marginnote{4.3.5} intercourse.” He became anxious … “You’ve committed an offense entailing suspension.” 

On\marginnote{4.3.8} one occasion a woman said to a monk who associated with her family, “Sir, how can I become more fortunate?” 

“Well,\marginnote{4.3.10} Sister, give the highest gift.” 

“What’s\marginnote{4.3.11} that?” 

“Sexual\marginnote{4.3.12} intercourse.” He became anxious … “You’ve committed an offense entailing suspension.” 

On\marginnote{4.4.1} one occasion a woman said to a monk who associated with her family, “Sir, what may I give you?” 

“The\marginnote{4.4.3} highest gift, Sister.” 

“What’s\marginnote{4.4.4} that?” 

“Sexual\marginnote{4.4.5} intercourse.” He became anxious … “You’ve committed an offense entailing suspension.” 

On\marginnote{4.5.1} one occasion a woman said to a monk who associated with her family, “Sir, how may I support you?” 

“With\marginnote{4.5.3} the highest gift, Sister.” 

“What’s\marginnote{4.5.4} that?” 

“Sexual\marginnote{4.5.5} intercourse.” He became anxious … “You’ve committed an offense entailing suspension.” 

On\marginnote{4.6.1} one occasion a woman said to a monk who associated with her family, “Sir, how can I get a good rebirth?” 

“Well,\marginnote{4.6.3} Sister, give the highest gift.” 

“What’s\marginnote{4.6.4} that?” 

“Sexual\marginnote{4.6.5} intercourse.” He became anxious … “You’ve committed an offense entailing suspension.” 

\scendsutta{The training rule on satisfying one’s own desires, the fourth, is finished. }

%
\section*{{\suttatitleacronym Bu Ss 5}{\suttatitletranslation 5. The training rule on matchmaking }{\suttatitleroot Sañcaritta}}
\addcontentsline{toc}{section}{\tocacronym{Bu Ss 5} \toctranslation{5. The training rule on matchmaking } \tocroot{Sañcaritta}}
\markboth{5. The training rule on matchmaking }{Sañcaritta}
\extramarks{Bu Ss 5}{Bu Ss 5}

\subsection*{Origin story }

\subsubsection*{First sub-story }

At\marginnote{1.1.1} one time when the Buddha was staying at \textsanskrit{Sāvatthī} in \textsanskrit{Anāthapiṇḍika}’s Monastery, Venerable \textsanskrit{Udāyī} was associating with and visiting a number of families in \textsanskrit{Sāvatthī}. 

When\marginnote{1.1.3} \textsanskrit{Udāyī} saw a young man without a wife and a young woman without a husband, he would praise the young woman to the parents of the young man,\footnote{The Pali reads \textit{yattha passati \textsanskrit{kumārakaṁ} \textsanskrit{vā} \textsanskrit{apajāpatikaṁ} \textsanskrit{kumārikaṁ} \textsanskrit{vā} \textsanskrit{apatikaṁ} \textsanskrit{kumārakassa}}, literally, “When \textsanskrit{Udāyi} saw a young man without a wife or a young woman without a husband”. However, since the “or” gives an awkward meaning in English, I have replaced it with “and”. } “The young woman of such-and-such a family is beautiful, intelligent, skilled, and diligent. She’s suitable for your son.” They would reply, “They don’t know who we are, venerable. If you could persuade them to give the girl, we would take her for our son.” 

And\marginnote{1.1.9} he praised the young man to the parents of the young woman, “The young man of such-and-such a family is handsome, intelligent, skilled, and diligent. He’s suitable for your daughter.” They would reply, “They don’t know who we are, venerable, and we would be ashamed to speak to them for the sake of the girl. But if you could persuade them to ask us, we would give our girl to the young man.” In this way he arranged for the taking of a bride, for the giving of a bride, and for marriage. 

At\marginnote{1.2.1} that time there was a former courtesan who had a beautiful daughter. On one occasion some lay followers of the \textsanskrit{Ājīvaka} religion came from another village and said to that courtesan, “Ma’am, please give your girl to our boy.” 

“I\marginnote{1.2.4} don’t know who you are, sirs, and I won’t give my only daughter to be taken to another village.” 

People\marginnote{1.2.6} asked those \textsanskrit{Ājīvaka} lay followers why they had come. They replied, “We came to ask that courtesan to give her daughter to our son, but she refused.” 

“But\marginnote{1.2.11} why did you ask the courtesan? You should speak to Venerable \textsanskrit{Udāyī}. He’ll persuade her.” 

They\marginnote{1.2.14} then went to \textsanskrit{Udāyī} and said, “Venerable, we’ve asked that courtesan here to give her daughter to our son, but she refused. Would you please persuade her to give her daughter?” 

\textsanskrit{Udāyī}\marginnote{1.2.19} agreed. Soon afterwards he went to that courtesan and said, “Why didn’t you give them your daughter?” 

“I\marginnote{1.2.21} don’t know who they are, sir, and I won’t give my only daughter to be taken to another village.” 

“Please\marginnote{1.2.23} give her to them. I know them.” 

“If\marginnote{1.2.25} you know them, I’ll give her away.” 

She\marginnote{1.3.1} then gave her daughter to those \textsanskrit{Ājīvaka} followers, and they took her away. For a month they treated her like a daughter-in-law, but then like a slave. 

The\marginnote{1.3.4} girl sent a message to her mother, saying, “Mom, I’m unhappy and miserable. For a month they treated me like a daughter-in-law, but then like a slave. Come, mom, and take me home.” 

Soon\marginnote{1.3.9} afterwards the courtesan went to those \textsanskrit{Ājīvakas} and said, “Please don’t treat my girl like a slave; treat her properly!” 

They\marginnote{1.3.12} replied, “We deal with the monastic, not with you. Go away! We don’t want anything to do with you.”\footnote{Literally, “We don’t know you.” } Being dismissed, she returned to \textsanskrit{Sāvatthī}. 

A\marginnote{1.3.17} second time the girl sent the same message to her mother. That courtesan then went to \textsanskrit{Udāyī} and said, “Venerable, my girl is unhappy and miserable. They treated her like a daughter-in-law for a month, but then like a slave. Please tell them to treat her properly.” 

\textsanskrit{Udāyī}\marginnote{1.3.29} went to the \textsanskrit{Ājīvakas} and said, “Please don’t treat this girl like a slave; treat her properly.” 

They\marginnote{1.3.32} replied, “We deal with the courtesan, not with you. A monastic shouldn’t get involved. You should behave like a proper monastic. So go away! We don’t want anything to do with you.” Being dismissed, he returned to \textsanskrit{Sāvatthī}. 

A\marginnote{1.3.39} third time that girl sent the same message to her mother, and a second time the courtesan went to \textsanskrit{Udāyī} and told him the same thing. 

He\marginnote{1.3.50} replied, “When I first went, they just dismissed me. Go yourself; I’m not going.” 

Then\marginnote{1.4.1} that courtesan complained and criticized him, “May Venerable \textsanskrit{Udāyī} be miserable and unhappy, just as my girl is miserable and unhappy because of her nasty mother-in-law, father-in-law, and husband.” 

The\marginnote{1.4.3} girl, too, complained and criticized him in the same way, 

as\marginnote{1.4.5} did other women who were unhappy with their mothers-in-law, fathers-in-law, and husbands. 

But\marginnote{1.4.7} those women who were happy with their in-laws wished him well, saying, “May Venerable \textsanskrit{Udāyī} be happy and well, just as we are happy and well because of our good mothers-in-law, fathers-in-law, and husbands.” 

The\marginnote{1.5.1} monks heard that some women were criticizing him whereas others were wishing him well. And the monks of few desires complained and criticized him, “How can Venerable \textsanskrit{Udāyī} act as a matchmaker?” 

They\marginnote{1.5.4} told the Buddha. The Buddha had the Sangha gathered and questioned \textsanskrit{Udāyī}: “Is it true, \textsanskrit{Udāyī}, that you do this?” 

“It’s\marginnote{1.5.7} true, sir.” 

The\marginnote{1.5.8} Buddha rebuked him … “Foolish man, how can you do this? This will affect people’s confidence …” … “And, monks, this training rule should be recited like this: 

\subsubsection*{Preliminary ruling }

\scrule{‘If a monk acts as a matchmaker, conveying a man’s intention to a woman or a woman’s intention to a man, for marriage or for an affair, he commits an offense entailing suspension.’” }

In\marginnote{1.5.13} this way the Buddha laid down this training rule for the monks. 

\subsubsection*{Second sub-story }

Soon\marginnote{2.1.1} afterwards a number of scoundrels who were enjoying themselves in a park sent a messenger to a sex worker, saying, “Please come, let’s enjoy ourselves in the park.” 

She\marginnote{2.1.3} replied, “Sirs, I don’t know who you are. I’m wealthy. I don’t want to go outside the city.” 

The\marginnote{2.1.7} messenger returned the message. A certain man then said to those men, “Why did you ask the sex worker? You should speak to Venerable \textsanskrit{Udāyī}. He’ll persuade her.” 

But\marginnote{2.1.12} a certain Buddhist lay follower said, “No way. That’s not allowable for the Sakyan monastics. He won’t do it.” And they made a bet on whether he would. 

Those\marginnote{2.1.17} scoundrels then went to \textsanskrit{Udāyī} and said, “Venerable, while we were enjoying ourselves in the park, we sent a message to such-and-such a sex worker, asking her to come, but she refused. Would you please persuade her?” 

\textsanskrit{Udāyī}\marginnote{2.1.24} agreed. He then went to that sex worker and said, “Why don’t you go to those men?” 

She\marginnote{2.1.26} told him why. 

“Please\marginnote{2.1.29} go. I know them.” 

“If\marginnote{2.1.30} you know them, sir, I’ll go.” And those men took her to the park. 

Then\marginnote{2.2.1} that lay follower complained and criticized him, “How could Venerable \textsanskrit{Udāyī} act as a matchmaker for a brief affair?” 

The\marginnote{2.2.3} monks heard it, and the monks of few desires complained and criticized him, “How could Venerable \textsanskrit{Udāyī} act as a matchmaker for a brief affair?” 

They\marginnote{2.2.6} rebuked \textsanskrit{Udāyī} in many ways and then told the Buddha. Soon afterwards he had the Sangha gathered and questioned \textsanskrit{Udāyī}: “Is it true, \textsanskrit{Udāyī}, that you did this?” 

“It’s\marginnote{2.2.8} true, sir.” 

The\marginnote{2.2.9} Buddha rebuked him … “Foolish man, how could you do this? This will affect people’s confidence …” … “And so, monks, this training rule should be recited like this: 

\subsection*{Final ruling }

\scrule{‘If a monk acts as a matchmaker, conveying a man’s intention to a woman or a woman’s intention to a man, for marriage or for an affair, even if just a brief one, he commits an offense entailing suspension.’” }

\subsection*{Definitions }

\begin{description}%
\item[A: ] whoever … %
\item[Monk: ] … The monk who has been given the full ordination by a unanimous Sangha through a legal procedure consisting of one motion and three announcements that is irreversible and fit to stand—this sort of monk is meant in this case. %
\item[Acts as a matchmaker, conveying: ] sent by a woman he goes to a man, or sent by a man he goes to a woman. %
\item[A man’s intention to a woman: ] he informs a woman of a man’s intention. %
\item[A woman’s intention to a man: ] he informs a man of a woman’s intention. %
\item[For marriage: ] “You should be his wife.” %
\item[For an affair: ] “You should be his mistress.” %
\item[Even if just a brief one: ] “You will have a short relationship.” %
\item[He commits an offense entailing suspension: ] … Therefore, too, it is called “an offense entailing suspension”. %
\end{description}

\subsection*{Permutations }

\subsubsection*{Permutations part 1 }

\paragraph*{Summary }

There\marginnote{4.1.1} are ten kinds of women: the one protected by her mother, the one protected by her father, the one protected by her parents, the one protected by her brother, the one protected by her sister, the one protected by her relatives, the one protected by her family, the one protected by her religion, the one otherwise protected, the one protected by the threat of punishment. 

There\marginnote{4.1.2} are ten kinds of wives: the bought wife, the wife by choice, the wife through property, the wife through clothes, the wife through the bowl-of-water ritual, the wife through removing the head pad, the slave wife, the servant wife, the captured wife, the momentary wife. 

\paragraph*{Definitions }

\begin{description}%
\item[The one protected by her mother: ] her mother protects, guards, wields authority, controls. %
\item[The one protected by her father: ] her father protects, guards, wields authority, controls. %
\item[The one protected by her parents: ] her parents protect, guard, wield authority, control. %
\item[The one protected by her brother: ] her brother protects, guards, wields authority, controls. %
\item[The one protected by her sister: ] her sister protects, guards, wields authority, controls. %
\item[The one protected by her relatives: ] her relatives protect, guard, wield authority, control. %
\item[The one protected by her family: ] her family protects, guards, wields authority, controls. %
\item[The one protected by her religion: ] her fellow believers protect, guard, wield authority, control. %
\item[The one otherwise protected:\footnote{Sp 1.303: \textit{Saha \textsanskrit{ārakkhenāti} \textsanskrit{sārakkhā}}, “\textit{\textsanskrit{Sārakkha}} means having protection.” Sp-\textsanskrit{ṭ} 1.303: \textit{\textsanskrit{Sasāmikā} \textsanskrit{sārakkhā}}; “\textit{\textsanskrit{Sārakkhā}} means having a husband.” But it clearly does not mean marriage in a modern sense. } ] even in the womb someone takes possession of her, thinking, “She is mine,” and so too for one engaged to be married. %
\item[The one protected by the threat of punishment: ] those who punish will punish anyone going to her with a fixed punishment. %
\item[The bought wife: ] after buying her with money, they live together. %
\item[The wife by choice: ] being dear to each other, they live together. %
\item[The wife through property: ] after giving property, they live together. %
\item[The wife through clothes: ] after giving clothes, they live together. %
\item[The wife through the bowl-of-water ritual: ] after touching a bowl of water, they live together.\footnote{Sp 1.303: \textit{\textsanskrit{Odapattakinīti} \textsanskrit{ubhinnaṁ} \textsanskrit{ekissā} \textsanskrit{udakapātiyā} hatthe \textsanskrit{otāretvā} “\textsanskrit{idaṁ} \textsanskrit{udakaṁ} viya \textsanskrit{saṁsaṭṭhā} \textsanskrit{abhejjā} \textsanskrit{hothā}”ti \textsanskrit{vatvā} \textsanskrit{pariggahitāya} \textsanskrit{vohāranāmametaṁ}, niddesepissa “\textsanskrit{tāya} saha \textsanskrit{udakapattaṁ} \textsanskrit{āmasitvā} \textsanskrit{taṁ} \textsanskrit{vāsetī}”ti evamattho veditabbo}; “\textit{\textsanskrit{Odapattakinī}}: this is an expression for both having entered their hand into a single bowl of water, saying, for the purpose of taking possession, ʻMay you not be split but be together like this water.’ Also, it may be specified like this: ʻAfter touching the bowl of water with you, he lives with you.’ The meaning is to be understood in this way.” } %
\item[The wife through removing the head pad: ] after removing the head pad, they live together.\footnote{Sp 1.303: \textit{\textsanskrit{Obhaṭaṁ} \textsanskrit{oropitaṁ} \textsanskrit{cumbaṭamassāti} \textsanskrit{obhaṭacumbaṭā}, \textsanskrit{kaṭṭhahārikādīnaṁ} \textsanskrit{aññatarā}, \textsanskrit{yassā} \textsanskrit{sīsato} \textsanskrit{cumbaṭaṁ} \textsanskrit{oropetvā} ghare \textsanskrit{vāseti}, \textsanskrit{tassā} \textsanskrit{etaṁ} \textsanskrit{adhivacanaṁ}}; “\textit{\textsanskrit{Obhaṭacumbaṭā}}: her head pad, which is for carrying sticks, etc., has been removed. This is a term for one who is made to live in a house (with another), after removing the cloth pad from her head.” } %
\item[The slave wife: ] she is a slave and a wife. %
\item[The servant wife: ] she is a servant and a wife. %
\item[The captured wife: ] one brought back as a captive is what is meant. %
\item[The momentary wife: ] a wife for one occasion is what is meant. %
\end{description}

\paragraph*{Exposition }

\subparagraph*{Acting as a matchmaker for a man and a single bought wife }

A\marginnote{4.4.1} man sends a monk, saying, “Sir, go to so-and-so protected by her mother and say, ‘Please be the bought wife of so-and-so.’” If he accepts the mission, finds out the response, and reports back, he commits an offense entailing suspension. 

A\marginnote{4.4.5} man sends a monk, saying, “Sir, go to so-and-so protected by her father and say … so-and-so protected by her parents and say … so-and-so protected by her brother and say … so-and-so protected by her sister and say … so-and-so protected by her relatives and say … so-and-so protected by her family and say … so-and-so protected by her religion and say … so-and-so otherwise protected and say … so-and-so protected by the threat of punishment and say, ‘Please be the bought wife of so-and-so.’” If he accepts the mission, finds out the response, and reports back, he commits an offense entailing suspension. 

\scend{The setting out of the steps is finished. }

\subparagraph*{Acting as a matchmaker for a man and two bought wives }

A\marginnote{4.4.18.1} man sends a monk, saying, “Sir, go to so-and-so protected by her mother and so-and-so protected by her father and say,\footnote{At first glance the Pali seems to concern a single wife who is protected in two ways, not two separate wives. However, since the immediately following question is phrased in the plural, \textit{hotha … \textsanskrit{bhariyāyo} \textsanskrit{dhanakkītā}}, “please be the bought wives,” it is clear that this is about two wives. To bring this out in translation, I have added “so-and-so” to each potential wife. } ‘Please be the bought wives of so-and-so.’” If he accepts the mission, finds out the response, and reports back, he commits an offense entailing suspension. 

A\marginnote{4.4.22} man sends a monk, saying, “Sir, go to so-and-so protected by her mother and so-and-so protected by her parents … so-and-so protected by her mother and so-and-so protected by her brother … so-and-so protected by her mother and so-and-so protected by her sister … so-and-so protected by her mother and so-and-so protected by her relatives … so-and-so protected by her mother and so-and-so protected by her family … so-and-so protected by her mother and so-and-so protected by her religion … so-and-so protected by her mother and so-and-so otherwise protected … so-and-so protected by her mother and so-and-so protected by the threat of punishment and say, ‘Please be the bought wives of so-and-so.’” If he accepts the mission, finds out the response, and reports back, he commits an offense entailing suspension. 

\scend{The unconnected permutation series is finished.\footnote{For an explanation of these sectional summaries, see Appendix of Specialized Vocabulary under \textit{\textsanskrit{khaṇḍa}} and \textit{cakka}. } }

A\marginnote{4.4.34} man sends a monk, saying, “Sir, go to so-and-so protected by her father and so-and-so protected by her parents and say, ‘Please be the bought wives of so-and-so.’” If he accepts the mission, finds out the response, and reports back, he commits an offense entailing suspension. 

A\marginnote{4.4.38} man sends a monk, saying, “Sir, go to so-and-so protected by her father and so-and-so protected by her brother … so-and-so protected by her father and so-and-so protected by her sister … so-and-so protected by her father and so-and-so protected by her relatives … so-and-so protected by her father and so-and-so protected by her family … so-and-so protected by her father and so-and-so protected by her religion … so-and-so protected by her father and so-and-so otherwise protected … so-and-so protected by her father and so-and-so protected by the threat of punishment and say, ‘Please be the bought wives of so-and-so.’” If he accepts the mission, finds out the response, and reports back, he commits an offense entailing suspension. 

A\marginnote{4.4.48} man sends a monk, saying, “Sir, go to so-and-so protected by her father and so-and-so protected by her mother and say, ‘Please be the bought wives of so-and-so.’” If he accepts the mission, finds out the response, and reports back, he commits an offense entailing suspension. 

\scend{The linked permutation series with the basis in brief is finished.\footnote{See Appendix of Specialized Vocabulary under \textit{baddha}, \textit{cakka}, \textit{\textsanskrit{mūla}} and, \textit{\textsanskrit{saṅkhitta}}. } }

…\marginnote{4.4.53} A man sends a monk, saying, “Sir, go to so-and-so protected by the threat of punishment and so-and-so protected by her mother and say, ‘Please be the bought wives of so-and-so.’” If he accepts the mission, finds out the response, and reports back, he commits an offense entailing suspension. 

A\marginnote{4.4.57} man sends a monk, saying, “Sir, go to so-and-so protected by the threat of punishment and so-and-so protected by her father … so-and-so protected by the threat of punishment and so-and-so protected by her parents … so-and-so protected by the threat of punishment and so-and-so protected by her brother … so-and-so protected by the threat of punishment and so-and-so protected by her sister … so-and-so protected by the threat of punishment and so-and-so protected by her relatives … so-and-so protected by the threat of punishment and so-and-so protected by her family … so-and-so protected by the threat of punishment and so-and-so protected by her religion … so-and-so protected by the threat of punishment and so-and-so otherwise protected and say, ‘Please be the bought wives of so-and-so.’” If he accepts the mission, finds out the response, and reports back, he commits an offense entailing suspension. 

\scend{The section based on one item is finished.\footnote{See Appendix of Specialized Vocabulary under \textit{\textsanskrit{mūlaka}}. } }

\subparagraph*{Acting as a matchmaker for a man and three to nine bought wives }

The\marginnote{4.4.69.1} sections based on two items, three items, up to nine items, are to be done in the same way. 

\subparagraph*{Acting as a matchmaker for a man and ten bought wives }

This\marginnote{4.4.70.1} is the section based on ten items: 

A\marginnote{4.4.71} man sends a monk, saying, “Sir, go to so-and-so protected by her mother and so-and-so protected by her father and so-and-so protected by her parents and so-and-so protected by her brother and so-and-so protected by her sister and so-and-so protected by her relatives and so-and-so protected by her family and so-and-so protected by her religion and so-and-so otherwise protected and so-and-so protected by the threat of punishment and say, ‘Please be the bought wives of so-and-so.’” If he accepts the mission, finds out the response, and reports back, he commits an offense entailing suspension. 

\scend{The permutation series on bought wives is finished. }

\subparagraph*{Acting as a matchmaker for a man and other kinds of wives }

A\marginnote{4.5.1} man sends a monk, saying, “Sir, go to so-and-so protected by her mother and say, ‘Please be the wife by choice of so-and-so.’” … the wife through property of so-and-so.’” … the wife through clothes of so-and-so.’” … the wife through the bowl-of-water ritual of so-and-so.’” … the wife through removing the head pad of so-and-so.’” … the slave wife of so-and-so.’” … the servant wife of so-and-so.’” … the captured wife of so-and-so.’” … the momentary wife of so-and-so.’” If he accepts the mission, finds out the response, and reports back, he commits an offense entailing suspension. 

\subparagraph*{Acting as a matchmaker for a man and a single momentary wife }

A\marginnote{4.5.13.1} man sends a monk, saying, “Sir, go to so-and-so protected by her father and say … so-and-so protected by her parents … so-and-so protected by her brother … so-and-so protected by her sister … so-and-so protected by her relatives … so-and-so protected by her family … so-and-so protected by her religion … so-and-so otherwise protected … so-and-so protected by the threat of punishment and say, ‘Please be the momentary wife of so-and-so.’” If he accepts the mission, finds out the response, and reports back, he commits an offense entailing suspension. 

\scend{The setting out of the steps is finished. }

\subparagraph*{Acting as a matchmaker for a man and two momentary wives }

A\marginnote{4.5.26.1} man sends a monk, saying, “Sir, go to so-and-so protected by her mother and so-and-so protected by her father and say, ‘Please be the momentary wives of so-and-so.’” If he accepts the mission, finds out the response, and reports back, he commits an offense entailing suspension. 

A\marginnote{4.5.30} man sends a monk, saying, “Sir, go to so-and-so protected by her mother and so-and-so protected by her parents … so-and-so protected by her mother and so-and-so protected by the threat of punishment and say, ‘Please be the momentary wives of so-and-so.’” If he accepts the mission, finds out the response, and reports back, he commits an offense entailing suspension. 

\scend{The unconnected permutation series is finished. }

A\marginnote{4.5.36} man sends a monk, saying, “Sir, go to so-and-so protected by her father and so-and-so protected by her parents and say, ‘Please be the momentary wives of so-and-so.’” If he accepts the mission, finds out the response, and reports back, he commits an offense entailing suspension. 

A\marginnote{4.5.40} man sends a monk, saying, “Sir, go to so-and-so protected by her father and so-and-so protected by her brother … so-and-so protected by her father and so-and-so protected by the threat of punishment and say, ‘Please be the momentary wives of so-and-so.’” If he accepts the mission, finds out the response, and reports back, he commits an offense entailing suspension. 

A\marginnote{4.5.45} man sends a monk, saying, “Sir, go to so-and-so protected by her father and so-and-so protected by her mother and say, ‘Please be the momentary wives of so-and-so.’” If he accepts the mission, finds out the response, and reports back, he commits an offense entailing suspension. 

\scend{The linked permutation series with the basis in brief is finished. }

A\marginnote{4.5.50} man sends a monk, saying, “Sir, go to so-and-so protected by the threat of punishment and so-and-so protected by her mother and say, ‘Please be the momentary wives of so-and-so.’” If he accepts the mission, finds out the response, and reports back, he commits an offense entailing suspension. 

A\marginnote{4.5.54} man sends a monk, saying, “Sir, go to so-and-so protected by the threat of punishment and so-and-so protected by her father … so-and-so protected by the threat of punishment and so-and-so otherwise protected and say, ‘Please be the momentary wives of so-and-so.’” If he accepts the mission, finds out the response, and reports back, he commits an offense entailing suspension. 

\scend{The section based on one item is finished. }

\subparagraph*{Acting as a matchmaker for a man and three to nine momentary wives }

The\marginnote{4.5.60.1} sections based on two items, etc., are to be done in the same way. 

\subparagraph*{Acting as a matchmaker for a man and ten momentary wives }

This\marginnote{4.5.61.1} is the section based on ten items: 

A\marginnote{4.5.62} man sends a monk, saying, “Sir, go to so-and-so protected by her mother and so-and-so protected by her father and so-and-so protected by her parents and so-and-so protected by her brother and so-and-so protected by her sister and so-and-so protected by her relatives and so-and-so protected by her family and so-and-so protected by her religion and so-and-so otherwise protected and so-and-so protected by the threat of punishment and say, ‘Please be the momentary wives of so-and-so.’” If he accepts the mission, finds out the response, and reports back, he commits an offense entailing suspension. 

\scend{The permutation series on momentary wives is finished. }

\subparagraph*{Acting as a matchmaker for a man and one protected by her mother: a single reason\footnote{The reason referred to here and below refers to the one or more reasons why a woman might become a man’s wife. } }

A\marginnote{4.6.1} man sends a monk, saying, “Sir, go to so-and-so protected by her mother and say, ‘Please be the bought wife of so-and-so.’” If he accepts the mission, finds out the response, and reports back, he commits an offense entailing suspension. 

A\marginnote{4.6.5} man sends a monk, saying, “Sir, go to so-and-so protected by her mother and say, ‘Please be the wife by choice of so-and-so.’” … the wife through property of so-and-so.’” … the wife through clothes of so-and-so.’” … the wife through the bowl-of-water ritual of so-and-so.’” … the wife through removing the head pad of so-and-so.’” … the slave wife of so-and-so.’” … the servant wife of so-and-so.’” … the captured wife of so-and-so.’” … the momentary wife of so-and-so.’” If he accepts the mission, finds out the response, and reports back, he commits an offense entailing suspension. 

\scend{The setting out of the steps is finished. }

\subparagraph*{Acting as a matchmaker for a man and one protected by her mother: combinations of two reasons }

A\marginnote{4.6.18.1} man sends a monk, saying, “Sir, go to so-and-so protected by her mother and say, ‘Please be the bought wife and the wife by choice of so-and-so.’” If he accepts the mission, finds out the response, and reports back, he commits an offense entailing suspension. 

A\marginnote{4.6.22} man sends a monk, saying, “Sir, go to so-and-so protected by her mother and say, ‘Please be the bought wife and the wife through property of so-and-so.’” … the bought wife and the wife through clothes of so-and-so.’” … the bought wife and the wife through the bowl-of-water ritual of so-and-so.’” … the bought wife and the wife through removing the head pad of so-and-so.’” … the bought wife and the slave wife of so-and-so.’” … the bought wife and the servant wife of so-and-so.’” … the bought wife and the captured wife of so-and-so.’” … the bought wife and the momentary wife of so-and-so.’” If he accepts the mission, finds out the response, and reports back, he commits an offense entailing suspension. 

\scend{The unconnected permutation series is finished. }

A\marginnote{4.6.34} man sends a monk, saying, “Sir, go to so-and-so protected by her mother and say, ‘Please be the wife by choice and the wife through property of so-and-so.’” … the wife by choice and the momentary wife of so-and-so.’” … the wife by choice and the bought wife of so-and-so.’” If he accepts the mission, finds out the response, and reports back, he commits an offense entailing suspension. 

\scend{The linked permutation series with the basis in brief is finished. }

A\marginnote{4.6.41} man sends a monk, saying, “Sir, go to so-and-so protected by her mother and say, ‘Please be the momentary wife and the bought wife of so-and-so.’” … the momentary wife and the wife by choice of so-and-so.’” … the momentary wife and the captured wife of so-and-so.’” If he accepts the mission, finds out the response, and reports back, he commits an offense entailing suspension. 

\scend{The section based on one item is finished. }

\subparagraph*{Acting as a matchmaker for a man and one protected by her mother: combinations of three to nine reasons }

The\marginnote{4.6.48.1} sections based on two items, etc., are to be done in the same way. 

\subparagraph*{Acting as a matchmaker for a man and one protected by her mother: ten reasons }

This\marginnote{4.6.49.1} is the section based on ten items: 

A\marginnote{4.6.50} man sends a monk, saying, “Sir, go to so-and-so protected by her mother and say, ‘Please be the bought wife and the wife by choice and the wife through property and the wife through clothes and the wife through the bowl-of-water ritual and the wife through removing the head pad and the slave wife and the servant wife and the captured wife and the momentary wife of so-and-so.’” If he accepts the mission, finds out the response, and reports back, he commits an offense entailing suspension. 

\scend{The permutation series on the one guarded by her mother is finished. }

\subparagraph*{Acting as a matchmaker for a man and one protected in various ways: a single reason }

A\marginnote{4.6.55.1} man sends a monk, saying, “Sir, go to so-and-so protected by her father … so-and-so protected by her parents … so-and-so protected by her brother … so-and-so protected by her sister … so-and-so protected by her relatives … so-and-so protected by her family … so-and-so protected by her religion … so-and-so otherwise protected … so-and-so protected by the threat of punishment and say, ‘Please be the bought wife of so-and-so.’” If he accepts the mission, finds out the response, and reports back, he commits an offense entailing suspension. 

\subparagraph*{Acting as a matchmaker for a man and one protected by the threat of punishment: a single reason }

A\marginnote{4.6.67.1} man sends a monk, saying, “Sir, go to so-and-so protected by the threat of punishment and say, ‘Please be the wife by choice of so-and-so.’” … the wife through property of so-and-so.’” … the wife through clothes of so-and-so.’” … the wife through the bowl-of-water ritual of so-and-so.’” … the wife through removing the head pad of so-and-so.’” … the slave wife of so-and-so.’” … the servant wife of so-and-so.’” … the captured wife of so-and-so.’” … the momentary wife of so-and-so.’” If he accepts the mission, finds out the response, and reports back, he commits an offense entailing suspension. 

\scend{The setting out of the steps is finished. }

\subparagraph*{Acting as a matchmaker for a man and one protected by the threat of punishment: combinations of two reasons }

A\marginnote{4.6.80.1} man sends a monk, saying, “Sir, go to so-and-so protected by the threat of punishment and say, ‘Please be the bought wife and the wife by choice of so-and-so.’” If he accepts the mission, finds out the response, and reports back, he commits an offense entailing suspension. 

A\marginnote{4.6.84} man sends a monk, saying, “Sir, go to so-and-so protected by the threat of punishment and say, ‘Please be the bought wife and the wife through property of so-and-so … the bought wife and the momentary wife of so-and-so.’” If he accepts the mission, finds out the response, and reports back, he commits an offense entailing suspension. 

\scend{The unconnected permutation series is finished. }

A\marginnote{4.6.90} man sends a monk, saying, “Sir, go to so-and-so protected by the threat of punishment and say, ‘Please be the wife by choice and the wife through property of so-and-so.’” … the wife by choice and the momentary wife of so-and-so.’” … the wife by choice and the bought wife of so-and-so.’” If he accepts the mission, finds out the response, and reports back, he commits an offense entailing suspension. 

\scend{The linked permutation series with the basis in brief is finished. }

A\marginnote{4.6.97} man sends a monk, saying, “Sir, go to so-and-so protected by the threat of punishment and say, ‘Please be the momentary wife and the bought wife of so-and-so.’” … the momentary wife and the wife by choice of so-and-so.’” … the momentary wife and the captured wife of so-and-so.’” If he accepts the mission, finds out the response, and reports back, he commits an offense entailing suspension. 

\scend{The section based on one item is finished. }

\subparagraph*{Acting as a matchmaker for a man and one protected by the threat of punishment: combinations of three to nine reasons }

The\marginnote{4.6.104.1} sections based on two items, three items, up to nine items, are to be done in the same way. 

\subparagraph*{Acting as a matchmaker for a man and one protected by the threat of punishment: ten reasons }

This\marginnote{4.6.105.1} is the section based on ten items: 

A\marginnote{4.6.106} man sends a monk, saying, “Sir, go to so-and-so protected by the threat of punishment and say, ‘Please be the bought wife and the wife by choice and the wife through property and the wife through clothes and the wife through the bowl-of-water ritual and the wife through removing the head pad and the slave wife and the servant wife and the captured wife and the momentary wife of so-and-so.’” If he accepts the mission, finds out the response, and reports back, he commits an offense entailing suspension. 

\scend{The permutation series for the one protected by the threat of punishment is finished. }

\subparagraph*{Incremental increase in both wives and reasons }

A\marginnote{4.7.1} man sends a monk, saying, “Sir, go to so-and-so protected by her mother and say, ‘Please be the bought wife of so-and-so.’” If he accepts the mission, finds out the response, and reports back, he commits an offense entailing suspension. 

A\marginnote{4.7.5} man sends a monk, saying, “Sir, go to so-and-so protected by her mother and so-and-so protected by her father and say, ‘Please be the bought wives and the wives by choice of so-and-so.’” If he accepts the mission, finds out the response, and reports back, he commits an offense entailing suspension. 

A\marginnote{4.7.9} man sends a monk, saying, “Sir, go to so-and-so protected by her mother and so-and-so protected by her father and so-and-so protected by her parents and say, ‘Please be the bought wives and the wives by choice and the wives through property of so-and-so.’” If he accepts the mission, finds out the response, and reports back, he commits an offense entailing suspension. 

\scend{In this way the increase of both items is to be done. }

A\marginnote{4.7.14} man sends a monk, saying, “Sir, go to so-and-so protected by her mother and so-and-so protected by her father and so-and-so protected by her parents and so-and-so protected by her brother and so-and-so protected by her sister and so-and-so protected by her relatives and so-and-so protected by her family and so-and-so protected by her religion and so-and-so otherwise protected and so-and-so protected by the threat of punishment and say, ‘Please be the bought wives and the wives by choice and the wives through property and the wives through clothes and the wives through the bowl-of-water ritual and the wives through removing the head pad and the slave wives and the servant wives and the captured wives and the momentary wives of so-and-so.’” If he accepts the mission, finds out the response, and reports back, he commits an offense entailing suspension. 

\scend{The increase of both items is finished. }

\subparagraph*{Relationships arranged for a man }

A\marginnote{4.8.1} man’s mother sends a monk … A man’s father sends a monk … A man’s parents send a monk … A man’s brother sends a monk … A man’s sister sends a monk … A man’s relatives send a monk … A man’s family sends a monk … A man’s fellow believers send a monk … 

To\marginnote{4.8.9} be expanded as for the successive series on a man. 

The\marginnote{4.8.10} increase of both items is to be expanded as before. 

\subparagraph*{Relationships arranged by the mother: a single reason }

The\marginnote{4.9.1} mother of one protected by her mother sends a monk, saying, “Sir, go to so-and-so and say, ‘I have a wife for you who can be your bought wife.’” If he accepts the mission, finds out the response, and reports back, he commits an offense entailing suspension. 

The\marginnote{4.9.5} mother of one protected by her mother sends a monk, saying, “Sir, go to so-and-so and say, ‘I have a wife for you who can be your wife by choice.’” … your wife through property.’” … your wife through clothes.’” … your wife through the bowl-of-water ritual.’” … your wife through removing the head pad.’” … your slave wife.’” … your servant wife.’” … your captured wife.’” … your momentary wife.’” If he accepts the mission, finds out the response, and reports back, he commits an offense entailing suspension. 

\scend{The setting out of the steps is finished. }

\subparagraph*{Relationships arranged by the mother: combinations of two reasons }

The\marginnote{4.9.18.1} mother of one protected by her mother sends a monk, saying, “Sir, go to so-and-so and say, ‘I have a wife for you who can be your bought wife and your wife by choice.’” … your bought wife and your wife through property.’” … your bought wife and your momentary wife.’” If he accepts the mission, finds out the response, and reports back, he commits an offense entailing suspension. 

\scend{The unconnected permutation series is finished. }

The\marginnote{4.9.25} mother of one protected by her mother sends a monk, saying, “Sir, go to so-and-so and say, ‘I have a wife for you who can be your wife by choice and your wife through property.’” … your wife by choice and your momentary wife.’” … your wife by choice and your bought wife.’” If he accepts the mission, finds out the response, and reports back, he commits an offense entailing suspension. 

\scend{The linked permutation series with the basis in brief is finished. }

The\marginnote{4.9.32} mother of one protected by her mother sends a monk, saying, “Sir, go to so-and-so and say, ‘I have a wife for you who can be your momentary wife and your bought wife.’” … your momentary wife and your wife by choice.’” … your momentary wife and your captured wife.’” If he accepts the mission, finds out the response, and reports back, he commits an offense entailing suspension. 

\scend{The section based on one item is finished. }

\subparagraph*{Relationships arranged by the mother: combinations of three to nine reasons }

The\marginnote{4.9.39.1} sections based on two items, three items, up to nine items, are to be done in the same way. 

\subparagraph*{Relationships arranged by the mother: ten reasons }

This\marginnote{4.9.40.1} is the section based on ten items: 

The\marginnote{4.9.41} mother of one protected by her mother sends a monk, saying, “Sir, go to so-and-so and say, ‘I have a wife for you who can be your bought wife and your wife by choice and your wife through property and your wife through clothes and your wife through the bowl-of-water ritual and your wife through removing the head pad and your slave wife and your servant wife and your captured wife and your momentary wife.’” If he accepts the mission, finds out the response, and reports back, he commits an offense entailing suspension. 

\scend{The permutation series on the mother is finished. }

\subparagraph*{Relationships arranged by various people: a single reason }

The\marginnote{4.10.1} father of one protected by her father sends a monk … The parents of one protected by her parents send a monk … The brother of one protected by her brother sends a monk … The sister of one protected by her sister sends a monk … The relatives of one protected by her relatives send a monk … The family of one protected by her family sends a monk … The fellow believers of one protected by her religion send a monk … The master of one otherwise protected sends a monk … The one who punishes in relation to one protected by the threat of punishment sends a monk, saying, “Sir, go to so-and-so and say, ‘I have a wife for you who can be your bought wife.’” If he accepts the mission, finds out the response, and reports back, he commits an offense entailing suspension. 

\subparagraph*{Relationships arranged by the one who punishes: a single reason }

The\marginnote{4.10.13.1} one who punishes in relation to one protected by the threat of punishment sends a monk, saying, “Sir, go to so-and-so and say, ‘I have a wife for you who can be your wife by choice.’” … your wife through property.’” … your wife through clothes.’” … your wife through the bowl-of-water ritual.’” … your wife through removing the head pad.’” … your slave wife.’” … your servant wife.’” … your captured wife.’” … your momentary wife.’” If he accepts the mission, finds out the response, and reports back, he commits an offense entailing suspension. 

\scend{The setting out of the steps is finished. }

\subparagraph*{Relationships arranged by the one who punishes: two reasons }

The\marginnote{4.10.26.1} one who punishes in relation to one protected by the threat of punishment sends a monk, saying, “Sir, go to so-and-so and say, ‘I have a wife for you who can be your bought wife and your wife by choice.’” … your bought wife and your wife through property.’” … your bought wife and your momentary wife.’” If he accepts the mission, finds out the response, and reports back, he commits an offense entailing suspension. 

\scend{The unconnected permutation series is finished. }

The\marginnote{4.10.33} one who punishes in relation to one protected by the threat of punishment sends a monk, saying, “Sir, go to so-and-so and say, ‘I have a wife for you who can be your wife by choice and your wife through property.’” … your wife by choice and your momentary wife.’” … your wife by choice and your bought wife.’” If he accepts the mission, finds out the response, and reports back, he commits an offense entailing suspension. 

\scend{The linked permutation series with the basis in brief is finished. }

The\marginnote{4.10.40} one who punishes in relation to one protected by the threat of punishment sends a monk, saying, “Sir, go to so-and-so and say, ‘I have a wife for you who can be your momentary wife and your bought wife.’” … who can be your momentary wife and your wife by choice.’” … who can be your momentary wife and your captured wife.’” If he accepts the mission, finds out the response, and reports back, he commits an offense entailing suspension. 

\scend{The section based on one item is finished. }

\subparagraph*{Relationships arranged by the one who punishes: three to nine reasons }

The\marginnote{4.10.47.1} sections based on two items, three items, up to nine items, are to be done in the same way. 

\subparagraph*{Relationships arranged by the one who punishes: ten reasons }

This\marginnote{4.10.48.1} is the section based on ten items: 

The\marginnote{4.10.49} one who punishes in relation to one protected by the threat of punishment sends a monk, saying, “Sir, go to so-and-so and say, ‘I have a wife for you who can be your bought wife and your wife by choice and your wife through property and your wife through clothes and your wife through the bowl-of-water ritual and your wife through removing the head pad and your slave wife and your servant wife and your captured wife and your momentary wife.’” If he accepts the mission, finds out the response, and reports back, he commits an offense entailing suspension. 

\scend{The permutation series on the one who punishes is finished. }

\subparagraph*{The one protected by her mother taking the initiative: a single reason }

The\marginnote{4.11.1} one protected by her mother sends a monk, saying, “Sir, go to so-and-so and say that I’ll be his bought wife.” If he accepts the mission, finds out the response, and reports back, he commits an offense entailing suspension. 

The\marginnote{4.11.5} one protected by her mother sends a monk, saying, “Sir, go to so-and-so and say that I’ll be his wife by choice.” … his wife through property.” … his wife through clothes.” … his wife through the bowl-of-water ritual.” … his wife through removing the head pad.” … his slave wife.” … his servant wife.” … his captured wife.” … his momentary wife.” If he accepts the mission, finds out the response, and reports back, he commits an offense entailing suspension. 

\scend{The setting out of the steps is finished. }

\subparagraph*{The one protected by her mother taking the initiative: two reasons }

The\marginnote{4.11.18.1} one protected by her mother sends a monk, saying, “Sir, go to so-and-so and say that I’ll be his bought wife and his wife by choice.” If he accepts the mission, finds out the response, and reports back, he commits an offense entailing suspension. 

The\marginnote{4.11.22} one protected by her mother sends a monk, saying, “Sir, go to so-and-so and say that I’ll be his bought wife and his wife through property.” … his bought wife and his wife through clothes.” … his bought wife and his momentary wife.” If he accepts the mission, finds out the response, and reports back, he commits an offense entailing suspension. 

\scend{The unconnected permutation series is finished. }

The\marginnote{4.11.29} one protected by her mother sends a monk, saying, “Sir, go to so-and-so and say that I’ll be his wife by choice and his wife through property.” … his wife by choice and his momentary wife.” … his wife by choice and his bought wife.” If he accepts the mission, finds out the response, and reports back, he commits an offense entailing suspension. 

\scend{The linked permutation series with the basis in brief is finished. }

The\marginnote{4.11.36} one protected by her mother sends a monk, saying, “Sir, go to so-and-so and say that I’ll be his momentary wife and his bought wife.” … his momentary wife and his wife by choice.” … his momentary wife and his captured wife.” If he accepts the mission, finds out the response, and reports back, he commits an offense entailing suspension. 

\scend{The section based on one item is finished. }

\subparagraph*{The one protected by her mother taking the initiative: three to nine reasons }

The\marginnote{4.11.43.1} sections based on two items, etc., are to be done in the same way. 

\subparagraph*{The one protected by her mother taking the initiative: ten reasons }

This\marginnote{4.11.44.1} is the section based on ten items: 

The\marginnote{4.11.45} one protected by her mother sends a monk, saying, “Sir, go to so-and-so and say that I’ll be his bought wife and his wife by choice and his wife through property and his wife through clothes and his wife through the bowl-of-water ritual and his wife through removing the head pad and his slave wife and his servant wife and his captured wife and his momentary wife.” If he accepts the mission, finds out the response, and reports back, he commits an offense entailing suspension. 

\scend{The further permutation series on the one guarded by her mother is finished. }

\subparagraph*{The ones protected by various people taking the initiative: a single reason }

The\marginnote{4.11.50.1} one protected by her father sends a monk … The one protected by her parents sends a monk … The one protected by her brother sends a monk … The one protected by her sister sends a monk … The one protected by her relatives sends a monk … The one protected by her family sends a monk … The one protected by her religion sends a monk … The one otherwise protected sends a monk … The one protected by the threat of punishment sends a monk, saying, “Sir, go to so-and-so and say that I’ll be his bought wife.” If he accepts the mission, finds out the response, and reports back, he commits an offense entailing suspension. 

\subparagraph*{The one protected by the threat of punishment taking the initiative: a single reason }

The\marginnote{4.11.62.1} one protected by the threat of punishment sends a monk, saying, “Sir, go to so-and-so and say that I’ll be his wife by choice.” … his wife through property.” … his wife through clothes.” … his wife through the bowl-of-water ritual.” … his wife through removing the head pad.” … his slave wife.” … his servant wife.” … his captured wife.” … his momentary wife.” If he accepts the mission, finds out the response, and reports back, he commits an offense entailing suspension. 

\scend{The setting out of the steps is finished. }

\subparagraph*{The one protected by the threat of punishment taking the initiative: two reasons }

The\marginnote{4.11.75.1} one protected by the threat of punishment sends a monk, saying, “Sir, go to so-and-so and say that I’ll be his bought wife and his wife by choice.” … his bought wife and his momentary wife.” If he accepts the mission, finds out the response, and reports back, he commits an offense entailing suspension. 

\scend{The unconnected permutation series is finished. }

The\marginnote{4.11.81} one protected by the threat of punishment sends a monk, saying, “Sir, go to so-and-so and say that I’ll be his wife by choice and his wife through property.” … his wife by choice and his momentary wife.” … his wife by choice and his bought wife.” If he accepts the mission, finds out the response, and reports back, he commits an offense entailing suspension. 

\scend{The linked permutation series with the basis in brief is finished. }

The\marginnote{4.11.88} one protected by the threat of punishment sends a monk, saying, “Sir, go to so-and-so and say that I’ll be his momentary wife and his bought wife.” … his momentary wife and his wife by choice.” … his momentary wife and his captured wife.” If he accepts the mission, finds out the response, and reports back, he commits an offense entailing suspension. 

\scend{The section based on one item is finished. }

\subparagraph*{The one protected by the threat of punishment taking the initiative: three to nine reasons }

The\marginnote{4.11.95.1} sections based on two items, etc., are to be done in the same way. 

\subparagraph*{The one protected by the threat of punishment taking the initiative: ten reasons }

This\marginnote{4.11.96.1} is the section based on ten items: 

The\marginnote{4.11.97} one protected by the threat of punishment sends a monk, saying, “Sir, go to so-and-so and say that I’ll be his bought wife and his wife by choice and his wife through property and his wife through clothes and his wife through the bowl-of-water ritual and his wife through removing the head pad and his slave wife and his servant wife and his captured wife and his momentary wife.” If he accepts the mission, finds out the response, and reports back, he commits an offense entailing suspension. 

\scend{The further permutation series on the one protected by the threat of punishment is finished. }

\scend{The whole successive permutation series is finished. }

\subsubsection*{Permutations part 2 }

If\marginnote{4.12.1} he accepts the mission, finds out the response, and reports back, he commits an offense entailing suspension. If he accepts the mission, and finds out the response, but does not report back, he commits a serious offense. If he accepts the mission, but does not find out the response, yet reports back, he commits a serious offense. If he accepts the mission, but neither finds out the response, nor reports back, he commits an offense of wrong conduct. If he does not accept the mission, yet finds out the response and reports back, he commits a serious offense. If he does not accept the mission, yet finds out the response, but does not report back, he commits an offense of wrong conduct. If he neither accepts the mission, nor finds out the response, yet reports back, he commits an offense of wrong conduct. If he does not accept the mission, nor finds out the response, nor reports back, there is no offense. 

A\marginnote{4.13.1} man tells a number of monks, “Venerables, find out about such-and-such a woman.” If they all accept the mission, all find out the response, and all report back, they all commit an offense entailing suspension. 

A\marginnote{4.13.4} man tells a number of monks, “Venerables, find out about such-and-such a woman.” If they all accept the mission, all find out the response, but only one reports back, they all commit an offense entailing suspension. 

A\marginnote{4.13.7} man tells a number of monks, “Venerables, find out about such-and-such a woman.” If they all accept the mission, but only one finds out the response, yet all report back, they all commit an offense entailing suspension. 

A\marginnote{4.13.10} man tells a number of monks, “Venerables, find out about such-and-such a woman.” If they all accept the mission, but only one finds out the response, and only one reports back, they all commit an offense entailing suspension. 

A\marginnote{4.14.1} man tells a monk, “Sir, find out about such-and-such a woman.” If he accepts the mission, finds out the response, and reports back, he commits an offense entailing suspension. 

A\marginnote{4.14.4} man tells a monk, “Sir, find out about such-and-such a woman.” If he accepts the mission, finds out the response, but gets a pupil to report back, he commits an offense entailing suspension. 

A\marginnote{4.14.7} man tells a monk, “Sir, find out about such-and-such a woman.” If he accepts the mission, but gets a pupil to find out the response, and then reports back himself, he commits an offense entailing suspension. 

A\marginnote{4.14.10} man tells a monk, “Sir, find out about such-and-such a woman.” If he accepts the mission, but gets a pupil to find out the response, and the pupil then reports back on his own initiative, they both commit a serious offense. 

\subsubsection*{Permutations part 3 }

If\marginnote{4.15.1} he fulfills the agreement when he goes, but not when he returns, he commits a serious offense. 

If\marginnote{4.15.2} he does not fulfill the agreement when he goes, but he does when he returns, he commits a serious offense. 

If\marginnote{4.15.3} he fulfills the agreement both when he goes and when he returns, he commits an offense entailing suspension. 

If\marginnote{4.15.4} he neither fulfills the agreement when he goes nor when he returns, there is no offense. 

\subsection*{Non-offenses }

There\marginnote{4.16.1} is no offense: if he goes because of business for the Sangha, for a shrine, or for one who is sick; if he is insane; if he is the first offender. 

\scuddanaintro{Summary verses of case studies }

\begin{scuddana}%
“Asleep,\marginnote{4.16.6} and dead, gone out, \\
Not a woman, a woman who lacks sexual organs; \\
He reconciled them after quarreling, \\
And he was a matchmaker for \textit{\textsanskrit{paṇḍakas}}.” 

%
\end{scuddana}

\subsubsection*{Case studies }

At\marginnote{5.1.1} one time a man told a monk, “Sir, please find out about such-and-such a woman.” When the monk got there, he asked some people, “Where’s so-and-so?” 

“She’s\marginnote{5.1.5} asleep, venerable.” 

He\marginnote{5.1.6} became anxious, thinking, “The Buddha has laid down a training rule. Could it be that I’ve committed an offense entailing suspension?” He told the Buddha, who said, “There’s no offense entailing suspension, but there’s an offense of wrong conduct.” 

At\marginnote{5.2.1} one time a man told a monk, “Sir, please find out about such-and-such a woman.” When the monk got there, he asked some people, “Where’s so-and-so?” 

“She’s\marginnote{5.2.5} dead, venerable.” … “She’s gone out, venerable.” … “That’s not a woman, venerable.” … “That’s a woman who lacks sexual organs, venerable.” 

He\marginnote{5.2.9} became anxious … “There’s no offense entailing suspension, but there’s an offense of wrong conduct.” 

At\marginnote{5.3.1} one time a certain woman quarreled with her husband and went to her mother’s house. A monk who associated with that family reconciled them. He became anxious … “Were they divorced, monk?” 

“No,\marginnote{5.3.5} sir.” 

“There’s\marginnote{5.3.6} no offense if they’re not divorced.” 

At\marginnote{5.4.1} one time a monk acted as a matchmaker for \textit{\textsanskrit{paṇḍakas}}. He became anxious … “There’s no offense entailing suspension, but there’s a serious offense.” 

\scendsutta{The training rule on matchmaking, the fifth, is finished. }

%
\section*{{\suttatitleacronym Bu Ss 6}{\suttatitletranslation 6. The training rule on building huts }{\suttatitleroot Kuṭikāra}}
\addcontentsline{toc}{section}{\tocacronym{Bu Ss 6} \toctranslation{6. The training rule on building huts } \tocroot{Kuṭikāra}}
\markboth{6. The training rule on building huts }{Kuṭikāra}
\extramarks{Bu Ss 6}{Bu Ss 6}

\subsection*{Origin story }

At\marginnote{1.1.1} one time the Buddha was staying at \textsanskrit{Rājagaha} in the Bamboo Grove, the squirrel sanctuary. At that time the monks of \textsanskrit{Āḷavī} were building huts by means of begging. The huts were intended for themselves, did not have a sponsoring owner, and were inappropriately large. And since they were never finished, the monks kept on begging and asking, “Please give a man, a servant, an ox, a cart, a machete, a hatchet, an ax, a spade, a chisel; give creepers, bamboo, reeds, grass, clay.”\footnote{I have rendered \textit{\textsanskrit{muñja}}-reed and \textit{pabbaja}-reed with the single word “reed”. I am not aware that these two kinds of reed can be distinguished in English. } People felt oppressed by all the begging and asking, so much so that when they saw a monk they became alarmed and fearful. They turned away, took a different path, ran off, and closed their doors. They even ran away when they saw cows, thinking they were monks. 

Just\marginnote{1.1.8} then Venerable \textsanskrit{Mahākassapa}, after completing the rainy-season residence at \textsanskrit{Rājagaha}, set out for \textsanskrit{Āḷavī}. When he eventually arrived, he stayed at the \textsanskrit{Aggāḷava} Shrine. One morning he robed up, took his bowl and robe, and entered \textsanskrit{Āḷavī} for alms. When people saw him, they became alarmed and fearful. They turned away, took a different path, ran off, and closed their doors. Then, when Venerable \textsanskrit{Mahākassapa} had eaten his meal and returned from almsround, he said to the monks: 

“There\marginnote{1.1.14} used to be plenty of almsfood in \textsanskrit{Āḷavī}, and it was easy to get by on alms. But now there’s a shortage, and it’s hard to get by. Why is that?” The monks told Venerable \textsanskrit{Mahākassapa} what had happened. 

Soon\marginnote{1.2.1} afterwards the Buddha too set out wandering toward \textsanskrit{Āḷavī} after staying at \textsanskrit{Rājagaha} for as long as he liked. When he eventually arrived, he too stayed at the \textsanskrit{Aggāḷava} Shrine. 

Venerable\marginnote{1.2.4} \textsanskrit{Mahākassapa} then went to see the Buddha, bowed, sat down, and told him what had happened. 

The\marginnote{1.2.6} Buddha had the Sangha gathered and questioned the monks of \textsanskrit{Āḷavī}: “Is it true, monks, that this is happening?” 

“It’s\marginnote{1.2.12} true, sir.” 

The\marginnote{1.2.13} Buddha rebuked them … “Foolish men, how can you act like this? This will affect people’s confidence …” After rebuking them, he gave a teaching and addressed the monks: 

\subparagraph*{\textsanskrit{Jātaka}\footnote{This story is parallel to the \textsanskrit{Maṇikaṇṭhajātaka}, story number 253 in the \textsanskrit{Jātaka} collection. } }

“Once\marginnote{1.3.1} upon a time, monks, two sages who were brothers lived near the river Ganges. On one occasion the dragon king \textsanskrit{Maṇikaṇṭha} emerged from the Ganges and went up to the younger sage. He encircled him with seven coils and spread his large hood over his head. Then, because of his fear of that dragon, the younger sage became thin, haggard, and pale, with veins protruding all over his body. The older sage saw him like this and asked him what was the matter. The younger sage told him. The elder sage said, ‘So, do you want that dragon to stay away?’ 

‘Yes.’\marginnote{1.3.10} 

‘Well\marginnote{1.3.11} then, did you see anything belonging to that dragon?’ 

‘I\marginnote{1.3.12} saw an ornamental gem on his neck.’ 

‘In\marginnote{1.3.13} that case, ask the dragon for that gem.’ 

Soon\marginnote{1.3.16} the dragon king again emerged from the Ganges and went up to the younger sage. And the sage said to him, ‘Sir, give me the gem. I want the gem.’ The dragon thought, ‘The monk is asking for the gem; he wants the gem,’ and he left in a hurry. 

Once\marginnote{1.3.22} more the dragon king emerged from the Ganges and approached the younger sage. The sage saw him coming and said to him, ‘Sir, give me the gem. I want the gem.’ When the dragon heard him, he turned around right there. 

Yet\marginnote{1.3.29} again the dragon king emerged from the Ganges. The younger sage saw him emerging and said to him, ‘Sir, give me the gem. I want the gem.’ The dragon king then spoke these verses to the sage: 

\begin{verse}%
‘My\marginnote{1.3.35} food and drink are abundant and sublime, \\
And they appear because of this gem. \\
I won’t give it to you—you ask too much—\\
Nor will I return to your hermitage. 

Like\marginnote{1.3.39} a youth holding a sword polished on a rock,\footnote{Sp 1.344: \textit{\textsanskrit{Sakkharā} vuccati \textsanskrit{kāḷasilā}, tattha dhoto asi “sakkharadhoto \textsanskrit{nāmā}”ti vuccati, sakkharadhoto \textsanskrit{pāṇimhi} \textsanskrit{assāti} \textsanskrit{sakkharadhotapāṇi}, \textsanskrit{pāsāṇe} dhotanisitakhaggahatthoti attho}; “A black rock is what is meant by \textit{\textsanskrit{sakkharā}}. A sword polished on that is termed \textit{sakkharadhota}. A sword in his hand polished on a rock is the meaning of \textit{\textsanskrit{sakkharadhotapāṇi}}. The meaning is a hand with a sword that is sharp and polished on a rock.” } \\
You frighten me, asking for this gem.\footnote{Sp 1.344: \textit{\textsanskrit{Yathā} so asihattho puriso \textsanskrit{tāseyya}, \textsanskrit{evaṁ} \textsanskrit{tāsesi} \textsanskrit{maṁ} \textsanskrit{selaṁ} \textsanskrit{yācamāno}, \textsanskrit{maṇiṁ} \textsanskrit{yācantoti} attho}, “The meaning is that just as that man with sword in hand would cause fear, so the one asking me for the rock, asking for the gem, caused fear.” } \\
I won’t give it to you—you ask too much—\\
Nor will I return to your hermitage.’ 

%
\end{verse}

And\marginnote{1.3.43} the dragon king \textsanskrit{Maṇikaṇṭha} thought, ‘The monk is asking for the gem; he wants the gem,’ and he left and never returned. 

Because\marginnote{1.3.46} he did not get to see that beautiful dragon, the young sage became even thinner, more haggard and pale, his veins protruding even more. The older sage saw him like this and asked what was the matter. He replied, ‘It’s because I no longer get to see that beautiful dragon.’ The older sage then spoke to him in verse: 

\begin{verse}%
‘One\marginnote{1.3.52} shouldn’t beg from those one wishes to be dear to; \\
One is detested for asking for too much. \\
When the brahmin asked the dragon for his gem, \\
It left and was never to be seen again.’ 

%
\end{verse}

One\marginnote{1.3.56} will be disliked even by animals, monks, for begging and asking, let alone by human beings.” 

\subparagraph*{Story\footnote{This story seems to be unique to this rule. } }

“At\marginnote{1.4.1} one time, monks, a certain monk lived in a forest grove on the slopes of the Himalayas. Not far from that grove was a large, low-lying marsh. A great flock of birds fed in the marsh during the day and entered the grove to roost at night. The monk was disturbed by the noise of the flocking birds, and so he came to see me. He bowed, sat down, and I said to him, ‘I hope you’re keeping well, monk, I hope you’re getting by? I hope you’re not tired from traveling?  And where have you come from?’ 

‘I’m\marginnote{1.4.9} keeping well, sir, I’m getting by. I’m not tired from traveling.’ He then explained where he had come from, adding, ‘That’s where I’ve come from, sir. I left because I was disturbed by the noise of that flock of birds.’ 

‘Do\marginnote{1.4.16} you want that flock of birds to stay away?’ 

‘Yes,\marginnote{1.4.17} sir.’ 

‘Well\marginnote{1.4.18} then, go back to that forest grove. In the first part of the night, call out three times and say, “Listen to me, good birds. I want a feather from anyone roosting in this forest grove. Each one of you must give me a feather.” And in the middle and last part of the night do the same thing.’ 

The\marginnote{1.4.27} monk returned to that forest grove and did as instructed. That flock of birds thought, ‘The monk is asking for a feather; he wants a feather,’ and they left that grove and never returned. One will be disliked even by animals, monks, for begging and asking, let alone by human beings. 

“\textsanskrit{Raṭṭhapāla}’s\marginnote{1.5.1} father, monks, once spoke to his son with this verse:\footnote{For the inspiring story of \textsanskrit{Raṭṭhapāla}, see \href{https://suttacentral.net/mn82/en/brahmali\#1.1}{MN 82:1.1}. } 

\begin{verse}%
‘All\marginnote{1.5.2} these people, \textsanskrit{Raṭṭhāpāla}, \\
Who come to me and beg—\\
I don’t even know them. \\
So why don’t \emph{you} beg from me?’ 

‘The\marginnote{1.5.6} beggar is disliked, \\
And so is one who doesn’t give when asked. \\
That’s why I do not beg from you; \\
Please don’t hate me for this.’ 

%
\end{verse}

If\marginnote{1.5.10} the gentleman \textsanskrit{Raṭṭhapāla} could say this to his own father, how much more can one person to another. 

It’s\marginnote{1.6.1} hard, monks, for householders to acquire and protect their possessions. And still, foolish men, you kept on begging and asking for all sorts of things. This will affect people’s confidence …” … “And, monks, this training rule should be recited like this: 

\subsection*{Final ruling }

\scrule{‘When a monk, by means of begging, builds a hut without a sponsoring owner and intended for himself, it is to be no more than twelve standard handspans long and seven wide inside. He must have monks approve a site where no harm will be done and which has space on all sides. If a monk, by means of begging, builds a hut on a site where harm will be done and which lacks space on all sides, or he does not have monks approve the site, or he exceeds the right size, he commits an offense entailing suspension.’” }

\subsection*{Definitions }

\begin{description}%
\item[By means of begging: ] having himself begged for a man, a servant, an ox, a cart, a machete, a hatchet, an ax, a spade, a chisel; creepers, bamboo, reed, grass, clay. %
\item[A hut: ] plastered inside or plastered outside or plastered both inside and outside. %
\item[Builds: ] building it himself or having it built. %
\item[Without a sponsoring owner: ] there is no other owner, either a woman or a man, either a lay person or one gone forth. %
\item[Intended for himself: ] for his own use. %
\item[\footnote{In the rule formulation above, I have not translated the phrase \textit{\textsanskrit{pamāṇikā} \textsanskrit{kāretabbā}}. This is for stylistic reasons. It is often awkward in English to replicate the exact phrasing of the Pali. This means that the same phrase is also not translated here in the word commentary. } It is to be no more than twelve standard handspans long:\footnote{“Standard” renders \textit{sugata}. See Appendix of Technical Terms for discussion. } ] measured outside. %
\item[And seven wide inside: ] measured inside. %
\item[He must have monks approve a site: ] the\marginnote{2.2.2} monk who wants to build a hut should clear a site. He should then approach the Sangha, arrange his upper robe over one shoulder, pay respect at the feet of the senior monks, squat on his heels, raise his joined palms, and say: 

“Venerables,\marginnote{2.2.3} I want to build a hut by means of begging, without a sponsoring owner and intended for myself. I request the Sangha to inspect the site for that hut.” 

He\marginnote{2.2.5} should make his request a second and a third time. If the whole Sangha is able to inspect the site, they should all go. If the whole Sangha is unable to inspect the site, then those monks there who are competent and capable—who know where harm will be done and where no harm will be done, who know what is meant by space on all sides and a lack of space on all sides—should be asked and then appointed. 

“And,\marginnote{2.2.9} monks, they should be appointed like this. A competent and capable monk should inform the Sangha: 

‘Please,\marginnote{2.2.11} venerables, I ask the Sangha to listen. Monk so-and-so wants to build a hut by means of begging, without a sponsoring owner and intended for himself. He is requesting the Sangha to inspect the site for that hut. If the Sangha is ready, it should appoint monk so-and-so and monk so-and-so to inspect the site for the hut of monk so-and-so. This is the motion. 

Please,\marginnote{2.2.16} venerables, I ask the Sangha to listen. Monk so-and-so wants to build a hut by means of begging, without a sponsoring owner and intended for himself. He is requesting the Sangha to inspect the site for that hut. The Sangha appoints monk so-and-so and monk so-and-so to inspect the site for the hut of monk so-and-so. Any monk who approves of appointing monk so-and-so and monk so-and-so to inspect the site for the hut of monk so-and-so should remain silent. Any monk who doesn’t approve should speak up. 

The\marginnote{2.2.22} Sangha has appointed monk so-and-so and monk so-and-so to inspect the site for the hut of monk so-and-so. The Sangha approves and is therefore silent. I’ll remember it thus.’ 

The\marginnote{2.2.25} appointed monks should go and inspect the site for the hut to find out if any harm will be done and if it has space on all sides. If harm will be done or it lacks space on all sides, they should say, ‘Don’t build here.’ If no harm will be done and it has space on all sides, they should inform the Sangha: ‘No harm will be done and it has space on all sides.’ The monk who wants to build the hut should then approach the Sangha, arrange his upper robe over one shoulder, pay respect at the feet of the senior monks, squat on his heels, raise his joined palms, and say: 

‘Venerables,\marginnote{2.2.30} I want to build a hut by means of begging, without a sponsoring owner and intended for myself. I request the Sangha to approve the site for the hut.’ 

He\marginnote{2.2.32} should make his request a second and a third time. A competent and capable monk should then inform the Sangha: 

‘Please,\marginnote{2.2.35} venerables, I ask the Sangha to listen. Monk so-and-so wants to build a hut by means of begging, without a sponsoring owner and intended for himself. He is requesting the Sangha to approve the site for that hut. If the Sangha is ready, it should approve the site for the hut of monk-so-and-so. This is the motion. 

Please,\marginnote{2.2.40} venerables, I ask the Sangha to listen. Monk so-and-so wants to build a hut by means of begging, without a sponsoring owner and intended for himself. He is requesting the Sangha to approve the site for that hut. The Sangha approves the site for the hut of monk so-and-so. Any monk who agrees to approving the site for the hut of monk-so-and-so should remain silent. Any monk who doesn’t agree should speak up. 

The\marginnote{2.2.46} Sangha has approved the site for the hut of monk so-and-so. The Sangha approves and is therefore silent. I’ll remember it thus.’” 

%
\item[Where harm will be done: ] it is the abode of ants, termites, rats, snakes, scorpions, centipedes, elephants, horses, lions, tigers, leopards, bears, or hyenas, or any other animal; or it is bordering on a field of grain, a field of vegetables, a place of slaughter, a place of execution, a charnel ground, a park, a royal property, an elephant stable, a horse stable, a prison, a bar, a slaughterhouse, a street, a crossroads, a public meeting hall, or a cul-de-sac—\footnote{“Elephant stable” renders \textit{\textsanskrit{hatthisālā}}. For the meaning of \textit{\textsanskrit{sālā}}, see Appendix of Technical Terms. } this is called “where harm will be done”. %
\item[Which lacks space on all sides: ] it is not possible to go around it with a yoked cart, or to go all the way around it with a ladder—this is called “which lacks space on all sides”. %
\item[Where no harm will be done: ] it is not the abode of ants, termites, rats, snakes, scorpions, centipedes … it is not bordering on … a cul-de-sac—this is called “where no harm will be done”. %
\item[Which has space on all sides: ] it is possible to go around it with a yoked cart, or to go all the way around it with a ladder—this is called “which has space on all sides”. %
\item[By means of begging: ] having himself begged for a man, a servant … clay. %
\item[A hut: ] plastered inside or plastered outside or plastered both inside and outside. %
\item[Builds: ] building it himself or having it built. %
\item[Or he does not have monks approve the site, or he exceeds the right size: ] if the site for the hut has not been approved through a legal procedure consisting of one motion and three announcements, or if he builds a hut or has one built that exceeds the allowable length or breadth even by the width of a hair, then for the effort there is an act of wrong conduct. When there is one piece left to complete the hut, he commits a serious offense. When the last piece is finished, he commits an offense entailing suspension. %
\item[He commits an offense entailing suspension: ] … Therefore, too, it is called “an offense entailing suspension”. %
\end{description}

\subsection*{Permutations }

\subsubsection*{Permutations part 1 }

\subparagraph*{Building oneself }

If\marginnote{3.1.1} a monk builds a hut whose site has not been approved, where harm will be done, and which lacks space on all sides, he commits one offense entailing suspension and two offenses of wrong conduct. If a monk builds a hut whose site has not been approved, where harm will be done, but which has space on all sides, he commits one offense entailing suspension and one offense of wrong conduct. If a monk builds a hut whose site has not been approved, where no harm will be done, but which lacks space on all sides, he commits one offense entailing suspension and one offense of wrong conduct. If a monk builds a hut whose site has not been approved, where no harm will be done, and which has space on all sides, he commits one offense entailing suspension. 

If\marginnote{3.1.5} a monk builds a hut whose site has been approved, where harm will be done, and which lacks space on all sides, he commits two offenses of wrong conduct. If a monk builds a hut whose site has been approved, where harm will be done, but which has space on all sides, he commits one offense of wrong conduct. If a monk builds a hut whose site has been approved, where no harm will be done, but which lacks space on all sides, he commits one offense of wrong conduct. If a monk builds a hut whose site has been approved, where no harm will be done, and which has space on all sides, there is no offense. 

If\marginnote{3.2.1} a monk builds a hut which exceeds the right size, where harm will be done, and which lacks space on all sides, he commits one offense entailing suspension and two offenses of wrong conduct. If a monk builds a hut which exceeds the right size, where harm will be done, but which has space on all sides, he commits one offense entailing suspension and one offense of wrong conduct. If a monk builds a hut which exceeds the right size, where no harm will be done, but which lacks space on all sides, he commits one offense entailing suspension and one offense of wrong conduct. If a monk builds a hut which exceeds the right size, where no harm will be done, and which has space on all sides, he commits one offense entailing suspension. 

If\marginnote{3.2.5} a monk builds a hut which is the right size, where harm will be done, and which lacks space on all sides, he commits two offenses of wrong conduct. If a monk builds a hut which is the right size, where harm will be done, but which has space on all sides, he commits one offense of wrong conduct. If a monk builds a hut which is the right size, where no harm will be done, but which lacks space on all sides, he commits one offense of wrong conduct. If a monk builds a hut which is the right size, where no harm will be done, and which has space on all sides, there is no offense. 

If\marginnote{3.3.1} a monk builds a hut whose site has not been approved, which exceeds the right size, where harm will be done, and which lacks space on all sides, he commits two offenses entailing suspension and two offenses of wrong conduct. If a monk builds a hut whose site has not been approved, which exceeds the right size, where harm will be done, but which has space on all sides, he commits two offenses entailing suspension and one offense of wrong conduct. If a monk builds a hut whose site has not been approved, which exceeds the right size, where no harm will be done, but which lacks space on all sides, he commits two offenses entailing suspension and one offense of wrong conduct. If a monk builds a hut whose site has not been approved, which exceeds the right size, where no harm will be done, and which has space on all sides, he commits two offenses entailing suspension. 

If\marginnote{3.4.1} a monk builds a hut whose site has been approved, which is the right size, where harm will be done, and which lacks space on all sides, he commits two offenses of wrong conduct. If a monk builds a hut whose site has been approved, which is the right size, where harm will be done, but which has space on all sides, he commits one offense of wrong conduct. If a monk builds a hut whose site has been approved, which is the right size, where no harm will be done, but which lacks space on all sides, he commits one offense of wrong conduct. If a monk builds a hut whose site has been approved, which is the right size, where no harm will be done, and which has space on all sides, there is no offense. 

\subparagraph*{Appointing someone else to build }

A\marginnote{3.5.1} monk appoints someone to build him a hut. If they build one whose site has not been approved, where harm will be done, and which lacks space on all sides, he commits one offense entailing suspension and two offenses of wrong conduct. … where harm will be done, but which has space on all sides, he commits one offense entailing suspension and one offense of wrong conduct. … where no harm will be done, but which lacks space on all sides, he commits one offense entailing suspension and one offense of wrong conduct. … where no harm will be done, and which has space on all sides, he commits one offense entailing suspension. 

A\marginnote{3.5.7} monk appoints someone to build him a hut. If they build one whose site has been approved, where harm will be done, and which lacks space on all sides, he commits two offenses of wrong conduct. … where harm will be done, but which has space on all sides, he commits one offense of wrong conduct. … where no harm will be done, but which lacks space on all sides, he commits one offense of wrong conduct. … where no harm will be done, and which has space on all sides, there is no offense. 

A\marginnote{3.5.13} monk appoints someone to build him a hut. If they build one which exceeds the right size, where harm will be done, and which lacks space on all sides, he commits one offense entailing suspension and two offenses of wrong conduct. … where harm will be done, but which has space on all sides, he commits one offense entailing suspension and one offense of wrong conduct. … where no harm will be done, but which lacks space on all sides, he commits one offense entailing suspension and one offense of wrong conduct. … where no harm will be done, and which has space on all sides, he commits one offense entailing suspension. 

A\marginnote{3.5.19} monk appoints someone to build him a hut. If they build one which is the right size, where harm will be done, and which lacks space on all sides, he commits two offenses of wrong conduct. … where harm will be done, but which has space on all sides, he commits one offense of wrong conduct. … where no harm will be done, but which lacks space on all sides, he commits one offense of wrong conduct. … where no harm will be done, and which has space on all sides, there is no offense. 

A\marginnote{3.5.25} monk appoints someone to build him a hut. If they build one whose site has not been approved, which exceeds the right size, where harm will be done, and which lacks space on all sides, he commits two offenses entailing suspension and two offenses of wrong conduct. … where harm will be done, but which has space on all sides, he commits two offenses entailing suspension and one offense of wrong conduct. … where no harm will be done, but which lacks space on all sides, he commits two offenses entailing suspension and one offense of wrong conduct. … where no harm will be done, and which has space on all sides, he commits two offenses entailing suspension. 

A\marginnote{3.5.31} monk appoints someone to build him a hut. If they build one whose site has been approved, which is the right size, where harm will be done, and which lacks space on all sides, he commits two offenses of wrong conduct. … where harm will be done, but which has space on all sides, he commits one offense of wrong conduct. … where no harm will be done, but which lacks space on all sides, he commits one offense of wrong conduct. … where no harm will be done, and which has space on all sides, there is no offense. 

\subparagraph*{Departing without informing of the proper building procedure }

A\marginnote{3.6.1} monk appoints someone to build him a hut. He then departs without telling them to build one whose site has been approved, where no harm will be done, and which has space on all sides. If they build one whose site has not been approved, where harm will be done, and which lacks space on all sides, he commits one offense entailing suspension and two offenses of wrong conduct. … where harm will be done, but which has space on all sides, he commits one offense entailing suspension and one offense of wrong conduct. … where no harm will be done, but which lacks space on all sides, he commits one offense entailing suspension and one offense of wrong conduct. … where no harm will be done, and which has space on all sides, he commits one offense entailing suspension. 

A\marginnote{3.6.7} monk appoints someone to build him a hut. He then departs without telling them to build one whose site has been approved, where no harm will be done, and which has space on all sides. If they build one whose site has been approved, where harm will be done, and which lacks space on all sides, he commits two offenses of wrong conduct. … where harm will be done, but which has space on all sides, he commits one offense of wrong conduct. … where no harm will be done, but which lacks space on all sides, he commits one offense of wrong conduct. … where no harm will be done, and which has space on all sides, there is no offense. 

A\marginnote{3.7.1} monk appoints someone to build him a hut. He then departs without telling them to build one which is the right size, where no harm will be done, and which has space on all sides. If they build one which exceeds the right size, where harm will be done, and which lacks space on all sides, he commits one offense entailing suspension and two offenses of wrong conduct. … where harm will be done, but which has space on all sides, he commits one offense entailing suspension and one offense of wrong conduct. … where no harm will be done, but which lacks space on all sides, he commits one offense entailing suspension and one offense of wrong conduct. … where no harm will be done, and which has space on all sides, he commits one offense entailing suspension. 

A\marginnote{3.7.7} monk appoints someone to build him a hut. He then departs without telling them to build one which is the right size, where no harm will be done, and which has space on all sides. If they build one which is the right size, where harm will be done, and which lacks space on all sides, he commits two offenses of wrong conduct. … where harm will be done, but which has space on all sides, he commits one offense of wrong conduct. … where no harm will be done, but which lacks space on all sides, he commits one offense of wrong conduct. … where no harm will be done, and which has space on all sides, there is no offense. 

A\marginnote{3.8.1} monk appoints someone to build him a hut. He then departs without telling them to build one whose site has been approved, which is the right size, where no harm will be done, and which has space on all sides. If they build one whose site has not been approved, which exceeds the right size, where harm will be done, and which lacks space on all sides, he commits two offenses entailing suspension and two offenses of wrong conduct. … where harm will be done, but which has space on all sides, he commits two offenses entailing suspension and one offense of wrong conduct. … where no harm will be done, but which lacks space on all sides, he commits two offenses entailing suspension and one offense of wrong conduct. … where no harm will be done, and which has space on all sides, he commits two offenses entailing suspension. 

A\marginnote{3.8.7} monk appoints someone to build him a hut. He then departs without telling them to build one whose site has been approved, which is the right size, where no harm will be done, and which has space on all sides. If they build one whose site has been approved, which is the right size, where harm will be done, and which lacks space on all sides, he commits two offenses of wrong conduct. … where harm will be done, but which has space on all sides, he commits one offense of wrong conduct. … where no harm will be done, but which lacks space on all sides, he commits one offense of wrong conduct. … where no harm will be done, and which has space on all sides, there is no offense. 

\subparagraph*{Departing and then hearing about wrong building procedure }

A\marginnote{3.9.1} monk appoints someone to build him a hut. He then departs, telling them to build one whose site has been approved, where no harm will be done, and which has space on all sides, but they build one whose site has not been approved, where harm will be done, and which lacks space on all sides. If he hears about it, he must either go there himself or send a message, telling them to build one whose site has been approved, where no harm will be done, and which has space on all sides. If he neither goes himself nor sends a message, he commits an offense of wrong conduct. 

A\marginnote{3.9.8} monk appoints someone to build him a hut. He then departs, telling them to build one whose site has been approved, where no harm will be done, and which has space on all sides, but they build one whose site has not been approved, where harm will be done, but which has space on all sides. If he hears about it, he must either go there himself or send a message, telling them to build one whose site has been approved and where no harm will be done. If he neither goes himself nor sends a message, he commits an offense of wrong conduct. 

A\marginnote{3.9.15} monk appoints someone to build him a hut. He then departs, telling them to build one whose site has been approved, where no harm will be done, and which has space on all sides, but they build one whose site has not been approved, where no harm will be done, but which lacks space on all sides. If he hears about it, he must either go there himself or send a message, telling them to build one whose site has been approved and which has space on all sides. If he neither goes himself nor sends a message, he commits an offense of wrong conduct. 

A\marginnote{3.9.22} monk appoints someone to build him a hut. He then departs, telling them to build one whose site has been approved, where no harm will be done, and which has space on all sides, but they build one whose site has not been approved, where no harm will be done, and which has space on all sides. If he hears about it, he must either go there himself or send a message, telling them to build one whose site has been approved. If he neither goes himself nor sends a message, he commits an offense of wrong conduct. 

A\marginnote{3.9.29} monk appoints someone to build him a hut. He then departs, telling them to build one whose site has been approved, where no harm will be done, and which has space on all sides, but they build one whose site has been approved, where harm will be done, and which lacks space on all sides. If he hears about it, he must either go there himself or send a message, telling them to build one where no harm will be done and which has space on all sides. If he neither goes himself nor sends a message, he commits an offense of wrong conduct. 

A\marginnote{3.9.36} monk appoints someone to build him a hut. He then departs, telling them to build one whose site has been approved, where no harm will be done, and which has space on all sides, but they build one whose site has been approved, where harm will be done, but which has space on all sides. If he hears about it, he must either go there himself or send a message, telling them to build one where no harm will be done. If he neither goes himself nor sends a message, he commits an offense of wrong conduct. 

A\marginnote{3.9.43} monk appoints someone to build him a hut. He then departs, telling them to build one whose site has been approved, where no harm will be done, and which has space on all sides, but they build one whose site has been approved, where no harm will be done, but which lacks space on all sides. If he hears about it, he must either go there himself or send a message, telling them to build one which has space on all sides. If he neither goes himself nor sends a message, he commits an offense of wrong conduct. 

A\marginnote{3.9.50} monk appoints someone to build him a hut. He then departs, telling them to build one whose site has been approved, where no harm will be done, and which has space on all sides, and they do build one whose site has been approved, where no harm will be done, and which has space on all sides. There is no offense. 

A\marginnote{3.10.1} monk appoints someone to build him a hut. He then departs, telling them to build one which is the right size, where no harm will be done, and which has space on all sides, but they build one which exceeds the right size, where harm will be done, and which lacks space on all sides. If he hears about it, he must either go there himself or send a message, telling them to build one which is the right size, where no harm will be done, and which has space on all sides. … telling them to build one which is the right size and where no harm will be done. … telling them to build one which is the right size and which has space on all sides. … telling them to build one which is the right size. If he neither goes himself nor sends a message, he commits an offense of wrong conduct. 

A\marginnote{3.10.11} monk appoints someone to build him a hut. He then departs, telling them to build one which is the right size, where no harm will be done, and which has space on all sides, but they build one which is the right size, where harm will be done, and which lacks space on all sides. If he hears about it, he must either go there himself or send a message, telling them to build one where no harm will be done and which has space on all sides. … telling them to build one where no harm will be done. … telling them to build one which has space on all sides. … There is no offense. 

A\marginnote{3.11.1} monk appoints someone to build him a hut. He then departs, telling them to build one whose site has been approved, which is the right size, where no harm will be done, and which has space on all sides, but they build one whose site has not been approved, which exceeds the right size, where harm will be done, and which lacks space on all sides. If he hears about it, he must either go there himself or send a message, telling them to build one whose site has been approved, which is the right size, where no harm will be done, and which has space on all sides. … telling them to build one whose site has been approved, which is the right size, and where no harm will be done. … telling them to build one whose site has been approved, which is the right size, and which has space on all sides. … telling them to build one whose site has been approved and which is the right size. If he neither goes himself nor sends a message, he commits an offense of wrong conduct. 

A\marginnote{3.11.11} monk appoints someone to build him a hut. He then departs, telling them to build one whose site has been approved, which is the right size, where no harm will be done, and which has space on all sides, but they build one whose site has been approved, which is the right size, where harm will be done, and which lacks space on all sides. If he hears about it, he must either go there himself or send a message, telling them to build one where no harm will be done and which has space on all sides. … telling them to build one where no harm will be done. … telling them to build one which has space on all sides. … There is no offense. 

\subparagraph*{Offenses for appointed builders }

A\marginnote{3.12.1} monk appoints someone to build him a hut. He then departs, telling them to build one whose site has been approved, where no harm will be done, and which has space on all sides. If they build one whose site has not been approved, where harm will be done, and which lacks space on all sides, the builders commit three offenses of wrong conduct. … where harm will be done, but which has space on all sides, the builders commit two offenses of wrong conduct. … where no harm will be done, but which lacks space on all sides, the builders commit two offenses of wrong conduct. … where no harm will be done, and which has space on all sides, the builders commit one offense of wrong conduct. 

A\marginnote{3.12.7} monk appoints someone to build him a hut. He then departs, telling them to build one whose site has been approved, where no harm will be done, and which has space on all sides. If they build one whose site has been approved, where harm will be done, and which lacks space on all sides, the builders commit two offenses of wrong conduct. … where harm will be done, but which has space on all sides, the builders commit one offense of wrong conduct. … where no harm will be done, but which lacks space on all sides, the builders commit one offense of wrong conduct. … where no harm will be done, and which has space on all sides, there is no offense. 

A\marginnote{3.13.1} monk appoints someone to build him a hut. He then departs, telling them to build one which is the right size, where no harm will be done, and which has space on all sides. If they build one which exceeds the right size, where harm will be done, and which lacks space on all sides, the builders commit three offenses of wrong conduct. … where harm will be done, but which has space on all sides, the builders commit two offenses of wrong conduct. … where no harm will be done, but which lacks space on all sides, the builders commit two offenses of wrong conduct. … where no harm will be done, and which has space on all sides, the builders commit one offense of wrong conduct. 

A\marginnote{3.13.7} monk appoints someone to build him a hut. He then departs, telling them to build one which is the right size, where no harm will be done, and which has space on all sides. If they build one which is the right size, where harm will be done, and which lacks space on all sides, the builders commit two offenses of wrong conduct. … where harm will be done, but which has space on all sides, the builders commit one offense of wrong conduct. … where no harm will be done, but which lacks space on all sides, the builders commit one offense of wrong conduct. … where no harm will be done, and which has space on all sides, there is no offense. 

A\marginnote{3.13.13} monk appoints someone to build him a hut. He then departs, telling them to build one whose site has been approved, which is the right size, where no harm will be done, and which has space on all sides. If they build one whose site has not been approved, which exceeds the right size, where harm will be done, and which lacks space on all sides, the builders commit four offenses of wrong conduct. … where harm will be done, but which has space on all sides, the builders commit three offenses of wrong conduct. … where no harm will be done, but which lacks space on all sides, the builders commit three offenses of wrong conduct. … where no harm will be done, and which has space on all sides, the builders commit two offenses of wrong conduct. 

A\marginnote{3.13.19} monk appoints someone to build him a hut. He then departs, telling them to build one whose site has been approved, which is the right size, where no harm will be done, and which has space on all sides. If they build one whose site has been approved, which is the right size, where harm will be done, and which lacks space on all sides, the builders commit two offenses of wrong conduct. … where harm will be done, but which has space on all sides, the builders commit one offense of wrong conduct. … where no harm will be done, but which lacks space on all sides, the builders commit one offense of wrong conduct. … where no harm will be done, and which has space on all sides, there is no offense. 

\subparagraph*{Unfinished when he returns }

A\marginnote{3.14.1} monk appoints someone to build him a hut and then departs. They build one whose site has not been approved, where harm will be done, and which lacks space on all sides. If it is unfinished when he returns, that hut is to be given to someone else, or it is to be demolished and rebuilt. If he neither gives it to someone else, nor demolishes and rebuilds it, he commits one offense entailing suspension and two offenses of wrong conduct. 

A\marginnote{3.14.5} monk appoints someone to build him a hut and then departs. They build one whose site has not been approved, where harm will be done, but which has space on all sides. If it is unfinished when he returns, that hut is to be given to someone else, or it is to be demolished and rebuilt. If he neither gives it to someone else, nor demolishes and rebuilds it, he commits one offense entailing suspension and one offense of wrong conduct. … where no harm will be done, but which lacks space on all sides. … he commits one offense entailing suspension and one offense of wrong conduct. … where no harm will be done, and which has space on all sides. … he commits one offense entailing suspension. 

A\marginnote{3.14.12} monk appoints someone to build him a hut and then departs. They build one whose site has been approved, where harm will be done, and which lacks space on all sides. If it is unfinished when he returns, that hut is to be given to someone else, or it is to be demolished and rebuilt. If he neither gives it to someone else, nor demolishes and rebuilds it, he commits two offenses of wrong conduct. … where harm will be done, but which has space on all sides … he commits one offense of wrong conduct. … where no harm will be done, but which lacks space on all sides … he commits one offense of wrong conduct. … where no harm will be done, and which has space on all sides. There is no offense. 

A\marginnote{3.14.19} monk appoints someone to build him a hut and then departs. They build one which exceeds the right size, where harm will be done, and which lacks space on all sides. If it is unfinished when he returns, that hut is to be given to someone else, or it is to be demolished and rebuilt. If he neither gives it to someone else, nor demolishes and rebuilds it, he commits one offense entailing suspension and two offenses of wrong conduct. … where harm will be done, but which has space on all sides … he commits one offense entailing suspension and one offense of wrong conduct. … where no harm will be done, but which lacks space on all sides … he commits one offense entailing suspension and one offense of wrong conduct. … where no harm will be done, and which has space on all sides … he commits one offense entailing suspension. 

A\marginnote{3.14.26} monk appoints someone to build him a hut and then departs. They build one which is the right size, where harm will be done, and which lacks space on all sides. If it is unfinished when he returns, that hut is to be given to someone else, or it is to be demolished and rebuilt. If he neither gives it to someone else, nor demolishes and rebuilds it, he commits two offenses of wrong conduct. … where harm will be done, but which has space on all sides … he commits one offense of wrong conduct. … where no harm will be done, but which lacks space on all sides … he commits one offense of wrong conduct. … where no harm will be done, and which has space on all sides. There is no offense. 

A\marginnote{3.14.33} monk appoints someone to build him a hut and then departs. They build one whose site has not been approved, which exceeds the right size, where harm will be done, and which lacks space on all sides. If it is unfinished when he returns, that hut is to be given to someone else, or it is to be demolished and rebuilt. If he neither gives it to someone else, nor demolishes and rebuilds it, he commits two offenses entailing suspension and two offenses of wrong conduct. … where harm will be done, but which has space on all sides … he commits two offenses entailing suspension and one offense of wrong conduct. … where no harm will be done, but which lacks space on all sides … he commits two offenses entailing suspension and one offense of wrong conduct. … where no harm will be done, and which has space on all sides … he commits two offenses entailing suspension. 

A\marginnote{3.14.40} monk appoints someone to build him a hut and then departs. They build one whose site has been approved, which is the right size, where harm will be done, and which lacks space on all sides. If it is unfinished when he returns, that hut is to be given to someone else, or it is to be demolished and rebuilt. If he neither gives it to someone else, nor demolishes and rebuilds it, he commits two offenses of wrong conduct. … where harm will be done, but which has space on all sides … he commits one offense of wrong conduct. … where no harm will be done, but which lacks space on all sides … he commits one offense of wrong conduct. 

A\marginnote{3.14.46} monk appoints someone to build him a hut and then departs. They build one whose site has been approved, which is the right size, where no harm will be done, and which has space on all sides. There is no offense. 

\subsubsection*{Permutations part 2 }

If\marginnote{3.15.1} he finishes what he began himself, he commits an offense entailing suspension. 

If\marginnote{3.15.2} he has others finish what he began himself, he commits an offense entailing suspension. 

If\marginnote{3.15.3} he finishes himself what was begun by others, he commits an offense entailing suspension. 

If\marginnote{3.15.4} he has others finish what was begun by others, he commits an offense entailing suspension. 

\subsection*{Non-offenses }

There\marginnote{3.16.1} is no offense: if it is a shelter; if it is a cave;\footnote{For the rendering of \textit{\textsanskrit{guhā}} as “cave”, see Appendix of Technical Terms. } if it is a grass hut; if it is built for someone else; if it is anything apart from a dwelling; if he is insane; if he is the first offender. 

\scendsutta{The training rule on building huts, the sixth, is finished. }

%
\section*{{\suttatitleacronym Bu Ss 7}{\suttatitletranslation 7. The training rule on building dwellings }{\suttatitleroot Vihārakāra}}
\addcontentsline{toc}{section}{\tocacronym{Bu Ss 7} \toctranslation{7. The training rule on building dwellings } \tocroot{Vihārakāra}}
\markboth{7. The training rule on building dwellings }{Vihārakāra}
\extramarks{Bu Ss 7}{Bu Ss 7}

\subsection*{Origin story }

At\marginnote{1.1} one time when the Buddha was staying at \textsanskrit{Kosambī} in Ghosita’s Monastery, a householder who was a supporter of Venerable Channa said to him, “I’ll have a dwelling built for you, venerable, if you would find a site for it.” 

While\marginnote{1.4} Venerable Channa was clearing a site for that dwelling, he felled a tree that served as a shrine and was revered by village, town, district, and kingdom. People complained and criticized him, “How could the Sakyan monastics fell a tree that serves as a shrine and is revered by village, town, district, and kingdom? They are hurting one-sensed life.” 

The\marginnote{1.8} monks heard the criticism of those people, and the monks of few desires complained and criticized Venerable Channa in the same way. 

After\marginnote{1.11} rebuking Venerable Channa in many ways, they told the Buddha. Soon afterwards he had the Sangha gathered and questioned Channa: “Is it true, Channa, that you did this?” 

“It’s\marginnote{1.13} true, sir.” 

The\marginnote{1.14} Buddha rebuked him … “Foolish man, how could you do this? People perceive trees as conscious. This will affect people’s confidence …” … “And, monks, this training rule should be recited like this: 

\subsection*{Final ruling }

\scrule{‘When a monk builds a large dwelling with a sponsoring owner and intended for himself,\footnote{I have rendered \textit{\textsanskrit{sassāmikaṁ}} as “with a sponsoring owner”. The word \textit{\textsanskrit{sāmika}} generally means “owner”, and there seems to be no reason why that should not also be the meaning here. However, although the monk does not own the dwelling, he is still the builder and the intended user once the building is finished. To indicate this, I have added the word “sponsoring”. } he must have monks approve a site where no harm will be done and which has space on all sides. If a monk builds a large dwelling on a site where harm will be done and which lacks space on all sides, or he does not have monks approve the site, he commits an offense entailing suspension.’” }

\subsection*{Definitions }

\begin{description}%
\item[A large dwelling: ] one with a sponsoring owner is what is meant.\footnote{I render \textit{\textsanskrit{vihāra}} as “dwelling”, the idea that it is a monastic dwelling being implied. In later usage, especially in the commentaries, \textit{\textsanskrit{vihāra}} comes to refer to entire monasteries, rather than individual dwellings. The commentaries seem to agree that in its early usage the word refers to a dwelling. Sp 1.493: \textit{\textsanskrit{Vihāro} nivesanasadiso}, “A \textit{\textsanskrit{vihāra}} is like a house.” } %
\item[Dwelling: ] plastered inside or plastered outside or plastered both inside and outside. %
\item[Builds: ] building it himself or having it built. %
\item[With a sponsoring owner: ] there is another owner, either a woman or a man, either a lay person or one gone forth. %
\item[Intended for himself: ] for his own use. %
\item[He must have monks approve a site: ] the\marginnote{2.12} monk who wants to build a dwelling should clear a site. He should then approach the Sangha, put his upper robe over one shoulder, pay respect at the feet of the senior monks, squat on his heels, raise his joined palms, and say: 

“Venerables,\marginnote{2.13} I want to build a large dwelling with a sponsoring owner and intended for myself. I request the Sangha to inspect the site for the dwelling.” 

He\marginnote{2.15} should make his request a second and a third time. If the whole Sangha is able to inspect the site, they should all go. If the whole Sangha is unable to inspect the site, then those monks there who are competent and capable—who know where harm will be done and where no harm will be done, who know what is meant by space on all sides and a lack of space on all sides—should be asked and then appointed. 

“And,\marginnote{2.20} monks, they should be appointed like this. A competent and capable monk should inform the Sangha: 

‘Please,\marginnote{2.22} venerables, I ask the Sangha to listen. Monk so-and-so wants to build a large dwelling with a sponsoring owner and intended for himself. He is requesting the Sangha to inspect the site for that dwelling. If the Sangha is ready, it should appoint monk so-and-so and monk so-and-so to inspect the site for the dwelling of monk-so-and-so. This is the motion. 

Please,\marginnote{2.27} venerables, I ask the Sangha to listen. Monk so-and-so wants to build a large dwelling with a sponsoring owner and intended for himself. He is requesting the Sangha to inspect the site for that dwelling. The Sangha appoints monk so-and-so and monk so-and-so to inspect the site for the dwelling of monk-so-and-so. Any monk who approves of appointing monk so-and-so and monk so-and-so to inspect the site for the dwelling of monk-so-and-so should remain silent. Any monk who doesn’t approve should speak up. 

The\marginnote{2.33} Sangha has appointed monk so-and-so and monk so-and-so to inspect the site for the dwelling of monk so-and-so. The Sangha approves and is therefore silent. I’ll remember it thus.’ 

The\marginnote{2.36} appointed monks should go and inspect the site for the dwelling to find out if any harm will be done and if it has space on all sides. If harm will be done or it lacks space on all sides, they should say, ‘Don’t build here.’ If no harm will be done and it has space on all sides, they should inform the Sangha: ‘No harm will be done and it has space on all sides.’ The monk who wants to build the dwelling should then approach the Sangha, arrange his upper robe over one shoulder, pay respect at the feet of the senior monks, squat on his heels, raise his joined palms, and say: 

‘Venerables,\marginnote{2.43} I want to build a large dwelling with a sponsoring owner and intended for myself. I request the Sangha to approve the site for the dwelling.’ 

He\marginnote{2.45} should make his request a second and a third time. A competent and capable monk should then inform the Sangha: 

‘Please,\marginnote{2.48} venerables, I ask the Sangha to listen. Monk so-and-so wants to build a large dwelling with a sponsoring owner and intended for himself. He is requesting the Sangha to approve the site for that dwelling. If the Sangha is ready, it should approve the site for the dwelling of monk-so-and-so. This is the motion. 

Please,\marginnote{2.53} venerables, I ask the Sangha to listen. Monk so-and-so wants to build a large dwelling with a sponsoring owner and intended for himself. He is requesting the Sangha to approve the site for that dwelling. The Sangha approves the site for the dwelling of monk so-and-so. Any monk who agrees to approving the site for the dwelling of monk-so-and-so should remain silent. Any monk who doesn’t agree should speak up. 

The\marginnote{2.59} Sangha has approved the site for the dwelling of monk so-and-so. The Sangha approves and is therefore silent. I’ll remember it thus.’” 

%
\item[Where harm will be done: ] it is the abode of ants, termites, rats, snakes, scorpions, centipedes, elephants, horses, lions, tigers, leopards, bears, or hyenas, or any other animal; or it is bordering on a field of grain, a field of vegetables, a place of slaughter, a place of execution, a charnel ground, a park, a royal property, an elephant stable, a horse stable, a prison, a bar, a slaughterhouse, a street, a crossroads, a public meeting hall, or a cul-de-sac—this is called “where harm will be done”. %
\item[Which lacks space on all sides: ] it is not possible to go around it with a yoked cart, or to go all the way around it with a ladder—this is called “which lacks space on all sides”. %
\item[Where no harm will be done: ] it is not the abode of ants … it is not bordering on … a cul-de-sac—this is called “where no harm will be done”. %
\item[Which has space on all sides: ] it is possible to go around it with a yoked cart, or to go all the way around it with a ladder—this is called “which has space on all sides”. %
\item[A large dwelling: ] one with a sponsoring owner is what is meant. %
\item[Dwelling: ] plastered inside or plastered outside or plastered both inside and outside. %
\item[Builds: ] building it himself or having it built. %
\item[Or he does not have monks approve the site: ] if the site has not been approved through a legal procedure consisting of one motion and three announcements, and he then builds a dwelling or has one built, then for the effort there is an act of wrong conduct. When there is one piece left to complete the dwelling, he commits a serious offense. When the last piece is finished, he commits an offense entailing suspension. %
\item[He commits an offense entailing suspension: ] … Therefore, too, it is called “an offense entailing suspension”. %
\end{description}

\subsection*{Permutations }

\subsubsection*{Permutations part 1 }

\subparagraph*{Building oneself }

If\marginnote{3.1.1} a monk builds a dwelling whose site has not been approved, where harm will be done, and which lacks space on all sides, he commits one offense entailing suspension and two offenses of wrong conduct. If a monk builds a dwelling whose site has not been approved, where harm will be done, but which has space on all sides, he commits one offense entailing suspension and one offense of wrong conduct. If a monk builds a dwelling whose site has not been approved, where no harm will be done, but which lacks space on all sides, he commits one offense entailing suspension and one offense of wrong conduct. If a monk builds a dwelling whose site has not been approved, where no harm will be done, and which has space on all sides, he commits one offense entailing suspension. 

If\marginnote{3.1.5} a monk builds a dwelling whose site has been approved, where harm will be done, and which lacks space on all sides, he commits two offenses of wrong conduct. If a monk builds a dwelling whose site has been approved, where harm will be done, but which has space on all sides, he commits one offense of wrong conduct. If a monk builds a dwelling whose site has been approved, where no harm will be done, but which lacks space on all sides, he commits one offense of wrong conduct. If a monk builds a dwelling whose site has been approved, where no harm will be done, and which has space on all sides, there is no offense. 

\subparagraph*{Appointing someone else to build }

A\marginnote{3.2.1} monk appoints someone to build him a dwelling. If they build a dwelling whose site has not been approved, where harm will be done, and which lacks space on all sides, he commits one offense entailing suspension and two offenses of wrong conduct. … where harm will be done, but which has space on all sides, he commits one offense entailing suspension and one offense of wrong conduct. … where no harm will be done, but which lacks space on all sides, he commits one offense entailing suspension and one offense of wrong conduct. … where no harm will be done, and which has space on all sides, he commits one offense entailing suspension. 

A\marginnote{3.2.7} monk appoints someone to build him a dwelling. If they build a dwelling whose site has been approved, where harm will be done, and which lacks space on all sides, he commits two offenses of wrong conduct. … where harm will be done, but which has space on all sides, he commits one offense of wrong conduct. … where no harm will be done, but which lacks space on all sides, he commits one offense of wrong conduct. … where no harm will be done, and which has space on all sides, there is no offense. 

\subparagraph*{Departing without informing of the proper building procedure }

A\marginnote{3.3.1} monk appoints someone to build him a dwelling and then departs, but he does not tell them to build a dwelling whose site has been approved, where no harm will be done, and which has space on all sides. If they build a dwelling whose site has not been approved, where harm will be done, and which lacks space on all sides, he commits one offense entailing suspension and two offenses of wrong conduct. … where harm will be done, but which has space on all sides, he commits one offense entailing suspension and one offense of wrong conduct. … where no harm will be done, but which lacks space on all sides, he commits one offense entailing suspension and one offense of wrong conduct. … where no harm will be done, and which has space on all sides, he commits one offense entailing suspension. 

A\marginnote{3.3.9} monk appoints someone to build him a dwelling and then departs, but he does not tell them to build a dwelling whose site has been approved, where no harm will be done, and which has space on all sides. If they build a dwelling whose site has been approved, where harm will be done, and which lacks space on all sides, he commits two offenses of wrong conduct. … where harm will be done, but which has space on all sides, he commits one offense of wrong conduct. … where no harm will be done, but which lacks space on all sides, he commits one offense of wrong conduct. … where no harm will be done, and which has space on all sides, there is no offense. 

\subparagraph*{Departing and then hearing about wrong building procedure }

A\marginnote{3.4.1} monk appoints someone to build him a dwelling and then departs. He tells them to build a dwelling whose site has been approved, where no harm will be done, and which has space on all sides, but they build a dwelling whose site has not been approved, where harm will be done, and which lacks space on all sides. If he hears about it, he must either go there himself or send a message, telling them to build a dwelling whose site has been approved, where no harm will be done, and which has space on all sides. … whose site has been approved and where no harm will be done. … whose site has been approved and which has space on all sides. … whose site has been approved. If he neither goes himself nor sends a message, he commits an offense of wrong conduct. 

A\marginnote{3.4.13} monk appoints someone to build him a dwelling and then departs. He tells them to build a dwelling whose site has been approved, where no harm will be done, and which has space on all sides, but they build a dwelling whose site has been approved, where harm will be done, and which lacks space on all sides. If he hears about it, he must either go there himself or send a message, telling them to build one where no harm will be done and which has space on all sides. … (To be expanded as in \href{https://suttacentral.net/pli-tv-bu-vb-ss6\#3.9.35}{Bu Ss 6:3.9.35}–Bu Ss 6:3.11.16.) …  where no harm will be done. …\footnote{These ellipses points, and those in the next segment, are not found in the Pali. I have added them for clarity. } which has space on all sides. … There is no offense. 

\subparagraph*{Offenses for appointed builders }

A\marginnote{3.5.1} monk appoints someone to build him a dwelling. He then departs, telling them to build a dwelling whose site has been approved, where no harm will be done, and which has space on all sides. If they build a dwelling whose site has not been approved, where harm will be done, and which lacks space on all sides, the builders commit three offenses of wrong conduct. … where harm will be done, but which has space on all sides, the builders commit two offenses of wrong conduct. … where no harm will be done, but which lacks space on all sides, the builders commit two offenses of wrong conduct. … where no harm will be done, and which has space on all sides, the builders commit one offense of wrong conduct. 

A\marginnote{3.5.7} monk appoints someone to build him a dwelling. He then departs, telling them to build a dwelling whose site has been approved, where no harm will be done, and which has space on all sides. If they build a dwelling whose site has been approved, where harm will be done, and which lacks space on all sides, the builders commit two offenses of wrong conduct. … where harm will be done, but which has space on all sides, the builders commit one offense of wrong conduct. … where no harm will be done, but which lacks space on all sides, the builders commit one offense of wrong conduct. … where no harm will be done, and which has space on all sides, there is no offense. 

\subparagraph*{Unfinished when he returns }

A\marginnote{3.6.1} monk appoints someone to build him a dwelling and then departs. They build a dwelling whose site has not been approved, where harm will be done, and which lacks space on all sides. If it is unfinished when he returns, that dwelling is to be given to someone else, or it is to be demolished and rebuilt. If he neither gives it to someone else, nor demolishes and rebuilds it, he commits one offense entailing suspension and two offenses of wrong conduct. … where harm will be done, but which has space on all sides … he commits one offense entailing suspension and one offense of wrong conduct. … where no harm will be done, but which lacks space on all sides … he commits one offense entailing suspension and one offense of wrong conduct. … where no harm will be done, and which has space on all sides … he commits one offense entailing suspension. 

A\marginnote{3.6.8} monk appoints someone to build him a dwelling and then departs. They build a dwelling whose site has been approved, where harm will be done, and which lacks space on all sides. If it is unfinished when he returns, that dwelling is to be given to someone else, or it is to be demolished and rebuilt. If he neither gives it to someone else, nor demolishes and rebuilds it, he commits two offenses of wrong conduct. … where harm will be done, but which has space on all sides … he commits one offense of wrong conduct. … where no harm will be done, but which lacks space on all sides … he commits one offense of wrong conduct. … where no harm will be done, and which has space on all sides … there is no offense. 

\subsubsection*{Permutations part 2 }

If\marginnote{3.7.1} he finishes what he began himself, he commits an offense entailing suspension. 

If\marginnote{3.7.2} he has others finish what he began himself, he commits an offense entailing suspension. 

If\marginnote{3.7.3} he finishes himself what was begun by others, he commits an offense entailing suspension. 

If\marginnote{3.7.4} he has others finish what was begun by others, he commits an offense entailing suspension. 

\subsection*{Non-offenses }

There\marginnote{3.7.5.1} is no offense: if it is a shelter, a cave, or a grass hut;\footnote{For the rendering of \textit{\textsanskrit{guhā}} as “cave”, see Appendix of Technical Terms. } if it is built for someone else; if it is anything apart from a dwelling; if he is insane; if he is the first offender. 

\scendsutta{The training rule on building dwellings, the seventh, is finished. }

%
\section*{{\suttatitleacronym Bu Ss 8}{\suttatitletranslation 8. The training rule on anger }{\suttatitleroot Duṭṭhadosa}}
\addcontentsline{toc}{section}{\tocacronym{Bu Ss 8} \toctranslation{8. The training rule on anger } \tocroot{Duṭṭhadosa}}
\markboth{8. The training rule on anger }{Duṭṭhadosa}
\extramarks{Bu Ss 8}{Bu Ss 8}

\subsection*{Origin story }

At\marginnote{1.1.1} one time when the Buddha was staying at \textsanskrit{Rājagaha} in the Bamboo Grove, Venerable Dabba the Mallian realized perfection at the age of seven. He had achieved all there is to achieve by a disciple and had nothing further to do. Then, while reflecting in private, he thought, “How can I be of service to the Sangha? 

Why\marginnote{1.1.10} don’t I assign the dwellings and designate the meals?” 

In\marginnote{1.2.1} the evening Dabba came out of seclusion and went to the Buddha. He bowed, sat down, and said, “Sir, while I was reflecting in private, it occurred to me that I’ve achieved all there is to achieve by a disciple, and I was wondering how I could be of service to the Sangha. I thought, ‘Why don’t I assign the dwellings and designate the meals?’” 

“Good,\marginnote{1.2.7} good, Dabba, please do so.” 

“Yes.”\marginnote{1.2.9} 

Soon\marginnote{1.3.1} afterwards the Buddha gave a teaching and addressed the monks: “Monks, the Sangha should appoint Dabba the Mallian as the assigner of dwellings and the designator of meals. And he should be appointed like this. First Dabba should be asked. A competent and capable monk should then inform the Sangha: 

‘Please,\marginnote{1.3.6} venerables, I ask the Sangha to listen. If the Sangha is ready, it should appoint Venerable Dabba the Mallian as assigner of dwellings and designator of meals. This is the motion. 

Please,\marginnote{1.3.9} venerables, I ask the Sangha to listen. The Sangha appoints Venerable Dabba the Mallian as assigner of dwellings and designator of meals. Any monk who approves of appointing Venerable Dabba as assigner of dwellings and designator of meals should remain silent. Any monk who doesn’t approve should speak up. 

The\marginnote{1.3.13} Sangha has appointed Venerable Dabba the Mallian as assigner of dwellings and designator of meals. The Sangha approves and is therefore silent. I’ll remember it thus.’” 

Dabba\marginnote{1.4.1} assigned dwellings to the monks according to their character. He assigned dwellings in the same place to those monks who were experts on the discourses, thinking, “They’ll recite the discourses to one another.” And he did likewise for the experts on the Monastic Law, thinking, “They’ll discuss the Monastic Law;” for the expounders of the Teaching, thinking, “They’ll discuss the Teaching;” for the meditators, thinking, “They won’t disturb each other;” and for the gossips and the body-builders, thinking, “In this way even these venerables will be happy.” 

When\marginnote{1.4.12} monks arrived at night, he entered the fire element and assigned dwellings with the help of that light. Monks even arrived late on purpose, hoping to see the marvel of Dabba’s supernormal powers. 

They\marginnote{1.4.15} would approach Dabba and say, “Venerable Dabba, please assign us a dwelling.” 

“Where\marginnote{1.4.17} would you like to stay?” 

They\marginnote{1.4.19} would intentionally suggest somewhere far away: “On the Vulture Peak,” “At Robbers’ Cliff,” “On Black Rock on the slope of Mount Isigili,” “In the \textsanskrit{Sattapaṇṇi} Cave on the slope of Mount \textsanskrit{Vebhāra},” “In Cool Grove on the hill at the Snake’s Pool,” “At Gotamaka Gorge,” “At Tinduka Gorge,” “At Tapoda Gorge,” “In Tapoda Park,” “In \textsanskrit{Jīvaka}’s Mango Grove,” “In the deer park at Maddakucchi.” 

Dabba\marginnote{1.4.31} then entered the fire element, and with his finger glowing, he walked in front of those monks. They followed behind him with the help of that light. And he would assign them dwellings: “This is the bed, this the bench, this the mattress, this the pillow, this the place for defecating, this the place for urinating, this the water for drinking, this the water for washing, this the walking stick; these are the Sangha’s agreements concerning the right time to enter and the right time to leave.”\footnote{See \href{https://suttacentral.net/pli-tv-kd18/en/brahmali\#1.2.22}{Kd 18:1.2.22} for the correct interpretation of this line. } Dabba then returned to the Bamboo Grove. 

At\marginnote{1.5.1} that time the monks Mettiya and \textsanskrit{Bhūmajaka} were only newly ordained. They had little merit,\footnote{\textit{\textsanskrit{Mettiyabhūmajakā} \textsanskrit{bhikkhū}} can be read either as referring to two monks, Mettiya and \textsanskrit{Bhūmajaka}, or as a group of monks led by these two. I have not been able to find any clear evidence that it refers to a group, and so I prefer the more straightforward reading that it only refers to the two monks. } getting inferior dwellings and meals. The people of \textsanskrit{Rājagaha} were keen to give specially prepared almsfood to the senior monks—ghee, oil, and special curries—but to the monks Mettiya and \textsanskrit{Bhūmajaka} they gave ordinary food of porridge and broken rice. 

When\marginnote{1.5.5} they had eaten their meal and returned from almsround, they asked the senior monks, “What did you get at the dining hall?” 

Some\marginnote{1.5.7} said, “We got ghee, oil, and special curries.” 

But\marginnote{1.5.9} the monks Mettiya and \textsanskrit{Bhūmajaka} said, “We didn’t get anything except ordinary food of porridge and broken rice.” 

At\marginnote{1.6.1} that time there was a householder who gave a regular meal of fine food to four monks. He made his offering in the dining hall together with his wives and children. Some of them offered rice, some bean curry, some oil, and some special curries. 

On\marginnote{1.6.4} one occasion the meal to be given by this householder on the following day had been designated to the monks Mettiya and \textsanskrit{Bhūmajaka}. Just then that householder went to the monastery on some business. He approached Dabba, bowed, and sat down. And Dabba instructed, inspired, and gladdened him with a teaching. After the talk, he asked Dabba, “Sir, who has been designated to receive tomorrow’s meal in our house?” 

“Mettiya\marginnote{1.6.10} and \textsanskrit{Bhūmajaka}.” 

He\marginnote{1.6.11} was disappointed, and thought, “Why should bad monks eat in our house?” After returning to his house, he told a female slave, “For those who are coming for tomorrow’s meal, prepare seats in the gatehouse and serve them broken rice and porridge.”\footnote{“Gatehouse” renders \textit{\textsanskrit{koṭṭhaka}}. See Appendix of Technical Terms for discussion. } 

“Yes,\marginnote{1.6.15} sir.” 

The\marginnote{1.7.1} monks Mettiya and \textsanskrit{Bhūmajaka} said to each other, “Yesterday we were designated a meal from that householder who offers fine food. Tomorrow he’ll serve us together with his wives and children. Some of them will offer us rice, some bean curry, some oil, and some special curries.” And because they were excited, they did not sleep properly that night. 

The\marginnote{1.7.6} following morning they robed up, took their bowls and robes, and went to the house of that householder. When the female slave saw them coming, she prepared seats in the gatehouse and said to them, “Please sit, venerables.” 

They\marginnote{1.7.9} thought, “The meal can’t be ready, since we’re given seats in the gatehouse.” 

She\marginnote{1.7.11} then brought them broken rice and porridge, and said, “Eat, sirs.” 

“But,\marginnote{1.7.13} Sister, we’ve come for the regular meal.” 

“I\marginnote{1.7.14} know. But yesterday I was told by the head of the household to serve you like this. Please eat.” 

They\marginnote{1.7.17} said to each other, “Yesterday this householder came to the monastery and spoke with Dabba. Dabba must be responsible for this split between the householder and us.” And because they were dejected, they did not eat as much as they had intended. When they had eaten their meal and returned from almsround, they put their bowls and robes away and squatted on their heels outside the monastery gatehouse, using their upper robes as back-and-knee straps. They were silent and humiliated, their shoulders drooping and their heads down, glum and speechless.\footnote{For the rendering of \textit{\textsanskrit{saṅghāṭi}} as “upper robe”, see Appendix of Technical Terms. } 

Just\marginnote{1.8.1} then the nun \textsanskrit{Mettiyā} came to them and said, “My respectful greetings to you, venerables.” But they did not respond. A second time and a third time she said the same thing, but they still did not respond. 

“Have\marginnote{1.8.8} I done something wrong? Why don’t you respond?” 

“It’s\marginnote{1.8.10} because we’ve been badly treated by Dabba the Mallian, and you’re not taking an interest.” 

“But\marginnote{1.8.11} what can I do?” 

“If\marginnote{1.8.12} you like, you could make the Buddha expel Dabba.” 

“And\marginnote{1.8.13} how can I do that?” 

“Go\marginnote{1.8.14} to the Buddha and say, ‘Sir, this isn’t proper or appropriate. There’s fear, distress, and oppression in this district, where none of these should exist. From where one would expect security, there’s insecurity. It’s as if water is burning. Venerable Dabba the Mallian has raped me.’”\footnote{“Rapist” renders \textit{\textsanskrit{dūsita}}. See \textit{\textsanskrit{dūseti}} in Appendix of Technical Terms for discussion. } 

Saying,\marginnote{1.8.20} “Alright, venerables,” she went to the Buddha, bowed, and then repeated what she had been told to say. 

Soon\marginnote{1.9.1} afterwards the Buddha had the Sangha gathered and questioned Dabba: “Dabba, do you remember doing as the nun \textsanskrit{Mettiyā} says?” 

“Sir,\marginnote{1.9.3} you know what I’m like.” 

A\marginnote{1.9.4} second and a third time the Buddha asked the same question and got the same response. He then said, “Dabba, the Dabbas don’t give such evasive answers. If it was done by you, say so; if it wasn’t, then say that.” 

“Since\marginnote{1.9.11} I was born, sir, I don’t recall having sexual intercourse even in a dream, let alone when awake.” 

The\marginnote{1.9.12} Buddha addressed the monks: “Well then, monks, expel the nun \textsanskrit{Mettiyā},\footnote{“Expel” renders \textit{\textsanskrit{nāsetha}}. For a discussion of the verb \textit{\textsanskrit{nāseti}}, see Appendix of Technical Terms. } and call these monks to account.” The Buddha then got up from his seat and entered his dwelling. 

When\marginnote{1.9.16} the monks had expelled the nun \textsanskrit{Mettiyā}, the monks Mettiya and \textsanskrit{Bhūmajaka} said to them, “Don’t expel the nun \textsanskrit{Mettiyā}; she’s done nothing wrong. She was urged on by us. We were angry and displeased, and trying to get Dabba to leave the monastic life.” 

“But\marginnote{1.9.21} did you groundlessly charge Venerable Dabba with an offense entailing expulsion?” 

“Yes.”\marginnote{1.9.22} 

The\marginnote{1.9.23} monks of few desires complained and criticized them, “How could the monks Mettiya and \textsanskrit{Bhūmajaka} groundlessly charge Venerable Dabba with an offense entailing expulsion?” 

They\marginnote{1.9.25} rebuked those monks in many ways and then told the Buddha. Soon afterwards he had the Sangha gathered and questioned those monks: “Is it true, monks, that you did this?” 

“It’s\marginnote{1.9.27} true, sir.” 

The\marginnote{1.9.28} Buddha rebuked them … “Foolish men, how could you do this? This will affect people’s confidence …” … “And, monks, this training rule should be recited like this: 

\subsection*{Final ruling }

\scrule{‘If a monk who is angry and displeased groundlessly charges a monk with an offense entailing expulsion, aiming to make him leave the monastic life, and then after some time, whether he is questioned or not, it is clear that the legal issue is groundless, and he admits to his ill will, he commits an offense entailing suspension.’” }

\subsection*{Definitions }

\begin{description}%
\item[A: ] whoever … %
\item[Monk: ] … The monk who has been given the full ordination by a unanimous Sangha through a legal procedure consisting of one motion and three announcements that is irreversible and fit to stand—this sort of monk is meant in this case. %
\item[A monk: ] another monk. %
\item[Angry: ] upset, dissatisfied, discontent, having hatred, hostile. %
\item[Displeased: ] because of that upset, that ill will, that dissatisfaction, and that discontent, he is displeased. %
\item[Groundlessly: ] not seen, not heard, not suspected. %
\item[With an offense entailing expulsion: ] with one of the four. %
\item[Charges: ] accuses him or has him accused. %
\item[To make him leave the monastic life: ] to make him leave the monkhood, leave the state of a monastic, leave his morality, leave the virtue of monasticism. %
\item[And then after some time: ] the moment, the instant, the second after he has laid the charge. %
\item[He is questioned: ] he is questioned about the grounds of his charge. %
\item[Not: ] he is not spoken to by anyone. %
\item[The legal issue: ] there are four kinds of legal issues: legal issues arising from disputes, legal issues arising from accusations, legal issues arising from offenses, legal issues arising from business. %
\item[And he admits to his ill will: ] “What I said was empty,” “What I said was false,” “What I said was unreal,” “I said it without knowing.” %
\item[He commits an offense entailing suspension: ] … Therefore, too, it is called “an offense entailing suspension”. %
\end{description}

\subsection*{Permutations }

\subsubsection*{Permutations part 1 }

\subparagraph*{Doing the accusing oneself }

Although\marginnote{3.1.1} he has not seen it, he accuses someone of having committed an offense entailing expulsion: “I’ve seen that you’ve committed an offense entailing expulsion. You’re not an ascetic, not a Sakyan monastic. You’re excluded from the observance-day ceremony, from the invitation ceremony, and from the legal procedures of the Sangha.”\footnote{It is interesting to note that both \textit{uposatha} and \textit{\textsanskrit{pavāraṇā}} are here used as distinct from \textit{\textsanskrit{saṅghakamma}}, suggesting neither was regarded as \textit{\textsanskrit{saṅghakamma}} proper. See Appendix of Technical Terms for a further discussion of these two words. } For each statement, he commits an offense entailing suspension. 

Although\marginnote{3.1.6} he has not heard it, he accuses someone of having committed an offense entailing expulsion: “I’ve heard that you’ve committed an offense entailing expulsion. You’re not an ascetic, not a Sakyan monastic. You’re excluded from the observance-day ceremony, from the invitation ceremony, and from the legal procedures of the Sangha.” For each statement, he commits an offense entailing suspension. 

Although\marginnote{3.1.12} he does not suspect it, he accuses someone of having committed an offense entailing expulsion: “I suspect that you’ve committed an offense entailing expulsion. You’re not an ascetic, not a Sakyan monastic. You’re excluded from the observance-day ceremony, from the invitation ceremony, and from the legal procedures of the Sangha.” For each statement, he commits an offense entailing suspension. 

Although\marginnote{3.2.1} he has not seen it, he accuses someone of having committed an offense entailing expulsion: “I’ve seen and I’ve heard that you’ve committed an offense entailing expulsion. You’re not an ascetic …” For each statement, he commits an offense entailing suspension. 

Although\marginnote{3.2.5} he has not seen it, he accuses someone of having committed an offense entailing expulsion: “I’ve seen and I suspect that you’ve committed an offense entailing expulsion. You’re not an ascetic …” For each statement, he commits an offense entailing suspension. 

Although\marginnote{3.2.9} he has not seen it, he accuses someone of having committed an offense entailing expulsion: “I’ve seen and I’ve heard and I suspect that you’ve committed an offense entailing expulsion. You’re not an ascetic …” For each statement, he commits an offense entailing suspension. 

Although\marginnote{3.2.13} he has not heard it, he accuses someone of having committed an offense entailing expulsion: “I’ve heard and I suspect …” … “I’ve heard and I’ve seen …” … “I’ve heard and I suspect and I’ve seen that you’ve committed an offense entailing expulsion. You’re not an ascetic …” For each statement, he commits an offense entailing suspension. 

Although\marginnote{3.2.20} he does not suspect it, he accuses someone of having committed an offense entailing expulsion: “I suspect and I’ve seen …” … “I suspect and I’ve heard …” … “I suspect and I’ve seen and I’ve heard that you’ve committed an offense entailing expulsion. You’re not an ascetic …” For each statement, he commits an offense entailing suspension. 

He\marginnote{3.3.1} has seen that someone has committed an offense entailing expulsion, but he accuses him like this: “I’ve heard that you’ve committed an offense entailing expulsion. You’re not an ascetic …” For each statement, he commits an offense entailing suspension. 

He\marginnote{3.3.6} has seen that someone has committed an offense entailing expulsion, but he accuses him like this: “I suspect that you’ve committed an offense entailing expulsion …” … “I’ve heard and I suspect that you’ve committed an offense entailing expulsion. You’re not an ascetic …” For each statement, he commits an offense entailing suspension. 

He\marginnote{3.3.12} has heard that someone has committed an offense entailing expulsion, but he accuses him like this: “I suspect that you’ve committed an offense entailing expulsion …” … “I’ve seen that you’ve committed an offense entailing expulsion …” … “I suspect and I’ve seen that you’ve committed an offense entailing expulsion. You’re not an ascetic …” For each statement, he commits an offense entailing suspension. 

He\marginnote{3.3.20} suspects that someone has committing an offense entailing expulsion, but he accuses him like this: “I’ve seen that you’ve committed an offense entailing expulsion …” … “I’ve heard that you’ve committed an offense entailing expulsion …” … “I’ve seen and I’ve heard that you’ve committed an offense entailing expulsion. You’re not an ascetic, not a Sakyan monastic. …” For each statement, he commits an offense entailing suspension. 

He\marginnote{3.4.1} has seen someone committing an offense entailing expulsion, but he is unsure of what he has seen, he does not believe what he has seen, he does not remember what he has seen, he is confused about what he has seen …\footnote{Sp 1.387: \textit{No \textsanskrit{kappetīti} na saddahati}, “\textit{No kappeti}: he does not believe.” } he is unsure of what he has heard, he does not believe what he has heard, he does not remember what he has heard, he is confused about what he has heard … he is unsure of what he suspects, he does not believe what he suspects, he does not remember what he suspects, he is confused about what he suspects. If he then accuses him like this: “I suspect and I’ve seen …” … “I suspect and I’ve heard …” … “I suspect and I’ve seen and I’ve heard that you’ve committed an offense entailing expulsion. You’re not an ascetic, not a Sakyan monastic. You’re excluded from the observance-day ceremony, from the invitation ceremony, and from the legal procedures of the Sangha.” For each statement, he commits an offense entailing suspension. 

\subparagraph*{Getting someone else to do the accusing }

Although\marginnote{3.5.1} he has not seen it, he has someone accused of having committed an offense entailing expulsion: “You’ve been seen. You’ve committed an offense entailing expulsion. You’re not an ascetic, not a Sakyan monastic. You’re excluded from the observance-day ceremony, from the invitation ceremony, and from the legal procedures of the Sangha.” For each statement, he commits an offense entailing suspension. 

Although\marginnote{3.5.7} he has not heard it … Although he does not suspect it, he has someone accused of having committed an offense entailing expulsion: “You’re suspected. You’ve committed an offense entailing expulsion. …” For each statement, he commits an offense entailing suspension. 

Although\marginnote{3.6.1} he has not seen it, he has someone accused of having committed an offense entailing expulsion: “You’ve been seen and you’ve been heard …” … “You’ve been seen and you’re suspected …” … “You’ve been seen and you’ve been heard and you’re suspected. You’ve committed an offense entailing expulsion …” … Although he has not heard it … Although he does not suspect it, he has someone accused of having committed an offense entailing expulsion: “You’re suspected and you’ve been seen …” … “You’re suspected and you’ve been heard …” … “You’re suspected and you’ve been seen and you’ve been heard. You’ve committed an offense entailing expulsion. You’re not an ascetic …” For each statement, he commits an offense entailing suspension. 

He\marginnote{3.7.1} has seen that someone has committed an offense entailing expulsion, but he has him accused like this: “You’ve been heard …” … but he has him accused like this: “You’re suspected …” … but he has him accused like this: “You’ve been heard and you’re suspected. You’ve committed an offense entailing expulsion. You’re not an ascetic …” For each statement, he commits an offense entailing suspension. 

He\marginnote{3.7.11} has heard that someone has committed an offense entailing expulsion … He suspects that someone has committed an offense entailing expulsion, but he has him accused like this: “You’ve been seen …” … but he has him accused like this: “You’ve been heard …” … but he has him accused like this: “You’ve been seen and you’ve been heard. You’ve committed an offense entailing expulsion. You’re not an ascetic …” For each statement, he commits an offense entailing suspension. 

He\marginnote{3.8.1} has seen that someone has committed an offense entailing expulsion, but he is unsure of what he has seen, he does not believe what he has seen, he does not remember what he has seen, he is confused about what he has seen … he is unsure of what he has heard, he does not believe what he has heard, he does not remember what he has heard, he is confused about what he has heard … he is unsure of what he suspects, he does not believe what he suspects, he does not remember what he suspects, he is confused about what he suspects. If he then has him accused like this: “You’re suspected and you’ve been seen …” … he is confused about what he suspects. If he then has him accused like this: “You’re suspected and you’ve been heard …” … he is confused about what he suspects. If he then has him accused like this: “You’re suspected and you’ve been seen and you’ve been heard. You’ve committed an offense entailing expulsion. You’re not an ascetic, not a Sakyan monastic. You’re excluded from the observance-day ceremony, from the invitation ceremony, and from the legal procedures of the Sangha.” For each statement, he commits an offense entailing suspension. 

\subsubsection*{Permutations part 2 }

\paragraph*{Summary }

Someone\marginnote{4.1.1} is impure, but viewed as pure; someone is pure, but viewed as impure; someone is impure and viewed as impure; someone is pure and viewed as pure. 

\paragraph*{Exposition }

\subparagraph*{Impure but viewed as pure }

An\marginnote{4.2.1} impure person has committed an offense entailing expulsion. If one views him as pure, but then, without having gotten his permission, speaks with the aim of making him leave the monastic life, one commits one offense entailing suspension and one offense of wrong conduct. 

An\marginnote{4.2.3} impure person has committed an offense entailing expulsion. If one views him as pure, but then, having gotten his permission, speaks with the aim of making him leave the monastic life, one commits an offense entailing suspension. 

An\marginnote{4.2.5} impure person has committed an offense entailing expulsion. If one views him as pure, but then, without having gotten his permission, speaks with the aim of abusing him, one commits one offense for abusive speech and one offense of wrong conduct.\footnote{See \href{https://suttacentral.net/pli-tv-bu-vb-pc2/en/brahmali\#1.2.33}{Bu Pc 2:1.2.33} for the rule on abusive speech. } 

An\marginnote{4.2.7} impure person has committed an offense entailing expulsion. If one views him as pure, but then, having gotten his permission, speaks with the aim of abusing him, one commits an offense for abusive speech. 

\subparagraph*{Pure but viewed as impure }

A\marginnote{4.3.1} pure person has not committed an offense entailing expulsion. If one views him as impure, and then, without having gotten his permission, speaks with the aim of making him leave the monastic life, one commits an offense of wrong conduct. 

A\marginnote{4.3.3} pure person has not committed an offense entailing expulsion. If one views him as impure, and then, having gotten his permission, speaks with the aim of making him leave the monastic life, there is no offense. 

A\marginnote{4.3.5} pure person has not committed an offense entailing expulsion. If one views him as impure, and then, without having gotten his permission, speaks with the aim of abusing him, one commits one offense for abusive speech and one offense of wrong conduct. 

A\marginnote{4.3.7} pure person has not committed an offense entailing expulsion. If one views him as impure, then, having gotten his permission, speaks with the aim of abusing him, one commits an offense for abusive speech. 

\subparagraph*{Impure and viewed as impure }

An\marginnote{4.4.1.1} impure person has committed an offense entailing expulsion. If one views him as impure, and then, without having gotten his permission, speaks with the aim of making him leave the monastic life, one commits an offense of wrong conduct. 

An\marginnote{4.4.3} impure person has committed an offense entailing expulsion. If one views him as impure, and then, having gotten his permission, speaks with the aim of making him leave the monastic life, there is no offense. 

An\marginnote{4.4.5} impure person has committed an offense entailing expulsion. If one views him as impure, and then, without having gotten his permission, speaks with the aim of abusing him, one commits one offense for abusive speech and one offense of wrong conduct.\footnote{See \href{https://suttacentral.net/pli-tv-bu-vb-pc2/en/brahmali\#1.2.33}{Bu Pc 2:1.2.33} for the rule on abusive speech. } 

An\marginnote{4.4.7} impure person has committed an offense entailing expulsion. If one views him as impure, and then, having gotten his permission, speaks with the aim of abusing him, one commits an offense for abusive speech. 

\subparagraph*{Pure and viewed as pure }

A\marginnote{4.5.1} pure person has not committed an offense entailing expulsion. If one views him as pure, but then, without having gotten his permission, speaks with the aim of making him leave the monastic life, one commits one offense entailing suspension and one offense of wrong conduct. 

A\marginnote{4.5.3} pure person has not committed an offense entailing expulsion. If one views him as pure, but then, having gotten his permission, speaks with the aim of making him leave the monastic life, one commits an offense entailing suspension. 

A\marginnote{4.5.5} pure person has not committed an offense entailing expulsion. If one views him as pure, but then, without having gotten his permission, speaks with the aim of abusing him, one commits one offense for abusive speech and one offense of wrong conduct. 

A\marginnote{4.5.7} pure person has not committed an offense entailing expulsion. If one views him as pure, but then, having gotten his permission, speaks with the aim of abusing him, one commits an offense for abusive speech. 

\subsection*{Non-offenses }

There\marginnote{4.5.8.1} is no offense: if he views a pure person as impure; if he views an impure person as impure; if he is insane; if he is the first offender. 

\scendsutta{The training rule on groundless, the eighth, is finished. }

%
\section*{{\suttatitleacronym Bu Ss 9}{\suttatitletranslation 9. The second training rule on anger }{\suttatitleroot Aññabhāgiya}}
\addcontentsline{toc}{section}{\tocacronym{Bu Ss 9} \toctranslation{9. The second training rule on anger } \tocroot{Aññabhāgiya}}
\markboth{9. The second training rule on anger }{Aññabhāgiya}
\extramarks{Bu Ss 9}{Bu Ss 9}

\subsection*{Origin story }

At\marginnote{1.1.1} one time when the Buddha was staying at \textsanskrit{Rājagaha} in the Bamboo Grove, the monks Mettiya and \textsanskrit{Bhūmajaka} were descending from the Vulture Peak when they saw two goats copulating. They said to each other, “Let’s give the he-goat the name Dabba the Mallian and the she-goat the name \textsanskrit{Mettiyā} the nun. We can then say, ‘Previously we spoke of what we had heard, but now we’ve seen Dabba copulating with the nun \textsanskrit{Mettiyā}.’” They then gave them those names and told the monks, “Previously we spoke of what we had heard, but now we’ve seen Dabba copulating with the nun \textsanskrit{Mettiyā}.” 

The\marginnote{1.1.14} monks replied, “Don’t say such things. Venerable Dabba wouldn’t do that.” 

The\marginnote{1.1.17} monks told the Buddha. Soon afterwards the Buddha had the Sangha gathered and questioned Dabba: “Dabba, do you remember doing as these monks say?” 

“Sir,\marginnote{1.1.20} you know what I’m like.” 

A\marginnote{1.1.21} second and a third time the Buddha asked the same question and got the same response. He then said, “Dabba, the Dabbas don’t give such evasive answers. If it was done by you, say so; if it wasn’t done by you, then say that.” 

“Since\marginnote{1.1.28} I was born, sir, I don’t recall having sexual intercourse even in a dream, let alone when awake.” 

“Well\marginnote{1.1.29} then, monks, call those monks to account.” And the Buddha got up from his seat and entered his dwelling. 

The\marginnote{1.2.1} monks then questioned Mettiya and \textsanskrit{Bhūmajaka}, who told them what had happened. The monks said, “So did you charge Venerable Dabba with an offense entailing expulsion, using an unrelated legal issue as a pretext?” 

“Yes.”\marginnote{1.2.4} 

The\marginnote{1.2.5} monks of few desires complained and criticized them, “How could Mettiya and \textsanskrit{Bhūmajaka} charge Venerable Dabba with an offense entailing expulsion, using an unrelated legal issue as a pretext?” 

They\marginnote{1.2.7} rebuked those monks in many ways and then told the Buddha. Soon afterwards he had the Sangha gathered and questioned those monks: “Is it true, monks, that you did this?” 

“It’s\marginnote{1.2.9} true, sir.” 

The\marginnote{1.2.10} Buddha rebuked them … “Foolish men, how could you do this? This will affect people’s confidence …” … “And, monks, this training rule should be recited like this: 

\subsection*{Final ruling }

\scrule{‘If a monk who is angry and displeased, uses an unrelated legal issue as a pretext to charge a monk with an offense entailing expulsion, aiming to make him leave the monastic life, and then after some time, whether he is questioned or not, it is clear that the legal issue is unrelated and was used as a pretext, and he admits to his ill will, he commits an offense entailing suspension.’” }

\subsection*{Definitions }

\begin{description}%
\item[A: ] whoever … %
\item[Monk: ] … The monk who has been given the full ordination by a unanimous Sangha through a legal procedure consisting of one motion and three announcements that is irreversible and fit to stand—this sort of monk is meant in this case. %
\item[A monk: ] another monk. %
\item[Angry: ] upset, dissatisfied, discontent, having hatred, hostile. %
\item[Displeased: ] because of that upset, that ill will, that dissatisfaction, and that discontent, he is displeased. %
\item[An unrelated legal issue: ] it\marginnote{2.2.2} is either unrelated in regard to offenses or unrelated in regard to legal issues. 

How\marginnote{2.2.3} is a legal issue unrelated to a legal issue? A legal issue arising from a dispute is unrelated to a legal issue arising from an accusation, a legal issue arising from an offense, and a legal issue arising from business. A legal issue arising from an accusation is unrelated to a legal issue arising from an offense, a legal issue arising from business, and a legal issue arising from a dispute. A legal issue arising from an offense is unrelated to a legal issue arising from business, a legal issue arising from a dispute, and a legal issue arising from an accusation. A legal issue arising from business is unrelated to a legal issue arising from a dispute, a legal issue arising from an accusation, and a legal issue arising from an offense. It is in this way that a legal issue is unrelated to a legal issue. 

How\marginnote{2.2.9} is a legal issue related to a legal issue? A legal issue arising from a dispute is related to a legal issue arising from a dispute. A legal issue arising from an accusation is related to a legal issue arising from an accusation. A legal issue arising from an offense may be either related or unrelated to a legal issue arising from an offense. 

How\marginnote{2.2.13} is a legal issue arising from an offense unrelated to a legal issue arising from an offense? An offense entailing expulsion in regard to sexual intercourse is unrelated to an offense entailing expulsion in regard to stealing, an offense entailing expulsion in regard to a human being, and an offense entailing expulsion in regard to a superhuman quality. An offense entailing expulsion in regard to stealing is unrelated to an offense entailing expulsion in regard to a human being, an offense entailing expulsion in regard to a superhuman quality, and an offense entailing expulsion in regard to sexual intercourse. An offense entailing expulsion in regard to a human being is unrelated to an offense entailing expulsion in regard to a superhuman quality, an offense entailing expulsion in regard to sexual intercourse, and an offense entailing expulsion in regard to stealing. An offense entailing expulsion in regard to a superhuman quality is unrelated to an offense entailing expulsion in regard to sexual intercourse, an offense entailing expulsion in regard to stealing, and an offense entailing expulsion in regard to a human being. It is in this way that a legal issue arising from an offense is unrelated to a legal issue arising from an offense. 

How\marginnote{2.2.19} is a legal issue arising from an offense related to a legal issue arising from an offense? An offense entailing expulsion in regard to sexual intercourse is related to an offense entailing expulsion in regard to sexual intercourse. An offense entailing expulsion in regard to stealing is related to an offense entailing expulsion in regard to stealing. An offense entailing expulsion in regard to a human being is related to an offense entailing expulsion in regard to a human being. An offense entailing expulsion in regard to a superhuman quality is related to an offense entailing expulsion in regard to a superhuman quality. It is in this way that a legal issue arising from an offense is related to a legal issue arising from an offense. 

A\marginnote{2.2.25} legal issue arising from business is related to a legal issue arising from business. It is in this way that a legal issue is related to a legal issue. 

%
\item[Uses as a pretext: ] A\marginnote{2.3.2} pretext: there are ten kinds of pretext—the pretext of caste, the pretext of name, the pretext of family, the pretext of characteristic, the pretext of offense, the pretext of almsbowl, the pretext of robe, the pretext of preceptor, the pretext of teacher, the pretext of dwelling. 

\begin{enumerate}%
\item The pretext of caste: a monk sees an aristocrat committing an offense entailing expulsion. If he then accuses another aristocrat, saying, “I’ve seen an aristocrat. You’ve committed an offense entailing expulsion. You’re not an ascetic, not a Sakyan monastic. You’re excluded from the observance-day ceremony, from the invitation ceremony, and from the legal procedures of the Sangha,” he commits an offense entailing suspension for each statement. A monk sees a brahmin … A monk sees a merchant … A monk sees a worker committing an offense entailing expulsion. If he then accuses another worker, saying, “I’ve seen a worker. You’ve committed an offense entailing expulsion. You’re not an ascetic, not a Sakyan monastic. …” he commits an offense entailing suspension for each statement. %
\item The pretext of name: a monk sees someone whose name is Buddharakkhita … Dhammarakkhita … \textsanskrit{Saṅgharakkhita} committing an offense entailing expulsion. If he then accuses another person called \textsanskrit{Saṅgharakkhita}, saying, “I’ve seen \textsanskrit{Saṅgharakkhita}. You’ve committed an offense entailing expulsion. You’re not an ascetic, not a Sakyan monastic. …” he commits an offense entailing suspension for each statement. %
\item The pretext of family: a monk sees someone whose family name is Gotama … \textsanskrit{Moggallāna} … \textsanskrit{Kaccāyana} … \textsanskrit{Vāsiṭṭha} committing an offense entailing expulsion. If he then accuses another person called \textsanskrit{Vāsiṭṭha}, saying, “I’ve seen \textsanskrit{Vāsiṭṭha}. You’ve committed an offense entailing expulsion. You’re not an ascetic, not a Sakyan monastic. …” he commits an offense entailing suspension for each statement. %
\item The pretext of characteristic: a monk sees someone tall … short … dark-skinned … light-skinned committing an offense entailing expulsion. If he then accuses another light-skinned person, saying, “I’ve seen a light-skinned person. You’ve committed an offense entailing expulsion. You’re not an ascetic, not a Sakyan monastic. …” he commits an offense entailing suspension for each statement. %
\item The pretext of offense: a monk sees someone committing a light offense. If he then accuses him of an offense entailing expulsion, saying, “You’re not an ascetic, not a Sakyan monastic. …” he commits an offense entailing suspension for each statement. %
\item The pretext of almsbowl: a monk sees someone carrying an iron bowl … a black clay bowl …\footnote{Sp 1.400: \textit{\textsanskrit{Sāṭakapattoti} lohapattasadiso \textsanskrit{susaṇṭhāno} succhavi siniddho \textsanskrit{bhamaravaṇṇo} \textsanskrit{mattikāpatto} vuccati}, “\textit{\textsanskrit{Sāṭakapatta}}: what is meant is a clay bowl like an iron bowl, which is well-formed, has a nice and glossy surface, and has the color of a bee.” \href{https://suttacentral.net/thig13.1/en/brahmali\#1.1}{Thig 13.1:1.1}: \textit{\textsanskrit{Kāḷakā} \textsanskrit{bhamaravaṇṇasādisā}}, “Black like the color of a bee.” It seems plausible that this kind of bowl was made of the ceramic that is now known among archeologists as “Northern Black Polished Ware”. } an ordinary clay bowl committing an offense entailing expulsion.\footnote{Sp 1.400: \textit{Sumbhakapattoti \textsanskrit{pakatimattikāpatto}}, “\textit{Sumbhakapatta}: an ordinary clay bowl.” } If he then accuses another person carrying an ordinary clay bowl, saying, “I’ve seen someone carrying an ordinary clay bowl. You’ve committed an offense entailing expulsion. You’re not an ascetic, not a Sakyan monastic. …” he commits an offense entailing suspension for each statement. %
\item The pretext of robe: a monk sees a rag-robe wearer … wearing robes given by householders committing an offense entailing expulsion. If he then accuses another person wearing robes given by householders, saying, “I’ve seen someone wearing robes given by householders. You’ve committed an offense entailing expulsion. You’re not an ascetic, not a Sakyan monastic. …” he commits an offense entailing suspension for each statement. %
\item The pretext of preceptor: a monk sees a student of so-and-so committing an offense entailing expulsion. If he then accuses another student of that person, saying, “I’ve seen the student of so-and-so. You’ve committed an offense entailing expulsion. You’re not an ascetic, not a Sakyan monastic. …” he commits an offense entailing suspension for each statement. %
\item The pretext of teacher: a monk sees a pupil of so-and-so committing an offense entailing expulsion. If he then accuses another pupil of that person, saying, “I’ve seen the pupil of so-and-so. You’ve committed an offense entailing expulsion. You’re not an ascetic, not a Sakyan monastic. …” he commits an offense entailing suspension for each statement. %
\item The pretext of dwelling: a monk sees one who dwells in such-and-such a dwelling committing an offense entailing expulsion. If he then accuses someone else who dwells in that dwelling, saying, “I’ve seen one who dwells in such-and-such a dwelling. You’ve committed an offense entailing expulsion. You’re not an ascetic, not a Sakyan monastic. You’re excluded from the observance-day ceremony, from the invitation ceremony, and from the legal procedures of the Sangha,” he commits an offense entailing suspension for each statement. %
\end{enumerate}

%
\item[With an offense entailing expulsion: ] with one of the four. %
\item[Charges: ] accuses him or has him accused. %
\item[To make him leave the monastic life: ] to make him leave the monkhood, leave the state of a monastic, leave his morality, leave the virtue of monasticism. %
\item[And then after some time: ] the moment, the instant, the second after he has laid the charge. %
\item[He is questioned: ] he is questioned about the grounds of his charge. %
\item[Not: ] he is not spoken to by anyone. %
\item[The legal issue: ] there are four kinds of legal issues: legal issues arising from disputes, legal issues arising from accusations, legal issues arising from offenses, legal issues arising from business. %
\item[Was used as a pretext: ] he has used a certain pretext among those listed above. %
\item[And he admits to his ill will: ] “What I said was empty,” “What I said was false,” “What I said was unreal,” “I said it without knowing.” %
\item[He commits an offense entailing suspension: ] … Therefore, too, it is called “an offense entailing suspension”. %
\end{description}

\subsection*{Permutations }

\subparagraph*{Doing the accusing oneself }

A\marginnote{3.1.1} monk sees a second monk committing an offense entailing suspension, and the first monk regards it as an offense entailing suspension. If he then accuses him of an offense entailing expulsion, saying, “You’re not an ascetic, not a Sakyan monastic. You’re excluded from the observance-day ceremony, from the invitation ceremony, and from the legal procedures of the Sangha,” thus using an unrelated offense as a pretext, he commits an offense entailing suspension for each statement. 

A\marginnote{3.1.6} monk sees a second monk committing an offense entailing suspension, but the first monk regards it as a serious offense … but the first monk regards it as an offense entailing confession … but the first monk regards it as an offense entailing acknowledgment … but the first monk regards it as an offense of wrong conduct … but the first monk regards it as an offense of wrong speech. If he then accuses him of an offense entailing expulsion, saying, “You’re not an ascetic …” thus using an unrelated offense as a pretext, he commits an offense entailing suspension for each statement. 

A\marginnote{3.1.14} monk sees a second monk committing a serious offense, and the first monk regards it as a serious offense … but the first monk regards it as an offense entailing confession … but the first monk regards it as an offense entailing acknowledgment … but the first monk regards it as an offense of wrong conduct … but the first monk regards it as an offense of wrong speech … but the first monk regards it as an offense entailing suspension. If he then accuses him of an offense entailing expulsion, saying, “You’re not an ascetic …” thus using an unrelated offense as a pretext, he commits an offense entailing suspension for each statement. 

A\marginnote{3.1.23} monk sees a second monk committing an offense entailing confession … an offense entailing acknowledgment … an offense of wrong conduct … an offense of wrong speech, and the first monk regards it as an offense of wrong speech … but the first monk regards it as an offense entailing suspension … but the first monk regards it as a serious offense … but the first monk regards it as an offense entailing confession … but the first monk regards it as an offense entailing acknowledgment … but the first monk regards it as an offense of wrong conduct. If he then accuses him of an offense entailing expulsion, saying, “You’re not an ascetic, not a Sakyan monastic. You’re excluded from the observance-day ceremony, from the invitation ceremony, and from the legal procedures of the Sangha,” thus using an unrelated offense as a pretext, he commits an offense entailing suspension for each statement. 

\scend{The permutation series is to be linked by doing the items one by one. }

\subparagraph*{Getting someone else to do the accusing }

A\marginnote{3.2.1} monk sees a second monk committing an offense entailing suspension and the first monk regards it as an offense entailing suspension. If he then has him accused of an offense entailing expulsion, saying, “You’re not an ascetic …” thus using an unrelated offense as a pretext, he commits an offense entailing suspension for each statement. 

A\marginnote{3.2.5} monk sees a second monk committing an offense entailing suspension, but the first monk regards it as a serious offense … but the first monk regards it as an offense entailing confession … but the first monk regards it as an offense entailing acknowledgment … but the first monk regards it as an offense of wrong conduct … but the first monk regards it as an offense of wrong speech. If he then has him accused of an offense entailing expulsion, saying, “You’re not an ascetic …” thus using an unrelated offense as a pretext, he commits an offense entailing suspension for each statement. 

A\marginnote{3.2.13} monk sees a second monk committing a serious offense, and the first monk regards it as a serious offense … but the first monk regards it as an offense entailing confession … but the first monk regards it as an offense entailing acknowledgment … but the first monk regards it as an offense of wrong conduct … but the first monk regards it as an offense of wrong speech … but the first monk regards it as an offense entailing suspension. If he then has him accused of an offense entailing expulsion, saying, “You’re not an ascetic …” thus using an unrelated offense as a pretext, he commits an offense entailing suspension for each statement. 

A\marginnote{3.2.22} monk sees a second monk committing an offense entailing confession … an offense entailing acknowledgment … an offense of wrong conduct … an offense of wrong speech, and the first monk regards it as an offense of wrong speech … but the first monk regards it as an offense entailing suspension … but the first monk regards it as a serious offense … but the first monk regards it as an offense entailing confession … but the first monk regards it as an offense entailing acknowledgment … but the first monk regards it as an offense of wrong conduct. If he then has him accused of an offense entailing expulsion, saying, “You’re not an ascetic, not a Sakyan monastic. You’re excluded from the observance-day ceremony, from the invitation ceremony, and from the legal procedures of the Sangha,” thus using an unrelated offense as a pretext, he commits an offense entailing suspension for each statement. 

\subsection*{Non-offenses }

There\marginnote{3.3.1} is no offense: if he accuses or has someone accused in accordance with his own perception; if he is insane; if he is the first offender. 

\scendsutta{The training rule on a (unrelated) pretext, the ninth, is finished. }

%
\section*{{\suttatitleacronym Bu Ss 10}{\suttatitletranslation 10. The training rule on schism in the Sangha }{\suttatitleroot Saṅghabheda}}
\addcontentsline{toc}{section}{\tocacronym{Bu Ss 10} \toctranslation{10. The training rule on schism in the Sangha } \tocroot{Saṅghabheda}}
\markboth{10. The training rule on schism in the Sangha }{Saṅghabheda}
\extramarks{Bu Ss 10}{Bu Ss 10}

\subsection*{Origin story }

At\marginnote{1.1.1} one time when the Buddha was staying at \textsanskrit{Rājagaha} in the Bamboo Grove, Devadatta went to \textsanskrit{Kokālika}, \textsanskrit{Kaṭamodakatissaka}, \textsanskrit{Khaṇḍadeviyāputta}, and Samuddadatta. He said to them, “Let’s cause a schism in the Sangha of the ascetic Gotama. Let’s break its authority.”\footnote{“Break its authority” renders \textit{cakkabheda}. Sp 1.410: \textit{\textsanskrit{Cakkabhedāyāti} \textsanskrit{āṇābhedāya}}, “\textit{\textsanskrit{Cakkabhedāya}}: by breaking the authority.” Vjb 4.343: \textit{Cakkabhedanti \textsanskrit{sāsanabhedaṁ}}, “\textit{Cakkabheda}: a break in the instruction.” The break in authority is presumably both from the Buddha and the Sangha. Although the Buddha was the only authority in laying down rules, the Sangha was autonomous in its decision making. For practical purposes, it was the Sangha that Devadatta was breaking with. } 

\textsanskrit{Kokālika}\marginnote{1.1.4} said to Devadatta, “The ascetic Gotama is powerful and mighty. How can we do this?” 

“Well,\marginnote{1.1.7} let’s go to the ascetic Gotama and request five things: ‘In many ways, sir, you praise fewness of wishes, contentment, self-effacement, ascetic practices, being inspiring, the reduction in things, and being energetic. And there are five things that lead to just that: It would be good, sir, 

\begin{enumerate}%
\item if the monks stayed in the wilderness for life, and whoever stayed near an inhabited area would commit an offense %
\item if they ate only almsfood for life, and whoever accepted an invitational meal would commit an offense %
\item if they were rag-robe wearers for life, and whoever accepted robe-cloth from a householder would commit an offense %
\item if they lived at the foot of a tree for life, and whoever took shelter would commit an offense %
\item if they didn’t eat fish or meat for life, and whoever did would commit an offense.’ %
\end{enumerate}

The\marginnote{1.1.21} ascetic Gotama won’t allow this. We’ll then be able to win people over with these five points.” 

\textsanskrit{Kokālika}\marginnote{1.1.23} said, “It might be possible to cause a schism in the Sangha with these five points, for people have confidence in austerity.” 

Devadatta\marginnote{1.2.1} and his followers then went to the Buddha, bowed, and sat down, and Devadatta made his request. The Buddha replied, “No, Devadatta. Those who wish may stay in the wilderness, and those who wish may live near inhabited areas.\footnote{“Inhabited area” renders \textit{\textsanskrit{gāma}}. See Appendix of Technical Terms for discussion. } Those who wish may eat only almsfood, and those who wish may accept invitational meals. Those who wish may be rag-robe wearers, and those who wish may accept robe-cloth from householders. I have allowed the foot of a tree as a resting place for eight months of the year, as well as fish and meat that are pure in three respects: one hasn’t seen, heard, or suspected that the animal was specifically killed to feed a monastic.” 

Devadatta\marginnote{1.2.16} thought, “The Buddha doesn’t allow the five points.” Glad and elated, he got up from his seat, bowed down, circumambulated the Buddha with his right side toward him, and left with his followers. 

Devadatta\marginnote{1.2.18} then entered \textsanskrit{Rājagaha} and won people over with the five points, saying, “The ascetic Gotama doesn’t agree to them, but we practice in accordance with them.” 

The\marginnote{1.3.1} foolish people with little faith and confidence said, “These Sakyan monastics are practicing asceticism and living with the aim of self-effacement. But the ascetic Gotama is extravagant and has chosen a life of indulgence.” But the wise people who had faith and confidence complained and criticized Devadatta, “How can Devadatta pursue schism in the Sangha of the Buddha? How can he break its authority?”\footnote{For the further development of these events, see \href{https://suttacentral.net/pli-tv-kd17/en/brahmali\#2.11}{Kd 17:2.11}–4.5.15. } 

The\marginnote{1.3.6} monks heard the criticism of those people, and the monks of few desires complained and criticized him in the same way. 

After\marginnote{1.3.9} rebuking Devadatta in many ways, they told the Buddha. Soon afterwards he had the Sangha gathered and questioned Devadatta: “Is it true, Devadatta, that you are doing this?” 

“It’s\marginnote{1.3.11} true, sir.” 

The\marginnote{1.3.12} Buddha rebuked him … “Foolish man, how can you do this? This will affect people’s confidence …” … “And, monks, this training rule should be recited like this: 

\subsection*{Final ruling }

\scrule{‘If a monk pursues schism in a united Sangha or persists in taking up a legal issue conducive to schism, the monks should correct him like this, “Venerable, don’t pursue schism in the united Sangha or persist in taking up a legal issue conducive to schism. Stay with the Sangha, for a united Sangha—in concord, in harmony, having a joint recitation—is at ease.” If that monk continues as before, the monks should press him up to three times to make him stop. If he then stops, all is well. If he does not stop, he commits an offense entailing suspension.’” }

\subsection*{Definitions }

\begin{description}%
\item[A: ] whoever … %
\item[Monk: ] … The monk who has been given the full ordination by a unanimous Sangha through a legal procedure consisting of one motion and three announcements that is irreversible and fit to stand—this sort of monk is meant in this case. %
\item[A united Sangha: ] those belonging to the same Buddhist sect and staying within the same monastery zone.\footnote{“Monastery zone” renders \textit{\textsanskrit{sīmā}}. See Appendix of Technical Terms for discussion. } %
\item[Pursues schism: ] thinking, “What can I do to split, separate, and divide them?” he searches for a faction and puts together a group. %
\item[A legal issue conducive to schism:\footnote{For these eighteen grounds, see \href{https://suttacentral.net/pli-tv-kd17/en/brahmali\#5.2.3}{Kd 17:5.2.3}. } ] the eighteen grounds for schism. %
\item[Taking up: ] having adopted. %
\item[Taking up:\footnote{In the rule “taking up” renders both \textit{\textsanskrit{samādāya}} and \textit{paggayha}. As a consequence, the two terms are here rendered in the same way. } ] he proclaims. %
\item[If he persists in: ] if he does not stop. %
\item[Him: ] the monk who is pursuing schism in the Sangha. %
\item[The monks: ] other\marginnote{2.20} monks, those who see it or hear it. They should correct him like this: 

“Venerable,\marginnote{2.21} don’t pursue schism in the united Sangha or persist in taking up a legal issue conducive to schism. Stay with the Sangha, for a united Sangha—in concord, in harmony, having a joint recitation—is at ease.” 

And\marginnote{2.24} they should correct him a second and a third time. If he stops, all is well. If he does not stop, he commits an offense of wrong conduct. 

If\marginnote{2.28} those who hear about it do not say anything, they commit an offense of wrong conduct. 

That\marginnote{2.29} monk, even if he has to be pulled into the Sangha, should be corrected like this: 

“Venerable,\marginnote{2.30} don’t pursue schism in the united Sangha or persist in taking up a legal issue conducive to schism. Stay with the Sangha, for a united Sangha—in concord, in harmony, having a joint recitation—is at ease.” 

They\marginnote{2.33} should correct him a second and a third time. If he stops, all is well. If he does not stop, he commits an offense of wrong conduct. 

%
\item[Should press him: ] “And,\marginnote{2.38} monks, he should be pressed like this. A competent and capable monk should inform the Sangha: 

‘Please,\marginnote{2.40} venerables, I ask the Sangha to listen. The monk so-and-so is pursuing schism in the united Sangha. And he keeps on doing it. If the Sangha is ready, it should press him to make him stop. This is the motion. 

Please,\marginnote{2.45} venerables, I ask the Sangha to listen. The monk so-and-so is pursuing schism in the united Sangha. And he keeps on doing it. The Sangha presses him to make him stop. Any monk who approves of pressing him to make him stop should remain silent. Any monk who doesn’t approve should speak up. 

For\marginnote{2.51} the second time, I speak on this matter. … For the third time, I speak on this matter. Please, venerables, I ask the Sangha to listen. The monk so-and-so is pursuing schism in the united Sangha. And he keeps on doing it. The Sangha presses him to make him stop. Any monk who approves of pressing him to make him stop should remain silent. Any monk who doesn’t approve should speak up. 

The\marginnote{2.59} Sangha has pressed monk so-and-so to make him stop. The Sangha approves and is therefore silent. I’ll remember it thus.’” 

After\marginnote{2.62} the motion, he commits an offense of wrong conduct.\footnote{The Pali just says \textit{\textsanskrit{dukkaṭa}}, without specifying that it is an \textit{\textsanskrit{āpatti}}, an offense. Yet just below the text says that the \textit{\textsanskrit{dukkaṭa}} is annulled if you commit the full offense of \textit{\textsanskrit{saṅghādisesa}}. The implication is that \textit{\textsanskrit{dukkaṭa}} should be read as \textit{\textsanskrit{āpatti} \textsanskrit{dukkaṭassa}}, “an offense of wrong conduct”. } After each of the first two announcements, he commits a serious offense. When the last announcement is finished, he commits an offense entailing suspension. For one who commits the offense entailing suspension, the offense of wrong conduct and the serious offenses are annulled. 

%
\item[He commits an offense entailing suspension: ] … Therefore, too, it is called “an offense entailing suspension”. %
\end{description}

\subsection*{Permutations }

If\marginnote{3.1.1} it is a legitimate legal procedure, and he perceives it as such, and he does not stop, he commits an offense entailing suspension. 

If\marginnote{3.1.2} it is a legitimate legal procedure, but he is unsure of it, and he does not stop, he commits an offense entailing suspension. 

If\marginnote{3.1.3} it is a legitimate legal procedure, but he perceives it as illegitimate, and he does not stop, he commits an offense entailing suspension. 

If\marginnote{3.1.4} it is an illegitimate legal procedure, but he perceives it as legitimate, he commits an offense of wrong conduct. 

If\marginnote{3.1.5} it is an illegitimate legal procedure, but he is unsure of it, he commits an offense of wrong conduct. 

If\marginnote{3.1.6} it is an illegitimate legal procedure, and he perceives it as such, he commits an offense of wrong conduct. 

\subsection*{Non-offenses }

There\marginnote{3.2.1} is no offense: if he has not been pressed; if he stops; if he is insane; if he is deranged; if he is overwhelmed by pain; if he is the first offender. 

\scendsutta{The training rule on schism in the Sangha, the tenth, is finished. }

%
\section*{{\suttatitleacronym Bu Ss 11}{\suttatitletranslation 11. The training rule on supporting a schism }{\suttatitleroot Bhedānuvattaka}}
\addcontentsline{toc}{section}{\tocacronym{Bu Ss 11} \toctranslation{11. The training rule on supporting a schism } \tocroot{Bhedānuvattaka}}
\markboth{11. The training rule on supporting a schism }{Bhedānuvattaka}
\extramarks{Bu Ss 11}{Bu Ss 11}

\subsection*{Origin story }

At\marginnote{1.1} one time the Buddha was staying at \textsanskrit{Rājagaha} in the Bamboo Grove, the squirrel sanctuary. At that time Devadatta was pursuing schism in the Sangha, a break in authority. The monks were saying, “Devadatta speaks contrary to the Teaching and the training.\footnote{“Training” renders \textit{vinaya}. See Appendix of Technical Terms for discussion. } How can he pursue schism in the Sangha?” 

But\marginnote{1.6} \textsanskrit{Kokālika}, \textsanskrit{Kaṭamodakatissaka}, \textsanskrit{Khaṇḍadeviyāputta}, and Samuddadatta said to those monks, “No, venerables, Devadatta speaks in accordance with the Teaching and the training. And he speaks with our consent and approval. He knows about us and speaks for us, and we approve of this.” 

The\marginnote{1.10} monks of few desires complained and criticized them, “How can these monks support Devadatta’s pursuit of schism in the Sangha?” 

They\marginnote{1.12} rebuked those monks in many ways and then told the Buddha. Soon afterwards he had the Sangha gathered and questioned the monks: “Is it true, monks, that there are monks who support this?” 

“It’s\marginnote{1.14} true, sir.” 

The\marginnote{1.15} Buddha rebuked them … “Monks, how can those foolish men support this? This will affect people’s confidence …” … “And, monks, this training rule should be recited like this: 

\subsection*{Final ruling }

\scrule{‘That monk may have one, two, or three monks who side with him and support him, and they may say, “Venerables, don’t correct this monk. He speaks in accordance with the Teaching and the training. And he speaks with our consent and approval. He knows about us and speaks for us, and we approve of this.” The monks should correct those monks like this, “No, venerables, this monk speaks contrary to the Teaching and the training. And don’t consent to schism in the Sangha. Stay with the Sangha, for a united Sangha—in concord, in harmony, having a joint recitation—is at ease.” If those monks still continue as before, the monks should press them up to three times to make them stop. If they then stop, all is well. If they do not stop, they commit an offense entailing suspension.’” }

\subsection*{Definitions }

\begin{description}%
\item[That: ] that monk who is pursuing schism in the Sangha. %
\item[May have monks: ] may have other monks. %
\item[Who side with him: ] they have the same view, the same belief, the same persuasion as he does. %
\item[Who support him: ] they praise him and take his side. %
\item[One, two, or three: ] there is one, or two, or three. They may say, “Venerables, don’t correct this monk. He speaks in accordance with the Teaching and the training.\footnote{“Correct” renders \textit{avacuttha}. See \textit{vadati} in Appendix of Technical Terms for discussion of this word. } And he speaks with our consent and approval. He knows about us and speaks for us, and we approve of this.” %
\item[Those monks: ] those monks who side with him. %
\item[The monks: ] other\marginnote{2.16} monks, those who see it or hear about it. They should correct them like this: 

“No,\marginnote{2.17} venerables, this monk speaks contrary to the Teaching and the training. And don’t consent to schism in the Sangha. Stay with the Sangha, for a united Sangha—in concord, in harmony, having a joint recitation—is at ease.” 

And\marginnote{2.22} they should correct them a second and a third time. If they stop, all is well. If they do not stop, they commit an offense of wrong conduct. If those who hear about it do not say anything, they commit an offense of wrong conduct. 

Those\marginnote{2.27} monks, even if they have to be pulled into the Sangha, should be corrected like this: 

“No,\marginnote{2.28} venerables, this monk speaks contrary to the Teaching and the training. And don’t consent to schism in the Sangha. Stay with the Sangha, for a united Sangha—in concord, in harmony, having a joint recitation—is at ease.” 

They\marginnote{2.33} should correct them a second and a third time. If they stop, all is well. If they do not stop, they commit an offense of wrong conduct. 

%
\item[Should press them: ] “And,\marginnote{2.38} monks, they should be pressed like this. A competent and capable monk should inform the Sangha: 

‘Please,\marginnote{2.40} venerables, I ask the Sangha to listen. Monks so-and-so and so-and-so are siding with and supporting monk so-and-so who is pursuing schism in the Sangha. And they keep on doing it. If the Sangha is ready, it should press them to make them stop. This is the motion. 

Please,\marginnote{2.45} venerables, I ask the Sangha to listen. Monks so-and-so and so-and-so are siding with and supporting monk so-and-so who is pursuing schism in the Sangha. And they keep on doing it. The Sangha presses them to make them stop. Any monk who approves of pressing them to make them stop should remain silent. Any monk who doesn’t approve should speak up. 

For\marginnote{2.51} the second time, I speak on this matter. … For the third time, I speak on this matter. Please, venerables, I ask the Sangha to listen. Monks so-and-so and so-and-so are siding with and supporting monk so-and-so who is pursuing schism in the Sangha. And they keep on doing it. The Sangha presses them to make them stop. Any monk who approves of pressing them to make them stop should remain silent. Any monk who doesn’t approve should speak up. 

The\marginnote{2.59} Sangha has pressed monks so-and-so and so-and-so to make them stop. The Sangha approves and is therefore silent. I’ll remember it thus.’” 

After\marginnote{2.62} the motion, they commit an offense of wrong conduct.\footnote{The Pali just says \textit{\textsanskrit{dukkaṭa}}, without specifying that it is an \textit{\textsanskrit{āpatti}}, an offense. Yet just below the text says that the \textit{\textsanskrit{dukkaṭa}} is annulled if you commit the full offense of \textit{\textsanskrit{saṅghādisesa}}. The implication is that \textit{\textsanskrit{dukkaṭa}} should be read as \textit{\textsanskrit{āpatti} \textsanskrit{dukkaṭassa}}, “an offense of wrong conduct”. } After each of the first two announcements, they commit a serious offense. When the last announcement is finished, they commit an offense entailing suspension. For those who commit the offense entailing suspension, the offense of wrong conduct and the serious offenses are annulled. Two or three may be pressed together, but not more than that. 

%
\item[They commit an offense entailing suspension: ] … Therefore, too, it is called “an offense entailing suspension”. %
\end{description}

\subsection*{Permutations }

If\marginnote{3.1.1} it is a legitimate legal procedure, and they perceive it as such, and they do not stop, they commit an offense entailing suspension. 

If\marginnote{3.1.2} it is a legitimate legal procedure, but they are unsure of it, and they do not stop, they commit an offense entailing suspension. 

If\marginnote{3.1.3} it is a legitimate legal procedure, but they perceive it as illegitimate, and they do not stop, they commit an offense entailing suspension. 

If\marginnote{3.1.4} it is an illegitimate legal procedure, but they perceive it as legitimate, they commit an offense of wrong conduct. 

If\marginnote{3.1.5} it is an illegitimate legal procedure, but they are unsure of it, they commit an offense of wrong conduct. 

If\marginnote{3.1.6} it is an illegitimate legal procedure, and they perceive it as such, they commit an offense of wrong conduct. 

\subsection*{Non-offenses }

There\marginnote{3.2.1} is no offense: if they have not been pressed; if they stop; if they are insane; if they are deranged; if they are overwhelmed by pain; if they are the first offenders. 

\scendsutta{The training rule on supporting a schism, the eleventh, is finished. }

%
\section*{{\suttatitleacronym Bu Ss 12}{\suttatitletranslation 12. The training rule on being difficult to correct }{\suttatitleroot Dubbaca}}
\addcontentsline{toc}{section}{\tocacronym{Bu Ss 12} \toctranslation{12. The training rule on being difficult to correct } \tocroot{Dubbaca}}
\markboth{12. The training rule on being difficult to correct }{Dubbaca}
\extramarks{Bu Ss 12}{Bu Ss 12}

\subsection*{Origin story }

At\marginnote{1.1} one time when the Buddha was staying at \textsanskrit{Kosambī} in Ghosita’s Monastery, Venerable Channa was misbehaving. The monks would tell him, “Don’t do that; it’s not allowable,” and he would reply, “Who are you to correct me? I should correct you! The Buddha is mine; the Teaching is mine. The Master realized the Truth because of me. Just as grass, sticks, and fallen leaves are whirled up by a strong wind all at once, just as various water plants are whirled up by a mountain stream all at once, so too have you–after going forth with various names, various families, various castes, various households–been lifted up all at once. So, who are you to correct me? I should correct you! The Buddha is mine; the Teaching is mine. The Master realized the Truth because of me.”\footnote{For the further development of these events, see \href{https://suttacentral.net/pli-tv-kd11/en/brahmali\#25.1.1}{Kd 11:25.1.1}–31.1.219. } 

The\marginnote{1.17} monks of few desires complained and criticized him, “How can Venerable Channa make himself incorrigible when he’s legitimately corrected by the monks?” 

They\marginnote{1.19} rebuked Channa in many ways and then told the Buddha. Soon afterwards he had the Sangha gathered and questioned Channa: “Is it true, Channa, that you do this? 

“It’s\marginnote{1.21} true, sir.” 

The\marginnote{1.22} Buddha rebuked him … “Foolish man, how can you do this? This will affect people’s confidence …” … “And, monks, this training rule should be recited like this: 

\subsection*{Final ruling }

\scrule{‘If a monk is difficult to correct, and he makes himself incorrigible when legitimately corrected by the monks concerning the training rules that are recited, saying, “Venerables, don’t say anything to me, either good or bad, and I won’t say anything to you, either good or bad. Please refrain from correcting me,” then the monks should correct him like this: “Be easy to correct, venerable, not incorrigible. And please give legitimate correction to the monks, and the monks will do the same to you. For it’s in this way that the Buddha’s community has grown, that is, through mutual correction and mutual clearing of offenses.” If that monk continues as before, the monks should press him up to three times to make him stop. If he then stops, all is well. If he does not stop, he commits an offense entailing suspension.’” }

\subsection*{Definitions }

\begin{description}%
\item[If a monk is difficult to correct: ] if he is hard to correct, endowed with qualities that make him hard to correct, resistant, not receiving instructions respectfully. %
\item[Concerning the training rules that are recited: ] concerning the training rules of the Monastic Code. %
\item[The monks: ] other monks. %
\item[Legitimately: ] the\marginnote{2.8} training rules laid down by the Buddha—this is called “legitimately”. When corrected in regard to this, he makes himself incorrigible, saying, “Venerables, don’t say anything to me, either good or bad, and I won’t say anything to you, either good or bad. Please refrain from correcting me.” 

%
\item[Him: ] the monk who is difficult to correct. %
\item[The monks: ] other\marginnote{2.14} monks, those who see it or hear about it. They should correct him like this: 

“Be\marginnote{2.15} easy to correct, venerable, not incorrigible. And please give legitimate correction to the monks, and the monks will do the same to you. For it’s in this way that the Buddha’s community has grown, that is, through mutual correction and mutual clearing of offenses.” 

And\marginnote{2.19} they should correct him a second and a third time. If he stops, all is well. If he does not stop, he commits an offense of wrong conduct. If those who hear about it do not say anything, they commit an offense of wrong conduct. 

That\marginnote{2.24} monk, even if he has to be pulled into the Sangha, should be corrected like this: 

“Be\marginnote{2.25} easy to correct, venerable, not incorrigible. And please give legitimate correction to the monks, and the monks will do the same to you. For it’s in this way that the Buddha’s community has grown, that is, through mutual correction and mutual clearing of offenses.” 

They\marginnote{2.27} should correct him a second and a third time. If he stops, all is well. If he does not stop, he commits an offense of wrong conduct. 

%
\item[Should press him: ] “And,\marginnote{2.32} monks, he should be pressed like this. A competent and capable monk should inform the Sangha: 

‘Please,\marginnote{2.34} venerables, I ask the Sangha to listen. The monk so-and-so makes himself incorrigible when legitimately corrected by the monks. And he keeps on doing it. If the Sangha is ready, it should press him to make him stop. This is the motion. 

Please,\marginnote{2.39} venerables, I ask the Sangha to listen. The monk so-and-so makes himself incorrigible when legitimately corrected by the monks. And he keeps on doing it. The Sangha presses him to make him stop. Any monk who approves of pressing him to make him stop should remain silent. Any monk who doesn’t approve should speak up. 

For\marginnote{2.45} the second time, I speak on this matter. … For the third time, I speak on this matter. Please, venerables, I ask the Sangha to listen. The monk so-and-so makes himself incorrigible when legitimately corrected by the monks. And he keeps on doing it. The Sangha presses him to make him stop. Any monk who approves of pressing him to make him stop should remain silent. Any monk who doesn’t approve should speak up. 

The\marginnote{2.53} Sangha has pressed monk so-and-so to make him stop. The Sangha approves and is therefore silent. I’ll remember it thus.’” 

After\marginnote{2.56} the motion, he commits an offense of wrong conduct.\footnote{The Pali just says \textit{\textsanskrit{dukkaṭa}}, without specifying that it is an \textit{\textsanskrit{āpatti}}, an offense. Yet just below the text says that the \textit{\textsanskrit{dukkaṭa}} is annulled if you commit the full offense of \textit{\textsanskrit{saṅghādisesa}}. The implication is that \textit{\textsanskrit{dukkaṭa}} should be read as \textit{\textsanskrit{āpatti} \textsanskrit{dukkaṭassa}}, “an offense of wrong conduct”. } After each of the first two announcements, he commits a serious offense. When the last announcement is finished, he commits an offense entailing suspension. For one who commits the offense entailing suspension, the offense of wrong conduct and the serious offenses are annulled. 

%
\item[He commits an offense entailing suspension: ] … Therefore, too, it is called “an offense entailing suspension”. %
\end{description}

\subsection*{Permutations }

If\marginnote{3.1.1} it is a legitimate legal procedure, and he perceives it as such, but he does not stop, he commits an offense entailing suspension. 

If\marginnote{3.1.2} it is a legitimate legal procedure, but he is unsure of it, and he does not stop, he commits an offense entailing suspension. 

If\marginnote{3.1.3} it is a legitimate legal procedure, but he perceives it as illegitimate, and he does not stop, he commits an offense entailing suspension. 

If\marginnote{3.1.4} it is an illegitimate legal procedure, but he perceives it as legitimate, he commits an offense of wrong conduct. 

If\marginnote{3.1.5} it is an illegitimate legal procedure, but he is unsure of it, he commits an offense of wrong conduct. 

If\marginnote{3.1.6} it is an illegitimate legal procedure, and he perceives it as such, he commits an offense of wrong conduct. 

\subsection*{Non-offenses }

There\marginnote{3.2.1} is no offense: if he has not been pressed; if he stops; if he is insane; if he is the first offender. 

\scendsutta{The  training rule on being difficult to correct, the twelfth, is finished. }

%
\section*{{\suttatitleacronym Bu Ss 13}{\suttatitletranslation 13. The training rule on corrupters of families }{\suttatitleroot Kuladūsaka}}
\addcontentsline{toc}{section}{\tocacronym{Bu Ss 13} \toctranslation{13. The training rule on corrupters of families } \tocroot{Kuladūsaka}}
\markboth{13. The training rule on corrupters of families }{Kuladūsaka}
\extramarks{Bu Ss 13}{Bu Ss 13}

\subsection*{Origin story }

At\marginnote{1.1.1} one time the Buddha was staying at \textsanskrit{Sāvatthī} in the Jeta Grove, \textsanskrit{Anāthapiṇḍika}’s Monastery. At that time the bad and shameless monks Assaji and Punabbasuka were staying at \textsanskrit{Kīṭāgiri}. They were misbehaving in many ways. 

They\marginnote{1.1.4} planted flowering trees, watered and plucked them, and then tied the flowers together. They made the flowers into garlands, garlands with stalks on one side and garlands with stalks on both sides. They made flower arrangements, wreaths, ornaments for the head, ornaments for the ears, and ornaments for the chest. And they had others do the same. They then took these things, or sent them, to the women, the daughters, the girls, the daughters-in-law, and the female slaves of good families. 

They\marginnote{1.1.8} ate from the same plates as these women and drank from the same vessels. They sat on the same seats as them, and they lay down on the same beds, on the same sheets, under the same covers, and both on the same sheets and under the same covers. They ate at the wrong time, drank alcohol, and wore garlands, perfumes, and cosmetics. They danced, sang, played instruments, and performed. While the women were dancing, singing, playing instruments, and performing, so would they. 

They\marginnote{1.2.1} played various games: eight-row checkers, ten-row checkers, imaginary checkers, hopscotch, pick-up-sticks, board games, tipcat, painting with the hand, dice, leaf flutes, toy plows, somersaults, pinwheels, toy measures, toy carriages, toy bows, guessing from syllables, thought guessing, mimicking deformities. 

They\marginnote{1.2.2} trained in elephant riding, in horsemanship, in carriage riding, in archery, in swordsmanship. And they ran in front of elephants, horses, and carriages, and they ran backward and forward. They whistled, clapped their hands, wrestled, and boxed. They spread their outer robe on a stage and said to the dancing girls, “Dance here, Sister,” and they made gestures of approval. And they misbehaved in a variety of ways. 

Just\marginnote{1.3.1} then a monk who had completed the rainy-season residence in \textsanskrit{Kāsi} was on his way to visit the Buddha at \textsanskrit{Sāvatthī} when he arrived at \textsanskrit{Kīṭāgiri}. In the morning he robed up, took his bowl and robe, and entered \textsanskrit{Kīṭāgiri} to collect almsfood. He was pleasing in his conduct: in going out and coming back, in looking ahead and looking aside, in bending and stretching his arms. His eyes were lowered, and he was perfect in deportment. When people saw him, they said, “Who’s this, acting like a moron and always frowning? Who’s gonna give almsfood to him? Almsfood should be given to our Venerables Assaji and Punabbasuka, for they are gentle, congenial, pleasant to speak with, greeting one with a smile, welcoming, friendly, open, the first to speak.” 

A\marginnote{1.3.5} certain lay follower saw that monk walking for alms in \textsanskrit{Kīṭāgiri}. He approached him, bowed, and said, “Venerable, have you received any almsfood?” 

“No,\marginnote{1.3.7} I haven’t.” 

“Come,\marginnote{1.3.8} let’s go to my house.” 

He\marginnote{1.4.1} took that monk to his house and gave him a meal. He then said, “Where are you going, venerable?” 

“I’m\marginnote{1.4.3} going to \textsanskrit{Sāvatthī} to see the Buddha.” 

“Well\marginnote{1.4.4} then, would you please pay respect at the Buddha’s feet in my name and say, ‘Sir, the monastery at \textsanskrit{Kīṭāgiri} has been corrupted. The bad and shameless monks Assaji and Punabbasuka are staying there. And they’re misbehaving in many ways. They plant flowering trees, water them … And they misbehave in a variety of ways. Those who previously had faith and confidence have now lost it, and there’s no longer any support for the Sangha. The good monks have left and the bad monks are staying on. Sir, please send monks to stay at the monastery at \textsanskrit{Kīṭāgiri}.’” 

The\marginnote{1.5.1} monk consented, got up, and set out for \textsanskrit{Sāvatthī}. When he eventually arrived, he went to the Buddha in \textsanskrit{Anāthapiṇḍika}’s Monastery. He bowed to the Buddha and sat down. Since it is the custom for Buddhas to greet newly-arrived monks, the Buddha said to him, “I hope you’re keeping well, monk, I hope you’re getting by?  I hope you’re not tired from traveling?  And where have you come from?” 

“I’m\marginnote{1.5.8} keeping well, sir, I’m getting by.  I’m not tired from traveling.” He then told the Buddha all that had happened at \textsanskrit{Kīṭāgiri}, 

adding,\marginnote{1.5.19} “That’s where I’ve come from, sir.” 

Soon\marginnote{1.6.1} afterwards the Buddha had the Sangha gathered and questioned the monks: “Is it true, monks, that the bad and shameless monks Assaji and Punabbasuka are staying at \textsanskrit{Kīṭāgiri} and misbehaving like this? And is it true that those people who previously had faith and confidence have now lost it, that there’s no longer any support for the Sangha, and that the good monks have left and the bad monks are staying on?” 

“It’s\marginnote{1.6.7} true, sir.” 

The\marginnote{1.6.8} Buddha rebuked them … “Monks, how can those foolish men misbehave in this way? 

This\marginnote{1.6.19} will affect people’s confidence …” He then gave a teaching and addressed \textsanskrit{Sāriputta} and \textsanskrit{Moggallāna}: “Go to \textsanskrit{Kīṭāgiri} and do a legal procedure of banishing the monks Assaji and Punabbasuka. They’re your students.” 

“Sir,\marginnote{1.6.26} how can we do a procedure of banishing these monks from \textsanskrit{Kīṭāgiri}? They’re temperamental and harsh.” 

“Well\marginnote{1.6.27} then, take many monks.” 

“Yes,\marginnote{1.6.28} sir.” 

“And,\marginnote{1.7.1} monks, this is how it should be done. First you should accuse the monks Assaji and Punabbasuka. They should then be reminded of what they have done, before they are charged with an offense. A competent and capable monk should then inform the Sangha: 

‘Please,\marginnote{1.7.5} venerables, I ask the Sangha to listen. These monks, Assaji and Punabbasuka, are corrupters of families and badly behaved. Their bad behavior has been seen and heard about, and the families corrupted by them have been seen and heard about. If the Sangha is ready, it should do a legal procedure of banishing them, prohibiting the monks Assaji and Punabbasuka from staying at \textsanskrit{Kīṭāgiri}. This is the motion. 

Please,\marginnote{1.7.9} venerables, I ask the Sangha to listen. These monks, Assaji and Punabbasuka, are corrupters of families and badly behaved. Their bad behavior has been seen and heard about, and the families corrupted by them have been seen and heard about. The Sangha does a legal procedure of banishing them, prohibiting the monks Assaji and Punabbasuka from staying at \textsanskrit{Kīṭāgiri}. Any monk who approves of doing this legal procedure should remain silent. Any monk who doesn’t approve should speak up. 

For\marginnote{1.7.14} the second time, I speak on this matter. … For the third time, I speak on this matter. Please, venerables, I ask the Sangha to listen. … should speak up. 

The\marginnote{1.7.16} Sangha has done a legal procedure of banishing them, prohibiting the monks Assaji and Punabbasuka from staying at \textsanskrit{Kīṭāgiri}. The Sangha approves and is therefore silent. I’ll remember it thus.’” 

Soon\marginnote{1.8.1} afterwards a sangha of monks, headed by \textsanskrit{Sāriputta} and \textsanskrit{Moggallāna}, went to \textsanskrit{Kīṭāgiri} and did the procedure of banishing Assaji and Punabbasuka, prohibiting them from staying at \textsanskrit{Kīṭāgiri}. After the Sangha had done the procedure, those monks did not conduct themselves properly or suitably so as to deserve to be released, nor did they ask the monks for forgiveness. Instead they abused and reviled them, and they slandered them as acting from favoritism, ill will, confusion, and fear. And they left and they disrobed.\footnote{The meaning of the first of these phrases, \textit{\textsanskrit{sammā} vattati}, is straightforward, but the last two, \textit{\textsanskrit{lomaṁ} \textsanskrit{pāteti}} and \textit{\textsanskrit{netthāraṁ} vattati}, are more difficult. Sp 1.435: \textit{Na \textsanskrit{lomaṁ} \textsanskrit{pātentīti} \textsanskrit{anulomapaṭipadaṁ} \textsanskrit{appaṭipajjanatāya} na \textsanskrit{pannalomā} honti. Na \textsanskrit{netthāraṁ} \textsanskrit{vattantīti} attano \textsanskrit{nittharaṇamaggaṁ} na \textsanskrit{paṭipajjanti}}, “\textit{Na \textsanskrit{lomaṁ} \textsanskrit{pātenti}}: because of their non-practicing in conformity with the path, their bodily hairs are not flat. \textit{Na \textsanskrit{netthāraṁ} vattanti}: they are not practicing the path for their own getting out (of the offense).” My rendering attempts to capture the meaning in a non-literal way. For the further development of these events, see \href{https://suttacentral.net/pli-tv-kd11/en/brahmali\#13.1.1}{Kd 11:13.1.1}–16.1.28. │ “Disrobe” renders \textit{\textsanskrit{vibbhamissāmi}}. According to PED and SED the general meaning of this word is something like “to go astray”. However, the implied meaning throughout the Vinaya \textsanskrit{Piṭaka} is that one leaves the Sangha, that is, one disrobes. I therefore take this word to express the functional equivalence of disrobing. This is supported by the commentaries. Sp 1.435: \textit{\textsanskrit{Vibbhamantīti} ekacce \textsanskrit{gihī} honti}, “\textit{Vibbhamanti}: some became householders.” Sp 4.434: \textit{Yadeva \textsanskrit{sā} \textsanskrit{vibbhantāti} \textsanskrit{yasmā} \textsanskrit{sā} \textsanskrit{vibbhantā} attano \textsanskrit{ruciyā} \textsanskrit{khantiyā} \textsanskrit{odātāni} \textsanskrit{vatthāni} \textsanskrit{nivatthā}, \textsanskrit{tasmāyeva} \textsanskrit{sā} \textsanskrit{abhikkhunī}, na \textsanskrit{sikkhāpaccakkhānenāti} dasseti}, “\textit{Yadeva \textsanskrit{sā} \textsanskrit{vibbhantā}} means she is no longer a nun because, according to her own preference and choice, she dresses in white.” } The monks of few desires complained and criticized them, “How can these monks act in this way when the Sangha has done a legal procedure of banishing them?” 

They\marginnote{1.8.5} rebuked the monks Assaji and Punabbasuka in many ways and then told the Buddha. Soon afterwards he had the Sangha gathered and questioned the monks: “Is it true, monks, that the monks Assaji and Punabbasuka acted in this way?” “It’s true, sir.” 

The\marginnote{1.8.8} Buddha rebuked them … “And, monks, this training rule should be recited like this: 

\subsection*{Final ruling }

\scrule{‘If a monk who lives supported by a village or town is a corrupter of families and badly behaved, and his bad behavior has been seen and heard about, and the families corrupted by him have been seen and heard about, then the monks should correct him like this: “Venerable, you’re a corrupter of families and badly behaved. Your bad behavior has been seen and heard about, and the families corrupted by you have been seen and heard about. Leave this monastery; you’ve stayed here long enough.” If he replies, “You’re acting out of favoritism, ill will, confusion, and fear. Because of this sort of offense, you only banish some, but not others,” the monks should correct him like this: “No, venerable, the monks are not acting out of favoritism, ill will, confusion, and fear. Venerable, you’re a corrupter of families and badly behaved. Your bad behavior has been seen and heard about, and the families corrupted by you have been seen and heard about. Leave this monastery; you’ve stayed here long enough.” If that monk continues as before, the monks should press him up to three times to make him stop. If he then stops, all is well. If he does not stop, he commits an offense entailing suspension.’” }

\subsection*{Definitions }

\begin{description}%
\item[A monk … a village or town: ] a village and a town and a city are included in just a village and a town. %
\item[Lives supported by: ] robe-cloth, almsfood, dwellings, and medicinal supplies can be obtained in that place. %
\item[A family: ] there are four kinds of families: the aristocratic family, the brahmin family, the merchant family, the worker family. %
\item[A corrupter of families: ] he corrupts families by means of flowers, fruit, bath powder, soap, tooth cleaners, bamboo, medical treatment, or by taking messages on foot.\footnote{For the rendering of \textit{mattika} and \textit{\textsanskrit{cuṇṇa}} as respectively “soap” and “bath powder”, see Appendix of Technical Terms. } %
\item[Badly behaved: ] he plants flowering trees and has it done; he waters them and has it done; he plucks them and has it done; he ties the flowers together and has it done. %
\item[Has been seen and heard about: ] those who are present see it; those who are absent hear about it. %
\item[The families corrupted by him: ] they have lost their faith because of him; they have lost their confidence because of him. %
\item[Have been seen and heard about: ] those who are present see it; those who are absent hear about it. %
\item[Him: ] that monk who is a corrupter of families. %
\item[The monks: ] other monks, those who see it or hear about it. They should correct him like this: “Venerable, you’re a corrupter of families and badly behaved. Your bad behavior has been seen and heard about, and the families corrupted by you have been seen and heard about. Leave this monastery; you’ve stayed here long enough.” If he replies, “You’re acting out of favoritism, ill will, confusion, and fear. Because of this sort of offense, you only banish some, but not others.” %
\item[Him: ] that monk who is having a legal procedure done against him. %
\item[The monks: ] other\marginnote{2.26} monks, those who see it or hear about it. They should correct him like this: “No, venerable, the monks are not acting out of favoritism, ill will, confusion, and fear. Venerable, you’re a corrupter of families and badly behaved. Your bad behavior has been seen and heard about, and the families corrupted by you have been seen and heard about. Leave this monastery; you’ve stayed here long enough.” And they should correct him a second and a third time. 

If\marginnote{2.31} he stops, all is well. If he does not stop, he commits an offense of wrong conduct. If those who hear about it do not say anything, they commit an offense of wrong conduct. 

That\marginnote{2.34} monk, even if he has to be pulled into the Sangha, should be corrected like this: “No, venerable, the monks are not acting out of favoritism, ill will, confusion, and fear. Venerable, you’re a corrupter of families and badly behaved. Your bad behavior has been seen and heard about, and the families corrupted by you have been seen and heard about. Leave this monastery; you’ve stayed here long enough.” They should correct him a second and a third time. If he stops, all is well. If he does not stop, he commits an offense of wrong conduct. 

%
\item[Should press him: ] “And,\marginnote{2.42} monks, he should be pressed like this. A competent and capable monk should inform the Sangha: 

‘Please,\marginnote{2.44} venerables, I ask the Sangha to listen. The monk so-and-so, who has had a legal procedure of banishment done against himself, is slandering the monks as acting out of favoritism, ill will, confusion, and fear. And he keeps on doing it. If the Sangha is ready, it should press him to make him stop. This is the motion. 

Please,\marginnote{2.49} venerables, I ask the Sangha to listen. The monk so-and-so, who has had a legal procedure of banishment done against himself, is slandering the monks as acting out of favoritism, ill will, confusion, and fear. And he keeps on doing it. The Sangha presses him to make him stop. Any monk who approves of pressing him to make him stop should remain silent. Any monk who doesn’t approve should speak up. 

For\marginnote{2.54} the second time, I speak on this matter. … For the third time, I speak on this matter. … 

The\marginnote{2.56} Sangha has pressed monk so-and-so to make him stop. The Sangha approves and is therefore silent. I’ll remember it thus.’” 

After\marginnote{2.58} the motion, he commits an offense of wrong conduct.\footnote{The Pali just says \textit{\textsanskrit{dukkaṭa}}, without specifying that it is an \textit{\textsanskrit{āpatti}}, an offense. Yet just below the text says that the \textit{\textsanskrit{dukkaṭa}} is annulled if you commit the full offense of \textit{\textsanskrit{saṅghādisesa}}. The implication is that \textit{\textsanskrit{dukkaṭa}} should be read as \textit{\textsanskrit{āpatti} \textsanskrit{dukkaṭassa}}, “an offense of wrong conduct”. } After each of the first two announcements, he commits a serious offense.  When the last announcement is finished, he commits an offense entailing suspension. For one who commits the offense entailing suspension, the offense of wrong conduct and the serious offenses are annulled. 

%
\item[He commits an offense entailing suspension: ] only the Sangha gives probation for that offense, sends back to the beginning, gives the trial period, and rehabilitates—not several monks, not an individual. Therefore it is called “an offense entailing suspension”.  This is the name and designation of this class of offense. Therefore, too, it is called “an offense entailing suspension”. %
\end{description}

\subsection*{Permutations }

If\marginnote{3.1.1} it is a legitimate legal procedure, and he perceives it as such, and he does not stop, he commits an offense entailing suspension. 

If\marginnote{3.1.2} it is a legitimate legal procedure, but he is unsure of it, and he does not stop, he commits an offense entailing suspension. 

If\marginnote{3.1.3} it is a legitimate legal procedure, but he perceives it as illegitimate, and he does not stop, he commits an offense entailing suspension. 

If\marginnote{3.1.4} it is an illegitimate legal procedure, but he perceives it as legitimate, he commits an offense of wrong conduct. 

If\marginnote{3.1.5} it is an illegitimate legal procedure, but he is unsure of it, he commits an offense of wrong conduct. 

If\marginnote{3.1.6} it is an illegitimate legal procedure, and he perceives it as such, he commits an offense of wrong conduct. 

\subsection*{Non-offenses }

There\marginnote{3.2.1} is no offense:  if he has not been pressed;  if he stops;  if he is insane;  if he is the first offender. 

\scendsutta{The training rule on corrupters of families, the thirteenth, is finished. }

“Venerables,\marginnote{3.2.7} the thirteen rules on suspension have been recited, nine being immediate offenses, four after the third announcement. If a monk commits any one of them, he is to undergo probation for the same number of days as he knowingly concealed that offense. When this is completed, he must undertake the trial period for a further six days. When this is completed, he is to be rehabilitated wherever there is a sangha of at least twenty monks. If that monk is rehabilitated by a sangha of even one less than twenty, that monk is not rehabilitated and those monks are at fault. This is the proper procedure. 

In\marginnote{3.2.12} regard to this I ask you, ‘Are you pure in this?’ A second time I ask, ‘Are you pure in this?’ A third time I ask, ‘Are you pure in this?’ You are pure in this and therefore silent. I’ll remember it thus.” 

\scendsutta{The group of thirteen is finished. }

\scuddanaintro{This is the summary: }

\begin{scuddana}%
“Emission,\marginnote{3.2.18} physical contact, \\
Indecent, and his own needs; \\
Matchmaking, and a hut, \\
And a dwelling, groundless. 

A\marginnote{3.2.22} pretext, and schism, \\
Those who side with him; \\
Difficult to correct, and corrupter of families—\\
The thirteen offenses entailing suspension.” 

%
\end{scuddana}

\scendkanda{The chapter on offenses entailing suspension is finished. }

%
\addtocontents{toc}{\let\protect\contentsline\protect\nopagecontentsline}
\chapter*{Undetermined }
\addcontentsline{toc}{chapter}{\tocchapterline{Undetermined }}
\addtocontents{toc}{\let\protect\contentsline\protect\oldcontentsline}

%
%
\section*{{\suttatitleacronym Bu Ay 1}{\suttatitletranslation The first undetermined training rule }{\suttatitleroot Paṭhamaaniyata}}
\addcontentsline{toc}{section}{\tocacronym{Bu Ay 1} \toctranslation{The first undetermined training rule } \tocroot{Paṭhamaaniyata}}
\markboth{The first undetermined training rule }{Paṭhamaaniyata}
\extramarks{Bu Ay 1}{Bu Ay 1}

Venerables,\marginnote{0.5} these two undetermined rules come up for recitation. 

\subsection*{Origin story }

At\marginnote{1.1} one time the Buddha was staying at \textsanskrit{Sāvatthī} in the Jeta Grove, \textsanskrit{Anāthapiṇḍika}’s Monastery. At that time Venerable \textsanskrit{Udāyī} was associating with and visiting a number of families in \textsanskrit{Sāvatthī}. On one occasion one of the families that supported him gave their daughter in marriage to the son of another family. Soon afterwards Venerable \textsanskrit{Udāyī} robed up in the morning, took his bowl and robe, and went to the first of those families. When he arrived, he asked where the daughter was, and he was told that she had been given to another family. That family too supported \textsanskrit{Udāyī}. He then went there and again asked where the girl was. They said, “She’s sitting in her room.” He went up to that girl, and the two of them sat down alone in private on a concealed seat suitable for the deed. When they were able, they chatted; otherwise he gave her a teaching.\footnote{Sp 1.443: \textit{Tattha \textsanskrit{kālayuttaṁ} samullapantoti \textsanskrit{kālaṁ} \textsanskrit{sallakkhetvā} \textsanskrit{yadā} na \textsanskrit{añño} koci \textsanskrit{samīpena} gacchati \textsanskrit{vā} \textsanskrit{āgacchati} \textsanskrit{vā} \textsanskrit{tadā} \textsanskrit{tadanurūpaṁ} “kacci na \textsanskrit{ukkaṇṭhasi}, na kilamasi, na \textsanskrit{chātāsī}”\textsanskrit{tiādikaṁ} \textsanskrit{gehassitaṁ} \textsanskrit{kathaṁ} kathento}, “‘When they were able, they chatted’: having considered the occasion, whether anyone else was coming or going nearby, they spoke about worldly things, such as, ‘I hope you are not yearning, fed up, and craving.’” } 

At\marginnote{1.14} that time \textsanskrit{Visākhā} \textsanskrit{Migāramātā} had many healthy children and grandchildren. As a consequence, she was considered auspicious. At sacrifices, ceremonies, and celebrations people would feed \textsanskrit{Visākhā} first. Just then she had been invited to that family that supported \textsanskrit{Udāyī}. When she arrived, she saw him sitting alone with that girl, and she said to him, “Venerable, it’s not appropriate for you to sit down in private alone with a woman on a concealed seat suitable for the deed. You may not be aiming at that act, but it’s hard to convince people with little confidence.” But \textsanskrit{Udāyī} did not listen. After leaving, \textsanskrit{Visākhā} told the monks what had happened. The monks of few desires complained and criticized him, “How could Venerable \textsanskrit{Udāyī} sit down in private alone with a woman on a concealed seat suitable for the deed?” 

After\marginnote{1.25} rebuking \textsanskrit{Udāyī} in many ways, they told the Buddha. Soon afterwards he had the Sangha gathered and questioned \textsanskrit{Udāyī}: “Is it true, \textsanskrit{Udāyī}, that you did this?” 

“It’s\marginnote{1.27} true, sir.” 

The\marginnote{1.28} Buddha rebuked him … “Foolish man, how could you do this? This will affect people’s confidence …” … “And, monks, this training rule should be recited like this: 

\subsection*{Final ruling }

\scrule{‘If a monk sits down in private alone with a woman on a concealed seat suitable for the deed, and a trustworthy female lay follower sees him and accuses him of an offense entailing expulsion, an offense entailing suspension, or an offense entailing confession, then, if he admits to the sitting, he is to be dealt with according to one of these three  or according to what that trustworthy female lay follower has said. This rule is undetermined.’” }

\subsection*{Definitions }

\begin{description}%
\item[A: ] whoever … %
\item[Monk: ] … The monk who has been given the full ordination by a unanimous Sangha through a legal procedure consisting of one motion and three announcements that is irreversible and fit to stand—this sort of monk is meant in this case. %
\item[A woman: ] a female human being, not a female spirit, not a female ghost, not a female animal; even a girl born that very day, let alone an older one. %
\item[With: ] together. %
\item[Alone: ] just the monk and the woman. %
\item[In private: ] private to the eye and private to the ear. %
\item[Private to the eye: ] one is unable to see them winking, raising an eyebrow, or nodding. %
\item[Private to the ear: ] one is unable to hear ordinary speech. %
\item[Concealed seat: ] it is concealed by a wall, a screen, a door, a cloth screen, a tree, a pillar, a grain container, or anything else. %
\item[Suitable for the deed: ] one is able to have sexual intercourse. %
\item[Sits down: ] the monk sits down or lies down next to the seated woman. The woman sits down or lies down next to the seated monk. Both are seated or both are lying down. %
\item[Trustworthy: ] she has attained the fruit, she has broken through, she has understood the instruction. %
\item[Female lay follower: ] she has gone for refuge to the Buddha, the Teaching, and the Sangha. %
\item[Sees: ] having seen. %
\item[If she accuses him of an offense entailing expulsion, an offense entailing suspension, or an offense entailing confession, then, if he admits to the sitting, he is to be dealt with according to one of these three or according to what that trustworthy female lay follower has said: ] If\marginnote{2.2.4} she accuses him like this: “I’ve seen you seated, having sexual intercourse with a woman,” and he admits to that, then he is to be dealt with for the offense. If she accuses him like this: “I’ve seen you seated, having sexual intercourse with a woman,” but he says, “It’s true that I was seated, but I didn’t have sexual intercourse,” then he is to be dealt with for the sitting. If she accuses him like this: “I’ve seen you seated, having sexual intercourse with a woman,” but he says, “I wasn’t seated, but lying down,” then he is to be dealt with for the lying down. If she accuses him like this: “I’ve seen you seated, having sexual intercourse with a woman,” but he says, “I wasn’t seated, but standing,” then he is not to be dealt with. 

If\marginnote{2.2.22} she accuses him like this: “I’ve seen you lying down, having sexual intercourse with a woman,” and he admits to that, then he is to be dealt with for the offense. If she accuses him like this: “I’ve seen you lying down, having sexual intercourse with a woman,” but he says, “It’s true that I was lying down, but I didn’t have sexual intercourse,” then he is to be dealt with for the lying down. If she accuses him like this: “I’ve seen you lying down, having sexual intercourse with a woman,” but he says, “I wasn’t lying down, but seated,” then he is to be dealt with for the sitting. If she accuses him like this: “I’ve seen you lying down, having sexual intercourse with a woman,” but he says, “I wasn’t lying down, but standing,” then he is not to be dealt with. 

If\marginnote{2.2.40} she accuses him like this: “I’ve seen you seated, making physical contact with a woman,” and he admits to that, then he is to be dealt with for the offense. … “It’s true that I was seated, but I didn’t make physical contact,” then he is to be dealt with for the sitting. … “I wasn’t seated, but lying down,” then he is to be dealt with for the lying down. … “I wasn’t seated, but standing,” then he is not to be dealt with. 

If\marginnote{2.2.49} she accuses him like this: “I’ve seen you lying down, making physical contact with a woman,” and he admits to that, then he is to be dealt with for the offense. … “It’s true that I was lying down, but I didn’t make physical contact,” then he is to be dealt with for the lying down. … “I wasn’t lying down, but seated,” then he is to be dealt with for the sitting. … “I wasn’t lying down, but standing,” then he is not to be dealt with. 

If\marginnote{2.2.58} she accuses him like this: “I’ve seen you seated in private alone with a woman on a concealed seat suitable for the deed,” and he admits to that, then he is to be dealt with for the sitting. … “I wasn’t seated, but lying down,” then he is to be dealt with for the lying down. … “I wasn’t seated, but standing,” then he is not to be dealt with. 

If\marginnote{2.2.65} she accuses him like this: “I’ve seen you lying down in private alone with a woman on a concealed seat suitable for the deed,” and he admits to that, then he is to be dealt with for the lying down. … “I wasn’t lying down, but seated,” then he is to be dealt with for the sitting. … “I wasn’t lying down, but standing,” then he is not to be dealt with. 

%
\item[Undetermined: ] not determined. It is either an offense entailing expulsion, an offense entailing suspension, or an offense entailing confession. %
\end{description}

\subsection*{Permutations }

If\marginnote{3.1} he admits to going, and he admits to sitting, and he admits to an offense, he is to be dealt with for the offense. If he admits to going, but he does not admit to sitting, yet he admits to an offense, he is to be dealt with for the offense. If he admits to going, and he admits to sitting, but he does not admit to an offense, he is to be dealt with for the sitting. If he admits to going, but he does not admit to sitting, nor does he admit to an offense, he is not to be dealt with. 

If\marginnote{3.5} he does not admit to going, but he admits to sitting, and he admits to an offense, he is to be dealt with for the offense. If he does not admit to going, nor does he admit to sitting, but he admits to an offense, he is to be dealt with for the offense. If he does not admit to going, but he admits to sitting, yet he does not admit to an offense, he is to be dealt with for the sitting. If he does not admit to going, nor does he admit to sitting, nor does he admit to an offense, he is not to be dealt with. 

\scendsutta{The first undetermined offense is finished. }

%
\section*{{\suttatitleacronym Bu Ay 2}{\suttatitletranslation The second undetermined training rule }{\suttatitleroot Dutiyaaniyata}}
\addcontentsline{toc}{section}{\tocacronym{Bu Ay 2} \toctranslation{The second undetermined training rule } \tocroot{Dutiyaaniyata}}
\markboth{The second undetermined training rule }{Dutiyaaniyata}
\extramarks{Bu Ay 2}{Bu Ay 2}

\subsection*{Origin story }

At\marginnote{1.1} one time the Buddha was staying at \textsanskrit{Sāvatthī} in the Jeta Grove, \textsanskrit{Anāthapiṇḍika}’s Monastery. At this time Venerable \textsanskrit{Udāyī} heard that the Buddha had prohibited sitting alone with a woman on a private and concealed seat suitable for the deed, and so instead he sat down in private alone with the same girl. When they were able, they just chatted; otherwise he gave her a teaching. 

A\marginnote{1.3} second time \textsanskrit{Visākhā} had been invited to that family. When she arrived, she saw \textsanskrit{Udāyī} sitting in private alone with the same girl, and she said to \textsanskrit{Udāyī}, “Venerable, it’s not appropriate for you to sit down in private alone with a woman. You may not be aiming at that act, but it’s hard to convince people with little confidence.” But \textsanskrit{Udāyī} did not listen. After leaving, \textsanskrit{Visākhā} told the monks what had happened. The monks of few desires complained and criticized him, “How could Venerable \textsanskrit{Udāyī} sit down in private alone with a woman?” 

After\marginnote{1.12} rebuking \textsanskrit{Udāyī} in many ways, they told the Buddha. Soon afterwards he had the Sangha gathered and questioned \textsanskrit{Udāyī}: “Is it true, \textsanskrit{Udāyī}, that you did this?” 

“It’s\marginnote{1.14} true, sir.” 

The\marginnote{1.15} Buddha rebuked him … “Foolish man, how could you do this? This will affect people’s confidence …” … “And, monks, this training rule should be recited like this: 

\subsection*{Final ruling }

\scrule{‘Although a seat is not concealed, nor suitable for the deed, it may be suitable for speaking indecently to a woman. If a monk sits down on such a seat in private alone with a woman, and a trustworthy female lay follower sees him and accuses him of an offense entailing suspension or an offense entailing confession, then, if he admits to the sitting, he is to be dealt with according to one of these two or according to what that trustworthy female lay follower has said. This rule too is undetermined.’” }

\subsection*{Definitions }

\begin{description}%
\item[Although a seat is not concealed: ] it is not concealed by a wall, a screen, a door, a cloth screen, a tree, a pillar, a grain container, or anything else. %
\item[Nor suitable for the deed: ] one is not able to have sexual intercourse. %
\item[It may be suitable for speaking indecently to a woman: ] one is able to speak indecently to a woman. %
\item[A: ] whoever … %
\item[Monk: ] … The monk who has been given the full ordination by a unanimous Sangha through a legal procedure consisting of one motion and three announcements that is irreversible and fit to stand—this sort of monk is meant in this case. %
\item[On such a seat: ] on that sort of seat. %
\item[A woman: ] a female human being, not a female spirit, not a female ghost, not a female animal. She understands and is capable of discerning bad speech and good speech, what is decent and what is indecent. %
\item[With: ] together. %
\item[Alone: ] just the monk and the woman. %
\item[In private: ] private to the eye and private to the ear. %
\item[Private to the eye: ] one is unable to see them winking, raising an eyebrow, or nodding. %
\item[Private to the ear: ] one is unable to hear ordinary speech. %
\item[Sits down: ] the monk sits down or lies down next to the seated woman. The woman sits down or lies down next to the seated monk. Both are seated or both are lying down. %
\item[Trustworthy: ] she has attained the fruit, she has broken through, she has understood the instruction. %
\item[Female lay follower: ] she has gone for refuge to the Buddha, the Teaching, and the Sangha. %
\item[Sees: ] having seen. %
\item[If she accuses him of an offense entailing suspension or an offense entailing confession, then, if he admits to the sitting, he is to be dealt with according to one of these two or according to what that trustworthy female lay follower has said. ] If\marginnote{2.2.4} she accuses him like this: “I’ve seen you seated, making physical contact with a woman,” and he admits to that, then he is to be dealt with for the offense. If she accuses him like this: “I’ve seen you seated, making physical contact with a woman,” but he says, “It’s true that I was seated, but I didn’t make physical contact,” then he is to be dealt with for the sitting. … “I wasn’t seated, but lying down,” then he is to be dealt with for the lying down. … “I wasn’t seated, but standing,” then he is not to be dealt with. 

If\marginnote{2.2.16} she accuses him like this: “I’ve seen you lying down, making physical contact with a woman,” and he admits to that, then he is to be dealt with for the offense. … “It’s true that I was lying down, but I didn’t make physical contact,” then he is to be dealt with for the lying down. … “I wasn’t lying down, but seated,” then he is to be dealt with for the sitting. … “I wasn’t seated, but standing,” then he is not to be dealt with. 

If\marginnote{2.2.25} she accuses him like this: “I’ve heard you speaking indecently to a woman while seated,” and he admits to that, then he is to be dealt with for the offense. If she accuses him like this: “I’ve heard you speaking indecently to a woman while seated,” but he says, “It’s true that I was seated, but I didn’t speak indecently,” then he is to be dealt with for the sitting. … “I wasn’t seated, but lying down,” then he is to be dealt with for the lying down. … “I wasn’t seated, but standing,” then he is not to be dealt with. 

If\marginnote{2.2.37} she accuses him like this: “I’ve heard you speaking indecently to a woman while lying down,” and he admits to that, then he is to be dealt with for the offense. … “It’s true that I was lying down, but I didn’t speak indecently,” then he is to be dealt with for the lying down. … “I wasn’t lying down, but seated,” then he is to be dealt with for the sitting. … “I wasn’t lying down, but standing,” then he is not to be dealt with. 

If\marginnote{2.2.46} she accuses him like this: “I’ve seen you seated in private alone with a woman,” and he admits to that, then he is to be dealt with for the sitting. … “I wasn’t seated, but lying down,” then he is to be dealt with for the lying down. … “I wasn’t seated, but standing,” then he is not to be dealt with. 

If\marginnote{2.2.53} she accuses him like this: “I’ve seen you lying down in private alone with a woman,” and he admits to that, then he is to be dealt with for the lying down. … “I wasn’t lying down, but seated,” then he is to be dealt with for the sitting. … “I wasn’t lying down, but standing,” then he is not to be dealt with. 

%
\item[This rule too: ] this is said with reference to the previous undetermined rule. %
\item[Undetermined: ] not determined. It is either an offense entailing suspension or an offense entailing confession. %
\end{description}

\subsection*{Permutations }

If\marginnote{3.1} he admits to going, and he admits to sitting, and he admits to an offense, he is to be dealt with for the offense. If he admits to going, but he does not admit to sitting, yet he admits to an offense, he is to be dealt with for the offense. If he admits to going, and he admits to sitting, but he does not admit to an offense, he is to be dealt with for the sitting. If he admits to going, but he does not admit to sitting, nor does he admit to an offense, he is not to be dealt with. 

If\marginnote{3.5} he does not admit to going, but he admits to sitting, and he admits to an offense, he is to be dealt with for the offense. If he does not admit to going, nor does he admit to sitting, but he admits to an offense, he is to be dealt with for the offense. If he does not admit to going, but he admits to sitting, yet he does not admit to an offense, he is to be dealt with for the sitting. If he does not admit to going, nor does he admit to sitting, nor does he admit to an offense, he is not to be dealt with. 

\scendsutta{The second undetermined offense is finished. }

“Venerables,\marginnote{3.10} the two undetermined rules have been recited. In regard to this I ask you, ‘Are you pure in this?’ A second time I ask, ‘Are you pure in this?’ A third time I ask, ‘Are you pure in this?’ You are pure in this and therefore silent. I’ll remember it thus.” 

\scuddanaintro{This is the summary: }

\begin{scuddana}%
“Suitable\marginnote{3.21} for the deed, \\
And then not so—\\
The undetermined offenses have been well laid down, \\
By the Stable One, the Buddha who is the best.” 

%
\end{scuddana}

\scendkanda{The chapter on undetermined offenses is finished. }

%
\backmatter%
%
\chapter*{Glossary}
\addcontentsline{toc}{chapter}{Glossary}
\markboth{Glossary}{Glossary}

This glossary focuses on Vinaya terminology or words that are used in a special sense in the Vinaya. With the latter kind of words, I normally only give the meaning that is applicable to the Vinaya, leaving our more general meanings or meanings that are relevant to other contexts. I also list some ordinary words that play a special role in the Vinaya \textsanskrit{Piṭaka}.

\begin{description}%
\item[\textit{\textsanskrit{aṁsabaddhaka}, \textsanskrit{aṁsabandhaka}}] a shoulder strap (see Appendix IV: Medical Terminology)%
\item[\textit{\textsanskrit{akaṭayūsa}}] mung-bean broth (see Appendix IV: Medical Terminology)%
\item[\textit{akata}] untreated (floor); not shaped (precious metal); invalid (legal procedure); not failed (in virtue)%
\item[\textit{akamma}] invalid (legal procedure)%
\item[\textit{\textsanskrit{akālacīvara}}] an out-of-season robe(-cloth); robe-cloth outside the robe season%
\item[\textit{akuppa}] irreversible (legal procedure) (see Appendix I: Technical Terms); composure%
\item[\textit{akka}] the crown flower (see Appendix V: Plants)%
\item[\textit{akkosa}] abuse, name-calling%
\item[\textit{akkhanti}] intolerance%
\item[\textit{\textsanskrit{agatiṁ} gacchati}] to be biased%
\item[\textit{\textsanskrit{agāmaka}}] an uninhabited area%
\item[\textit{\textsanskrit{agāra}}] a house, a home%
\item[\textit{\textsanskrit{aggaḷa}}] a door; a patch (of cloth) (see Appendix I: Technical Terms)%
\item[\textit{\textsanskrit{aggaḷaṁ} acchupeti}] to patch (cloth)%
\item[\textit{\textsanskrit{aggaḷavaṭṭi}}] a door jamb%
\item[\textit{\textsanskrit{aggiṭṭhāna}}] a fireplace%
\item[\textit{\textsanskrit{aggisālā}, \textsanskrit{agyāgāra}}] a water-boiling shed; a fire hut (see Appendix I: Technical Terms)%
\item[\textit{\textsanskrit{aṅgajāta}}] a penis; a vagina; genitals%
\item[\textit{\textsanskrit{aṅgavāta}}] arthritis of the hands and feet (see Appendix IV: Medical Terminology)%
\item[\textit{\textsanskrit{aṅgula}}] a fingerbreadth (approximately 1.67 cm) (see Appendix I: Technical Terms)%
\item[\textit{acittaka}] unintentionally%
\item[\textit{acelaka}] a naked ascetic%
\item[\textit{accaya}] mistake; offense%
\item[\textit{\textsanskrit{acchakañjī}}] clear rice broth (see Appendix IV: Medical Terminology)%
\item[\textit{acchindati}] to steal, to rob; to confiscate; to take back; to take%
\item[\textit{\textsanskrit{ajinappaveṇī}}] a rug made of black antelope hide (see Appendix III: Furniture)%
\item[\textit{ajjuka}] shrubby basil (see Appendix V: Plants)%
\item[\textit{\textsanskrit{ajjhācāra}}] misconduct; conduct%
\item[\textit{\textsanskrit{ajjhokāsa}, \textsanskrit{abbhokāsa}}] outside; out in the open (see Appendix I: Technical Terms)%
\item[\textit{\textsanskrit{ajjhohāra}}] a mouthful%
\item[\textit{\textsanskrit{añjana}}] an ointment%
\item[\textit{\textsanskrit{añjanitthavika}}] an ointment-box bag%
\item[\textit{\textsanskrit{añjanisalāka}}] an ointment stick%
\item[\textit{\textsanskrit{añjanī}}] an ointment box%
\item[\textit{\textsanskrit{añjalī}}] raising (one’s) joined palms (to someone)%
\item[\textit{\textsanskrit{aṭṭa},}] see \textit{\textsanskrit{aḍḍa}}%
\item[\textit{\textsanskrit{aṭṭakārikā},}] see \textit{\textsanskrit{aḍḍakārikā}}%
\item[\textit{\textsanskrit{aṭṭhapadaka}}] a darning, a cross weaving%
\item[\textit{\textsanskrit{aṭṭhānāraha}}] unfit to stand (an attribute of an improperly performed legal procedure)%
\item[\textit{\textsanskrit{aḍḍa}, \textsanskrit{aṭṭa}}] a lawsuit%
\item[\textit{\textsanskrit{aḍḍakārikā}, \textsanskrit{aṭṭakārikā}}] an initiator of a lawsuit%
\item[\textit{\textsanskrit{aḍḍhakusi}}] a short inter-panel strip (referring to robes)%
\item[\textit{\textsanskrit{aḍḍhapallaṅka}}] semi-cross-legged%
\item[\textit{\textsanskrit{aḍḍhamaṇḍala}}] a medium-sized panel (referring to robes)%
\item[\textit{\textsanskrit{aḍḍhayoga}}] a stilt house; a small stilt house (see Appendix I: Technical Terms)%
\item[\textit{atimuttaka}] the sandan tree (see Appendix V: Plants)%
\item[\textit{atireka}] extra, more than, more; at least, in excess of%
\item[\textit{\textsanskrit{ativisā}}] the atis root (see Appendix V: Plants)%
\item[\textit{atekiccha}] irredeemable%
\item[\textit{\textsanskrit{attādāna}}] raising an issue%
\item[\textit{\textsanskrit{attharaṇa}}] a sheet; a spread; a rug; bedding%
\item[\textit{\textsanskrit{adinnādāna}}] stealing%
\item[\textit{\textsanskrit{aduṭṭhulla}}] minor (offense); decent (speech)%
\item[\textit{\textsanskrit{addhānamaggappaṭipanna}}] traveling%
\item[\textit{adhamma}] illegitimate%
\item[\textit{\textsanskrit{adhammavādī}}] one who speaks contrary to the Teaching%
\item[\textit{\textsanskrit{adhikaraṇa}}] a legal issue, a legal case%
\item[\textit{\textsanskrit{adhikaraṇakāraka}}] one who creates legal issues%
\item[\textit{\textsanskrit{adhiṭṭhāti}}] to determine; to supervise%
\item[\textit{ananulomika}] improper, not proper%
\item[\textit{\textsanskrit{anapadāna}}] lacking in boundaries%
\item[\textit{anabhirati}] discontent (with the spiritual life)%
\item[\textit{\textsanskrit{anabhiratiyā} \textsanskrit{pīḷita}}] plagued by lust%
\item[\textit{\textsanskrit{anavasesā} \textsanskrit{āpatti}}] an incurable offense%
\item[\textit{\textsanskrit{anācāra}}] misbehavior%
\item[\textit{\textsanskrit{anāvāsa}}] a non-monastery%
\item[\textit{\textsanskrit{animittā}}] a woman without genitals%
\item[\textit{\textsanskrit{animittamattā}}] a woman with incomplete genitals%
\item[\textit{aniyata}] undetermined%
\item[\textit{\textsanskrit{anujānanā}}] an allowance; an instruction (to act in a certain way)%
\item[\textit{\textsanskrit{anujānāti}}] to allow, to permit; should (plus main verb), to require, to instruct (see Appendix I: Technical Terms)%
\item[\textit{\textsanskrit{anudūta}}] a companion messenger%
\item[\textit{\textsanskrit{anuddhaṁseti}}] to charge with%
\item[\textit{\textsanskrit{anupaññatti}}] an addition to a rule%
\item[\textit{anuposathika}] on every observance day%
\item[\textit{\textsanskrit{anubyañjana}}] the next phrase; a detailed exposition%
\item[\textit{\textsanskrit{anumodanā}}] an expression of appreciation%
\item[\textit{anuvattaka}] to side with%
\item[\textit{anuvassa}] every year, annual%
\item[\textit{\textsanskrit{anuvāta}}] a lengthwise border (referring to robes); a lengthwise edge (of a wooden frame)%
\item[\textit{\textsanskrit{anuvāda}}] an accusation%
\item[\textit{\textsanskrit{anuvādaṁ} \textsanskrit{paṭṭhapeti}}] to give instructions to%
\item[\textit{\textsanskrit{anuvādādhikaraṇa}}] a legal issue arising from an accusation%
\item[\textit{\textsanskrit{anuvivaṭṭa}}] an intermediate section (referring to robes)%
\item[\textit{\textsanskrit{anussāvana}}] an announcement; a proclamation%
\item[\textit{\textsanskrit{antagaṇṭhābādha}}] a twisted gut%
\item[\textit{\textsanskrit{antaggāhikādiṭṭhi}}] a side view%
\item[\textit{antaraghara}] an inhabited area%
\item[\textit{\textsanskrit{antaravāsaka}}] a sarong%
\item[\textit{\textsanskrit{antarāya}}] an obstacle; a danger, a threat%
\item[\textit{antarika dhamma}] an obstacle (to ordination)%
\item[\textit{antimavatthu}] the worst kind of offense%
\item[\textit{antepura}] a (royal) compound (see Appendix I: Technical Terms)%
\item[\textit{\textsanskrit{antevāsī}, \textsanskrit{antevāsinī}}] a pupil%
\item[\textit{apacaya}] reduction in things%
\item[\textit{\textsanskrit{apamāra}}] epilepsy%
\item[\textit{\textsanskrit{aparaṇṇa}}] vegetables%
\item[\textit{\textsanskrit{aparimāṇa}}] unspecified (offense)%
\item[\textit{apalokanakamma}] a legal procedure consisting of getting permission%
\item[\textit{apaloketi}] to get permission; to take leave%
\item[\textit{apassenaphalaka}] a leaning board%
\item[\textit{\textsanskrit{apidhāna}}] a lid, a cover%
\item[\textit{\textsanskrit{appaṭikamma}}] not making amends (after committing an offense)%
\item[\textit{\textsanskrit{appaṭicchanna}}] unconcealed (offense)%
\item[\textit{appamattakavissajjaka}] a distributor of minor requisites%
\item[\textit{\textsanskrit{appamāṇikā}}] inappropriate in size%
\item[\textit{appaharite}] no cultivated plants%
\item[\textit{\textsanskrit{appāṇaka}}] without life, not containing living beings%
\item[\textit{abbhantara}] 11.2 meters (approximately) (see Appendix I: Technical Terms)%
\item[\textit{\textsanskrit{abbhāna}}] rehabilitation%
\item[\textit{\textsanskrit{abbhokāsa},}] see \textit{\textsanskrit{ajjhokāsa}}%
\item[\textit{abyatta, avyatta}] incompetent%
\item[\textit{\textsanskrit{abyākata}}] indeterminate%
\item[\textit{abhidhamma}] philosophy%
\item[\textit{\textsanskrit{abhivādeti}}] to bow down to%
\item[\textit{\textsanskrit{abhisannakāya}}] full of bodily impurities (see Appendix IV: Medical Terminology)%
\item[\textit{amagga}] the mouth (see Appendix I: Technical Terms)%
\item[\textit{amata}] freedom from death%
\item[\textit{amanussa}] a spirit%
\item[\textit{\textsanskrit{amanussikābādha}}] spirit possession (see Appendix IV: Medical Terminology)%
\item[\textit{\textsanskrit{amūlaka}}] groundless, without proper reason%
\item[\textit{\textsanskrit{amūḷha}}] no longer insane; non-delusion, free from delusion%
\item[\textit{\textsanskrit{amūḷhavinaya}}] resolution because of past insanity%
\item[\textit{ayo}] iron%
\item[\textit{\textsanskrit{arañña}}] the wilderness%
\item[\textit{\textsanskrit{araṇisahita}}] a fire-making implement%
\item[\textit{araha}] to deserve, eligible; the value; subject to%
\item[\textit{ariyaka}] an Indo-Aryan (person)%
\item[\textit{\textsanskrit{aruṇa}, \textsanskrit{aruṇuggamana}}] dawn%
\item[\textit{\textsanskrit{alajjī}}] shameless%
\item[\textit{\textsanskrit{avakkārapāti}}] a bowl for leftovers%
\item[\textit{avalitta}] plastered outside%
\item[\textit{\textsanskrit{avalekhanakaṭṭha}}] a wiping stick%
\item[\textit{avassuta}] having lust%
\item[\textit{\textsanskrit{avāpuraṇa}}] a key%
\item[\textit{avinaya}] contrary to the Monastic Law%
\item[\textit{\textsanskrit{avinayavādī}}] one who speaks contrary to the Monastic Law%
\item[\textit{\textsanskrit{avippavāsa}}] may-stay-apart (zone)%
\item[\textit{avyatta,}] see \textit{abyatta}%
\item[\textit{\textsanskrit{asaṁvāsa}}] excluded from the community%
\item[\textit{asakkacca}] contemptuously%
\item[\textit{asambhoga}] prohibited from living with; prohibited from interacting with%
\item[\textit{asammukha}] absence%
\item[\textit{asura}] an antigod%
\item[\textit{assattha}] the Bodhi tree (see Appendix V: Plants)%
\item[\textit{assatthara}] a horse-back rug (see Appendix III: Furniture)%
\item[\textit{\textsanskrit{assāmaṇaka}}] not worthy of an ascetic%
\item[\textit{\textsanskrit{assāvo}}] a running sore%
\item[\textit{ahata}] new%
\item[\textit{ahatakappa}] nearly new%
\item[\textit{\textsanskrit{ākāra}}] a motive%
\item[\textit{\textsanskrit{āgatāgama}}] a master of the tradition%
\item[\textit{\textsanskrit{āgantuka}}] newly arrived, just arrived, a new arrival%
\item[\textit{\textsanskrit{āgama}}] a tradition%
\item[\textit{\textsanskrit{ācamana}}] washing (after defecating)%
\item[\textit{\textsanskrit{ācamanakumbhī}}] a restroom ablutions pot%
\item[\textit{\textsanskrit{ācamanapāduka}}] an ablution foot stand%
\item[\textit{\textsanskrit{ācamanasarāvaka}}] a (restroom) ablutions scoop%
\item[\textit{\textsanskrit{ācarinī}}] a (female) teacher%
\item[\textit{\textsanskrit{ācariya}}] a teacher%
\item[\textit{\textsanskrit{āciṇṇa}}] a custom; practiced%
\item[\textit{\textsanskrit{āṇicoḷaka}}] a menstruation pad%
\item[\textit{\textsanskrit{ādikammika}}] a first offender%
\item[\textit{\textsanskrit{āpaṇasālā}}] a shop%
\item[\textit{\textsanskrit{āpattādhikaraṇa}}] a legal issue arising from an offense%
\item[\textit{\textsanskrit{āpatti}}] an offense%
\item[\textit{\textsanskrit{āpatti} \textsanskrit{paṭiggaṇhāti}}] to receive a confession (of an offense)%
\item[\textit{\textsanskrit{āpattiṁ} ropeti, \textsanskrit{āropeti}}] to charge with an offense%
\item[\textit{\textsanskrit{āpattikkhandha}}] a class of offenses%
\item[\textit{\textsanskrit{āpadā}}] an emergency (see Appendix I: Technical Terms)%
\item[\textit{\textsanskrit{āpucchati}}] to ask; to ask permission; to inform; to take leave%
\item[\textit{\textsanskrit{ābādha}}] a disease, an illness%
\item[\textit{\textsanskrit{āmakamaṁsa}}] raw meat%
\item[\textit{\textsanskrit{āmakalohita}}] raw blood%
\item[\textit{\textsanskrit{āmalaka}}] emblic myrobalan (see Appendix V: Plants)%
\item[\textit{\textsanskrit{āmalakavaṭṭika} \textsanskrit{pīṭha}}] a bench with many legs (see Appendix III: Furniture)%
\item[\textit{\textsanskrit{āmisa}}] food; worldly gain; material (thing); a requisite%
\item[\textit{\textsanskrit{āmisakhāra}}] lye%
\item[\textit{\textsanskrit{āyoga}}] a back-and-knee strap%
\item[\textit{\textsanskrit{āraññika}}] a wilderness dweller, one who stays in the wilderness%
\item[\textit{\textsanskrit{ārāma}}] a park; a monastery (see Appendix I: Technical Terms)%
\item[\textit{\textsanskrit{ārāmika}}] a monastery worker%
\item[\textit{\textsanskrit{āropeti},}] see \textit{\textsanskrit{āpatti} ropeti}%
\item[\textit{\textsanskrit{ālambanabāha}}] a rail%
\item[\textit{\textsanskrit{āḷinda}}] a porch%
\item[\textit{\textsanskrit{ālepa}}] an ointment%
\item[\textit{\textsanskrit{ālokasandhi}}] a window; a window opening%
\item[\textit{\textsanskrit{āvaraṇa}}] a restriction%
\item[\textit{\textsanskrit{āvasatha}}] a (public) guesthouse; lodging%
\item[\textit{\textsanskrit{āvasathacīvara}}] a communal robe%
\item[\textit{\textsanskrit{āvasathāgāra}}] a guesthouse%
\item[\textit{\textsanskrit{āvāsa}}] a monastery%
\item[\textit{\textsanskrit{āvāsika}}] a resident (monastic); a local (monastic)%
\item[\textit{\textsanskrit{āvikaroti}}] to reveal (an offense); to state (a view)%
\item[\textit{\textsanskrit{āviñchanachidda}, \textsanskrit{āviñchanacchidda}}] a door-pulling hole%
\item[\textit{\textsanskrit{āviñchanarajju}}] a door-pulling rope%
\item[\textit{\textsanskrit{āsana}}] a seat%
\item[\textit{\textsanskrit{āsandika}}] a square bench (see Appendix III: Furniture)%
\item[\textit{\textsanskrit{āsandī}}] a high couch (see Appendix III: Furniture)%
\item[\textit{\textsanskrit{āsāvacchedikā}}] ending when an expectation is disappointed (referring to the robe season)%
\item[\textit{\textsanskrit{ikkāsa}}] sap%
\item[\textit{\textsanskrit{iṭṭhakā}}] a brick; a tile%
\item[\textit{\textsanskrit{itthannāma}}] so-and-so; such-and-such%
\item[\textit{\textsanskrit{itthipaṇḍaka}}] a woman who lacks sexual organs%
\item[\textit{\textsanskrit{ukkā}}] a torch%
\item[\textit{\textsanskrit{ukkuṭika} \textsanskrit{nisīdati}}] squat on the heels%
\item[\textit{\textsanskrit{ukkoṭanaka}}] reopening (of a settled legal issue)%
\item[\textit{ukkhitta}] ejected%
\item[\textit{ukkhepaka}] one who has ejected (someone else)%
\item[\textit{\textsanskrit{ukkhepanīya}}] ejection%
\item[\textit{\textsanskrit{uklāpa}}] dirty%
\item[\textit{\textsanskrit{ugghaṁseti}}] to rub%
\item[\textit{\textsanskrit{uccāra}}] feces%
\item[\textit{\textsanskrit{uccāvaca}}] luxurious; various%
\item[\textit{ucchu}] a sugarcane%
\item[\textit{\textsanskrit{ujjhāyati}, \textsanskrit{ujjhāna}, \textsanskrit{ujjhāpanaka}}] to complain; to find fault%
\item[\textit{utu}] a season%
\item[\textit{\textsanskrit{utunī}}] menstruating; the fertile period%
\item[\textit{\textsanskrit{utuppamāna}}] a date%
\item[\textit{\textsanskrit{uttarattharaṇa}}] a bedspread%
\item[\textit{\textsanskrit{uttarapāsaka}}] an upper hinge%
\item[\textit{\textsanskrit{uttarāsaṅga}}] an upper robe%
\item[\textit{\textsanskrit{uttaribhaṅga}}] a non-bean curry, a special curry%
\item[\textit{uttarimanussadhamma}] a superhuman quality%
\item[\textit{\textsanskrit{udakakoṭṭhaka}}] a bathtub (see Appendix IV: Medical Terminology)%
\item[\textit{\textsanskrit{udakapuñchanī}}] a water wiper%
\item[\textit{\textsanskrit{udakasāṭika}}] a bathing robe (for nuns)%
\item[\textit{\textsanskrit{udapāna}}] a well%
\item[\textit{\textsanskrit{udapānasāla}}] a well house%
\item[\textit{\textsanskrit{udaravātābādha}}] a stomachache%
\item[\textit{udukkhala}] a mortar%
\item[\textit{udukkhalika}] a lower hinge%
\item[\textit{udumbara}] a cluster fig (see Appendix V: Plants)%
\item[\textit{uddisati}] to recite; to teach; to designate%
\item[\textit{uddesa}] recitation%
\item[\textit{uddesaka}] a designator; a reciter%
\item[\textit{uddesabhatta}] a meal for designated monastics%
\item[\textit{uddosita, udosita}] a storehouse%
\item[\textit{\textsanskrit{uddhalomī}}] a woolen rug with long fleece on one side (see Appendix III: Furniture)%
\item[\textit{\textsanskrit{uddhāra}}] end (of the robe season)%
\item[\textit{\textsanskrit{upacāra}}] vicinity; access (see Appendix I: Technical Terms)%
\item[\textit{\textsanskrit{upajjhāya}}] a preceptor%
\item[\textit{\textsanskrit{upaṭṭhāka}}] an attendant; a nurse; a supporter%
\item[\textit{\textsanskrit{upaṭṭhāti}, \textsanskrit{upaṭṭhahati}}] to attend on; to nurse; to support%
\item[\textit{\textsanskrit{upaṭṭhānasālā}}] an assembly hall%
\item[\textit{\textsanskrit{upadhāna}}] a cushion%
\item[\textit{upabhoga}] what is valuable%
\item[\textit{\textsanskrit{upalāpeti}}] to befriend; to bribe%
\item[\textit{\textsanskrit{upasampadā}}] the full ordination%
\item[\textit{upassaya}] a dwelling place (for nuns)%
\item[\textit{\textsanskrit{upāsaka}}] a (male) lay follower%
\item[\textit{\textsanskrit{upāsika}}] a female lay follower%
\item[\textit{\textsanskrit{upāhana}}] a sandal%
\item[\textit{\textsanskrit{upāhanāpuñchanacoḷaka}}] a sandal-wiping cloth%
\item[\textit{uposatha}] the observance day; the observance-day ceremony (see Appendix I: Technical Terms)%
\item[\textit{\textsanskrit{uposathāgāra}, uposathagga}] the observance-day hall%
\item[\textit{uposathika}] on the observance day%
\item[\textit{uppajjati}] to be offered, to be given (to monastics); to obtain; to get; to acquire (faith)%
\item[\textit{uppanna}] offered, given; obtained%
\item[\textit{\textsanskrit{ubbāhika}}] a committee%
\item[\textit{\textsanskrit{ubbhāra}}] end (of the robe season)%
\item[\textit{ubbhida}] soil salt%
\item[\textit{\textsanskrit{ubhatobyañjanaka}}] a hermaphrodite (see Appendix I: Technical Terms)%
\item[\textit{\textsanskrit{ubhatolohitakūpadhāna}}] a seat with red cushions at each end (see Appendix III: Furniture)%
\item[\textit{ummatta, ummattaka}] insane%
\item[\textit{\textsanskrit{uyyāna}}] a park%
\item[\textit{ullitta}] plastered inside%
\item[\textit{\textsanskrit{ullittāvalitta}}] plastered inside and outside%
\item[\textit{ulloka}] a ceiling cloth; an underlay%
\item[\textit{\textsanskrit{usīra}}] vetiver root; vetiver grass (see Appendix V: Plants)%
\item[\textit{\textsanskrit{ussacālinī}}] a cloth sieve%
\item[\textit{\textsanskrit{ussayavādikā}}] one who takes legal action%
\item[\textit{\textsanskrit{ussāvanantika}}] a (food-store) building made according to a proclamation%
\item[\textit{\textsanskrit{ekatoupasampannā}}] fully ordained (only) on one side%
\item[\textit{\textsanskrit{ekantalomī}}] a woolen rug with long fleece on both side (see Appendix III: Furniture)%
\item[\textit{ekuddesa}] a joint recitation%
\item[\textit{\textsanskrit{okāsa}}] permission; estate%
\item[\textit{\textsanskrit{okāsaṁ} \textsanskrit{kāreti}}] to get permission (from someone to correct them)%
\item[\textit{\textsanskrit{oguṇṭhitāsīsa}}] wearing headgear%
\item[\textit{ogumpheti}] to firm up (the structure of a building)%
\item[\textit{\textsanskrit{oṇojana}}] a meal invitation%
\item[\textit{\textsanskrit{otiṇṇa}}] overcome by lust%
\item[\textit{odana}] cooked grain; boiled rice, rice%
\item[\textit{onaddha}] upholstered%
\item[\textit{\textsanskrit{onītapattapāṇi},}] see \textit{\textsanskrit{bhuttāvī}}%
\item[\textit{omasati}] to speak abusively; to stroke downward%
\item[\textit{omukka}] second-hand (sandals); loose%
\item[\textit{\textsanskrit{ovaṭṭika}}] a hem%
\item[\textit{ovadati}] to instruct%
\item[\textit{ovaddheyya}] a patch (for a robe)%
\item[\textit{\textsanskrit{ovāda}}] an instruction, a half-monthly instruction%
\item[\textit{\textsanskrit{ovādaka}}] an instructor%
\item[\textit{\textsanskrit{osāraka}}] an entrance roof%
\item[\textit{\textsanskrit{osāraṇā}}] admittance, readmittance%
\item[\textit{\textsanskrit{kaṁsa}}] bronze; a bronze coin%
\item[\textit{kakudha}] the arjun tree (see Appendix V: Plants)%
\item[\textit{kacchaka}] the Indian cedar tree (see Appendix V: Plants)%
\item[\textit{kacchu}] an itch (see Appendix IV: Medical Terminology)%
\item[\textit{\textsanskrit{kacchurogābādha}}] an itchy skin disease%
\item[\textit{\textsanskrit{kañcuka}}] a close-fitting jacket%
\item[\textit{\textsanskrit{kañjika}}] rice broth%
\item[\textit{\textsanskrit{kaṭākaṭa}}] oily mung-bean broth (see Appendix IV: Medical Terminology)%
\item[\textit{\textsanskrit{kaṭāha}}] a bowl; a pot%
\item[\textit{\textsanskrit{kaṭisuttaka}}] a girdle%
\item[\textit{\textsanskrit{kaṭukarohiṇī}}] black hellebore (see Appendix V: Plants)%
\item[\textit{\textsanskrit{kaṭula}}] pungent%
\item[\textit{\textsanskrit{kaṭṭissa}}] a sheet of silk embroidered with gems (see Appendix III: Furniture)%
\item[\textit{\textsanskrit{kaṭhala}, kathala}] a pebble%
\item[\textit{\textsanskrit{kaṇḍu}}] an itch (see Appendix IV: Medical Terminology)%
\item[\textit{\textsanskrit{kaṇḍuppaṭicchādi}}] an itch-covering cloth%
\item[\textit{\textsanskrit{kaṇḍusa}, \textsanskrit{kaṇḍusaka}}] a strip of cloth for marking%
\item[\textit{\textsanskrit{kaṇṇakita}}] moldy; rusted%
\item[\textit{\textsanskrit{kaṇṇamalaharaṇi}}] an earpick%
\item[\textit{kataka}] a ceramic foot scrubber%
\item[\textit{\textsanskrit{katikā}, \textsanskrit{katikasaṇṭhāna}}] an agreement%
\item[\textit{\textsanskrit{kattaradaṇḍa}}] a walking stick%
\item[\textit{kathina}] a (robe-making) frame; the robe-making ceremony; the robe season (see Appendix I: Technical Terms)%
\item[\textit{kathina + attharati}] to perform the robe-making ceremony, to participate in the robe-making ceremony%
\item[\textit{kathina + uddharati}] to end the robe season%
\item[\textit{kathina + \textsanskrit{ubbhāra}}] to end the robe season%
\item[\textit{\textsanskrit{kathinatthāraka}}] one who has performed the robe-making ceremony%
\item[\textit{\textsanskrit{kathinasāla}}] a sewing shed%
\item[\textit{\textsanskrit{kadalīmigapavarapaccattharaṇa}}] a seat with red cushions at each end (see Appendix III: Furniture)%
\item[\textit{kapalla}] soot%
\item[\textit{kapitthana, \textsanskrit{kapītana}}] the portia tree (see Appendix V: Plants)%
\item[\textit{\textsanskrit{kapisīsaka}}] a bolt socket%
\item[\textit{kappa}] a mark (on a monastic robe)%
\item[\textit{\textsanskrit{kappāsa}}] a cotton plant (see Appendix V: Plants)%
\item[\textit{\textsanskrit{kappāsika}}] cotton%
\item[\textit{kappiya}] allowable%
\item[\textit{\textsanskrit{kappiyakāraka}}] an attendant, a service provider%
\item[\textit{\textsanskrit{kappiyakuṭi}}] a food-storage hut%
\item[\textit{\textsanskrit{kappiyabhūmi}}] a food-storage area%
\item[\textit{\textsanskrit{kabaḷika}}] flour paste (for treating a wound) (see Appendix IV: Medical Terminology)%
\item[\textit{kamala}] grass (of a particular kind)%
\item[\textit{kambala}] wool, a woolen cloth; a woolen cloak; a sarong%
\item[\textit{kamma}] a legal procedure, a procedure; occupation%
\item[\textit{kammapatta}] who should take part in a legal procedure; able to do a legal procedure%
\item[\textit{\textsanskrit{kammavācā}}] an announcement%
\item[\textit{\textsanskrit{kammāraha}}] deserving a legal procedure (done against them)%
\item[\textit{\textsanskrit{kayavikkayaṁ} \textsanskrit{samāpajjati}}] to barter%
\item[\textit{\textsanskrit{karaṇīya}}] business, something to be done; need%
\item[\textit{\textsanskrit{kavaṭa}}] weft (the crosswise thread on a loom)%
\item[\textit{\textsanskrit{kavāṭa}}] a door; a (window) shutter%
\item[\textit{\textsanskrit{kasāva}}] bitter (substance)%
\item[\textit{\textsanskrit{kasāvodaka}}] astringent water%
\item[\textit{\textsanskrit{kāca}}] glass%
\item[\textit{\textsanskrit{kāja}}] a carrying pole%
\item[\textit{\textsanskrit{kāṇo}}] one who is blind in one eye%
\item[\textit{\textsanskrit{kāmabhogī}, \textsanskrit{kāmabhoginī}}] one who indulges in worldly pleasures%
\item[\textit{\textsanskrit{kāyabandhana}}] a belt%
\item[\textit{\textsanskrit{kāreti}}] to deal with (as prescribed by the Monastic Law)%
\item[\textit{\textsanskrit{kāla}}] the right time, the appropriate time; (out of or in) season; in the morning, before midday%
\item[\textit{\textsanskrit{kāḷa}}] the waning phase of the moon%
\item[\textit{\textsanskrit{kālacīvara}}] in-season robe(-cloth)%
\item[\textit{\textsanskrit{kāḷañjana}}] a black ointment%
\item[\textit{\textsanskrit{kāḷānusāriya}}] Indian valerian (see Appendix IV: Medical Terminology)%
\item[\textit{\textsanskrit{kāsāya}}] ocher%
\item[\textit{kicca}] business; a duty%
\item[\textit{\textsanskrit{kiccādhikaraṇa}}] a legal issue arising from business%
\item[\textit{\textsanskrit{kiṭika}}] a screen%
\item[\textit{\textsanskrit{kilañja}}] a screen; reed%
\item[\textit{\textsanskrit{kilāsa}}] mild leprosy (see Appendix I: Technical Terms)%
\item[\textit{kukkucca}] anxiety%
\item[\textit{kukkuccaka}] (a monastic) who is afraid of wrongdoing%
\item[\textit{\textsanskrit{kucchivikāra}}] dysentery%
\item[\textit{\textsanskrit{kuṭaja}}] the arctic snow (see Appendix V: Plants)%
\item[\textit{\textsanskrit{kuṭi}, \textsanskrit{kuṭika}}] a hut%
\item[\textit{\textsanskrit{kuṭṭa}, \textsanskrit{kuḍḍa}}] a wall%
\item[\textit{\textsanskrit{kuṭṭha}}] leprosy (see Appendix I: Technical Terms)%
\item[\textit{\textsanskrit{kuṭhārī}}] an ax%
\item[\textit{\textsanskrit{kuḍḍa},}] see \textit{\textsanskrit{kuṭṭa}}%
\item[\textit{\textsanskrit{kuṇī}}] one with a crooked limb%
\item[\textit{kuttaka}] a woolen rug like a dancer’s rug (see Appendix III: Furniture)%
\item[\textit{\textsanskrit{kudāla}}] a spade%
\item[\textit{kuppa}] reversible (see Appendix I: Technical Terms); to disturb; (to make) a scene%
\item[\textit{\textsanskrit{kumāribhūta}}] an unmarried woman%
\item[\textit{kumbha, \textsanskrit{kumbhī}}] a pot, a ceramic pot; a jar; a container; a forehead%
\item[\textit{\textsanskrit{kumbhakāra}}] a potter%
\item[\textit{kulaputta}] gentleman%
\item[\textit{\textsanskrit{kuḷīrapādaka}}] having crooked legs (of a bed or bench) (see Appendix III: Furniture)%
\item[\textit{\textsanskrit{kulūpaka}}] (a monastic) who associates with families%
\item[\textit{\textsanskrit{kusapāta} (karoti)}] to draw lots%
\item[\textit{kusi}] a long inter-panel strip (referring to robes)%
\item[\textit{koccha}] a stool; a (hair) brush%
\item[\textit{kojava}] a woolen fleecy robe%
\item[\textit{\textsanskrit{koṭṭhaka}}] a gatehouse (see Appendix I: Technical Terms)%
\item[\textit{kolamba}] a basin%
\item[\textit{\textsanskrit{kolāhala}}] an uproar, a racket%
\item[\textit{\textsanskrit{koviḷāra},}] see \textit{\textsanskrit{pāricchattaka}}%
\item[\textit{koseyya}] silk; a silken sheet (see Appendix III: Furniture)%
\item[\textit{khajja, khajjaka}] fresh food%
\item[\textit{\textsanskrit{khañja}}] one who is lame%
\item[\textit{khanti}] a belief (in connection with \textit{ruci} and \textit{\textsanskrit{diṭṭhi}}); patience%
\item[\textit{khamati}] to forgive; to approve, to agree to%
\item[\textit{\textsanskrit{khamāpeti}}] to ask for forgiveness%
\item[\textit{\textsanskrit{khādanīya}}] fresh food (see Appendix I: Technical Terms)%
\item[\textit{\textsanskrit{khādanīya} + \textsanskrit{bhojanīya}}] food, various kinds of food%
\item[\textit{khiyyati, khiyyanaka, \textsanskrit{khīyana}}] to criticize%
\item[\textit{khittacitta}] deranged%
\item[\textit{\textsanskrit{khīra}}] milk%
\item[\textit{\textsanskrit{khuddānukhuddaka}}] minor (training rule)%
\item[\textit{khura}] a razor%
\item[\textit{\textsanskrit{khurabhaṇḍa}}] barber equipment%
\item[\textit{khurasila}] a whetstone%
\item[\textit{\textsanskrit{kheḷamallaka}}] a spittoon%
\item[\textit{khoma}] linen, a linen cloak%
\item[\textit{\textsanskrit{gaṇa}}] a group, a community, several (monastics); companions%
\item[\textit{\textsanskrit{gaṇṭhika}}] a toggle%
\item[\textit{\textsanskrit{gaṇṭhikaphalaka}}] a toggle shield (for a robe)%
\item[\textit{\textsanskrit{gaṇḍa}}] an abscess (see Appendix I: Technical Terms)%
\item[\textit{\textsanskrit{gaṇḍābādha}}] ulcer disease%
\item[\textit{gandha}] scent, perfume%
\item[\textit{\textsanskrit{gabbhapātana}, \textsanskrit{gabbhaṁ} \textsanskrit{pāteti}}] abortion%
\item[\textit{gamika}] departing (monastic)%
\item[\textit{garuka}] a heavy (offense); serious (matter); valuable%
\item[\textit{garudhamma}] an important principle (to be kept by the nuns); a heavy offense%
\item[\textit{\textsanskrit{garubhaṇḍa}}] valuable goods, a valuable belonging%
\item[\textit{\textsanskrit{gahapatānī}}] a female householder%
\item[\textit{gahapati}] a householder; the head of a family; a building given by a householder (to serve as a food store)%
\item[\textit{\textsanskrit{gahapaticīvara}}] a robe (or robe-cloth) given by a householder%
\item[\textit{\textsanskrit{gāma}}] a village, an inhabited area, habitation (see Appendix I: Technical Terms)%
\item[\textit{\textsanskrit{gāmantara}}] the next village%
\item[\textit{\textsanskrit{gāhāpaka}}] a distributor, an allocator%
\item[\textit{\textsanskrit{gāheti}}] to allocate, to distribute, to give away; to offer%
\item[\textit{gimha}] the summer%
\item[\textit{\textsanskrit{gilāna}}] sick%
\item[\textit{\textsanskrit{gilānupaṭṭhāka}}] a nurse%
\item[\textit{gihigata}] a married girl (see Appendix I: Technical Terms)%
\item[\textit{\textsanskrit{gihī}}] a householder%
\item[\textit{\textsanskrit{gīveyyaka}}] a strengthening piece for the neck (referring to robes)%
\item[\textit{\textsanskrit{guṇa}}] a layer%
\item[\textit{\textsanskrit{guṇaṅguṇa}}] multi-layered (soles)%
\item[\textit{gutta}] protected, guarded, looked after%
\item[\textit{\textsanskrit{guḷa}}] sugar%
\item[\textit{\textsanskrit{guhā}}] a cave (see Appendix I: Technical Terms)%
\item[\textit{\textsanskrit{gūtha}}] feces%
\item[\textit{geruka}] red ocher%
\item[\textit{\textsanskrit{goṭṭhaphala}}] the crepe ginger (see Appendix V: Plants)%
\item[\textit{gotta}] category (of offense)%
\item[\textit{gonaka}] a long-fleeced woolen rug (see Appendix III: Furniture)%
\item[\textit{\textsanskrit{gonisādika}}] a cow stall (used to store food)%
\item[\textit{golomika}] goatee%
\item[\textit{\textsanskrit{ghaṁsati}}] to rub; to scratch; to scrub; to grind%
\item[\textit{\textsanskrit{ghaṭa}, \textsanskrit{ghaṭaka}, \textsanskrit{ghaṭi}}] a waterpot, a pot%
\item[\textit{\textsanskrit{ghaṭika}}] a latch (for a door)%
\item[\textit{\textsanskrit{gharadinnakābādha}}] sickness from a drug (see Appendix IV: Medical Terminology)%
\item[\textit{cakkabheda}] a break in authority (in the Sangha)%
\item[\textit{\textsanskrit{cakkhurogābādho}}] eye disease%
\item[\textit{\textsanskrit{caṅkama}}] a walking-meditation path%
\item[\textit{\textsanskrit{caṅkamati}}] to walk back and forth, to do walking meditation%
\item[\textit{\textsanskrit{caṅkamanasālā}}] an indoor walking-meditation path%
\item[\textit{caturassaka}] sideburns%
\item[\textit{candana}] sandalwood, sandal%
\item[\textit{candanika}] a waste-water collection tank%
\item[\textit{camma}] a hide, a skin; leather; a shield%
\item[\textit{\textsanskrit{cammakhaṇḍa}}] a hide%
\item[\textit{caya}] a (raised) foundation; a mound%
\item[\textit{\textsanskrit{cāṭi}}] a large earthenware pot%
\item[\textit{\textsanskrit{cātuddisa}}] everywhere%
\item[\textit{\textsanskrit{cāpalasuṇa}}] spring onion (see Appendix V: Plants)%
\item[\textit{\textsanskrit{cārika}}] wandering%
\item[\textit{\textsanskrit{cālinī}}] a sieve%
\item[\textit{cikkhalla}] mud, muddy%
\item[\textit{cittaka}] a multi-colored woolen rug (see Appendix III: Furniture)%
\item[\textit{\textsanskrit{cittavipariyāsakata}}] suffering from psychosis%
\item[\textit{\textsanskrit{cittāgāra}}] a gallery%
\item[\textit{cimilika}] a mat underlay (see Appendix III: Furniture)%
\item[\textit{\textsanskrit{cīra}}] a robe%
\item[\textit{\textsanskrit{cīvara}}] a robe; a robe-cloth (see Appendix I: Technical Terms)%
\item[\textit{\textsanskrit{cīvarakārasamaya}}] a time of making robes%
\item[\textit{\textsanskrit{cīvarakālasamaya}}] the robe season%
\item[\textit{\textsanskrit{cīvaranidahaka}}] a keeper of robe-cloth%
\item[\textit{\textsanskrit{cīvararajju}}] a clothesline (for robes)%
\item[\textit{\textsanskrit{cīvaravaṁsa}}] a bamboo robe rack%
\item[\textit{\textsanskrit{cuṇṇa}}] bath powder, powder; dust (see Appendix I: Technical Terms)%
\item[\textit{\textsanskrit{cuṇṇacālinī}}] a powder sieve%
\item[\textit{\textsanskrit{cuṇṇāni} \textsanskrit{bhesajjāni}}] medicinal powders%
\item[\textit{\textsanskrit{cetasā} \textsanskrit{cetoparivitakkamaññāya}}] to read the mind%
\item[\textit{\textsanskrit{cetāpana}, \textsanskrit{cetāpanna}}] a fund%
\item[\textit{\textsanskrit{cetāpeti}}] to buy, to exchange%
\item[\textit{cetiya}] a shrine%
\item[\textit{coca}] a banana with seeds%
\item[\textit{codeti}] to accuse; to prompt; to confront (see Appendix I: Technical Terms)%
\item[\textit{\textsanskrit{coḷa}, \textsanskrit{coḷaka}}] cloth, a cloth; a towel%
\item[\textit{\textsanskrit{chakaṇa}}] dung, detergent (see Appendix I: Technical Terms)%
\item[\textit{chatta}] a sunshade%
\item[\textit{chadana}] a roof; roofing, roofing material%
\item[\textit{chanda}] consent; favoritism%
\item[\textit{chandaso}] metrical form%
\item[\textit{\textsanskrit{chandahāraka}}] one who is conveying (another’s) consent%
\item[\textit{\textsanskrit{chandārahā}}] one who is eligible to give their consent%
\item[\textit{\textsanskrit{chamā}}] the ground%
\item[\textit{\textsanskrit{chavakuṭi}}] a charnel house%
\item[\textit{\textsanskrit{chāyā}}] a shadow (of a sundial); time%
\item[\textit{\textsanskrit{chārika}}] ash%
\item[\textit{chinnaka}] made of pieces (referring to robes)%
\item[\textit{chinnabhatta}] a missed meal%
\item[\textit{\textsanskrit{jaṅgheyyaka}}] a strengthening piece for the calves (referring to robes)%
\item[\textit{jacca}] caste%
\item[\textit{\textsanskrit{jaṭila}}] a dreadlocked ascetic%
\item[\textit{jatu}] gum, resin%
\item[\textit{\textsanskrit{jantāghara}}] a sauna (see Appendix I: Technical Terms)%
\item[\textit{\textsanskrit{jantāgharapīṭha}}] a sauna stool%
\item[\textit{\textsanskrit{jantāgharasālā}}] a sauna shed%
\item[\textit{jambu}] a rose apple (tree) (see Appendix V: Plants)%
\item[\textit{\textsanskrit{jambudīpa}}] India%
\item[\textit{\textsanskrit{jātarūpa}}] gold%
\item[\textit{\textsanskrit{jātarūparajata}}] “gold, silver, and money” (see Appendix I: Technical Terms)%
\item[\textit{\textsanskrit{juṇha}}] the waxing phase of the moon%
\item[\textit{\textsanskrit{ñatti}}] a motion (of a legal procedure)%
\item[\textit{\textsanskrit{ñattiṁ} \textsanskrit{ṭhapeti}}] to put forward a motion%
\item[\textit{\textsanskrit{ñattikamma}}] a legal procedure consisting of one motion%
\item[\textit{\textsanskrit{ñatticatutthakamma}}] a legal procedure consisting of one motion and three announcements%
\item[\textit{\textsanskrit{ñattidutiyakamma}}] a legal procedure consisting of one motion and one announcement%
\item[\textit{\textsanskrit{ñāpeti}}] to inform (in connection with a legal procedure)%
\item[\textit{\textsanskrit{ṭhapana}}] cancellation (of the observance-day procedure, etc.)%
\item[\textit{\textsanskrit{ṭhapeti}}] to cancel (the observance-day procedure, etc.); to put forward (a motion)%
\item[\textit{\textsanskrit{ṭhānāraha}}] fit to stand (an attribute of a properly performed legal procedure)%
\item[\textit{\textsanskrit{ḍāka}}] potherb%
\item[\textit{taka}] the taka tree (see Appendix V: Plants)%
\item[\textit{takka}] buttermilk%
\item[\textit{tagara}] crepe jasmine (see Appendix IV: Medical Terminology)%
\item[\textit{\textsanskrit{tajjanīya}}] condemnation%
\item[\textit{\textsanskrit{taṭṭika}}] a straw mat%
\item[\textit{\textsanskrit{Tathāgata}}] the Buddha; I (when the Buddha is referring to himself)%
\item[\textit{tantaka}] warp (the lengthwise thread on a loom)%
\item[\textit{tambaloha}] copper%
\item[\textit{\textsanskrit{tassapāpiyasikā}}] a further penalty (a legal procedure of giving)%
\item[\textit{\textsanskrit{tāḷa}}] key%
\item[\textit{\textsanskrit{tāḷacchidda}}] a keyhole%
\item[\textit{\textsanskrit{tālavaṇṭa}}] a palm-leaf (fan)%
\item[\textit{\textsanskrit{tālīsa}}] the coffee plum tree (see Appendix V: Plants)%
\item[\textit{\textsanskrit{tāvakālika}}] borrowing; lending%
\item[\textit{\textsanskrit{ticīvarena} \textsanskrit{avippavāsa}}] a may-stay-apart-from-the-three-robes (area)%
\item[\textit{\textsanskrit{tiṇa}}] grass, straw%
\item[\textit{\textsanskrit{tiṇavatthāraka}}] covering over as if with grass%
\item[\textit{\textsanskrit{tiṇasanthāraka}}] a spread of grass%
\item[\textit{tittha}] a ford%
\item[\textit{\textsanskrit{titthāyatanā}}] (another) religious community%
\item[\textit{titthiya}] an ascetic of another religion%
\item[\textit{titthiyapakkantaka}] one who has previously left to join the monastics of another religion%
\item[\textit{\textsanskrit{titthiyasāvaka}}] a lay follower of another religion%
\item[\textit{titthiyaseyya}] the dwelling place of the ascetics of another religion%
\item[\textit{tipu}] tin (the metal)%
\item[\textit{\textsanskrit{timbarūsaka}}] the Gaub tree (see Appendix V: Plants)%
\item[\textit{\textsanskrit{tiracchānakatha}}] worldly talk, gossip%
\item[\textit{\textsanskrit{tiracchānavijjā}}] a worldly subject%
\item[\textit{\textsanskrit{tirīṭaka}}] the lodh tree (see Appendix V: Plants)%
\item[\textit{\textsanskrit{tirokaraṇī}}] a curtain%
\item[\textit{tilakakka}] sesame paste (for treating a wound)%
\item[\textit{tunna}] mending (cloth)%
\item[\textit{tumba}] a vessel; a gourd%
\item[\textit{\textsanskrit{tūlika}}] a cotton-down quilt (see Appendix III: Furniture)%
\item[\textit{\textsanskrit{tekaṭulayāgu}}] threefold pungent congee (see Appendix IV: Medical Terminology)%
\item[\textit{\textsanskrit{tecīvarika}}] a three-robe owner%
\item[\textit{tela}] oil%
\item[\textit{\textsanskrit{telapāka}}] a heated concoction of oil%
\item[\textit{\textsanskrit{toraṇa}}] an arch%
\item[\textit{thambha}] a pillar, a post%
\item[\textit{\textsanskrit{thavikā}}] a bag%
\item[\textit{\textsanskrit{thālaka}}] a vessel%
\item[\textit{\textsanskrit{thullakacchābādha}}] a carbuncle (see Appendix IV: Medical Terminology)%
\item[\textit{thullaccaya}] a serious offense%
\item[\textit{thullavajja}] a heavy offense%
\item[\textit{\textsanskrit{thūpa}}] a stupa; a heap%
\item[\textit{\textsanskrit{theyyasaṁvāsaka}}] a fake monastic%
\item[\textit{thera}] a senior (monastic), the most senior (monastic), an elder%
\item[\textit{theva, thevaka}] a drop (of water); dripping%
\item[\textit{\textsanskrit{daṇḍa}}] a staff; a handle; punishment, fine%
\item[\textit{\textsanskrit{daṇḍakamma}}] a penalty%
\item[\textit{dadhi}] curd%
\item[\textit{\textsanskrit{dantakaṭṭha}}] a tooth cleaner%
\item[\textit{dantapona}] a tooth cleaner%
\item[\textit{\textsanskrit{daḷhīkamma}}] strengthening (of cloth)%
\item[\textit{dassana}] recognizing (an offense)%
\item[\textit{\textsanskrit{dassanūpacāra}}] the range of sight%
\item[\textit{\textsanskrit{dāna}}] passing on (purity or invitation), giving (one’s consent)%
\item[\textit{\textsanskrit{dāru}}] timber, wood%
\item[\textit{\textsanskrit{diṭṭhi}}] a view; to regard%
\item[\textit{\textsanskrit{diṭṭhigata}}] wrong view%
\item[\textit{\textsanskrit{divasabhāga}}] part of the day (morning or afternoon)%
\item[\textit{\textsanskrit{disā}}] a district, a region, a different region%
\item[\textit{\textsanskrit{disābhāga}}] a cardinal direction%
\item[\textit{\textsanskrit{dīpeti}}] to proclaim%
\item[\textit{\textsanskrit{dukkaṭa}}] an offense of wrong conduct; an act of wrong conduct, wrong conduct, badly done, badly made%
\item[\textit{\textsanskrit{duṭṭha}}] angry; malicious%
\item[\textit{\textsanskrit{duṭṭhagahaṇika}}] indigestion (see Appendix IV: Medical Terminology)%
\item[\textit{\textsanskrit{duṭṭhulla}}] indecent; grave; coarse%
\item[\textit{dutiya, \textsanskrit{dutiyikā}}] a companion; a wife%
\item[\textit{dubbaca}] difficult to correct%
\item[\textit{\textsanskrit{dubbhāsita}}] bad speech; an offense of wrong speech%
\item[\textit{\textsanskrit{durakkhāta}}] flawed%
\item[\textit{\textsanskrit{duvūpasanta}}] improperly disposed of (legal procedure)%
\item[\textit{dussa}] cloth, fabric; a dressing gown%
\item[\textit{\textsanskrit{dūseti}, \textsanskrit{dūsaka}}] to rape; to molest; to have sex with; to be intimate with; to corrupt, to spoil (see Appendix I: Technical Terms)%
\item[\textit{deti}] to pass on (purity or invitation), to give (one’s consent); to put in charge of (work)%
\item[\textit{\textsanskrit{desanā}, deseti}] confession (of an offense); approving (a site for a dwelling)%
\item[\textit{\textsanskrit{desanāgāminiyā}}] (an offense) clearable by confession%
\item[\textit{\textsanskrit{doṇa}}] a liter (a measure of volume)%
\item[\textit{\textsanskrit{doṇī}, \textsanskrit{doṇika}}] a trough%
\item[\textit{\textsanskrit{dvāra}}] a gate, a gateway, an entrance, a door%
\item[\textit{\textsanskrit{dvārakosa}}] a doorcase%
\item[\textit{\textsanskrit{dvāramūla}}] an entrance, a main entrance, a gateway%
\item[\textit{dhajabandha}] notorious (criminal)%
\item[\textit{dhanu}] a bow-length (approximately 1.6 m) (see Appendix I: Technical Terms)%
\item[\textit{dhamma}] the Teaching, a teaching; the Truth; a rule, a principle; a quality; legitimate, appropriate%
\item[\textit{dhammakathika}] an expounder of the Teaching%
\item[\textit{\textsanskrit{dhammakaraṇa}}] a water strainer%
\item[\textit{dhammadhara}] an expert on the Teaching%
\item[\textit{\textsanskrit{dhammapatirūpaka}}] legitimate-like (legal procedure)%
\item[\textit{\textsanskrit{dhammavādī}}] one who speaks in accordance with the Teaching%
\item[\textit{dhammavinaya}] a spiritual path (see Appendix I: Technical Terms)%
\item[\textit{dhammika}] legitimate%
\item[\textit{\textsanskrit{dhāreti}}] to keep; to wear; to use; to master%
\item[\textit{\textsanskrit{dhutaṅga}}] ascetic practices%
\item[\textit{dhutta}] a scoundrel%
\item[\textit{\textsanskrit{dhūmaṁ} \textsanskrit{kātuṁ}}] fumigation%
\item[\textit{\textsanskrit{dhūmaṁ} \textsanskrit{pātuṁ}}] smoke inhalation%
\item[\textit{\textsanskrit{dhūmanetta}}] a tube (to inhale smoke); a chimney%
\item[\textit{\textsanskrit{dhūmanettathavika}}] a bag for the tube (to inhale smoke)%
\item[\textit{\textsanskrit{dhotapādaka}}] a towel%
\item[\textit{nakkhattapada}] a constellation%
\item[\textit{\textsanskrit{nattamāla}}] Indian beech (see Appendix V: Plants)%
\item[\textit{natthukamma}] treatment through the nose%
\item[\textit{\textsanskrit{natthukaraṇī}}] a nose dropper%
\item[\textit{namataka}] felt%
\item[\textit{nava}] new; junior (monastic)%
\item[\textit{navaka}] a junior (monastic), a newly ordained (monastic)%
\item[\textit{navakamma}] building work%
\item[\textit{\textsanskrit{navakammadāna}}] putting in charge of building work%
\item[\textit{\textsanskrit{navanīta}}] butter%
\item[\textit{nassati}] to be lost%
\item[\textit{\textsanskrit{nāga}}] a (male) dragon%
\item[\textit{\textsanskrit{nāgadanta}(ka)}] a wall peg%
\item[\textit{\textsanskrit{nāgī}}] a female dragon%
\item[\textit{\textsanskrit{nānāsaṁvāsaka}}] one who belongs to a different Buddhist sect (see Appendix I: Technical Terms)%
\item[\textit{\textsanskrit{nāma}}] :%
\item[\textit{\textsanskrit{nāsanantika}}] ending when the robe-cloth is lost (referring to the robe season)%
\item[\textit{\textsanskrit{nāseti}}] to expel (see Appendix I: Technical Terms)%
\item[\textit{\textsanskrit{nikāya}}] collection%
\item[\textit{\textsanskrit{nikhādana}}] a chisel%
\item[\textit{nigrodha}] the banyan tree (see Appendix V: Plants)%
\item[\textit{niccabhatta}] a regular meal%
\item[\textit{\textsanskrit{nijjhāpeti}}] to convince%
\item[\textit{\textsanskrit{niṭṭhānantika}}] ending when the robe is finished (referring to the robe season)%
\item[\textit{nidahaka}] a storer (of cloth, etc.)%
\item[\textit{\textsanskrit{nidāna}}] an origin story; an introduction; a source%
\item[\textit{niddesa}] detailed explanation%
\item[\textit{niddhamana}] a (water) drain%
\item[\textit{nibbakosa}] eaves%
\item[\textit{nibbematika}] sure%
\item[\textit{nimantana}] an invitational meal%
\item[\textit{nimanteti}] to offer (food)%
\item[\textit{nimitta}] genitals; a sign; a marker (for a boundary)%
\item[\textit{nimba}] the neem tree (see Appendix V: Plants)%
\item[\textit{niyassa}] demotion (see Appendix I: Technical Terms)%
\item[\textit{nirutti}] a way of speaking, an expression%
\item[\textit{\textsanskrit{nivāsana}}] a sarong%
\item[\textit{nivesana}] a house%
\item[\textit{nisada, nisadapotaka}] a grinding stone%
\item[\textit{nisadapotaka,}] see \textit{nisada}%
\item[\textit{\textsanskrit{nisīdana}}] a sitting mat (see Appendix I: Technical Terms)%
\item[\textit{nissaggiya}] to be relinquished, entailing relinquishment, subject to relinquishment; (something) released%
\item[\textit{nissaggiya \textsanskrit{pācittiya}}] an offense entailing relinquishment and confession%
\item[\textit{nissaya, \textsanskrit{nissāya}, nissita}] formal support, support (see Appendix I: Technical Terms)%
\item[\textit{\textsanskrit{nissāraṇā}}] sending away%
\item[\textit{\textsanskrit{nissāraṇīya}}] entailing sending away%
\item[\textit{nissita,}] see \textit{nissaya}%
\item[\textit{negama}] a householder association%
\item[\textit{\textsanskrit{netthāraṁ} vattati}] to deserve to be released%
\item[\textit{\textsanskrit{paṁsukūla}}] discarded; a rag%
\item[\textit{\textsanskrit{paṁsukūlika}}] a rag-robe wearer%
\item[\textit{pakata, pakati, \textsanskrit{pākaṭa}}] regular (monastic); ordinary, normal; nature (of a person)%
\item[\textit{\textsanskrit{pakāsanīya}}] an announcement%
\item[\textit{\textsanskrit{pakuṭṭa}}] an encircling corridor%
\item[\textit{pakka}] ripe; ready; heated, cooked, baked; digested%
\item[\textit{pakkamanantika}] ending when one departs (referring to the robe season)%
\item[\textit{pakkha}] a side (in a dispute), a faction, supporters; a group; a (lunar) half-month%
\item[\textit{\textsanskrit{pakkhasaṅkanta}}] joined another faction%
\item[\textit{pakkhahata}] one who is paralyzed%
\item[\textit{pakkhika}] half-monthly (meal)%
\item[\textit{\textsanskrit{paggharantī}}] incontinent%
\item[\textit{\textsanskrit{paghaṇa}}] a screened doorstep%
\item[\textit{\textsanskrit{paccakkhāti}}] to verbally renounce (the monastic life), to renounce (the monastic life)%
\item[\textit{\textsanskrit{paccattharaṇa}}] a sheet%
\item[\textit{\textsanskrit{paccantimā} \textsanskrit{janapadā}}] outside the central Ganges plain%
\item[\textit{paccaya}] a requisite (for a monastic); a reason because (of)%
\item[\textit{\textsanskrit{pacchāsamaṇa}}] an attendant%
\item[\textit{pajja}] a foot salve%
\item[\textit{\textsanskrit{pañcapaṭika}}] the fivefold pattern%
\item[\textit{\textsanskrit{paññatti}}] a rule%
\item[\textit{\textsanskrit{paññāpaka}}] an assigner%
\item[\textit{\textsanskrit{paññapeti}, \textsanskrit{paññāpeti}}] to lay down (a rule); to prepare; to assign%
\item[\textit{\textsanskrit{paṭaggi}}] a counterfire%
\item[\textit{\textsanskrit{paṭalika}}] a crimson woolen rug (see Appendix III: Furniture)%
\item[\textit{\textsanskrit{paṭika}}] a white woolen rug (see Appendix III: Furniture)%
\item[\textit{\textsanskrit{paṭikamma}}] making amends (after committing an offense); treatment (for a sore)%
\item[\textit{\textsanskrit{paṭikaroti}}] to make amends (for an offense)%
\item[\textit{\textsanskrit{paṭikuṭṭha}}] objected to%
\item[\textit{\textsanskrit{paṭikkosana}}] objecting%
\item[\textit{\textsanskrit{paṭikkhitta}}] prohibited%
\item[\textit{\textsanskrit{paṭikkhepa}}] a prohibition; a refusal%
\item[\textit{\textsanskrit{paṭiggaṇhāti}}] to receive (a confession), to accept; to take (dye); to agree%
\item[\textit{\textsanskrit{paṭiggaha}}] a thimble; a (refuse) receptacle; a trash can%
\item[\textit{\textsanskrit{paṭiggāhaka}}] a receiver, one who receives%
\item[\textit{\textsanskrit{paṭicodeti}}] to confront; to counter accuse%
\item[\textit{\textsanskrit{paṭicchanna}}] concealed (offense)%
\item[\textit{\textsanskrit{paṭicchādanīya}}] meat broth (see Appendix IV: Medical Terminology)%
\item[\textit{\textsanskrit{paṭijānāti}, \textsanskrit{paṭiññā}}] to admit; to claim, to assert%
\item[\textit{\textsanskrit{paṭiññātakaraṇa}}] acting according to what has been admitted%
\item[\textit{\textsanskrit{paṭideseti}}] to acknowledge; to confess%
\item[\textit{\textsanskrit{paṭipāṭiya}}] order; a row%
\item[\textit{\textsanskrit{paṭipādaka}}] a support (for a bed)%
\item[\textit{\textsanskrit{paṭipucchā}}] questioning%
\item[\textit{\textsanskrit{paṭippassaddhi}}] lifting (of a legal procedure); ending (of formal support); annulling (an offense)%
\item[\textit{\textsanskrit{paṭibaddha}}] connected%
\item[\textit{\textsanskrit{paṭibaddhacitta}}] in love with%
\item[\textit{\textsanskrit{paṭibala}}] capable%
\item[\textit{\textsanskrit{paṭibāhana}}] blocking; reserving; excluding; defending against; obstructing%
\item[\textit{\textsanskrit{paṭibhāna}}] articulate%
\item[\textit{\textsanskrit{paṭibhānacitta}}] a picture%
\item[\textit{\textsanskrit{paṭivīsa}}] a share%
\item[\textit{\textsanskrit{paṭisallīna},}] see \textit{rahogata}%
\item[\textit{\textsanskrit{paṭisāraṇīya}}] reconciliation%
\item[\textit{\textsanskrit{paṭola}}] a pointed gourd (see Appendix V: Plants)%
\item[\textit{\textsanskrit{paṇāmeti}}] to dismiss%
\item[\textit{\textsanskrit{paṇīta}}] fine (food), superior%
\item[\textit{\textsanskrit{paṇḍaka}}] a \textit{\textsanskrit{paṇḍaka}} (see Appendix I: Technical Terms)%
\item[\textit{\textsanskrit{paṇḍurogābādha}}] jaundice%
\item[\textit{\textsanskrit{patirūpaka}}] -like%
\item[\textit{patta (1)}] a bowl, an almsbowl%
\item[\textit{patta (2)}] a panel (of a robe)%
\item[\textit{pattakalla}] ready%
\item[\textit{\textsanskrit{pattakuṇḍolika}}] a container for bowls%
\item[\textit{\textsanskrit{pattatthavikā}}] a bowl bag%
\item[\textit{\textsanskrit{pattamaṇḍala}}] a circular bowl rest%
\item[\textit{\textsanskrit{pattamāḷaka}}] a platform for bowls%
\item[\textit{\textsanskrit{pattādhāraka}}] a bowl rack%
\item[\textit{padarasila}] a paving stone%
\item[\textit{\textsanskrit{padīpa}}] a lamp%
\item[\textit{pabbaja}] a reed (of a particular kind)%
\item[\textit{\textsanskrit{pabbajjā}}] going forth%
\item[\textit{\textsanskrit{pabbavāta}}] arthritis%
\item[\textit{\textsanskrit{pabbājanīya}}] banishment, banishing%
\item[\textit{\textsanskrit{pamāṇika}}] the right size%
\item[\textit{pamukha}] a forecourt, an entryway; headed by%
\item[\textit{\textsanskrit{payirupāsati}}] to visit; to attend to (an expectation of robe-cloth)%
\item[\textit{payoga}] an effort; an action%
\item[\textit{parasu}] a hatchet%
\item[\textit{parikamma}] a massage; treatment; assistance%
\item[\textit{\textsanskrit{parikkhāra}}] a requisite, equipment; goods; possession; ingredient (see Appendix I: Technical Terms)%
\item[\textit{\textsanskrit{parikkhāracoḷa}}] a cloth for requisites%
\item[\textit{parikkhitta}] enclosed%
\item[\textit{paricarati}] to provide a service; to satisfy%
\item[\textit{\textsanskrit{paripañhati}}] to enquire%
\item[\textit{\textsanskrit{paripucchā}}] questioning, testing (see Appendix I: Technical Terms)%
\item[\textit{\textsanskrit{paribbājaka}, \textsanskrit{paribbājika}}] a male wanderer; a female wanderer%
\item[\textit{\textsanskrit{paribhaṇḍa}}] a crosswise border (of a robe); a crosswise edge (of a wooden frame); a shelf; the area immediately outside (a building); the plastering of a floor%
\item[\textit{paribhoga}] equipment; a possession; things; using; eating%
\item[\textit{\textsanskrit{paribhojanīya}}] water for washing%
\item[\textit{\textsanskrit{parimāṇa}}] a specified (offense)%
\item[\textit{parivatteti}] to trade; to purchase; to turn upside down%
\item[\textit{\textsanskrit{parivāra}}] accompanying (food); the Compendium (when referring to the last book if the Vinaya \textsanskrit{Piṭaka})%
\item[\textit{\textsanskrit{parivāsa}, parivasati}] probation%
\item[\textit{\textsanskrit{pariveṇa}}] a yard (see Appendix I: Technical Terms)%
\item[\textit{\textsanskrit{parisadūsaka}}] one with abnormal appearance%
\item[\textit{\textsanskrit{parisuddhā}}] pure; free from obstructions (for ordination)%
\item[\textit{\textsanskrit{parisuddhiṁ} deti}] to pass on one’s purity%
\item[\textit{\textsanskrit{parissāvana}}] a water filter%
\item[\textit{palita}] grey hair%
\item[\textit{palibuddha}] obstructed%
\item[\textit{palibundheti}] to take possession of; to detain%
\item[\textit{\textsanskrit{palibodhā}}] an obstacle; responsibility%
\item[\textit{\textsanskrit{pallaṅka}}] (sitting) cross-legged; a luxurious couch (see Appendix III: Furniture)%
\item[\textit{\textsanskrit{pallatthikā}}] clasping the knees (while sitting); a back-and-knee strap%
\item[\textit{\textsanskrit{pavattinī}}] a mentor (for nuns)%
\item[\textit{\textsanskrit{pavāraṇahāraka}}] one who is conveying the invitation to correct%
\item[\textit{\textsanskrit{pavāraṇā}}] an invitation, an invitation to correct (one); the invitation ceremony; the invitation day (see Appendix I: Technical Terms)%
\item[\textit{\textsanskrit{pavāraṇaṁ} deti}] to pass on one’s invitation%
\item[\textit{\textsanskrit{pavāreti}}] to invite, to invite correction; to do the invitation ceremony; to invite (someone) to ask (for something); to refuse an invitation (occasionally “offer”) of more food%
\item[\textit{\textsanskrit{pasukā}}] a domesticated animal%
\item[\textit{passati}] to recognize (an offense)%
\item[\textit{\textsanskrit{passāva}}] urine%
\item[\textit{\textsanskrit{passāvakumbhī}}] a urine-collection pot%
\item[\textit{\textsanskrit{passāvadoṇika}}] a urinal%
\item[\textit{\textsanskrit{passāvamagga}}] a vagina; private parts (when used with \textit{vaccamagga})%
\item[\textit{\textsanskrit{paharaṇī}, \textsanskrit{paharaṇa}}] a weapon%
\item[\textit{\textsanskrit{pākāra}}] a wall, an encircling wall; a (cloth) screen%
\item[\textit{\textsanskrit{pācittiya}}] an offense entailing confession%
\item[\textit{\textsanskrit{pāṭaṅkī}}] a litter%
\item[\textit{\textsanskrit{pāṭidesanīya}}] an offense entailing acknowledgement%
\item[\textit{\textsanskrit{pāṭipadika}}] the day after the observance day; a meal on the day after the observance day%
\item[\textit{\textsanskrit{pāṭihāriya}}] a wonder%
\item[\textit{\textsanskrit{pātimokkha}}] the Monastic Code%
\item[\textit{\textsanskrit{pātī}}] a basin, a bowl%
\item[\textit{\textsanskrit{pātheyya}}] provisions%
\item[\textit{\textsanskrit{pādakathalika}}] a foot scraper%
\item[\textit{\textsanskrit{pādakhilābādha}}] corns on the feet%
\item[\textit{\textsanskrit{pādaghaṁsanī}}] a foot scrubber%
\item[\textit{\textsanskrit{pādapīṭha}}] a foot stool%
\item[\textit{\textsanskrit{pādapuñchani}}] a doormat%
\item[\textit{\textsanskrit{pādabbhañjana}}] a foot salve%
\item[\textit{\textsanskrit{pāduka}}] a shoe; a foot stand%
\item[\textit{\textsanskrit{pādodaka}}] water for washing the feet%
\item[\textit{\textsanskrit{pānāgāra}}] a bar%
\item[\textit{\textsanskrit{pānīya}}] drinking water; a drink%
\item[\textit{\textsanskrit{pānīyasālā}}] a drinking-water shed%
\item[\textit{\textsanskrit{pāpa}, \textsanskrit{pāpika}}] miserable; bad; serious (sickness)%
\item[\textit{\textsanskrit{pāpaka} \textsanskrit{diṭṭhgata}}] a bad and erroneous view%
\item[\textit{\textsanskrit{pāpikā} \textsanskrit{diṭṭhi}}] a bad view%
\item[\textit{\textsanskrit{pāpiccha}}] a bad desire%
\item[\textit{\textsanskrit{pāmaṅga}}] an ornamental hanging string%
\item[\textit{\textsanskrit{pārājika}}] an offense entailing expulsion%
\item[\textit{\textsanskrit{pāricchattaka}}] the orchard tree (see Appendix V: Plants)%
\item[\textit{\textsanskrit{pārisuddhi}}] purity%
\item[\textit{\textsanskrit{pārisuddhiṁ} deti}] to pass on (one’s) purity; (although I render \textit{\textsanskrit{chandaṁ} deti} as “to give (one’s) consent”, it seems more natural to use “pass on” with the word “purity”, as well as with “invitation”)%
\item[\textit{\textsanskrit{pārisuddhihāraka}}] one who is conveying the purity%
\item[\textit{\textsanskrit{pāvāra}}] a fleecy robe%
\item[\textit{\textsanskrit{pāvuraṇa}}] a cloak; a cover%
\item[\textit{\textsanskrit{pāsaka}}] a loop%
\item[\textit{\textsanskrit{pāsakaphalaka}}] a loop shield (for a robe)%
\item[\textit{\textsanskrit{pāsāṇa}}] a rock%
\item[\textit{\textsanskrit{pāsāda}}] a stilt house (see Appendix I: Technical Terms)%
\item[\textit{\textsanskrit{piṭaka}}] a collection; a basket%
\item[\textit{\textsanskrit{piṭṭha} (1)}] flour%
\item[\textit{\textsanskrit{piṭṭha} (2)}] a door frame; the back%
\item[\textit{\textsanskrit{piṭṭhamadda}}] flour paste%
\item[\textit{\textsanskrit{piṭṭhasaṅghāṭo}}] a door frame%
\item[\textit{\textsanskrit{piṭṭhivaṁsaṁ}}] a ridge beam (of a house)%
\item[\textit{\textsanskrit{piṇḍaka}}] almsfood%
\item[\textit{\textsanskrit{piṇḍacārika}}] alms collector, alms-collecting (monastic)%
\item[\textit{\textsanskrit{piṇḍapāta}}] almsfood; almsround%
\item[\textit{\textsanskrit{piṇḍapātika}}] one who only eats almsfood%
\item[\textit{\textsanskrit{pippalī}}] the long pepper (see Appendix V: Plants)%
\item[\textit{\textsanskrit{piḷakā}}] a boil (see Appendix IV: Medical Terminology)%
\item[\textit{pilakkha}] an Indian rock fig (see Appendix V: Plants)%
\item[\textit{pilotika}] an old cloth%
\item[\textit{\textsanskrit{pisāca}}] a demon%
\item[\textit{\textsanskrit{pisācillikā}}] a goblin%
\item[\textit{\textsanskrit{pīṭha}, \textsanskrit{pīṭhaka}}] a bench (both for sitting and sleeping on) (see Appendix III: Furniture)%
\item[\textit{puggala}] a person, an individual, a single monastic; a man%
\item[\textit{pucimanda}] the neem tree (see Appendix V: Plants)%
\item[\textit{putta}] a son, a child, an offspring%
\item[\textit{puppha}] the fertile period%
\item[\textit{\textsanskrit{pubbaṇṇa}}] grain%
\item[\textit{\textsanskrit{purekkhāra}}] aiming at; giving priority to%
\item[\textit{purohita}] a brahmin counselor%
\item[\textit{\textsanskrit{pūga}}] an association%
\item[\textit{\textsanskrit{pūva}}] a cookie (see Appendix I: Technical Terms)%
\item[\textit{pekkheti}] to make see%
\item[\textit{peta}] a (male) ghost%
\item[\textit{\textsanskrit{petī}}] a female ghost%
\item[\textit{pesala}] good (monastic)%
\item[\textit{\textsanskrit{pokkharaṇī}}] a (lotus) pond, a (lotus) bathing tank (see Appendix I: Technical Terms)%
\item[\textit{potthaka}] jute%
\item[\textit{phaggava, pakkava}] the white fig (see Appendix V: Plants)%
\item[\textit{\textsanskrit{phaṇahatthaka}}] one with joined fingers%
\item[\textit{\textsanskrit{phaṇijjaka}}] rajmahal hemp (see Appendix V: Plants)%
\item[\textit{phalaka, \textsanskrit{phalakapīṭha}}] a plank bench (see Appendix III: Furniture)%
\item[\textit{phalika}] a crystal%
\item[\textit{\textsanskrit{phāṇita}}] syrup (see Appendix I: Technical Terms)%
\item[\textit{\textsanskrit{phārusaka}}] a falsa fruit%
\item[\textit{\textsanskrit{phāsu}}] comfortable, at ease%
\item[\textit{\textsanskrit{phāsukā}}] a rib%
\item[\textit{bandhanasuttaka}] a string for fastening (see Appendix IV: Medical Terminology)%
\item[\textit{\textsanskrit{bandhanāgāra}}] a prison%
\item[\textit{\textsanskrit{bāla}}] one who is ignorant, a fool%
\item[\textit{\textsanskrit{bāhanta}}] an outer section (referring to robes)%
\item[\textit{\textsanskrit{bidalamañcaka}}] a wicker bed (see Appendix III: Furniture)%
\item[\textit{bibbohana}] a pillow%
\item[\textit{bila}] red salt%
\item[\textit{\textsanskrit{bilaṅga}}] false black pepper (see Appendix V: Plants)%
\item[\textit{\textsanskrit{bīja}}] a seed; propagation; capable of propagation%
\item[\textit{\textsanskrit{bījanī}, \textsanskrit{vījanī}}] a fan, a whisk%
\item[\textit{\textsanskrit{bundikābaddha}}] having legs and frame (of a bed or bench) (see Appendix III: Furniture)%
\item[\textit{bodhirukkha}] the Bodhi tree%
\item[\textit{\textsanskrit{byañjana}}] curry; wording%
\item[\textit{byatta, vyatta}] competent%
\item[\textit{\textsanskrit{byūha}}] a cul-de-sac; a heap; a massing (of an army)%
\item[\textit{brahmacariya}] a spiritual life, monastic life; celibacy%
\item[\textit{\textsanskrit{brahmadaṇḍa}}] the supreme penalty%
\item[\textit{\textsanskrit{brāhmaṇa}}] a brahmin%
\item[\textit{\textsanskrit{bhagandalābādha}}] hemorrhoids%
\item[\textit{\textsanskrit{bhaṅga}}] hemp (see Appendix V: Plants)%
\item[\textit{\textsanskrit{bhaṅgodaka}}] herb water (as a treatment) (see Appendix IV: Medical Terminology)%
\item[\textit{\textsanskrit{bhañjana}}] a salve%
\item[\textit{\textsanskrit{bhañjanaka}}] a shallot (see Appendix V: Plants)%
\item[\textit{\textsanskrit{bhaṇḍa}}] goods, equipment; an article, a possession%
\item[\textit{\textsanskrit{bhaṇḍāgāra}}] a storeroom%
\item[\textit{\textsanskrit{bhaṇḍāgārika}}] a storeman%
\item[\textit{\textsanskrit{bhaṇḍika}}] a bundle; a bag; a cornice%
\item[\textit{bhatta}] a meal, rice%
\item[\textit{bhattagga}] a dining hall (see Appendix I: Technical Terms)%
\item[\textit{bhattudesaka}] a designator of meals%
\item[\textit{\textsanskrit{bhaddapīṭha}}] a cane bench (see Appendix III: Furniture)%
\item[\textit{bhaddamuttaka}] nut grass (see Appendix V: Plants)%
\item[\textit{bhante}] Sir, Venerable Sir (the latter especially when \textit{bhante} is combined with \textit{\textsanskrit{bhagavā}}, that is, it refers to the Buddha)%
\item[\textit{\textsanskrit{bhājaka}}] a distributor%
\item[\textit{\textsanskrit{bhājana}}] a vessel%
\item[\textit{\textsanskrit{bhāṇaka}}] a reciter; a jar%
\item[\textit{\textsanskrit{bhātar}}] a brother, a sibling%
\item[\textit{\textsanskrit{bhikkhā}}] alms, food%
\item[\textit{bhikkhu}] a monk, a Buddhist monk%
\item[\textit{\textsanskrit{bhikkhunidūsaka}, \textsanskrit{bhikkhunīdūsaka}}] one who has raped a nun%
\item[\textit{\textsanskrit{bhikkhunī}}] a nun, a Buddhist nun%
\item[\textit{\textsanskrit{bhikkhunovādaka}}] an instructor of the nuns%
\item[\textit{bhitti}] a (interior) wall%
\item[\textit{bhittikhila}] a wall peg%
\item[\textit{bhisi}] a mattress%
\item[\textit{bhisicchavi}] a mattress cover%
\item[\textit{\textsanskrit{bhuttāvasesa}}] leftover (food)%
\item[\textit{\textsanskrit{bhuttāvī} \textsanskrit{onītapattapāṇi}, \textsanskrit{onītapattapāṇi}}] who has finished the meal%
\item[\textit{\textsanskrit{bhūtagāma}}] plant%
\item[\textit{\textsanskrit{bhūmattharaṇa}}] a floor cover%
\item[\textit{\textsanskrit{bhūmi}}] the ground; a foundation; a floor; a ground%
\item[\textit{bheda, bhedana}] a schism (in the Sangha); division; breaking, destruction%
\item[\textit{bhesajja}] a medicine, a tonic (see Appendix I: Technical Terms)%
\item[\textit{\textsanskrit{bhesajjaparikkhāra}}] medicinal supplies%
\item[\textit{bhojana, \textsanskrit{bhojanīya}}] cooked food, food (see Appendix I: Technical Terms)%
\item[\textit{\textsanskrit{bhojanīya} + \textsanskrit{khādanīya}}] food, various kinds of food%
\item[\textit{\textsanskrit{bhojjayāgu}}] rice porridge%
\item[\textit{makaradantaka}] a shark-teeth pattern%
\item[\textit{\textsanskrit{makasakuṭika}}] a mosquito tent%
\item[\textit{magga}] orifice; private part (see Appendix I: Technical Terms)%
\item[\textit{\textsanskrit{maṅku}}] humiliated%
\item[\textit{majjhima}] (a monastic) of middle standing%
\item[\textit{\textsanskrit{majjhimā} \textsanskrit{janapadā}}] the central Ganges plain%
\item[\textit{\textsanskrit{mañca}}] a bed%
\item[\textit{\textsanskrit{mañcapaṭipādakā}}] a bed support%
\item[\textit{\textsanskrit{mañjiṭṭhā}, \textsanskrit{mañjiṭṭhaka}}] magenta; red rot (disease)%
\item[\textit{\textsanskrit{maṇi}}] a gem%
\item[\textit{\textsanskrit{maṇḍapa}}] a roof cover%
\item[\textit{\textsanskrit{maṇḍala}, \textsanskrit{maṇḍalī}}] a panel (of a robe), a large panel (of a robe); a circular bowl rest%
\item[\textit{\textsanskrit{maṇḍalika}}] an encircling trench%
\item[\textit{\textsanskrit{mattikā}}] clay; ceramics; soap (see Appendix I: Technical Terms)%
\item[\textit{madhu}] honey%
\item[\textit{madhumeha}] diabetes%
\item[\textit{madhusitthaka}] beeswax%
\item[\textit{\textsanskrit{madhūka}, madhuka}] the licorice plant (see Appendix V: Plants)%
\item[\textit{manta}] the Vedas; a mantra; a plan%
\item[\textit{mantha}] cracker (see Appendix I: Technical Terms)%
\item[\textit{\textsanskrit{mandārava}}] the coral tree (see Appendix V: Plants)%
\item[\textit{\textsanskrit{maraṇamatta}}] death-like%
\item[\textit{marica}] black pepper%
\item[\textit{marumba}] gravel%
\item[\textit{\textsanskrit{masāraka}}] having legs and frame (of a bed or bench) (see Appendix III: Furniture)%
\item[\textit{\textsanskrit{mahācamma}}] a luxurious skin%
\item[\textit{\textsanskrit{mahānāmaratta}}] beige%
\item[\textit{\textsanskrit{mahāmatta}}] a (government) official%
\item[\textit{\textsanskrit{mahāraṅgaratta}}] orange (color)%
\item[\textit{\textsanskrit{mahāvikaṭa}}] a filthy edible%
\item[\textit{\textsanskrit{mahāsayana}}] a luxurious bed or resting place%
\item[\textit{\textsanskrit{mātikā}}] a key term or phrase; a (water) channel%
\item[\textit{\textsanskrit{mātikādhara}}] an expert on the summaries%
\item[\textit{\textsanskrit{mānatta}}] the trial period%
\item[\textit{\textsanskrit{mānattacārika}, \textsanskrit{mānattacārinī}}] who is undertaking the trial period%
\item[\textit{\textsanskrit{māḷa}}] stilt house (see Appendix I: Technical Terms)%
\item[\textit{\textsanskrit{māla}}] a garland%
\item[\textit{\textsanskrit{māsa}}] black gram%
\item[\textit{\textsanskrit{miḍḍhi}}] a bench (see Appendix III: Furniture)%
\item[\textit{\textsanskrit{mukhapuñchanacoḷa}(ka)}] a washcloth%
\item[\textit{mukhodaka}] water for rinsing the mouth%
\item[\textit{mugga}] mung bean (see Appendix V: Plants)%
\item[\textit{mucalinda}] the powder-puff tree (see Appendix V: Plants)%
\item[\textit{\textsanskrit{muñja}}] a reed (of a particular kind)%
\item[\textit{\textsanskrit{muṭṭhassati}}] absentminded%
\item[\textit{mutta}] urine, pee%
\item[\textit{\textsanskrit{muddikā}}] a grape%
\item[\textit{muddhani telaka}] oil for the head%
\item[\textit{musala}] a pestle%
\item[\textit{\textsanskrit{musāvāda}}] lying%
\item[\textit{\textsanskrit{mūgabbata}}] a vow of silence%
\item[\textit{\textsanskrit{mūla}}] root; (at) the foot (of); basis%
\item[\textit{\textsanskrit{mūlacīvara}}] original robe-cloth%
\item[\textit{\textsanskrit{mūlaṭṭha}}] an instigator%
\item[\textit{\textsanskrit{mūlāya} \textsanskrit{paṭikassana}}] sending back to the beginning%
\item[\textit{\textsanskrit{mūḷha}}] insane; deluded; gone astray%
\item[\textit{moca}] a seedless banana%
\item[\textit{modaka}] a cake%
\item[\textit{\textsanskrit{moragū}}] chaff-flower grass (see Appendix V: Plants)%
\item[\textit{yakkha}] a (male) spirit%
\item[\textit{\textsanskrit{yakkhī}}] a female spirit%
\item[\textit{\textsanskrit{yathāvuḍḍha}}] according to seniority%
\item[\textit{yamakathavika}] a bag with two compartments%
\item[\textit{\textsanskrit{yamakanatthukaraṇī}}] a double nose dropper%
\item[\textit{\textsanskrit{yāgu}}] congee%
\item[\textit{\textsanskrit{yāna}}] a vehicle%
\item[\textit{\textsanskrit{yāma}}] a part (of the night)%
\item[\textit{\textsanskrit{yāmakālika}}] a post-midday tonic%
\item[\textit{\textsanskrit{yāvakālika}}] ordinary food%
\item[\textit{\textsanskrit{yāvajīvika}}] a lifetime tonic%
\item[\textit{\textsanskrit{yūsa}}] mung-bean broth%
\item[\textit{\textsanskrit{yebhuyyasikā}}] a majority decision%
\item[\textit{yojana}] 13 kilometers (approximately) (see Appendix I: Technical Terms)%
\item[\textit{rajata}] silver; money%
\item[\textit{rajana}] dye%
\item[\textit{rajananippakka}] a cleaning agent (see Appendix IV: Medical Terminology)%
\item[\textit{rajju}] a rope, a string, a clothesline%
\item[\textit{ratta}] a day (as in a twenty-four-hour period); a time period; a night; desire%
\item[\textit{ratha}] a carriage, a chariot%
\item[\textit{rathatthara}] a carriage-seat rug (see Appendix III: Furniture)%
\item[\textit{rathika}] a street%
\item[\textit{rasa}] juice%
\item[\textit{\textsanskrit{rasañjana}}] a mixed ointment (see Appendix IV: Medical Terminology)%
\item[\textit{raho}] private%
\item[\textit{\textsanskrit{rājantepura},}] see \textit{antepura}%
\item[\textit{\textsanskrit{rājāyatana}}] the Indian ape-flower tree (see Appendix V: Plants)%
\item[\textit{ruci}] preference; persuasion; acceptance, consent, approval%
\item[\textit{rudhita}] a wound%
\item[\textit{\textsanskrit{rūpiya}}] money, silver%
\item[\textit{ropeti}] see \textit{\textsanskrit{āpatti} ropeti}%
\item[\textit{romanthaka}] a regurgitator%
\item[\textit{\textsanskrit{lajjī}}] who has a sense of conscience%
\item[\textit{\textsanskrit{lajjīdhammo} okkamati}] overcome by guilt%
\item[\textit{lahuka}] light (offense); ordinary%
\item[\textit{\textsanskrit{lahubhaṇḍa}}] ordinary goods, an ordinary belonging%
\item[\textit{\textsanskrit{lābha}}] material support, material things; obtaining; allowance%
\item[\textit{likhitaka}] a wanted criminal%
\item[\textit{lujjati}] to be torn apart; to collapse%
\item[\textit{\textsanskrit{lūkha}}] coarse, rough; haggard%
\item[\textit{\textsanskrit{leṇa}}] a shelter%
\item[\textit{lesa}] a pretext%
\item[\textit{\textsanskrit{lokāyata}}] cosmology%
\item[\textit{\textsanskrit{loṇasakkharika}}] a razor%
\item[\textit{\textsanskrit{loṇasakkharikāya} \textsanskrit{chindituṁ}}] cut with a razor (to remove flesh) (see Appendix IV: Medical Terminology)%
\item[\textit{\textsanskrit{loṇasovīraka}}] salty purgative%
\item[\textit{\textsanskrit{lomaṁ} \textsanskrit{pāteti}}] to conduct oneself suitably%
\item[\textit{loha}] metal, iron, copper%
\item[\textit{lohita, lohitaka}] red; blood; a ruby%
\item[\textit{\textsanskrit{lohituppādaka}}] one who has caused the Buddha to bleed%
\item[\textit{vagga}] incomplete, an incomplete assembly; a group%
\item[\textit{\textsanskrit{vaggavādaka}}] to support%
\item[\textit{vacatta, vacattha}] white sweet flag (see Appendix V: Plants)%
\item[\textit{\textsanskrit{vacā}}] sweet flag (see Appendix V: Plants)%
\item[\textit{vacca}] feces%
\item[\textit{\textsanskrit{vaccakuṭi}}] a restroom%
\item[\textit{\textsanskrit{vaccakūpa}}] a cesspit%
\item[\textit{\textsanskrit{vaccaṭṭhāna}}] a place for defecating%
\item[\textit{\textsanskrit{vaccadoṇika}}] a toilet%
\item[\textit{vaccamagga}] the anus; the private parts (when used with \textit{\textsanskrit{passāvamagga}})%
\item[\textit{\textsanskrit{vaṇa}}] a sore%
\item[\textit{\textsanskrit{vaṇabandhanacoḷa}}] a dressing (for sores)%
\item[\textit{\textsanskrit{vaṇṇaka}}] cosmetics%
\item[\textit{vatta}] proper conduct%
\item[\textit{vatti,}] see \textit{vadati}%
\item[\textit{vatthikamma}] an enema (see Appendix IV: Medical Terminology)%
\item[\textit{vatthu}] a topic, a point, a case, a thing; a base (for a building), a site, land; (an action that is) the basis (for an offense, etc.), an offense; a ground; (a person or thing as) an object (of a legal procedure); a practice%
\item[\textit{\textsanskrit{vatthudesanā}}] approving a site (for a dwelling)%
\item[\textit{vadati, vatti}] to correct, to accuse; to ask (see Appendix I: Technical Terms)%
\item[\textit{vaddhika}] a strap (for a sandal)%
\item[\textit{vandati}] to pay respect%
\item[\textit{vayha}] a wagon%
\item[\textit{valli}] a creeper%
\item[\textit{\textsanskrit{vallikā}}] an earring; a creeper%
\item[\textit{vassa, \textsanskrit{vassāna}}] the rainy season, the rainy-season residence, the rains residence, the rains; a year%
\item[\textit{\textsanskrit{vassaṁvuṭṭha}}] who has completed the rains residence, who has spent the rains residence%
\item[\textit{\textsanskrit{vassāna},}] see \textit{vassa}%
\item[\textit{\textsanskrit{vassāvāsa}}] the rainy-season residence, the rains residence%
\item[\textit{\textsanskrit{vassikasāṭikā}}] a rainy-season robe%
\item[\textit{\textsanskrit{vātapāna}}] a window%
\item[\textit{\textsanskrit{vātapānakavāṭaka}}] a kind of shutter%
\item[\textit{\textsanskrit{vātapānabhisika}}] a kind of shutter%
\item[\textit{\textsanskrit{vātābādha}}] a certain disease (see Appendix IV: Medical Terminology)%
\item[\textit{\textsanskrit{vāraka}}] a bucket%
\item[\textit{\textsanskrit{vāḷa}}] a predatory animal, predatory%
\item[\textit{\textsanskrit{vāsāgāra}}] a dwelling%
\item[\textit{\textsanskrit{vāsi}}] an adz%
\item[\textit{\textsanskrit{vikatikā}}] a woolen rug decorated with the images of wild animals (see Appendix III: Furniture)%
\item[\textit{\textsanskrit{vikappanā}}] assignment (to another), assignment of ownership (to another) (see Appendix I: Technical Terms)%
\item[\textit{\textsanskrit{vikāla}}] the wrong time, after midday, late (in the day), at night%
\item[\textit{\textsanskrit{vikāsika}}] a bandage%
\item[\textit{vigarahati}] to rebuke%
\item[\textit{\textsanskrit{viññāpeti}}] to ask for, to request; to make understood, to declare%
\item[\textit{\textsanskrit{viññū}}] who understands, discerning%
\item[\textit{\textsanskrit{vitāna}}] a canopy%
\item[\textit{vidatthi}] a handspan (approximately 20 cm) (see Appendix I: Technical Terms)%
\item[\textit{vidha}] a buckle%
\item[\textit{\textsanskrit{vidhūpana}}] a fan%
\item[\textit{vinaya}] training (see Appendix I: Technical Terms); the Monastic Law; resolution (of a legal issue)%
\item[\textit{vinayadhara}] an expert on the Monastic Law%
\item[\textit{vinicchaya}] a discussion; an investigation; a decision%
\item[\textit{\textsanskrit{vinītavatthu}}] a case; a ground of training%
\item[\textit{vipatti}] a failure%
\item[\textit{vipanna}] a failure; deficient%
\item[\textit{\textsanskrit{vipariṇata}}] distorted%
\item[\textit{vibbhamati}] to disrobe (see Appendix I: Technical Terms)%
\item[\textit{\textsanskrit{vibhaṅga}}] analysis%
\item[\textit{\textsanskrit{vibhītaka}}] a belleric myrobalan (see Appendix V: Plants)%
\item[\textit{virecana}] a purgative, purging%
\item[\textit{\textsanskrit{vilaṅga},}] see \textit{\textsanskrit{bilaṅga}}%
\item[\textit{\textsanskrit{vivaṭṭa}}] a middle section (referring to robes)%
\item[\textit{\textsanskrit{vivāda}}] a dispute%
\item[\textit{\textsanskrit{vivādādhikaraṇa}}] a legal issue arising from a dispute%
\item[\textit{\textsanskrit{visaṅketa}}] not according to the arrangement%
\item[\textit{\textsanskrit{visāṇena} \textsanskrit{gāhetuṁ}}] receiving it in a horn (in connection with bloodletting) (see Appendix IV: Medical Terminology)%
\item[\textit{\textsanskrit{visuddhā}}] free from obstacles (for ordination)%
\item[\textit{\textsanskrit{visuddhāpekkha}}] desiring purification%
\item[\textit{\textsanskrit{vissāsa}}] trust%
\item[\textit{\textsanskrit{vihāra}}] a (monastic) dwelling; meditation (see Appendix I: Technical Terms)%
\item[\textit{\textsanskrit{vuṭṭhāti}}] to clear (an offense)%
\item[\textit{\textsanskrit{vuṭṭhāna}}] clearing (of an offense)%
\item[\textit{\textsanskrit{vuṭṭhāpanā}, \textsanskrit{vuṭṭhāna}}] full admission (of a woman as a nun)%
\item[\textit{\textsanskrit{vuḍḍha}}] senior, seniority, old, old age%
\item[\textit{\textsanskrit{vūpasama}}] resolving (a legal issue)%
\item[\textit{\textsanskrit{veṭhitasīsa}}] wearing a turban%
\item[\textit{vetta}] a cane%
\item[\textit{\textsanskrit{vedanāṭṭa}}] one who is overwhelmed by pain%
\item[\textit{vedika}] a railing%
\item[\textit{\textsanskrit{vepurisikā}}] a manlike woman%
\item[\textit{\textsanskrit{veyyāvacca}}] a service%
\item[\textit{\textsanskrit{veyyāvaccakara}}] a service provider%
\item[\textit{vema}] reed (a weaver’s instrument for making the weave firm)%
\item[\textit{vematika}] unsure%
\item[\textit{\textsanskrit{veḷuriya}}] a beryl%
\item[\textit{\textsanskrit{vehāsa}}] above ground, the air%
\item[\textit{\textsanskrit{vehāsakuṭi}}] an upper story (in a monastic dwelling)%
\item[\textit{vyatta,}] see \textit{byatta}%
\item[\textit{sauttaracchada}] a seat with a canopy (see Appendix III: Furniture)%
\item[\textit{\textsanskrit{saṁvacchara}}] a twelve-month period%
\item[\textit{\textsanskrit{saṁvāsa}}] a community; a formal meeting of the community%
\item[\textit{\textsanskrit{saṁvidhāya}}] by arrangement%
\item[\textit{\textsanskrit{saṁvelliya}}] a loin cloth%
\item[\textit{\textsanskrit{saṁsaṭṭhā}}] socializing%
\item[\textit{\textsanskrit{sakaṭa}}] a cart%
\item[\textit{sakkacca}] carefully, with care, respectfully%
\item[\textit{sakkhara}] a stone%
\item[\textit{sakkharika}] a small stone%
\item[\textit{sakkhali}] a pastry%
\item[\textit{sakyaputta}] a Sakyan%
\item[\textit{sakyaputtiya}] a (Sakyan) monastic%
\item[\textit{\textsanskrit{saṅkaccikā}}] a chest wrap%
\item[\textit{\textsanskrit{saṅkamati}}] to join%
\item[\textit{\textsanskrit{saṅkāra}}] trash%
\item[\textit{\textsanskrit{saṅketa}}] an appointment%
\item[\textit{\textsanskrit{saṅgīti}}] a Council, a communal recitation%
\item[\textit{\textsanskrit{saṅgha}}] the Sangha%
\item[\textit{\textsanskrit{saṅghakamma}}] a legal procedure of the Sangha, a legal procedure%
\item[\textit{\textsanskrit{saṅghabheda}}] a schism in the Sangha%
\item[\textit{\textsanskrit{saṅghabhedaka}}] one who has caused a schism in the Sangha; a schismatic; one who is pursuing schism in the Sangha%
\item[\textit{\textsanskrit{saṅgharāji}}] a fracture in the Sangha%
\item[\textit{\textsanskrit{saṅghāṭi}}] an outer robe; an upper robe; a robe (see Appendix I: Technical Terms)%
\item[\textit{\textsanskrit{saṅghādisesa}}] an offense entailing suspension%
\item[\textit{\textsanskrit{saṅghika}}] belonging to the Sangha%
\item[\textit{sacittaka}] intentionally%
\item[\textit{sajjulasa}] resin%
\item[\textit{\textsanskrit{sañcetanika}}] intentional%
\item[\textit{\textsanskrit{saññāpeti}}] to win over, to persuade; to instruct%
\item[\textit{\textsanskrit{saṇḍāsa}}] tweezers%
\item[\textit{sativinaya}] resolution through recollection%
\item[\textit{\textsanskrit{sattaṅga}}] a sofa (see Appendix III: Furniture)%
\item[\textit{\textsanskrit{sattāhakaraṇīya}}] seven-day business; for seven days%
\item[\textit{\textsanskrit{sattāhakālika}}] seven-day tonics; seven-day allowance (to go travelling during the rainy-season residence)%
\item[\textit{sattu}] flour; flour products (see Appendix I: Technical Terms)%
\item[\textit{sattha}] a group (of travelers), a caravan%
\item[\textit{satthaka}] a knife%
\item[\textit{satthakamma}] surgery%
\item[\textit{\textsanskrit{satthusāsana}}] the Teacher’s instruction%
\item[\textit{\textsanskrit{saddhivihārī}, \textsanskrit{saddhivihārinī}}] a student%
\item[\textit{\textsanskrit{santānaka}}] a cobweb%
\item[\textit{santhata}] a blanket, a mat; covered (see Appendix I: Technical Terms)%
\item[\textit{santhara}] a mat; a decking, a floor%
\item[\textit{\textsanskrit{santhāgāra}}] a public hall%
\item[\textit{\textsanskrit{santhāraka}}] a mat%
\item[\textit{\textsanskrit{sandamānikā}}] a chariot%
\item[\textit{\textsanskrit{sanniṭṭhānantika}}] ending when one makes a decision (referring to the robe season)%
\item[\textit{\textsanskrit{sannipāta}}] an assembly%
\item[\textit{\textsanskrit{sapadāna}}] continuous, in order%
\item[\textit{\textsanskrit{sappaṭibhaya}}] dangerous%
\item[\textit{\textsanskrit{sappāṇaka}}] containing living beings%
\item[\textit{sappi}] ghee%
\item[\textit{\textsanskrit{sabrahmacārī}}] a fellow monastic%
\item[\textit{\textsanskrit{sabhā}}] a public meeting hall%
\item[\textit{\textsanskrit{sabhāgāpatti}}] a shared offense, the same offense%
\item[\textit{samagga}] unanimous, a unanimous assembly, united, complete, a complete assembly%
\item[\textit{\textsanskrit{samaṇa}}] an ascetic, monastic%
\item[\textit{\textsanskrit{samaṇuddesa}}] a novice monastic%
\item[\textit{samatha}] settling (of a legal issue)%
\item[\textit{samathadhamma}] principle of settling (a legal issue)%
\item[\textit{\textsanskrit{samanubhāsati}}] to press (in connection with a \textit{\textsanskrit{saṅghakamma}})%
\item[\textit{samaya}] time; season; an appropriate occasion%
\item[\textit{\textsanskrit{samānasaṁvāsa}}] who belongs to the same community (see Appendix I: Technical Terms)%
\item[\textit{\textsanskrit{samānasaṁvāsako}}] one who belongs to the same Buddhist sect (see Appendix I: Technical Terms)%
\item[\textit{\textsanskrit{samānasaṁvāsasīmā}}] a boundary that defines who belongs to the same community%
\item[\textit{\textsanskrit{samānasīmā}}] within the same boundary%
\item[\textit{\textsanskrit{samānācariyaka}}] a co-pupil%
\item[\textit{\textsanskrit{samānupajjhāyaka}}] a co-student%
\item[\textit{samuccaya}] gathering up%
\item[\textit{\textsanskrit{samuṭṭhāna}}] origination%
\item[\textit{\textsanskrit{samodhānaparivāsa}}] simultaneous probation%
\item[\textit{sampajojeti}] to associate inappropriately with%
\item[\textit{sampatti}] success (in performing a legal procedure)%
\item[\textit{samparivattaka}] turning over%
\item[\textit{sambahula}] a number, several, three (see Appendix I: Technical Terms)%
\item[\textit{\textsanskrit{sambādha}}] private parts; crowded%
\item[\textit{\textsanskrit{sambhāraseda}}] sweating with herbs (as a treatment) (see Appendix IV: Medical Terminology)%
\item[\textit{\textsanskrit{sambhinnā}}] fistula%
\item[\textit{sambhoga}] living together with, interacting with%
\item[\textit{sammajjati}] to sweep%
\item[\textit{\textsanskrit{sammajjanī}}] a broom%
\item[\textit{sammannati}] to appoint, to designate, to establish; to approve, to agree upon; to honor%
\item[\textit{\textsanskrit{sammā} vattati}] to conduct oneself properly%
\item[\textit{sammukha}] face-to-face, presence%
\item[\textit{\textsanskrit{sammukhāvinaya}}] resolution face-to-face%
\item[\textit{sammuti}] appointment, designation, establishment, approval, agreement, permission%
\item[\textit{sammuti}] a (food-store) building agreed upon by the Sangha%
\item[\textit{sayana}] a resting place, a bed; lying down%
\item[\textit{sayanagata}] lying down%
\item[\textit{sayanighara}] a bedroom%
\item[\textit{\textsanskrit{sarabhañña}}] chanting%
\item[\textit{\textsanskrit{sarāvaka}}] a (water) scoop%
\item[\textit{\textsanskrit{salāka}}] a straw; a rib (of a sunshade); a spacer (for cloth attached to a \textit{kathina} frame); a shuttle (for weaving); a ticket, a ballot; a decision%
\item[\textit{\textsanskrit{salākaggāhā}}] a vote%
\item[\textit{\textsanskrit{salākaggāhāpaka}}] a manager of voting%
\item[\textit{\textsanskrit{salākabhatta}}] a meal for which lots are drawn%
\item[\textit{\textsanskrit{salākodhāniya}}] a case for an (ointment) stick%
\item[\textit{\textsanskrit{savacanīya} karoti}] to direct (someone)%
\item[\textit{savanantika}] ending when one hears about it (referring to the robe season)%
\item[\textit{\textsanskrit{savanūpacāra}}] the range of hearing%
\item[\textit{\textsanskrit{sahajīvinī}}] a disciple (of a nun)%
\item[\textit{\textsanskrit{sahatthā}}] personally, oneself; with (one’s) own hands%
\item[\textit{sahadhammika}] legitimately; a fellow believer%
\item[\textit{\textsanskrit{sahubbhārā}}] ending together (referring to the robe season)%
\item[\textit{\textsanskrit{sāṇa}}] sunn hemp (see Appendix V: Plants)%
\item[\textit{\textsanskrit{sāṭaka}}] a wrap garment, a wrap%
\item[\textit{\textsanskrit{sāṭiyaggāhāpaka}}] a distributor of rainy-season bathing cloths%
\item[\textit{\textsanskrit{sāṇipākāra}}] a cloth screen%
\item[\textit{\textsanskrit{sādiyati}}] to consent, to accept, to like%
\item[\textit{\textsanskrit{sādhāraṇa}}] common (offenses for monks and nuns)%
\item[\textit{\textsanskrit{sāmaggī}}] unity, a complete assembly, unanimous%
\item[\textit{\textsanskrit{sāmaṇera}}] a novice monk, novice%
\item[\textit{\textsanskrit{sāmaṇerī}}] a novice nun%
\item[\textit{\textsanskrit{sāmīci}}] proper%
\item[\textit{\textsanskrit{sāmīcikamma}}] to do acts of respect (toward)%
\item[\textit{\textsanskrit{sāmudda}}] sea salt%
\item[\textit{\textsanskrit{sāratta}}] lustful%
\item[\textit{\textsanskrit{sāreti}}] to remind (someone of an offense)%
\item[\textit{\textsanskrit{sālā}}] a building, a hall, a shed (see Appendix I: Technical Terms)%
\item[\textit{\textsanskrit{sālūka}}] a lotus tuber%
\item[\textit{\textsanskrit{sāvana},}] see \textit{\textsanskrit{anussāvana}}%
\item[\textit{\textsanskrit{sāvasesā} \textsanskrit{āpatti}}] a curable offense%
\item[\textit{\textsanskrit{sāsaṅka}}] risky%
\item[\textit{\textsanskrit{sāsana}}] instruction; Buddhism%
\item[\textit{\textsanskrit{sāsapakuṭṭa}}] mustard powder%
\item[\textit{\textsanskrit{sikkā}}] a carrying net%
\item[\textit{\textsanskrit{sikharaṇī}}] a woman who has genital prolapse%
\item[\textit{\textsanskrit{sikkhaṁ} \textsanskrit{paccakkhāti}, \textsanskrit{paccācikkhāti}}] to renounce the training%
\item[\textit{\textsanskrit{sikkhamānā}}] a trainee nun%
\item[\textit{\textsanskrit{sikkhākāma}}] one who is fond of the training%
\item[\textit{\textsanskrit{sikkhāpada}}] a training rule%
\item[\textit{\textsanskrit{siṅgivera}}] ginger%
\item[\textit{\textsanskrit{siṅghāṭaka}}] an intersection%
\item[\textit{sitthatelaka}] beeswax%
\item[\textit{sindhava}] hill salt%
\item[\textit{\textsanskrit{sipāṭika}}] a case%
\item[\textit{sippa}] a profession%
\item[\textit{\textsanskrit{silā}}] a stone, slate, quartz%
\item[\textit{sivika}] a palanquin%
\item[\textit{\textsanskrit{sītāloḷī}}] mud from a plow (see Appendix IV: Medical Terminology)%
\item[\textit{\textsanskrit{sīmā}}] a monastery zone, a zone (see Appendix I: Technical Terms)%
\item[\textit{\textsanskrit{sīmātikkantika}}] ending when one is outside the monastery zone (referring to the robe season)%
\item[\textit{\textsanskrit{sīsa}}] lead (the metal)%
\item[\textit{\textsanskrit{sīsābhitāpa}}] a headache%
\item[\textit{sugata}] standard (when used with measurements) (see Appendix I: Technical Terms)%
\item[\textit{\textsanskrit{sugataṅgula}}] the standard fingerbreadth (approximately 5 cm) (see Appendix I: Technical Terms)%
\item[\textit{\textsanskrit{sugatacīvarappamāṇa}}] the standard robe measure%
\item[\textit{sugatavidatthi}] the standard handspan (approximately 60 cm) (see Appendix I: Technical Terms)%
\item[\textit{sutta}] the Monastic Code, the rules (of the Monastic Code); a string, thread, yarn%
\item[\textit{suttantika}] an expert on the discourses%
\item[\textit{\textsanskrit{sudhā}}] plaster%
\item[\textit{\textsanskrit{sulasī}, \textsanskrit{tulasī}}] holy basil (see Appendix V: Plants)%
\item[\textit{\textsanskrit{suvaṇṇa}}] gold (see Appendix I: Technical Terms)%
\item[\textit{\textsanskrit{suvūpasanta}}] properly disposed of (legal procedure)%
\item[\textit{\textsanskrit{susāna}}] a charnel ground%
\item[\textit{\textsanskrit{sūci}}] a needle%
\item[\textit{\textsanskrit{sūcika}}] a needle; a bolt%
\item[\textit{\textsanskrit{sūpa}}] curry, bean curry%
\item[\textit{sekhiya}] training%
\item[\textit{\textsanskrit{seṭṭhi}}] a wealthy merchant%
\item[\textit{\textsanskrit{setaṭṭhikā}, \textsanskrit{setaṭṭikā}}] a whitehead%
\item[\textit{sedakamma}] treatment through sweating%
\item[\textit{\textsanskrit{senāsana}}] a dwelling; furniture; a resting place (see Appendix I: Technical Terms)%
\item[\textit{seyya}] a sleeping place%
\item[\textit{\textsanskrit{seyyaṁ} kappeti}] to lie down, to lie down in a sleeping place; to arrange a sleeping place%
\item[\textit{sevanacitta}] intention of sexual relations%
\item[\textit{\textsanskrit{sevanādhippāya}}] aiming at connection%
\item[\textit{sesaka}] leftovers; remainder%
\item[\textit{\textsanskrit{sotañjana}}] river ointment (see Appendix IV: Medical Terminology)%
\item[\textit{\textsanskrit{sopāna}}] stairs, a staircase%
\item[\textit{\textsanskrit{sovaṇṇa}}] gold%
\item[\textit{sosa}] tuberculosis%
\item[\textit{hattha}] a forearm (approximately 40 cm) (see Appendix I: Technical Terms)%
\item[\textit{hatthato}] directly from (someone)%
\item[\textit{hatthatthara}] an elephant-back rug (see Appendix III: Furniture)%
\item[\textit{\textsanskrit{hatthapāsa}}] an arm’s reach%
\item[\textit{\textsanskrit{hatthavaṭṭaka}}] a rickshaw%
\item[\textit{\textsanskrit{hatthavikāra}}] a hand signal%
\item[\textit{\textsanskrit{hatthavilaṅghaka}}] together%
\item[\textit{hammiya}] a stilt house (see Appendix I: Technical Terms)%
\item[\textit{harita}] a cultivated plant%
\item[\textit{\textsanskrit{harītaka}}] chebulic myrobalan (see Appendix V: Plants)%
\item[\textit{haliddi}] turmeric (see Appendix V: Plants)%
\item[\textit{\textsanskrit{hāpeti}}] to omit; to neglect%
\item[\textit{\textsanskrit{hiṅgu}}] the shrub asafoetida (see Appendix V: Plants)%
\item[\textit{\textsanskrit{hintāla}}] a fishtail palm (see Appendix V: Plants)%
\item[\textit{\textsanskrit{hirañña}}] money, a gold coin (see Appendix I: Technical Terms)%
\item[\textit{hirivera}] the fragrant swamp mallow (see Appendix V: Plants)%
\item[\textit{hetu}] cause%
\item[\textit{hemanta}] the winter%
\end{description}

%
\chapter*{Appendix I: Technical Terms}
\addcontentsline{toc}{chapter}{Appendix I: Technical Terms}
\markboth{Appendix I: Technical Terms}{Appendix I: Technical Terms}

The discussion in this appendix focuses on words that in previous Vinaya translations have not been properly understood or only understood in part. As such, I do not give all relevant meanings of a particular word, but focus on those aspects where I propose a new, improved, or additional understanding. In some cases, I also discuss historical developments and try to discern to what extent the meaning of certain words has changed over time. The focus is on core Vinaya terminology.

\subsection*{\textit{Akuppa}: “irreversible”}

See \textit{kuppa}.

\subsection*{\textit{\textsanskrit{Aggaḷa}}: “door”}

I. B. Horner translates \textit{\textsanskrit{aggaḷa}} as “bolt”,\footnote{E.g. at BD 2.258 and BD 4.342. } as does Bhikkhu Bodhi in the “Numerical Discourses of the Buddha”, yet it is far from clear that this is the correct rendering. In several places, the Vinaya \textsanskrit{Piṭaka} has a list of door appurtenances. This list includes a detailed description of things that pertain to doors: a door panel (\textit{\textsanskrit{kavāṭa}}), a lintel (\textit{\textsanskrit{piṭṭha}}),\footnote{In some cases, however, \textit{\textsanskrit{piṭṭha}} seems more likely to mean “doorpost” or “doorframe”. \href{https://suttacentral.net/pli-tv-kd1/en/brahmali\#25.15.1}{Kd1:25.15.1}: \textit{\textsanskrit{Mañco} \textsanskrit{nīcaṁ} \textsanskrit{katvā} \textsanskrit{sādhukaṁ} \textsanskrit{appaṭighaṁsantena}, \textsanskrit{asaṅghaṭṭentena} \textsanskrit{kavāṭapiṭṭhaṁ}, \textsanskrit{nīharitvā} \textsanskrit{ekamantaṁ} nikkhipitabbo}, “Holding the bed low, he should carefully take it out without scratching it or knocking it against the door or the door frame, and he should put it aside.” Perhaps the compound \textit{\textsanskrit{piṭṭhisaṅghāṭa}} simply refers to a doorframe. } door posts (\textit{\textsanskrit{saṅghāṭa}}), a lower hinge (\textit{udukkhalika}), an upper hinge (\textit{\textsanskrit{uttarapāsaka}}), a door jamb (\textit{\textsanskrit{aggaḷavaṭṭi}}), a bolt eye (\textit{\textsanskrit{kapisīsaka}}), a bolt (\textit{\textsanskrit{sūcika}}), a lockable bolt (\textit{\textsanskrit{ghaṭika}}), a keyhole (\textit{\textsanskrit{tāḷacchidda}}), a door-pulling hole (\textit{\textsanskrit{āviñchanacchidda}}), and a door-pulling rope (\textit{\textsanskrit{āviñchanarajju}}). \textit{\textsanskrit{Aggaḷa}}, however, is conspicuously absent. This makes it unlikely that \textit{\textsanskrit{aggaḷa}} should simply be equated with “bolt” or any other basic part of a door, especially since the above list contains two words that specifically mean bolt, namely, \textit{\textsanskrit{sūcika}} and \textit{\textsanskrit{ghaṭika}}.

Another possible meaning for \textit{\textsanskrit{aggaḷa}} is “door”, as suggested by DOP. Let us have a look at some relevant contexts.

\begin{quotation}%
“Monks, imagine a light ball of thread placed on an \textit{\textsanskrit{aggaḷa}}-plank made entirely of heartwood.” (\href{https://suttacentral.net/mn119/en/sujato\#26.2}{MN~119:26.2})

%
\end{quotation}

Here it is hard to imagine that \textit{\textsanskrit{aggaḷa}} could have anything to do with a bolt. That it should refer to a door, however, makes good sense, and \textit{\textsanskrit{aggaḷaphalake}} would then mean something like a “door-panel.”

At \href{https://suttacentral.net/pli-tv-bu-vb-pc19/en/brahmali\#2.1.9}{Bu~Pc~19} \textit{\textsanskrit{aggaḷaṭṭhapanāya}} is glossed as \textit{\textsanskrit{dvāraṭṭhapanāya}}. \textit{\textsanskrit{Dvāra}} unambiguously means “gate” or “door”, and so that must be the meaning of \textit{\textsanskrit{aggaḷa}} too, at least in this context.

Then we have the fairly common expression \textit{\textsanskrit{aggaḷaṁ} \textsanskrit{ākoteti}}:

\begin{quotation}%
Having entered the porch, having coughed, he knocked the \textit{\textsanskrit{aggaḷa}}. The Buddha opened the door. (\href{https://suttacentral.net/dn3/en/sujato\#1.9.1}{DN~3:1.9.1})

%
\end{quotation}

To translate this as “knocking (on) the door-bolt,” as I. B. Horner does, is not meaningful. “Knocking on the door/door-panel”, however, is straightforward.

The commentaries support this understanding of \textit{\textsanskrit{aggaḷaṁ} \textsanskrit{ākoṭesi}}:

\begin{quotation}%
The \textit{\textsanskrit{aggaḷa}} is the door in the doorway.\footnote{DN‑a 1.260: \textit{\textsanskrit{Aggaḷanti} \textsanskrit{dvārakavāṭaṁ}}. }

%
\end{quotation}

\begin{quotation}%
Knocking on the \textit{\textsanskrit{aggaḷa}} means apply a sign on the door with the tip of the nails.\footnote{MN‑a 1.273: \textit{\textsanskrit{Aggaḷaṁ} \textsanskrit{ākoṭesīti} agganakhena \textsanskrit{kavāṭe} \textsanskrit{saññaṁ} \textsanskrit{adāsi}}. }

%
\end{quotation}

\begin{quotation}%
Knocking on the \textit{\textsanskrit{aggaḷa}} means knocking on the door-panel with the tip of the nails.\footnote{AN‑a 9.4: \textit{\textsanskrit{Aggaḷaṁ} \textsanskrit{ākoṭesīti} agganakhena \textsanskrit{dvārakavāṭaṁ} \textsanskrit{ākoṭesi}}. }

%
\end{quotation}

At MN 21 we have the compound \textit{\textsanskrit{aggaḷasūci}}, translated by \textsanskrit{Ñāṇamoli} and Bodhi in the “Middle Length Discourses of the Buddha” as “rolling-pin,” but it is hard to see how they might justify this. We have seen above that \textit{\textsanskrit{sūci}}, when used in conjunction with doors, means “bolt”. If \textit{\textsanskrit{aggaḷa}} too means “bolt”, the compound would not make sense. If \textit{\textsanskrit{aggaḷa}} means door, then an \textit{\textsanskrit{aggaḷasūci}} would be a “door-bolt.” This fits with the story in MN 21, where Mistress \textsanskrit{Vedehikā} hits her servant \textsanskrit{Kālī} on the head and causes her to bleed.

If we take \textit{\textsanskrit{aggaḷa}} to mean a door or a door-panel, this also makes sense in the use of \textit{\textsanskrit{aggaḷa}} as a patch of cloth, as found in the non-offense clause to \href{https://suttacentral.net/pli-tv-bu-vb-pc58/en/brahmali\#2.3.6}{Bu~Pc~58} and in the Chapter on Robes at \href{https://suttacentral.net/pli-tv-kd8/en/brahmali\#14.1.9}{Kd~8:14.1.9}. A patch and a panel have much in common, in that they both cover a hole. This explains the shared name. On the other hand, there is no obvious reason a patch and a bolt would share a name.

I conclude from the above that “door” or “door-panel” is the main meaning of \textit{\textsanskrit{aggaḷa}} in the Vinaya \textsanskrit{Piṭaka}. There are a few more references to \textit{\textsanskrit{aggaḷa}} that I have not mentioned, but they do not add much to the above. The only exception is the compound \textit{\textsanskrit{aggaḷaguttivihāro}}, “a dwelling kept safe by an \textit{\textsanskrit{aggaḷa}}.” In this case the obvious meaning of \textit{\textsanskrit{aggaḷa}} is “bolt” or “lock.” But even here it could refer to a door, with the existence of a bolt/lock being implied.

It is this last usage of \textit{\textsanskrit{aggaḷa}} which perhaps gives us the final clue to its meaning. I would suggest the \textit{\textsanskrit{aggaḷa}} is a complete door, including all the parts that go into a door. This is why \textit{\textsanskrit{aggaḷa}} does not appear as a separate part in the list quoted above. This is also why there is no need to mention the lock when a hut is said to be guarded by an \textit{\textsanskrit{aggaḷa}}—the lock is implied. In contrast, \textit{\textsanskrit{kavāṭa}} refers to a door-panel, and as such it is included in the door-part list. And a \textit{\textsanskrit{dvāra}} is a door in the sense of a door-way. It includes grand “doors” such as gates and gateways found at the entry points to villages and towns.

So \textit{\textsanskrit{aggaḷa}} means “door.” If it ever means “bolt,” this is no more than an extended meaning. It makes sense that this fairly rare word should only have a single overarching meaning rather than two quite distinct ones.

\subsection*{\textit{\textsanskrit{Aggisālā}}: “water-boiling shed”}

\textit{\textsanskrit{Aggisālā}} literally means “a fire shed”. The nature of the \textit{\textsanskrit{aggisālā}} is not clear from the Canonical texts, the term being found almost exclusively in lists of buildings and stock passages. We need to turn to the commentaries to get a clearer view.

\begin{quotation}%
After boiling dye outside, the dye-vessel, the dye-ladle, and the dye-trough should all be put away in the \textit{\textsanskrit{aggisālā}}.\footnote{Sp 2.112: \textit{\textsanskrit{Ajjhokāse} \textsanskrit{rajanaṁ} \textsanskrit{pacitvā} \textsanskrit{rajanabhājanaṁ} \textsanskrit{rajanauḷuṅko} \textsanskrit{rajanadoṇikāti} \textsanskrit{sabbaṁ} \textsanskrit{aggisālāya} \textsanskrit{paṭisāmetabbaṁ}.} }

%
\end{quotation}

\begin{quotation}%
\textit{\textsanskrit{Aggisālā}}: the shed for boiling with a fire.\footnote{Sp-yoj 2.112: \textit{\textsanskrit{Aggisālāyanti} \textsanskrit{agginā} \textsanskrit{pacanasālāyaṁ}.} }

%
\end{quotation}

This suggests that the purpose of this building was to boil water, for instance to dye cloth. Occasionally \textit{\textsanskrit{aggisālā}} can also refer to a “fire hut”, as used by ascetics who tended the sacred fire, as in \href{https://suttacentral.net/pli-tv-kd1/en/brahmali\#15.6.4}{Kd~1:15.5.4}.

See also entry on \textit{\textsanskrit{sālā}}.

\subsection*{\textit{\textsanskrit{Aṅgula}}: “fingerbreadth”}

See \textit{sugata}.

\subsection*{\textit{\textsanskrit{Ajjhokāsa}/\textsanskrit{abbhokāsa}}: “out in the open”, “outside”}

Both DOP and CPD give “the open air” for \textit{\textsanskrit{ajjhokāsa}}, which suggests it just means outdoors. \textit{\textsanskrit{Ajjhokāsa}}, however, is regularly contrasted with \textit{\textsanskrit{rukkhamūla}}, “the foot of a tree”. A particularly telling juxtaposition is found in the Chapter on Entering the Rainy-season Residence (Kd 3):

\begin{quotation}%
At one time monks entered the rains residence \textit{\textsanskrit{ajjhokāsa}}. When it was raining, they ran for cover under trees (\textit{\textsanskrit{rukkhamūla}}) and eaves. (\href{https://suttacentral.net/pli-tv-kd3/en/brahmali\#12.5.1}{Kd~3:12.5.1})

%
\end{quotation}

It is clear from this that \textit{\textsanskrit{ajjhokāsa}} is opposed to “under cover” and must therefore mean “out in the open”. In some contexts, however, “outside” seems to be a better rendering:

\begin{quotation}%
Then, after spending much of the night \textit{\textsanskrit{ajjhokāsa}}, the Buddha entered the dwelling … (\href{https://suttacentral.net/pli-tv-kd5/en/brahmali\#13.9.1}{Kd~5:13.9.1})

%
\end{quotation}

I vary my translation accordingly.

\subsection*{\textit{\textsanskrit{Aḍḍhayoga}:} “stilt house”}

In the Vinaya \textsanskrit{Piṭaka}, the \textit{\textsanskrit{aḍḍhayoga}} is normally grouped with the \textit{hammiya} and the \textit{\textsanskrit{pāsāda}}. According to the commentaries, all three buildings are different kinds of “stilt houses”. Rather than try to differentiate between them, which is not necessary from a practical perspective, I have grouped them together as “stilt house”. Here is what the commentaries have to say:

\begin{quotation}%
“An \textit{\textsanskrit{aḍḍhayoga}} is a house bent like a \textit{\textsanskrit{supaṇṇa}}.”\footnote{Sp 4.294: \textit{\textsanskrit{Aḍḍhayogoti} \textsanskrit{supaṇṇavaṅkagehaṁ}}. }

%
\end{quotation}

\begin{quotation}%
“A house bent like a \textit{\textsanskrit{supaṇṇa}}: a house made in the shape of the wings of a \textit{\textsanskrit{garuḷa}}.”\footnote{Sp-\textsanskrit{ṭ} 4.294: \textit{\textsanskrit{Supaṇṇavaṅkagehanti} \textsanskrit{garuḷapakkhasaṇṭhānena} \textsanskrit{katagehaṁ}}. }

%
\end{quotation}

A \textit{\textsanskrit{garuḷa}}, better known in its Sanskrit form \textit{\textsanskrit{garuḍa}}, is a mythological bird. The commentary continues:

\begin{quotation}%
“A \textit{\textsanskrit{pāsāda}} is a long stilt house. A \textit{hammiya} is just a \textit{\textsanskrit{pāsāda}} that has an upper room on top of its flat roof.”\footnote{Sp 4.294: \textit{\textsanskrit{Pāsādoti} \textsanskrit{dīghapāsādo}. Hammiyanti \textsanskrit{upariākāsatale} \textsanskrit{patiṭṭhitakūṭāgāro} \textsanskrit{pāsādoyeva}}. }

%
\end{quotation}

At Sp-\textsanskrit{ṭ} 3.74 we find slightly different explanations. It is clear, however, that all three are stilt houses that are distinguished according to their shape and the kind of roof they possess. See also \textit{\textsanskrit{pāsāda}} in this same Appendix.

\subsection*{\textit{\textsanskrit{Anujānāti}}: “should”, “to instruct”}

This verb ranges in meaning from “allowing” to “requiring”. Here are a couple of examples to illustrate this point:

\begin{quotation}%
“We allow Sudinna to go forth from home to homelessness.” (\href{https://suttacentral.net/pli-tv-bu-vb-pj1/en/brahmali\#5.4.8}{Bu~Pj~1:5.4.8})

%
\end{quotation}

The context—parents allowing a young man to go forth—makes it clear that this must be an allowance. With the following example, however, it is equally clearly a requirement:

\begin{quotation}%
“Monks, you should recite the Monastic Code.” (\href{https://suttacentral.net/pli-tv-kd2/en/brahmali\#3.2.4}{Kd~2:3.2.4})

%
\end{quotation}

There is no clear cut-off point between the two meanings, one tending to segue into the other. The only aid to decide which meaning is appropriate is context. I vary my translation from “to allow” to “to instruct”, but often I just use “should” plus a main verb. The “should” represents \textit{\textsanskrit{anujānāti}}, whereas the main verb represents the action that is either prescribed or proscribed.

\subsection*{\textit{Antepura/\textsanskrit{rājantepura}}: “royal compound”}

Some of the best evidence for the meaning of this word comes from \href{https://suttacentral.net/pli-tv-bu-vb-pc83/en/brahmali\#0.5}{Bu~Pc~83}. According to the origin story to this rule, the following things relate to the \textit{\textsanskrit{rājantepura}}: a harem, secret deliberations, a father longing for his son, someone being promoted/demoted, the army being dispatched, and the trampling of elephants, horses, and chariots. This is more than a mere royal residence. It seems to suggest the seat of government and perhaps a little town in its own right.

This picture is supported by evidence from other passages. Criminals are sentenced in the \textit{antepura}, (\href{https://suttacentral.net/pli-tv-kd1/en/brahmali\#43.1.2}{Kd~1:43.1.2}); the king’s home is only one building among many inside the \textit{antepura}, (\href{https://suttacentral.net/pli-tv-kd8/en/brahmali\#1.13.15}{Kd~8:1.13.15}); a carriage reaches the gate of the \textit{antepura}, (\href{https://suttacentral.net/mn24/en/sujato\#14.11}{MN~24:14.11}); the \textit{antepura} seems to be a place within a town, (\href{https://suttacentral.net/pli-tv-kd17/en/brahmali\#1.6.8}{Kd~17:1.6.8}); the King’s assembly is seated in the \textit{antepura}, (\href{https://suttacentral.net/an3.60/en/sujato\#7.5}{AN~3.60:7.5}).

Then there is the compound \textit{\textsanskrit{uparipāsādavaragata}}, “ascended (his) finest stilt house”, normally referring to kings. If the \textit{antepura} is a whole compound, then presumably it included a number of \textit{\textsanskrit{pāsādas}}, “stilt houses”, one of which may have been the best one, perhaps the king’s favorite.

Moreover, the fact that the \textit{antepura} had a gate would seem to mean that it was a clearly defined area, perhaps surrounded by walls or at least a fence. According to \href{https://suttacentral.net/pli-tv-bu-vb-pc83/en/brahmali\#1.2.11,%20monastics%20would%20not%20normally%20go%20for%20alms%20to%20the%20%3Ci%20translate='no'%20lang='pli'%3Eantepura%3C/i%3E}{Bu~Pc~83:1.2.11}, yet they would go for alms in \textsanskrit{Sāvatthī}. From this it would again appear that the \textit{antepura} was a compound within a city, which fits the literal meaning of \textit{antepura}, “within the city”.

\subsection*{\textit{Abbhantara}}

See \textit{sugata}.

\subsection*{\textit{Amagga}: “mouth”}

See \textit{magga}.

\subsection*{\textit{\textsanskrit{Āpadā}}: “emergency”}

\textit{\textsanskrit{Āpadā}} is essentially the opposite of \textit{\textsanskrit{sampadā}}, “success”, “accomplishment”, “good fortune”, etc., and as such it might be rendered as “misfortune”. In the Vinaya \textsanskrit{Piṭaka} \textit{\textsanskrit{āpadā}} points to external dangers and stressful situations. Possible renderings are “times of distress” or “difficult circumstance”. In the non-offense clauses of the Sutta-\textsanskrit{vibhaṅga}, however, which is where this word is most commonly found, the meaning is closer to an immediate danger that stops you from taking a normally required action. Examples include a monastic who is unable to put away furniture before departing because of an \textit{\textsanskrit{āpadā}} (\href{https://suttacentral.net/pli-tv-bu-vb-pc14/en/brahmali\#2.3.7}{Bu~Pc~14:2.3.7} and \href{https://suttacentral.net/pli-tv-bu-vb-pc15/en/brahmali\#2.3.8}{Bu~Pc~15:2.3.8}), and 73 out of the 75 \textit{sekhiya} rules on proper behaviors, which need not be adhered to when there is an \textit{\textsanskrit{āpadā}}. The idea of an immediate danger is best captured by “emergency”.

\subsection*{\textit{\textsanskrit{Ārāma}: “}a park”, “a monastery”}

“Park” is the more fundamental meaning of \textit{\textsanskrit{ārāma}}. However, since such parks were sometimes given to the Sangha to serve as monasteries, the monasteries, too, became known by the same name. It is the latter meaning which predominates in the Vinaya \textsanskrit{Piṭaka}.

\subsection*{\textit{\textsanskrit{Upacāra}}: “access”, “vicinity”}

\textit{\textsanskrit{Upacāra}}, which is composed of the root \textit{car}, “move”, and the prefix \textit{upa}, “toward”, can perhaps most comprehensively be rendered as “approach”. The word “approach” has the benefit of encompassing two common translations of \textit{\textsanskrit{upacāra}}, “vicinity” and “access”. Ideally, we would use “approach” for all contexts of \textit{\textsanskrit{upacāra}}. For practical application, however, “approach” may not be precise enough to convey the contextual meaning in any given Pali passage.

In the Vinaya \textsanskrit{Piṭaka}, \textit{\textsanskrit{upacāra}} is mostly used in conjunction with physical spaces. As a matter of common sense, “vicinity” fits better whenever \textit{\textsanskrit{upacāra}} relates to an “unenclosed” space, whereas “access” fits better with “enclosed” spaces. The Canonical text aligns well with this understanding, with \textit{\textsanskrit{upacāra}} as “access” referring specifically to the entryway through an enclosure. This is the word definition at Bu Pj 2:

\begin{quotation}%
The \textit{\textsanskrit{upacāra}} to an inhabited area: of an enclosed inhabited area: the stone-throw of a man of average height standing at the threshold of the gateway to the inhabited area; of an unenclosed inhabited area: the stone-throw of a man of average height standing at the \textit{\textsanskrit{upacāra}} to a house. (\href{https://suttacentral.net/pli-tv-bu-vb-pj2/en/brahmali\#3.7}{Bu~Pj~2:3.7})

%
\end{quotation}

Here is what the commentary says:

\begin{quotation}%
Enclosed: … It is to this extent that the state of having a single \textit{\textsanskrit{upacāra}} for a one-clan village is shown. … Unenclosed: with this the state of having many \textit{\textsanskrit{upacāras}} for that same village is shown.\footnote{Sp 2.177: \textit{Parikkhittoti … \textsanskrit{Ettāvatā} \textsanskrit{ekakulagāmassa} \textsanskrit{ekūpacāratā} \textsanskrit{dassitā}. … Aparikkhittoti \textsanskrit{iminā} tasseva \textsanskrit{gāmassa} \textsanskrit{nānūpacāratā} \textsanskrit{dassitā}}. }

%
\end{quotation}

The commentary is here making the case that an unenclosed inhabited area has as many \textit{\textsanskrit{upacāras}} as there are houses. However, there are rules such as \href{https://suttacentral.net/pli-tv-bu-vb-np29/en/brahmali\#2.9}{Bu~NP~29} and \href{https://suttacentral.net/pli-tv-bu-vb-pc15/en/brahmali\#2.1.19}{Bu~Pc~15} where a village or monastery is said to have one \textit{\textsanskrit{upacāra}} (singular). To serve us better in such rules, it seems reasonable to regard the \textit{\textsanskrit{upacāra}} of an unenclosed inhabited area as the sum of the \textit{\textsanskrit{upacāras}} to the individual houses, that is, as the “vicinity” of the inhabited area. Given that \textit{\textsanskrit{upacāra}} spans the meanings “access” and “vicinity”, we are probably justified in using both terms, choosing the appropriate one according to context.

There are a few contexts where neither of the above two renderings is satisfactory, for instance in the cases of \textit{dassan’\textsanskrit{ūpacāra}} and \textit{savan’\textsanskrit{ūpacāra}}., e.g. in \href{https://suttacentral.net/pli-tv-bu-vb-pc42/en/brahmali\#2.1.15}{Bu~Pc~42}. In these cases, \textit{\textsanskrit{upacāra}} refers to what is accessible by sight or by hearing, literally, “seeing access” and “hearing access”. To bring out this meaning in idiomatic English, I render these as “within (the range of) sight/hearing”.

\subsection*{\textit{Uposatha}: “observance-day ceremony”}

The \textit{uposatha} is the day of religious observance. On this day the monastic community meets to recite the monastic rules, whereas lay Buddhists often go to a monastery to keep the eight precepts and to listen to teachings. This ancient practice is described both in the Suttas and the Vinaya \textsanskrit{Piṭaka}. My most general rendering of \textit{uposatha} is therefore “observance day”.

“To do the \textit{uposatha}”, \textit{\textsanskrit{uposathaṁ} karoti}, refers the act of reciting the monastic rules. Although the \textit{uposatha}, together with the \textit{\textsanskrit{pavāraṇā}} (“invitation”) ceremony, has historically been considered a \textit{\textsanskrit{saṅghakamma}}, a monastic legal procedure, this may not have been the case when it was first instituted, as can be seen from the following quote:

\begin{quotation}%
“You’re excluded from the observance-day ceremony, from the invitation ceremony, and from the legal procedures of the Sangha.” (\href{https://suttacentral.net/pli-tv-bu-vb-ss8/en/brahmali\#3.1.4}{Bu~Ss~8:3.1.4})

%
\end{quotation}

A similar distinction between the \textit{uposatha} and \textit{\textsanskrit{saṅghakamma}} is found here:

\begin{quotation}%
“From today on, Ānanda, I’ll do the observance-day ceremony and the legal procedures of the Sangha separate from the Buddha and the Sangha of monks.” (\href{https://suttacentral.net/pli-tv-kd17/en/brahmali\#3.17.4}{Kd~17:3.17.4})

%
\end{quotation}

In both these cases, \textit{uposatha} is placed side-by-side with \textit{\textsanskrit{saṅghakamma}}, suggesting that the \textit{uposatha} was not regarded as a \textit{\textsanskrit{saṅghakamma}}, or at least as a \textit{\textsanskrit{saṅghakamma}} with special significance. Moreover, according to the Chapter on the Observance Day, even groups of less than four monastics, down to a single monk or nun, are said to do the \textit{uposatha} (\href{https://suttacentral.net/pli-tv-kd2/en/brahmali\#26.2.5}{Kd~2:26.2.5}–26.9.4). This cannot refer to \textit{\textsanskrit{saṅghakamma}} proper since legal procedures require a minimum of four monastics. For these reasons I do not render the \textit{uposatha} proceedings as “observance-day (legal) procedure”, but rather as “observance-day ceremony”. Only when \textit{uposatha} and \textit{kamma} are used together in the single compound \textit{uposathakamma}, which may well be a later development, do I render the combination as “observance-day procedure”.

\subsection*{\textit{\textsanskrit{Ubhatobyañjanaka}}: “hermaphrodite”}

\textit{\textsanskrit{Ubhatobyañjanaka}}, literally, “one who has both characteristics”, is a reference to a person who has both male and female genitalia. Exactly what is meant by this in a Vinaya context is debatable. It does not seem to be equivalent to what we now term an intersex person.

The term is not found in Sutta \textsanskrit{Piṭaka} and only rarely in the Vinaya \textsanskrit{Piṭaka}. Almost all references are found in contexts that are likely to be relatively late, especially passages that set out elaborate permutation series, such as in Bu Pj 1 or in Kd 9.\footnote{\href{https://suttacentral.net/pli-tv-bu-vb-pj1/en/brahmali\#9.1.3}{Bu Pj 1:9.1.3}–9.6.31 and \href{https://suttacentral.net/pli-tv-kd9/en/brahmali\#4.2.39}{Kd 9:4.2.39}–4.7.25. } Bhikkhu*\textsanskrit{nī} Vimala, in the essay Yellow Gate, judges the term to be a relatively late addition to the Vinaya \textsanskrit{Piṭaka}, possibly added at the second Council, \textit{\textsanskrit{saṅgīti}}. It seems unlikely that the Buddha himself used the term.

For an explanation of \textit{\textsanskrit{ubhatobyañjanaka}} we need to turn to the commentaries. The main explanation is as follows:

\begin{quotation}%
‘\textit{\textsanskrit{Ubhatobyañjanaka}}’ means: because of \textit{kamma} giving rise to female characteristics and because of \textit{kamma} giving rise to male characteristics, they have the characteristics of both. ‘Had sex’ means: with the male characteristic they act to transgress through sexual intercourse with women. ‘Made others have it’ means: having encouraged another, they cause action in their own female characteristic.

They are twofold: the female \textit{\textsanskrit{ubhatobyañjanaka}} and the male \textit{\textsanskrit{ubhatobyañjanaka}}. In regard to this, the female characteristic of the female \textit{\textsanskrit{ubhatobyañjanaka}} is apparent, but the male characteristic is hidden. The male characteristic of the male \textit{\textsanskrit{ubhatobyañjanaka}} is apparent, but the female characteristic is hidden.

When the female \textit{\textsanskrit{ubhatobyañjanaka}} is acting with manliness among women, the female characteristic is hidden, whereas the male characteristic is apparent. When the male \textit{\textsanskrit{ubhatobyañjanaka}} takes on the state of a woman for the sake of men, the male characteristic is hidden, whereas the female characteristic is apparent. The female \textit{\textsanskrit{ubhatobyañjanaka}} both becomes pregnant and causes others to become pregnant. The male \textit{\textsanskrit{ubhatobyañjanaka}} does not become pregnant, but causes others to become pregnant. This is the difference between them.

But in the Kurundi it is said: ‘If male characteristics should occur in the rebirth-link, then female characteristics appear at rebirth, and if female characteristics should occur in the rebirth-link, then male characteristics appear at rebirth.’ In regard to this, as to the details, it is to be understood according to the \textsanskrit{Atthasālinī}, the commentary on the \textsanskrit{Dhammasaṅgaṇī}. And for this twofold \textit{\textsanskrit{ubhatobyañjanaka}} there is no going forth, nor full ordination. This is to be understood here.\footnote{Sp 3.116: \textit{\textsanskrit{Ubhatobyañjanako} bhikkhaveti \textsanskrit{itthinimittuppādanakammato} ca \textsanskrit{purisanimittuppādanakammato} ca ubhato \textsanskrit{byañjanamassa} \textsanskrit{atthīti} \textsanskrit{ubhatobyañjanako}. \textsanskrit{Karotīti} purisanimittena \textsanskrit{itthīsu} \textsanskrit{methunavītikkamaṃ} karoti. \textsanskrit{Kārāpetīti} paraṃ \textsanskrit{samādapetvā} attano itthinimitte \textsanskrit{kārāpeti}, so duvidho hoti – \textsanskrit{itthiubhatobyañjanako}, \textsanskrit{purisaubhatobyañjanakoti}. Tattha \textsanskrit{itthiubhatobyañjanakassa} itthinimittaṃ \textsanskrit{pākaṭaṃ} hoti, purisanimittaṃ \textsanskrit{paṭicchannaṃ}. \textsanskrit{Purisaubhatobyañjanakassa} purisanimittaṃ \textsanskrit{pākaṭaṃ}, itthinimittaṃ \textsanskrit{paṭicchannaṃ}. \textsanskrit{Itthiubhatobyañjanakassa} \textsanskrit{itthīsu} purisattaṃ karontassa itthinimittaṃ \textsanskrit{paṭicchannaṃ} hoti, purisanimittaṃ \textsanskrit{pākaṭaṃ} hoti. \textsanskrit{Purisaubhatobyañjanakassa} \textsanskrit{purisānaṃ} \textsanskrit{itthibhāvaṃ} upagacchantassa purisanimittaṃ \textsanskrit{paṭicchannaṃ} hoti, itthinimittaṃ \textsanskrit{pākaṭaṃ} hoti. \textsanskrit{Itthiubhatobyañjanako} \textsanskrit{sayañca} gabbhaṃ \textsanskrit{gaṇhāti}, \textsanskrit{parañca} \textsanskrit{gaṇhāpeti}. \textsanskrit{Purisaubhatobyañjanako} pana sayaṃ na \textsanskrit{gaṇhāti}, paraṃ \textsanskrit{gaṇhāpetīti}, idametesaṃ \textsanskrit{nānākaraṇaṃ}. Kurundiyaṃ pana vuttaṃ – “yadi \textsanskrit{paṭisandhiyaṃ} \textsanskrit{purisaliṅgaṃ} pavatte \textsanskrit{itthiliṅgaṃ} nibbattati, yadi \textsanskrit{paṭisandhiyaṃ} \textsanskrit{itthiliṅgaṃ} pavatte \textsanskrit{purisaliṅgaṃ} \textsanskrit{nibbattatī}”ti}\textit{. Tattha \textsanskrit{vicāraṇakkamo} \textsanskrit{vitthārato} \textsanskrit{aṭṭhasāliniyā} \textsanskrit{dhammasaṅgahaṭṭhakathāya} veditabbo. Imassa pana \textsanskrit{duvidhassāpi} \textsanskrit{ubhatobyañjanakassa} neva \textsanskrit{pabbajjā} atthi, na \textsanskrit{upasampadāti} idamidha veditabbaṃ.} }

%
\end{quotation}

To which the sub-commentary adds the following:

\begin{quotation}%
For both \textit{\textsanskrit{ubhatobyañjanakas}}, when lust for a woman arises, then the male characteristic is apparent, whereas the other is hidden, and when lust for a man arises, then the female characteristic is apparent, whereas the other is hidden.\footnote{Vmv 3.116: \textit{Ubhinnampi cesaṃ \textsanskrit{ubhatobyañjanakānaṃ} \textsanskrit{yadā} \textsanskrit{itthiyā} \textsanskrit{rāgo} uppajjati, \textsanskrit{tadā} \textsanskrit{purisabyañjanaṃ} \textsanskrit{pākaṭaṃ} hoti, itaraṃ \textsanskrit{paṭicchannaṃ}. \textsanskrit{Yadā} purise \textsanskrit{rāgo} uppajjati, \textsanskrit{tadā} \textsanskrit{itthibyañjanaṃ} \textsanskrit{pākaṭaṃ} hoti, itaraṃ \textsanskrit{paṭicchannaṃ}}. }

%
\end{quotation}

It seems clear from this explanation that the \textit{\textsanskrit{ubhatobyañjanaka}} is a mythological rather than real person. People such as the ones described above are not found in nature. It is impossible for a single person to both impregnate and be impregnated.\footnote{See Wikipedia on True Hermaphrodism. } Moreover, the idea that the sexual characteristics of the person change dependent on who they lust for is hard to make sense of.

Bhikkhu*\textsanskrit{nī} Vimala argues persuasively that the Buddhist texts inherited the idea of the \textit{\textsanskrit{ubhatobyañjanaka}} (but not the word) from the contemporary Indian culture, especially the Vedas. In this culture the \textit{\textsanskrit{ubhatobyañjanaka}} was mythological rather than real. The Buddhists, as can be seen from the commentarial extracts above, inherited this mythological understanding. For this reason, it is difficult to see how the rule prohibiting \textit{\textsanskrit{ubhatobyañjanakas}} from the full ordination could ever be applied to real candidates.

In sum, I render \textit{\textsanskrit{ubhatobyañjanaka}} as “hermaphrodite”, a term that is now considered archaic in describing humans. I use this term to give an approximation of the meaning, while suggesting that the \textit{\textsanskrit{ubhatobyañjanaka}} does not actually exist.

\subsection*{\textit{Kathina}: “a (robe-making) frame”, “the robe-making ceremony”, “the robe season”}

The \textit{kathina} is a frame that was used for sewing robes at the time of the Buddha. Rules relating to this frame are found in Kd 15, the Khuddakavatthu-kkhandhaka, the Chapter on Minor Topics. I render \textit{kathina} as “frame” only in the context of this chapter.

In other contexts, \textit{kathina} is used largely metaphorically. Rather than rendering \textit{kathina} as “frame” in these instances—since the metaphor would normally be lost on a modern audience—I translate according to the verb it is associated with. When \textit{kathina} is used with \textit{attharati}—literally, “to cover the (\textit{kathina}) frame”—the implied meaning is to do the robe-making ceremony, commonly referred to as the “kathina ceremony”, whereby the robe season is extended. I then translate it as “(to perform/to participate in) the robe-making ceremony”. And when \textit{kathina} is used with \textit{uddharati} or \textit{ubbharati}—literally, “to remove the (\textit{kathina}) frame”—the implied meaning is to end the robe season, and I translate accordingly.

This rendering of \textit{kathina} + \textit{uddharati/ubbharati} needs a bit more explanation. The robe season, properly known as the \textit{\textsanskrit{cīvarakālasamaya}}, begins when the rainy season retreat finishes, and ends either one month later or, if one participates in the robe-making ceremony, at the end of the cold season, four months later. The purpose of the so-called “removal of the \textit{kathina} frame” is in fact to end the robe season. Again, since the literal expression is now largely meaningless, I translate according to the purpose.

Although the robe-making ceremony includes a \textit{\textsanskrit{saṅghakamma}}, a monastic legal procedure, this is no more than the initial step of an extended process. The purpose of the \textit{\textsanskrit{saṅghakamma}} is to formally bestow a cloth given by lay supporters on an individual monk or nun. The next step in the ceremony is the sewing of the robe from that cloth, upon the completion of which the recipient declares that the ceremony has been done. Finally, the \textit{\textsanskrit{saṅghakamma}} participants individually express their appreciation for the properly performed ceremony. This is the process in outline. Because the ceremony has so many steps, it would be inaccurate to render \textit{kathina} + \textit{attharati} as “(to do) the robe-making legal procedure”. A broader expression such as “robe-making ceremony” is required.

There are two additional complications. First, because of the slightly different roles, as seen above, played by those involved in the ceremony, I render the various forms of \textit{kathina} + \textit{attharati} a little differently depending on the context. The person who is chosen to receive the robe “performs the robe-making ceremony”, whereas all the other monastics who take part in the \textit{\textsanskrit{saṅghakamma}} and the expression of appreciation “participate in the robe-making ceremony”. In the \textsanskrit{Parivāra} the performance and the participation are sometimes grouped together, in which case I still use the expression “to participate in the robe-making ceremony”.

Second, \textit{kathina} is also used apart from the three verbs mentioned above (\textit{attharati}, \textit{uddharati}, and \textit{ubbharati}), sometimes compounded with another word, for instance, \textit{\textsanskrit{kathinānujānanā}}, “the allowance for a robe-making ceremony”. The meaning of the word in these instances is generally as if it was used in conjunction with \textit{attharati}. I translate accordingly.

\subsection*{\textit{\textsanskrit{Kilāsa}}: “mild leprosy”}

\textit{\textsanskrit{Kilāsa}} is closely related to \textit{\textsanskrit{kuṭṭha}} (leprosy):

\begin{quotation}%
\textit{\textsanskrit{Kilāsa}} is leprosy without lesions, without discharge, and with the color of red and white lotuses.\footnote{Sp 3.88: \textit{\textsanskrit{Kilāsoti} na \textsanskrit{bhijjanakaṁ} na \textsanskrit{paggharaṇakaṁ} \textsanskrit{padumapuṇḍarīkapattavaṇṇaṁ} \textsanskrit{kuṭṭhaṁ}}. }

%
\end{quotation}

It seems modern medical science distinguishes between \emph{tuberculoid leprosy}, which is mild and has few lesions, and \emph{lepromatous leprosy}, which is severe and has widespread lesions. It seems plausible to identify \textit{\textsanskrit{kilāsa}} with the former and \textit{\textsanskrit{kuṭṭha}} with the latter. But to avoid the technical Latinate medical vocabulary, I render them respectively as “mild leprosy” and “leprosy”.

\subsection*{\textit{\textsanskrit{Kuṭṭha}}: “leprosy”}

See \textit{\textsanskrit{kilāsa}}.

\subsection*{\textit{Kuppa}: “reversible”}

\textit{Kuppa} is a future passive participle derived from the verb \textit{kuppati}, “is disturbed”. The participle means something like “can be disturbed” or “disturbable”. The meaning of the negative participle \textit{akuppa} is well established, since it occurs in a common \textit{sutta} context: \textit{akuppa me vimutti}, often translated as “my liberation is unshakeable”. In this context \textit{akuppa} means that the liberation is firm and irreversible. There seems to be no reason why \textit{akuppa} should not mean the same in the Vinaya \textsanskrit{Piṭaka}, that is, an \textit{akuppa \textsanskrit{saṅghakamma}} is a legal procedure of the Sangha that cannot be disturbed. In other words, it is valid and irreversible.

While the meaning of \textit{akuppa} is straightforward, the same is not true of \textit{kuppa}. Being the antonym of \textit{akuppa}, it would seem that “reversible” would be a suitable rendering. Yet there is a good reason to question this, namely that the general Vinaya context suggests otherwise.

According to the Vinaya, a legal procedure, a \textit{\textsanskrit{saṅghakamma}}, is either valid or invalid, depending on whether a certain number of technical conditions has been fulfilled. If it is valid, the procedure stands and is therefore “irreversible”. If it is invalid, then it is as if the procedure was never done. There is no need to reverse it or undo it, because by definition an invalid procedure is equivalent to a procedure that has not been performed at all. Thus, to translate \textit{kuppa} as “reversible” is misleading, because the invalidity or otherwise is already established. When all the facts of a legal procedure are known and undisputed, there can never be any reversibility.

So perhaps it would be better to render \textit{kuppa} as “invalid”, as suggested in DOP. This would fit perfectly with how the validity of \textit{\textsanskrit{saṅghakamma}} is normally understood and would leave out the awkward “reversible”, which does not fit the broader Vinaya context.

Yet, there are at least two serious problems with this. The first is that there is no evidence that \textit{kuppa}, or the related verb \textit{kuppati}, ever means “invalid”. \textit{Kuppati} means “to disturb” or “to shake”, for which “reversible” is a good fit. “Disturbable” sends a different message from “invalid”. Unless absolutely demanded by the context, we need to be careful not to impose artificial meanings on words.

The second problem is that although the broader Vinaya context does not seem to allow for the meaning “reversible”, in a narrower context this meaning may in fact be quite apt. In the context of \textit{\textsanskrit{saṅghakamma}}, \textit{kuppa} is almost always used together with the synonym \textit{\textsanskrit{aṭṭhān}’\textsanskrit{āraha}}. This latter word means “not worthy of remaining” or “not fit to stand”. Rather than referring to “invalidity”, the implication of this word is that the \textit{\textsanskrit{saṅghakamma}} actually is valid, but that it should be objected to and annulled. Another important synonym for \textit{kuppa} is \textit{adhammika/\textsanskrit{adhammattā}}, “illegitimate”. This word, too, is short of saying the \textit{\textsanskrit{saṅghakamma}} is invalid. Since these are the closest synonyms of \textit{kuppa}, it seems likely that \textit{kuppa} should be understood in a similar way.

Given the above, we must question whether the prevailing and contemporary understanding of \textit{\textsanskrit{saṅghakamma}} is correct. Is it really the case that the validity of \textit{\textsanskrit{saṅghakamma}} depends purely on technical issues, thus making the word “reversible” ill-suited as a translation of \textit{kuppa}? Or could it be that our understanding of \textit{\textsanskrit{saṅghakamma}} is a product of evolution and that in the earliest stages there actually was such a thing as the reversing of the Sangha’s legal procedures?

The idea that a \textit{\textsanskrit{saṅghakamma}} that has been improperly performed is inherently invalid is quite likely rooted in the word \textit{akamma}, which is sometimes used in the canonical Vinaya to qualify an improperly performed \textit{\textsanskrit{saṅghakamma}}. \textit{Akamma} means “non-action”, implying that the \textit{\textsanskrit{saṅghakamma}} has not actually been performed at all. Although the word is found a number of times in the Vinaya, it is restricted to the highly technical expositions found in Kd 9 and in Kd 12.\footnote{\href{https://suttacentral.net/pli-tv-kd9/en/brahmali\#2.3.1}{Kd~9:2.3.1}–3.2.12 and \href{https://suttacentral.net/pli-tv-kd12/en/brahmali\#1.4.28}{Kd~12:1.4.28}. } The more technical and impersonal the exposition, the more likely it is to be relatively late.\footnote{This accords with the common-sense principle that detailed and complex ideas emerge from simpler ones. } Moreover, \textit{kuppa} and \textit{\textsanskrit{aṭṭhān}’\textsanskrit{āraha}} are only found twice in these chapters. Apart from this, it seems that it is only in the commentaries that this technical and absolute distinction between a valid and an invalid \textit{\textsanskrit{saṅghakamma}} is established once and for all.

It seems reasonable to suppose, then, that the Vinaya \textsanskrit{Piṭaka} may have gone through an evolution in how the validity of \textit{\textsanskrit{saṅghakamma}} was understood. I am suggesting that in the earliest stages a legal procedure of the Sangha needed to be successfully challenged before it was considered invalid. The implication of this is that even if a \textit{\textsanskrit{saṅghakamma}} has been improperly performed it stands as valid if no-one is aware of the problem or no-one brings attention to it.

This view is supported by a passage in Kd 2 that seems to say that perception is a factor in deciding whether the \textit{uposatha} procedure has been performed correctly or not. Here it is:

\begin{quotation}%
On the observance day, four or more resident monks may have gathered together in a certain monastery. They don’t know there are other resident monks who haven’t arrived. Perceiving that they’re acting according to the Teaching and the Monastic Law, perceiving that the assembly is complete although it’s not, they do the observance-day ceremony and recite the Monastic Code. When they’ve just finished, and the entire gathering has left, a smaller number of resident monks arrive. In such a case, what has been recited is valid, and the late arrivals should announce their purity in the presence of the others. There’s no offense for the reciters. (\href{https://suttacentral.net/pli-tv-kd2/en/brahmali\#28.7.15}{Kd~2:28.7.15}–28.7.21)

%
\end{quotation}

Note the word “valid”—here \textit{\textsanskrit{sūddiṭṭhaṁ}}, literally, “well-recited”—in the second last sentence. This passage is clear that an improperly performed \textit{\textsanskrit{saṅghakamma}}—in this case the \textit{uposatha} procedure\footnote{The \textit{uposatha} was arguably not regarded as a \textit{\textsanskrit{saṅghakamma}} in the earliest period, for which see discussion in the introduction to volume 4 of this series. Yet the rules that govern the validity of the assembly are the same for the \textit{uposatha} ceremony as for \textit{\textsanskrit{saṅghakamma}}. }—is valid so long as that is one’s perception. The fact that it was technically invalid, because not all the monks were present, is irrelevant.

This understanding of \textit{\textsanskrit{saṅghakamma}} has several benefits. One of these is that the obsession with purity of lineage, which is common in Theravada countries, is misplaced. It does not matter if an ordination procedure—taking one of the most obvious examples of a \textit{\textsanskrit{saṅghakamma}}—is wrongly performed. As long as the procedure is not challenged, it remains valid. According to this argument, the common practice among monastics of doubting the validity of one’s own or another’s ordination is misguided. Once again, we see evidence that the Buddha was more pragmatic than later generations of Buddhists.

So, despite my initial misgivings, I choose to translate \textit{kuppa} as “reversible” after all.

\subsection*{\textit{\textsanskrit{Koṭṭhaka}}: “gatehouse”}

I. B. Horner renders \textit{\textsanskrit{koṭṭhaka}} as “store-room”, which is not specific enough.

In Kd 15, which among other things discusses allowable buildings and other structures, we find several allowances for encircling walls (\textit{\textsanskrit{pākāra}}), especially in cases where privacy was important, such as with saunas, places for bathing, or restrooms.\footnote{See for instance \href{https://suttacentral.net/pli-tv-kd15/en/brahmali\#14.3.44}{Kd~15:14.3.44}. } In many of the cases where the Buddha gives an allowance for such a wall, he afterwards gives an allowance for a \textit{\textsanskrit{koṭṭhaka}}, which in the context means an access point through the encircling wall. But since the word for “gateway” is \textit{\textsanskrit{dvāra}}, it is not immediately obvious what \textit{\textsanskrit{koṭṭhaka}} refers to. Still, we find the compound \textit{\textsanskrit{dvārakoṭṭhaka}}, “the gateway \textit{\textsanskrit{koṭṭhaka}}” in a number of places, which shows that although the \textit{\textsanskrit{koṭṭhaka}} was not the gateway itself, it was closely associated with it.

In fact, we know from Kd 15 that \textit{\textsanskrit{koṭṭhakas}} were rooms or houses, because it is said that grass and dust fell into them (\href{https://suttacentral.net/pli-tv-kd15/en/brahmali\#14.4.15}{Kd~15:14.4.15}). This would make no sense if it was merely an open area with a gate. Indeed, all instances where grass and dust fall into something, it refers to an enclosed space, usually a building of some kind. It is then said that it may be plastered inside and outside, again suggesting an enclosed space. We also know that this “gatehouse” would have often been small. For instance, we see all sorts of buildings, including saunas, having encircling walls and therefore also a gatehouse. But since the gatehouse is merely a subsidiary building to the main building, and the main building in this case is in itself quite humble, we can infer that the gatehouse would in all likelihood have been tiny.

What was the \textit{\textsanskrit{koṭṭhaka}} used for? In the Suttas we find references to a \textit{\textsanskrit{dovārika}}, a “doorkeeper” or “gatekeeper”. They would inform the householders of any guest arriving at the gateway/gatehouse (\textit{\textsanskrit{bahidvārakoṭṭhaka}}). It seems reasonable, then, to think the \textit{\textsanskrit{koṭṭhaka}} was a room or house for the gatekeeper, where he either lived or stayed during his working hours. In many instances, however, the \textit{\textsanskrit{koṭṭhaka}} would have been no more than a storeroom, that is, a place to put aside shoes, sunshades, or other equipment one did not wish to bring into the main building. This might have been the function of the \textit{\textsanskrit{koṭṭhakas}} built in connection with saunas, for instance.

\subsection*{\textit{\textsanskrit{Khādanīya}}: “fresh food”}

See \textit{\textsanskrit{bhojanīya}.}

\subsection*{\textit{\textsanskrit{Gaṇḍa}}: “abscess”}

The basic meaning of \textit{\textsanskrit{gaṇḍa}} is “swelling”, but only occasionally does it appear in the Canonical texts with this meaning, as in the case of stuffing one’s cheeks while eating:

\begin{quotation}%
“If a monk, out of disrespect, eats making a swelling on one or both sides, he commits an offense of wrong conduct.” (\href{https://suttacentral.net/pli-tv-bu-vb-sk46/en/brahmali\#1.5}{Sk~46})

%
\end{quotation}

Normally \textit{\textsanskrit{gaṇḍa}} is more specific than the generic term “swelling”. To start with, \textit{\textsanskrit{gaṇḍa}} is one of the five common diseases found in ancient India:

\begin{quotation}%
At that time a monk had the \textit{\textsanskrit{gaṇḍa}} illness. (\href{https://suttacentral.net/pli-tv-kd6/en/brahmali\#14.4.13}{Kd~6:14.4.13})

%
\end{quotation}

And:

\begin{quotation}%
At that time in Magadha, there were five common diseases: leprosy, \textit{\textsanskrit{gaṇḍa}}, mild leprosy, tuberculosis, and epilepsy. (\href{https://suttacentral.net/pli-tv-kd1/en/brahmali\#39.1.1}{Kd~1:39.1.1})

%
\end{quotation}

If you have a \textit{\textsanskrit{gaṇḍa}}, you should not be ordained. Since a swelling is not always an illness—certainly not a serious one that would bar you from ordination—it must have a narrower meaning in the context of illness, which is what we are considering here.

One common rendering of \textit{\textsanskrit{gaṇḍa}} is “boil”. A boil, however, is usually a superficial skin disease that often does not need much treatment, if any. A \textit{\textsanskrit{gaṇḍa}}, on the other hand, was often treated, even at the time of the Buddha:

\begin{quotation}%
“If a nun … has a \textit{\textsanskrit{gaṇḍa}} or a wound … ruptured, or split open, washed, anointed, bandaged, or unwrapped …” (\href{https://suttacentral.net/pli-tv-bi-vb-pc60/en/brahmali\#1.16.1}{Bi~Pc~60:1.16.1})

%
\end{quotation}

Here a \textit{\textsanskrit{gaṇḍa}} appears to be a skin disease that requires a significant amount of attention. “A boil” does not seem quite right, whereas “an abscess” or perhaps “a cyst” seems appropriate.

Another passage that sheds light on the meaning on this word is as follows:

\begin{quotation}%
“Monks, suppose there was a \textit{\textsanskrit{gaṇḍa}} that was many years old. And that \textit{\textsanskrit{gaṇḍa}} had nine continually open wounds. Whatever oozed out of them would be filthy, stinking, and disgusting.” (\href{https://suttacentral.net/an9.15/en/sujato\#1.1}{AN~9.15})

%
\end{quotation}

This passage suggests that “ulcer” may be the appropriate rendering.

It may be that \textit{\textsanskrit{gaṇḍa}} does not have an exact equivalent in modern medical terminology. It seems, in fact, that it covers a number of related skin diseases, especially abscesses and ulcers. Given that the clearest description of \textit{\textsanskrit{gaṇḍa}} in the Vinaya concerns its rupturing and splitting open, I have settled on “abscess” as the closest equivalent.

\subsection*{\textit{\textsanskrit{Gāma}}: “inhabited area”}

\textit{\textsanskrit{Gāma}} is usually rendered as “village”. In the Vinaya \textsanskrit{Piṭaka}, however, this is often too narrow. For instance, we often find \textit{\textsanskrit{gāma}} used in contrast to \textit{\textsanskrit{arañña}}, “wilderness”. In these cases, it clearly means any kind of inhabited area, not just the more restrictive sense of “village”. Moreover, according to the word commentary,\footnote{At \href{https://suttacentral.net/pli-tv-bu-vb-pj2/en/brahmali\#3.6}{Bu~Pj~2:3.6}. } even a single house or a caravan stationary for an extended period can be regarded as a \textit{\textsanskrit{gāma}}. Again, “inhabited area” is a better fit than “village”. Unless the context or the narrative flow requires otherwise, I have given preference to the more general and technical meaning of “inhabited area”.

\subsection*{\textit{Gihigata}: “married girl”}

Occasionally \textit{gihigata} is a generic term for “householder” (\href{https://suttacentral.net/pli-tv-kd21/en/brahmali\#1.9.22}{Kd~21:1.9.22}). Normally, however, it refers to a female, presumably a female householder. The definition from the Vinaya \textsanskrit{Piṭaka} is as follows:

\begin{quotation}%
A woman who has gone to the place of a man is called a \textit{gihigata}. \href{https://suttacentral.net/pli-tv-bi-vb-pc65/en/brahmali\#2.1.7}{Bi~Pc~65:2.1.7}

%
\end{quotation}

The commentaries elaborate as follows:

\begin{quotation}%
\textit{Gihigata}: who has gone to the place of a man; the meaning is “one who has attained cohabitation with a man”.\footnote{Vin-vn-\textsanskrit{ṭ} 2365: \textit{Gihigatanti \textsanskrit{purisantaragataṁ}, \textsanskrit{purisasamāgamappattanti} attho}. }

%
\end{quotation}

\begin{quotation}%
\textit{\textsanskrit{Gihigatāya}}: here a woman who has gone to the place of a man is called \textit{gihigata}. She is called a \textit{gihigata} because, on account of sex, she has gone to a householder considered a man or he has gone to her.\footnote{Sp‑yoj 2.1119: \textit{\textsanskrit{Gihigatāyāti} ettha \textsanskrit{gihigatā} \textsanskrit{nāma} \textsanskrit{purisantaragatā} vuccati. \textsanskrit{Sā} hi \textsanskrit{yasmā} \textsanskrit{purisasaṅkhātena} \textsanskrit{gihinā} gamiyittha, \textsanskrit{ajjhācāravasena}, \textsanskrit{gihiṁ} \textsanskrit{vā} gamittha, \textsanskrit{tasmā} \textsanskrit{gihigatāti} vuccati}. }

%
\end{quotation}

Although the above would clearly include marriage, it is impossible to know whether it is restricted to this. Less formal kinds of relationship might also be implied. Given the conservative nature of ancient Indian society, however, it seems reasonable to think that it refers, in the main, to marriage.

The evidence from the Vinayas of other schools also suggest that a \textit{gihigata} is a married girl. I have received the following information from Ven. \textsanskrit{Vimalañāṇī} (private communication):

\begin{itemize}%
\item \textsanskrit{Mahāsāṅghika} \textit{\textsanskrit{pācittiya}} 100: \langlzh{適他婦}, “wife, gone to another”.\\
Word definition: \langlzh{適他婦者壞梵行}, “\textit{\textsanskrit{Gihigatā}}: means she has broken \textit{brahmacariya}.”%
\item \textsanskrit{Lokuttaravāda} \textit{\textsanskrit{pācittiya}} 100: \textit{\textsanskrit{yā} puna \textsanskrit{bhikṣuṇī} \textsanskrit{ūna}-\textsanskrit{dvādaśa}-\textsanskrit{varṣāṁ} \textsanskrit{gṛhi}-\textsanskrit{caritām} \textsanskrit{upasthāpayet} \textsanskrit{pācattikaṁ}}, “If a nun ordains someone less than twelve years old who has gone to a man, she incurs a \textit{\textsanskrit{pācittiya}} offense.”\\
Word definition: \textit{\textsanskrit{gṛhi}-\textsanskrit{caritā} ti vikopita-\textsanskrit{brahmacaryām}}, “Who has gone to a man: who has destroyed celibacy.”%
\item Dharmaguptaka \textsanskrit{pācittiya} 125: \langlzh{嫁女人}, “A married woman.”%
\item \textsanskrit{Mahīśāsaka} \textit{\textsanskrit{pācittiya}} 104: \langlzh{嫁女}, “A married woman.”\\
Word definition: \langlzh{嫁者已經男子}, “Married means already subject to a man.”%
\item \textsanskrit{Sarvāstivāda} \textit{\textsanskrit{pācittiya}} 108: \langlzh{嫁女}, “A married woman.”%
\item \textsanskrit{Mūlasarvāstivāda} \textsanskrit{pācittiya} 108: \langlzh{曾嫁女人}, “A previously married woman.”\\
Word definition: \langlzh{曾嫁女者,謂曾適他氏}, “A previously married woman: previously gone to another family.”%
\end{itemize}

\subsection*{\textit{\textsanskrit{Guhā}}: “cave”}

The term \textit{\textsanskrit{guhā}} does not have a precise equivalent in English. The general meaning seems to be “enclosed place” or “hiding place”, derived from the root \textit{guh}, meaning “to cover” or “to hide”. In the Vinaya \textsanskrit{Piṭaka} such places are either naturally occurring, as in \textit{\textsanskrit{giriguhā}} (\href{https://suttacentral.net/pli-tv-kd16/en/brahmali\#1.1.4}{Kd~16:1.1.4}), or man-made (\href{https://suttacentral.net/pli-tv-kd3/en/brahmali\#5.6.4}{Kd~3:5.6.4}). In all cases a \textit{\textsanskrit{guhā}} was used to shelter or cover something, for instance, to shelter food from rain (\href{https://suttacentral.net/pli-tv-kd6/en/brahmali\#33.2.1}{Kd~6:33.2.1}), to store robes away from destruction by insects (\href{https://suttacentral.net/pli-tv-kd8/en/brahmali\#7.1.3}{Kd~8:7.1.3}), and to shelter monastics while they do the \textit{uposatha} procedure (\href{https://suttacentral.net/pli-tv-kd2/en/brahmali\#8.1.7}{Kd~2:8.1.7}).

The commentaries add the following:

\begin{quotation}%
\textit{\textsanskrit{Guhā}}: a \textit{\textsanskrit{guhā}} made of bricks, a \textit{\textsanskrit{guhā}} made of rocks, a \textit{\textsanskrit{guhā}} made of wood, a \textit{\textsanskrit{guhā}} made of dirt.\footnote{Sp 4.294: \textit{\textsanskrit{Guhāti} \textsanskrit{iṭṭhakāguhā} \textsanskrit{silāguhā} \textsanskrit{dāruguhā} \textsanskrit{paṁsuguhā}}. }

%
\end{quotation}

\begin{quotation}%
\textit{Giriguha}: in the middle of two hills or a great opening like a tunnel in just one (of them).\footnote{Sp‑\textsanskrit{ṭ} 4.294: \textit{\textsanskrit{Giriguhā} \textsanskrit{nāma} \textsanskrit{dvinnaṁ} \textsanskrit{pabbatānaṁ} \textsanskrit{antarā}, \textsanskrit{ekasmiṁyeva} \textsanskrit{vā} \textsanskrit{umaṅgasadisaṁ} \textsanskrit{mahāvivaraṁ}.} }

%
\end{quotation}

In sum, it is likely that “cave” does not properly capture all the nuances of \textit{\textsanskrit{guhā}}, but in the absence of a more fitting term, it is perhaps satisfactory.

\subsection*{\textit{\textsanskrit{Cīvara}}: “robe”, “robe-cloth”}

This word has two fairly distinct meanings: “a (finished) robe” and “robe-cloth”. Often the context makes it clear which one is meant, but sometimes it does not. We need a rule to decide how to translate in ambiguous cases.

I will argue here that the meaning “robe” is primary, whereas “robe-cloth” is secondary. This follows from several different lines of reasoning. First, the meaning “robe” is by far the most common one. The word \textit{\textsanskrit{cīvara}} is found approximately 630 times in the main four \textsanskrit{Nikāyas}. A simple search and perusal of the material is enough to make it clear that the vast majority of these refer to “robe”. There are a few indeterminate uses of \textit{\textsanskrit{cīvara}}, but as far as I am aware none that unambiguously means “robe-cloth”. Second, and connected to the previous reason, \textit{\textsanskrit{cīvara}} is the normal word for a “(monastic) robe” in Pali literature. For a monastic, the concept of a “robe” is far more significant than the idea of “robe-cloth”. Because of the importance of the word \textit{\textsanskrit{cīvara}} in the meaning of “robe” and its frequent usage in this meaning, it is natural to read it as “robe”, unless “robe-cloth” is specifically required by the context. Third, “robe-cloth” is a derived meaning that only makes sense in the context of robes, whereas the reverse is not true. It follows that the meaning “robe” is more fundamental and therefore more likely to be intended in ambiguous cases. Fourth, in the Sutta-\textsanskrit{vibhaṅga} we often find \textit{\textsanskrit{cīvara}} used as “robe” in the origin stories. It is then natural to read \textit{\textsanskrit{cīvara}} as “robe” also in the rule. Only in the word analysis is \textit{\textsanskrit{cīvara}} then interpreted as “robe-cloth”. This pattern is found in \href{https://suttacentral.net/pli-tv-bu-vb-np1/en/brahmali\#3.1.9}{Bu~NP~1}, \href{https://suttacentral.net/pli-tv-bu-vb-np5/en/brahmali\#3.1.9}{5}, \href{https://suttacentral.net/pli-tv-bu-vb-np6/en/brahmali\#3.1.11}{6}, \href{https://suttacentral.net/pli-tv-bu-vb-np25/en/brahmali\#2.9}{25}, and \href{https://suttacentral.net/pli-tv-bu-vb-np27/en/brahmali\#2.11}{27}, and in \href{https://suttacentral.net/pli-tv-bu-vb-pc59/en/brahmali\#2.1.17}{Bu~Pc~59} and \href{https://suttacentral.net/pli-tv-bu-vb-pc60/en/brahmali\#2.1.9}{60}. This suggests that “robe-cloth” in many cases is a secondary derivation that emerged as the evolving Vinaya strove for clear definitions.

Based on the above considerations, my default translation of \textit{\textsanskrit{cīvara}} is “robe”. I render it as “robe-cloth” only when required by the context, of which there are three in particular.

The first is where \textit{\textsanskrit{cīvara}} is specifically interpreted as any cloth that is larger than the smallest cloth that can be assigned to another (\textit{vikappita}), meaning larger than 8 by 4 standard fingerbreadths. In these cases, while not excluding the meaning “robe”, \textit{\textsanskrit{cīvara}} clearly includes “robe-cloth”. In any rule where this definition is found, I render all subsequent occurrences of \textit{\textsanskrit{cīvara}} in that rule as “robe-cloth”.

The second context is the \textit{kathina} ceremony and the robe season, which are closely linked. This was a time when the Sangha collected cloth for distribution to its members. In a number of instances, it is stated that a monk did not have enough cloth for a robe and was awaiting further gifts. The implication is that the cloth he had was robe-cloth, not a complete robe. So, in the context of the robe season, I always render \textit{\textsanskrit{cīvara}} as “robe-cloth”.

The third context is where lay people give \textit{\textsanskrit{cīvara}} to the monastics. Going by the description in the Chapter on Robes, lay people would normally (but not always, it seems) give cloth that the monastics would then make into finished robes (\href{https://suttacentral.net/pli-tv-kd8/en/brahmali\#9.2.4}{Kd~8:9.2.4}). So, unless the context requires otherwise, whenever a lay person is giving \textit{\textsanskrit{cīvara}} to monastics, including the time it is stored in a monastery, I render it as “robe-cloth”.

Finally, the texts also distinguish between the singular \textit{\textsanskrit{bahuṁ} \textsanskrit{cīvaraṁ}} and the plural (\textit{\textsanskrit{bahūni}}) \textit{\textsanskrit{cīvarāni}}, where the former most naturally refers to “much robe-cloth”, whereas the latter usually refers to “many robes”.

\subsection*{\textit{\textsanskrit{Cuṇṇa}}: “bath powder”}

See \textit{mattika}.

\subsection*{\textit{Codeti}: “to accuse”, “to prompt”, “to confront”}

The semantic range of \textit{codeti} is similar to that of “to admonish”. In particular, it covers both the meaning of “to urge” and “to reprimand”. In \href{https://suttacentral.net/pli-tv-bu-vb-np10/en/brahmali\#1.3.13}{Bu~NP~10} \textit{codeti} means to urge or prompt someone to offer a robe, that is, a robe that has already been promised. In one passage in the Great Chapter, the lay people \textit{codeti} the monks, here meaning “to confront” them or “to reprimand” them (\href{https://suttacentral.net/pli-tv-kd1/en/brahmali\#24.5.5}{Kd~1:24.5.5}). In both of these instances “admonish” would work well. In most other contexts \textit{codeti} means “to confront” or “to accuse” someone of a wrongdoing. In these cases, too, “admonish” would fit. However, since “admonish” is perhaps a slightly old-fashioned and increasingly uncommon word, I have instead rendered \textit{codeti} according to context, using “to prompt” at \href{https://suttacentral.net/pli-tv-bu-vb-np10),%20%E2%80%9Cto%20confront%E2%80%9D%20at%20[Kd%C2%A01/en/brahmali\#24.5.5](pli-tv-kd1:24.5.5}{Bu~NP~10}, and “to accuse” elsewhere.

\subsection*{\textit{\textsanskrit{Chakaṇa}}: “detergent”}

See \textit{mattika}.

\subsection*{\textit{\textsanskrit{Jantāghara}}: “sauna”}

The Buddha’s personal physician, \textsanskrit{Jīvaka}, recommended a \textit{\textsanskrit{janṭāghara}} for the Sangha to keep the monastics healthy (\href{https://suttacentral.net/pli-tv-kd15/en/brahmali\#14.1.8}{Kd~15:14.1.8}). The \textit{\textsanskrit{jantāghara}} was a room with a fireplace and a flue (\href{https://suttacentral.net/pli-tv-kd15/en/brahmali\#14.3.16}{Kd~15:14.3.16}). It was used to make one sweat (\href{https://suttacentral.net/pli-tv-kd15/en/brahmali\#14.3.31}{Kd~15:14.3.31}) and was also a place for cleaning oneself (\href{https://suttacentral.net/pli-tv-kd1/en/brahmali\#25.12.4}{Kd~1:25.12.4}). Moreover, the verb for what one does in a \textit{\textsanskrit{jantāghara}} is “bathe”, \textit{\textsanskrit{nahāyati}} (\href{https://suttacentral.net/pli-tv-kd20/en/brahmali\#27.4.15}{Kd~20:27.4.15}). All in all it comes close to what we now call a sauna.

\subsection*{\textit{\textsanskrit{Jātarūparajata}}: “gold, silver, and money”}

\textit{\textsanskrit{Jātarūparajata}} literally means “gold and silver”. Yet, as can be seen from other passages, this includes gold and silver coins, that is, money. For instance, at \textit{bhikkhu nissaggiya \textsanskrit{pācittiya}} 18 \textit{rajata} is defined as money:

\begin{quotation}%
Silver: a \textit{\textsanskrit{kahāpaṇa}} coin, a copper \textit{\textsanskrit{māsaka}} coin, a wooden \textit{\textsanskrit{māsaka}} coin, a resin \textit{\textsanskrit{māsaka}} coin—whatever is used in commerce. (\href{https://suttacentral.net/pli-tv-bu-vb-np18/en/brahmali\#2.7}{Bu~NP~18:2.7})

%
\end{quotation}

Similarly, the context in Kd 22 makes it clear that \textit{\textsanskrit{jātarūparajata}} refers to money:

\begin{quotation}%
“Don’t give a \textit{\textsanskrit{kahāpaṇa}} to the Sangha, or half a \textit{\textsanskrit{kahāpaṇa}}, or a \textit{\textsanskrit{pāda}}, or a \textit{\textsanskrit{māsaka}}. Gold, silver, and money aren’t allowable for the Sakyan monastics.” (\href{https://suttacentral.net/pli-tv-kd22/en/brahmali\#1.1.9}{Kd~22:1.19})

%
\end{quotation}

See also separate entry for \textit{\textsanskrit{hirañña}}, where I make the case that this word means “gold coins”.

\subsection*{\textit{\textsanskrit{Dūseti}}/\textit{\textsanskrit{dūsaka}}: “to rape”/“rapist”}

The basic meaning of this word—the two words in the heading being respectively the verbal form and the noun form of the same root—is “to spoil” something or someone. One way “to spoil” a person is through sex, especially forced sex. This is what one of the sub-commentaries has to say:

\begin{quotation}%
He is called a \textit{\textsanskrit{dūsaka}} of a nun: having had sexual intercourse with an ordinary \textit{\textsanskrit{bhikkhunī}}, because of “spoiling” her, he has “spoiled” a \textit{\textsanskrit{bhikkhunī}}.\footnote{Vin-vn-\textsanskrit{ṭ} 322: \textit{\textsanskrit{Dūsakoti} \textsanskrit{pakatattāya} \textsanskrit{bhikkhuniyā} \textsanskrit{methunaṁ} \textsanskrit{paṭisevitvā} \textsanskrit{tassā} \textsanskrit{dūsitattā} \textsanskrit{bhikkhuniṁ} \textsanskrit{dūsetīti} \textsanskrit{bhikkhunidūsakoti} vutto ca}. }

%
\end{quotation}

Assuming that the nun did not consent, the meaning of \textit{\textsanskrit{dūseti}} must be “to rape”. This meaning is even more explicit in several Canonical passages, for instance:

\begin{quotation}%
A monk rapes a sleeping monk: if he wakes up and consents, both should be expelled. If he wakes up but does not consent, then the “spoiler” should be expelled. (\href{https://suttacentral.net/pli-tv-bu-vb-pj1/en/brahmali\#9.7.11}{Bu~Pj~1:9.7.11})

%
\end{quotation}

Since the action results in expulsion, it must refer to sexual intercourse, in which case the lack of consent means that “rapist” is the appropriate translation of \textit{\textsanskrit{dūsaka}}. The same is true for the story of the rape of the nun \textsanskrit{Uppalavaṇṇā}, told in the same rule. Here, too, \textit{\textsanskrit{dūseti}} must refer to rape, since she is “grabbed hold of” and the potential offense for the nun is a \textit{\textsanskrit{pārājika}}. Also, in the origin story to Bu Ss 8, where Dabba the Mallian is accused of having \textit{\textsanskrit{dūsita}} a \textit{\textsanskrit{bhikkhunī}}, the most likely meaning is “rape” (\href{https://suttacentral.net/pli-tv-bu-vb-ss8/en/brahmali\#1.8.19}{Bu~Ss~8:1.8.19}). The context makes it clear that it refers to sexual intercourse, and the nature of the accusation suggests rape. There are also a number of cases where various kinds of people are said to \textit{\textsanskrit{dūseti}} women who are alone (see \href{https://suttacentral.net/pli-tv-bu-vb-pc27/en/brahmali\#2.8}{Bu~Pc~27:2.8} and \href{https://suttacentral.net/pli-tv-bu-vb-pc28/en/brahmali\#2.9}{Bu Pc 28:2.9}, \href{https://suttacentral.net/pli-tv-bu-vb-pd4/en/brahmali\#1.8}{Bu~Pd~4:1.8}, \href{https://suttacentral.net/pli-tv-bi-vb-ss3/en/brahmali\#2.7}{Bi~Ss~3:2.7}, \href{https://suttacentral.net/pli-tv-bi-vb-ss3/en/brahmali\#4.3}{Bi~Ss~3:4.3}, \href{https://suttacentral.net/pli-tv-bi-vb-pc37/en/brahmali\#1.3}{Bi~Pc~37:1.3} and \href{https://suttacentral.net/pli-tv-bi-vb-pc38/en/brahmali\#1.3}{38:1.3}, and two instances in the Khandhakas at \href{https://suttacentral.net/pli-tv-kd20/en/brahmali\#23.1.1}{Kd~20:23.1.1} and \href{https://suttacentral.net/pli-tv-kd20/en/brahmali\#27.4.23}{Kd~20:27.4.23}). The most obvious meaning of the word in these cases is “rape”. From these examples it seems the dominant meaning of \textit{\textsanskrit{dūseti}} in the Canonical Vinaya is “rape”.

Still, there are several contexts where the word “rape” is too strong. At \href{https://suttacentral.net/pli-tv-bi-vb-pj5/en/brahmali\#1.14}{Bi~Pj~5:1.14} a layman is said to desire to \textit{\textsanskrit{dūseti}} a \textit{\textsanskrit{bhikkhunī}}. Eventually, when they find themselves in a private place, they touch each other lustfully. A similar situation is found in the origin story at \href{https://suttacentral.net/pli-tv-bu-vb-ss2/en/brahmali\#1.1.29}{Bu~Ss~2:1.1.29}. In the former case “to be intimate with” seems an appropriate rendering, whereas in the latter “to molest” is more to the point. In Kd 1 we find a few cases where \textit{\textsanskrit{dūseti}} seems to refer to consensual sex, such as the case of two novices who \textit{\textsanskrit{dūseti}} each other (\href{https://suttacentral.net/pli-tv-kd1/en/brahmali\#52.1.3}{Kd~1:52.1.3}).

In still other contexts the meaning of \textit{\textsanskrit{dūseti}} is not related to sexuality. In these cases, the usual meaning is “to spoil”. For instance, in one case the faith of the laypeople is said to be \textit{\textsanskrit{dūseti}}, “spoiled” or “corrupted” (\href{https://suttacentral.net/pli-tv-bu-vb-ss13/en/brahmali\#2.8}{Bu~Ss~13:2.8}). In other contexts, fields are said to be \textit{\textsanskrit{dūseti}}, “spoiled”, when someone stands in them or someone discards waste there (\href{https://suttacentral.net/pli-tv-bu-vb-pc19/en/brahmali\#1.5}{Bu~Pc~19:1.5} and \href{https://suttacentral.net/pli-tv-bi-vb-pc9/en/brahmali\#1.5}{Bi~Pc~9:1.5}). In the Khandhakas the word of the Buddha is said to be \textit{\textsanskrit{dūseti}}, “spoiled” or “corrupted”, by those who teach the Dhamma using their own expressions (\href{https://suttacentral.net/pli-tv-kd15/en/brahmali\#33.1.5}{Kd~15:33.1.5}).

\subsection*{\textit{Dhanu}: “bow-length”}

See \textit{sugata}.

\subsection*{\textit{Dhammavinaya}: “spiritual path”}

\textit{Dhammavinaya} is the compounded form of \textit{Dhamma} and \textit{vinaya}. In this context \textit{Dhamma} must mean “teaching”, whether that of the Buddha or some other teaching, whereas \textit{vinaya} seems to mean “training”, see the separate entry for \textit{vinaya}. \textit{Dhammavinaya} is often used to refer to the Buddha’s system of teaching and training, thus referring to his religion or spiritual path. It is close in meaning to \textit{\textsanskrit{sāsana}} (“(spiritual) instruction” or “Buddhism”) and \textit{brahmacariya} (“spiritual life”). It is interesting, however, that in the Chapter on Nuns we find \textit{dhammavinaya} apparently used to refer to all religions or spiritual paths that existed in ancient India (\href{https://suttacentral.net/pli-tv-kd20/en/brahmali\#1.6.8}{Kd~20:1.6.8}). The same is true of the \textsanskrit{Mahāparinibbāna} Sutta, where the Buddha explains to the wanderer Subhadda that awakening can be achieved in whatever \textit{dhammavinaya} there is the noble eightfold path (\href{https://suttacentral.net/dn16/en/sujato\#5.27.2}{DN~16:5.27.2}). And so, in the broadest sense \textit{dhammavinaya} is a reference to any “spiritual path”, and I render it as such.

\subsection*{\textit{\textsanskrit{Nānāsaṁvāsa}/\textsanskrit{nānāsaṁvāsako}: “}who belongs to a different community” "one who belongs to a different Buddhist sect”}

\textit{\textsanskrit{Nānāsaṁvāsaka}} (and \textit{\textsanskrit{samānasaṁvāsaka}}) need to be carefully distinguished from \textit{\textsanskrit{nānāsaṁvāsa}} (and \textit{\textsanskrit{samānasaṁvāsa}}). Only the former means “one belonging to a different Buddhist sect”. The latter means “belonging to a different community”, as decided by \textit{\textsanskrit{sīmās}}, “monastic zones”.

\subsection*{\textit{\textsanskrit{Nāseti}}: “to expel”}

I render \textit{\textsanskrit{pārājika}} as “offense entailing expulsion” and \textit{\textsanskrit{nāseti}} as “expulsion”. I use the same rendering for the two words because they are closely related in the Vinaya \textsanskrit{Piṭaka}. At Bu Pj 1, we find the following use of \textit{\textsanskrit{nāseti}}:

\begin{quotation}%
A monk rapes a sleeping monk: if he wakes up and consents, both should be expelled. (\href{https://suttacentral.net/pli-tv-bu-vb-pj1/en/brahmali\#9.7.11}{Bu~Pj~1:9.7.11})

%
\end{quotation}

Both monks have committed a \textit{\textsanskrit{pārājika}} offense, and the consequence is that they are \textit{\textsanskrit{nāseti}}, which here, then, must mean “expelled”. Since there is no verbal form of \textit{\textsanskrit{pārājika}}, it seems \textit{\textsanskrit{nāseti}} takes on that function. Moreover, \textit{\textsanskrit{nāseti}} is a less technical term than \textit{\textsanskrit{pārājika}}, which only applies to fully ordained monastics. \textit{\textsanskrit{Nāseti}}, on the other hand, is used more broadly, including for novices:

\begin{quotation}%
A novice rapes a sleeping novice: if he wakes up and consents, both should be expelled. (\href{https://suttacentral.net/pli-tv-bu-vb-pj1/en/brahmali\#9.7.20}{Bu~Pj~1:9.7.20})

%
\end{quotation}

\subsection*{\textit{Niyassa}: “demotion”}

The literal meaning of \textit{niyassa}, \textit{ni + yasa}, is something like “disrepute”. The only use of this word in the Vinaya, however, is in connection with the eponymous legal procedure, the \textit{niyassakamma}. The main characteristic that distinguishes this procedure from other \textit{\textsanskrit{daṇḍakammas}}, “punishment procedures”, is that the person against whom it is done must live with formal support, \textit{nissaya}. In effect they are treated like a junior monastic. For this reason “demotion” seems like an appropriate rendering.

\subsection*{\textit{\textsanskrit{Nisīdana}}: “sitting mat”}

\textit{\textsanskrit{Nisīdana}} is usually rendered as “sitting cloth”. This translation seems to be a result of the current practice of using the \textit{\textsanskrit{nisīdana}} wherever one sits down, especially in public. For most monastics, this means it is used almost exclusively indoors, where a sitting cloth is not really required. In effect, it ends up having more of a ceremonial value than any real purpose. The evidence from the Suttas, however, shows that the \textit{\textsanskrit{nisīdana}} was used outdoors as an underlay when sitting on the ground, presumably to protect the robes.\footnote{See \href{https://suttacentral.net/dn16/en/sujato\#3.1.3}{DN~16}, \href{https://suttacentral.net/sn51.10/en/sujato\#1.5}{SN~51.10}, \href{https://suttacentral.net/an8.70/en/sujato\#1.4}{AN~8.70}, \href{https://suttacentral.net/mn24/en/sujato\#7.1}{MN~24}, \href{https://suttacentral.net/mn147/en/sujato\#1.8}{MN~147}, and \href{https://suttacentral.net/sn35.121/en/sujato\#1.7}{SN~35.12}. } \href{https://suttacentral.net/pli-tv-bu-vb-np15/en/brahmali\#1.3.10.1}{Bu~NP~15}, which concerns the \textit{\textsanskrit{nisīdanasanthata}}, the “sitting blanket”, points to \textit{\textsanskrit{nisīdanas}} being made of extra thick material, which again fits with outdoor usage. For this reason, I have opted to render \textit{\textsanskrit{nisīdana}} as “sitting mat”.

\subsection*{\textit{Nissaya/\textsanskrit{nissāya}}: “formal support”, “support”}

“Formal support” renders \textit{nissaya}, often translated as “dependence”. Yet “dependence” only gives one nuance of this word. Just as common in the Canonical texts, perhaps more common, is the idea of “support”. “Support” gives a different psychological perspective on \textit{nissaya}, one that is perhaps more positive. Here are a couple of instances where \textit{nissaya} is closer in meaning to “support” than “dependence”:

\begin{quotation}%
As Sudinna was eating the previous evening’s porridge, supported by the base of a certain wall … (\href{https://suttacentral.net/pli-tv-bu-vb-pj1/en/brahmali\#5.6.9}{Bu~Pj~1:5.6.9})

%
\end{quotation}

\begin{quotation}%
“But since we’re staying here, we look to you for support.” (\href{https://suttacentral.net/pli-tv-bu-vb-np30/en/brahmali\#1.10}{Bu~NP~30:1.10})

%
\end{quotation}

\textit{Nissaya} as a technical term of the Vinaya \textsanskrit{Piṭaka} does not just mean any kind of support. Rather, it refers to the institutionalized relationship between teacher and student. This is a \textit{formal} relationship, and thus my rendering.

\subsection*{\textit{\textsanskrit{Paṇḍaka}}: “\textit{\textsanskrit{paṇḍaka}}”}

I have left the term \textit{\textsanskrit{paṇḍaka}} untranslated. \textit{\textsanskrit{Paṇḍaka}} is not interpreted in the Canonical literature, and there are few contextual hints as to its meaning. One of the few Canonical passages that deals with the \textit{\textsanskrit{paṇḍaka}} in any detail is found in Kd 1:

\begin{quotation}%
At one time a certain \textit{\textsanskrit{paṇḍaka}} had gone forth as a monk. He went to the young monks and said, “Come, Venerables, have sex with me.”

The monks dismissed him, “Go away, \textit{\textsanskrit{paṇḍaka}}. We don’t want you.”

He went to the big and fat novices, said the same thing, and got the same response. He then went to the elephant keepers and the horse keepers, and once again said the same thing. And they had sex with him.

They complained and criticized him, “These Sakyan monastics are \textit{\textsanskrit{paṇḍakas}}. And those who are not have sex with them. None of them is celibate.”

The monks heard their complaints and told the Buddha.

“A \textit{\textsanskrit{paṇḍaka}} shouldn’t be given the full ordination. If it has been given, he should be expelled.” (\href{https://suttacentral.net/pli-tv-kd1/en/brahmali\#61.1.1}{Kd~1:61.1.1}–61.1.19)

%
\end{quotation}

The description here suggests that \textit{\textsanskrit{paṇḍakas}} had an especially high libido. From other passages it is clear that the \textit{\textsanskrit{paṇḍaka}} was considered a kind of third sex, neither male nor female. At Bu Pj 1 we find a list of four kinds of humans: women, hermaphrodites (\textit{\textsanskrit{ubhatobyañjanakas}}, see separate entry), \textit{\textsanskrit{paṇḍakas}}, and men (\href{https://suttacentral.net/pli-tv-bu-vb-pj1/en/brahmali\#9.1.1}{Bu~Pj~1:9.1.1}–9.1.8). In other words, the \textit{\textsanskrit{paṇḍaka}} was considered neither male nor female. A similar distinction is found at Bu Ss 2 and 4, where one perceives a person as either a \textit{\textsanskrit{paṇḍaka}}, an animal, a man, or a woman (\href{https://suttacentral.net/pli-tv-bu-vb-ss2/en/brahmali\#3.1.21}{Bu~Ss~2:3.1.21}–3.1.27 and \href{https://suttacentral.net/pli-tv-bu-vb-ss4/en/brahmali\#3.1.11}{Bu~Ss~4:3.1.11}–3.1.13). Further, it seems that the designation \textit{\textsanskrit{paṇḍaka}} as a term for a kind of libidinous third-sex individual fits with how it was used in the broader Indian culture, especially as a development from the earlier and probably more general term \textit{\textsanskrit{napuṁsaka}}, literally, “not a man”. See Bhikkhu*\textsanskrit{nī} Vimala’s essay, \textit{Yellow Gate}, on the historical and semantic development of these terms.

For further information on the \textit{\textsanskrit{paṇḍaka}} as it relates to the Vinaya, we need to consult the Pali commentaries, which interpret the term as follows:

\begin{quotation}%
A \textit{\textsanskrit{paṇḍaka}}: a sexless person (\textit{\textsanskrit{napuṁsaka}}) with strong defilements, one whose (sexual) fevers are not allayed. Overcome by the force of his fevers, he desires friendship with anyone.\footnote{Sp 3.87: \textit{\textsanskrit{Paṇḍakāti} \textsanskrit{ussannakilesā} \textsanskrit{avūpasantapariḷāhā} \textsanskrit{napuṁsakā}; te \textsanskrit{pariḷāhavegābhibhūtā} yena kenaci \textsanskrit{saddhiṁ} \textsanskrit{mittabhāvaṁ} patthenti}. }

%
\end{quotation}

\begin{quotation}%
“A \textit{\textsanskrit{paṇḍaka}}, monks”: here there are five kinds of \textit{\textsanskrit{paṇḍaka}}: the \textit{\textsanskrit{āsittapaṇḍaka},} the \textit{\textsanskrit{usūyapaṇḍaka},} the \textit{\textsanskrit{opakkamikapaṇḍaka},} the \textit{\textsanskrit{pakkhapaṇḍaka},} and the \textit{\textsanskrit{napuṁsakapaṇḍaka}}. The \textit{\textsanskrit{āsittapaṇḍaka}} allays his sexual fevers by taking the penis of another in his mouth and being sprinkled by semen. The \textit{\textsanskrit{usūyapaṇḍaka}} allays his sexual fevers by jealously watching others having sex. The \textit{\textsanskrit{opakkamikapaṇḍaka}} has had his testicles removed. The \textit{\textsanskrit{pakkhapaṇḍaka}} is a \textit{\textsanskrit{paṇḍaka}} who, as a result of bad \textit{kamma}, is a \textit{\textsanskrit{paṇḍaka}} during the fortnight when the moon is waning, but whose sexual fevers are calmed during the fortnight when the moon is waxing. The \textit{\textsanskrit{napuṁsakapaṇḍaka}} is one who is undeveloped from the time of conception. Of these, the \textit{\textsanskrit{āsittapaṇḍaka}} and the \textit{\textsanskrit{usūyapaṇḍaka}} are not barred from the going forth, but the other three are. It is said in the Kurundi that the \textit{\textsanskrit{pakkhapaṇḍaka}} is barred from the going forth during the fortnight he is a \textit{\textsanskrit{paṇḍaka}}. That: “A \textit{\textsanskrit{paṇḍaka}} should not be given the full ordination. If it has been given, he should be expelled” was said [in the Canonical text] with reference to those for whom the going forth is barred. And he should be expelled because of his sexual characteristics.\footnote{Sp 3.109: \textit{\textsanskrit{Paṇḍako} bhikkhaveti ettha \textsanskrit{āsittapaṇḍako} \textsanskrit{usūyapaṇḍako} \textsanskrit{opakkamikapaṇḍako} \textsanskrit{pakkhapaṇḍako} \textsanskrit{napuṁsakapaṇḍakoti} \textsanskrit{pañca} \textsanskrit{paṇḍakā}. Tattha yassa \textsanskrit{paresaṁ} \textsanskrit{aṅgajātaṁ} mukhena \textsanskrit{gahetvā} \textsanskrit{asucinā} \textsanskrit{āsittassa} \textsanskrit{pariḷāho} \textsanskrit{vūpasammati}, \textsanskrit{ayaṁ} \textsanskrit{āsittapaṇḍako}. Yassa \textsanskrit{paresaṁ} \textsanskrit{ajjhācāraṁ} passato \textsanskrit{usūyāya} \textsanskrit{uppannāya} \textsanskrit{pariḷāho} \textsanskrit{vūpasammati}, \textsanskrit{ayaṁ} \textsanskrit{usūyapaṇḍako}. Yassa upakkamena \textsanskrit{bījāni} \textsanskrit{apanītāni}, \textsanskrit{ayaṁ} \textsanskrit{opakkamikapaṇḍako}. Ekacco pana \textsanskrit{akusalavipākānubhāvena} \textsanskrit{kāḷapakkhe} \textsanskrit{paṇḍako} hoti, \textsanskrit{juṇhapakkhe} panassa \textsanskrit{pariḷāho} \textsanskrit{vūpasammati}, \textsanskrit{ayaṁ} \textsanskrit{pakkhapaṇḍako}. Yo pana \textsanskrit{paṭisandhiyaṁyeva} \textsanskrit{abhāvako} uppanno, \textsanskrit{ayaṁ} \textsanskrit{napuṁsakapaṇḍakoti}. Tesu \textsanskrit{āsittapaṇḍakassa} ca \textsanskrit{usūyapaṇḍakassa} ca \textsanskrit{pabbajjā} na \textsanskrit{vāritā}, \textsanskrit{itaresaṁ} \textsanskrit{tiṇṇaṁ} \textsanskrit{vāritā}. Tesupi \textsanskrit{pakkhapaṇḍakassa} \textsanskrit{yasmiṁ} pakkhe \textsanskrit{paṇḍako} hoti, \textsanskrit{tasmiṁyevassa} pakkhe \textsanskrit{pabbajjā} \textsanskrit{vāritāti} \textsanskrit{kurundiyaṁ} \textsanskrit{vuttaṁ}. Yassa cettha \textsanskrit{pabbajjā} \textsanskrit{vāritā}, \textsanskrit{taṁ} \textsanskrit{sandhāya} \textsanskrit{idaṁ} \textsanskrit{vuttaṁ} – “anupasampanno na \textsanskrit{upasampādetabbo} upasampanno \textsanskrit{nāsetabbo}”ti. Sopi \textsanskrit{liṅganāsaneneva} \textsanskrit{nāsetabbo}.} }

%
\end{quotation}

It seems from the above that \textit{\textsanskrit{paṇḍakas}} constitute a varied group, all of whom are regarded as a kind of third sex.\footnote{The Canonical texts nevertheless make a distinction between a \textit{\textsanskrit{paṇḍaka}} and an \textit{\textsanskrit{itthipaṇḍaka}}, the latter meaning “a female \textit{\textsanskrit{paṇḍaka}}”, see \href{https://suttacentral.net/pli-tv-bu-vb-ss2/en/brahmali\#3.1.32}{Bu Ss 2:3.1.32}, \href{https://suttacentral.net/pli-tv-bu-vb-ss5/en/brahmali\#5.2.8}{Bu Ss 5:5.2.8}, and \href{https://suttacentral.net/pli-tv-kd20/en/brahmali\#17.1.2}{Kd 20:17.1.2}. And so it seems that although they were not regarded as fully male or female, they were still considered as closer to one gender or the other. For this reason, I use gendered pronouns in referring to them. (I have rendered \textit{\textsanskrit{itthipaṇḍaka}} in accordance with the commentary as “a woman who lacks sexual organs”.) } What they have in common is either a strong libido or perhaps sexual urges that cannot be properly satisfied. It is impossible to come up with an English umbrella term that satisfactorily captures all these individuals. The matter is further complicated by the commentarial explanation that some \textit{\textsanskrit{paṇḍakas}} are allowed to go forth whereas others are not. All things considered, I have decided to leave \textit{\textsanskrit{paṇḍaka}} untranslated.

Since the idea of a \textit{\textsanskrit{paṇḍaka}} is so tightly tied up with ancient Indian social and cultural categories, and there appear to be no comparable categories in modern society, at least in the West, it would seem that the rule prohibiting \textit{\textsanskrit{paṇḍakas}} from ordaining is largely meaningless in a modern context.

\subsection*{\textit{\textsanskrit{Parikkhāra}}: “requisite(s)”, “ingredient(s)”}

\textit{\textsanskrit{Parikkhāra}}, “requisite” or “ingredient”, is used in both the plural and the singular. Often, however, even the singular form is plural in meaning, similar to how the word “equipment” is used in English. Here are a couple unambiguous instances:

\begin{quotation}%
They prepared a number of monastic requisites (\textit{\textsanskrit{parikkhāraṁ}}, singular): a bowl, a robe, a sitting mat, a needle case, a belt, a water filter, and a water strainer. (\href{https://suttacentral.net/pli-tv-kd22/en/brahmali\#2.1.9}{Kd 22:2.1.9})

%
\end{quotation}

\begin{quotation}%
On one occasion a nun who was dying said, “When I’m dead, give my requisites (\textit{\textsanskrit{parikkhāro}}, singular) to the Sangha.” (\href{https://suttacentral.net/pli-tv-kd20/en/brahmali\#11.1.1}{Kd 20:11.1.1})

%
\end{quotation}

Given this precedent, I render the grammatical singular \textit{\textsanskrit{parikkhāra}} as the plural “requisites” in \textit{\textsanskrit{bhikkhunī} nissaggiya \textsanskrit{pācittiyas}} 6, 7, and 10, and as the plural “ingredients” in \textit{\textsanskrit{bhikkhunī} nissaggiya \textsanskrit{pācittiyas}} 8 and 9. In the context, the plural seems more appropriate.

\subsection*{\textit{\textsanskrit{Paripucchā}}: “questioning”, “testing”}

The basic meaning of \textit{\textsanskrit{paripucchā}} is “to question” or “to ask”, as used for instance at \href{https://suttacentral.net/pli-tv-bu-vb-pc71/en/brahmali\#1.19.1}{Bu~Pc~71:1.19.1}. Often, however, it refers to a teacher questioning his student, in the sense of finding out how much the student knows. In such cases I render the word as “testing”.

\subsection*{\textit{\textsanskrit{Pariveṇa}}: “yard”}

I. B. Horner renders \textit{\textsanskrit{pariveṇa}} as “cell”, for which there is no proper basis. The \textit{\textsanskrit{pariveṇa}} is the area surrounding a building. That it is not an indoor area can be seen from \href{https://suttacentral.net/pli-tv-kd15/en/brahmali\#14.5.1}{Kd~15:14.5.1} where a \textit{\textsanskrit{pariveṇa}} is said to get muddy, upon which the Buddha allows gravel, stepping stones, and drains. The \textit{\textsanskrit{pariveṇa}} is often mentioned in connection with keeping a building tidy, specifically as an area that is to be swept, e.g. at \href{https://suttacentral.net/pli-tv-bu-vb-ss2/en/brahmali\#1.1.6}{Bu~Ss~2:1.1.6}. It seems natural to conclude that this must refer to the area, or part of it, that surrounds a building. This is confirmed by the commentaries:

\begin{quotation}%
\textit{\textsanskrit{Pariveṇa}}: 11 meters surrounding each individual dwelling.\footnote{Sp‑\textsanskrit{ṭ} 2.117: \textit{\textsanskrit{Pariveṇanti} ekekassa \textsanskrit{vihārassa} \textsanskrit{parikkhepabbhantaraṁ}}. }

%
\end{quotation}

Sometimes, but not always it seems, this area was walled in:

\begin{quotation}%
\textit{\textsanskrit{Pariveṇa}}: a place separately delimited by a wall within a large monastery.\footnote{Sp‑\textsanskrit{ṭ} 1.103: \textit{\textsanskrit{Pariveṇanti} \textsanskrit{mahāvihārassa} abbhantare \textsanskrit{visuṁ} \textsanskrit{visuṁ} \textsanskrit{pākāraparicchinnaṭṭhānaṁ}}. }

%
\end{quotation}

The existence of surrounding walls fits with the fact that \textit{\textsanskrit{pariveṇas}} are said to be constructed (\href{https://suttacentral.net/pli-tv-kd3/en/brahmali\#5.6.5}{Kd~3:5.6.5}). So does the frequent mention in the Vinaya \textsanskrit{Piṭaka} of entry points, known as \textit{\textsanskrit{koṭṭhakas}}, “gatehouses”, through the surrounding walls. See the separate entry on \textit{\textsanskrit{koṭṭhakas}}.

On a more general note, the term \textit{\textsanskrit{pariveṇa}} is not found in the four main \textsanskrit{Nikāyas}, and in the Vinaya it is found only in narratives, minor rules on conduct, and explanatory material. This suggests it may not have existed in the earliest period of Buddhist monasticism. The earliest buildings and monasteries would have been simple, without such embellishments, especially the surrounding walls.

\subsection*{\textit{\textsanskrit{Pavāraṇā}/\textsanskrit{pavāreti}}: “invitation ceremony”/“to invite (to eat more)”}

The cognate words \textit{\textsanskrit{pavāraṇā}} and \textit{\textsanskrit{pavāreti}} have the general sense of “offer” or “invitation”, with the exact meaning depending on the context. Because these contexts are mostly distinct and because I try to stay as close to English idiom as I can, I render \textit{\textsanskrit{pavāraṇā}}/\textit{\textsanskrit{pavāreti}} slightly differently in each case.

The most important context for \textit{\textsanskrit{pavāraṇā}}/\textit{\textsanskrit{pavāreti}} is the ceremony named after it. This is explained in detail in \href{https://suttacentral.net/pli-tv-kd4/en/brahmali}{Kd~4}. This \textit{khandhaka} begins with a story of monks who spent the rainy season in complete silence. The Buddha does not approve and lays down a rule that this should not be done. He then establishes the \textit{\textsanskrit{pavāraṇā}}, during which each monastic is to say the following in the presence of the Sangha:

\begin{quotation}%
“I \textit{\textsanskrit{pavāraṇā}} the Sangha concerning what you have seen, heard, or suspect. Please correct me, venerables, out of compassion. If I see (a fault), I will make amends.” (\href{https://suttacentral.net/pli-tv-kd4/en/brahmali\#1.14.7}{Kd~4:1.14.7})

%
\end{quotation}

From this it is clear that the \textit{\textsanskrit{pavāraṇā}} is an invitation to one’s fellow monastics to admonish or point out wrong behavior. In other words, it is a specific kind of invitation to correct. It follows that rendering \textit{\textsanskrit{pavāraṇā}}, in this context, simply as “invitation” is misleading. To bring out this narrow meaning I use “invitation to correct”.

As can be seen from the above quote, the \textit{\textsanskrit{pavāraṇā}} is a formal invitation that is part of a set ritual performed by the Sangha in any given monastery. Historically this has been regarded as a \textit{\textsanskrit{saṅghakamma}}, “a legal procedure”, but its status in the Canonical Vinaya is actually ambiguous. Although the \textit{\textsanskrit{pavāraṇā}} is occasionally called the \textit{\textsanskrit{pavāraṇākamma}}, “the invitation (legal) procedure”, this usage is rare in the Canonical texts. Apart from a few instances in the \textsanskrit{Parivāra}, in the entire Vinaya \textsanskrit{Piṭaka} this usage is only found in one short subchapter at \href{https://suttacentral.net/pli-tv-kd4/en/brahmali\#3.2.2}{Kd~4:3.2.2}–3.2.17. In fact, the \textit{\textsanskrit{pavāraṇā}} is sometimes mentioned as an alternative to \textit{\textsanskrit{saṅghakamma}}, giving the impression that it was not originally considered a legal procedure at all, or at least that it was considered special enough to be singled out. (The same is true for the \textit{uposatha}, “the observance day”.) Here is an example from Bu Ss 8:

\begin{quotation}%
“There is no (doing of) observance days, \textit{\textsanskrit{pavāraṇā}}, or legal procedures with you.” (\href{https://suttacentral.net/pli-tv-bu-vb-ss8/en/brahmali\#3.1.4}{Bu~Ss~8:3.1.4})

%
\end{quotation}

And another from the Chapter on Schism in the Sangha:

\begin{quotation}%
“If, based on any of these eighteen grounds, they pull away and separate, and they do the observance-day ceremony, the \textit{\textsanskrit{pavāraṇā}}, or legal procedures of the Sangha separately, then there is a schism in the Sangha.” (\href{https://suttacentral.net/pli-tv-kd17/en/brahmali\#5.2.21}{Kd~17:5.2.21})

%
\end{quotation}

This suggests that it is only appropriate to use the expression “invitation procedure” when \textit{\textsanskrit{pavāraṇā}} is compounded with the word \textit{kamma} as in \textit{\textsanskrit{pavāraṇākamma}}. Given these considerations, whenever \textit{\textsanskrit{pavāraṇā}} refers to the overall ritual as performed by the Sangha, without being specifically called an “invitation procedure”, I render it as “invitation ceremony”.

Although the invitation ceremony is the most significant context for \textit{\textsanskrit{pavāraṇā}}/\textit{\textsanskrit{pavāreti}}, they are also used for other purposes. One important instance is where the verb \textit{\textsanskrit{pavāreti}} is used in conjunction with a meal invitation. At such invitations, the lay host would normally keep on serving the monastics until the monastics were satisfied and declined any further offer of food. The monastic is then said to be \textit{\textsanskrit{pavārita}}, “invited”—that is, they have been invited to eat more but have refused—the implication being that they have had enough. In such contexts English idiom requires “offered” rather than “invited”, and so I render \textit{\textsanskrit{pavārita}} as “refused an offer to eat more”. This meaning is especially relevant at \href{https://suttacentral.net/pli-tv-bu-vb-pc35/en/brahmali\#3.1.7}{Bu~Pc~35} and \href{https://suttacentral.net/pli-tv-bu-vb-pc36/en/brahmali\#2.1.9}{36}, as well as \href{https://suttacentral.net/pli-tv-bi-vb-pc54/en/brahmali\#2.1.7}{Bi~Pc~54}.

Another significant context is where \textit{\textsanskrit{pavārita}} refers to a lay person who has given a monastic an invitation to ask for requisites or to specify anything regarding a requisite, for instance at \href{https://suttacentral.net/pli-tv-bu-vb-np6/en/brahmali\#3.4.4}{Bu~NP~6} and \href{https://suttacentral.net/pli-tv-bi-vb-np11/en/brahmali\#2.3.5}{Bi NP~11}. In these cases, the word is rendered as “those who have given an invitation”.

\subsection*{\textit{\textsanskrit{Pāsāda}}: “stilt house”}

\textit{\textsanskrit{Pāsāda}} is often rendered as “palace” or “mansion”. This seems to be based entirely on commentarial explanations. The Canonical texts do not lend any support to either of these two renderings.

Let us take a brief look at a couple of commentarial descriptions. In the \textsanskrit{Aṅguttara} \textsanskrit{Nikāya} the Buddha-to-be is said to have had three \textit{\textsanskrit{pāsādas}} before he left the household life, one for each of the Indian seasons (\href{https://suttacentral.net/an3.39/en/sujato\#2.1}{AN~3.39:2.1}). According to the commentary, these were respectively nine, seven, and five stories tall.\footnote{AN-a 3.39: \textit{Tattha hemantiko \textsanskrit{pāsādo} \textsanskrit{navabhūmako} ahosi … Gimhiko pana \textsanskrit{pañcabhūmako} ahosi. … Vassiko \textsanskrit{sattabhūmako} ahosi}, “There, the winter \textit{\textsanskrit{pāsāda}} was nine stories … The summer \textit{\textsanskrit{pāsāda}} was five stories … The rainy season \textit{\textsanskrit{pāsāda}} was seven stories.” } Similarly, the commentary to the Majjhima \textsanskrit{Nikāya} speaks of a seven story \textit{\textsanskrit{pāsāda}}.\footnote{MN-a 3.74: \textit{\textsanskrit{Sattabhūmiko} \textsanskrit{pāsādo}}, “a seven-story \textit{\textsanskrit{pāsāda}.}” }

These commentarial descriptions are hard to take seriously. There are no obvious mentions in the four \textsanskrit{Nikāyas} or the Vinaya \textsanskrit{Piṭaka} of multi-story buildings, except for occasional references to a single upper story or a loft, such as the \textit{\textsanskrit{uparivehāsakuṭi}} of \href{https://suttacentral.net/pli-tv-bu-vb-pc18/en/brahmali\#1.18.1}{Bu~Pc~18}. Moreover, the archaeological evidence from the period suggests that the towns were modest in size, even the big ones like \textsanskrit{Sāvatthī}. Academic estimates for the size of the largest cities are typically under 10,000 inhabitants.\footnote{Erdosy (1988) has a detailed discussion of a large number of archaeological excavations in the Allahabad district in the Ganges plane, relating to urbanization in the period 600–350 BCE. On p. 59 he states that “The dominant settlement in the hierarchy was Kausambi, whose size, estimated as 50 hectares, and elaborate fortification place it in a separate category.” On p. 45 he assumes a population density of 160 people per hectare. Taken together, we can estimate the population of Kausambi at the time of the Buddha as around 8,000 people. According to the \textsanskrit{Mahāparinibbāna} Sutta (DN 16), Kausambi was one of the six great cities of ancient India. We can infer, then, a population size of approximately 10,000 people for the largest cities in the Ganges plane at that time. } It seems unlikely that these would have had enough material resources and wealth to build very large or tall structures. Even more important is the fact that all buildings were built of perishable materials, which is why there are no material remains from the earliest period of Buddhism, except for occasional embankments and city walls. Once again, it seems unlikely they would have built very large buildings, especially very tall ones, out of wood. Everything in the archaeological record from the Ganges plain for the relevant period points to small-scale buildings.

So, if the commentarial notions are fanciful, what were these \textit{\textsanskrit{pāsādas}}? A suitable point of departure is to note that \textit{\textsanskrit{pāsādas}} were allowed by the Buddha for monastics. Now one of the main principles of the Vinaya is that monastics should not live in luxury. On this basis alone we can conclude that the \textit{\textsanskrit{pāsādas}} were not generally luxurious. This immediately eliminates “palace” and “mansion” as suitable translations.

Another thing that stands out about \textit{\textsanskrit{pāsādas}} is that they are always ascended to and descended from. This means that they must have been relatively tall buildings. An interesting point is that the Suttas do not seem to mention any other way of accessing these buildings—apart from ascent/descent—which must mean that they were not directly accessible from ground level. They must have been built on some sort of high foundation. A passage in the Kosala \textsanskrit{Saṁyutta} is particularly instructive:

\begin{quotation}%
“Suppose, great king, a man would climb from the ground on to a palanquin, or from a palanquin on to horseback, or from horseback to an elephant mount, or from an elephant mount to a \textit{\textsanskrit{pāsāda}} …” (\href{https://suttacentral.net/sn3.21/en/sujato\#5.1}{SN~3.21:5.1})

%
\end{quotation}

According to a number of passages, the \textit{\textsanskrit{pāsādas}} had external staircases (\textit{\textsanskrit{sopāṇakaḷevara}} at \href{https://suttacentral.net/mn85/en/sujato\#5.1}{MN~85:5.1}) or ladders (\textit{\textsanskrit{nisseṇi}} at \href{https://suttacentral.net/dn9/en/sujato\#37.1}{DN~9:37.1} and \href{https://suttacentral.net/dn13/en/sujato\#21.1}{DN~13:21.1}), including the \textit{\textsanskrit{pāsāda}} given to the Sangha by \textsanskrit{Visākha} (\href{https://suttacentral.net/mn107/en/sujato\#2.2}{MN~107:2.2}), and they seem always to have been entered with the help of these. The similes in DN 9 and DN 13 speak of making the ladder before building a hypothetical \textit{\textsanskrit{pāsāda}}. These similes seem to imply that \textit{all \textsanskrit{pāsādas}} were accessed by climbing. In addition, the use of the word “ladder” shows that \textit{\textsanskrit{pāsādas}} were often quite humble buildings.

The need for climbing to get access also seems to be why—in one of the case studies at \href{https://suttacentral.net/pli-tv-bu-vb-pj2/en/brahmali\#7.47.3}{Bu~Pj~2:7.47.3}—Ven. Pilindavaccha was able to make use of a \textit{\textsanskrit{pāsāda}} to hide children from kidnappers. If \textit{\textsanskrit{pāsādas}} were high off the ground and the access ladder was removed, then the children were presumably safe. We also find the common compound \textit{\textsanskrit{uparipāsāda}}, “up the \textit{\textsanskrit{pāsāda}}”, which fits this description. No other names for types of buildings are compounded with \textit{upari} in this way.

The final clue to the meaning of \textit{\textsanskrit{pāsāda}} is that they seem to have had a space underneath. In the passage from AN 3.39 mentioned above, where the Buddha speaks of his life before going forth, it is said that he stayed in the rainy-season \textit{\textsanskrit{pāsāda}} for the full four months of the rainy season, without \textit{\textsanskrit{heṭṭhāpāsādaṁ} \textsanskrit{orohāmi}}. The verb \textit{orohati} means “to descend” and \textit{\textsanskrit{heṭṭhā}} means “below”. \textit{\textsanskrit{Heṭṭhā}} is used in other compounds such as \textit{\textsanskrit{heṭṭhāmañcaṁ}} which unambiguously means “below the bed (\textit{\textsanskrit{mañca}})”. It follows that \textit{\textsanskrit{heṭṭhāpāsādaṁ} \textsanskrit{orohāmi}} can only really mean “I descended to below the \textit{\textsanskrit{pāsāda}}”, which must mean that \textit{\textsanskrit{pāsādas}} had open spaces underneath them. The most obvious sort of building that fits with this is a stilt house.

Now a stilt house would normally be a fairly humble building compared to a palace. But it seems likely to me that in an ancient Indian setting they would have had considerable advantages, and thus were probably regarded as high-end housing. In a climate where heavy rain and flooding were common, stilt houses would be particularly sought after. It is also possible, as the story with Pilindavaccha shows, that these houses were considered relatively safe from burglars and the like.

There are a couple of passages in the Suttas, however, where \textit{\textsanskrit{pāsāda}} clearly does refer to a “palace”, such as Sakka’s and \textsanskrit{Mahāsudassana}’s palaces (respectively at \href{https://suttacentral.net/mn37/en/sujato\#8.9}{MN~37:8.9} and \href{https://suttacentral.net/dn17/en/sujato\#1.25.8}{DN~17:1.25.8}). In both these cases the \textit{\textsanskrit{pāsādas}} were very grand structures, and even the word “palace” may not do full justice to their magnificence. Yet in both instances the setting is mythological; they do not refer to historical India. Perhaps the word \textit{\textsanskrit{pāsāda}} was used because the palaces were envisaged as grander versions of the best buildings they had at the time.

\subsection*{\textit{\textsanskrit{Pūva}} and \textit{mantha}: “cookie” and “cracker”}

I. B. Horner renders \textit{\textsanskrit{pūva}} as “cake”. Yet according to \href{https://suttacentral.net/pli-tv-bu-vb-pc34/en/brahmali\#2.1.6}{Bu~Pc~34}, they were used as gifts to be sent away, which presumably meant they were dry edibles. The commentary has this to say:

\begin{quotation}%
Whatever is for the purpose of sending away: whatever cake having a superior taste, etc., is prepared as a gift, all that is here counted as \textit{\textsanskrit{pūva}.}\footnote{Sp 2.233: \textit{\textsanskrit{Yaṅkiñci} \textsanskrit{paheṇakatthāyāti} \textsanskrit{paṇṇākāratthāya} \textsanskrit{paṭiyattaṁ} \textsanskrit{yaṅkiñci} \textsanskrit{atirasakamodakasakkhalikādi} \textsanskrit{sabbaṁ} idha \textsanskrit{pūvotveva} \textsanskrit{saṅkhyaṁ} gacchati}. }

%
\end{quotation}

The same rule says that \textit{\textsanskrit{pūvas}} where cooked, \textit{paci}. According to \href{https://suttacentral.net/pli-tv-bu-vb-pc41/en/brahmali\#1.1.4}{Bu~Pc~41}, on several occasions Ānanda accidently gave two \textit{\textsanskrit{pūvas}} to a female wanderer when intending to give only one. All this fits with the idea of a cookie, but is more difficult to reconcile with “cake”. Finally, according to \href{https://suttacentral.net/pli-tv-bu-vb-pc51/en/brahmali\#2.1.2}{Bu~Pc~51}, \textit{\textsanskrit{pūvas}} where used to make alcohol, which presumably means they contained starch and/or sugar. Although a \textit{\textsanskrit{pūva}} is unlikely to be an exact equivalent of what we now call a “cookie”, it would not seem to be far off the mark.

I. B. Horner then translates \textit{mantha} as “barley gruel”, which seems even further off the mark. According to \href{https://suttacentral.net/pli-tv-bu-vb-pc34/en/brahmali\#2.1.8}{Bu~Pc~34}, \textit{manthas} were used as provisions for journeys, which yet again must mean they were dry edibles. Here is the commentarial gloss:

\begin{quotation}%
Whatever provisions for a journey: whatever flour made from sesame, grains, etc., whether made into goods or not, and that is prepared for the journey of those traveling, all that is here counted as \textit{mantha}.\footnote{Sp 2.233\textit{: \textsanskrit{Yaṅkiñci} \textsanskrit{pātheyyatthāyāti} \textsanskrit{maggaṁ} \textsanskrit{gacchantānaṁ} \textsanskrit{antarāmaggatthāya} \textsanskrit{paṭiyattaṁ} \textsanskrit{yaṅkiñci} \textsanskrit{baddhasattuabaddhasattutilataṇḍulādi} \textsanskrit{sabbaṁ} idha manthotveva \textsanskrit{saṅkhyaṁ} gacchati}. }

%
\end{quotation}

Moreover, since they were used as provisions rather than as gifts, it seems reasonable to assume they were not sweet.

\subsection*{\textit{\textsanskrit{Pokkharaṇī}}: “(lotus) pond” or “(lotus) tank”}

The usual meaning of \textit{\textsanskrit{pokkharaṇī}} as “lotus pond” is well established in the \textit{suttas}. Much of the time they seem to have been decorative, but there are few instances in the Vinaya where they are used for washing, such as in \textsanskrit{Mahā}-khandhaka where the Buddha washes a cloth in a \textit{\textsanskrit{pokkharaṇī}} (\href{https://suttacentral.net/pli-tv-kd1/en/brahmali\#20.1.4}{Kd~1:20.1.4}). In fact, this distinction in use is reflected in the commentaries, which speak of \textit{\textsanskrit{nahāna}-\textsanskrit{pokkharaṇī}}, “\textit{\textsanskrit{pokkharaṇī}} for bathing”, and \textit{\textsanskrit{kīḷana}-\textsanskrit{pokkharaṇī}}, “\textit{\textsanskrit{pokkharaṇī}} for playing”. Because \textit{\textsanskrit{pokkharaṇīs}} were given to monasteries “for the benefit of the Sangha” (\href{https://suttacentral.net/pli-tv-kd15/en/brahmali\#17.2.1}{Kd~15:17.2.1}), they were probably meant for bathing and washing, not just for decoration. Moreover, it seems the \textit{\textsanskrit{pokkharaṇīs}} could be quite elaborate structures with foundations, staircases, and rails (\href{https://suttacentral.net/pli-tv-kd15/en/brahmali\#17.2.6}{Kd~15:17.2.6}). For these reasons, I vary my translation according to context, sometimes using “pond”, at other times “tank”, and sometimes adding the qualifier “lotus”.

\subsection*{\textit{\textsanskrit{Phāṇita}: “}syrup”}

I. B. Horner translates \textit{\textsanskrit{phāṇita}} as “molasses”, which doesn’t quite hit the mark. SED defines \textit{\textsanskrit{phāṇita}} as “the inspissated juice of the sugar cane or other plants”, in other words, “cane syrup”. According to the commentary at Sp 1.623 it can be either cooked or uncooked, the difference presumably being whether it is raw or concentrated. “Syrup” seems closer to the mark than “molasses”.

\subsection*{\textit{Bhattagga}: “dining hall”}

\textit{Bhattagga} is literally “a meal house”. The name suggests that the \textit{bhattagga} was a separate building for eating. They were found both in private houses and in monasteries (\href{https://suttacentral.net/pli-tv-kd10/en/brahmali\#4.5.7}{Kd~10:4.5.7}). Since they were part of houses or a compound of private buildings, “refectory” is not a satisfactory rendering. The fact that kitchens are not mentioned separately may mean that they were part of the \textit{bhattagga}, except in monasteries. This is supported by a passage at (\href{https://suttacentral.net/pli-tv-bu-vb-pj3/en/brahmali\#5.3.1}{Bu~Pj~3:5.3.1}) that mentions a cooking implement, a pestle, being stored in a village \textit{bhattagga}.

\subsection*{\textit{Bhesajja}: “tonic”, “medicine”}

The Pali word \textit{bhesajja} is broader than the English “medicine”. It includes certain foods that provide nourishment without being considered substantial. In most contexts this latter group concerns a standard set of five non-substantial foods: ghee, butter, oil, honey, and syrup. Occasionally it also refers to other edibles that are closer to food than medicine proper, such as fats (which are to be mixed with oil) and in one case even rice porridge. I have called these “tonics”, following the precedent set by others, such as Bhikkhu \textsanskrit{Ṭhānissaro} and Ajahn Brahm. In other contexts, \textit{bhesajja} is rendered as “medicine”, including cases where \textit{bhesajja} is used as an umbrella term for all of the above and cases of medicines used externally.

\subsection*{\textit{\textsanskrit{Bhojanīya}, \textsanskrit{khādanīya}}: “cooked food”, “fresh food”}

\textit{\textsanskrit{Bhojanīya}}, together with \textit{\textsanskrit{khādanīya}}, is a core concept of the monastic rules. It is perhaps surprising, then, how difficult it is to render these terms into English. A typical translation of the pair, used for instance by I. B. Horner, is “soft food” and “hard food”, respectively for \textit{\textsanskrit{bhojanīya}} and \textit{\textsanskrit{khādanīya}}. This is in turn derived from underlying verbs that mean “to savor” and “to chew”. So presumably the difference in these foods relate to the way they tend to be eaten. Still, it is not obvious that foods to be savored do not also need to be chewed, nor that foods that need chewing cannot also be savored. So, at best “soft food” and “hard food” are no more than convenient approximations. Moreover, “soft food” and “hard food” are quite meaningless in English translation. They do not correspond to any typical division of food we are used to in contemporary Western society.

A more recent rendering for the same pair, used for instance by Bhikkhu \textsanskrit{Ṭhānissaro}, is “staple food” and “non-staple food”, respectively. This rendering has the advantage of fitting well with a typical classification of food in English. However, it also has drawbacks. The way \textit{\textsanskrit{bhojanīya}}, “staple food”, is interpreted in the \textsanskrit{Vibhaṅga}, it includes foods that were not staples in ancient India, such as meat and fish. We can only conclude that this translation does not properly reflect the usage of the underlying Pali.

Even more recently, Bhikkhu \textsanskrit{Ñāṇatusita} has suggested that \textit{\textsanskrit{bhojanīya}} and \textit{\textsanskrit{khādanīya}} should be rendered as “cooked food” and “uncooked food” (“Analysis of the Bhikkhu Pātimokkha”, p. 193). He bases this suggestion on the fact that the definition of \textit{\textsanskrit{bhojanīya}} lists foods that are normally cooked, whereas the definition of \textit{\textsanskrit{khādanīya}} consists of foods that are normally or often served raw. This translation is certainly more meaningful than “soft food” and “hard food”. Now let us compare it to the pair “staple food” and “non-staple food”.

\textit{\textsanskrit{Khādanīya}} occurs in a number of contexts where “uncooked food” fits better than “non-staple food”. \textit{\textsanskrit{Khādanīya}} was often stored in a monastery, whereas there is no indication that this was the case for \textit{\textsanskrit{bhojanīya}}. For instance, in the origin story to \href{https://suttacentral.net/pli-tv-bu-vb-pc41/en/brahmali\#1.1.1}{Bu~Pc~41}, we find that the Sangha had an abundance of \textit{\textsanskrit{khādanīya}}. This food was then distributed in the monastery by an official specifically elected to the task (\href{https://suttacentral.net/pli-tv-bu-vb-pj2/en/brahmali\#7.12.6}{Bu~Pj~2:7.12.6}), the \textit{\textsanskrit{khajjakabhājaka}} (\href{https://suttacentral.net/pli-tv-kd16/en/brahmali\#21.2.10}{Kd~16:21.2.10}). Uncooked food could well be stored in this way, whereas this would not be possible with cooked food at a time when there were no means of refrigeration. We find a particularly instructive passage in \href{https://suttacentral.net/pli-tv-bu-vb-pc37/en/brahmali\#1.4}{Bu~Pc~37}, where monks are first given food to eat (\textit{\textsanskrit{bhojetvā}}, “having been made to eat \textit{\textsanskrit{bhojanīya}}”), and then given \textit{\textsanskrit{khādanīya}} to take away. Again, the cooked/uncooked distinction fits well, but not so much the staple/non-staple duality. Another passage with a similarly suggestive value is found at \href{https://suttacentral.net/pli-tv-bu-vb-pc46/en/brahmali\#2.11}{Bu~Pc~46:2.11}, where \textit{\textsanskrit{khādanīya}} is sent to the monastery and then returned to the owners. There is no mention of \textit{\textsanskrit{bhojanīya}}, and it does seem unlikely cooked food would be treated in this way.

Yet another relevant passage is found in Kd 6, where people are said to bring salt, oil, rice (\textit{\textsanskrit{taṇḍula}}), and \textit{\textsanskrit{khādanīya}} to the monastery and then store it there (\href{https://suttacentral.net/pli-tv-kd6/en/brahmali\#33.1.1}{Kd~6:33.1.1}). Oil and salt are uncooked foods, as is \textit{\textsanskrit{taṇḍula}}, which refers to uncooked rice. (The \textit{\textsanskrit{taṇḍula}} is then cooked, becoming \textit{bhatta} or \textit{\textsanskrit{bhojanīya}}, before it is served to the monks.) Given that all these foods are stored, it is reasonable to infer that the \textit{\textsanskrit{khādanīya}} is also uncooked. Again, the distinction between cooked and uncooked is meaningful, not so much the contrast between staple and non-staple. A similar passage is found earlier on in the same text at \href{https://suttacentral.net/pli-tv-kd6/en/brahmali\#24.1.2}{Kd~6:24.1.2}. In this case the same group of foods is loaded onto carts, which then follow behind the Buddha and the monks, apparently for a period of two months or more. This too only makes sense if \textit{\textsanskrit{khādanīya}} is uncooked food.

What about counterarguments? In \href{https://suttacentral.net/pli-tv-bi-vb-pc44/en/brahmali\#2.1.6}{Bi~Pc~44} \textit{\textsanskrit{khādanīya}} is said to be cooked. However, this does not necessarily contradict the idea that \textit{\textsanskrit{khādanīya}} generally was uncooked or given as uncooked food. Almost any food \emph{can} be cooked, but certain foods tend to be cooked more often than others, such as rice or meat. It follows that \textit{\textsanskrit{bhojanīya}} and \textit{\textsanskrit{khādanīya}} are not exclusive categories, and that some degree of overlap is to be expected.

Then there is the curious lack of mention of \textit{\textsanskrit{sūpa}}, a bean-based curry, which was a cooked staple in ancient India, but is not included in the standard definition of \textit{\textsanskrit{bhojanīya}}. It might be that it was considered too marginal to be included in either \textit{\textsanskrit{bhojanīya}} or \textit{\textsanskrit{khādanīya}}. Perhaps it was seen as no more than a sauce to spice up the rice.

I conclude that staple/non-staple is not a good match for \textit{\textsanskrit{bhojanīya}}/\textit{\textsanskrit{khādanīya}}. For this reason, I follow Bhikkhu \textsanskrit{Ñāṇatusita} in his rendering of these terms, except I use “fresh” in place of “uncooked”.

\subsection*{\textit{Magga, amagga}: “orifice” or “private part”, “mouth”}

The word \textit{magga} normally means “path” or “road”, as in the \textit{\textsanskrit{aṭṭhaṅgika} magga}, “the eightfold path”. In the Vinaya \textsanskrit{Piṭaka}, however, \textit{magga} is sometimes used in a specialized sense.

In part 1 of the permutation series to \textit{bhikkhu \textsanskrit{pārājika}} 1, \textit{magga} explicitly refers to the three orifices—vagina, anus, or mouth—of a female or hermaphrodite partner that a monk might have sexual intercourse with (\href{https://suttacentral.net/pli-tv-bu-vb-pj1/en/brahmali\#9.1.9.1}{Bu~Pj~1:9.1.9.1} and (\href{https://suttacentral.net/pli-tv-bu-vb-pj1/en/brahmali\#9.1.16}{Bu~Pj~1:9.1.16}). In part 2 of the same permutation series, the text refers to both \textit{magga} and \textit{amagga}, but without explaining them (\href{https://suttacentral.net/pli-tv-bu-vb-pj1/en/brahmali\#9.7.7.1}{Bu~Pj~1:9.7.7.1}–9.7.10). The commentary understands \textit{magga} here to have the same meaning as in part 1, whereas it interprets \textit{amagga} to refer to a sore on the body. This interpretation appears to stem from one of the case studies to this rule, which, interestingly, does not actually use the word \textit{amagga}.

On the face of it this does seem strange. Part 2 of the permutation series begins with a short summary of offenses committed under this rule, consisting of no more than four short statements. It would be rather extraordinary if the marginal case of sexual intercourse through a sore on the body should appear here. An even more serious problem is that a kiss involving penetration of the mouth would then be a \textit{\textsanskrit{pārājika}} offense. There is nothing in the Canonical text to substantiate this, and in fact all the evidence points to this not being the case. The rule itself speaks of sexual intercourse, \textit{methunadhamma}, which is then defined as “genital with genital”, \textit{\textsanskrit{aṅgajātena} \textsanskrit{aṅgajātaṁ}}. It makes no sense to interpret this as including mouth to mouth. In addition, there is no mention in the \textsanskrit{Vibhaṅga}, including the case studies, of any offense for mouth-to-mouth contact, whether penetrative or otherwise. In fact, there is no need to mention mouth-to-mouth cases under this rule because non-genital bodily contact falls squarely under \href{https://suttacentral.net/pli-tv-bu-vb-ss2/en/brahmali\#1.2.15.1}{Bu~Ss~2}.

Given the above problems, it seems preferable to understand \textit{magga} in part 2 of the permutation series as restricted to the genitals and the anus, that is, “private part”, whether of the partner or the monastic. The mouth would then be represented by \textit{amagga}, “non-path” or “non-private part”. This understanding fits with how \textit{magga} is used in other contexts. The anus is regularly called \textit{vaccamagga}, “the path of feces”, and the genitals \textit{\textsanskrit{passāvamagga}}, “the path of urine”. The mouth, however, with the curious exception of part 1 of the permutation series in \textit{bhikkhu \textsanskrit{pārājika}} 1, is never called \textit{magga}, whether as a separate word or compounded with some other term. It seems likely that even in part 2 of the present permutation series \textit{magga} would be used in the more established sense.

However, there is one further problem. If we do take \textit{amagga} to mean the mouth, the question arises as to why mouth-to-mouth penetration would be a serious offense, a \textit{thullaccaya}, and not an offense entailing suspension, a \textit{\textsanskrit{saṅghādisesa}}. A possible answer is that this rule does not include intention as a subfactor. In other words, if the mouth-to-mouth contact is motivated by lust, then the action would fall under \textit{\textsanskrit{saṅghādisesa}} 2. If there is no lust—perhaps in the case of mouth-to-mouth resuscitation—it would be a serious offense under this rule.

We still need to explain why the word \textit{magga} would be used in such different ways in two closely connected parts of the same rule. To start with, it seems likely that part 2 of the permutation series is earlier than part 1. Part 2 is short and concise, whereas part 1 is long and elaborate. As a general principle of interpretation, it can be assumed that the short and less elaborate section is likely to be earlier. If this is so, then the threefold classification of \textit{magga} found in part 1 would not yet have existed when part 2 was formulated. In other words, there would have been no contradiction in the text. Later on, when a more detailed exegesis gradually emerged, the shorthand of three \textit{maggas} may have been invented for the purpose of concision. They would no doubt have been aware of the discrepancy, but probably assumed—wrongly it seems!—that it would not be a problem.

I conclude that \textit{magga} is used in slightly different ways in the two sections. In the first part it refers to the three orifices, whereas in the second part it is restricted to the “private parts”. In interpretating \textit{magga} in this way, I agree with Bhikkhu \textsanskrit{Ṭhānissaro}’s “The Buddhist Monastic Code I”.

\subsection*{\textit{Mattika}, \textit{\textsanskrit{cuṇṇa}}, \textit{\textsanskrit{chakaṇa}}: “soap”, “bath powder”, “detergent”}

Where required by context, I render \textit{\textsanskrit{cuṇṇa}}, \textit{mattika}, and \textit{\textsanskrit{chakaṇa}} according to function rather than substance. \textit{\textsanskrit{Cuṇṇa}} is a powder used for a number of purposes, including medical ones, and \textit{mattika} is clay. There are several Canonical passages, however, that show them as the two principal substances for cleaning the body. At \href{https://suttacentral.net/pli-tv-kd1/en/brahmali\#25.12.4}{Kd~1:25.12.4} and a number of similar passages, we find that the two would always be made ready before entering a \textit{\textsanskrit{jantāghara}}, a sauna or hot bathroom. At \href{https://suttacentral.net/pli-tv-bu-vb-pc57/en/brahmali\#7.1.8}{Bu~Pc~57:7.1.8} bathing is defined as bathing with powder and/or clay. At \href{https://suttacentral.net/pli-tv-kd20/en/brahmali\#27.4.5}{Kd~20:27.4.5} and \href{https://suttacentral.net/pli-tv-kd20/en/brahmali\#27.4.12}{Kd~20:27.4.12} the nuns are prohibited from bathing with powder and scented clay, but should use ordinary clay instead. In the commentaries, we find that clay is used for washing one’s hair.\footnote{Sp 3.34: \textit{\textsanskrit{Kesā} panassa sayameva \textsanskrit{mattikaṁ} \textsanskrit{gahetvā} \textsanskrit{dhovitabbā}}. }

When it comes to \textit{\textsanskrit{chakaṇa}}, literally, “dung”, there are several Canonical and commentarial passages that show it was used as cleaning agent.

\begin{quotation}%
\textit{\textsanskrit{Chakaṇa}} is cow dung.\footnote{Sp 3.264: \textit{\textsanskrit{Chakaṇanti} \textsanskrit{gomayaṁ}}. }

%
\end{quotation}

\begin{quotation}%
For the word \textit{\textsanskrit{chakaṇa}}, ‘cow dung’, is said for horses, etc., because of the use in regard to stains.\footnote{Sp‑yoj 3.264: \textit{\textsanskrit{Chakaṇasaddassa} \textsanskrit{assādīnaṁ} malepi pavattanato \textsanskrit{vuttaṁ} “gomayan”ti}. }

%
\end{quotation}

There is a particularly clear passage in the Khandha-\textsanskrit{saṁyutta}:

\begin{quotation}%
“Suppose there is a dirty and stained cloth. The owners take it to a dyer. The dyer kneads it with salt or lye or cow dung and then rinses it in clean water.” (\href{https://suttacentral.net/sn22.89/en/sujato\#12.1}{SN~22.89:12.1})

%
\end{quotation}

A commentarial passage reinforces the point that cow dung is used in this way:

\begin{quotation}%
But the young boy having himself entered the water, having rubbed with cow dung and clay, he should bathe.\footnote{Sp 3.34: \textit{\textsanskrit{Daharakumārako} pana \textsanskrit{sayaṁ} \textsanskrit{udakaṁ} \textsanskrit{otaritvā} \textsanskrit{gomayamattikāhi} \textsanskrit{ghaṁsitvā} \textsanskrit{nahāpetabbo}.} }

%
\end{quotation}

\subsection*{\textit{Mantha}: “cracker”}

See \textit{\textsanskrit{pūva}}.

\subsection*{\textit{\textsanskrit{Māḷa}: “}stilt house”}

The \textit{\textsanskrit{māḷa}} is hard to distinguish from the \textit{\textsanskrit{pāsāda}} and the \textit{hammiya}, the \textit{\textsanskrit{māḷa}} always occurring together with the other two in the Vinaya. In fact, according to the commentary, they are all different kinds of \textit{\textsanskrit{pāsāda}}, that is, different kinds of “stilt houses”:

\begin{quotation}%
“A \textit{\textsanskrit{māḷa}} is a square stilt house with (a roof in) a single peak. A \textit{\textsanskrit{pāsāda}} is a long stilt house. A \textit{hammiya} is a stilt house with a bald roof.”\footnote{Sp 1.482: \textit{\textsanskrit{Māḷoti} \textsanskrit{ekakūṭasaṅgahito} \textsanskrit{caturassapāsādo}. \textsanskrit{Pāsādoti} \textsanskrit{dīghapāsādo}. Hammiyanti \textsanskrit{muṇḍacchadanapāsādo}}. }

%
\end{quotation}

Rather than try to name each of these buildings, which in any case would not be useful from a practical perspective, I have instead grouped them together as “stilt house”. For practical purposes, especially in relation to \href{https://suttacentral.net/pli-tv-bu-vb-np2/en/brahmali\#3.4.1}{Bu NP~2:3.4.1}–3.5.4, what these three buildings have in common is that they do not have an enclosed category, that is, there is nothing to create an enclosure, such a fence or wall. For a discussion of the \textit{\textsanskrit{pāsāda}}, see separate entry in this Appendix.

\subsection*{\textit{Yojana}}

See \textit{sugata}.

\subsection*{\textit{Vadati}: “to correct”}

PED gives “to speak, say, tell” as the meanings of \textit{vadati}, which is far from comprehensive. An important meaning of \textit{vadati}, especially in a Vinaya context, is “to admonish” or “to correct”. This usage, which is common in the Vinaya, is a close parallel to the English expression “to speak to someone”. Here is a clear example:

\begin{quotation}%
“I invite the Sangha concerning what you have seen, heard, or suspect. Please correct (\textit{vadantu}) me, venerables, out of compassion. If I see (a fault), I will make amends.” (\href{https://suttacentral.net/pli-tv-kd4/en/brahmali\#1.14.7}{Kd~4:1.14.7})

%
\end{quotation}

Because “admonish” is sometimes considered formal and perhaps a bit dated, I prefer “correct”.

Closely related to the meaning “correct” is the occasional meaning “to accuse”, for instance, at \textit{bhikkhu aniyata} 1 and 2:

\begin{quotation}%
If she accuses him like this: “I’ve seen you seated, having sexual intercourse with a woman,” and he admits to that, then he is to be dealt with for the offense. (\href{https://suttacentral.net/pli-tv-bu-vb-ay1/en/brahmali\#2.2.4}{Bu~Ay~1:2.2.4})

%
\end{quotation}

In these cases, it is perhaps just an alternative form of \textit{anuvadati}, which always means “to accuse” in the Canonical texts. Sometimes \textit{vadati} can also mean “to ask”:

\begin{quotation}%
The other should ask, “Who is your friend or companion?” (\href{https://suttacentral.net/pli-tv-bu-vb-pc59/en/brahmali\#2.1.25}{Bu~Pc~59})

%
\end{quotation}

\subsection*{\textit{\textsanskrit{Vikappanā}}: “assignment”}

The word \textit{\textsanskrit{vikappanā}} is hard to pin down. What is clear is that it refers to a requisite somehow being transferred to another monastic. The exact nature of this “transfer”, however, has not been well understood. For instance, \textit{\textsanskrit{vikappanā}} is sometimes rendered as “shared ownership”, yet there is little evidence in support of this. SED suggests “a distributor, apportioner” for \textit{vikalpaka}, but again, this does not quite fit the usage in the Vinaya \textsanskrit{Piṭaka}. Yet another rendering favored by some translators is “transfer (of ownership)”. The problem here is that “transfer” suggests a physical change of ownership, which does not seem required by \textit{\textsanskrit{vikappanā}}. To clear this up, let us have a look at a few suggestive contexts from the Vinaya \textsanskrit{Piṭaka}.

To start with, we need to look at the main \textsanskrit{Pātimokkha} rule that deals with \textit{\textsanskrit{vikappanā}}, that is, \textit{bhikkhu \textsanskrit{pācittiya}} 59. According to the \textit{\textsanskrit{vibhaṅga}} to this rule, it is clear that \textit{\textsanskrit{vikappanā}} refers to a real change of ownership. This is what it says:

\begin{quotation}%
\textit{\textsanskrit{Vikappanā}} in the absence of: one should say, “I give this robe-cloth to you for the purpose of \textit{\textsanskrit{vikappanā}}.” The other should ask, “Who is your friend or companion?” One should reply, “So-and-so and so-and-so.” The other should say, “I give it to them (\textit{\textsanskrit{ahaṁ} \textsanskrit{tesaṁ} dammi}). Please use their property (\textit{\textsanskrit{tesaṁ} \textsanskrit{santakaṁ}}), give it away, or do as you like with it.”\footnote{\textit{\textsanskrit{Parammukhāvikappanā} \textsanskrit{nāma} “\textsanskrit{imaṁ} \textsanskrit{cīvaraṁ} \textsanskrit{vikappanatthāya} \textsanskrit{tuyhaṁ} \textsanskrit{dammī}”ti. Tena vattabbo: “ko te mitto \textsanskrit{vā} \textsanskrit{sandiṭṭho} \textsanskrit{vā}”ti? “\textsanskrit{Itthannāmo} ca \textsanskrit{itthannāmo} \textsanskrit{cā}”ti. Tena vattabbo: “\textsanskrit{ahaṁ} \textsanskrit{tesaṁ} dammi, \textsanskrit{tesaṁ} \textsanskrit{santakaṁ} \textsanskrit{paribhuñja} \textsanskrit{vā} vissajjehi \textsanskrit{vā} \textsanskrit{yathāpaccayaṁ} \textsanskrit{vā} \textsanskrit{karohī}”ti}. (\href{https://suttacentral.net/pli-tv-bu-vb-pc59/en/brahmali\#2.1.23}{Bu~Pc~59:2.1.23}) }

%
\end{quotation}

“\textit{\textsanskrit{Vikappanā}} in the absence of” means that the person one is “transferring to” is not present. The conversation above is between the original owner and a middle person. The sentence “I give this robe-cloth to you for the purpose of \textit{\textsanskrit{vikappanā}}” refers to the original owner giving the robe-cloth they want to “transfer” to this middle person. This person is then responsible for completing the \textit{\textsanskrit{vikappanā}} by effecting the transfer with the words “I give it to them”.

Two interesting points emerge from this description. (1) The expressions “I give it to them” and “their property” make it unmistakably clear that we are dealing with a change of ownership. There is no support here for the idea of “shared ownership”. (2) The phrases “please use their property, give it away, or do as you like with it” suggest that the requisite stays in the hands of the original owner. In other words, the idea of a physical “transfer” is equally unsupported.

Putting the two together, we get the impression that the change of ownership is in name only. I will come back to this unusual situation—that is, how there can be a change of ownership without a physical transfer—toward the end of this entry. Before I do so, let us look at some other contexts for \textit{\textsanskrit{vikappanā}} to find out whether they support this initial analysis.

In the Chapter on Robes, we find a ruling on which requisites are to be determined and which are to be dealt with through \textit{\textsanskrit{vikappanā}}:

\begin{quotation}%
“You should determine the three robes, not \textit{\textsanskrit{vikappanā}} them; you should determine the rainy-season robe for the four months of the rainy season and then \textit{\textsanskrit{vikappanā}} it; you should determine the sitting mat, not \textit{\textsanskrit{vikappanā}} it; you should determine a sheet, not \textit{\textsanskrit{vikappanā}} it; you should determine an itch-covering cloth for as long as you’re sick and then \textit{\textsanskrit{vikappanā}} it; you should determine a washcloth, not \textit{\textsanskrit{vikappanā}} it; you should determine a cloth for requisites, not \textit{\textsanskrit{vikappanā}} it.”\footnote{\textit{“\textsanskrit{Anujānāmi}, bhikkhave, \textsanskrit{ticīvaraṁ} \textsanskrit{adhiṭṭhātuṁ} na \textsanskrit{vikappetuṁ}; \textsanskrit{vassikasāṭikaṁ} \textsanskrit{vassānaṁ} \textsanskrit{cātumāsaṁ} \textsanskrit{adhiṭṭhātuṁ}, tato \textsanskrit{paraṁ} \textsanskrit{vikappetuṁ}; \textsanskrit{nisīdanaṁ} \textsanskrit{adhiṭṭhātuṁ} na \textsanskrit{vikappetuṁ}; \textsanskrit{paccattharaṇaṁ} \textsanskrit{adhiṭṭhātuṁ} na \textsanskrit{vikappetuṁ}; \textsanskrit{kaṇḍuppaṭicchādiṁ} \textsanskrit{yāva} \textsanskrit{ābādhā} \textsanskrit{adhiṭṭhātuṁ} tato \textsanskrit{paraṁ} \textsanskrit{vikappetuṁ}; \textsanskrit{mukhapuñchanacoḷaṁ} \textsanskrit{adhiṭṭhātuṁ} na \textsanskrit{vikappetuṁ}; \textsanskrit{parikkhāracoḷaṁ} \textsanskrit{adhiṭṭhātuṁ} na vikappetun”ti.} (\href{https://suttacentral.net/pli-tv-kd8/en/brahmali\#20.2.4}{Kd~8:20.2.4}) }

%
\end{quotation}

It is clear from this that \textit{\textsanskrit{vikappanā}} was used for storage of cloth that was not in use. If storage is the purpose of \textit{\textsanskrit{vikappanā}}, then a physical transfer of the cloth or requisite is presumably not required, perhaps not even desirable. This means that “distribution” or “transference” don’t quite hit the mark, since both of these imply that the new owner gets their hands on the requisite with all the associated powers of ownership, including the ability to use the item as they wish. “Shared ownership” might work, except there is no evidence from any sources that \textit{\textsanskrit{vikappanā}} can actually have this meaning.

\textit{\textsanskrit{Vikappanā}} is also found in the following standard sequence of terms:

\begin{quotation}%
He determines, he does \textit{\textsanskrit{vikappanā}}, he gives away.\footnote{\textit{\textsanskrit{Adhiṭṭheti}, vikappeti, vissajjeti}, e.g. at \href{https://suttacentral.net/pli-tv-bu-vb-np1/en/brahmali\#4.16}{Bu~NP~1:4.16}. }

%
\end{quotation}

Since the last of these refers to a physical change of ownership, \textit{\textsanskrit{vikappanā}} is unlikely to refer to the same thing.

Perhaps the most revealing passage for our purposes is found at \textit{bhikkhu \textsanskrit{pācittiya}} 33, where we find a conversation between the Buddha and Ānanda. The Buddha encourages Ānanda to accept food that is offered to him:

\begin{quotation}%
The Buddha said, “Accept it, Ānanda.”—“I can’t, sir, I’m expecting another meal.”—“Well then, Ānanda, \textit{\textsanskrit{vikappetvā}} that meal to someone else and then receive this.”\footnote{\textit{“\textsanskrit{Gaṇhāhi}, \textsanskrit{ānandā}”ti. “\textsanskrit{Alaṁ}, \textsanskrit{bhagavā}, atthi me \textsanskrit{bhattapaccāsā}”ti. “\textsanskrit{Tenahānanda}, \textsanskrit{vikappetvā} \textsanskrit{gaṇhāhī}”ti.} (\href{https://suttacentral.net/pli-tv-bu-vb-pc33/en/brahmali\#4.4}{Bu~Pc~33:4.4}) }

%
\end{quotation}

The context suggests that Ānanda is to “transfer” that other meal to someone who is not present. If the other person were present, they too would be unable to accept the food that was being offered. Moreover, it is clear from this that the rendering “shared ownership” does not work. Ānanda had to give up the other meal entirely to be able to eat where he was. As a follow up, Ānanda would then presumably be obliged to inform the other person of the meal to ensure the donor was not let down.

To sum up our findings so far, it seems that all contexts for \textit{\textsanskrit{vikappanā}} either support or are compatible with our suggestion that it refers to a change of ownership without a physical transfer. We now need to consider how this is to be understood.

As we have seen above, \textit{bhikkhu \textsanskrit{pācittiya}} 59 includes the following two sentences in the section on “\textit{\textsanskrit{vikappanā}} in the absence of”:

\begin{quotation}%
“I give it to them. Please use their property, give it away, or do as you like with it.”

%
\end{quotation}

Here the intermediary monastic is saying to the original owner that they may do what they wish with the requisite that now has new owners. To get a handle on this, let us first consider the commentarial explanation:

\begin{quotation}%
When this is said: “Please use so-and-so’s property, give it away, or do as you like with it,” it is relinquishment that is meant.\footnote{Sp 1.469: \textit{“\textsanskrit{Itthannāmassa} \textsanskrit{santakaṁ} \textsanskrit{paribhuñja} \textsanskrit{vā} vissajjehi \textsanskrit{vā} \textsanskrit{yathāpaccayaṁ} \textsanskrit{vā} \textsanskrit{karohī}”ti vutte \textsanskrit{paccuddhāro} \textsanskrit{nāma} hoti}. }

%
\end{quotation}

The commentary seems to say that this refers to the new owner relinquishing the cloth so that the old owner may use the requisite as they like. This works technically in that it fits with the phrasing of the actual rule. But it does not work well when the two sentences from the \textit{\textsanskrit{vibhaṅga}} are considered together, that is, “I give it to them” immediately preceding “Please use their property …”. In the Pali these are said by the same person, which can only be the intermediary monastic, not the new owner. Moreover, there is no indication in the Pali, as one would expect if the requisite changed hands, of a period of storage before the relinquishment happens.

We are compelled to look for an alternative explanation. One possibility is that we are dealing with \textit{\textsanskrit{vissāsa}}, “taking on trust”. On this understanding, we have a situation where the intermediary monastic gives the requisite to the new owners and immediately says to the old owner that they may take the item on trust at any time. If this is correct, it becomes clear why, as we find in the rule, the intermediary monastic asks who are the friends and companions of the original owner. The reason for this question is that one can only take on trust from close associates.

This interpretation allows one to do \textit{\textsanskrit{vikappanā}} and then keep the item in one’s own possession, before taking it on trust whenever one needs it. The only formality one needs to go through is the initial establishing of \textit{\textsanskrit{vikappanā}}. For the conditions for \textit{\textsanskrit{vissāsa}}, “taking on trust”, see \href{https://suttacentral.net/pli-tv-kd8/en/brahmali\#19.1.5}{Kd~8:19.1.5}.

We are left with a word that expresses the change of ownership of a requisite, without it necessarily changing hands. In addition, it seems clear that the rendering “shared ownership” does not work. Based on this, I suggest the best translation of \textit{\textsanskrit{vikappanā}} is “assignment”.

\subsection*{\textit{Vidatthi}: “handspan”}

See \textit{sugata}.

\subsection*{\textit{Vinaya}: “training”}

The word \textit{vinaya} has a number of nuances in the Pali Canon. On the one hand it is used in the context of resolving issues within the Sangha, in which case it means something like “resolution”. On the other hand, it is often interpreted as shorthand for the Vinaya \textsanskrit{Piṭaka}, the collection of monastic rules and regulations, that is, the Monastic Law. The latter of these, however, needs to be used with circumspection. It would have taken time for the monastic rules to become a collection in their own right, and thus the Vinaya \textsanskrit{Piṭaka} as a collection would not have existed from the beginning. We need to be careful not to backread a later historical development into the earliest stratum of texts.

The word \textit{vinaya} is ubiquitous in the Canonical texts and as such seems to have existed from the earliest period. If it did not mean Monastic Law in this earliest period, what did it mean? An indication is given by the cognate verb \textit{vineti}, which in several places has the unambiguous meaning “to train”. For instance, in \href{https://suttacentral.net/an4.111/en/sujato\#1.3}{AN~4.111} the verb \textit{vineti} is used to describe the training of a horse. Indeed, in \href{https://suttacentral.net/mn107/en/sujato\#3.3}{MN~107} and \href{https://suttacentral.net/mn125/en/sujato\#15.1}{MN~125}, the Buddha uses the term \textit{vineti} to describe the full training of a monastic all the way to the final goal of awakening. Moreover, at MN 125 the “training” of a monastic is directly compared to the taming or training of a wild elephant. In these cases, \textit{vineti} describes “the training”, that is, the practical side of the Dhamma, the Dhamma being the theory.

That this is an appropriate understanding of \textit{vinaya} more generally is clear from a number of other Canonical usages. Significantly, the compound \textit{dhammavinaya} is sometimes used about non-Buddhist teachings, e.g. in the Chapter on Nuns, where we find the following interesting passage:

\begin{quotation}%
“Just as families with many women and few men are easily robbed by thieves, so too, the \textit{dhammavinaya} doesn’t last long on whatever spiritual path where women are allowed to go forth.” (\href{https://suttacentral.net/pli-tv-kd20/en/brahmali\#1.6.8}{Kd~20:1.6.8})

%
\end{quotation}

We find a number of similar passages throughout the Canon, for instance at \href{https://suttacentral.net/dn16/en/sujato\#5.27.1}{DN~16}. What these passages have in common is that \textit{dhammavinaya} does not, in these cases, refer specifically to the teachings of the Buddha, but to any spiritual teaching. It follows from this that \textit{vinaya} cannot mean the Monastic Law. There is no evidence that other religions or spiritual schools had a Monastic Law that was equivalent to the rules and regulations that we find in the Vinaya \textsanskrit{Piṭaka}. Here is what Bhikkhu \textsanskrit{Ñāṇatusita} has to say:

\begin{quotation}%
The \textit{\textsanskrit{pātimokkha}} in terms of a word, as well as a code of discipline and the recitation of it, is unique to the Buddhist tradition and no other Indian religious traditions, such as the Jain tradition (which has \textit{\textsanskrit{sūtra}}s with rules but no \textsanskrit{Pātimokkha} recitation or the like) have anything corresponding to it; see Dutt 72. (“Analysis of the Bhikkhu Pātimokkha”, p. 46.)

%
\end{quotation}

We can only conclude that \textit{vinaya} in these contexts does not refer to a legal framework or anything of that nature. It must refer to something else, most likely the general idea of “training”, as we have seen above. In fact, this understanding is quite natural. All religions had a certain theoretical framework, their \textit{dhamma}, as well as a practical application of that framework, their \textit{vinaya}. \textit{Dhammavinaya} thus comes to stand for theory and praxis, that is, the whole of the religious life, and thus I render it as “spiritual path”.

Over time, however, as the number of rules and regulations grew, this general sense of “training” got narrowed down to mean a fixed Monastic Law. In this way, Vinaya becomes a reference to the Monastic Law as a collection of scriptures. We find examples of this in the Canonical texts themselves, e.g. in the Chapter on the Cancellation of the Monastic Code:

\begin{quotation}%
Before accusing another, a monk should consider, “Have I properly learned both Monastic Codes in detail; have I analyzed them well, thoroughly mastered them, and investigated them well, both in terms of the rules and their detailed exposition? Is this quality found in me or not?” If it’s not, then when he’s asked, “Where was this said by the Buddha?” he won’t be able to reply. And there will be those who say, “Please learn the Monastic Law first.” (\href{https://suttacentral.net/pli-tv-kd19/en/brahmali\#5.1.20}{Kd~19:5.1.20}–5.1.24)

%
\end{quotation}

Here \textit{vinaya} seems to be equivalent to the two \textsanskrit{Pātimokkhas} together with their analyses, that is, roughly what is now the \textsanskrit{Mahā}-\textsanskrit{vibhaṅga} and the \textsanskrit{Bhikkhunī}-\textsanskrit{vibhaṅga}. We are seeing the beginning of the formation of a collection of texts. The fact that this usage is found in a Canonical text, however, does not necessarily mean that the word \textit{vinaya} is always a reference to the Canonical text. Given the above considerations, it is more likely that the Vinaya \textsanskrit{Piṭaka} itself is a chronologically stratified text, where we can see the development over time of certain terminology. I suggest, then, that the meaning of the word \textit{vinaya} is an example of such development.

\subsection*{\textit{Vibbhamati}: “to disrobe”}

According to PED and SED the general meaning of this word is something like “to go astray”. However, the implied meaning throughout the Vinaya \textsanskrit{Piṭaka} is that one leaves the Sangha, that is, one disrobes. I therefore take this word to express the functional equivalence of disrobing. This is supported by the commentaries:

\begin{quotation}%
\textit{Vibbhamanti}: some became householders.\footnote{Sp 1.435: \textit{\textsanskrit{Vibbhamantīti} ekacce \textsanskrit{gihī} honti}. }

%
\end{quotation}

\begin{quotation}%
\textit{Yadeva \textsanskrit{sā} \textsanskrit{vibbhantā}} means she is no longer a nun because, according to her own preference and choice, she dresses in white.\footnote{Sp 4.434: \textit{Yadeva \textsanskrit{sā} \textsanskrit{vibbhantāti} \textsanskrit{yasmā} \textsanskrit{sā} \textsanskrit{vibbhantā} attano \textsanskrit{ruciyā} \textsanskrit{khantiyā} \textsanskrit{odātāni} \textsanskrit{vatthāni} \textsanskrit{nivatthā}, \textsanskrit{tasmāyeva} \textsanskrit{sā} \textsanskrit{abhikkhunī}, na \textsanskrit{sikkhāpaccakkhānenāti} dasseti.} }

%
\end{quotation}

\subsection*{\textit{\textsanskrit{Vihāra}: “}a (monastic) dwelling”, “meditation”}

A \textit{\textsanskrit{vihāra}} is a dwelling, the idea that it is a monastic dwelling being implied. In later usage, especially in the commentaries, \textit{\textsanskrit{vihāra}} comes to refer to entire monasteries, rather than individual dwellings. However, the commentaries seem to agree that in its early usage the word does refer to a dwelling: \textit{\textsanskrit{Vihāro} nivesanasadiso}, “A \textit{\textsanskrit{vihāra}} is like a house.” (Sp 1.493)

In other contexts, especially in the Suttas, \textit{\textsanskrit{vihāra}} can refer to meditation or specific states of meditation such as the \textit{\textsanskrit{jhānas}}, for instance in the phrase \textit{\textsanskrit{diṭṭhadhammasukhavihāra}}, “a happy (meditation) abiding in this very life”. Similarly, in the Vinaya we encounter the phrase \textit{\textsanskrit{phāsuvihārika}}. \textit{\textsanskrit{Phāsu}} means “comfortable” or “at ease”, and so we have “one who dwells at ease”, that is, “one whose meditation is going well”.

\subsection*{\textit{\textsanskrit{Saṅghāṭi}}: “outer robe”, “upper robe”, “robe”}

A \textit{\textsanskrit{saṅghāṭi}} is normally understood to refer to one of the three robes of a \textit{bhikkhu} or one of the five of a \textit{\textsanskrit{bhikkhunī}}. It can then be translated as “outer robe” to distinguish it from the \textit{\textsanskrit{uttarāsaṅga}}, the upper robe, and the \textit{\textsanskrit{antaravāsaka}}, the lower robe or sarong.

“Outer robe”, however, is not always a satisfactory rendering of \textit{\textsanskrit{saṅghāṭi}}, as can be seen from a number of different contexts in the Vinaya \textsanskrit{Piṭaka}. There are in fact three different uses of the word, ranging in meaning from “robe” in general to “upper robe”—denoting either the \textit{\textsanskrit{uttarāsaṅga}}, \textit{\textsanskrit{saṅghāṭi}}, or both—to “outer robe”, which is the only case when it specifically refers an individual robe. That there is such a diversity in meaning can be seen from the following passages in the Vinaya.

There are two rules in the \textit{\textsanskrit{Bhikkhunī} \textsanskrit{Pātimokkha}} that mention \textit{\textsanskrit{saṅghāṭi}}, \href{https://suttacentral.net/pli-tv-bi-vb-pj8/en/brahmali\#2.1.14}{Bi~Pj~8} and \href{https://suttacentral.net/pli-tv-bi-vb-pc24/en/brahmali\#2.1.5}{Bi~Pc~24}. In the former case the \textsanskrit{Vibhaṅga} explains \textit{\textsanskrit{saṅghāṭi}} as \textit{\textsanskrit{nivatthaṁ} \textsanskrit{vā} \textsanskrit{pārutaṁ} \textsanskrit{vā}}, “dressed below and dressed above”, which are the standard words for dressing in a sarong and an upper robe respectively.\footnote{See for instance \href{https://suttacentral.net/pli-tv-bu-vb-sk1/en/brahmali\#1.16.1}{Sk 1} and \href{https://suttacentral.net/pli-tv-bu-vb-sk2/en/brahmali\#1.3.1}{Sk 2}. } In this case, it seems any of the three or five robes are intended. In the latter case, \textit{\textsanskrit{saṅghāṭi}} is in fact explained as \textit{\textsanskrit{pañca} \textsanskrit{cīvarāni}}, “the five robes”, that is, any of the five robes used by \textit{\textsanskrit{bhikkhunīs}}. From this I conclude that in certain contexts \textit{\textsanskrit{saṅghāṭi}} should simply be rendered as “robe”.

\textsanskrit{Bhikkhunī} \textsanskrit{Vimalañāṇī} (private communication) has shown that parallel rules preserved in other Vinayas “strongly supports” the idea that \textit{\textsanskrit{saṅghāṭi}} often just means “robe”. The following is a summary of her findings.

There are two Pali \textit{\textsanskrit{bhikkhunī}} rules that mention the \textit{\textsanskrit{saṅghāṭi}}: Bi Pj 8 and Bi Pc 24.

\begin{itemize}%
\item The Chinese parallels for Bi Pj 8 all say “robe”, except the \textsanskrit{Mūlasarvāstivāda}, which in both Chinese and Tibetan does not mention this factor. In addition, the Hybrid Sanskrit text of the \textsanskrit{Lokuttaravāda} also says “robe” (\textit{\textsanskrit{cīvara}}).%
\item As for Bi Pc 24, four Vinayas (\textsanskrit{Mahīśāsaka}, \textsanskrit{Sarvāstivāda}, and \textsanskrit{Mūlasarvāstivāda} in both Chinese and Tibetan) say “robes” (although there is some confusion in the translation of the Tibetan). The Dharmaguptaka is alone in specifying a \textit{\textsanskrit{saṅghāṭi}}; it says that other robes incur a \textit{\textsanskrit{dukkaṭa}} offense. The \textsanskrit{Lokuttaravāda} and \textsanskrit{Mahāsāṁghika} do not have this rule.%
\end{itemize}

Conversely, there are two rules that mention \textit{\textsanskrit{saṅghāṭi}} in some other Vinayas, but \textit{\textsanskrit{cīvara}} in the Pali.

\begin{itemize}%
\item Pali Bi Pc 23 mentions the \textit{\textsanskrit{cīvara}}, as do the \textsanskrit{Mahīśāsaka} and \textsanskrit{Sarvāstivāda} parallels. However, the Dharmaguptaka, \textsanskrit{Lokuttaravāda}, and \textsanskrit{Mahāsāṁghika} parallels have \textit{\textsanskrit{saṅghāṭi}}. The \textsanskrit{Mūlasarvāstivāda} omits this rule.%
\item Pali Bi NP 3 also includes \textit{\textsanskrit{cīvara}}, followed by the Dharmaguptaka, \textsanskrit{Mahīśāsaka}, and \textsanskrit{Sarvāstivāda}. The \textsanskrit{Lokuttaravāda} and \textsanskrit{Mahāsāṁghika}, however, have \textit{\textsanskrit{saṅghāṭi}}, with the \textsanskrit{Lokuttaravāda} word definition identifying the two (\textit{\textsanskrit{saṁghāṭī} ti \textsanskrit{cīvaraṁ}}). Again, the \textsanskrit{Mūlasarvāstivāda} has no parallel.%
\end{itemize}

We see two patterns in the Chinese translations of the Vinaya. First, that \textit{\textsanskrit{saṅghāṭi}} is often defined as \textit{\textsanskrit{cīvara}}. Second, where one version of the Vinaya uses \textit{\textsanskrit{saṅghāṭi}} another may well use \textit{\textsanskrit{cīvara}}. Both of these patterns suggest that the two were sometimes used interchangeably. (I take it that “robe” is a translation of a Chinese term that ultimately refers to \textit{\textsanskrit{cīvara}}.)

Now let us look at the other meanings of \textit{\textsanskrit{saṅghāṭi}.} In the Khandhakas we find \textit{\textsanskrit{saṅghāṭi}} used in the plural, but in a context that excludes the sarong. In Kd 1, in the context of a student’s duties to their preceptor or teacher, the student puts the two \textit{\textsanskrit{saṅghāṭis}} together and then hands them as one to the teacher (\href{https://suttacentral.net/pli-tv-kd1/en/brahmali\#25.9.1}{Kd~1:25.9.1}). This cannot refer to the sarong because the teacher is specifically said to have put it on just before. The student is then said to dress in the same manner. Moreover, in \href{https://suttacentral.net/pli-tv-bi-vb-pc96/en/brahmali\#1.3}{Bi~Pc~96}, the \textit{\textsanskrit{saṅghāṭis}} (plural) of a \textit{\textsanskrit{bhikkhunī}} are lifted up by a whirlwind and the upper part of her body is exposed, while nothing is said of the lower part.

In addition to this, there are a few of contexts where the \textit{\textsanskrit{saṅghāṭi}} is specifically distinguished from the other two robes, the \textit{\textsanskrit{uttarāsaṅga}} and the \textit{\textsanskrit{antaravāsaka}}, for instance at \href{https://suttacentral.net/pli-tv-bu-vb-np2/en/brahmali\#3.1.5}{Bu~NP~2:3.1.5}.

We are thus left with three different meanings of the word \textit{\textsanskrit{saṅghāṭi}}. In practice, there are a number of passages in the Vinaya \textsanskrit{Piṭaka} where the precise meaning of \textit{\textsanskrit{saṅghāṭi}} cannot be decided with certainty. In such cases one needs to make the most of the context to decide which meaning is the most likely. To sum up, I translate \textit{\textsanskrit{saṅghāṭi}} in three different ways, as either “robe”, “upper robe”, or “outer robe”. And I use the context to decide which one is appropriate in each particular case.

There is perhaps one additional criterion that can be used in rendering this term. In the above analysis it can be seen that \textit{\textsanskrit{saṅghāṭi}} ranges from the generic term “robe” via the less general “upper robe” to the specific “outer robe”. It seems reasonable to assume that these three meanings were not used interchangeably at all historical stages, but that a certain meaning predominated at any specific time. The question, then, is whether we can discern a development in usage over time.

It seems likely to me that the more general meaning would have been the earlier one, whereas the more specific meanings developed later. It is natural for any institution such as Buddhism to start off with a less specialized vocabulary, mostly inherited from the broader society, including existing religions, and then gradually develop its own specialized terminology. This seems to be no more than the natural way in which any organization tends to develop. If this is correct, then the more generic meaning of \textit{\textsanskrit{saṅghāṭi}} would be the earlier one, whereas the most specific meaning would be the latest. We may then ascribe the more generic meaning to the earliest parts of the texts, such as the \textsanskrit{Pātimokkha} rules, and the more specialized meaning to the later parts, such as the explanatory material. This gives an additional criterion for deciding the translation at any particular point.

\subsection*{\textit{Sattu}: “flour products”}

“Flour products” renders \textit{sattu}. \textit{Sattu} is sometimes translated as “flour” or “meal”. Basing myself on the commentary, I take it that flour products, such as baked goods, are included in \textit{sattu}:

\begin{quotation}%
\textit{Sattu}: means flour made of rice or cereals. Having roasted millet, beans, or \textit{\textsanskrit{kudrūsa}}, having pounded a little, having removed the husk, again having pounded strongly, they make powder. Also, if because of wetness it is lumped into one, it is still reckoned as \textit{sattu}.\footnote{Sp 2.238: \textit{Sattu \textsanskrit{nāma} \textsanskrit{sālivīhiyavehi} katasattu. \textsanskrit{Kaṅguvarakakudrūsakasīsānipi} \textsanskrit{bhajjitvā} \textsanskrit{īsakaṁ} \textsanskrit{koṭṭetvā} thuse \textsanskrit{palāpetvā} puna \textsanskrit{daḷhaṁ} \textsanskrit{koṭṭetvā} \textsanskrit{cuṇṇaṁ} karonti. Sacepi \textsanskrit{taṁ} \textsanskrit{allattā} \textsanskrit{ekābaddhaṁ} hoti, \textsanskrit{sattusaṅgahameva} gacchati}. }

%
\end{quotation}

Commenting on Bu Pc 34, the sub-commentary Kkh-\textsanskrit{pṭ} says, \textit{\textsanskrit{sattūti} baddhasattu, abaddhasattu ca}, “sattu means bound flour and unbound flour”, which would seem to refer to products made of flour and loose flour, respectively. Also, commenting on Bu Pc 34, another sub-commentary adds:

\begin{quotation}%
\textit{Sattu}: the taking of flour, whether made into goods or not; sesame, etc., which are characterized by this.\footnote{Vin-vn-\textsanskrit{ṭ} 1233: \textit{\textsanskrit{Sattūti} \textsanskrit{baddhasattuabaddhasattūnaṁ} \textsanskrit{gahaṇaṁ}, \textsanskrit{imināva} \textsanskrit{tilādīni} \textsanskrit{upalakkhitāni}.} }

%
\end{quotation}

\subsection*{\textit{Santhata}: “blanket”}

The \textit{santhata} is a cloth requisite that is used as an underlay for sitting or lying on,\footnote{\textit{\textsanskrit{Nisīdanasanthataṁ}}, “sitting blanket”, at \href{https://suttacentral.net/pli-tv-bu-vb-np15/en/brahmali\#1.3.10.1}{Bu~NP~15}. Sp 1.567: \textit{\textsanskrit{Sakiṁ} nivatthampi \textsanskrit{sakiṁ} \textsanskrit{pārutampīti} \textsanskrit{sakiṁ} \textsanskrit{nisinnañceva} \textsanskrit{nipannañca}}, “Even worn once: even sat down once, or lain down once.” } as a blanket to keep one warm,\footnote{\textit{\textsanskrit{Mayhañca} \textsanskrit{vinā} \textsanskrit{santhatā} na \textsanskrit{phāsu} hoti}, “I am not comfortable without my \textit{santhata}.” (\href{https://suttacentral.net/pli-tv-bu-vb-np14/en/brahmali\#2.10}{Bu~NP~14:2.10}) } or as a fourth robe.\footnote{\textit{\textsanskrit{Purāṇasanthataṁ} \textsanskrit{nāma} \textsanskrit{sakiṁ} nivatthampi \textsanskrit{sakiṁ} \textsanskrit{pārutampi}}, “An old blanket: even worn once,” at \href{https://suttacentral.net/pli-tv-bu-vb-np15/en/brahmali\#2.7}{Bu~NP~15:2.7}. Sp 1.566: \textit{\textsanskrit{Santhatāni} \textsanskrit{ujjhitvāti} santhate \textsanskrit{catutthacīvarasaññitāya} \textsanskrit{sabbasanthatāni} \textsanskrit{ujjhitvā}}, “They discarded their blankets: because they perceived the \textit{santhata} as a fourth robe, they discarded all blankets.” } The commentarial explanation of how a \textit{santhata} is made seems to suggest it was quite thick:

\begin{quotation}%
It is made by spreading layer upon layer (\textit{\textsanskrit{uparūpari}}) of silk thread on an even stretch of ground, and sprinkling with rice water etc.\footnote{Sp 1.542: \textit{Same \textsanskrit{bhūmibhāge} \textsanskrit{kosiyaṁsūni} \textsanskrit{uparūpari} \textsanskrit{santharitvā} \textsanskrit{kañjikādīhi} \textsanskrit{siñcitvā} \textsanskrit{kataṁ} hoti}. }

%
\end{quotation}

However, thickness alone is not what distinguishes a \textit{santhata} from a \textit{\textsanskrit{cīvara}}, the usual word for a monastic robe. In the Vinaya there is no limit to the thickness or material quality of a \textit{\textsanskrit{cīvara}}, which means that all types of woven cloth can be classified under this word. The reason for the separate category of a \textit{santhata}, then, would seem to be that it is not woven cloth. Indeed, in the Vinaya \textsanskrit{Piṭaka} the \textit{santhata} is always defined by its method of production, which, as we have just seen, is then described in greater detail in the commentary. It is a kind of unwoven cloth.

Yet the specific way a \textit{santhata} was made does not seem to be in use in modern societies. It follows that defining a \textit{santhata} strictly according to the way it is manufactured would make all rules relating to it defunct, specifically \href{https://suttacentral.net/pli-tv-bu-vb-np11/en/brahmali\#1.23.1}{Bu~NP~11}–15. It therefore seems preferable to classify it according to its usage. It is for this reason I have chosen the rendering “blanket”, which means that these rules are relevant also in a modern context.

\subsection*{\textit{\textsanskrit{Samānasaṁvāsa}/\textsanskrit{samānasaṁvāsako}: “}who belongs to the same community” “one who belongs to the same Buddhist sect”}

\textit{\textsanskrit{Samānasaṁvāsaka}} (and \textit{\textsanskrit{nānāsaṁvāsaka}}) need to be carefully distinguished from \textit{\textsanskrit{samānasaṁvāsa}} (and \textit{\textsanskrit{nānāsaṁvāsa}}). Only the former means “one belonging to the same Buddhist sect”. The latter means “belonging to the same community”, as decided by \textit{\textsanskrit{sīmās}}.

\subsection*{\textit{Sambahula}: “several”, “three”}

\textit{Sambahula} normally means “many” or “a number of”. In the Vinaya \textsanskrit{Piṭaka}, however, it is also used technically to mean a number of monastics greater than one but less than a \textit{sangha}, in other words, two or three monastics. For instance, in the \textit{nissaggiya \textsanskrit{pācittiya}} rules, a monastic who has breached a rule relinquishes the item in question to a single monastic, to \textit{sambahula} monastics, or to a \textit{sangha}. In these and similar instances I render \textit{sambahula} as “several”.

In the Chapter Connected with \textsanskrit{Campā} (Kd 9), \textit{sambahula} is used in an even more restricted sense. Here it refers to more than two monastics but less than a \textit{sangha}, that is, exactly three monastics. In this context I render it simply as “three”.

\subsection*{\textit{\textsanskrit{Sālā}}: “building”}

\textit{\textsanskrit{Sālā}} is etymologically related to the English word “hall” and is often translated as such. In some contexts, this may be appropriate, but \textit{\textsanskrit{sālā}} is in fact used more broadly than “hall”, referring to buildings of all sizes. At the lower end of the spectrum, it refers to small buildings, probably no more than sheds, for instance, the \textit{\textsanskrit{aggisālā}}, “water-boiling shed”, the \textit{\textsanskrit{udapānasāla}}, “well house”, and the \textit{\textsanskrit{pānīyasālā}}, “drinking-water shed”. Buildings that were presumably intermediate in size, were also called \textit{\textsanskrit{sālā}}, such as the \textit{\textsanskrit{kathinasālā}}, “sewing shed”, the \textit{\textsanskrit{jantāgharasālā}}, “sauna”, and the \textit{\textsanskrit{āpaṇasālā}}, “shop”. \textit{\textsanskrit{Sālā}} also includes buildings that would have been quite substantial, such as the \textit{\textsanskrit{upaṭṭhānasālā}}, “assembly hall”, and the \textit{\textsanskrit{mahāsālās}}, “large houses”, of wealthy people. In this latter usage \textit{\textsanskrit{sālā}} seems to be indistinguishable from a “house”. We may deduce from this that the \textit{\textsanskrit{kūṭāgārasālā}} of \textsanskrit{Vesālī}, where the Buddha often stayed, should probably be rendered simply as “the house with a peaked roof”. In addition to this, \textit{\textsanskrit{sālā}} covers large but simple buildings, such as \textit{\textsanskrit{caṅkamasālā}}, “covered walking path”, \textit{\textsanskrit{gosālā}}, “cow shed”, \textit{\textsanskrit{assasālā}}, “horse stable”, and \textit{\textsanskrit{hatthisālā}}, “elephant stable”.

Apart from the issue of size, there is no indication that a \textit{\textsanskrit{sālā}} had to consist of a single room, as implied by the rendering “hall”. Another piece of evidence is found in \href{https://suttacentral.net/pli-tv-bu-vb-pc31/en/brahmali\#3.1.6}{Bu~Pc~31} where the \textit{\textsanskrit{sālā}} as a venue for eating is contrasted with the \textit{\textsanskrit{maṇḍapa}}, “roof cover”, and open-air venues such as the \textit{\textsanskrit{rukkhamūla}}, “the foot of a tree”, and \textit{\textsanskrit{ajjhokāsa}}, “out in the open”. In this last instance, \textit{\textsanskrit{sālā}} seems to be a generic reference to an indoor venue. Given this diversity in usage, it seems \textit{\textsanskrit{sālā}}, in its generic sense, may best be rendered as “building”. In more specific circumstances, I render it according to context.

\subsection*{\textit{\textsanskrit{Sīmā}}: “(monastery) zone”}

\textit{\textsanskrit{Sīmā}} is commonly rendered as “boundary”. It is perhaps worth noting straightaway that there is another Pali word that unambiguously means “boundary”, namely, \textit{\textsanskrit{mariyāda}}. With \textit{\textsanskrit{sīmā}}, however, the situation is more equivocal. There is in fact some evidence in the Canonical texts that “area” is closer to the mark. At Kd 2, we find the following rule:

\begin{quotation}%
“A whole river, a whole ocean, or a whole lake cannot be a \textit{\textsanskrit{sīmā}} in its own right.” (\href{https://suttacentral.net/pli-tv-kd2/en/brahmali\#12.7.3}{Kd~2:12.7.3})

%
\end{quotation}

In these cases, it is more intuitive to think of the \textit{\textsanskrit{sīmā}} as the area of the lake, etc., not as its boundary. A little bit earlier in the same \textit{khandhaka}, we find the following:

\begin{quotation}%
“You shouldn’t establish a \textit{\textsanskrit{sīmā}} that is too large, whether 50, 65, or 80 kilometers across. If you do, you commit an offense of wrong conduct. You should establish a \textit{\textsanskrit{sīmā}} that is 40 kilometers across at the most.” (\href{https://suttacentral.net/pli-tv-kd2/en/brahmali\#7.1.6}{Kd~2:7.1.6})

%
\end{quotation}

It is not immediately obvious what the distances here refer to. If a \textit{\textsanskrit{sīmā}} is to be regarded as a “boundary”, then it would be natural to see these distances as the length of the boundary. According to the commentary, however, the distances here are the maximum lengths \emph{across} the \textit{\textsanskrit{sīmā}}:

\begin{quotation}%
Three \textit{yojanas} at the most: here three \textit{yojanas} at the most is its measure. This is three \textit{yojanas} at the most. One who is establishing (a \textit{\textsanskrit{sīmā}}), standing in the middle, should establish (a \textit{\textsanskrit{sīmā}}) that is one and a half \textit{yojanas} in the four directions. If, standing in the middle, one makes it three \textit{yojanas} in each direction, it will be six \textit{yojanas}, which is not allowable. One who is establishing (a \textit{\textsanskrit{sīmā}}) that is quadrangular or triangular should establish (a \textit{\textsanskrit{sīmā}}) that is three \textit{yojanas} corner to corner. If one exceeds three \textit{yojanas} even by a hair’s breadth on any side, one commits an offense and the \textit{\textsanskrit{sīmā}} is not actually a \textit{\textsanskrit{sīmā}}.\footnote{Sp 3.140: \textit{Tiyojanaparamanti ettha \textsanskrit{tiyojanaṁ} \textsanskrit{paramaṁ} \textsanskrit{pamāṇametissāti} \textsanskrit{tiyojanaparamā}; \textsanskrit{taṁ} \textsanskrit{tiyojanaparamaṁ}. Sammannantena pana majjhe \textsanskrit{ṭhatvā} \textsanskrit{yathā} \textsanskrit{catūsupi} \textsanskrit{disāsu} \textsanskrit{diyaḍḍhadiyaḍḍhayojanaṁ} hoti, \textsanskrit{evaṁ} \textsanskrit{sammannitabbā}. Sace pana majjhe \textsanskrit{ṭhatvā} ekekadisato \textsanskrit{tiyojanaṁ} karonti, \textsanskrit{chayojanaṁ} \textsanskrit{hotīti} na \textsanskrit{vaṭṭati}. \textsanskrit{Caturassaṁ} \textsanskrit{vā} \textsanskrit{tikoṇaṁ} \textsanskrit{vā} sammannantena \textsanskrit{yathā} \textsanskrit{koṇato} \textsanskrit{koṇaṁ} \textsanskrit{tiyojanaṁ} hoti, \textsanskrit{evaṁ} \textsanskrit{sammannitabbā}. Sace hi yena kenaci pariyantena kesaggamattampi \textsanskrit{tiyojanaṁ} \textsanskrit{atikkāmeti}, \textsanskrit{āpattiñca} \textsanskrit{āpajjati} \textsanskrit{sīmā} ca \textsanskrit{asīmā} hoti.} }

%
\end{quotation}

Again, this suggests that “area” is a better fit for \textit{\textsanskrit{sīmā}} than “boundary”. In fact, the word “zone” is probably more accurate than “area”. A zone is normally understood as an area used for a specific purpose, in this case to delimit the precise extent of a monastery.

A complicating factor is that the word \textit{\textsanskrit{sīmā}} is used for all sorts of areas, not just those concerned with monasteries. For instance, there are \textit{\textsanskrit{gāmasīmās}}, “village zones” or “zones of inhabited areas”. To avoid confusion, when \textit{\textsanskrit{sīmā}} refers specifically to the area of a monastery, I therefore usually render it as “monastery zone”, or where the meaning is self-evident, simply as “zone”.

\subsection*{\textit{Sugata}: “standard”}

In contrast to the Suttas, where \textit{sugata} is used as an epithet of the Buddha, in the Vinaya \textsanskrit{Piṭaka} it is mostly used in combination with various measures, such as the \textit{sugatavidatthi}, the “\textit{sugata} handspan”. This might be rendered literally as the handspan of the Buddha, but the question is whether this is too literal. The Vinaya \textsanskrit{Piṭaka} contains the same measures both with and without the prefix \textit{sugata}. These measures are based on the body, in particular the \textit{\textsanskrit{aṅgula}} (the “finger-breadth”), the \textit{vidatthi}, (the “handspan”), and the \textit{hattha}, (the “forearm”), the latter being found mostly in later literature. Measures based purely on the body are obviously going to be quite imprecise. It would seem likely, then, that \textit{sugata} was added to give more precision to these measures, that is, to give a standard. The question is what kind of standard.

It might seem natural to conclude that the \textit{sugata} standard is a direct reference to the bodily measures of the Buddha. Yet, is it likely that the monastic community actually measured the Buddha? And even if it did, how long would such measures have survived after the Buddha’s death? Could it be that the \textit{sugata} measures instead were standards \emph{laid down} by the Buddha, or at least attributed to him? This latter possibility gains some credence when we consider the following passage from \textsanskrit{Kauṭilya}’s \textsanskrit{Arthaśāstra}:

\begin{quotation}%
1 \textit{angula} [‘finger-breadth’] or the middlemost joint of the middle finger of a man of medium size may be taken to be equal to an \textit{angula}. … 12 \textit{angulas} are equal to 1 \textit{vitasti} … 2 \textit{vitastis} are equal to 1 \textit{aratni} or 1 \textit{\textsanskrit{prājāpatya} hasta}.\footnote{\textsanskrit{Kauṭilya} \textsanskrit{Arthaśāstra} 2.20.07–12: \textit{Madhyamasya \textsanskrit{puruṣasya} \textsanskrit{madhyamāyā} \textsanskrit{anugulyā} \textsanskrit{madhyaprakarṣo} \textsanskrit{vāṅgulam} … \textsanskrit{dvādaśaaṅgulā} vitastiḥ … dvivitastiraratniḥ \textsanskrit{prājāpatyo} hastaḥ.} }

%
\end{quotation}

These measures, including the relationships between them, correspond closely to what we find in the Vinaya \textsanskrit{Piṭaka}, the exception being that \textit{sugata} is replaced by \textit{\textsanskrit{prājāpati}}. According to Patrick Olivelle,\footnote{Patrick Olivelle, 2013. } the \textsanskrit{Kauṭilya} \textsanskrit{Arthaśāstra} was composed and edited in the period 2nd century BCE to 3rd century CE. These dates overlap reasonably well with dates of composition and editing of the Vinaya \textsanskrit{Piṭaka}, especially if we assume that the ideas described in the \textsanskrit{Arthaśāstra} existed in Indian culture prior to its composition. It seems plausible, then, that the \textit{sugata} measures were modeled on the existing Indian tradition.

As can be seen from the above quote, the existing tradition based its measures on \textit{\textsanskrit{Prājāpati}}, the ancient Indian creator god. Clearly this cannot refer to the actual measures of this god, but rather must refer to a standard. It could be, for instance, that these measures were seen as \emph{laid down} by \textit{\textsanskrit{Prājāpati}}. Or it could be that they were regarded as the actual measures of \textit{\textsanskrit{Prājāpati}}, but even so it would be a standard, since no actual measurement of \textit{\textsanskrit{Prājāpati}} would be possible.

I am suggesting the \textit{sugata} measures were based on this pre-existing norm and should be understood through the same paradigm: as measures \emph{laid down} by the Buddha. In this way they become standards, which were not necessarily directly associated with or even related to the physical measures of the Buddha.

It is in this light, I believe, that the commentarial understanding of the \textit{sugata} measures being three times the size of an ordinary man must be understood.\footnote{Sp 1.348: \textit{\textsanskrit{Sugatavidatthiyāti} sugatavidatthi \textsanskrit{nāma} \textsanskrit{idāni} majjhimassa purisassa tisso vidatthiyo}, “The standard handspan: three handspans of an average person now are called a standard handspan.” } If taken literally, this idea makes a mockery of the Pali commentaries. Anyone who knows the early Suttas will be aware that such ideas are flagrantly opposed to how the Buddha is portrayed. It is hard to imagine that the commentaries were unaware of this. It seems more likely that the commentaries regarded the \textit{sugata} measures as standards, which then explains, at least in part, how they could end up with such large numbers.

Given the uncertainties involved, I translate \textit{sugata-vidatthi} as “standard handspan” and \textit{sugata-\textsanskrit{aṅgula}} as “standard fingerbreadth”. Rather than come up with precise numbers, it seems that broader criteria from the Vinaya—such as not being indulgent or luxurious—should be used to decide what is appropriate in specific instances. Additionally, the \textit{sugata} measures are usually used to denote maximum allowable sizes, and as such they must have been quite ample to be suitable for all monastics. Both the Buddha and \textit{\textsanskrit{Prājāpati}} can perhaps be regarded as kings writ large, in the sense that “king size” normally denotes a particularly large measure. I would therefore propose that the \textit{sugata} measures should be understood in such a way that all monastics would be able to live comfortably within them.

For those who prefer exact numbers, I will provide a brief example of how some of the ancient Indian measures might be estimated, if only very approximately.

Apart from the \textit{sugata} measures, there are also a significant number of cases where measures of length are not prefixed with \textit{sugata}. To get an estimate of these measures, I start by assuming that they are the measures of an average man, as suggested by the commentaries:

\begin{quotation}%
The standard handspan (\textit{sugatavidatthi}): three handspans (\textit{vidatthi}) of an average (\textit{majjhima}) person now are called a standard handspan.\footnote{Sp 1.348: \textit{\textsanskrit{Sugatavidatthiyāti} sugatavidatthi \textsanskrit{nāma} \textsanskrit{idāni} majjhimassa purisassa tisso vidatthiyo}. Moreover, when \textit{vidatthi} is used without the prefix \textit{sugata}, it means the \textit{vidatthi} of an average person. Sp‑\textsanskrit{ṭ} 1.462: \textit{Majjhimassa purisassa \textsanskrit{vidatthiṁ} \textsanskrit{sandhāya} “dve vidatthiyo”\textsanskrit{tiādi} \textsanskrit{vuttaṁ}}, “Two handspans (\textit{vidatthi}) was said with reference to the handspan of an average man.” }

%
\end{quotation}

\begin{quotation}%
Eight fingerbreadths of the standard fingerbreadth: three fingerbreadths of an average man now are here called a standard fingerbreadth.\footnote{Kkh-\textsanskrit{pṭ}: \textit{\textsanskrit{Aṭṭhaṅgulaṁ} \textsanskrit{sugataṅgulenāti} ettha \textsanskrit{sugataṅgulaṁ} \textsanskrit{nāma} \textsanskrit{idāni} majjhimassa purimassa \textsanskrit{tīṇi} \textsanskrit{aṅgulāni}, tena \textsanskrit{sugataṅgulena} \textsanskrit{aṭṭhaṅgulaṁ} \textsanskrit{vaḍḍhakihatthappamāṇanti} attho}. }

%
\end{quotation}

Next, we need an estimate of the height of an average man. According to research at the University of Tuebingen,\footnote{Https://ourworldindata.org/grapher/average-height-of-men-for-selected-countries?country=IND. } an average Indian man in 1840 was 160 centimeters tall. According to the website “Our World in Data”,\footnote{“Over the last two millennia, human height, based on skeletal remains, has stayed fairly steady, oscillating around 170 cm. With the onset of modernity, we see a massive spike in heights in the developed world.” Https://ourworldindata.org/human-height. } human height was essentially stable before the modern era, which means we can assume, very roughly, that the average height of an Indian man at the time of the Buddha was 160 centimeters. Additionally, if we follow the commentarial tradition that the forearm (\textit{hattha}) is one quarter the length of a person’s height,\footnote{Sp‑\textsanskrit{ṭ} 1.33: \textit{\textsanskrit{Majjhimappamāṇoti} catuhattho puriso \textsanskrit{majjhimappamāṇo}}, “Average measure: a man of four forearms is the average measure.” } we have a forearm measure of 40 cm. Based on this, we can estimate other measures, as follows:

\begin{itemize}%
\item \textit{Vidatthi} = 20 cm. The \textit{vidatthi}, the “hand-span”, is half the length of a \textit{hattha}.%
\item \textit{\textsanskrit{Aṅgula}} = 1.67 cm, which gives roughly 3.5 cm for 2 \textit{\textsanskrit{aṅgulas}} and 7 cm for 4 \textit{\textsanskrit{aṅgulas}}. The \textit{\textsanskrit{aṅgula}}, the “finger-breadth”, is one twelfth of a \textit{vidatthi}.%
\item \textit{Dhanu} = 1.6 m, which gives 800 m for 500 \textit{dhanus}. The \textit{dhanu}, the “bow-length”, is equal to four \textit{hatthas}.\footnote{Sp‑\textsanskrit{ṭ} 1.92: \textit{Ācariyadhanu \textsanskrit{nāma} pakatihatthena \textsanskrit{navavidatthippamāṇaṁ}, \textsanskrit{jiyāya} pana \textsanskrit{āropitāya} \textsanskrit{catuhatthappamāṇa}}, “A measure of nine hand-spans of an ordinary hand, or a measure of four forearms when the bowstring is attached, this is called a teacher’s bow length.” According to Sp 1.92 a teacher’s bow length is the standard measure for bow lengths: \textit{\textsanskrit{Ācariyadhanunā} \textsanskrit{pañcadhanusatappamāṇanti} \textsanskrit{veditabbaṁ}}, “A measure of five hundred bow-lengths is to be understood according to a teacher’s bow length.” SED (sv. \textit{dhanu}) says the same, that is, one bow length is equivalent to four \textit{hatthas}, or four lengths of the forearm. }%
\item \textit{Abbhantara} = 11.2 m, which gives almost 80 m for 7 \textit{abbhantaras}.\footnote{Sp 1.489: \textit{\textsanskrit{Ekaṁ} \textsanskrit{abbhantaraṁ} \textsanskrit{aṭṭhavīsatihatthaṁ} hoti}, “One \textit{abbhantara} is twenty-eight \textit{hatthas}.” } The \textit{abbhantara} is equal to 28 \textit{hatthas}.%
\end{itemize}

Then there is the \textit{yojana}, an important measure of distances related to travel on land. Unfortunately, there is much uncertainty about this measure, and perhaps it never was a precise distance. Still, in his book on ancient Indian measures, T. W. Rhys Davids estimates the \textit{yojana} at 8 miles or approximately 13 kilometers.\footnote{See T. W. Rhys Davids, “On the Ancient Coins and Measures of Ceylon: with a discussion of the Ceylon date of the Buddha’s death”, p. 16. } I follow Rhys Davids, but due to the uncertainties involved, I use round numbers when calculating distances of two or more \textit{yojanas}.

In sum, we have the following estimates for the various measures used in the Vinaya \textsanskrit{Piṭaka}:

\begin{itemize}%
\item \textit{\textsanskrit{Aṅgula}} = 1.67 cm%
\item \textit{\textsanskrit{Sugataṅgula}} = 5 cm%
\item \textit{Vidatthi} = 20 cm%
\item \textit{Sugatavidatthi} = 60 cm%
\item \textit{Hattha} = 40 cm%
\item \textit{Dhanu} = 1.8 m%
\item \textit{Abbhantara} = 11.2 m%
\item \textit{Yojana} = 13 km.%
\end{itemize}

\subsection*{\textit{\textsanskrit{Suvaṇṇa}}: “gold”}

See \textit{\textsanskrit{hirañña}}.

\subsection*{\textit{\textsanskrit{Senāsana}}: “resting place”, “dwelling”, “furniture”}

\textit{\textsanskrit{Senāsana}}, literally “bed and seat”, refers broadly to any kind of “resting place”, ranging from huts to furniture and bedding, including even resting places at the foot of a tree.\footnote{For huts see \href{https://suttacentral.net/pli-tv-kd16/en/brahmali\#1.1.2}{Kd~16:1.1.2}, for furniture \href{https://suttacentral.net/pli-tv-bu-vb-pc14/en/brahmali\#1.1.5}{Bu Pc14:1.1.5}, for bedding \href{https://suttacentral.net/pli-tv-bu-vb-pc15/en/brahmali\#1.5}{Bu Pc15:1.5}, and for resting places at the foot of a tree \href{https://suttacentral.net/pli-tv-kd1/en/brahmali\#77.1.10}{Kd~1:77.1.10}. } It can also mean an individual resting place within a single dwelling (\href{https://suttacentral.net/pli-tv-kd5/en/brahmali\#13.8.3}{Kd~5:13.8.3}–13.8.6). This broad range of meaning is confirmed by the sub-commentary:

\begin{quotation}%
There are four kinds of \textit{\textsanskrit{senāsana}}: the dwelling \textit{\textsanskrit{senāsana}}, the bed-and-bench \textit{\textsanskrit{senāsana}}, the bedding (or “mat”) \textit{\textsanskrit{senāsana}}, the place \textit{\textsanskrit{senāsana}.} … Wherever monks retire to, all that is called “\textit{\textsanskrit{senāsana}}”.\footnote{Sp‑\textsanskrit{ṭ} 3.294: \textit{\textsanskrit{Catubbidhañhi} \textsanskrit{senāsanaṁ} \textsanskrit{vihārasenāsanaṁ} \textsanskrit{mañcapīṭhasenāsanaṁ} \textsanskrit{santhatasenāsanaṁ} \textsanskrit{okāsasenāsananti}. … Yattha \textsanskrit{vā} pana \textsanskrit{bhikkhū} \textsanskrit{paṭikkamanti}, \textsanskrit{sabbametaṁ} \textsanskrit{senāsananti}}. }

%
\end{quotation}

It is almost impossible to capture this range of meaning in a single English word, with the possible exception of “resting place”. In practice, however, “resting place” is often too vague to be properly meaningful. For this reason, I vary my translation according to context, using “dwelling”, “furniture”, and “resting place” as appropriate.

\subsection*{\textit{Hattha}: “forearm”}

See \textit{sugata}.

\subsection*{\textit{Hammiya: “}a stilt house”}

In the Vinaya \textsanskrit{Piṭaka}, the \textit{hammiya} is normally grouped with either the \textit{\textsanskrit{māḷa}} and the \textit{\textsanskrit{pāsāda}}, or with the \textit{\textsanskrit{aḍḍhayoga}} and the \textit{\textsanskrit{pāsāda}}. According to the commentaries, all these buildings are different kinds of “stilt houses”. Rather than try to differentiate between them, which is not necessary from a practical perspective, I have grouped them together as “stilt house”. Here is what the commentaries have to say:

\begin{quotation}%
“A \textit{\textsanskrit{pāsāda}} is a long stilt house. A \textit{hammiya} is just a \textit{\textsanskrit{pāsāda}} that has an upper room on top of its flat roof.”\footnote{Sp 4.294: \textit{\textsanskrit{Pāsādoti} \textsanskrit{dīghapāsādo}. Hammiyanti \textsanskrit{upariākāsatale} \textsanskrit{patiṭṭhitakūṭāgāro} \textsanskrit{pāsādoyeva}}. }

%
\end{quotation}

\begin{quotation}%
“An \textit{\textsanskrit{aḍḍhayoga}} is a house bent like a \textit{\textsanskrit{supaṇṇa}}.”\footnote{Sp 4.294: \textit{\textsanskrit{Aḍḍhayogoti} \textsanskrit{supaṇṇavaṅkagehaṁ}}. }

%
\end{quotation}

\begin{quotation}%
“A house bent like a \textit{\textsanskrit{supaṇṇa}}: a house made in the shape of the wings of a \textit{\textsanskrit{garuḷa}}.”\footnote{Sp-\textsanskrit{ṭ} 4.294: \textit{\textsanskrit{Supaṇṇavaṅkagehanti} \textsanskrit{garuḷapakkhasaṇṭhānena} \textsanskrit{katagehaṁ}}. }

%
\end{quotation}

A \textit{\textsanskrit{garuḷa}}, better known in its Sanskrit form \textit{\textsanskrit{garuḍa}}, is a mythological bird.

At Sp-\textsanskrit{ṭ} 3.74 we find slightly different explanations. It is clear, however, that all three are stilt houses that are distinguished according to their shape and the kind of roof they possess. See also \textit{\textsanskrit{pāsāda}} in this same Appendix.

\subsection*{\textit{\textsanskrit{Hirañña}}, \textit{\textsanskrit{suvaṇṇa}}: “gold coin” or “money”, “gold”}

\textit{\textsanskrit{Suvaṇṇa}}, \textit{\textsanskrit{hirañña}}, and \textit{\textsanskrit{jātarūpa}} are closely related, all referring to gold in one way or another. It seems, however, that \textit{\textsanskrit{hirañña}} is used in a slightly different way from the other two. At \href{https://suttacentral.net/pli-tv-bu-vb-pj1/en/brahmali\#5.6.22}{Bu~Pj~1:5.6.22}, Sudinna’s parents make one pile of \textit{\textsanskrit{suvaṇṇa}} and one of \textit{\textsanskrit{hirañña}}. The use of two different words seems to necessitate that the piles are not exactly the same. According to the commentary \textit{\textsanskrit{hirañña}} here refers to money, presumably gold coins:

\begin{quotation}%
Here \textit{\textsanskrit{hirañña}} should be understood as the \textit{\textsanskrit{kahāpaṇa}} coin.\footnote{Sp 1.33: \textit{Ettha \textsanskrit{hiraññanti} \textsanskrit{kahāpaṇo} veditabbo.} }

%
\end{quotation}

This commentarial interpretation fits with how \textit{\textsanskrit{hirañña}} is used elsewhere in the Vinaya. In the Chapter on Robes (Kd 8), we find that doctors are paid in \textit{\textsanskrit{hirañña}}; in the Chapter on Resting Places (Kd 16), \textsanskrit{Anāthapiṇḍika} pays Prince Jeta in \textit{\textsanskrit{hirañña}} for the Jeta Grove; and in the Chapter on the Group of Seven Hundred (Kd 22) the monks ask for money and get \textit{\textsanskrit{hirañña}}.\footnote{See respectively \href{https://suttacentral.net/pli-tv-kd8/en/brahmali\#1.8.8}{Kd~8:1.8.8}, \href{https://suttacentral.net/pli-tv-kd16/en/brahmali\#4.9.11}{Kd~16:4.9.11}, and \href{https://suttacentral.net/pli-tv-kd22/en/brahmali\#1.1.15}{Kd~22:1.1.15}. } In all these cases, the context suggests that \textit{\textsanskrit{hirañña}} was used as money, that is, they were gold coins.

In most other cases of \textit{\textsanskrit{hirañña}} as found in the Vinaya Pitaka, it is paired with \textit{\textsanskrit{suvaṇṇa}}. For instance, \textit{bhikkhu \textsanskrit{pārājika}} 1 mentions a number of different temptations that might entice a monastic to disrobe, among them \textit{\textsanskrit{hirañña}} and \textit{\textsanskrit{suvaṇṇa}} (\href{https://suttacentral.net/pli-tv-bu-vb-pj1/en/brahmali\#8.2.73}{Bu~Pj~1:8.2.73}). If \textit{\textsanskrit{hirañña}} means gold coins, \textit{\textsanskrit{suvaṇṇa}} must refer to gold in some other form, otherwise it hard to see why they are both mentioned. In fact, all cases in the Vinaya where \textit{\textsanskrit{hirañña}} and \textit{\textsanskrit{suvaṇṇa}} are used side by side are such that we must conclude that they refer to gold in different forms.

The evidence suggests that \textit{\textsanskrit{suvaṇṇa}}/\textit{\textsanskrit{sovaṇṇa}} refers to gold in a more general sense. For instance, at Kd 6 we find the expressions \textit{\textsanskrit{suvaṇṇamālā}}, “a golden garland” and \textit{\textsanskrit{pāsāda} \textsanskrit{suvaṇṇa}}, “a stilt house made of gold” (\href{https://suttacentral.net/pli-tv-kd6/en/brahmali\#15.6.4}{Kd~6:15.6.4} and \href{https://suttacentral.net/pli-tv-kd6/en/brahmali\#15.8.6}{Kd~6:15.8.6}). At \textit{\textsanskrit{bhikkhunī} \textsanskrit{pācittiya}} 1, we find the phrase \textit{\textsanskrit{sabbasovaṇṇamayā} \textsanskrit{pattā}}, “all feathers made of gold” (\href{https://suttacentral.net/pli-tv-bi-vb-pc1/en/brahmali\#1.29}{Bi~Pc~1:1.29}), and in the Khandhakas we find a number of other things made of \textit{\textsanskrit{suvaṇṇa}}/\textit{\textsanskrit{sovaṇṇa}}. In most of the remaining cases, \textit{\textsanskrit{suvaṇṇa}} is paired either with \textit{\textsanskrit{hirañña}} or \textit{\textsanskrit{jātarūpa}}, both of which mean money, suggesting, once again, that \textit{\textsanskrit{suvaṇṇa}} must mean something else, that is, gold in all usages except as coins.

From the above I conclude that \textit{\textsanskrit{hirañña}} means gold coins in all contexts, whereas \textit{\textsanskrit{suvaṇṇa}} means gold in general. This is true even for Bu Ss 2, where we find the following definition in connection with the offense for physical contact: “Hair: just the hair; or the hair with strings in it, with a garland, with \textit{\textsanskrit{hirañña}}, with \textit{\textsanskrit{suvaṇṇa}}, with pearls, or with gems.”\footnote{\textit{\textsanskrit{Veṇī} \textsanskrit{nāma} \textsanskrit{suddhakesā} \textsanskrit{vā}, \textsanskrit{suttamissā} \textsanskrit{vā}, \textsanskrit{mālāmissā} \textsanskrit{vā}, \textsanskrit{hiraññamissā} \textsanskrit{vā}, \textsanskrit{suvaṇṇamissā} \textsanskrit{vā}, \textsanskrit{muttāmissā} \textsanskrit{vā}, \textsanskrit{maṇimissā} \textsanskrit{vā}}. (\href{https://suttacentral.net/pli-tv-bu-vb-ss2/en/brahmali\#2.1.20}{Bu Ss2:2.1.20}) } The commentary confirms this distinction between the two terms, with the implication that women used gold coins in their hair as an adornment.\footnote{Sp 1.271\textit{: \textsanskrit{Hiraññamissāti} \textsanskrit{kahāpaṇamālāya} \textsanskrit{missetvā} \textsanskrit{katā}. \textsanskrit{Suvaṇṇamissāti} \textsanskrit{suvaṇṇacīrakehi} \textsanskrit{vā} \textsanskrit{pāmaṅgādīhi} \textsanskrit{vā} \textsanskrit{missetvā} \textsanskrit{katā}}, “With \textit{\textsanskrit{hirañña}}: it is done by mixing with a garland of coins. With \textit{\textsanskrit{suvaṇṇa}}: it is done by mixing with golden strips or ornamental hanging stings.” } This might seem unlikely, except that this is done in India to the present day! In conclusion, I vary my translation between “money” and “gold coins” dependent on what is appropriate in any given context.

%
\chapter*{Appendix II: Specialized Vocabulary (used to section the Pali text)}
\addcontentsline{toc}{chapter}{Appendix II: Specialized Vocabulary (used to section the Pali text)}
\markboth{Appendix II: Specialized Vocabulary (used to section the Pali text)}{Appendix II: Specialized Vocabulary (used to section the Pali text)}

\subsection*{\textit{\textsanskrit{Kaṇḍa}}: “a chapter”}

\subsection*{\textit{Kucchi}: “core”}

The \textit{kucchi} permutation series is the first, and thus the “core”, in a series of very similar permutations. It is contrasted with a number of \textit{\textsanskrit{piṭṭhi}} permutation series, which are all subsequent, or “additional”, to the \textit{kucchi} series.

\begin{quotation}%
Sp 1.240: \textit{Tato \textsanskrit{paraṁ} \textsanskrit{sabbapacchimapadaṁ} \textsanskrit{nīlādīhi} navahi padehi \textsanskrit{saddhiṁ} \textsanskrit{yojetvā} \textsanskrit{kucchicakkaṁ} \textsanskrit{nāma} \textsanskrit{vuttaṁ}.}\\

“After this, all the subsequent words, having combined the nine words starting with blue, etc.—this is called the \textit{kucchi} permutation series.”

%
\end{quotation}

\subsection*{\textit{\textsanskrit{Khaṇḍa}}: “unconnected”}

The basic idea of \textit{\textsanskrit{khaṇḍa}} is “broken”. It is used with permutation series to show a series that does not link back to the first item of a series, that is, it is does not form a complete circle or loop, thus being “unconnected”. \textit{\textsanskrit{Khaṇḍa}} contrasts with \textit{baddha}, which refers to a complete circle, that is, a “linked” permutation series. (\textit{\textsanskrit{Khaṇḍa}} is also a commentarial term for a section of a monastic robe.)

\begin{quotation}%
Sp 1.200: \textit{Tato \textsanskrit{paṭhamañca} \textsanskrit{jhānaṁ}, \textsanskrit{dutiyañca} \textsanskrit{jhānanti} \textsanskrit{evaṁ} \textsanskrit{paṭhamajjhānena} \textsanskrit{saddhiṁ} \textsanskrit{ekamekaṁ} \textsanskrit{padaṁ} \textsanskrit{ghaṭentena} \textsanskrit{sabbapadāni} \textsanskrit{ghaṭetvā} teneva \textsanskrit{vitthārena} \textsanskrit{khaṇḍacakkaṁ} \textsanskrit{nāma} \textsanskrit{vuttaṁ}. \textsanskrit{Tañhi} puna \textsanskrit{ānetvā} \textsanskrit{paṭhamajjhānādīhi} na \textsanskrit{yojitaṁ}, \textsanskrit{tasmā} “\textsanskrit{khaṇḍacakka}”nti vuccati.}\\

“Beginning with ʻthe first absorption and the second absorption,’ in this way having connected all the steps by connecting each step with the first absorption, just by that expansion it is called ʻan unconnected permutation series’. It is called ʻan unconnected permutation series’ because it is not, having again brought it back, connected with the first absorption, etc.”

%
\end{quotation}

\subsection*{\textit{Gamana}: “a round”}

\textit{Gamana} is used in the permutation series to denote a section in a series of sections that are based on the same template.

\subsection*{\textit{Cakka}: “a permutation series”}

A technical exposition that sets out the various combinations of factors in which a rule applies.

\subsection*{\textit{Pali}: “a Canonical text”}

As used by the commentaries.

\subsection*{\textit{\textsanskrit{Piṭṭhi}}: “additional”}

The \textit{\textsanskrit{piṭṭhi}} permutation series contrast with the \textit{kucchi} permutation series, which is the first in the sequence. In this way, the \textit{kucchi} series is the “core”, whereas the \textit{\textsanskrit{piṭṭhi}} series are “additional”.

\begin{quotation}%
Sp 1.240: \textit{Tato \textsanskrit{pītakādīni} nava \textsanskrit{padāni} ekena \textsanskrit{nīlapadeneva} \textsanskrit{saddhiṁ} \textsanskrit{yojetvā} \textsanskrit{piṭṭhicakkaṁ} \textsanskrit{nāma} \textsanskrit{vuttaṁ}. Tato \textsanskrit{lohitakādīni} nava \textsanskrit{padāni} ekena \textsanskrit{pītakapadeneva} \textsanskrit{saddhiṁ} \textsanskrit{yojetvā} \textsanskrit{dutiyaṁ} \textsanskrit{piṭṭhicakkaṁ} \textsanskrit{vuttaṁ}. \textsanskrit{Evaṁ} \textsanskrit{lohitakapadādīhi} \textsanskrit{saddhiṁ} \textsanskrit{itarāni} nava nava \textsanskrit{padāni} \textsanskrit{yojetvā} \textsanskrit{aññānipi} \textsanskrit{aṭṭha} \textsanskrit{cakkāni} \textsanskrit{vuttānīti} \textsanskrit{evaṁ} \textsanskrit{dasagatikaṁ} \textsanskrit{piṭṭhicakkaṁ} \textsanskrit{veditabbaṁ}.}\\

“After this, having combined the nine words starting with yellow with the one word blue—this is called the \textit{\textsanskrit{piṭṭhi}} permutation series. After this, having combined the nine words starting with red with the one word yellow—this is called the second \textit{\textsanskrit{piṭṭhi}} permutation series. In this way, having combined each next group of nine words with the word red, etc., eight other permutation series are mentioned, thus the ten-fold \textit{\textsanskrit{piṭṭhi}} permutation series is to be known.”

%
\end{quotation}

\subsection*{\textit{\textsanskrit{Peyyāla}}: “successive”, “repetition”}

Often equivalent to etcetera in English.

\subsection*{\textit{Baddha}: “linked”}

\textit{Baddha} contrasts with \textit{\textsanskrit{khaṇḍa}}, which refers to a permuation series that is “unconnected” in the sense of not forming a complete loop. A \textit{baddhacakka}, “a linked permutation series”, forms a complete loop, linking back to the first item of the series.

\begin{quotation}%
Sp 1.200: \textit{Tato \textsanskrit{dutiyañca} \textsanskrit{jhānaṁ}, \textsanskrit{tatiyañca} \textsanskrit{jhānanti} \textsanskrit{evaṁ} \textsanskrit{dutiyajjhānena} \textsanskrit{saddhiṁ} \textsanskrit{ekamekaṁ} \textsanskrit{padaṁ} \textsanskrit{ghaṭetvā} puna \textsanskrit{ānetvā} \textsanskrit{paṭhamajjhānena} \textsanskrit{saddhiṁ} \textsanskrit{sambandhitvā} teneva \textsanskrit{vitthārena} \textsanskrit{baddhacakkaṁ} \textsanskrit{nāma} \textsanskrit{vuttaṁ}.}\\

“Then, ‘the second absorption and the third absorption\textit{’}, having thus joined the words one by one with the second absorption, having brought it back again, having bound it together with the first absorption, a \textit{baddha} permutation series is spoken of through that detail.”

%
\end{quotation}

\subsection*{\textit{\textsanskrit{Bhāṇavāro}}: “a section for recitation”}

\subsection*{\textit{Bheda}: “a subdivision”}

Only used in this way in the \textsanskrit{Parivāra}.

\subsection*{\textit{\textsanskrit{Mūla}}: “basis”}

Used in repetition series to denote a basic pattern to be followed.

\subsection*{\textit{\textsanskrit{Mūlaka}}: “based on”}

Used in repetition series to denote the number of items that form the basis for the series.

\subsection*{\textit{Vagga}: “a division”, “a subchapter”}

A \textit{vagga} either denotes a large division of text, as when used in the compounds \textsanskrit{Mahāvagga} or Cullavagga, or the smallest division of the text, in which case it forms part of the larger division \textit{\textsanskrit{kaṇḍa}}, “a chapter”.

\subsection*{\textit{\textsanskrit{Vāra}}: “a section”}

\textit{\textsanskrit{Vāra}} denotes the end of a particular section of the text in the Bhikkhu-\textsanskrit{pārājikas} and the Khandhakas.

\subsection*{\textit{Vidhi}: “a process”}

\textit{Vidhi} is used to denote certain processes used to achieve particular results, such as the ordination process. It is found only in headings and summary verses.

\subsection*{\textit{\textsanskrit{Saṅkhitta}}: “in brief”, “contracted”}

\subsection*{\textit{\textsanskrit{Saṅkhepa}}: “a contraction”}

Used only in the \textsanskrit{Parivāra}.

\subsection*{\textit{Suddhika}: “a basic series”}

\textit{Suddhika}, the “basic series”, thus refers to the fundamental template on which the more complicated series are based.

\begin{quotation}%
Sp 1.200: \textit{Tesu \textsanskrit{suddhikavāre} \textsanskrit{paṭhamajjhānaṁ} \textsanskrit{ādiṁ} \textsanskrit{katvā} \textsanskrit{yāva} \textsanskrit{mohā} \textsanskrit{cittaṁ} \textsanskrit{vinīvaraṇapadaṁ}, \textsanskrit{tāva} \textsanskrit{ekamekasmiṁ} pade \textsanskrit{samāpajjiṁ}, \textsanskrit{samāpajjāmi}, \textsanskrit{samāpanno}, \textsanskrit{lābhīmhi}, \textsanskrit{vasīmhi}, \textsanskrit{sacchikataṁ} \textsanskrit{mayāti} imesu chasu padesu \textsanskrit{ekamekaṁ} \textsanskrit{padaṁ} \textsanskrit{tīhākārehi}, \textsanskrit{catūhi}, \textsanskrit{pañcahi}, chahi, \textsanskrit{sattahākārehīti} \textsanskrit{evaṁ} \textsanskrit{pañcakkhattuṁ} \textsanskrit{yojetvā} suddhikanayo \textsanskrit{nāma} vutto.}\\

“Among them, in the \textit{suddhika} section, having placed the first absorption at the beginning, as far as the phrase ‘a mind free from the hindrance of confusion’, so far each phrase one by one. Having thus connected the fivefold ‘when three conditions are fulfilled, when four, when five, when six, when seven conditions are fulfilled’ phrase by phrase with these six phrases ‘I attained, I’m attaining, I’ve attained, I obtain, I master, I’ve realized’, the method of the ‘basic series’ is spoken of.”

%
\end{quotation}

%
\chapter*{Appendix III: Furniture, Furnishings, and Bedding}
\addcontentsline{toc}{chapter}{Appendix III: Furniture, Furnishings, and Bedding}
\markboth{Appendix III: Furniture, Furnishings, and Bedding}{Appendix III: Furniture, Furnishings, and Bedding}

\begin{description}%
\item[\textit{\textsanskrit{Ajinappaveṇī}}: “A rug made of black antelope hide”.] Sp 3.254: \textit{\textsanskrit{Ajinappaveṇīti} ajinacammehi \textsanskrit{mañcappamāṇena} \textsanskrit{sibbitvā} \textsanskrit{katā} \textsanskrit{paveṇī}}, “The \textit{\textsanskrit{ajinappaveṇī}} is a mat made by sewing \textit{ajina}-hides to the size of a bed.” Sp‑\textsanskrit{ṭ} 3.254: \textit{\textsanskrit{Ajinacammehīti} ajinamigacammehi}, “\textit{Ajina}-hides are hides of the \textit{ajina}-antelope.” According to DOP the hide is black.%
\item[\textit{Assatthara}: “A horse-back rug”.] Sp 3.254: \textit{\textsanskrit{Hatthattharaassattharāti} \textsanskrit{hatthiassapiṭṭhīsu} \textsanskrit{attharaṇakaattharaṇā} eva}, “The \textit{hatthattharaassatthara} are rugs to cover over the backs of elephants and horses.”%
\item[\textit{\textsanskrit{Āmalakavaṭṭika} \textsanskrit{pīṭha}}: “A bench with many legs”.] Sp 4.297: \textit{\textsanskrit{Āmalakavaṭṭikapīṭhaṁ} \textsanskrit{nāma} \textsanskrit{āmalakākārena} \textsanskrit{yojitaṁ} \textsanskrit{bahupādakapīṭhaṁ}}, “\textit{\textsanskrit{Āmalakavaṭṭikapīṭhaṁ}}: a bench with many legs connected in the manner of an emblic myrobalan.”%
\item[\textit{Āsandika}: “A square bench”.] Sp 4.297: \textit{Āsandikoti \textsanskrit{caturassapīṭhaṁ} vuccati}, “A bench with four sides is called an \textit{\textsanskrit{āsandika}}.” Vmv 4.297 adds: \textit{\textsanskrit{Caturassapīṭhanti} \textsanskrit{samacaturassaṁ}}, “A bench with four sides is one with four equal sides.”%
\item[\textit{\textsanskrit{Āsandī}}: “A high couch”.] Sp 3.254: \textit{\textsanskrit{āsandīti} \textsanskrit{pamāṇātikkantāsanaṁ}}, “The \textit{\textsanskrit{āsandī}} is a seat that is too large.” Sp‑yoj 2.254: \textit{\textsanskrit{Pamāṇātikkantāsananti} \textsanskrit{dīghāsanaṁ}}, “A seat that is too large is a seat that is too high.” \href{https://suttacentral.net/pli-tv-bi-vb-pc42/en/brahmali\#2.1.10}{Bi~Pc~42:2.1.10} shows that the \textit{\textsanskrit{āsandī}} was used both for sitting on and for lying down.%
\item[\textit{\textsanskrit{Uddhalomī}}: “A woolen rug with long fleece on one side”.] Sp 3.254: \textit{\textsanskrit{Uddalomīti} ekato \textsanskrit{uggatalomaṁ} \textsanskrit{uṇṇāmayattharaṇaṁ}}, “The \textit{\textsanskrit{uddalomī}} is a rug made of wool with tall hairs on one side.”%
\item[\textit{\textsanskrit{Ubhatolohitakūpadhāna}}: “A seat with red cushions at each end”.] Sp 3.254: \textit{\textsanskrit{Ubhatolohitakūpadhānanti} \textsanskrit{sīsūpadhānañca} \textsanskrit{pādūpadhānañcāti} \textsanskrit{mañcassa} \textsanskrit{ubhatolohitakūpadhānaṁ}, \textsanskrit{etaṁ} na kappati}, “\textit{\textsanskrit{Ubhatolohitakūpadhāna}}: a cushion for the head and a cushion for the feet, this is the meaning of red cushions at both ends of a bed, which is not allowed.”%
\item[\textit{\textsanskrit{Ekantalomī}}: “A woolen rug with long fleece on both sides”.] Sp 3.254: \textit{\textsanskrit{Ekantalomīti} ubhato \textsanskrit{uggatalomaṁ} \textsanskrit{uṇṇāmayattharaṇaṁ}}, “The \textit{\textsanskrit{ekantalomī}} is a rug made of wool with tall hairs on both sides.”%
\item[\textit{\textsanskrit{Kaṭṭissa}}: “A sheet of silk embroidered with gems”.] Sp 3.254: \textit{\textsanskrit{Kaṭṭissanti} \textsanskrit{ratanaparisibbitaṁ} \textsanskrit{koseyyakaṭṭissamayaṁ} \textsanskrit{paccattharaṇaṁ}}, “The \textit{\textsanskrit{kaṭṭissa}} is a sheet made of silken \textit{\textsanskrit{kaṭṭissa}} with gems sewn into it.”%
\item[\textit{\textsanskrit{Kadalīmigapavarapaccattharaṇa}}: “An exquisite sheet made of \textit{\textsanskrit{kadalī}}-deer hide”.] Sp 3.254: \textit{\textsanskrit{Kadalīmigapavarapaccattharaṇanti} \textsanskrit{kadalīmigacammaṁ} \textsanskrit{nāma} atthi, tena \textsanskrit{kataṁ} \textsanskrit{pavarapaccattharaṇaṁ}, \textsanskrit{uttamapaccattharaṇanti} attho; \textsanskrit{taṁ} kira setavatthassa upari \textsanskrit{kadalīmigacammaṁ} \textsanskrit{pattharitvā} \textsanskrit{sibbitvā} karonti}, “\textit{\textsanskrit{Kadalīmigapavarapaccattharaṇa}}: an exquisite sheet made with the hide of a \textit{\textsanskrit{kadalī}}-deer; the meaning is ‘the best kind of sheet’. They make it by spreading a white cloth on top and sewing them together.”%
\item[\textit{Kuttaka}: “A woolen rug like a dancer’s rug”.] Sp 3.254: \textit{Kuttakanti \textsanskrit{soḷasannaṁ} \textsanskrit{nāṭakitthīnaṁ} \textsanskrit{ṭhatvā} \textsanskrit{naccanayoggaṁ} \textsanskrit{uṇṇāmayaattharaṇaṁ}}, “The \textit{kuttaka} is a rug made of wool suitable for sixteen dancing girls to dance.”%
\item[\textit{\textsanskrit{Kuḷīrapādaka}}: “A bed/bench with crooked legs”.] Sp 2.111: \textit{\textsanskrit{Kuḷīrapādakoti} \textsanskrit{assameṇḍakādīnaṁ} \textsanskrit{pādasadisehi} \textsanskrit{pādehi} kato. Yo \textsanskrit{vā} pana koci \textsanskrit{vaṅkapādako}, \textsanskrit{ayaṁ} vuccati \textsanskrit{kuḷīrapādako}}, “It is made with legs like the legs of horses or rams, etc. Or whatever has crooked legs, this is called a \textit{\textsanskrit{kuḷīrapādaka}}.” \textit{\textsanskrit{Kuḷīrapādaka}} describes both beds and benches, \textit{\textsanskrit{maṇca}} and \textit{\textsanskrit{pīṭha}} (\href{https://suttacentral.net/pli-tv-bu-vb-pc14/en/brahmali}{Bu~Pc~14}).%
\item[\textit{Koseyya}: “A silken sheet”.] Sp 3.254: \textit{Koseyyanti \textsanskrit{ratanaparisibbitaṁ} \textsanskrit{kosiyasuttamayaṁ} \textsanskrit{paccattharaṇaṁ}; \textsanskrit{suddhakoseyyaṁ} pana \textsanskrit{vaṭṭati}}, “The \textit{koseyya} is a sheet made of silken threads with gems sewn into it; but pure silk is also allowed.” The name of the sheet suggests that the silk is its main attribute.%
\item[\textit{Gonaka}: “A long-fleeced woolen rug”.] Sp 3.254: \textit{Gonakoti \textsanskrit{dīghalomako} \textsanskrit{mahākojavo}; \textsanskrit{caturaṅgulādhikāni} kira tassa \textsanskrit{lomāni}}, “The \textit{gonaka} is a large cover with long fleece. Its fleece is more than eight centimeters long.”%
\item[\textit{Cittaka}: “A multi-colored woolen rug”.] Sp 3.254: \textit{\textsanskrit{Cittakāti} \textsanskrit{vānacitro} \textsanskrit{uṇṇāmayattharaṇo}}, “The \textit{cittaka} is a rug made of wool woven with many colors.”%
\item[\textit{\textsanskrit{Cimilikā}}: “A mat underlay”.] Sp 4.297: \textit{\textsanskrit{Cimilikā} \textsanskrit{nāma} \textsanskrit{parikammakatāya} \textsanskrit{bhūmiyā} \textsanskrit{chavisaṁrakkhaṇatthāya} \textsanskrit{attharaṇaṁ} vuccati}, “A mat to protect the skin from a treated floor is called a \textit{\textsanskrit{cimilikā}}.” But at Sp 2.112 we find the following definition: \textit{\textsanskrit{Cimilikā} \textsanskrit{nāma} \textsanskrit{sudhādiparikammakatāya} \textsanskrit{bhūmiyā} \textsanskrit{vaṇṇānurakkhaṇatthaṁ} \textsanskrit{katā} hoti, \textsanskrit{taṁ} \textsanskrit{heṭṭhā} \textsanskrit{pattharitvā} upari \textsanskrit{kaṭasārakaṁ} pattharanti}, “A \textit{\textsanskrit{cimilikā}} is made to protect the color of a floor that has been plastered, etc. It is spread out underneath, with a straw-mat spread out on top.”%
\item[\textit{\textsanskrit{Tūlikā}}: “A cotton-down quilt”.] Sp 3.254: \textit{\textsanskrit{Tūlikāti} \textsanskrit{pakatitūlikāyeva}}, “The \textit{\textsanskrit{tūlikā}}: just ordinary cotton.” Sp‑\textsanskrit{ṭ} 3.254: \textit{\textsanskrit{Pakatitūlikāti} \textsanskrit{rukkhatūlalatātūlapoṭakītūlasaṅkhātānaṁ} \textsanskrit{tiṇṇaṁ} \textsanskrit{tūlānaṁ} \textsanskrit{aññatarapuṇṇā} \textsanskrit{tūlikā}}, “Ordinary cotton: a \textit{\textsanskrit{tūlikā}} is filled with one of the three cotton downs, either cotton down from trees, cotton down from creepers, or cotton down from grass.” It is not clear whether the \textit{\textsanskrit{tūlikā}} was used as an underlay or as a cover. Quite possibly it was used as both.%
\item[\textit{\textsanskrit{Paṭalika}}: “A red woolen rug”.] Sp 3.254: \textit{\textsanskrit{Paṭalikāti} ghanapupphako \textsanskrit{uṇṇāmayalohitattharaṇo}; yo \textsanskrit{āmalakapaṭṭotipi} vuccati}, “The \textit{\textsanskrit{paṭalika}} is a red rug made of wool, dyed with deep red. Also, what has the pattern of an emblic myrobalan is so called.”%
\item[\textit{\textsanskrit{Paṭika}}: “A white woolen rug”.] Sp 3.254: \textit{\textsanskrit{Paṭikāti} \textsanskrit{uṇṇāmayo} \textsanskrit{setattharaṇo}}, “The \textit{\textsanskrit{paṭika}} is a white rug made of wool.”%
\item[\textit{\textsanskrit{Pallaṅka}}: “A luxurious couch”.] Sp 3.254: \textit{\textsanskrit{Pallaṅkoti} \textsanskrit{pādesu} \textsanskrit{vāḷarūpāni} \textsanskrit{ṭhapetvā} kato}, “The \textit{\textsanskrit{pallaṅka}} is made with images of wild animals on its legs.” \href{https://suttacentral.net/pli-tv-bi-vb-pc42/en/brahmali\#2.1.10}{Bi~Pc~42:2.1.10} shows that the \textit{\textsanskrit{pallaṅka}} was used both for sitting on and for lying down.%
\item[\textit{\textsanskrit{Pīṭha}}: “A bench”.] Certain kinds of \textit{\textsanskrit{pīṭha}}, which I normally render as “bench”, could seat two or more people, and could also be used for sleeping on. This kind of \textit{\textsanskrit{pīṭha}} would have been a large piece of furniture. Other kinds of \textit{\textsanskrit{pīṭha}} were much smaller. For instance, there is the \textit{\textsanskrit{pādapīṭha}}, “the footstool”, which was used for washing one’s feet. It is not known exactly how it was used, but presumably it was small. This is presumably true also of the \textit{\textsanskrit{jantāgharapīṭha}}, “the sauna bench”, which a student would carry to the sauna every time their teacher wanted to use it. Again, this must have been a relatively small piece of furniture. Nevertheless, I render \textit{\textsanskrit{pīṭha}} as “bench” in all contexts except \textit{\textsanskrit{pādapīṭha}}, which I translate as footstool.%
\item[\textit{\textsanskrit{Pīṭhika}}: “A small bench bound with cloth”.] Sp 4.297: \textit{\textsanskrit{Pīṭhikāti} \textsanskrit{pilotikābaddhapīṭhameva}}, “\textit{\textsanskrit{Pīṭhika}}: just a bench bound with pieces of cloth.” The diminutive ending -\textit{ika} suggests it was small.%
\item[\textit{Phalaka}/\textit{\textsanskrit{phalakapīṭha}}: “A plank bench”.] Vin-vn-\textsanskrit{ṭ} 1064: \textit{Phalaka’nti \textsanskrit{iminā} \textsanskrit{pāṭhāgataṁ} \textsanskrit{phalakapīṭhameva} \textsanskrit{dassitaṁ}}, “\textit{Phalaka}: by this the reading plank bench is shown.” This is a comment on \textit{bhikkhu \textsanskrit{pācittiya}} 14, but I take this to be the meaning also at \href{https://suttacentral.net/pli-tv-kd16/en/brahmali\#2.4.25}{Kd~16:2.4.25}, the context of which is all about seats.%
\item[\textit{\textsanskrit{Bidalamañcaka}}: “A wicker bed”.] Sp 4.296: \textit{\textsanskrit{Bidalamañcakanti} \textsanskrit{vettamañcaṁ}; \textsanskrit{veḷuvilīvehi} \textsanskrit{vā} \textsanskrit{vītaṁ}}, “\textit{\textsanskrit{Bidalamañcaka}}: a bed of cane or one woven with bamboo or reeds.”%
\item[\textit{\textsanskrit{Bundikābaddha}}: “A bed/bench with legs and frame”.] Sp 2.111: \textit{\textsanskrit{Bundikābaddhoti} \textsanskrit{aṭanīhi} \textsanskrit{mañcapāde} \textsanskrit{ḍaṁsāpetvā} \textsanskrit{pallaṅkasaṅkhepena} kato}, “\textit{\textsanskrit{Bundikābaddha}}: it is made by making the rails of the frame ‘bite’ into the legs of the bed, in the way of a luxurious bed (\textit{\textsanskrit{pallaṅka}}).” Sp‑yoj 2.111: \textit{Bundo eva bundiko, \textsanskrit{pādo}, \textsanskrit{tasmiṁ} \textsanskrit{ābaddhā} \textsanskrit{bandhitā} \textsanskrit{aṭanī} \textsanskrit{yassāti} \textsanskrit{bundikābaddho}}, “\textit{\textsanskrit{Bundikābaddha}}: a \textit{bundika} is just a \textit{bunda}, a leg, in that to which the rail is bound, attached.” \textit{\textsanskrit{Bundikābaddha}} describes both beds and benches, \textit{\textsanskrit{maṇca}} and \textit{\textsanskrit{pīṭha}} (\href{https://suttacentral.net/pli-tv-bu-vb-pc14/en/brahmali}{Bu~Pc~14}).%
\item[\textit{\textsanskrit{Bhaddapīṭha}}: “A cane bench”.] Sp 4.297: \textit{\textsanskrit{Bhaddapīṭhanti} \textsanskrit{vettamayaṁ} \textsanskrit{pīṭhaṁ} vuccati}, “A bench made of cane is called a \textit{\textsanskrit{bhaddapīṭha}}.”%
\item[\textit{\textsanskrit{Masāraka}}: “A bed/bench with legs and frame”.] Sp 2.111: \textit{\textsanskrit{Masārakoti} \textsanskrit{mañcapāde} \textsanskrit{vijjhitvā} tattha \textsanskrit{aṭaniyo} \textsanskrit{pavesetvā} kato}, “\textit{\textsanskrit{Masāraka}}: it is made by making a hole in the legs of the bed and then inserting the rails of the frame there.” \textit{\textsanskrit{Masāraka}} describes both beds and benches, \textit{\textsanskrit{maṇca}} and \textit{\textsanskrit{pīṭha}} (\href{https://suttacentral.net/pli-tv-bu-vb-pc14/en/brahmali}{Bu~Pc~14}).%
\item[\textit{\textsanskrit{Miḍḍhi}}: “A bench”.] Vmv 4.296: \textit{\textsanskrit{Miḍḍhakanti} \textsanskrit{mañcākārena} \textsanskrit{kaṭṭhamattikādīhi} katave}, “\textit{\textsanskrit{Miḍḍhaka}}: having the appearance of a bed and made of wood, clay, etc.”%
\item[\textit{Rathatthara}: “A carriage-seat rug”.] Sp 3.254: \textit{\textsanskrit{Hatthattharaassattharāti} \textsanskrit{hatthiassapiṭṭhīsu} \textsanskrit{attharaṇakaattharaṇā} eva. Rathattharepi eseva nayo}, “The \textit{hatthattharaassatthara} are rugs to cover over the backs of elephants and horses. The \textit{rathatthara}, too, is to be understood the same way.”%
\item[\textit{\textsanskrit{Vikatikā}}: “A woolen rug decorated with the images of predatory animals”.] Sp 3.254: \textit{\textsanskrit{Vikatikāti} \textsanskrit{sīhabyagghādirūpavicitro} \textsanskrit{uṇṇāmayattharaṇo}}, “The \textit{\textsanskrit{vikatikā}} is a rug made of wool decorated with images of lions, tigers, etc.”%
\item[\textit{Sauttaracchada}: “A seat with a canopy”.] Sp 3.254: \textit{Sauttaracchadanti saha uttaracchadanena; uparibaddhena \textsanskrit{rattavitānena} saddhinti attho}, “\textit{Sauttaracchada}: together with a canopy. The meaning is: together with a dyed canopy which is fastened above.”%
\item[\textit{\textsanskrit{Sattaṅga}}: “A sofa”.] Sp 4.297: \textit{\textsanskrit{Sattaṅgo} \textsanskrit{nāma} \textsanskrit{tīsu} \textsanskrit{disāsu} \textsanskrit{apassayaṁ} \textsanskrit{katvā} \textsanskrit{katamañco}}, “A bed made with support on three sides is called a \textit{\textsanskrit{sattaṅga}}.”%
\item[\textit{Hatthatthara}: “An elephant-back rug”.] Sp 3.254: \textit{\textsanskrit{Hatthattharaassattharāti} \textsanskrit{hatthiassapiṭṭhīsu} \textsanskrit{attharaṇakaattharaṇā} eva}, “The \textit{hatthattharaassatthara} are rugs to cover over the backs of elephants and horses.”%
\end{description}

%
\chapter*{Appendix IV: Medical Terminology}
\addcontentsline{toc}{chapter}{Appendix IV: Medical Terminology}
\markboth{Appendix IV: Medical Terminology}{Appendix IV: Medical Terminology}

\section*{Illnesses}

\begin{description}%
\item[\textit{\textsanskrit{Aṅgavāta}}: “arthritis of the hands and feet”.] \textit{\textsanskrit{Aṅgavāta}} is literally “wind of the limbs”. I follow the commentarial explanation. Sp 3.267: \textit{\textsanskrit{Aṅgavātoti} \textsanskrit{hatthapāde} \textsanskrit{vāto}}, “\textit{\textsanskrit{Aṅgavāta}} means wind in the hands and the feet.”%
\item[\textit{\textsanskrit{Apamāra}}: “epilepsy”.] Sp 3.269: \textit{\textsanskrit{Abhisannakāyoti} \textsanskrit{ussannadosakāyo}}, “\textit{\textsanskrit{Abhisannakāya}} means the body is full of impurities.”%
\item[\textit{\textsanskrit{Abhisannakāya}}: “full of bodily impurities”.] Sp 3.264: \textit{\textsanskrit{Āmakamaṁsañca} \textsanskrit{khādi} \textsanskrit{āmakalohitañca} \textsanskrit{pivīti} na \textsanskrit{taṁ} bhikkhu \textsanskrit{khādi} na pivi, amanusso \textsanskrit{khāditvā} ca \textsanskrit{pivitvā} ca pakkanto, tena \textsanskrit{vuttaṁ} – tassa so \textsanskrit{amanussikābādho} \textsanskrit{paṭippassambhīti}}, “He ate raw meat and drank blood: that monk did not eat or drink it; the spirit ate and drank it, and then departed. Because of that it was said that ‘his spirit possession subsided.’”%
\item[\textit{\textsanskrit{Amanussikābādha}}: “spirit possession”.] There is no clear difference between \textit{kacchu} and \textit{\textsanskrit{kaṇḍu}}. Sp 2.539: \textit{\textsanskrit{Kaṇḍūti} kacchu}, “\textit{\textsanskrit{Kaṇḍu}} means \textit{kacchu}.”%
\item[\textit{\textsanskrit{Assāvo}}: “a running sore”.] There is no clear difference between \textit{kacchu} and \textit{\textsanskrit{kaṇḍu}}. Sp 2.539: \textit{\textsanskrit{Kaṇḍūti} kacchu}, “\textit{\textsanskrit{Kaṇḍu}} means \textit{kacchu}.”%
\item[\textit{\textsanskrit{Udaravātābādha}}: “a stomachache”.] See Appendix I: Technical Terms.%
\item[\textit{Kacchu}: “an itch”.] See Appendix I: Technical Terms.%
\item[\textit{\textsanskrit{Kacchurogābādha}}: “itchy skin disease”.] See Appendix I: Technical Terms.%
\item[\textit{\textsanskrit{Kaṇḍu}}: “an itch”.] Sp 3.269: \textit{\textsanskrit{Gharadinnakābādhoti} \textsanskrit{vasīkaraṇapānakasamuṭṭhitarogo}, “\textsanskrit{Gharadinnakābādha}} is a sickness coming from drinking an overpowering drink.” Sp‑\textsanskrit{ṭ} 3.269: \textit{\textsanskrit{Gharadinnakābādho} \textsanskrit{nāma} \textsanskrit{vasīkaraṇatthāya} \textsanskrit{gharaṇiyā} \textsanskrit{dinnabhesajjasamuṭṭhito} \textsanskrit{ābādho}}, “\textit{\textsanskrit{Gharadinnakābādha}} is the name of a sickness coming from medicine given by a housewife for the purpose of overpowering.” The point seems to be that one is given a substance so that one can subsequently be overpowered.%
\item[\textit{\textsanskrit{Kilāsa}}: “mild leprosy”.] Sp 2.539: \textit{Thullakacchu \textsanskrit{vā} \textsanskrit{ābādhoti} \textsanskrit{mahāpiḷakābādho} vuccati}, “\textit{Thullakacchu \textsanskrit{vā} \textsanskrit{ābādha}} is a sickness with large boils.”%
\item[\textit{\textsanskrit{Kuṭṭha}}: “leprosy”.] Sp 3.269: \textit{\textsanskrit{Duṭṭhagahaṇikoti} \textsanskrit{vipannagahaṇiko}; kicchena \textsanskrit{uccāro} \textsanskrit{nikkhamatīti} attho}, “\textit{\textsanskrit{Duṭṭhagahaṇiko}}: one whose stomach has failed; the meaning is one has difficulty excreting feces.”%
\item[\textit{\textsanskrit{Gaṇḍa}}: “an abscess”.] Sp 2.539: \textit{\textsanskrit{Piḷakāti} \textsanskrit{lohitatuṇḍikā} \textsanskrit{sukhumapiḷakā}}, “\textit{\textsanskrit{Piḷaka}} is a minor \textit{\textsanskrit{piḷaka}} with blood on the tip.”%
\item[\textit{\textsanskrit{Gaṇḍābādha}}: “an abscess”.] Literally, “a wind disease”, but this is according to the Indian system of classification, where many diseases are classified under this heading, including arthritis. Nothing is specified in the commentaries.%
\end{description}

\section*{Medicines for internal use}

See also Medicinal Plants in Appendix IV.

\begin{description}%
\item[\textit{\textsanskrit{Akaṭayūsa}}: “mung-bean broth”.\footnote{This list does not include the medicines rendered as “tonics”, including fats, nor medicines classified as salts. For these medicines, see respectively, \href{https://suttacentral.net/pli-tv-kd6/en/brahmali\#1.2.6}{Kd~6:1.2.6} (tonics), \href{https://suttacentral.net/pli-tv-kd6/en/brahmali\#2.1.3}{Kd~6:2.1.3} (fats), and \href{https://suttacentral.net/pli-tv-kd6/en/brahmali\#8.1.3}{Kd~6:8.1.3} (salts). }] Sp 3.269: \textit{\textsanskrit{Akaṭayusanti} asiniddho \textsanskrit{muggapacitapānīyo}}, “\textit{\textsanskrit{Akaṭayūsa}} is drinkable mung beans that have been boiled without oil.” Sp‑\textsanskrit{ṭ} 3.269, however, says: \textit{\textsanskrit{Akaṭayūsenāti} \textsanskrit{anabhisaṅkhatena} \textsanskrit{muggayūsena}}, “\textit{\textsanskrit{Akaṭayūsena}} means the juice of unprepared mung beans.” This would seem to mean the raw juice of mung beans. I follow the more ancient authority.%
\item[\textit{\textsanskrit{Acchakañjī}}: “clear congee”.] Sp 3.269: \textit{\textsanskrit{Acchakañjiyanti} \textsanskrit{taṇḍulodakamaṇḍo}}, “\textit{\textsanskrit{Acchakañjiya}}: the cream of rice water.”%
\item[\textit{\textsanskrit{Āmakamaṁsa}}: “raw meat”.] Sp 3.269: \textit{\textsanskrit{Kaṭākaṭanti} sova dhotasiniddho}, “\textit{\textsanskrit{Kaṭākaṭa}} the same (as the previous) but washed in oil.” Sp‑\textsanskrit{ṭ} 3.269, however, says: \textit{\textsanskrit{Kaṭākaṭenāti} mugge \textsanskrit{pacitvā} \textsanskrit{acāletvāva} \textsanskrit{parissāvitena} \textsanskrit{muggasūpenāti}}, “\textit{\textsanskrit{Kaṭākaṭa}} means mung-bean soup made by boiling mung beans and then filtering it without stirring.” But this seems indistinguishable from the \textit{\textsanskrit{akaṭayūsa}}, “the mung-bean broth”, mentioned above.%
\item[\textit{Āmakalohita}: “raw blood”.] \textit{\textsanskrit{Tekaṭulayāgu}} is commonly rendered as “rice porridge having three pungent ingredients”. The three are sesame seeds, rice, and mung beans, yet rice and mung beans can hardly be called pungent. I would suggest it is the taste of the combination of the three that is pungent.%
\item[\textit{\textsanskrit{Kaṭākaṭa}}: “oily mung-bean broth”.] Sp 3.269: \textit{\textsanskrit{Paṭicchādanīyenāti} \textsanskrit{maṁsarasena}}, “\textit{\textsanskrit{Paṭicchādanīyena}} means having the juice of meat.”%
\item[\textit{\textsanskrit{Kasāvodaka}}: “bitter water”.] \textit{\textsanskrit{Sītāloḷī}} literally means “what is mixed in a furrow”. Sp 3.269: \textit{\textsanskrit{Sītāloḷinti} \textsanskrit{naṅgalena} kasantassa \textsanskrit{phāle} \textsanskrit{laggamattikaṁ} udakena \textsanskrit{āloḷetvā} \textsanskrit{pāyetuṁ} \textsanskrit{anujānāmīti} atth}o, “\textit{\textsanskrit{Sītāloḷī}}: the meaning is ‘I allow you to drink a mixture of water and the clay sticking to a plowshare of one plowing with a plow.’”%
\end{description}

\section*{Medicines for external use}

\begin{description}%
\item[\textit{Kapalla}: “soot”.] Sp 3.267: \textit{\textsanskrit{Kabaḷikanti} \textsanskrit{vaṇamukhe} \textsanskrit{sattupiṇḍaṁ} \textsanskrit{pakkhipituṁ}}, “\textit{\textsanskrit{Kabaḷika}} means to place a lump of flour on the sore.” Vmv 3.267: \textit{\textsanskrit{Kabaḷikāti} \textsanskrit{upanāhabhesajjaṁ}}, “\textit{\textsanskrit{Kabaḷika}}: a lasting medicine.” The definition in DOP is not quite right.%
\item[\textit{\textsanskrit{Kabaḷika}}: “flour paste”.] The \textit{\textsanskrit{kāḷānusāriya}} is identified as \textit{Dalbergia sissoo} in SED, sv. \textit{\textsanskrit{kālānusārya}}, and as the \textit{Parmelia perlata}, “stone flower”, in SAF, p. 111. Yet according to SN 45.142, the \textit{\textsanskrit{kāḷānusāriya}} was a fragrant root (\textit{ye keci \textsanskrit{mūlagandhā}, \textsanskrit{kāḷānusāriyaṁ} \textsanskrit{tesaṁ} \textsanskrit{aggamakkhāyati}}), which fits neither with the \textit{Dalbergia sissoo} nor the \textit{Parmelia perlata}. There is one Ayurvedic plant with the name \textit{\textsanskrit{kālānusārya}} (the Indian valerian), however, that fits this description. It seems this plant is used as an eye ointment, which fits its description at \href{https://suttacentral.net/pli-tv-kd6/en/brahmali\#11.2.8}{Kd~6:11.2.8}. The Indian valerian is closely related to the Spikenard, which might be an alternative rendering.%
\item[\textit{\textsanskrit{Kāḷañjana}}: “black ointment”.] See SED and DOP.%
\item[\textit{\textsanskrit{Kāḷānusāriya}}: “Indian valerian”.] Sp 3.267: \textit{\textsanskrit{Bhaṅgodakanti} \textsanskrit{nānāpaṇṇabhaṅgakuthitaṁ} \textsanskrit{udakaṁ}; tehi \textsanskrit{paṇṇehi} ca udakena ca \textsanskrit{siñcitvā} \textsanskrit{siñcitvā} sedetabbo}, “\textit{\textsanskrit{Bhaṅgodaka}}: water with various putrid, shredded leaves. One is made to sweat by repeated pouring the water and the leaves.” The commentary brings in the idea of sweating, saying that the hemp water was for external use, yet neither is mentioned in the Canonical text. In fact, although the use of \textit{\textsanskrit{bhaṅgodaka}} in the Canonical text is immediately preceded by the three separate treatments that involve sweating (\href{https://suttacentral.net/pli-tv-kd6/en/brahmali\#14.3.3}{Kd~6:14.3.3}–14.3.9), it is not mentioned in connection with \textit{\textsanskrit{bhaṅgodaka}}. Moreover, the commentary interprets \textit{\textsanskrit{bhaṅga}} to mean shredded (leaves), with the idea of leaves merely implied. The more straightforward interpretation is that \textit{\textsanskrit{bhaṅga}} refers to hemp, which is how I. B. Horner understands it. It seems possible, then, that this refers to hemp water, or cannabis water, that was taken as an internal medicine. Given that cannabis is known to alleviate arthritis symptoms, this is perhaps not as surprising as it may seem.%
\item[\textit{Geruka}: “red ocher”.] Sp 3.365: \textit{\textsanskrit{Rasañjanaṁ} \textsanskrit{nānāsambhārehi} \textsanskrit{kataṁ}}, “\textit{\textsanskrit{Rasañjana}} is made with many ingredients.”%
\item[\textit{Candana}: “sandalwood”.] Sp 3.365: \textit{\textsanskrit{Sotañjananti} \textsanskrit{nadīsotādīsu} \textsanskrit{uppajjanakaṁ} \textsanskrit{añjanaṁ}}, “\textit{\textsanskrit{Sotañjana}}: an ointment being produced in the stream of rivers.”%
\end{description}

\section*{Medical equipment and instruments}

\begin{description}%
\item[\textit{\textsanskrit{Aṁsabaddhaka}}: “a shoulder strap”.] Vin-\textsanskrit{ālaṅ}-\textsanskrit{ṭ} 34.67: \textit{\textsanskrit{Añjanitthavikāya} \textsanskrit{aṁse} \textsanskrit{lagganatthāya} \textsanskrit{aṁsabaddhakampi} bandhanasuttakampi \textsanskrit{vaṭṭati}}, “A shoulder strap and also a \textit{bandhanasuttaka} is allowed for the purpose of the hanging of the ointment-box bag from the shoulder.”%
\item[\textit{\textsanskrit{Añjanitthavika}}: “an ointment-box bag”.] Sp 3.267: \textit{\textsanskrit{Udakakoṭṭhakanti} \textsanskrit{udakakoṭṭhe} \textsanskrit{cāṭiṁ} \textsanskrit{vā} \textsanskrit{doṇiṁ} \textsanskrit{vā} \textsanskrit{uṇhodakassa} \textsanskrit{pūretvā} tattha \textsanskrit{pavisitvā} \textsanskrit{sedakammakaraṇaṁ} \textsanskrit{anujānāmīti} attho}, “\textit{\textsanskrit{Udakakoṭṭhaka}}: the meaning is ‘I allow the causing of sweating by entering a tank or trough filled with hot water in a bathroom.’”%
\item[\textit{\textsanskrit{Añjanisalāka}}: “an ointment stick”.] \textit{\textsanskrit{Chakaṇa}} literally means “dung”, but functions as a cleaning agent. See under \textit{mattika} in Appendix I: Technical Terms.%
\item[\textit{\textsanskrit{Añjanī}}: “an ointment box”.] Vin-\textsanskrit{ālaṅ}-\textsanskrit{ṭ} 34.67: \textit{\textsanskrit{Añjanitthavikāya} \textsanskrit{aṁse} \textsanskrit{lagganatthāya} \textsanskrit{aṁsabaddhakampi} bandhanasuttakampi \textsanskrit{vaṭṭati}}, “A shoulder strap and also a \textit{bandhanasuttaka} is allowed for the purpose of the hanging of the ointment-box bag from the shoulder.”%
\item[\textit{\textsanskrit{Apidhāna}}: “a lid”.] \textit{Mattika} literally means “clay”, but functions as a cleaning agent. See Appendix I: Technical Terms.%
\item[\textit{\textsanskrit{Udakakoṭṭhaka}}: “a bathtub”.] \textit{Rajananippakka} literally means “dye that has been cooked”, yet its function is that of cleaning. Sp 3.264: \textit{Rajananippakkanti \textsanskrit{rajanakasaṭaṁ}. \textsanskrit{Pākatikacuṇṇampi} \textsanskrit{koṭṭetvā} udakena \textsanskrit{temetvā} \textsanskrit{nhāyituṁ} \textsanskrit{vaṭṭati}; etampi \textsanskrit{rajananippakkasaṅkhepameva} gacchati}, “\textit{Rajananippakka} are the dregs from dyeing. Having ground regular bathing powder, having moistened it, it is allowable to bathe. The same goes for \textit{rajananippakka}.” Since dyeing and cleaning was often the same process in ancient India, the dyeing agent would have had cleansing properties. The dregs could therefore be used as a cleaning agent.%
\end{description}

\section*{Medical treatments}

\begin{description}%
\item[\textit{\textsanskrit{Dhūmaṁ} \textsanskrit{kātuṁ}}: “fumigation”.] Sp 3.267: \textit{\textsanskrit{Loṇasakkharikāya} chinditunti khurena \textsanskrit{chindituṁ}}, “\textit{\textsanskrit{Loṇasakkharikāya} \textsanskrit{chindituṁ}} means to cut with a razor.”%
\item[\textit{\textsanskrit{Dhūmaṁ} \textsanskrit{pātuṁ}}: “smoke inhalation”.] Sp 3.279: \textit{Yena kenaci pana cammena \textsanskrit{vā} vatthena \textsanskrit{vā} \textsanskrit{vatthipīḷanampi} na \textsanskrit{kātabbaṁ},} “One should not do bladder-action, \textit{\textsanskrit{vatthipīḷana}}, with whatever skin or cloth.” Vmv 3.279: \textit{\textsanskrit{Vatthipīḷananti} \textsanskrit{yathā} \textsanskrit{vatthigatatelādi} \textsanskrit{antosarīre} \textsanskrit{ārohanti}, \textsanskrit{evaṁ} hatthena \textsanskrit{vatthimaddanaṁ}}, “\textit{\textsanskrit{Vatthipīḷana}}: in order for oils, etc., in a bladder to go up inside the body, thus one squeezes the bladder with the hand.” The meaning is not entirely clear. My rendering is no more than a suggestion.%
\item[\textit{Natthukamma}: “treatment through the nose”.] I have not been able to trace any explanation of this seemingly strange practice, either in the commentaries or elsewhere.%
\item[\textit{\textsanskrit{Paṭikamma}}: “treatment”.] Sp 3.267: \textit{\textsanskrit{Sambhārasedanti} \textsanskrit{nānāvidhapaṇṇabhaṅgasedaṁ}}, “\textit{\textsanskrit{Sambhārasedanti}}: sweating with various shredded leaves.”%
\end{description}

%
\chapter*{Appendix V: Plants}
\addcontentsline{toc}{chapter}{Appendix V: Plants}
\markboth{Appendix V: Plants}{Appendix V: Plants}

\section*{Medicinal plants}

\begin{description}%
\item[\textit{\textsanskrit{Ativisā}}: “(Indian) atis root”, \textit{Aconitum heterophyllum}.] Identified in SED as \textit{Aconitum ferox}, sv. \textit{ati}. SAF, p. 96, and the Pandanus Database of Plants, however, identifies it (\textit{\textsanskrit{ativiṣā}}) with \textit{Aconitum heterophyllum} (http://iu.ff.cuni.cz/pandanus/database/details.php?id=1235).%
\item[\textit{Āmalaka}: “emblic myrobalan (fruit)”, \textit{Phyllanthus emblica}.] Also known as “yellow myrobalan”. See DOP and CPD.%
\item[\textit{\textsanskrit{Usīra}}: “vetiver grass”, “vetiver root”, \textit{Andropogon muricatus} or \textit{Vetiveria zizanioides}.] Also known as khus. See DOP.%
\item[\textit{\textsanskrit{Kaṭukarohiṇī}}: “black hellebore”, \textit{Helleborus niger.}] See SED and DOP.%
\item[\textit{\textsanskrit{Kappāsa}}: “cotton plant”, \textit{Gossypium hirsutum}.] Identified in SAF, p. 66.%
\item[\textit{\textsanskrit{Kāḷānusāriya}}: “Indian valerian”.] The \textit{\textsanskrit{kāḷānusāriya}} is identified as \textit{Dalbergia sissoo} in SED, sv. \textit{\textsanskrit{kālānusārya}}, and as the \textit{Parmelia perlata}, “stone flower”, in SAF, p. 111. Yet according to SN 45.142, the \textit{\textsanskrit{kāḷānusāriya}} was a fragrant root (\textit{ye keci \textsanskrit{mūlagandhā}, \textsanskrit{kāḷānusāriyaṁ} \textsanskrit{tesaṁ} \textsanskrit{aggamakkhāyati}}), which fits neither with the \textit{Dalbergia sissoo} nor the \textit{Parmelia perlata}. There is one Ayurvedic plant with the name \textit{\textsanskrit{kālānusārya}} (the Indian valerian), however, that fits this description. It seems this plant is used as an eye ointment, which fits its description at \href{https://suttacentral.net/pli-tv-kd6/en/brahmali\#11.2.8}{Kd~6:11.2.8}. The Indian valerian is closely related to the Spikenard, which might be an alternative rendering.%
\item[\textit{\textsanskrit{Kuṭaja}}: “arctic snow”, \textit{Wrightia antidysenterica}.] See SED and DOP. SAF identifies it with \textit{Holarrhena pubescens}, but this is wrong according to Wikipedia: https://en.wikipedia.org/wiki/Wrightia\_antidysenterica.%
\item[\textit{\textsanskrit{Goṭṭhaphala}}: “crepe ginger”, \textit{Costus speciosus.}] See SED. This is the same as the Skt. \textit{\textsanskrit{kuṣṭha}}. But there are other opinions. First, Khuddas-\textsanskrit{pṭ} 93: \textit{\textsanskrit{Goṭṭhaphalanti} madanaphalanti vadanti}, “They say the \textit{\textsanskrit{goṭṭhaphala}} is the ‘\textit{madana} fruit’”, which is identified by SAF, p. 106, as the \textit{Catunaregam spinosa}. Second, Vjb 2.249: \textit{\textsanskrit{Goṭṭhaphalaṁ} \textsanskrit{pūvaphalanti} eke. \textsanskrit{Koṭṭhase} kira acchiva}., “Some say the \textit{\textsanskrit{goṭṭhaphala}} is the ‘biscuit fruit’, but a faction says it’s the \textit{acchiva}.” DOP identifies the \textit{acchiva} with the \textit{Hyperanthera moringa}.%
\item[\textit{Candana}: “sandal (tree)”.] I have not been able to find any information to identify \textit{taka}. It is not even clear whether it is the name of a plant or a word for gum, with Vjb 3.263 suggesting it could be either: \textit{\textsanskrit{Takaṁ} \textsanskrit{nāma} tassa rukkhassa \textsanskrit{tacapākodakaṁ} … Atha \textsanskrit{vā} \textsanskrit{takaṁ} \textsanskrit{nāma} \textsanskrit{lākhā}}, “The water from the boiled bark of that tree is called \textit{taka} … alternatively the gum is called \textit{taka}.” In any case, this gum and the next two (as listed at \href{https://suttacentral.net/pli-tv-kd6/en/brahmali\#7.1.4}{Kd~6:7.1.4}) are closely related, all stemming from the same plant, the \textit{taka}. Sp 3.363: \textit{\textsanskrit{Takatakapattitakapaṇṇiyo} \textsanskrit{lākhājātiyo}}, “\textit{\textsanskrit{Takatakapattitakapaṇṇiyo}} are kinds of gum.” Sp‑\textsanskrit{ṭ} 2.249 elaborates: \textit{Takanti \textsanskrit{aggakoṭiyā} nikkhantasileso. Takapattinti pattato nikkhantasileso. \textsanskrit{Takapaṇṇinti} \textsanskrit{palāse} \textsanskrit{bhajjitvā} katasileso. \textsanskrit{Daṇḍato} nikkhantasileso tipi vadanti}, “\textit{Taka}: the gum exuding at the highest point. \textit{Takapatti}: the gum exuding from the leaves. \textit{\textsanskrit{Takapaṇṇi}}: the gum produced from roasting the foliage. They also say it is the gum exuding from sticks.”%
\item[\textit{Taka}: “the \textit{taka} tree”.] See SED and DOP.%
\item[\textit{Tagara}: “crape jasmine”, \textit{Tabernaemontana coronaria}.] See SED and DOP.%
\item[\textit{\textsanskrit{Tālīsa}}: “coffee plum”, \textit{Flacourtia cataphracta}.] See DOP and SAF, p. 83.%
\item[\textit{\textsanskrit{Nattamāla}}: “Indian beech”, \textit{Pongamia pinnata}.] See DOP and SAF, p. 86.%
\item[\textit{Nimba}: “neem tree”, \textit{Azadirachta indica}.] See DOP and SAF, p. 87.%
\item[\textit{\textsanskrit{Paṭola}}: “pointed gourd”, \textit{Trichosanthes dioica}.] See DOP and SAF, p. 89.%
\item[\textit{\textsanskrit{Pippalī}}: “long pepper”, \textit{Piper longum}.] N\&E suggests \textit{Ficus rumphii}, sv. \textit{pakkava}, but this seems little more than an educated guess based on equating it with Hindi word \textit{pakar}. It seems more likely to be related to the Sanskrit \textit{\textsanskrit{parkaṭa}}/\textit{plaksha}. In Sinhala script \textit{\textsanskrit{ṭ}} and \textit{v} are hardly distinguishable and so it is quite conceivable that there has been a mix-up in the scribal process, whereby \textit{\textsanskrit{pakkaṭa}} has become \textit{pakkava}. According to SED, the \textit{\textsanskrit{parkaṭa}} is the \textit{Ficus virens/Ficus infectoria}.%
\item[\textit{Phaggava}/\textit{pakkava}/\textit{paggava}/\textit{vaggava}: “the white fig”, \textit{Ficus virens/Ficus infectoria.}] See SED and SAF, p. 104, which I follow. PED suggests \textit{Erycibe paniculata}.%
\item[\textit{\textsanskrit{Bilaṅga}}/\textit{\textsanskrit{vilaṅga}}: “false black pepper”, \textit{Embelia ribes}.] SED says: “\textit{\textsanskrit{Bhāṅga}}, mf(\textsanskrit{ī})n, (fr. \textit{\textsanskrit{bhaṅgā}}) hempen, made or consisting of hemp …”, and “\textit{\textsanskrit{Bhaṅgā}}, f. hemp (\textit{Cannabis Sativa}); an intoxicating beverage (or narcotic drug commonly called ‘bhang’) prepared from the hemp plant”. But Sp 1.636: \textit{\textsanskrit{Bhaṅganti} \textsanskrit{pāṭekkaṁ} \textsanskrit{vākasuttamevāti} eke. Etehi \textsanskrit{pañcahi} \textsanskrit{missetvā} \textsanskrit{katasuttaṁ} pana “\textsanskrit{bhaṅga}”nti \textsanskrit{veditabbaṁ}}, “\textit{\textsanskrit{Bhaṅga}}: some say it is just a separate thread from bark. But \textit{\textsanskrit{bhaṅga}} is to be understood as the thread made by mixing the (other) five.”%
\item[\textit{\textsanskrit{Bhaṅga}}: “hemp”, \textit{Cannabis} s\textit{ativa.}] See SAF, p. 74. See also SED, sv. \textit{Bhadramustaka.}%
\item[\textit{Bhaddamuttaka}: “nut grass”, \textit{Cyperus rotundus}.] See SED and PED.%
\item[\textit{Marica}: “black pepper”.] Sp 3.263: \textit{Vacattanti \textsanskrit{setavacaṁ}}, “\textit{Vacatta} is the ‘white sweet flag’.” According to SAF, p. 75, there are two varieties of \textit{\textsanskrit{vacā}}, one red and one white. So it seems reasonable to regard \textit{vacatta} as a sub-species of \textit{\textsanskrit{vacā}} and to render it as “white sweet flag”.%
\item[\textit{Vacatta}/\textit{vacattha}: “white sweet flag”] See SAF, p. 75, and also SED.%
\item[\textit{\textsanskrit{Vacā}}: “sweet flag”, \textit{Acorus calamus}.] See SED and SAF, p. 55.%
\item[\textit{\textsanskrit{Vibhītaka}}: “belleric myrobalan (fruit)”, \textit{Terminalia belleric}.] See SED and PED.%
\item[\textit{\textsanskrit{Siṅgivera}}: “ginger”] See DOP, sv. \textit{tulasi}. That \textit{\textsanskrit{tulasī}} is the same as \textit{\textsanskrit{sulasī}} is suggested in SED, sv. \textit{\textsanskrit{sulasā}}, defined as “holy basil”.%
\item[\textit{\textsanskrit{Sulasī}}/\textit{\textsanskrit{tulasī}}: “holy basil”, \textit{Ocimum sanctum}.] See SED and SAF, p. 57. It is also known as “black myrobalan”.%
\item[\textit{\textsanskrit{Harītaka}}: “chebulic myrobalan (fruit)”, \textit{Terminalia chebula}.] See SED.%
\item[\textit{Haliddi}: “turmeric”, \textit{Curcuma longa}.] See SED. \textit{\textsanskrit{Hiṅgu}}, \textit{\textsanskrit{hiṅgujatu}}, \textit{\textsanskrit{hiṅgusipāṭika}} are all derived from the \textit{\textsanskrit{hiṅgu}} plant. Sp 3.263: \textit{\textsanskrit{Hiṅguhiṅgujatuhiṅgusipāṭikā} \textsanskrit{hiṅgujātiyoyeva}}, “\textit{\textsanskrit{Hiṅguhiṅgujatuhiṅgusipāṭikā}} are all kinds of \textit{\textsanskrit{hiṅgu}}.” Sp‑\textsanskrit{ṭ} 2.249 clarifies the distinction between the three: \textit{\textsanskrit{Hiṅgūti} \textsanskrit{hiṅgurukkhato} \textsanskrit{paggharitaniyyāso}. \textsanskrit{Hiṅgujatuādayopi} \textsanskrit{hiṅguvikatiyo} eva. Tattha \textsanskrit{hiṅgujatu} \textsanskrit{nāma} \textsanskrit{hiṅgurukkhassa} \textsanskrit{daṇḍapattāni} \textsanskrit{pacitvā} \textsanskrit{kataniyyāso}, \textsanskrit{hiṅgusipāṭikaṁ} \textsanskrit{nāma} \textsanskrit{hiṅgupattāni} \textsanskrit{pacitvā} \textsanskrit{kataniyyāso}}, “\textit{\textsanskrit{Hiṅgu}} is the gum exuded from the \textit{\textsanskrit{hiṅgu}} tree. \textit{\textsanskrit{Hiṅgujatu}}, etc. are also made from \textit{\textsanskrit{hiṅgu}}. The gum produced from boiling sticks and leaves of the \textit{\textsanskrit{hiṅgu}} tree is called \textit{\textsanskrit{hiṅgujatu}}. The gum produced from boiling \textit{\textsanskrit{hiṅgu}} leaves is called \textit{\textsanskrit{hiṅgusipāṭika}}.”%
\end{description}

\section*{Trees}

\begin{description}%
\item[\textit{Atimuttaka}: “the Sandan tree”, \textit{Ougeinia oojeinensis}.] Suggested in DOP.%
\item[\textit{Assattha}: “the Bodhi tree”, \textit{Ficus religiosa}.] See CPD.%
\item[\textit{Āmalaka}: “the emblic myrobalan tree”, \textit{Phyllanthus emblica}.] See DOP and CPD. It is also known as “yellow myrobalan”.%
\item[\textit{Udumbara}: “the cluster fig”, \textit{Ficus glomerata} or \textit{Ficus racemosa}.] See CPD and DOP.%
\item[\textit{Kakudha}: “the arjun tree”, \textit{Terminalia arjuna}.] See CPD and DOP.%
\item[\textit{Kacchaka}: “the Indian cedar tree”, \textit{Cedrela toona} or \textit{Toona ciliata}.] See CPD and DOP.%
\item[\textit{Kapitthana}: “the portia tree”, \textit{Thespesia populnea}.] See DOP and SAF, p. 73, sv. \textit{\textsanskrit{Kapītana}}.%
\item[\textit{Candana}: “the sandal tree”.] See DOP.%
\item[\textit{Jambu}: “the rose-apple tree” or “the black plum tree”, \textit{Eugenia jambolana} or \textit{Syzigium cumini}.] See PED and N\&E, p. 68.%
\item[\textit{Taka}: “the taka tree”.] It is not even clear whether it is the name of a plant or a word for gum, with Vjb 3.263 suggesting it could be either: \textit{\textsanskrit{Takaṁ} \textsanskrit{nāma} tassa rukkhassa \textsanskrit{tacapākodakaṁ} … Atha \textsanskrit{vā} \textsanskrit{takaṁ} \textsanskrit{nāma} \textsanskrit{lākhā}}, “The water from the boiled bark of that tree is called \textit{taka} … alternatively the gum is called \textit{taka}.” In any case, \textit{taka}, \textit{\textsanskrit{takapattiṁ}}, amd \textit{\textsanskrit{takapaṇṇiṁ}} (as listed at \href{https://suttacentral.net/pli-tv-kd6/en/brahmali\#7.1.4}{Kd~6:7.1.4}) are closely related, all apparently stemming from the same plant or being a related kind of gum. Sp 3.363: \textit{\textsanskrit{Takatakapattitakapaṇṇiyo} \textsanskrit{lākhājātiyo}}, “\textit{\textsanskrit{Takatakapattitakapaṇṇiyo}} are kinds of gum.” Sp‑\textsanskrit{ṭ} 2.249 elaborates: \textit{Takanti \textsanskrit{aggakoṭiyā} nikkhantasileso. Takapattinti pattato nikkhantasileso. \textsanskrit{Takapaṇṇinti} \textsanskrit{palāse} \textsanskrit{bhajjitvā} katasileso. \textsanskrit{Daṇḍato} nikkhantasileso tipi vadanti}, “\textit{Taka}: the gum exuding at the highest point. \textit{Takapatti}: the gum exuding from the leaves. \textit{\textsanskrit{Takapaṇṇi}}: the gum produced from roasting the foliage. They also say it is the gum exuding from sticks.”%
\item[\textit{\textsanskrit{Tālīsa}}: “the coffee plum”, \textit{Flacourtia cataphracta}.] See SED and DOP.%
\item[\textit{\textsanskrit{Timbarūsaka}}: “the Gaub tree”, \textit{Diospyros malabarica}.] See SAF, p. 81.%
\item[\textit{\textsanskrit{Tirīṭaka}}: “the lodh tree”, \textit{Symplocos racemosa}.] See SED and SAF, p. 73.%
\item[\textit{\textsanskrit{Nattamāla}}: “the Indian beech”, \textit{Pongamia pinnata}.] See DOP and SAF, p. 83.%
\item[\textit{Nigrodha}: “the banyan tree”, \textit{Ficus benghalensis}.] See SAF, p. 85.%
\item[\textit{Nimba}: “the neem tree”, \textit{Azadirachta indica}.] See DOP and SAF, p. 86.%
\item[\textit{\textsanskrit{Pāricchattaka}}/\textit{\textsanskrit{kovilāra}}: “the orchid tree”, \textit{Bauhinia variegata}.] See DOP.%
\item[\textit{Pilakkha}: “the (Indian) rock fig tree”, \textit{Ficus arnottiana}.] See SAF, p. 90.%
\item[\textit{Pucimanda}: “the neem tree”, \textit{Azadirachta indica}.] See DOP and SAF, p. 86.%
\item[\textit{Phaggava}/\textit{pakkava}/\textit{paggava}/\textit{vaggava}: “the white fig tree”, \textit{Ficus virens/Ficus infectoria.}] N\&E suggests \textit{Ficus rumphii}, sv. \textit{pakkava}, but this seems to be little more than an educated guess based on equating it with Hindi word \textit{pakar}. It seems more likely, however, to be related to the Sanskrit \textit{\textsanskrit{parkaṭa}}/\textit{plaksha}. In Sinhala script \textit{\textsanskrit{ṭ}} and \textit{v} are hardly distinguishable and so it is quite conceivable that there has been a mix-up in the scribal process, whereby \textit{\textsanskrit{pakkaṭa}} has become \textit{pakkava}. According to SED, the \textit{\textsanskrit{parkaṭa}} is the \textit{Ficus virens/Ficus infectoria}.%
\item[\textit{Bodhirukkha}: “the Bodhi tree”.] See SED.%
\item[\textit{\textsanskrit{Mandārava}}: “the coral tree”, \textit{Erythrina indica}.] See SAF, p. 86.%
\item[\textit{Mucalinda}: “the powderpuff tree”, \textit{Barringtonia racemosa}.] See SAF, p. 73.%
\item[\textit{\textsanskrit{Rājāyatana}}: “the (Indian) ape-flower tree”, \textit{Buchanania axillaris}.] See SED and SAF, p. 55.%
\item[\textit{\textsanskrit{Vibhītaka}}: “the belleric myrobalan tree”, \textit{Terminalia belleric}.] See SED and SAF, p. 57. It is also known as “black myrobalan”.%
\end{description}

\section*{Other plants}

\begin{description}%
\item[\textit{Akka}: “the crown flower”, \textit{Calotropis gigantea}.] See CPD, DOP, and SED. But SAF, p. 55, says \textit{Calotropis procera}, “apple of sodom”. I follow the majority view.%
\item[\textit{Ajjuka}: “the shrubby basil”.] SAF says \textit{Orthosiphon pallidus}, whereas SED and CPD say \textit{Ocimum gratissimum}. It seems they are closely related.%
\item[\textit{\textsanskrit{Cāpalasuṇa}}: “spring onion”.] “Spring onion” is suggested with a question mark by DOP. The identity of the \textit{\textsanskrit{cāpalasuṇa}} therefore remains uncertain. Sp 2.797: \textit{\textsanskrit{Cāpalasuṇo} \textsanskrit{amiñjako}, \textsanskrit{aṅkuramattameva} hi tassa hoti}, “The \textit{\textsanskrit{cāpalasuṇa}} does not have cloves. It is just a sprout.”%
\item[\textit{\textsanskrit{Ḍāka}}: “potherb”.] See DOP and SED, sv. \textit{\textsanskrit{śāka}}.%
\item[\textit{\textsanskrit{Phaṇijjaka}}: “the rajmahal hemp”, \textit{Marsdenia tenacissima}.] See SAF, p. 106.%
\item[\textit{\textsanskrit{Bhañjanaka}}: “shallot”.] DOP: “A kind of onion or similar vegetable”. The commentarial description is an almost perfect fit for a shallot. Sp 2.797: \textit{\textsanskrit{Bhañjanako} \textsanskrit{lohitavaṇṇo}. … \textsanskrit{Miñjāya} pana … \textsanskrit{bhañjanakassa} dve}, “The \textit{\textsanskrit{bhañjanaka}} is red. … But in regard to cloves … the \textit{\textsanskrit{bhañjanaka}} has two.”%
\item[\textit{\textsanskrit{Madhūka}}/\textit{madhuka}: “the licorice plant”, \textit{Glycyrrhiza glabra}.] See SED and SAF, p. 96. But according to PED it is the \textit{Bassia latifolia}, “the honey tree”. Sp 3.300: \textit{\textsanskrit{Madhukapānanti} \textsanskrit{madhukānaṁ} \textsanskrit{jātirasena} \textsanskrit{katapānaṁ}}, “\textit{\textsanskrit{Madhukapāna}}: a drink made with the natural juice from licorice.”%
\item[\textit{Mugga}: “the mung bean”, \textit{Vigna radiata} or \textit{Phaseolus radiatus}.] The identity of the \textit{mugga} is hardly in doubt, since this bean goes under the Sanskrit term, \textit{mudga}, to the present day. SED has \textit{Phaseolus mungo}, which refers to the closely related black gram.%
\item[\textit{\textsanskrit{Moragū}}: “chaff-flower grass”, \textit{Achyranthes aspera} or \textit{Celosia cristata}.] See SED. Sp 3.257: \textit{\textsanskrit{Eragū}, \textsanskrit{moragū}, \textsanskrit{majjārū}, \textsanskrit{jantūti} \textsanskrit{imā} catassopi \textsanskrit{tiṇajātiyo}; etehi \textsanskrit{kaṭasārake} ca \textsanskrit{taṭṭikāyo} ca karonti. … \textsanskrit{Moragūtiṇaṁ} \textsanskrit{tambasīsaṁ} \textsanskrit{mudukaṁ} \textsanskrit{sukhasamphassaṁ}, tena \textsanskrit{katataṭṭikā} \textsanskrit{nipajjitvā} \textsanskrit{vuṭṭhitamatte} puna \textsanskrit{uddhumātā} \textsanskrit{hutvā} \textsanskrit{tiṭṭhati}.}, “\textit{\textsanskrit{Eragū}}, \textit{\textsanskrit{moragū}}, \textit{\textsanskrit{majjārū}}, and \textit{jantu}: these are four kinds of grass. Straw mats and grass mats are made with these. … The \textit{\textsanskrit{moragū}} grass has a copper head, is soft, and is pleasant to touch. Soon after getting up after laying down on it, the mat regains its shape and remains like that.” The \textit{eragu} and \textit{jantu} grass have not been identified. SED (sv. \textit{\textsanskrit{mārjāra}}) identifies the ma\textit{\textsanskrit{jjāru}} as either the \textit{Plumbago rosea}, \textit{Terminalia katappa}, or \textit{Agati grandiflora}, but this seems doubtful since the former is a shrub and the two latter are trees.%
\item[\textit{\textsanskrit{Sāṇa}}: “sunn hemp”, \textit{Crotolaria juncea.}] Sp 1.636: \textit{\textsanskrit{Sāṇanti} \textsanskrit{sāṇavākasuttaṁ}}. “\textit{\textsanskrit{Sāṇa}}: thread from the bark of hemp.” SED says: “\textit{\textsanskrit{śaṇa}}, m. L also n.) a kind of hemp, Cannabis Sativa or Crotolaria Juncea …”. Since SED identifies \textit{\textsanskrit{bhaṅga}} with \textit{Cannabis sativa}, I take \textit{\textsanskrit{sāṇa}} to be \textit{Crotolaria juncea}, otherwise known as “sunn hemp”.%
\item[\textit{\textsanskrit{Hintāla}}: “the fishtail palm”, \textit{Caryota urens}.] See SAF, p. 112.%
\item[\textit{Hirivera}: “the Vicks plant”, \textit{Plectranthus zeylanicus} or \textit{Plectranthus hadiensis}.] See SAF, p. 113. According to Tropical Plants Database at https://tropical.theferns.info/viewtropical.php?id=Plectranthus+elegans, “cuttings of \textit{Plectranthus} species generally root easily”, which fits with its categorization at Bu Pc 11.%
\end{description}

%
\chapter*{Colophon}
\addcontentsline{toc}{chapter}{Colophon}
\markboth{Colophon}{Colophon}

\section*{The Translator}

Bhikkhu Brahmali was born Norway in 1964. He first became interested in Buddhism and meditation in his early 20s after a visit to Japan. Having completed degrees in engineering and finance, he began his monastic training as an anagarika (keeping the eight precepts) in England at Amaravati and Chithurst Buddhist Monastery.

After hearing teachings from Ajahn Brahm he decided to travel to Australia to train at Bodhinyana Monastery. Bhikkhu Brahmali has lived at Bodhinyana Monastery since 1994, and was ordained as a Bhikkhu, with Ajahn Brahm as his preceptor, in 1996. In 2015 he entered his 20th Rains Retreat as a fully ordained monastic and received the title Maha Thera (Great Elder).

Bhikkhu Brahmali’s knowledge of the Pali language and of the Suttas is excellent. Bhikkhu Bodhi, who translated most of the Pali Canon into English for Wisdom Publications, called him one of his major helpers for the 2012 translation of \emph{The Numerical Discourses of the Buddha}. He has also published two essays on Dependent Origination and a book called \emph{The Authenticity of the Early Buddhist Texts} with the Buddhist Publication Society in collaboration with Bhante Sujato.

The monastics of the Buddhist Society of WA (BSWA) often turn to him to clarify Vinaya (monastic discipline) or Sutta questions. They also greatly appreciate his Sutta and Pali classes. Furthermore he has been instrumental in most of the building and maintenance projects at Bodhinyana Monastery and at the emerging Hermit Hill property in Serpentine.

\section*{Creation Process}

Translated from the Pali. The primary source was the \textsanskrit{Mahāsaṅgīti} edition, with occasional reference to other Pali editions, especially the \textsanskrit{Chaṭṭha} \textsanskrit{Saṅgāyana} edition and the Pali Text Society edition. I cross-checked with I.B. Horner’s English translation, “The Book of the Discipline”, as well as Bhikkhu \textsanskrit{Ñāṇatusita}’s “A Translation and Analysis of the \textsanskrit{Pātimokkha}” and Ajahn \textsanskrit{Ṭhānissaro}’s “Buddhist Monastic Code”.

\section*{The Translation}

This is the first complete translation of the Vinaya \textsanskrit{Piṭaka} in English. The aim has been to produce a translation that is easy to read, clear, and accurate, and also modern in vocabulary and style.

\section*{About SuttaCentral}

SuttaCentral publishes early Buddhist texts. Since 2005 we have provided root texts in Pali, Chinese, Sanskrit, Tibetan, and other languages, parallels between these texts, and translations in many modern languages. Building on the work of generations of scholars, we offer our contribution freely.

SuttaCentral is driven by volunteer contributions, and in addition we employ professional developers. We offer a sponsorship program for high quality translations from the original languages. Financial support for SuttaCentral is handled by the SuttaCentral Development Trust, a charitable trust registered in Australia.

\section*{About Bilara}

“Bilara” means “cat” in Pali, and it is the name of our Computer Assisted Translation (CAT) software. Bilara is a web app that enables translators to translate early Buddhist texts into their own language. These translations are published on SuttaCentral with the root text and translation side by side.

\section*{About SuttaCentral Editions}

The SuttaCentral Editions project makes high quality books from selected Bilara translations. These are published in formats including HTML, EPUB, PDF, and print.

You are welcome to print any of our Editions.

%
\end{document}