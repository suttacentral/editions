\documentclass[12pt,openany]{book}%
\usepackage{lastpage}%
%
\usepackage{ragged2e}
\usepackage{verse}
\usepackage[a-3u]{pdfx}
\usepackage[inner=1in, outer=1in, top=.7in, bottom=1in, papersize={6in,9in}, headheight=13pt]{geometry}
\usepackage{polyglossia}
\usepackage[12pt]{moresize}
\usepackage{soul}%
\usepackage{microtype}
\usepackage{tocbasic}
\usepackage{realscripts}
\usepackage{epigraph}%
\usepackage{setspace}%
\usepackage{sectsty}
\usepackage{fontspec}
\usepackage{marginnote}
\usepackage[bottom]{footmisc}
\usepackage{enumitem}
\usepackage{fancyhdr}
\usepackage{emptypage}
\usepackage{extramarks}
\usepackage{graphicx}
\usepackage{relsize}
\usepackage{etoolbox}

% improve ragged right headings by suppressing hyphenation and orphans. spaceskip plus and minus adjust interword spacing; increase rightskip stretch to make it want to push a word on the first line(s) to the next line; reduce parfillskip stretch to make line length more equal . spacefillskip and xspacefillskip can be deleted to use defaults.
\protected\def\BalancedRagged{
\leftskip     0pt
\rightskip    0pt plus 10em
\spaceskip=1\fontdimen2\font plus .5\fontdimen3\font minus 1.5\fontdimen4\font
\xspaceskip=1\fontdimen2\font plus 1\fontdimen3\font minus 1\fontdimen4\font
\parfillskip  0pt plus 15em
\relax
}

\hypersetup{
colorlinks=true,
urlcolor=black,
linkcolor=black,
citecolor=black,
allcolors=black
}

% use a small amount of tracking on small caps
\SetTracking[ spacing = {25*,166, } ]{ encoding = *, shape = sc }{ 25 }

% add a blank page
\newcommand{\blankpage}{
\newpage
\thispagestyle{empty}
\mbox{}
\newpage
}

% define languages
\setdefaultlanguage[]{english}
\setotherlanguage[script=Latin]{sanskrit}

%\usepackage{pagegrid}
%\pagegridsetup{top-left, step=.25in}

% define fonts
% use if arno sanskrit is unavailable
%\setmainfont{Gentium Plus}
%\newfontfamily\Marginalfont[]{Gentium Plus}
%\newfontfamily\Allsmallcapsfont[RawFeature=+c2sc]{Gentium Plus}
%\newfontfamily\Noligaturefont[Renderer=Basic]{Gentium Plus}
%\newfontfamily\Noligaturecaptionfont[Renderer=Basic]{Gentium Plus}
%\newfontfamily\Fleuronfont[Ornament=1]{Gentium Plus}

% use if arno sanskrit is available. display is applied to \chapter and \part, subhead to \section and \subsection.
\setmainfont[
  FontFace={sb}{n}{Font = {Arno Pro Semibold}},
  FontFace={sb}{it}{Font = {Arno  Pro Semibold Italic}}
]{Arno Pro}

% create commands for using semibold
\DeclareRobustCommand{\sbseries}{\fontseries{sb}\selectfont}
\DeclareTextFontCommand{\textsb}{\sbseries}

\newfontfamily\Marginalfont[RawFeature=+subs]{Arno Pro Regular}
\newfontfamily\Allsmallcapsfont[RawFeature=+c2sc]{Arno Pro}
\newfontfamily\Noligaturefont[Renderer=Basic]{Arno Pro}
\newfontfamily\Noligaturecaptionfont[Renderer=Basic]{Arno Pro Caption}

% chinese fonts
\newfontfamily\cjk{Noto Serif TC}
\newcommand*{\langlzh}[1]{\cjk{#1}\normalfont}%

% logo
\newfontfamily\Logofont{sclogo.ttf}
\newcommand*{\sclogo}[1]{\large\Logofont{#1}}

% use subscript numerals for margin notes
\renewcommand*{\marginfont}{\Marginalfont}

% ensure margin notes have consistent vertical alignment
\renewcommand*{\marginnotevadjust}{-.17em}

% use compact lists
\setitemize{noitemsep,leftmargin=1em}
\setenumerate{noitemsep,leftmargin=1em}
\setdescription{noitemsep, style=unboxed, leftmargin=1em}

% style ToC
\DeclareTOCStyleEntries[
  raggedentrytext,
  linefill=\hfill,
  pagenumberwidth=.5in,
  pagenumberformat=\normalfont,
  entryformat=\normalfont
]{tocline}{chapter,section}


  \setlength\topsep{0pt}%
  \setlength\parskip{0pt}%

% define new \centerpars command for use in ToC. This ensures centering, proper wrapping, and no page break after
\def\startcenter{%
  \par
  \begingroup
  \leftskip=0pt plus 1fil
  \rightskip=\leftskip
  \parindent=0pt
  \parfillskip=0pt
}
\def\stopcenter{%
  \par
  \endgroup
}
\long\def\centerpars#1{\startcenter#1\stopcenter}

% redefine part, so that it adds a toc entry without page number
\let\oldcontentsline\contentsline
\newcommand{\nopagecontentsline}[3]{\oldcontentsline{#1}{#2}{}}

    \makeatletter
\renewcommand*\l@part[2]{%
  \ifnum \c@tocdepth >-2\relax
    \addpenalty{-\@highpenalty}%
    \addvspace{0em \@plus\p@}%
    \setlength\@tempdima{3em}%
    \begingroup
      \parindent \z@ \rightskip \@pnumwidth
      \parfillskip -\@pnumwidth
      {\leavevmode
       \setstretch{.85}\large\scshape\centerpars{#1}\vspace*{-1em}\llap{#2}}\par
       \nobreak
         \global\@nobreaktrue
         \everypar{\global\@nobreakfalse\everypar{}}%
    \endgroup
  \fi}
\makeatother

\makeatletter
\def\@pnumwidth{2em}
\makeatother

% define new sectioning command, which is only used in volumes where the pannasa is found in some parts but not others, especially in an and sn

\newcommand*{\pannasa}[1]{\clearpage\thispagestyle{empty}\begin{center}\vspace*{14em}\setstretch{.85}\huge\itshape\scshape\MakeLowercase{#1}\end{center}}

    \makeatletter
\newcommand*\l@pannasa[2]{%
  \ifnum \c@tocdepth >-2\relax
    \addpenalty{-\@highpenalty}%
    \addvspace{.5em \@plus\p@}%
    \setlength\@tempdima{3em}%
    \begingroup
      \parindent \z@ \rightskip \@pnumwidth
      \parfillskip -\@pnumwidth
      {\leavevmode
       \setstretch{.85}\large\itshape\scshape\lowercase{\centerpars{#1}}\vspace*{-1em}\llap{#2}}\par
       \nobreak
         \global\@nobreaktrue
         \everypar{\global\@nobreakfalse\everypar{}}%
    \endgroup
  \fi}
\makeatother

% don't put page number on first page of toc (relies on etoolbox)
\patchcmd{\chapter}{plain}{empty}{}{}

% global line height
\setstretch{1.05}

% allow linebreak after em-dash
\catcode`\—=13
\protected\def—{\unskip\textemdash\allowbreak}

% style headings with secsty. chapter and section are defined per-edition
\partfont{\setstretch{.85}\normalfont\centering\textsc}
\subsectionfont{\setstretch{.95}\normalfont\BalancedRagged}%
\subsubsectionfont{\setstretch{1}\normalfont\itshape\BalancedRagged}

% style elements of suttatitle
\newcommand*{\suttatitleacronym}[1]{\smaller[2]{#1}\vspace*{.3em}}
\newcommand*{\suttatitletranslation}[1]{\linebreak{#1}}
\newcommand*{\suttatitleroot}[1]{\linebreak\smaller[2]\itshape{#1}}

\DeclareTOCStyleEntries[
  indent=3.3em,
  dynindent,
  beforeskip=.2em plus -2pt minus -1pt,
]{tocline}{section}

\DeclareTOCStyleEntries[
  indent=0em,
  dynindent,
  beforeskip=.4em plus -2pt minus -1pt,
]{tocline}{chapter}

\newcommand*{\tocacronym}[1]{\hspace*{-3.3em}{#1}\quad}
\newcommand*{\toctranslation}[1]{#1}
\newcommand*{\tocroot}[1]{(\textit{#1})}
\newcommand*{\tocchapterline}[1]{\bfseries\itshape{#1}}


% redefine paragraph and subparagraph headings to not be inline
\makeatletter
% Change the style of paragraph headings %
\renewcommand\paragraph{\@startsection{paragraph}{4}{\z@}%
            {-2.5ex\@plus -1ex \@minus -.25ex}%
            {1.25ex \@plus .25ex}%
            {\noindent\normalfont\itshape\small}}

% Change the style of subparagraph headings %
\renewcommand\subparagraph{\@startsection{subparagraph}{5}{\z@}%
            {-2.5ex\@plus -1ex \@minus -.25ex}%
            {1.25ex \@plus .25ex}%
            {\noindent\normalfont\itshape\footnotesize}}
\makeatother

% use etoolbox to suppress page numbers on \part
\patchcmd{\part}{\thispagestyle{plain}}{\thispagestyle{empty}}
  {}{\errmessage{Cannot patch \string\part}}

% and to reduce margins on quotation
\patchcmd{\quotation}{\rightmargin}{\leftmargin 1.2em \rightmargin}{}{}
\AtBeginEnvironment{quotation}{\small}

% titlepage
\newcommand*{\titlepageTranslationTitle}[1]{{\begin{center}\begin{large}{#1}\end{large}\end{center}}}
\newcommand*{\titlepageCreatorName}[1]{{\begin{center}\begin{normalsize}{#1}\end{normalsize}\end{center}}}

% halftitlepage
\newcommand*{\halftitlepageTranslationTitle}[1]{\setstretch{2.5}{\begin{Huge}\uppercase{\so{#1}}\end{Huge}}}
\newcommand*{\halftitlepageTranslationSubtitle}[1]{\setstretch{1.2}{\begin{large}{#1}\end{large}}}
\newcommand*{\halftitlepageFleuron}[1]{{\begin{large}\Fleuronfont{{#1}}\end{large}}}
\newcommand*{\halftitlepageByline}[1]{{\begin{normalsize}\textit{{#1}}\end{normalsize}}}
\newcommand*{\halftitlepageCreatorName}[1]{{\begin{LARGE}{\textsc{#1}}\end{LARGE}}}
\newcommand*{\halftitlepageVolumeNumber}[1]{{\begin{normalsize}{\Allsmallcapsfont{\textsc{#1}}}\end{normalsize}}}
\newcommand*{\halftitlepageVolumeAcronym}[1]{{\begin{normalsize}{#1}\end{normalsize}}}
\newcommand*{\halftitlepageVolumeTranslationTitle}[1]{{\begin{Large}{\textsc{#1}}\end{Large}}}
\newcommand*{\halftitlepageVolumeRootTitle}[1]{{\begin{normalsize}{\Allsmallcapsfont{\textsc{\itshape #1}}}\end{normalsize}}}
\newcommand*{\halftitlepagePublisher}[1]{{\begin{large}{\Noligaturecaptionfont\textsc{#1}}\end{large}}}

% epigraph
\renewcommand{\epigraphflush}{center}
\renewcommand*{\epigraphwidth}{.85\textwidth}
\newcommand*{\epigraphTranslatedTitle}[1]{\vspace*{.5em}\footnotesize\textsc{#1}\\}%
\newcommand*{\epigraphRootTitle}[1]{\footnotesize\textit{#1}\\}%
\newcommand*{\epigraphReference}[1]{\footnotesize{#1}}%

% map
\newsavebox\IBox

% custom commands for html styling classes
\newcommand*{\scnamo}[1]{\begin{Center}\textit{#1}\end{Center}\bigskip}
\newcommand*{\scendsection}[1]{\begin{Center}\begin{small}\textit{#1}\end{small}\end{Center}\addvspace{1em}}
\newcommand*{\scendsutta}[1]{\begin{Center}\textit{#1}\end{Center}\addvspace{1em}}
\newcommand*{\scendbook}[1]{\bigskip\begin{Center}\uppercase{#1}\end{Center}\addvspace{1em}}
\newcommand*{\scendkanda}[1]{\begin{Center}\textbf{#1}\end{Center}\addvspace{1em}} % use for ending vinaya rule sections and also samyuttas %
\newcommand*{\scend}[1]{\begin{Center}\begin{small}\textit{#1}\end{small}\end{Center}\addvspace{1em}}
\newcommand*{\scendvagga}[1]{\begin{Center}\textbf{#1}\end{Center}\addvspace{1em}}
\newcommand*{\scrule}[1]{\textsb{#1}}
\newcommand*{\scadd}[1]{\textit{#1}}
\newcommand*{\scevam}[1]{\textsc{#1}}
\newcommand*{\scspeaker}[1]{\hspace{2em}\textit{#1}}
\newcommand*{\scbyline}[1]{\begin{flushright}\textit{#1}\end{flushright}\bigskip}
\newcommand*{\scexpansioninstructions}[1]{\begin{small}\textit{#1}\end{small}}
\newcommand*{\scuddanaintro}[1]{\medskip\noindent\begin{footnotesize}\textit{#1}\end{footnotesize}\smallskip}

\newenvironment{scuddana}{%
\setlength{\stanzaskip}{.5\baselineskip}%
  \vspace{-1em}\begin{verse}\begin{footnotesize}%
}{%
\end{footnotesize}\end{verse}
}%

% custom command for thematic break = hr
\newcommand*{\thematicbreak}{\begin{center}\rule[.5ex]{6em}{.4pt}\begin{normalsize}\quad\Fleuronfont{•}\quad\end{normalsize}\rule[.5ex]{6em}{.4pt}\end{center}}

% manage and style page header and footer. "fancy" has header and footer, "plain" has footer only

\pagestyle{fancy}
\fancyhf{}
\fancyfoot[RE,LO]{\thepage}
\fancyfoot[LE,RO]{\footnotesize\lastleftxmark}
\fancyhead[CE]{\setstretch{.85}\Noligaturefont\MakeLowercase{\textsc{\firstrightmark}}}
\fancyhead[CO]{\setstretch{.85}\Noligaturefont\MakeLowercase{\textsc{\firstleftmark}}}
\renewcommand{\headrulewidth}{0pt}
\fancypagestyle{plain}{ %
\fancyhf{} % remove everything
\fancyfoot[RE,LO]{\thepage}
\fancyfoot[LE,RO]{\footnotesize\lastleftxmark}
\renewcommand{\headrulewidth}{0pt}
\renewcommand{\footrulewidth}{0pt}}
\fancypagestyle{plainer}{ %
\fancyhf{} % remove everything
\fancyfoot[RE,LO]{\thepage}
\renewcommand{\headrulewidth}{0pt}
\renewcommand{\footrulewidth}{0pt}}

% style footnotes
\setlength{\skip\footins}{1em}

\makeatletter
\newcommand{\@makefntextcustom}[1]{%
    \parindent 0em%
    \thefootnote.\enskip #1%
}
\renewcommand{\@makefntext}[1]{\@makefntextcustom{#1}}
\makeatother

% hang quotes (requires microtype)
\microtypesetup{
  protrusion = true,
  expansion  = true,
  tracking   = true,
  factor     = 1000,
  patch      = all,
  final
}

% Custom protrusion rules to allow hanging punctuation
\SetProtrusion
{ encoding = *}
{
% char   right left
  {-} = {    , 500 },
  % Double Quotes
  \textquotedblleft
      = {1000,     },
  \textquotedblright
      = {    , 1000},
  \quotedblbase
      = {1000,     },
  % Single Quotes
  \textquoteleft
      = {1000,     },
  \textquoteright
      = {    , 1000},
  \quotesinglbase
      = {1000,     }
}

% make latex use actual font em for parindent, not Computer Modern Roman
\AtBeginDocument{\setlength{\parindent}{1em}}%
%

% Default values; a bit sloppier than normal
\tolerance 1414
\hbadness 1414
\emergencystretch 1.5em
\hfuzz 0.3pt
\clubpenalty = 10000
\widowpenalty = 10000
\displaywidowpenalty = 10000
\hfuzz \vfuzz
 \raggedbottom%

\title{Theravāda Collection on Monastic Law}
\author{Bhikkhu Brahmali}
\date{}%
% define a different fleuron for each edition
\newfontfamily\Fleuronfont[Ornament=9]{Arno Pro}

% Define heading styles per edition for chapter and section. Suttatitle can be either of these, depending on the volume. 

\let\oldfrontmatter\frontmatter
\renewcommand{\frontmatter}{%
\chapterfont{\setstretch{.85}\normalfont\centering}%
\sectionfont{\setstretch{.85}\normalfont\BalancedRagged}%
\oldfrontmatter}

\let\oldmainmatter\mainmatter
\renewcommand{\mainmatter}{%
\chapterfont{\thispagestyle{empty}\setstretch{.85}\normalfont\centering}%
\sectionfont{\clearpage\thispagestyle{plain}\setstretch{.85}\normalfont\centering}%
\oldmainmatter}

\let\oldbackmatter\backmatter
\renewcommand{\backmatter}{%
\chapterfont{\setstretch{.85}\normalfont\centering}%
\sectionfont{\setstretch{.85}\normalfont\BalancedRagged}%
\pagestyle{plainer}%
\oldbackmatter}
%
%
\begin{document}%
\normalsize%
\frontmatter%
\setlength{\parindent}{0cm}

\pagestyle{empty}

\maketitle

\blankpage%
\begin{center}

\vspace*{2.2em}

\halftitlepageTranslationTitle{Theravāda Collection on Monastic Law}

\vspace*{1em}

\halftitlepageTranslationSubtitle{A translation of the Pali Vinaya Piṭaka into English}

\vspace*{2em}

\halftitlepageFleuron{•}

\vspace*{2em}

\halftitlepageByline{translated and introduced by}

\vspace*{.5em}

\halftitlepageCreatorName{Bhikkhu Brahmali}

\vspace*{4em}

\halftitlepageVolumeNumber{Volume 2}

\smallskip

\halftitlepageVolumeAcronym{Bu Vb}

\smallskip

\halftitlepageVolumeTranslationTitle{Analysis of Rules for Monks (2)}

\smallskip

\halftitlepageVolumeRootTitle{Bhikkhu Vibhaṅga}

\vspace*{\fill}

\sclogo{0}
 \halftitlepagePublisher{SuttaCentral}

\end{center}

\newpage
%
\setstretch{1.05}

\begin{footnotesize}

\textit{Theravāda Collection on Monastic Law} is a translation of the Theravāda Vinayapiṭaka by Bhikkhu Brahmali.

\medskip

Creative Commons Zero (CC0)

To the extent possible under law, Bhikkhu Brahmali has waived all copyright and related or neighboring rights to \textit{Theravāda Collection on Monastic Law}.

\medskip

This work is published from Australia.

\begin{center}
\textit{This translation is an expression of an ancient spiritual text that has been passed down by the Buddhist tradition for the benefit of all sentient beings. It is dedicated to the public domain via Creative Commons Zero (CC0). You are encouraged to copy, reproduce, adapt, alter, or otherwise make use of this translation. The translator respectfully requests that any use be in accordance with the values and principles of the Buddhist community.}
\end{center}

\medskip

\begin{description}
    \item[Web publication date] 2021
    \item[This edition] 2025-01-13 01:01:43
    \item[Publication type] hardcover
    \item[Edition] ed3
    \item[Number of volumes] 6
    \item[Publication ISBN] 978-1-76132-006-4
    \item[Volume ISBN] 978-1-76132-008-8
    \item[Publication URL] \href{https://suttacentral.net/editions/pli-tv-vi/en/brahmali}{https://suttacentral.net/editions/pli-tv-vi/en/brahmali}
    \item[Source URL] \href{https://github.com/suttacentral/bilara-data/tree/published/translation/en/brahmali/vinaya}{https://github.com/suttacentral/bilara-data/tree/published/translation/en/brahmali/vinaya}
    \item[Publication number] scpub8
\end{description}

\medskip

Map of Jambudīpa is by Jonas David Mitja Lang, and is released by him under Creative Commons Zero (CC0).

\medskip

Published by SuttaCentral

\medskip

\textit{SuttaCentral,\\
c/o Alwis \& Alwis Pty Ltd\\
Kaurna Country,\\
Suite 12,\\
198 Greenhill Road,\\
Eastwood,\\
SA 5063,\\
Australia}

\end{footnotesize}

\newpage

\setlength{\parindent}{1em}%%
\tableofcontents
\newpage
\pagestyle{fancy}
%
\chapter*{Introduction to the Monks’ Analysis, part II}
\addcontentsline{toc}{chapter}{Introduction to the Monks’ Analysis, part II}
\markboth{Introduction to the Monks’ Analysis, part II}{Introduction to the Monks’ Analysis, part II}

\scbyline{Bhikkhu Brahmali, 2024}

The present volume is the second of six, the total of which comprises a complete translation of the Pali Vinaya \textsanskrit{Piṭaka}, the Monastic Law, of the Theravada school of Buddhism. For a general introduction to the Monastic Law, see volume 1. In the present introduction, I will survey the contents of volume 2 and make observations of points of particular interest.

This volume contains the second and last part of the Bhikkhu-\textsanskrit{vibhaṅga}, “the Monks’ Analysis”, the first part of which is contained in volume 1. Whereas the first volume contains the heavy offenses of \textit{\textsanskrit{pārājika}} and \textit{\textsanskrit{saṅghādisesa}}, also known as \textit{\textsanskrit{garukāpatti}}, the current volume contains the light offenses, or \textit{\textsanskrit{lahukāpatti}}. These offenses are divided into the following classes:

\begin{enumerate}%
\item The \textit{nissaggiya \textsanskrit{pācittiyas}} (NP), “the offenses entailing relinquishment and confession”%
\item The \textit{\textsanskrit{pācittiyas}} (Pc), “the offenses entailing confession”%
\item The \textit{\textsanskrit{pāṭidesanīyas}} (Pd), “the offenses entailing acknowledgment”%
\item The \textit{sekhiyas} (Sk), “the rules of training”%
\item The \textit{\textsanskrit{adhikaraṇasamathadhammas}}, or just \textit{\textsanskrit{adhikaraṇasamathas}} (As), “the principles for settling legal issues”.%
\end{enumerate}

\section*{The \textit{nissaggiya \textsanskrit{pācittiyas}} (NP)}

The \textit{nissaggiya \textsanskrit{pācittiyas}} are offenses entailing relinquishment and confession. They concern cases where a monastic either acquires or uses a requisite in an inappropriate fashion, or acquires something altogether unallowable. The requisite in question then needs to be relinquished before the offense is confessed. There is no further penalty. A curious detail is that the relinquished requisites must be returned to the offender, except if the item is altogether unallowable for a monastic, as in the case of money, gold, or gems. The point of this largely ceremonial relinquishment might be to make the rule stick better in memory. Not returning a relinquished item is itself an offense. In total there are thirty such rules for the monks.

The majority of \textit{nissaggiya \textsanskrit{pācittiya}} rules, in total 23 out of 30, concern cloth, cloth requisites, or thread for such requisites. Of the remaining seven rules, there are two rules about bowls, one about medicinal tonics, one concerned with diverting requisites meant for the Sangha, one about money, and finally two about trading. In what follows I will briefly note a few points of interest drawn from the current chapter.

At \href{https://suttacentral.net/pli-tv-bu-vb-np2/en/brahmali\#2.39.1}{Bu~NP~2} the rule includes the expression “except if the monks have agreed”, \textit{\textsanskrit{aññatra} bhikkhusammutiya}, which is then explained in the word commentary using the synonymous expression \textit{\textsanskrit{ṭhapetvā} \textsanskrit{bhikkhusammutiṁ}}. It is only in the origin story that this is specified as a legal procedure of one motion and one announcement. We find a similar situation in a number of other rules. This gives the impression that the “agreement of the monks” initially was an informal process that over time got formalized as a specific legal procedure. This also points to the word commentaries being older than the origin stories.

At \href{https://suttacentral.net/pli-tv-bu-vb-np5/en/brahmali\#1.1.1}{Bu~NP~5} we find the touching case of a compassionate criminal who makes an offering to the nun \textsanskrit{Uppalavaṇṇā} by hanging some meat from a branch, announcing in her presence that this is offered to any monastic who sees it. She takes the meat, has it prepared, and takes it to the Buddha. At no point is the meat offered into her hands. One is left with a sense that giving to a monastic was quite informal in the earliest period. It is also noteworthy what she says to the monk \textsanskrit{Udāyī}, namely, that it is hard for women to get material support. Some things have not changed much since the time of the Buddha.

In the rule formulation to \href{https://suttacentral.net/pli-tv-bu-vb-np7/en/brahmali\#1.31.1}{Bu~NP~7}, we find the phrase “that monk”, referring to the monk in the previous rule, \href{https://suttacentral.net/pli-tv-bu-vb-np6/en/brahmali\#2.18.1}{Bu~NP~6}. This sort of direct connection between the rules shows unmistakably that at an earlier time the \textsanskrit{Pātimokkha} must have existed as a separate text, presumably going back to a time when the \textsanskrit{Vibhaṅga} did not yet exist.\footnote{At present the two \textsanskrit{Pātimokkhas} of Theravada Buddhism are only found among the sub-commentaries, where they are included in a text called the \textsanskrit{Dvemātikāpāḷi}. It is interesting that it has the suffix \textit{\textsanskrit{pāḷi}}, which indicates its provenance as a Canonical text. At some point the two \textsanskrit{Pātimokkhas} must have been moved to their present position in the sub-commentaries. } \href{https://suttacentral.net/pli-tv-bu-vb-np10/en/brahmali\#1.3.1}{Bu~NP~10}, the longest rule of the Bhikkhu-\textsanskrit{pātimokkha}, includes a full description of how to establish and use a fund set up for the benefit of a monastic, which makes it much more than just a training rule. \href{https://suttacentral.net/pli-tv-bu-vb-np15/en/brahmali\#1.1.1}{Bu~NP~15} starts with the striking story of the monk Upasena who visits the Buddha while the latter is on retreat. It turns out that the local Sangha has laid down a rule that anyone who visits the Buddha commits a \textit{\textsanskrit{pācittiya}} offense. Upasena expresses his disapproval of this and is supported by the Buddha. This reinforces the message of the \textsanskrit{Mahāparinibbāna} Sutta that the Sangha should not add new rules or remove existing ones, but should practice according to what has been laid down by the Buddha (\href{https://suttacentral.net/dn16/en/sujato\#1.6.13}{DN~16:1.6.13}).

At \href{https://suttacentral.net/pli-tv-bu-vb-np18/en/brahmali\#1.28.1}{Bu~NP~18} we find the rule against a monastic accepting gold or silver, which is then defined as money in the word commentary. This rule is significant because the Buddha elsewhere warns about the danger of money for a monastic (\href{https://suttacentral.net/pli-tv-kd22/en/brahmali\#1.3.1}{Kd~22:1.3.1}–1.5.1). At \href{https://suttacentral.net/pli-tv-bu-vb-np23/en/brahmali\#1.1.1}{Bu~NP~23}, we have the extraordinary story of Pilindavaccha who uses his psychic powers to turn a pad of grass into a golden garland and then does the same with the king’s house. The same story is found at \href{https://suttacentral.net/pli-tv-kd6/en/brahmali\#15.1.1}{Kd~6}, in which Pilindavaccha appears a number of times. He also appears in a case study at \href{https://suttacentral.net/pli-tv-bu-vb-pj2/en/brahmali\#7.47.1}{Bu~Pj~2}. Interestingly, although he was clearly an inspiring monk one might expect to encounter in the four main \textsanskrit{Nikāyas}, he is not mentioned there at all except for a single and probably late occurrence at \href{https://suttacentral.net/an1.209%E2%80%93218/en/brahmali\#1.1}{AN~1.215}. This is yet another indication that much of the Vinaya \textsanskrit{Piṭaka} does not belong to the earliest period of Buddhism.

\section*{The \textit{\textsanskrit{pācittiyas}} (Pc)}

The \textit{\textsanskrit{pācittiyas}}, “the offenses entailing confession”, are the largest class of rules in the monks’ \textsanskrit{Pātimokkha}, numbering in total 92. There is no penalty for committing these offenses apart from the confession itself. The confession formula, which is found at \href{https://suttacentral.net/pli-tv-kd2/en/brahmali\#27.1.8}{Kd~2:27.1.8}–27.1.14, involves recognizing that one has committed an offense and undertaking restraint for the future. The severity of committing a \textit{\textsanskrit{pācittiya}} offense is illustrated through a simile at \href{https://suttacentral.net/an4.244/en/sujato\#3.1}{AN~4.244}, which compares it to being struck on the head with a sack of ashes.\footnote{The Canonical text just says a sack of ashes, which the commentary says is used to strike one on the head. }

The \textit{\textsanskrit{pācittiyas}} are classed as \textit{khuddaka}, that is, minor offenses, and are called by this name at the end of the chapter. This is interesting, for it sheds light on expressions such as the \textit{\textsanskrit{khuddānukhuddaka}},\footnote{For instance at \href{https://suttacentral.net/pli-tv-bu-vb-pc72/en/brahmali\#1.20.1}{Bu~Pc~72}. } which must then be a reference to the \textit{\textsanskrit{pācittiyas}} and perhaps other offenses of even lesser importance. The \textit{\textsanskrit{pācittiya}} chapter is divided into nine subchapters with ten rules each, except for the penultimate subchapter which has twelve.

The first subchapter begins with the fundamental moral rule against lying (\href{https://suttacentral.net/pli-tv-bu-vb-pc1/en/brahmali\#1.20.1}{Bu~Pc~1}), followed by a rule against verbal abuse (\href{https://suttacentral.net/pli-tv-bu-vb-pc2/en/brahmali\#1.2.1}{Bu~Pc~2}), which includes a retelling of the \textsanskrit{Nandivisāla} \textsanskrit{Jātaka}, number 28 of that collection. The \textsanskrit{Vibhaṅga} to this rule shows that at the time of the Buddha, or shortly thereafter, India was already a society divided by caste, name, and occupation. This rule is also an example of a tendency I mentioned earlier of monastic victims of misconduct being better protected by the Vinaya than lay victims. There is a full offense of \textit{\textsanskrit{pācittiya}} for abusing another monastic, but just \textit{\textsanskrit{dukkaṭa}} (an offense of wrong conduct) for abusing a lay person. The same difference in offense is also found for the next rule, \href{https://suttacentral.net/pli-tv-bu-vb-pc3/en/brahmali\#2.3.4}{Bu~Pc~3}, as well as for \href{https://suttacentral.net/pli-tv-bu-vb-pc13/en/brahmali\#3.2.3.}{Bu~Pc~13} and elsewhere.

\href{https://suttacentral.net/pli-tv-bu-vb-pc8/en/brahmali\#1.2.26.1}{Bu~Pc~8} is an notable rule that prohibits monastics from telling lay people about their own superhuman qualities, including deep meditations, deep insights, and psychic powers. According to the origin story, monks told lay people of their superhuman qualities in order to receive sufficient food during a famine. Although the \textit{suttas} regard psychic powers as a testimony to one’s spiritual development, they are mostly a private concern. For instance, the Buddha compares showing off psychic powers for the sake of material support to a sex worker showing her private parts for the sake of money (\href{https://suttacentral.net/pli-tv-kd15/en/brahmali\#8.2.15}{Kd~15:8.2.15}–8.2.18). In the Kevaddha Sutta, the Buddha takes this one step further (\href{https://suttacentral.net/dn11/en/sujato\#5.7}{DN~11:5.7}). In the context of using psychic powers to strengthen people’s faith, he says that he detests and abhors them. This is very strong language coming from the Buddha, who normally tends to be understated. He explains his attitude by saying that people without confidence will simply dismiss such powers as a magic trick. The overall point seems to be that talking about any of these things to lay people tends to cheapen them, which is the opposite of what one should try to achieve.

\href{https://suttacentral.net/pli-tv-bu-vb-pc10/en/brahmali\#1.16.1}{Bu~Pc~10} and \href{https://suttacentral.net/pli-tv-bu-vb-pc11/en/brahmali\#1.29.1}{Bu~Pc~11} prohibit a monastic from digging the earth or destroying vegetation. The origin stories to both rules tell of lay people complaining that monastics were destroying life with one sense faculty. In other words, the earth and plants were regarded by the general population, or at least some people, as having a rudimentary form of sentience. The Buddha seems to dismiss this as superstition, saying, “People regard the earth/trees as conscious”, but nevertheless lays down a rule, apparently to satisfy the lay people. Interestingly, the issue of whether plants, especially, are conscious is never finally resolved, for the Buddha never makes an explicit statement either way. What we can say, however, is that either action was considered a minor issue. The \textsanskrit{Vibhaṅga} to both these rules allows monastics to give broad hints if they need a hole in the ground or vegetation removed. The remaining rules in the second subchapter mostly concern proper conduct in relation to dwellings and furniture.

Subchapter three concerns the relationship between monks and nuns. A significant aspect of this relationship is that the monks have a duty to give a half-monthly instruction to the nuns. To ensure the quality of this instruction, \href{https://suttacentral.net/pli-tv-bu-vb-pc21/en/brahmali\#1.44.1}{Bu~Pc~21} lays down that an instructing monk has to be appointed by the Sangha and meet a set of minimum standards, which include good moral conduct, detailed knowledge of both \textsanskrit{Pātimokkhas}, and 20 years of seniority. Most of the remaining rules in this subchapter prohibit various inappropriate interactions between monks and nuns, including being together in private.

Subchapter four is all about food. It is here that we find the rule that a monastic can only eat between dawn and noon, \href{https://suttacentral.net/pli-tv-bu-vb-pc37/en/brahmali\#1.22.1}{Bu~Pc~37}, and that they cannot store food, \href{https://suttacentral.net/pli-tv-bu-vb-pc38/en/brahmali\#1.18.1}{Bu~Pc~38}. Both of these rules are fundamental to how the monastic life works, in that they make it necessary for monastics and their lay supporters to have daily contact. \href{https://suttacentral.net/pli-tv-bu-vb-pc39/en/brahmali\#2.10.1}{Bu~Pc~39} and \href{https://suttacentral.net/pli-tv-bu-vb-sk37/en/brahmali\#2.10.1}{Sk~37} prohibit a monastic from asking for specific foods. A monastic is supposed to be content with whatever they receive. The last rule of this subchapter prohibits a monastic from eating anything that has not been given.

\href{https://suttacentral.net/pli-tv-bu-vb-pc44/en/brahmali\#1.14.1}{Bu~Pc~44} and \href{https://suttacentral.net/pli-tv-bu-vb-pc45/en/brahmali\#1.14.1}{Bu~Pc~45} prohibit a monastic from being alone with someone of the opposite gender, rules that can be surprisingly important to sustain the monastic life. \href{https://suttacentral.net/pli-tv-bu-vb-pc67/en/brahmali\#1.28.1}{Bu~Pc~67} is of the same sort. \href{https://suttacentral.net/pli-tv-bu-vb-pc47/en/brahmali\#1.4.27.1}{Bu~Pc~47}, which allows a monastic to ask a lay person for requisites if invited beforehand, is another ever-relevant rule that governs the relationship between monastics and lay people. The offense is incurred when the monastic goes beyond the limits of the invitation. \href{https://suttacentral.net/pli-tv-bu-vb-pc48/en/brahmali\#2.13.1}{Bu~Pc~48}–50 place strict limits on monastics’ visit to the military. This is in keeping with the fundamentally non-violent nature of Buddhism, especially monasticism.

\href{https://suttacentral.net/pli-tv-bu-vb-pc51/en/brahmali\#1.1}{Bu~Pc~51} opens with the entertaining story of a monk fighting a battle with a dragon, the details of which are similar to the story of the Buddha at \href{https://suttacentral.net/pli-tv-kd1/en/brahmali\#15.3.1}{Kd~1:15.3.1}–15.4.3. At a later time, the same monk becomes so drunk that he collapses and has to be carried back to the monastery by his fellow monastics. The Buddha remarks that this previously powerful monk would now be unable to fight even a lizard! He then lays down the rule against drinking alcohol. Not drinking alcohol is a fundamental aspect of the monastic life (\href{https://suttacentral.net/an4.50/en/sujato}{AN~4.50}).

\href{https://suttacentral.net/pli-tv-bu-vb-pc61/en/brahmali\#1.16.1}{Bu~Pc~61} prohibits a monastic from killing living beings, another fundamental aspect of Buddhist morality. Apart from the present rule and \textit{\textsanskrit{pārājika}} 3, there are several other rules that also concern the killing of living beings, such as \href{https://suttacentral.net/pli-tv-bu-vb-pc20/en/brahmali\#1.12.1}{Bu~Pc~20} and \href{https://suttacentral.net/pli-tv-bu-vb-pc62/en/brahmali\#1.11.1}{Bu~Pc~62}, and indirectly \href{https://suttacentral.net/pli-tv-bu-vb-np11/en/brahmali\#1.23.1}{Bu~NP~11}. Other acts of physical violence are covered by rules such as \href{https://suttacentral.net/pli-tv-bu-vb-pc74/en/brahmali\#1.15.1}{Bu~Pc~74} and \href{https://suttacentral.net/pli-tv-bu-vb-pc75/en/brahmali\#1.15.1}{Bu~Pc~75}.

\href{https://suttacentral.net/pli-tv-bu-vb-pc63/en/brahmali\#1.12.1}{Bu~Pc~63} is one among a number of rules that concern legal issues (\textit{\textsanskrit{adhikaraṇas}}) and their resolution through legal procedures (\textit{\textsanskrit{saṅghakamma}}), all of which are principal topics of the Khandhakas. Other rules that directly concern the \textit{\textsanskrit{adhikaraṇas}} and \textit{\textsanskrit{saṅghakamma}} are \href{https://suttacentral.net/pli-tv-bu-vb-pc79/en/brahmali\#1.22.1}{Bu~Pc~79}–81. Many more rules use \textit{\textsanskrit{saṅghakamma}} to fulfill their purpose, such as \href{https://suttacentral.net/pli-tv-bu-vb-ss10/en/brahmali\#1.3.16.1}{Bu~Ss~10}–13, the four rules that include the phrase “except if the monks have agreed”,\footnote{\href{https://suttacentral.net/pli-tv-bu-vb-np2/en/brahmali\#2.39.1}{Bu~NP~2}, \href{https://suttacentral.net/pli-tv-bu-vb-np14/en/brahmali\#2.39}{14}, and \href{https://suttacentral.net/pli-tv-bu-vb-np29/en/brahmali\#1.2.17}{29}, as well as \href{https://suttacentral.net/pli-tv-bu-vb-pc9/en/brahmali\#1.20.1}{Bu~Pc~9}. } and assorted other rules. As we will see in the introduction to the Khandhakas in volume 4, \textit{\textsanskrit{saṅghakamma}} is fundamental to the workings of monasticism as an institution.

\href{https://suttacentral.net/pli-tv-bu-vb-pc64/en/brahmali\#1.23.1}{Bu~Pc~64} prohibits a monk from concealing another monk’s grave offense. Monasticism is an honor system that relies on trust. If the trust is broken, the whole system starts to fall apart. For this reason, it is important that significantly bad behavior is rooted out as soon as possible. For the nuns, concealing a \textit{\textsanskrit{pārājika}} offense is itself a \textit{\textsanskrit{pārājika}} at \href{https://suttacentral.net/pli-tv-bi-vb-pj6/en/brahmali\#1.23.1}{Bi~Pj~6}.

\href{https://suttacentral.net/pli-tv-bu-vb-pc65/en/brahmali\#1.53.1}{Bu~Pc~65} sets the age limit for the full ordination of men at 20. The offense is committed by the act of ordaining a man below this age. (I will discuss the rules for the ordination of women in the introduction to the Nuns’ Analysis in volume 3.) As one might expect, ordination is an important topic in the Vinaya \textsanskrit{Piṭaka}, with the longest \textit{khandhaka}, Kd 1, devoted to it.

\href{https://suttacentral.net/pli-tv-bu-vb-pc68/en/brahmali\#1.49.1}{Bu~Pc~68}–70 concern the wrong view that sexual intercourse is not an obstacle on the spiritual path. Such views are heavily censured in the Vinaya \textsanskrit{Piṭaka}. The initial offense is one of confession under Bu Pc 68, but for anyone who refuses to relinquish such a view the final result is ejection from the Sangha (\href{https://suttacentral.net/pli-tv-kd11/en/brahmali\#32.1.1}{Kd~11:32.1.1}). Again, it is striking how the Buddha’s view of sensuality is so at odds with the way it tends to be celebrated in regular society.

The origin story to \href{https://suttacentral.net/pli-tv-bu-vb-pc83/en/brahmali\#1.3.56.1}{Bu~Pc~83}, the first rule of the ninth and last subchapter, includes a long section on the dangers of entering a royal compound, which is a parallel to \href{https://suttacentral.net/an10.45/en/sujato}{AN 10.45}. It is essentially a warning against getting too close to power, a very real problem for some monastics, both ancient and modern. \href{https://suttacentral.net/pli-tv-bu-vb-pc84/en/brahmali\#3.17.1}{Bu~Pc~84} prohibits a monastic from picking up valuables, which, besides money, precious metals, and gems, include anything people regard as valuable or useful. The only exception is safekeeping. This is one of only two instances where a monastic is allowed to pick up money, the other being if they have been chosen as a “discarder of money” under \href{https://suttacentral.net/pli-tv-bu-vb-np18/en/brahmali\#2.36}{Bu~NP~18} or \href{https://suttacentral.net/pli-tv-bu-vb-np19/en/brahmali\#2.46}{Bu~NP~19}. \href{https://suttacentral.net/pli-tv-bu-vb-pc85/en/brahmali\#4.9.1}{Bu~Pc~85} prohibits monks from entering an inhabited area at the wrong time, that is, any time after noon until the following dawn. This shows that monks in the early period lived exclusively outside of inhabited areas. The nuns, who were not permitted to live in the wilderness, do not have this rule.

The remainder of the rules in subchapter nine concern requisites that are inappropriate for monastics, including rules on the proper size for cloth requisites. It is here, especially, that we meet with the idea of \textit{sugata} measures. \textit{Sugata} is normally an epithet of the Buddha, meaning something like “the one who has gone to a good destination”, but here seems to be used in the sense of “standard” measures. See the Appendix I: Technical Terms for further discussion.

\section*{The \textit{\textsanskrit{pāṭidesanīyas}} (Pd)}

The next class of offenses is known as the \textit{\textsanskrit{pāṭidesanīyas}}, “offenses entailing acknowledgment”. These, too, are clearable by confession, but the formula to be used is different from that of the \textit{\textsanskrit{pācittiyas}}. The monks have four such offenses.

It is not clear why there are two separate classes of offenses both entailing confession, that is, \textit{\textsanskrit{pācittiyas}} and \textit{\textsanskrit{pāṭidesanīyas}}. According to the similes at \href{https://suttacentral.net/an4.244/en/sujato\#4.1}{AN~4.244}, the committing of a \textit{\textsanskrit{pāṭidesanīya}} offense is equivalent to wearing black and submitting oneself to the wishes of a crowd. The committing a \textit{\textsanskrit{pācittiya}} offense, as we have seen, is compared to being beaten with a sack of ashes, and so it would seem that \textit{\textsanskrit{pācittiya}} offenses are regarded as the more serious. Still, it is hard to see that there is any systematic difference between the two classes. Perhaps the \textit{\textsanskrit{pāṭidesanīyas}} were an early class of offenses that was later abandoned as unnecessary, which may explain why there are so few of them. The unusual formula of confession might perhaps be a remnant from a time when the standard formula of confession had not yet been laid down. Yet, abolishing the class in toto was presumably not practical since the monks were already practicing the existing rules.

The first two of the four involve the nuns. \href{https://suttacentral.net/pli-tv-bu-vb-pd1/en/brahmali\#1.35.1}{Bu~Pd~1} prohibits a monk from receiving almsfood from a nun within an inhabited area. Again, this shows how difficult it could be for women to find support in their practice of the monastic life. Here and elsewhere, such as \href{https://suttacentral.net/pli-tv-bu-vb-np5/en/brahmali\#1.1.1}{Bu~NP~5}, the monks are required to be sensitive to the needs of the nuns. In confessing this offense, and also the last two \textit{\textsanskrit{pāṭidesanīyas}}, a monk is required to say that “I have done a blameworthy and unsuitable thing that is to be acknowledged. I acknowledge it.”

\href{https://suttacentral.net/pli-tv-bu-vb-pd2/en/brahmali\#1.15.1}{Bu~Pd~2} concerns nuns ordering lay people to give food to specific monks. The offense is incurred if the monks do nothing to stop such a nun. The confession formula to this rule is unique and involves the monks confessing together: “We have done a blameworthy and unsuitable thing that is to be acknowledged. We acknowledge it.” \href{https://suttacentral.net/pli-tv-bu-vb-pd3/en/brahmali\#3.15.1}{Bu~Pd~3} and \href{https://suttacentral.net/pli-tv-bu-vb-pd4/en/brahmali\#2.12.1}{Bu~Pd~4} concern unusual circumstances that are unlikely to occur in practice.

\section*{The \textit{sekhiyas} (Sk)}

The second last class of offenses is known as the \textit{sekhiyas}, “rules of training”. There is no automatic offense for breaking these rules, but an offense of \textit{\textsanskrit{dukkaṭa}} if broken out of disrespect. There are 75 such rules. They are the same for the monks and the nuns.

The \textit{sekhiyas} are mostly about etiquette and as such about what was considered socially appropriate in India 2,500 years ago. For this reason, it is convenient that these rules are only offended against when broken out of disrespect. In practice, this means that circumstances such as the prevailing social norms may be taken into account in deciding whether a rule needs to be followed or not. An obvious example is \href{https://suttacentral.net/pli-tv-bu-vb-sk70/en/brahmali\#1.3.1}{Sk~70}, which prohibits a monastic from teaching while standing if the audience is sitting. This rule does not fit the modern practice of speakers standing in the presence of their audience, as in the case of a talk given in an auditorium. In such circumstances one can reasonably argue that this \textit{sekhiya} rule does not apply. Similar considerations apply for a large number of these rules. It follows that the \textit{sekhiyas} have a lower status than the other classes of rules. It is no doubt for this reason that it was considered acceptable to add new \textit{sekhiyas} to the \textsanskrit{Pātimokkha} after the time of the Buddha, as we have discussed in the introduction to volume 1. Nevertheless, even these added rules often draw on material found elsewhere in the Vinaya or the Suttas, and as such may well stem from the earliest period.

That the \textit{sekhiyas} draw on material that exists elsewhere is especially obvious in the case of 55 of the first 56 \textit{sekhiyas}, which are found in similar form in Kd 18.\footnote{See \href{https://suttacentral.net/pli-tv-kd18/en/brahmali\#4.3.1}{Kd~18:4.3.1}–4.6.8. The missing rule is \href{https://suttacentral.net/pli-tv-bu-vb-sk2/en/brahmali\#1.3.1}{Sk~2}. } The question then arises of where they are likely to have appeared first. Normally, in the Khandhakas, when someone is depicted as breaking a \textsanskrit{Pātimokkha} rule, the rule is simply referred to, which means one needs to look it up in the \textsanskrit{Vibhaṅga}. This is not the case for these \textit{sekhiya} rules, which suggests they did not exist in the \textsanskrit{Vibhaṅga} at the time they were laid down in Kd 18. Moreover, the wording in Kd 18 is simpler than the wording in the \textsanskrit{Vibhaṅga}, which includes the refrain \textit{\textsanskrit{sikkhā} \textsanskrit{karaṇīyā}}, “a training to be done”, for every \textit{sekhiya}. This too suggests a movement from the Khandhakas to the \textsanskrit{Vibhaṅga}. Lastly, most of the \textit{sekhiya} rules have artificially created origin stories, with all of them except three featuring the group of six monks. This suggests that they were not really independent rules with separate origin stories, but rather part of a general description of right conduct with a single origin story, which is what we find in the Khandhakas (\href{https://suttacentral.net/pli-tv-kd18/en/brahmali\#4.2.1}{Kd~18:4.2.1}). And indeed, this single origin story does feature the group of six monks, which then, presumably, got distributed to all the origin stories of \textit{sekhiyas} 1–56, with the exception of \href{https://suttacentral.net/pli-tv-bu-vb-sk51/en/brahmali\#1.1}{Sk~51}, \href{https://suttacentral.net/pli-tv-bu-vb-sk55/en/brahmali\#1.1}{Sk~55}, and \href{https://suttacentral.net/pli-tv-bu-vb-sk56/en/brahmali\#1.1}{Sk~56}.

A brief review of the correspondence tables in Pachow’s comparative study of the Bhikkhu-\textsanskrit{pātimokkha} suggests to me that the rules of the different schools are more closely aligned for the last 19 \textit{sekhiyas} than for the first 56. This is especially so if we remove \href{https://suttacentral.net/pli-tv-bu-vb-sk65/en/brahmali\#1.3.1}{Sk~65} and \href{https://suttacentral.net/pli-tv-bu-vb-sk68/en/brahmali\#1.3.1}{Sk~68}, which are not attested in many of the non-Pali schools.\footnote{Pachow, 1955, appendix IV, pp. 15–22. I am taking the Pali rules as the baseline. } If this is correct, and given the possibility that many of the \textit{sekhiyas} may have been moved from the Khandhakas to the \textsanskrit{Vibhaṅga}, it may be that the remaining 17 rules, more or less, are the original common core around which the \textit{sekhiyas} developed as a chapter. A number of these rules do in fact have a word commentary, which is rare among the \textit{sekhiyas}. This too suggests that they developed as independent rules and were not merely transplanted from elsewhere.

As to the content of the \textit{sekhiyas}, the first 26 concern the proper behavior when walking for alms and sitting in inhabited areas. The next 30 rules lay down the proper conduct and etiquette in relation to eating. We then have 16 rules on not teaching anyone not showing appropriate respect. The last three rules are about spitting and toilet etiquette.

Despite the general homogeneity in the \textsanskrit{Vibhaṅga} material of the \textit{sekhiyas}, there are occasional points of interest. At \href{https://suttacentral.net/pli-tv-bu-vb-sk51/en/brahmali\#1.1}{Sk~51}, a monk makes a joke when he hears the slurping of his fellow monks, upon which the Buddha lays down a \textit{\textsanskrit{dukkaṭa}} offense for joking about the Buddha, Dhamma, or Sangha. This is not, however, a general rule against joking, with the Buddha himself occasionally making humorous remarks. In the origin story to \href{https://suttacentral.net/pli-tv-bu-vb-sk69/en/brahmali\#1.12.1}{Sk~69}, we find a \textsanskrit{Jātaka} story, the Chavaka \textsanskrit{Jātaka}, number 309 of that collection.

\section*{The \textit{\textsanskrit{adhikaraṇasamathadhammas}} (As)}

The final class consists of the \textit{\textsanskrit{adhikaraṇasamathadhammas}}, “the principles for settling legal issues”. As I have pointed out in the introduction to volume 1, these principles stand out as not fitting very well with the other rules of the \textsanskrit{Pātimokkha}, which are all rules about individual conduct. These seven principles, in contrast, are regulations to be used by the Sangha to resolve legal issues that may arise in the monastic community.

It is not just the nature of these principles that are different from the rest of the \textsanskrit{Pātimokkha} rules, but also their presentation. There is virtually no \textit{\textsanskrit{vibhaṅga}} material, with the seven principles presented as little more than seven key words, not very different from an index. It is impossible to understand even the meaning of these principles from these key words alone, let alone how they are to be applied to resolve legal issues. For this reason, we are compelled to assume that this section must have been much longer earlier on, with the explanatory \textit{\textsanskrit{vibhaṅga}} material eventually moved as the Vinaya expanded and became unwieldy.

In the earliest period, this explanatory material may have looked like the exposition we now find in the \textsanskrit{Sāmagāma} Sutta, where we see a succinct explanation of these principles.\footnote{\href{https://suttacentral.net/mn104/en/sujato\#12.1}{MN~104:12.1}–20.13. } In addition to this, the \textit{\textsanskrit{adhikaraṇasamathas}} may have been the container for the various \textit{\textsanskrit{saṅghakammas}} that must have been laid down early on, such as the observance-day ceremony, the \textit{uposatha}, and the ordination ceremony. Over time, as the Vinaya developed, it would have been impractical to keep all this material as part of the \textsanskrit{Pātimokkha} or even the \textsanskrit{Vibhaṅga}. A whole new section was then created, which became the Khandhakas. The material specific to the \textit{\textsanskrit{adhikaraṇasamathas}} was collected in a purpose-made chapter, now known as the Samatha-kkhandhaka, that is, the Chapter on the Settling of Legal issues (Kd 14). Other material, such as the \textit{\textsanskrit{saṅghakammas}}, was distributed, as appropriate, in specially created chapters. Over time, more material would have been added. The end result of this process is the twenty-two Khandhakas as we have them now. I will discuss this process further in my introduction to part I of the Khandhakas in volume 4.

We may speculate as to why a mere list of these principles was retained in the \textsanskrit{Pātimokkha}. It could be that the number of rules was regarded as fixed and should not be changed. Or it could be that this bridge between the \textsanskrit{Pātimokkha} and the Khandhakas was considered useful. Whatever the case may be, we now have a reasonable explanation for why we only have a stub of these principles left in the \textsanskrit{Pātimokkha} itself.

Further discussion of the seven \textit{\textsanskrit{adhikaraṇasamathas}} is found in the introduction to part II of the Khandhakas in volume 5.

%
\chapter*{Abbreviations}
\addcontentsline{toc}{chapter}{Abbreviations}
\markboth{Abbreviations}{Abbreviations}

\begin{description}%
\item[AN] \textsanskrit{Aṅguttara} Nikāya (references are to Nipāta and \textit{sutta} numbers)%
\item[AN-a] \textsanskrit{Aṅguttara} Nikāya \textsanskrit{aṭṭhakathā}, the commentary on the \textsanskrit{Aṅguttara} Nikāya%
\item[As] \textit{\textsanskrit{adhikaraṇasamathadhamma}}%
\item[Ay] \textit{aniyata}%
\item[Bi] \textit{\textsanskrit{bhikkhunī}}%
\item[Bu] \textit{bhikkhu}%
\item[CPD] Critical Pali Dictionary%
\item[DN] \textsanskrit{Dīgha} \textsanskrit{Nikāya} (references are to \textit{sutta} numbers)%
\item[DN-a] \textsanskrit{Dīgha} \textsanskrit{Nikāya} \textsanskrit{aṭṭhakathā}, the commentary on the \textsanskrit{Dīgha} \textsanskrit{Nikāya}%
\item[DOP] Dictionary of Pali%
\item[f, ff] and the following page, pages%
\item[Iti] Itivuttaka (references are to verse numbers)%
\item[Ja] \textsanskrit{Jātaka} and \textsanskrit{Jātaka} \textsanskrit{aṭṭhakathā}%
\item[Kd] Khandhaka%
\item[Khuddas-\textsanskrit{pṭ}] \textsanskrit{Khuddasikkhā}-\textsanskrit{purāṇaṭīkā} (references are to paragraph numbers)%
\item[Khuddas-\textsanskrit{nṭ}] \textsanskrit{Khuddasikkhā}-\textsanskrit{abhinavaṭīkā} (references are to paragraph numbers)%
\item[Kkh] \textsanskrit{Kaṅkha}̄\textsanskrit{vitaraṇi}̄%
\item[Kkh-\textsanskrit{pṭ}] \textsanskrit{Kaṅkhāvitaraṇīpurāṇa}-\textsanskrit{ṭīkā}%
\item[MN] Majjhima \textsanskrit{Nikāya} (references are to \textit{sutta} numbers)%
\item[MN-a] Majjhima \textsanskrit{Nikāya} \textsanskrit{aṭṭhakathā}, the commentary on the Majjhima \textsanskrit{Nikāya}%
\item[MS] \textsanskrit{Mahāsaṅgīti} \textsanskrit{Tipiṭaka} (the version of the \textsanskrit{Tipiṭaka} found on SuttaCentral)%
\item[N\&E] “Nature and the Environment in Early Buddhism”, Bhante Dhammika%
\item[Nidd-a] \textsanskrit{Mahāniddesa} \textsanskrit{aṭṭhakathā} (references are to VRI edition paragraph numbers)%
\item[NP] \textit{nissaggiya \textsanskrit{pācittiya}}%
\item[p., pp.] page, pages%
\item[Pc] \textit{\textsanskrit{pācittiya}}%
\item[Pd] \textit{\textsanskrit{pāṭidesanīya}}%
\item[PED] Pali English Dictionary%
\item[Pj] \textit{\textsanskrit{pārājika}}%
\item[PTS] Pali Text Society%
\item[Pvr] \textsanskrit{Parivāra}%
\item[SAF] “South Asian Flora as reflected in the twelfth-century Pali lexicon \textsanskrit{Abhidhānapadīpikā}”, J. Liyanaratne%
\item[SED] Sanskrit English Dictionary%
\item[Sk] \textit{sekhiya}%
\item[SN] \textsanskrit{Saṁyutta} \textsanskrit{Nikāya} (references are to \textsanskrit{Saṁyutta} and \textit{sutta} numbers)%
\item[SN-a] \textsanskrit{Saṁyutta} \textsanskrit{Nikāya} \textsanskrit{aṭṭhakathā}, the commentary on the \textsanskrit{Saṁyutta} \textsanskrit{Nikāya} (references are to volume number and paragraph numbers of the VRI version)%
\item[Sp] Samantapāsādikā, the commentary on the Vinaya \textsanskrit{Piṭaka} (references are to volume and paragraph numbers of the VRI version)%
\item[Sp‑ṭ] Sāratthadīpanī-ṭīkā (references follow the division into five volumes of the Canonical text and then add the paragraph number of the VRI version of the sub-commentary)%
\item[Sp‑yoj] \textsanskrit{Pācityādiyojanā} (volume numbers match those of Sp of the online VRI version, which, given that Sp‑yoj starts with the \textit{bhikkhu \textsanskrit{pācittiyas}}, means that Sp‑yoj is divided into four volumes, starting at volume 2; paragraph numbers are those of the VRI version)%
\item[SRT] Siamrath \textsanskrit{Tipiṭaka}, official edition of the \textsanskrit{Tipiṭaka} published in Thailand%
\item[Ss] \textit{\textsanskrit{saṅghādisesa}}%
\item[sv.] \textit{sub voce}, see under%
\item[\textsanskrit{Thīg}] \textsanskrit{Therīgāthā}%
\item[Ud-a] \textsanskrit{Udāna} \textsanskrit{aṭṭhakathā}, the commentary on the \textsanskrit{Udāna} (references are to \textit{sutta} number)%
\item[Vb] \textsanskrit{Vibhaṅga}, the second book of the Abhidhamma \textsanskrit{Piṭaka}%
\item[Vin-\textsanskrit{ālaṅ}-\textsanskrit{ṭ}] \textsanskrit{Vinayālaṅkāra}-\textsanskrit{ṭīkā} (references are to chapter number and paragraph numbers of the VRI version)%
\item[Vin-vn-\textsanskrit{ṭ}] \textsanskrit{Vinayavinicchayaṭīkā} (references are to paragraph numbers of the VRI version)%
\item[Vjb] \textsanskrit{Vajirabuddhiṭīkā} (references are to volume and paragraph numbers of the VRI version)%
\item[Vmv] \textsanskrit{Vimativinodanī}-\textsanskrit{ṭīkā} (references are to volume and paragraph numbers of the VRI version)%
\item[VRI] Vipassana Research Institute, the publisher of the online version of the Sixth Council edition of the Pali Canon at https://www.tipitaka.org%
\item[Vv-a] \textsanskrit{Vimānavatthu} \textsanskrit{aṭṭhakathā}, the commentary on the \textsanskrit{Vimānavatthu} (references are to paragraph numbers of the VRI edition).%
\end{description}

%
\mainmatter%
\pagestyle{fancy}%
\addtocontents{toc}{\let\protect\contentsline\protect\nopagecontentsline}
\part*{Analysis of Rules for Monks (2)}
\addcontentsline{toc}{part}{Analysis of Rules for Monks (2)}
\markboth{}{}
\addtocontents{toc}{\let\protect\contentsline\protect\oldcontentsline}

%
\addtocontents{toc}{\let\protect\contentsline\protect\nopagecontentsline}
\chapter*{Relinquishment With Confession }
\addcontentsline{toc}{chapter}{\tocchapterline{Relinquishment With Confession }}
\addtocontents{toc}{\let\protect\contentsline\protect\oldcontentsline}

%
%
\section*{{\suttatitleacronym Bu Np 1}{\suttatitletranslation 1. The training rule on the robe season }{\suttatitleroot Kathina}}
\addcontentsline{toc}{section}{\tocacronym{Bu Np 1} \toctranslation{1. The training rule on the robe season } \tocroot{Kathina}}
\markboth{1. The training rule on the robe season }{Kathina}
\extramarks{Bu Np 1}{Bu Np 1}

Venerables,\marginnote{0.6} these thirty rules on relinquishment and confession come up for recitation. 

\subsection*{Origin story }

\subsubsection*{First sub-story }

At\marginnote{1.1} one time when the Buddha was staying at \textsanskrit{Vesālī} at the Gotamaka Shrine,\footnote{In the heading above, and also in the heading to \href{https://suttacentral.net/pli-tv-bu-vb-np3/en/brahmali\#1.1.1}{Bu NP 3:1.1.1} (and also at the end of each of these rules), I translate \textit{kathina} as “robe season” rather than “robe-making ceremony”, since this is the contextual meaning in these rules. For further discussion of the meaning of \textit{kathina}, see Appendix of Technical Terms. } he allowed the three robes for the monks. When they heard about this, the monks from the group of six went to the village in one set of three robes, stayed in the monastery in another set, and went to bathe in yet another set. The monks of few desires complained and criticized them, “How can the monks from the group of six keep extra robes?” 

After\marginnote{1.6} rebuking those monks in many ways, they told the Buddha. Soon afterwards he had the Sangha gathered and questioned the monks: “Is it true, monks, that you do this?” 

“It’s\marginnote{1.8} true, sir.” 

The\marginnote{1.9} Buddha rebuked them … “Foolish men, how can you do this? This will affect people’s confidence …” … “And, monks, this training rule should be recited like this: 

\subsubsection*{Preliminary ruling }

\scrule{‘If a monk keeps an extra robe, he commits an offense entailing relinquishment and confession.’” }

In\marginnote{1.14} this way the Buddha laid down this training rule for the monks. 

\subsubsection*{Second sub-story }

Soon\marginnote{2.1.1} afterwards Venerable Ānanda was offered an extra robe.\footnote{“Was offered” renders \textit{uppanna}. This word, which literally means “arisen”, varies slightly in meaning dependent on the context. Often it refers to a requisite that has just been given to the Sangha or an individual monastic. I then render \textit{uppanna} as “given”. Occasionally however, such as here, this does not fit the context, because Ānanda would have incurred a \textit{nissaggiya \textsanskrit{pācittiya}} offense had he received the robe. In other words, here \textit{uppanna} happens first, and only afterwards is the robe given. The meaning, then, must be that Ānanda had been given an offer or a promise of a robe, but had not yet received it. In this sense the robe had “become available” to him. The most common way for a requisite to become available to a monastic is that an offer is made. I translate accordingly. See also DOP for this meaning of \textit{uppanna}. } He wanted to give it to Venerable \textsanskrit{Sāriputta} who was staying at \textsanskrit{Sāketa}. Knowing that the Buddha had laid down a rule against having extra robes, Ānanda thought, “What should I do now?” He told the Buddha, who said, “How long is it, Ānanda, before \textsanskrit{Sāriputta} returns?” 

“Nine\marginnote{2.13} or ten days.” 

Soon\marginnote{2.14} afterwards the Buddha gave a teaching and addressed the monks: “Monks, you should keep an extra robe for ten days at the most. And so, monks, this training rule should be recited like this: 

\subsection*{Final ruling }

\scrule{‘When his robe is finished and the robe season has ended, a monk should keep an extra robe for ten days at the most. If he keeps it longer than that, he commits an offense entailing relinquishment and confession.’” }

\subsection*{Definitions }

\begin{description}%
\item[When his robe is finished: ] the monk has made a robe; or the robe-cloth has been lost, destroyed, or burned; or his expectation of receiving further robe-cloth is disappointed.\footnote{This refers to robe-cloth received during the robe season. For details see \href{https://suttacentral.net/pli-tv-kd7/en/brahmali\#1.7.3}{Kd 7:1.7.3}–13.2.7. “Finished” refers to the monk either having made a robe or having given up on making a robe. } %
\item[The robe season has ended: ] it has ended according to one of the eight key phrases or the Sangha has ended it.\footnote{For an explanation of this, see \href{https://suttacentral.net/pli-tv-kd7/en/brahmali\#1.7.3}{Kd 7:1.7.3}–12.1.37. } %
\item[For ten days at the most: ] it should be kept ten days at a maximum. %
\item[An extra robe: ] it has not been determined nor assigned to another.\footnote{For an explanation of the idea of \textit{\textsanskrit{vikappanā}}, see Appendix of Technical Terms. } %
\item[Robe: ] one of the six kinds of robe-cloth, but not smaller than what can be assigned to another.\footnote{The six are linen, cotton, silk, wool, sunn hemp, and hemp, see \href{https://suttacentral.net/pli-tv-kd8/en/brahmali\#3.1.6}{Kd 8:3.1.6}. According to \href{https://suttacentral.net/pli-tv-kd8/en/brahmali\#21.1.4}{Kd 8:21.1.4}, the size referred to here is no smaller than 8 by 4 \textit{\textsanskrit{sugataṅgula}}, “standard fingerbreadths”. For a discussion of \textit{\textsanskrit{sugataṅgula}} and \textit{sugata}, see Appendix of Technical Terms. For the rendering of \textit{\textsanskrit{cīvara}} as “robe-cloth”, see the same appendix. } %
\item[If he keeps it longer than that, it becomes subject to relinquishment:\footnote{At first sight this looks like a phrase from the rule to be defined, but, interestingly, it is not actually identical with the phrasing of the rule. } ] it becomes subject to relinquishment at dawn on the eleventh day.\footnote{According to the commentary this means the tenth dawn after one received the robe(-cloth). Sp 1.462: \textit{Tattha \textsanskrit{ekādase} \textsanskrit{aruṇuggamaneti} ettha \textsanskrit{yaṁ} \textsanskrit{divasaṁ} \textsanskrit{cīvaraṁ} \textsanskrit{uppannaṁ} tassa yo \textsanskrit{aruṇo}, so uppannadivasanissito, \textsanskrit{tasmā} \textsanskrit{cīvaruppādadivasaena} \textsanskrit{saddhiṁ} \textsanskrit{ekādase} \textsanskrit{aruṇuggamane} \textsanskrit{nissaggiyaṁ} \textsanskrit{hotīti} \textsanskrit{veditabbaṁ}}, “In this case ‘at dawn on the eleventh day’: here, the dawn of the day that the robe was given depends on the day of giving, and therefore it is to be understood that it is subject to relinquishment on the eleventh dawn including the day that the robe was given.” } %
\end{description}

The\marginnote{3.2.3} robe-cloth should be relinquished to a sangha, a group, or an individual. “And, monks, it should be relinquished like this. After approaching the Sangha, that monk should arrange his upper robe over one shoulder and pay respect at the feet of the senior monks. He should then squat on his heels, raise his joined palms, and say: 

‘Venerables,\marginnote{3.2.6} this robe-cloth, which I have kept over ten days, is to be relinquished. I relinquish it to the Sangha.’ 

After\marginnote{3.2.8} relinquishing it, he should confess the offense. The confession should be received by a competent and capable monk. The relinquished robe-cloth is then to be given back: 

‘Please,\marginnote{3.2.11} venerables, I ask the Sangha to listen. This robe-cloth, which was to be relinquished by monk so-and-so, has been relinquished to the Sangha. If the Sangha is ready, it should give this robe-cloth back to monk so-and-so.’ 

Or:\marginnote{3.2.14} after approaching several monks, that monk should arrange his upper robe over one shoulder and pay respect at the feet of the senior monks. He should then squat on his heels, raise his joined palms, and say:\footnote{“Several” renders \textit{sambahula}.  See Appendix of Technical Terms for a discussion. } 

‘Venerables,\marginnote{3.2.15} this robe-cloth, which I have kept over ten days, is to be relinquished. I relinquish it to you.’ 

After\marginnote{3.2.17} relinquishing it, he should confess the offense. The confession should be received by a competent and capable monk. The relinquished robe-cloth is then to be given back: 

‘Please,\marginnote{3.2.20} venerables, I ask you to listen. This robe-cloth, which was to be relinquished by monk so-and-so, has been relinquished to you. If the venerables are ready, you should give this robe-cloth back to monk so-and-so.’ 

Or:\marginnote{3.2.23} after approaching a single monk, that monk should arrange his upper robe over one shoulder, squat on his heels, raise his joined palms, and say: 

‘This\marginnote{3.2.24} robe-cloth, which I have kept over ten days, is to be relinquished. I relinquish it to you.’ 

After\marginnote{3.2.26} relinquishing it, he should confess the offense. The confession should be received by that monk. The relinquished robe-cloth is then to be given back: 

‘I\marginnote{3.2.29} give this robe-cloth back to you.’” 

\subsection*{Permutations }

If\marginnote{4.1} it is more than ten days and he perceives it as more, he commits an offense entailing relinquishment and confession. If it is more than ten days, but he is unsure of it, he commits an offense entailing relinquishment and confession. If it is more than ten days, but he perceives it as less, he commits an offense entailing relinquishment and confession. 

If\marginnote{4.4} it has not been determined, but he perceives that it has, he commits an offense entailing relinquishment and confession. If it has not been assigned to another, but he perceives that it has, he commits an offense entailing relinquishment and confession. If it has not been given away, but he perceives that it has, he commits an offense entailing relinquishment and confession. If it has not been lost, but he perceives that it has, he commits an offense entailing relinquishment and confession. If it has not been destroyed, but he perceives that it has, he commits an offense entailing relinquishment and confession. If it has not been burned, but he perceives that it has, he commits an offense entailing relinquishment and confession. If it has not been stolen, but he perceives that it has, he commits an offense entailing relinquishment and confession. 

If\marginnote{4.11} he uses robe-cloth that should be relinquished without first relinquishing it, he commits an offense of wrong conduct. If it is less than ten days, but he perceives it as more, he commits an offense of wrong conduct. If it is less than ten days, but he is unsure of it, he commits an offense of wrong conduct. If it is less than ten days and he perceives it as less, there is no offense. 

\subsection*{Non-offenses }

There\marginnote{4.15.1} is no offense: if within ten days it has been determined, assigned to another, given away, lost, destroyed, burned, stolen, or taken on trust;\footnote{“Taken on trust” refers to a situation where you have an agreement with a close friend that you may take their belongings on trust. The conditions for taking on trust are set out at \href{https://suttacentral.net/pli-tv-kd8/en/brahmali\#19.1.5}{Kd 8:19.1.5}. } if he is insane; if he is the first offender. 

Soon\marginnote{5.1} afterwards the monks from the group of six did not give back relinquished robe-cloth. They told the Buddha. 

\scrule{“Monks, relinquished robe-cloth should be given back. If you don’t give it back, you commit an offense of wrong conduct.” }

\scendsutta{The training rule on the robe season, the first, is finished. }

%
\section*{{\suttatitleacronym Bu Np 2}{\suttatitletranslation 2. The training rule on storehouses }{\suttatitleroot Udosita}}
\addcontentsline{toc}{section}{\tocacronym{Bu Np 2} \toctranslation{2. The training rule on storehouses } \tocroot{Udosita}}
\markboth{2. The training rule on storehouses }{Udosita}
\extramarks{Bu Np 2}{Bu Np 2}

\subsection*{Origin story }

\subsubsection*{First sub-story }

At\marginnote{1.1} one time the Buddha was staying at \textsanskrit{Sāvatthī} in the Jeta Grove, \textsanskrit{Anāthapiṇḍika}’s Monastery. At that time the monks stored one of their robes with other monks and then left to wander the country in a sarong and an upper robe. Because they were stored for a long time, the robes became moldy. The monks put them out in the sun. 

Then,\marginnote{1.5} while walking about the dwellings, Venerable Ānanda noticed that the monks were sunning those robes. He asked them, “Whose moldy robes are these?” And they told him what had happened. Venerable Ānanda complained and criticized them, “How can those monks store a robe with other monks and then leave to wander the country in a sarong and an upper robe?” 

After\marginnote{1.11} rebuking those monks in many ways, Venerable Ānanda told the Buddha. Soon afterwards the Buddha had the Sangha gathered and questioned the monks: “Is it true, monks, that there are monks who do this?” 

“It’s\marginnote{1.13} true, sir.” 

The\marginnote{1.14} Buddha rebuked them … “How can those foolish men do this? This will affect people’s confidence …” … “And, monks, this training rule should be recited like this: 

\subsubsection*{Preliminary ruling }

\scrule{‘When his robe is finished and the robe season has ended, if a monk stays apart from his three robes even for a single day, he commits an offense entailing relinquishment and confession.’” }

In\marginnote{1.19} this way the Buddha laid down this training rule for the monks. 

\subsubsection*{Second sub-story }

At\marginnote{2.1} one time a certain monk at \textsanskrit{Kosambī} was sick. His relatives sent him a message, saying, “Come, venerable, we’ll nurse you.” The monks urged him to go, but he said, “The Buddha has laid down a training rule that you can’t be apart from your three robes. Now because I’m sick, I’m unable to travel with my three robes. So I can’t go.” 

They\marginnote{2.11} told the Buddha. Soon afterwards he gave a teaching and addressed the monks: 

\scrule{“Monks, I allow you to give permission to a sick monk to stay apart from his three robes. }

And\marginnote{2.14} it should be given like this. After approaching the Sangha, the sick monk should arrange his upper robe over one shoulder and pay respect at the feet of the senior monks. He should then squat on his heels, raise his joined palms, and say, ‘Venerables, I’m sick. I’m unable to travel with my three robes. I ask the Sangha for permission to stay apart from my three robes.’ And he should ask a second and a third time. A competent and capable monk should then inform the Sangha: 

‘Please,\marginnote{2.22} venerables, I ask the Sangha to listen. The monk so-and-so is sick. He is unable to travel with his three robes. He is asking the Sangha for permission to stay apart from his three robes. If the Sangha is ready, it should give permission to monk so-and-so to stay apart from his three robes. This is the motion. 

Please,\marginnote{2.28} venerables, I ask the Sangha to listen. The monk so-and-so is sick. He is unable to travel with his three robes. He is asking the Sangha for permission to stay apart from his three robes. The Sangha gives permission to monk so-and-so to stay apart from his three robes. Any monk who approves of giving permission to monk so-and-so to stay apart from his three robes should remain silent. Any monk who doesn’t approve should speak up. 

The\marginnote{2.35} Sangha has given permission to monk so-and-so to stay apart from his three robes. The Sangha approves and is therefore silent. I’ll remember it thus.’ 

And\marginnote{2.38} so, monks, this training rule should be recited like this: 

\subsection*{Final ruling }

\scrule{‘When his robe is finished and the robe season has ended, if a monk stays apart from his three robes even for a single day, except if the monks have agreed, he commits an offense entailing relinquishment and confession.’” }

\subsection*{Definitions }

\begin{description}%
\item[When his robe is finished: ] the monk has made a robe; or the robe-cloth has been lost, destroyed, or burned; or his expectation of receiving further robe-cloth is disappointed.\footnote{This refers to the robe-cloth received during the robe season. For details see \href{https://suttacentral.net/pli-tv-kd7/en/brahmali\#1.7.3}{Kd 7:1.7.3}–13.2.7. } %
\item[The robe season has ended: ] it has ended according to one of the eight key phrases or the Sangha has ended it.\footnote{For an explanation of this see \href{https://suttacentral.net/pli-tv-kd7/en/brahmali\#1.7.3}{Kd 7:1.7.3}–12.1.37. } %
\item[If a monk stays apart from his three robes even for a single day: ] from the outer robe, the upper robe, or the sarong. %
\item[Except if the monks have agreed: ] unless the monks have agreed. %
\item[Entailing relinquishment: ] it becomes subject to relinquishment at dawn. %
\end{description}

The\marginnote{3.1.11} robe should be relinquished to a sangha, a group, or an individual. “And, monks, it should be relinquished like this. (To be expanded as in \href{https://suttacentral.net/pli-tv-bu-vb-np1\#3.2.5}{Bu NP 1:3.2.5}–3.2.29, with appropriate substitutions.) 

‘Venerables,\marginnote{3.1.14} this robe, which I have stayed apart from for one day without the agreement of the monks, is to be relinquished. I relinquish it to the Sangha.’ … the Sangha should give … you should give … ‘I give this robe back to you.’” 

\subsection*{Permutations }

\paragraph*{Summary }

An\marginnote{3.2.1} inhabited area may have a single access or many;\footnote{For the rendering of \textit{\textsanskrit{upacāra}} as “access” and \textit{\textsanskrit{gāma}} as “inhabited area”, see Appendix of Technical Terms. } a house may have a single access or many; a storehouse may have a single access or many; a watchtower may have a single access or many; a stilt house may have a single access or many;\footnote{I have here included three kinds of building—the \textit{\textsanskrit{māḷa}}, the \textit{\textsanskrit{pāsāda}}, and the \textit{hammiya}—into one category, “stilt house”. See below for further discussion. } a boat may have a single access or many; a caravan may have a single access or many; a field may have a single access or many; a threshing floor may have a single access or many; a monastery may have a single access or many; a dwelling may have a single access or many; the foot of a tree may have a single access or many; out-in-the-open may have a single access or many. 

\paragraph*{Exposition }

\subparagraph*{An inhabited area }

“An\marginnote{3.3.1} inhabited area with a single access” refers to the following. 

An\marginnote{3.3.2} enclosed inhabited area belonging to one clan: if the robe is kept within the inhabited area, one must stay within the inhabited area. An unenclosed inhabited area belonging to one clan: one must stay in the house where the robe is kept, or not go beyond arm’s reach of the house.\footnote{“Of the house” is added from the commentary. Sp 1.477: \textit{\textsanskrit{Hatthapāsā} \textsanskrit{vā} na vijahitabbanti atha \textsanskrit{vā} \textsanskrit{taṁ} \textsanskrit{gharaṁ} samantato \textsanskrit{hatthapāsā} na \textsanskrit{vijahitabbaṁ}}, “\textit{\textsanskrit{Hatthapāsā} \textsanskrit{vā} na vijahitabbanti}: or one is not to go beyond arm’s reach from any point of that house.” Without this explanation it is hard to make out whether the arm’s reach is from the house or from the robe. In each instance below I have added the respective commentarial explanation, so as to make it clear what the arm’s reach relates to. } 

An\marginnote{3.3.4} enclosed inhabited area belonging to many clans: if the robe is kept in a house, one must stay in that house, in the public meeting hall, or at the gateway to the inhabited area, or not go beyond arm’s reach of the public meeting hall or the gateway.\footnote{Sp 1.479: \textit{Tattha \textsanskrit{saddasaṅghaṭṭanena} \textsanskrit{vā} \textsanskrit{janasambādhena} \textsanskrit{vā} \textsanskrit{vasituṁ} asakkontena \textsanskrit{sabhāye} \textsanskrit{vā} \textsanskrit{vatthabbaṁ} \textsanskrit{nagaradvāramūle} \textsanskrit{vā}. Tatrapi \textsanskrit{vasituṁ} asakkontena yattha katthaci \textsanskrit{phāsukaṭṭhāne} \textsanskrit{vasitvā} \textsanskrit{antoaruṇe} \textsanskrit{āgamma} \textsanskrit{tesaṁyeva} \textsanskrit{sabhāyadvāramūlānaṁ} \textsanskrit{hatthapāsā} \textsanskrit{vā} na \textsanskrit{vijahitabbaṁ}. Gharassa pana \textsanskrit{cīvarassa} \textsanskrit{vā} \textsanskrit{hatthapāse} vattabbameva natthi}, “If one is unable to stay there because of the impact of noise or being confined by people, one should stay in the public meeting hall or at the main gateway to the town. If one is also unable to stay there, one should stay in a comfortable place wherever. Then, before dawn, one should go there, or one should not go beyond arm’s reach of that meeting hall or of the main gateway to the town. But there is no staying within arm’s reach of the house or the robe.” } If one puts aside the robe within arm’s reach while going to the public meeting hall, one must stay in the public meeting hall, or at the gateway to the inhabited area, or not go beyond arm’s reach of either.\footnote{Sp 1.479: \textit{Purimanayeneva \textsanskrit{sabhāye} \textsanskrit{vā} \textsanskrit{vatthabbaṁ} \textsanskrit{dvāramūle} \textsanskrit{vā}, \textsanskrit{hatthapāsā} \textsanskrit{vā} na \textsanskrit{vijahitabbaṁ}}, “According to the previous method, one should stay in the meeting hall or at the main gateway, or not go beyond arm’s reach of either.” } If the robe is kept in the public meeting hall, one must stay in the public meeting hall, or at the gateway to the inhabited area, or not go beyond arm’s reach of either.\footnote{Sp 1.479: \textit{\textsanskrit{Sabhāye} \textsanskrit{nikkhipitvā} pana \textsanskrit{sabhāye} \textsanskrit{vā} tassa sammukhe \textsanskrit{nagaradvāramūle} \textsanskrit{vā} \textsanskrit{tesaṁyeva} \textsanskrit{hatthapāse} \textsanskrit{vā} \textsanskrit{aruṇaṁ} \textsanskrit{uṭṭhāpetabbaṁ}}, “But having put it aside in the public meeting hall, dawn should arrive (while one is) in the meeting hall, in its presence, at the main gateway to the town, or within arm’s reach of these.” } An unenclosed inhabited area belonging to many clans: one must stay in the house where the robe is kept, or not go beyond arm’s reach of the house. 

\subparagraph*{A house }

An\marginnote{3.4.1} enclosed house belonging to one clan and having many rooms: if the robe is kept in the house, one must stay within the house. An unenclosed house belonging to one clan and having many rooms: one must stay in the room where the robe is kept, or not go beyond arm’s reach of the room.\footnote{Sp 1.480: \textit{\textsanskrit{Hatthapāsā} \textsanskrit{vāti} gabbhassa \textsanskrit{hatthapāsā}}, “\textit{\textsanskrit{Hatthapāsā} \textsanskrit{vā}}: beyond arm’s reach of the room.” } 

An\marginnote{3.4.3} enclosed house belonging to many clans and having many rooms: if the robe is kept in a room, one must stay in that room, or at the main entrance, or not go beyond arm’s reach of either.\footnote{“The main entrance” renders \textit{\textsanskrit{dvāramūle}}. Sp 1.480: \textit{\textsanskrit{Sabbesaṁ} \textsanskrit{sādhāraṇe} \textsanskrit{gharadvāramūle}}, “At the base of that door of the house which is common to all.” And then: \textit{\textsanskrit{Hatthapāsā} \textsanskrit{vāti} gabbhassa \textsanskrit{vā} \textsanskrit{gharadvāramūlassa} \textsanskrit{vā} \textsanskrit{hatthapāsā}}, “\textit{\textsanskrit{Hatthapāsā} \textsanskrit{vā}}: beyond arm’s reach of the room or the main entrance to the house.” } An unenclosed house belonging to many clans and having many rooms: one must stay in the room where the robe is kept, or not go beyond arm’s reach of the room. 

\subparagraph*{A storehouse }

An\marginnote{3.5.1} enclosed storehouse belonging to one clan and having many rooms: if the robe is kept in the building, one must stay within the building.\footnote{“Storehouse” renders \textit{uddosita}. Sp 1.482: \textit{Udositoti \textsanskrit{yānādīnaṁ} \textsanskrit{bhaṇḍānaṁ} \textsanskrit{sālā}}, “An \textit{udosita} is a shed for goods such as vehicles, etc.” | Sp 1.482: \textit{Ito \textsanskrit{paṭṭhāya} ca nivesane vuttanayeneva vinicchayo veditabbo}, “Starting from here, the judgment is to be understood by the same method as in regard to a house.” } An unenclosed storehouse belonging to one clan and having many rooms: one must stay in the room where the robe is kept, or not go beyond arm’s reach of the room. 

An\marginnote{3.5.3} enclosed storehouse belonging to many clans and having many rooms: if the robe is kept in a room, one must stay in that room, or at the main entrance, or not go beyond arm’s reach of either. An unenclosed storehouse belonging to many clans and having many rooms: one must stay in the room where the robe is kept, or not go beyond arm’s reach of the room. 

\subparagraph*{A watchtower }

A\marginnote{3.6.1} watchtower belonging to one clan: if the robe is kept in the watchtower, one must stay within the watchtower. 

A\marginnote{3.6.2} watchtower belonging to many clans and having many rooms: one must stay in the room where the robe is kept, or at the main entrance, or not go beyond arm’s reach of either. 

\subparagraph*{A stilt house }

A\marginnote{3.7.1} stilt house belonging to one clan: if the robe is kept in the stilt house, one must stay within the stilt house.\footnote{The building at \href{https://suttacentral.net/pli-tv-bu-vb-np2/en/brahmali\#3.7.1}{Bu NP 2:3.7.1}, the \textit{\textsanskrit{māḷa}}, is hard to distinguish from the next two buildings, the \textit{\textsanskrit{pāsāda}} and the \textit{hammiya}. In fact, according to Sp 1.482 they are all different kinds of \textit{\textsanskrit{pāsāda}}, that is, different kinds of “stilt houses”: \textit{\textsanskrit{Māḷoti} \textsanskrit{ekakūṭasaṅgahito} \textsanskrit{caturassapāsādo}. \textsanskrit{Pāsādoti} \textsanskrit{dīghapāsādo}. Hammiyanti \textsanskrit{muṇḍacchadanapāsādo}}, “A \textit{\textsanskrit{māḷa}} is a square stilt house with (a roof in) a single peak. A \textit{\textsanskrit{pāsāda}} is a long stilt house. A \textit{hammiya} is a stilt house with a bald roof.” Rather than try to name each of these buildings, which in any case would not be useful for practical purposes, I have instead grouped them together as “stilt house”. For practical purposes, what these three buildings have in common—and this is what distinguishes them from the buildings at \href{https://suttacentral.net/pli-tv-bu-vb-np2/en/brahmali\#3.4.1}{Bu NP 2:3.4.1}–3.5.4, the \textit{nivesana} and the \textit{uddosita}—is that they do not have an enclosed category. For a discussion of the \textit{\textsanskrit{pāsāda}}, see Appendix of Technical Terms. } 

A\marginnote{3.7.2} stilt house belonging to many clans and having many rooms: one must stay in the room where the robe is kept, or at the main entrance, or not go beyond arm’s reach of either. 

\subparagraph*{A boat }

A\marginnote{3.10.1} boat belonging to one clan: if the robe is kept on the boat, one must stay on the boat. 

A\marginnote{3.10.2} boat belonging to many clans and having many rooms: one must stay in the room where the robe is kept, or not go beyond arm’s reach of the room. 

\subparagraph*{A caravan }

A\marginnote{3.11.1} caravan belonging to one clan: if the robe is kept within the caravan, one must not go further than 80 meters in front of or behind the caravan, and no further than 11 meters from either side.\footnote{“80 meters” and “11 meters” render seven \textit{abbhantara} and one \textit{abbhantara} respectively. For further discussion of the \textit{abbhantara}, see \textit{sugata} in Appendix of Technical Terms. } 

A\marginnote{3.11.2} caravan belonging to many clans: if the robe is kept within the caravan, one must not go beyond arm’s reach of the caravan.\footnote{Sp 2.1.489: \textit{Satthoti \textsanskrit{jaṅghasattho} \textsanskrit{sakaṭasattho} \textsanskrit{vā} … satthe \textsanskrit{cīvaraṁ} \textsanskrit{nikkhipitvā} \textsanskrit{hatthapāsā} na vijahitabbanti ettha \textsanskrit{hatthapāso} \textsanskrit{nāma} satthassa \textsanskrit{hatthapāsoti} \textsanskrit{veditabbaṁ}}, “Caravan: a caravan of walkers or a caravan of carts … If the robe is kept within the caravan, one must not go beyond arm’s reach: here ‘arm’s reach’ is to be understood as arm’s reach from the caravan.” } 

\subparagraph*{A field }

An\marginnote{3.12.1} enclosed field belonging to one clan: if the robe is kept within the field, one must stay within that field. An unenclosed field belonging to one clan: one must not go beyond arm’s reach of the robe.\footnote{Sp 1.490: \textit{Aparikkhitte \textsanskrit{cīvarasseva} \textsanskrit{hatthapāso}}, “When it is unenclosed, it is just arm’s reach from the robe.” } 

An\marginnote{3.12.3} enclosed field belonging to many clans: if the robe is kept within the field, one must stay at the entrance to the field, or not go beyond arm’s reach of the entrance or the robe.\footnote{Sp 1.490: \textit{\textsanskrit{Nānākulassa} khette \textsanskrit{hatthapāso} \textsanskrit{nāma} \textsanskrit{khettadvārassa} \textsanskrit{hatthapāso}}, “In a field belonging to many clans, arm’s reach means arm’s reach from the entrance to the field.” Sp-\textsanskrit{ṭ} 1.490: adds: \textit{\textsanskrit{Dvāramūlato} \textsanskrit{aññattha} antokhettepi vasantena \textsanskrit{cīvaraṁ} \textsanskrit{hatthapāse} \textsanskrit{katvāyeva} \textsanskrit{vasitabbaṁ}}, “For one who is living within the fleld, but not at the main entrance, one should stay within arm’s reach of the robe.” } An unenclosed field belonging to many clans: one must not go beyond arm’s reach of the robe.\footnote{Sp 1.490: \textit{Aparikkhitte \textsanskrit{cīvarasseva} \textsanskrit{hatthapāso}}, “When it is unenclosed, it is just arm’s reach from the robe.” } 

\subparagraph*{A threshing floor }

An\marginnote{3.13.1} enclosed threshing floor belonging to one clan: if the robe is kept on the threshing floor, one must stay on that threshing floor.\footnote{For the threshing floor and the next category of monastery, arm’s reach has the same meaning as for field. Sp 1.491: \textit{\textsanskrit{Dvīsupi} khette vuttasadisova vinicchayo}, “Also in regard to the two, the judgment is just the same as said in regard to a field.” } An unenclosed threshing floor belonging to one clan: one must not go beyond arm’s reach of the robe. 

An\marginnote{3.13.3} enclosed threshing floor belonging to many clans: if the robe is kept on the threshing floor, one must stay at the entrance to the threshing floor, or not go beyond arm’s reach of the entrance or the robe. An unenclosed threshing floor belonging to many clans: one must not go beyond arm’s reach of the robe. 

\subparagraph*{A monastery }

An\marginnote{3.14.1} enclosed monastery belonging to one clan: if the robe is kept within the monastery, one must stay within that monastery.\footnote{\textit{\textsanskrit{Ārāma}} could be rendered as “park”, which is the more fundamental meaning of the word. However, since such parks were sometimes given to the Sangha to serve as monasteries, as in \textsanskrit{Anāthapiṇḍika}’s \textit{\textsanskrit{ārāma}}, the monasteries too became known by the same name. It is the latter meaning which predominates in the Vinaya \textsanskrit{Piṭaka}. } An unenclosed monastery belonging to one clan: one must not go beyond arm’s reach of the robe. 

An\marginnote{3.14.3} enclosed monastery belonging to many clans: if the robe is kept within the monastery, one must stay at the entrance to the monastery, or not go beyond arm’s reach of the entrance or the robe. An unenclosed monastery belonging to many clans: one must not go beyond arm’s reach of the robe. 

\subparagraph*{A dwelling }

An\marginnote{3.15.1} enclosed dwelling belonging to one clan: if the robe is kept within the dwelling, one must stay within that dwelling.\footnote{Sp 1.493: \textit{\textsanskrit{Vihāro} nivesanasadiso}, “A dwelling is like a house.” In other words, it is to be treated like a house for the purposes of this rule. } An unenclosed dwelling belonging to one clan: one must stay in the dwelling where the robe is kept, or not go beyond arm’s reach of that dwelling. 

An\marginnote{3.15.3} enclosed dwelling belonging to many clans: one must stay in the dwelling where the robe is kept, or at the main entrance to the dwelling, or not go beyond arm’s reach of either. An unenclosed dwelling belonging to many clans: one must stay in the dwelling where the robe is kept, or not go beyond arm’s reach of the dwelling. 

\subparagraph*{The foot of a tree }

At\marginnote{3.16.1} the foot of a tree belonging to one clan: if the robe is kept within the area of the midday shadow of the tree, one must stay within that area. 

At\marginnote{3.16.2} the foot of a tree belonging to many clans: one must not go beyond arm’s reach of the robe.\footnote{Sp 1.494: \textit{\textsanskrit{Idhāpi} \textsanskrit{hatthapāso} \textsanskrit{cīvarahatthapāsoyeva}}, “Also here arm’s reach is just arm’s reach from the robe.” } 

\subparagraph*{In the open }

Out-in-the-open\marginnote{3.17.1} with a single access: in an uninhabited area, in the wilderness, the area covered by a circle with 80 meters radius has a single access.\footnote{“Out-in-the-open” renders \textit{\textsanskrit{ajjhokāsa}}. See Appendix of Technical Terms for discussion. } Whatever lies beyond that has many accesses. 

If\marginnote{3.18.1} he has stayed apart and he perceives that he has, then, except if the monks have agreed, he commits an offense entailing relinquishment and confession. If he has stayed apart, but he is unsure of it, then, except if the monks have agreed, he commits an offense entailing relinquishment and confession. If he has stayed apart, but he does not perceive that he has, then, except if the monks have agreed, he commits an offense entailing relinquishment and confession. 

If\marginnote{3.18.4} it has not been relinquished, but he perceives that it has, then, except if the monks have agreed, he commits an offense entailing relinquishment and confession. If it has not been given away, but he perceives that it has, then, except if the monks have agreed, he commits an offense entailing relinquishment and confession. If it has not been lost, but he perceives that it has, then, except if the monks have agreed, he commits an offense entailing relinquishment and confession. If it has not been destroyed, but he perceives that it has, then, except if the monks have agreed, he commits an offense entailing relinquishment and confession. If it has not been burned, but he perceives that it has, then, except if the monks have agreed, he commits an offense entailing relinquishment and confession. If it has not been stolen, but he perceives that it has, then, except if the monks have agreed, he commits an offense entailing relinquishment and confession. 

If\marginnote{3.18.10} he uses a robe that should be relinquished without first relinquishing it, he commits an offense of wrong conduct. If he has not stayed apart, but he perceives that he has, he commits an offense of wrong conduct. If he has not stayed apart, but he is unsure of it, he commits an offense of wrong conduct. If he has not stayed apart and he does not perceive that he has, there is no offense. 

\subsection*{Non-offenses }

There\marginnote{3.19.1} is no offense: if before dawn it has been relinquished, given away, lost, destroyed, burned, stolen, or taken on trust;\footnote{“Taken on trust” refers to a situation where you have an agreement with a close friend that you may take their belongings on trust. The conditions for taking on trust are set out at \href{https://suttacentral.net/pli-tv-kd8/en/brahmali\#19.1.5}{Kd 8:19.1.5}. } if he has the permission of the monks; if he is insane; if he is the first offender. 

\scendsutta{The training rule on storehouses, the second, is finished. }

%
\section*{{\suttatitleacronym Bu Np 3}{\suttatitletranslation 3. The third training rule on the robe season }{\suttatitleroot Akālacīvara}}
\addcontentsline{toc}{section}{\tocacronym{Bu Np 3} \toctranslation{3. The third training rule on the robe season } \tocroot{Akālacīvara}}
\markboth{3. The third training rule on the robe season }{Akālacīvara}
\extramarks{Bu Np 3}{Bu Np 3}

\subsection*{Origin story }

At\marginnote{1.1.1} one time when the Buddha was staying at \textsanskrit{Sāvatthī} in \textsanskrit{Anāthapiṇḍika}’s Monastery, a monk had been given robe-cloth outside the robe season. While he was making the robe, he realized there was not enough cloth. Lifting it up, he smoothed it out again and again.\footnote{Sp 1.497: \textit{Tattha \textsanskrit{ussāpetvā} \textsanskrit{punappunaṁ} \textsanskrit{vimajjatīti} “\textsanskrit{valīsu} \textsanskrit{naṭṭhāsu} \textsanskrit{idaṁ} \textsanskrit{mahantaṁ} \textsanskrit{bhavissatī}”ti \textsanskrit{maññamāno} udakena \textsanskrit{siñcitvā} \textsanskrit{pādehi} \textsanskrit{akkamitvā} hatthehi \textsanskrit{ussāpetvā} \textsanskrit{ukkhipitvā} \textsanskrit{piṭṭhiyaṁ} \textsanskrit{ghaṁsati}, \textsanskrit{taṁ} \textsanskrit{ātape} \textsanskrit{sukkhaṁ} \textsanskrit{paṭhamappamāṇameva} hoti. So punapi \textsanskrit{tathā} karoti}, “In regard to this \textit{\textsanskrit{ussāpetvā} \textsanskrit{punappunaṁ} vimajjati} means: thinking, ‘When the wrinkles disappear, it will be larger,’ he would sprinkle it with water, step on it with his feet, and then lifting and holding it up with his hands, he would rub it against his back. But when dried in the sun, it returned to its initial size. He would then do it again.” } 

While\marginnote{1.1.5} walking about the dwellings, the Buddha saw that monk acting in this way. He went up to him and said, “What are you doing, monk?” 

“Sir,\marginnote{1.1.8} I’ve been given this out-of-season robe-cloth, but it’s not enough to make a robe. That’s why I lift it up and smooth it out again and again.” 

“Are\marginnote{1.1.11} you expecting to receive more cloth?” 

“I\marginnote{1.1.12} am.” 

Soon\marginnote{1.1.13} afterwards the Buddha gave a teaching and addressed the monks: 

\scrule{“Monks, I allow you to keep out-of-season robe-cloth if you are expecting to receive more.” }

When\marginnote{1.2.1} they heard about this, some monks kept out-of-season robe-cloth for more than a month,  keeping them in bundles on a bamboo robe rack.  While walking about the dwellings, Venerable Ānanda saw that robe-cloth,  and he asked the monks,  “Whose cloth is this?” 

“It’s\marginnote{1.3.1} our out-of-season robe-cloth, which we’re keeping because we’re expecting more.” 

“But\marginnote{1.3.2} how long have you kept it?” 

“More\marginnote{1.3.3} than a month.” 

Venerable\marginnote{1.3.4} Ānanda complained and criticized them, “How can these monks keep out-of-season robe-cloth for more than a month?” 

After\marginnote{1.3.6} rebuking those monks in many ways, Venerable Ānanda told the Buddha. Soon afterwards he had the Sangha gathered and questioned the monks: “Is it true, monks, that there are monks who do this?” 

“It’s\marginnote{1.3.8} true, sir.” 

The\marginnote{1.3.9} Buddha rebuked them … “How can those foolish men keep out-of-season robe-cloth for more than a month? This will affect people’s confidence …” … “And, monks, this training rule should be recited like this: 

\subsection*{Final ruling }

\scrule{‘When his robe is finished and the robe season has ended, if out-of-season robe-cloth is offered to a monk, he may receive it if he wishes. If he receives it, he should quickly make a robe. If there is not enough cloth, but he is expecting more, he should keep it at most one month to make up the lack. If he keeps it longer than that, then even if he expects more cloth, he commits an offense entailing relinquishment and confession.’” }

\subsection*{Definitions }

\begin{description}%
\item[When his robe is finished: ] the monk has made a robe; or the robe-cloth has been lost, destroyed, or burned; or his expectation of receiving further robe-cloth is disappointed.\footnote{This refers to the robe-cloth received during the robe season. For details see \href{https://suttacentral.net/pli-tv-kd7/en/brahmali\#1.7.3}{Kd 7:1.7.3}–13.2.7. } %
\item[The robe season has ended: ] it has ended according to one of the eight key phrases or the Sangha has ended it.\footnote{For an explanation of this see \href{https://suttacentral.net/pli-tv-kd7/en/brahmali\#1.7.3}{Kd 7:1.7.3}–12.1.37. } %
\item[Out-of-season robe-cloth: ] for one who has not participated in the robe-making ceremony, it is robe-cloth given during the eleven months.\footnote{That is, in-season robe-cloth is cloth obtained during the last month of the rainy season, while out-of-season robe-cloth is cloth obtained during the remaining eleven months of the year. \href{https://suttacentral.net/pli-tv-bu-vb-np28/en/brahmali\#2.9}{Bu NP 28:2.9}: \textit{\textsanskrit{Cīvarakālasamayo} \textsanskrit{nāma} anatthate kathine \textsanskrit{vassānassa} pacchimo \textsanskrit{māso}}, “\textit{\textsanskrit{Cīvarakālasamaya}}: for one who has not participated in the robe-making ceremony, it is the last month of the rainy season.” Kkh: \textit{\textsanskrit{Akālacīvaraṁ} \textsanskrit{nāma} \textsanskrit{yvāyaṁ} “anatthate kathine \textsanskrit{vassānassa} pacchimo \textsanskrit{māso} … ”ti \textsanskrit{cīvarakālo} vutto, \textsanskrit{taṁ} \textsanskrit{ṭhapetvā} \textsanskrit{aññadā} \textsanskrit{uppannaṁ} … \textsanskrit{etaṁ} \textsanskrit{akālacīvaraṁ} \textsanskrit{nāma}}, “\textit{\textsanskrit{Akālacīvara}}: when the robe-making ceremony has not been performed, then the last month of the rainy season … is called the robe season. Apart from that, what is given at other times … this is called out-of-season robe-cloth.” “Robe-making ceremony” refers to the \textit{kathina \textsanskrit{saṅghakamma}}, the making of the \textit{kathina} robe, and the rejoicing in the process, the three together represented by the words \textit{(an)atthate kathine}. For further discussion of the meaning of \textit{kathina}, see Appendix of Technical Terms. For the rendering of \textit{\textsanskrit{cīvara}} as “robe-cloth”, see the same appendix. } for one who has participated in the robe-making ceremony, it is robe-cloth given during the seven months.\footnote{That is, in-season robe-cloth is cloth obtained during the last month of the rainy season or during the cold season, while out-of-season robe-cloth is cloth obtained during the remaining seven months of the year. \href{https://suttacentral.net/pli-tv-bu-vb-np28/en/brahmali\#2.9}{Bu NP 28:2.9}: \textit{\textsanskrit{Cīvarakālasamayo} \textsanskrit{nāma} … atthate kathine \textsanskrit{pañcamāsā}}, “\textit{\textsanskrit{Cīvarakālasamaya}}: … for one who has participated in the robe-making ceremony, it is the five month period.”  Kkh: \textit{\textsanskrit{Akālacīvaraṁ} \textsanskrit{nāma} \textsanskrit{yvāyaṁ} “… atthate kathine \textsanskrit{pañcamāsā}”ti \textsanskrit{cīvarakālo} vutto, \textsanskrit{taṁ} \textsanskrit{ṭhapetvā} \textsanskrit{aññadā} \textsanskrit{uppannaṁ} … \textsanskrit{etaṁ} \textsanskrit{akālacīvaraṁ} \textsanskrit{nāma}}, “\textit{\textsanskrit{Akālacīvara}}: … when the robe-making ceremony has been performed, then the five-month period is called the robe season. Apart from that, what is given at other times … this is called out-of-season robe-cloth.” } Also, if it is given in the robe season, but the cloth is designated, it is called “out-of-season robe-cloth”.\footnote{Sp 1.499: \textit{\textsanskrit{Kālepi} \textsanskrit{ādissa} dinnanti \textsanskrit{saṅghassa} \textsanskrit{vā} “\textsanskrit{idaṁ} \textsanskrit{akālacīvaran}”ti \textsanskrit{uddisitvā} \textsanskrit{dinnaṁ}, ekapuggalassa \textsanskrit{vā} “\textsanskrit{idaṁ} \textsanskrit{tuyhaṁ} \textsanskrit{dammī}”ti \textsanskrit{dinnaṁ}}, “‘Also, if it is given in the robe season’: it is given to the Sangha after designating it by saying, ‘This is out-of-season robe-cloth,’ or it is given to an individual by saying, ‘I give this to you.’” In other words, “designated” means designated as out-of-season cloth or designated to an individual. Sp 2.740, commenting on Bi Np 2, adds that designating to a group is included in designated cloth: \textit{Ādissa dinnanti \textsanskrit{sampattā} \textsanskrit{bhājentūti} \textsanskrit{vatvāpi} \textsanskrit{idaṁ} \textsanskrit{gaṇassa} \textsanskrit{idaṁ} \textsanskrit{tumhākaṁ} \textsanskrit{dammīti} \textsanskrit{vatvā} \textsanskrit{vā} \textsanskrit{dātukamyatāya} \textsanskrit{pādamūle} \textsanskrit{ṭhapetvā} \textsanskrit{vā} dinnampi \textsanskrit{ādissa} \textsanskrit{dinnaṁ} \textsanskrit{nāma} hoti; \textsanskrit{etaṁ} sabbampi \textsanskrit{akālacīvaraṁ}}, “‘Given after designating’: also if they give after saying, ‘Let those who are present share it out,’ or after saying, ‘I give this to the group, to you,’ or they place it at the feet (of the recipient) wishing to give, this is called ‘given after designating’. All this is called out-of-season robe-cloth.” } %
\item[If it is offered: ] If it is offered by a sangha, by a group, by a relative, or by a friend, or if it is rags, or if he got it by means of his own property. %
\item[If he wishes: ] if he desires, he may receive it. %
\item[If he receives it, he should quickly make a robe: ] it should be made within ten days.\footnote{That is, in accordance with the stipulations of \href{https://suttacentral.net/pli-tv-bu-vb-np1/en/brahmali\#2.17.1}{Bu NP 1:2.17.1}. } %
\item[If there is not enough cloth: ] if there is not enough cloth when the robe is being made. %
\item[He should keep it at most one month: ] he should keep it one month at a maximum. %
\item[To make up the lack: ] for the purpose of making up the lack. %
\item[But he is expecting more: ] he is expecting more from a sangha, from a group, from a relative, or from a friend, or he is expecting to get rags, or he is expecting to get it by means of his own property. %
\item[If he keeps it longer than that, then even if he expects more cloth: ] if\marginnote{2.2.2} he is given the expected robe-cloth on the same day as he was given the original robe-cloth, it must be made into a robe within ten days. If he is given the expected robe-cloth the day after he was given the original robe-cloth, it must be made into a robe within ten days.\footnote{The Pali idiom is such that \textit{\textsanskrit{dvīhuppanne}}, literally, “given two days ago”, actually means given on the day after he received the original piece of cloth. The way this works, it seems, is that the day on which the cloth was received counts as one. I do the equivalent adjustment for each case below. See also the definition of ten days at \href{https://suttacentral.net/pli-tv-bu-vb-np1/en/brahmali\#3.2.2}{Bu NP 1:3.2.2}, where the eleventh dawn means the tenth dawn after one received the cloth. } If he is given the expected robe-cloth two days after … three days after … eighteen days after … … nineteen days after he was given the original robe-cloth, it must be made into a robe within ten days. If he is given the expected robe-cloth twenty days after he was given the original robe-cloth, it must be made into a robe within nine days. If he is given the expected robe-cloth twenty-one days after he was given the original robe-cloth, it must be made into a robe within eight days. … twenty-two days after … twenty-seven days after … If he is given the expected robe-cloth twenty-eight days after he was given the original robe-cloth, it must be made into a robe within one day. If he is given the expected robe-cloth twenty-nine days after he was given the original robe-cloth, it must be determined, assigned to another, or given away on that very day.\footnote{Again, in the Pali idiom, \textit{\textsanskrit{tiṁse} uppanne}, literally, “given thirty days ago”, actually means given twenty-nine days ago. } If he does not determine it, assign it to another, or give it away, it becomes subject to relinquishment at dawn on the thirtieth day. 

%
\end{description}

The\marginnote{2.2.33} robe-cloth should be relinquished to a sangha, a group, or an individual. “And, monks, it should be relinquished like this. (To be expanded as in \href{https://suttacentral.net/pli-tv-bu-vb-np1\#3.2.5}{Bu NP 1:3.2.5}–3.2.29, with appropriate substitutions.) 

‘Venerables,\marginnote{2.2.36} this out-of-season robe-cloth, which I have kept for more than a month, is to be relinquished. I relinquish it to the Sangha.’ … the Sangha should give … you should give … ‘I give this robe-cloth back to you.’” 

If\marginnote{2.2.41} he is given the expected robe-cloth, but it is different from the robe-cloth originally given to him, and there are days remaining, he does not have to make a robe if he does not want to. 

\subsection*{Permutations }

If\marginnote{2.3.1} it is more than a month and he perceives it as more, he commits an offense entailing relinquishment and confession. If it is more than a month, but he is unsure of it, he commits an offense entailing relinquishment and confession. If it is more than a month, but he perceives it as less, he commits an offense entailing relinquishment and confession. 

If\marginnote{2.3.4} it has not been determined, but he perceives that it has, he commits an offense entailing relinquishment and confession. If it has not been assigned to another, but he perceives that it has, he commits an offense entailing relinquishment and confession.\footnote{For an explanation of the idea of \textit{\textsanskrit{vikappanā}}, see Appendix of Technical Terms. } If it has not been given away, but he perceives that it has, he commits an offense entailing relinquishment and confession. If it has not been lost, but he perceives that it has, he commits an offense entailing relinquishment and confession. If it has not been destroyed, but he perceives that it has, he commits an offense entailing relinquishment and confession. If it has not been burned, but he perceives that it has, he commits an offense entailing relinquishment and confession. If it has not been stolen, but he perceives that it has, he commits an offense entailing relinquishment and confession. 

If\marginnote{2.3.11} he uses robe-cloth that should be relinquished without first relinquishing it, he commits an offense of wrong conduct. If it is less than a month, but he perceives it as more, he commits an offense of wrong conduct. If it is less than a month, but he is unsure of it, he commits an offense of wrong conduct. If it is less than a month and he perceives it as less, there is no offense. 

\subsection*{Non-offenses }

There\marginnote{2.4.1} is no offense: if within a month it has been determined, assigned to another, given away, lost, destroyed, burned, stolen, or taken on trust;\footnote{“Taken on trust” refers to a situation where you have an agreement with a close friend that you may take their belongings on trust. The conditions for taking on trust are set out at \href{https://suttacentral.net/pli-tv-kd8/en/brahmali\#19.1.5}{Kd 8:19.1.5}. } if he is insane; if he is the first offender. 

\scendsutta{The third training rule on the robe season, the third, is finished. }

%
\section*{{\suttatitleacronym Bu Np 4}{\suttatitletranslation 4. The training rule on used robes }{\suttatitleroot Purāṇacīvara}}
\addcontentsline{toc}{section}{\tocacronym{Bu Np 4} \toctranslation{4. The training rule on used robes } \tocroot{Purāṇacīvara}}
\markboth{4. The training rule on used robes }{Purāṇacīvara}
\extramarks{Bu Np 4}{Bu Np 4}

\subsection*{Origin story }

At\marginnote{1.1} one time when the Buddha was staying at \textsanskrit{Sāvatthī} in \textsanskrit{Anāthapiṇḍika}’s Monastery, Venerable \textsanskrit{Udāyī}’s ex-wife became a nun. She frequently visited \textsanskrit{Udāyī}, and he frequently visited her. And \textsanskrit{Udāyī} shared his meals with that nun. 

One\marginnote{1.6} morning \textsanskrit{Udāyī} robed up, took his bowl and robe, and went to her. He then uncovered his genitals in front of her and sat down on a seat. She too uncovered her genitals in front of him and sat down on a seat. Lustfully staring at her genitals, he emitted semen. 

He\marginnote{1.9} then said to that nun: “Sister, get some water. I’ll wash the robe.” 

“Give\marginnote{1.11} it to me, venerable, I’ll wash it.” 

She\marginnote{1.12} then took some of the semen in her mouth and inserted some into her vagina. Because of that she became pregnant. The nuns said, “This nun doesn’t abstain from sex. She’s pregnant.” 

She\marginnote{1.16} said, “Venerables, I do abstain from sex,” and she told them what had happened. 

The\marginnote{1.17} nuns complained and criticized \textsanskrit{Udāyī}, “How could Venerable \textsanskrit{Udāyī} get a nun to wash a used robe?” They then told the monks. The monks of few desires complained and criticized him, “How could Venerable \textsanskrit{Udāyī} get a nun to wash a used robe?” 

After\marginnote{1.22} rebuking him in many ways, they told the Buddha. Soon afterwards he had the Sangha gathered and questioned \textsanskrit{Udāyī}: “Is it true, \textsanskrit{Udāyī}, that you did this?” 

“It’s\marginnote{1.24} true, sir.” 

“Is\marginnote{1.25} she a relative of yours?” 

“No.”\marginnote{1.26} 

“Foolish\marginnote{1.27} man, a man and a woman who are unrelated don’t know what’s appropriate and inappropriate, what’s inspiring and uninspiring, in dealing with each other. And still you did this. This will affect people’s confidence …” … “And, monks, this training rule should be recited like this: 

\subsection*{Final ruling }

\scrule{‘If a monk has an unrelated nun wash, dye, or beat a used robe, he commits an offense entailing relinquishment and confession.’” }

\subsection*{Definitions }

\begin{description}%
\item[A: ] whoever … %
\item[Monk: ] … The monk who has been given the full ordination by a unanimous Sangha through a legal procedure consisting of one motion and three announcements that is irreversible and fit to stand—this sort of monk is meant in this case. %
\item[Unrelated: ] anyone who is not a descendant of one’s male ancestors going back eight generations, either on the mother’s side or on the father’s side.\footnote{Sp 1.505: \textit{Tattha \textsanskrit{yāva} \textsanskrit{sattamā} \textsanskrit{pitāmahayugāti} \textsanskrit{pitupitā} \textsanskrit{pitāmaho}, \textsanskrit{pitāmahassa} \textsanskrit{yugaṁ} \textsanskrit{pitāmahayugaṁ}}, “In this \textit{\textsanskrit{yāva} \textsanskrit{sattamā} \textsanskrit{pitāmahayuga}} means: the father of a father is a grandfather. The generation of a grandfather is called a \textit{\textsanskrit{pitāmahayuga}}.” So the PaIi phrase \textit{\textsanskrit{yāva} \textsanskrit{sattamā} \textsanskrit{pitāmahayuga}} means “as far as the seventh generation of grandfathers”, that is, eight generations back. This can be counted as follows: (1) one’s grandfather; (2) his father; (3) 2’s father; (4) 3’s father; (5) 4’s father; (6) 5’s father; and (7) 6’s father. This applies to both one’s paternal and maternal grandfathers. This gives a total of 14 ancestors. Anyone who is a descendent of these fourteen is considered a relative. Anyone who is not such a descendent is not regarded as a relative. } %
\item[A nun: ] she has been given the full ordination by both Sanghas. %
\item[A used robe: ] a sarong or an upper robe, even worn once. %
\end{description}

If\marginnote{2.1.11} he tells her to wash it, he commits an offense of wrong conduct. When it has been washed, it becomes subject to relinquishment. If he tells her to dye it, he commits an offense of wrong conduct. When it has been dyed, it becomes subject to relinquishment. If he tells her to beat it, he commits an offense of wrong conduct. When she has struck it once with her hand or with an implement, it becomes subject to relinquishment. 

The\marginnote{2.1.17} robe should be relinquished to a sangha, a group, or an individual. “And, monks, it should be relinquished like this. (To be expanded as in \href{https://suttacentral.net/pli-tv-bu-vb-np1\#3.2.5}{Bu NP 1:3.2.5}–3.2.29, with appropriate substitutions.) 

‘Venerables,\marginnote{2.1.20} this used robe, which I got an unrelated nun to wash, is to be relinquished. I relinquish it to the Sangha.’ … the Sangha should give … you should give … ‘I give this robe back to you.’” 

\subsection*{Permutations }

If\marginnote{2.2.1} she is unrelated and he perceives her as such, and he has her wash a used robe, he commits one offense entailing relinquishment and confession. If she is unrelated and he perceives her as such, and he has her wash and dye a used robe, he commits one offense entailing relinquishment and one offense of wrong conduct. If she is unrelated and he perceives her as such, and he has her wash and beat a used robe, he commits one offense entailing relinquishment and one offense of wrong conduct. If she is unrelated and he perceives her as such, and he has her wash, dye, and beat a used robe, he commits one offense entailing relinquishment and two offenses of wrong conduct. 

If\marginnote{2.2.5} she is unrelated and he perceives her as such, and he has her dye a used robe, he commits one offense entailing relinquishment and confession. If she is unrelated and he perceives her as such, and he has her dye and beat a used robe, he commits one offense entailing relinquishment and one offense of wrong conduct. If she is unrelated and he perceives her as such, and he has her dye and wash a used robe, he commits one offense entailing relinquishment and one offense of wrong conduct. If she is unrelated and he perceives her as such, and he has her dye, beat, and wash a used robe, he commits one offense entailing relinquishment and two offenses of wrong conduct. 

If\marginnote{2.2.9} she is unrelated and he perceives her as such, and he has her beat a used robe, he commits one offense entailing relinquishment and confession. If she is unrelated and he perceives her as such, and he has her beat and wash a used robe, he commits one offense entailing relinquishment and one offense of wrong conduct. If she is unrelated and he perceives her as such, and he has her beat and dye a used robe, he commits one offense entailing relinquishment and one offense of wrong conduct. If she is unrelated and he perceives her as such, and he has her beat, wash, and dye a used robe, he commits one offense entailing relinquishment and two offenses of wrong conduct. 

If\marginnote{2.2.13} she is unrelated, but he is unsure of it …\footnote{Sp 1.506: \textit{\textsanskrit{Aññātikāya} vematiko \textsanskrit{aññātikāya} \textsanskrit{ñātikasaññīti} \textsanskrit{imānipi} \textsanskrit{padāni} \textsanskrit{vuttānaṁyeva} \textsanskrit{tiṇṇaṁ} \textsanskrit{catukkānaṁ} vasena \textsanskrit{vitthārato} \textsanskrit{veditabbāni}}, “‘If she is unrelated, but he is unsure of it; if she is unrelated, but he perceives her as related’: these sentences should be understood in detail just as the stated three tetrads (above).” } If she is unrelated, but he perceives her as related … 

If\marginnote{2.2.15} he has her wash a used robe belonging to someone else, he commits an offense of wrong conduct. If he has her wash a sitting mat or a sheet, he commits an offense of wrong conduct. If he has a nun who is fully ordained only on one side do the washing, he commits an offense of wrong conduct. 

If\marginnote{2.2.18} she is related, but he perceives her as unrelated, he commits an offense of wrong conduct. If she is related, but he is unsure of it, he commits an offense of wrong conduct. If she is related and he perceives her as such, there is no offense. 

\subsection*{Non-offenses }

There\marginnote{2.2.21.1} is no offense: if a related nun does the washing and an unrelated nun helps her; if a nun does the washing without being asked; if he has a nun wash an unused robe; if he has a nun wash any requisite apart from a robe; if it is a trainee nun; if it is a novice nun; if he is insane; if he is the first offender. 

\scendsutta{The training rule on used robes, the fourth, is finished. }

%
\section*{{\suttatitleacronym Bu Np 5}{\suttatitletranslation 5. The training rule on receiving robes }{\suttatitleroot Cīvarapaṭiggahana}}
\addcontentsline{toc}{section}{\tocacronym{Bu Np 5} \toctranslation{5. The training rule on receiving robes } \tocroot{Cīvarapaṭiggahana}}
\markboth{5. The training rule on receiving robes }{Cīvarapaṭiggahana}
\extramarks{Bu Np 5}{Bu Np 5}

\subsection*{Origin story }

\subsubsection*{First sub-story }

At\marginnote{1.1.1} one time when the Buddha was staying at \textsanskrit{Rājagaha} in the Bamboo Grove, the nun \textsanskrit{Uppalavaṇṇā} was staying at \textsanskrit{Sāvatthī}. One morning she robed up, took her bowl and robe, and entered \textsanskrit{Sāvatthī} to collect almsfood. When she had completed her almsround, eaten her meal, and returned, she went to the Blind Men’s Grove where she sat down at the foot of a tree for the day’s meditation. 

Just\marginnote{1.1.6} then some bandits who had stolen and slaughtered a cow were taking the meat to the Blind Men’s Grove. The head bandit saw \textsanskrit{Uppalavaṇṇā} sitting at the foot of that tree. He thought, “If my sons and brothers see this nun, they’ll harass her,” and he took a different path. Soon afterwards when the meat was cooked, he took the best part, tied it up with a palm-leaf wrap, hung it from a tree not far from \textsanskrit{Uppalavaṇṇā}, and said, “Whatever ascetic or brahmin sees this gift, please take it.” And he left. 

\textsanskrit{Uppalavaṇṇā}\marginnote{1.1.13} had just come out from the stillness of meditation when she heard the head bandit making that statement. She took the meat and returned to her dwelling place. The following morning she prepared the meat and made it into a bundle with her upper robe. She then rose into the air and landed in the Bamboo Grove. 

When\marginnote{1.2.1} she arrived, the Buddha had already gone to the village for alms, but Venerable \textsanskrit{Udāyī} had been left behind to look after the dwellings. \textsanskrit{Uppalavaṇṇā} approached \textsanskrit{Udāyī} and said, “Sir, where’s the Buddha?” 

“He’s\marginnote{1.2.5} gone to the village for alms.” 

“Please\marginnote{1.2.6} give this meat to the Buddha.” 

“You’ll\marginnote{1.2.7} please the Buddha with that meat. If you give me your sarong, you’ll please me too.” 

“It’s\marginnote{1.2.9} hard for women to get material support, and this is one of my five robes. I don’t have another. I can’t give it away.” 

“Sister,\marginnote{1.2.11} just as a man giving an elephant might decorate it with a girdle, so should you, when giving meat to the Buddha, decorate me with your sarong.” 

Being\marginnote{1.2.13} pressured by \textsanskrit{Udāyī}, \textsanskrit{Uppalavaṇṇā} gave him her sarong and then returned to her dwelling place. The nuns who received \textsanskrit{Uppalavaṇṇā}’s bowl and robe asked her where her sarong was. And she told them what had happened. The nuns complained and criticized \textsanskrit{Udāyī}, “How could Venerable \textsanskrit{Udāyī} receive a robe from a nun? It’s hard for women to get material support.” 

The\marginnote{1.2.19} nuns told the monks. The monks of few desires complained and criticized \textsanskrit{Udāyī}, “How could Venerable \textsanskrit{Udāyī} receive a robe from a nun?” 

After\marginnote{1.2.22} rebuking him in many ways, they told the Buddha. Soon afterwards he had the Sangha gathered and questioned \textsanskrit{Udāyī}: “Is it true, \textsanskrit{Udāyī}, that you did this?” 

“It’s\marginnote{1.2.24} true, sir.” 

“Is\marginnote{1.2.25} she a relative of yours?” 

“No.”\marginnote{1.2.26} 

“Foolish\marginnote{1.2.27} man, a man and a woman who are unrelated don’t know what’s appropriate and inappropriate, what’s good and bad, in dealing with each other. And still you did this. This will affect people’s confidence …” … “And, monks, this training rule should be recited like this: 

\subsubsection*{Preliminary ruling }

\scrule{‘If a monk receives a robe directly from an unrelated nun, he commits an offense entailing relinquishment and confession.’” }

In\marginnote{1.2.32} this way the Buddha laid down this training rule for the monks. 

\subsubsection*{Second sub-story }

Then,\marginnote{2.1} being afraid of wrongdoing, the monks did not receive robes from nuns even in exchange. The nuns complained and criticized them, “How can they not receive robes from us in exchange?” 

The\marginnote{2.4} monks heard the criticism of those nuns and they told the Buddha. Soon afterwards the Buddha gave a teaching and addressed the monks: 

\scrule{“Monks, I allow you to receive things in exchange from five kinds of people: monks, nuns, trainee nuns, novice monks, and novice nuns. }

And\marginnote{2.9} so, monks, this training rule should be recited like this: 

\subsection*{Final ruling }

\scrule{‘If a monk receives a robe directly from an unrelated nun, except in exchange, he commits an offense entailing relinquishment and confession.’” }

\subsection*{Definitions }

\begin{description}%
\item[A: ] whoever … %
\item[Monk: ] … The monk who has been given the full ordination by a unanimous Sangha through a legal procedure consisting of one motion and three announcements that is irreversible and fit to stand—this sort of monk is meant in this case. %
\item[Unrelated: ] anyone who is not a descendant of one’s male ancestors going back eight generations, either on the mother’s side or on the father’s side.\footnote{Sp 1.505: \textit{Tattha \textsanskrit{yāva} \textsanskrit{sattamā} \textsanskrit{pitāmahayugāti} \textsanskrit{pitupitā} \textsanskrit{pitāmaho}, \textsanskrit{pitāmahassa} \textsanskrit{yugaṁ} \textsanskrit{pitāmahayugaṁ}}, “In this \textit{\textsanskrit{yāva} \textsanskrit{sattamā} \textsanskrit{pitāmahayuga}} means: the father of a father is a grandfather. The generation of a grandfather is called a \textit{\textsanskrit{pitāmahayuga}}.” So the PaIi phrase \textit{\textsanskrit{yāva} \textsanskrit{sattamā} \textsanskrit{pitāmahayuga}} means “as far as the seventh generation of grandfathers”, that is, eight generations back. This can be counted as follows: (1) one’s grandfather; (2) his father; (3) 2’s father; (4) 3’s father; (5) 4’s father; (6) 5’s father; and (7) 6’s father. This applies to both one’s paternal and maternal grandfathers. This gives a total of 14 ancestors. Anyone who is a descendent of these fourteen is considered a relative. Anyone who is not such a descendent is not regarded as a relative. } %
\item[A nun: ] she has been given the full ordination by both Sanghas. %
\item[A robe: ] one of the six kinds of robe-cloth, but not smaller than what can be assigned to another.\footnote{The six are linen, cotton, silk, wool, sunn hemp, and hemp; see \href{https://suttacentral.net/pli-tv-kd8/en/brahmali\#3.1.6}{Kd 8:3.1.6}. According to \href{https://suttacentral.net/pli-tv-kd8/en/brahmali\#21.1.4}{Kd 8:21.1.4}, the size referred to here is no smaller than 8 by 4 \textit{\textsanskrit{sugataṅgula}}, “standard fingerbreadths”. For an explanation of the idea of \textit{\textsanskrit{vikappanā}}, see Appendix of Technical Terms. For the rendering of \textit{\textsanskrit{cīvara}} as “robe-cloth”, see the same appendix. } %
\item[Except in exchange: ] unless there is an exchange. %
\end{description}

If\marginnote{3.1.13} he accepts, then for the effort there is an act of wrong conduct. When he gets the robe-cloth, it becomes subject to relinquishment. 

The\marginnote{3.1.15} robe-cloth should be relinquished to a sangha, a group, or an individual. “And, monks, it should be relinquished like this. (To be expanded as in \href{https://suttacentral.net/pli-tv-bu-vb-np1\#3.2.5}{Bu NP 1:3.2.5}–3.2.29, with appropriate substitutions.) 

‘Venerables,\marginnote{3.1.18} this robe-cloth, which I received directly from an unrelated nun without anything in exchange, is to be relinquished. I relinquish it to the Sangha.’ … the Sangha should give … you should give … ‘I give this robe-cloth back to you.’” 

\subsection*{Permutations }

If\marginnote{3.2.1} she is unrelated and he perceives her as such, and he receives robe-cloth from her, except in exchange, he commits an offense entailing relinquishment and confession. If she is unrelated, but he is unsure of it, and he receives robe-cloth from her, except in exchange, he commits an offense entailing relinquishment and confession. If she is unrelated, but he perceives her as related, and he receives robe-cloth from her, except in exchange, he commits an offense entailing relinquishment and confession. 

If\marginnote{3.2.4} he receives robe-cloth directly from a nun who is fully ordained only on one side, except in exchange, he commits an offense of wrong conduct. If she is related, but he perceives her as unrelated, he commits an offense of wrong conduct. If she is related, but he is unsure of it, he commits an offense of wrong conduct. If she is related and he perceives her as such, there is no offense. 

\subsection*{Non-offenses }

There\marginnote{3.3.1} is no offense: if the nun is related; if much is exchanged with little or little is exchanged with much; if he takes it on trust;\footnote{This refers to a situation where you have an agreement with a close friend that you may take their belongings on trust. The conditions for taking on trust are set out at \href{https://suttacentral.net/pli-tv-kd8/en/brahmali\#19.1.5}{Kd 8:19.1.5}. } if he borrows it; if he receives any requisite apart from robe-cloth; if it is a trainee nun; if it is a novice nun; if he is insane; if he is the first offender. 

\scendsutta{The training rule on receiving robes, the fifth, is finished. }

%
\section*{{\suttatitleacronym Bu Np 6}{\suttatitletranslation 6. The training rule on asking non-relations }{\suttatitleroot Aññātakaviññatti}}
\addcontentsline{toc}{section}{\tocacronym{Bu Np 6} \toctranslation{6. The training rule on asking non-relations } \tocroot{Aññātakaviññatti}}
\markboth{6. The training rule on asking non-relations }{Aññātakaviññatti}
\extramarks{Bu Np 6}{Bu Np 6}

\subsection*{Origin story }

\subsubsection*{First sub-story }

At\marginnote{1.1.1} one time the Buddha was staying at \textsanskrit{Sāvatthī} in the Jeta Grove, \textsanskrit{Anāthapiṇḍika}’s Monastery. At that time Venerable Upananda the Sakyan was skilled at teaching. On one occasion the son of a wealthy merchant went to Upananda, bowed, and sat down. And Upananda instructed, inspired, and gladdened him with a teaching. Afterwards that merchant’s son said to Upananda: 

“Venerable,\marginnote{1.1.6} please tell me what you need. I can give you robe-cloth, almsfood, a dwelling, and medicinal supplies.” 

“If\marginnote{1.1.8} you wish to give me something, give me one of your wrap garments.” 

“It’s\marginnote{1.1.9} shameful, venerable, for a gentleman to walk around in only one wrap. Please wait until I get back home. I’ll send you this wrap or a better one.” 

A\marginnote{1.1.12} second time and a third time Upananda said the same thing to that merchant’s son, and he got the same reply. He then said, “What’s the point of inviting me if you don’t want to give?” 

Being\marginnote{1.1.19} pressured by Upananda, that merchant’s son gave him one of his wraps and left. People asked him why he was walking around in only one wrap, and he told them what had happened. People complained and criticized him, “These Sakyan monastics have great desires. They’re not content. Even to make them an appropriate offer isn’t easy. How could they take his wrap when the merchant’s son was making an appropriate offer?” 

The\marginnote{1.2.7} monks heard the complaints of those people, and the monks of few desires complained and criticized Upananda, “How could Venerable Upananda ask the merchant’s son for a robe?” 

After\marginnote{1.2.10} rebuking him in many ways, they told the Buddha. Soon afterwards he had the Sangha gathered and questioned Upananda: “Is it true, Upananda, that you did this?” 

“It’s\marginnote{1.2.12} true, sir.” 

“Is\marginnote{1.2.13} he a relative of yours?” 

“No.”\marginnote{1.2.14} 

“Foolish\marginnote{1.2.15} man, people who are unrelated don’t know what’s appropriate and inappropriate, what’s good and bad, in dealing with each other. And still you did this. This will affect people’s confidence …” … “And, monks, this training rule should be recited like this: 

\subsubsection*{Preliminary ruling }

\scrule{‘If a monk asks an unrelated male or female householder for a robe, he commits an offense entailing relinquishment and confession.’” }

In\marginnote{1.2.20} this way the Buddha laid down this training rule for the monks. 

\subsubsection*{Second sub-story }

Soon\marginnote{2.1} afterwards a number of monks who were traveling from \textsanskrit{Sāketa} to \textsanskrit{Sāvatthī} were robbed by bandits. Knowing that the Buddha had laid down this training rule and being afraid of wrongdoing, they did not ask for robes. As a consequence, they walked naked to \textsanskrit{Sāvatthī}, where they bowed down to the monks. The monks there said, “These \textsanskrit{Ājīvaka} ascetics are good people, as they bow down to the monks.” 

“We’re\marginnote{2.5} not \textsanskrit{Ājīvakas}! We’re monks!” 

The\marginnote{2.6} monks asked Venerable \textsanskrit{Upāli} to examine them. 

When\marginnote{2.7} the naked monks told him what had happened, \textsanskrit{Upāli} said to the monks, “They are monks. Please give them robes.” 

The\marginnote{2.8} monks of few desires complained and criticized them, “How can monks go naked? Shouldn’t they have covered up with grass and leaves?” 

After\marginnote{2.11} rebuking those monks in many ways, they told the Buddha. Soon afterwards he gave a teaching and addressed the monks: 

\scrule{“Monks, if a monk’s robes are stolen or lost, I allow him to ask an unrelated householder for robes. At the first monastery where he arrives, if the Sangha has a communal robe, a bedspread, a floor cover, or a mattress cover, he should take that and put it on, thinking, ‘When I get a robe, I’ll return it.’\footnote{“A communal robe” renders \textit{\textsanskrit{vihāracīvara}}, literally, “a dwelling robe”. Since “dwelling robe” is awkward in English, and because these robes did not belong to individual monastics, I prefer the given rendering. “A bedspread” renders \textit{\textsanskrit{uttarattharaṇa}}. Sp 1.517: \textit{\textsanskrit{Uttarattharaṇanti} \textsanskrit{mañcakassa} upari \textsanskrit{attharaṇakaṁ} vuccati}, “What is spread on top of a bed is called an \textit{\textsanskrit{uttarattharaṇa}}.” } If there’s none of these things, he should cover up with grass and leaves before going on. He should not go on while naked. If he does, he commits an offense of wrong conduct. }

And\marginnote{2.17} so, monks, this training rule should be recited like this: 

\subsection*{Final ruling }

\scrule{‘If a monk asks an unrelated male or female householder for a robe, except on an appropriate occasion, he commits an offense entailing relinquishment and confession. These are the appropriate occasions: his robes are stolen or his robes are lost.’” }

\subsection*{Definitions }

\begin{description}%
\item[A: ] whoever … %
\item[Monk: ] … The monk who has been given the full ordination by a unanimous Sangha through a legal procedure consisting of one motion and three announcements that is irreversible and fit to stand—this sort of monk is meant in this case. %
\item[Unrelated: ] anyone who is not a descendant of one’s male ancestors going back eight generations, either on the mother’s side or on the father’s side.\footnote{Sp 1.505: \textit{Tattha \textsanskrit{yāva} \textsanskrit{sattamā} \textsanskrit{pitāmahayugāti} \textsanskrit{pitupitā} \textsanskrit{pitāmaho}, \textsanskrit{pitāmahassa} \textsanskrit{yugaṁ} \textsanskrit{pitāmahayugaṁ}}, “In this \textit{\textsanskrit{yāva} \textsanskrit{sattamā} \textsanskrit{pitāmahayuga}} means: the father of a father is a grandfather. The generation of a grandfather is called a \textit{\textsanskrit{pitāmahayuga}}.” So the PaIi phrase \textit{\textsanskrit{yāva} \textsanskrit{sattamā} \textsanskrit{pitāmahayuga}} means “as far as the seventh generation of grandfathers”, that is, eight generations back. This can be counted as follows: (1) one’s grandfather; (2) his father; (3) 2’s father; (4) 3’s father; (5) 4’s father; (6) 5’s father; and (7) 6’s father. This applies to both one’s paternal and maternal grandfathers. This gives a total of 14 ancestors. Anyone who is a descendent of these fourteen is considered a relative. Anyone who is not such a descendent is not regarded as a relative. } %
\item[A male householder: ] any man who lives at home.\footnote{\textit{\textsanskrit{Agāraṁ}} is typically rendered as “in a house”. The problem with this is that it is not unallowable for a monastic to live in a building that is the equivalent of a house. What a monastic should not do is own a home and then live there. } %
\item[A female householder: ] any woman who lives at home. %
\item[A robe: ] one of the six kinds of robe-cloth, but not smaller than what can be assigned to another.\footnote{The six are linen, cotton, silk, wool, sunn hemp, and hemp; see \href{https://suttacentral.net/pli-tv-kd8/en/brahmali\#3.1.6}{Kd 8:3.1.6}. According to \href{https://suttacentral.net/pli-tv-kd8/en/brahmali\#21.1.4}{Kd 8:21.1.4}, the size referred to here is no smaller than 8 by 4 \textit{\textsanskrit{sugataṅgula}}, “standard fingerbreadths”. For an explanation of the idea of \textit{\textsanskrit{vikappanā}}, see Appendix of Technical Terms. For the rendering of \textit{\textsanskrit{cīvara}} as “robe-cloth”, see the same appendix. } %
\item[Except on an appropriate occasion: ] unless it is an appropriate occasion. %
\item[His robes are stolen: ] a monk’s robe is taken by kings, bandits, scoundrels, or whoever. %
\item[His robes are lost: ] a monk’s robe is burned by fire, carried away by flooding, eaten by rats or termites, or worn through use. %
\end{description}

If\marginnote{3.2.1} he asks, except on an appropriate occasion, then for the effort there is an act of wrong conduct. When he gets the robe-cloth, it becomes subject to relinquishment. 

The\marginnote{3.2.3} robe-cloth should be relinquished to a sangha, a group, or an individual. “And, monks, it should be relinquished like this. (To be expanded as in \href{https://suttacentral.net/pli-tv-bu-vb-np1\#3.2.5}{Bu NP 1:3.2.5}–3.2.29, with appropriate substitutions.) 

‘Venerables,\marginnote{3.2.6} this robe-cloth, which I received after asking an unrelated householder, but not on an appropriate occasion, is to be relinquished. I relinquish it to the Sangha.’ … the Sangha should give … you should give … ‘I give this robe-cloth back to you.’” 

\subsection*{Permutations }

If\marginnote{3.3.1} the person is unrelated and the monk perceives them as such, and he asks them for robe-cloth, except on an appropriate occasion, he commits an offense entailing relinquishment and confession. If the person is unrelated, but the monk is unsure of it, and he asks them for robe-cloth, except on an appropriate occasion, he commits an offense entailing relinquishment and confession. If the person is unrelated, but the monk perceives them as related, and he asks them for robe-cloth, except on an appropriate occasion, he commits an offense entailing relinquishment and confession. 

If\marginnote{3.3.4} the person is related, but the monk perceives them as unrelated, he commits an offense of wrong conduct. If the person is related, but the monk is unsure of it, he commits an offense of wrong conduct. If the person is related and the monk perceives them as such, there is no offense. 

\subsection*{Non-offenses }

There\marginnote{3.4.1} is no offense: if it is an appropriate occasion; if he asks relatives; if he asks those who have given an invitation; if he asks for the benefit of someone else; if it is by means of his own property; if he is insane; if he is the first offender. 

\scendsutta{The training rule on asking non-relations, the sixth, is finished. }

%
\section*{{\suttatitleacronym Bu Np 7}{\suttatitletranslation 7. The training rule on more than that }{\suttatitleroot Tatuttari}}
\addcontentsline{toc}{section}{\tocacronym{Bu Np 7} \toctranslation{7. The training rule on more than that } \tocroot{Tatuttari}}
\markboth{7. The training rule on more than that }{Tatuttari}
\extramarks{Bu Np 7}{Bu Np 7}

\subsection*{Origin story }

At\marginnote{1.1} one time when the Buddha was staying at \textsanskrit{Sāvatthī} in \textsanskrit{Anāthapiṇḍika}’s Monastery, the monks from the group of six said to the monks whose robes had been stolen, “The Buddha has allowed a monk whose robes have been stolen or lost to ask an unrelated householder for a robe. You should ask for a robe.” 

“It’s\marginnote{1.6} not necessary. We’ve already gotten robes.” 

“We’ll\marginnote{1.7} ask for you.” 

“Do\marginnote{1.8} as you please.”\footnote{\textit{\textsanskrit{Viññāpetha} \textsanskrit{āvuso}}, literally, “Ask, friends.” } 

The\marginnote{1.9} monks from the group of six then went to householders and said, “Monks have arrived whose robes have been stolen. Please give them robes.” And they asked for many robes. 

Soon\marginnote{1.13} afterwards in the public meeting hall a certain man said to another, “Sir, monks have arrived whose robes have been stolen. I’ve given them robes.” 

The\marginnote{1.16} other replied, “So have I.” And another said the same. 

They\marginnote{1.18} complained and criticized those monks, “How can the Sakyan monastics ask for many robes without moderation? Are they going to trade in cloth or set up shop?” 

The\marginnote{1.21} monks heard the complaints of those people, and the monks of few desires complained and criticized those monks, “How could the monks from the group of six ask for many robes without moderation?” 

After\marginnote{1.24} rebuking those monks in many ways, they told the Buddha. Soon afterwards he had the Sangha gathered and questioned the monks: “Is it true, monks, that you did this?” 

“It’s\marginnote{1.26} true, sir.” 

The\marginnote{1.27} Buddha rebuked them … “Foolish men, how could you do this? This will affect people’s confidence …” … “And, monks, this training rule should be recited like this: 

\subsection*{Final ruling }

\scrule{‘If an unrelated male or female householder invites that monk to take many robes, he should accept at most one sarong and one upper robe. If he accepts more than that, he commits an offense entailing relinquishment and confession.’” }

\subsection*{Definitions }

\begin{description}%
\item[That monk: ] the monk whose robes have been stolen. %
\item[Unrelated: ] anyone who is not a descendant of one’s male ancestors going back eight generations, either on the mother’s side or on the father’s side.\footnote{Sp 1.505: \textit{Tattha \textsanskrit{yāva} \textsanskrit{sattamā} \textsanskrit{pitāmahayugāti} \textsanskrit{pitupitā} \textsanskrit{pitāmaho}, \textsanskrit{pitāmahassa} \textsanskrit{yugaṁ} \textsanskrit{pitāmahayugaṁ}}, “In this \textit{\textsanskrit{yāva} \textsanskrit{sattamā} \textsanskrit{pitāmahayuga}} means: the father of a father is a grandfather. The generation of a grandfather is called a \textit{\textsanskrit{pitāmahayuga}}.” So the PaIi phrase \textit{\textsanskrit{yāva} \textsanskrit{sattamā} \textsanskrit{pitāmahayuga}} means “as far as the seventh generation of grandfathers”, that is, eight generations back. This can be counted as follows: (1) one’s grandfather; (2) his father; (3) 2’s father; (4) 3’s father; (5) 4’s father; (6) 5’s father; and (7) 6’s father. This applies to both one’s paternal and maternal grandfathers. This gives a total of 14 ancestors. Anyone who is a descendent of these fourteen is considered a relative. Anyone who is not such a descendent is not regarded as a relative. } %
\item[A male householder: ] any man who lives at home.\footnote{\textit{\textsanskrit{Agāraṁ}} is typically rendered as “in a house”. The problem with this is that it is not unallowable for a monastic to live in a building that is the equivalent of a house. What a monastic should not do is own a home and then live there. } %
\item[A female householder: ] any woman who lives at home. %
\item[Many robes: ] a lot of robes. %
\item[Invites to take: ] saying, “Take as many as you like.” %
\item[He should accept at most one sarong and one upper robe: ] if three robes are lost, he should accept two; if two robes are lost, he should accept one; if one robe is lost, he should not accept any. %
\item[If he accepts more than that: ] if he asks for more than that, then for the effort there is an act of wrong conduct. When he gets the robe, it becomes subject to relinquishment. %
\end{description}

The\marginnote{2.1.18} robe should be relinquished to a sangha, a group, or an individual. “And, monks, it should be relinquished like this. (To be expanded as in \href{https://suttacentral.net/pli-tv-bu-vb-np1\#3.2.5}{Bu NP 1:3.2.5}–3.2.29, with appropriate substitutions.) 

‘Venerables,\marginnote{2.1.21} this robe, which I received after asking an unrelated householder for too many, is to be relinquished. I relinquish it to the Sangha.’ … the Sangha should give … you should give … ‘I give this robe back to you.’” 

\subsection*{Permutations }

If\marginnote{2.2.1.1} the person is unrelated and the monk perceives them as such, and he asks them for too many robes, he commits an offense entailing relinquishment and confession. If the person is unrelated, but the monk is unsure of it, and he asks them for too many robes, he commits an offense entailing relinquishment and confession. If the person is unrelated, but the monk perceives them as related, and he asks them for too many robes, he commits an offense entailing relinquishment and confession. 

If\marginnote{2.2.4} the person is related, but the monk perceives them as unrelated, he commits an offense of wrong conduct. If the person is related, but the monk is unsure of it, he commits an offense of wrong conduct. If the person is related and the monk perceives them as such, there is no offense. 

\subsection*{Non-offenses }

There\marginnote{2.3.1} is no offense: if he takes too much, but with the intention of returning the remainder;\footnote{Sp 1.526: \textit{\textsanskrit{Sesakaṁ} \textsanskrit{āharissāmīti} dve \textsanskrit{cīvarāni} \textsanskrit{katvā} \textsanskrit{sesaṁ} puna \textsanskrit{āharissāmīti} attho}, “\textit{\textsanskrit{Sesakaṁ} \textsanskrit{āharissāmi}}: the meaning is that he says, ‘After making two robes, I will return the remainder.’” } if they give, saying, “The remainder is for you;”\footnote{Presumably this refers to any leftover cloth after the robe(s) have been made. } if they give, but not because his robes were stolen; if they give, but not because his robes were lost; if it is from relatives; if it is from those who have given an invitation; if it is by means of his own property; if he is insane; if he is the first offender. 

\scendsutta{The training rule on more than that, the seventh, is finished. }

%
\section*{{\suttatitleacronym Bu Np 8}{\suttatitletranslation 8. The training rule on what is set aside }{\suttatitleroot Upakkhata}}
\addcontentsline{toc}{section}{\tocacronym{Bu Np 8} \toctranslation{8. The training rule on what is set aside } \tocroot{Upakkhata}}
\markboth{8. The training rule on what is set aside }{Upakkhata}
\extramarks{Bu Np 8}{Bu Np 8}

\subsection*{Origin story }

At\marginnote{1.1.1} one time when the Buddha was staying at \textsanskrit{Sāvatthī} in \textsanskrit{Anāthapiṇḍika}’s Monastery, a certain man said to his wife, “I’m going to give robe-cloth to Venerable Upananda.” 

An\marginnote{1.1.4} alms-collecting monk heard that man making that statement. He then went to Upananda the Sakyan and said, “Upananda, you have much merit. In such-and-such a place I heard a man tell his wife that he’s going to give you robe-cloth.” 

“He’s\marginnote{1.1.9} my supporter.” 

Upananda\marginnote{1.1.10} then went to that man and said, “Is it true that you want to give me robe-cloth?” 

“Yes,\marginnote{1.1.12} that’s just what I was thinking.” 

“If\marginnote{1.1.14} that’s the case, give me such-and-such robe-cloth. For what’s the point of giving robe-cloth that I won’t use?” 

That\marginnote{1.1.16} man complained and criticized him, “These Sakyan monastics have great desires. They’re not content. It’s no easy matter to give them robe-cloth. How could Venerable Upananda come to me and say what kind of robe-cloth he wants without first being invited by me to do so?” 

The\marginnote{1.1.20} monks heard the complaints of that man, and the monks of few desires complained and criticized him, “How could Venerable Upananda go to a householder and say what kind of robe-cloth he wants without first being invited to do so?” 

After\marginnote{1.1.23} rebuking him in many ways, they told the Buddha. Soon afterwards he had the Sangha gathered and questioned Upananda: “Is it true, Upananda, that you did this?” 

“It’s\marginnote{1.1.25} true, sir.” 

“Is\marginnote{1.1.26} he a relative of yours?” 

“No.”\marginnote{1.1.27} 

“Foolish\marginnote{1.1.28} man, people who are unrelated don’t know what’s appropriate and inappropriate, what’s good and bad, in dealing with each other. And still you did this. This will affect people’s confidence …” … “And, monks, this training rule should be recited like this: 

\subsection*{Final ruling }

\scrule{‘If a male or female householder has set aside a robe fund for an unrelated monk, thinking, “With this robe fund I will buy robe-cloth and give it to monk so-and-so;” and if that monk, without first being invited, goes to them and specifies the kind of robe-cloth he wants, saying, “Please use this robe fund to buy such-and-such robe-cloth and then give it to me,” and he does so because he wants something fine, he commits an offense entailing relinquishment and confession.’” }

\subsection*{Definitions }

\begin{description}%
\item[For a monk: ] for the benefit of a monk; making a monk the object of consideration, one wants to give to him. %
\item[Unrelated: ] anyone who is not a descendant of one’s male ancestors going back eight generations, either on the mother’s side or on the father’s side.\footnote{Sp 1.505: \textit{Tattha \textsanskrit{yāva} \textsanskrit{sattamā} \textsanskrit{pitāmahayugāti} \textsanskrit{pitupitā} \textsanskrit{pitāmaho}, \textsanskrit{pitāmahassa} \textsanskrit{yugaṁ} \textsanskrit{pitāmahayugaṁ}}, “In this \textit{\textsanskrit{yāva} \textsanskrit{sattamā} \textsanskrit{pitāmahayuga}} means: the father of a father is a grandfather. The generation of a grandfather is called a \textit{\textsanskrit{pitāmahayuga}}.” So the PaIi phrase \textit{\textsanskrit{yāva} \textsanskrit{sattamā} \textsanskrit{pitāmahayuga}} means “as far as the seventh generation of grandfathers”, that is, eight generations back. This can be counted as follows: (1) one’s grandfather; (2) his father; (3) 2’s father; (4) 3’s father; (5) 4’s father; (6) 5’s father; and (7) 6’s father. This applies to both one’s paternal and maternal grandfathers. This gives a total of 14 ancestors. Anyone who is a descendent of these fourteen is considered a relative. Anyone who is not such a descendent is not regarded as a relative. } %
\item[A male householder: ] any man who lives at home.\footnote{\textit{\textsanskrit{Agāraṁ}} is typically rendered as “in a house”. The problem with this is that it is not unallowable for a monastic to live in a building that is the equivalent of a house. What a monastic should not do is own a home and then live there. } %
\item[A female householder: ] any woman who lives at home. %
\item[A robe fund: ] money, gold, a pearl, a gem, a coral, a crystal, cloth, thread, or cotton wool.\footnote{“Money” renders \textit{\textsanskrit{hirañña}}.  See Appendix of Technical Terms for discussion. } %
\item[With this robe fund: ] with that which is at one’s disposal. %
\item[I will buy: ] having traded. %
\item[I will give: ] I will donate. %
\item[If that monk: ] the monk the robe fund has been set aside for. %
\item[Without first being invited: ] he or she has not said beforehand: “Venerable, what kind of robe-cloth do you need? What kind of robe-cloth can I buy for you?” %
\item[Goes to them: ] having gone to their house or having gone wherever. %
\item[Specifies the kind of robe-cloth he wants: ] please make it long or wide or closely woven or soft. %
\item[This robe fund: ] that which is at one’s disposal. %
\item[Such-and-such: ] long or wide or closely woven or soft. %
\item[To buy: ] having traded. %
\item[Give: ] donate. %
\item[Because he wants something fine: ] wanting something good, wanting something expensive. %
\end{description}

If\marginnote{2.1.35} that lay person buys robe-cloth that is long, wide, closely woven, or soft because of the monk’s statement, then for the effort there is an act of wrong conduct. When he gets the robe-cloth, it becomes subject to relinquishment. 

The\marginnote{2.1.37} robe-cloth should be relinquished to a sangha, a group, or an individual. “And, monks, it should be relinquished like this. (To be expanded as in \href{https://suttacentral.net/pli-tv-bu-vb-np1\#3.2.5}{Bu NP 1:3.2.5}–3.2.29, with appropriate substitutions.) 

‘Venerables,\marginnote{2.1.40} this robe-cloth, which I received after going to an unrelated householder and saying what kind of robe-cloth I wanted without first being invited, is to be relinquished. I relinquish it to the Sangha.’ … the Sangha should give … you should give … ‘I give this robe-cloth back to you.’” 

\subsection*{Permutations }

If\marginnote{2.2.1} the householder is unrelated and the monk perceives them as such, and, without first being invited, he goes to them and specifies the kind of robe-cloth he wants, he commits an offense entailing relinquishment and confession. If the householder is unrelated, but the monk is unsure of it, and, without first being invited, he goes to them and specifies the kind of robe-cloth he wants, he commits an offense entailing relinquishment and confession. If the householder is unrelated, but the monk perceives them as related, and, without first being invited, he goes to them and specifies the kind of robe-cloth he wants, he commits an offense entailing relinquishment and confession. 

If\marginnote{2.2.4} the householder is related, but the monk perceives them as unrelated, he commits an offense of wrong conduct. If the householder is related, but the monk is unsure of it, he commits an offense of wrong conduct. If the householder is related and the monk perceives them as such, there is no offense. 

\subsection*{Non-offenses }

There\marginnote{2.3.1} is no offense: if it is from relatives; if it is from those who have given an invitation; if it is for the benefit of another; if it is by means of his own property; if the householder wishes to buy something expensive, but the monk has them buy something inexpensive; if he is insane; if he is the first offender. 

\scendsutta{The training rule on what is set aside, the eighth, is finished. }

%
\section*{{\suttatitleacronym Bu Np 9}{\suttatitletranslation 9. The second training rule on what is set aside }{\suttatitleroot Dutiyaupakkhata}}
\addcontentsline{toc}{section}{\tocacronym{Bu Np 9} \toctranslation{9. The second training rule on what is set aside } \tocroot{Dutiyaupakkhata}}
\markboth{9. The second training rule on what is set aside }{Dutiyaupakkhata}
\extramarks{Bu Np 9}{Bu Np 9}

\subsection*{Origin story }

At\marginnote{1.1} one time when the Buddha was staying at \textsanskrit{Sāvatthī} in \textsanskrit{Anāthapiṇḍika}’s Monastery, a certain man said to another man, “I’m going to give robe-cloth to Venerable Upananda.” And he replied, “So am I.” 

An\marginnote{1.6} alms-collecting monk heard that conversation. He then went to Upananda the Sakyan and said, “Upananda, you have much merit. In such-and-such a place I heard two men telling each other that they’re each going to give you robe-cloth.” 

“They\marginnote{1.13} are my supporters.” 

Upananda\marginnote{1.14} then went to those men and said, “Is it true that you want to give me robe-cloth?” 

“Yes,\marginnote{1.16} that’s just what we were thinking.” 

“If\marginnote{1.18} that’s the case, then give me such-and-such robe-cloth. For what’s the point of giving robe-cloth that I won’t use?” 

Those\marginnote{1.20} men complained and criticized him, “These Sakyan monastics have great desires. They’re not content. It’s no easy matter to give them robe-cloth. How could Venerable Upananda come to us and say what kind of robe-cloth he wants without first being invited by us to do so?” 

The\marginnote{1.24} monks heard the complaints of those men, and the monks of few desires complained and criticized him, “How could Venerable Upananda go to householders and say what kind of robe-cloth he wants without first being invited to do so?” 

After\marginnote{1.27} rebuking him in many ways, they told the Buddha. Soon afterwards he had the Sangha gathered and questioned Upananda: “Is it true, Upananda, that you did this?” 

“It’s\marginnote{1.29} true, sir.” 

“Are\marginnote{1.30} they relatives of yours?” 

“No.”\marginnote{1.31} 

“Foolish\marginnote{1.32} man, people who are unrelated don’t know what’s appropriate and inappropriate, what’s good and bad, in dealing with each other. And still you did this. This will affect people’s confidence …” … “And, monks, this training rule should be recited like this: 

\subsection*{Final ruling }

\scrule{‘If two male or female householders have set aside separate robe funds for an unrelated monk, thinking, “With these separate robe funds we’ll buy separate robe-cloths and give them to monk so-and-so;” and if that monk, without first being invited, goes to them and specifies the kind of robe-cloth he wants, saying, “Please put these separate robe funds together to buy such-and-such robe-cloth and then give it to me,”\footnote{The phrase \textit{ubhova \textsanskrit{santā} ekena}, here translated as “put together”, is not immediately clear. It seems to me that this phrase needs to be related to the main sentence verb \textit{\textsanskrit{acchādetha}}, “present” or “give”, which takes the instrumental of the thing given: “present (him) with one (robe)”. \textit{Ubhova \textsanskrit{santā}} is explained below in the word commentary as \textit{dvepi \textsanskrit{janā}}, “both people”, and the overall phrase then becomes, “two people present (him) with one (robe)”. This means that the funds are put together, and I translate accordingly. }  and he does so because he wants something fine, he commits an offense entailing relinquishment and confession.’” }

\subsection*{Definitions }

\begin{description}%
\item[For a monk: ] for the benefit of a monk; making a monk the object of consideration, they want to give to him. %
\item[Two: ] a pair. %
\item[Unrelated: ] anyone who is not a descendant of one’s male ancestors going back eight generations, either on the mother’s side or on the father’s side.\footnote{Sp 1.505: \textit{Tattha \textsanskrit{yāva} \textsanskrit{sattamā} \textsanskrit{pitāmahayugāti} \textsanskrit{pitupitā} \textsanskrit{pitāmaho}, \textsanskrit{pitāmahassa} \textsanskrit{yugaṁ} \textsanskrit{pitāmahayugaṁ}}, “In this \textit{\textsanskrit{yāva} \textsanskrit{sattamā} \textsanskrit{pitāmahayuga}} means: the father of a father is a grandfather. The generation of a grandfather is called a \textit{\textsanskrit{pitāmahayuga}}.” So the PaIi phrase \textit{\textsanskrit{yāva} \textsanskrit{sattamā} \textsanskrit{pitāmahayuga}} means “as far as the seventh generation of grandfathers”, that is, eight generations back. This can be counted as follows: (1) one’s grandfather; (2) his father; (3) 2’s father; (4) 3’s father; (5) 4’s father; (6) 5’s father; and (7) 6’s father. This applies to both one’s paternal and maternal grandfathers. This gives a total of 14 ancestors. Anyone who is a descendent of these fourteen is considered a relative. Anyone who is not such a descendent is not regarded as a relative. } %
\item[Male householders: ] any men who live at home.\footnote{\textit{\textsanskrit{Agāraṁ}} is typically rendered as “in a house”. The problem with this is that it is not unallowable for a monastic to live in a building that is the equivalent of a house. What a monastic should not do is own a home and then live there. } %
\item[Female householders: ] any women who live at home. %
\item[Robe funds: ] money, gold, pearls, gems, corals, crystals, cloth, thread, or cotton wool. %
\item[With these separate robe funds: ] with that which is at their disposal. %
\item[We’ll buy: ] having traded. %
\item[We’ll give: ] we’ll donate. %
\item[If that monk: ] the monk the robe funds have been set aside for. %
\item[Without first being invited: ] they have not said beforehand: “Venerable, what kind of robe-cloth do you need? What kind of robe-cloth can we buy for you?” %
\item[Goes to them: ] having gone to their house or having gone wherever. %
\item[Specifies the kind of robe-cloth he wants: ] please make it long or wide or closely woven or soft. %
\item[These separate robe funds: ] that which is at their disposal. %
\item[Such-and-such: ] long or wide or closely woven or soft. %
\item[To buy: ] having traded. %
\item[Give: ] donate. %
\item[Put together: ] two people supplying him with one robe-cloth. %
\item[Because he wants something fine: ] wanting something good, wanting something expensive. %
\end{description}

If\marginnote{2.39} those householders buy robe-cloth that is long, wide, closely woven, or soft because of his statement, then for the effort there is an act of wrong conduct. When he gets the robe-cloth, it becomes subject to relinquishment. 

The\marginnote{2.41} robe-cloth should be relinquished to a sangha, a group, or an individual. “And, monks, it should be relinquished like this. (To be expanded as in \href{https://suttacentral.net/pli-tv-bu-vb-np1\#3.2.5}{Bu NP 1:3.2.5}–3.2.29, with appropriate substitutions.) 

‘Venerables,\marginnote{2.44} this robe-cloth, which I received after going to unrelated householders and saying what kind of robe-cloth I wanted without first being invited, is to be relinquished. I relinquish it to the Sangha.’ … the Sangha should give … you should give … ‘I give this robe-cloth back to you.’” 

\subsection*{Permutations }

If\marginnote{2.49.1} the householders are unrelated and he perceives them as such, and, without first being invited, he goes to them and specifies the kind of robe-cloth he wants, he commits an offense entailing relinquishment and confession. If the householders are unrelated, but he is unsure of it, and, without first being invited, he goes to them and specifies the kind of robe-cloth he wants, he commits an offense entailing relinquishment and confession. If the householders are unrelated, but he perceives them as related, and, without first being invited, he goes to them and specifies the kind of robe-cloth he wants, he commits an offense entailing relinquishment and confession. 

If\marginnote{2.52} the householders are related, but he perceives them as unrelated, he commits an offense of wrong conduct. If the householders are related, but he is unsure of it, he commits an offense of wrong conduct. If the householders are related and he perceives them as such, there is no offense. 

\subsection*{Non-offenses }

There\marginnote{2.55.1} is no offense: if it is from relatives; if it is from those who have given an invitation; if it is for the benefit of another; if it is by means of his own property; if the householders wish to buy something expensive, but he has them buy something inexpensive; if he is insane; if he is the first offender. 

\scendsutta{The second training rule on what is set aside, the ninth, is finished. }

%
\section*{{\suttatitleacronym Bu Np 10}{\suttatitletranslation 10. The training rule on kings }{\suttatitleroot Rāja}}
\addcontentsline{toc}{section}{\tocacronym{Bu Np 10} \toctranslation{10. The training rule on kings } \tocroot{Rāja}}
\markboth{10. The training rule on kings }{Rāja}
\extramarks{Bu Np 10}{Bu Np 10}

\subsection*{Origin story }

At\marginnote{1.1.1} one time the Buddha was staying at \textsanskrit{Sāvatthī} in the Jeta Grove, \textsanskrit{Anāthapiṇḍika}’s Monastery. At that time a certain government official was a supporter of Venerable Upananda the Sakyan. On one occasion that official sent a robe fund by messenger, saying, “Buy robe-cloth with this fund and give it to Venerable Upananda.” 

The\marginnote{1.1.4} messenger went to Upananda and said, “Venerable, I’ve brought a robe fund for you. Please receive it.” 

“We\marginnote{1.1.7} don’t receive robe funds, but we do receive allowable robe-cloth at the right time.” 

“Is\marginnote{1.1.9} there anyone who provides services for you?” 

Just\marginnote{1.1.11} then a lay follower had come to the monastery on some business. Upananda told the messenger, “This lay follower provides services for the monks.” 

The\marginnote{1.1.14} messenger instructed that lay follower and then returned to Upananda, saying, “I’ve instructed the lay follower you pointed out to me. Please go to them at the right time and they’ll give you robe-cloth.” 

Later\marginnote{1.1.17} on that government official sent a message to Upananda, saying, “Please use the robe-cloth. I would like you to use the robe-cloth.” When Upananda did not say anything to that lay follower, that government official sent him a second message saying the same thing. When Upananda still did not say anything to that lay follower, that government official sent him a third message. 

At\marginnote{1.2.1} that time the householder association had made an agreement that whoever came late to a meeting would be fined fifty coins. And now they were having a meeting. Just then Upananda went to that lay follower and said, “I need robe-cloth.” 

“Please\marginnote{1.2.5} wait one day, venerable, for today there’s a meeting of the householder association. They’ve made an agreement that whoever comes late gets fined fifty coins.” 

Saying,\marginnote{1.2.7} “Give me the robe-cloth today,” he grabbed him by the belt. 

Being\marginnote{1.2.8} pressured by Upananda, the lay follower bought him robe-cloth, and as a consequence he was late for the meeting. People asked him, “Sir, why are you late? You’ve just lost fifty coins.” 

When\marginnote{1.2.11} that lay follower told them what had happened, they complained and criticized Upananda, “These Sakyan monastics have great desires. They’re not content. It’s no easy matter to provide them with a service. How could Venerable Upananda not agree when asked by a lay follower to wait for a day?” 

The\marginnote{1.2.17} monks heard the complaints of those people, and the monks of few desires complained and criticized him, “How could Venerable Upananda not agree when asked by a lay follower to wait for a day?” 

After\marginnote{1.2.21} rebuking him in many ways, they told the Buddha. Soon afterwards he had the Sangha gathered and questioned Upananda: “Is it true, Upananda, that you did this?” 

“It’s\marginnote{1.2.24} true, sir.” 

The\marginnote{1.2.25} Buddha rebuked him … “Foolish man, how could you do this? This will affect people’s confidence …” … “And, monks, this training rule should be recited like this: 

\subsection*{Final ruling }

\scrule{‘If a king, a king’s employee, a brahmin, or a householder sends a robe fund for a monk by messenger, saying, “Buy robe-cloth with this robe fund and give it to monk so-and-so,” and the messenger goes to that monk and says, “Venerable, I have brought a robe fund for you. Please receive it,” then that monk should reply, ‘We don’t receive robe funds, but we do receive allowable robe-cloth at the right time.’ If that messenger says, “Is there anyone who provides services for you?” the monk, if he needs robe-cloth, should point out a monastery worker or a lay follower and say, “They provide services for the monks.” If the messenger instructs that service provider and then returns to the monk and says, “Venerable, I have instructed the service provider you pointed out. Please go to them at the right time and they’ll give you robe-cloth,” then, if that monk needs robe-cloth, he should go to that service provider and prompt them and remind them two or three times, saying, “I need robe-cloth.” If he then gets robe-cloth, all is well. If he does not get it, he should stand in silence for it at most six times. If he then gets robe-cloth, all is well. If he makes any further effort and then gets robe-cloth, he commits an offense entailing relinquishment and confession. If he does not get robe-cloth, he should go to the owners of that robe fund, or send a message, saying, “That monk hasn’t received any benefit from the robe fund you sent for him. Please recover what’s yours, or it might perish.” This is the proper procedure.’” }

\subsection*{Definitions }

\begin{description}%
\item[For a monk: ] for the benefit of a monk; making a monk the object of consideration, one wants to give to him. %
\item[A king: ] whoever rules. %
\item[A king’s employee: ] whoever gets food and wages from a king. %
\item[A brahmin: ] a brahmin by birth. %
\item[A householder: ] anyone apart from a king, a king’s employee, and a brahmin. %
\item[A robe fund: ] money, gold, a pearl, or a gem. %
\item[With this robe fund: ] with that which is at one’s disposal. %
\item[Buy: ] having traded. %
\item[Give: ] donate. %
\item[And the messenger goes to that monk and says, “Venerable, I have brought a robe fund for you. Please receive it,” then that monk should reply, “We don’t receive robe funds, but we do receive allowable robe-cloth at the right time.” If that messenger says, “Is there anyone who provides services for you?” the monk, if he needs robe-cloth, should point out a monastery worker or a lay follower and say, “They provides services for the monks”: ] he should not say, “Give it to them,” “They’ll put it aside,” “They’ll trade it,” “They’ll buy it.” %
\item[If the messenger instructs that service provider and then returns to the monk and says, “Venerable, I have instructed the service provider you pointed out. Please go to them at the right time and they’ll give you robe-cloth,” then, if that monk needs robe-cloth, he should go to that service provider and prompt them and remind them two or three times, saying,\footnote{“Prompt” renders \textit{codetabbo}. For a discussion of the meaning of the verb \textit{codeti}, see Appendix of Technical Terms. } “I need robe-cloth”: ] he\marginnote{2.1.32} should not say, “Give me robe-cloth,” “Get me robe-cloth,” “Trade me robe-cloth,” “Buy me robe-cloth.” 

He\marginnote{2.1.33} should say it a second and a third time. 

%
\item[If he gets it, all is well. If he does not get it, he should go there and stand in silence for it: ] he\marginnote{2.1.37} should not sit down on a seat; he should not receive food; he should not give a teaching. If he is asked, “Why have you come?” he should say, “Think about it.”\footnote{\textit{\textsanskrit{Jānāhi}}. This verb, which normally means “to understand” or “to find out”, is very flexible in its usage. } If he sits down on a seat, or he receives food, or he gives a teaching, he loses one allowance to stand. 

He\marginnote{2.1.41} should stand a second and a third time. If he prompts four times, he can stand four times.\footnote{The point here and below is that each prompting is equivalent to two standings. You can choose one or the other. } If he prompts five times, he can stand twice. If he prompts six times, he cannot stand at all. 

%
\item[If he makes any further effort and then gets robe-cloth: ] for\marginnote{2.2.2} the effort there is an act of wrong conduct. When he gets the robe-cloth, it becomes subject to relinquishment. 

The\marginnote{2.2.4} robe-cloth should be relinquished to a sangha, a group, or an individual. “And, monks, it should be relinquished like this. … (To be expanded as in \href{https://suttacentral.net/pli-tv-bu-vb-np1\#3.2.5}{Bu NP 1:3.2.5}–3.2.29, with appropriate substitutions.) 

‘Venerables,\marginnote{2.2.7} this robe-cloth, which I got after prompting more than three times and standing more than six times, is to be relinquished. I relinquish it to the Sangha.’ … the Sangha should give … you should give … ‘I give this robe-cloth back to you.’” 

%
\item[If he does not get robe-cloth, he should go to the owner of that robe fund, or send a message, saying, “That monk hasn’t received any benefit from the robe fund you sent for him. Please recover what’s yours, or it might perish.” ] this is the right method. %
\end{description}

\subsection*{Permutations }

If\marginnote{2.3.1} he prompts more than three times and stands more than six times, and he perceives it as more, and he gets robe-cloth, he commits an offense entailing relinquishment and confession. If he prompts more than three times and stands more than six times, but he is unsure of it, and he gets robe-cloth, he commits an offense entailing relinquishment and confession. If he prompts more than three times and stands more than six times, but he perceives it as less, and he gets robe-cloth, he commits an offense entailing relinquishment and confession. 

If\marginnote{2.3.4} he prompts less than three times and stands less than six times, but he perceives it as more, he commits an offense of wrong conduct. If he prompts less than three times and stands less than six times, but he is unsure of it, he commits an offense of wrong conduct. If he prompts less than three times and stands less than six times, and he perceives it as less, there is no offense. 

\subsection*{Non-offenses }

There\marginnote{2.4.1} is no offense: if he prompts three times and stands six times; if he prompts less than three times and stands less than six times; if it is given without prompting; if the owners prompt and then it is given;\footnote{I translate according to the gloss in the \textsanskrit{Kaṅkhāvitaraṇī} commentary, \textit{\textsanskrit{sāmikehi} \textsanskrit{codetvā} dinne}, “After prompting by the owners, it is given.” } if he is insane; if he is the first offender. 

\scendsutta{The training rule on kings, the tenth, is finished. }

\scendvagga{The first subchapter on the robe season is finished. }

\scuddanaintro{This is the summary: }

\begin{scuddana}%
“Three\marginnote{2.4.11} on the ended robe season, \\
And washing, receiving; \\
Three on those who are unrelated, \\
Of both, and with messenger.” 

%
\end{scuddana}

%
\section*{{\suttatitleacronym Bu Np 11}{\suttatitletranslation 11. The training rule on silk }{\suttatitleroot Kosiya}}
\addcontentsline{toc}{section}{\tocacronym{Bu Np 11} \toctranslation{11. The training rule on silk } \tocroot{Kosiya}}
\markboth{11. The training rule on silk }{Kosiya}
\extramarks{Bu Np 11}{Bu Np 11}

\subsection*{Origin story }

At\marginnote{1.1} one time when the Buddha was staying at \textsanskrit{Āḷavī} at the \textsanskrit{Aggāḷava} Shrine, the monks from the group of six went to silk-makers and said, “Please boil a heap of silkworms and give us silk. We want to make a blanket containing silk.” The silk-makers complained and criticized them, “How can the Sakyan monastics come and say such things to us? It’s our misfortune that we must kill many small creatures because of our livelihoods and because of our wives and children.”\footnote{“Children” renders \textit{putta/\textsanskrit{ā}}. In Pali the male gender takes precedent if a group contains people of both sexes. For instance, the plural \textit{\textsanskrit{puttā}}, “sons”, may mean “children” or “offsping”, depending on the context. In the same way, the plural \textit{\textsanskrit{bhātāro}}, “brothers”, can mean “siblings”. This way of understanding male-gender nouns is confirmed in the introduction to the Pali lexical work the \textsanskrit{Abhidhānappadīpikāṭīkā}: \textit{Ettha hi \textsanskrit{mātā} ca \textsanskrit{pitā} ca pitaro, putto ca \textsanskrit{dhītā} ca \textsanskrit{puttā}, sassu ca sasuro ca \textsanskrit{sasurā}, \textsanskrit{bhātā} ca \textsanskrit{bhaginī} ca \textsanskrit{bhātaroti} \textsanskrit{bhinnaliṅgānampi} ekaseso dassitoti}, “Mother and father are fathers; son and daughter are sons; mother-in-law and father-in-law are fathers-in-law; brother and sister are brothers;’ in this case the split gender is shown with only one gender remaining.” The \textsanskrit{Abhidhānappadīpikāṭīkā} is available online at tipitaka.org. } 

The\marginnote{1.9} monks heard the complaints of those silk-makers, and the monks of few desires complained and criticized those monks, “How could the monks from the group of six go to silk-makers and say such a thing?” 

After\marginnote{1.13} rebuking them in many ways, they told the Buddha. Soon afterwards he had the Sangha gathered and questioned the monks: “Is it true, monks, that you did this?” 

“It’s\marginnote{1.17} true, sir.” 

The\marginnote{1.18} Buddha rebuked them … “Foolish men, how could you do this? This will affect people’s confidence …” … “And, monks, this training rule should be recited like this: 

\subsection*{Final ruling }

\scrule{‘If a monk has a blanket made that contains silk, he commits an offense entailing relinquishment and confession.’” }

\subsection*{Definitions }

\begin{description}%
\item[A: ] whoever … %
\item[Monk: ] … The monk who has been given the full ordination by a unanimous Sangha through a legal procedure consisting of one motion and three announcements that is irreversible and fit to stand—this sort of monk is meant in this case. %
\item[A blanket: ] it is made by strewing, not by weaving.\footnote{For the method of making a \textit{santhata}, see Appendix of Technical Terms. } %
\item[Has made: ] If he makes a blanket that contains even one thread of silk, or he has one made, then for the effort there is an act of wrong conduct. When he gets the blanket, it becomes subject to relinquishment. %
\end{description}

The\marginnote{2.1.10} blanket should be relinquished to a sangha, a group, or an individual. “And, monks, it should be relinquished like this. (To be expanded as in \href{https://suttacentral.net/pli-tv-bu-vb-np1\#3.2.5}{Bu NP 1:3.2.5}–3.2.29, with appropriate substitutions.) 

‘Venerables,\marginnote{2.1.13} this blanket containing silk, which I got made, is to be relinquished. I relinquish it to the Sangha.’ … the Sangha should give … you should give … ‘I give this blanket back to you.’” 

\subsection*{Permutations }

If\marginnote{2.2.1} he finishes what he began himself, he commits an offense entailing relinquishment and confession. If he has others finish what he began himself, he commits an offense entailing relinquishment and confession. If he finishes himself what was begun by others, he commits an offense entailing relinquishment and confession. If he has others finish what was begun by others, he commits an offense entailing relinquishment and confession. 

If\marginnote{2.2.5} he makes one, or has one made, for someone else, he commits an offense of wrong conduct. If he gets one that was made by someone else and then uses it, he commits an offense of wrong conduct. 

\subsection*{Non-offenses }

There\marginnote{2.2.7.1} is no offense: if he makes a canopy, a floor cover, a cloth screen, a mattress, or a pillow; if he is insane; if he is the first offender. 

\scendsutta{The training rule on silk, the first, is finished.\footnote{It is called the first rule since it is the first rule of the second chapter of \textit{nissaggiya \textsanskrit{pācittiyas}} for \textit{bhikkhus}. If, however, one disregards the division into chapters, it is the eleventh rule of the \textit{nissaggiya \textsanskrit{pācittiyas}}. } }

%
\section*{{\suttatitleacronym Bu Np 12}{\suttatitletranslation 12. The training rule on entirely black }{\suttatitleroot Suddhakāla}}
\addcontentsline{toc}{section}{\tocacronym{Bu Np 12} \toctranslation{12. The training rule on entirely black } \tocroot{Suddhakāla}}
\markboth{12. The training rule on entirely black }{Suddhakāla}
\extramarks{Bu Np 12}{Bu Np 12}

\subsection*{Origin story }

At\marginnote{1.1} one time when the Buddha was staying in the hall with the peaked roof in the Great Wood near \textsanskrit{Vesālī}, the monks from the group of six were having blankets made entirely of black wool. People who were walking about the dwellings complained and criticized them, “How could the Sakyan monastics have blankets made entirely of black wool? They’re just like householders who indulge in worldly pleasures!” 

The\marginnote{1.5} monks heard the complaints of those people, and the monks of few desires complained and criticized those monks, “How could the monks from the group of six have blankets made entirely of black wool?” 

After\marginnote{1.8} rebuking those monks in many ways, they told the Buddha. Soon afterwards he had the Sangha gathered and questioned the monks: “Is it true, monks, that you do this?” 

“It’s\marginnote{1.10} true, sir.” 

The\marginnote{1.11} Buddha rebuked them … “Foolish men, how could you do this? This will affect people’s confidence …” … “And, monks, this training rule should be recited like this: 

\subsection*{Final ruling }

\scrule{‘If a monk has a blanket made entirely of black wool, he commits an offense entailing relinquishment and confession.’” }

\subsection*{Definitions }

\begin{description}%
\item[A: ] whoever … %
\item[Monk: ] … The monk who has been given the full ordination by a unanimous Sangha through a legal procedure consisting of one motion and three announcements that is irreversible and fit to stand—this sort of monk is meant in this case. %
\item[Black: ] there are two kinds of black: natural black and dyed black. %
\item[A blanket: ] it is made by strewing, not by weaving. %
\item[Has made: ] if he makes the blanket, or has it made, then for the effort there is an act of wrong conduct. When he gets the blanket, it becomes subject to relinquishment. %
\end{description}

The\marginnote{2.13} blanket should be relinquished to a sangha, a group, or an individual. “And, monks, it should be relinquished like this. (To be expanded as in \href{https://suttacentral.net/pli-tv-bu-vb-np1\#3.2.5}{Bu NP 1:3.2.5}–3.2.29, with appropriate substitutions.) 

‘Venerables,\marginnote{2.16} this blanket, which I got made entirely of black wool, is to be relinquished. I relinquish it to the Sangha.’ … the Sangha should give … you should give … ‘I give this blanket back to you.’” 

\subsection*{Permutations }

If\marginnote{2.21.1} he finishes what he began himself, he commits an offense entailing relinquishment and confession. If he has others finish what he began himself, he commits an offense entailing relinquishment and confession. If he finishes himself what was begun by others, he commits an offense entailing relinquishment and confession. If he has others finish what was begun by others, he commits an offense entailing relinquishment and confession. 

If\marginnote{2.25} he makes one, or has one made, for someone else, he commits an offense of wrong conduct. If he gets one that was made by someone else and then uses it, he commits an offense of wrong conduct. 

\subsection*{Non-offenses }

There\marginnote{2.27.1} is no offense: if he makes a canopy, a floor cover, a cloth screen, a mattress, or a pillow; if he is insane; if he is the first offender. 

\scendsutta{The training rule on entirely black, the second, is finished. }

%
\section*{{\suttatitleacronym Bu Np 13}{\suttatitletranslation 13. The training rule on two parts }{\suttatitleroot Dvebhāga}}
\addcontentsline{toc}{section}{\tocacronym{Bu Np 13} \toctranslation{13. The training rule on two parts } \tocroot{Dvebhāga}}
\markboth{13. The training rule on two parts }{Dvebhāga}
\extramarks{Bu Np 13}{Bu Np 13}

\subsection*{Origin story }

At\marginnote{1.1} one time the Buddha was staying at \textsanskrit{Sāvatthī} in the Jeta Grove, \textsanskrit{Anāthapiṇḍika}’s Monastery. At that time, when the monks from the group of six knew that the Buddha had prohibited having a blanket made entirely of black wool, they added just a little bit of white on the edge, effectively having a blanket made entirely of black wool just as before. The monks of few desires complained and criticized them, “How could the monks from the group of six do this?” 

After\marginnote{1.5} rebuking those monks in many ways, they told the Buddha. Soon afterwards he had the Sangha gathered and questioned the monks: “Is it true, monks, that you do this?” 

“It’s\marginnote{1.7} true, sir.” 

The\marginnote{1.8} Buddha rebuked them … “Foolish men, how could you do this? This will affect people’s confidence …” … “And, monks, this training rule should be recited like this: 

\subsection*{Final ruling }

\scrule{‘If a monk is having a new blanket made, he should use two parts of entirely black wool, a third part of white, and a fourth part of brown. If he has a new blanket made without using two parts of entirely black wool, a third part of white, and a fourth part of brown, he commits an offense entailing relinquishment and confession.’” }

\subsection*{Definitions }

\begin{description}%
\item[New: ] newly made is what is meant. %
\item[A blanket: ] it is made by strewing, not by weaving. %
\item[Is having made: ] making it himself or having it made. %
\item[He should use two parts of entirely black wool: ] having weighed it, he should use two measures. %
\item[A third part of white: ] one measure of white. %
\item[A fourth part of brown: ] one measure of brown. %
\item[Without using two parts of entirely black wool, a third part of white, and a fourth part of brown: ] if he makes it, or has it made, without using two measures of entirely black wool, one measure of white, and one measure of brown, then for the effort there is an act of wrong conduct. When he gets the blanket, it becomes subject to relinquishment. %
\end{description}

The\marginnote{2.16} blanket should be relinquished to a sangha, a group, or an individual. “And, monks, it should be relinquished like this. (To be expanded as in \href{https://suttacentral.net/pli-tv-bu-vb-np1\#3.2.5}{Bu NP 1:3.2.5}–3.2.29, with appropriate substitutions.) 

‘Venerables,\marginnote{2.19} this blanket, which I got made without using two measures of entirely black wool, one measure of white, and one measure of brown, is to be relinquished. I relinquish it to the Sangha.’ … the Sangha should give … you should give … ‘I give this blanket back to you.’” 

\subsection*{Permutations }

If\marginnote{2.24.1} he finishes what he began himself, he commits an offense entailing relinquishment and confession. If he has others finish what he began himself, he commits an offense entailing relinquishment and confession. If he finishes himself what was begun by others, he commits an offense entailing relinquishment and confession. If he has others finish what was begun by others, he commits an offense entailing relinquishment and confession. 

If\marginnote{2.28} he makes one, or has one made, for someone else, he commits an offense of wrong conduct. If he gets one that was made by someone else and then uses it, he commits an offense of wrong conduct. 

\subsection*{Non-offenses }

There\marginnote{2.30.1} is no offense: if he makes one using one measure of white and one measure of brown; if he makes one using more than one measure of white and more than one measure of brown; if he makes one using just white and brown; if he makes a canopy, a floor cover, a cloth screen, a mattress, or a pillow; if he is insane; if he is the first offender. 

\scendsutta{The training rule on two parts, the third, is finished. }

%
\section*{{\suttatitleacronym Bu Np 14}{\suttatitletranslation 14. The training rule on six years }{\suttatitleroot Chabbassa}}
\addcontentsline{toc}{section}{\tocacronym{Bu Np 14} \toctranslation{14. The training rule on six years } \tocroot{Chabbassa}}
\markboth{14. The training rule on six years }{Chabbassa}
\extramarks{Bu Np 14}{Bu Np 14}

\subsection*{Origin story }

\subsubsection*{First sub-story }

At\marginnote{1.1} one time the Buddha was staying at \textsanskrit{Sāvatthī} in the Jeta Grove, \textsanskrit{Anāthapiṇḍika}’s Monastery. At that time monks were having blankets made every year. They kept on begging and asking, “Please give wool! We need wool!” People complained and criticized them, “How can the Sakyan monastics have blankets made every year, begging and asking, ‘Please give wool! We need wool!’? We only make blankets for ourselves every five or six years, even though our children defecate and urinate on them and they are eaten by rats.” 

The\marginnote{1.12} monks heard the complaints of those people, and the monks of few desires complained and criticized those monks, “How can those monks do this?” 

After\marginnote{1.16} rebuking those monks in many ways, they told the Buddha. Soon afterwards he had the Sangha gathered and questioned the monks: “Is it true, monks, that there are monks who do this?” 

“It’s\marginnote{1.19} true, sir.” 

The\marginnote{1.20} Buddha rebuked them, “How can those foolish men do this? This will affect people’s confidence …” … “And, monks, this training rule should be recited like this: 

\subsubsection*{Preliminary ruling }

\scrule{‘If a monk has had a new blanket made, he should keep it for six years. Whether that blanket has been given away or not, if he has another new blanket made in less than six years, he commits an offense entailing relinquishment and confession.’” }

In\marginnote{1.27} this way the Buddha laid down this training rule for the monks. 

\subsubsection*{Second sub-story }

At\marginnote{2.1} one time a certain monk at \textsanskrit{Kosambī} was sick. His relatives sent him a message, saying, “Come, venerable, we’ll nurse you.” The monks urged him to go, but he said, “The Buddha has laid down a training rule that a monk who has had a new blanket made must keep it for six years. Now because I’m sick, I’m unable to travel with my blanket. And because I’m not comfortable without it, I can’t go.” 

They\marginnote{2.12} told the Buddha. 

Soon\marginnote{2.13} afterwards he gave a teaching and addressed the monks: 

\scrule{“Monks, I allow you to give permission to a sick monk to make a blanket.\footnote{“Permission to make a blanket”, renders \textit{santhatasammuti}. The literal meaning, “blanket permission”, is ambiguous in English. } }

And\marginnote{2.15} it should be given like this. After approaching the Sangha, the sick monk should arrange his upper robe over one shoulder and pay respect at the feet of the senior monks. He should then squat on his heels, raise his joined palms, and say, ‘Venerables, I’m sick. I’m unable to travel with my blanket. I ask the Sangha for permission to make a blanket.’ And he should ask a second and a third time. A competent and capable monk should then inform the Sangha: 

‘Please,\marginnote{2.23} venerables, I ask the Sangha to listen. The monk so-and-so is sick. He’s unable to travel with his blanket. He’s asking the Sangha for permission to make a blanket. If the Sangha is ready, it should give permission to monk so-and-so to make a blanket. This is the motion. 

Please,\marginnote{2.28} venerables, I ask the Sangha to listen. The monk so-and-so is sick. He’s unable to travel with his blanket. He’s asking the Sangha for permission to make a blanket. The Sangha gives permission to monk so-and-so to make a blanket. Any monk who approves of giving permission to monk so-and-so to make a blanket should remain silent. Any monk who doesn’t approve should speak up. 

The\marginnote{2.34} Sangha has given permission to monk so-and-so to make a blanket. The Sangha approves and is therefore silent. I’ll remember it thus.’ 

And\marginnote{2.37} so, monks, this training rule should be recited like this: 

\subsection*{Final ruling }

\scrule{‘If a monk has had a new blanket made, he should keep it for six years. Whether that blanket has been given away or not, if he has another new blanket made in less than six years, except if the monks have agreed, he commits an offense entailing relinquishment and confession.’” }

\subsection*{Definitions }

\begin{description}%
\item[New: ] newly made is what is meant. %
\item[A blanket: ] it is made by strewing, not by weaving. %
\item[Has had made: ] has made or has had made. %
\item[He should keep it for six years: ] he should keep it for six years at a minimum.\footnote{“At a minimum” renders \textit{\textsanskrit{paramatā}}, which normally means “at a maximum”. Yet \textit{\textsanskrit{paramatā}} also means “in the extreme” and thus in the present case “at a minimum”. } %
\item[In less than six years: ] fewer than six years. %
\item[That blanket has been given away: ] it has been given to others. %
\item[Not: ] it has not been given to anyone. %
\item[Except if the monks have agreed: ] if he makes another new blanket, or has one made, unless the monks have agreed, then for the effort there is an act of wrong conduct. When he gets the blanket, it becomes subject to relinquishment. %
\end{description}

The\marginnote{3.18} blanket should be relinquished to a sangha, a group, or an individual. “And, monks, it should be relinquished like this. (To be expanded as in \href{https://suttacentral.net/pli-tv-bu-vb-np1\#3.2.5}{Bu NP 1:3.2.5}–3.2.29, with appropriate substitutions.) 

‘Venerables,\marginnote{3.21} this blanket, which I got made after less than six years without the agreement of the monks, is to be relinquished. I relinquish it to the Sangha.’ … the Sangha should give … you should give … ‘I give this blanket back to you.’” 

\subsection*{Permutations }

If\marginnote{3.26.1} he finishes what he began himself, he commits an offense entailing relinquishment and confession. If he has others finish what he began himself, he commits an offense entailing relinquishment and confession. If he finishes himself what was begun by others, he commits an offense entailing relinquishment and confession. If he has others finish what was begun by others, he commits an offense entailing relinquishment and confession. 

\subsection*{Non-offenses }

There\marginnote{3.30.1} is no offense: if he makes one after six years; if he makes one after more than six years; if he makes one, or has one made, for the sake of another; if he gets what was made by another and then uses it; if he makes a canopy, a floor cover, a cloth screen, a mattress, or a pillow; if he has the permission of the monks; if he is insane; if he is the first offender. 

\scendsutta{The training rule on six years, the fourth, is finished. }

%
\section*{{\suttatitleacronym Bu Np 15}{\suttatitletranslation 15. The training rule on sitting blankets }{\suttatitleroot Nisīdanasanthata}}
\addcontentsline{toc}{section}{\tocacronym{Bu Np 15} \toctranslation{15. The training rule on sitting blankets } \tocroot{Nisīdanasanthata}}
\markboth{15. The training rule on sitting blankets }{Nisīdanasanthata}
\extramarks{Bu Np 15}{Bu Np 15}

\subsection*{Origin story }

At\marginnote{1.1.1} one time the Buddha was staying at \textsanskrit{Sāvatthī} in the Jeta Grove, \textsanskrit{Anāthapiṇḍika}’s Monastery. There the Buddha addressed the monks: “Monks, I wish to do a solitary retreat for three months. No one should visit me except the one who brings me almsfood.” 

“Yes,\marginnote{1.1.4} sir,” they replied, and no one visited him except the one who brought him almsfood. 

Soon\marginnote{1.1.5} afterwards the Sangha at \textsanskrit{Sāvatthī} made the following agreement: “The Buddha wishes to be on solitary retreat for three months. No one should visit him except the one who brings him almsfood. Anyone who does must confess an offense entailing confession.” 

Just\marginnote{1.1.9} then Venerable Upasena of \textsanskrit{Vaṅganta} and his followers went to the Buddha, bowed, and sat down. Since it is the custom for Buddhas to greet newly-arrived monks, the Buddha said this to Upasena, “I hope you’re keeping well, Upasena, I hope you’re getting by? I hope you’re not tired from traveling?” 

“We’re\marginnote{1.1.13} keeping well, sir, we’re getting by. We’re not tired from traveling.” 

One\marginnote{1.1.15} of Upasena’s students was seated not far from the Buddha, and the Buddha said to him, “Do you like rag-robes, monk?” 

“I\marginnote{1.1.17} don’t like rag-robes, sir.” 

“Why\marginnote{1.1.18} then do you wear them?” 

“My\marginnote{1.1.19} preceptor wears them, and so I do it too.” 

And\marginnote{1.1.20} the Buddha said to Upasena, “Upasena, your followers are inspiring. How do you train them?” 

“When\marginnote{1.1.22} anyone asks me for the full ordination, I tell him this: ‘I stay in the wilderness, I eat only almsfood, and I wear rag-robes. If you do the same, I’ll give you the full ordination.’ If he agrees, I ordain him. Otherwise I don’t. And I do the same when anyone asks me for support.\footnote{“Support” renders \textit{nissaya}. See Appendix of Technical Terms for discussion. } It’s in this way that I train my followers.” 

“Good,\marginnote{1.2.1} Upasena, you train your followers well. But do you know the agreement made by the Sangha at \textsanskrit{Sāvatthī}?” 

“No.”\marginnote{1.2.3} 

“The\marginnote{1.2.4} Sangha at \textsanskrit{Sāvatthī} has made the following agreement: ‘The Buddha wishes to be on solitary retreat for three months. No one should visit him except the one who brings him almsfood. Anyone who does must confess an offense entailing confession.’” 

“Sir,\marginnote{1.2.8} let the Sangha at \textsanskrit{Sāvatthī} be known for this agreement. We, however, don’t lay down new rules, nor do we get rid of the existing ones. We practice and undertake the training rules as they are.” 

“Good,\marginnote{1.2.11} Upasena. One should not lay down new rules, nor should one get rid of the existing ones. One should practice and undertake the training rules as they are. 

\scrule{And, Upasena, I allow those monks who stay in the wilderness, who eat only almsfood, and who wear rag-robes to visit me whenever they please.” }

Upasena\marginnote{1.2.15} and his followers got up from their seats, bowed down, circumambulated the Buddha with their right sides toward him, and left. Just then a number of monks were standing outside the gateway, thinking, “We’ll make Venerable Upasena confess an offense entailing confession.”\footnote{“Outside the gateway” renders \textit{\textsanskrit{bahidvārakoṭṭhaka}} (in the previous sentence). See Appendix of Technical Terms for a discussion of the word \textit{\textsanskrit{koṭṭhaka}}. } And they said to Upasena, “Upasena, do you know the agreement of the Sangha at \textsanskrit{Sāvatthī}?” 

“The\marginnote{1.2.18} Buddha asked me the same question, and I replied that I didn’t. He then told me what it was, and I said, ‘Sir, let the Sangha at \textsanskrit{Sāvatthī} be known for this agreement. We, however, don’t lay down new rules, nor do we get rid of the existing ones. We practice and undertake the training rules as they are.’ Also, the Buddha has allowed those monks who stay in the wilderness, who eat only almsfood, and who wear rag-robes to visit him whenever they please.” 

Those\marginnote{1.2.27} monks thought, “It’s true what Venerable Upasena says.” 

The\marginnote{1.3.1} monks heard that the Buddha had allowed those monks who stay in the wilderness, who eat only almsfood, and who wear rag-robes to visit him whenever they please. Longing to see the Buddha, they discarded their blankets and undertook the practice of staying in the wilderness, of eating only almsfood, and of wearing rag-robes. 

Soon\marginnote{1.3.3} afterwards, when the Buddha and a number of monks were walking about the dwellings, he saw discarded blankets here and there. He asked the monks, “Who owns these discarded blankets?” 

The\marginnote{1.3.5} monks told him. Soon afterwards he gave a teaching and addressed the monks: “Well then, monks, I will lay down a training rule for the following ten reasons: for the well-being of the Sangha, for the comfort of the Sangha, for the restraint of bad people, for the ease of good monks, for the restraint of the corruptions relating to the present life, for the restraint of the corruptions relating to future lives, to give rise to confidence in those without it, to increase the confidence of those who have it, for the longevity of the true Teaching, and for supporting the training. And, monks, this training rule should be recited like this: 

\subsection*{Final ruling }

\scrule{‘If a monk is having a sitting blanket made, he must incorporate a piece of one standard handspan from the border of an old blanket in order to make it ugly. If he has a new sitting blanket made without incorporating a piece of one standard handspan from the border of an old blanket, he commits an offense entailing relinquishment and confession.’” }

\subsection*{Definitions }

\begin{description}%
\item[Sitting (blanket):\footnote{“Sitting (blanket)” renders \textit{\textsanskrit{nisīdana}}. See Appendix of Technical Terms for discussion. } ] one with a border is what is meant. %
\item[A blanket: ] it is made by strewing, not by weaving. %
\item[Is having made: ] making it himself or having it made. %
\item[An old blanket: ] even worn once. %
\item[He must incorporate a piece of one standard handspan from the border in order to make it ugly: ] to make it strong, he cuts out a circular or a rectangular piece, and he then incorporates it in one place or he strews it on after pulling it apart. %
\item[Without incorporating a piece of one standard handspan from the border of an old blanket: ] if he makes a new sitting blanket, or has one made, without incorporating a piece of one standard handspan from the border of an old blanket, then for the effort there is an act of wrong conduct. When he gets the sitting blanket, it becomes subject to relinquishment. %
\end{description}

The\marginnote{2.14} sitting blanket should be relinquished to a sangha, a group, or an individual. “And, monks, it should be relinquished like this. (To be expanded as in \href{https://suttacentral.net/pli-tv-bu-vb-np1\#3.2.5}{Bu NP 1:3.2.5}–3.2.29, with appropriate substitutions.) 

‘Venerables,\marginnote{2.17} this sitting blanket, which I got made without incorporating a piece of one standard handspan from the border of an old blanket, is to be relinquished. I relinquish it to the Sangha.’ … the Sangha should give … you should give … ‘I give this sitting blanket back to you.’” 

\subsection*{Permutations }

If\marginnote{2.22.1} he finishes what he began himself, he commits an offense entailing relinquishment and confession. If he has others finish what he began himself, he commits an offense entailing relinquishment and confession. If he finishes himself what was begun by others, he commits an offense entailing relinquishment and confession. If he has others finish what was begun by others, he commits an offense entailing relinquishment and confession. 

If\marginnote{2.26} he makes one, or has one made, for the sake of another, he commits an offense of wrong conduct. 

\subsection*{Non-offenses }

There\marginnote{2.27.1} is no offense: if he makes one incorporating a piece of one standard handspan from the border of an old blanket; if he is unable to get hold of such a piece and he makes one incorporating a smaller piece; if he is unable to get hold of such a smaller piece and he makes one without; if he gets what was made by another and then uses it; if he makes a canopy, a floor cover, a cloth screen, a mattress, or a pillow; if he is insane; if he is the first offender. 

\scendsutta{The training rule on sitting blankets, the fifth, is finished. }

%
\section*{{\suttatitleacronym Bu Np 16}{\suttatitletranslation 16. The training rule on wool }{\suttatitleroot Eḷakaloma}}
\addcontentsline{toc}{section}{\tocacronym{Bu Np 16} \toctranslation{16. The training rule on wool } \tocroot{Eḷakaloma}}
\markboth{16. The training rule on wool }{Eḷakaloma}
\extramarks{Bu Np 16}{Bu Np 16}

\subsection*{Origin story }

At\marginnote{1.1} one time when the Buddha was staying at \textsanskrit{Sāvatthī} in \textsanskrit{Anāthapiṇḍika}’s Monastery, a certain monk was given wool as he was walking through the Kosalan country on his way to \textsanskrit{Sāvatthī}. He bound it into a bundle with his upper robe and carried on. People who saw him teased him, “Venerable, how much did it cost you? How much will the profit be?” As a result he was humiliated. 

When\marginnote{1.8} he arrived at \textsanskrit{Sāvatthī}, he threw the wool to the ground. The monks asked him why. 

“People\marginnote{1.11} have been teasing me because of this wool.” 

“But\marginnote{1.12} how far have you carried it?” 

“Over\marginnote{1.13} 40 kilometers.” 

The\marginnote{1.14} monks of few desires complained and criticized him, “How could a monk carry wool more than 40 kilometers?” 

After\marginnote{1.16} rebuking that monk in many ways, they told the Buddha. Soon afterwards he had the Sangha gathered and questioned the monks: “Is it true, monk, that you did this?” 

“It’s\marginnote{1.18} true, sir.” 

The\marginnote{1.19} Buddha rebuked him … “Foolish man, how could you do this? This will affect people’s confidence …” … “And, monks, this training rule should be recited like this: 

\subsection*{Final ruling }

\scrule{‘If wool is given to a monk who is traveling, he may receive it if he wishes. If he receives it and there is no one else to carry it, he may carry it himself for at most 40 kilometers.\footnote{That is, three \textit{yojanas}. For a discussion of measures and distances, see \textit{sugata} in Appendix of Technical Terms. } If he carries it further than that, even if there is no one else to carry it, he commits an offense entailing relinquishment and confession.’” }

\subsection*{Definitions }

\begin{description}%
\item[To a monk who is traveling: ] to one who is walking on a road. %
\item[If wool is given: ] if it is given by a sangha, by a group, by a relative, by a friend, or if it is discarded wool, or if he got it by means of his own property. %
\item[If he wishes: ] if he desires, he may receive it. %
\item[If he receives it, he may carry it himself for at most 40 kilometers: ] he may carry it himself a maximum of 40 kilometers. %
\item[There is no one else to carry it: ] there is no other person who can carry it, either a woman or a man, either a lay person or a monastic. %
\item[If he carries it further than that, even if there is no one else to carry it: ] when he goes beyond 40 kilometers with the first foot, he commits an offense of wrong conduct. When he goes beyond with the second foot, he commits an offense entailing relinquishment and confession. If he stands within the 40 kilometer limit, but drops it beyond the 40 kilometer limit, he commits an offense entailing relinquishment and confession. If he places it in the vehicle or among the goods of another without their knowledge, and it goes more than 40 kilometers, it becomes subject to relinquishment. %
\end{description}

The\marginnote{2.16} wool should be relinquished to a sangha, a group, or an individual. “And, monks, it should be relinquished like this. … (To be expanded as in \href{https://suttacentral.net/pli-tv-bu-vb-np1\#3.2.5}{Bu NP 1:3.2.5}–3.2.29, with appropriate substitutions.) 

‘Venerables,\marginnote{2.19} this wool, which I have taken more than 40 kilometers, is to be relinquished. I relinquish it to the Sangha.’ … the Sangha should give … you should give … ‘I give this wool back to you.’” 

\subsection*{Permutations }

If\marginnote{2.24.1} he takes it more than 40 kilometers and he perceives it as more, he commits an offense entailing relinquishment and confession. If he takes it more than 40 kilometers, but he is unsure of it, he commits an offense entailing relinquishment and confession. If he takes it more than 40 kilometers, but he perceives it as less, he commits an offense entailing relinquishment and confession. 

If\marginnote{2.27} he takes it less than 40 kilometers, but he perceives it as more, he commits an offense of wrong conduct. If he takes it less than 40 kilometers, but he is unsure of it, he commits an offense of wrong conduct. If he takes it less than 40 kilometers and he perceives it as less, there is no offense. 

\subsection*{Non-offenses }

There\marginnote{2.30.1} is no offense: if he carries it 40 kilometers; if he carries it less than 40 kilometers; if he carries it 40 kilometers and then carries it back; if he takes it 40 kilometers with the intention of staying there, but then takes it further; if he gets back what had been taken from him and then carries it on; if he gets back what he had given up and then carries it on; if he gets someone else to carry it; if it is a finished article; if he is insane; if he is the first offender. 

\scendsutta{The training rule on wool, the sixth, is finished. }

%
\section*{{\suttatitleacronym Bu Np 17}{\suttatitletranslation 17. The training rule on having wool washed }{\suttatitleroot Eḷakalomadhovāpana}}
\addcontentsline{toc}{section}{\tocacronym{Bu Np 17} \toctranslation{17. The training rule on having wool washed } \tocroot{Eḷakalomadhovāpana}}
\markboth{17. The training rule on having wool washed }{Eḷakalomadhovāpana}
\extramarks{Bu Np 17}{Bu Np 17}

\subsection*{Origin story }

At\marginnote{1.1} one time the Buddha was staying in the Sakyan country in the Banyan Tree Monastery at Kapilavatthu. At that time the monks from the group of six had the nuns wash, dye, and comb wool. Because of this, the nuns neglected recitation, testing, the higher morality, the higher mind, and the higher wisdom.\footnote{“Testing” renders \textit{paripuccha}. The basic meaning of \textit{\textsanskrit{paripucchā}} is “to question” or “to ask”, as used for instance in \href{https://suttacentral.net/pli-tv-bu-vb-pc71/en/brahmali\#1.19.1}{Bu Pc 71}. Often, however, as in the present case, it refers to a teacher questioning his student, in the sense of finding out how much the student knows. In such cases I render the word as “testing”. } 

\textsanskrit{Mahāpajāpati}\marginnote{1.4} \textsanskrit{Gotamī} then went to the Buddha and bowed down to him. And the Buddha said to her, “\textsanskrit{Gotamī}, I hope the nuns are heedful, energetic, and diligent?” 

“How\marginnote{1.6} could the nuns be heedful, venerable sir?” And she told him what was happening. 

The\marginnote{1.9} Buddha then instructed, inspired, and gladdened her with a teaching. She bowed down, circumambulated him with her right side toward him, and left. 

Soon\marginnote{1.11} afterwards the Buddha had the Sangha gathered and questioned the monks from the group of six: “Is it true, monks, that you do this?” 

“It’s\marginnote{1.13} true, sir.” 

“Are\marginnote{1.14} they your relatives?” 

“No.”\marginnote{1.15} 

“Foolish\marginnote{1.16} men, people who are unrelated don’t know what’s appropriate and inappropriate, what’s inspiring and uninspiring, in dealing with each other. And still you do this. This will affect people’s confidence …” … “And, monks, this training rule should be recited like this: 

\subsection*{Final ruling }

\scrule{‘If a monk has an unrelated nun wash, dye, or comb wool, he commits an offense entailing relinquishment and confession.’” }

\subsection*{Definitions }

\begin{description}%
\item[A: ] whoever … %
\item[Monk: ] … The monk who has been given the full ordination by a unanimous Sangha through a legal procedure consisting of one motion and three announcements that is irreversible and fit to stand—this sort of monk is meant in this case. %
\item[Unrelated: ] anyone who is not a descendant of one’s male ancestors going back eight generations, either on the mother’s side or on the father’s side.\footnote{Sp 1.505: \textit{Tattha \textsanskrit{yāva} \textsanskrit{sattamā} \textsanskrit{pitāmahayugāti} \textsanskrit{pitupitā} \textsanskrit{pitāmaho}, \textsanskrit{pitāmahassa} \textsanskrit{yugaṁ} \textsanskrit{pitāmahayugaṁ}}, “In this \textit{\textsanskrit{yāva} \textsanskrit{sattamā} \textsanskrit{pitāmahayuga}} means: the father of a father is a grandfather. The generation of a grandfather is called a \textit{\textsanskrit{pitāmahayuga}}.” So the PaIi phrase \textit{\textsanskrit{yāva} \textsanskrit{sattamā} \textsanskrit{pitāmahayuga}} means “as far as the seventh generation of grandfathers”, that is, eight generations back. This can be counted as follows: (1) one’s grandfather; (2) his father; (3) 2’s father; (4) 3’s father; (5) 4’s father; (6) 5’s father; and (7) 6’s father. This applies to both one’s paternal and maternal grandfathers. This gives a total of 14 ancestors. Anyone who is a descendent of these fourteen is considered a relative. Anyone who is not such a descendent is not regarded as a relative. } %
\item[A nun: ] she has been given the full ordination by both Sanghas. %
\end{description}

If\marginnote{2.9} he tells her to wash it, he commits an offense of wrong conduct. When it has been washed, it becomes subject to relinquishment. If he tells her to dye it, he commits an offense of wrong conduct. When it has been dyed, it becomes subject to relinquishment. If he tells her to comb it, he commits an offense of wrong conduct. When it has been combed, it becomes subject to relinquishment. 

The\marginnote{2.15} wool should be relinquished to a sangha, a group, or an individual. “And, monks, it should be relinquished like this. … (To be expanded as in \href{https://suttacentral.net/pli-tv-bu-vb-np1\#3.2.5}{Bu NP 1:3.2.5}–3.2.29, with appropriate substitutions.) 

‘Venerables,\marginnote{2.18} this wool, which I got washed by an unrelated nun, is to be relinquished. I relinquish it to the Sangha.’ … the Sangha should give … you should give … ‘I give this wool back to you.’” 

\subsection*{Permutations }

If\marginnote{2.23.1} she is unrelated and he perceives her as such, and he has her wash wool, he commits one offense entailing relinquishment and confession. If she is unrelated and he perceives her as such, and he has her wash and dye wool, he commits one offense entailing relinquishment and one offense of wrong conduct. If she is unrelated and he perceives her as such, and he has her wash and comb wool, he commits one offense entailing relinquishment and one offense of wrong conduct. If she is unrelated and he perceives her as such, and he has her wash, dye, and comb wool, he commits one offense entailing relinquishment and two offenses of wrong conduct. 

If\marginnote{2.27} she is unrelated and he perceives her as such, and he has her dye wool, he commits one offense entailing relinquishment and confession. If she is unrelated and he perceives her as such, and he has her dye and comb wool, he commits one offense entailing relinquishment and one offense of wrong conduct. If she is unrelated and he perceives her as such, and he has her dye and wash wool, he commits one offense entailing relinquishment and one offense of wrong conduct. If she is unrelated and he perceives her as such, and he has her dye, comb, and wash wool, he commits one offense entailing relinquishment and two offenses of wrong conduct. 

If\marginnote{2.31} she is unrelated and he perceives her as such, and he has her comb wool, he commits one offense entailing relinquishment and confession. If she is unrelated and he perceives her as such, and he has her comb and wash wool, he commits one offense entailing relinquishment and one offense of wrong conduct. If she is unrelated and he perceives her as such, and he has her comb and dye wool, he commits one offense entailing relinquishment and one offense of wrong conduct. If she is unrelated and he perceives her as such, and he has her comb, wash, and dye wool, he commits one offense entailing relinquishment and two offenses of wrong conduct. 

If\marginnote{2.35} she is unrelated, but he is unsure of it … If she is unrelated, but he perceives her as related … 

If\marginnote{2.37} he has her wash wool belonging to someone else, he commits an offense of wrong conduct. If he has a nun who is fully ordained only on one side do the washing, he commits an offense of wrong conduct. If she is related, but he perceives her as unrelated, he commits an offense of wrong conduct. If she is related, but he is unsure of it, he commits an offense of wrong conduct. If she is related and he perceives her as such, there is no offense. 

\subsection*{Non-offenses }

There\marginnote{2.42.1} is no offense: if a related nun does the washing and an unrelated nun helps her; if a nun does the washing without being asked; if he has a nun wash an unused and finished article; if it is a trainee nun; if it is a novice nun; if he is insane; if he is the first offender. 

\scendsutta{The training rule on having wool washed, the seventh, is finished. }

%
\section*{{\suttatitleacronym Bu Np 18}{\suttatitletranslation 18. The training rule on money }{\suttatitleroot Rūpiya}}
\addcontentsline{toc}{section}{\tocacronym{Bu Np 18} \toctranslation{18. The training rule on money } \tocroot{Rūpiya}}
\markboth{18. The training rule on money }{Rūpiya}
\extramarks{Bu Np 18}{Bu Np 18}

\subsection*{Origin story }

At\marginnote{1.1} one time the Buddha was staying at \textsanskrit{Rājagaha} in the Bamboo Grove, the squirrel sanctuary. At that time Venerable Upananda was associating with a family from which he received a regular meal. Whenever that family obtained food, they put aside a portion for Upananda. And that’s what they did when one evening they obtained some meat. 

The\marginnote{1.6} following morning their son got up early and cried, “Give me meat!” The man said to his wife, “Give him the venerable’s portion. We’ll buy something else for the venerable.” 

On\marginnote{1.9} the same morning Upananda robed up, took his bowl and robe, and went to that family where he sat down on the prepared seat. The man of the house approached Upananda, bowed, sat down, and said, “Yesterday evening, venerable, we obtained some meat, and we put aside a portion for you. But then in the morning our son got up early and cried, ‘Give me meat!’ and we gave him your portion. What can we get you for a \textit{\textsanskrit{kahāpaṇa}}?” 

“Are\marginnote{1.14} you giving up a \textit{\textsanskrit{kahāpaṇa}} coin for me?” 

“Yes.”\marginnote{1.15} 

“Then\marginnote{1.16} just give me that \textit{\textsanskrit{kahāpaṇa}}.” 

After\marginnote{1.17} giving a \textit{\textsanskrit{kahāpaṇa}} to Upananda, that man complained and criticized him, “The Sakyan monastics accept money just as we do.” 

The\marginnote{1.18} monks heard the complaints of that man, and the monks of few desires complained and criticized him, “How could Venerable Upananda receive money?” 

After\marginnote{1.21} rebuking him in many ways, they told the Buddha. Soon afterwards he had the Sangha gathered and questioned Upananda: “Is it true, Upananda, that you did this?” 

“It’s\marginnote{1.23} true, sir.” 

The\marginnote{1.24} Buddha rebuked him … “Foolish man, how could you do this? This will affect people’s confidence …” … “And, monks, this training rule should be recited like this: 

\subsection*{Final ruling }

\scrule{‘If a monk takes, has someone else take, or consents to gold, silver, or money being deposited for him, he commits an offense entailing relinquishment and confession.’”\footnote{“Gold, silver, or money” renders \textit{\textsanskrit{jātarūparajata}}. For a discussion of this compound, see Appendix of Technical Terms. } }

\subsection*{Definitions }

\begin{description}%
\item[A: ] whoever … %
\item[Monk: ] … The monk who has been given the full ordination by a unanimous Sangha through a legal procedure consisting of one motion and three announcements that is irreversible and fit to stand—this sort of monk is meant in this case. %
\item[Gold: ] that which has the color of the Teacher is what is meant. %
\item[Silver: ] a \textit{\textsanskrit{kahāpaṇa}} coin, a copper \textit{\textsanskrit{māsaka}} coin, a wooden \textit{\textsanskrit{māsaka}} coin, a resin \textit{\textsanskrit{māsaka}} coin—whatever is used in commerce. %
\item[Takes: ] if he takes hold of it himself, he commits an offense entailing relinquishment and confession. %
\item[Has someone else take: ] if he has another take hold of it, he commits an offense entailing relinquishment and confession. %
\item[Consents to … being deposited for him: ] if someone says, “This is for you,” and he consents to it being deposited for him, it becomes subject to relinquishment. %
\end{description}

It\marginnote{2.15} should be relinquished in the midst of the Sangha. “And, monks, it should be relinquished like this. After approaching the Sangha, that monk should arrange his upper robe over one shoulder and pay respect at the feet of the senior monks. He should then squat on his heels, raise his joined palms, and say: 

‘Venerables,\marginnote{2.18} I have received money. It is to be relinquished. I relinquish it to the Sangha.’” 

After\marginnote{2.21} relinquishing it, he is to confess the offense. The confession should be received by a competent and capable monk. 

If\marginnote{2.23} a monastery worker or a lay follower is available, you should tell him, “Look into this.” If he says, “What can I get you with this?” one should not say, “Get this or that;” one should point out what is allowable: ghee, oil, honey, or syrup. If he makes a purchase and brings back what is allowable, everyone may enjoy it except the one who received the money. 

If\marginnote{2.32} this is what happens, all is well. If not, he should be told, “Discard it.” If he discards it, all is well. If he does not, a monk who has five qualities should be appointed as the money discarder: one who is not biased by desire, ill will, confusion, or fear, and who knows what has and has not been discarded. 

“And,\marginnote{2.38} monks, this is how he should be appointed. First the monk should be asked and then a competent and capable monk should inform the Sangha: 

‘Please,\marginnote{2.41} venerables, I ask the Sangha to listen. If the Sangha is ready, it should appoint monk so-and-so as the money discarder. This is the motion. 

Please,\marginnote{2.44} venerables, I ask the Sangha to listen. The Sangha appoints monk so-and-so as the money discarder. Any monk who agrees to appointing monk so-and-so as the money discarder should remain silent. Any monk who doesn’t agree should speak up. 

The\marginnote{2.48} Sangha has appointed monk so-and-so as the money discarder. The Sangha approves and is therefore silent. I’ll remember it thus.’” 

The\marginnote{2.51} appointed monk should throw it away without taking note of where. If he takes note of where he throws it, he commits an offense of wrong conduct. 

\subsection*{Permutations }

If\marginnote{2.53.1} it is money, and he perceives it as such, and he receives it, he commits an offense entailing relinquishment and confession. If it is money, but he is unsure of it, and he receives it, he commits an offense entailing relinquishment and confession. If it is money, but he does not perceive it as such, and he receives it, he commits an offense entailing relinquishment and confession. 

If\marginnote{2.56} it is not money, but he perceives it as such, he commits an offense of wrong conduct. If it is not money, but he is unsure of it, he commits an offense of wrong conduct. If it is not money, and he does not perceive it as such, there is no offense. 

\subsection*{Non-offenses }

There\marginnote{2.59.1} is no offense: if, within a monastery or a lodging, he takes it or has someone take it, and he then puts it aside with the thought, “Whoever owns it will come and get it;” if he is insane; if he is the first offender. 

\scendsutta{The training rule on money, the eighth, is finished. }

%
\section*{{\suttatitleacronym Bu Np 19}{\suttatitletranslation 19. The training rule on trades involving money }{\suttatitleroot Rūpiyasaṁvohāra}}
\addcontentsline{toc}{section}{\tocacronym{Bu Np 19} \toctranslation{19. The training rule on trades involving money } \tocroot{Rūpiyasaṁvohāra}}
\markboth{19. The training rule on trades involving money }{Rūpiyasaṁvohāra}
\extramarks{Bu Np 19}{Bu Np 19}

\subsection*{Origin story }

At\marginnote{1.1} one time the Buddha was staying at \textsanskrit{Sāvatthī} in the Jeta Grove, \textsanskrit{Anāthapiṇḍika}’s Monastery. At that time the monks from the group of six engaged in various kinds of trades involving money. People complained and criticized them, “How can the Sakyan monastics engage in trades that involve money? They’re just like householders who indulge in worldly pleasures!” 

The\marginnote{1.5} monks heard the complaints of those people, and the monks of few desires complained and criticized those monks, “How can the monks from the group of six do this?” 

After\marginnote{1.8} rebuking those monks in many ways, they told the Buddha. Soon afterwards he had the Sangha gathered and questioned those monks: “Is it true, monks, that you do this?” 

“It’s\marginnote{1.10} true, sir.” 

The\marginnote{1.11} Buddha rebuked them … “Foolish men, how can you do this? This will affect people’s confidence …” … “And, monks, this training rule should be recited like this: 

\subsection*{Final ruling }

\scrule{‘If a monk engages in various kinds of trades involving money, he commits an offense entailing relinquishment and confession.’” }

\subsection*{Definitions }

\begin{description}%
\item[A: ] whoever … %
\item[Monk: ] … The monk who has been given the full ordination by a unanimous Sangha through a legal procedure consisting of one motion and three announcements that is irreversible and fit to stand—this sort of monk is meant in this case. %
\item[Various kinds: ] what is shaped, what is not shaped, and what is both shaped and not shaped. %
\item[What is shaped: ] what is meant for the head, what is meant for the neck, what is meant for the hands, what is meant for the feet, what is meant for the waist.\footnote{Sp 2.589: \textit{\textsanskrit{Sīsūpagantiādīsu} \textsanskrit{sīsaṁ} \textsanskrit{upagacchatīti} \textsanskrit{sīsūpagaṁ}, potthakesu pana “\textsanskrit{sīsūpaka}”nti \textsanskrit{likhitaṁ}, yassa kassaci \textsanskrit{sīsālaṅkārassetaṁ} \textsanskrit{adhivacanaṁ}. Esa nayo sabbattha}, “In regard to ʻWhat is meant for the head’, etc., ʻWhat is meant for the head’ means what goes to the head. In the books, however, it is written ʻ\textit{\textsanskrit{sīsūpaka}}’. This is a term for whatever is an ornament for the head. This is the method everywhere.” The basic meaning of the Pali word \textit{\textsanskrit{rūpiya}}, here rendered as “money”, is silver. We can infer from this that the ornaments mentioned in the commentary must be made of silver. The commentary suggests that gold is also included: Sp 1.587: \textit{\textsanskrit{Rūpiyasaṁvohāranti} \textsanskrit{jātarūparajataparivattanaṁ}}, “\textit{\textsanskrit{Rūpiyasaṁvohāra}} means exchanging gold and silver.” } %
\item[What is not shaped: ] what is shaped in a lump is what is meant. %
\item[What is both shaped and not shaped: ] both of them. %
\item[Money: ] a golden \textit{\textsanskrit{kahāpaṇa}} coin, a copper \textit{\textsanskrit{māsaka}} coin, a wooden \textit{\textsanskrit{māsaka}} coin, a resin \textit{\textsanskrit{māsaka}} coin—whatever is used in commerce. %
\item[Engages: ] If\marginnote{2.16} he exchanges what is shaped with what is shaped, he commits an offense entailing relinquishment and confession. If he exchanges what is not shaped with what is shaped, he commits an offense entailing relinquishment and confession. If he exchanges what is both shaped and not shaped with what is shaped, he commits an offense entailing relinquishment and confession. 

If\marginnote{2.19} he exchanges what is shaped with what is not shaped, he commits an offense entailing relinquishment and confession. If he exchanges what is not shaped with what is not shaped, he commits an offense entailing relinquishment and confession. If he exchanges what is both shaped and not shaped with what is not shaped, he commits an offense entailing relinquishment and confession. 

If\marginnote{2.22} he exchanges what is shaped with what is both shaped and not shaped, he commits an offense entailing relinquishment and confession. If he exchanges what is not shaped with what is both shaped and not shaped, he commits an offense entailing relinquishment and confession. If he exchanges what is both shaped and not shaped with what is both shaped and not shaped, he commits an offense entailing relinquishment and confession. 

%
\end{description}

It\marginnote{2.25} should be relinquished in the midst of the Sangha. “And, monks, it should be relinquished like this. After approaching the Sangha, that monk should arrange his upper robe over one shoulder and pay respect at the feet of the senior monks. He should then squat on his heels, raise his joined palms, and say, 

‘Venerables,\marginnote{2.28} I have engaged in various kinds of trades involving money. This is to be relinquished. I relinquish it to the Sangha.’” 

After\marginnote{2.31} relinquishing it, he is to confess the offense. The confession should be received by a competent and capable monk. 

If\marginnote{2.33} a monastery worker or a lay follower is available, you should tell him, “Look into this.” If he says, “What can I get you with this?” one should not say, “Get this or that;” one should point out what is allowable: ghee, oil, honey, or syrup. If he makes a purchase and brings back what is allowable, everyone may enjoy it except the one who did the exchange involving money. 

If\marginnote{2.42} this is what happens, all is well. If not, he should be told, “Discard this.” If he discards it, all is well. If he does not, a monk who has five qualities should be appointed as the money discarder: one who is not biased by desire, ill will, confusion, or fear, and who knows what has and has not been discarded. 

“And,\marginnote{2.48} monks, this is how he should be appointed. First the monk should be asked and then a competent and capable monk should inform the Sangha: 

‘Please,\marginnote{2.51} venerables, I ask the Sangha to listen. If the Sangha is ready, it should appoint monk so-and-so as the money discarder. This is the motion. 

Please,\marginnote{2.54} venerables, I ask the Sangha to listen. The Sangha appoints monk so-and-so as the money discarder. Any monk who agrees to appointing monk so-and-so as the money discarder should remain silent. Any monk who doesn’t agree should speak up. 

The\marginnote{2.58} Sangha has appointed monk so-and-so as the money discarder. The Sangha approves and is therefore silent. I’ll remember it thus.’” 

The\marginnote{2.61} appointed monk should throw it away without taking note of where. If he takes note of where he throws it, he commits an offense of wrong conduct. 

\subsection*{Permutations }

If\marginnote{2.63.1} it is money, and he perceives it as such, and he exchanges it for money, he commits an offense entailing relinquishment and confession.\footnote{The commentarial explanation makes it clear that what is meant here is that one exchanges money for money. The same is true for the next two cases. Sp 1.589: \textit{\textsanskrit{purimasikkhāpade} \textsanskrit{vuttavatthūsu} \textsanskrit{nissaggiyavatthunā} \textsanskrit{nissaggiyavatthuṁ} \textsanskrit{cetāpentassa}}, “For one who is exchanging an object to be relinquished with another object to be relinquished, it is in accordance with the cases spoken of in the previous training rule.” } If it is money, but he is unsure of it, and he exchanges it for money, he commits an offense entailing relinquishment and confession. If it is money, but he does not perceive it as such, and he exchanges it for money, he commits an offense entailing relinquishment and confession. 

If\marginnote{2.66} it is not money, but he perceives it as such, and he exchanges it for money, he commits an offense entailing relinquishment and confession.\footnote{The commentarial explanation makes it clear that what is meant here is that one exchanges what is not money for money, that is, one is selling something and therefore ending up with money. Sp 1.589: \textit{Attano \textsanskrit{vā} hi \textsanskrit{arūpiyena} parassa \textsanskrit{rūpiyaṁ} \textsanskrit{cetāpeyya}}, “Or one exchanges one’s own non-money for someone else’s money.” The same is true for the next two cases. } If it is not money, but he is unsure of it, and he exchanges it for money, he commits an offense entailing relinquishment and confession. If it is not money, and he does not perceive it as such, but he exchanges it for money, he commits an offense entailing relinquishment and confession. 

If\marginnote{2.69} it is not money, but he perceives it as such, he commits an offense of wrong conduct. If it is not money, but he is unsure of it, he commits an offense of wrong conduct. If it is not money, and he does not perceive it as such, there is no offense. 

\subsection*{Non-offenses }

There\marginnote{2.72.1} is no offense: if he is insane; if he is the first offender. 

\scendsutta{The training rule on trades involving money, the ninth, is finished. }

%
\section*{{\suttatitleacronym Bu Np 20}{\suttatitletranslation 20. The training rule on bartering }{\suttatitleroot Kayavikkaya}}
\addcontentsline{toc}{section}{\tocacronym{Bu Np 20} \toctranslation{20. The training rule on bartering } \tocroot{Kayavikkaya}}
\markboth{20. The training rule on bartering }{Kayavikkaya}
\extramarks{Bu Np 20}{Bu Np 20}

\subsection*{Origin story }

At\marginnote{1.1} one time the Buddha was staying at \textsanskrit{Sāvatthī} in the Jeta Grove, \textsanskrit{Anāthapiṇḍika}’s Monastery. At that time Venerable Upananda the Sakyan had become skilled at making robes. He made an upper robe of old cloth, well-dyed and beautifully made, and he wore it.\footnote{For the rendering of \textit{\textsanskrit{saṅghāṭi}} as “upper robe”, see Appendix of Technical Terms. } 

Just\marginnote{1.4} then a certain wanderer who was wearing an expensive robe went up to Upananda and said, “Your upper robe is beautiful. Please give it to me in exchange for my robe.” 

“Are\marginnote{1.7} you sure?” 

“I\marginnote{1.8} am.” 

Saying,\marginnote{1.9} “Alright, then,” he gave it. 

The\marginnote{1.10} wanderer put on the upper robe and went to the wanderers’ monastery. The wanderers said to him, “This upper robe of yours is beautiful. Where did you get it?” 

“I\marginnote{1.14} traded it for my robe.” 

“But\marginnote{1.15} how long will it last? Your other robe was better.” 

The\marginnote{1.17} wanderer realized they were right, and so he returned to Upananda and said, “Here’s your upper robe. Please give me back mine.” 

“But\marginnote{1.20} didn’t I ask you if you were sure? I won’t give it back.” 

Then\marginnote{1.22} that wanderer complained and criticized him, “Even householders give back to each other when they regret a trade. How, then, can a monastic not do the same?” 

The\marginnote{1.25} monks heard the complaints of that wanderer, and the monks of few desires complained and criticized Upananda, “How could Venerable Upananda barter with a wanderer?” 

After\marginnote{1.28} rebuking him in many ways, they told the Buddha. Soon afterwards he had the Sangha gathered and questioned Upananda: “Is it true, Upananda, that you did this?” 

“It’s\marginnote{1.30} true, sir.” 

The\marginnote{1.31} Buddha rebuked him … “Foolish man, how could you do this? This will affect people’s confidence …” … “And, monks, this training rule should be recited like this: 

\subsection*{Final ruling }

\scrule{‘If a monk engages in various kinds of barter, he commits an offense entailing relinquishment and confession.’” }

\subsection*{Definitions }

\begin{description}%
\item[A: ] whoever … %
\item[Monk: ] … The monk who has been given the full ordination by a unanimous Sangha through a legal procedure consisting of one motion and three announcements that is irreversible and fit to stand—this sort of monk is meant in this case. %
\item[Various kinds: ] robes, almsfood, a dwelling, or medicinal supplies; even a bit of bath powder, a tooth cleaner, or a piece of string. %
\item[Engages in barter: ] if he misbehaves, saying, “Give that for this,” “Bring that for this,” “Trade that with this,” “Exchange that for this,” he commits an offense of wrong conduct. When it has been bartered—his own goods are in the hands of the other and the other’s goods are in his own hands—it becomes subject to relinquishment. %
\end{description}

The\marginnote{2.10} goods should be relinquished to a sangha, a group, or an individual. “And, monks, they should be relinquished like this. (To be expanded as in \href{https://suttacentral.net/pli-tv-bu-vb-np1\#3.2.5}{Bu NP 1:3.2.5}–3.2.29, with appropriate substitutions.) 

‘Venerables,\marginnote{2.13} I have engaged in various kinds of barter. This is to be relinquished. I relinquish it to the Sangha.’ … the Sangha should give … you should give … ‘I give this back to you.’” 

\subsection*{Permutations }

If\marginnote{2.19.1} it is bartering, and he perceives it as such, he commits an offense entailing relinquishment and confession. If it is bartering, but he is unsure of it, he commits an offense entailing relinquishment and confession. If it is bartering, but he does not perceive it as such, he commits an offense entailing relinquishment and confession. 

If\marginnote{2.22} it is not bartering, but he perceives it as such, he commits an offense of wrong conduct. If it is not bartering, but he is unsure of it, he commits an offense of wrong conduct. If it is not bartering, and he does not perceive it as such, there is no offense. 

\subsection*{Non-offenses }

There\marginnote{2.25.1} is no offense: if he asks about the value; if he tells an attendant; if he says, “I have this and I have need of such-and-such;” if he is insane; if he is the first offender. 

\scendsutta{The training rule on bartering, the tenth, is finished. }

\scendvagga{The second subchapter on silk is finished. }

\scuddanaintro{This is the summary: }

\begin{scuddana}%
“Silk,\marginnote{2.34} entirely, two parts, \\
Six years, sitting mat; \\
And two on wool, should take, \\
Two on various kinds.” 

%
\end{scuddana}

%
\section*{{\suttatitleacronym Bu Np 21}{\suttatitletranslation 21. The training rule on almsbowls }{\suttatitleroot Patta}}
\addcontentsline{toc}{section}{\tocacronym{Bu Np 21} \toctranslation{21. The training rule on almsbowls } \tocroot{Patta}}
\markboth{21. The training rule on almsbowls }{Patta}
\extramarks{Bu Np 21}{Bu Np 21}

\subsection*{Origin story }

\subsubsection*{First sub-story }

At\marginnote{1.1} one time when the Buddha was staying at \textsanskrit{Sāvatthī} in \textsanskrit{Anāthapiṇḍika}’s Monastery, the monks from the group of six were storing up many almsbowls. When people walking about dwellings saw this, they complained and criticized them, “How can the Sakyan monastics store up many bowls? Are they starting up as bowl merchants or setting up a bowl shop?” 

The\marginnote{1.6} monks heard the complaints of those people, and the monks of few desires complained and criticized those monks, “How can the monks from the group of six keep extra bowls?” 

After\marginnote{1.9} rebuking those monks in many ways, they told the Buddha. Soon afterwards he had the Sangha gathered and questioned the monks: “Is it true, monks, that you do this?” 

“It’s\marginnote{1.11} true, sir.” 

The\marginnote{1.12} Buddha rebuked them … “Foolish men, how can you do this? This will affect people’s confidence …” … “And, monks, this training rule should be recited like this: 

\subsubsection*{Preliminary ruling }

\scrule{‘If a monk keeps an extra almsbowl, he commits an offense entailing relinquishment and confession.’” }

In\marginnote{1.17} this way the Buddha laid down this training rule for the monks. 

\subsubsection*{Second sub-story }

Soon\marginnote{2.1} afterwards Venerable Ānanda was offered an extra bowl. He wanted to give it to Venerable \textsanskrit{Sāriputta}, who was staying at \textsanskrit{Sāketa}. Knowing that the Buddha had laid down a rule against keeping an extra bowl, Venerable Ānanda thought, “What should I do now?” He told the Buddha, who said, “How long is it, Ānanda, before \textsanskrit{Sāriputta} returns?” 

“Nine\marginnote{2.13} or ten days, venerable sir.” 

Soon\marginnote{2.14} afterwards the Buddha gave a teaching and addressed the monks: “Monks, you should keep an extra almsbowl for ten days at the most. And so, monks, this training rule should be recited like this: 

\subsection*{Final ruling }

\scrule{‘A monk should keep an extra almsbowl for ten days at the most. If he keeps it longer than that, he commits an offense entailing relinquishment and confession.’” }

\subsection*{Definitions }

\begin{description}%
\item[For ten days at the most: ] it should be kept ten days at a maximum. %
\item[An extra almsbowl: ] a bowl that is not determined, nor assigned to another.\footnote{For an explanation of the idea of \textit{\textsanskrit{vikappanā}}, see Appendix of Technical Terms. } %
\item[An almsbowl: ] there\marginnote{3.6} are two kinds of bowls: the iron bowl and the clay bowl. 

And\marginnote{3.7} there are three sizes of bowls: the large bowl, the medium bowl, and the small bowl.\footnote{Sp 1.602: \textit{Tayo pattassa \textsanskrit{vaṇṇāti} \textsanskrit{tīṇi} pattassa \textsanskrit{pamāṇāni}},  “\textit{Tayo pattassa \textsanskrit{vaṇṇā}}:  there are three measures of bowls.” } 

The\marginnote{3.8} large bowl: it takes half an \textit{\textsanskrit{āḷhaka}} measure of boiled rice, a fourth part of fresh food, and a suitable amount of curry.\footnote{Sp 1.602: \textit{\textsanskrit{Aḍḍhāḷhakodanaṁ} \textsanskrit{gaṇhātīti} \textsanskrit{magadhanāḷiyā} \textsanskrit{dvinnaṁ} \textsanskrit{taṇḍulanāḷīnaṁ} \textsanskrit{odanaṁ} \textsanskrit{gaṇhāti}}, “\textit{\textsanskrit{Aḍḍhāḷhakodanaṁ} \textsanskrit{gaṇhāti}}: it takes rice to the quantity of two \textit{\textsanskrit{nāḷī}} measures of rice of the Magadha \textit{\textsanskrit{nāḷī}}.” } 

The\marginnote{3.9} medium bowl: it takes a \textit{\textsanskrit{nāḷika}} measure of boiled rice, a fourth part of fresh food, and a suitable amount of curry.\footnote{The medium size bowl is half the volume of the large bowl. Sp 1.602: \textit{\textsanskrit{Nāḷikodananti} \textsanskrit{magadhanāḷiyā} \textsanskrit{ekāya} \textsanskrit{taṇḍulanāḷiyā} \textsanskrit{odanaṁ}}, “\textit{\textsanskrit{Nāḷikodananti}}: rice to the quantity of one \textit{\textsanskrit{nāḷī}} measure of rice of the Magadha \textit{\textsanskrit{nāḷī}}.” } 

The\marginnote{3.10} small bowl: it takes a \textit{pattha} measure of boiled rice, a fourth part of fresh food, and a suitable amount of curry.\footnote{The small bowl is half the volume of the medium size bowl. Sp 1.602: \textit{Patthodananti \textsanskrit{magadhanāḷiyā} \textsanskrit{upaḍḍhanāḷikodanaṁ}}, “\textit{Patthodananti}: rice to the quantity of half a \textit{\textsanskrit{nāḷī}} of the Magadha \textit{\textsanskrit{nāḷī}} .” } Anything larger than this is not a bowl, nor anything smaller. 

%
\item[If he keeps it longer than that, it becomes subject to relinquishment: ] it becomes subject to relinquishment at dawn on the eleventh day.\footnote{According to the commentary this means the tenth dawn after one received the bowl. Sp 1.463: \textit{\textsanskrit{Cīvaruppādadivasaena} \textsanskrit{saddhiṁ} \textsanskrit{ekādase} \textsanskrit{aruṇuggamane} \textsanskrit{nissaggiyaṁ} \textsanskrit{hotīti} \textsanskrit{veditabbaṁ}}, “It is to be understood in this way: it is to be relinquished on the eleventh dawn including the day when the robe (or bowl) was given.” } %
\end{description}

The\marginnote{3.14} bowl should be relinquished to a sangha, a group, or an individual. “And, monks, it should be relinquished like this. After approaching the Sangha, that monk should arrange his upper robe over one shoulder and pay respect at the feet of the senior monks. He should then squat on his heels, raise his joined palms, and say, 

‘Venerables,\marginnote{3.17} this almsbowl, which I have kept over ten days, is to be relinquished. I relinquish it to the Sangha.’ 

After\marginnote{3.19} relinquishing it, he should confess the offense. The confession should be received by a competent and capable monk. The relinquished bowl is then to be given back: 

‘Please,\marginnote{3.22} venerables, I ask the Sangha to listen. This almsbowl, which was to be relinquished by monk so-and-so, has been relinquished to the Sangha. If the Sangha is ready, it should give this bowl back to monk so-and-so.’ 

Or:\marginnote{3.25} after approaching several monks, that monk should arrange his upper robe over one shoulder and pay respect at the feet of the senior monks. He should then squat on his heels, raise his joined palms, and say: 

‘Venerables,\marginnote{3.26} this almsbowl, which I have kept over ten days, is to be relinquished. I relinquish it to you.’ 

After\marginnote{3.28} relinquishing it, he should confess the offense. The confession should be received by a competent and capable monk. The relinquished bowl is then to be given back: 

‘Please,\marginnote{3.31} venerables, I ask you to listen. This almsbowl, which was to be relinquished by monk so-and-so, has been relinquished to you. If the venerables are ready, you should give this bowl back to monk so-and-so.’ 

Or:\marginnote{3.34} after approaching a single monk, that monk should arrange his upper robe over one shoulder, squat on his heels, raise his joined palms, and say: ‘This almsbowl, which I have kept over ten days, is to be relinquished. I relinquish it to you.’ After relinquishing it, he should confess the offense. The confession should be received by that monk. The relinquished bowl is then to be given back: 

‘I\marginnote{3.40} give this almsbowl back to you.’” 

\subsection*{Permutations }

If\marginnote{3.41.1} it is more than ten days and he perceives it as more, he commits an offense entailing relinquishment and confession. If it is more than ten days, but he is unsure of it, he commits an offense entailing relinquishment and confession. If it is more than ten days, but he perceives it as less, he commits an offense entailing relinquishment and confession. 

If\marginnote{3.44} it has not been determined, but he perceives that it has, he commits an offense entailing relinquishment and confession. If it has not been assigned to another, but he perceives that it has, he commits an offense entailing relinquishment and confession. If it has not been given away, but he perceives that it has, he commits an offense entailing relinquishment and confession. If it has not been lost, but he perceives that it has, he commits an offense entailing relinquishment and confession. If it has not been destroyed, but he perceives that it has, he commits an offense entailing relinquishment and confession. If it has not been broken, but he perceives that it has, he commits an offense entailing relinquishment and confession. If it has not been stolen, but he perceives that it has, he commits an offense entailing relinquishment and confession. 

If\marginnote{3.51} he uses an almsbowl that should be relinquished without first relinquishing it, he commits an offense of wrong conduct. If it is less than ten days, but he perceives it as more, he commits an offense of wrong conduct. If it is less than ten days, but he is unsure of it, he commits an offense of wrong conduct. If it is less than ten days and he perceives it as less, there is no offense. 

\subsection*{Non-offenses }

There\marginnote{3.55.1} is no offense: if, within ten days, it has been determined, assigned to another, given away, lost, destroyed, broken, stolen, or taken on trust;\footnote{“Taken on trust” refers to a situation where you have an agreement with a close friend that you may take their belongings on trust. The conditions for taking on trust are set out at \href{https://suttacentral.net/pli-tv-kd8/en/brahmali\#19.1.5}{Kd 8:19.1.5}. } if he is insane; if he is the first offender. 

Soon\marginnote{4.1} afterwards the monks from the group of six did not give back a relinquished bowl. They told the Buddha. 

\scrule{“Monks, a relinquished almsbowl should be given back. If you don’t give it back, you commit an offense of wrong conduct.” }

\scendsutta{The training rule on almsbowls, the first, is finished. }

%
\section*{{\suttatitleacronym Bu Np 22}{\suttatitletranslation 22. The training rule on fewer than five mends }{\suttatitleroot Ūnapañcabandhana}}
\addcontentsline{toc}{section}{\tocacronym{Bu Np 22} \toctranslation{22. The training rule on fewer than five mends } \tocroot{Ūnapañcabandhana}}
\markboth{22. The training rule on fewer than five mends }{Ūnapañcabandhana}
\extramarks{Bu Np 22}{Bu Np 22}

\subsection*{Origin story }

At\marginnote{1.1.1} one time the Buddha was staying in the Sakyan country in the Banyan Tree Monastery at Kapilavatthu. At that time a potter had invited the monks, saying, “If any of you needs an almsbowl, I’ll provide it.” But the monks had no sense of moderation, and they asked for many bowls. Those who had small bowls asked for large ones, and those who had large bowls asked for small ones. The potter was so busy making bowls for the monks that he was unable to make goods for sale. He could not make a living for himself, and his wives and children suffered. People complained and criticized them, “How can the Sakyan monastics have no sense of moderation and ask for many bowls? This potter is so busy making bowls for them that he is unable to make goods for sale. He is unable to make a living for himself, and his wives and children are suffering.” 

The\marginnote{1.1.13} monks heard the complaints of those people, and the monks of few desires complained and criticized those monks, “How can those monks have no sense of moderation and ask for many bowls?” 

After\marginnote{1.1.16} rebuking those monks in many ways, they told the Buddha. Soon afterwards he had the Sangha gathered and questioned the monks: “Is it true, monks, that there are monks who do this?” 

“It’s\marginnote{1.1.18} true, sir.” 

The\marginnote{1.1.19} Buddha rebuked them … “How can those foolish men do this? This will affect people’s confidence …” … After rebuking them, he gave a teaching and addressed the monks: 

“Monks,\marginnote{1.1.23} a monk should not ask for an almsbowl. If he does, he commits an offense of wrong conduct.” 

Soon\marginnote{1.2.1} afterwards the bowl of a certain monk broke. Knowing that the Buddha had prohibited asking for a bowl and being afraid of wrongdoing, he did not ask for a new one. As a consequence, he collected almsfood with his hands. People complained and criticized him, “How can the Sakyan monastics collect almsfood with their hands, just like the monastics of other religions?” 

The\marginnote{1.2.6} monks heard the complaints of those people and they told the Buddha. Soon afterwards the Buddha gave a teaching and addressed the monks: 

“Monks,\marginnote{1.2.9} I allow you to ask for a new almsbowl if your bowl has been lost or is broken.” 

When\marginnote{1.3.1} they heard about the Buddha’s allowance, the monks from the group of six asked for many bowls even when their existing bowls only had a minor chip or scratch. Once again that potter was so busy making bowls for the monks that he was unable to make goods for sale. He could not make a living for himself, and his wives and children suffered.\footnote{“Children” renders \textit{putta/\textsanskrit{ā}}. In Pali the male gender takes precedent if a group contains people of both sexes. For instance, the plural \textit{\textsanskrit{puttā}}, “sons”, may mean “children” or “offsping”, depending on the context. In the same way, the plural \textit{\textsanskrit{bhātāro}}, “brothers”, can mean “siblings”. This way of understanding male-gender nouns is confirmed in the introduction to the Pali lexical work the \textsanskrit{Abhidhānappadīpikāṭīkā}: \textit{Ettha hi \textsanskrit{mātā} ca \textsanskrit{pitā} ca pitaro, putto ca \textsanskrit{dhītā} ca \textsanskrit{puttā}, sassu ca sasuro ca \textsanskrit{sasurā}, \textsanskrit{bhātā} ca \textsanskrit{bhaginī} ca \textsanskrit{bhātaroti} \textsanskrit{bhinnaliṅgānampi} ekaseso dassitoti}, “Mother and father are fathers; son and daughter are sons; mother-in-law and father-in-law are fathers-in-law; brother and sister are brothers;’ in this case the split gender is shown with only one gender remaining.” The \textsanskrit{Abhidhānappadīpikāṭīkā} is available online at tipitaka.org. } And people complained and criticized them as before. 

The\marginnote{1.3.7} monks heard the complaints of those people, and the monks of few desires complained and criticized those monks, “How can the monks from the group of six ask for many bowls even when their existing bowls only has a minor chip or scratch?” 

After\marginnote{1.3.10} rebuking those monks in many ways, they told the Buddha. Soon afterwards he had the Sangha gathered and questioned the monks: “Is it true, monks, that you do this?” 

“It’s\marginnote{1.3.12} true, sir.” 

The\marginnote{1.3.13} Buddha rebuked them … “Foolish men, how could you do this? This will affect people’s confidence …” … “And, monks, this training rule should be recited like this: 

\subsection*{Final ruling }

\scrule{‘If a monk exchanges an almsbowl with fewer than five mends for a new almsbowl, he commits an offense entailing relinquishment and confession. That monk should relinquish that almsbowl to a gathering of monks. He should then be given the last almsbowl belonging to that gathering: “Monk, this bowl is yours. Keep it until it breaks.” This is the proper procedure.’” }

\subsection*{Definitions }

\begin{description}%
\item[A: ] whoever … %
\item[Monk: ] … The monk who has been given the full ordination by a unanimous Sangha through a legal procedure consisting of one motion and three announcements that is irreversible and fit to stand—this sort of monk is meant in this case. %
\item[An almsbowl with fewer than five mends: ] it has no mends, one mend, two mends, three mends, or four mends. %
\item[An almsbowl with a mend that does not count: ] one that does not have a fracture of 3.5 cm.\footnote{That is, a fracture of two fingerbreadths, \textit{\textsanskrit{dvaṅgulā}}. } %
\item[An almsbowl with a mend that counts: ] one that does have a fracture of 3.5 cm. %
\item[New almsbowl: ] asked for is what is meant. %
\item[Exchanges: ] if he asks, then for the effort there is an act of wrong conduct. When he gets the bowl, it becomes subject to relinquishment. %
\end{description}

It\marginnote{2.1.16} should be relinquished in the midst of the Sangha. All determined bowls should be brought together. One should not determine an inferior bowl, thinking, “I’ll get a valuable one.” 

If\marginnote{2.1.20} one determines an inferior bowl, thinking, “I’ll get a valuable one,” one commits an offense of wrong conduct. 

“And,\marginnote{2.1.21} monks, it should be relinquished like this. After approaching the Sangha, that monk should arrange his upper robe over one shoulder and pay respect at the feet of the senior monks. He should then squat on his heels, raise his joined palms, and say: 

‘Venerables,\marginnote{2.1.23} this almsbowl, which I got in exchange for a bowl that had fewer than five mends, is to be relinquished. I relinquish it to the Sangha.’” 

After\marginnote{2.1.25} relinquishing it, he should confess the offense. The confession should be received by a competent and capable monk. 

A\marginnote{2.1.27} monk who has five qualities should be appointed as the distributor of almsbowls: one who is not biased by favoritism, ill will, confusion, or fear, and who knows what has and has not been distributed. “And, monks, this is how he should be appointed. First the monk should be asked, and then a competent and capable monk should inform the Sangha: 

‘Please,\marginnote{2.1.32} venerables, I ask the Sangha to listen. If the Sangha is ready, it should appoint monk so-and-so as the distributor of almsbowls. This is the motion. 

Please,\marginnote{2.1.35} venerables, I ask the Sangha to listen. The Sangha appoints monk so-and-so as the distributor of almsbowls. Any monk who agrees to appointing monk so-and-so as the distributor of almsbowls should remain silent. Any monk who doesn’t agree should speak up. 

The\marginnote{2.1.39} Sangha has appointed monk so-and-so as the distributor of almsbowls. The Sangha approves and is therefore silent. I’ll remember it thus.’” 

The\marginnote{2.1.42} appointed monk should give away that relinquished bowl. He should tell the most senior monk, “Sir, would you like this bowl?” If the most senior monk takes it, his old bowl should be offered to the next monk. 

He\marginnote{2.1.46} should not not take that bowl out of sympathy. If he does, he commits an offense of wrong conduct. 

It\marginnote{2.1.47} is not to be offered to anyone who does not have bowl. In this manner it should be offered all the way to the most junior monk in the Sangha. 

\begin{description}%
\item[He should then be given the last almsbowl belonging to that gathering: “Monk, this bowl is yours. Keep it until it breaks”: ] That monk is not to store that bowl in an unsuitable place, use it in an unsuitable way, or give it away, thinking, “How may this bowl be lost, destroyed, or broken?” If he stores it in an unsuitable place, uses it in an unsuitable way, or gives it away, he commits an offense of wrong conduct. %
\item[This is the proper procedure: ] this is the right method. %
\end{description}

\subsection*{Permutations }

If\marginnote{2.2.1} he exchanges a bowl without mends for a bowl without mends, he commits an offense entailing relinquishment and confession. If he exchanges a bowl without mends for a bowl with one mend, he commits an offense entailing relinquishment and confession. If he exchanges a bowl without mends for a bowl with two mends, he commits an offense entailing relinquishment and confession. If he exchanges a bowl without mends for a bowl with three mends, he commits an offense entailing relinquishment and confession. If he exchanges a bowl without mends for a bowl with four mends, he commits an offense entailing relinquishment and confession. 

If\marginnote{2.2.6} he exchanges a bowl with one mend for a bowl without mends, he commits an offense entailing relinquishment and confession. If he exchanges a bowl with one mend for a bowl with one mend, he commits an offense entailing relinquishment and confession. If he exchanges a bowl with one mend for a bowl with two mends, he commits an offense entailing relinquishment and confession. If he exchanges a bowl with one mend for a bowl with three mends, he commits an offense entailing relinquishment and confession. If he exchanges a bowl with one mend for a bowl with four mends, he commits an offense entailing relinquishment and confession. 

If\marginnote{2.2.11} he exchanges a bowl with two mends for a bowl without mends, he commits an offense entailing relinquishment and confession. If he exchanges a bowl with two mends for a bowl with one mend … for a bowl with two mends … for a bowl with three mends … If he exchanges a bowl with two mends for a bowl with four mends, he commits an offense entailing relinquishment and confession. 

If\marginnote{2.2.16} he exchanges a bowl with three mends for a bowl without mends … for a bowl with one mend … for a bowl with two mends … for a bowl with three mends … If he exchanges a bowl with three mends for a bowl with four mends, he commits an offense entailing relinquishment and confession. 

If\marginnote{2.2.21} he exchanges a bowl with four mends for a bowl without mends … for a bowl with one mend … for a bowl with two mends … for a bowl with three mends … If he exchanges a bowl with four mends for a bowl with four mends, he commits an offense entailing relinquishment and confession. 

If\marginnote{2.2.26} he exchanges a bowl without mends for a bowl without mends that count, he commits an offense entailing relinquishment and confession. If he exchanges a bowl without mends for a bowl with one mend that counts … for a bowl with two mends that count … for a bowl with three mends that count … If he exchanges a bowl without mends for a bowl with four mends that count, he commits an offense entailing relinquishment and confession. 

If\marginnote{2.2.31} he exchanges a bowl with one mend for a bowl without mends that count … for a bowl with one mend that counts … for a bowl with two mends that count … for a bowl with three mends that count … If he exchanges a bowl with one mend for a bowl with four mends that count, he commits an offense entailing relinquishment and confession. 

If\marginnote{2.2.36} he exchanges a bowl with two mends for a bowl without mends that count … If he exchanges a bowl with two mends for a bowl with four mends that count, he commits an offense entailing relinquishment and confession. 

If\marginnote{2.2.38} he exchanges a bowl with three mends for a bowl without mends that count … If he exchanges a bowl with three mends for a bowl with four mends that count, he commits an offense entailing relinquishment and confession. 

If\marginnote{2.2.40} he exchanges a bowl with four mends for a bowl without mends that count … for a bowl with one mend that counts … for a bowl with two mends that count … for a bowl with three mends that count … If he exchanges a bowl with four mends for a bowl with four mends that count, he commits an offense entailing relinquishment and confession. 

If\marginnote{2.2.45} he exchanges a bowl without mends that count for a bowl without mends, he commits an offense entailing relinquishment and confession. If he exchanges a bowl without mends that count for a bowl with one mend … for a bowl with two mends … for a bowl with three mends … for a bowl with four mends, he commits an offense entailing relinquishment and confession. …\footnote{The Pali seems to be missing ellipses points after \textit{\textsanskrit{pācittiyaṁ}}. Otherwise the three four-item series beginning with \textit{\textsanskrit{ekabandhanokāsena}}, \textit{\textsanskrit{dvibandhanokāsena}}, and \textit{\textsanskrit{tibandhanokāsena}} are not accounted for. } 

If\marginnote{2.2.50} he exchanges a bowl with four mends that count for a bowl without mends … If he exchanges a bowl with four mends that count for a bowl with one mend … for a bowl with two mends … for a bowl with three mends … If he exchanges a bowl with four mends that count for a bowl with four mends, he commits an offense entailing relinquishment and confession. 

If\marginnote{2.2.55} he exchanges a bowl without mends that count for a bowl without mends that count … for a bowl with one mend that counts … for a bowl with two mends that count … for a bowl with three mends that count … If he exchanges a bowl without mends that count for a bowl with four mends that count, he commits an offense entailing relinquishment and confession. …\footnote{Again, the Pali seems to be missing ellipses points after \textit{\textsanskrit{pācittiyaṁ}}. Otherwise the three four-item series beginning with \textit{\textsanskrit{ekabandhanokāsena}}, \textit{\textsanskrit{dvibandhanokāsena}}, and \textit{\textsanskrit{tibandhanokāsena}} are not accounted for. } 

If\marginnote{2.2.60} he exchanges a bowl with four mends that count for a bowl without mends that count … for a bowl with one mend that counts … for a bowl with two mends that count … for a bowl with three mends that count … If he exchanges a bowl with four mends that count for a bowl with four mends that count, he commits an offense entailing relinquishment and confession. 

\subsection*{Non-offenses }

There\marginnote{2.2.65.1} is no offense: if his almsbowl is lost; if his almsbowl is broken; if it is from relatives; if it is from those who have given an invitation; if it is for the benefit of someone else; if it is by means of his own property; if he is insane; if he is the first offender. 

\scendsutta{The training rule on fewer than five mends, the second, is finished. }

%
\section*{{\suttatitleacronym Bu Np 23}{\suttatitletranslation 23. The training rule on tonics }{\suttatitleroot Bhesajja}}
\addcontentsline{toc}{section}{\tocacronym{Bu Np 23} \toctranslation{23. The training rule on tonics } \tocroot{Bhesajja}}
\markboth{23. The training rule on tonics }{Bhesajja}
\extramarks{Bu Np 23}{Bu Np 23}

\subsection*{Origin story }

At\marginnote{1.1.1} one time when the Buddha was staying at \textsanskrit{Sāvatthī} in \textsanskrit{Anāthapiṇḍika}’s Monastery, Venerable Pilindavaccha was clearing a slope near \textsanskrit{Rājagaha}, intending to build a shelter. Just then King Seniya \textsanskrit{Bimbisāra} of Magadha went to Pilindavaccha, bowed, sat down, and said, “Venerable, what are you having made?” 

“I’m\marginnote{1.1.4} clearing a slope, great king. I want to build a shelter.” 

“Do\marginnote{1.1.5} you need a monastery worker?” 

“The\marginnote{1.1.6} Buddha hasn’t allowed monastery workers.” 

“Well\marginnote{1.1.7} then, sir, please ask the Buddha and tell me the outcome.” 

“Yes,\marginnote{1.1.8} great king.” 

Pilindavaccha\marginnote{1.1.9} then instructed, inspired, and gladdened King \textsanskrit{Bimbisāra} with a teaching, after which the king got up from his seat, bowed down, circumambulated Pilindavaccha with his right side toward him, and left. 

Soon\marginnote{1.1.11} afterwards Pilindavaccha sent a message to the Buddha: “Sir, King Seniya \textsanskrit{Bimbisāra} of Magadha wishes to provide a monastery worker. What should I tell him?” The Buddha then gave a teaching and addressed the monks: 

\scrule{“Monks, I allow monastery workers.” }

Once\marginnote{1.1.16} again King \textsanskrit{Bimbisāra} went to Pilindavaccha, bowed, sat down, and said, “Sir, has the Buddha allowed monastery workers?” 

“Yes,\marginnote{1.1.18} great king.” 

“Well\marginnote{1.1.19} then, I’ll provide you with a monastery worker.” 

But\marginnote{1.1.20} after making this promise, he forgot, and only remembered after a long time. He then addressed the official in charge of all practical affairs: “Listen, has the monastery worker I promised been provided?” 

“No,\marginnote{1.1.21} sir, he hasn’t.” 

“How\marginnote{1.1.22} long has it been since we made that promise?” 

The\marginnote{1.1.23} official counted the days and said, “It’s been five hundred days.” 

“Well\marginnote{1.1.24} then, provide him with five hundred monastery workers.” 

“Yes.”\marginnote{1.1.25} 

The\marginnote{1.1.26} official provided Pilindavaccha with those monastery workers and a separate village was established. They called it “The Monastery Workers’ Village” and “Pilinda Village”. 

And\marginnote{1.2.1} Pilindavaccha began associating with the families in that village. 

After\marginnote{1.2.2} robing up one morning, he took his bowl and robe and went to Pilinda Village for alms. At that time they were holding a celebration in that village and the children were dressed up with ornaments and garlands. As Pilindavaccha was walking on continuous almsround, he came to the house of a certain monastery worker where he sat down on the prepared seat. Just then the daughter of that house had seen the other children dressed up in ornaments and garlands. She cried, saying, “Give me a garland! Give me ornaments!” Pilindavaccha asked her mother why the girl was crying. She told him, adding, “Poor people like us can’t afford garlands and ornaments.” Pilindavaccha took a pad of grass and said to the mother, “Here, place this on the girl’s head.” She did, and it turned into a beautiful golden garland. Even the royal compound had nothing like it. 

People\marginnote{1.2.11} told King \textsanskrit{Bimbisāra}, “In the house of such-and-such a monastery worker there’s a beautiful golden garland. Even in your court, sir, there’s nothing like it. So how did those poor people get it? They must have stolen it.” King \textsanskrit{Bimbisāra} had that family imprisoned. 

Once\marginnote{1.2.16} again Pilindavaccha robed up in the morning, took his bowl and robe, and went to Pilinda Village for alms. As he was walking on continuous almsround, he came to the house of that monastery worker. He then asked the neighbors what had happened to that family. 

“The\marginnote{1.2.18} king has jailed them, venerable, because of that golden garland.” 

Pilindavaccha\marginnote{1.3.1} then went to King \textsanskrit{Bimbisāra}’s house and sat down on the prepared seat. King \textsanskrit{Bimbisāra} approached Pilindavaccha, bowed, and sat down. Pilindavaccha said, “Great king, why have you jailed the family of that monastery worker?” 

“Sir,\marginnote{1.3.4} in the house of that monastery worker there was a beautiful golden garland. Even the royal compound has nothing like it. So how did those poor people get it? They must have stolen it.” 

Pilindavaccha\marginnote{1.3.8} then focused his mind on turning King \textsanskrit{Bimbisāra}’s stilt house into gold. As a result, the whole house became gold. He said, “Great king, how did you get so much gold?” 

“Understood,\marginnote{1.3.11} sir! It’s your supernormal power.” He then released that family. 

People\marginnote{1.3.13} said, “They say Venerable Pilindavaccha has performed a superhuman feat, a wonder of supernormal power, for the king and his court!” Being delighted and gaining confidence in Pilindavaccha, they brought him the five tonics: ghee, butter, oil, honey, and syrup. Ordinarily, too, Pilindavaccha was getting the five tonics. Since he was getting so much, he gave it away to his followers, who ended up with an abundance of tonics. After filling up basins and waterpots and setting these aside, they filled their water filters and bags and hung these in the windows. But the tonics were dripping, and the dwellings became infested with rats.\footnote{Sp 1.621: \textit{\textsanskrit{Olīnavilīnānīti} \textsanskrit{heṭṭhā} ca ubhatopassesu ca \textsanskrit{gaḷitāni}}, “\textit{\textsanskrit{Olīnavilīnāni}}: dripping from below and from both sides.” } When people walking about the dwellings saw this, they complained and criticized them, “These Sakyan monastics are hoarding things indoors, just like King Seniya \textsanskrit{Bimbisāra} of Magadha.” 

The\marginnote{1.3.22} monks heard the complaints of those people and the monks of few desires complained and criticized those monks, “How can these monks choose to live with such abundance?” 

After\marginnote{1.3.25} rebuking those monks in many ways, they told the Buddha. Soon afterwards he had the Sangha gathered and questioned the monks: “Is it true, monks, that there are monks who live like this?” 

“It’s\marginnote{1.3.27} true, sir.” 

The\marginnote{1.3.28} Buddha rebuked them … “How can those foolish men live like this? This will affect people’s confidence …” … “And, monks, this training rule should be recited like this: 

\subsection*{Final ruling }

\scrule{‘After being received, the tonics allowable for sick monks—that is, ghee, butter, oil, honey, and syrup—should be used from storage for at most seven days.\footnote{For the rendering of \textit{bhesajja} as “tonic”, see Appendix of Technical Terms. } If one uses them longer than that, one commits an offense entailing relinquishment and confession.’” }

\subsection*{Definitions }

\begin{description}%
\item[The tonics allowable for sick monks: Ghee: ] ghee from cows, ghee from goats, ghee from buffaloes, or ghee from whatever animal whose meat is allowable. %
\item[Butter: ] butter from the same animals. %
\item[Oil: ] sesame oil, mustard oil, honey-tree oil, castor oil, oil from fat. %
\item[Honey: ] honey from bees. %
\item[Syrup: ] from sugarcane.\footnote{“Syrup” renders \textit{\textsanskrit{phāṇita}}. I. B. Horner instead translates it as “molasses”, which doesn’t quite hit the mark. SED defines \textit{\textsanskrit{phāṇita}} as “the inspissated juice of the sugar cane or other plants”, in other words, “cane syrup”. According to the commentary at Sp 1.623 it can be either cooked or uncooked, the difference presumably whether it is raw or concentrated. “Syrup” seems closer to the mark than “molasses”. } %
\item[After being received, they should be used from storage for at most seven days: ] they are to be used for seven days at a maximum. %
\item[If one uses them longer than that, one commits an offense entailing relinquishment: ] it becomes subject to relinquishment at dawn on the eighth day.\footnote{According to the commentary the counting of dawns includes the dawn on which the robe was received, see \href{https://suttacentral.net/pli-tv-bu-vb-np1/en/brahmali\#3.2.2}{Bu NP 1:3.2.2}. It seems reasonable to assume that the same method of counting should be employed in this rule. } %
\end{description}

The\marginnote{2.16} tonics should be relinquished to a sangha, a group, or an individual. “And, monks, they should be relinquished like this: (To be expanded as in \href{https://suttacentral.net/pli-tv-bu-vb-np1\#3.2.5}{Bu NP 1:3.2.5}–3.2.29, with appropriate substitutions.) 

‘Venerables,\marginnote{2.19} these tonics, which I have kept over seven days, are to be relinquished. I relinquish them to the Sangha.’ … the Sangha should give … you should give … ‘I give these tonics back to you.’” 

\subsection*{Permutations }

If\marginnote{2.24.1} it is more than seven days and he perceives it as more, he commits an offense entailing relinquishment and confession. If it is more than seven days, but he is unsure of it, he commits an offense entailing relinquishment and confession. If it is more than seven days, but he perceives it as less, he commits an offense entailing relinquishment and confession. 

If\marginnote{2.27} they have not been determined, but he perceives that they have, he commits an offense entailing relinquishment and confession.\footnote{In connection with the non-offense clause below, Sp 1.224 says the following: \textit{\textsanskrit{Anāpatti} \textsanskrit{antosattāhaṁ} \textsanskrit{adhiṭṭhetīti} \textsanskrit{sattāhabbhantare} \textsanskrit{sappiñca} \textsanskrit{telañca} \textsanskrit{vasañca} \textsanskrit{muddhanitelaṁ} \textsanskrit{vā} \textsanskrit{abbhañjanaṁ} \textsanskrit{vā} \textsanskrit{madhuṁ} \textsanskrit{arumakkhanaṁ} \textsanskrit{phāṇitaṁ} \textsanskrit{gharadhūpanaṁ} \textsanskrit{adhiṭṭheti}, \textsanskrit{anāpatti}}, “\textit{\textsanskrit{Anāpatti} \textsanskrit{antosattāhaṁ} \textsanskrit{adhiṭṭhetīti}}: there is no offense if, within seven days, one determines ghee, oil, or fat as oil for the head or as ointment; honey as ointment for wounds; and syrup as a scent in a dwelling.” } If they have not been given away, but he perceives that they have, he commits an offense entailing relinquishment and confession. If they have not been lost, but he perceives that they have, he commits an offense entailing relinquishment and confession. If they have not been destroyed, but he perceives that they have, he commits an offense entailing relinquishment and confession. If they have not been burned, but he perceives that they have, he commits an offense entailing relinquishment and confession. If they have not been stolen, but he perceives that they have, he commits an offense entailing relinquishment and confession. 

After\marginnote{2.33} the relinquished tonics have been returned, they are not to be used on the body, nor are they to be eaten. They may be used in lamps or as a black coloring agent. Other monks may use them on the body, but they may not eat them. 

If\marginnote{2.36} it is less than seven days, but he perceives it as more, he commits an offense of wrong conduct. If it is less than seven days, but he is unsure of it, he commits an offense of wrong conduct. If it is less than seven days and he perceives it as less, there is no offense. 

\subsection*{Non-offenses }

There\marginnote{2.39.1} is no offense: if within seven days they have been determined, given away, lost, destroyed, burned, stolen, or taken on trust;\footnote{“Taken on trust” refers to a situation where you have an agreement with a close friend that you may take their belongings on trust. The conditions for taking on trust are set out at \href{https://suttacentral.net/pli-tv-kd8/en/brahmali\#19.1.5}{Kd 8:19.1.5}. } if, without any desire for them, he gives them up to a person who is not fully ordained, and he then obtains them again and then uses them; if he is insane; if he is the first offender. 

\scendsutta{The training rule on tonics, the third, is finished. }

%
\section*{{\suttatitleacronym Bu Np 24}{\suttatitletranslation 24. The training rule on the rainy-season robe }{\suttatitleroot Vassikasāṭika}}
\addcontentsline{toc}{section}{\tocacronym{Bu Np 24} \toctranslation{24. The training rule on the rainy-season robe } \tocroot{Vassikasāṭika}}
\markboth{24. The training rule on the rainy-season robe }{Vassikasāṭika}
\extramarks{Bu Np 24}{Bu Np 24}

\subsection*{Origin story }

At\marginnote{1.1} one time when the Buddha was staying at \textsanskrit{Sāvatthī} in \textsanskrit{Anāthapiṇḍika}’s Monastery, he allowed the rainy-season robe for the monks. Knowing that this was the case, the monks from the group of six went looking for cloth for their rainy-season robes in advance. And after sewing them in advance, they wore them. Then, because their rainy-season robes were worn out, they bathed naked in the rain. 

The\marginnote{1.7} monks of few desires complained and criticized them, “How could the monks from the group of six go looking for cloth for their rainy-season robes in advance, sew them in advance, and then wear them, and then, because their rainy-season robes are worn out, bathe naked in the rain?” 

After\marginnote{1.9} rebuking the monks from the group of six in many ways, they told the Buddha. Soon afterwards he had the Sangha gathered and questioned the monks: “Is it true, monks, that you did this?” 

“It’s\marginnote{1.13} true, sir.” 

The\marginnote{1.14} Buddha rebuked them … “Foolish men, how could you do this? This will affect people’s confidence …” … “And, monks, this training rule should be recited like this: 

\subsection*{Final ruling }

\scrule{‘When there is a month left of summer, a monk may go looking for cloth for his rainy-season robe. When there is a half-month left, he may sew it and then wear it. If he goes looking for cloth for his rainy-season robe when there is more than a month left of summer, or if he sews it and then wears it when there is more than a half-month left, he commits an offense entailing relinquishment and confession.’” }

\subsection*{Definitions }

\begin{description}%
\item[When there is a month left of summer, a monk may go looking for cloth for his rainy-season robe: ] after going to those people who previously, too, have given cloth for the rainy-season robes, he should say, “It’s time for the rainy-season robe,” “It’s the occasion for the rainy-season robe,” “Other people, too, are giving cloth for the rainy-season robe.” He should not say, “Give me cloth for the rainy-season robe,” “Bring me cloth for the rainy-season robe,” “Trade me cloth for the rainy-season robe,” “Buy me cloth for the rainy-season robe.” %
\item[When there is a half-month left, he may sew it and then wear it: ] after sewing it during the last half-month of summer, he may wear it. %
\item[When there is more than a month left of summer: ] if he goes looking for cloth for the rainy-season robe when there is more than a month left of summer, he commits an offense entailing relinquishment and confession. %
\item[When there is more than a half-month left: ] if\marginnote{2.9} he wears it after sewing it when there is more than a half-month left of summer, it becomes subject to relinquishment. 

The\marginnote{2.10} rainy-season robe should be relinquished to a sangha, a group, or an individual. “And, monks, it should be relinquished like this. (To be expanded as in \href{https://suttacentral.net/pli-tv-bu-vb-np1\#3.2.5}{Bu NP 1:3.2.5}–3.2.29, with appropriate substitutions.) 

‘Venerables,\marginnote{2.13} this cloth for the rainy-season robe, which I went looking for when there was more than a month left of summer or which I wore after sewing it when there was more than a half-month left of summer, is to be relinquished. I relinquish it to the Sangha.’ … the Sangha should give … you should give … ‘I give this cloth for the rainy-season robe back to you.’” 

%
\end{description}

\subsection*{Permutations }

If\marginnote{2.18.1} there is more than a month left of summer, and he perceives it as more, and he goes looking for cloth for a rainy-season robe, he commits an offense entailing relinquishment and confession. If there is more than a month left of summer, but he is unsure of it, and he goes looking for cloth for a rainy-season robe, he commits an offense entailing relinquishment and confession. If there is more than a month left of summer, but he perceives it as less, and he goes looking for cloth for a rainy-season robe, he commits an offense entailing relinquishment and confession. 

If\marginnote{2.21} there is more than a half-month left of summer, and he perceives it as more, and he wears the rainy-season robe after sewing it, he commits an offense entailing relinquishment and confession. If there is more than a half-month left of summer, but he is unsure of it, and he wears the rainy-season robe after sewing it, he commits an offense entailing relinquishment and confession. If there is more than a half-month left of summer, but he perceives it as less, and he wears the rainy-season robe after sewing it, he commits an offense entailing relinquishment and confession. 

If\marginnote{2.24} he has a rainy-season robe, but he bathes naked in the rain, he commits an offense of wrong conduct. If there is less than a month left of summer, but he perceives it as more, he commits an offense of wrong conduct. If there is less than a month left of summer, but he is unsure of it, he commits an offense of wrong conduct. If there is less than a month left of summer, and he perceives it as less, there is no offense. 

If\marginnote{2.28} there is less than a half-month left of summer, but he perceives it as more, he commits an offense of wrong conduct. If there is less than a half-month left of summer, but he is unsure of it, he commits an offense of wrong conduct. If there is less than a half-month left of summer, and he perceives it as less, there is no offense. 

\subsection*{Non-offenses }

There\marginnote{2.31.1} is no offense: if he goes looking for cloth for the rainy-season robe when there is a month left of summer; if he wears the rainy-season robe after sewing it when there is a half-month left of summer; if he goes looking for cloth for the rainy-season robe when there is less than a month left of summer; if he wears the rainy-season robe after sewing it when there is less than a half-month left of summer; if, after looking for a rainy-season robe, he postpones the rainy-season residence; if, after wearing a rainy-season robe, he postpones the rainy-season residence (in which case he should wash it and store it and then use it at the right time); if his robe has been stolen; if his robe has been lost; if there is an emergency;\footnote{“Emergency” renders \textit{\textsanskrit{āpadāsu}}. See Appendix of Technical Terms for discussion. } if he is insane; if he is the first offender. 

\scendsutta{The training rule on the rainy-season robe, the fourth, is finished. }

%
\section*{{\suttatitleacronym Bu Np 25}{\suttatitletranslation 25. The training rule on taking back a robe }{\suttatitleroot Cīvaraacchindana}}
\addcontentsline{toc}{section}{\tocacronym{Bu Np 25} \toctranslation{25. The training rule on taking back a robe } \tocroot{Cīvaraacchindana}}
\markboth{25. The training rule on taking back a robe }{Cīvaraacchindana}
\extramarks{Bu Np 25}{Bu Np 25}

\subsection*{Origin story }

At\marginnote{1.1} one time when the Buddha was staying at \textsanskrit{Sāvatthī} in \textsanskrit{Anāthapiṇḍika}’s Monastery, Venerable Upananda the Sakyan said to his brother’s student, “Come, let’s go wandering the country.” 

“I\marginnote{1.4} can’t, venerable, my robes are worn out.” 

“I’ll\marginnote{1.6} give you a robe.” And he gave him a robe. 

Soon\marginnote{1.8} afterwards that monk heard that the Buddha was about to go wandering the country. He thought, “Now I’ll go wandering with the Buddha instead.” Then, when Upananda said, “Let’s go,” he replied, “I’m not going with you, but with the Buddha.” 

“Well,\marginnote{1.15} that robe I gave you is going with me,” and he just took it back in anger. 

That\marginnote{1.17} monk told other monks what had happened. And the monks of few desires complained and criticized Upananda, “How could Venerable Upananda give away a robe and then take it back in anger?” 

After\marginnote{1.20} rebuking him in many ways, they told the Buddha. Soon afterwards he had the Sangha gathered and questioned the monks: “Is it true, Upananda, that you did this?” 

“It’s\marginnote{1.22} true, sir.” 

The\marginnote{1.23} Buddha rebuked him … “Foolish man, how could you do this? This will affect people’s confidence …” … “And, monks, this training rule should be recited like this: 

\subsection*{Final ruling }

\scrule{‘If a monk himself gives a robe to a monk, but then, in anger, takes it back or has it taken back, he commits an offense entailing relinquishment and confession.’” }

\subsection*{Definitions }

\begin{description}%
\item[A: ] whoever … %
\item[Monk: ] … The monk who has been given the full ordination by a unanimous Sangha through a legal procedure consisting of one motion and three announcements that is irreversible and fit to stand—this sort of monk is meant in this case. %
\item[To a monk: ] to another monk. %
\item[Himself: ] he himself has given it. %
\item[A robe: ] one of the six kinds of robe-cloth, but not smaller than what can be assigned to another.\footnote{The six are linen, cotton, silk, wool, sunn hemp, and hemp; see \href{https://suttacentral.net/pli-tv-kd8/en/brahmali\#3.1.6}{Kd 8:3.1.6}.  According to \href{https://suttacentral.net/pli-tv-kd8/en/brahmali\#21.1.4}{Kd 8:21.1.4}, the size referred to here  is no smaller than 8 by 4 \textit{\textsanskrit{sugataṅgula}}, “standard fingerbreadths”. For an explanation of the idea of \textit{\textsanskrit{vikappanā}}, see Appendix of Technical Terms. For the rendering of \textit{\textsanskrit{cīvara}} as “robe-cloth”, see the same appendix. } %
\item[In anger: ] discontent, having hatred, hostile. %
\item[Takes back: ] if he takes it back himself, he commits an offense entailing relinquishment and confession. %
\item[Has taken back: ] if he asks another, he commits an offense of wrong conduct. If he only asks once, then even if the other takes back many, it becomes subject to relinquishment.\footnote{“Many” renders \textit{\textsanskrit{bahukaṁ}}. Sp 1.633: \textit{\textsanskrit{Āṇatto} \textsanskrit{bahūni} \textsanskrit{gaṇhāti}, \textsanskrit{ekaṁ} \textsanskrit{pācittiyaṁ}}, “If the one who is asked takes many, there is (only) one offense entailing confession.” } %
\end{description}

The\marginnote{2.18} robe-cloth should be relinquished to a sangha, a group, or an individual. “And, monks, it should be relinquished like this. (To be expanded as in \href{https://suttacentral.net/pli-tv-bu-vb-np1\#3.2.5}{Bu NP 1:3.2.5}–3.2.29, with appropriate substitutions.) 

‘Venerables,\marginnote{2.21} this robe-cloth, which I took back after giving it to a monk, is to be relinquished. I relinquish it to the Sangha.’ … the Sangha should give … you should give … ‘I give this robe-cloth back to you.’” 

\subsection*{Permutations }

If\marginnote{2.25.1} the other person is fully ordained and he perceives them as such, and after giving them robe-cloth, he takes it back in anger or has it taken back, he commits an offense entailing relinquishment and confession. If the other person is fully ordained, but he is unsure of it, and after giving them robe-cloth, he takes it back in anger or has it taken back, he commits an offense entailing relinquishment and confession. If the other person is fully ordained, but he does not perceive them as such, and after giving them robe-cloth, he takes it back in anger or has it taken back, he commits an offense entailing relinquishment and confession. 

If,\marginnote{2.28} after giving them another requisite, he takes it back in anger or has it taken back, he commits an offense of wrong conduct. If, after giving robe-cloth or another requisite to a person who is not fully ordained, he takes it back in anger or has it taken back, he commits an offense of wrong conduct. 

If\marginnote{2.30} the other person is not fully ordained, but he perceives them as such, he commits an offense of wrong conduct. If the other person is not fully ordained, but he is unsure of it, he commits an offense of wrong conduct. If the other person is not fully ordained, and he does not perceive them as such, he commits an offense of wrong conduct. 

\subsection*{Non-offenses }

There\marginnote{2.33.1} is no offense: if the other person gives it back; if he takes it on trust from them;\footnote{This refers to a situation where you have an agreement with a close friend that you may take their belongings on trust. The conditions for taking on trust are set out at \href{https://suttacentral.net/pli-tv-kd8/en/brahmali\#19.1.5}{Kd 8:19.1.5}. } if he is insane; if he is the first offender. 

\scendsutta{The training rule on taking back a robe, the fifth, is finished. }

%
\section*{{\suttatitleacronym Bu Np 26}{\suttatitletranslation 26. The training rule on asking for thread }{\suttatitleroot Suttaviññatti}}
\addcontentsline{toc}{section}{\tocacronym{Bu Np 26} \toctranslation{26. The training rule on asking for thread } \tocroot{Suttaviññatti}}
\markboth{26. The training rule on asking for thread }{Suttaviññatti}
\extramarks{Bu Np 26}{Bu Np 26}

\subsection*{Origin story }

At\marginnote{1.1} one time when the Buddha was staying at \textsanskrit{Rājagaha} in the Bamboo Grove, the monks from the group of six were making robes and they asked for a large amount of thread.\footnote{From the origin story to Bu Pc 32 it seems \textit{\textsanskrit{cīvarakārasamaya}} refers to any time one is making a robe, see \href{https://suttacentral.net/pli-tv-bu-vb-pc32/en/brahmali\#9.1.10}{Bu Pc 32:9.1.10}. } But when their robes were finished, there was much thread left over. They said, “Well, let’s ask for even more thread and get weavers to weave us robe-cloth.” Yet even when that robe-cloth had been woven, there was much thread left over. A second time they asked for more thread and had weavers weave them robe-cloth. Once again there was much thread left over. A third time they asked for more thread and had weavers weave them robe-cloth. People complained and criticized them, “How could the Sakyan monastics ask for thread and then get weavers to weave them robe-cloth?” 

The\marginnote{1.13} monks heard the complaints of those people, and the monks of few desires complained and criticized those monks, “How could the monks from the group of six ask for thread and then get weavers to weave them robe-cloth?” 

After\marginnote{1.16} rebuking those monks in many ways, they told the Buddha. Soon afterwards he had the Sangha gathered and questioned those monks: “Is it true, monks, that you did this?” 

“It’s\marginnote{1.18} true, sir.” 

The\marginnote{1.19} Buddha rebuked them … “Foolish men, how could you do this? This will affect people’s confidence …” … “And, monks, this training rule should be recited like this: 

\subsection*{Final ruling }

\scrule{‘If a monk himself asks for thread, and then has weavers weave him robe-cloth, he commits an offense entailing relinquishment and confession.’” }

\subsection*{Definitions }

\begin{description}%
\item[A: ] whoever … %
\item[Monk: ] … The monk who has been given the full ordination by a unanimous Sangha through a legal procedure consisting of one motion and three announcements that is irreversible and fit to stand—this sort of monk is meant in this case. %
\item[Himself: ] he himself has asked. %
\item[Thread: ] there are six kinds of thread: linen, cotton, silk, wool, sunn hemp, and hemp.\footnote{Sp 1.636: \textit{\textsanskrit{Sāṇanti} \textsanskrit{sāṇavākasuttaṁ}. \textsanskrit{Bhaṅganti} \textsanskrit{pāṭekkaṁ} \textsanskrit{vākasuttamevāti} eke. Etehi \textsanskrit{pañcahi} \textsanskrit{missetvā} \textsanskrit{katasuttaṁ} pana “\textsanskrit{bhaṅga}”nti \textsanskrit{veditabbaṁ}}, “\textit{\textsanskrit{Sāṇa}}: thread from the bark of hemp. \textit{\textsanskrit{Bhaṅga}}: some say it is just a separate thread from bark. But \textit{\textsanskrit{bhaṅga}} is to be understood as the thread made by mixing the (other) five.” SED says: “\textit{\textsanskrit{śaṇa}}, m. L also n.) a kind of hemp, Cannabis Sativa or Crotolaria Juncea …” And: “\textit{\textsanskrit{bhāṅga}}, mf(\textit{\textsanskrit{ī}})n, (fr. \textit{\textsanskrit{bhaṅgā}}) hempen, made or consisting of hemp …” and “\textsanskrit{Bhaṅgā}, f. hemp (Cannabis Sativa); an intoxicating beverage (or narcotic drug commonly called ‘bhang’) prepared from the hemp plant”. Since SED identifies \textsanskrit{bhaṅga} with Cannabis sativa, I take \textsanskrit{sāṇa} to be Crotolaria juncea, otherwise known as “sunn hemp”. } %
\item[Weavers: ] if he has it woven by weavers, then for every effort there is an act of wrong conduct. When he gets the robe-cloth, it becomes subject to relinquishment. %
\end{description}

The\marginnote{2.13} robe-cloth should be relinquished to a sangha, a group, or an individual. “And, monks, it should be relinquished like this. (To be expanded as in \href{https://suttacentral.net/pli-tv-bu-vb-np1\#3.2.5}{Bu NP 1:3.2.5}–3.2.29, with appropriate substitutions.) 

‘Venerables,\marginnote{2.16} this robe-cloth, which I got weavers to weave after asking for the thread myself, is to be relinquished. I relinquish it to the Sangha.’ … the Sangha should give … you should give … ‘I give this robe-cloth back to you.’” 

\subsection*{Permutations }

If\marginnote{2.21.1} he had it woven, and he perceives that he did, he commits an offense entailing relinquishment and confession. If he had it woven, but he is unsure of it, he commits an offense entailing relinquishment and confession. If he had it woven, but he does not perceive that he did, he commits an offense entailing relinquishment and confession. 

If\marginnote{2.24} he did not have it woven, but he perceives that he did, he commits an offense of wrong conduct. If he did not have it woven, but he is unsure of it, he commits an offense of wrong conduct. If he did not have it woven, and he does not perceive that he did, there is no offense. 

\subsection*{Non-offenses }

There\marginnote{2.27.1} is no offense: if it is to sew a robe; if it is for a back-and-knee strap;\footnote{The \textit{\textsanskrit{āyoga}} is used as a support for the \textit{\textsanskrit{pallattikā}} sitting posture. See Bhikkhu Ñā\textsanskrit{ṇatusita}, “Analysis of the Bhikkhu Pātimokkha”, p. 259, (re. \href{https://suttacentral.net/pli-tv-bu-vb-sk26/en/brahmali\#1.3.1}{Bu Sk 26:1.3.1}). } if it is for a belt; if it is for a shoulder strap; if it is for a bowl bag; if it is for a water filter; if it is from relatives; if it is from those who have given an invitation; if it for the benefit of someone else; if it is by means of one’s own property; if he is insane; if he is the first offender. 

\scendsutta{The training rule on asking for thread, the sixth, is finished. }

%
\section*{{\suttatitleacronym Bu Np 27}{\suttatitletranslation 27. The long training rule on weavers }{\suttatitleroot Mahāpesakāra}}
\addcontentsline{toc}{section}{\tocacronym{Bu Np 27} \toctranslation{27. The long training rule on weavers } \tocroot{Mahāpesakāra}}
\markboth{27. The long training rule on weavers }{Mahāpesakāra}
\extramarks{Bu Np 27}{Bu Np 27}

\subsection*{Origin story }

At\marginnote{1.1} one time when the Buddha was staying at \textsanskrit{Sāvatthī} in \textsanskrit{Anāthapiṇḍika}’s Monastery, a man who was going away said to his wife, “Please weigh some thread, take it to the weavers, get them to weave robe-cloth, and put the robe-cloth aside. When I return, I’ll give it to Venerable Upananda.” 

An\marginnote{1.4} alms-collecting monk heard that man speaking those words. He then went to Upananda the Sakyan and said, “Upananda, you have much merit. In such-and-such a place I heard a man, as he was going away, tell his wife to get robe-cloth woven so that he could give it to you when he returned.” 

“He’s\marginnote{1.7} my supporter.” And the weaver was Upananda’s supporter too. 

Upananda\marginnote{1.9} then went to that weaver and said, “This robe-cloth that you’re weaving for me, make it long and wide. And make it closely woven, well-woven, well-stretched, well-scraped, and well-combed.” 

“Venerable,\marginnote{1.11} they’ve already weighed the thread and given it to me, telling me to weave the robe-cloth with that. I won’t be able to make it long, wide, or closely woven. But I’m able to make it well-woven, well-stretched, well-scraped, and well-combed.” 

“Just\marginnote{1.15} make it long, wide, and closely woven. There’ll be enough thread.” 

Then,\marginnote{1.17} when all the thread had been used up, that weaver went to that woman and said, “Ma’am, I need more thread.” 

“But\marginnote{1.18} didn’t I tell you to weave the robe-cloth with that thread?” 

“You\marginnote{1.19} did. But Venerable Upananda told me to make it long, wide, and closely woven. And he said there would be enough thread.” That woman then gave him as much thread again as she had done the first time. 

When\marginnote{1.23} Upananda heard that the husband had returned from his travels, he went to his house and sat down on the prepared seat. That man approached him, bowed, and sat down. He then said to his wife, “Has the robe-cloth been woven?” 

“Yes,\marginnote{1.26} it has.” 

“Please\marginnote{1.27} bring it. I’ll give it to Venerable Upananda.” 

She\marginnote{1.28} then got the robe-cloth, gave it to her husband, and told him what had happened. After giving the robe-cloth to Upananda, he complained and criticized him, “These Sakyan monastics have great desires; they’re not content. It’s no easy matter to give them robe-cloth. How could Venerable Upananda go to the weavers and say what kind of robe-cloth he wanted without first being invited by me?” 

The\marginnote{1.33} monks heard the complaints of that man, and the monks of few desires complained and criticized Upananda, “How could Venerable Upananda go to a householder’s weavers and say what kind of robe-cloth he wants without first being invited?” 

After\marginnote{1.36} rebuking him in many ways, they told the Buddha. Soon afterwards he had the Sangha gathered and questioned Upananda: “Is it true, Upananda, that you did this?” 

“It’s\marginnote{1.38} true, sir.” 

“Is\marginnote{1.39} he a relative of yours?” 

“No,\marginnote{1.40} sir.” 

“Foolish\marginnote{1.41} man, people who are unrelated don’t know what’s appropriate and inappropriate, what’s good and bad, in dealing with each other. And still you did this. This will affect people’s confidence …” … “And, monks, this training rule should be recited like this: 

\subsection*{Final ruling }

\scrule{‘If a male or female householder is having robe-cloth woven by weavers for an unrelated monk and, without first being invited, that monk goes to those weavers and specifies the kind of robe-cloth he wants, saying, ‘This robe-cloth that you are weaving for me, make it long and wide; make it closely woven, well-woven, well-stretched, well-scraped, and well-combed, and perhaps I will even give you a small gift,’ then, in saying that and afterwards giving them a small gift, even a bit of almsfood, he commits an offense entailing relinquishment and confession.’” }

\subsection*{Definitions }

\begin{description}%
\item[For a monk: ] for the benefit of a monk; making a monk the object of consideration, one wants to give to him. %
\item[Unrelated: ] anyone who is not a descendant of one’s male ancestors going back eight generations, either on the mother’s side or on the father’s side.\footnote{Sp 1.505: \textit{Tattha \textsanskrit{yāva} \textsanskrit{sattamā} \textsanskrit{pitāmahayugāti} \textsanskrit{pitupitā} \textsanskrit{pitāmaho}, \textsanskrit{pitāmahassa} \textsanskrit{yugaṁ} \textsanskrit{pitāmahayugaṁ}}, “In this \textit{\textsanskrit{yāva} \textsanskrit{sattamā} \textsanskrit{pitāmahayuga}} means: the father of a father is a grandfather. The generation of a grandfather is called a \textit{\textsanskrit{pitāmahayuga}}.” So the PaIi phrase \textit{\textsanskrit{yāva} \textsanskrit{sattamā} \textsanskrit{pitāmahayuga}} means “as far as the seventh generation of grandfathers”, that is, eight generations back. This can be counted as follows: (1) one’s grandfather; (2) his father; (3) 2’s father; (4) 3’s father; (5) 4’s father; (6) 5’s father; and (7) 6’s father. This applies to both one’s paternal and maternal grandfathers. This gives a total of 14 ancestors. Anyone who is a descendent of these fourteen is considered a relative. Anyone who is not such a descendent is not regarded as a relative. } %
\item[A male householder: ] any man who lives at home.\footnote{\textit{\textsanskrit{Agāraṁ}} is typically rendered as “in a house”. The problem with this is that it is not unallowable for a monastic to live in a building that is the equivalent of a house. What a monastic should not do is own a home and then live there. } %
\item[A female householder: ] any woman who lives at home. %
\item[By weavers: ] by those who weave. %
\item[Robe-cloth: ] one of the six kinds of robe-cloth, but not smaller than what can be assigned to another.\footnote{The six are linen, cotton, silk, wool, sunn hemp, and hemp; see \href{https://suttacentral.net/pli-tv-kd8/en/brahmali\#3.1.6}{Kd 8:3.1.6}. According to \href{https://suttacentral.net/pli-tv-kd8/en/brahmali\#21.1.4}{Kd 8:21.1.4} this is no smaller than 8 by 4 \textit{\textsanskrit{sugataṅgula}}, “standard fingerbreadths”. For an explanation of the idea of \textit{\textsanskrit{vikappanā}}, see Appendix of Technical Terms. } %
\item[Is having woven: ] is causing to weave. %
\item[If that monk: ] the monk the robe-cloth is being woven for. %
\item[Without first being invited: ] without it first being said, “Venerable, what kind of robe-cloth do you need? What kind of robe-cloth should I get woven for you?” %
\item[Goes to those weavers: ] having gone to their house, having gone up to them wherever. %
\item[Specifies the kind of robe-cloth he wants: ] “This robe-cloth that you are weaving for me, make it long and wide; make it closely woven, well-woven, well-stretched, well-scraped, and well-combed; and perhaps I will even give you a small gift.” %
\item[Then in saying that and afterwards giving them a small gift, even a bit of almsfood—Almsfood: ] congee,\marginnote{2.27} a meal, fresh food, a bit of bath powder, a tooth cleaner, a piece of string, and even if he gives a teaching.\footnote{“Fresh food” renders \textit{\textsanskrit{khādanīya}}. See Appendix of Technical Terms for discussion. } If the weaver makes it long or wide or closely woven because of the monk’s statement, then for the effort there is an act of wrong conduct. When he gets the robe-cloth, it becomes subject to relinquishment. 

The\marginnote{2.30} robe-cloth should be relinquished to a sangha, a group, or an individual. “And, monks, it should be relinquished like this. (To be expanded as in \href{https://suttacentral.net/pli-tv-bu-vb-np1\#3.2.5}{Bu NP 1:3.2.5}–3.2.29, with appropriate substitutions.) 

‘Venerables,\marginnote{2.33} this robe-cloth, for which I went to the weavers of an unrelated householder and said what kind of robe-cloth I wanted without first being invited, is to be relinquished. I relinquish it to the Sangha.’ … the Sangha should give … you should give … ‘I give this robe-cloth back to you.’” 

%
\end{description}

\subsection*{Permutations }

If\marginnote{2.38.1} the householder is unrelated and the monk perceives them as such and, without first being invited, he goes to their weavers and specifies the kind of robe-cloth he wants, he commits an offense entailing relinquishment and confession. If the householder is unrelated, but the monk is unsure of it and, without first being invited, he goes to their weavers and specifies the kind of robe-cloth he wants, he commits an offense entailing relinquishment and confession. If the householder is unrelated, but the monk perceives them as related and, without first being invited, he goes to their weavers and specifies the kind of robe-cloth he wants, he commits an offense entailing relinquishment and confession. 

If\marginnote{2.41} the householder is related, but the monk perceives them as unrelated, he commits an offense of wrong conduct. If the householder is related, but the monk is unsure of it, he commits an offense of wrong conduct. If the householder is related and the monk perceives them as such, there is no offense. 

\subsection*{Non-offenses }

There\marginnote{2.44.1} is no offense: if it is from relatives; if it is from those who have given an invitation; if it is for the benefit of someone else; if it is by means of his own property; if someone wants to have expensive robe-cloth woven, but he has them weave inexpensive robe-cloth instead; if he is insane; if he is the first offender. 

\scendsutta{The long training rule on weavers, the seventh, is finished. }

%
\section*{{\suttatitleacronym Bu Np 28}{\suttatitletranslation 28. The training rule on haste-cloth }{\suttatitleroot Accekacīvara}}
\addcontentsline{toc}{section}{\tocacronym{Bu Np 28} \toctranslation{28. The training rule on haste-cloth } \tocroot{Accekacīvara}}
\markboth{28. The training rule on haste-cloth }{Accekacīvara}
\extramarks{Bu Np 28}{Bu Np 28}

\subsection*{Origin story }

At\marginnote{1.1.1} one time when the Buddha was staying at \textsanskrit{Sāvatthī} in \textsanskrit{Anāthapiṇḍika}’s Monastery, a government official who was about to set out on a journey sent a message to the monks, saying, “Come, venerables, I wish to give robe-cloth to those who have completed the rainy-season residence.” 

The\marginnote{1.1.4} monks thought, “The Buddha has allowed such robes only for those who have completed the rains residence,” and being afraid of wrongdoing they did not go. That government official complained and criticized them, “How could they not come when I send a message? I’m about to set out with the army. It’s hard to know whether I’ll live or die.” 

The\marginnote{1.1.11} monks heard the complaints of that government official, and they told the Buddha. Soon afterwards the Buddha gave a teaching and addressed the monks: 

\scrule{“Monks, I allow you to receive a haste-cloth, and then store it.” }

When\marginnote{1.2.1} they heard about this, monks received haste-cloths and stored them beyond the robe season, keeping them in bundles on a bamboo robe rack. 

While\marginnote{1.2.5} walking about the dwellings, Venerable Ānanda saw that cloth, and he asked the monks, “Whose cloth is this?” 

“It’s\marginnote{1.2.7} our haste-cloth.” 

“But\marginnote{1.2.8} how long have you stored it?” 

They\marginnote{1.2.9} told him. Ānanda then complained and criticized them, “How could these monks receive haste-cloth and then store it beyond the robe season?” 

After\marginnote{1.2.12} rebuking those monks in many ways, Ānanda told the Buddha. Soon afterwards he had the Sangha gathered and questioned the monks: “Is it true, monks, that there are monks who do this?” 

“It’s\marginnote{1.2.14} true, sir.” 

The\marginnote{1.2.15} Buddha rebuked them … “Monks, how could those foolish men do this? This will affect people’s confidence …” … “And, monks, this training rule should be recited like this: 

\subsection*{Final ruling }

\scrule{‘When there are ten days left to the Kattika full moon that ends the first rainy-season residence and haste-cloth is given to a monk, he may receive it if he regards it as urgent. He may then store it until the end of the robe season. If he stores it beyond that, he commits an offense entailing relinquishment and confession.’” }

\subsection*{Definitions }

\begin{description}%
\item[There are ten days left: ] the invitation ceremony is ten days in the future. %
\item[The Kattika full moon that ends the first rainy-season residence: ] the Kattika full moon of the invitation ceremony is what is meant.\footnote{For a discussion of the word \textit{\textsanskrit{pavāraṇā}}, see Appendix of Technical Terms. } %
\item[Haste-cloth: ] when someone intends to set out with the army, when someone intends to set out on a journey, when someone is sick, when someone is pregnant, when someone without faith acquires faith, when someone without confidence acquires confidence—if that person sends a message to the monks, saying, “Come, venerables, I wish to give a robe to those who have completed the rainy-season residence,” this is called “haste-cloth”. %
\item[He may receive it if he regards it as urgent. He may then store it until the end of the robe season: ] establishing the perception of it as a haste-cloth, he may store it. %
\item[The robe season: ] for one who has not participated in the robe-making ceremony, it is the last month of the rainy season;\footnote{“Robe-making ceremony” refers to the \textit{kathina \textsanskrit{saṅghakamma}}, the making of the \textit{kathina} robe, and the rejoicing in the process, all three together represented by the words \textit{(an)atthate kathine }. } for one who has participated in the robe-making ceremony, it is the five month period.\footnote{“The five month period” is the last month of the rainy season plus the four months of the cold season. } %
\item[If he stores it beyond that: ] for one who has not participated in the robe-making ceremony, if he stores it beyond the last day of the rainy season, he commits an offense entailing relinquishment and confession. For one who has participated in the robe-making ceremony, if he stores it beyond the day on which the robe season ends, the cloth becomes subject to relinquishment.\footnote{The robe season ends if the Sangha decides to forgo the robe-season privileges, or if the monk leaves the monastery where he spent the rains residence and gives up any intention of making a robe before the end of the cold season, see \href{https://suttacentral.net/pli-tv-bu-vb-np1/en/brahmali\#3.1.4}{Bu NP 1:3.1.4} and \href{https://suttacentral.net/pli-tv-kd7/en/brahmali\#13.2.7}{Kd 7:13.2.7}. } %
\end{description}

The\marginnote{2.15} cloth should be relinquished to a sangha, a group, or an individual. “And, monks, it should be relinquished like this. (To be expanded as in \href{https://suttacentral.net/pli-tv-bu-vb-np1\#3.2.5}{Bu NP 1:3.2.5}–3.2.29, with appropriate substitutions.) 

‘Venerables,\marginnote{2.18} this haste-cloth, which I have stored beyond the robe season, is to be relinquished. I relinquish it to the Sangha.’ … the Sangha should give … you should give … ‘I give this cloth back to you.’” 

\subsection*{Permutations }

If\marginnote{2.23.1} it is haste-cloth and he perceives it as such, and he stores it beyond the robe season, he commits an offense entailing relinquishment and confession. If it is haste-cloth, but he is unsure of it, and he stores it beyond the robe season, he commits an offense entailing relinquishment and confession. If it is haste-cloth, but he does not perceive it as such, and he stores it beyond the robe season, he commits an offense entailing relinquishment and confession. 

If\marginnote{2.26} it has not been determined, but he perceives that it has … If it has not been assigned to another, but he perceives that it has …\footnote{For an explanation of the idea of \textit{\textsanskrit{vikappanā}}, see Appendix of Technical Terms. } If it has not been given away, but he perceives that it has … If it has not been lost, but he perceives that it has … If it has not been destroyed, but he perceives that it has … If it has not been burned, but he perceives that it has … If it has not been stolen, but he perceives that it has, and he stores it beyond the robe season, he commits an offense entailing relinquishment and confession. 

If\marginnote{2.33} he uses a cloth that should be relinquished without first relinquishing it, he commits an offense of wrong conduct. If it is not haste-cloth, but he perceives it as such, he commits an offense of wrong conduct. If it is not haste-cloth, but he is unsure of it, he commits an offense of wrong conduct. If it is not haste-cloth and he does not perceive it as such, there is no offense. 

\subsection*{Non-offenses }

There\marginnote{2.37.1} is no offense: if within the robe season the haste-cloth has been determined, assigned to another, given away, lost, destroyed, burned, stolen, or taken on trust;\footnote{“Taken on trust” refers to a situation where you have an agreement with a close friend that you may take their belongings on trust. The conditions for taking on trust are set out at \href{https://suttacentral.net/pli-tv-kd8/en/brahmali\#19.1.5}{Kd 8:19.1.5}. } if he is insane; if he is the first offender. 

\scendsutta{The training rule on haste-cloth, the eighth, is finished. }

%
\section*{{\suttatitleacronym Bu Np 29}{\suttatitletranslation 29. The training rule on what is risky }{\suttatitleroot Sāsaṅka}}
\addcontentsline{toc}{section}{\tocacronym{Bu Np 29} \toctranslation{29. The training rule on what is risky } \tocroot{Sāsaṅka}}
\markboth{29. The training rule on what is risky }{Sāsaṅka}
\extramarks{Bu Np 29}{Bu Np 29}

\subsection*{Origin story }

At\marginnote{1.1.1} one time the Buddha was staying at \textsanskrit{Sāvatthī} in the Jeta Grove, \textsanskrit{Anāthapiṇḍika}’s Monastery. At that time monks who had completed the rainy-season residence were staying in wilderness dwellings. Thieves who were active during the month of Kattika attacked those monks, thinking, “They have been given things.” 

The\marginnote{1.1.5} monks told the Buddha. Soon afterwards the Buddha gave a teaching and addressed the monks: 

\scrule{“Monks, I allow monks who are staying in wilderness dwellings to store one of their three robes in an inhabited area.” }

When\marginnote{1.2.1} they heard about this, monks stored one of their three robes in inhabited areas, staying apart from them for more than six days. The robes were lost, destroyed, burned, and eaten by rats. As a consequence, those monks became poorly dressed. Other monks asked them why, and they told them what had happened. The monks of few desires complained and criticized them, “How could those monks store one of their three robes in an inhabited area and then stay apart from it for more than six days?” 

After\marginnote{1.2.9} rebuking those monks in many ways, they told the Buddha. Soon afterwards he had the Sangha gathered and questioned the monks: “Is it true, monks, that there are monks who do this?” 

“It’s\marginnote{1.2.11} true, sir.” 

The\marginnote{1.2.12} Buddha rebuked them … “Monks, how could those foolish men do this? This will affect people’s confidence …” … “And, monks, this training rule should be recited like this: 

\subsection*{Final ruling }

\scrule{‘There are wilderness dwellings that are considered risky and dangerous. After observing the Kattika full moon that ends the rainy season, a monk who is staying in such a dwelling may, if he so desires, store one of his three robes in an inhabited area so long as he has a reason for staying apart from that robe. He should stay apart from that robe for six days at the most. If he stays apart from it longer than that, except if the monks have agreed, he commits an offense entailing relinquishment and confession.’” }

\subsection*{Definitions }

\begin{description}%
\item[After observing: ] after completing the rainy season. %
\item[The Kattika full moon that ends the rainy season: ] the fourth full moon of the rainy season in the month of Kattika is what is meant. %
\item[There are wilderness dwellings: ] a wilderness dwelling: it is at least 800 meters away from any inhabited area.\footnote{That is, five hundred bow-lengths. For further discussion of the \textit{dhanu}, see \textit{sugata} in the Appendix of Technical Terms. } %
\item[Risky: ] in the monastery, or in the vicinity of the monastery, thieves have been seen camping, eating, standing, sitting, or lying down. %
\item[Dangerous: ] in the monastery, or in the vicinity of the monastery, thieves have been seen injuring, robbing, or beating people. %
\item[A monk who is staying in such a dwelling: ] a monk who is staying in that kind of dwelling. %
\item[If he so desires: ] if he so wishes. %
\item[One of his three robes: ] the outer robe, the upper robe, or the sarong. %
\item[May store in an inhabited area: ] may store it anywhere in his alms village. %
\item[So long as he has a reason for staying apart from that robe: ] if there is a reason, if there is something to be done. %
\item[He should stay apart from that robe for six days at the most: ] he should stay apart from it for six days at a maximum. %
\item[Except if the monks have agreed: ] unless the monks have agreed. %
\item[If he stays apart from it longer than that: ] the robe becomes subject to relinquishment at dawn on the seventh day. %
\end{description}

The\marginnote{2.28} robe should be relinquished to a sangha, a group, or an individual. “And, monks, it should be relinquished like this. (To be expanded as in \href{https://suttacentral.net/pli-tv-bu-vb-np1\#3.2.5}{Bu NP 1:3.2.5}–3.2.29, with appropriate substitutions.) 

‘Venerables,\marginnote{2.31} this robe, which I have stayed apart from for more than six days without the agreement of the monks, is to be relinquished. I relinquish it to the Sangha.’ … the Sangha should give … you should give … ‘I give this robe back to you.’” 

\subsection*{Permutations }

If\marginnote{2.36.1} it is more than six days and he perceives it as more, and he is staying apart from it, then, except if the monks have agreed, he commits an offense entailing relinquishment and confession. If it is more than six days, but he is unsure of it, and he is staying apart from it, then, except if the monks have agreed, he commits an offense entailing relinquishment and confession. If it is more than six days, but he perceives it as less, and he is staying apart from it, then, except if the monks have agreed, he commits an offense entailing relinquishment and confession. 

If\marginnote{2.39} the determination has not been given up, but he perceives that it has … If it has not been given away, but he perceives that it has … If it has not been lost, but he perceives that it has … If it has not been destroyed, but he perceives that it has … If it has not been burned, but he perceives that it has … If it has not been stolen, but he perceives that it has, and he is staying apart from it, then, except if the monks have agreed, he commits an offense entailing relinquishment and confession. 

If\marginnote{2.45} he uses a robe that should be relinquished without first relinquishing it, he commits an offense of wrong conduct. If it is less than six days, but he perceives it as more, he commits an offense of wrong conduct. If it is less than six days, but he is unsure of it, he commits an offense of wrong conduct. If it is less than six days and he perceives it as less, there is no offense. 

\subsection*{Non-offenses }

There\marginnote{2.49.1} is no offense: if he stays apart from the robe for six days; if he stays apart from the robe for less than six days; if, after staying apart from it for six days, he stays overnight within the village zone and then leaves;\footnote{“Zone” renders \textit{\textsanskrit{sīmā}}.  See Appendix of Technical Terms for discussion. } if within the six days he gives up the determination, or the robe has been given away, lost, destroyed, burned, stolen, or taken on trust;\footnote{“Taken on trust” refers to a situation where you have an agreement with a close friend that you may take their belongings on trust. The conditions for taking on trust are set out at \href{https://suttacentral.net/pli-tv-kd8/en/brahmali\#19.1.5}{Kd 8:19.1.5}. } if he has the permission of the monks; if he is insane; if he is the first offender. 

\scendsutta{The training rule on what is risky, the ninth, is finished. }

%
\section*{{\suttatitleacronym Bu Np 30}{\suttatitletranslation 30. The training rule on what was intended }{\suttatitleroot Pariṇata}}
\addcontentsline{toc}{section}{\tocacronym{Bu Np 30} \toctranslation{30. The training rule on what was intended } \tocroot{Pariṇata}}
\markboth{30. The training rule on what was intended }{Pariṇata}
\extramarks{Bu Np 30}{Bu Np 30}

\subsection*{Origin story }

At\marginnote{1.1} one time when the Buddha was staying at \textsanskrit{Sāvatthī} in \textsanskrit{Anāthapiṇḍika}’s Monastery, an association had prepared a meal together with robe-cloth for the Sangha, intending to offer the robe-cloth after giving the meal. 

But\marginnote{1.4} the monks from the group of six went to that association and said, “Please give this robe-cloth to us.” 

“Venerables,\marginnote{1.7} we can’t do that. We’ve prepared our annual alms-offering together with robe-cloth for the Sangha.” 

“The\marginnote{1.9} Sangha has many donors and supporters. But since we’re staying here, we look to you for support. If you don’t give to us, who will? So give us the robe-cloth.” Being pressured by the monks from the group of six, that association gave the prepared robe-cloth to them and served the food to the Sangha. 

The\marginnote{1.14} monks who knew that a meal together with robe-cloth had been prepared for the Sangha, but who did not know that the robe-cloth had been given to the monks from the group of six, said, “Please offer the robe-cloth.” 

“There\marginnote{1.16} isn’t any. The monks from the group of six have diverted to themselves the robe-cloth we had prepared.” 

The\marginnote{1.18} monks of few desires complained and criticized those monks, “How could the monks from the group of six divert to themselves things they knew were intended for the Sangha?” 

After\marginnote{1.20} rebuking those monks in many ways, they told the Buddha. Soon afterwards he had the Sangha gathered and questioned the monks: “Is it true, monks, that you did this?” 

“It’s\marginnote{1.22} true, sir.” 

The\marginnote{1.23} Buddha rebuked them … “Foolish men, how could you do this? This will affect people’s confidence …” … “And, monks, this training rule should be recited like this: 

\subsection*{Final ruling }

\scrule{‘If a monk diverts to himself material support that he knows was intended for the Sangha, he commits an offense entailing relinquishment and confession.’” }

\subsection*{Definitions }

\begin{description}%
\item[A: ] whoever … %
\item[Monk: ] … The monk who has been given the full ordination by a unanimous Sangha through a legal procedure consisting of one motion and three announcements that is irreversible and fit to stand—this sort of monk is meant in this case. %
\item[He knows: ] he knows by himself or others have told him or the donor has told him.\footnote{The meaning of the last of these three ways of knowing, \textit{so \textsanskrit{vā} \textsanskrit{āroceti}}, is not clear. CPD suggests: “\textit{sa (\textsanskrit{sā}) \textsanskrit{āroceti}} (?). Perhaps this last form is conformable to sa. \textit{\textsanskrit{ārocayate}} med. caus. in the meaning: he or she makes inquiries (of others).” However, this does not fit with the parallel usage at \href{https://suttacentral.net/pli-tv-bu-vb-pc29/en/brahmali\#3.1.6}{Bu Pc 29:3.1.6} where the text says that she tells (\textit{\textsanskrit{sā} \textsanskrit{vā} \textsanskrit{āroceti}}) him, presumably referring to the nun telling the monk. In this case \textit{\textsanskrit{āroceti}} cannot refer to the monk making inquiries. The commentaries are silent, and I therefore assume that a straightforward meaning is the most likely one. I would suggest, then, that it simply refers to the donor telling the monk directly. } %
\item[For the Sangha: ] given to the Sangha, given up to the Sangha. %
\item[Material support: ] robe-cloth, almsfood, a dwelling, and medicinal supplies; even a bit of bath powder, a tooth cleaner, or a piece of string. %
\item[Intended: ] they have said, “We’ll give,” “We’ll prepare.” If he diverts it to himself, then for the effort there is an act of wrong conduct. When he gets it, it becomes subject to relinquishment. %
\end{description}

It\marginnote{2.15} should be relinquished to a sangha, a group, or an individual. “And, monks, it should be relinquished like this. (To be expanded as in \href{https://suttacentral.net/pli-tv-bu-vb-np1\#3.2.5}{Bu NP 1:3.2.5}–3.2.29, with appropriate substitutions.) 

‘Venerables,\marginnote{2.18} this thing, which I diverted to myself knowing that it was intended for the Sangha, is to be relinquished. I relinquish it to the Sangha.’ … the Sangha should give … you should give … ‘I give this back to you.’” 

\subsection*{Permutations }

If\marginnote{2.23.1} it was intended for the Sangha and he perceives it as such, and he diverts it to himself, he commits an offense entailing relinquishment and confession. 

If\marginnote{2.24} it was intended for the Sangha, but he is unsure of it, and he diverts it to himself, he commits an offense of wrong conduct. If it was intended for the Sangha, but he does not perceive it as such, and he diverts it to himself, there is no offense. 

If\marginnote{2.26} it was intended for one Sangha and he diverts it to another Sangha or to a shrine, he commits an offense of wrong conduct. If it was intended for one shrine and he diverts it to another shrine or to a sangha or to an individual, he commits an offense of wrong conduct. If it was intended for an individual and he diverts it to another individual or to a sangha or to a shrine, he commits an offense of wrong conduct. 

If\marginnote{2.29} it was not intended for the Sangha, but he perceives it as such, he commits an offense of wrong conduct. If it was not intended for the Sangha, but he is unsure of it, he commits an offense of wrong conduct. If it was not intended for the Sangha and he does not perceive it as such, there is no offense. 

\subsection*{Non-offenses }

There\marginnote{2.32.1} is no offense: if being asked, “Where may we give?” he says, “Give where your gift will be useful;” “Give where it goes toward repairs;” “Give where it will last for a long time;” “Give where you feel inspired;” if he is insane; if he is the first offender. 

\scendsutta{The training rule on what was intended, the tenth, is finished. }

\scendvagga{The third subchapter on almsbowls is finished. }

\scuddanaintro{This is the summary: }

\begin{scuddana}%
“Two\marginnote{2.39} on bowls, and tonics, \\
Rainy season, the fifth on a gift; \\
Oneself, having woven, haste, \\
Risky, and with the Sangha.” 

%
\end{scuddana}

“Venerables,\marginnote{2.43} the thirty rules on relinquishment and confession have been recited. In regard to this I ask you, ‘Are you pure in this?’ A second time I ask, ‘Are you pure in this?’ A third time I ask, ‘Are you pure in this?’ You are pure in this and therefore silent. I’ll remember it thus.” 

\scendkanda{The chapter on offenses entailing relinquishment is finished. }

\scendbook{The canonical text beginning with offenses entailing expulsion is finished. }

%
\addtocontents{toc}{\let\protect\contentsline\protect\nopagecontentsline}
\chapter*{Confession }
\addcontentsline{toc}{chapter}{\tocchapterline{Confession }}
\addtocontents{toc}{\let\protect\contentsline\protect\oldcontentsline}

%
%
\section*{{\suttatitleacronym Bu Pc 1}{\suttatitletranslation 1. The training rule on lying }{\suttatitleroot Musāvāda}}
\addcontentsline{toc}{section}{\tocacronym{Bu Pc 1} \toctranslation{1. The training rule on lying } \tocroot{Musāvāda}}
\markboth{1. The training rule on lying }{Musāvāda}
\extramarks{Bu Pc 1}{Bu Pc 1}

\scnamo{Homage to the Buddha, the Perfected One, the fully Awakened One }

Venerables,\marginnote{1.0.2} these ninety-two rules on confession come up for recitation. 

Hatthaka\marginnote{1.1} the Sakyan was beaten in debate. While talking with the monastics of other religions, he would assert things after denying them, and he would deny things after asserting them. He evaded the issues, lied, and made sham appointments.\footnote{Sp 2.1: \textit{\textsanskrit{Purebhattādīsu}, “\textsanskrit{asukasmiṁ} \textsanskrit{nāma} \textsanskrit{kāle} \textsanskrit{asukasmiṁ} \textsanskrit{nāma} padese \textsanskrit{vādo} \textsanskrit{hotū}”ti \textsanskrit{saṅketaṁ} \textsanskrit{katvā} \textsanskrit{saṅketato} pure \textsanskrit{vā} \textsanskrit{pacchā} \textsanskrit{vā} \textsanskrit{gantvā}, “passatha bho, \textsanskrit{titthiyā} na \textsanskrit{āgatā} \textsanskrit{parājitā}”ti pakkamati}, “Before the meal, etc., he would make an appointment, saying, ‘Let there be a debate at such-and-such a time and at such-and-such a place,’ and he would then go there before or after the time of the appointment, saying, ‘Look, sirs, the ascetics of other religions have not come; they are defeated’, and he would then depart.” } The monastics of other religions complained and criticized him, “When Hatthaka talks with us, how can he assert things after denying them, deny things after asserting them, evade the issues, lie, and make sham appointments?” 

The\marginnote{1.5} monks heard the complaints of those monastics of other religions. They then went to Hatthaka and said, “Is it true, Hatthaka, that you’re doing this?” 

“These\marginnote{1.8} monastics of other religions should be beaten, whatever it takes! They shouldn’t be allowed to win.” 

The\marginnote{1.10} monks of few desires complained and criticized him, “When Hatthaka talks with the monastics of other religions, how can he assert things after denying them, deny things after asserting them, evade the issues, lie, and make sham appointments?” 

After\marginnote{1.12} rebuking Hatthaka in many ways, they told the Buddha. Soon afterwards the Buddha had the Sangha gathered and questioned Hatthaka: “Is it true, Hatthaka, that you’re doing this?” 

“It’s\marginnote{1.15} true, sir.” 

The\marginnote{1.16} Buddha rebuked him … “Foolish man, how can you do this? This will affect people’s confidence …” … “And, monks, this training rule should be recited like this: 

\subsection*{Final ruling }

\scrule{‘If a monk lies in full awareness, he commits an offense entailing confession.’” }

\subsection*{Definitions }

\begin{description}%
\item[Lies in full awareness: ] the speech of one who is aiming to deceive—his words, his way of speaking, his breaking into speech, his verbal expression, his eight kinds of ignoble speech: he says that he has seen what he has not seen; he says that he has heard what he has not heard; he says that he has sensed what he has not sensed; he says that he has mentally experienced what he has not mentally experienced; he says that he has not seen what he has seen; he says that he has not heard what he has heard; he says that he has not sensed what he has sensed; he says that he has not mentally experienced what he has mentally experienced.\footnote{Although \textit{muta} means “thought”, translating it as “sensed” is necessary in light of the definitions below. As for \textit{\textsanskrit{aviññāta}}, “not mentally experienced”, the commentary at Sp 2.3 says: \textit{\textsanskrit{Aññatra} \textsanskrit{pañcahi} indriyehi suddhena \textsanskrit{viññāṇeneva} \textsanskrit{aggahitaṁ} \textsanskrit{aviññātanti} \textsanskrit{veditabbaṁ}}, “\textit{\textsanskrit{Aviññāta}} is to be understood as what is grasped by mere consciousness apart from the five senses.” } %
\end{description}

\subsection*{Permutations }

\paragraph*{Definitions }

\begin{description}%
\item[Not seen: ] not seen with the eye. %
\item[Not heard: ] not heard with the ear. %
\item[Not sensed: ] not smelled with the nose, not tasted with the tongue, not touched with the body. %
\item[Not mentally experienced: ] not mentally experienced with the mind. %
\item[Seen: ] seen with the eye. %
\item[Heard: ] heard with the ear. %
\item[Sensed: ] smelled with the nose, tasted with the tongue, touched with the body. %
\item[Mentally experienced: ] mentally experienced with the mind. %
\end{description}

\paragraph*{Exposition }

\subparagraph*{Falsely claiming to have experienced what he has not experienced: a single sense door }

If\marginnote{2.2.1} he lies in full awareness, saying that he has seen what he has not seen, he commits an offense entailing confession when three conditions are fulfilled: before he has lied, he knows he is going to lie; while lying, he knows he is lying; after he has lied, he knows he has lied. 

If\marginnote{2.2.3} he lies in full awareness, saying that he has seen what he has not seen, he commits an offense entailing confession when four conditions are fulfilled: before he has lied, he knows he is going to lie; while lying, he knows he is lying; after he has lied, he knows he has lied; he misrepresents his view of what is true.\footnote{“Of what is true” is not in the Pali, but has been added for clarity. } 

If\marginnote{2.2.5} he lies in full awareness, saying that he has seen what he has not seen, he commits an offense entailing confession when five conditions are fulfilled: before he has lied, he knows he is going to lie; while lying, he knows he is lying; after he has lied, he knows he has lied; he misrepresents his view of what is true; he misrepresents his belief of what is true. 

If\marginnote{2.2.7} he lies in full awareness, saying that he has seen what he has not seen, he commits an offense entailing confession when six conditions are fulfilled: before he has lied, he knows he is going to lie; while lying, he knows he is lying; after he has lied, he knows he has lied; he misrepresents his view of what is true; he misrepresents his belief of what is true; he misrepresents his acceptance of what is true. 

If\marginnote{2.2.9} he lies in full awareness, saying that he has seen what he has not seen, he commits an offense entailing confession when seven conditions are fulfilled: before he has lied, he knows he is going to lie; while lying, he knows he is lying; after he has lied, he knows he has lied; he misrepresents his view of what is true; he misrepresents his belief of what is true; he misrepresents his acceptance of what is true; he misrepresents his sentiment of what is true. 

If\marginnote{2.2.11} he lies in full awareness, saying that he says that he has heard what he has not heard … saying that he has sensed what he has not sensed … saying that he has mentally experienced what he has not mentally experienced, he commits an offense entailing confession when three conditions are fulfilled: before he has lied, he knows he is going to lie; while lying, he knows he is lying; after he has lied, he knows he has lied. 

…\marginnote{2.2.15} when four conditions are fulfilled … when five conditions are fulfilled … when six conditions are fulfilled … If he lies in full awareness, saying that he has mentally experienced what he has not mentally experienced, he commits an offense entailing confession when seven conditions are fulfilled: before he has lied, he knows he is going to lie; while lying, he knows he is lying; after he has lied, he knows he has lied; he misrepresents his view of what is true; he misrepresents his belief of what is true; he misrepresents his acceptance of what is true; he misrepresents his sentiment of what is true. 

\subparagraph*{Falsely claiming to have experienced what he has not experienced: multiple sense doors }

If\marginnote{2.3.1} he lies in full awareness, saying that he has seen and heard what he has not seen, he commits an offense entailing confession when three conditions are fulfilled … If he lies in full awareness, saying that he has seen and sensed what he has not seen, he commits an offense entailing confession when three conditions are fulfilled … If he lies in full awareness, saying that he has seen and mentally experienced what he has not seen, he commits an offense entailing confession when three conditions are fulfilled … If he lies in full awareness, saying that he has seen and heard and sensed what he has not seen, he commits an offense entailing confession when three conditions are fulfilled … If he lies in full awareness, saying that he has seen and heard and mentally experienced what he has not seen, he commits an offense entailing confession when three conditions are fulfilled … If he lies in full awareness, saying that he has seen and heard and sensed and mentally experienced what he has not seen, he commits an offense entailing confession when three conditions are fulfilled …\footnote{The combination “… saying that he has seen and sensed and mentally experienced what he has not seen …” is missing. The corresponding combination is missing from the next three series of permutations, ending at \href{https://suttacentral.net/pli-tv-bu-vb-pc1/en/brahmali\#2.3.12}{Bu Pc 1:2.3.12}, \href{https://suttacentral.net/pli-tv-bu-vb-pc1/en/brahmali\#2.3.18}{Bu Pc 1:2.3.18}, and \href{https://suttacentral.net/pli-tv-bu-vb-pc1/en/brahmali\#2.3.24}{Bu Pc 1:2.3.24}. } 

If\marginnote{2.3.7} he lies in full awareness, saying that he has heard and sensed what he has not heard, he commits an offense entailing confession when three conditions are fulfilled … If he lies in full awareness, saying that he has heard and mentally experienced what he has not heard, he commits an offense entailing confession when three conditions are fulfilled … If he lies in full awareness, saying that he has heard and seen what he has not heard, he commits an offense entailing confession when three conditions are fulfilled … If he lies in full awareness, saying that he has heard and sensed and mentally experienced what he has not heard, he commits an offense entailing confession when three conditions are fulfilled … If he lies in full awareness, saying that he has heard and sensed and seen what he has not heard, he commits an offense entailing confession when three conditions are fulfilled … If he lies in full awareness, saying that he has heard and sensed and mentally experienced and seen what he has not heard, he commits an offense entailing confession when three conditions are fulfilled … 

If\marginnote{2.3.13} he lies in full awareness, saying that he has sensed and mentally experienced what he has not sensed, he commits an offense entailing confession when three conditions are fulfilled … If he lies in full awareness, saying that he has sensed and seen what he has not sensed, he commits an offense entailing confession when three conditions are fulfilled … If he lies in full awareness, saying that he has sensed and heard what he has not sensed, he commits an offense entailing confession when three conditions are fulfilled … If he lies in full awareness, saying that he has sensed and mentally experienced and seen what he has not sensed, he commits an offense entailing confession when three conditions are fulfilled … If he lies in full awareness, saying that he has sensed and mentally experienced and heard what he has not sensed, he commits an offense entailing confession when three conditions are fulfilled … If he lies in full awareness, saying that he has sensed and mentally experienced and seen and heard what he has not sensed, he commits an offense entailing confession when three conditions are fulfilled … 

If\marginnote{2.3.19} he lies in full awareness, saying that he has mentally experienced and seen what he has not mentally experienced, he commits an offense entailing confession when three conditions are fulfilled … If he lies in full awareness, saying that he has mentally experienced and heard what he has not mentally experienced, he commits an offense entailing confession when three conditions are fulfilled … If he lies in full awareness, saying that he has mentally experienced and sensed what he has not mentally experienced, he commits an offense entailing confession when three conditions are fulfilled … If he lies in full awareness, saying that he has mentally experienced and seen and heard what he has not mentally experienced, he commits an offense entailing confession when three conditions are fulfilled … If he lies in full awareness, saying that he has mentally experienced and seen and sensed what he has not mentally experienced, he commits an offense entailing confession when three conditions are fulfilled … If he lies in full awareness, saying that he has mentally experienced and seen and heard and sensed what he has not mentally experienced, he commits an offense entailing confession when three conditions are fulfilled … 

\subparagraph*{Falsely claiming not to have experienced what he has experienced }

If\marginnote{2.4.1} he lies in full awareness, saying that he has not seen what he has seen, he commits an offense entailing confession when three conditions are fulfilled … saying that he has not heard what he has heard … saying that he has not sensed what he has sensed … saying that he has not mentally experienced what he has mentally experienced, he commits an offense entailing confession when three conditions are fulfilled … 

\subparagraph*{Falsely claiming to have experienced with one sense what he has experienced with another }

If\marginnote{2.5.1} he lies in full awareness, saying that he has heard what he has seen, he commits an offense entailing confession when three conditions are fulfilled … saying that he has sensed what he has seen … saying that he has mentally experienced what he has seen, he commits an offense entailing confession when three conditions are fulfilled … If he lies in full awareness, saying that he has heard and sensed what he has seen, he commits an offense entailing confession when three conditions are fulfilled … saying that he has heard and mentally experienced what he has seen … saying that he has heard and sensed and mentally experienced what he seen, he commits an offense entailing confession when three conditions are fulfilled …\footnote{The combination “… saying that he has sensed and mentally experienced what he has seen …” is missing. The corresponding combination is missing from the next three series of permutations, ending at \href{https://suttacentral.net/pli-tv-bu-vb-pc1/en/brahmali\#2.5.12}{Bu Pc 1:2.5.12}, \href{https://suttacentral.net/pli-tv-bu-vb-pc1/en/brahmali\#2.5.18}{Bu Pc 1:2.5.18}, and \href{https://suttacentral.net/pli-tv-bu-vb-pc1/en/brahmali\#2.5.24}{Bu Pc 1:2.5.24}. } 

If\marginnote{2.5.7} he lies in full awareness, saying that he has sensed what he has heard, he commits an offense entailing confession when three conditions are fulfilled … saying that he has mentally experienced what he has heard … saying that he has seen what he has heard, he commits an offense entailing confession when three conditions are fulfilled … If he lies in full awareness, saying that he has sensed and mentally experienced what he has heard, he commits an offense entailing confession when three conditions are fulfilled … saying that he has sensed and seen what he has heard … saying that he has sensed and mentally experienced and seen what he heard, he commits an offense entailing confession when three conditions are fulfilled … 

If\marginnote{2.5.13} he lies in full awareness, saying that he has mentally experienced what he has sensed, he commits an offense entailing confession when three conditions are fulfilled … saying that he has seen what he has sensed … saying that he has heard what he has sensed, he commits an offense entailing confession when three conditions are fulfilled … If he lies in full awareness, saying that he has mentally experienced and seen what he has sensed, he commits an offense entailing confession when three conditions are fulfilled … saying that he has mentally experienced and heard what he has sensed … saying that he has mentally experienced and seen and heard what he sensed, he commits an offense entailing confession when three conditions are fulfilled … 

If\marginnote{2.5.19} he lies in full awareness, saying that he has seen what he has mentally experienced, he commits an offense entailing confession when three conditions are fulfilled … saying that he has heard what he has mentally experienced … saying that he has sensed what he has mentally experienced, he commits an offense entailing confession when three conditions are fulfilled … If he lies in full awareness, saying that he has seen and heard what he has mentally experienced, he commits an offense entailing confession when three conditions are fulfilled … saying that he has seen and sensed what he has mentally experienced … saying that he has seen and heard and sensed what he mentally experienced, he commits an offense entailing confession when three conditions are fulfilled … 

\subparagraph*{Making claims while having doubts }

If\marginnote{2.6.1} he is unsure of what he has seen, has doubts about what he has seen, does not remember what he has seen, is confused about what he has seen … If he is unsure of what he has heard, has doubts about what he has heard, does not remember what he has heard, is confused about what he has heard … If he is unsure of what he has sensed, has doubts about what he has sensed, does not remember what he has sensed, is confused about what he has sensed … If he is unsure of what he has mentally experienced, has doubts about what he has mentally experienced, does not remember what he has mentally experienced, is confused about what he has mentally experienced, but he lies in full awareness, saying that he has mentally experienced and seen … is confused about what he has mentally experienced, but he lies in full awareness, saying that he has mentally experienced and heard … is confused about what he has mentally experienced, but he lies in full awareness, saying that he has mentally experienced and sensed … is confused about what he has mentally experienced, but he lies in full awareness, saying that he has mentally experienced and seen and heard … is confused about what he has mentally experienced, but he lies in full awareness, saying that he has mentally experienced and seen and sensed … is confused about what he has mentally experienced, but he lies in full awareness, saying that he has mentally experienced and seen and heard and sensed, he commits an offense entailing confession when three conditions are fulfilled …\footnote{The series that ends here is also missing one combination: “… saying that he has mentally experienced and heard and sensed what he has seen …”. } 

when\marginnote{2.6.10} four conditions are fulfilled … when five conditions are fulfilled … when six conditions are fulfilled … is confused about what he has mentally experienced, but he lies in full awareness, saying that he has mentally experienced and seen and heard and sensed, he commits an offense entailing confession when seven conditions are fulfilled: before he has lied, he knows he is going to lie; while lying, he knows he is lying; after he has lied, he knows he has lied; he misrepresents his view of what is true; he misrepresents his belief of what is true; he misrepresents his acceptance of what is true; he misrepresents his sentiment of what is true. 

\subsection*{Non-offenses }

There\marginnote{2.7.1} is no offense: if he speaks playfully; if he speaks too fast; 

\begin{description}%
\item[(Speaks playfully means: ] speaking quickly. %
\item[Speaks too fast means: ] meaning to say one thing, he says something else.) %
\end{description}

if\marginnote{2.7.8} he is insane; if he is the first offender. 

\scendsutta{The training rule on lying, the first, is finished. }

%
\section*{{\suttatitleacronym Bu Pc 2}{\suttatitletranslation 2. The training rule on abusive speech }{\suttatitleroot Omasavāda}}
\addcontentsline{toc}{section}{\tocacronym{Bu Pc 2} \toctranslation{2. The training rule on abusive speech } \tocroot{Omasavāda}}
\markboth{2. The training rule on abusive speech }{Omasavāda}
\extramarks{Bu Pc 2}{Bu Pc 2}

\subsection*{Origin story }

At\marginnote{1.1.1} one time when the Buddha was staying at \textsanskrit{Sāvatthī} in \textsanskrit{Anāthapiṇḍika}’s Monastery, the monks from the group of six were arguing with and speaking abusively to the good monks. They reviled and insulted them about their caste, name, family, occupation, profession, illnesses, physical traits, defilements, and offenses, and by calling them names. The monks of few desires complained and criticized them, “How can the monks from the group of six argue with and abuse good monks? How can they revile and insult them about these things?” 

After\marginnote{1.1.7} rebuking those monks in many ways, they told the Buddha. Soon afterwards he had the Sangha gathered and questioned the monks: “Is it true, monks, that you’re doing this?” 

“It’s\marginnote{1.1.9} true, sir.” 

The\marginnote{1.1.10} Buddha rebuked them … “Foolish men, how can you do this? This will affect people’s confidence …” After rebuking them … he gave a teaching and addressed the monks: 

\subparagraph*{Jataka }

“Once\marginnote{1.2.1} upon a time, monks, there was a brahmin in \textsanskrit{Takkasilā} who had an ox called \textsanskrit{Nandivisāla}. On one occasion the ox said to that brahmin, ‘Go, brahmin, and bet a thousand coins with the wealthy merchant that your ox will pull one hundred carts tied together.’ And that brahmin did just that. Then, after tying one hundred carts together and yoking \textsanskrit{Nandivisāla} to them, he said, ‘Go, you fraud! Pull, you liar!’ But \textsanskrit{Nandivisāla} didn’t budge. 

Then\marginnote{1.2.10} that brahmin became depressed because he had lost a thousand coins. \textsanskrit{Nandivisāla} said to him, ‘Why are you depressed?’ 

‘Because\marginnote{1.2.13} I lost one thousand coins because of you.’ 

‘But\marginnote{1.2.14} why did you disgrace me by calling me a fraud when I’m not? Now go, brahmin, and make the same bet with that merchant, but increase the stakes to two thousand coins. Just don’t disgrace me by calling me a fraud.’ Once again that brahmin did just that. Then, after tying one hundred carts together and yoking \textsanskrit{Nandivisāla} to them, he said, ‘Go, good ox! Pull, good ox!’ And \textsanskrit{Nandivisāla} pulled those one hundred carts. 

\begin{verse}%
‘One\marginnote{1.2.23} should say what’s pleasant, \\
Never what’s unpleasant. \\
Because of his pleasant speech, \\
Heavy loads were pulled, \\
And he obtained wealth; \\
And he was delighted with that.’ 

%
\end{verse}

Even\marginnote{1.2.29} at that time, monks, reviling and insulting was unpleasant to me. How then could reviling and insulting be pleasant now? This will affect people’s confidence …” … “And, monks, this training rule should be recited like this: 

\subsection*{Final ruling }

\scrule{‘If a monk speaks abusively, he commits an offense entailing confession.’” }

\subsection*{Definitions }

\begin{description}%
\item[Speaks abusively: ] he speaks abusively in ten ways: about caste, about name, about family, about occupation, about profession, about illnesses, about physical traits, about defilements, about offenses, and by name-calling. %
\end{description}

\subsection*{Permutations }

\paragraph*{Definitions }

\begin{description}%
\item[Caste: ] there are two kinds of castes: low castes and high castes. %
\item[Low castes: ] outcasts, bamboo workers, hunters, carriage makers, waste removers—these are called “low castes”. %
\item[High castes: ] aristocrats and brahmins—these are called “high castes”. %
\item[Name: ] there are two kinds of names: low names and high names. %
\item[Low names: ] \textsanskrit{Avakaṇṇaka}, \textsanskrit{Javakaṇṇaka}, \textsanskrit{Dhaniṭṭhaka}, \textsanskrit{Saviṭṭhaka}, \textsanskrit{Kulavaḍḍhaka}, or names, in whatever countries, that are despised, looked down upon, scorned, treated with contempt, disregarded—these are called “low names”. %
\item[High names: ] those connected with the Buddha, connected with the Teaching, or connected with the Sangha, or names, in whatever countries, that are highly regarded, highly thought of, respected, valued, esteemed—these are called “high names”. %
\item[Family: ] there are two kinds of families: low families and high families. %
\item[Low families: ] the Kosiya family, the \textsanskrit{Bhāradvāja} family, or families, in whatever countries, that are despised, looked down upon, scorned, treated with contempt, disregarded—these are called “low families”. %
\item[High families: ] the Gotama family, the \textsanskrit{Moggallāna} family, the \textsanskrit{Kaccāna} family, the \textsanskrit{Vāsiṭṭha} family, or families, in whatever countries, that are highly regarded, highly thought of, respected, valued, esteemed—these are called “high families”. %
\item[Occupation: ] there are two kinds of occupations: low occupations and high occupations. %
\item[Low occupations: ] carpentry, waste removing, or occupations, in whatever countries, that are despised, looked down upon, scorned, treated with contempt, disregarded—these are called “low occupations”. %
\item[High occupations: ] farming, trade, cattle keeping, or occupations, in whatever countries, that are highly regarded, highly thought of, respected, valued, esteemed—these are called “high occupations”. %
\item[Profession: ] there are two kinds of professions: low professions and high professions. %
\item[Low professions: ] reed work, pottery, weaving, leather work, hairdressing, or professions, in whatever countries, that are despised, looked down upon, scorned, treated with contempt, disregarded—these are called “low professions”. %
\item[High professions: ] arithmetic, accounting, writing, or professions, in whatever countries, that are highly regarded, highly thought of, respected, valued, esteemed—these are called “high professions”. %
\item[Illnesses: ] all are low, but there is the illness of diabetes which is high. %
\item[Physical traits: ] there are two kinds of physical traits: low physical traits and high physical traits. %
\item[Low physical traits: ] too tall, too short, too dark, too fair—these are called “low physical traits”. %
\item[High physical traits: ] not too tall, not too short, not too dark, not too fair—these are called “high physical traits”. %
\item[Defilements: ] all are low. %
\item[Offenses: ] all are low, but there is the attainment of stream-entry which is high.\footnote{The English translation cannot properly capture the Pali, in which the word for “offense” and “attainment” is the same, \textit{\textsanskrit{āpatti}}. } %
\item[Name-calling: ] there are two kinds of name-calling: low name-calling and high name-calling. %
\item[Low name-calling: ] “You’re a camel,” “You’re a ram,” “You’re an ox,” “You’re a donkey,” “You’re an animal,” “You’re bound for hell,” “You’re not going to a good destination,” “You can only expect a bad destination,” or adding disparaging endings to someone’s name, or calling someone words for the male or female genitals—this is called “low name-calling”. %
\item[High name-calling: ] “You’re wise,” “You’re competent,” “You’re intelligent,” “You’re learned,” “You’re an expounder of the Teaching,” “You’re not going to a bad destination,” “You can only expect a good destination”—this is called “high name-calling”. %
\end{description}

\paragraph*{Exposition }

\subparagraph*{Abuse about caste }

If\marginnote{2.2.1} one who is fully ordained, wishing to revile, wishing to insult, wishing to humiliate another who is fully ordained, says what is low to one who is low—an outcast, a bamboo worker, a hunter, a carriage maker, a waste remover—saying, “You’re an outcast,” “You’re a bamboo worker,” “You’re a hunter,” “You’re a carriage maker,” “You’re a waste remover,” then for every statement, he commits an offense entailing confession. 

If\marginnote{2.2.2} one who is fully ordained, wishing to revile, wishing to insult, wishing to humiliate another who is fully ordained, says what is low to one who is high—an aristocrat, a brahmin—saying, “You’re an outcast,” “You’re a bamboo worker,” “You’re a hunter,” “You’re a carriage maker,” “You’re a waste remover,” then for every statement, he commits an offense entailing confession. 

If\marginnote{2.2.3} one who is fully ordained, wishing to revile, wishing to insult, wishing to humiliate another who is fully ordained, says what is high to one who is low—an outcast, a bamboo worker, a hunter, a carriage maker, a waste remover—saying, “You’re an aristocrat,” “You’re a brahmin,” then for every statement, he commits an offense entailing confession. 

If\marginnote{2.2.4} one who is fully ordained, wishing to revile, wishing to insult, wishing to humiliate another who is fully ordained, says what is high to one who is high—an aristocrat, a brahmin—saying, “You’re an aristocrat,” “You’re a brahmin,” then for every statement, he commits an offense entailing confession. 

\subparagraph*{Abuse about name }

If\marginnote{2.2.5.1} one who is fully ordained, wishing to revile, wishing to insult, wishing to humiliate another who is fully ordained, says what is low to one who is low—an \textsanskrit{Avakaṇṇaka}, a \textsanskrit{Javakaṇṇaka}, a \textsanskrit{Dhaniṭṭhaka}, a \textsanskrit{Saviṭṭhaka}, a \textsanskrit{Kulavaḍḍhaka}—saying, “You’re an \textsanskrit{Avakaṇṇaka},” “You’re a \textsanskrit{Javakaṇṇaka},” “You’re a \textsanskrit{Dhaniṭṭhaka},” “You’re a \textsanskrit{Saviṭṭhaka},” “You’re a \textsanskrit{Kulavaḍḍhaka},” then for every statement, he commits an offense entailing confession. 

If\marginnote{2.2.7} one who is fully ordained, wishing to revile, wishing to insult, wishing to humiliate another who is fully ordained, says what is low to one who is high—a Buddharakkhita, a Dhammarakkhita, a Sangharakkhita—saying, “You’re an \textsanskrit{Avakaṇṇaka},” “You’re a \textsanskrit{Javakaṇṇaka},” “You’re a \textsanskrit{Dhaniṭṭhaka},” “You’re a \textsanskrit{Saviṭṭhaka},” “You’re a \textsanskrit{Kulavaḍḍhaka},” then for every statement, he commits an offense entailing confession. 

If\marginnote{2.2.8} one who is fully ordained, wishing to revile, wishing to insult, wishing to humiliate another who is fully ordained, says what is high to one who is low—an \textsanskrit{Avakaṇṇaka}, a \textsanskrit{Javakaṇṇaka}, a \textsanskrit{Dhaniṭṭhaka}, a \textsanskrit{Saviṭṭhaka}, a \textsanskrit{Kulavaḍḍhaka}—saying, “You’re a Buddharakkhita,” “You’re a Dhammarakkhita,” “You’re a Sangharakkhita,” then for every statement, he commits an offense entailing confession. 

If\marginnote{2.2.9} one who is fully ordained, wishing to revile, wishing to insult, wishing to humiliate another who is fully ordained, says what is high to one who is high—a Buddharakkhita, a Dhammarakkhita, a Sangharakkhita—saying, “You’re a Buddharakkhita,” “You’re a Dhammarakkhita,” “You’re a Sangharakkhita,” then for every statement, he commits an offense entailing confession. 

\subparagraph*{Abuse about family }

If\marginnote{2.2.10.1} one who is fully ordained, wishing to revile, wishing to insult, wishing to humiliate another who is fully ordained, says what is low to one who is low—one from the Kosiya family, one from the \textsanskrit{Bhāradvāja} family—saying, “You’re a Kosiya,” “You’re a \textsanskrit{Bhāradvāja},” then for every statement, he commits an offense entailing confession. 

If\marginnote{2.2.11} one who is fully ordained, wishing to revile, wishing to insult, wishing to humiliate another who is fully ordained, says what is low to one who is high—one from the Gotama family, one from the \textsanskrit{Moggallāna} family, one from the \textsanskrit{Kaccāna} family, one from the \textsanskrit{Vāsiṭṭha} family—saying, “You’re a Kosiya,” “You’re a \textsanskrit{Bhāradvāja},” then for every statement, he commits an offense entailing confession. 

If\marginnote{2.2.12} one who is fully ordained, wishing to revile, wishing to insult, wishing to humiliate another who is fully ordained, says what is high to one who is low—one from the Kosiya family, one from the \textsanskrit{Bhāradvāja} family—saying, “You’re a Gotama,” “You’re a \textsanskrit{Moggallāna},” “You’re a \textsanskrit{Kaccāna},” “You’re a \textsanskrit{Vāsiṭṭha},” then for every statement, he commits an offense entailing confession. 

If\marginnote{2.2.13} one who is fully ordained, wishing to revile, wishing to insult, wishing to humiliate another who is fully ordained, says what is high to one who is high—one from the Gotama family, one from the \textsanskrit{Moggallāna} family, one from the \textsanskrit{Kaccāna} family, one from the \textsanskrit{Vāsiṭṭha} family—saying, “You’re a Gotama,” “You’re a \textsanskrit{Moggallāna},” “You’re a \textsanskrit{Kaccāna},” “You’re a \textsanskrit{Vāsiṭṭha},” then for every statement, he commits an offense entailing confession. 

\subparagraph*{Abuse about occupation }

If\marginnote{2.2.14.1} one who is fully ordained, wishing to revile, wishing to insult, wishing to humiliate another who is fully ordained, says what is low to one who is low—a carpenter, a waste remover—saying, “You’re a carpenter,” “You’re a waste remover,” then for every statement, he commits an offense entailing confession. 

If\marginnote{2.2.15} one who is fully ordained, wishing to revile, wishing to insult, wishing to humiliate another who is fully ordained, says what is low to one who is high—a farmer, a trader, a cattle keeper—saying, “You’re a carpenter,” “You’re a waste remover,” then for every statement, he commits an offense entailing confession. 

If\marginnote{2.2.17} one who is fully ordained, wishing to revile, wishing to insult, wishing to humiliate another who is fully ordained, says what is high to one who is low—a carpenter, a waste remover—saying, “You’re a farmer,” “You’re a trader,” “You’re a cattle keeper,” then for every statement, he commits an offense entailing confession. 

If\marginnote{2.2.18} one who is fully ordained, wishing to revile, wishing to insult, wishing to humiliate another who is fully ordained, says what is high to one who is high—a farmer, a trader, a cattle keeper—saying, “You’re a farmer,” “You’re a trader,” “You’re a cattle keeper,” then for every statement, he commits an offense entailing confession. 

\subparagraph*{Abuse about profession }

If\marginnote{2.2.19.1} one who is fully ordained, wishing to revile, wishing to insult, wishing to humiliate another who is fully ordained, says what is low to one who is low—a reed worker, a potter, a weaver, a leather worker, a barber—saying, “You’re a reed worker,” “You’re a potter,” “You’re a weaver,” “You’re a leather worker,” “You’re a barber,” then for every statement, he commits an offense entailing confession. 

If\marginnote{2.2.20} one who is fully ordained, wishing to revile, wishing to insult, wishing to humiliate another who is fully ordained, says what is low to one who is high—an arithmetician, an accountant, a clerk—saying, “You’re a reed worker,” “You’re a potter,” “You’re a weaver,” “You’re a leather worker,” “You’re a barber,” then for every statement, he commits an offense entailing confession. 

If\marginnote{2.2.21} one who is fully ordained, wishing to revile, wishing to insult, wishing to humiliate another who is fully ordained, says what is high to one who is low—a reed worker, a potter, a weaver, a leather worker, a barber—saying, “You’re an arithmetician,” “You’re an accountant,” “You’re a clerk,” then for every statement, he commits an offense entailing confession. 

If\marginnote{2.2.22} one who is fully ordained, wishing to revile, wishing to insult, wishing to humiliate another who is fully ordained, says what is high to one who is high—an arithmetician, an accountant, a clerk—saying, “You’re an arithmetician,” “You’re an accountant,” “You’re a clerk,” then for every statement, he commits an offense entailing confession. 

\subparagraph*{Abuse about illnesses }

If\marginnote{2.2.23.1} one who is fully ordained, wishing to revile, wishing to insult, wishing to humiliate another who is fully ordained, says what is low to one who is low—a leper, one with abscesses, one with mild leprosy, one with tuberculosis, an epileptic—saying, “You’re a leper,” “You have abscesses,” “You have mild leprosy,” “You have tuberculosis,” “You’re an epileptic,” then for every statement, he commits an offense entailing confession.\footnote{For an explanation of these, see Appendix of Medical Terminology. } 

If\marginnote{2.2.24} one who is fully ordained, wishing to revile, wishing to insult, wishing to humiliate another who is fully ordained, says what is low to one who is high—a diabetic—saying, “You’re a leper,” “You have abscesses,” “You have mild leprosy,” “You have tuberculosis,” “You’re an epileptic,” then for every statement, he commits an offense entailing confession. 

If\marginnote{2.2.25} one who is fully ordained, wishing to revile, wishing to insult, wishing to humiliate another who is fully ordained, says what is high to one who is low—a leper, one with abscesses, one with mild leprosy, one with tuberculosis, an epileptic—saying, “You’re a diabetic,” then for every statement, he commits an offense entailing confession. 

If\marginnote{2.2.26} one who is fully ordained, wishing to revile, wishing to insult, wishing to humiliate another who is fully ordained, says what is high to one who is high—a diabetic—saying, “You’re a diabetic,” then for every statement, he commits an offense entailing confession. 

\subparagraph*{Abuse about physical traits }

If\marginnote{2.2.27.1} one who is fully ordained, wishing to revile, wishing to insult, wishing to humiliate another who is fully ordained, says what is low to one who is low—one who is too tall, one who is too short, one who is too dark, one who is too fair—saying, “You’re too tall,” “You’re too short,” “You’re too dark,” “You’re too fair,” then for every statement, he commits an offense entailing confession. 

If\marginnote{2.2.28} one who is fully ordained, wishing to revile, wishing to insult, wishing to humiliate another who is fully ordained, says what is low to one who is high—one who is not too tall, one who is not too short, one who is not too dark, one who is not too fair—saying, “You’re too tall,” “You’re too short,” “You’re too dark,” “You’re too fair,” then for every statement, he commits an offense entailing confession. 

If\marginnote{2.2.29} one who is fully ordained, wishing to revile, wishing to insult, wishing to humiliate another who is fully ordained, says what is high to one who is low—one who is too tall, one who is too short, one who is too dark, one who is too fair—saying, “You’re not too tall,” “You’re not too short,” “You’re not too dark,” “You’re not too fair,” then for every statement, he commits an offense entailing confession. 

If\marginnote{2.2.30} one who is fully ordained, wishing to revile, wishing to insult, wishing to humiliate another who is fully ordained, says what is high to one who is high—one who is not too tall, one who is not too short, one who is not too dark, one who is not too fair—saying, “You’re not too tall,” “You’re not too short,” “You’re not too dark,” “You’re not too fair,” then for every statement, he commits an offense entailing confession. 

\subparagraph*{Abuse about defilements }

If\marginnote{2.2.31.1} one who is fully ordained, wishing to revile, wishing to insult, wishing to humiliate another who is fully ordained, says what is low to one who is low—one full of sensual desire, one full of ill will, one full of confusion—saying, “You’re full of sensual desire,” “You’re full of ill will,” “You’re full of confusion,” then for every statement, he commits an offense entailing confession. 

If\marginnote{2.2.32} one who is fully ordained, wishing to revile, wishing to insult, wishing to humiliate another who is fully ordained, says what is low to one who is high—one without sensual desire, one without ill will, one without confusion—saying, “You’re full of sensual desire,” “You’re full of ill will,” “You’re full of confusion,” then for every statement, he commits an offense entailing confession. 

If\marginnote{2.2.33} one who is fully ordained, wishing to revile, wishing to insult, wishing to humiliate another who is fully ordained, says what is high to one who is low—one full of sensual desire, one full of ill will, one full of confusion—saying, “You’re without sensual desire,” “You’re without ill will,” “You’re without confusion,” then for every statement, he commits an offense entailing confession. 

If\marginnote{2.2.34} one who is fully ordained, wishing to revile, wishing to insult, wishing to humiliate another who is fully ordained, says what is high to one who is high—one without sensual desire, one without ill will, one without confusion—saying, “You’re without sensual desire,” “You’re without ill will,” “You’re without confusion,” then for every statement, he commits an offense entailing confession. 

\subparagraph*{Abuse about offenses }

If\marginnote{2.2.35.1} one who is fully ordained, wishing to revile, wishing to insult, wishing to humiliate another who is fully ordained, says what is low to one who is low—one who has committed an offense entailing expulsion, one who has committed an offense entailing suspension, one who has committed a serious offense, one who has committed an offense entailing confession, one who has committed an offense entailing acknowledgment, one who has committed an offense of wrong conduct, one who has committed an offense of wrong speech—saying, “You’ve committed an offense entailing expulsion,” “You’ve committed an offense entailing suspension,” “You’ve committed a serious offense,” “You’ve committed an offense entailing confession,” “You’ve committed an offense entailing acknowledgment,” “You’ve committed an offense of wrong conduct,” “You’ve committed an offense of wrong speech,” then for every statement, he commits an offense entailing confession. 

If\marginnote{2.2.36} one who is fully ordained, wishing to revile, wishing to insult, wishing to humiliate another who is fully ordained, says what is low to one who is high—a stream-enterer—saying, “You’ve committed an offense entailing expulsion,” “You’ve committed an offense entailing suspension,” “You’ve committed a serious offense,” “You’ve committed an offense entailing confession,” “You’ve committed an offense entailing acknowledgment,” “You’ve committed an offense of wrong conduct,” “You’ve committed an offense of wrong speech,” then for every statement, he commits an offense entailing confession. 

If\marginnote{2.2.37} one who is fully ordained, wishing to revile, wishing to insult, wishing to humiliate another who is fully ordained, says what is high to one who is low—one who has committed an offense entailing expulsion, one who has committed an offense entailing suspension, one who has committed a serious offense, one who has committed an offense entailing confession, one who has committed an offense entailing acknowledgment, one who has committed an offense of wrong conduct, one who has committed an offense of wrong speech—saying, “You’re a stream-enterer,” then for every statement, he commits an offense entailing confession. 

If\marginnote{2.2.38} one who is fully ordained, wishing to revile, wishing to insult, wishing to humiliate another who is fully ordained, says what is high to one who is high—a stream-enterer—saying, “You’re a stream-enterer,” then for every statement, he commits an offense entailing confession. 

\subparagraph*{Insulting abuse }

If\marginnote{2.2.39.1} one who is fully ordained, wishing to revile, wishing to insult, wishing to humiliate another who is fully ordained, says what is low to one who is low—a camel, a ram, an ox, a donkey, an animal, one bound for hell—saying, “You’re a camel,” “You’re a ram,” “You’re an ox,” “You’re a donkey,” “You’re an animal,” “You’re bound for hell,” “You’re not going to a good destination,” “You can only expect a bad destination,” then for every statement, he commits an offense entailing confession. 

If\marginnote{2.2.40} one who is fully ordained, wishing to revile, wishing to insult, wishing to humiliate another who is fully ordained, says what is low to one who is high—one who is wise, one who is competent, one who is intelligent, one who is learned, one who is an expounder of the Teaching—saying, “You’re a camel,” “You’re a ram,” “You’re an ox,” “You’re a donkey,” “You’re an animal,” “You’re bound for hell,” “You’re not going to a good destination,” “You can only expect a bad destination,” then for every statement, he commits an offense entailing confession. 

If\marginnote{2.2.41} one who is fully ordained, wishing to revile, wishing to insult, wishing to humiliate another who is fully ordained, says what is high to one who is low—a camel, a ram, an ox, a donkey, an animal, one bound for hell—saying, “You’re wise,” “You’re competent,” “You’re intelligent,” “You’re learned,” “You’re an expounder of the Teaching,” “You’re not going to a bad destination,” “You can only expect a good destination,” then for every statement, he commits an offense entailing confession. 

If\marginnote{2.2.42} one who is fully ordained, wishing to revile, wishing to insult, wishing to humiliate another who is fully ordained, says what is high to one who is high—one who is wise, one who is competent, one who is intelligent, one who is learned, one who is an expounder of the Teaching—saying, “You’re wise,” “You’re competent,” “You’re intelligent,” “You’re learned,” “You’re an expounder of the Teaching,” “You’re not going to a bad destination,” “You can only expect a good destination,” then for every statement, he commits an offense entailing confession. 

\subparagraph*{Indirect abuse }

If\marginnote{2.3.1} one who is fully ordained, wishing to revile, wishing to insult, wishing to humiliate another who is fully ordained, says, “There are outcasts right here,” “There are bamboo workers right here,” “There are hunters right here,” “There are carriage makers right here,” “There are waste removers right here,” then for every statement, he commits an offense of wrong conduct. 

If\marginnote{2.3.2} one who is fully ordained, wishing to revile, wishing to insult, wishing to humiliate another who is fully ordained, says, “There are aristocrats right here,” “There are brahmins right here,” then for every statement, he commits an offense of wrong conduct. 

If\marginnote{2.3.3} one who is fully ordained, wishing to revile, wishing to insult, wishing to humiliate another who is fully ordained, says, “There are \textsanskrit{Avakaṇṇakas} right here,” “There are \textsanskrit{Javakaṇṇakas} right here,” “There are \textsanskrit{Dhaniṭṭhakas} right here,” “There are \textsanskrit{Saviṭṭhakas} right here,” “There are \textsanskrit{Kulavaḍḍhakas} right here,” … says, “There are Buddharakkhitas right here,” “There are Dhammarakkhitas right here,” “There are Sangharakkhitas right here,” … says, “There are Kosiyas right here,” “There are \textsanskrit{Bhāradvājas} right here,” … says, “There are Gotamas right here,” “There are \textsanskrit{Moggallānas} right here,” “There are \textsanskrit{Kaccānas} right here,” “There are \textsanskrit{Vāsiṭṭhas} right here,” … says, “There are carpenters right here,” “There are waste removers right here,” … says, “There are farmers right here,” “There are traders right here,” “There are cattle keepers right here,” … says, “There are reed workers right here,” “There are potters right here,” “There are weavers right here,” “There are leather workers right here,” “There are barbers right here,” … says, “There are arithmeticians right here,” “There are accountants right here,” “There are clerks right here,” … says, “There are lepers right here,” “There are some with abscesses right here,” “There are some with mild leprosy right here,” “There are some with tuberculosis right here,” “There are epileptics right here,” … says, “There are diabetics right here,” … says, “There are some who are too tall right here,” “There are some who are too short right here,” “There are some who are too dark right here,” “There are some who are too fair right here,” … says, “There are some who are not too tall right here,” “There are some who are not too short right here,” “There are some who are not too dark right here,” “There are some who are not too fair right here,” … says, “There are some who are full of sensual desire right here,” “There are some who are full of ill will right here,” “There are some who are full of confusion right here,” … says, “There are some without sensual desire right here,” “There are some without ill will right here,” “There are some without confusion right here,” … says, “There are some who have committed an offense entailing expulsion right here … etc. … some who have committed an offense of wrong speech right here,” … says, “There are stream-enterers right here,” … says, “There are camels right here,” “There are rams right here,” “There are oxen right here,” “There are donkeys right here,” “There are animals right here,” “There are those bound for hell right here,” “There are those not going to a good destination right here,” “There are those who can only expect a bad destination right here,” then for every statement, he commits an offense of wrong conduct. 

If\marginnote{2.3.21} one who is fully ordained, wishing to revile, wishing to insult, wishing to humiliate another who is fully ordained, says, “There are wise ones right here,” “There are competent ones right here,” “There are intelligent ones right here,” “There are learned ones right here,” “There are expounders of the Teaching right here,” “There are those not going to a bad destination right here,” “There are those who can only expect a good destination right here,” then for every statement, he commits an offense of wrong conduct. 

If\marginnote{2.4.1} one who is fully ordained, wishing to revile, wishing to insult, wishing to humiliate another who is fully ordained, says, “Perhaps these are outcasts,” “Perhaps these are bamboo workers,” “Perhaps these are hunters,” “Perhaps these are carriage makers,” “Perhaps these are waste removers,” then for every statement, he commits an offense of wrong conduct. … 

If\marginnote{2.4.2} one who is fully ordained, wishing to revile, wishing to insult, wishing to humiliate another who is fully ordained, says, “Perhaps these are wise ones,” “Perhaps these are competent ones,” “Perhaps these are intelligent ones,” “Perhaps these are learned ones,” “Perhaps these are expounders of the Teaching,” then for every statement, he commits an offense of wrong conduct. 

If\marginnote{2.5.1} one who is fully ordained, wishing to revile, wishing to insult, wishing to humiliate another who is fully ordained, says, “We’re not outcasts,” “We’re not bamboo workers,” “We’re not hunters,” “We’re not carriage makers,” “We’re not waste removers,” … “We’re not wise ones,” “We’re not competent ones,” “We’re not intelligent ones,” “We’re not learned ones,” “We’re not expounders of the Teaching,” “We’re not going to a bad destination,” “We can only expect a good destination,” then for every statement, he commits an offense of wrong conduct.\footnote{This looks like an editing mistake. Presumably this should read, “We’re not going to a good destination/we can only expect a bad destination.” And below \href{https://suttacentral.net/pli-tv-bu-vb-pc2/en/brahmali\#2.6.10}{Bu Pc 2:2.6.10}, \href{https://suttacentral.net/pli-tv-bu-vb-pc2/en/brahmali\#2.7.15}{Bu Pc 2:2.7.15}, and \href{https://suttacentral.net/pli-tv-bu-vb-pc2/en/brahmali\#2.8.11}{Bu Pc 2:2.8.11}. } 

\subparagraph*{Abuse of one who is not fully ordained }

If\marginnote{2.6.1} one who is fully ordained, wishing to revile, wishing to insult, wishing to humiliate someone who is not fully ordained, says what is low to one who is low … says what is low to one who is high … says what is high to one who is low … says what is high to one who is high—one who is wise, one who is competent, one who is intelligent, one who is learned, one who is an expounder of the Teaching—saying, “You’re wise,” “You’re competent,” “You’re intelligent,” “You’re learned,” “You’re an expounder of the Teaching,” “You’re not going to a bad destination,” “You can only expect a good destination,” then for every statement, he commits an offense of wrong conduct. 

If\marginnote{2.6.5} one who is fully ordained, wishing to revile, wishing to insult, wishing to humiliate someone who is not fully ordained, says, “There are outcasts right here,” “There are bamboo workers right here,” “There are hunters right here,” “There are carriage makers right here,” “There are waste removers right here,” … “There are wise ones right here,” “There are competent ones right here,” “There are intelligent ones right here,” “There are learned ones right here,” “There are expounders of the Teaching right here,” “There are those not going to a bad destination right here,” “There are those who can only expect a good destination right here,” then for every statement, he commits an offense of wrong conduct. 

If\marginnote{2.6.7} one who is fully ordained, wishing to revile, wishing to insult, wishing to humiliate someone who is not fully ordained, says, “Perhaps these are outcasts,” “Perhaps these are bamboo workers,” “Perhaps these are hunters,” “Perhaps these are carriage makers,” “Perhaps these are waste removers,” … “Perhaps these are wise ones,” “Perhaps these are competent ones,” “Perhaps these are intelligent ones,” “Perhaps these are learned ones,” “Perhaps these are expounders of the Teaching,” then for every statement, he commits an offense of wrong conduct. 

If\marginnote{2.6.9} one who is fully ordained, wishing to revile, wishing to insult, wishing to humiliate someone who is not fully ordained, says, “We’re not outcasts,” “We’re not bamboo workers,” “We’re not hunters,” “We’re not carriage makers,” “We’re not waste removers,” … “We’re not wise ones,” “We’re not competent ones,” “We’re not intelligent ones,” “We’re not learned ones,” “We’re not expounders of the Teaching,” “We’re not going to a bad destination,” “We can only expect a good destination,” then for every statement, he commits an offense of wrong conduct. 

\subparagraph*{Not intending to abuse, direct speech }

If\marginnote{2.7.1} one who is fully ordained, not wishing to revile, not wishing to insult, not wishing to humiliate another who is fully ordained, but wanting to have fun, says what is low to one who is low—an outcast, a bamboo worker, a hunter, a carriage maker, a waste remover—saying, “You’re an outcast,” “You’re a bamboo worker,” “You’re a hunter,” “You’re a carriage maker,” “You’re a waste remover,” then for every statement, he commits an offense of wrong speech. 

If\marginnote{2.7.2} one who is fully ordained, not wishing to revile, not wishing to insult, not wishing to humiliate another who is fully ordained, but wanting to have fun, says what is low to one who is high—an aristocrat, a brahmin—saying, “You’re an outcast,” “You’re a bamboo worker,” “You’re a hunter,” “You’re a carriage maker,” “You’re a waste remover,” then for every statement, he commits an offense of wrong speech. 

If\marginnote{2.7.3} one who is fully ordained, not wishing to revile, not wishing to insult, not wishing to humiliate another who is fully ordained, but wanting to have fun, says what is high to one who is low—an outcast, a bamboo worker, a hunter, a carriage maker, a waste remover—saying, “You’re an aristocrat,” “You’re a brahmin,” then for every statement, he commits an offense of wrong speech. 

If\marginnote{2.7.4} one who is fully ordained, not wishing to revile, not wishing to insult, not wishing to humiliate another who is fully ordained, but wanting to have fun, says what is high to one who is high—an aristocrat, a brahmin—saying, “You’re an aristocrat,” “You’re a brahmin,” then for every statement, he commits an offense of wrong speech. 

If\marginnote{2.7.5} one who is fully ordained, not wishing to revile, not wishing to insult, not wishing to humiliate another who is fully ordained, but wanting to have fun, says what is low to one who is low … says what is low to one who is high … says what is high to one who is low … says what is high to one who is high—one who is wise, one who is competent, one who is intelligent, one who is learned, one who is an expounder of the Teaching—saying, “You’re wise,” “You’re competent,” “You’re intelligent,” “You’re learned,” “You’re an expounder of the Teaching,” “You’re not going to a bad destination,” “You can only expect a good destination,” then for every statement, he commits an offense of wrong speech. 

\subparagraph*{Not intending to abuse, indirect speech }

If\marginnote{2.7.10.1} one who is fully ordained, not wishing to revile, not wishing to insult, not wishing to humiliate another who is fully ordained, but wanting to have fun, says, “There are outcasts right here,” “There are bamboo workers right here,” “There are hunters right here,” “There are carriage makers right here,” “There are waste removers right here,” … “There are wise ones right here,” “There are competent ones right here,” “There are intelligent ones right here,” “There are learned ones right here,” “There are expounders of the Teaching right here,” “There are those not going to a bad destination right here,” “There are those who can only expect a good destination right here,” then for every statement, he commits an offense of wrong speech. 

If\marginnote{2.7.12} one who is fully ordained, not wishing to revile, not wishing to insult, not wishing to humiliate another who is fully ordained, but wanting to have fun, says, “Perhaps these are outcasts,” “Perhaps these are bamboo workers,” “Perhaps these are hunters,” “Perhaps these are carriage makers,” “Perhaps these are waste removers,” … “Perhaps these are wise ones,” “Perhaps these are competent ones,” “Perhaps these are intelligent ones,” “Perhaps these are learned ones,” “Perhaps these are expounders of the Teaching,” then for every statement, he commits an offense of wrong speech. 

If\marginnote{2.7.14} one who is fully ordained, not wishing to revile, not wishing to insult, not wishing to humiliate another who is fully ordained, but wanting to have fun, says, “We’re not outcasts,” “We’re not bamboo workers,” “We’re not hunters,” “We’re not carriage makers,” “We’re not waste removers,” … “We’re not wise ones,” “We’re not competent ones,” “We’re not intelligent ones,” “We’re not learned ones,” “We’re not expounders of the Teaching,” “We’re not going to a bad destination,” “We can only expect a good destination,” then for every statement, he commits an offense of wrong speech. 

\subparagraph*{Not intending to abuse one who is not fully ordained }

If\marginnote{2.8.1} one who is fully ordained, not wishing to revile, not wishing to insult, not wishing to humiliate someone who is not fully ordained, but wanting to have fun, says what is low to one who is low … says what is low to one who is high … says what is high to one who is low … says what is high to one who is high—one who is wise, one who is competent, one who is intelligent, one who is learned, one who is an expounder of the Teaching—saying, “You’re wise,” “You’re competent,” “You’re intelligent,” “You’re learned,” “You’re an expounder of the Teaching,” “You’re not going to a bad destination,” “You can only expect a good destination,” then for every statement, he commits an offense of wrong speech. 

If\marginnote{2.8.6} one who is fully ordained, not wishing to revile, not wishing to insult, not wishing to humiliate someone who is not fully ordained, but wanting to have fun, says, “There are outcasts right here,” “There are bamboo workers right here,” “There are hunters right here,” “There are carriage makers right here,” “There are waste removers right here,” … “There are wise ones right here,” “There are competent ones right here,” “There are intelligent ones right here,” “There are learned ones right here,” “There are expounders of the Teaching right here,” “There are those not going to a bad destination right here,” “There are those who can only expect a good destination right here,” then for every statement, he commits an offense of wrong speech. 

If\marginnote{2.8.8} one who is fully ordained, not wishing to revile, not wishing to insult, not wishing to humiliate someone who is not fully ordained, but wanting to have fun, says, “Perhaps these are outcasts,” “Perhaps these are bamboo workers,” “Perhaps these are hunters,” “Perhaps these are carriage makers,” “Perhaps these are waste removers,” … “Perhaps these are wise ones,” “Perhaps these are competent ones,” “Perhaps these are intelligent ones,” “Perhaps these are learned ones,” “Perhaps these are expounders of the Teaching,” then for every statement, he commits an offense of wrong speech. 

If\marginnote{2.8.10} one who is fully ordained, not wishing to revile, not wishing to insult, not wishing to humiliate someone who is not fully ordained, but wanting to have fun, says, “We’re not outcasts,” “We’re not bamboo workers,” “We’re not hunters,” “We’re not carriage makers,” “We’re not waste removers,” … “We’re not wise ones,” “We’re not competent ones,” “We’re not intelligent ones,” “We’re not learned ones,” “We’re not expounders of the Teaching,” “We’re not going to a bad destination,” “We can only expect a good destination,” then for every statement, he commits an offense of wrong speech. 

\subsection*{Non-offenses }

There\marginnote{2.9.1} is no offense: if he is aiming at something beneficial; if he is aiming at giving a teaching; if he is aiming at giving an instruction; if he is insane; if he is deranged; if he is overwhelmed by pain; if he is the first offender. 

\scendsutta{The training rule on abusive speech, the second, is finished. }

%
\section*{{\suttatitleacronym Bu Pc 3}{\suttatitletranslation 3. The training rule on malicious talebearing }{\suttatitleroot Pesuñña}}
\addcontentsline{toc}{section}{\tocacronym{Bu Pc 3} \toctranslation{3. The training rule on malicious talebearing } \tocroot{Pesuñña}}
\markboth{3. The training rule on malicious talebearing }{Pesuñña}
\extramarks{Bu Pc 3}{Bu Pc 3}

\subsection*{Origin story }

At\marginnote{1.1} one time when the Buddha was staying at \textsanskrit{Sāvatthī} in \textsanskrit{Anāthapiṇḍika}’s Monastery, the monks from the group of six were engaged in malicious talebearing between monks who were arguing. After hearing something on one side they reported it to the other side, and vice versa, in order to create division between them. In this way they started new quarrels and made existing quarrels worse. 

The\marginnote{1.5} monks of few desires complained and criticized them, “How can the monks from the group of six engage in malicious talebearing between monks who are arguing? How can they report to one side what they have heard on the other side, and vice versa, in order to create division, and in this way start new quarrels and make existing quarrels worse?” 

After\marginnote{1.9} rebuking those monks in many ways, they told the Buddha. Soon afterwards he had the Sangha gathered and questioned those monks: “Is it true, monks, that you do this?” 

“It’s\marginnote{1.12} true, sir.” 

The\marginnote{1.13} Buddha rebuked them … “Foolish men, how can you do this? This will affect people’s confidence …” … “And, monks, this training rule should be recited like this: 

\subsection*{Final ruling }

\scrule{‘If a monk engages in malicious talebearing between monks, he commits an offense entailing confession.’”\footnote{This is translated somewhat freely in accordance with the explanation below and in the commentary. According to \href{https://suttacentral.net/pli-tv-bu-vb-pc3/en/brahmali\#2.3.1}{Bu Pc 3:2.3.1} the offense is only incurred when a monk brings the words of another monk to the attention of yet another monk, and thus the translation “between monks”. Sp 2.37 supports the plural rendering “monks”: \textit{\textsanskrit{Bhikkhūnaṁ} \textsanskrit{pesuññe}}, “Malicious talebearing of monks”. } }

\subsection*{Definitions }

\begin{description}%
\item[Malicious talebearing: ] there is malicious talebearing in two ways: for one wanting to endear himself and for one aiming at division. One engages in malicious talebearing in ten ways: about caste, about name, about family, about occupation, about profession, about illnesses, about physical traits, about defilements, about offenses, and by name-calling. %
\end{description}

\subsection*{Permutations }

\subsubsection*{Permutations part 1 }

\paragraph*{Definitions }

\begin{description}%
\item[Caste: ] there are two kinds of castes: low castes and high castes. %
\item[Low castes: ] outcasts, bamboo workers, hunters, carriage makers, waste removers—these are called “low castes”. %
\item[High castes: ] aristocrats and brahmins—these are called “high castes”. %
\end{description}

(To\marginnote{2.1.10} be expanded as in previous rule.) 

\begin{description}%
\item[Name-calling: ] there are two kinds of name-calling: low name-calling and high name-calling. %
\item[Low name-calling: ] “You’re a camel,” “You’re a ram,” “You’re an ox,” “You’re a donkey,” “You’re an animal,” “You’re bound for hell,” “You’re not going to a good destination,” “You can only expect a bad destination,” or adding disparaging endings to someone’s name, or calling someone words for the male and female genitals—this is called “low name-calling”. %
\item[High name-calling: ] “You’re wise,” “You’re competent,” “You’re intelligent,” “You’re learned,” “You’re an expounder of the Teaching,” “You’re not going to a bad destination,” “You can only expect a good destination”—this is called “high name-calling”. %
\end{description}

\paragraph*{Exposition }

\subparagraph*{Direct abuse }

If\marginnote{2.2.1} one who is fully ordained, after hearing it from another who is fully ordained, engages in malicious talebearing by saying to yet another who is fully ordained, “So-and-so says this about you, ‘He’s an outcast,’ ‘He’s a bamboo worker,’ ‘He’s a hunter,’ ‘He’s a carriage maker,’ ‘He’s a waste remover,’” then for every statement, he commits an offense entailing confession. 

If\marginnote{2.2.2} one who is fully ordained, after hearing it from another who is fully ordained, engages in malicious talebearing by saying to yet another who is fully ordained, “So-and-so says this about you, ‘He’s an aristocrat,’ ‘He’s a brahmin,’” then for every statement, he commits an offense entailing confession. 

If\marginnote{2.2.3} one who is fully ordained, after hearing it from another who is fully ordained, engages in malicious talebearing by saying to yet another who is fully ordained, “So-and-so says this about you, ‘He’s an \textsanskrit{Avakaṇṇaka},’ ‘He’s a \textsanskrit{Javakaṇṇaka},’ ‘He’s a \textsanskrit{Dhaniṭṭhaka},’ ‘He’s a \textsanskrit{Saviṭṭhaka},’ ‘He’s a \textsanskrit{Kulavaḍḍhaka},’” then for every statement, he commits an offense entailing confession. 

If\marginnote{2.2.4} one who is fully ordained, after hearing it from another who is fully ordained, engages in malicious talebearing by saying to yet another who is fully ordained, “So-and-so says this about you, ‘He’s a Buddharakkhita,’ ‘He’s a Dhammarakkhita,’ ‘He’s a Sangharakkhita,’” then for every statement, he commits an offense entailing confession. 

If\marginnote{2.2.5} one who is fully ordained, after hearing it from another who is fully ordained, engages in malicious talebearing by saying to yet another who is fully ordained, “So-and-so says this about you, ‘He’s a Kosiya,’ ‘He’s a \textsanskrit{Bhāradvāja},’” then for every statement, he commits an offense entailing confession. 

If\marginnote{2.2.6} one who is fully ordained, after hearing it from another who is fully ordained, engages in malicious talebearing by saying to yet another who is fully ordained, “So-and-so says this about you, ‘He’s a Gotama,’ ‘He’s a \textsanskrit{Moggallāna},’ ‘He’s a \textsanskrit{Kaccāna},’ ‘He’s a \textsanskrit{Vāsiṭṭha},’” then for every statement, he commits an offense entailing confession. 

If\marginnote{2.2.7} one who is fully ordained, after hearing it from another who is fully ordained, engages in malicious talebearing by saying to yet another who is fully ordained, “So-and-so says this about you, ‘He’s a carpenter,’ ‘He’s a waste remover,’” then for every statement, he commits an offense entailing confession. 

If\marginnote{2.2.8} one who is fully ordained, after hearing it from another who is fully ordained, engages in malicious talebearing by saying to yet another who is fully ordained, “So-and-so says this about you, ‘He’s a farmer,’ ‘He’s a trader,’ ‘He’s a cattle keeper,’” then for every statement, he commits an offense entailing confession. 

If\marginnote{2.2.9} one who is fully ordained, after hearing it from another who is fully ordained, engages in malicious talebearing by saying to yet another who is fully ordained, “So-and-so says this about you, ‘He’s a reed worker,’ ‘He’s a potter,’ ‘He’s a weaver,’ ‘He’s a leather worker,’ ‘He’s a hairdresser,’” then for every statement, he commits an offense entailing confession. 

If\marginnote{2.2.10} one who is fully ordained, after hearing it from another who is fully ordained, engages in malicious talebearing by saying to yet another who is fully ordained, “So-and-so says this about you, ‘He’s an arithmetician,’ ‘He’s an accountant,’ ‘He’s a clerk,’” then for every statement, he commits an offense entailing confession. 

If\marginnote{2.2.11} one who is fully ordained, after hearing it from another who is fully ordained, engages in malicious talebearing by saying to yet another who is fully ordained, “So-and-so says this about you, ‘He’s a leper,’ ‘He has abscesses,’ ‘He has mild leprosy,’ ‘He has tuberculosis,’ ‘He’s an epileptic,’” then for every statement, he commits an offense entailing confession. 

If\marginnote{2.2.12} one who is fully ordained, after hearing it from another who is fully ordained, engages in malicious talebearing by saying to yet another who is fully ordained, “So-and-so says this about you, ‘He’s a diabetic,’” then for every statement, he commits an offense entailing confession. 

If\marginnote{2.2.13} one who is fully ordained, after hearing it from another who is fully ordained, engages in malicious talebearing by saying to yet another who is fully ordained, “So-and-so says this about you, ‘He’s too tall,’ ‘He’s too short,’ ‘He’s too dark,’ ‘He’s too fair,’” then for every statement, he commits an offense entailing confession. 

If\marginnote{2.2.14} one who is fully ordained, after hearing it from another who is fully ordained, engages in malicious talebearing by saying to yet another who is fully ordained, “So-and-so says this about you, ‘He’s not too tall,’ ‘He’s not too short,’ ‘He’s not too dark,’ ‘He’s not too fair,’” then for every statement, he commits an offense entailing confession. 

If\marginnote{2.2.15} one who is fully ordained, after hearing it from another who is fully ordained, engages in malicious talebearing by saying to yet another who is fully ordained, “So-and-so says this about you, ‘He’s full of sensual desire,’ ‘He’s full of ill will,’ ‘He’s full of confusion,’” then for every statement, he commits an offense entailing confession. 

If\marginnote{2.2.16} one who is fully ordained, after hearing it from another who is fully ordained, engages in malicious talebearing by saying to yet another who is fully ordained, “So-and-so says this about you, ‘He’s without sensual desire,’ ‘He’s without ill will,’ ‘He’s without confusion,’” then for every statement, he commits an offense entailing confession. 

If\marginnote{2.2.17} one who is fully ordained, after hearing it from another who is fully ordained, engages in malicious talebearing by saying to yet another who is fully ordained, “So-and-so says this about you, ‘He has committed an offense entailing expulsion,’ ‘He has committed an offense entailing suspension,’ ‘He has committed a serious offense,’ ‘He has committed an offense entailing confession,’ ‘He has committed an offense entailing acknowledgment,’ ‘He has committed an offense of wrong conduct,’ ‘He has committed an offense of wrong speech,’” then for every statement, he commits an offense entailing confession. 

If\marginnote{2.2.18} one who is fully ordained, after hearing it from another who is fully ordained, engages in malicious talebearing by saying to yet another who is fully ordained, “So-and-so says this about you, ‘He’s a stream-enterer,’” then for every statement, he commits an offense entailing confession. 

If\marginnote{2.2.19} one who is fully ordained, after hearing it from another who is fully ordained, engages in malicious talebearing by saying to yet another who is fully ordained, “So-and-so says this about you, ‘He’s a camel,’ ‘He’s a ram,’ ‘He’s an ox,’ ‘He’s a donkey,’ ‘He’s an animal,’ ‘He’s bound for hell,’ ‘He’s not going to a good destination,’ ‘He can only expect a bad destination,’” then for every statement, he commits an offense entailing confession. 

If\marginnote{2.2.20} one who is fully ordained, after hearing it from another who is fully ordained, engages in malicious talebearing by saying to yet another who is fully ordained, “So-and-so says this about you, ‘He’s wise,’ ‘He’s competent,’ ‘He’s intelligent,’ ‘He’s learned,’ ‘He’s an expounder of the Teaching,’ ‘He’s not going to a bad destination,’ ‘He can only expect a good destination,’” then for every statement, he commits an offense entailing confession. 

\subparagraph*{Indirect abuse }

If\marginnote{2.2.21.1} one who is fully ordained, after hearing it from another who is fully ordained, engages in malicious talebearing by saying to yet another who is fully ordained, “So-and-so says, ‘There are outcasts right here,’ ‘There are bamboo workers right here,’ ‘There are hunters right here,’ ‘There are carriage makers right here,’ ‘There are waste removers right here,’ and he’s not speaking about someone else, he’s speaking about you,” then for every statement, he commits an offense of wrong conduct. 

If\marginnote{2.2.22} one who is fully ordained, after hearing it from another who is fully ordained, engages in malicious talebearing by saying to yet another who is fully ordained, “So-and-so says, ‘There are aristocrats right here,’ ‘There are brahmins right here,’ and he’s not speaking about someone else, he’s speaking about you,” then for every statement, he commits an offense of wrong conduct. … 

If\marginnote{2.2.23} one who is fully ordained, after hearing it from another who is fully ordained, engages in malicious talebearing by saying to yet another who is fully ordained, “So-and-so says, ‘There are wise ones right here,’ ‘There are competent ones right here,’ ‘There are intelligent ones right here,’ ‘There are learned ones right here,’ ‘There are expounders of the Teaching right here,’ ‘There are those not going to a bad destination right here,’ ‘There are those who can only expect a good destination right here,’ and he’s not speaking about someone else, he’s speaking about you,” then for every statement, he commits an offense of wrong conduct. 

If\marginnote{2.2.24} one who is fully ordained, after hearing it from another who is fully ordained, engages in malicious talebearing by saying to yet another who is fully ordained, “So-and-so says, ‘Perhaps these are outcasts,’ ‘Perhaps these are bamboo workers,’ ‘Perhaps these are hunters,’ ‘Perhaps these are carriage makers,’ ‘Perhaps these are waste removers,’ and he’s not speaking about someone else, he’s speaking about you,” then for every statement, he commits an offense of wrong conduct. …\footnote{I have added the ellipses points, which are presumably missing in the Pali due to an editing mistake, cf. \href{https://suttacentral.net/pli-tv-bu-vb-pc3/en/brahmali\#2.2.22}{Bu Pc 3:2.2.22}. The ellipses points are also missing at \href{https://suttacentral.net/pli-tv-bu-vb-pc3/en/brahmali\#2.2.26}{Bu Pc 3:2.2.26}. } 

If\marginnote{2.2.25} one who is fully ordained, after hearing it from another who is fully ordained, engages in malicious talebearing by saying to yet another who is fully ordained, “So-and-so says, ‘Perhaps these are wise ones,’ ‘Perhaps these are competent ones,’ ‘Perhaps these are intelligent ones,’ ‘Perhaps these are learned ones,’ ‘Perhaps these are expounders of the Teaching,’ and he’s not speaking about someone else, he’s speaking about you,” then for every statement, he commits an offense of wrong conduct. 

If\marginnote{2.2.26} one who is fully ordained, after hearing it from another who is fully ordained, engages in malicious talebearing by saying to yet another who is fully ordained, “So-and-so says, ‘We’re not outcasts,’ ‘We’re not bamboo workers,’ ‘We’re not hunters,’ ‘We’re not carriage makers,’ ‘We’re not waste removers,’ and he’s not speaking about someone else, he’s speaking about you,” then for every statement, he commits an offense of wrong conduct. … 

If\marginnote{2.2.27} one who is fully ordained, after hearing it from another who is fully ordained, engages in malicious talebearing by saying to yet another who is fully ordained, “So-and-so says, ‘We’re not wise ones,’ ‘We’re not competent ones,’ ‘We’re not intelligent ones,’ ‘We’re not learned ones,’ ‘We’re not expounders of the Teaching,’ ‘We’re not going to a bad destination,’ ‘We can only expect a good destination,’ and he’s not speaking about someone else, he’s speaking about you,” then for every statement, he commits an offense of wrong conduct.\footnote{This looks like an editing mistake. Presumably this should read, “We’re not going to a good destination/we can only expect a bad destination.” } 

\subsubsection*{Permutations part 2 }

If\marginnote{2.3.1} one who is fully ordained, after hearing it from another who is fully ordained, engages in malicious talebearing to yet another who is fully ordained, then for every statement, he commits an offense entailing confession. 

If\marginnote{2.3.2} one who is fully ordained, after hearing it from another who is fully ordained, engages in malicious talebearing to one who is not fully ordained, then for every statement, he commits an offense of wrong conduct. 

If\marginnote{2.3.3} one who is fully ordained, after hearing it from one who is not fully ordained, engages in malicious talebearing to another who is fully ordained, then for every statement, he commits an offense of wrong conduct. 

If\marginnote{2.3.4} one who is fully ordained, after hearing it from one who is not fully ordained, engages in malicious talebearing to another who is not fully ordained, then for every statement, he commits an offense of wrong conduct. 

\subsection*{Non-offenses }

There\marginnote{2.4.1} is no offense: if he does not want to endear himself and he is not aiming at division; if he is insane; if he is the first offender. 

\scendsutta{The training rule on malicious talebearing, the third, is finished. }

%
\section*{{\suttatitleacronym Bu Pc 4}{\suttatitletranslation 4. The training rule on memorizing the Teaching }{\suttatitleroot Padasodhamma}}
\addcontentsline{toc}{section}{\tocacronym{Bu Pc 4} \toctranslation{4. The training rule on memorizing the Teaching } \tocroot{Padasodhamma}}
\markboth{4. The training rule on memorizing the Teaching }{Padasodhamma}
\extramarks{Bu Pc 4}{Bu Pc 4}

\subsection*{Origin story }

At\marginnote{1.1} one time when the Buddha was staying at \textsanskrit{Sāvatthī} in \textsanskrit{Anāthapiṇḍika}’s Monastery, the monks from the group of six were instructing lay followers to memorize the Teaching. Those lay followers became disrespectful, undeferential, and rude toward the monks. 

The\marginnote{1.4} monks of few desires complained and criticized them, “How can the monks from the group of six instruct lay followers to memorize the Teaching?” 

After\marginnote{1.7} rebuking those monks in many ways, they told the Buddha. Soon afterwards he had the Sangha gathered and questioned the monks: “Is it true, monks, that you do this?” 

“It’s\marginnote{1.10} true, sir.” 

The\marginnote{1.11} Buddha rebuked them … “Foolish men, how can you do this?” This will affect people’s confidence …” … “And, monks, this training rule should be recited like this: 

\subsection*{Final ruling }

\scrule{‘If a monk instructs a person who is not fully ordained to memorize the Teaching, he commits an offense entailing confession.’” }

\subsection*{Definitions }

\begin{description}%
\item[A: ] whoever … %
\item[Monk: ] … The monk who has been given the full ordination by a unanimous Sangha through a legal procedure consisting of one motion and three announcements that is irreversible and fit to stand—this sort of monk is meant in this case. %
\item[A person who is not fully ordained: ] anyone except a fully ordained monk or a fully ordained nun. %
\item[To memorize:\footnote{\textit{Padaso} does not literally mean “memorize”, but something like “by the line”. The reason I have “memorize” here is simply an artifact of  my translation of the rule just above. The expression \textit{padaso \textsanskrit{dhammaṁ} \textsanskrit{vāceyya}} literally means “causing (someone) to say the Teaching by the line”, which implies helping another to memorize the teaching, hence my rendering. } ] a line, the next line, the next syllable, the next phrase. %
\item[A line: ] they start together and finish together. %
\item[The next line: ] one of them starts, but they finish together. %
\item[The next syllable: ] when ‘\textit{\textsanskrit{Rūpaṁ} \textsanskrit{aniccaṁ}}’ is being said, he prompts him, saying, ‘\textit{\textsanskrit{rū}}’. %
\item[The next phrase: ] when ‘\textit{\textsanskrit{Rūpaṁ} \textsanskrit{aniccaṁ}},’ is being said, the other says, ‘\textit{\textsanskrit{Vedanā} \textsanskrit{aniccā}}.’ %
\end{description}

And\marginnote{2.1.17} whatever line there is, whatever next line, whatever next syllable, whatever next phrase—this is all called “to memorize”. 

\begin{description}%
\item[The Teaching: ] what has been spoken by the Buddha, what has been spoken by disciples, what has been spoken by sages, what has been spoken by gods, what is connected with what is beneficial, what is connected with the Teaching. %
\item[Instructs: ] if he instructs by the line, then for every line he commits an offense entailing confession. If he instructs by the syllable, then for every syllable he commits an offense entailing confession. %
\end{description}

\subsection*{Permutations }

If\marginnote{2.2.1} the person is not fully ordained, and the monk does not perceive them as such, and he instructs them to memorize the Teaching, he commits an offense entailing confession. If the person is not fully ordained, but the monk is unsure of it, and he instructs them to memorize the Teaching, he commits an offense entailing confession. If the person is not fully ordained, but the monk perceives them as such, and he instructs them to memorize the Teaching, he commits an offense entailing confession. 

If\marginnote{2.2.4} the person is fully ordained, but the monk does not perceive them as such, he commits an offense of wrong conduct.\footnote{Because the gender of the ordained person is not given, I use “them”. } If the person is fully ordained, but the monk is unsure of it, he commits an offense of wrong conduct. If the person is fully ordained, and the monk perceives them such, there is no offense. 

\subsection*{Non-offenses }

There\marginnote{2.3.1} is no offense: if the recitation is done together; if they practice together; if he prompts one who is speaking a mostly familiar text; if he prompts one who is reciting; if he is insane; if he is the first offender. 

\scendsutta{The training rule on memorizing the Teaching, the fourth, is finished. }

%
\section*{{\suttatitleacronym Bu Pc 5}{\suttatitletranslation 5. The training rule on the same sleeping place }{\suttatitleroot Anupasampannasahaseyya}}
\addcontentsline{toc}{section}{\tocacronym{Bu Pc 5} \toctranslation{5. The training rule on the same sleeping place } \tocroot{Anupasampannasahaseyya}}
\markboth{5. The training rule on the same sleeping place }{Anupasampannasahaseyya}
\extramarks{Bu Pc 5}{Bu Pc 5}

\subsection*{Origin story }

\subsubsection*{First sub-story }

At\marginnote{1.1} one time the Buddha was staying at \textsanskrit{Āḷavī} at the \textsanskrit{Aggāḷava} Shrine. At that time the lay followers were coming to the monastery to listen to the Teaching. When the instruction was over, the senior monks went to their own dwellings, but the newly ordained monks lay down right there in the assembly hall together with the lay followers—absentminded, heedless, naked, muttering, and snoring. The lay followers complained and criticized them, “How can the venerables lie down absentminded, heedless, naked, muttering, and snoring?” 

The\marginnote{1.7} monks heard the complaints of those lay followers, and the monks of few desires complained and criticized those monks, “How can monks lie down in the same sleeping place as people who are not fully ordained?” 

After\marginnote{1.10} rebuking those newly ordained monks in many ways, they told the Buddha. Soon afterwards he had the Sangha gathered and questioned the monks: “Is it true, monks, that monks did this?” 

“It’s\marginnote{1.12} true, sir.” 

The\marginnote{1.13} Buddha rebuked them … “How could those foolish men do this? This will affect people’s confidence …” … “And, monks, this training rule should be recited like this: 

\subsubsection*{Preliminary ruling }

\scrule{‘If a monk lies down in the same sleeping place as a person who is not fully ordained, he commits an offense entailing confession.’” }

In\marginnote{1.18} this way the Buddha laid down this training rule for the monks. 

\subsubsection*{Second sub-story }

After\marginnote{2.1} staying at \textsanskrit{Āḷavī} for as long as he wanted, the Buddha set out wandering toward \textsanskrit{Kosambī}. When he eventually arrived, he stayed at the \textsanskrit{Badarikā} Monastery. 

Just\marginnote{2.3} then the monks there said to Venerable \textsanskrit{Rāhula}, “\textsanskrit{Rāhula}, the Buddha has laid down a training rule that we can’t lie down in the same sleeping place as a person who’s not fully ordained. Please find another sleeping place.” Since \textsanskrit{Rāhula} was not able to find a sleeping place, he lay down in the restroom. 

Then,\marginnote{2.7} after rising early in the morning, the Buddha went to the restroom where he cleared his throat. \textsanskrit{Rāhula}, too, cleared his throat. 

“Who’s\marginnote{2.9} there?” 

“It’s\marginnote{2.10} me, sir, \textsanskrit{Rāhula}.” 

“Why\marginnote{2.11} are you sitting here, \textsanskrit{Rāhula}?” 

\textsanskrit{Rāhula}\marginnote{2.12} told the Buddha what had happened. Soon afterwards the Buddha gave a teaching and addressed the monks: 

\scrule{“Monks, I allow you to lie down in the same sleeping place as a person who isn’t fully ordained for two or three nights. }

And\marginnote{2.15} so, monks, this training rule should be recited like this: 

\subsection*{Final ruling }

\scrule{‘If a monk lies down more than two or three nights in the same sleeping place as a person who is not fully ordained, he commits an offense entailing confession.’” }

\subsection*{Definitions }

\begin{description}%
\item[A: ] whoever … %
\item[Monk: ] … The monk who has been given the full ordination by a unanimous Sangha through a legal procedure consisting of one motion and three announcements that is irreversible and fit to stand—this sort of monk is meant in this case. %
\item[A person who is not fully ordained: ] anyone except a fully ordained monk. %
\item[More than two or three nights: ] in excess of two or three nights. %
\item[Same: ] together. %
\item[Sleeping place: ] fully roofed, fully walled; mostly roofed, mostly walled. %
\item[Lies down in the same sleeping place: ] at dawn on the fourth day: if he lies down when the person who is not fully ordained is already lying down, he commits an offense entailing confession; if the person who is not fully ordained lies down when he is already lying down, he commits an offense entailing confession; if they both lie down together, he commits an offense entailing confession; every time they get up and then lie down again, he commits an offense entailing confession. %
\end{description}

\subsection*{Permutations }

If\marginnote{3.2.1} they are not fully ordained, and the monk does not perceive them as such, and he lies down more than two or three nights in the same sleeping place as them, he commits an offense entailing confession. If they are not fully ordained, but the monk is unsure of it, and he lies down more than two or three nights in the same sleeping place as them, he commits an offense entailing confession. If they are is not fully ordained, but the monk perceives them as such, and he lies down more than two or three nights in the same sleeping place as them, he commits an offense entailing confession. 

If\marginnote{3.2.4} it is half-roofed and half-walled, he commits an offense of wrong conduct. If they are fully ordained, but the monk does not perceive them as such, he commits an offense of wrong conduct. If they are fully ordained, but the monk is unsure of it, he commits an offense of wrong conduct. If they are fully ordained, and the monk perceives them as such, there is no offense. 

\subsection*{Non-offenses }

There\marginnote{3.3.1} is no offense: if he stays together with them for two or three nights; if he stays together with them for less than two or three nights; if, after staying together for two nights, he leaves before dawn on the third night and then stays together again; if it is fully roofed, but not walled; if it is fully walled, but not roofed; if it is mostly not roofed; if it is mostly not walled; if the monk sits when the person who is not fully ordained is lying down; if the person who is not fully ordained sits when the monk is lying down; if they both sit; if he is insane; if he is the first offender. 

\scendsutta{The training rule on the same sleeping place, the fifth, is finished. }

%
\section*{{\suttatitleacronym Bu Pc 6}{\suttatitletranslation 6. The second training rule on the same sleeping place }{\suttatitleroot Mātugāmasahaseyya}}
\addcontentsline{toc}{section}{\tocacronym{Bu Pc 6} \toctranslation{6. The second training rule on the same sleeping place } \tocroot{Mātugāmasahaseyya}}
\markboth{6. The second training rule on the same sleeping place }{Mātugāmasahaseyya}
\extramarks{Bu Pc 6}{Bu Pc 6}

\subsection*{Origin story }

At\marginnote{1.1} one time when the Buddha was staying at \textsanskrit{Sāvatthī} in \textsanskrit{Anāthapiṇḍika}’s Monastery, Venerable Anuruddha was walking through the Kosalan country on his way to \textsanskrit{Sāvatthī}, when one evening he arrived at a certain village. Just then a woman in that village had prepared her guesthouse.\footnote{Sp 2.55: \textit{\textsanskrit{Āvasathāgāranti} \textsanskrit{āgantukānaṁ} \textsanskrit{vasanāgāraṁ}}, “A house for visitors to stay.” } Anuruddha went to that woman and said, “If it’s not inconvenient for you, I’d like to stay in your guesthouse for one night.” 

“Please\marginnote{1.5} stay, venerable.” 

Other\marginnote{1.6} travelers also went to that woman and said, “Ma’am, if it’s not troublesome for you, we’d like to stay in your guesthouse for one night.” 

“Sirs,\marginnote{1.7} a monastic is already staying there. If he agrees, you may stay.” 

Those\marginnote{1.8} travelers then approached Anuruddha and said, “If you don’t mind, venerable, we’d like to stay one night in the guesthouse.” 

“No\marginnote{1.9} problem.” 

Now\marginnote{1.10} as soon as that woman had seen Anuruddha, she had fallen in love with him. She now went to him and said, “Sir, you won’t be comfortable surrounded by these people. Why don’t I prepare a bed for you in the main house?” Anuruddha consented by remaining silent. 

After\marginnote{1.14} preparing a bed in the main house, she put on jewelery and perfume, and she went to Anuruddha and said, “You’re attractive, sir, and so am I. Why don’t you take me as your wife?” But Anuruddha remained silent. She said the same thing a second time, but again got no response. And a third time she said, “You’re attractive, sir, and so am I. Why don’t you take me and all this property?” Once again Anuruddha remained silent. She then threw off her wrap, and she walked back and forth, stood, sat down, and lay down in front of him. But Anuruddha controlled his senses and neither looked at nor spoke to her. Then that woman said, “It’s astonishing and amazing. Many people pay a hundred or a thousand coins to be with me. But this monastic doesn’t want me and all this property, even when I beg him!” After dressing, she bowed down with her head at Anuruddha’s feet and said, “Sir, I’ve made a mistake. I’ve been foolish, confused, and unskillful. Please forgive me so that I may restrain myself in the future.” 

“You\marginnote{1.31} have certainly made a mistake. You’ve been foolish, confused, and unskillful. But since you acknowledge your mistake and make proper amends, I forgive you. For this is called growth in the training of the noble ones: acknowledging a mistake, making proper amends, and undertaking restraint for the future.”\footnote{“Training” renders \textit{vinaya}.  See Appendix of Technical Terms for a discussion. } 

The\marginnote{1.34} following morning that woman personally served and satisfied Anuruddha with various kinds of fine foods. When he had finished his meal, she bowed and sat down to one side. And Anuruddha instructed, inspired, and gladdened her with a teaching. She then said to him, “Wonderful, sir, wonderful! Just as one might set upright what’s overturned, or reveal what’s hidden, or show the way to one who’s lost, or bring a lamp into the darkness so that one with eyes might see what is there—just so has the Buddha made the Teaching clear in many ways. I go for refuge to the Buddha, the Teaching, and the Sangha of monks. Please accept me as a lay follower who’s gone for refuge for life.” 

Soon\marginnote{1.42} afterwards, after arriving at \textsanskrit{Sāvatthī}, Anuruddha told the monks what had happened. The monks of few desires complained and criticized him, “How can Venerable Anuruddha lie down in the same sleeping place as a woman?” 

After\marginnote{1.45} rebuking him in many ways, they told the Buddha. Soon afterwards he had the Sangha gathered and questioned Anuruddha: “Is it true, Anuruddha, that you did this?” 

“It’s\marginnote{1.47} true, sir.” 

The\marginnote{1.48} Buddha rebuked him … “Anuruddha, how could you do this? This will affect people’s confidence …” … “And, monks, this training rule should be recited like this: 

\subsection*{Final ruling }

\scrule{‘If a monk lies down in the same sleeping place as a woman, he commits an offense entailing confession.’” }

\subsection*{Definitions }

\begin{description}%
\item[A: ] whoever … %
\item[Monk: ] … The monk who has been given the full ordination by a unanimous Sangha through a legal procedure consisting of one motion and three announcements that is irreversible and fit to stand—this sort of monk is meant in this case. %
\item[A woman: ] a female human being, not a female spirit, not a female ghost, not a female animal; even a girl born on that very day, let alone an older one. %
\item[Same: ] together. %
\item[Sleeping place: ] fully roofed, fully walled; mostly roofed, mostly walled. %
\item[Lies down in the same sleeping place: ] when the sun has set—if the monk lies down when the woman is already lying down, he commits an offense entailing confession; if the woman lies down when the monk is already lying down, he commits an offense entailing confession; if they both lie down together, he commits an offense entailing confession; every time they get up and then lie down again, he commits an offense entailing confession. %
\end{description}

\subsection*{Permutations }

If\marginnote{2.2.1} it is a woman, and he perceives her as such, and he lies down in the same sleeping place as her, he commits an offense entailing confession. If it is a woman, but he is unsure of it, and he lies down in the same sleeping place as her, he commits an offense entailing confession. If it is a woman, but he does not perceive her as such, and he lies down in the same sleeping place as her, he commits an offense entailing confession. 

If\marginnote{2.2.4} it is half-roofed and half-walled, he commits an offense of wrong conduct. If he lies down in the same sleeping place as a female spirit, a female ghost, a \textit{\textsanskrit{paṇḍaka}}, or a female animal, he commits an offense of wrong conduct. If it is not a woman, but he perceives them as such, he commits an offense of wrong conduct. If it is not a woman, but he is unsure of it, he commits an offense of wrong conduct. If it is not a woman, and he does not perceive them as such, there is no offense. 

\subsection*{Non-offenses }

There\marginnote{2.3.1} is no offense: if it is fully roofed, but not walled; if it is fully walled, but not roofed; if it is mostly not roofed; if it is mostly not walled; if the monk sits when the woman is lying down; if the woman sits when the monk is lying down; if they both sit down together; if he is insane; if he is the first offender. 

\scendsutta{The second training rule on the same sleeping place, the sixth, is finished. }

%
\section*{{\suttatitleacronym Bu Pc 7}{\suttatitletranslation 7. The training rule on teaching }{\suttatitleroot Dhammadesanā}}
\addcontentsline{toc}{section}{\tocacronym{Bu Pc 7} \toctranslation{7. The training rule on teaching } \tocroot{Dhammadesanā}}
\markboth{7. The training rule on teaching }{Dhammadesanā}
\extramarks{Bu Pc 7}{Bu Pc 7}

\subsection*{Origin story }

\subsubsection*{First sub-story }

At\marginnote{1.1} one time the Buddha was staying at \textsanskrit{Sāvatthī} in the Jeta Grove, \textsanskrit{Anāthapiṇḍika}’s Monastery. At that time Venerable \textsanskrit{Udāyī} was associating with and visiting a number of families in \textsanskrit{Sāvatthī}. After robing up one morning, he took his bowl and robe and went to a certain family. Just then the housewife was sitting at the door to the house, while the daughter-in-law at the door to guesthouse. \textsanskrit{Udāyī} went up to the housewife and gave her a teaching, whispering in her ear. And the daughter-in-law thought, “Is this monastic my mother-in-law’s lover, or is he speaking indecently?” 

After\marginnote{1.8} teaching the housewife in this way, \textsanskrit{Udāyī} went up to the daughter-in-law and gave her a teaching in the same way. Then the housewife thought, “Is this monastic my daughter-in-law’s lover, or is he speaking indecently?” 

When\marginnote{1.11} \textsanskrit{Udāyī} had left, the housewife said to her daughter-in-law, “Hey, what did that monastic say to you?” 

“He\marginnote{1.14} gave me a teaching, ma’am. But what did he say to you?” 

“He\marginnote{1.16} gave me a teaching, too.” 

And\marginnote{1.17} they complained and criticized him, “How can Venerable \textsanskrit{Udāyī} give teachings by whispering in the ear? Should not teachings be given audibly and openly?” 

The\marginnote{1.20} monks heard the complaints of those women, and the monks of few desires complained and criticized \textsanskrit{Udāyī}, “How can Venerable \textsanskrit{Udāyī} give teachings to women?” 

After\marginnote{1.23} rebuking him in many ways, they told the Buddha. Soon afterwards he had the Sangha gathered and questioned \textsanskrit{Udāyī}: “Is it true, \textsanskrit{Udāyī}, that you did this?” 

“It’s\marginnote{1.25} true, sir.” 

The\marginnote{1.26} Buddha rebuked him … “Foolish man, how can you do this? This will affect people’s confidence …” … “And, monks, this training rule should be recited like this: 

\subsubsection*{First preliminary ruling }

\scrule{‘If a monk gives a teaching to a woman, he commits an offense entailing confession.’” }

In\marginnote{1.31} this way the Buddha laid down this training rule for the monks. 

\subsubsection*{Second sub-story }

Soon\marginnote{2.1} afterwards some female lay followers saw some monks and said to them, “Venerables, please give a teaching.” 

“It’s\marginnote{2.3} not allowable for us to teach women.” 

“Just\marginnote{2.4} teach five or six sentences. That might be enough for us to understand.” 

“It’s\marginnote{2.5} not allowable for us to teach women.” And being afraid of wrongdoing, they did not teach them. 

Those\marginnote{2.7} female lay followers complained and criticized them, “How can they not teach us when asked?” 

The\marginnote{2.9} monks heard the complaints of those female lay followers, and they told the Buddha. Soon afterwards the Buddha gave a teaching and addressed the monks: 

\scrule{“Monks, I allow you to teach five or six sentences to a woman. }

And\marginnote{2.13} so, monks, this training rule should be recited like this: 

\subsubsection*{Second preliminary ruling }

\scrule{‘If a monk gives a teaching of more than five or six sentences to a woman, he commits an offense entailing confession.’” }

In\marginnote{2.15} this way the Buddha laid down this training rule for the monks. 

\subsubsection*{Third sub-story }

When\marginnote{3.1} the monks from the group of six heard that the Buddha had made this allowance, they taught women more than five or six sentences with a man who did not understand sitting nearby. The monks of few desires complained and criticized them, “How can the monks from the group of six teach women more than five or six sentences with a man who doesn’t understand sitting nearby?” 

After\marginnote{3.4} rebuking those monks in many ways, they told the Buddha. Soon afterwards he had the Sangha gathered and questioned the monks: “Is it true, monks, that you do this?” 

“It’s\marginnote{3.6} true, sir.” 

The\marginnote{3.7} Buddha rebuked them … “Foolish men, how can you do this? This will affect people’s confidence …” … “And so, monks, this training rule should be recited like this: 

\subsection*{Final ruling }

\scrule{‘If a monk gives a teaching of more than five or six sentences to a woman, except in the presence of a man who understands, he commits an offense entailing confession.’” }

\subsection*{Definitions }

\begin{description}%
\item[A: ] whoever … %
\item[Monk: ] …The monk who has been given the full ordination by a unanimous Sangha through a legal procedure consisting of one motion and three announcements that is irreversible and fit to stand—this sort of monk is meant in this case. %
\item[A woman: ] a female human being, not a female spirit, not a female ghost, not a female animal; one who understands and is capable of discerning bad speech and good speech, what is indecent and what is decent. %
\item[More than five or six sentences: ] in excess of five or six sentences. %
\item[A teaching: ] what has been spoken by the Buddha, what has been spoken by disciples, what has been spoken by sages, what has been spoken by gods, what is connected with what is beneficial, what is connected with the Teaching. %
\item[Gives: ] if he teaches by the line, then for every line he commits an offense entailing confession. If he teaches by the syllable, then for every syllable he commits an offense entailing confession. %
\item[Except in the presence of a man who understands: ] unless a man who understands is present.\footnote{“Is present” is not actually found in the Pali, but it is implied. } %
\item[A man who understands: ] one who is capable of discerning bad speech and good speech, what is indecent and what is decent. %
\end{description}

\subsection*{Permutations }

If\marginnote{4.2.1} it is a woman, and he perceives her as such, and he teaches her more than five or six sentences, except in the presence of a man who understands, he commits an offense entailing confession. If it is a woman, but he is unsure of it, and he teaches her more than five or six sentences, except in the presence of a man who understands, he commits an offense entailing confession. If it is a woman, but he does not perceive her as such, and he teaches her more than five or six sentences, except in the presence of a man who understands, he commits an offense entailing confession. 

If\marginnote{4.2.4} he teaches more than five or six sentences to a female spirit, a female ghost, a \textit{\textsanskrit{paṇḍaka}}, or a female animal in the form of a woman, except in the presence of a man who understands, he commits an offense of wrong conduct.\footnote{For the meaning of \textit{\textsanskrit{paṇḍaka}}, see Appendix of Technical Terms. } If it is not a woman, but he perceives them as such, he commits an offense of wrong conduct. If it is not a woman, but he is unsure of it, he commits an offense of wrong conduct. If it is not a woman, and he does not perceive them as such, there is no offense. 

\subsection*{Non-offenses }

There\marginnote{4.3.1} is no offense: if a man who understands is present; if he teaches five or six sentences; if he teaches fewer than five or six sentences; if he gets up, sits down again, and then teaches; if the woman gets up and sits down again, and he then teaches her;\footnote{Sp 2.66: \textit{\textsanskrit{Sampadānatthe} \textsanskrit{vā} \textsanskrit{etaṁ} \textsanskrit{bhummavacanaṁ}. \textsanskrit{Tassā} \textsanskrit{desetīti} attho}, “Or this locative case is in the meaning of the dative. The meaning is: he teaches her.” In other words, the locative \textit{\textsanskrit{tasmiṁ}} should be read as a dative. } if he teaches another woman; if he asks a question; if he is asked a question and then speaks; if he is speaking for the benefit of someone else and a woman listens in; if he is insane; if he is the first offender. 

\scendsutta{The training rule on teaching, the seventh, is finished. }

%
\section*{{\suttatitleacronym Bu Pc 8}{\suttatitletranslation 8. The training rule on telling truthfully }{\suttatitleroot Bhūtārocana}}
\addcontentsline{toc}{section}{\tocacronym{Bu Pc 8} \toctranslation{8. The training rule on telling truthfully } \tocroot{Bhūtārocana}}
\markboth{8. The training rule on telling truthfully }{Bhūtārocana}
\extramarks{Bu Pc 8}{Bu Pc 8}

\subsection*{Origin story }

At\marginnote{1.1.1} one time when the Buddha was staying in the hall with the peaked roof in the Great Wood near \textsanskrit{Vesālī}, a number of monks who were friends had entered the rainy-season residence on the banks of the river \textsanskrit{Vaggumudā}. At that time \textsanskrit{Vajjī} was short of food and afflicted with hunger, with crops affected by whiteheads and turned to straw. It was not easy to get by on almsfood.\footnote{“Whiteheads” renders \textit{\textsanskrit{setaṭṭ}(h)\textsanskrit{ikā}}, literally, “white bones”. Sp 4.403: \textit{\textsanskrit{Setaṭṭhikā} \textsanskrit{nāma} \textsanskrit{rogajātīti} eko \textsanskrit{pāṇako} \textsanskrit{nāḷimajjhagataṁ} \textsanskrit{kaṇḍaṁ} vijjhati, yena \textsanskrit{viddhattā} nikkhantampi \textsanskrit{sālisīsaṁ} \textsanskrit{khīraṁ} \textsanskrit{gahetuṁ} na sakkoti}, “The disease called \textit{\textsanskrit{setaṭṭhikā}} means: an insect penetrates the stem, goes to the middle of the stalk, from the penetration of which the rice grains are not able to get sap.” This seems to be a description of so-called “whiteheads”, pale panicles without rice grains, caused by stem borers. } 

The\marginnote{1.1.4} monks considered the difficult circumstances, and they thought, “How can we live together in peace and harmony, have a comfortable rains, and get almsfood without trouble?” 

Some\marginnote{1.1.5} said, “We could work for the householders, and they’ll support us in return.” 

Others\marginnote{1.1.7} said, “There’s no need to work for the householders. Let’s instead take messages for them, and they’ll support us in return.” 

Still\marginnote{1.1.10} others said, “There’s no need to do work or take messages for them. Let’s instead talk up one another’s superhuman qualities to the householders: ‘That monk has the first absorption, that monk the second absorption, that monk the third, that monk the fourth; that monk is a stream-enterer, that monk a once-returner, that a non-returner, that a perfected one; that monk has the three true insights, and that the six direct knowledges.’ Then they’ll support us. In this way we’ll have a comfortable rains, live together in peace and harmony, and get almsfood without trouble. This is the way to go.” 

Then\marginnote{1.1.16} those monks did just that. And the people there thought, “We’re so fortunate that such monks have come to us for the rainy-season residence. Such virtuous and good monks have never before entered the rains residence with us.” And they gave such food and drink to those monks that they did not even eat and drink themselves, or give to their parents, to their wives and children, to their slaves, servants, and workers, to their friends and companions, or to their relatives. Soon those monks had a good color, bright faces, clear skin, and sharp senses. 

Now\marginnote{1.2.1} it was the custom for monks who had completed the rainy-season residence to go and visit the Buddha. And so, when the three months were over and they had completed the rains residence, those monks put their dwellings in order, took their bowls and robes, and set out for \textsanskrit{Vesālī}. When they eventually arrived, they went to the hall with the peaked roof in the Great Wood. There they approached the Buddha, bowed, and sat down. 

At\marginnote{1.2.5} that time the monks who had completed the rains residence in that region were thin, haggard and pale, with veins protruding all over their bodies. Yet the monks from the banks of the \textsanskrit{Vaggumudā} had a good color, bright faces, clear skin, and  sharp senses. 

Since\marginnote{1.2.7} it is the custom for Buddhas to greet newly-arrived monks, the Buddha said to them,  “I hope you’re keeping well, monks, I hope you’re getting by?  I hope you had a comfortable and harmonious rains, and got almsfood without trouble?” 

“We’re\marginnote{1.2.11} keeping well, sir, we’re getting by. We had a comfortable and harmonious rains, and got almsfood without trouble.” 

When\marginnote{1.2.13} Buddhas know what is going on, sometimes they ask and sometimes not. They know the right time to ask and when not to ask. Buddhas ask when it is beneficial, otherwise not, for Buddhas are incapable of doing what is unbeneficial.\footnote{“Incapable of doing” renders \textit{\textsanskrit{setughāta}}, literally, “destroyed the bridge”. Sp 1.16: \textit{Setu vuccati maggo, maggeneva \textsanskrit{tādisassa} vacanassa \textsanskrit{ghāto}, samucchedoti \textsanskrit{vuttaṁ} hoti}, “The path is called the bridge. What is said is that there is the destruction and cutting off of such speech by the path.” The commentary seems to take \textit{setu}, “bridge”, as a reference to the eightfold path. I prefer to understand “bridge” as a metaphor for access, that is, the Buddhas no longer have the possibility of doing what is unbeneficial. } Buddhas question the monks for two reasons: to give a teaching or to lay down a training rule. 

And\marginnote{1.2.18} the Buddha said to those monks, “In what way, monks, did you have a comfortable and harmonious rains? And how did you get almsfood without trouble?” 

They\marginnote{1.2.19} then told him. 

“But\marginnote{1.2.20} did you really have those superhuman qualities?” 

“Yes,\marginnote{1.2.21} sir.” 

The\marginnote{1.2.22} Buddha rebuked them, “How could you for the sake of your stomachs talk up one another’s superhuman qualities to householders? This will affect people’s confidence …” … “And, monks, this training rule should be recited like this: 

\subsection*{Final ruling }

\scrule{‘If a monk truthfully tells a person who is not fully ordained of a superhuman quality, he commits an offense entailing confession.’” }

\subsection*{Definitions }

\begin{description}%
\item[A: ] whoever … %
\item[Monk: ] … The monk who has been given the full ordination by a unanimous Sangha through a legal procedure consisting of one motion and three announcements that is irreversible and fit to stand—this sort of monk is meant in this case. %
\item[A person who is not fully ordained: ] anyone except a fully ordained monk or a fully ordained nun. %
\item[A superhuman quality: ] absorption, release, stillness, attainment, knowledge and vision, development of the path, realization of the fruits, abandoning the defilements, a mind without hindrances, delighting in solitude.\footnote{“Delighting in solitude” renders \textit{\textsanskrit{suññāgāra}}, literally, “delighting in an empty dwelling”. According to the commentaries, this is often an idiom for solitude. AN-a 1.88: \textit{Tattha ca \textsanskrit{rukkhamūlānīti} \textsanskrit{iminā} \textsanskrit{rukkhamūlasenāsanaṁ} dasseti. \textsanskrit{Suññāgārānīti} \textsanskrit{iminā} \textsanskrit{janavivittaṭṭhānaṁ}}; “And there \textit{\textsanskrit{rukkhamūlāni}}: by this is shown dwellings at the foot of a tree; \textit{\textsanskrit{suññāgārāni}}: by this (is shown) a place free from people.” } %
\end{description}

\subsection*{Permutations }

\paragraph*{Definitions }

\begin{description}%
\item[Absorption: ] the first absorption, the second absorption, the third absorption, the fourth absorption. %
\item[Release: ] emptiness release, signless release, desireless release. %
\item[Stillness: ] emptiness stillness, signless stillness, desireless stillness. %
\item[Attainment: ] emptiness attainment, signless attainment, desireless attainment. %
\item[Knowledge and vision: ] the three true insights. %
\item[Development of the path: ] the four applications of mindfulness, the four right efforts, the four foundations for supernormal power, the five spiritual faculties, the five spiritual powers, the seven factors of awakening, the noble eightfold path. %
\item[Realization of the fruits: ] realization of the fruit of stream-entry, realization of the fruit of once-returning, realization of the fruit of non-returning, realization of perfection. %
\item[Abandoning the defilements: ] the abandoning of sensual desire, the abandoning of ill will, the abandoning of confusion. %
\item[A mind without hindrances: ] a mind without sensual desire, a mind without ill will, a mind without confusion. %
\item[Delighting in solitude: ] because of the first absorption there is delight in solitude, because of the second absorption there is delight in solitude, because of the third absorption there is delight in solitude, because of the fourth absorption there is delight in solitude. %
\end{description}

\paragraph*{Exposition }

\subparagraph*{First absorption }

\begin{description}%
\item[Tells: ] if a monk says to a person who is not fully ordained, “I attained the first absorption,” he commits an offense entailing confession. %
\item[Tells: ] if a monk says to a person who is not fully ordained, “I’m attaining the first absorption,” he commits an offense entailing confession. %
\item[Tells: ] if a monk says to a person who is not fully ordained, “I’ve attained the first absorption,” he commits an offense entailing confession. %
\item[Tells: ] if a monk says to a person who is not fully ordained, “I obtain the first absorption,” he commits an offense entailing confession. %
\item[Tells: ] if a monk says to a person who is not fully ordained, “I master the first absorption,” he commits an offense entailing confession. %
\item[Tells: ] if a monk says to a person who is not fully ordained, “I’ve realized the first absorption,” he commits an offense entailing confession. %
\end{description}

\subparagraph*{Other individual attainments }

\begin{description}%
\item[Tells: ] if a monk says to a person who is not fully ordained, “I attained the second absorption … the third absorption … the fourth absorption … I’m attaining … I’ve attained … I obtain … I master … I’ve realized the fourth absorption,” he commits an offense entailing confession. %
\item[Tells: ] if a monk says to a person who is not fully ordained, “I attained the emptiness release … the signless release … the desireless release … the emptiness stillness … the signless stillness … the desireless stillness … I’m attaining … I’ve attained … I obtain … I master … I’ve realized the desireless stillness,” he commits an offense entailing confession. %
\item[Tells: ] if a monk says to a person who is not fully ordained, “I attained the emptiness attainment … the signless attainment … the desireless attainment … I’m attaining … I’ve attained … I obtain … I master … I’ve realized the desireless attainment,” he commits an offense entailing confession. %
\item[Tells: ] if a monk says to a person who is not fully ordained, “I attained the three true insights … I’m attaining … I’ve attained … I obtain … I master … I’ve realized the three true insights,” he commits an offense entailing confession. %
\item[Tells: ] if a monk says to a person who is not fully ordained, “I attained the four applications of mindfulness … the four right efforts … the four foundations for supernormal power … I’m attaining … I’ve attained … I obtain … I master … I’ve realized the four foundations for supernormal power,” he commits an offense entailing confession. %
\item[Tells: ] if a monk says to a person who is not fully ordained, “I attained the five spiritual faculties … the five spiritual powers … I’m attaining … I’ve attained … I obtain … I master … I’ve realized the five spiritual powers,” he commits an offense entailing confession. %
\item[Tells: ] if a monk says to a person who is not fully ordained, “I attained the seven factors of awakening … I’m attaining … I’ve attained … I obtain … I master … I’ve realized the seven factors of awakening,” he commits an offense entailing confession. %
\item[Tells: ] if a monk says to a person who is not fully ordained, “I attained the noble eightfold path … I’m attaining … I’ve attained … I obtain … I master … I’ve realized the noble eightfold path,” he commits an offense entailing confession. %
\item[Tells: ] if a monk says to a person who is not fully ordained, “I attained the fruit of stream-entry … the fruit of once-returning … the fruit of non-returning … perfection … I’m attaining … I’ve attained … I obtain … I master … I’ve realized perfection,” he commits an offense entailing confession. %
\item[Tells: ] if a monk says to a person who is not fully ordained, “I’ve given up sensual desire … I’ve given up ill will … I’ve given up confusion, I’ve renounced it, I’ve let it go, I’ve abandoned it, I’ve relinquished it, I’ve forsaken it, I’ve thrown it aside,” he commits an offense entailing confession. %
\item[Tells: ] if a monk says to a person who is not fully ordained, “My mind is free from the hindrance of sensual desire … My mind is free from the hindrance of ill will … My mind is free from the hindrance of confusion,” he commits an offense entailing confession. %
\item[Tells: ] if a monk says to a person who is not fully ordained, “I attained the first absorption in solitude … the second absorption … the third absorption … the fourth absorption … I’m attaining … I’ve attained … I obtain … I master … I’ve realized the fourth absorption in solitude,” he commits an offense entailing confession. %
\end{description}

\subparagraph*{Combinations of two attainments }

\begin{description}%
\item[Tells: ] if a monk says to a person who is not fully ordained, “I attained the first absorption and the second absorption … I’m attaining … I’ve attained … I obtain … I master … I’ve realized the first absorption and the second absorption,” he commits an offense entailing confession. %
\item[Tells: ] if a monk says to a person who is not fully ordained, “I attained the first absorption and the third absorption … the first absorption and the fourth absorption … I’m attaining … I’ve attained … I obtain … I master … I’ve realized the first absorption and the fourth absorption,” he commits an offense entailing confession. %
\item[Tells: ] if a monk says to a person who is not fully ordained, “I attained the first absorption and the emptiness release … and the signless release … and the desireless release … and the emptiness stillness … and the signless stillness … and the desireless stillness … I’m attaining … I’ve attained … I obtain … I master … I’ve realized the first absorption and the desireless stillness,” he commits an offense entailing confession. %
\item[Tells: ] if a monk says to a person who is not fully ordained, “I attained the first absorption and the emptiness attainment … the signless attainment … the desireless attainment … I’m attaining … I’ve attained … I obtain … I master … I’ve realized the first absorption and the desireless attainment,” he commits an offense entailing confession. %
\item[Tells: ] if a monk says to a person who is not fully ordained, “I attained the first absorption and the three true insights … I’m attaining … I’ve attained … I obtain … I master … I’ve realized the first absorption and the three true insights,” he commits an offense entailing confession. %
\item[Tells: ] if a monk says to a person who is not fully ordained, “I attained the first absorption and the four applications of mindfulness … the four right efforts … the four foundations for supernormal power … I’m attaining … I’ve attained … I obtain … I master … I’ve realized the first absorption and the four foundations for supernormal power,” he commits an offense entailing confession. %
\item[Tells: ] if a monk says to a person who is not fully ordained, “I attained the first absorption and the five spiritual faculties … the five spiritual powers … I’m attaining … I’ve attained … I obtain … I master … I’ve realized the first absorption and the five spiritual powers,” he commits an offense entailing confession. %
\item[Tells: ] if a monk says to a person who is not fully ordained, “I attained the first absorption and the seven factors of awakening … I’m attaining … I’ve attained … I obtain … I master … I’ve realized the first absorption and the seven factors of awakening,” he commits an offense entailing confession. %
\item[Tells: ] if a monk says to a person who is not fully ordained, “I attained the first absorption and the noble eightfold path … I’m attaining … I’ve attained … I obtain … I master … I’ve realized the first absorption and the noble eightfold path,” he commits an offense entailing confession. %
\item[Tells: ] if a monk says to a person who is not fully ordained, “I attained the first absorption and the fruit of stream-entry … and the fruit of once-returning … and the fruit of non-returning … and perfection … I’m attaining … I’ve attained … I obtain … I master … I’ve realized the first absorption and perfection,” he commits an offense entailing confession. %
\item[Tells: ] if a monk says to a person who is not fully ordained, “I attained the first absorption and I’ve given up sensual desire … and I’ve given up ill will … and I’ve given up confusion … I’m attaining … I’ve attained … I obtain … I master … I’ve realized the first absorption and I’ve given up confusion,” he commits an offense entailing confession. %
\item[Tells: ] if a monk says to a person who is not fully ordained, “I attained the first absorption and my mind is free from the hindrance of sensual desire … and my mind is free from the hindrance of ill will … and my mind is free from the hindrance of confusion … I’m attaining … I’ve attained … I obtain … I master … I’ve realized the first absorption and my mind is free from the hindrance of confusion,” he commits an offense entailing confession. %
\end{description}

\begin{description}%
\item[Tells: ] if a monk says to a person who is not fully ordained, “I attained the second absorption and the third absorption … the second absorption and the fourth absorption … I’m attaining … I’ve attained … I obtain … I master … I’ve realized the second absorption and the fourth absorption,” he commits an offense entailing confession. %
\item[Tells: ] if a monk says to a person who is not fully ordained, “I attained the second absorption and the emptiness release … and my mind is free from the hindrance of confusion,” he commits an offense entailing confession. %
\item[Tells: ] if a monk says to a person who is not fully ordained, “I attained the second absorption and the first absorption … I’m attaining … I’ve attained … I obtain … I master … I’ve realized the second absorption and the first absorption,” he commits an offense entailing confession. … %
\end{description}

\scend{The basis in brief is finished.\footnote{See Appendix of Specialized Vocabulary in volume 1 under \textit{\textsanskrit{mūla}} and \textit{\textsanskrit{saṅkhitta}}. } }

\begin{description}%
\item[Tells: ] if a monk says to a person who is not fully ordained, “My mind is free from the hindrance of confusion and I attained the first absorption … I’m attaining … I’ve attained … I obtain … I master … My mind is free from the hindrance of confusion and I’ve realized the first absorption,” he commits an offense entailing confession. … %
\item[Tells: ] if a monk says to a person who is not fully ordained, “My mind is free from the hindrance of confusion and my mind is free from the hindrance of ill will,” he commits an offense entailing confession. … %
\end{description}

\subparagraph*{Combination of all attainments }

\begin{description}%
\item[Tells: ] if a monk says to a person who is not fully ordained, “I attained the first absorption and the second absorption and the third absorption and the fourth absorption and the emptiness release and the signless release and the desireless release and the emptiness stillness and the signless stillness and the desireless stillness and the emptiness attainment and the signless attainment and the desireless attainment and the three true insights and the four applications of mindfulness and the four right efforts and the four foundations for supernormal power and the five spiritual faculties and the five spiritual powers and the seven factors of awakening and the noble eightfold path and the fruit of stream entry and the fruit of once-returning and the fruit of non-returning and perfection … and I’ve given up sensual desire, renounced it, let it go, abandoned it, relinquished it, forsaken it, thrown it aside; and I’ve given up ill will, renounced it, let it go, abandoned it, relinquished it, forsaken it, thrown it aside; and I’ve given up confusion, renounced it, let it go, abandoned it, relinquished it, forsaken it, thrown it aside; and my mind is free from the hindrance of sensual desire; and my mind is free from the hindrance of ill will; and my mind is free from the hindrance of confusion,” he commits an offense entailing confession. %
\end{description}

\subparagraph*{Meaning to say one thing, but saying something else }

\begin{description}%
\item[Tells: ] if a monk means to say to a person who is not fully ordained, “I attained the first absorption,” but actually says, “I attained the second absorption,” then, if the listener understands, he commits an offense entailing confession; if the listener does not understand, he commits an offense of wrong conduct.\footnote{“If the listener understands” renders \textit{\textsanskrit{paṭivijānantassa}}. This is in accordance with Sp 1.219, which has this to say: \textit{Atha pana yassa \textsanskrit{āroceti}, so na \textsanskrit{jānāti} “ki \textsanskrit{ayaṁ} \textsanskrit{bhaṇatī}”ti, \textsanskrit{saṁsayaṁ} \textsanskrit{vā} \textsanskrit{āpajjati}, \textsanskrit{ciraṁ} \textsanskrit{vīmaṁsitvā} \textsanskrit{vā} \textsanskrit{pacchā} \textsanskrit{jānāti}, \textsanskrit{appaṭivijānanto} icceva \textsanskrit{saṅkhyaṁ} gacchati}, “When he who is informed does not understand, thinking, ‘What does he say?’ or he has doubt, or he understands after reflecting for a long time, then it is considered ‘one who does not understand.’” Grammatically \textit{\textsanskrit{paṭivijānantassa}} could refer to either the speaker or the listener (it can be regarded as a genitive agreeing with \textit{\textsanskrit{bhaṇantassa}}, thus referring to the speaker, or it can be regarded as a dative of the person spoken to, that is, the listener), but logically it seems it must refer to the listener. In accordance with common usage, “understanding” (\textit{\textsanskrit{paṭivijānantassa}}) must refer to understanding the overall meaning of what is said, not to knowing the exact words that have been spoken. Since the speaker knows he is lying, he understands the overall meaning. It follows that the understanding here must refer to the listener. A parallel construction is found at \href{https://suttacentral.net/pli-tv-bu-vb-pj1/en/brahmali\#8.4.10}{Bu Pj 1:8.4.10} where \textit{\textsanskrit{paṭivijānāti}} is used in connection with giving up the monastic training. Here the verb clearly refers to understanding on the part of the listener, that is, one has only succeeded in renouncing the training if the listener understands what one is saying. At \href{https://suttacentral.net/pli-tv-bu-vb-ss3/en/brahmali\#4.1.4}{Bu Ss 3:4.1.4} the same verb, this time in the aorist form \textit{\textsanskrit{paṭivijāni}}, again refers to the listener. } %
\item[Tells: ] if a monk means to say to a person who is not fully ordained, “I attained the first absorption,” but actually says, “I attained the third absorption … the fourth absorption …  the emptiness release …  the signless release …  the desireless release …  the emptiness stillness …  the signless stillness …  the desireless stillness …  the emptiness attainment …  the signless attainment …  the desireless attainment …  the three true insights …  the four applications of mindfulness …  the four right efforts …  the four foundations for supernormal power …  the five spiritual faculties …  the five spiritual powers …  the seven factors of awakening …  the noble eightfold path …  the fruit of stream entry …  the fruit of once-returning …  the fruit of non-returning …  perfection … etc. … I’ve given up sensual desire …  I’ve given up ill will …  I’ve given up confusion, renounced it, let it go, abandoned it, relinquished it, forsaken it, thrown it aside …  my mind is free from the hindrance of sensual desire …  my mind is free from the hindrance of ill will …  my mind is free from the hindrance of confusion,” then, if the listener understands, he commits an offense entailing confession; if the listener does not understand, he commits an offense of wrong conduct. %
\item[Tells: ] the fourth absorption …  the emptiness release …  the signless release …  the desireless release …  the emptiness stillness …  the signless stillness …  the desireless stillness …  the emptiness attainment …  the signless attainment …  the desireless attainment …  the three true insights …  the four applications of mindfulness …  the four right efforts …  the four foundations for supernormal power …  the five spiritual faculties …  the five spiritual powers …  the seven factors of awakening …  the noble eightfold path …  the fruit of stream entry …  the fruit of once-returning …  the fruit of non-returning …  perfection … etc. … I’ve given up sensual desire …  I’ve given up ill will …  I’ve given up confusion, renounced it, let it go, abandoned it, relinquished it, forsaken it, thrown it aside …  my mind is free from the hindrance of sensual desire …  my mind is free from the hindrance of ill will …  my mind is free from the hindrance of confusion,” then, if the listener understands, he commits an offense entailing confession; if the listener does not understand, he commits an offense of wrong conduct. %
\item[Tells: ] if a monk means to say to a person who is not fully ordained, “I attained the second absorption” …  but actually says, “My mind is free from the hindrance of confusion,” then, if the listener understands, he commits an offense entailing confession; if the listener does not understand, he commits an offense of wrong conduct. … %
\end{description}

\scend{The basis in brief is finished. }

\begin{description}%
\item[Tells: ] if a monk means to say to a person who is not fully ordained, “My mind is free from the hindrance of confusion,” but actually says, “I attained the first absorption,” then, if the listener understands, he commits an offense entailing confession; if the listener does not understand, he commits an offense of wrong conduct. … %
\item[Tells: ] if a monk means to say to a person who is not fully ordained, “My mind is free from the hindrance of confusion,” but actually says, “My mind is free from the hindrance of ill will,” then, if the listener understands, he commits an offense entailing confession; if the listener does not understand, he commits an offense of wrong conduct. … %
\end{description}

\begin{description}%
\item[Tells: ] if a monk means to say to a person who is not fully ordained, “I attained the first absorption and the second absorption and the third absorption and the fourth absorption … and my mind is free from the hindrance of ill will,” but actually says, “My mind is free from the hindrance of confusion,” then, if the listener understands, he commits an offense entailing confession; if the listener does not understand, he commits an offense of wrong conduct. %
\item[Tells: ] if a monk means to say to a person who is not fully ordained, “I attained the second absorption and the third absorption and the fourth absorption … and my mind is free from the hindrance of confusion,” but actually says, “I attained the first absorption,” then, if the listener understands, he commits an offense entailing confession; if the listener does not understand, he commits an offense of wrong conduct. … %
\end{description}

\subparagraph*{Gross hinting }

\begin{description}%
\item[Tells: ] if a monk says to a person who is not fully ordained, “The monk who stayed in your dwelling attained the first absorption … is attaining … has attained … obtains … masters … has realized the first absorption,” he commits an offense of wrong conduct. %
\item[Tells: ] if a monk says to a person who is not fully ordained, “The monk who stayed in your dwelling attained the second absorption …  etc. … the third absorption …  the fourth absorption … is attaining … has attained … obtains … masters … has realized the fourth absorption,” he commits an offense of wrong conduct. %
\item[Tells: ] if a monk says to a person who is not fully ordained, “The monk who stayed in your dwelling attained the emptiness release … etc. … the signless release …  the desireless release …  the emptiness stillness …  the signless stillness …  the desireless stillness … is attaining … has attained … obtains … masters … has realized the desireless stillness,” he commits an offense of wrong conduct. %
\item[Tells: ] if a monk says to a person who is not fully ordained, “The monk who stayed in your dwelling attained the emptiness attainment … etc. … the signless attainment …  the desireless attainment … is attaining … has attained … obtains … masters … has realized the desireless attainment,” he commits an offense of wrong conduct. %
\item[Tells: ] if a monk says to a person who is not fully ordained, “The monk who stayed in your dwelling attained the three true insights … etc. … the four applications of mindfulness …  the four right efforts …  the four foundations for supernormal power …  the five spiritual faculties …  the five spiritual powers …  the seven factors of awakening …  the noble eightfold path …  the fruit of stream entry …  the fruit of once-returning …  the fruit of non-returning …  perfection … is attaining … has attained … obtains … masters … has realized … etc. … has given up sensual desire …  has given up ill will …  has given up confusion, renounced it, let it go, abandoned it, relinquished it, forsaken it, thrown it aside … has a mind free from the hindrance of sensual desire …  has a mind free from the hindrance of ill will …  has a mind free from the hindrance of confusion,” he commits an offense of wrong conduct. %
\end{description}

\begin{description}%
\item[Tells: ] if a monk says to a person who is not fully ordained, “The monk who stayed in your dwelling attained the first absorption in solitude … etc. … the second absorption … the third absorption … the fourth absorption …  is attaining …  has attained … obtains … masters … has realized the fourth absorption in solitude,” he commits an offense of wrong conduct. %
\item[Tells: ] if a monk says to a person who is not fully ordained, “The monk who made use of your robe-cloth …  who made use of your almsfood …  who made use of your dwelling …  who made use of your medicinal supplies attained the fourth absorption in solitude … is attaining … has attained … obtains … masters … has realized the fourth absorption in solitude,” he commits an offense of wrong conduct. %
\item[Tells: ] if a monk says to a person who is not fully ordained, “The monk who has made use of your dwelling … etc. …  who has made use of your robe-cloth …  who has made use of your almsfood …  who has made use of your furniture …\footnote{Because \textit{\textsanskrit{vihāra}}, “dwelling”, is mentioned just before, I here render \textit{\textsanskrit{senāsana}} as furniture. }  who has made use of your medicinal supplies attained the fourth absorption in solitude … is attaining … has attained … obtains … masters … has realized the fourth absorption in solitude,” he commits an offense of wrong conduct. %
\item[Tells: ] if a monk says to a person who is not fully ordained, “The monk you gave a dwelling to … etc. …  you gave robe-cloth to …  you gave almsfood to …  you gave furniture to …  you gave medicinal supplies to attained the fourth absorption in solitude …  is attaining … has attained … obtains … masters … has realized the fourth absorption in solitude,” he commits an offense of wrong conduct. %
\end{description}

\subsection*{Non-offenses }

There\marginnote{2.6.1} is no offense: if he truthfully tells one who is fully ordained; if he is the first offender. 

\scendsutta{The training rule on telling truthfully, the eighth, is finished. }

%
\section*{{\suttatitleacronym Bu Pc 9}{\suttatitletranslation 9. The training rule on telling about what is grave }{\suttatitleroot Duṭṭhullārocana}}
\addcontentsline{toc}{section}{\tocacronym{Bu Pc 9} \toctranslation{9. The training rule on telling about what is grave } \tocroot{Duṭṭhullārocana}}
\markboth{9. The training rule on telling about what is grave }{Duṭṭhullārocana}
\extramarks{Bu Pc 9}{Bu Pc 9}

\subsection*{Origin story }

At\marginnote{1.1} one time when the Buddha was staying at \textsanskrit{Sāvatthī} in \textsanskrit{Anāthapiṇḍika}’s Monastery, Venerable Upananda the Sakyan was quarreling with the monks from the group of six. 

He\marginnote{1.3} then committed an offense of intentional emission of semen. He asked the Sangha for probation for that offense, which he received. Just then an association in \textsanskrit{Sāvatthī} was offering a meal to the Sangha. Because Upananda was on probation, he sat on the last seat in the dining hall. The monks from the group of six then told those lay followers, “This Venerable Upananda, the esteemed associate of your families, eats the food given in faith with the same hand he uses to masturbate. After committing an offense of intentional emission of semen, he asked the Sangha for probation for that offense, which he received. And because he’s on probation, he now sits on the last seat.” 

The\marginnote{1.12} monks of few desires complained and criticized them, “How could the monks from the group of six tell a person who’s not fully ordained about a monk’s grave offense?” … “Is it true, monks, that you did this?” 

“It’s\marginnote{1.15} true, sir.” 

The\marginnote{1.16} Buddha rebuked them … “Foolish men, how could you do this? This will affect people’s confidence …” … “And, monks, this training rule should be recited like this: 

\subsection*{Final ruling }

\scrule{‘If a monk tells a person who is not fully ordained about a monk’s grave offense, except if the monks have agreed, he commits an offense entailing confession.’” }

\subsection*{Definitions }

\begin{description}%
\item[A: ] whoever … %
\item[Monk: ] … The monk who has been given the full ordination by a unanimous Sangha through a legal procedure consisting of one motion and three announcements that is irreversible and fit to stand—this sort of monk is meant in this case. %
\item[A monk’s: ] another monk’s. %
\item[A grave offense: ] the four offenses entailing expulsion and the thirteen entailing suspension. %
\item[A person who is not fully ordained: ] anyone except a fully ordained monk or a fully ordained nun. %
\item[Tells: ] tells a woman or a man, a lay person or a monastic. %
\item[Except if the monks have agreed: ] unless the monks have agreed. %
\end{description}

\subsection*{Permutations }

\subsubsection*{Permutations part 1 }

\paragraph*{Summary }

There\marginnote{2.1.15.1} is agreement of the monks with a limit on offenses, but not on families. There is agreement of the monks with a limit on families, but not on offenses. There is agreement of the monks with a limit both on offenses and on families. There is agreement of the monks with neither a limit on offenses nor on families. 

\paragraph*{Definitions }

\begin{description}%
\item[With a limit on offenses: ] offenses are specified: “These particular offenses can be told about.” %
\item[With limit on families: ] families are specified: “These particular families can be told.” %
\item[With a limit both on offenses and on families: ] both offenses and families are specified: “These particular offenses can be told about, and these particular families can be told.” %
\item[With neither a limit on offenses nor on families: ] neither offenses nor families are specified in this way. %
\end{description}

\paragraph*{Exposition }

When\marginnote{2.1.25.1} there is a limit on offenses, if he tells about other offenses than those that are specified, he commits an offense entailing confession. 

When\marginnote{2.1.26} there is a limit on families, if he tells other families than those that are specified, he commits an offense entailing confession. 

When\marginnote{2.1.27} there is a limit both on offenses and on families, if he tells about other offenses than those that are specified or he tells other families than those that are specified, he commits an offense entailing confession. 

When\marginnote{2.1.28} there is neither a limit on offenses nor on families, there is no offense. 

\subsubsection*{Permutations part 2 }

If\marginnote{2.2.1} the offense is grave, and he perceives it as such, and he tells a person who is not fully ordained, then, except if the monks have agreed, he commits an offense entailing confession. 

If\marginnote{2.2.2} the offense is grave, but he is unsure of it, and he tells a person who is not fully ordained, then, except if the monks have agreed, he commits an offense entailing confession. 

If\marginnote{2.2.3} the offense is grave, but he perceives it as minor, and he tells a person who is not fully ordained, then, except if the monks have agreed, he commits an offense entailing confession. 

If\marginnote{2.2.4} he tells about a minor offense, he commits an offense of wrong conduct. 

If\marginnote{2.2.5} he tells about the misconduct of a person who is not fully ordained, whether grave or minor, he commits an offense of wrong conduct. 

If\marginnote{2.2.6} the offense is minor, but he perceives it as grave, he commits an offense of wrong conduct. 

If\marginnote{2.2.7} the offense is minor, but he is unsure of it, he commits an offense of wrong conduct. 

If\marginnote{2.2.8} the offense is minor, and he perceives it as such, he commits an offense of wrong conduct. 

\subsection*{Non-offenses }

There\marginnote{2.3.1} is no offense: if he tells about the action that was the basis for the offense, but not the class of offense; if he tells about the class of offense, but not the action that was the basis for the offense; if the monks have agreed; if he is insane; if he is the first offender. 

\scendsutta{The training rule on telling about what is grave, the ninth, is finished. }

%
\section*{{\suttatitleacronym Bu Pc 10}{\suttatitletranslation 10. The training rule on digging the earth }{\suttatitleroot Pathavīkhaṇana}}
\addcontentsline{toc}{section}{\tocacronym{Bu Pc 10} \toctranslation{10. The training rule on digging the earth } \tocroot{Pathavīkhaṇana}}
\markboth{10. The training rule on digging the earth }{Pathavīkhaṇana}
\extramarks{Bu Pc 10}{Bu Pc 10}

\subsection*{Origin story }

At\marginnote{1.1} one time the Buddha was staying at \textsanskrit{Āḷavī} at the \textsanskrit{Aggāḷava} Shrine. At that time the monks of \textsanskrit{Āḷavī} were doing building work, and they dug the earth and had it dug. People complained and criticized them, “How can the Sakyan monastics dig the earth and have it dug? They are hurting one-sensed life.” 

The\marginnote{1.6} monks heard the complaints of those people, and the monks of few desires complained and criticized those monks, “How can those monks at \textsanskrit{Āḷavī} dig the earth and have it dug?” … “Is it true, monks, that you do this?” 

“It’s\marginnote{1.10} true, sir.” 

The\marginnote{1.11} Buddha rebuked them … “Foolish men, how can you do this? People regard the earth as conscious. This will affect people’s confidence …” … “And, monks, this training rule should be recited like this: 

\subsection*{Final ruling }

\scrule{‘If a monk digs the earth or has it dug, he commits an offense entailing confession.’” }

\subsection*{Definitions }

\begin{description}%
\item[A: ] whoever … %
\item[Monk: ] … The monk who has been given the full ordination by a unanimous Sangha through a legal procedure consisting of one motion and three announcements that is irreversible and fit to stand—this sort of monk is meant in this case. %
\item[The earth: ] there are two kinds of earth: productive earth and unproductive earth. %
\item[Productive earth: ] pure soil, pure clay, with few stones, with few pebbles, with few potsherds, with little gravel, with little sand; mostly soil, mostly clay. If it is unburned, it is also called “productive earth”. A pile of soil or clay that has been rained on for more than four months—this too is called “productive earth”. %
\item[Unproductive earth: ] just stones, just pebbles, just potsherds, just gravel, just sand, with little soil, with little clay; mostly stones, mostly pebbles, mostly potsherds, mostly gravel, mostly sand. If it is burned, it is also called “unproductive earth”. A pile of soil or clay that has been rained on for less than four months—this too is called “unproductive earth”. %
\item[Digs: ] if he digs it himself, he commits an offense entailing confession. %
\item[Has dug: ] if he asks another, he commits an offense entailing confession. If he only asks once, then even if the other digs a lot, he commits one offense entailing confession.\footnote{“A lot” renders \textit{\textsanskrit{bahukaṁ}}. Sp 2.86: \textit{\textsanskrit{Sakiṁ} \textsanskrit{āṇatto} bahukampi \textsanskrit{khaṇatīti} sacepi \textsanskrit{sakaladivasaṁ} \textsanskrit{khaṇati}, \textsanskrit{āṇāpakassa} \textsanskrit{ekaṁyeva} \textsanskrit{pācittiyaṁ}}, “\textit{\textsanskrit{Sakiṁ} \textsanskrit{āṇatto} bahukampi \textsanskrit{khaṇatīti}}: there is only one offense entailing confession for the one who asks, even if the other digs the entire day.” } %
\end{description}

\subsection*{Permutations }

If\marginnote{2.2.1} it is earth, and he perceives it as such, and he digs it or has it dug, or he breaks it or has it broken, or he burns it or has it burned, he commits an offense entailing confession. 

If\marginnote{2.2.2} it is earth, but he is unsure of it, and he digs it or has it dug, or he breaks it or has it broken, or he burns it or has it burned, he commits an offense of wrong conduct. 

If\marginnote{2.2.3} it is earth, but he does not perceive it as such, and he digs it or has it dug, or he breaks it or has it broken, or he burns it or has it burned, there is no offense. 

If\marginnote{2.2.4} it is not earth, but he perceives it as such, he commits an offense of wrong conduct. If it is not earth, but he is unsure of it, he commits an offense of wrong conduct. If it is not earth, and he does not perceive it as such, there is no offense. 

\subsection*{Non-offenses }

There\marginnote{2.3.1} is no offense: if he says, “Consider this,” “Give this,” “Bring this,” “There’s need for this,” “Make this allowable;” if it is unintentional; if he is not mindful; if he does not know; if he is insane; if he is the first offender. 

\scendsutta{The training rule on digging the earth, the tenth, is finished. }

\scendvagga{The first subchapter on lying is finished. }

\scuddanaintro{This is the summary: }

\begin{scuddana}%
“Falsely,\marginnote{2.3.11} abusive, and malicious talebearing, \\
Memorizing, and two on beds; \\
Except with one who understands, true, \\
Grave offense, digging.” 

%
\end{scuddana}

%
\section*{{\suttatitleacronym Bu Pc 11}{\suttatitletranslation 11. The training rule on plants }{\suttatitleroot Bhūtagāma}}
\addcontentsline{toc}{section}{\tocacronym{Bu Pc 11} \toctranslation{11. The training rule on plants } \tocroot{Bhūtagāma}}
\markboth{11. The training rule on plants }{Bhūtagāma}
\extramarks{Bu Pc 11}{Bu Pc 11}

\subsection*{Origin story }

At\marginnote{1.1} one time the Buddha was staying at \textsanskrit{Āḷavī} at the \textsanskrit{Aggāḷava} Shrine. At that time the monks of \textsanskrit{Āḷavī} were doing building work, and they cut down trees and had them cut down. Then, when a certain monk was cutting down a tree, the deity that lived in it said to him, “Venerable, don’t cut down our dwelling because you want to build a dwelling for yourself.” Not taking any heed, he just cut it down, and he hurt the arm of that deity’s child. The deity thought, “Why don’t I just kill this monk?” But then it reconsidered, “It wouldn’t be right to just kill this monk. Let me instead tell the Buddha about this matter.” And it approached the Buddha and told him what had happened. 

“Well\marginnote{1.13} done, deity! It’s good that you didn’t kill that monk. If you had killed that monk, you would have made much demerit. The tree over there is empty. Take that as your dwelling.” 

People\marginnote{1.17} complained and criticized the monks, “How can the Sakyan monastics cut down trees and have them cut down? They are hurting life with one sense.” 

The\marginnote{1.19} monks heard the complaints of those people, and the monks of few desires complained and criticized those monks, “How can those monks at \textsanskrit{Āḷavī} cut down trees and have them cut down?”… “Is it true, monks, that you do this?” 

“It’s\marginnote{1.23} true, sir.” 

The\marginnote{1.24} Buddha rebuked them … “Foolish men, how can you do this? People regard trees as conscious. This will affect people’s confidence …” … “And, monks, this training rule should be recited like this: 

\subsection*{Final ruling }

\scrule{‘If a monk destroys a plant, he commits an offense entailing confession.’” }

\subsection*{Definitions }

\begin{description}%
\item[Plant: ] there are five kinds of propagation: propagation from roots, propagation from stems, propagation from joints, propagation from cuttings, propagation from seeds. %
\item[Propagation from roots: ] turmeric, ginger, sweet flag, white sweet flag, atis root, black hellebore, vetiver root, nutgrass, or any other plant produced from roots, that grows from roots—this is called “propagation from roots”.\footnote{For further details of these names and the names below, see Appendix of Plants. } %
\item[Propagation from stems: ] the Bodhi tree, the banyan tree, the Indian rock fig, the cluster fig, the Indian cedar, the portia tree, or any other plant produced from stems, that grows from stems—this is called “propagation from stems”. %
\item[Propagation from joints: ] sugarcane, bamboo, reed, or any other plant produced from joints, that grows from joints—this is called “propagation from joints”. %
\item[Propagation from cuttings: ] shrubby basil, rajmahal hemp, Vicks plant, or any other plant produced from cuttings, that grows from cuttings—this is called “propagation from cuttings”. %
\item[Propagation from seeds: ] grains, vegetables, or any other plant produced from seeds, that grows from seeds—this is called “propagation from seeds”. %
\end{description}

\subsection*{Permutations }

If\marginnote{2.2.1} it is capable of propagation, and he perceives that it is, and he cuts it down or has it cut down, or he breaks it or has it broken, or he cooks it or has it cooked, he commits an offense entailing confession.\footnote{“Capable of propagation” renders \textit{\textsanskrit{bīja}}. Normally \textit{\textsanskrit{bīja}} just means seed, but in the present context living seeds is implied. Thus my rendering. } If it is capable of propagation, but he is unsure of it, and  he cuts it down or has it cut down, or he breaks it or has it broken, or he cooks it or has it cooked, he commits an offense of wrong conduct. If it is capable of propagation, but he perceives that it is not, and he cuts it down or has it cut down, or he breaks it or has it broken, or he cooks it or has it cooked, there is no offense. 

If\marginnote{2.2.4} it is not capable of propagation, but he perceives that it is, he commits an offense of wrong conduct. If it is not capable of propagation, but he is unsure of it, he commits an offense of wrong conduct. If it is not capable of propagation, and he perceives that it is not, there is no offense. 

\subsection*{Non-offenses }

There\marginnote{2.3.1} is no offense: if he says, “Consider this”, “Give this”, “Bring this”, “There’s need for this”, “Make this allowable;” if it is unintentional; if he is not mindful; if he does not know; if he is insane; if he is the first offender. 

\scendsutta{The training rule on plants, the first, is finished. }

%
\section*{{\suttatitleacronym Bu Pc 12}{\suttatitletranslation 12. The training rule on evasive speech }{\suttatitleroot Aññavādaka}}
\addcontentsline{toc}{section}{\tocacronym{Bu Pc 12} \toctranslation{12. The training rule on evasive speech } \tocroot{Aññavādaka}}
\markboth{12. The training rule on evasive speech }{Aññavādaka}
\extramarks{Bu Pc 12}{Bu Pc 12}

\subsection*{Origin story }

\subsubsection*{First sub-story }

At\marginnote{1.1} one time when the Buddha was staying at \textsanskrit{Kosambī} in Ghosita’s Monastery, Venerable Channa was misbehaving. Then, when he was examined about an offense in the midst of the Sangha, he spoke evasively, “Who has committed an offense? What offense was committed? In regard to what was it committed? How was it committed? Who are you talking about? What are you talking about?” 

The\marginnote{1.4} monks of few desires complained and criticized him, “How can Venerable Channa speak evasively when examined about an offense in the midst of the Sangha?”… “Is it true, Channa, that you did this?” 

“It’s\marginnote{1.9} true, sir.” 

The\marginnote{1.10} Buddha rebuked him … “Foolish man, how could you do this? This will affect people’s confidence …” After rebuking him … he gave a teaching and addressed the monks: “Well then, monks, the Sangha should charge Channa with evasive speech. And he is to be charged like this. A competent and capable monk should inform the Sangha: 

‘Please,\marginnote{1.19} venerables, I ask the Sangha to listen. The monk Channa speaks evasively when examined about an offense in the midst of the Sangha. If the Sangha is ready, it should charge him with evasive speech. This is the motion. 

Please,\marginnote{1.23} venerables, I ask the Sangha to listen. The monk Channa speaks evasively when examined about an offense in the midst of the Sangha. The Sangha is charging him with evasive speech. Any monk who approves of charging him with evasive speech should remain silent. Any monk who doesn’t approve should speak up. 

The\marginnote{1.28} Sangha has charged the monk Channa with evasive speech. The Sangha approves and is therefore silent. I’ll remember it thus.’” 

After\marginnote{1.30} rebuking Channa in many ways, the Buddha spoke in dispraise of being difficult to support … “And, monks, this training rule should be recited like this: 

\subsubsection*{Preliminary ruling }

\scrule{‘If a monk speaks evasively, he commits an offense entailing confession.’” }

In\marginnote{1.33} this way the Buddha laid down this training rule for the monks. 

\subsubsection*{Second sub-story }

Later,\marginnote{2.1} when Channa was again being examined about an offense in the midst of the Sangha, he thought, “By speaking evasively I’ll commit an offense,” and he instead harassed the Sangha by remaining silent. 

The\marginnote{2.3} monks of few desires complained and criticized him, “When he’s examined about an offense in the midst of the Sangha, how can Venerable Channa harass the Sangha by remaining silent?” … “Is it true, Channa, that you did this?” 

“It’s\marginnote{2.6} true, sir.” 

The\marginnote{2.7} Buddha rebuked him … “Foolish man, how could you do this? This will affect people’s confidence …” After rebuking him … he gave a teaching and addressed the monks: “Well then, monks, the Sangha should charge Channa with harassment. And he is to be charged like this. A competent and capable monk should inform the Sangha: 

‘Please,\marginnote{2.15} venerables, I ask the Sangha to listen. The monk Channa, when examined about an offense in the midst of the Sangha, harasses the Sangha by remaining silent. If the Sangha is ready, it should charge him with harassment. This is the motion. 

Please,\marginnote{2.19} venerables, I ask the Sangha to listen. The monk Channa, when examined about an offense in the midst of the Sangha, harasses the Sangha by remaining silent. The Sangha is charging him with harassment. Any monk who approves of charging him of harassment should remain silent. Any monk who doesn’t approve should speak up. 

The\marginnote{2.24} Sangha has charged the monk Channa with harassment. The Sangha approves and is therefore silent. I’ll remember it thus.’” 

After\marginnote{2.26} rebuking Channa in many ways, the Buddha spoke in dispraise of being difficult to support … “And so, monks, this training rule should be recited like this: 

\subsection*{Final ruling }

\scrule{‘If a monk speaks evasively or harasses, he commits an offense entailing confession.’”\footnote{There is no equivalent of the “or” in the Pali, but it is implied by the origin story, which gives a separate offense for speaking evasively. } }

\subsection*{Definitions }

\begin{description}%
\item[One who speaks evasively: ] when being examined in the midst of the Sangha about an action that was the basis for an offense or about the class of an offense, he speaks evasively because he does not want to talk about it or reveal it, saying, “Who has committed an offense? What offense was committed? In regard to what was it committed? How was it committed? Who are you talking about? What are you talking about?”—this is called “one who speaks evasively”. %
\item[One who harasses: ] when being examined in the midst of the Sangha about an action that was the basis for an offense or about the class of an offense, he harasses the Sangha by remaining silent because he does not want to talk about it or reveal it—this is called “one who harasses”. %
\end{description}

\subsection*{Permutations }

If\marginnote{3.1.5.1} he has not been charged with evasive speech, but he is being examined in the midst of the Sangha about an action that was the basis for an offense or about the class of an offense, and he then speaks evasively because he does not want to talk about it or reveal it, saying, “Who has committed an offense? What offense was committed? In regard to what was it committed? How was it committed? Who are you talking about? What are you talking about?” then he commits an offense of wrong conduct. 

If\marginnote{3.1.7} he has not been charged with harassment, but he is being examined in the midst of the Sangha about an action that was the basis for an offense or about the class of an offense, and he then harasses the Sangha by remaining silent because he does not want to talk about it or reveal it, then he commits an offense of wrong conduct. 

If\marginnote{3.1.8} he has been charged with evasive speech, and he is being examined in the midst of the Sangha about an action that was the basis for an offense or about the class of an offense, and he then speaks evasively because he does not want to talk about it or reveal it, saying, 

“Who\marginnote{3.1.9} has committed an offense? What offense was committed? In regard to what was it committed? How was it committed? Who are you talking about? What are you talking about?” then he commits an offense entailing confession. 

If\marginnote{3.1.10} he has been charged with harassment, and he is being examined in the midst of the Sangha about an action that was the basis for an offense or about the class of an offense, and he then harasses the Sangha by remaining silent because he does not want to talk about it or reveal it, then he commits an offense entailing confession. 

If\marginnote{3.2.1} it is a legitimate legal procedure, and he perceives it as such, and he speaks evasively or he harasses, he commits an offense entailing confession.\footnote{The legal procedure referred to here is the procedure of charging the monk with evasive speech or harassment of the Sangha. Sp 2.101: \textit{Dhammakamme \textsanskrit{dhammakammasaññītiādīsu} \textsanskrit{yaṁ} \textsanskrit{taṁ} \textsanskrit{aññavādakavihesakaropanakammaṁ} \textsanskrit{kataṁ}}, “In regard to \textit{dhammakamme \textsanskrit{dhammakammasaññī}} etc.: the legal procedure done to charge (a monk) with evasive speech or harassment.” } If it is a legitimate legal procedure, but he is unsure of it, and he speaks evasively or he harasses, he commits an offense entailing confession. If it is a legitimate legal procedure, but he perceives it as illegitimate, and he speaks evasively or he harasses, he commits an offense entailing confession. 

If\marginnote{3.2.4} it is an illegitimate legal procedure, but he perceives it as legitimate, he commits an offense of wrong conduct. If it is an illegitimate legal procedure, but he is unsure of it, he commits an offense of wrong conduct. If it is an illegitimate legal procedure, and he perceives it as such, he commits an offense of wrong conduct. 

\subsection*{Non-offenses }

There\marginnote{3.3.1} is no offense: if he asks because he does not know; if he does not speak because he is sick; if he does not speak because he thinks there will be quarrels or disputes in the Sangha; if he does not speak because he thinks there will be a fracture or schism in the Sangha; if he does not speak because he thinks the legal procedure will be illegitimate, done by an incomplete assembly, or done to one who does not deserve a legal procedure; if he is insane; if he is the first offender. 

\scendsutta{The training rule on evasive speech, the second, is finished. }

%
\section*{{\suttatitleacronym Bu Pc 13}{\suttatitletranslation 13. The training rule on complaining }{\suttatitleroot Ujjhāpanaka}}
\addcontentsline{toc}{section}{\tocacronym{Bu Pc 13} \toctranslation{13. The training rule on complaining } \tocroot{Ujjhāpanaka}}
\markboth{13. The training rule on complaining }{Ujjhāpanaka}
\extramarks{Bu Pc 13}{Bu Pc 13}

\subsection*{Origin story }

\subsubsection*{First sub-story }

At\marginnote{1.1} one time the Buddha was staying at \textsanskrit{Rājagaha} in the Bamboo Grove, the squirrel sanctuary. At that time Venerable Dabba the Mallian was assigning the dwellings and designating the meals, and the monks Mettiya and \textsanskrit{Bhūmajaka} were newly ordained. They had little merit, getting inferior dwellings and meals. They then complained about Dabba to other monks, “Dabba the Mallian assigns dwellings and designates meals based on favoritism.” 

The\marginnote{1.7} monks of few desires complained and criticized them, “How can the monks Mettiya and \textsanskrit{Bhūmajaka} complain about Venerable Dabba to other monks?”… “Is it true, monks, that you do this?” 

“It’s\marginnote{1.10} true, sir.” 

The\marginnote{1.11} Buddha rebuked them … “Foolish men, how can you do this? This will affect people’s confidence …” … “And, monks, this training rule should be recited like this: 

\subsubsection*{Preliminary ruling }

\scrule{‘If a monk complains, he commits an offense entailing confession.’” }

In\marginnote{1.16} this way the Buddha laid down this training rule for the monks. 

\subsubsection*{Second sub-story }

Knowing\marginnote{2.1} that the Buddha had prohibited complaining, the monks Mettiya and \textsanskrit{Bhūmajaka} thought of other ways of getting the monks to hear about their grievances. They then criticized Dabba the Mallian in the vicinity of other monks, “Dabba assigns dwellings and designates meals based on favoritism.” 

The\marginnote{2.4} monks of few desires complained and criticized them, “How can the monks Mettiya and \textsanskrit{Bhūmajaka} criticize Venerable Dabba?”… “Is it true, monks, that you do this?” 

“It’s\marginnote{2.7} true, sir.” 

The\marginnote{2.8} Buddha rebuked them … “Foolish men, how can you do this? This will affect people’s confidence …” … “And so, monks, this training rule should be recited like this: 

\subsection*{Final ruling }

\scrule{‘If a monk complains or criticizes, he commits an offense entailing confession.’”\footnote{There is no equivalent of the “or” in the Pali, but it is implied by the origin story, which gives a separate offense for complaining. } }

\subsection*{Definitions }

\begin{description}%
\item[Complaining: ] when someone who is fully ordained is the assigner of dwellings or the designator of meals or the distributor of congee or the distributor of fruit or the distributor of fresh foods or the distributor of small requisites, and he has been appointed by the Sangha as such, then if a monk complains about him or criticizes him to one who is fully ordained—desiring to disparage him, desiring to give him a bad reputation, desiring to humiliate him—he commits an offense entailing confession.\footnote{According to Sp 2.615 (commenting on the \textit{sekhiya} rules) \textit{khajja}/\textit{khajjaka} refers to all fresh foods: \textit{Ettha \textsanskrit{mūlakhādanīyādi} \textsanskrit{sabbaṁ} \textsanskrit{gahetabbaṁ}}, “Here the fresh foods which are roots, etc., may all be taken.” } %
\end{description}

\subsection*{Permutations }

If\marginnote{3.1.3.1} it is a legitimate legal procedure, and he perceives it as such, and he complains or criticizes, he commits an offense entailing confession.\footnote{The legal procedure referred to here is the procedure of making a monk an officer of the Sangha. Sp 2.106: \textit{Dhammakamme \textsanskrit{dhammakammasaññītiādīsu} \textsanskrit{yaṁ} tassa upasampannassa \textsanskrit{sammutikammaṁ} \textsanskrit{kataṁ}}, “In regard to \textit{dhammakamme \textsanskrit{dhammakammasaññī}} etc.: the legal procedure done to approve a fully ordained person.” } If it is a legitimate legal procedure, but he is unsure of it, and he complains or criticizes, he commits an offense entailing confession. If it is a legitimate legal procedure, but he perceives it as illegitimate, and he complains or criticizes, he commits an offense entailing confession. 

If\marginnote{3.2.1} he complains or criticizes him to one who is not fully ordained, he commits an offense of wrong conduct. 

When\marginnote{3.2.2} someone who is fully ordained is the assigner of dwellings or the designator of meals or the distributor of congee or the distributor of fruit or the distributor of fresh foods or the distributor of small requisites, but he has not been appointed by the Sangha as such, then if a monk complains about him or criticizes him to one who is fully ordained or to one who is not fully ordained—desiring to disparage him, desiring to give him a bad reputation, desiring to humiliate him—he commits an offense of wrong conduct. 

When\marginnote{3.2.3} someone who is not fully ordained is the assigner of dwellings or the designator of meals or the distributor of congee or the distributor of fruit or the distributor of fresh foods or the distributor of small requisites, whether he has been appointed by the Sangha as such or not, then if a monk complains about him or criticizes him to one who is fully ordained or to one who is not fully ordained—desiring to disparage him, desiring to give him a bad reputation, desiring to humiliate him—he commits an offense of wrong conduct. 

If\marginnote{3.2.4} it is an illegitimate legal procedure, but he perceives it as legitimate, he commits an offense of wrong conduct. If it is an illegitimate legal procedure, but he is unsure of it, he commits an offense of wrong conduct. If it is an illegitimate legal procedure, and he perceives it as such, he commits an offense of wrong conduct. 

\subsection*{Non-offenses }

There\marginnote{3.3.1} is no offense: if he complains about or criticizes one who regularly acts out of favoritism, ill will, confusion, or fear; if he is insane; if he is the first offender. 

\scendsutta{The training rule on complaining, the third, is finished. }

%
\section*{{\suttatitleacronym Bu Pc 14}{\suttatitletranslation 14. The training rule on furniture }{\suttatitleroot Mañcasanthārana}}
\addcontentsline{toc}{section}{\tocacronym{Bu Pc 14} \toctranslation{14. The training rule on furniture } \tocroot{Mañcasanthārana}}
\markboth{14. The training rule on furniture }{Mañcasanthārana}
\extramarks{Bu Pc 14}{Bu Pc 14}

\subsection*{Origin story }

\subsubsection*{First sub-story }

At\marginnote{1.1.1} one time when the Buddha was staying at \textsanskrit{Sāvatthī} in \textsanskrit{Anāthapiṇḍika}’s Monastery during winter, the monks put furniture out in the open in order to warm themselves in the sun. But when the time for departure was announced, they departed without putting it away, having it put away, or informing anyone.\footnote{For the meaning of \textit{\textsanskrit{senāsana}}, see Appendix of Technical Terms. } The furniture was rained on. 

The\marginnote{1.1.4} monks of few desires complained and criticized them, “How could those monks put furniture out in the open and then depart without putting it away, having it put away, or informing anyone? The furniture was rained on.” 

After\marginnote{1.1.6} rebuking those monks in many ways, they told the Buddha. Soon afterwards he had the Sangha gathered and questioned the monks: “Is it true, monks, that you did this?” … “And, monks, this training rule should be recited like this: 

\subsection*{Final ruling }

\scrule{‘If a monk takes a bed, a bench, a mattress, or a stool belonging to the Sangha and puts it out in the open or has it put out in the open, and he then departs without putting it away, having it put away, or informing anyone, he commits an offense entailing confession.’” }

In\marginnote{1.1.10} this way the Buddha laid down this training rule for the monks. 

\subsubsection*{Second sub-story }

Soon\marginnote{1.2.1} afterwards, monks who had stayed out in the open brought the furniture back even though it was not the rainy season. The Buddha saw this. After giving a teaching, he addressed the monks: 

\scrule{“Monks, during the eight months outside of the rainy season, I allow you to store furniture under a roof cover or at the foot of a tree or wherever crows or ravens don’t leave droppings.” }

\subsection*{Definitions }

\begin{description}%
\item[A: ] whoever … %
\item[Monk: ] …The monk who has been given the full ordination by a unanimous Sangha through a legal procedure consisting of one motion and three announcements that is irreversible and fit to stand—this sort of monk is meant in this case. %
\item[Belonging to the Sangha: ] given to the Sangha, given up to the Sangha. %
\item[A bed: ] there are four kinds of beds: one with legs and frame, called \textit{\textsanskrit{masāraka}}; one with legs and frame, called \textit{\textsanskrit{bundikābaddha}}; one with crooked legs; one with detachable legs.\footnote{For a discussion of these and those below, see Appendix of Furniture. } %
\item[A bench: ] there are four kinds of benches: one with legs and frame, called \textit{\textsanskrit{masāraka}}; one with legs and frame, called \textit{\textsanskrit{bundikābaddha}}; one with crooked legs; one with detachable legs. %
\item[A mattress: ] there are five kinds of mattresses: a mattress stuffed with wool, a mattress stuffed with cloth, a mattress stuffed with bark, a mattress stuffed with grass, a mattress stuffed with leaves. %
\item[A stool: ] one made of bark, one made of vetiver grass, one made of reed. It is upholstered and then bound together.\footnote{I have rendered \textit{\textsanskrit{muñja}}-reed and \textit{pabbaja}-reed with the single word “reed”. I am not aware that these different species of reed can be properly distinguished in English. } %
\item[Puts it: ] puts it oneself. %
\item[Has it put: ] gets another to put it. If he gets one who is not fully ordained to put it, it is the responsibility of the monk. If he gets one who is fully ordained to put it, it is the responsibility of the one who puts it. %
\item[Departs without putting it away: ] he does not put it away himself. %
\item[Having it put away: ] he does not get another to put it away. %
\item[Or informing anyone: ] if he does not inform a monk, a novice monk, or a monastery worker, then when he goes beyond the distance of a stone’s throw of an average man, he commits an offense entailing confession. %
\end{description}

\subsection*{Permutations }

If\marginnote{2.2.1} it belongs to the Sangha, and he perceives it as such, and he puts it out in the open or has it put out in the open, and he then departs without putting it away or having it put away or informing anyone, he commits an offense entailing confession.\footnote{“Out in the open” renders \textit{\textsanskrit{ajjhokāsa}}. See Appendix of Technical Terms for discussion. } If it belongs to the Sangha, but he is unsure of it … If it belongs to the Sangha, but he perceives it as belonging to an individual, and he puts it out in the open or has it put out in the open, and he then departs without putting it away or having it put away or informing anyone, he commits an offense entailing confession. 

If\marginnote{2.2.4} it is a mat underlay, a bedspread, a floor cover, a straw mat, a hide, a foot-wiping cloth, or a plank bench, and he puts it out in the open or has it put out in the open, and he then departs without putting it away or having it put away or informing anyone, he commits an offense of wrong conduct.\footnote{Sp-yoj 2.112: \textit{\textsanskrit{Cammakhaṇḍoti} ettha \textsanskrit{cammaṁyeva} ante \textsanskrit{khaṇḍattā} \textsanskrit{chinnattā} \textsanskrit{cammakhaṇḍoti} vuccati}, “Here \textit{\textsanskrit{cammakhaṇḍa}} is just a hide. Because it is cut up within and made of pieces, it called a \textit{\textsanskrit{cammakhaṇḍa}}.” For the meaning of \textit{\textsanskrit{cimilikā}}, “a mat underlay”, see Appendix of Furniture. } 

If\marginnote{2.2.5} it belongs to an individual, but he perceives it as belonging to the Sangha, he commits an offense of wrong conduct. If it belongs to an individual, but he is unsure of it, he commits an offense of wrong conduct. If it belongs to an individual, and he perceives it as such, but that individual is not himself, he commits an offense of wrong conduct. If it belongs to himself, there is no offense. 

\subsection*{Non-offenses }

There\marginnote{2.3.1} is no offense: if he departs after putting it away; if he departs after having it put away; if he departs after informing someone; if he departs while he is sunning it; if the furniture is obstructed;\footnote{Sp 2.113: \textit{Kenaci \textsanskrit{palibuddhaṁ} \textsanskrit{hotīti} \textsanskrit{senāsanaṁ} kenaci \textsanskrit{upaddutaṁ} \textsanskrit{hotīti} attho}, “\textit{Kenaci \textsanskrit{palibuddhaṁ} hoti}: the meaning is that the furniture is controlled by something.” } if there is an emergency;\footnote{“Emergency” renders \textit{\textsanskrit{āpadāsu}}. See Appendix of Technical Terms for discussion. } if he is insane; if he is the first offender. 

\scendsutta{The training rule on furniture, the fourth, is finished. }

%
\section*{{\suttatitleacronym Bu Pc 15}{\suttatitletranslation 15. The second training rule on furniture }{\suttatitleroot Seyyasanthārana}}
\addcontentsline{toc}{section}{\tocacronym{Bu Pc 15} \toctranslation{15. The second training rule on furniture } \tocroot{Seyyasanthārana}}
\markboth{15. The second training rule on furniture }{Seyyasanthārana}
\extramarks{Bu Pc 15}{Bu Pc 15}

\subsection*{Origin story }

At\marginnote{1.1} one time the Buddha was staying at \textsanskrit{Sāvatthī} in the Jeta Grove, \textsanskrit{Anāthapiṇḍika}’s Monastery. At that time the monks from the group of seventeen were friends. The lived together, and when going somewhere they would leave together. On one occasion they put out bedding in a dwelling belonging to the Sangha, but then departed without putting it away, having it put away, or informing anyone. The furniture was eaten by termites.\footnote{For the meaning of \textit{\textsanskrit{senāsana}}, see Appendix of Technical Terms. } 

The\marginnote{1.6} monks of few desires complained and criticized them, “How could those monks from the group of seventeen put out bedding in a dwelling belonging to the Sangha, and then depart without putting it away, getting it put away, or informing anyone? The furniture was eaten by termites.” 

After\marginnote{1.8} rebuking those monks in many ways, they told the Buddha. Soon afterwards he had the Sangha gathered and questioned the monks: “Is it true, monks, that those monks did this?” 

“It’s\marginnote{1.10} true, sir.” 

The\marginnote{1.11} Buddha rebuked them … “How could those foolish men do this? This will affect people’s confidence …” … “And, monks, this training rule should be recited like this: 

\subsection*{Final ruling }

\scrule{‘If a monk puts out bedding in a dwelling belonging to the Sangha, or has it put out, and he then departs without putting it away, having it put away, or informing anyone, he commits an offense entailing confession.’” }

\subsection*{Definitions }

\begin{description}%
\item[A: ] whoever … %
\item[Monk: ] …The monk who has been given the full ordination by a unanimous Sangha through a legal procedure consisting of one motion and three announcements that is irreversible and fit to stand—this sort of monk is meant in this case. %
\item[A dwelling belonging to the Sangha:\footnote{I render \textit{\textsanskrit{vihāra}} as “dwelling”, the idea that it is a monastic dwelling being implied. In later usage, especially in the commentaries, \textit{\textsanskrit{vihāra}} comes to refer to entire monasteries, rather than individual dwellings. The commentaries seem to agree that in its early usage the word refers to a dwelling. Sp 1.493: \textit{\textsanskrit{Vihāro} nivesanasadiso}, “A \textit{\textsanskrit{vihāra}} is like a house.” } ] given to the Sangha, given up to the Sangha. %
\item[Bedding: ] a mattress, a mat underlay, a bedspread, a floor cover, a straw mat, a hide, a sitting mat, a sheet, a spread of grass, a spread of leaves.\footnote{Sp-yoj 2.112: \textit{\textsanskrit{Cammakhaṇḍoti} ettha \textsanskrit{cammaṁyeva} ante \textsanskrit{khaṇḍattā} \textsanskrit{chinnattā} \textsanskrit{cammakhaṇḍoti} vuccati}, “Here \textit{\textsanskrit{cammakhaṇḍa}} is just a hide. Because it is cut up within and made of pieces, it is called a \textit{\textsanskrit{cammakhaṇḍa}}.” } %
\item[Puts out: ] puts out oneself. %
\item[Has put out: ] gets another to put out. %
\item[Departs without putting it away: ] he does not put it away himself. %
\item[Having it put away: ] he does not get another to put it away. %
\item[Or informing anyone: ] if he does not inform a monk, a novice monk, or a monastery worker, and he crosses the boundary of an enclosed monastery, he commits an offense entailing confession. If he goes beyond the vicinity of an unenclosed monastery, he commits an offense entailing confession. %
\end{description}

\subsection*{Permutations }

If\marginnote{2.2.1} it belongs to the Sangha, and he perceives it as such, and he puts out bedding there or has it put out, and he then departs without putting it away or having it put away or informing anyone, he commits an offense entailing confession. If it belongs to the Sangha, but he is unsure of it, and he puts out bedding there or has it put out, and he then departs without putting it away or having it put away or informing anyone, he commits an offense entailing confession. If it belongs to the Sangha, but he perceives it as belonging to an individual, and he puts out bedding there or has it put out, and he then departs without putting it away or having it put away or informing anyone, he commits an offense entailing confession. 

If\marginnote{2.2.4} he puts out bedding, or has it put out, in the vicinity of a dwelling, in the assembly hall, under a roof-cover, or at the foot of a tree, and he then departs without putting it away or having it put away or informing anyone, he commits an offense of wrong conduct. If he puts out a bed or a bench, or has it put out, in a dwelling, in the vicinity of a dwelling, in the assembly hall, under a roof cover, or at the foot of a tree, and he then departs without putting it away or having it put away or informing anyone, he commits an offense of wrong conduct. 

If\marginnote{2.2.6} it belongs to an individual, but he perceives it as belonging to the Sangha, he commits an offense of wrong conduct. If it belongs to an individual, but he is unsure of it, he commits an offense of wrong conduct. If it belongs to an individual, and he perceives it as such, but that individual is not himself, he commits an offense of wrong conduct. If it belongs to himself, there is no offense. 

\subsection*{Non-offenses }

There\marginnote{2.3.1} is no offense: if he departs after putting it away; if he departs after having it put away; if he departs after informing someone; if the bedding is obstructed;\footnote{Sp 2.118: \textit{Kenaci \textsanskrit{palibuddhaṁ} \textsanskrit{hotīti} \textsanskrit{vuḍḍhatarabhikkhūissariyayakkhasīhavāḷamigakaṇhasappādīsu} yena kenaci \textsanskrit{senāsanaṁ} \textsanskrit{palibuddhaṁ} hoti}, “\textit{Kenaci \textsanskrit{palibuddhaṁ} hoti}: by whatever the furniture is obstructed, such as the authority of more senior monks, a spirit, a lion, a wild beast, or a black snake.” } if he abandons his intention to return, and at that spot informs anyone; if he is obstructed; if there is an emergency; if he is insane; if he is the first offender. 

\scendsutta{The second training rule on furniture, the fifth, is finished. }

%
\section*{{\suttatitleacronym Bu Pc 16}{\suttatitletranslation 16. The training rule on encroaching }{\suttatitleroot Anupakhajja}}
\addcontentsline{toc}{section}{\tocacronym{Bu Pc 16} \toctranslation{16. The training rule on encroaching } \tocroot{Anupakhajja}}
\markboth{16. The training rule on encroaching }{Anupakhajja}
\extramarks{Bu Pc 16}{Bu Pc 16}

\subsection*{Origin story }

At\marginnote{1.1} one time when the Buddha was staying at \textsanskrit{Sāvatthī} in \textsanskrit{Anāthapiṇḍika}’s Monastery, the monks from the group of six had taken possession of the best sleeping places. When the the senior monks evicted them, they thought, “How can we get to stay here during the rainy season?” They then arranged their sleeping places so as to encroach on the senior monks, thinking, “Whoever feels crowded will leave.” 

The\marginnote{1.7} monks of few desires complained and criticized them, “How could the monks from the group of six arrange their sleeping places so as to encroach on the senior monks?” 

After\marginnote{1.9} rebuking those monks in many ways, they told the Buddha. Soon afterwards he had the Sangha gathered and questioned the monks: “Is it true, monks, that you did this?” 

“It’s\marginnote{1.11} true, sir.” 

The\marginnote{1.12} Buddha rebuked them … “Foolish men, how could you do this? This will affect people’s confidence …” … “And, monks, this training rule should be recited like this: 

\subsection*{Final ruling }

\scrule{‘If, in a dwelling belonging to the Sangha, a monk arranges his sleeping place in a way that encroaches on a monk that he knows arrived there before him, with the intention that anyone who feels crowded will leave, and he does so only for this reason and no other, he commits an offense entailing confession.’” }

\subsection*{Definitions }

\begin{description}%
\item[A: ] whoever … %
\item[Monk: ] …The monk who has been given the full ordination by a unanimous Sangha through a legal procedure consisting of one motion and three announcements that is irreversible and fit to stand—this sort of monk is meant in this case. %
\item[A dwelling belonging to the Sangha: ] given to the Sangha, given up to the Sangha. %
\item[He knows: ] he knows that he is senior, he knows that he is sick, he knows that it was given to him by the Sangha. %
\item[To encroach on: ] enters after. %
\item[Arranges his sleeping place: ] if he puts out his sleeping place at the access to the bed, the bench, the entrance, or the exit, he commits an offense of wrong conduct. If he sits down or lies down on it, he commits an offense entailing confession. %
\item[He does so only for this reason and no other: ] there is no other reason for arranging his sleeping place in a way that encroaches. %
\end{description}

\subsection*{Permutations }

If\marginnote{2.2.1} it belongs to the Sangha, and he perceives it as such, and he arranges his sleeping place there in a way that encroaches, he commits an offense entailing confession. If it belongs to the Sangha, but he is unsure of it, and he arranges his sleeping place there in a way that encroaches, he commits an offense entailing confession. If it belongs to the Sangha, but he perceives it as belonging to an individual, and he arranges his sleeping place there in a way that encroaches, he commits an offense entailing confession. 

If\marginnote{2.2.4} he puts out his sleeping place, or has it put out, anywhere apart from the access to the bed, the bench, the entrance, or the exit, he commits an offense of wrong conduct. If he sits down or lies down on it, he commits an offense of wrong conduct. If he puts out his sleeping place, or has it put out, in the vicinity of a dwelling, in an assembly hall, under a roof cover, at the foot of a tree, or out in the open, he commits an offense of wrong conduct. If he sits down or lies down on it, he commits an offense of wrong conduct. 

If\marginnote{2.2.8} it belongs to an individual, but he perceives it as belonging to the Sangha, he commits an offense of wrong conduct. If it belongs to an individual, but he is unsure of it, he commits an offense of wrong conduct. If it belongs to an individual, and he perceives it as such, but that individual is not himself, he commits an offense of wrong conduct. If it belongs to himself, there is no offense. 

\subsection*{Non-offenses }

There\marginnote{2.3.1} is no offense: if he enters because he is sick; if he enters because he is feeling cold or hot; if there is an emergency; if he is insane; if he is the first offender. 

\scendsutta{The training rule on encroaching, the sixth, is finished. }

%
\section*{{\suttatitleacronym Bu Pc 17}{\suttatitletranslation 17. The training rule on throwing out }{\suttatitleroot Nikkaḍḍhana}}
\addcontentsline{toc}{section}{\tocacronym{Bu Pc 17} \toctranslation{17. The training rule on throwing out } \tocroot{Nikkaḍḍhana}}
\markboth{17. The training rule on throwing out }{Nikkaḍḍhana}
\extramarks{Bu Pc 17}{Bu Pc 17}

\subsection*{Origin story }

At\marginnote{1.1} one time the Buddha was staying at \textsanskrit{Sāvatthī} in the Jeta Grove, \textsanskrit{Anāthapiṇḍika}’s Monastery. At that time the monks from the group of seventeen were repairing a large dwelling nearby, intending to stay there for the rainy season. The monks from the group of six saw this and said, “These monks from the group of seventeen are repairing a dwelling. Let’s throw them out.” But some of them said, “Let’s wait until they’ve finished repairing it.” 

Soon\marginnote{1.10} afterwards the monks from the group of six said to those from the group of seventeen, “Leave, this dwelling is ours.” 

“Shouldn’t\marginnote{1.12} you have told us beforehand? We would’ve repaired another one.” 

“Doesn’t\marginnote{1.13} this dwelling belong to the Sangha?” 

“Yes,\marginnote{1.14} it does.” 

“Well\marginnote{1.15} then, leave! This dwelling is ours.” 

“The\marginnote{1.16} dwelling is large. You can stay here and so can we.” 

But\marginnote{1.18} they said, “Leave, this dwelling is ours,” and they grabbed them by the necks and threw them out in anger. 

The\marginnote{1.19} monks from the group of seventeen cried. When other monks asked them why, they told them what had happened. 

The\marginnote{1.22} monks of few desires complained and criticized them, “How could the monks from the group of six angrily throw other monks out of a dwelling belonging to the Sangha?” 

After\marginnote{1.24} rebuking those monks in many ways, they told the Buddha. Soon afterwards he had the Sangha gathered and questioned the monks: “Is it true, monks, that you did this?” 

“It’s\marginnote{1.26} true, sir.” 

The\marginnote{1.27} Buddha rebuked them … “Foolish men, how could you do this? This will affect people’s confidence …” … “And, monks, this training rule should be recited like this: 

\subsection*{Final ruling }

\scrule{‘If a monk, in anger, throws a monk out of a dwelling belonging to the Sangha, or has him thrown out, he commits an offense entailing confession.’” }

\subsection*{Definitions }

\begin{description}%
\item[A: ] whoever … %
\item[Monk: ] …The monk who has been given the full ordination by a unanimous Sangha through a legal procedure consisting of one motion and three announcements that is irreversible and fit to stand—this sort of monk is meant in this case. %
\item[A monk: ] another monk. %
\item[In anger: ] discontent, having hatred, hostile. %
\item[A dwelling belonging to the Sangha: ] given to the Sangha, given up to the Sangha. %
\item[Throws out: ] if he takes hold of him in a room and throws him out to the entryway, he commits an offense entailing confession. If he takes hold of him in the entryway and throws him outside, he commits an offense entailing confession. Even if he makes him go through many doors with a single effort, he commits one offense entailing confession. %
\item[Has thrown out: ] if he asks another, he commits an offense entailing confession. If he only asks once, then even if the other makes him go through many doors, he commits one offense entailing confession. %
\end{description}

\subsection*{Permutations }

If\marginnote{2.2.1} it belongs to the Sangha, and he perceives it as such, and in anger he throws him out, or has him thrown out, he commits an offense entailing confession. If it belongs to the Sangha, but he is unsure of it, and in anger he throws him out, or has him thrown out, he commits an offense entailing confession. If it belongs to the Sangha, but he perceives it as belonging to an individual, and in anger he throws him out, or has him thrown out, he commits an offense entailing confession. 

If\marginnote{2.2.4} he throws out one of his requisites, or he has it thrown out, he commits an offense of wrong conduct. If he throws him out, or has him thrown out, from the vicinity of a dwelling, from an assembly hall, from under a roof cover, from the foot of a tree, or from a space out in the open, he commits an offense of wrong conduct. If he throws out one of his requisites from any of these places, or he has it thrown out, he commits an offense of wrong conduct. If he throws out one who is not fully ordained, or has them thrown out, from a dwelling, from the vicinity of a dwelling, from an assembly hall, from under a roof cover, from the foot of a tree, or from a space out in the open, he commits an offense of wrong conduct. If he throws out one of their requisites from any of these places, or has it thrown out, he commits an offense of wrong conduct. 

If\marginnote{2.2.9} it belongs to an individual, but he perceives it as belonging to the Sangha, he commits an offense of wrong conduct. If it belongs to an individual, but he is unsure of it, he commits an offense of wrong conduct. If it belongs to an individual, and he perceives it as such, but that individual is not himself, he commits an offense of wrong conduct. If it belongs to himself, there is no offense. 

\subsection*{Non-offenses }

There\marginnote{2.3.1} is no offense: if he throws out one who is shameless, or has him thrown out; if he throws out one of the requisites belonging to that person, or has them thrown out; if he throws out one who is insane, or has him thrown out; if he throws out one of the requisites belonging to that person, or has them thrown out; if he throws out one who is quarrelsome and argumentative, who creates legal issues in the Sangha, or has them thrown out; if he throws out one of the requisites belonging to that person, or has them thrown out; if he throws out a pupil or student who is not conducting himself properly, or has him thrown out; if he throws out one of the requisites belonging to such a person, or has them thrown out; if he is insane; if he is the first offender. 

\scendsutta{The training rule on throwing out, the seventh, is finished. }

%
\section*{{\suttatitleacronym Bu Pc 18}{\suttatitletranslation 18. The training rule on upper stories }{\suttatitleroot Vehāsakuṭi}}
\addcontentsline{toc}{section}{\tocacronym{Bu Pc 18} \toctranslation{18. The training rule on upper stories } \tocroot{Vehāsakuṭi}}
\markboth{18. The training rule on upper stories }{Vehāsakuṭi}
\extramarks{Bu Pc 18}{Bu Pc 18}

\subsection*{Origin story }

At\marginnote{1.1} one time the Buddha was staying at \textsanskrit{Sāvatthī} in the Jeta Grove, \textsanskrit{Anāthapiṇḍika}’s Monastery. At that time two monks were staying in a dwelling with an upper story belonging to the Sangha, one staying below and one above. The monk above sat down hastily on a bed with detachable legs. A leg fell off and hit the monk below on the head. He cried out. Monks rushed up and asked him why, and he told them what had happened. 

The\marginnote{1.9} monks of few desires complained and criticized him, 

“How\marginnote{1.10} could a monk sit down hastily on a bed with detachable legs on an upper story in a dwelling belonging to the Sangha?” 

After\marginnote{1.11} rebuking that monk in many ways, they told the Buddha. Soon afterwards he had the Sangha gathered and questioned the monks: “Is it true, monk, that you did this?” 

“It’s\marginnote{1.13} true, sir.” 

The\marginnote{1.14} Buddha rebuked him … “Foolish man, how could you do this? This will affect people’s confidence …” … “And, monks, this training rule should be recited like this: 

\subsection*{Final ruling }

\scrule{‘If a monk sits down or lies down on a bed or a bench with detachable legs on an upper story in a dwelling belonging to the Sangha, he commits an offense entailing confession.’”\footnote{I am here taking \textit{\textsanskrit{kuṭi}} to refer to the upper story itself, rather than the whole building, which already has the separate name \textit{\textsanskrit{vihāra}}. } }

\subsection*{Definitions }

\begin{description}%
\item[A: ] whoever … %
\item[Monk: ] …The monk who has been given the full ordination by a unanimous Sangha through a legal procedure consisting of one motion and three announcements that is irreversible and fit to stand—this sort of monk is meant in this case. %
\item[A dwelling belonging to the Sangha: ] given to the Sangha, given up to the Sangha. %
\item[An upper story: ] a man of average height does not hit his head. %
\item[A bed with detachable legs: ] it stands after inserting the limbs. %
\item[A bench with detachable legs: ] it stands after inserting the limbs. %
\item[Sits down: ] if he sits down on it, he commits an offense entailing confession. %
\item[Lies down: ] if he lies down on it, he commits an offense entailing confession. %
\end{description}

\subsection*{Permutations }

If\marginnote{2.2.1} it belongs to the Sangha, and he perceives it as such, and he sits down or lies down on a bed or a bench with detachable legs on an upper story, he commits an offense entailing confession. If it belongs to the Sangha, but he is unsure of it … If it belongs to the Sangha, but he perceives it as belonging to an individual, and he sits down or lies down on a bed or a bench with detachable legs on an upper story, he commits an offense entailing confession. 

If\marginnote{2.2.4} it belongs to an individual, but he perceives it as belonging to the Sangha, he commits an offense of wrong conduct. If it belongs to an individual, but he is unsure of it, he commits an offense of wrong conduct. If it belongs to an individual, and he perceives it as such, but that individual is not himself, he commits an offense of wrong conduct. If it belongs to himself, there is no offense. 

\subsection*{Non-offenses }

There\marginnote{2.3.1} is no offense: if there is no upper story; if the upper story is so low that one hits the head; if the lower story is not in use; if the upper story has floorboards; if the legs are fastened by bolts; if he stands on it to get hold of something or to put something up;\footnote{Sp 2.133: \textit{\textsanskrit{Tasmiṁ} \textsanskrit{ṭhitoti} \textsanskrit{āhaccapādake} \textsanskrit{mañce} \textsanskrit{vā} \textsanskrit{pīṭhe} \textsanskrit{vā} \textsanskrit{ṭhito} upari \textsanskrit{nāgadantakādīsu} \textsanskrit{laggitakaṁ} \textsanskrit{cīvaraṁ} \textsanskrit{vā} \textsanskrit{kiñci} \textsanskrit{vā} \textsanskrit{gaṇhāti} \textsanskrit{vā}, \textsanskrit{aññaṁ} \textsanskrit{vā} laggeti, \textsanskrit{tassāpi} \textsanskrit{anāpatti}}, “\textit{\textsanskrit{Tasmiṁ} \textsanskrit{ṭhito}}: standing on the bed or bench with detachable legs, he takes hold of a robe or anything else hanging from a wall peg, etc., above, or he puts something up; for this too there is no offense.” } if he is insane; if he is the first offender. 

\scendsutta{The training rule on upper stories, the eighth, is finished. }

%
\section*{{\suttatitleacronym Bu Pc 19}{\suttatitletranslation 19. The training rule on large dwellings }{\suttatitleroot Mahallakavihāra}}
\addcontentsline{toc}{section}{\tocacronym{Bu Pc 19} \toctranslation{19. The training rule on large dwellings } \tocroot{Mahallakavihāra}}
\markboth{19. The training rule on large dwellings }{Mahallakavihāra}
\extramarks{Bu Pc 19}{Bu Pc 19}

\subsection*{Origin story }

At\marginnote{1.1} one time when the Buddha was staying at \textsanskrit{Kosambī} in Ghosita’s Monastery, a government official who was Venerable Channa’s supporter was making him a dwelling. When the dwelling was finished, Channa had it roofed and plastered over and over. Being overloaded, the dwelling collapsed. Then, while collecting grass and sticks, Channa spoiled the barley field belonging to a certain brahmin. That brahmin complained and criticized him, “How can the venerables spoil my barley field?” 

The\marginnote{1.8} monks heard the complaints of that brahmin, and the monks of few desires complained and criticized Channa, “How could Venerable Channa have a finished dwelling roofed and plastered over and over until it collapsed from overloading?” 

After\marginnote{1.11} rebuking him in many ways, they told the Buddha. Soon afterwards he had the Sangha gathered and questioned Channa: “Is it true, Channa, that you did this?” 

“It’s\marginnote{1.13} true, sir.” 

The\marginnote{1.14} Buddha rebuked him … “Foolish man, how could you do this? This will affect people’s confidence …” … “And, monks, this training rule should be recited like this: 

\subsection*{Final ruling }

\scrule{‘When a monk is building a large dwelling, then standing where there are no cultivated plants, he may apply two or three layers of roofing material, taking it as far as the doorcase and using it for fixing the door and for treating the window openings. If he applies more than that, even if he stands where there are no cultivated plants, he commits an offense entailing confession.’” }

\subsection*{Definitions }

\begin{description}%
\item[A large dwelling: ] one with an owner is what is meant. %
\item[A dwelling: ] plastered inside or plastered outside or plastered both inside and outside. %
\item[Is building: ] building it himself or having it built. %
\item[As far as the doorcase: ] a distance of an arm’s reach next to the door frame. %
\item[For fixing the door:\footnote{Sp 3.135: \textit{\textsanskrit{Aggaḷaṭṭhapanāyāti} \textsanskrit{sakavāṭakadvārabandhaṭṭhapanāya}; \textsanskrit{sakavāṭakassa} \textsanskrit{dvārabandhassa} \textsanskrit{niccalabhāvatthāyāti} attho}, “\textit{\textsanskrit{Aggaḷaṭṭhapanāya}}: for the fixing of the doorframe together with the door panel. It is for the purpose of making the doorframe together with the door panel stable: this is the meaning.” } ] for the fixing of the door. %
\item[For treating the window openings: ] for treating the window openings there is white color, black color, and treating with red ocher; and there is making a garland pattern, a creeper pattern, a shark-teeth pattern, and the fivefold pattern.\footnote{“A shark-teeth pattern” renders \textit{makaradantaka}, literally, “with teeth like a \textit{makara}”. In later Buddhism the \textit{makara} is the name of a mythological marine animal, but what it refers to in this context is not clear. According to Sp-yoj 4.243 \textit{makara} is the name of a certain species of fish: \textit{Makaradantaketi \textsanskrit{makaranāmakassa} macchassa dantasadise dante}, “Teeth like the teeth of a fish called \textit{makara}.” Vin-vn-\textsanskrit{ṭ} 3048, \textit{Makaradantakanti \textsanskrit{girikūṭākāraṁ}}, “\textit{Makaradantaka} means making (a design) like the peak of a hill.” PED suggests “the tooth of a swordfish”, but apparently swordfish do not have teeth. Given that the \textit{makara} were fearsome creatures and that their teeth looked like the peak of a hill, presumably meaning that their teeth were pointed, “shark teeth” seems like a reasonable approximation. | “The fivefold pattern” renders \textit{\textsanskrit{pañcapaṭika}}. Vmv 4.299: \textit{\textsanskrit{Pāḷiyaṁ} \textsanskrit{pañcapaṭikanti} \textsanskrit{jātiādipañcappakāravaṇṇamaṭṭhaṁ}}, “\textit{\textsanskrit{Pañcapaṭika}} in the canonical text means treated with the color of the five kinds, starting with jasmine.” The meaning is not clear. It seems unlikely, however, that it should refer to colors, since such have just been listed. } %
\item[Standing where there are no cultivated plants, he may apply two or three layers of roofing material: ] cultivated plants: grain and vegetables; if he applies it while standing where there are cultivated plants, he commits an offense of wrong conduct. For someone covering by the line, after covering with two layers, he may ask for a third layer, and he should then leave.\footnote{“By the line” renders \textit{maggena}. Sp 2.136: \textit{\textsanskrit{Aparikkhipitvā} ujukameva \textsanskrit{chādanaṁ}}, “Not encircling, just straight covering.” As for “two layers”, \textit{dve magge}, the context requires that this refers to lines on top of each other rather than next to each other. The implied purpose of “leave”, \textit{\textsanskrit{pakkamitabbaṁ}}, according to Sp 2.136, is to show that he may not keep on asking: \textit{Sace na pakkamati, \textsanskrit{tuṇhībhūtena} \textsanskrit{ṭhātabbaṁ}}, “If he does not leave, he should stand in silence.” } For someone covering by the layer, after covering with two layers, he may ask for a third layer, and he should then leave.\footnote{Sp 3.126: \textit{\textsanskrit{Pariyāyenāti} parikkhepena}, “\textit{\textsanskrit{Pariyāya}}: by encircling.” However, since the word \textit{\textsanskrit{pariyāya}} is used in the sense of “layer” just before, I am assuming the meaning must be the same here. } %
\item[If he applies more than that, even if he stands where there are no cultivated plants: ] if he is covering with bricks, then for every brick, he commits an offense entailing confession. If he is covering with slate, then for every piece of slate, he commits an offense entailing confession. If he is covering with plaster, then for every lump, he commits an offense entailing confession. If he is covering with grass, then for every handful, he commits an offense entailing confession. If he is covering with leaves, then for every leaf, he commits an offense entailing confession. %
\end{description}

\subsection*{Permutations }

If\marginnote{2.2.1} it is more than two or three layers, and he perceives it as more, and he applies it, he commits an offense entailing confession. If it is more than two or three layers, but he is unsure of it, and he applies it, he commits an offense entailing confession. If it is more than two or three layers, but he perceives it as less, and he applies it, he commits an offense entailing confession. 

If\marginnote{2.2.4} it is less than two or three layers, but he perceives it as more, he commits an offense of wrong conduct. If it is less than two or three layers, but he is unsure of it, he commits an offense of wrong conduct. If it is less than two or three layers, and he perceives it as less, there is no offense. 

\subsection*{Non-offenses }

There\marginnote{2.3.1} is no offense: if he applies two or three layers; if he applies less than two or three layers; if it is a shelter; if it is a cave;\footnote{For the rendering of \textit{\textsanskrit{guhā}} as “cave”, see Appendix of Technical Terms. } if it is a grass hut; if it is for the benefit of someone else; if it is by means of his own property; if it is anything apart from a dwelling; if he is insane; if he is the first offender. 

\scendsutta{The training rule on large dwellings, the ninth, is finished. }

%
\section*{{\suttatitleacronym Bu Pc 20}{\suttatitletranslation 20. The training rule on containing living beings }{\suttatitleroot Sappāṇaka}}
\addcontentsline{toc}{section}{\tocacronym{Bu Pc 20} \toctranslation{20. The training rule on containing living beings } \tocroot{Sappāṇaka}}
\markboth{20. The training rule on containing living beings }{Sappāṇaka}
\extramarks{Bu Pc 20}{Bu Pc 20}

\subsection*{Origin story }

At\marginnote{1.1} one time when the Buddha was staying at \textsanskrit{Āḷavī} at the \textsanskrit{Aggāḷava} Shrine, the monks there were doing building work. They poured water that they knew contained living beings onto grass and clay, and they had others do the same. The monks of few desires complained and criticized them, “How can the monks at \textsanskrit{Āḷavī} pour water that they know contains living beings onto grass and clay, and have others do the same?” 

After\marginnote{1.5} rebuking those monks in many ways, they told the Buddha. Soon afterwards he had the Sangha gathered and questioned the monks: “Is it true, monks, that you do this?” 

“It’s\marginnote{1.7} true, sir.” 

The\marginnote{1.8} Buddha rebuked them … “Foolish men, how can you do this? This will affect people’s confidence …” … “And, monks, this training rule should be recited like this: 

\subsection*{Final ruling }

\scrule{‘If a monk pours water that he knows contains living beings onto grass or clay, or has it poured, he commits an offense entailing confession.’” }

\subsection*{Definitions }

\begin{description}%
\item[A: ] whoever … %
\item[Monk: ] …The monk who has been given the full ordination by a unanimous Sangha through a legal procedure consisting of one motion and three announcements that is irreversible and fit to stand—this sort of monk is meant in this case. %
\item[He knows: ] he knows by himself or others have told him. %
\item[Pours: ] if he pours it himself, he commits an offense entailing confession. %
\item[Has poured: ] if he asks another, he commits an offense entailing confession. If he only asks once, then even if the other pours a lot, he commits one offense entailing confession.\footnote{“A lot” renders \textit{\textsanskrit{bahukaṁ}}. This is based on the commentary to Bu Pc 10, Sp 2.86: \textit{\textsanskrit{Sakiṁ} \textsanskrit{āṇatto} bahukampi \textsanskrit{khaṇatīti} sacepi \textsanskrit{sakaladivasaṁ} \textsanskrit{khaṇati}, \textsanskrit{āṇāpakassa} \textsanskrit{ekaṁyeva} \textsanskrit{pācittiyaṁ}}, “\textit{\textsanskrit{Sakiṁ} \textsanskrit{āṇatto} bahukampi \textsanskrit{khaṇatīti}}: there is only one offense entailing confession for the one who asks, even if the other digs the entire day.” } %
\end{description}

\subsection*{Permutations }

If\marginnote{2.2.1} it contains living beings, and he perceives it as such, and he pours it onto grass or clay, or he has it poured, he commits an offense entailing confession. If it contains living beings, but he is unsure of it, and he pours it onto grass or clay, or he has it poured, he commits an offense of wrong conduct. If it contains living beings, but he does not perceive it as such, and he pours it onto grass or clay, or he has it poured, there is no offense. 

If\marginnote{2.2.4} it does not contain living beings, but he perceives it as such, he commits an offense of wrong conduct. If it does not contain living beings, but he is unsure of it, he commits an offense of wrong conduct. If it does not contain living beings, and he does not perceive it as such, there is no offense. 

\subsection*{Non-offenses }

There\marginnote{2.3.1} is no offense: if it is unintentional; if he is not mindful; if he does not know; if he is insane; if he is the first offender. 

\scendsutta{The training rule on containing living beings, the tenth, is finished. }

\scendvagga{The second subchapter on plants is finished. }

\scuddanaintro{This is the summary: }

\begin{scuddana}%
“Plant,\marginnote{2.3.10} with evasion, complaining, \\
The two with departing; \\
Before, throwing out, detachable, \\
Door, and containing living beings.” 

%
\end{scuddana}

%
\section*{{\suttatitleacronym Bu Pc 21}{\suttatitletranslation 21. The training rule on the instruction }{\suttatitleroot Ovāda}}
\addcontentsline{toc}{section}{\tocacronym{Bu Pc 21} \toctranslation{21. The training rule on the instruction } \tocroot{Ovāda}}
\markboth{21. The training rule on the instruction }{Ovāda}
\extramarks{Bu Pc 21}{Bu Pc 21}

\subsection*{Origin story }

\subsubsection*{First sub-story }

At\marginnote{1.1} one time the Buddha was staying at \textsanskrit{Sāvatthī} in the Jeta Grove, \textsanskrit{Anāthapiṇḍika}’s Monastery. At that time the senior monks who were instructing the nuns received robe-cloth, almsfood, dwellings, and medicinal supplies. When the monks from the group of six found out about this, they thought, “Well then, let’s instruct the nuns.” They then went to the nuns and said, “Come to us, Sisters, and we too will instruct you.” 

Soon\marginnote{1.7} afterwards those nuns went to the monks from the group of six, bowed, and sat down. But after giving just a short teaching, those monks spent the day on worldly talk. They then dismissed the nuns, saying, “Go, Sisters.” 

The\marginnote{1.10} nuns went to the Buddha and bowed, and the Buddha said to them, “Nuns, I hope the instruction was effective?” 

“Sir,\marginnote{1.11} how could the instruction be effective? After giving just a short teaching, the monks from the group of six spent the day on worldly talk, and then dismissed us.” 

The\marginnote{1.13} Buddha instructed, inspired, and gladdened those nuns with a teaching. The nuns bowed, circumambulated the Buddha with their right sides toward him, and left. 

Soon\marginnote{1.15} afterwards the Buddha had the Sangha gathered and questioned the monks from the group of six: “Is it true, monks, that you acted like this?” 

“It’s\marginnote{1.17} true, sir.” 

The\marginnote{1.18} Buddha rebuked them … “Foolish men, how could you act like this? This will affect people’s confidence …” After rebuking them … he gave a teaching and addressed the monks: 

“Monks,\marginnote{1.23} you should appoint an instructor of the nuns.\footnote{“Should” renders \textit{\textsanskrit{anujānāmi}}. See Appendix of Technical Terms for discussion. } And this is how he should be appointed. First a monk should be asked and then a competent and capable monk should inform the Sangha: 

‘Please,\marginnote{1.27} venerables, I ask the Sangha to listen. If the Sangha is ready, it should appoint monk so-and-so as an instructor of the nuns. This is the motion. 

Please,\marginnote{1.30} venerables, I ask the Sangha to listen. The Sangha appoints monk so-and-so as an instructor of the nuns. Any monk who approves of appointing monk so-and-so as an instructor of the nuns should remain silent. Any monk who doesn’t approve should speak up. 

For\marginnote{1.34} the second time, I speak on this matter. … For the third time, I speak on this matter. Please, venerables, I ask the Sangha to listen. The Sangha appoints monk so-and-so as an instructor of the nuns. Any monk who approves of appointing monk so-and-so as an instructor of the nuns should remain silent. Any monk who doesn’t approve should speak up. 

The\marginnote{1.40} Sangha has appointed monk so-and-so as an instructor of the nuns. The Sangha approves and is therefore silent. I’ll remember it thus.’” 

Then,\marginnote{1.42} after rebuking the monks from the group of six in many ways, the Buddha spoke in dispraise of being difficult to support … “And, monks, this training rule should be recited like this: 

\subsection*{Final ruling }

\scrule{‘If a monk who has not been appointed instructs the nuns, he commits an offense entailing confession.’” }

In\marginnote{1.45} this way the Buddha laid down this training rule for the monks. 

\subsubsection*{Second sub-story }

After\marginnote{2.1} being appointed to do so, the senior monks who instructed the nuns still received robe-cloth, almsfood, dwellings, and medicinal supplies. When they found out about this, the monks from the group of six said, “Well then, let’s go outside the monastery zone, appoint each other as instructors of the nuns, and then instruct them.” After doing just that, they again went to the nuns and said, “Sisters, we too have been appointed. Come to us, and we will instruct you.” 

Once\marginnote{2.7} again the nuns went to the monks from the group of six and everything unfolded as before. 

The\marginnote{2.16} Buddha then had the Sangha gathered and questioned the monks from the group of six: “Is it true, monks, that you acted like this?” 

“It’s\marginnote{2.20} true, sir.” 

The\marginnote{2.21} Buddha rebuked them … “Foolish men, how could you act like this? This will affect people’s confidence …” After rebuking them … he gave a teaching and addressed the monks: 

“Monks,\marginnote{2.26} you may appoint a monk who possesses eight qualities as an instructor of the nuns:\footnote{This set of eight qualities is also found at \href{https://suttacentral.net/an8.52/en/brahmali\#2.1}{AN 8.52:2.1}. } 

\begin{enumerate}%
\item He is virtuous and restrained by the Monastic Code. His conduct is good, he associates with the right people, and he sees danger in minor faults. He undertakes and trains in the training rules. %
\item He has learned much, and he retains and accumulates what he has learned. Those teachings that are good in the beginning, good in the middle, and good in the end, that have a true goal and are well articulated, and that set out the perfectly complete and pure spiritual life—he has learned many such teachings, retained them in mind, recited them verbally, mentally investigated them, and penetrated them well by view. %
\item He has properly learned both Monastic Codes in detail. He has analyzed them well, thoroughly mastered them, and investigated them well, both in terms of the rules and their detailed exposition. %
\item He is well-spoken and has a good voice. %
\item He is generally liked by and pleasing to the nuns. %
\item He is capable of instructing the nuns. %
\item He has never committed a heavy offense against a Buddhist nun. %
\item He has been fully ordained for at least twenty years. %
\end{enumerate}

Monks,\marginnote{2.35} you may appoint a monk who possesses these eight qualities as an instructor of the nuns.” 

\subsection*{Definitions }

\begin{description}%
\item[A: ] whoever … %
\item[Monk: ] … The monk who has been given the full ordination by a unanimous Sangha through a legal procedure consisting of one motion and three announcements that is irreversible and fit to stand—this sort of monk is meant in this case. %
\item[Who has not been appointed: ] who has not been appointed through a legal procedure consisting of one motion and three announcements. %
\item[Nuns: ] they have been given the full ordination by both Sanghas. %
\item[Instructs: ] if\marginnote{3.1.10} he instructs in the eight important principles, he commits an offense entailing confession. If he instructs in any other teaching, he commits an offense of wrong conduct. If he instructs a nun who is fully ordained only on one side, he commits an offense of wrong conduct. 

The\marginnote{3.1.13} monk who has been appointed should sweep the yard, set out water for drinking and water for washing, prepare a seat, get hold of a companion, and then sit down.\footnote{For the rendering of \textit{\textsanskrit{pariveṇa}} as “yard”, see Appendix of Technical Terms. } The nuns should go there, bow down to the monk, and sit down. The monk should then ask them, “Are you all here, Sisters?”\footnote{“Are you all here” renders \textit{\textsanskrit{samaggāttha}}, literally, “Are you united”. Sp 2.149: \textit{\textsanskrit{Samaggātthāti} \textsanskrit{sabbā} \textsanskrit{āgatatthāti} attho}, “Are you united: the meaning is, ‘Have you all come?’” } 

If\marginnote{3.1.17} they say, “We’re all here, venerable,” he should say, “Are you keeping the eight important principles?” 

If\marginnote{3.1.18} they say, “We are,” he should say, “This is the instruction.”\footnote{For the meaning of \textit{\textsanskrit{niyyādetabbo}} in this context, see the usage in the origin story to the next rule, \href{https://suttacentral.net/pli-tv-bu-vb-pc22/en/brahmali\#1.9}{Bu Pc 22:1.9}. Sp 2.149: \textit{\textsanskrit{Niyyādetabboti} appetabbo} … \textit{\textsanskrit{Ovādaṁ} \textsanskrit{aniyyādetvāti} eso bhaginiyo \textsanskrit{ovādoti} \textsanskrit{avatvā}}, “\textit{\textsanskrit{Niyyādetabbo}}: should display it. … \textit{\textsanskrit{Ovādaṁ} \textsanskrit{aniyyādetvā}}: not having said, ‘Sisters, this is the instruction.’” } 

If\marginnote{3.1.19} they say, “We’re not,” he should recite the important principles:\footnote{Sp 2.149: \textit{\textsanskrit{Osāretabboti} \textsanskrit{pāḷi} \textsanskrit{vattabbā}}, “\textit{\textsanskrit{Osāretabbo}}: the Canonical text is to be spoken.” } 

\begin{enumerate}%
\item “A nun who has been fully ordained for a hundred years should bow down to a monk who was given the full ordination on that very day, and she should stand up for him, raise her joined palms to him, and do acts of respect toward him.\footnote{“Acts of respect” renders \textit{\textsanskrit{sāmīcikammaṁ}}. Sp 2.149: \textit{\textsanskrit{sāmīcikammanti} \textsanskrit{maggasampadānabījanapānīyāpucchanādikaṁ} \textsanskrit{anucchavikavattaṁ}}, “\textit{\textsanskrit{Sāmīcikammaṁ}} means appropriate duties such as giving way, fanning, offering drinking water, etc.” } This principle is to be honored and respected all one’s life, and is not to be breached. %
\item A nun shouldn’t spend the rainy-season residence in a monastery without monks.\footnote{The Pali word behind the translation “monastery” is \textit{\textsanskrit{āvāsa}}. This refers to the area of the \textit{\textsanskrit{simā}}, the monastery zone, as established through a legal procedure. This area may be much larger than the space occupied by any actual buildings. } This principle too is to be honored and respected all one’s life, and is not to be breached. %
\item Every half-month a nun should seek two things from the Sangha of monks:  asking it about the observance day and going to it for the instruction.  This principle too is to be honored and respected all one’s life, and is not to be breached. %
\item A nun who has completed the rainy-season residence should invite correction from both Sanghas in regard to three things:  what has been seen, heard, or suspected. This principle too is to be honored and respected all one’s life, and is not to be breached. %
\item A nun who has committed a heavy offense must undertake a trial period for a half-month toward both Sanghas.\footnote{Heavy offense, \textit{garudhamma}, here refers to the \textit{\textsanskrit{saṅghādisesa}} offenses, the offenses entailing suspension. } This principle too is to be honored and respected all one’s life, and is not to be breached. %
\item A trainee nun who has trained for two years in the six rules may seek for full ordination in both Sanghas. This principle too is to be honored and respected all one’s life, and is not to be breached. %
\item A nun may not in any way abuse or revile a monk. This principle too is to be honored and respected all one’s life, and is not to be breached. %
\item From today onwards, nuns may not correct monks, but monks may correct nuns.\footnote{“Correct” renders \textit{vacanapatha}. For the meaning of this word, see Bhikkhu Sujato, “Bhikkhuni Vinaya Studies”, pp. 73–76. } This principle too is to be honored and respected all one’s life, and is not to be breached.” %
\end{enumerate}

If\marginnote{3.1.38} they say, “We’re all here, venerable,” and he instructs them with another teaching, then he commits an offense of wrong conduct.\footnote{It is not entirely clear how this offense comes about, but it must somehow relate to what we find in segment 3.1.17 above. It seems the text here assumes that the nuns reply “no” to the follow-up question in segment 3.1.17 of whether they are keeping the important principles, in which case the appointed monk is supposed to recite them. Only if he does not do so at this point, does he incur an offense of wrong conduct under this rule. } If they say, “We’re not all here, venerable,” and he instructs them in the eight important principles, then he commits an offense of wrong conduct. If he does not give the instruction, but he gives them another teaching, then he commits an offense of wrong conduct.\footnote{“He does not give the instruction” renders \textit{\textsanskrit{ovādaṁ} \textsanskrit{aniyyādetvā}}. Sp 2.149: \textit{\textsanskrit{Ovādaṁ} \textsanskrit{aniyyādetvāti} eso bhaginiyo \textsanskrit{ovādoti} \textsanskrit{avatvā}} “He does not give the instruction: not having said, ‘Sisters, this is the instruction.’” This seems a bit contrived, but in light of segment 3.1.18 above and the origin story to \href{https://suttacentral.net/pli-tv-bu-vb-pc22/en/brahmali\#1.9}{Bu Pc 22:1.9} perhaps it is nevertheless the right interpretation. } 

%
\end{description}

\subsection*{Permutations }

If\marginnote{3.2.1} it is an illegitimate legal procedure and he perceives it as such, and the Sangha of nuns is incomplete and he perceives it as such, then if he instructs those nuns, he commits an offense entailing confession.\footnote{The legal procedure in question is the one that gives approval to a monk to instruct the nuns. Sp 2.150: \textit{\textsanskrit{Adhammakammetiādīsu} \textsanskrit{bhikkhunovādakasammutikammaṁ} kammanti \textsanskrit{veditabbaṁ}}, “\textit{Adhammakamme}: the legal procedure that gives approval to instruct the nuns should be understood.” } If it is an illegitimate legal procedure and he perceives it as such, and the Sangha of nuns is incomplete but he is unsure of it, then if he instructs those nuns, he commits an offense entailing confession. If it is an illegitimate legal procedure and he perceives it as such, and the Sangha of nuns is incomplete but he perceives it as complete, then if he instructs those nuns, he commits an offense entailing confession. 

If\marginnote{3.2.4} it is an illegitimate legal procedure but he is unsure of it, and the Sangha of nuns is incomplete and he perceives it as such, then if he instructs those nuns, he commits an offense entailing confession. If it is an illegitimate legal procedure but he is unsure of it, and the Sangha of nuns is incomplete but he is unsure of it, then if he instructs those nuns, he commits an offense entailing confession. If it is an illegitimate legal procedure but he is unsure of it, and the Sangha of nuns is incomplete but he perceives it as complete, then if he instructs those nuns, he commits an offense entailing confession. 

If\marginnote{3.2.7} it is an illegitimate legal procedure but he perceives it as legitimate, and the Sangha of nuns is incomplete and he perceives it as such, then if he instructs those nuns, he commits an offense entailing confession. If it is an illegitimate legal procedure but he perceives it as legitimate, and the Sangha of nuns is incomplete but he is unsure of it, then if he instructs those nuns, he commits an offense entailing confession. If it is an illegitimate legal procedure but he perceives it as legitimate, and the Sangha of nuns is incomplete but he perceives it as complete, then if he instructs those nuns, he commits an offense entailing confession. 

If\marginnote{3.2.10} it is an illegitimate legal procedure and he perceives it as such, and the Sangha of nuns is complete but he perceives it as incomplete, then if he instructs those nuns, he commits an offense entailing confession. If it is an illegitimate legal procedure and he perceives it as such, and the Sangha of nuns is complete but he is unsure of it, then if he instructs those nuns, he commits an offense entailing confession. If it is an illegitimate legal procedure and he perceives it as such, and the Sangha of nuns is complete and he perceives it as such, then if he instructs those nuns, he commits an offense entailing confession. 

If\marginnote{3.2.13} it is an illegitimate legal procedure but he is unsure of it, and the Sangha of nuns is complete but he perceives it as incomplete, then if he instructs those nuns, he commits an offense entailing confession. If it is an illegitimate legal procedure but he is unsure of it, and the Sangha of nuns is complete but he is unsure of it, then if he instructs those nuns, he commits an offense entailing confession. If it is an illegitimate legal procedure but he is unsure of it, and the Sangha of nuns is complete and he perceives it as such, then if he instructs those nuns, he commits an offense entailing confession. 

If\marginnote{3.2.16} it is an illegitimate legal procedure but he perceives it as legitimate, and the Sangha of nuns is complete but he perceives it as incomplete, then if he instructs those nuns, he commits an offense entailing confession. If it is an illegitimate legal procedure but he perceives it as legitimate, and the Sangha of nuns is complete but he is unsure of it, then if he instructs those nuns, he commits an offense entailing confession. If it is an illegitimate legal procedure but he perceives it as legitimate, and the Sangha of nuns is complete and he perceives it as such, then if he instructs those nuns, he commits an offense entailing confession. 

If\marginnote{3.2.19} it is a legitimate legal procedure but he perceives it as illegitimate, and the Sangha of nuns is incomplete and he perceives it as such, then if he instructs those nuns, he commits an offense of wrong conduct. If it is a legitimate legal procedure but he perceives it as illegitimate, and the Sangha of nuns is incomplete but he is unsure of it, then if he instructs those nuns, he commits an offense of wrong conduct. If it is a legitimate legal procedure but he perceives it as illegitimate, and the Sangha of nuns is incomplete but he perceives it as complete, then if he instructs those nuns, he commits an offense of wrong conduct. 

If\marginnote{3.2.22} it is a legitimate legal procedure but he is unsure of it, and the Sangha of nuns is incomplete and he perceives it as such, then if he instructs those nuns, he commits an offense of wrong conduct. If it is a legitimate legal procedure but he is unsure of it, and the Sangha of nuns is incomplete but he is unsure of it, then if he instructs those nuns, he commits an offense of wrong conduct. If it is a legitimate legal procedure but he is unsure of it, and the Sangha of nuns is incomplete but he perceives it as complete, then if he instructs those nuns, he commits an offense of wrong conduct. 

If\marginnote{3.2.25} it is a legitimate legal procedure and he perceives it as such, and the Sangha of nuns is incomplete and he perceives it as such, then if he instructs those nuns, he commits an offense of wrong conduct. If it is a legitimate legal procedure and he perceives it as such, and the Sangha of nuns is incomplete but he is unsure of it, then if he instructs those nuns, he commits an offense of wrong conduct. If it is a legitimate legal procedure and he perceives it as such, and the Sangha of nuns is incomplete but he perceives it as complete, then if he instructs those nuns, he commits an offense of wrong conduct. 

If\marginnote{3.2.28} it is a legitimate legal procedure but he perceives it as illegitimate, and the Sangha of nuns is complete but he perceives it as incomplete, then if he instructs those nuns, he commits an offense of wrong conduct. If it is a legitimate legal procedure but he perceives it as illegitimate, and the Sangha of nuns is complete but he is unsure of it, then if he instructs those nuns, he commits an offense of wrong conduct. If it is a legitimate legal procedure but he perceives it as illegitimate, and the Sangha of nuns is complete and he perceives it as such, then if he instructs those nuns, he commits an offense of wrong conduct. 

If\marginnote{3.2.31} it is a legitimate legal procedure but he is unsure of it, and the Sangha of nuns is complete but he perceives it as incomplete, then if he instructs those nuns, he commits an offense of wrong conduct. If it is a legitimate legal procedure but he is unsure of it, and the Sangha of nuns is complete but he is unsure of it, then if he instructs those nuns, he commits an offense of wrong conduct. If it is a legitimate legal procedure but he is unsure of it, and the Sangha of nuns is complete and he perceives it as such, then if he instructs those nuns, he commits an offense of wrong conduct. 

If\marginnote{3.2.34} it is a legitimate legal procedure and he perceives it as such, and the Sangha of nuns is complete but he perceives it as incomplete, then if he instructs those nuns, he commits an offense of wrong conduct. If it is a legitimate legal procedure and he perceives it as such, and the Sangha of nuns is complete but he is unsure of it, then if he instructs those nuns, he commits an offense of wrong conduct. If it is a legitimate legal procedure and he perceives it as such, and the Sangha of nuns is complete and he perceives it as such, then if he instructs those nuns, there is no offense. 

\subsection*{Non-offenses }

There\marginnote{3.3.1} is no offense: if he recites; if he tests them;\footnote{“Testing” renders \textit{paripuccha}. The basic meaning of \textit{\textsanskrit{paripucchā}} is “to question” or “to ask”, as used for instance in \href{https://suttacentral.net/pli-tv-bu-vb-pc71/en/brahmali\#1.19.1}{Bu Pc 71}. Often, however, as in the present case, it refers to a teacher questioning his student, in the sense of finding out how much the student knows. In such cases I render the word as “testing”. } if he recites when asked to do so;\footnote{That \textit{\textsanskrit{osāreti}} must mean “recite” can be seen from \href{https://suttacentral.net/pli-tv-bu-vb-pc4/en/brahmali\#2.3.5}{Bu Pc 4:2.3.5} where there is no offense if \textit{\textsanskrit{osārentaṁ} \textsanskrit{opāteti}}, “he prompts one who is \textit{\textsanskrit{osārentaṁ}}.” The difference between \textit{\textsanskrit{uddesaṁ} deti} and \textit{\textsanskrit{osāreti}} is not clear to me. } if he asks a question; if he replies when asked a question; if he is speaking for the benefit of someone else and the nuns listen in; if it is to a trainee nun; if it is to a novice nun; if he is insane; if he is the first offender. 

\scendsutta{The training rule on the instruction, the first, is finished. }

%
\section*{{\suttatitleacronym Bu Pc 22}{\suttatitletranslation 22. The training rule on set }{\suttatitleroot Atthaṅgata}}
\addcontentsline{toc}{section}{\tocacronym{Bu Pc 22} \toctranslation{22. The training rule on set } \tocroot{Atthaṅgata}}
\markboth{22. The training rule on set }{Atthaṅgata}
\extramarks{Bu Pc 22}{Bu Pc 22}

\subsection*{Origin story }

At\marginnote{1.1} one time when the Buddha was staying at \textsanskrit{Sāvatthī} in \textsanskrit{Anāthapiṇḍika}’s Monastery, the senior monks were taking turns instructing the nuns. Then, on one occasion, it was Venerable \textsanskrit{Cūḷapanthaka}’s turn. The nuns said, “Today the instruction won’t be effective. Venerable \textsanskrit{Cūḷapanthaka} will just be repeating the same thing over and over.” 

Soon\marginnote{1.4} afterwards those nuns went to \textsanskrit{Cūḷapanthaka}, bowed, and sat down. \textsanskrit{Cūḷapanthaka} then said to them, “Are you all here, Sisters?”\footnote{“Are you all here” renders \textit{\textsanskrit{samaggāttha}}, literally, “are you united”. Sp 2.149: \textit{\textsanskrit{Samaggātthāti} \textsanskrit{sabbā} \textsanskrit{āgatatthāti} attho}, “Are you united: the meaning is ‘Have you all come?’” } 

“We’re\marginnote{1.6} all here, venerable.” 

“Are\marginnote{1.7} you keeping the eight important principles?” 

“We\marginnote{1.8} are.” 

After\marginnote{1.9} saying, “This is the instruction,” he uttered the same heartfelt exclamation again and again: 

\begin{verse}%
“For\marginnote{1.10} the heedful one possessed of the higher mind, \\
For the sage training on the path to sagacity—\\
There are no sorrows for such a one, \\
The peaceful one, who is always mindful.” 

%
\end{verse}

And\marginnote{1.14} the nuns said, “Didn’t we say that the instruction wouldn’t be effective, that Venerable \textsanskrit{Cūḷapanthaka} would just be repeating the same thing over and over?” \textsanskrit{Cūḷapanthaka} overheard that conversation between the nuns. He then rose up into the air, walked back and forth in space, and he stood, sat down, and lay down there. He emitted smoke and fire, and he disappeared, all the while uttering the same heartfelt exclamation and many other sayings by the Buddha. The nuns said, “It’s astonishing and amazing! No previous instruction has been as effective as this one from Venerable \textsanskrit{Cūḷapanthaka}!” And \textsanskrit{Cūḷapanthaka} kept on instructing those nuns until the dark of night and then dismissed them, saying, “Go, Sisters.” 

But\marginnote{1.21} because the gates to town were closed, those nuns spent the night outside and only entered early in the morning. People complained and criticized them, “These nuns aren’t celibate. They spend the night in the monastery with the monks and only now do they enter town.” 

The\marginnote{1.24} monks heard the complaints of those people, and the monks of few desires complained and criticized \textsanskrit{Cūḷapanthaka}, “How could Venerable \textsanskrit{Cūḷapanthaka} instruct the nuns after the sun has set?”… “Is it true, \textsanskrit{Cūḷapanthaka}, that you did this?” 

“It’s\marginnote{1.28} true, sir.” 

The\marginnote{1.29} Buddha rebuked him … “\textsanskrit{Cūḷapanthaka}, how could you do this? This will affect people’s confidence …” … “And, monks, this training rule should be recited like this: 

\subsection*{Final ruling }

\scrule{‘Even if he has been appointed, if a monk instructs the nuns after sunset, he commits an offense entailing confession.’” }

\subsection*{Definitions }

\begin{description}%
\item[Has been appointed: ] has been appointed through a legal procedure consisting of one motion and three announcements. %
\item[After sunset: ] after the sun has gone down. %
\item[The nuns: ] they have been given the full ordination by both Sanghas. %
\item[Instructs: ] if he instructs in the eight important principles or he gives any other teaching, he commits an offense entailing confession. %
\end{description}

\subsection*{Permutations }

If\marginnote{2.2.1} the sun has set, and he perceives that it has, and he instructs the nuns, he commits an offense entailing confession. If the sun has set, but he is unsure of it, and he instructs the nuns, he commits an offense entailing confession. If the sun has set, but he perceives that it has not, and he instructs the nuns, he commits an offense entailing confession. 

If\marginnote{2.2.4} he instructs a nun who is fully ordained only on one side, he commits an offense of wrong conduct. If the sun has not set, but he perceives that it has, he commits an offense of wrong conduct. If the sun has not set, but he is unsure of it, he commits an offense of wrong conduct. If the sun has not set, and he perceives that it has not, there is no offense. 

\subsection*{Non-offenses }

There\marginnote{2.3.1} is no offense: if he recites; if he tests them; if he recites when asked to do so;\footnote{That \textit{\textsanskrit{osāreti}} must mean “recite” can be seen from \href{https://suttacentral.net/pli-tv-bu-vb-pc4/en/brahmali\#2.3.5}{Bu Pc 4:2.3.5} where there is no offense if \textit{\textsanskrit{osārentaṁ} \textsanskrit{opāteti}}, “he prompts one who is \textit{\textsanskrit{osārentaṁ}}.” The difference between \textit{\textsanskrit{uddesaṁ} deti} and \textit{\textsanskrit{osāreti}} is not clear to me. } if he asks a question; if he replies when asked a question; if he is speaking for the benefit of someone else and the nuns listen in; if it is to a trainee nun; if it is to a novice nun; if he is insane; if he is the first offender. 

\scendsutta{The training rule on set, the second, is finished. }

%
\section*{{\suttatitleacronym Bu Pc 23}{\suttatitletranslation 23. The training rule on the nuns’ dwelling place }{\suttatitleroot Bhikkhunupassaya}}
\addcontentsline{toc}{section}{\tocacronym{Bu Pc 23} \toctranslation{23. The training rule on the nuns’ dwelling place } \tocroot{Bhikkhunupassaya}}
\markboth{23. The training rule on the nuns’ dwelling place }{Bhikkhunupassaya}
\extramarks{Bu Pc 23}{Bu Pc 23}

\subsection*{Origin story }

\subsubsection*{First sub-story }

At\marginnote{1.1} one time when the Buddha was staying in the Sakyan country in the Banyan Tree Monastery at Kapilavatthu, the monks from the group of six went to the nuns’ dwelling place and instructed the nuns from the group of six. 

Soon\marginnote{1.3} afterwards, other nuns said to the nuns from the group of six, “Come, venerables, let’s go to the instruction.” 

“There’s\marginnote{1.5} no need. The monks from the group of six came and instructed us right here.” 

The\marginnote{1.6} nuns complained and criticized the monks from the group of six, “How could the monks from the group of six go and instruct the nuns at their dwelling place?” Then those nuns told the monks. 

The\marginnote{1.9} monks of few desires complained and criticized those monks, “How could those monks do this?”… “Is it true, monks, that you did this?” 

“It’s\marginnote{1.12} true, sir.” 

The\marginnote{1.13} Buddha rebuked them … “Foolish men, how could you do this? This will affect people’s confidence …” … “And, monks, this training rule should be recited like this: 

\subsubsection*{Preliminary ruling }

\scrule{‘If a monk goes to the nuns’ dwelling place, and then instructs them, he commits an offense entailing confession.’” }

In\marginnote{1.18} this way the Buddha laid down this training rule for the monks. 

\subsubsection*{Second sub-story }

Soon\marginnote{2.1} afterwards \textsanskrit{Mahāpajāpati} \textsanskrit{Gotamī} became sick. The senior monks went to see her and said, “We hope you’re bearing up, \textsanskrit{Gotamī}, we hope you’re getting better?” 

“I’m\marginnote{2.5} not bearing up, venerables, and I’m not getting better. Please give me a teaching.” 

“It’s\marginnote{2.7} not allowable for us to go and teach the nuns at their dwelling place.” And being afraid of wrongdoing, they did not teach her. 

Soon\marginnote{2.8} afterwards, after robing up in the morning, the Buddha took his bowl and robe and went to \textsanskrit{Mahāpajāpati} \textsanskrit{Gotamī} where he sat down on the prepared seat. He said to her, “I hope you’re bearing up, \textsanskrit{Gotamī}, I hope you’re getting better?” 

“Previously,\marginnote{2.11} sir, the senior monks would come and teach me, and because of that I would be comfortable. But now that this has been prohibited by the Buddha, they don’t teach because they’re afraid of wrongdoing. Because of that I’m not comfortable.” 

After\marginnote{2.15} instructing, inspiring, and gladdening her with a teaching, the Buddha got up from his seat and left. Soon afterwards the Buddha gave a teaching and addressed the monks: 

\scrule{“Monks, I allow you to go and instruct a sick nun at her dwelling place. }

And\marginnote{2.18} so, monks, this training rule should be recited like this: 

\subsection*{Final ruling }

\scrule{‘If a monk goes to the nuns’ dwelling place and then instructs them, except on an appropriate occasion, he commits an offense entailing confession. This is the appropriate occasion: a nun is sick.’” }

\subsection*{Definitions }

\begin{description}%
\item[A: ] whoever … %
\item[Monk: ] … The monk who has been given the full ordination by a unanimous Sangha through a legal procedure consisting of one motion and three announcements that is irreversible and fit to stand—this sort of monk is meant in this case. %
\item[The nuns’ dwelling place: ] wherever nuns stay, even for a single night. %
\item[Goes to: ] goes there. %
\item[A nun: ] she has been given the full ordination by both Sanghas. %
\item[Instructs: ] if he instructs about the eight important principles, he commits an offense entailing confession. %
\item[Except on an appropriate occasion: ] unless it is an appropriate occasion. %
\item[A sick nun: ] she is not able to go to the instruction or to a formal meeting of the community.\footnote{“A formal meeting of the community”, \textit{\textsanskrit{saṁvāsa}}, is defined at \href{https://suttacentral.net/pli-tv-bu-vb-pc69/en/brahmali\#2.1.21}{Bu Pc 69:2.1.21}, as the observance-day ceremony, the invitation ceremony, or a legal procedure. } %
\end{description}

\subsection*{Permutations }

If\marginnote{3.2.1} she is fully ordained, and he perceives her as such, and he goes to her dwelling place and then instructs her, except on an appropriate occasion, he commits an offense entailing confession. If she is fully ordained, but he is unsure of it, and he goes to her dwelling place and then instructs her, except on an appropriate occasion, he commits an offense entailing confession. If she is fully ordained, but he does not perceive her as such, and he goes to her dwelling place and then instructs her, except on an appropriate occasion, he commits an offense entailing confession. 

If\marginnote{3.2.4} he instructs her with another teaching, he commits an offense of wrong conduct. If he instructs a nun who is fully ordained only on one side, he commits an offense of wrong conduct. If she is not fully ordained, but he perceives her as such, he commits an offense of wrong conduct. If she is not fully ordained, but he is unsure of it, he commits an offense of wrong conduct. If she is not fully ordained, and he does not perceive her as such, there is no offense. 

\subsection*{Non-offenses }

There\marginnote{3.3.1} is no offense: if it is an appropriate occasion; if he recites; if he tests them; if he recites when asked to do so;\footnote{That \textit{\textsanskrit{osāreti}} must mean “recite” can be seen from \href{https://suttacentral.net/pli-tv-bu-vb-pc4/en/brahmali\#2.3.5}{Bu Pc 4:2.3.5} where there is no offense if \textit{\textsanskrit{osārentaṁ} \textsanskrit{opāteti}}, “he prompts one who is \textit{\textsanskrit{osārentaṁ}}.” The difference between \textit{\textsanskrit{uddesaṁ} deti} and \textit{\textsanskrit{osāreti}} is not clear to me. } if he asks a question; if he replies when asked a question; if he is speaking for the benefit of someone else and the nuns listen in; if it is to a trainee nun; if it is to a novice nun; if he is insane; if he is the first offender. 

\scendsutta{The training rule on the nuns’ dwelling place, the third, is finished. }

%
\section*{{\suttatitleacronym Bu Pc 24}{\suttatitletranslation 24. The training rule on worldly gain }{\suttatitleroot Āmisa}}
\addcontentsline{toc}{section}{\tocacronym{Bu Pc 24} \toctranslation{24. The training rule on worldly gain } \tocroot{Āmisa}}
\markboth{24. The training rule on worldly gain }{Āmisa}
\extramarks{Bu Pc 24}{Bu Pc 24}

\subsection*{Origin story }

At\marginnote{1.1} one time the Buddha was staying at \textsanskrit{Sāvatthī} in the Jeta Grove, \textsanskrit{Anāthapiṇḍika}’s Monastery. At that time the senior monks who were instructing the nuns received robe-cloth, almsfood, dwellings, and medicinal supplies. And the monks from the group of six said this about them, “The senior monks aren’t instructing the nuns as a service, but for the sake of worldly gain.” 

The\marginnote{1.5} monks of few desires complained and criticized them, “How can the monks from the group of six say that the senior monks aren’t instructing the nuns to render a service, but for the sake of worldly gain?” … “Is it true, monks, that you say this?” 

“It’s\marginnote{1.10} true, sir.” 

The\marginnote{1.11} Buddha rebuked them … “Foolish men, how can you say this? This will affect people’s confidence …” … “And, monks, this training rule should be recited like this: 

\subsection*{Final ruling }

\scrule{‘If a monk says that the senior monks are instructing the nuns for the sake of worldly gain, he commits an offense entailing confession.’” }

\subsection*{Definitions }

\begin{description}%
\item[A: ] whoever … %
\item[Monk: ] … The monk who has been given the full ordination by a unanimous Sangha through a legal procedure consisting of one motion and three announcements that is irreversible and fit to stand—this sort of monk is meant in this case. %
\item[For the sake of worldly gain: ] for the sake of robe-cloth, for the sake of almsfood, for the sake of a dwelling, for the sake of medicinal supplies, for the sake of honor, for the sake of respect, for the sake of deference, for the sake of veneration, for the sake of worship. %
\item[Says: ] if, concerning one who is fully ordained and who has been appointed by the Sangha as an instructor of nuns—desiring to disparage him, desiring to give him a bad reputation, desiring to humiliate him—he says, “He instructs for the sake of robe-cloth,” “… for the sake of almsfood,” “… for the sake of a dwelling,” “… for the sake of medicinal supplies,” “… for the sake of honor,” “… for the sake of respect,” “… for the sake of deference,” “… for the sake of veneration,” “… for the sake of worship,” he commits an offense entailing confession. %
\end{description}

\subsection*{Permutations }

If\marginnote{2.2.1} it is a legitimate legal procedure, and he perceives it as such, and he says such a thing, he commits an offense entailing confession.\footnote{The legal procedure in question is the one that gives approval to a monk to instruct the nuns. Sp 2.150: \textit{\textsanskrit{Adhammakammetiādīsu} \textsanskrit{bhikkhunovādakasammutikammaṁ} kammanti \textsanskrit{veditabbaṁ}}, “\textit{Adhammakamme}: the legal procedure that gives approval to instruct the nuns should be understood.” } If it is a legitimate legal procedure, but he is unsure of it, and he says such a thing, he commits an offense entailing confession. If it is a legitimate legal procedure, but he perceives it as illegitimate, and he says such a thing, he commits an offense entailing confession. 

When\marginnote{2.2.4} someone who is fully ordained is an instructor of nuns, but he has not been appointed by the Sangha as such, and a monk—desiring to disparage him, desiring to give him a bad reputation, desiring to humiliate him—says, “He instructs for the sake of robe-cloth,” “… for the sake of almsfood,” “… for the sake of a dwelling,” “… for the sake of medicines,” “… for the sake of honor,” “… for the sake of respect,” “… for the sake of deference,” “… for the sake of veneration,” “… for the sake of worship,” he commits an offense of wrong conduct. When someone who is not fully ordained is an instructor of nuns, whether or not he has been appointed by the Sangha as such, and a monk—desiring to disparage him, desiring to give him a bad reputation, desiring to humiliate him—says, “He instructs for the sake of robe-cloth,” “… for the sake of almsfood,” “… for the sake of a dwelling,” “… for the sake of medicinal supplies,” “… for the sake of honor,” “… for the sake of respect,” “… for the sake of deference,” “… for the sake of veneration,” “… for the sake of worship,” he commits an offense of wrong conduct. 

If\marginnote{2.2.8} it is an illegitimate legal procedure, but he perceives it as legitimate, he commits an offense of wrong conduct. If it is an illegitimate legal procedure, but he is unsure of it, he commits an offense of wrong conduct. If it is an illegitimate legal procedure, and he perceives it as such, he commits an offense of wrong conduct. 

\subsection*{Non-offenses }

There\marginnote{2.3.1} is no offense: if he says it to one who regularly gives the instruction for the sake of robe-cloth, … for the sake of almsfood, … for the sake of a dwelling, … for the sake of medicinal supplies, … for the sake of honor, … for the sake of respect, … for the sake of deference, … for the sake of veneration, … for the sake of worship; if he is insane; if he is the first offender. 

\scendsutta{The training rule on worldly gain, the fourth, is finished. }

%
\section*{{\suttatitleacronym Bu Pc 25}{\suttatitletranslation 25. The training rule on giving robe-cloth }{\suttatitleroot Cīvaradāna}}
\addcontentsline{toc}{section}{\tocacronym{Bu Pc 25} \toctranslation{25. The training rule on giving robe-cloth } \tocroot{Cīvaradāna}}
\markboth{25. The training rule on giving robe-cloth }{Cīvaradāna}
\extramarks{Bu Pc 25}{Bu Pc 25}

\subsection*{Origin story }

\subsubsection*{First sub-story }

On\marginnote{1.1} one occasion when the Buddha was staying at \textsanskrit{Sāvatthī} in \textsanskrit{Anāthapiṇḍika}’s Monastery, a certain monk was walking for almsfood along a street in \textsanskrit{Sāvatthī}, as was a certain nun. That monk said to that nun, “Go to such-and-such a place, Sister, and you’ll get alms,” and she said the same to him. And because they met frequently, they became friends. 

Just\marginnote{1.6} then robe-cloth belonging to the Sangha was being distributed. Then, after going to the instruction, that nun went to that monk and bowed. He then said to her, “Sister, will you accept my share of the robe-cloth?” 

“Yes,\marginnote{1.9} venerable, my robes are worn.” 

And\marginnote{1.10} he gave his robe-cloth to that nun. As a consequence, his robes, too, became worn. Other monks said to him, “Why don’t you make a robe for yourself?” And he told them what had happened. 

The\marginnote{1.14} monks of few desires complained and criticized him, “How could a monk give robe-cloth to a nun?” … “Is it true, monk, that you did this?” 

“It’s\marginnote{1.17} true, sir.” 

“Is\marginnote{1.18} she a relative of yours?” 

“No.”\marginnote{1.19} 

“Foolish\marginnote{1.20} man, a man and a woman who are unrelated don’t know what’s appropriate and inappropriate, what’s good and bad, in dealing with each other. And still you did this. This will affect people’s confidence …” … “And, monks, this training rule should be recited like this: 

\subsubsection*{Preliminary ruling }

\scrule{‘If a monk gives robe-cloth to an unrelated nun, he commits an offense entailing confession.’” }

In\marginnote{1.25} this way the Buddha laid down this training rule for the monks. 

\subsubsection*{Second sub-story }

Once\marginnote{2.1} this had happened, the monks did not even give robe-cloth to the nuns in exchange, being afraid of wrongdoing. The nuns complained and criticized them, “How can they not give us robe-cloth in exchange?” 

The\marginnote{2.4} monks heard the complaints of those nuns and they told the Buddha. Soon afterwards the Buddha gave a teaching and addressed the monks: 

\scrule{“Monks, I allow you to give things in exchange to five kinds of people: monks, nuns, trainee nuns, novice monks, and novice nuns. }

And\marginnote{2.10} so, monks, this training rule should be recited like this: 

\subsection*{Final ruling }

\scrule{‘If a monk gives robe-cloth to an unrelated nun, except in exchange, he commits an offense entailing confession.’” }

\subsection*{Definitions }

\begin{description}%
\item[A: ] whoever … %
\item[Monk: ] … The monk who has been given the full ordination by a unanimous Sangha through a legal procedure consisting of one motion and three announcements that is irreversible and fit to stand—this sort of monk is meant in this case. %
\item[Unrelated: ] anyone who is not a descendant of one’s male ancestors going back eight generations, either on the mother’s side or on the father’s side.\footnote{Sp 1.505: \textit{Tattha \textsanskrit{yāva} \textsanskrit{sattamā} \textsanskrit{pitāmahayugāti} \textsanskrit{pitupitā} \textsanskrit{pitāmaho}, \textsanskrit{pitāmahassa} \textsanskrit{yugaṁ} \textsanskrit{pitāmahayugaṁ}}, “In this \textit{\textsanskrit{yāva} \textsanskrit{sattamā} \textsanskrit{pitāmahayuga}} means: the father of a father is a grandfather. The generation of a grandfather is called a \textit{\textsanskrit{pitāmahayuga}}.” So the PaIi phrase \textit{\textsanskrit{yāva} \textsanskrit{sattamā} \textsanskrit{pitāmahayuga}} means “as far as the seventh generation of grandfathers”, that is, eight generations back. This can be counted as follows: (1) one’s grandfather; (2) his father; (3) 2’s father; (4) 3’s father; (5) 4’s father; (6) 5’s father; and (7) 6’s father. This applies to both one’s paternal and maternal grandfathers. This gives a total of 14 ancestors. Anyone who is a descendent of these fourteen is considered a relative. Anyone who is not such a descendent is not regarded as a relative. } %
\item[A nun: ] she has been given the full ordination by both Sanghas. %
\item[Robe-cloth: ] one of the six kinds of robe-cloth, but not smaller than what can be assigned to another.\footnote{The six are linen, cotton, silk, wool, sunn hemp, and hemp; see \href{https://suttacentral.net/pli-tv-kd8/en/brahmali\#3.1.6}{Kd 8:3.1.6}. According to \href{https://suttacentral.net/pli-tv-kd8/en/brahmali\#21.1.4}{Kd 8:21.1.4} the size mentioned here is no smaller than 8 by 4 \textit{\textsanskrit{sugataṅgula}}, “standard fingerbreadths”. For an explanation of \textit{\textsanskrit{sugataṅgula}}, the idea of \textit{\textsanskrit{vikappanā}}, and the rendering of \textit{\textsanskrit{cīvara}} as “robe-cloth”, see Appendix of Technical Terms. } %
\item[Except in exchange: ] unless there is an exchange. %
\end{description}

\subsection*{Permutations }

If\marginnote{3.2.1} she is unrelated and he perceives her as such, and he gives her robe-cloth, except in exchange, he commits an offense entailing confession. If she is unrelated, but he is unsure of it, and he gives her robe-cloth, except in exchange, he commits an offense entailing confession. If she is unrelated, but he perceives her as related, and he gives her robe-cloth, except in exchange, he commits an offense entailing confession. 

If\marginnote{3.2.4} he gives robe-cloth to a nun who is fully ordained only on one side, except in exchange, he commits an offense of wrong conduct. If she is related, but he perceives her as unrelated, he commits an offense of wrong conduct. If she is related, but he is unsure of it, he commits an offense of wrong conduct. If she is related and he perceives her as such, there is no offense. 

\subsection*{Non-offenses }

There\marginnote{3.3.1} is no offense: if she is related; if much is exchanged with little or little is exchanged with much; if the nun takes it on trust;\footnote{This refers to a situation where you have an agreement with a close friend that you may take their belongings on trust. The conditions for taking on trust are set out at \href{https://suttacentral.net/pli-tv-kd8/en/brahmali\#19.1.5}{Kd 8:19.1.5}. } if she borrows it; if he gives any requisite apart from robe-cloth; if it is a trainee nun; if it is a novice nun; if he is insane; if he is the first offender. 

\scendsutta{The training rule on giving robe-cloth, the fifth, is finished. }

%
\section*{{\suttatitleacronym Bu Pc 26}{\suttatitletranslation 26. The training rule on sewing robes }{\suttatitleroot Cīvarasibbana}}
\addcontentsline{toc}{section}{\tocacronym{Bu Pc 26} \toctranslation{26. The training rule on sewing robes } \tocroot{Cīvarasibbana}}
\markboth{26. The training rule on sewing robes }{Cīvarasibbana}
\extramarks{Bu Pc 26}{Bu Pc 26}

\subsection*{Origin story }

At\marginnote{1.1} one time the Buddha was staying at \textsanskrit{Sāvatthī} in the Jeta Grove, \textsanskrit{Anāthapiṇḍika}’s Monastery. At that time Venerable \textsanskrit{Udāyī} had become skilled in making robes. On one occasion a certain nun went to \textsanskrit{Udāyī} and said to him, “Venerable, would you please sew me a robe?” 

He\marginnote{1.5} then sewed a robe for that nun, well-dyed and beautifully made, and he drew a picture in the middle of it. He then folded it and put it aside.\footnote{“Picture” renders \textit{\textsanskrit{paṭibhānacittaṁ}}. Sp 2.175: \textit{\textsanskrit{Paṭibhānacittanti} attano \textsanskrit{paṭibhānena} \textsanskrit{katacittaṁ}, so kira \textsanskrit{cīvaraṁ} \textsanskrit{rajitvā} tassa majjhe \textsanskrit{nānāvaṇṇehi} \textsanskrit{vippakatamethunaṁ} \textsanskrit{itthipurisarūpamakāsi}}, “\textit{\textsanskrit{Paṭibhānacitta}}: a picture made through one’s own impromptu imagination. After dyeing the robe, he made a multi-colored picture in the middle of a man and a woman having sexual intercourse.” } Soon afterwards that nun went to \textsanskrit{Udāyī} and said, “Sir, where’s the robe?” 

“Now,\marginnote{1.8} Sister, take this robe as it’s folded and put it aside. When the Sangha of nuns goes to the instruction, then put it on and follow right behind the other nuns.” 

And\marginnote{1.9} that nun did just that. People complained and criticized her, “How indecent these nuns are, what shameless scoundrels, in drawing pictures on their robes!” 

The\marginnote{1.12} nuns asked her, “Who did this?” 

“Venerable\marginnote{1.14} \textsanskrit{Udāyī}.” 

“This\marginnote{1.15} sort of work would not even make an indecent, shameless scoundrel look good, let alone Venerable \textsanskrit{Udāyī}.” 

The\marginnote{1.16} nuns told the monks, and the monks of few desires complained and criticized him, “How could Venerable \textsanskrit{Udāyī} sew a robe for a nun?” … “Is it true, \textsanskrit{Udāyī}, that you did this?” 

“It’s\marginnote{1.20} true, sir.” 

“Is\marginnote{1.21} she a relative of yours?” 

“No.”\marginnote{1.22} 

“Foolish\marginnote{1.23} man, a man and a woman who are unrelated don’t know what’s appropriate and inappropriate, what’s inspiring and uninspiring, in dealing with each other. So how could you do this? This will affect people’s confidence …” … “And, monks, this training rule should be recited like this: 

\subsection*{Final ruling }

\scrule{‘If a monk sews a robe for an unrelated nun, or has one sewn, he commits an offense entailing confession.’” }

\subsection*{Definitions }

\begin{description}%
\item[A: ] whoever … %
\item[Monk: ] … The monk who has been given the full ordination by a unanimous Sangha through a legal procedure consisting of one motion and three announcements that is irreversible and fit to stand—this sort of monk is meant in this case. %
\item[Unrelated: ] anyone who is not a descendant of one’s male ancestors going back eight generations, either on the mother’s side or on the father’s side.\footnote{Sp 1.505: \textit{Tattha \textsanskrit{yāva} \textsanskrit{sattamā} \textsanskrit{pitāmahayugāti} \textsanskrit{pitupitā} \textsanskrit{pitāmaho}, \textsanskrit{pitāmahassa} \textsanskrit{yugaṁ} \textsanskrit{pitāmahayugaṁ}}, “In this \textit{\textsanskrit{yāva} \textsanskrit{sattamā} \textsanskrit{pitāmahayuga}} means: the father of a father is a grandfather. The generation of a grandfather is called a \textit{\textsanskrit{pitāmahayuga}}.” So the PaIi phrase \textit{\textsanskrit{yāva} \textsanskrit{sattamā} \textsanskrit{pitāmahayuga}} means “as far as the seventh generation of grandfathers”, that is, eight generations back. This can be counted as follows: (1) one’s grandfather; (2) his father; (3) 2’s father; (4) 3’s father; (5) 4’s father; (6) 5’s father; and (7) 6’s father. This applies to both one’s paternal and maternal grandfathers. This gives a total of 14 ancestors. Anyone who is a descendent of these fourteen is considered a relative. Anyone who is not such a descendent is not regarded as a relative. } %
\item[A nun: ] she has been given the full ordination by both Sanghas. %
\item[A robe: ] one of the six kinds of robes.\footnote{The six are linen, cotton, silk, wool, sunn hemp, and hemp; see \href{https://suttacentral.net/pli-tv-kd8/en/brahmali\#3.1.6}{Kd 8:3.1.6}. } %
\item[Sews: ] if he sews it himself, then for each stitch he commits an offense entailing confession. %
\item[Has sewn: ] if he asks another, he commits an offense entailing confession. If he only asks once, then even if the other sews a lot, he commits one offense entailing confession.\footnote{“A lot” renders \textit{\textsanskrit{bahukaṁ}}. Sp 2.176: \textit{Bahukampi \textsanskrit{sibbatīti} sacepi \textsanskrit{sabbaṁ} \textsanskrit{sūcikammaṁ} \textsanskrit{pariyosāpetvā} \textsanskrit{cīvaraṁ} \textsanskrit{niṭṭhāpeti}, ekameva \textsanskrit{pācittiyaṁ}}, “\textit{Bahukampi sibbati}: even if he finishes the robe after completing all the needle work, there is just one offense entailing confession.” } %
\end{description}

\subsection*{Permutations }

If\marginnote{2.2.1} she is unrelated and he perceives her as such, and he sews her a robe or has one sewn, he commits an offense entailing confession. If she is unrelated, but he is unsure of it, and he sews her a robe or has one sewn, he commits an offense entailing confession. If she is unrelated, but he perceives her as related, and he sews her a robe or has one sewn, he commits an offense entailing confession. 

If\marginnote{2.2.4} he sews a robe, or has one sewn, for a nun who is fully ordained only on one side, he commits an offense of wrong conduct. If she is related, but he perceives her as unrelated, he commits an offense of wrong conduct. If she is related, but he is unsure of it, he commits an offense of wrong conduct. If she is related and he perceives her as such, there is no offense. 

\subsection*{Non-offenses }

There\marginnote{2.3.1} is no offense: if she is related; if he sews any requisite apart from a robe, or has it sewn; if it is a trainee nun; if it is a novice nun; if he is insane; if he is the first offender. 

\scendsutta{The training rule on sewing robes, the sixth, is finished. }

%
\section*{{\suttatitleacronym Bu Pc 27}{\suttatitletranslation 27. The training rule on arrangements }{\suttatitleroot Saṁvidhāna}}
\addcontentsline{toc}{section}{\tocacronym{Bu Pc 27} \toctranslation{27. The training rule on arrangements } \tocroot{Saṁvidhāna}}
\markboth{27. The training rule on arrangements }{Saṁvidhāna}
\extramarks{Bu Pc 27}{Bu Pc 27}

\subsection*{Origin story }

\subsubsection*{First sub-story }

At\marginnote{1.1} one time the Buddha was staying at \textsanskrit{Sāvatthī} in the Jeta Grove, \textsanskrit{Anāthapiṇḍika}’s Monastery. At that time the monks from the group of six traveled by arrangement with nuns. People complained and criticized them, “Just as we walk about with our wives, so these Sakyan monastics walk about by arrangement with nuns.” 

The\marginnote{1.5} monks heard the complaints of those people, and the monks of few desires complained and criticized those monks, “How can the monks from the group of six travel by arrangement with the nuns?” … “Is it true, monks, that you do this?” 

“It’s\marginnote{1.9} true, sir.” 

The\marginnote{1.10} Buddha rebuked them … “Foolish men, how can you do this? This will affect people’s confidence …” … “And, monks, this training rule should be recited like this: 

\subsubsection*{Preliminary ruling }

\scrule{‘If a monk travels by arrangement with a nun, even just to the next inhabited area, he commits an offense entailing confession.’” }

In\marginnote{1.15} this way the Buddha laid down this training rule for the monks. 

\subsubsection*{Second sub-story }

Soon\marginnote{2.1} afterwards a number of monks and nuns were traveling from \textsanskrit{Sāketa} to \textsanskrit{Sāvatthī}. The nuns said to the monks, “Let’s go together.” 

“Sisters,\marginnote{2.4} it’s not allowable for us to travel by arrangement with nuns. Either you go first, or we will.” 

“You\marginnote{2.6} have the higher status, venerables. Please go first.” 

But\marginnote{2.8} because the nuns went behind, they were robbed and raped by bandits.\footnote{For the rendering of \textit{\textsanskrit{dūsesuṁ}} as “raped”, see Appendix of Technical Terms. } When they arrived at \textsanskrit{Sāvatthī}, they told the nuns there what had happened. The nuns then told the monks, who in turn told the Buddha. 

Soon\marginnote{2.12} afterwards the Buddha gave a teaching and addressed the monks: 

\scrule{“Monks, I allow you to travel by arrangement with a nun if it’s a risky and dangerous road that should be traveled with a group. }

And\marginnote{2.14} so, monks, this training rule should be recited like this: 

\subsection*{Final ruling }

\scrule{‘If a monk travels by arrangement with a nun, even just to the next inhabited area, except on an appropriate occasion, he commits an offense entailing confession. This is the appropriate occasion: the road is considered risky and dangerous and should be traveled with a group.’”\footnote{“Group” renders \textit{sattha}, often translated as “caravan”. Sp-\textsanskrit{ṭ} 1.489: \textit{Satthoti \textsanskrit{jaṅghasattho} \textsanskrit{sakaṭasattho} \textsanskrit{vā}}, “\textit{Sattho}: a \textit{sattha} of travelers on foot or a \textit{sattha} of carts.” } }

\subsection*{Definitions }

\begin{description}%
\item[A: ] whoever … %
\item[Monk: ] … The monk who has been given the full ordination by a unanimous Sangha through a legal procedure consisting of one motion and three announcements that is irreversible and fit to stand—this sort of monk is meant in this case. %
\item[A nun: ] she has been given the full ordination by both Sanghas. %
\item[With: ] together. %
\item[By arrangement: ] if he makes an arrangement like this: he says, “Let’s go, Sister,” and she replies, “Yes, let’s go, venerable;” or she says, “Let’s go, venerable,” and he replies, “Yes, let’s go, Sister;” or he says, “Let’s go today,” “Let’s go tomorrow,” “Let’s go the day after tomorrow,” then he commits an offense of wrong conduct. %
\item[Even just to the next inhabited area:\footnote{For the rendering of \textit{\textsanskrit{gāma}} as “inhabited area”, see Appendix of Technical Terms. } ] when the inhabited areas are a chicken’s flight apart, then for every next inhabited area he commits an offense entailing confession. When it is an uninhabited area, a wilderness, then for every six kilometers he commits an offense entailing confession.\footnote{For a discussion of the \textit{yojana}, see \textit{sugata} in Appendix of Technical Terms. } %
\item[Except on an appropriate occasion: ] unless it is an appropriate occasion. %
\item[The road should be traveled with a group: ] it is not possible to travel without a group. %
\item[Risky: ] a place has been seen along that road where criminals are camping, eating, standing, sitting, or lying down. %
\item[Dangerous: ] criminals have been seen along that road, injuring, robbing, or beating people. If they go together thinking it is dangerous, but then see that it is not, the nuns are to be dismissed, “Go, Sisters.” %
\end{description}

\subsection*{Permutations }

If\marginnote{3.2.1} there is an arrangement, and he perceives that there is, and he travels with a nun, even just to the next inhabited area, except on an appropriate occasion, he commits an offense entailing confession. If there is an arrangement, but he is unsure of it, and he travels with a nun, even just to the next inhabited area, except on an appropriate occasion, he commits an offense entailing confession. If there is an arrangement, but he does not perceive that there is, and he travels with a nun, even just to the next inhabited area, except on an appropriate occasion, he commits an offense entailing confession. 

If\marginnote{3.2.4} the monk makes an arrangement, but the nun does not express her agreement, he commits an offense of wrong conduct. If there is no arrangement, but he perceives that there is, he commits an offense of wrong conduct. If there is no arrangement, but he is unsure of it, he commits an offense of wrong conduct. If there is no arrangement, and he does not perceive that there is, there is no offense. 

\subsection*{Non-offenses }

There\marginnote{3.3.1} is no offense: if it is an appropriate occasion; if he goes without an arrangement; if the nun has made an arrangement, but he has not expressed his agreement; if they go, but not according to the arrangement; if there is an emergency; if he is insane; if he is the first offender. 

\scendsutta{The  training rule on arrangements, the seventh, is finished. }

%
\section*{{\suttatitleacronym Bu Pc 28}{\suttatitletranslation 28. The training rule on boarding boats }{\suttatitleroot Nāvābhiruhana}}
\addcontentsline{toc}{section}{\tocacronym{Bu Pc 28} \toctranslation{28. The training rule on boarding boats } \tocroot{Nāvābhiruhana}}
\markboth{28. The training rule on boarding boats }{Nāvābhiruhana}
\extramarks{Bu Pc 28}{Bu Pc 28}

\subsection*{Origin story }

\subsubsection*{First sub-story }

At\marginnote{1.1} one time the Buddha was staying at \textsanskrit{Sāvatthī} in the Jeta Grove, \textsanskrit{Anāthapiṇḍika}’s Monastery. At that time the monks from the group of six were boarding boats by arrangement with nuns. People complained and criticized them, “Just as we enjoy ourselves on boats with our wives, so these Sakyan monastics make arrangements with the nuns and then enjoy themselves on boats.” 

The\marginnote{1.5} monks heard the complaints of those people, and the monks of few desires complained and criticized those monks, “How can the monks from the group of six board boats by arrangement with nuns?” … “Is it true, monks, that you do this?” 

“It’s\marginnote{1.9} true, sir.” 

The\marginnote{1.10} Buddha rebuked them … “Foolish men, how can you do this? This will affect people’s confidence …” … “And, monks, this training rule should be recited like this: 

\subsubsection*{Preliminary ruling }

\scrule{‘If a monk boards a boat by arrangement with a nun, either to go upstream or downstream, he commits an offense entailing confession.’” }

In\marginnote{1.15} this way the Buddha laid down this training rule for the monks. 

\subsubsection*{Second sub-story }

Soon\marginnote{2.1} afterwards a number of monks and nuns were traveling from \textsanskrit{Sāketa} to \textsanskrit{Sāvatthī}. On the way they needed to cross a river. The nuns said to the monks, “Let’s cross together.” 

“Sisters,\marginnote{2.5} it’s not allowable for us to board a boat by arrangement with a nun. Either you cross first, or we will.” 

“You\marginnote{2.7} have the higher status, venerables. Please go first.” 

But\marginnote{2.9} because the nuns crossed afterwards, they were robbed and raped by bandits. When they arrived at \textsanskrit{Sāvatthī} they told the nuns there what had happened. The nuns then told the monks, who in turn told the Buddha. 

Soon\marginnote{2.13} afterwards the Buddha gave a teaching and addressed the monks: 

\scrule{“Monks, I allow you to board a boat by arrangement with a nun if it’s for the purpose of crossing. }

And\marginnote{2.15} so, monks, this training rule should be recited like this: 

\subsection*{Final ruling }

\scrule{‘If a monk boards a boat by arrangement with a nun, either to go upstream or downstream, except for the purpose of crossing, he commits an offense entailing confession.’” }

\subsection*{Definitions }

\begin{description}%
\item[A: ] whoever … %
\item[Monk: ] … The monk who has been given the full ordination by a unanimous Sangha through a legal procedure consisting of one motion and three announcements that is irreversible and fit to stand—this sort of monk is meant in this case. %
\item[A nun: ] she has been given the full ordination by both Sanghas. %
\item[With: ] together. %
\item[By arrangement: ] if he makes an arrangement like this: he says, “Let’s board, Sister,” and she replies, “Yes, let’s board, venerable;” or she says, “Let’s board, venerable,” and he replies, “Yes, let’s board, Sister;” or he says, “Let’s board today,” “Let’s board tomorrow,” “Let’s board the day after tomorrow,” then he commits an offense of wrong conduct. If the monk boards when the nun has already boarded, he commits an offense entailing confession. If the nun boards when the monk has already boarded, he commits an offense entailing confession. If they both board together, he commits an offense entailing confession. %
\item[To go upstream: ] for the purpose of going against the stream. %
\item[To go downstream: ] for the purpose of going with the stream. %
\item[Except for the purpose of crossing: ] unless it is to go across. %
\end{description}

When\marginnote{3.1.20} the inhabited areas are a chicken’s flight apart, then for every next inhabited area he commits an offense entailing confession. When it is an uninhabited area, a wilderness, then for every six kilometers he commits an offense entailing confession.\footnote{“Six kilometers” renders \textit{addhayojana}, “half a \textit{yojana}”. For further discussion of the \textit{yojana}, see \textit{sugata} in Appendix of Technical Terms. } 

\subsection*{Permutations }

If\marginnote{3.2.1} there is an arrangement, and he perceives that there is, and he boards a boat with a nun, either to go upstream or downstream, except for the purpose of crossing, he commits an offense entailing confession. If there is an arrangement, but he is unsure of it, and he boards a boat with a nun, either to go upstream or downstream, except for the purpose of crossing, he commits an offense entailing confession. If there is an arrangement, but he does not perceive that there is, and he boards a boat with a nun, either to go upstream or downstream, except for the purpose of crossing, he commits an offense entailing confession. 

If\marginnote{3.2.4} the monk makes an arrangement, but the nun does not express her agreement, he commits an offense of wrong conduct. If there is no arrangement, but he perceives that there is, he commits an offense of wrong conduct. If there is no arrangement, but he is unsure of it, he commits an offense of wrong conduct. If there is no arrangement, and he does not perceive that there is, there is no offense. 

\subsection*{Non-offenses }

There\marginnote{3.3.1} is no offense: if it is for the purpose of crossing; if they board without an arrangement; if the nun has made an arrangement, but he has not expressed his agreement; if they board, but not according to the arrangement; if there is an emergency; if he is insane; if he is the first offender. 

\scendsutta{The training rule on boarding boats, the eighth, is finished. }

%
\section*{{\suttatitleacronym Bu Pc 29}{\suttatitletranslation 29. The training rule on had prepared }{\suttatitleroot Paripācita}}
\addcontentsline{toc}{section}{\tocacronym{Bu Pc 29} \toctranslation{29. The training rule on had prepared } \tocroot{Paripācita}}
\markboth{29. The training rule on had prepared }{Paripācita}
\extramarks{Bu Pc 29}{Bu Pc 29}

\subsection*{Origin story }

\subsubsection*{First sub-story }

At\marginnote{1.1} one time the Buddha was staying at \textsanskrit{Rājagaha} in the Bamboo Grove, the squirrel sanctuary. At that time the nun \textsanskrit{Thullanandā} was associating with a family from which she received a regular meal. 

Now\marginnote{1.3} on one occasion the head of that family had invited some senior monks. On the same day, the nun \textsanskrit{Thullanandā} robed up in the morning, took her bowl and robe, and went to that family. And she asked the head of the family, “Why have you prepared so much food?” 

“’Cause,\marginnote{1.6} venerable, I’ve invited the senior monks.” 

“But\marginnote{1.7} who are those senior monks?” 

“Venerable\marginnote{1.8} \textsanskrit{Sāriputta}, Venerable \textsanskrit{Mahāmoggallāna}, Venerable \textsanskrit{Mahākaccāna}, Venerable \textsanskrit{Mahākoṭṭhika}, Venerable \textsanskrit{Mahākappina}, Venerable \textsanskrit{Mahācunda}, Venerable Anuruddha, Venerable Revata, Venerable \textsanskrit{Upāli}, Venerable Ānanda, and Venerable \textsanskrit{Rāhula}.” 

“But\marginnote{1.9} why do you invite such inferior monks instead of the great ones?” 

“Who\marginnote{1.10} are these great monks?” “Venerable Devadatta, Venerable \textsanskrit{Kokālika}, Venerable \textsanskrit{Kaṭamodakatissaka}, Venerable \textsanskrit{Khaṇḍadeviyāputta}, and Venerable Samuddadatta.” 

While\marginnote{1.12} this conversation was taking place, the senior monks entered. \textsanskrit{Thullanandā} said, “Is it true that you’ve invited these great monks?” 

“Just\marginnote{1.14} before you called them inferior and now you call them great.” And that lay person threw her out of the house and made an end of her regular meal. 

The\marginnote{1.16} monks of few desires complained and criticized Devadatta, “How could Devadatta eat almsfood knowing that a nun had it prepared?” … “Is it true, Devadatta, that you did this?” 

“It’s\marginnote{1.19} true, sir.” 

The\marginnote{1.20} Buddha rebuked him … “Foolish man, how could you do this? This will affect people’s confidence …” … “And, monks, this training rule should be recited like this: 

\subsubsection*{Preliminary ruling }

\scrule{‘If a monk eats almsfood knowing that a nun had it prepared, he commits an offense entailing confession.’” }

In\marginnote{1.25} this way the Buddha laid down this training rule for the monks. 

\subsubsection*{Second sub-story }

Soon\marginnote{2.1} afterwards a monk who had earlier left \textsanskrit{Rājagaha} returned to see his family. Because it was a long time since he had last returned, people prepared food for him respectfully. And the nun who was associating with that family said to them, “Give food to that monk.” The monk thought, “The Buddha has prohibited us from eating almsfood knowing that a nun had it prepared,” and being afraid of wrongdoing, he did not accept it. And because he was unable to walk for alms, he missed his meal. 

After\marginnote{2.8} returning to the monastery, he told the monks what had happened, and they in turn told the Buddha. 

Soon\marginnote{2.10} afterwards the Buddha gave a teaching and addressed the monks: 

\scrule{“Monks, I allow you to eat almsfood knowing that a nun had it prepared if the householder had intended to prepare it anyway. }

And\marginnote{2.12} so, monks, this training rule should be recited like this: 

\subsection*{Final ruling }

\scrule{‘If a monk eats almsfood knowing that a nun had it prepared, except if the householder had intended to prepare it anyway, he commits an offense entailing confession.’” }

\subsection*{Definitions }

\begin{description}%
\item[A: ] whoever … %
\item[Monk: ] … The monk who has been given the full ordination by a unanimous Sangha through a legal procedure consisting of one motion and three announcements that is irreversible and fit to stand—this sort of monk is meant in this case. %
\item[Knowing: ] he knows by himself or others have told him or the nun has told him.\footnote{The meaning of the last of these three ways of knowing, \textit{\textsanskrit{sā} \textsanskrit{vā} \textsanskrit{āroceti}}, presumably refers to the nun telling the monk directly. } %
\item[A nun: ] she has been given the full ordination by both Sanghas. %
\item[Has it prepared: ] if she says to those who do not already want to give or want to prepare, “This monk is a reciter,” “This monk is learned,” “This monk is an expert on the discourses,” “This monk is an expert on the Monastic Law,” “This monk is an expounder of the Teaching;” “Give to this monk,” “Prepare for this monk”—this is called “has it prepared”. %
\item[Almsfood: ] any of the five cooked foods. %
\item[Except if the householder had intended to prepare it anyway: ] unless the householder had intended to prepare it. %
\item[The householder had intended to prepare it: ] they are relatives or they have invited or they give regularly. %
\end{description}

If\marginnote{3.1.17} he receives it with the intention of eating it, except if the householder had intended to prepare it anyway, he commits an offense of wrong conduct. For every mouthful swallowed, he commits an offense entailing confession. 

\subsection*{Permutations }

If\marginnote{3.2.1} a nun had it prepared, and he perceives it as such, and he eats it, except if the householder had intended to prepare it anyway, he commits an offense entailing confession. If a nun had it prepared, but he is unsure of it, and he eats it, except if the householder had intended to prepare it anyway, he commits an offense of wrong conduct. If a nun had it prepared, but he does not perceive it as such, and he eats it, except if the householder had intended to prepare it anyway, there is no offense. 

If\marginnote{3.2.4} a nun who is fully ordained only on one side had it prepared, and he eats it, except if the householder had intended to prepare it anyway, he commits an offense of wrong conduct. If a nun did not have it prepared, but he perceives it as such, he commits an offense of wrong conduct. If a nun did not have it prepared, but he is unsure of it, he commits an offense of wrong conduct. If a nun did not have it prepared, and he does not perceive it as such, there is no offense. 

\subsection*{Non-offenses }

There\marginnote{3.3.1} is no offense: if the householder had intended to prepare it anyway; if a trainee nun has it prepared; if a novice nun has it prepared; if it is anything apart from the five cooked foods; if he is insane; if he is the first offender. 

\scendsutta{The training rule on had prepared, the ninth, is finished. }

%
\section*{{\suttatitleacronym Bu Pc 30}{\suttatitletranslation 30. The training rule on sitting in private }{\suttatitleroot Rahonisajja}}
\addcontentsline{toc}{section}{\tocacronym{Bu Pc 30} \toctranslation{30. The training rule on sitting in private } \tocroot{Rahonisajja}}
\markboth{30. The training rule on sitting in private }{Rahonisajja}
\extramarks{Bu Pc 30}{Bu Pc 30}

\subsection*{Origin story }

At\marginnote{1.1} one time when the Buddha was staying at \textsanskrit{Sāvatthī} in \textsanskrit{Anāthapiṇḍika}’s Monastery, Venerable \textsanskrit{Udāyī}’s ex-wife became a nun. She often went to see \textsanskrit{Udāyī}, and he often went to see her. And \textsanskrit{Udāyī} would sit down in private alone with that nun. 

The\marginnote{1.5} monks of few desires complained and criticized him, “How can Venerable \textsanskrit{Udāyī} sit down in private alone with a nun?” … “Is it true, \textsanskrit{Udāyī}, that you do this?” 

“It’s\marginnote{1.8} true, sir.” 

The\marginnote{1.9} Buddha rebuked him … “Foolish man, how can you do this? This will affect people’s confidence …” … “And, monks, this training rule should be recited like this: 

\subsection*{Final ruling }

\scrule{‘If a monk sits down in private alone with a nun, he commits an offense entailing confession.’” }

\subsection*{Definitions }

\begin{description}%
\item[A: ] whoever … %
\item[Monk: ] … The monk who has been given the full ordination by a unanimous Sangha through a legal procedure consisting of one motion and three announcements that is irreversible and fit to stand—this sort of monk is meant in this case. %
\item[A nun: ] she has been given the full ordination by both Sanghas. %
\item[With: ] together. %
\item[Alone: ] just the monk and the nun. %
\item[In private: ] private to the eye and private to the ear. %
\item[Private to the eye: ] one is unable to see them winking, raising an eyebrow, or nodding. %
\item[Private to the ear: ] one is unable to hear ordinary speech. %
\item[Sits: ] if the monk sits down or lies down next to the seated nun, he commits an offense entailing confession. If the nun sits down or lies down next to the seated monk, he commits an offense entailing confession. If both are seated or both are lying down, he commits an offense entailing confession. %
\end{description}

\subsection*{Permutations }

If\marginnote{2.2.1} it is private, and he perceives it as such, and he sits down alone with a nun, he commits an offense entailing confession. If it is private, but he is unsure of it, and he sits down alone with a nun, he commits an offense entailing confession. If it is private, but he does not perceive it as such, and he sits down alone with a nun, he commits an offense entailing confession. 

If\marginnote{2.2.4} it is not private, but he perceives it as such, he commits an offense of wrong conduct. If it is not private, but he is unsure of it, he commits an offense of wrong conduct. If it is not private, and he does not perceive it as such, there is no offense. 

\subsection*{Non-offenses }

There\marginnote{2.3.1} is no offense: if he has a companion who understands; if he stands and does not sit down; if he is not seeking privacy; if he sits down preoccupied with something else;\footnote{Sp 5.467: \textit{\textsanskrit{Aññavihitoti} \textsanskrit{aññaṁ} \textsanskrit{cintayamāno}}, “\textit{\textsanskrit{Aññavihita}}: thinking of something else.” } if he is insane; if he is the first offender. 

\scendsutta{The training rule on sitting in private, the tenth, is finished. }

\scendvagga{The third subchapter on the instruction is finished. }

\scuddanaintro{This is the summary: }

\begin{scuddana}%
“Not\marginnote{2.3.11} appointed, set, \\
Dwelling place, worldly gain, and with giving; \\
He sews, a road, a boat, should eat, \\
Alone: those are the ten.” 

%
\end{scuddana}

%
\section*{{\suttatitleacronym Bu Pc 31}{\suttatitletranslation 31. The training rule on almsmeals at public guesthouses }{\suttatitleroot Āvasathapiṇḍa}}
\addcontentsline{toc}{section}{\tocacronym{Bu Pc 31} \toctranslation{31. The training rule on almsmeals at public guesthouses } \tocroot{Āvasathapiṇḍa}}
\markboth{31. The training rule on almsmeals at public guesthouses }{Āvasathapiṇḍa}
\extramarks{Bu Pc 31}{Bu Pc 31}

\subsection*{Origin story }

\subsubsection*{First sub-story }

At\marginnote{1.1} one time when the Buddha was staying at \textsanskrit{Sāvatthī} in \textsanskrit{Anāthapiṇḍika}’s Monastery, a certain association was preparing an almsmeal at a public guesthouse not far from \textsanskrit{Sāvatthī}. 

Then,\marginnote{1.3} after robing up in the morning, the monks from the group of six took their bowls and robes and entered \textsanskrit{Sāvatthī} for alms. Not getting anything, they went to that public guesthouse. Because it was a long time since they had been there, people served them respectfully. 

A\marginnote{1.5} second and a third day those monks did the same thing. Then they thought, “What’s the point of returning to the monastery? Tomorrow we’ll just have to come back here.” So they stayed on and on right there, eating alms at the guesthouse, while the monastics of other religions left. People complained and criticized them, “How can the Sakyan monastics stay on and on, eating alms at the guesthouse? We don’t prepare the almsfood just for them; we prepare it for everyone.” 

The\marginnote{1.15} monks heard the complaints of those people, and the monks of few desires complained and criticized those monks, “How could the monks from the group of six stay on and on, eating alms at a public guesthouse?” … “Is it true, monks, that you did this?” 

“It’s\marginnote{1.19} true, sir.” 

The\marginnote{1.20} Buddha rebuked them … “Foolish men, how could you do this? This will affect people’s confidence …” … “And, monks, this training rule should be recited like this: 

\subsubsection*{Preliminary ruling }

\scrule{‘If a monk eats more than one almsmeal at a public guesthouse, he commits an offense entailing confession.’” }

In\marginnote{1.25} this way the Buddha laid down this training rule for the monks. 

\subsubsection*{Second sub-story }

Soon\marginnote{2.1} afterwards Venerable \textsanskrit{Sāriputta} was traveling through the Kosalan country on his way to \textsanskrit{Sāvatthī} when he came to a public guesthouse. Because it was a long time since he had been there, people served him respectfully. After he had eaten, \textsanskrit{Sāriputta} became severely ill, and he was unable to leave that guesthouse. 

On\marginnote{2.4} the second day, too, those people said to him, “Please eat, venerable.” But since he knew that the Buddha had prohibited eating alms at a public guesthouse after staying on and on, and because he was afraid of wrongdoing, he did not accept. As a consequence, he missed his meal. 

When\marginnote{2.9} he arrived at \textsanskrit{Sāvatthī}, he told the monks what had happened, and they in turn told the Buddha. 

Soon\marginnote{2.11} afterwards the Buddha gave a teaching and addressed the monks: 

\scrule{“Monks, I allow a sick monk to stay on at a public guesthouse and eat alms there. }

And\marginnote{2.13} so, monks, this training rule should be recited like this: 

\subsection*{Final ruling }

\scrule{‘If a monk who is not sick eats more than one almsmeal at a public guesthouse, he commits an offense entailing confession.’” }

\subsection*{Definitions }

\begin{description}%
\item[Who is not sick: ] he is able to leave that public guesthouse. %
\item[Who is sick: ] he is unable to leave that public guesthouse. %
\item[Almsmeal at a public guesthouse: ] as much as one needs of any of the five cooked foods, prepared for the general public, in a building, under a roof cover, at the foot of a tree, or out in the open.\footnote{For the rendering of \textit{\textsanskrit{sālā}} as “building”, see Appendix of Technical Terms. } A monk who is not sick may eat there once. If he receives food beyond that with the intention of eating it, he commits an offense of wrong conduct. For every mouthful swallowed, he commits an offense entailing confession. %
\end{description}

\subsection*{Permutations }

If\marginnote{3.2.1} he is not sick, and he perceives himself as not sick, and he eats more than one almsmeal at a public guesthouse, he commits an offense entailing confession. If he is not sick, but he is unsure of it, and he eats more than one almsmeal at a public guesthouse, he commits an offense entailing confession. If he is not sick, but he perceives himself as sick, and he eats more than one almsmeal at a public guesthouse, he commits an offense entailing confession. 

If\marginnote{3.2.4} he is sick, but he perceives himself as not sick, he commits an offense of wrong conduct. If he is sick, but he is unsure of it, he commits an offense of wrong conduct. If he is sick, and he perceives himself as sick, there is no offense. 

\subsection*{Non-offenses }

There\marginnote{3.3.1} is no offense: if he is sick; if he is not sick and he eats once; if he eats while coming or going;\footnote{Sp 2.208: \textit{Gacchanto \textsanskrit{vāti} yo gacchanto \textsanskrit{antarāmagge} \textsanskrit{ekadivasaṁ} \textsanskrit{gataṭṭhāne} ca \textsanskrit{ekadivasaṁ} \textsanskrit{bhuñjati}, \textsanskrit{tassāpi} \textsanskrit{anāpatti}. Āgacchantepi eseva nayo. \textsanskrit{Gantvā} \textsanskrit{paccāgacchantopi} \textsanskrit{antarāmagge} \textsanskrit{ekadivasaṁ} \textsanskrit{āgataṭṭhāne} ca \textsanskrit{ekadivasaṁ} \textsanskrit{bhuñjituṁ} labhati}, “Or going means: whoever is going, if he eats one day on the road and one day at the place he has gone to, there is no offense. The same method applies for one who is coming. After going, even when returning, he may eat one day on the road and one day at the place returned to.” The point seems to be that one may return to the same guesthouse even after one day of traveling. } if he eats after being invited by the owners; if the food is prepared specifically for him; if there is not as much as he needs; if it is anything apart from the five cooked foods; if he is insane; if he is the first offender. 

\scendsutta{The training rule on almsmeals at public guesthouses, the first, is finished. }

%
\section*{{\suttatitleacronym Bu Pc 32}{\suttatitletranslation 32. The training rule on eating in a group }{\suttatitleroot Gaṇabhojana}}
\addcontentsline{toc}{section}{\tocacronym{Bu Pc 32} \toctranslation{32. The training rule on eating in a group } \tocroot{Gaṇabhojana}}
\markboth{32. The training rule on eating in a group }{Gaṇabhojana}
\extramarks{Bu Pc 32}{Bu Pc 32}

\subsection*{Origin story }

\subsubsection*{First sub-story }

At\marginnote{1.1} one time the Buddha was staying at \textsanskrit{Rājagaha} in the Bamboo Grove, the squirrel sanctuary. At that time, because of his loss of material support and honor, Devadatta and his followers had to ask families repeatedly to get invited to meals. People complained and criticized him, “How can the Sakyan monastics repeatedly ask families to get invited to meals? Who doesn’t like nice food? Who doesn’t prefer tasty food?” 

The\marginnote{1.6} monks heard the complaints of those people, and the monks of few desires complained and criticized those monks, “How can Devadatta and his followers repeatedly ask families to get invited to meals?” … “Is it true, Devadatta, that you do this?” 

“It’s\marginnote{1.10} true, sir.” 

The\marginnote{1.11} Buddha rebuked him … “Foolish man, how can you do this? This will affect people’s confidence …” … “And, monks, this training rule should be recited like this: 

\subsubsection*{First preliminary ruling }

\scrule{‘If a monk eats in a group, he commits an offense entailing confession.’” }

In\marginnote{1.16} this way the Buddha laid down this training rule for the monks. 

\subsubsection*{Second sub-story }

Soon\marginnote{2.1} afterwards people invited sick monks to a meal. But knowing that the Buddha had prohibited eating in a group and being afraid of wrongdoing, they did not accept. They told the Buddha. Soon afterwards he gave a teaching and addressed the monks: 

\scrule{“Monks, I allow a sick monk to eat in a group. }

And\marginnote{2.6} so, monks, this training rule should be recited like this: 

\subsubsection*{Second preliminary ruling }

\scrule{‘If a monk eats in a group, except on an appropriate occasion, he commits an offense entailing confession. This is the appropriate occasion: he is sick.’” }

In\marginnote{2.9} this way the Buddha laid down this training rule for the monks. 

\subsubsection*{Third sub-story }

Soon\marginnote{3.1} afterwards, during the robe-giving season, people prepared a meal together with robe-cloth and then invited the monks, saying, “We wish to offer a meal and then give robe-cloth.” But knowing that the Buddha had prohibited eating in a group and being afraid of wrongdoing, they did not accept. As a result, they only got a small amount of robe-cloth. They told the Buddha. … 

\scrule{“Monks, I allow you to eat in a group during the robe-giving season. }

And\marginnote{3.7} so, monks, this training rule should be recited like this: 

\subsubsection*{Third preliminary ruling }

\scrule{‘If a monk eats in a group, except on an appropriate occasion, he commits an offense entailing confession. These are the appropriate occasions: he is sick; it is the robe-giving season.’” }

In\marginnote{3.10} this way the Buddha laid down this training rule for the monks. 

\subsubsection*{Fourth sub-story }

Soon\marginnote{4.1} afterwards people invited the robe-making monks for a meal. But knowing that the Buddha had prohibited eating in a group and being afraid of wrongdoing, they did not accept. They told the Buddha. … 

\scrule{“Monks, I allow you to eat in a group at a time when you are making robes. }

And\marginnote{4.5} so, monks, this training rule should be recited like this: 

\subsubsection*{Fourth preliminary ruling }

\scrule{‘If a monk eats in a group, except on an appropriate occasion, he commits an offense entailing confession. These are the appropriate occasions: he is sick; it is the robe-giving season; it is a time of making robes.’” }

In\marginnote{4.8} this way the Buddha laid down this training rule for the monks. 

\subsubsection*{Fifth sub-story }

Soon\marginnote{5.1} afterwards some monks went traveling with a group of people. The monks said to those people, “Please wait a moment while we walk for alms.” They replied, “Venerables, please eat right here.” But knowing that the Buddha had prohibited eating in a group and being afraid of wrongdoing, they did not accept. They told the Buddha. … 

\scrule{“Monks, I allow you to eat in a group when you’re traveling. }

And\marginnote{5.9} so, monks, this training rule should be recited like this: 

\subsubsection*{Fifth preliminary ruling }

\scrule{‘If a monk eats in a group, except on an appropriate occasion, he commits an offense entailing confession. These are the appropriate occasions: he is sick; it is the robe-giving season; it is a time of making robes; he is traveling.’” }

In\marginnote{5.12} this way the Buddha laid down this training rule for the monks. 

\subsubsection*{Sixth sub-story }

Soon\marginnote{6.1} afterwards some monks were traveling by boat with a group of people. The monks said to those people, “Please go to the shore for a moment while we walk for alms.” They replied, “Venerables, please eat right here.” But knowing that the Buddha had prohibited eating in a group and being afraid of wrongdoing, they did not accept. They told the Buddha. … 

\scrule{“Monks, I allow you to eat in a group when on board a boat. }

And\marginnote{6.7} so, monks, this training rule should be recited like this: 

\subsubsection*{Sixth preliminary ruling }

\scrule{‘If a monk eats in a group, except on an appropriate occasion, he commits an offense entailing confession. These are the appropriate occasions: he is sick; it is the robe-giving season; it is a time of making robes; he is traveling; he is on a boat.’” }

In\marginnote{6.10} this way the Buddha laid down this training rule for the monks. 

\subsubsection*{Seventh sub-story }

Soon\marginnote{7.1} afterwards monks who had completed the rainy-season residence in various regions were coming to \textsanskrit{Rājagaha} to visit the Buddha. People saw those monks who had come from various countries and invited them for a meal. But knowing that the Buddha had prohibited eating in a group and being afraid of wrongdoing, they did not accept. They told the Buddha. … 

\scrule{“Monks, I allow you to eat in a group on big occasions. }

And\marginnote{7.6} so, monks, this training rule should be recited like this: 

\subsubsection*{Seventh preliminary ruling }

\scrule{‘If a monk eats in a group, except on an appropriate occasion, he commits an offense entailing confession. These are the appropriate occasions: he is sick; it is the robe-giving season; it is a time of making robes; he is traveling; he is on a boat; it is a big occasion.’” }

In\marginnote{7.9} this way the Buddha laid down this training rule for the monks. 

\subsubsection*{Eighth sub-story }

Soon\marginnote{8.1} afterwards a relative of King Seniya \textsanskrit{Bimbisāra} of Magadha who had gone forth with the \textsanskrit{Ājīvaka} ascetics went to the king and said, “Great king, I wish to make a meal for the monastics of all religions.” 

“That’s\marginnote{8.2} fine, sir, if you first feed the Sangha of monks headed by the Buddha.” 

“I’ll\marginnote{8.3} do that.” 

And\marginnote{8.4} he sent a message to the monks: “Please accept a meal from me tomorrow.” But knowing that the Buddha had prohibited eating in a group and being afraid of wrongdoing, they did not accept. That \textsanskrit{Ājīvaka} ascetic then went to the Buddha, exchanged pleasantries with him, and said, “Good Gotama has gone forth and so have I. One who has gone forth should receive alms from another who has gone forth. Good Gotama, please accept a meal from me tomorrow together with the Sangha of monks.” The Buddha consented by remaining silent. The \textsanskrit{Ājīvaka} understood that the Buddha had consented, and he left. 

Soon\marginnote{8.12} afterwards the Buddha gave a teaching and addressed the monks: 

\scrule{“Monks, I allow you to eat in a group when the meal is given by a monastic. }

And\marginnote{8.14} so, monks, this training rule should be recited like this: 

\subsection*{Final ruling }

\scrule{‘If a monk eats in a group, except on an appropriate occasion, he commits an offense entailing confession. These are the appropriate occasions: he is sick; it is the robe-giving season; it is a time of making robes; he is traveling; he is on a boat; it is a big occasion; it is a meal given by a monastic.’” }

\subsection*{Definitions }

\begin{description}%
\item[Eats in a group: ] wherever four monks, after being invited, eat any of the five cooked foods—this is called “eats in a group”. %
\item[Except on an appropriate occasion: ] unless it is an appropriate occasion. %
\item[He is sick: ] even if he has cracked feet, he may eat in a group. %
\item[It is the robe-giving season: ] if he has not participated in the robe-making ceremony, he may eat in a group during the last month of the rainy season. If he has participated in the robe-making ceremony, he may eat in a group during the five month period.\footnote{“Robe-making ceremony” refers to the \textit{kathina \textsanskrit{saṅghakamma}}, the making of the \textit{kathina} robe, and the rejoicing in the process, all three together represented by the words \textit{(an)atthate kathine }. “The five month period” is the last month of the rainy season plus the four months of the cold season. } %
\item[It is a time of making robes: ] when he is making robes, he may eat in a group.\footnote{Sp 2.218 says: \textit{Yo tattha \textsanskrit{cīvare} \textsanskrit{kattabbaṁ} \textsanskrit{yaṅkiñci} \textsanskrit{kammaṁ} karoti, \textsanskrit{mahāpaccariyañhi} “antamaso \textsanskrit{sūcivedhanako}”tipi \textsanskrit{vuttaṁ}, tena \textsanskrit{cīvarakārasamayoti} \textsanskrit{bhuñjitabbaṁ}}, “Whoever does any work to be done in regard to the robe—the \textsanskrit{Mahāpaccariya} says, ‘Even one who just threads the needle’—may eat in a group because he is making a robe.” } %
\item[He is traveling: ] he may eat in a group if he intends to travel at least six kilometers, while traveling, and after traveling.\footnote{For a discussion of the \textit{yojana}, see \textit{sugata} in Appendix of Technical Terms. } %
\item[He is on a boat: ] he may eat in a group if he intends to board a boat, while on board, and after disembarking. %
\item[It is a big occasion: ] if two or three monks can get by on walking for alms, but not a group of four, he may eat in a group. %
\item[It is a meal given by a monastic: ] if any kind of wanderer is making the meal, he may eat in a group. %
\end{description}

If\marginnote{9.1.19} he receives something intending to eat it, except on an appropriate occasion, he commits an offense of wrong conduct. For every mouthful swallowed, he commits an offense entailing confession. 

\subsection*{Permutations }

If\marginnote{9.2.1} he eats in a group, and he perceives it as such, except on an appropriate occasion, he commits an offense entailing confession. If he eats in a group, but he is unsure of it, except on an appropriate occasion, he commits an offense entailing confession. If he eats in a group, but he does not perceive it as such, except on an appropriate occasion, he commits an offense entailing confession. 

If\marginnote{9.2.4} he does not eat in a group, but he perceives it as such, he commits an offense of wrong conduct. If he does not eat in a group, but he is unsure of it, he commits an offense of wrong conduct. If he does not eat in a group, and he does not perceive it as such, there is no offense. 

\subsection*{Non-offenses }

There\marginnote{9.3.1} is no offense: if it is an appropriate occasion; if two or three eat together; if they eat together after walking for alms; if it is a regular meal invitation; if it is a meal for which lots are drawn; if it is a half-monthly meal;\footnote{Sp-\textsanskrit{ṭ} 1.30: \textit{\textsanskrit{Ekasmiṁ} pakkhe \textsanskrit{ekadivasaṁ} \textsanskrit{dātabbaṁ} \textsanskrit{bhattaṁ} \textsanskrit{pakkhikaṁ}}, “\textit{\textsanskrit{Pakkhikaṁ}} means: meal to be given one day in one half-month.” } if it is on the observance day; if it is on the day after the observance day; if it is anything apart from the five cooked foods; if he is insane; if he is the first offender. 

\scendsutta{The training rule on eating in a group, the second, is finished. }

%
\section*{{\suttatitleacronym Bu Pc 33}{\suttatitletranslation 33. The training rule on eating a meal before another }{\suttatitleroot Paramparabhojana}}
\addcontentsline{toc}{section}{\tocacronym{Bu Pc 33} \toctranslation{33. The training rule on eating a meal before another } \tocroot{Paramparabhojana}}
\markboth{33. The training rule on eating a meal before another }{Paramparabhojana}
\extramarks{Bu Pc 33}{Bu Pc 33}

\subsection*{Origin story }

\subsubsection*{First sub-story }

At\marginnote{1.1} one time when the Buddha was staying in the hall with the peaked roof in the Great Wood near \textsanskrit{Vesālī}, a succession of fine meals had been arranged in \textsanskrit{Vesālī}. A certain poor worker thought, “Why don’t I prepare a meal? It must be really worthwhile, seeing as these people prepare a meal with such respect.” 

He\marginnote{1.4} then went to his boss Kira and said, “Sir, I wish to prepare a meal for the Sangha of monks headed by the Buddha. Please give me my salary.” Because Kira also had faith and confidence, he gave the worker his salary and much extra. Soon afterwards that worker went to the Buddha, bowed, sat down, and said, “Sir, please accept a meal from me tomorrow together with the Sangha of monks.” 

“The\marginnote{1.10} Sangha is large.” 

“No\marginnote{1.11} problem! I’ve prepared much jujube, supplemented with jujube drinks.” The Buddha consented by remaining silent, and the worker understood. 

He\marginnote{1.14} got up from his seat, circumambulated the Buddha with his right side toward him, and left. 

The\marginnote{1.15} monks heard that a poor worker had invited the Sangha of monks headed by the Buddha on the following day for a meal supplemented with jujube drinks. And so they ate in the morning after walking for alms. 

When\marginnote{1.17} people heard that a poor worker had invited the Sangha of monks headed by the Buddha for a meal, they brought much food of various kinds to him. The following morning that worker prepared his food, and then had the Buddha informed that the meal was ready. 

The\marginnote{1.19} Buddha robed up, took his bowl and robe and, together with the Sangha of monks, went to the house of that poor worker where he sat down on the prepared seat in the dining hall. The worker served the monks, but they kept saying, “Give just a little.” 

“Venerables,\marginnote{1.22} don’t accept so little because you think I’m just a poor worker. I’ve prepared much food of various kinds. Please accept as much as you like.” 

“We’re\marginnote{1.25} not accepting so little because of that, but because we ate in the morning after walking for alms.” 

That\marginnote{1.27} poor worker complained and criticized them, “How could the venerables eat elsewhere when invited by me? Am I not able to give as much as they need?” 

The\marginnote{1.30} monks heard the complaints of that worker, and the monks of few desires complained and criticized those monks, “How could those monks eat elsewhere when invited for a meal?” … “Is it true, monks, that monks did this?” 

“It’s\marginnote{1.34} true, sir.” 

The\marginnote{1.35} Buddha rebuked them … “How could those foolish men do this? This will affect people’s confidence …” … “And, monks, this training rule should be recited like this: 

\subsubsection*{First preliminary ruling }

\scrule{‘If a monk eats a meal before another, he commits an offense entailing confession.’” }

In\marginnote{1.40} this way the Buddha laid down this training rule for the monks. 

\subsubsection*{Second sub-story }

Soon\marginnote{2.1} afterwards a certain monk became sick. Another monk took some almsfood, went to that monk, and told him to eat it. 

“I\marginnote{2.3} can’t. I’m expecting another meal.” 

But\marginnote{2.4} since that almsfood only arrived at midday, that monk did not get to eat as much as he had intended. They told the Buddha. Soon afterwards the Buddha gave a teaching and addressed the monks: 

\scrule{“Monks, I allow a sick monk to eat a meal before another. }

And\marginnote{2.8} so, monks, this training rule should be recited like this: 

\subsubsection*{Second preliminary ruling }

\scrule{‘If a monk eats a meal before another, except on an appropriate occasion, he commits an offense entailing confession. This is the appropriate occasion: he is sick.’” }

In\marginnote{2.11} this way the Buddha laid down this training rule for the monks. 

\subsubsection*{Third sub-story }

Soon\marginnote{3.1} afterwards, during the robe-giving season, people prepared a meal together with robe-cloth and then invited the monks, saying, “We wish to offer a meal and give robe-cloth.” But knowing that the Buddha had prohibited eating a meal before another and being afraid of wrongdoing, they did not accept. As a result, they only got a small amount of robe-cloth. They told the Buddha. … 

\scrule{“Monks, I allow you to eat a meal before another during the robe-giving season. }

And\marginnote{3.6} so, monks, this training rule should be recited like this: 

\subsubsection*{Third preliminary ruling }

\scrule{‘If a monk eats a meal before another, except on an appropriate occasion, he commits an offense entailing confession. These are the appropriate occasions: he is sick; it is the robe-giving season.’” }

In\marginnote{3.9} this way the Buddha laid down this training rule for the monks. 

\subsubsection*{Fourth sub-story }

Soon\marginnote{3.10.1} afterwards people invited the robe-making monks for a meal. But knowing that the Buddha had prohibited eating a meal before another and being afraid of wrongdoing, they did not accept. They told the Buddha. … 

\scrule{“Monks, I allow you to eat a meal before another at a time when you are making robes. }

And\marginnote{3.14} so, monks, this training rule should be recited like this: 

\subsection*{Final ruling }

\scrule{‘If a monk eats a meal before another, except on an appropriate occasion, he commits an offense entailing confession. These are the appropriate occasions: he is sick; it is the robe-giving season; it is a time of making robes.’” }

In\marginnote{3.17} this way the Buddha laid down this training rule for the monks. 

\subsubsection*{Fifth sub-story }

Soon\marginnote{4.1} afterwards, after robing up in the morning, the Buddha took his bowl and robe and went to a certain family with Venerable Ānanda as his attendant. He sat down on the prepared seat, and the people there gave cooked food. Being afraid of wrongdoing, Ānanda did not accept it. The Buddha said, “Accept it, Ānanda.” 

“I\marginnote{4.5} can’t, sir, I’m expecting another meal.” 

“Well\marginnote{4.6} then, Ānanda, assign that meal to someone else and then receive this.” 

Soon\marginnote{4.7} afterwards the Buddha gave a teaching and addressed the monks: 

\scrule{“Monks, I allow you to eat a meal before another if you assign the other meal to someone else. }

And,\marginnote{4.9} monks, this is how it should be assigned: ‘I give my expected meal to so-and-so.’” 

\subsection*{Definitions }

\begin{description}%
\item[Eats a meal before another: ] if he has been invited to eat any of the five cooked foods, and he then eats any of the five cooked foods elsewhere—this is called “eats a meal before another”. %
\item[Except on an appropriate occasion: ] unless it is an appropriate occasion. %
\item[He is sick: ] if he is not able to eat as much as he needs in one sitting, he may eat a meal before another. %
\item[It is the robe-giving season: ] if he has not participated in the robe-making ceremony, he may eat a meal before another during the last month of the rainy season. If he has participated in the robe-making ceremony, he may eat a meal before another during the five month period.\footnote{“The five month period” is the last month of the rainy season plus the four months of the cold season. “Robe-making procedure” refers to the \textit{kathina \textsanskrit{saṅghakamma}}, here represented by the words \textit{(an)atthate kathine }. } %
\item[It is a time of making robes: ] when he is making robes, he may eat a meal before another. %
\end{description}

If\marginnote{5.1.11} he receives food with the intention of eating it, except on an appropriate occasion, he commits an offense of wrong conduct. For every mouthful swallowed, he commits an offense entailing confession. 

\subsection*{Permutations }

If\marginnote{5.2.1} it is a meal before another, and he perceives it as such, and he eats it, except on an appropriate occasion, he commits an offense entailing confession. If it is a meal before another, but he is unsure of it, and he eats it, except on an appropriate occasion, he commits an offense entailing confession. If it is a meal before another, but he does not perceive it as such, and he eats it, except on an appropriate occasion, he commits an offense entailing confession. 

If\marginnote{5.2.4} it is not a meal before another, but he perceives it as such, he commits an offense of wrong conduct. If it is not a meal before another, but he is unsure of it, he commits an offense of wrong conduct. If it is not a meal before another, and he does not perceive it as such, there is no offense. 

\subsection*{Non-offenses }

There\marginnote{5.3.1} is no offense: if it is an appropriate occasion; if he assigns his other meal to someone else and then eats; if he eats the food from two or three invitational meals together; if he eats the meals in the same order that the invitations were received; if he is invited by a whole village and he eats anywhere in that village; if he is invited by a whole association and he eats anywhere that belongs to that association; if, when being invited, he says, “I’ll get almsfood;” if it is a regular meal invitation; if it is a meal for which lots are drawn; if it is a half-monthly meal; if it is on the observance day; if it is on the day after the observance day; if it is anything apart from the five cooked foods; if he is insane; if he is the first offender. 

\scendsutta{The training rule on eating a meal before another, the third, is finished. }

%
\section*{{\suttatitleacronym Bu Pc 34}{\suttatitletranslation 34. The training rule on Kāṇamātā }{\suttatitleroot Dvittipattapūrapaṭiggahaṇa}}
\addcontentsline{toc}{section}{\tocacronym{Bu Pc 34} \toctranslation{34. The training rule on Kāṇamātā } \tocroot{Dvittipattapūrapaṭiggahaṇa}}
\markboth{34. The training rule on Kāṇamātā }{Dvittipattapūrapaṭiggahaṇa}
\extramarks{Bu Pc 34}{Bu Pc 34}

\subsection*{Origin story }

At\marginnote{1.1.1} one time the Buddha was staying at \textsanskrit{Sāvatthī} in the Jeta Grove, \textsanskrit{Anāthapiṇḍika}’s Monastery. At that time there was a female lay follower \textsanskrit{Kāṇamātā} who had faith and confidence. She had a daughter, \textsanskrit{Kāṇā}, who had been given in marriage to a man in a certain village. 

On\marginnote{1.1.4} one occasion \textsanskrit{Kāṇā} went to her mother’s house on some business. \textsanskrit{Kāṇā}’s husband sent her a message: “Please come, \textsanskrit{Kāṇā}, I want you back.” \textsanskrit{Kāṇamātā} thought, “It’s shameful to go empty-handed,” and she baked some cookies. Just when the cookies were finished, an alms-collecting monk entered \textsanskrit{Kāṇamātā}’s house, and she gave him some. After leaving, he told another monk, and he too was given cookies. And the same happened a third time. At that, all the cookies were gone. 

A\marginnote{1.1.11} second time \textsanskrit{Kāṇā}’s husband sent her the same message, and everything unfolded as before. 

A\marginnote{1.1.18} third time he sent the same message, adding, “If \textsanskrit{Kāṇā} doesn’t come, I’ll find another wife.” But once again all the cookies were given to monks. \textsanskrit{Kāṇā}’s husband found another wife, and when \textsanskrit{Kāṇā} heard what had happened, she cried. 

Soon\marginnote{1.1.28} afterwards, after robing up in the morning, the Buddha took his bowl and robe and went to \textsanskrit{Kāṇamātā}’s house where he sat down on the prepared seat. \textsanskrit{Kāṇamātā} approached the Buddha, bowed, and sat down. The Buddha asked her why \textsanskrit{Kāṇā} was crying, and she told him what had happened. After instructing, inspiring, and gladdening her with a teaching, the Buddha got up from his seat and left. 

Soon\marginnote{1.2.1} afterwards a certain caravan was ready to go south from \textsanskrit{Rājagaha}. An alms-collecting monk went up to that caravan to get almsfood, and a lay follower gave him flour products.\footnote{“Flour products” renders \textit{sattu}. See discussion in Appendix of Technical Terms. } After leaving, that monk told another monk, and he too was given flour products. And the same happened a third time. At that, all his provisions were gone. 

That\marginnote{1.2.7} lay follower said to the other people in the caravan, “Sirs, please wait one day. I’ve given my provisions to the monks. I need to prepare more.” 

“We\marginnote{1.2.10} can’t wait. The caravan is already on its way.” And they left. 

After\marginnote{1.2.11} preparing provisions, that lay follower followed after the caravan, but he was robbed by thieves. People complained and criticized them, “How can the Sakyan monastics receive without moderation? This man gave to them, and then because he was following after the caravan he was robbed by thieves.” 

The\marginnote{1.2.15} monks heard the complaints of those people and they told the Buddha. Soon afterwards he gave a teaching and addressed the monks: “Well then, monks, I will lay down a training rule for the following ten reasons: for the well-being of the Sangha, for the comfort of the Sangha, for the restraint of bad people, for the ease of good monks, for the restraint of the corruptions relating to the present life, for the restraint of the corruptions relating to future lives, to give rise to confidence in those without it, to increase the confidence of those who have it, for the longevity of the true Teaching, and for supporting the training. And, monks, this training rule should be recited like this: 

\subsection*{Final ruling }

\scrule{‘If a monk goes to a family and is invited to take cookies or crackers, he may accept two or three bowlfuls if he wishes. If he accepts more than that, he commits an offense entailing confession. If he accepts two or three bowlfuls, he should take it away and share it with the monks. This is the proper procedure.’” }

\subsection*{Definitions }

\begin{description}%
\item[A monk goes to a family: ] a family: there are four kinds of families: the aristocratic family, the brahmin family, the merchant family, the worker family. %
\item[Goes to: ] he has gone there. %
\item[Cookies: ] whatever has been prepared for the purpose of sending away.\footnote{“Cookies” renders \textit{\textsanskrit{pūva}}. See discussion in Appendix of Technical Terms. } %
\item[Crackers: ] whatever has been prepared as provisions for a journey.\footnote{“Crackers” renders \textit{mantha}. See discussion in Appendix of Technical Terms. } %
\item[Is invited to take: ] “Take as much as you like.” %
\item[If he wishes: ] if he desires. %
\item[He may accept two or three bowlfuls: ] two or three bowlfuls can be accepted. %
\item[If he accepts more than that: ] when he accepts more than that, he commits an offense entailing confession. %
\item[If he accepts two or three bowlfuls: ] leaving that place and seeing a monk, he should tell him, “I’ve accepted two or three bowlfuls from such-and-such a place; don’t accept anything from there.” If he sees a monk, but does not tell him, he commits an offense of wrong conduct. If the one who is told accepts from there regardless, he commits an offense of wrong conduct.\footnote{That is, the one who is told commits the offense. } %
\item[He should take it away and share it with the monks: ] he should take it away when returning from almsround and then share it. %
\item[This is the proper procedure: ] this is the right method. %
\end{description}

\subsection*{Permutations }

If\marginnote{2.2.1} it is more than two or three bowlfuls, and he perceives it as more, and he accepts it, he commits an offense entailing confession. If it is more than two or three bowlfuls, but he is unsure of it, and he accepts it, he commits an offense entailing confession. If it is more than two or three bowlfuls, but he perceives it as less, and he accepts it, he commits an offense entailing confession. 

If\marginnote{2.2.4} it is less than two or three bowlfuls, but he perceives it as more, he commits an offense of wrong conduct. If it is less than two or three bowlfuls, but he is unsure of it, he commits an offense of wrong conduct. If it is less than two or three bowlfuls, and he perceives it as less, there is no offense. 

\subsection*{Non-offenses }

There\marginnote{2.3.1} is no offense: if he accepts two or three bowlfuls; if he accepts less than two or three bowlfuls; if they give anything that has not been prepared for sending away or as provisions for a journey; if they give the leftovers from what was prepared for sending away or as provisions for a journey; if they give after the plans to travel have been canceled; if it is from relatives; if it is from those who have given an invitation; if it is for the benefit of someone else; if it is by means of his own property; if he is insane; if he is the first offender. 

\scendsutta{The training rule on \textsanskrit{Kāṇamātā}, the fourth, is finished. }

%
\section*{{\suttatitleacronym Bu Pc 35}{\suttatitletranslation 35. The training rule on invitations }{\suttatitleroot Pavāraṇā}}
\addcontentsline{toc}{section}{\tocacronym{Bu Pc 35} \toctranslation{35. The training rule on invitations } \tocroot{Pavāraṇā}}
\markboth{35. The training rule on invitations }{Pavāraṇā}
\extramarks{Bu Pc 35}{Bu Pc 35}

\subsection*{Origin story }

\subsubsection*{First sub-story }

At\marginnote{1.1} one time when the Buddha was staying at \textsanskrit{Sāvatthī} in \textsanskrit{Anāthapiṇḍika}’s Monastery, a certain brahmin had invited the monks for a meal. When the monks had finished and refused an invitation to eat more, they went to their respective families, where some ate and some took away almsfood. 

Soon\marginnote{1.4} afterwards that brahmin said to his neighbors, “The monks have been satisfied by me. Come, and I’ll satisfy you, too.” “How could you satisfy us? Those monks who were invited by you came to our houses. Some of them ate there and some took away almsfood.” 

That\marginnote{1.9} brahmin complained and criticized those monks, “How could the venerables eat in our house and afterwards eat elsewhere? Am I not able to give them as much as they need?” 

The\marginnote{1.12} monks heard the complaints of that brahmin, and the monks of few desires complained and criticized those monks, “How could those monks finish their meal and refuse an invitation to eat more, and then eat elsewhere?” … “Is it true, monks, that monks did this?” 

“It’s\marginnote{1.16} true, sir.” 

The\marginnote{1.17} Buddha rebuked them … “How could those foolish men do this? This will affect people’s confidence …” … “And, monks, this training rule should be recited like this: 

\subsubsection*{Preliminary ruling }

\scrule{‘If a monk has finished his meal and refused an invitation to eat more, and then eats fresh or cooked food, he commits an offense entailing confession.’”\footnote{I normally render \textit{\textsanskrit{pavārita}} as “invited”, but the word also means “satisfied”. In the present case, the contextual meaning (see below) is that the monk has expressed his satisfaction after being invited to take more, thus “refused an invitation to eat more”. } }

In\marginnote{1.22} this way the Buddha laid down this training rule for the monks. 

\subsubsection*{Second sub-story }

Soon\marginnote{2.1} afterwards the monks were bringing back fine almsfood for the sick monks. But because the sick monks were unable to eat as much as they had intended, the monks threw the leftovers away. When the Buddha heard the loud sound of crows cawing, he asked Venerable Ānanda, “Ānanda, why is there this loud sound of crows cawing?” Ānanda told him what had happened, and the Buddha said, 

“But\marginnote{2.8} Ānanda, don’t the monks eat the leftovers from those who are sick?” 

“No,\marginnote{2.9} sir.” 

Soon\marginnote{2.10} afterwards the Buddha gave a teaching and addressed the monks: 

\scrule{“Monks, I allow you to eat the leftovers both from those who are sick and from those who are not sick. }

And,\marginnote{2.12} monks, this is how you make food left over: ‘I’ve had enough.’ And so, monks, this training rule should be recited like this: 

\subsection*{Final ruling }

\scrule{‘If a monk has finished his meal and refused an invitation to eat more, and then eats fresh or cooked food that is not left over, he commits an offense entailing confession.’” }

\subsection*{Definitions }

\begin{description}%
\item[A: ] whoever … %
\item[Monk: ] … The monk who has been given the full ordination by a unanimous Sangha through a legal procedure consisting of one motion and three announcements that is irreversible and fit to stand—this sort of monk is meant in this case. %
\item[Has finished his meal: ] has eaten any of the five cooked foods, even what fits on the tip of a blade of grass. %
\item[Refused an invitation to eat more:\footnote{For a discussion of the word \textit{\textsanskrit{pavāreti}}, see Appendix of Technical Terms. } ] there is eating; there is cooked food; they stand within arm’s reach; there is an offering; there is a refusal.\footnote{The punctuation of the Pali is wrong. According to \href{https://suttacentral.net/pli-tv-pvr17/en/brahmali\#66.1}{Pvr 17:66.1} there are five aspects to such a refusal, and thus an additional comma is required between \textit{\textsanskrit{ṭhito}} and \textit{abhiharati}. “They” refers to the donor, whether male or female. } %
\item[Not left over: ] the making it left over is done with food that is unallowable;\footnote{Literally, “It is done with what is unallowable.” } it is done with food that has not been received;\footnote{The “it”, both here and below, refers to the ceremony of making it left over. } it is done with food that is not held in hand;\footnote{Literally, “It is not done lifted.” } it is done by one who is not within arm’s reach; it is done by one who has not finished his meal; it is done by one who has finished his meal, who has refused an invitation to eat more, but who has risen from his seat; “I’ve had enough,” has not been said; it is not left over from one who is sick—this is called “not left over”. %
\item[Left over: ] the making it left over is done with food that is allowable; it is done with food that has been received; it is done with food that is held in hand; it is done by one who is within arm’s reach; it is done by one who has finished his meal; it is done by one who has finished his meal, who has refused an invitation to eat more, and who has not risen from his seat; “I’ve had enough,” has been said; it is left over from one who is sick—this is called “left over”. %
\item[Fresh food: ] apart from the five cooked foods, the post-midday tonics, the seven-day tonics, and the lifetime tonics—the rest is called “fresh food”.\footnote{“Fresh food” renders \textit{\textsanskrit{khādanīya}}. See Appendix of Technical Terms for discussion. } %
\item[Cooked food:\footnote{“Cooked food” renders \textit{\textsanskrit{bhojanīya}}. See Appendix of Technical Terms for discussion. } ] there are five kinds of cooked food: cooked grain, porridge, flour products, fish, and meat. %
\end{description}

If\marginnote{3.1.33} he receives it intending to eat it, he commits an offense of wrong conduct. For every mouthful swallowed, he commits an offense entailing confession. 

\subsection*{Permutations }

If\marginnote{3.2.1} it is not left over, and he does not perceive it as such, and he eats fresh or cooked food, he commits an offense entailing confession. If it is not left over, but he is unsure of it, and he eats fresh or cooked food, he commits an offense entailing confession. If it is not left over, but he perceives it as such, and he eats fresh or cooked food, he commits an offense entailing confession. 

If\marginnote{3.2.4} he receives post-midday tonics, seven-day tonics, or lifetime tonics for the purpose of food, he commits an offense of wrong conduct. For every mouthful swallowed, he commits an offense of wrong conduct. 

If\marginnote{3.2.6} it is left over, but he does not perceive it as such, he commits an offense of wrong conduct. If it is left over, but he is unsure of it, he commits an offense of wrong conduct. If it is left over, and he perceives it as such, there is no offense. 

\subsection*{Non-offenses }

There\marginnote{3.3.1} is no offense: if he eats it after having it made left over; if he receives it with the intention of having it made left over and then eating it; if he is taking food for the benefit of someone else; if he eats the leftovers from a sick person; if, when there is a reason, he uses post-midday tonics, seven-day tonics, or lifetime tonics; if he is insane; if he is the first offender. 

\scendsutta{The training rule on invitations, the fifth, is finished. }

%
\section*{{\suttatitleacronym Bu Pc 36}{\suttatitletranslation 36. The second training rule on invitations }{\suttatitleroot Dutiyapavāraṇā}}
\addcontentsline{toc}{section}{\tocacronym{Bu Pc 36} \toctranslation{36. The second training rule on invitations } \tocroot{Dutiyapavāraṇā}}
\markboth{36. The second training rule on invitations }{Dutiyapavāraṇā}
\extramarks{Bu Pc 36}{Bu Pc 36}

\subsection*{Origin story }

At\marginnote{1.1} one time the Buddha was staying at \textsanskrit{Sāvatthī} in the Jeta Grove, \textsanskrit{Anāthapiṇḍika}’s Monastery. At that time two monks were traveling through the Kosalan country on their way to \textsanskrit{Sāvatthī}. One monk misbehaved and the other said to him, “Don’t do that! It’s not allowable.” Because of that the first monk became resentful. They then carried on to \textsanskrit{Sāvatthī}. 

Soon\marginnote{1.8} afterwards an association in \textsanskrit{Sāvatthī} was offering a meal to the Sangha. When the second monk had finished his meal and refused an invitation to eat more, the resentful monk brought back some almsfood from his own family. He then said to the other, “Please eat!” 

“There’s\marginnote{1.12} no need. I’m full.” 

“The\marginnote{1.13} almsfood is nice, please eat.” 

And\marginnote{1.14} because he was pressured, he ate the almsfood. The resentful monk then said to him, “Who are you to correct me when you eat food that’s not left over even though you have finished your meal and refused an invitation to eat more?” 

“Shouldn’t\marginnote{1.17} you have told me?”\footnote{That is, why did he not say that the food was not left over. } 

“Shouldn’t\marginnote{1.18} you have asked?” 

The\marginnote{1.19} second monk told the monks what had happened, and the monks of few desires complained and criticized the other, “How could a monk invite another monk to eat food that’s not left over, when the other has finished his meal and refused an invitation to eat more ?” … “Is it true, monk, that you did this?” 

“It’s\marginnote{1.23} true, sir.” 

The\marginnote{1.24} Buddha rebuked him … “Foolish man, how could you do this? This will affect people’s confidence …” … “And, monks, this training rule should be recited like this: 

\subsection*{Final ruling }

\scrule{‘If a monk invites a monk, whom he knows has finished his meal and refused an invitation to eat more, to eat fresh or cooked food that is not left over, saying, “Here, monk, eat,” aiming to criticize him, then when the other has eaten, he commits an offense entailing confession.’” }

\subsection*{Definitions }

\begin{description}%
\item[A: ] whoever … %
\item[Monk: ] … The monk who has been given the full ordination by a unanimous Sangha through a legal procedure consisting of one motion and three announcements that is irreversible and fit to stand—this sort of monk is meant in this case. %
\item[A monk: ] another monk. %
\item[Whom has finished his meal: ] whom has eaten any of the five cooked foods, even what fits on the tip of a blade of grass. %
\item[Refused an invitation to eat more: ] there is eating; there is cooked food; they stand within arm’s reach; there is an offering; there is a refusal.\footnote{As with the previous rule, the punctuation of the Pali is wrong. An additional comma is required between \textit{\textsanskrit{ṭhito}} and \textit{abhiharati}. “They” refers to the donor, whether male or female. } %
\item[Not left over: ] the making it left over is done with food that is unallowable;\footnote{Literally, “It is done with what is not allowable.” } it is done with food that has not been received;\footnote{The “it”, both here and below, refers to the ceremony of making it left over. } it is done with food that is not held in hand;\footnote{Literally, “It is not done lifted.” } it is done by one who is not within arm’s reach; it is done by one who has not finished his meal; it is done by one who has finished his meal, who has refused an invitation to eat more, but who has risen from his seat; “I’ve had enough,” has not been said; it is not left over from one who is sick—this is called “not left over”. %
\item[Fresh food: ] apart from the five cooked foods, the post-midday tonics, the seven-day tonics, and the lifetime tonics—the rest is called “fresh food”. %
\item[Cooked food: ] there are five kinds of cooked food: cooked grain, porridge, flour products, fish, and meat. %
\item[Invites to eat: ] saying, “Take as much as you like.” %
\item[He knows: ] he knows by himself or others have told him or the monk has told him.\footnote{The meaning of the last of these three ways of knowing, \textit{so \textsanskrit{vā} \textsanskrit{āroceti}}, is not clear. CPD suggests: “\textit{sa (\textsanskrit{sā}) \textsanskrit{āroceti}} (?). Perhaps this last form is conformable to sa. \textit{\textsanskrit{ārocayate}} med. caus. in the meaning: he or she makes inquiries (of others).” However, this does not fit with the parallel usage at \href{https://suttacentral.net/pli-tv-bu-vb-pc29/en/brahmali\#3.1.6}{Bu Pc 29:3.1.6} where the text says that she tells (\textit{\textsanskrit{sā} \textsanskrit{vā} \textsanskrit{āroceti}}) him, presumably referring to the nun telling the monk. In this case \textit{\textsanskrit{āroceti}} cannot refer to the monk making inquiries. The commentaries are silent, and I therefore assume that a straightforward meaning is the most likely one. I would suggest, then, that it simply refers to the recipient telling the resentful monk directly. } %
\item[Aiming to criticize him: ] if he offers it to him, thinking, “With this I’ll accuse him,” “I’ll remind him,” “I’ll counter accuse him,” “I’ll counter remind him,” “I’ll humiliate him,” he commits an offense of wrong conduct. %
\end{description}

If,\marginnote{2.1.31} because of what he says, the other monk receives it with the intention of eating it, then the donor commits an offense of wrong conduct.  For every mouthful swallowed, the donor commits an offense of wrong conduct.  When the other monk has finished eating, the donor commits an offense entailing confession. 

\subsection*{Permutations }

If\marginnote{2.2.1} the other monk has refused an invitation to eat more, and the donor perceives that he has, and he invites him to eat fresh or cooked food that is not left over, he commits an offense entailing confession. If the other monk has refused an invitation to eat more, but the donor is unsure of it, and he invites him to eat fresh or cooked food that is not left over, he commits an offense of wrong conduct. If the other monk has refused an invitation to eat more, but the donor does not perceive that he has, and he invites him to eat fresh or cooked food that is not left over, there is no offense. 

If\marginnote{2.2.4} he invites him to eat post-midday tonics, seven-day tonics, or lifetime tonics for the purpose of food, he commits an offense of wrong conduct. If, because of what he says, the other monk receives it with the intention of eating it, then the donor commits an offense of wrong conduct. For every mouthful swallowed, the donor commits an offense of wrong conduct. 

If\marginnote{2.2.7} the other monk has not refused an invitation to eat more, but the donor perceives that he has, he commits an offense of wrong conduct. If the other monk has not refused an invitation to eat more, but the donor is unsure of it, he commits an offense of wrong conduct. If the other monk has not refused an invitation to eat more, and the donor does not perceive that he has, there is no offense. 

\subsection*{Non-offenses }

There\marginnote{2.3.1} is no offense: if he gives it after having it made left over; if he gives it, saying, “Have it made left over and then eat it;” if he gives it, saying, “Take this food for the benefit of someone else;” if he gives the leftovers from a sick person; if he gives, saying, “When there’s a reason, use these post-midday tonics,” “… use these seven-day tonics,” “… use  these lifetime tonics;” if he is insane; if he is the first offender. 

\scendsutta{The second training rule on invitations, the sixth, is finished. }

%
\section*{{\suttatitleacronym Bu Pc 37}{\suttatitletranslation 37. The training rule on eating at the wrong time }{\suttatitleroot Vikālabhojana}}
\addcontentsline{toc}{section}{\tocacronym{Bu Pc 37} \toctranslation{37. The training rule on eating at the wrong time } \tocroot{Vikālabhojana}}
\markboth{37. The training rule on eating at the wrong time }{Vikālabhojana}
\extramarks{Bu Pc 37}{Bu Pc 37}

\subsection*{Origin story }

At\marginnote{1.1} one time when the Buddha was staying at \textsanskrit{Rājagaha} in the Bamboo Grove, there was a hilltop fair in \textsanskrit{Rājagaha}, which the monks from the group of seventeen went to see. When people saw the monks, they bathed them, anointed them, fed them cooked food, and gave them fresh food. They took that food and brought it back to the monastery. And they said to the monks from the group of six, “Help yourselves!” 

“But\marginnote{1.7} where did you get this food?” And they told them what had happened. 

“So,\marginnote{1.9} do you eat at the wrong time?” 

“Yes.”\marginnote{1.10} 

The\marginnote{1.11} monks from the group of six complained and criticized them, “How can those monks from the group of seventeen eat at the wrong time?” 

They\marginnote{1.13} told the monks, and the monks of few desires complained and criticized them, “How can the monks from the group of seventeen eat at the wrong time?” … “Is it true, monks, that you do this?” 

“It’s\marginnote{1.17} true, sir.” 

The\marginnote{1.18} Buddha rebuked them … “Foolish men, how can you do this? This will affect people’s confidence …” … “And, monks, this training rule should be recited like this: 

\subsection*{Final ruling }

\scrule{‘If a monk eats fresh or cooked food at the wrong time, he commits an offense entailing confession.’” }

\subsection*{Definitions }

\begin{description}%
\item[A: ] whoever … %
\item[Monk: ] … The monk who has been given the full ordination by a unanimous Sangha through a legal procedure consisting of one motion and three announcements that is irreversible and fit to stand—this sort of monk is meant in this case. %
\item[At the wrong time: ] when the middle of the day has passed, until dawn. %
\item[Fresh food: ] apart from the five cooked foods, the post-midday tonics, the seven-day tonics, and the lifetime tonics—the rest is called “fresh food”. %
\item[Cooked food: ] there are five kinds of cooked food: cooked grain, porridge, flour products, fish, and meat. %
\end{description}

If\marginnote{2.1.11} he receives fresh or cooked food with the intention of eating it, he commits an offense of wrong conduct. For every mouthful swallowed, he commits an offense entailing confession. 

\subsection*{Permutations }

If\marginnote{2.2.1} it is the wrong time, and he perceives it as such, and he eats fresh or cooked food, he commits an offense entailing confession. If it is the wrong time, but he is unsure of it, and he eats fresh or cooked food, he commits an offense entailing confession. If it is the wrong time, but he perceives it as the right time, and he eats fresh or cooked food, he commits an offense entailing confession. 

If\marginnote{2.2.4} he receives post-midday tonics, seven-day tonics, or lifetime tonics for the purpose of food, he commits an offense of wrong conduct. For every mouthful swallowed, he commits an offense of wrong conduct. 

If\marginnote{2.2.6} it is the right time, but he perceives it as the wrong time, he commits an offense of wrong conduct. If it is the right time, but he is unsure of it, he commits an offense of wrong conduct. If it is the right time, and he perceives it as such, there is no offense. 

\subsection*{Non-offenses }

There\marginnote{2.3.1} is no offense: if, when there is a reason, he uses post-midday tonics, seven-day tonics, or lifetime tonics; if he is insane; if he is the first offender. 

\scendsutta{The training rule on eating at the wrong time, the seventh, is finished. }

%
\section*{{\suttatitleacronym Bu Pc 38}{\suttatitletranslation 38. The training rule on storing }{\suttatitleroot Sannidhikāraka}}
\addcontentsline{toc}{section}{\tocacronym{Bu Pc 38} \toctranslation{38. The training rule on storing } \tocroot{Sannidhikāraka}}
\markboth{38. The training rule on storing }{Sannidhikāraka}
\extramarks{Bu Pc 38}{Bu Pc 38}

\subsection*{Origin story }

At\marginnote{1.1} one time the Buddha was staying at \textsanskrit{Sāvatthī} in the Jeta Grove, \textsanskrit{Anāthapiṇḍika}’s Monastery. At that time Venerable \textsanskrit{Belaṭṭhasīsa}, Venerable Ānanda’s preceptor, was staying in the wilderness. After walking for alms, he brought plain boiled rice back to the monastery where he dried and stored it. Whenever he got hungry, he moistened and ate it. As a result, he only went the village for alms after a long time. 

The\marginnote{1.5} monks asked him, “Why do you only go for alms after such a long time?” And he told them. 

“But\marginnote{1.8} do you eat food that you’ve stored?” 

“Yes.”\marginnote{1.9} 

The\marginnote{1.10} monks of few desires complained and criticized him, “How can Venerable \textsanskrit{Belaṭṭhasīsa} eat food that he has stored?” … “Is it true, \textsanskrit{Belaṭṭhasīsa}, that you do this?” 

“It’s\marginnote{1.13} true, sir.” 

The\marginnote{1.14} Buddha rebuked him … “\textsanskrit{Belaṭṭhasīsa}, how can you do this? This will affect people’s confidence …” … “And, monks, this training rule should be recited like this: 

\subsection*{Final ruling }

\scrule{‘If a monk eats fresh or cooked food that he has stored, he commits an offense entailing confession.’” }

\subsection*{Definitions }

\begin{description}%
\item[A: ] whoever … %
\item[Monk: ] … The monk who has been given the full ordination by a unanimous Sangha through a legal procedure consisting of one motion and three announcements that is irreversible and fit to stand—this sort of monk is meant in this case. %
\item[That he has stored: ] received today and eaten on the following day. %
\item[Fresh food: ] apart from the five cooked foods, the post-midday tonics, the seven-day tonics, and the lifetime tonics—the rest is called “fresh food”. %
\item[Cooked food: ] there are five kinds of cooked food: cooked grain, porridge, flour products, fish, and meat. %
\end{description}

If\marginnote{2.1.11} he receives fresh or cooked food with the intention of eating it, he commits an offense of wrong conduct. For every mouthful swallowed, he commits an offense entailing confession. 

\subsection*{Permutations }

If\marginnote{2.2.1} it has been stored, and he perceives that it has, and he eats the fresh or cooked food, he commits an offense entailing confession. If it has been stored, but he is unsure of it, and he eats the fresh or cooked food, he commits an offense entailing confession. If it has been stored, but he does not perceive that it has, and he eats the fresh or cooked food, he commits an offense entailing confession. 

If\marginnote{2.2.4} he receives post-midday tonics, seven-day tonics, or lifetime tonics for the purpose of food, he commits an offense of wrong conduct. For every mouthful swallowed, he commits an offense of wrong conduct. 

If\marginnote{2.2.6} it has not been stored, but he perceives that it has, he commits an offense of wrong conduct. If it has not been stored, but he is unsure of it, he commits an offense of wrong conduct. If it has not been stored, and he does not perceive that it has, there is no offense. 

\subsection*{Non-offenses }

There\marginnote{2.3.1} is no offense: if he both stores and eats it during the right time; if he both stores and eats post-midday tonics during the remainder of the day; if he both stores and eats seven-day tonics during the seven-day period; if he uses lifetime tonics when there is a reason; if he is insane; if he is the first offender. 

\scendsutta{The training rule on storing, the eighth, is finished. }

%
\section*{{\suttatitleacronym Bu Pc 39}{\suttatitletranslation 39. The training rule on fine foods }{\suttatitleroot Paṇītabhojana}}
\addcontentsline{toc}{section}{\tocacronym{Bu Pc 39} \toctranslation{39. The training rule on fine foods } \tocroot{Paṇītabhojana}}
\markboth{39. The training rule on fine foods }{Paṇītabhojana}
\extramarks{Bu Pc 39}{Bu Pc 39}

\subsection*{Origin story }

\subsubsection*{First sub-story }

At\marginnote{1.1} one time when the Buddha was staying at \textsanskrit{Sāvatthī} in \textsanskrit{Anāthapiṇḍika}’s Monastery, the monks from the group of six were eating fine foods that they had requested for themselves. People complained and criticized them, “How can the Sakyan monastics eat fine foods that they have requested for themselves? Who doesn’t like nice food? Who doesn’t prefer tasty food?” 

The\marginnote{1.6} monks heard the complaints of those people, and the monks of few desires complained and criticized those monks, “How can the monks from the group of six eat fine foods that they have requested for themselves?” … “Is it true, monks, that you do this?” 

“It’s\marginnote{1.10} true, sir.” 

The\marginnote{1.11} Buddha rebuked them … “Foolish men, how can you do this? This will affect people’s confidence …” … “And, monks, this training rule should be recited like this: 

\subsubsection*{Preliminary ruling }

\scrule{‘If a monk asks for any of these kinds of fine foods for himself—that is, ghee, butter, oil, honey, syrup, fish, meat, milk, and curd—and then eats it, he commits an offense entailing confession.’” }

In\marginnote{1.16} this way the Buddha laid down this training rule for the monks. 

\subsubsection*{Second sub-story }

At\marginnote{2.1} one time a number of monks were sick. The monks who were looking after them asked, “I hope you’re bearing up? I hope you’re getting better?” 

“Previously\marginnote{2.4} we ate fine foods that we had requested ourselves, and then we were comfortable. But now that the Buddha has prohibited this, we don’t request because we’re afraid of wrongdoing. And because of that we’re not comfortable.” 

They\marginnote{2.6} told the Buddha. Soon afterwards he gave a teaching and addressed the monks: 

\scrule{“Monks, I allow a sick monk to eat fine foods that he has requested for himself. }

And\marginnote{2.9} so, monks, this training rule should be recited like this: 

\subsection*{Final ruling }

\scrule{‘If a monk who is not sick asks for any of these kinds of fine foods for himself—that is, ghee, butter, oil, honey, syrup, fish, meat, milk, and curd—and then eats it, he commits an offense entailing confession.’” }

\subsection*{Definitions }

\begin{description}%
\item[These kinds of fine foods: ] ghee from cows, ghee from goats, ghee from buffaloes, or ghee from whatever animal whose meat is allowable. %
\item[Ghee: ] butter from those same animals. %
\item[Butter: ] sesame oil, mustard oil, honey-tree oil, castor oil, oil from fat. %
\item[Oil: ] honey from bees. %
\item[Honey: ] from sugarcane.\footnote{“Syrup” renders \textit{\textsanskrit{phāṇita}}. I. B. Horner instead translates it as “molasses”, which doesn’t quite hit the mark. SED defines \textit{\textsanskrit{phāṇita}} as “the inspissated juice of the sugar cane or other plants”, in other words, “cane syrup”. According to the commentary at Sp 1.623 it can be either cooked or uncooked, the difference presumably whether it is raw or concentrated. “Syrup” seems closer to the mark than “molasses”. } %
\item[Syrup: ] what lives in water is what is meant.\footnote{Here I follow the alternative reading \textit{udakacaro} found in the PTS version. \textit{Udako} must be an editing mistake. } %
\item[Fish: ] the meat of those animals whose meat is allowable. %
\item[Meat: ] milk from cows, milk from goats, milk from buffaloes, or milk from whatever animal whose meat is allowable. %
\item[Milk: ] curd from those same animals. %
\item[Curd: ] whoever … %
\item[A: ] … The monk who has been given the full ordination by a unanimous Sangha through a legal procedure consisting of one motion and three announcements that is irreversible and fit to stand—this sort of monk is meant in this case. %
\item[Monk: ] such kinds of fine foods. %
\item[Any of these kinds of fine foods: ] who is comfortable without fine foods. %
\item[Who is not sick: ] who is not comfortable without fine foods. %
\end{description}

If\marginnote{3.1.30} he is not sick and he requests for himself, then for the effort there is an act of wrong conduct.  When he receives it with the intention of eating it, he commits an offense of wrong conduct.  For every mouthful swallowed, he commits an offense entailing confession. 

\subsection*{Permutations }

If\marginnote{3.2.1} he is not sick, and he does not perceive himself as sick, and he eats fine foods that he has requested for himself, he commits an offense entailing confession. If he is not sick, but he is unsure of it, and he eats fine foods that he has requested for himself, he commits an offense entailing confession. If he is not sick, but he perceives himself as sick, and he eats fine foods that he has requested for himself, he commits an offense entailing confession. 

If\marginnote{3.2.4} he is sick, but he does not perceive himself as sick, he commits an offense of wrong conduct. If he is sick, but he is unsure of it, he commits an offense of wrong conduct. If he is sick, and he perceives himself as sick, there is no offense. 

\subsection*{Non-offenses }

There\marginnote{3.3.1} is no offense: if he is sick; if he asked for it when he was sick, but eats it when he is no longer sick; if he eats the leftovers from one who is sick; if it is from relatives; if it is from those who have given an invitation; if it is for the benefit of someone else; if it is by means of his own property; if he is insane; if he is the first offender. 

\scendsutta{The training rule on fine foods, the ninth, is finished. }

%
\section*{{\suttatitleacronym Bu Pc 40}{\suttatitletranslation 40. The training rule on tooth cleaners }{\suttatitleroot Dantapona}}
\addcontentsline{toc}{section}{\tocacronym{Bu Pc 40} \toctranslation{40. The training rule on tooth cleaners } \tocroot{Dantapona}}
\markboth{40. The training rule on tooth cleaners }{Dantapona}
\extramarks{Bu Pc 40}{Bu Pc 40}

\subsection*{Origin story }

\subsubsection*{First sub-story }

At\marginnote{1.1} one time the Buddha was staying in the hall with the peaked roof in the Great Wood near \textsanskrit{Vesālī}. At that time a monk who only used discarded things was staying in a charnel ground. He disliked receiving things from people. Instead he would take whatever was offered to the dead at the charnel ground, at the foot of trees, or at the threshold, and he would use that.\footnote{“The threshold”, \textit{\textsanskrit{ummāra}}, would seem to refer the entry point to the charnel ground. Sp-yoj 2.263: \textit{\textsanskrit{Ummārepīti} \textsanskrit{susānassa} \textsanskrit{indakhīlepi}}, “\textit{\textsanskrit{Ummārepī}}: also at the entry post to the charnel ground.” } People complained and criticized him, “How can this monk take the offerings to our ancestors and use them?\footnote{Although this is not specified by the text, such offerings would mostly have been food. } This monk is big and strong. He probably eats human flesh too!” 

The\marginnote{1.7} monks heard the complaints of those people, and the monks of few desires complained and criticized him, “How can this monk eat food that hasn’t been given?” … “Is it true, monk, that you do this?” 

“It’s\marginnote{1.11} true, sir.” 

The\marginnote{1.12} Buddha rebuked him … “Foolish man, how can you do this? This will affect people’s confidence …” … “And, monks, this training rule should be recited like this: 

\subsubsection*{Preliminary ruling }

\scrule{‘If a monk eats food that has not been given, he commits an offense entailing confession.’” }

In\marginnote{1.17} this way the Buddha laid down this training rule for the monks. 

\subsubsection*{Second sub-story }

Soon\marginnote{2.1} afterwards the monks did not use water or tooth cleaners because they were afraid of wrongdoing. They told the Buddha. Soon afterwards he gave a teaching and addressed the monks: 

\scrule{“Monks, I allow you to use water and tooth cleaners after taking them yourselves. }

And\marginnote{2.4} so, monks, this training rule should be recited like this: 

\subsection*{Final ruling }

\scrule{‘If a monk eats food that has not been given, except for water and tooth cleaners, he commits an offense entailing confession.’” }

\subsection*{Definitions }

\begin{description}%
\item[A: ] whoever … %
\item[Monk: ] … The monk who has been given the full ordination by a unanimous Sangha through a legal procedure consisting of one motion and three announcements that is irreversible and fit to stand—this sort of monk is meant in this case. %
\item[That has not been given: ] what has not been received is what is meant. %
\item[Given: ] standing within arm’s reach of one giving by body or by what is connected to his body or by releasing, he receives it by body or by what is connected to his body—this is called “given”. %
\item[Food: ] whatever is edible, apart from water and tooth cleaners—this is called “food”. %
\item[Except for water and tooth cleaners: ] apart from water and tooth cleaners. %
\end{description}

If\marginnote{3.1.13} he takes it with the intention of eating it, he commits an offense of wrong conduct. For every mouthful swallowed, he commits an offense entailing confession. 

\subsection*{Permutations }

If\marginnote{3.2.1} it has not been received, and he does not perceive it as such, and he eats it, except for water and tooth cleaners, he commits an offense entailing confession. If it has not been received, but he is unsure of it, and he eats it, except for water and tooth cleaners, he commits an offense entailing confession. If it has not been not received, but he perceives it as such, and he eats it, except for water and tooth cleaners, he commits an offense entailing confession. 

If\marginnote{3.2.4} it has been received, but he does not perceive it as such, he commits an offense of wrong conduct. If it has been received, but he is unsure of it, he commits an offense of wrong conduct. If it has been received, and he perceives it as such, there is no offense. 

\subsection*{Non-offenses }

There\marginnote{3.3.1} is no offense: if it is water or tooth cleaners; if, when there is a reason, but there is no attendant, he himself takes the four foul edibles and eats them;\footnote{These are dung, urine, ash, and clay, see \href{https://suttacentral.net/pli-tv-kd6/en/brahmali\#14.6.4}{Kd 6:14.6.4}. } if he is insane; if he is the first offender. 

\scendsutta{The training rule on tooth cleaners, the tenth, is finished. }

\scendvagga{The fourth subchapter on eating is finished. }

\scuddanaintro{This is the summary: }

\begin{scuddana}%
“Alms,\marginnote{3.3.9} group, another, cookie, \\
And two are spoken on invitations; \\
At the wrong time, store, milk, \\
With tooth cleaner—those are the ten.” 

%
\end{scuddana}

%
\section*{{\suttatitleacronym Bu Pc 41}{\suttatitletranslation 41. The training rule on naked ascetics }{\suttatitleroot Acelaka}}
\addcontentsline{toc}{section}{\tocacronym{Bu Pc 41} \toctranslation{41. The training rule on naked ascetics } \tocroot{Acelaka}}
\markboth{41. The training rule on naked ascetics }{Acelaka}
\extramarks{Bu Pc 41}{Bu Pc 41}

\subsection*{Origin story }

At\marginnote{1.1.1} one time when the Buddha was staying in the hall with the peaked roof in the Great Wood near \textsanskrit{Vesālī}, the Sangha had an abundance of fresh food. Venerable Ānanda told the Buddha, who said, “Well then, Ānanda, give the cookies to those who take leftovers.” 

“Yes,\marginnote{1.1.4} sir.” Ānanda had them sit in a row and gave them one cookie each, until he accidentally gave two to a female wanderer. The female wanderers sitting next to her said to her, “This monastic is your lover.” 

“He’s\marginnote{1.1.6} not. He gave me two, thinking they were one.” 

And\marginnote{1.1.7} a second time … And a third time Ānanda gave them one cookie each, until he accidentally gave two to that same female wanderer. Once again the female wanderers sitting next to her said to her, “This monastic is your lover.” 

“He’s\marginnote{1.1.10} not. He gave me two, thinking they were one.” 

And\marginnote{1.1.11} they started to argue about whether or not they were lovers. 

A\marginnote{1.2.1} certain \textsanskrit{Ājīvaka} ascetic, too, went to that distribution of food. A monk mixed rice with a large amount of ghee and gave him a large lump. He took it and left. Another \textsanskrit{Ājīvaka} asked him, “Where did you get that lump?” 

“From\marginnote{1.2.5} the food distribution of the ascetic Gotama, that shaven-headed householder.” 

Some\marginnote{1.2.6} lay followers overheard that conversation between those \textsanskrit{Ājīvaka} ascetics. They then went to the Buddha, bowed, sat down, and said, “Sir, these monastics of other religions want to disparage the Buddha, the Teaching, and the Sangha. It would be good if the monks didn’t personally give anything to the monastics of other religions.” 

After\marginnote{1.2.10} the Buddha had instructed, inspired, and gladdened those lay followers with a teaching, they got up from their seats, bowed down, circumambulated him with their right sides toward him, and left. Soon afterwards the Buddha gave a teaching and addressed the monks: “Well then, monks, I will lay down a training rule for the following ten reasons: for the well-being of the Sangha, for the comfort of the Sangha, for the restraint of bad people, for the ease of good monks, for the restraint of the corruptions relating to the present life, for the restraint of the corruptions relating to future lives, to give rise to confidence in those without it, to increase the confidence of those who have it, for the longevity of the true Teaching, and for supporting the training. And, monks, this training rule should be recited like this: 

\subsection*{Final ruling }

\scrule{‘If a monk personally gives fresh or cooked food to a naked ascetic, to a male wanderer, or to a female wanderer, he commits an offense entailing confession.’” }

\subsection*{Definitions }

\begin{description}%
\item[A: ] whoever … %
\item[Monk: ] … The monk who has been given the full ordination by a unanimous Sangha through a legal procedure consisting of one motion and three announcements that is irreversible and fit to stand—this sort of monk is meant in this case. %
\item[A naked ascetic: ] any wanderer who is naked. %
\item[A male wanderer: ] any male wanderer apart from Buddhist monks and novice monks. %
\item[A female wanderer: ] any female wanderer apart from Buddhist nuns, trainee nuns, and novice nuns. %
\item[Fresh food: ] apart from the five cooked foods, water, and tooth cleaners, the rest is called “fresh food”. %
\item[Cooked food: ] there are five kinds of cooked food: cooked grain, porridge, flour products, fish, and meat. %
\item[Gives: ] if he gives by body or by what is connected to the body or by releasing, he commits an offense entailing confession. %
\end{description}

\subsection*{Permutations }

If\marginnote{2.2.1} it is a monastic of another religion, and he perceives them as such, and he personally gives them fresh or cooked food, he commits an offense entailing confession. If it is a monastic of another religion, but he is unsure of it, and he personally gives them fresh or cooked food, he commits an offense entailing confession. If it is a monastic of another religion, but he does not perceive them as such, and he personally gives them fresh or cooked food, he commits an offense entailing confession. 

If\marginnote{2.2.4} he gives water or a tooth cleaner, he commits an offense of wrong conduct. If it is not a monastic of another religion, but he perceives them as such, he commits an offense of wrong conduct. If it is not a monastic of another religion, but he is unsure of it, he commits an offense of wrong conduct. If it is not a monastic of another religion, and he does not perceive them as such, there is no offense. 

\subsection*{Non-offenses }

There\marginnote{2.3.1} is no offense: if he does not give, but has it given; if he gives by placing it near the person; if he gives ointments for external use; if he is insane; if he is the first offender. 

\scendsutta{The training rule on naked ascetics, the first, is finished. }

%
\section*{{\suttatitleacronym Bu Pc 42}{\suttatitletranslation 42. The training rule on sending away }{\suttatitleroot Uyyojana}}
\addcontentsline{toc}{section}{\tocacronym{Bu Pc 42} \toctranslation{42. The training rule on sending away } \tocroot{Uyyojana}}
\markboth{42. The training rule on sending away }{Uyyojana}
\extramarks{Bu Pc 42}{Bu Pc 42}

\subsection*{Origin story }

On\marginnote{1.1} one occasion when the Buddha was staying at \textsanskrit{Sāvatthī} in \textsanskrit{Anāthapiṇḍika}’s Monastery, Venerable Upananda the Sakyan said to his brother’s student, “Come, let’s go to the village for alms.” Then, without getting him any food, he sent him away, saying, “Go away! I’m not comfortable talking or sitting with you, but only if I talk and sit by myself.” But since the right time for eating was coming to an end, the student was unable to walk for alms. As he returned to the monastery, there was nobody offering food, and so he missed his meal. 

He\marginnote{1.7} then went to the monastery and told the monks what had happened. The monks of few desires complained and criticized Upananda, “How could Venerable Upananda say to a monk, ‘Come, let’s go to the village for alms,’ and then send him away without getting him any food?” … “Is it true, Upananda, that you did this?” 

“It’s\marginnote{1.11} true, sir.” 

The\marginnote{1.12} Buddha rebuked him … “Foolish man, how could you do this? This will affect people’s confidence …” … “And, monks, this training rule should be recited like this: 

\subsection*{Final ruling }

\scrule{‘If a monk says to a monk, “Come, let’s go to the village or town for alms,” and then, whether he has had food given to him or not, sends him away, saying, “Go away, I’m not comfortable talking or sitting with you, but only if I talk and sit by myself,” and he does so only for this reason and no other, he commits an offense entailing confession.’” }

\subsection*{Definitions }

\begin{description}%
\item[A: ] whoever … %
\item[Monk: ] … The monk who has been given the full ordination by a unanimous Sangha through a legal procedure consisting of one motion and three announcements that is irreversible and fit to stand—this sort of monk is meant in this case. %
\item[To a monk: ] to another monk. %
\item[Come … to the village or town: ] a village, also a town, also a city; both a village and a town. %
\item[He has had food given to him: ] he has had congee, a meal, fresh food, or cooked food given to him. %
\item[Not: ] he has not had anything given to him. %
\item[Sends away: ] if, wanting to laugh with a woman, wanting to enjoy himself with her, wanting to sit down in private with her, wanting to misbehave with her, he says, “Go away! I’m not comfortable talking or sitting with you, but only if I talk and sit by myself,” and he sends him away, he commits an offense of wrong conduct. If the second monk is in the process of going beyond sight or beyond hearing, the first monk commits an offense of wrong conduct. When the second monk has gone beyond, the first monk commits an offense entailing confession. %
\item[He does so only for this reason and no other: ] there is no other reason for sending him away. %
\end{description}

\subsection*{Permutations }

If\marginnote{2.2.1} the second monk is fully ordained, and the first monk perceives him as such, and he sends him away, he commits an offense entailing confession. If the second monk is fully ordained, but the first monk is unsure of it, and he sends him away, he commits an offense entailing confession. If the second monk is fully ordained, but the first monk does not perceive him as such, and he sends him away, he commits an offense entailing confession. 

If\marginnote{2.2.4} he puts him down, he commits an offense of wrong conduct.\footnote{“Puts him down” renders \textit{\textsanskrit{kalisāsanaṁ} \textsanskrit{āropeti}}. Sp 2.277: \textit{\textsanskrit{Kalisāsanaṁ} \textsanskrit{āropetīti} “\textsanskrit{kalī}”ti kodho; tassa \textsanskrit{sāsanaṁ} \textsanskrit{āropeti}; kodhassa \textsanskrit{āṇaṁ} \textsanskrit{āropeti}; kodhavasena \textsanskrit{ṭhānanisajjādīsu} \textsanskrit{dosaṁ} \textsanskrit{dassetvā} “passatha bho imassa \textsanskrit{ṭhānaṁ}, \textsanskrit{nisajjaṁ} \textsanskrit{ālokitaṁ} \textsanskrit{vilokitaṁ} \textsanskrit{khāṇuko} viya \textsanskrit{tiṭṭhati}, sunakho viya \textsanskrit{nisīdati}, \textsanskrit{makkaṭo} viya ito cito ca \textsanskrit{viloketī}”ti \textsanskrit{evaṁ} \textsanskrit{amanāpavacanaṁ} vadati “appeva \textsanskrit{nāma} \textsanskrit{imināpi} \textsanskrit{ubbāḷho} \textsanskrit{pakkameyyā}”ti}. “\textit{\textsanskrit{Kalisāsanaṁ} \textsanskrit{āropeti}}: \textit{kali} means anger. He inflicts a teaching because of that; he inflicts punishment because of anger. He finds faults because of anger in regard to the standing, sitting, etc., saying, ‘See how he stands, sits, looks this way and that. He stands like a post, sits like a dog, looks here and there like a monkey.’ He speaks such unpleasant speech hoping he will depart in a huff.” } If he sends away one who is not fully ordained, he commits an offense of wrong conduct. If he puts him down, he commits an offense of wrong conduct. 

If\marginnote{2.2.7} the other is not fully ordained, but he perceives them as such, he commits an offense of wrong conduct. If the other is not fully ordained, but he is unsure of it, he commits an offense of wrong conduct. If the other is not fully ordained, and he does not perceive them as such, he commits an offense of wrong conduct. 

\subsection*{Non-offenses }

There\marginnote{2.3.1} is no offense: if he sends him away, thinking, “Together we won’t get enough;” if he sends him away, thinking, “If he sees these valuable goods, he’ll become greedy;” if he sends him away, thinking, “If he sees this woman, he’ll become lustful;” if he sends him away, saying, “Take congee or a meal or fresh food or cooked food to the one who is sick or to the one who is left behind or to the one who is guarding the dwellings;” if he does not want to misbehave; if he sends him away when there is something to be done; if he is insane; if he is the first offender. 

\scendsutta{The training rule on sending away, the second, is finished. }

%
\section*{{\suttatitleacronym Bu Pc 43}{\suttatitletranslation 43. The training rule on lustful }{\suttatitleroot Sabhojana}}
\addcontentsline{toc}{section}{\tocacronym{Bu Pc 43} \toctranslation{43. The training rule on lustful } \tocroot{Sabhojana}}
\markboth{43. The training rule on lustful }{Sabhojana}
\extramarks{Bu Pc 43}{Bu Pc 43}

\subsection*{Origin story }

On\marginnote{1.1} one occasion when the Buddha was staying at \textsanskrit{Sāvatthī} in \textsanskrit{Anāthapiṇḍika}’s Monastery, Venerable Upananda the Sakyan went to the house of a friend and sat down with his wife in their bedroom. The husband approached Upananda, bowed, and sat down. He then said to his wife, “Please give him alms.” And she did so. 

Soon\marginnote{1.6} afterwards he said, “Please leave, sir, the alms have been given.” 

But\marginnote{1.7} the woman, knowing that her husband was lustful, said, “Please sit, sir, don’t go.” 

A\marginnote{1.8} second time and a third time he repeated his request, and both times his wife repeated hers. 

He\marginnote{1.11} then left the house and complained to the monks, “Venerables, Venerable Upananda is seated with my wife in our bedroom. When I ask him to leave because we’re busy, he doesn’t want to go.” 

The\marginnote{1.15} monks of few desires complained and criticized Upananda, “How could Venerable Upananda sit down intruding on a lustful couple?” … “Is it true, Upananda, that you did this?” 

“It’s\marginnote{1.18} true, sir.” 

The\marginnote{1.19} Buddha rebuked him … “Foolish man, how could you do this? This will affect people’s confidence …” … “And, monks, this training rule should be recited like this: 

\subsection*{Final ruling }

\scrule{‘If a monk sits down intruding on a lustful couple, he commits an offense entailing confession.’” }

\subsection*{Definitions }

\begin{description}%
\item[A: ] whoever … %
\item[Monk: ] … The monk who has been given the full ordination by a unanimous Sangha through a legal procedure consisting of one motion and three announcements that is irreversible and fit to stand—this sort of monk is meant in this case. %
\item[A lustful couple: ] both a woman and a man are present. The woman and the man have not both left, and both are not without lust. %
\item[Intruding on: ] enters after. %
\item[Sits down: ] in a large house, if he sits down more than one arm’s reach inside the door frame, he commits an offense entailing confession. In a small house, if he sits down beyond the ridge beam, he commits an offense entailing confession.\footnote{“Ridge beam” renders \textit{\textsanskrit{piṭṭhivaṁsa}}. Sp 2.280: \textit{\textsanskrit{Piṭṭhivaṁsaṁ} \textsanskrit{atikkamitvāti} \textsanskrit{iminā} \textsanskrit{majjhātikkamaṁ} dasseti}; “\textit{\textsanskrit{Piṭṭhivaṁsaṁ} \textsanskrit{atikkamitvā}}: by this going beyond the middle is shown.” } %
\end{description}

\subsection*{Permutations }

If\marginnote{2.2.1} it is a bedroom, and he perceives it as such, and he sits down intruding on a lustful couple, he commits an offense entailing confession.\footnote{“Bedroom” renders \textit{sayanighara}, literally, “house for sleeping”. Kkh-\textsanskrit{pṭ}: \textit{Sayanigharanti \textsanskrit{sayanīyagharaṁ}, \textsanskrit{vāsagehanti} attho}, “Bedroom: a house to be slept in. The meaning is a house for living in.” } If it is a bedroom, but he is unsure of it, and he sits down intruding on a lustful couple, he commits an offense entailing confession. If it is a bedroom, but he does not perceive it as such, and he sits down intruding on a lustful couple, he commits an offense entailing confession. 

If\marginnote{2.2.4} it is not a bedroom, but he perceives it as such, he commits an offense of wrong conduct. If it is not a bedroom, but he is unsure of it, he commits an offense of wrong conduct. If it is not a bedroom, and he does not perceive it as such, there is no offense. 

\subsection*{Non-offenses }

There\marginnote{2.3.1} is no offense: if, in a large house, he sits down, but not more one arm’s reach inside the door frame; if, in a small house, he sits down, but not beyond the ridge beam; if he has a companion monk; if both have left; if both are without lust; if it is not a bedroom; if he is insane; if he is the first offender. 

\scendsutta{The training rule on lustful, the third, is finished. }

%
\section*{{\suttatitleacronym Bu Pc 44}{\suttatitletranslation 44. The training rule on private and concealed }{\suttatitleroot Rahopaṭicchanna}}
\addcontentsline{toc}{section}{\tocacronym{Bu Pc 44} \toctranslation{44. The training rule on private and concealed } \tocroot{Rahopaṭicchanna}}
\markboth{44. The training rule on private and concealed }{Rahopaṭicchanna}
\extramarks{Bu Pc 44}{Bu Pc 44}

\subsection*{Origin story }

On\marginnote{1.1} one occasion when the Buddha was staying at \textsanskrit{Sāvatthī} in \textsanskrit{Anāthapiṇḍika}’s Monastery, Venerable Upananda the Sakyan went to the house of a friend and sat down in private on a concealed seat with his wife. The husband complained and criticized him, “How could Venerable Upananda sit down in private on a concealed seat with my wife?” 

The\marginnote{1.5} monks heard the complaints of that man, and the monks of few desires complained and criticized Upananda, “How could Venerable Upananda sit down in private on a concealed seat with a woman?” … “Is it true, Upananda, that you did this?” 

“It’s\marginnote{1.9} true, sir.” 

The\marginnote{1.10} Buddha rebuked him … “Foolish man, how could you do this? This will affect people’s confidence …” … “And, monks, this training rule should be recited like this: 

\subsection*{Final ruling }

\scrule{‘If a monk sits down in private on a concealed seat with a woman, he commits an offense entailing confession.’” }

\subsection*{Definitions }

\begin{description}%
\item[A: ] whoever … %
\item[Monk: ] … The monk who has been given the full ordination by a unanimous Sangha through a legal procedure consisting of one motion and three announcements that is irreversible and fit to stand—this sort of monk is meant in this case. %
\item[A woman: ] a female human being, not a female spirit, not a female ghost, not a female animal; even a girl born on that very day, let alone an older one. %
\item[With: ] together. %
\item[In private: ] private to the eye and private to the ear. %
\item[Private to the eye: ] one is unable to see them winking, raising an eyebrow, or nodding. %
\item[Private to the ear: ] one is unable to hear ordinary speech. %
\item[A concealed seat: ] it is concealed by a wall, a screen, a door, a cloth screen, a tree, a pillar, a grain container, or anything else. %
\item[Sits down: ] if the monk sits down or lies down next to the seated woman, he commits an offense entailing confession. If the woman sits down or lies down next to the seated monk, he commits an offense entailing confession. If both are seated or both are lying down, he commits an offense entailing confession. %
\end{description}

\subsection*{Permutations }

If\marginnote{2.2.1} it is a woman, and he perceives her as such, and he sits down in private on a concealed seat with her, he commits an offense entailing confession. If it is a woman, but he is unsure of it, and he sits down in private on a concealed seat with her, he commits an offense entailing confession. If it is a woman, but he does not perceive her as such, and he sits down in private on a concealed seat with her, he commits an offense entailing confession. 

If\marginnote{2.2.4} he sits down in private on a concealed seat with a female spirit, a female ghost, a \textit{\textsanskrit{paṇḍaka}}, or a female animal in the form of a woman, he commits an offense of wrong conduct.\footnote{The Pali reads \textit{\textsanskrit{tiracchānagatāya} \textsanskrit{vā} \textsanskrit{manussaviggahitthiyā} \textsanskrit{vā}}, but it is not clear what \textit{\textsanskrit{manussaviggahitthiyā}} would refer to in this context, and so I take this to be an editing mistake. I here follow the PTS reading, \textit{\textsanskrit{tiracchānagatamanussaviggahitthiyā} \textsanskrit{vā}}, which makes better sense and is consistent with the following rule. } If it is not a woman, but he perceives them as such, he commits an offense of wrong conduct. If it is not a woman, but he is unsure of it, he commits an offense of wrong conduct. If it is not a woman, and he does not perceive them as such, there is no offense. 

\subsection*{Non-offenses }

There\marginnote{2.3.1} is no offense: if he has a male companion who understands; if he stands and does not sit down; if he is not seeking privacy; if he sits down preoccupied with something else;\footnote{Sp 5.467: \textit{\textsanskrit{Aññavihitoti} \textsanskrit{aññaṁ} \textsanskrit{cintayamāno}}, “\textit{\textsanskrit{Aññavihita}}: thinking of something else.” } if he is insane; if he is the first offender. 

\scendsutta{The training rule on private and concealed, the fourth, is finished. }

%
\section*{{\suttatitleacronym Bu Pc 45}{\suttatitletranslation 45. The training rule on sitting down in private }{\suttatitleroot Rahonisajja}}
\addcontentsline{toc}{section}{\tocacronym{Bu Pc 45} \toctranslation{45. The training rule on sitting down in private } \tocroot{Rahonisajja}}
\markboth{45. The training rule on sitting down in private }{Rahonisajja}
\extramarks{Bu Pc 45}{Bu Pc 45}

\subsection*{Origin story }

On\marginnote{1.1} one occasion when the Buddha was staying at \textsanskrit{Sāvatthī} in \textsanskrit{Anāthapiṇḍika}’s Monastery, Venerable Upananda the Sakyan went to the house of a friend and sat down in private alone with his wife. The husband complained and criticized him, “How could Venerable Upananda sit down in private alone with my wife?” 

The\marginnote{1.5} monks heard the complaints of that man, and the monks of few desires complained and criticized Upananda, “How could Venerable Upananda sit down in private alone with a woman?” … “Is it true, Upananda, that you did this?” 

“It’s\marginnote{1.9} true, sir.” 

The\marginnote{1.10} Buddha rebuked him … “Foolish man, how could you do this? This will affect people’s confidence …” … “And, monks, this training rule should be recited like this: 

\subsection*{Final ruling }

\scrule{‘If a monk sits down in private alone with a woman, he commits an offense entailing confession.’” }

\subsection*{Definitions }

\begin{description}%
\item[A: ] whoever … %
\item[Monk: ] … The monk who has been given the full ordination by a unanimous Sangha through a legal procedure consisting of one motion and three announcements that is irreversible and fit to stand—this sort of monk is meant in this case. %
\item[A woman: ] a female human being, not a female spirit, not a female ghost, not a female animal. She understands and is capable of discerning bad speech and good speech, what is decent and what is indecent. %
\item[With: ] together. %
\item[Alone: ] just the monk and the woman. %
\item[In private: ] private to the eye and private to the ear. %
\item[Private to the eye: ] one is unable to see them winking, raising an eyebrow, or nodding. %
\item[Private to the ear: ] one is unable to hear ordinary speech. %
\item[Sits down: ] if the monk sits down or lies down next to the seated woman, he commits an offense entailing confession. If the woman sits down or lies down next to the seated monk, he commits an offense entailing confession. If both are seated or both are lying down, he commits an offense entailing confession. %
\end{description}

\subsection*{Permutations }

If\marginnote{2.2.1} it is a woman, and he perceives her as such, and he sits down in private alone with her, he commits an offense entailing confession. If it is a woman, but he is unsure of it, and he sits down in private alone with her, he commits an offense entailing confession. If it is a woman, but he does not perceive her as such, and he sits down in private alone with her, he commits an offense entailing confession. 

If\marginnote{2.2.4} he sits down in private alone with a female spirit, a female ghost, a \textit{\textsanskrit{paṇḍaka}}, or a female animal in the form of a woman, he commits an offense of wrong conduct. 

If\marginnote{2.2.5} it is not a woman, but he perceives them as such, he commits an offense of wrong conduct. If it is not a woman, but he is unsure of it, he commits an offense of wrong conduct. If it is not a woman, and he does not perceive them as such, there is no offense. 

\subsection*{Non-offenses }

There\marginnote{2.3.1} is no offense: if he has a male companion who understands; if he stands and does not sit down; if he is not seeking privacy; if he sits down preoccupied with something else;\footnote{Sp 5.467: \textit{\textsanskrit{Aññavihitoti} \textsanskrit{aññaṁ} \textsanskrit{cintayamāno}}, “\textit{\textsanskrit{Aññavihita}}: thinking of something else.” } if he is insane; if he is the first offender. 

\scendsutta{The training rule on sitting in private, the fifth, is finished. }

%
\section*{{\suttatitleacronym Bu Pc 46}{\suttatitletranslation 46. The training rule on visiting }{\suttatitleroot Cāritta}}
\addcontentsline{toc}{section}{\tocacronym{Bu Pc 46} \toctranslation{46. The training rule on visiting } \tocroot{Cāritta}}
\markboth{46. The training rule on visiting }{Cāritta}
\extramarks{Bu Pc 46}{Bu Pc 46}

\subsection*{Origin story }

\subsubsection*{First sub-story }

At\marginnote{1.1} one time when the Buddha was staying at \textsanskrit{Rājagaha} in the Bamboo Grove, a family that was supporting Venerable Upananda the Sakyan had invited him to a meal, and they had invited other monks too. But since Upananda was visiting other families before that meal, the other monks said to that family, “Please give the meal.” 

“Please\marginnote{1.7} wait, sirs, until Venerable Upananda arrives.” 

A\marginnote{1.8} second time … A third time those monks said, “Please give the meal before it is too late.” 

“But\marginnote{1.11} we prepared the meal because of Venerable Upananda. Please wait until he arrives.” 

Then,\marginnote{1.13} after visiting those families, Upananda arrived late, and those monks did not eat as much as they had intended. The monks of few desires complained and criticized Upananda, “How can Venerable Upananda visit families first when invited to a meal?” … “Is it true, Upananda, that you do this?” 

“It’s\marginnote{1.18} true, sir.” 

The\marginnote{1.19} Buddha rebuked him … “Foolish man, how can you do this? This will affect people’s confidence …” … “And, monks, this training rule should be recited like this: 

\subsubsection*{First preliminary ruling }

\scrule{‘If a monk who has been invited to a meal visits families beforehand, he commits an offense entailing confession.’” }

In\marginnote{1.24} this way the Buddha laid down this training rule for the monks. 

\subsubsection*{Second sub-story }

Soon\marginnote{2.1} afterwards a family that was supporting Upananda sent fresh food to the Sangha. They instructed that the food should be shown to Upananda and then given to the Sangha. 

But\marginnote{2.3} on that occasion Upananda had gone to the village for alms. When those people arrived at the monastery, they asked for Upananda, and they were told where he was. They said, “Venerables, after showing it to Venerable Upananda, this fresh food is to be given to the Sangha.” The monks told the Buddha, who then gave a teaching and addressed the monks: “Well then, monks, receive it and put it aside until Upananda returns.” 

When\marginnote{2.11} he heard that the Buddha had prohibited visiting families before the meal, Upananda visited them after the meal instead. As a consequence, he returned late to the monastery, and the food had to be returned to the donors. 

The\marginnote{2.12} monks of few desires complained and criticized Upananda, “How can Venerable Upananda visit families after the meal?” … “Is it true, Upananda, that you do this?” 

“It’s\marginnote{2.15} true, sir.” 

The\marginnote{2.16} Buddha rebuked him … “Foolish man, how can you do this? This will affect people’s confidence …” … “And so, monks, this training rule should be recited like this: 

\subsubsection*{Second preliminary ruling }

\scrule{‘If a monk who has been invited to a meal visits families beforehand or afterwards, he commits an offense entailing confession.’” }

In\marginnote{2.21} this way the Buddha laid down this training rule for the monks. 

\subsubsection*{Third sub-story }

Soon\marginnote{3.1} afterwards it was the robe-giving season. But being afraid of wrongdoing, the monks did not visit families. As a result, they only got a small amount of robe-cloth. They told the Buddha. … 

\scrule{“Monks, I allow you to visit families during the robe-giving season. }

And\marginnote{3.5} so, monks, this training rule should be recited like this: 

\subsubsection*{Third preliminary ruling }

\scrule{‘If a monk who has been invited to a meal visits families beforehand or afterwards, except on an appropriate occasion, he commits an offense entailing confession. This is the appropriate occasion: it is the robe-giving season.’” }

In\marginnote{3.8} this way the Buddha laid down this training rule for the monks. 

\subsubsection*{Fourth sub-story }

Soon\marginnote{4.1} afterwards the monks were making robes, and they needed a needle, thread, and scissors. But being afraid of wrongdoing, they did not visit families. They told the Buddha. … 

\scrule{“Monks, I allow you to visit families at a time of making robes. }

And\marginnote{4.5} so, monks, this training rule should be recited like this: 

\subsubsection*{Fourth preliminary ruling }

\scrule{‘If a monk who has been invited to a meal visits families beforehand or afterwards, except on an appropriate occasion, he commits an offense entailing confession. These are the appropriate occasions: it is the robe-giving season; it is a time of making robes.’” }

In\marginnote{4.8} this way the Buddha laid down this training rule for the monks. 

\subsubsection*{Fifth sub-story }

Soon\marginnote{5.1} afterwards there were sick monks who needed medicines. But being afraid of wrongdoing, the monks did not visit families. They told the Buddha. … 

\scrule{“Monks, I allow you to visit families after informing an available monk. }

And\marginnote{5.5} so, monks, this training rule should be recited like this: 

\subsection*{Final ruling }

\scrule{‘If a monk who has been invited to a meal visits families beforehand or afterwards without informing an available monk, except on an appropriate occasion, he commits an offense entailing confession. These are the appropriate occasions: it is the robe-giving season; it is a time of making robes.’” }

\subsection*{Definitions }

\begin{description}%
\item[A: ] whoever … %
\item[Monk: ] … The monk who has been given the full ordination by a unanimous Sangha through a legal procedure consisting of one motion and three announcements that is irreversible and fit to stand—this sort of monk is meant in this case. %
\item[Invited: ] invited to eat any of the five cooked foods. %
\item[To a meal: ] the invitation includes a meal. %
\item[An available monk: ] he is able to inform and then enter. %
\item[No available monk: ] he is not able to inform and then enter. %
\item[Beforehand: ] he has not yet eaten what he has been invited to eat. %
\item[Afterwards: ] even if he has just eaten what fits on the tip of a blade of grass from what he has been invited to eat. %
\item[A family: ] there are four kinds of families: the aristocratic family, the brahmin family, the merchant family, the worker family. %
\item[Visits families: ] if he enters the vicinity of someone else’s house, he commits an offense of wrong conduct. If he crosses the threshold with the first foot, he commits an offense of wrong conduct. If he crosses the threshold with the second foot, he commits an offense entailing confession. %
\item[Except on an appropriate occasion: ] unless it is an appropriate occasion. %
\item[It is the robe-giving season: ] for one who has not participated in the robe-making ceremony, it is the last month of the rainy season. For one who has participated in the robe-making ceremony, it is the five-month period.\footnote{“Robe-making ceremony” refers to the \textit{kathina \textsanskrit{saṅghakamma}}, the making of the \textit{kathina} robe, and the rejoicing in the process, all three together represented by the words \textit{(an)atthate kathine }. “The five month period” is the last month of the rainy season plus the four months of the cold season. } %
\item[It is a time of making robes: ] when he is making robes. %
\end{description}

\subsection*{Permutations }

If\marginnote{6.2.1} he has been invited, and he perceives that he has, and he visits families beforehand or afterwards without informing an available monk, except on an appropriate occasion, he commits an offense entailing confession. If he has been invited, but he is unsure of it, and he visits families beforehand or afterwards without informing an available monk, except on an appropriate occasion, he commits an offense entailing confession. If he has been invited, but he does not perceive that he has, and he visits families beforehand or afterwards without informing an available monk, except on an appropriate occasion, he commits an offense entailing confession. 

If\marginnote{6.2.4} he has not been invited, but he perceives that he has, he commits an offense of wrong conduct. If he has not been invited, but he is unsure of it, he commits an offense of wrong conduct. If he has not been invited, and he does not perceive that he has, there is no offense. 

\subsection*{Non-offenses }

There\marginnote{6.3.1} is no offense: if it is an appropriate occasion; if he enters after informing an available monk; if, when there is no available monk, he enters without informing anyone; if the road passes someone else’s house; if the road passes the vicinity of someone else’s house; if he is going between monasteries; if he is going to the dwelling place of nuns; if he is going to the dwelling place of the monastics of another religion; if he is returning to the monastery; if he is going to the house where he has been invited;\footnote{Sp 2.302: \textit{Bhattiyagharanti \textsanskrit{nimantitagharaṁ} \textsanskrit{vā} \textsanskrit{salākabhattādidāyakānaṁ} \textsanskrit{vā} \textsanskrit{gharaṁ}}, “\textit{Bhattiyaghara}: the house where he has been invited or the house of donors of a meal drawn by lots, etc.” } if there is an emergency; if he is insane; if he is the first offender. 

\scendsutta{The training rule on visiting, the sixth, is finished. }

%
\section*{{\suttatitleacronym Bu Pc 47}{\suttatitletranslation 47. The training rule on Mahānāma }{\suttatitleroot Catumāsappaccayapavāraṇā}}
\addcontentsline{toc}{section}{\tocacronym{Bu Pc 47} \toctranslation{47. The training rule on Mahānāma } \tocroot{Catumāsappaccayapavāraṇā}}
\markboth{47. The training rule on Mahānāma }{Catumāsappaccayapavāraṇā}
\extramarks{Bu Pc 47}{Bu Pc 47}

\subsection*{Origin story }

At\marginnote{1.1.1} one time when the Buddha was staying in the Sakyan country in the Banyan Tree Monastery at Kapilavatthu, \textsanskrit{Mahānāma} the Sakyan had an abundance of tonics.\footnote{“Tonics” renders \textit{bhesajja}. In the origin story below \textit{bhesajja} seems to refer to the five standard tonics, and I translate it accordingly. The commentary at Sp 2.306 glosses \textit{bhesajja} with \textit{\textsanskrit{sappitelādīsu}}, “Ghee, oil, etc.”, which lends some support to this view. In the actual rule, however, the word \textit{paccaya} is used, which suggests a broader meaning. See comment to the actual rule below. } He went to the Buddha, bowed, sat down, and said, “Sir, I wish to invite the Sangha to ask for tonics for four months.” 

“Good,\marginnote{1.1.5} good, \textsanskrit{Mahānāma}. Please do so.” 

But\marginnote{1.1.7} the monks were afraid of wrongdoing and did not accept. They then told the Buddha what had happened. … 

\scrule{“Monks, I allow you to accept an invitation to ask for tonics for four months.” }

Yet\marginnote{1.2.1} the monks only asked \textsanskrit{Mahānāma} for a small amount of tonics, and so he still had an abundance. A second time he went to the Buddha, bowed, sat down, and said, “Sir, I wish to invite the Sangha to ask for tonics for a further four months.” 

“Good,\marginnote{1.2.4} good, \textsanskrit{Mahānāma}. Please do so.” 

Again\marginnote{1.2.6} the monks were afraid of wrongdoing and did not accept. They told the Buddha. … 

\scrule{“Monks, I allow you to accept a further invitation.” }

Once\marginnote{1.3.1} again the monks only asked \textsanskrit{Mahānāma} for a small amount of tonics, and so he still had an abundance. A third time he went to the Buddha, bowed, sat down, and said, “Sir, I wish to invite the Sangha to ask for tonics for life.” 

“Good,\marginnote{1.3.4} good, \textsanskrit{Mahānāma}. Please do so.” 

Yet\marginnote{1.3.6} again the monks were afraid of wrongdoing and did not accept. They told the Buddha. … 

\scrule{“Monks, I allow you to accept a permanent invitation.” }

At\marginnote{1.4.1} that time the monks from the group of six were shabbily dressed and improper in appearance. \textsanskrit{Mahānāma} criticized them, “Venerables, why are you shabbily dressed and improper in appearance? Shouldn’t one who has gone forth be suitably dressed and proper in appearance?” 

The\marginnote{1.4.4} monks from the group of six developed a grudge against \textsanskrit{Mahānāma}. Thinking of ways to humiliate him, it occurred to them, “\textsanskrit{Mahānāma} has invited the Sangha to ask for tonics. Let’s ask him for ghee.” 

They\marginnote{1.4.7} then went to \textsanskrit{Mahānāma} and said, “We need a \textit{\textsanskrit{doṇa}} measure of ghee.”\footnote{According T. W. Rhys Davids in “On the Ancient Coins and Measures of Ceylon: with a discussion of the Ceylon date of the Buddha's death”, p. 18, one \textit{\textsanskrit{doṇa}} is equivalent to 64 handfuls. This is perhaps roughly equivalent to one liter. } 

“Please\marginnote{1.4.8} wait until tomorrow. People have gone to the cow-pen to get ghee. You may come and get it in the morning.” 

A\marginnote{1.4.11} second time and a third time the monks from the group of six said the same thing, and \textsanskrit{Mahānāma} replied as before. They then said, “Why do you give an invitation if you don’t wish to give?” 

\textsanskrit{Mahānāma}\marginnote{1.4.17} complained and criticized them, “How can they not wait for one day when asked?” 

The\marginnote{1.4.18} monks heard the complaints of \textsanskrit{Mahānāma}, and the monks of few desires complained and criticized those monks, “How could the monks from the group of six not wait for one day when asked by \textsanskrit{Mahānāma}?” … “Is it true, monks, that you acted like this?” 

“It’s\marginnote{1.4.22} true, sir.” 

The\marginnote{1.4.23} Buddha rebuked them … “Foolish men, how could you act like this? This will affect people’s confidence …” … “And, monks, this training rule should be recited like this: 

\subsection*{Final ruling }

\scrule{‘A monk who is not sick may accept an invitation to ask for requisites for four months. If he accepts one beyond that limit, except if it is a further invitation or a permanent invitation, he commits an offense entailing confession.’”\footnote{I render \textit{paccaya} as “requisites”, not as “tonics”. It might seem from the origin story and the word commentary below that this rule deals with tonics or medicines, \textit{bhesajja}. The problem with this view, however, is that \textit{paccaya}, when used alone, never has this meaning in the Vinaya, nor in the four main \textsanskrit{Nikāyas}. When \textit{paccaya} is used by itself, it invariably means cause, condition, foundation, support, or something to this effect. It only refers to medicines in compounds such as \textit{\textsanskrit{gilāna}-ppaccaya-bhesajja-\textsanskrit{parikkhāra}}, “medicines and requisites to \textit{support} the sick”, where it is the context that gives it its specific reference. In the present rule there is no such context. The meaning of the word remains “support”. Given that the rules must have existed before the \textsanskrit{Vibhaṅga}, the meaning of the rule is independent of it. We can only conclude that the meaning of \textit{paccaya} must be “support”, that is, any material requisite that supports the monastic life. See \href{https://suttacentral.net/an4.79/en/brahmali\#2.1}{AN 4.79} for the use of \textit{paccaya} in this sense. } }

\subsection*{Definitions }

\begin{description}%
\item[A monk who is not sick may accept an invitation to ask for requisites for four months: ] he may accept an invitation to ask for requisites for the sick. %
\item[He may also accept a further invitation: ] he should think, “I’ll ask when I’m sick.” %
\item[He may also accept a permanent invitation: ] he should think, “I’ll ask when I’m sick.” %
\item[If he accepts one beyond that limit: ] there are invitations that have a limit on the tonics, but no limit on the time period; there are invitations that have a limit on the time period, but no limit on the tonics; there are invitations that have a limit on both the tonics and the time period; there are invitations that have neither a limit on the tonics nor on the time period. %
\item[Limit on the tonics: ] the tonics are restricted: “I invite you to ask for these particular tonics.” %
\item[Limit on the time period: ] the time period is restricted: “I invite you to ask during this particular period of time.” %
\item[Limit on both the tonics and the time period: ] both the tonics and the time period are restricted: “I invite you to ask for these particular tonics during this particular period of time.” %
\item[Neither a limit on the tonics nor on the time period: ] neither the tonics nor the time period is restricted. %
\end{description}

When\marginnote{2.1.20} there is a limit on the tonics, if he asks for tonics other than those he has been invited to ask for, he commits an offense entailing confession. When there is a limit on the time period, if he asks outside of the period during which he has been invited to ask, he commits an offense entailing confession. When there is a limit on both the tonics and on the time period, if he asks for tonics other than those he has been invited to ask for or he asks outside of the period during which he has been invited to ask, he commits an offense entailing confession. When there is neither a limit on the tonics nor on the time period, there is no offense. 

If\marginnote{2.2.1} he asks for tonics when he has no need for tonics, he commits an offense entailing confession. If he asks for a tonic other than the tonic he needs, he commits an offense entailing confession. 

\subsection*{Permutations }

If\marginnote{2.2.3.1} it is beyond the limit, and he perceives it as such, and he asks for tonics, he commits an offense entailing confession. If it is beyond the limit, but he is unsure of it, and he asks for tonics, he commits an offense entailing confession. If it is beyond the limit, but he does not perceive it as such, and he asks for tonics, he commits an offense entailing confession. 

If\marginnote{2.2.6} it is not beyond the limit, but he perceives it as such, he commits an offense of wrong conduct. If it is not beyond the limit, but he is unsure of it, he commits an offense of wrong conduct. If it is not beyond the limit, and he does not perceive it as such, there is no offense. 

\subsection*{Non-offenses }

There\marginnote{2.3.1} is no offense: if he asks for those tonics for which he was invited to ask; if he asks during the time period for which he was invited to ask; if he asks by informing, “You have invited me to ask for these tonics, but I need such-and-such a tonic;” if he asks by informing, “The time period during which you invited me to ask has passed, but I need tonics;” if it is from relatives; if it is from those who have given an invitation; if it is for the benefit of someone else; if it is by means of his own property; if he is insane; if he is the first offender. 

\scendsutta{The training rule on \textsanskrit{Mahānāma}, the seventh, is finished. }

%
\section*{{\suttatitleacronym Bu Pc 48}{\suttatitletranslation 48. The training rule on armies }{\suttatitleroot Uyyuttasenā}}
\addcontentsline{toc}{section}{\tocacronym{Bu Pc 48} \toctranslation{48. The training rule on armies } \tocroot{Uyyuttasenā}}
\markboth{48. The training rule on armies }{Uyyuttasenā}
\extramarks{Bu Pc 48}{Bu Pc 48}

\subsection*{Origin story }

\subsubsection*{First sub-story }

On\marginnote{1.1} one occasion when the Buddha was staying at \textsanskrit{Sāvatthī} in \textsanskrit{Anāthapiṇḍika}’s Monastery, King Pasenadi of Kosala was marching out with the army, and the monks from the group of six went to see it. When King Pasenadi saw the monks coming, he summoned them and said, “Venerables, why have you come here?” 

“We\marginnote{1.7} wish to see the great king.” 

“What’s\marginnote{1.8} the use of seeing me finding pleasure in battle? Shouldn’t you see the Buddha?” 

And\marginnote{1.10} people complained and criticized them, “How can the Sakyan monastics go to see the army? It’s our misfortune that we must go out with the army for the sake of our livelihoods and because of our wives and children.” 

The\marginnote{1.13} monks heard the complaints of those people, and the monks of few desires complained and criticized those monks, “How could the monks from the group of six go to see the army?” … “Is it true, monks, that you did this?” 

“It’s\marginnote{1.17} true, sir.” 

The\marginnote{1.18} Buddha rebuked them … “Foolish men, how could you do this? This will affect people’s confidence …” … “And, monks, this training rule should be recited like this: 

\subsubsection*{Preliminary ruling }

\scrule{‘If a monk goes to see an army, he commits an offense entailing confession.’” }

In\marginnote{1.23} this way the Buddha laid down this training rule for the monks. 

\subsubsection*{Second sub-story }

Soon\marginnote{2.1} afterwards a certain monk had a sick uncle in the army. The uncle sent a message to that monk: “I’m with the army and I’m sick. Please come, venerable. I want you to come.” 

Knowing\marginnote{2.6} that the Buddha had laid down a rule against going to see an army, that monk thought, “I have a sick uncle in the army. What should I do now?” And he told the Buddha. Soon afterwards the Buddha gave a teaching and addressed the monks: 

\scrule{“Monks, I allow you to go to the army when there’s a suitable reason. }

And\marginnote{2.12} so, monks, this training rule should be recited like this: 

\subsection*{Final ruling }

\scrule{‘If a monk goes to see an army, except if there is a suitable reason, he commits an offense entailing confession.’”\footnote{“Army” renders \textit{uyyutta sena}, which might be rendered “a mobilized army”. At the time of the Buddha it seems armies consisted entirely of reserve troops that would then be mobilized at times of war. With the advent of professional armies, however, armies are always more or less mobilized. Thus my rendering. } }

\subsection*{Definitions }

\begin{description}%
\item[A: ] whoever … %
\item[Monk: ] … The monk who has been given the full ordination by a unanimous Sangha through a legal procedure consisting of one motion and three announcements that is irreversible and fit to stand—this sort of monk is meant in this case. %
\item[An army: ] it has left the inhabited area and is either encamped or marching. %
\item[An army: ] elephants, horses, chariots, infantry. An elephant has twelve men; a horse has three men; a chariot has four men; an infantry unit has four men with arrows in hand. %
\end{description}

If\marginnote{3.1.10} he is on his way to see it, he commits an offense of wrong conduct. Wherever he stands to see it, he commits an offense entailing confession. Every time he goes beyond the range of sight and then sees it again, he commits an offense entailing confession. 

\begin{description}%
\item[Except if there is a suitable reason: ] unless there is a suitable reason. %
\end{description}

\subsection*{Permutations }

If\marginnote{3.2.1} it is an army, and he perceives it as such, and he goes to see it, except if there is a suitable reason, he commits an offense entailing confession. If it is an army, but he is unsure of it, and he goes to see it, except if there is a suitable reason, he commits an offense entailing confession. If it is an army, but he does not perceive it as such, and he goes to see it, except if there is a suitable reason, he commits an offense entailing confession. 

If\marginnote{3.2.4} he is on his way to see one division of a fourfold army, he commits an offense of wrong conduct. Wherever he stands to see it, he commits an offense of wrong conduct. Every time he goes beyond the range of sight and then sees it again, he commits an offense of wrong conduct. 

If\marginnote{3.2.7} it is not an army, but he perceives it as such, he commits an offense of wrong conduct. If it is not an army, but he is unsure of it, he commits an offense of wrong conduct. If it is not an army, and he does not perceive it as such, there is no offense. 

\subsection*{Non-offenses }

There\marginnote{3.3.1} is no offense: if he sees it while standing in a monastery; if the army comes to where the monk is standing, sitting, or lying down; if he sees it while walking in the opposite direction; if he has a suitable reason; if there is an emergency; if he is insane; if he is the first offender. 

\scendsutta{The training rule on armies, the eighth, is finished. }

%
\section*{{\suttatitleacronym Bu Pc 49}{\suttatitletranslation 49. The training rule on staying with armies }{\suttatitleroot Senāvāsa}}
\addcontentsline{toc}{section}{\tocacronym{Bu Pc 49} \toctranslation{49. The training rule on staying with armies } \tocroot{Senāvāsa}}
\markboth{49. The training rule on staying with armies }{Senāvāsa}
\extramarks{Bu Pc 49}{Bu Pc 49}

\subsection*{Origin story }

On\marginnote{1.1} one occasion when the Buddha was staying at \textsanskrit{Sāvatthī} in \textsanskrit{Anāthapiṇḍika}’s Monastery, the monks from the group of six went to the army on some business, and they stayed there for more than three nights. People complained and criticized them, “How can the Sakyan monastics stay with the army? It’s our misfortune that we must stay with the army for the sake of our livelihoods and because of our wives and children.” 

The\marginnote{1.6} monks heard the complaints of those people, and the monks of few desires complained and criticized those monks, “How could the monks from the group of six stay with the army for more than three nights?” … “Is it true, monks, that you did this?” 

“It’s\marginnote{1.10} true, sir.” 

The\marginnote{1.11} Buddha rebuked them … “Foolish men, how could you do this? This will affect people’s confidence …” … “And, monks, this training rule should be recited like this: 

\subsection*{Final ruling }

\scrule{‘If that monk has a reason for going to the army, he may stay with the army for two or three nights. If he stays longer than that, he commits an offense entailing confession.’” }

\subsection*{Definitions }

\begin{description}%
\item[If that monk has a reason for going to the army: ] if he has a reason, if he has something to  do. %
\item[He may stay with the army for two or three nights: ] he may stay for two or for three nights. %
\item[If he stays longer than that: ] if he is staying with the army at sunset on the fourth day, he commits an offense entailing confession. %
\end{description}

\subsection*{Permutations }

If\marginnote{2.2.1} it is more than three nights, and he perceives it as more, and he is staying with the army, he commits an offense entailing confession. If it is more than three nights, but he is unsure of it, and he is staying with the army, he commits an offense entailing confession. If it is more than three nights, but he perceives it as less, and he is staying with the army, he commits an offense entailing confession. 

If\marginnote{2.2.4} it is less than three nights, but he perceives it as more, he commits an offense of wrong conduct. If it is less than three nights, but he is unsure of it, he commits an offense of wrong conduct. If it is less than three nights, and he perceives it as less, there is no offense. 

\subsection*{Non-offenses }

There\marginnote{2.3.1} is no offense: if he stays for two or three nights; if he stays for less than two or three nights; if he stays for two nights, then leaves before dawn on the third night, and then stays again; if he stays because he is sick; if he stays because he has to take care of someone who is sick; if the army is obstructed by an enemy army; if he is obstructed from leaving; if there is an emergency; if he is insane; if he is the first offender. 

\scendsutta{The training rule on staying with armies, the ninth, is finished. }

%
\section*{{\suttatitleacronym Bu Pc 50}{\suttatitletranslation 50. The training rule on battles }{\suttatitleroot Uyyodhika}}
\addcontentsline{toc}{section}{\tocacronym{Bu Pc 50} \toctranslation{50. The training rule on battles } \tocroot{Uyyodhika}}
\markboth{50. The training rule on battles }{Uyyodhika}
\extramarks{Bu Pc 50}{Bu Pc 50}

\subsection*{Origin story }

On\marginnote{1.1} one occasion when the Buddha was staying at \textsanskrit{Sāvatthī} in \textsanskrit{Anāthapiṇḍika}’s Monastery, the monks from the group of six were staying with the army for two or three nights. They went to see battles, troop reviews, the massing of the army, and troop inspections. One of the monks who went to a battle was struck by an arrow. People teased him, “We hope you had a good battle, venerable. How many targets did you hit?” And because they teased him, he felt humiliated. 

People\marginnote{1.7} complained and criticized those monks, “How can the Sakyan monastics go to see a battle? It’s our misfortune that we must go to battles for the sake of our livelihoods and because of our wives and children.” 

The\marginnote{1.10} monks heard the complaints of those people, and the monks of few desires complained and criticized those monks, “How could the monks from the group of six go to see a battle?” … “Is it true, monks, that you did this?” 

“It’s\marginnote{1.14} true, sir.” 

The\marginnote{1.15} Buddha rebuked them … “Foolish men, how could you do this? This will affect people’s confidence …” … “And, monks, this training rule should be recited like this: 

\subsection*{Final ruling }

\scrule{‘If a monk who is staying with an army for two or three nights goes to a battle, a troop review, a massing of the army, or a troop inspection, he commits an offense entailing confession.’” }

\subsection*{Definitions }

\begin{description}%
\item[If a monk who is staying with an army for two or three nights: ] he is staying for two or for three nights. %
\item[A battle: ] wherever fighting is seen. %
\item[A troop review: ] so many elephants, so many horses, so many chariots, so much infantry. %
\item[A massing of the army: ] the elephants should set out from here; the horses should set out from here; the chariots should set out from here; the infantry should set out from here. %
\item[A troop: ] an elephant troop, a horse troop, a chariot troop, an infantry troop. The smallest troop of elephants is three elephants; the smallest troop of horses is three horses; the smallest troop of chariots is three chariots; the smallest troop of infantry is four men with arrows in hand. %
\end{description}

If\marginnote{2.1.12} he is on his way to see it, he commits an offense of wrong conduct. Wherever he stands to see it, he commits an offense entailing confession. Every time he goes beyond the range of sight and then sees it again, he commits an offense entailing confession. 

\subsection*{Permutations }

If\marginnote{2.2.1} he is on his way to see one division of a fourfold army, he commits an offense of wrong conduct. Wherever he stands to see it, he commits an offense of wrong conduct. Every time he goes beyond the range of sight and then sees it again, he commits an offense of wrong conduct. 

\subsection*{Non-offenses }

There\marginnote{2.3.1} is no offense: if he sees it while standing in a monastery; if the army comes to where the monk is standing, sitting, or lying down, and he then sees fighting; if he sees it while walking in the opposite direction; if he goes because there is something to be done, and he then sees it; if there is an emergency; if he is insane; if he is the first offender. 

\scendsutta{The training rule on battles, the tenth, is finished. }

\scendvagga{The fifth subchapter on naked ascetics is finished. }

\scuddanaintro{This is the summary: }

\begin{scuddana}%
“Cookie,\marginnote{2.3.12} talking, three on Upananda, \\
And indeed supporting; \\
\textsanskrit{Mahānāma}, Pasenadi, \\
Army, and struck: those are the ten.” 

%
\end{scuddana}

%
\section*{{\suttatitleacronym Bu Pc 51}{\suttatitletranslation 51. The training rule on drinking alcoholic drinks }{\suttatitleroot Surāpāna}}
\addcontentsline{toc}{section}{\tocacronym{Bu Pc 51} \toctranslation{51. The training rule on drinking alcoholic drinks } \tocroot{Surāpāna}}
\markboth{51. The training rule on drinking alcoholic drinks }{Surāpāna}
\extramarks{Bu Pc 51}{Bu Pc 51}

\subsection*{Origin story }

On\marginnote{1.1} one occasion when the Buddha was wandering in the country of Ceti on his way to \textsanskrit{Bhaddavatikā}, he was seen by a number of cowherds, shepherds, farmers, and travelers.\footnote{This a standard group of people also found at \href{https://suttacentral.net/mn50/en/brahmali\#10.3}{MN 50} and \href{https://suttacentral.net/mn86/en/brahmali\#3.3}{MN 86}. } They said to him, “Sir, don’t go to Ambatittha. There’s a highly venomous dragon with supernormal powers there, in the hermitage of a dreadlocked ascetic. Don’t let it harm you.” The Buddha was silent. They repeated their request a second and a third time, and the Buddha remained silent. 

The\marginnote{1.13} Buddha then continued on to \textsanskrit{Bhaddavatikā}, and he stayed there. 

Just\marginnote{1.14} then Venerable \textsanskrit{Sāgata} went to the hermitage of that dreadlocked ascetic and entered his fire hut. After preparing a spread of grass, he sat down, crossed his legs, straightened his body, and established mindfulness in front of him. Seeing that \textsanskrit{Sāgata} had entered the fire hut, the dragon was upset and emitted smoke. \textsanskrit{Sāgata}, too, emitted smoke. The dragon was not able to contain his rage and emitted flames. \textsanskrit{Sāgata} entered the fire element and he, too, emitted flames. Then, after conquering fire with fire, \textsanskrit{Sāgata} went to \textsanskrit{Bhaddavatikā}. 

After\marginnote{1.20} staying at \textsanskrit{Bhaddavatikā} for as long as he liked, the Buddha set out wandering toward \textsanskrit{Kosambī}. 

When\marginnote{1.22} he arrived, the lay followers there received him. 

But\marginnote{1.23} the lay followers at \textsanskrit{Kosambī} had heard about \textsanskrit{Sāgata} fighting the Ambatittha dragon. And so after receiving the Buddha, they went to see \textsanskrit{Sāgata}. They bowed, stood to one side, and said, “Venerable, what can we prepare for you that’s delicious but hard to get?” 

The\marginnote{1.25} monks from the group of six replied, “There’s a delicious liquor called \textsanskrit{Kāpotikā}, which is hard for the monks to get. Prepare that.” 

And\marginnote{1.26} the lay followers prepared \textsanskrit{Kāpotikā} in house after house. Then, when they saw that \textsanskrit{Sāgata} had entered the town for alms, they said to him, “Drink, venerable, drink the \textsanskrit{Kāpotikā} liquor.” \textsanskrit{Sāgata} drank that liquor in house after house, and as he was leaving town, he collapsed at the town gate. 

Just\marginnote{1.28} then the Buddha, together with a number of monks, was also leaving town, and he saw \textsanskrit{Sāgata} at the town gate. He said, “Monks, pick up \textsanskrit{Sāgata}.” Saying, “Yes, sir,” they led him to the monastery, where they put him down with his head toward the Buddha. But \textsanskrit{Sāgata} turned around, pointing his feet toward the Buddha. 

The\marginnote{1.32} Buddha said, “Previously, monks, wasn’t \textsanskrit{Sāgata} respectful and deferential toward me?” 

“Yes.”\marginnote{1.33} 

“But\marginnote{1.34} is he now?” 

“Certainly\marginnote{1.35} not.” 

“Just\marginnote{1.36} recently, didn’t \textsanskrit{Sāgata} fight the Ambatittha dragon?” 

“Yes.”\marginnote{1.37} 

“Would\marginnote{1.38} he now be able to fight a dragon?” 

“Certainly\marginnote{1.39} not.” 

“So,\marginnote{1.40} monks, should one drink that which makes one senseless?” 

“Certainly\marginnote{1.41} not, sir.” 

“It’s\marginnote{1.42} not suitable, monks, it’s not proper for \textsanskrit{Sāgata}, it’s not worthy of a monastic, it’s not allowable, it’s not to be done. How could \textsanskrit{Sāgata} drink alcoholic drinks? This will affect people’s confidence …” … “And, monks, this training rule should be recited like this: 

\subsection*{Final ruling }

\scrule{‘If a monk drinks this or that kind of alcoholic drink, he commits an offense entailing confession.’” }

\subsection*{Definitions }

\begin{description}%
\item[This kind of alcoholic drink: ] alcoholic drinks made from flour, alcoholic drinks made from cookies, alcoholics drink made from rice, those with yeast added, those made from a combination of ingredients. %
\item[That kind of alcoholic drink: ] alcoholic drinks made from flowers, alcoholic drinks made from fruit, alcoholic drinks made from honey, alcoholic drinks made from sugar, those made from a combination of ingredients. %
\item[Drinks: ] if he drinks even what fits on the tip of a blade of grass, he commits an offense entailing confession. %
\end{description}

\subsection*{Permutations }

If\marginnote{2.2.1} it is an alcoholic drink, and he perceives it as such, and he drinks it, he commits an offense entailing confession. If it is an alcoholic drink, but he is unsure of it, and he drinks it, he commits an offense entailing confession. If it is an alcoholic drink, but he perceives it as non-alcoholic, and he drinks it, he commits an offense entailing confession. 

If\marginnote{2.2.4} it is a non-alcoholic drink, but he perceives it as alcoholic, he commits an offense of wrong conduct. If it is a non-alcoholic drink, but he is unsure of it, he commits an offense of wrong conduct. If it is a non-alcoholic drink, and he perceives it as such, there is no offense. 

\subsection*{Non-offenses }

There\marginnote{2.3.1} is no offense: if he drinks a non-alcoholic drink that has the color, smell, or taste of an alcoholic drink;  if it is cooked in a bean curry;  if it is cooked with meat;  if it is cooked with oil;  if it is in syrup from emblic myrobalan;  if he drinks a drink that is normally alcoholic, but which is actually without alcohol;  if he is insane;  if he is the first offender. 

\scendsutta{The training rule on drinking alcoholic drinks, the first, is finished. }

%
\section*{{\suttatitleacronym Bu Pc 52}{\suttatitletranslation 52. The training rule on tickling }{\suttatitleroot Aṅgulipatodaka}}
\addcontentsline{toc}{section}{\tocacronym{Bu Pc 52} \toctranslation{52. The training rule on tickling } \tocroot{Aṅgulipatodaka}}
\markboth{52. The training rule on tickling }{Aṅgulipatodaka}
\extramarks{Bu Pc 52}{Bu Pc 52}

\subsection*{Origin story }

On\marginnote{1.1} one occasion when the Buddha was staying at \textsanskrit{Sāvatthī} in \textsanskrit{Anāthapiṇḍika}’s Monastery, the monks from the group of six tickled one of the monks from the group of seventeen to make him laugh. Not being able to catch his breath, he died. 

The\marginnote{1.4} monks of few desires complained and criticized them, “How could the monks from the group of six tickle a monk to make him laugh?” … “Is it true, monks, that you did this?” 

“It’s\marginnote{1.7} true, sir.” 

The\marginnote{1.8} Buddha rebuked them … “Foolish men, how could you do this? This will affect people’s confidence …” … “And, monks, this training rule should be recited like this: 

\subsection*{Final ruling }

\scrule{‘If a monk tickles someone, he commits an offense entailing confession.’” }

\subsection*{Definitions }

\begin{description}%
\item[Tickles someone: ] if one who is fully ordained makes physical contact with another who is fully ordained, body to body, with the aim of making him laugh, he commits an offense entailing confession. %
\end{description}

\subsection*{Permutations }

If\marginnote{2.2.1} the other person is fully ordained, and he perceives him as such, and he tickles him to make him laugh, he commits an offense entailing confession. If the other person is fully ordained, but he is unsure of it, and he tickles him to make him laugh, he commits an offense entailing confession. If the other person is fully ordained, but he does not perceive him as such, and he tickles him to make him laugh, he commits an offense entailing confession. 

If,\marginnote{2.2.4} with his own body, he makes physical contact with something connected to the other monk’s body, he commits an offense of wrong conduct. If, with something connected to his own body, he makes physical contact with the other monk’s body, he commits an offense of wrong conduct. If, with something connected to his own body, he makes physical contact with something connected to the other monk’s body, he commits an offense of wrong conduct. 

If,\marginnote{2.2.7} by releasing something, he makes physical contact with the other monk’s body, he commits an offense of wrong conduct. If, by releasing something, he makes physical contact with something connected to the other monk’s body, he commits an offense of wrong conduct. If, by releasing something, he makes physical contact with something released by the other monk, he commits an offense of wrong conduct. 

If,\marginnote{2.2.10} with his own body, he makes physical contact with the body of someone who is not fully ordained, he commits an offense of wrong conduct. If, with his own body, he makes physical contact with something connected to the body of someone who is not fully ordained, he commits an offense of wrong conduct. If, with something connected to his own body, he makes physical contact with the body of someone who is not fully ordained, he commits an offense of wrong conduct. If, with something connected to his own body, he makes physical contact with something connected to the body of someone who is not fully ordained, he commits an offense of wrong conduct. 

If,\marginnote{2.2.14} by releasing something, he makes physical contact with the body of someone who is not fully ordained, he commits an offense of wrong conduct. If, by releasing something, he makes physical contact with something connected to the body of someone who is not fully ordained, he commits an offense of wrong conduct. If, by releasing something, he makes physical contact with something released by the someone who is not fully ordained, he commits an offense of wrong conduct. 

If\marginnote{2.2.17} the other person is not fully ordained, but he perceives them as such, he commits an offense of wrong conduct. If the other person is not fully ordained, but he is unsure of it, he commits an offense of wrong conduct. If the other person is not fully ordained, and he does not perceive them as such, he commits an offense of wrong conduct. 

\subsection*{Non-offenses }

There\marginnote{2.3.1} is no offense: if he is not aiming to make him laugh;  if he makes physical contact with him when there is a need;  if he is insane;  if he is the first offender. 

\scendsutta{The training rule on tickling, the second, is finished. }

%
\section*{{\suttatitleacronym Bu Pc 53}{\suttatitletranslation 53. The training rule on playing }{\suttatitleroot Hasadhamma}}
\addcontentsline{toc}{section}{\tocacronym{Bu Pc 53} \toctranslation{53. The training rule on playing } \tocroot{Hasadhamma}}
\markboth{53. The training rule on playing }{Hasadhamma}
\extramarks{Bu Pc 53}{Bu Pc 53}

\subsection*{Origin story }

On\marginnote{1.1} one occasion when the Buddha was staying at \textsanskrit{Sāvatthī} in \textsanskrit{Anāthapiṇḍika}’s Monastery, the monks from the group of seventeen were playing in the water of the river \textsanskrit{Aciravatī}. Just then, while King Pasenadi of Kosala was up in his finest stilt house with Queen \textsanskrit{Mallikā}, he saw the monks from the group of seventeen playing in the river. He said to Queen \textsanskrit{Mallikā}, “\textsanskrit{Mallikā}, these perfected ones are playing in the water.” 

“Great\marginnote{1.7} king, no doubt the Buddha hasn’t laid down a rule. Either that, or these monks are ignorant.” 

King\marginnote{1.9} Pasenadi thought, “How can the Buddha find out about these monks playing in the water without me telling him?” 

Having\marginnote{1.11} sent for those monks, King Pasenadi gave them a large lump of sugar, saying, “Sirs, please give this lump of sugar to the Buddha.” 

The\marginnote{1.13} monks took the lump of sugar, went to the Buddha, and said, “Sir, this lump of sugar is a gift from King Pasenadi.” 

“But,\marginnote{1.16} monks, where did you see the king?” 

“From\marginnote{1.17} the river \textsanskrit{Aciravatī}, while playing in the water.” 

The\marginnote{1.18} Buddha rebuked them … “Foolish men, how can you play in water? This will affect people’s confidence …” … “And, monks, this training rule should be recited like this: 

\subsection*{Final ruling }

\scrule{‘If a monk plays in water, he commits an offense entailing confession.’” }

\subsection*{Definitions }

\begin{description}%
\item[Plays in water: ] if, aiming to have fun, he immerses himself or emerges on the surface or swims in water that is more than ankle deep, he commits an offense entailing confession. %
\end{description}

\subsection*{Permutations }

If\marginnote{2.2.1} he is playing in water, and he perceives that he is, he commits an offense entailing confession. If he is playing in water, but he is unsure of it, he commits an offense entailing confession. If he is playing in water, but he does not perceive that he is, he commits an offense entailing confession. 

If\marginnote{2.2.4} he is playing in water less than ankle deep, he commits an offense of wrong conduct. 

If\marginnote{2.2.5} he is playing in a boat in water, he commits an offense of wrong conduct. 

If\marginnote{2.2.6} he strikes the water with his hand, with his foot, with a stick, or with a stone, he commits an offense of wrong conduct. 

If\marginnote{2.2.7} he plays with water in a vessel, or with congee, milk, buttermilk, dye, urine, or mud in a vessel, he commits an offense of wrong conduct. 

If\marginnote{2.2.8} he is not playing in water, but he perceives that he is, he commits an offense of wrong conduct. If he is not playing in water, but he is unsure of it, he commits an offense of wrong conduct. If he is not playing in water, and he does not perceive that he is, there is no offense. 

\subsection*{Non-offenses }

There\marginnote{2.3.1} is no offense: if he is not aiming to have fun;  if, when there is something to be done, he enters the water and then immerses himself or emerges on the surface or swims;  if, while crossing a body of water, he immerses himself or emerges on the surface or swims;  if there is an emergency;  if he is insane;  if he is the first offender. 

\scendsutta{The training rule on playing, the third, is finished. }

%
\section*{{\suttatitleacronym Bu Pc 54}{\suttatitletranslation 54. The training rule on disrespect }{\suttatitleroot Anādariya}}
\addcontentsline{toc}{section}{\tocacronym{Bu Pc 54} \toctranslation{54. The training rule on disrespect } \tocroot{Anādariya}}
\markboth{54. The training rule on disrespect }{Anādariya}
\extramarks{Bu Pc 54}{Bu Pc 54}

\subsection*{Origin story }

At\marginnote{1.1} one time when the Buddha was staying at \textsanskrit{Kosambī} in Ghosita’s Monastery, Venerable Channa was misbehaving. The monks would tell him, “Channa, don’t do that; it’s not allowable,” and he just did it again out of disrespect. 

The\marginnote{1.7} monks of few desires complained and criticized him, “How can Venerable Channa act disrespectfully?” … “Is it true, Channa, that you do this?” 

“It’s\marginnote{1.10} true, sir.” 

The\marginnote{1.11} Buddha rebuked him … “Foolish man, how can you do this? This will affect people’s confidence …” … “And, monks, this training rule should be recited like this: 

\subsection*{Final ruling }

\scrule{‘If a monk is disrespectful, he commits an offense entailing confession.’” }

\subsection*{Definitions }

\begin{description}%
\item[Disrespectful: ] there are two kinds of disrespect: disrespect for the person and disrespect for the rule. %
\item[Disrespect for the person: ] if, when corrected by one who is fully ordained about a rule that has been laid down, he thinks, “They’ve been ejected,” “They’ve been reproved,” or “They’ve been censured,” and then, “I won’t do what they say,” and he acts disrespectfully, then he commits an offense entailing confession. %
\item[Disrespect for the rule: ] if, when corrected by one who is fully ordained about a rule that has been laid down, he thinks, “What can be done so that this rule is lost?” “What can be done so that it perishes?”  or “What can be done so that it disappears?” or he does not want to train in that rule, and he acts disrespectfully, then he commits an offense entailing confession. %
\end{description}

\subsection*{Permutations }

If\marginnote{2.2.1} the other person is fully ordained, and he perceives them as such, and he acts disrespectfully, he commits an offense entailing confession. If the other person is fully ordained, but he is unsure of it, and he acts disrespectfully, he commits an offense entailing confession. If the other person is fully ordained, but he does not perceive them as such, and he acts disrespectfully, he commits an offense entailing confession. 

If,\marginnote{2.2.4} when corrected about something that has not been laid down, he thinks, “This isn’t conducive to self-effacement,” “This isn’t conducive to ascetic practices,” “This isn’t conducive to being inspiring,” “This isn’t conducive to a reduction in things,” or “This isn’t conducive to being energetic,” and he acts disrespectfully, he commits an offense of wrong conduct. If, when corrected by one who is not fully ordained, whether or not it has been laid down, he thinks, “This isn’t conducive to self-effacement,” “This isn’t conducive to ascetic practices,” “This isn’t conducive to being inspiring,” “This isn’t conducive to a reduction in things,” or “This isn’t conducive to being energetic,” and he acts disrespectfully, he commits an offense of wrong conduct. 

If\marginnote{2.2.8} the other person is not fully ordained, but he perceives them as such, he commits an offense of wrong conduct. If the other person is not fully ordained, but he is unsure of it, he commits an offense of wrong conduct. If the other person is not fully ordained, and he does not perceive them as such, he commits an offense of wrong conduct. 

\subsection*{Non-offenses }

There\marginnote{2.3.1} is no offense: if he says, “This is how we were taught and tested by our teachers;”\footnote{“Testing” renders \textit{paripuccha}. The basic meaning of \textit{\textsanskrit{paripucchā}} is “to question” or “to ask”, as used for instance in \href{https://suttacentral.net/pli-tv-bu-vb-pc71/en/brahmali\#1.19.1}{Bu Pc 71}. Often, however, as in the present case, it refers to a teacher questioning his student, in the sense of finding out how much the student knows. In such cases I render the word as “testing”. }  if he is insane;  if he is the first offender. 

\scendsutta{The training rule on disrespect, the fourth, is finished. }

%
\section*{{\suttatitleacronym Bu Pc 55}{\suttatitletranslation 55. The training rule on scaring }{\suttatitleroot Bhiṁsāpana}}
\addcontentsline{toc}{section}{\tocacronym{Bu Pc 55} \toctranslation{55. The training rule on scaring } \tocroot{Bhiṁsāpana}}
\markboth{55. The training rule on scaring }{Bhiṁsāpana}
\extramarks{Bu Pc 55}{Bu Pc 55}

\subsection*{Origin story }

At\marginnote{1.1} one time when the Buddha was staying at \textsanskrit{Sāvatthī} in \textsanskrit{Anāthapiṇḍika}’s Monastery, the monks from the group of six were scaring the monks from the group of seventeen. They cried. Other monks asked them why, and they told them. 

The\marginnote{1.7} monks of few desires complained and criticized them, “How can the monks from the group of six scare other monks?” … “Is it true, monks, that you do this?” 

“It’s\marginnote{1.10} true, sir.” 

The\marginnote{1.11} Buddha rebuked them … “Foolish men, how can you do this? This will affect people’s confidence …” … “And, monks, this training rule should be recited like this: 

\subsection*{Final ruling }

\scrule{‘If a monk scares a monk, he commits an offense entailing confession.’” }

\subsection*{Definitions }

\begin{description}%
\item[A: ] whoever … %
\item[Monk: ] …The monk who has been given the full ordination by a unanimous Sangha through a legal procedure consisting of one motion and three announcements that is irreversible and fit to stand—this sort of monk is meant in this case. %
\item[A monk: ] another monk. %
\item[Scares: ] if one who is fully ordained, wishing to scare another who is fully ordained, arranges a sight, a sound, a smell, a taste, or a physical contact, then whether the other monk is scared or not, he commits an offense entailing confession. If he tells him about a wilderness inhabited by criminals, predatory animals, or demons, then whether the other monk is scared or not, he commits an offense entailing confession. %
\end{description}

\subsection*{Permutations }

If\marginnote{2.2.1} the other person is fully ordained, and he perceives him as such, and he scares him, he commits an offense entailing confession. If the other person is fully ordained, but he is unsure of it, and he scares him, he commits an offense entailing confession. If the other person is fully ordained, but he does not perceive him as such, and he scares him, he commits an offense entailing confession. 

If,\marginnote{2.2.4} wishing to scare someone who is not fully ordained, he arranges a sight, a sound, a smell, a taste, or a physical contact, then whether the other person is scared or not, he commits an offense of wrong conduct. If he tells him about a wilderness inhabited by criminals, predatory animals, or demons, then whether the other person is scared or not, he commits an offense of wrong conduct. 

If\marginnote{2.2.8} the other person is not fully ordained, but he perceives them as such, he commits an offense of wrong conduct. If the other person is not fully ordained, but he is unsure of it, he commits an offense of wrong conduct. If the other person is not fully ordained, and he does not perceive them as such, he commits an offense of wrong conduct. 

\subsection*{Non-offenses }

There\marginnote{2.3.1} is no offense: if he arranges a sight, a sound, a smell, a taste, or a physical contact, or he tells about a wilderness inhabited by criminals, predatory animals, or demons, but not because he wishes to scare anyone;  if he is insane;  if he is the first offender. 

\scendsutta{The training rule on scaring, the fifth, is finished. }

%
\section*{{\suttatitleacronym Bu Pc 56}{\suttatitletranslation 56. The training rule on fire }{\suttatitleroot Jotika}}
\addcontentsline{toc}{section}{\tocacronym{Bu Pc 56} \toctranslation{56. The training rule on fire } \tocroot{Jotika}}
\markboth{56. The training rule on fire }{Jotika}
\extramarks{Bu Pc 56}{Bu Pc 56}

\subsection*{Origin story }

\subsubsection*{First sub-story }

At\marginnote{1.1.1} one time the Buddha was staying in the Bhagga country at \textsanskrit{Susumāragira} in the \textsanskrit{Bhesakaḷā} Grove, the deer park. At that time, during winter, the monks were warming themselves after setting fire to a hollow log. Heated by the fire, a black snake came out of the log and attacked the monks. The monks ran here and there. 

The\marginnote{1.1.5} monks of few desires complained and criticized them, “How could those monks light a fire to warm themselves?” … “Is it true, monks, that monks did this?” 

“It’s\marginnote{1.1.8} true, sir.” 

The\marginnote{1.1.9} Buddha rebuked them … “How could those foolish men do this? This will affect people’s confidence …” … “And, monks, this training rule should be recited like this: 

\subsubsection*{First preliminary ruling }

\scrule{‘If a monk lights a fire to warm himself, or has one lit, he commits an offense entailing confession.’” }

In\marginnote{1.1.14} this way the Buddha laid down this training rule for the monks. 

\subsubsection*{Second sub-story }

At\marginnote{1.2.1} one time a number of monks were sick. The monks who were looking after them asked, “I hope you’re bearing up? I hope you’re getting better?” 

“Previously\marginnote{1.2.4} we lit a fire to warm ourselves, and then we were comfortable. But now that the Buddha has prohibited this, we don’t warm ourselves because we’re afraid of wrongdoing. Because of that we’re not comfortable.” 

They\marginnote{1.2.7} told the Buddha. Soon afterwards he gave a teaching and addressed the monks: 

\scrule{“Monks, I allow a sick monk to light a fire to warm himself, or to have one lit. }

And\marginnote{1.2.9} so, monks, this training rule should be recited like this: 

\subsubsection*{Second preliminary ruling }

\scrule{‘If a monk who is not sick lights a fire to warm himself, or has one lit, he commits an offense entailing confession.’” }

In\marginnote{1.2.11} this way the Buddha laid down this training rule for the monks. 

\subsubsection*{Third sub-story }

Soon\marginnote{2.1} afterwards the monks did not light lamps, small fires, or saunas because they were afraid of wrongdoing.\footnote{Sp 2.352: \textit{\textsanskrit{Jotikepīti} \textsanskrit{pattapacanasedakammādīsu} \textsanskrit{jotikaraṇe}}, “\textit{Jotikepi}: making a fire to fire a bowl, to make sweat, etc.” } They told the Buddha … 

\scrule{“Monks, I allow you to light a fire, or to have one lit, if there’s a suitable reason. }

And\marginnote{2.4} so, monks, this training rule should be recited like this: 

\subsection*{Final ruling }

\scrule{‘If a monk who is not sick lights a fire to warm himself, or has one lit, except if there is a suitable reason, he commits an offense entailing confession.’” }

\subsection*{Definitions }

\begin{description}%
\item[A: ] whoever … %
\item[Monk: ] …The monk who has been given the full ordination by a unanimous Sangha through a legal procedure consisting of one motion and three announcements that is irreversible and fit to stand—this sort of monk is meant in this case. %
\item[Who is not sick: ] who is comfortable without a fire. %
\item[Who is sick: ] who is not comfortable without a fire. %
\item[To warm himself: ] wanting to heat himself. %
\item[A fire: ] flames are what is meant. %
\item[Lights: ] if he lights it himself, he commits an offense entailing confession. %
\item[Has one lit: ] if he asks another, he commits an offense entailing confession. If he only asks once, then even if the other lights many fires, he commits one offense entailing confession. %
\item[Except if there is a suitable reason: ] unless there is a suitable reason. %
\end{description}

\subsection*{Permutations }

If\marginnote{3.2.1} he is not sick, and he does not perceive himself as sick, and he lights a fire to warm himself, or has one lit, except if there is a suitable reason, he commits an offense entailing confession. If he is not sick, but he is unsure of it, and he lights a fire to warm himself, or has one lit, except if there is a suitable reason, he commits an offense entailing confession. If he is not sick, but he perceives himself as sick, and he lights a fire to warm himself, or has one lit, except if there is a suitable reason, he commits an offense entailing confession. 

If\marginnote{3.2.4} he puts back a burning piece of wood that has fallen off, he commits an offense of wrong conduct.\footnote{Sp 2.355: \textit{\textsanskrit{Paṭilātaṁ} \textsanskrit{ukkhipatīti} \textsanskrit{dayhamānaṁ} \textsanskrit{alātaṁ} \textsanskrit{patitaṁ} ukkhipati, puna \textsanskrit{yathāṭhāne} \textsanskrit{ṭhapetīti} attho}, “\textit{\textsanskrit{Paṭilātaṁ} ukkhipati}: he picks up a burning firebrand that has fallen. The meaning is that he puts it back in its original place.” } If he is sick, but he does not perceive himself as sick, he commits an offense of wrong conduct. If he is sick, but he is unsure of it, he commits an offense of wrong conduct. If he is sick, and he perceives himself as sick, there is no offense. 

\subsection*{Non-offenses }

There\marginnote{3.3.1} is no offense: if he is sick;  if he warms himself over a fire lit by another;  if he warms himself over flameless coals;  if he lights a lamp, a small fire, or a sauna, when there is a suitable reason;  if there is an emergency;  if he is insane;  if he is the first offender. 

\scendsutta{The training rule on fire, the sixth, is finished. }

%
\section*{{\suttatitleacronym Bu Pc 57}{\suttatitletranslation 57. The training rule on bathing }{\suttatitleroot Nahāna}}
\addcontentsline{toc}{section}{\tocacronym{Bu Pc 57} \toctranslation{57. The training rule on bathing } \tocroot{Nahāna}}
\markboth{57. The training rule on bathing }{Nahāna}
\extramarks{Bu Pc 57}{Bu Pc 57}

\subsection*{Origin story }

\subsubsection*{First sub-story }

At\marginnote{1.1} one time when the Buddha was staying at \textsanskrit{Rājagaha} in the Bamboo Grove, the monks were bathing in the hot springs. Just then King Seniya \textsanskrit{Bimbisāra} of Magadha went to the hot springs, intending to wash his hair. He waited respectfully for the monks to finish, but they kept on bathing until dark. Only then was King \textsanskrit{Bimbisāra} able to wash his hair. And because the town gates had been shut, he had to spend the night outside the city.\footnote{\textit{Asambhinnena vilepanena} is an unusual expression. Sp-\textsanskrit{ṭ} 2.357: \textit{\textsanskrit{Asambhinnenāti} amakkhitena, \textsanskrit{anaṭṭhenāti} attho}, “\textit{Asambhinnena}: the meaning is ‘with (make-up) not smeared, not lost’.” } 

In\marginnote{1.6} the morning, with his make-up still on, he went to the Buddha, bowed, and sat down. The Buddha said to him, “Great king, why have you come so early in the morning, with your make-up still on?” The king told him what had happened. The Buddha then instructed, inspired, and gladdened him with a teaching, after which the king got up from his seat, bowed down, circumambulated the Buddha with his right side toward him, and left. 

Soon\marginnote{1.10} afterwards the Buddha had the Sangha gathered and questioned the monks: “Is it true, monks, that monks bathed without moderation, even after seeing the king?” 

“It’s\marginnote{1.12} true, sir.” 

The\marginnote{1.13} Buddha rebuked them … “How could those foolish men act in this way? This will affect people’s confidence …” … “And, monks, this training rule should be recited like this: 

\subsubsection*{First preliminary ruling }

\scrule{‘If a monk bathes at intervals of less than a half-month, he commits an offense entailing confession.’” }

In\marginnote{1.18} this way the Buddha laid down this training rule for the monks. 

\subsubsection*{Second sub-story }

Soon\marginnote{2.1} afterwards, because they were afraid of wrongdoing, the monks did not bathe when it was hot or when they had a fever, and they went to sleep covered in sweat. As a consequence, their robes and beds got dirty. They told the Buddha. Soon afterwards he gave a teaching and addressed the monks: 

\scrule{“Monks, if it’s hot or you have a fever, I allow you to bathe at intervals of less than a half-month. }

And\marginnote{2.5} so, monks, this training rule should be recited like this: 

\subsubsection*{Second preliminary ruling }

\scrule{‘If a monk bathes at intervals of less than a half-month, except on an appropriate occasion, he commits an offense entailing confession. This is the appropriate occasion: it is the two-and-a-half-month period of summer and the fever season, comprising the last one-and-a-half months of summer and the first month of the rainy season.’” }

In\marginnote{2.8} this way the Buddha laid down this training rule for the monks. 

\subsubsection*{Third sub-story }

Soon\marginnote{3.1} afterwards some monks were sick. The monks who were looking after them asked,  “I hope you’re bearing up? I hope you’re getting better?” 

“Previously\marginnote{3.4} we bathed at intervals of less than a half-month, and then we were comfortable. But now that the Buddha has prohibited this, we don’t bathe because we’re afraid of wrongdoing. Because of that we’re not comfortable.” 

They\marginnote{3.6} told the Buddha. … 

\scrule{“Monks, I allow a sick monk to bathe at intervals of less than a half-month. }

And\marginnote{3.8} so, monks, this training rule should be recited like this: 

\subsubsection*{Third preliminary ruling }

\scrule{‘If a monk bathes at intervals of less than a half-month, except on an appropriate occasion, he commits an offense entailing confession. These are the appropriate occasions: it is the two-and-a-half-month period of summer and the fever season, comprising the last one-and-a-half months of summer and the first month of the rainy season; he is sick.’” }

In\marginnote{3.11} this way the Buddha laid down this training rule for the monks. 

\subsubsection*{Fourth sub-story }

Soon\marginnote{4.1} afterwards the monks were doing building work, but because they were afraid of wrongdoing they did not bathe. As a consequence, they went to sleep covered in sweat, and their robes and beds got dirty. They told the Buddha. … 

\scrule{“Monks, I allow you to bathe at intervals of less than a half-month when you’re working. }

And\marginnote{4.5} so, monks, this training rule should be recited like this: 

\subsubsection*{Fourth preliminary ruling }

\scrule{‘If a monk bathes at intervals of less than a half-month, except on an appropriate occasion, he commits an offense entailing confession. These are the appropriate occasions: it is the two-and-a-half-month period of summer and the fever season, comprising the last one-and-a-half months of summer and the first month of the rainy season; he is sick; he is working.’” }

In\marginnote{4.8} this way the Buddha laid down this training rule for the monks. 

\subsubsection*{Fifth sub-story }

Soon\marginnote{5.1} afterwards the monks were traveling, but because they were afraid of wrongdoing they did not bathe. As a consequence, they went to sleep covered in sweat, and their robes and beds got dirty. They told the Buddha. … 

\scrule{“Monks, I allow you to bathe at intervals of less than a half-month when you’re traveling. }

And\marginnote{5.5} so, monks, this training rule should be recited like this: 

\subsubsection*{Fifth preliminary ruling }

\scrule{‘If a monk bathes at intervals of less than a half-month, except on an appropriate occasion, he commits an offense entailing confession. These are the appropriate occasions: it is the two-and-a-half-month period of summer and the fever season, comprising the last one-and-a-half months of summer and the first month of the rainy season; he is sick; he is working; he is traveling.’” }

In\marginnote{5.8} this way the Buddha laid down this training rule for the monks. 

\subsubsection*{Sixth sub-story }

Soon\marginnote{6.1} afterwards a number of monks were making robes out in the open, when they were hit by dusty winds and fine rain. But because they were afraid of wrongdoing, they did not bathe afterwards, and they went to sleep while still wet. As a consequence, their robes and beds got dirty. They told the Buddha. … 

\scrule{“Monks, if there is wind and rain, I allow you to bathe at intervals of less than a half-month. }

And\marginnote{6.6} so, monks, this training rule should be recited like this: 

\subsection*{Final ruling }

\scrule{‘If a monk bathes at intervals of less than a half-month, except on an appropriate occasion, he commits an offense entailing confession. These are the appropriate occasions: it is the two-and-a-half-month period of summer and the fever season, comprising the last one-and-a-half months of summer and the first month of the rainy season; he is sick; he is working; he is traveling; there is wind and rain.’” }

\subsection*{Definitions }

\begin{description}%
\item[A: ] whoever … %
\item[Monk: ] … The monk who has been given the full ordination by a unanimous Sangha through a legal procedure consisting of one motion and three announcements that is irreversible and fit to stand—this sort of monk is meant in this case. %
\item[At intervals of less than a half-month: ] after less than a half-month. %
\item[Bathes: ] if he bathes with bath powder or soap, then for every effort there is an act of wrong conduct.\footnote{For an explanation of the renderings “bath powder” and “soap” for \textit{\textsanskrit{cuṇṇa}} and \textit{mattika} respectively, see Appendix of Technical Terms. } When the bath is finished, he commits an offense entailing confession. %
\item[Except on an appropriate occasion: ] unless it is an appropriate occasion. %
\item[Summer: ] the last month-and-a-half of summer. %
\item[The fever season: ] the first month of the rainy season. During the two-and-a-half-month period of summer and the fever season, he may bathe. %
\item[He is sick: ] he is not comfortable without bathing. If he is sick, he may bathe. %
\item[He is working: ] even if he just sweeps the yard of a building. If he is working, he may bathe. %
\item[He is traveling: ] if he intends to travel six kilometers, he may bathe; while traveling, he may bathe; after he has traveled, he may bathe.\footnote{For a discussion of the \textit{yojana}, see \textit{sugata} in Appendix of Technical Terms. } %
\item[There is wind and rain: ] monks are hit by dusty winds, and two or three drops of rain fall on their bodies. If there is wind and rain, they may bathe. %
\end{description}

\subsection*{Permutations }

If\marginnote{7.2.1} it is an interval of less than a half-month, and he perceives it as less, and he bathes, except on an appropriate occasion, he commits an offense entailing confession. If it is an interval of less than a half-month, but he is unsure of it, and he bathes, except on an appropriate occasion, he commits an offense entailing confession. If it is an interval of less than a half-month, but he perceives it as more, and he bathes, except on an appropriate occasion, he commits an offense entailing confession. 

If\marginnote{7.2.4} it is an interval of more than a half-month, but he perceives it as less, he commits an offense of wrong conduct. If it is an interval of more than a half-month, but he is unsure of it, he commits an offense of wrong conduct. If it is an interval of more than a half-month, and he perceives it as more, there is no offense. 

\subsection*{Non-offenses }

There\marginnote{7.3.1} is no offense: if it is an appropriate occasion;  if he bathes at intervals of a half-month;  if he bathes at intervals of more than a half-month;  if he bathes while crossing a body of water;  if he is outside the central Ganges plain;  if there is an emergency;  if he is insane;  if he is the first offender. 

\scendsutta{The training rule on bathing, the seventh is finished. }

%
\section*{{\suttatitleacronym Bu Pc 58}{\suttatitletranslation 58. The training rule on making stains }{\suttatitleroot Dubbaṇṇakaraṇa}}
\addcontentsline{toc}{section}{\tocacronym{Bu Pc 58} \toctranslation{58. The training rule on making stains } \tocroot{Dubbaṇṇakaraṇa}}
\markboth{58. The training rule on making stains }{Dubbaṇṇakaraṇa}
\extramarks{Bu Pc 58}{Bu Pc 58}

\subsection*{Origin story }

At\marginnote{1.1} one time when the Buddha was staying at \textsanskrit{Sāvatthī} in \textsanskrit{Anāthapiṇḍika}’s Monastery, a number of monks and wanderers were robbed while traveling from \textsanskrit{Sāketa} to \textsanskrit{Sāvatthī}. The king’s men set out from \textsanskrit{Sāvatthī} and caught the thieves and their loot. They then sent a message to the monks: “Venerables, please come and pick out your own robes.” But the monks did not recognize them. The people complained and criticized them, “How can they not recognize their own robes?” 

The\marginnote{1.8} monks heard the complaints of those people, and they told the Buddha. He had the Sangha gathered, gave a teaching on what is right and proper, and then addressed the monks: “Well then, monks, I will lay down a training rule for the following ten reasons: for the well-being of the Sangha, for the comfort of the Sangha, for the restraint of bad people, for the ease of good monks, for the restraint of the corruptions relating to the present life, for the restraint of the corruptions relating to future lives, to give rise to confidence in those without it, to increase the confidence of those who have it, for the longevity of the true Teaching, and for supporting the training. And, monks, this training rule should be recited like this: 

\subsection*{Final ruling }

\scrule{‘When a monk gets a new robe, he should apply one of three kinds of stains: blue-green, mud-color, or dark brown. If a monk uses a new robe without applying any of the three kinds of stains, he commits an offense entailing confession.’” }

\subsection*{Definitions }

\begin{description}%
\item[New: ] a mark has not been made is what is meant. %
\item[A robe: ] one of the six kinds of robes.\footnote{The six are linen, cotton, silk, wool, sunn hemp, and hemp; see \href{https://suttacentral.net/pli-tv-kd8/en/brahmali\#3.1.6}{Kd 8:3.1.6}. } %
\item[He should apply one of three kinds of stains: ] even if he just applies what fits on the tip of a blade of grass. %
\item[Blue-green: ] there are two kinds of blue-green: the color of copper sulfate and the color of leaves. %
\item[Mud-color: ] watery is what is meant.\footnote{According to Vin-vn-\textsanskrit{ṭ} 1626 (presumably commenting on \textit{kaddamena} in the \textsanskrit{Kaṅkhāvitaraṇī} commentary): \textit{\textsanskrit{Kaddamenāti} \textsanskrit{udakānukaddamasukkhakaddamādiṁ} \textsanskrit{saṅgaṇhāti}}, “With mud means: water with mud, dry mud, etc., is included.” } %
\item[Dark brown: ] whatever is dark brownish. %
\item[If a monk … without applying any of the three kinds of stains: ] if he uses a new robe without first applying one of the three kinds of stains, even just the amount on the tip of a blade of grass, he commits an offense entailing confession. %
\end{description}

\subsection*{Permutations }

If\marginnote{2.2.1} it has not been applied, and he perceives that it has not, and he uses the robe, he commits an offense entailing confession. If it has not been applied, but he is unsure of it, and he uses the robe, he commits an offense entailing confession. If it has not been applied, but he perceives that it has, and he uses the robe, he commits an offense entailing confession. 

If\marginnote{2.2.4} it has been applied, but he perceives that it has not, he commits an offense of wrong conduct. If it has been applied, but he is unsure of it, he commits an offense of wrong conduct. If it has been applied, and he perceives that it has, there is no offense. 

\subsection*{Non-offenses }

There\marginnote{2.3.1} is no offense: if he applies the stain and then uses it;  if the mark has disappeared;  if the area where the mark was applied is worn;  if what had been marked is sewn together with what has not been marked;  if it is a patch;  if it is a lengthwise border;\footnote{I understand \textit{\textsanskrit{anuvāta}} and \textit{\textsanskrit{paribhaṇḍa}} to refer to long borders and short borders respectively. }  if it is a crosswise border;  if he is insane;  if he is the first offender. 

\scendsutta{The training rule on making stains, the eighth, is finished. }

%
\section*{{\suttatitleacronym Bu Pc 59}{\suttatitletranslation 59. The training rule on assigning ownership to another }{\suttatitleroot Vikappana}}
\addcontentsline{toc}{section}{\tocacronym{Bu Pc 59} \toctranslation{59. The training rule on assigning ownership to another } \tocroot{Vikappana}}
\markboth{59. The training rule on assigning ownership to another }{Vikappana}
\extramarks{Bu Pc 59}{Bu Pc 59}

\subsection*{Origin story }

At\marginnote{1.1} one time when the Buddha was staying at \textsanskrit{Sāvatthī} in \textsanskrit{Anāthapiṇḍika}’s Monastery, Venerable Upananda the Sakyan assigned the ownership of a robe to a monk who was his brother’s student. He then used that robe without that monk having relinquished it. That monk told the monks, “Venerable Upananda is using a robe that he had assigned to me, even though I haven’t relinquished it.” 

The\marginnote{1.5} monks of few desires complained and criticized Upananda, “How could Venerable Upananda use a robe he had assigned to a monk, without that monk first relinquishing it?” … “Is it true, Upananda, that you did this?” 

“It’s\marginnote{1.8} true, sir.” 

The\marginnote{1.9} Buddha rebuked him … “Foolish man, how could you do this? This will affect people’s confidence …” … “And, monks, this training rule should be recited like this: 

\subsection*{Final ruling }

\scrule{‘If a monk himself assigns the ownership of a robe to a monk, a nun, a trainee nun, a novice monk, or a novice nun, and he then uses it without the other first relinquishing it, he commits an offense entailing confession.’” }

\subsection*{Definitions }

\begin{description}%
\item[A: ] whoever … %
\item[Monk: ] … The monk who has been given the full ordination by a unanimous Sangha through a legal procedure consisting of one motion and three announcements that is irreversible and fit to stand—this sort of monk is meant in this case. %
\item[To a monk: ] to another monk. %
\item[A nun: ] she has been given the full ordination by both Sanghas. %
\item[A trainee nun: ] one training for two years in the six rules. %
\item[A novice monk: ] a male training in the ten training rules. %
\item[A novice nun: ] a female training in the ten training rules. %
\item[Himself: ] having himself done the assignment. %
\item[A robe: ] one of the six kinds of robe-cloth, but not smaller than what can be assigned to another.\footnote{The six are linen, cotton, silk, wool, sunn hemp, and hemp; see \href{https://suttacentral.net/pli-tv-kd8/en/brahmali\#3.1.6}{Kd 8:3.1.6}. According to \href{https://suttacentral.net/pli-tv-kd8/en/brahmali\#21.1.4}{Kd 8:21.1.4} the size referred to is no smaller than 8 by 4 \textit{\textsanskrit{sugataṅgula}}, “standard fingerbreadths”. } %
\item[Assigns the ownership of: ] there are two kinds of assignment: assignment in the presence of and assignment in the absence of.\footnote{For an explanation of the idea of \textit{\textsanskrit{vikappanā}}, see Appendix of Technical Terms. } %
\item[Assignment in the presence of: ] one should say, “I assign this robe-cloth to you,” or “I assign this robe-cloth to so-and-so.” %
\item[Assignment in the absence of: ] one should say, “I give this robe-cloth to you for the purpose of assigning it.” The other should ask, “Who is your friend or companion?” One should reply, “So-and-so and so-and-so.” The other should say, “I give it to them. Please use their property, give it away, or do as you like with it.”\footnote{This seems to be a reference to \textit{\textsanskrit{vissāsa}}, the idea of “taking on trust”. If you have an agreement with a close friend, you may take their belongings on trust. In this particular case, you assign the robe-cloth to such a friend, which then enables you to take the robe-cloth back should you need it. (The place where the robe-cloth is stored seems to be irrelevant.) This explains why the other monk says, “Please use their property, give it away, or do as you like with it.” The conditions for taking on trust are set out at \href{https://suttacentral.net/pli-tv-kd8/en/brahmali\#19.1.5}{Kd 8:19.1.5}. } %
\item[Without it first being relinquished: ] if it is not given to him or he uses it without taking it on trust, he commits an offense entailing confession. %
\end{description}

\subsection*{Permutations }

If\marginnote{2.2.1} it has not been relinquished, and he perceives that it has not, and he uses it, he commits an offense entailing confession. If it has not been relinquished, but he is unsure of it, and he uses it, he commits an offense entailing confession. If it has not been relinquished, but he perceives that it has, and he uses it, he commits an offense entailing confession.\footnote{The Pali here reads \textit{\textsanskrit{appaccuddhāraṇasaññī}} , which must be an editing mistake. I instead follow the PTS reading, \textit{\textsanskrit{paccuddhāraṇasaññī}}. } 

If\marginnote{2.2.4} he determines it or gives it away, he commits an offense of wrong conduct. If it has been relinquished, but he perceives that it has not, he commits an offense of wrong conduct. If it has been relinquished, but he is unsure of it, he commits an offense of wrong conduct. If it has been relinquished, and he perceives that it has, there is no offense. 

\subsection*{Non-offenses }

There\marginnote{2.3.1} is no offense: if the other person gives it or he uses it after taking the other person’s property on trust;   if he is insane;  if he is the first offender. 

\scendsutta{The training rule on assigning ownership to another, the ninth, is finished. }

%
\section*{{\suttatitleacronym Bu Pc 60}{\suttatitletranslation 60. The training rule on hiding robes }{\suttatitleroot Cīvaraapanidhāna}}
\addcontentsline{toc}{section}{\tocacronym{Bu Pc 60} \toctranslation{60. The training rule on hiding robes } \tocroot{Cīvaraapanidhāna}}
\markboth{60. The training rule on hiding robes }{Cīvaraapanidhāna}
\extramarks{Bu Pc 60}{Bu Pc 60}

\subsection*{Origin story }

At\marginnote{1.1} one time the Buddha was staying at \textsanskrit{Sāvatthī} in the Jeta Grove, \textsanskrit{Anāthapiṇḍika}’s Monastery. On one occasion when the monks from the group of seventeen had not put away their requisites, the monks from the group of six hid their bowls and robes. The monks from the group of seventeen said to them, “Give us our bowls and robes.” The monks from the group of six laughed, but the monks from the group of seventeen cried. 

The\marginnote{1.7} monks asked them, “Why are you crying?” 

“’Cause\marginnote{1.9} the monks from the group of six have hidden our bowls and robes.” 

The\marginnote{1.10} monks of few desires complained and criticized them, “How could the monks from the group of six hide the bowls and robes of other monks?” … “Is it true, monks, that you did this?” 

“It’s\marginnote{1.13} true, sir.” 

The\marginnote{1.14} Buddha rebuked them … “Foolish men, how could you do this? This will affect people’s confidence …” … “And, monks, this training rule should be recited like this: 

\subsection*{Final ruling }

\scrule{‘If a monk hides a monk’s bowl, robe, sitting mat, needle case, or belt, or he has it hidden, even just for a laugh, he commits an offense entailing confession.’” }

\subsection*{Definitions }

\begin{description}%
\item[A: ] whoever … %
\item[Monk: ] … The monk who has been given the full ordination by a unanimous Sangha through a legal procedure consisting of one motion and three announcements that is irreversible and fit to stand—this sort of monk is meant in this case. %
\item[A monk’s: ] another monk’s. %
\item[Bowl: ] there are two kinds of bowls: iron bowls and ceramic bowls. %
\item[Robe: ] one of the six kinds of robe-cloth, but not smaller than what can be assigned to another.\footnote{The six are linen, cotton, silk, wool, sunn hemp, and hemp; see \href{https://suttacentral.net/pli-tv-kd8/en/brahmali\#3.1.6}{Kd 8:3.1.6}. According to \href{https://suttacentral.net/pli-tv-kd8/en/brahmali\#21.1.4}{Kd 8:21.1.4} the size referred to is no smaller than 8 by 4 \textit{\textsanskrit{sugataṅgula}}, “standard fingerbreadths”. For an explanation of the idea of \textit{\textsanskrit{vikappanā}}, see Appendix of Technical Terms. } %
\item[Sitting mat: ] one with a border is what is meant. %
\item[Needle case: ] with or without needles. %
\item[Belt: ] there are two kinds of belts: those made from strips of cloth and those made from pigs’ intestines. %
\item[Hides: ] if he hides it himself, he commits an offense entailing confession. %
\item[Has hidden: ] if he asks another, he commits an offense entailing confession. If he only asks once, then even if the other hides many things, he commits one offense entailing confession. %
\item[Even just for a laugh: ] aiming to have fun. %
\end{description}

\subsection*{Permutations }

If\marginnote{2.2.1} the other monk is fully ordained, and he perceives him as such, and he hides his bowl or robe or sitting mat or needle case or belt, or he has it hidden, even just for a laugh, he commits an offense entailing confession. If the other monk is fully ordained, but he is unsure of it, and he hides his bowl or robe or sitting mat or needle case or belt, or he has it hidden, even just for a laugh, he commits an offense entailing confession. If the other monk is fully ordained, but he does not perceive him as such, and he hides his bowl or robe or sitting mat or needle case or belt, or he has it hidden, even just for a laugh, he commits an offense entailing confession. 

If\marginnote{2.2.4} he hides another requisite, or he has it hidden, even just for a laugh, he commits an offense of wrong conduct. If he hides the bowl or robe or other requisite of someone who is not fully ordained, or he has it hidden, even just for a laugh, he commits an offense of wrong conduct. 

If\marginnote{2.2.6} the other person is not fully ordained, but he perceives them as such, he commits an offense of wrong conduct. If the other person is not fully ordained, but he is unsure of it, he commits an offense of wrong conduct. If the other person is not fully ordained, and he does not perceive them as such, he commits an offense of wrong conduct. 

\subsection*{Non-offenses }

There\marginnote{2.3.1} is no offense: if he is not aiming to have fun;  if he puts away what has been improperly put away;  if he puts something away with the thought, “After giving a teaching, I’ll give it back;”  if he is insane;  if he is the first offender. 

\scendsutta{The training rule on hiding robes, the tenth, is finished. }

\scendvagga{The sixth subchapter on drinking alcohol is finished. }

\scuddanaintro{This is the summary: }

\begin{scuddana}%
“Alcohol,\marginnote{2.3.10} finger, and laughter, \\
And disrespect, scaring; \\
Fire, bathing, stain, \\
Himself, and with hiding.” 

%
\end{scuddana}

%
\section*{{\suttatitleacronym Bu Pc 61}{\suttatitletranslation 61. The training rule on intentionally }{\suttatitleroot Sañcicca}}
\addcontentsline{toc}{section}{\tocacronym{Bu Pc 61} \toctranslation{61. The training rule on intentionally } \tocroot{Sañcicca}}
\markboth{61. The training rule on intentionally }{Sañcicca}
\extramarks{Bu Pc 61}{Bu Pc 61}

\subsection*{Origin story }

At\marginnote{1.1} one time the Buddha was staying at \textsanskrit{Sāvatthī} in the Jeta Grove, \textsanskrit{Anāthapiṇḍika}’s Monastery. At that time Venerable \textsanskrit{Udāyī} was skilled in archery. And because he disliked crows, he shot them. He cut off their heads and then set them out in a row impaled on stakes. The monks asked him, “Who killed these crows?” 

“I\marginnote{1.6} did. I don’t like crows.” 

The\marginnote{1.8} monks of few desires complained and criticized him, “How can Venerable \textsanskrit{Udāyī} intentionally kill living beings?” … “Is it true, \textsanskrit{Udāyī}, that you do this?” 

“It’s\marginnote{1.11} true, sir.” 

The\marginnote{1.12} Buddha rebuked him … “Foolish man, how can you do this? This will affect people’s confidence …” … “And, monks, this training rule should be recited like this: 

\subsection*{Final ruling }

\scrule{‘If a monk intentionally kills a living being, he commits an offense entailing confession.’” }

\subsection*{Definitions }

\begin{description}%
\item[A: ] whoever … %
\item[Monk: ] … The monk who has been given the full ordination by a unanimous Sangha through a legal procedure consisting of one motion and three announcements that is irreversible and fit to stand—this sort of monk is meant in this case. %
\item[Intentionally: ] knowing, perceiving, having intended, having decided, he transgresses. %
\item[A living being: ] an animal is what is meant. %
\item[Kills: ] if he cuts off and makes an end of the life faculty, if he destroys its continuity, he commits an offense entailing confession. %
\end{description}

\subsection*{Permutations }

If\marginnote{2.2.1} it is a living being, and he perceives it as such, and he kills it, he commits an offense entailing confession. If it is a living being, but he is unsure of it, and he kills it, he commits an offense of wrong conduct. If it is a living being, but he does not perceive it as such, and he kills it, there is no offense. 

If\marginnote{2.2.4} it is not a living being, but he perceives it as such, he commits an offense of wrong conduct. If it is not a living being, but he is unsure of it, he commits an offense of wrong conduct. If it is not a living being, and he does not perceive it as such, there is no offense. 

\subsection*{Non-offenses }

There\marginnote{2.3.1} is no offense: if it is unintentional;  if he is not mindful;  if he does not know;  if he is not aiming at death;  if he is insane;  if he is the first offender. 

\scendsutta{The training rule on intentionally, the first, is finished. }

%
\section*{{\suttatitleacronym Bu Pc 62}{\suttatitletranslation 62. The training rule on containing living beings }{\suttatitleroot Sappāṇaka}}
\addcontentsline{toc}{section}{\tocacronym{Bu Pc 62} \toctranslation{62. The training rule on containing living beings } \tocroot{Sappāṇaka}}
\markboth{62. The training rule on containing living beings }{Sappāṇaka}
\extramarks{Bu Pc 62}{Bu Pc 62}

\subsection*{Origin story }

At\marginnote{1.1} one time when the Buddha was staying at \textsanskrit{Sāvatthī} in \textsanskrit{Anāthapiṇḍika}’s Monastery, the monks from the group of six were using water that they knew contained living beings. 

The\marginnote{1.3} monks of few desires complained and criticized them, “How can the monks from the group of six use water that they know contains living beings?” … “Is it true, monks, that you do this?” 

“It’s\marginnote{1.6} true, sir.” 

The\marginnote{1.7} Buddha rebuked them … “Foolish men, how can you do this? This will affect people’s confidence …” … “And, monks, this training rule should be recited like this: 

\subsection*{Final ruling }

\scrule{‘If a monk uses water that he knows contains living beings, he commits an offense entailing confession.’” }

\subsection*{Definitions }

\begin{description}%
\item[A: ] whoever … %
\item[Monk: ] … The monk who has been given the full ordination by a unanimous Sangha through a legal procedure consisting of one motion and three announcements that is irreversible and fit to stand—this sort of monk is meant in this case. %
\item[He knows: ] he knows by himself or others have told him. If he uses it, knowing that it contains living beings and knowing that they will die if the water is used, he commits an offense entailing confession. %
\end{description}

\subsection*{Permutations }

If\marginnote{2.2.1} it contains living beings, and he perceives it as such, and he uses it, he commits an offense entailing confession. If it contains living beings, but he is unsure of it, and he uses it, he commits an offense of wrong conduct. If it contains living beings, but he does not perceive it as such, and he uses it, there is no offense. 

If\marginnote{2.2.4} it does not contain living beings, but he perceives it as such, he commits an offense of wrong conduct. If it does not contain living beings, but he is unsure of it, he commits an offense of wrong conduct. If it does not contain living beings, and he does not perceive it as such, there is no offense. 

\subsection*{Non-offenses }

There\marginnote{2.3.1} is no offense: if he does not know that it contains living beings;  if he knows that it does not contain living beings;  if he uses it knowing that they will not die;  if he is insane;  if he is the first offender. 

\scendsutta{The training rule on containing living beings, the second, is finished. }

%
\section*{{\suttatitleacronym Bu Pc 63}{\suttatitletranslation 63. The training rule on reopening }{\suttatitleroot Ukkoṭana}}
\addcontentsline{toc}{section}{\tocacronym{Bu Pc 63} \toctranslation{63. The training rule on reopening } \tocroot{Ukkoṭana}}
\markboth{63. The training rule on reopening }{Ukkoṭana}
\extramarks{Bu Pc 63}{Bu Pc 63}

\subsection*{Origin story }

At\marginnote{1.1} one time the Buddha was staying at \textsanskrit{Sāvatthī} in the Jeta Grove, \textsanskrit{Anāthapiṇḍika}’s Monastery. At that time the monks from the group of six were reopening a legal issue that they knew had been legitimately settled, saying, “The legal procedure hasn’t been done;” “It’s been done badly;” “It should be done again;” “It hasn’t been settled;” “It’s been badly settled;” “It should be settled again.”\footnote{The Pali appears as a single sentence enclosed in quotes. Yet it is hard to see how this could have been spoken in one go, since some of the constituent clauses are mutually contradictory. For instance, the same legal procedure cannot have been both not done and badly done. I therefore take it that the individual clauses were spoken at different times. This method of putting together a number of independent utterances into a single quote is in fact quite common in the Vinaya \textsanskrit{Piṭaka}. In most such cases I separate out the individual quotes. } 

The\marginnote{1.4} monks of few desires complained and criticized them, “How can the monks from the group of six reopen a legal issue that they know has been legitimately settled?” … “Is it true, monks, that you’re doing this?” 

“It’s\marginnote{1.7} true, sir.” 

The\marginnote{1.8} Buddha rebuked them … “Foolish men, how can you do this? This will affect people’s confidence …” … “And, monks, this training rule should be recited like this: 

\subsection*{Final ruling }

\scrule{‘If a monk reopens a legal issue that he knows has been legitimately settled, he commits an offense entailing confession.’” }

\subsection*{Definitions }

\begin{description}%
\item[A: ] whoever … %
\item[Monk: ] … The monk who has been given the full ordination by a unanimous Sangha through a legal procedure consisting of one motion and three announcements that is irreversible and fit to stand—this sort of monk is meant in this case. %
\item[He knows: ] he knows by himself or others have told him or the Sangha has told him.\footnote{The meaning of the last of these three ways of knowing, \textit{so \textsanskrit{vā} \textsanskrit{āroceti}}, is not clear. CPD suggests: “\textit{sa (\textsanskrit{sā}) \textsanskrit{āroceti}} (?). Perhaps this last form is conformable to sa. \textit{\textsanskrit{ārocayate}} med. caus. in the meaning: he or she makes inquiries (of others).” However, this does not fit with the parallel usage at \href{https://suttacentral.net/pli-tv-bu-vb-pc29/en/brahmali\#3.1.6}{Bu Pc 29:3.1.6} where the text says that she tells (\textit{\textsanskrit{sā} \textsanskrit{vā} \textsanskrit{āroceti}}) him, presumably referring to the nun telling the monk. In this case \textit{\textsanskrit{āroceti}} cannot refer to the monk making inquiries. The commentaries are silent, and I therefore assume that a straightforward meaning is the most likely one. I would suggest, then, that it refers to the Sangha, or perhaps one of the monks who participated in the \textit{\textsanskrit{saṅghakamma}}, having told him directly. } %
\item[Legitimately: ] done according to the Teaching, according to the Monastic Law, according to the Teacher’s instruction—this is called “legitimately”. %
\item[A legal issue: ] there are four kinds of legal issues: legal issues arising from disputes, legal issues arising from accusations, legal issues arising from offenses, legal issues arising from business. %
\item[Reopens: ] if he reopens it, saying, “The legal procedure hasn’t been done;” “It’s been done badly;” “It should be done again;” “It hasn’t been settled;” “It’s been badly settled;” “It should be settled again,” he commits an offense entailing confession. %
\end{description}

\subsection*{Permutations }

If\marginnote{2.2.1} it is a legitimate legal procedure, and he perceives it as such, and he reopens it, he commits an offense entailing confession. If it is a legitimate legal procedure, but he is unsure of it, and he reopens it, he commits an offense of wrong conduct. If it is a legitimate legal procedure, but he perceives it as illegitimate, and he reopens it, there is no offense. 

If\marginnote{2.2.4} it is an illegitimate legal procedure, but he perceives it as legitimate, he commits an offense of wrong conduct. If it is an illegitimate legal procedure, but he is unsure of it, he commits an offense of wrong conduct. If it is an illegitimate legal procedure, and he perceives it as such, there is no offense. 

\subsection*{Non-offenses }

There\marginnote{2.3.1} is no offense: if he reopens it because he knows that the legal procedure was illegitimate, done by an incomplete assembly, or done against one who did not deserve it;  if he is insane;  if he is the first offender. 

\scendsutta{The training rule on reopening, the third, is finished. }

%
\section*{{\suttatitleacronym Bu Pc 64}{\suttatitletranslation 64. The training rule on what is grave }{\suttatitleroot Duṭṭhulla}}
\addcontentsline{toc}{section}{\tocacronym{Bu Pc 64} \toctranslation{64. The training rule on what is grave } \tocroot{Duṭṭhulla}}
\markboth{64. The training rule on what is grave }{Duṭṭhulla}
\extramarks{Bu Pc 64}{Bu Pc 64}

\subsection*{Origin story }

At\marginnote{1.1} one time when the Buddha was staying at \textsanskrit{Sāvatthī} in \textsanskrit{Anāthapiṇḍika}’s Monastery, Venerable Upananda the Sakyan had committed an offense of intentional emission of semen. He told his brother’s student about this, adding, “Don’t tell anyone.” 

Soon\marginnote{1.5} afterwards another monk also committed an offense of intentional emission of semen. He asked the Sangha for probation, which he got. While he was on probation, he saw Upananda’s brother’s student and said to him, “I’ve committed an offense of intentional emission of semen. I asked the Sangha for probation, which I got. I’m now undergoing probation. Please remember me as such.” 

“Do\marginnote{1.9} others who have committed this offense need to do the same?” 

“Yes.”\marginnote{1.10} 

“Venerable\marginnote{1.11} Upananda committed this offense and told me not to tell anyone.” 

“So\marginnote{1.12} did you conceal it?” 

“Yes.”\marginnote{1.13} 

That\marginnote{1.14} monk then told other monks. The monks of few desires complained and criticized him, “How could a monk knowingly conceal a monk’s grave offense?” … “Is it true, monk, that you did this?” 

“It’s\marginnote{1.18} true, sir.” 

The\marginnote{1.19} Buddha rebuked him … “Foolish man, how could you do this? This will affect people’s confidence …” … “And, monks, this training rule should be recited like this: 

\subsection*{Final ruling }

\scrule{‘If a monk knowingly conceals a monk’s grave offense, he commits an offense entailing confession.’” }

\subsection*{Definitions }

\begin{description}%
\item[A: ] whoever … %
\item[Monk: ] … The monk who has been given the full ordination by a unanimous Sangha through a legal procedure consisting of one motion and three announcements that is irreversible and fit to stand—this sort of monk is meant in this case. %
\item[A monk’s: ] another monk’s. %
\item[Knowingly: ] he knows by himself or others have told him or the offender has told him.\footnote{The meaning of the last of these three ways of knowing, \textit{so \textsanskrit{vā} \textsanskrit{āroceti}}, is not clear. CPD suggests: “\textit{sa (\textsanskrit{sā}) \textsanskrit{āroceti}} (?). Perhaps this last form is conformable to sa. \textit{\textsanskrit{ārocayate}} med. caus. in the meaning: he or she makes inquiries (of others).” However, this does not fit with the parallel usage at \href{https://suttacentral.net/pli-tv-bu-vb-pc29/en/brahmali\#3.1.6}{Bu Pc 29:3.1.6} where the text says that she tells (\textit{\textsanskrit{sā} \textsanskrit{vā} \textsanskrit{āroceti}}) him, presumably referring to the nun telling the monk. In this case \textit{\textsanskrit{āroceti}} cannot refer to the monk making inquiries. The commentaries are silent, and I therefore assume that a straightforward meaning is the most likely one. I would suggest, then, that it simply refers to the other monk telling him directly. } %
\item[Grave offense: ] the four offenses entailing expulsion and the thirteen entailing suspension. %
\item[Conceals: ] thinking, “If they find out about this, they’ll accuse him, remind him, scold him, censure him, humiliate him; I won’t tell,” then by the mere fact of abandoning his duty, he commits an offense entailing confession. %
\end{description}

\subsection*{Permutations }

If\marginnote{2.2.1} it is a grave offense, and he perceives it as such, and he conceals it, he commits an offense entailing confession. If it is a grave offense, but he is unsure of it, and he conceals it, he commits an offense of wrong conduct. If it is a grave offense, but he perceives it as minor, and he conceals it, he commits an offense of wrong conduct. 

If\marginnote{2.2.4} he conceals a minor offense, he commits an offense of wrong conduct. If he conceals the grave or minor misconduct of one who is not fully ordained, he commits an offense of wrong conduct. 

If\marginnote{2.2.6} it is a minor offense, but he perceives it as grave, he commits an offense of wrong conduct. 

If\marginnote{2.2.7} it is a minor offense, but he is unsure of it, he commits an offense of wrong conduct. If it is a minor offense, and he perceives it as such, he commits an offense of wrong conduct. 

\subsection*{Non-offenses }

There\marginnote{2.3.1} is no offense: if he does not tell because he thinks there will be quarrels or disputes in the Sangha;  if he does not tell because he thinks there will be a fracture or schism in the Sangha;  if he does not tell because he thinks the person he is telling about is cruel and harsh and that he might become a threat to life or to the monastic life;  if he does not tell because he does not see any suitable monks;  if he does not tell, but not because he wants to conceal;  if he does not tell because he thinks the other person will be known through his own actions;  if he is insane;  if he is the first offender. 

\scendsutta{The training rule on what is grave, the fourth, is finished. }

%
\section*{{\suttatitleacronym Bu Pc 65}{\suttatitletranslation 65. The training rule on less than twenty years old }{\suttatitleroot Ūnavīsativassa}}
\addcontentsline{toc}{section}{\tocacronym{Bu Pc 65} \toctranslation{65. The training rule on less than twenty years old } \tocroot{Ūnavīsativassa}}
\markboth{65. The training rule on less than twenty years old }{Ūnavīsativassa}
\extramarks{Bu Pc 65}{Bu Pc 65}

\subsection*{Origin story }

At\marginnote{1.1} one time the Buddha was staying at \textsanskrit{Rājagaha} in the Bamboo Grove, the squirrel sanctuary. At that time in \textsanskrit{Rājagaha} there was a group of seventeen boys who were friends, with \textsanskrit{Upāli} as their leader. 

\textsanskrit{Upāli}’s\marginnote{1.4} parents considered, “How can we make sure that \textsanskrit{Upāli} is able to live happily without exhausting himself after we’ve passed away? He could become a clerk, but then his fingers will hurt. Or he could become an accountant, but then his chest will hurt. Or he could become a banker, but then his eyes will hurt. But these Sakyan monastics have pleasant habits and a happy life. They eat nice food and sleep in beds sheltered from the wind. If \textsanskrit{Upāli} goes forth with them, he’ll be able to live happily without exhausting himself after we’ve passed away.” 

\textsanskrit{Upāli}\marginnote{1.18} overheard this conversation between his parents. He then went to the other boys and said, “Come, let’s go forth with the Sakyan monastics.” 

“If\marginnote{1.21} you go forth, so will we.” 

The\marginnote{1.22} boys went each to his own parents and said, “Please allow me to go forth into homelessness.” Since the parents knew that the boys all had the same desire and good intention, they gave their approval. The boys then went to the monks and asked for the going forth. And the monks gave them the going forth and the full ordination. 

Soon\marginnote{1.28} afterwards they got up early in the morning and cried, “Give us congee, give us a meal, give us fresh food.” 

The\marginnote{1.30} monks said, “Wait until it gets light. If any of that becomes available then, you can have it. If not, you’ll eat after walking for alms.” 

But\marginnote{1.36} they carried on as before. And they defecated and urinated on the furniture. 

After\marginnote{1.39} rising early in the morning, the Buddha heard the sound of those boys. He asked Venerable Ānanda, who told him what was happening. Soon afterwards the Buddha had the Sangha gathered and questioned the monks: “Is it true, monks, that the monks give the full ordination to people they know are less than twenty years old?” 

“It’s\marginnote{1.45} true, sir.” 

The\marginnote{1.46} Buddha rebuked them … “How can those foolish men do this? A person who is less than twenty years old is unable to endure cold and heat; hunger and thirst; horseflies, mosquitoes, wind, and the burning sun; creeping animals and insects; and rude and unwelcome speech. And they are unable to bear up with bodily feelings that are painful, severe, sharp, and destructive of life.\footnote{Sp-\textsanskrit{ṭ} 4.295: \textit{\textsanskrit{Sarīsapeti} ye keci sarante gacchante \textsanskrit{dīghajātike}}, “\textit{\textsanskrit{Sarīsape}}: whatever long creatures are moving by flowing.” } But a person who’s twenty is able to endure these things. This will affect people’s confidence …” … “And, monks, this training rule should be recited like this: 

\subsection*{Final ruling }

\scrule{‘If a monk gives the full ordination to a person he knows is less than twenty years old, he commits an offense entailing confession. Moreover, that person has not received the full ordination and those monks are blameworthy.’” }

\subsection*{Definitions }

\begin{description}%
\item[A: ] whoever … %
\item[Monk: ] … The monk who has been given the full ordination by a unanimous Sangha through a legal procedure consisting of one motion and three announcements that is irreversible and fit to stand—this sort of monk is meant in this case. %
\item[He knows: ] he knows by himself or others have told him or the candidate has told him.\footnote{The meaning of the last of these three ways of knowing, \textit{so \textsanskrit{vā} \textsanskrit{āroceti}}, is not clear. CPD suggests: “\textit{sa (\textsanskrit{sā}) \textsanskrit{āroceti}} (?). Perhaps this last form is conformable to sa. \textit{\textsanskrit{ārocayate}} med. caus. in the meaning: he or she makes inquiries (of others).” However, this does not fit with the parallel usage at \href{https://suttacentral.net/pli-tv-bu-vb-pc29/en/brahmali\#3.1.6}{Bu Pc 29:3.1.6} where the text says that she tells (\textit{\textsanskrit{sā} \textsanskrit{vā} \textsanskrit{āroceti}}) him, presumably referring to the nun telling the monk. In this case \textit{\textsanskrit{āroceti}} cannot refer to the monk making inquiries. The commentaries are silent, and I therefore assume that a straightforward meaning is the most likely one. I would suggest, then, that it simply refers to the person wrongly ordained telling the monk directly. } %
\item[Less than twenty years old: ] who has not reached twenty years. %
\end{description}

If,\marginnote{2.1.9} intending to give the full ordination, he searches for a group, a teacher, a bowl, or a robe, or he establishes a monastery zone, he commits an offense of wrong conduct.\footnote{“Monastery zone” renders \textit{\textsanskrit{sīmā}}.  See Appendix of Technical Terms for discussion. } After the motion, he commits an offense of wrong conduct.\footnote{The Pali just says \textit{\textsanskrit{dukkaṭa}}, without specifying that it is an \textit{\textsanskrit{āpatti}}, “an offense”. Yet elsewhere, such as at \href{https://suttacentral.net/pli-tv-bu-vb-ss10/en/brahmali\#2.65}{Bu Ss 10:2.65}, the \textit{\textsanskrit{dukkaṭa}} is annulled if you commit the full offense of \textit{\textsanskrit{saṅghādisesa}}. The implication is that in these contexts \textit{\textsanskrit{dukkaṭa}} should be read as \textit{\textsanskrit{āpatti} \textsanskrit{dukkaṭassa}}, “an offense of wrong conduct”. } After each of the first two announcements, he commits an offense of wrong conduct. When the last announcement is finished, the preceptor commits an offense entailing confession, while the group and the teacher commit an offense of wrong conduct. 

\subsection*{Permutations }

If\marginnote{2.2.1} the other person is less than twenty years old, and he perceives them as less, and he gives them the full ordination, he commits an offense entailing confession. If the other person is less than twenty years old, but he is unsure of it, and he gives them the full ordination, he commits an offense of wrong conduct. If the other person is less than twenty years old, but he perceives them as more, and he gives them the full ordination, there is no offense. 

If\marginnote{2.2.4} the other person is more than twenty years old, but he perceives them as less, he commits an offense of wrong conduct. If the other person is more than twenty years old, but he is unsure of it, he commits an offense of wrong conduct. If the other person is more than twenty years old, and he perceives them as more, there is no offense. 

\subsection*{Non-offenses }

There\marginnote{2.3.1} is no offense: if he gives the full ordination to someone less than twenty years old, but he perceives them as more than twenty;  if he gives the full ordination to someone more than twenty years old, and he perceives them as more than twenty;  if he is insane;  if he is the first offender. 

\scendsutta{The training rule on less than twenty years old, the fifth, is finished. }

%
\section*{{\suttatitleacronym Bu Pc 66}{\suttatitletranslation 66. The training rule on a group of traveling thieves }{\suttatitleroot Theyyasattha}}
\addcontentsline{toc}{section}{\tocacronym{Bu Pc 66} \toctranslation{66. The training rule on a group of traveling thieves } \tocroot{Theyyasattha}}
\markboth{66. The training rule on a group of traveling thieves }{Theyyasattha}
\extramarks{Bu Pc 66}{Bu Pc 66}

\subsection*{Origin story }

At\marginnote{1.1} one time when the Buddha was staying at \textsanskrit{Sāvatthī} in \textsanskrit{Anāthapiṇḍika}’s Monastery, a certain group of travelers was about to go south from \textsanskrit{Rājagaha}. A monk said to those people, “Let me travel with you.” 

“But\marginnote{1.5} we’re smuggling goods.” 

“That’s\marginnote{1.6} your business.”\footnote{\textit{\textsanskrit{Pajānāth}’\textsanskrit{āvuso}}, literally, “You know, friends.” According to the commentary to DN-a 2.436: \textit{\textsanskrit{Tvaṁ} \textsanskrit{pajānāhīti} \textsanskrit{tvaṁ} \textsanskrit{jāna}. Sace \textsanskrit{gaṇhitukāmosi}, \textsanskrit{gaṇhāhīti} \textsanskrit{vuttaṁ} hoti}, “\textit{\textsanskrit{Tvaṁ} \textsanskrit{pajānāhi}}: you know. If you wish to take it, please do: this is what is meant.” In other words, it’s up to you to do what you like. } 

The\marginnote{1.7} customs officers heard about that group of travelers. They then blocked the road, caught the group, confiscated the goods, and asked that monk, “Venerable, why are you knowingly traveling with a group of thieves?” And they detained him. 

After\marginnote{1.13} being released, that monk went to \textsanskrit{Sāvatthī}, where he told the monks what had happened. The monks of few desires complained and criticized him, “How could a monk knowingly travel by arrangement with a group of thieves?” … “Is it true, monk, that you did this?” 

“It’s\marginnote{1.17} true, sir.” 

The\marginnote{1.18} Buddha rebuked him … “Foolish man, how could you do this? This will affect people’s confidence …” … “And, monks, this training rule should be recited like this: 

\subsection*{Final ruling }

\scrule{‘If a monk knowingly travels by arrangement with a group of thieves, even just to the next inhabited area, he commits an offense entailing confession.’” }

\subsection*{Definitions }

\begin{description}%
\item[A: ] whoever … %
\item[Monk: ] … The monk who has been given the full ordination by a unanimous Sangha through a legal procedure consisting of one motion and three announcements that is irreversible and fit to stand—this sort of monk is meant in this case. %
\item[Knowingly: ] he knows by himself or others have told him or the group of travelers has told him.\footnote{The meaning of the last of these three ways of knowing, \textit{so \textsanskrit{vā} \textsanskrit{āroceti}}, is not clear. CPD suggests: “\textit{sa (\textsanskrit{sā}) \textsanskrit{āroceti}} (?). Perhaps this last form is conformable to sa. \textit{\textsanskrit{ārocayate}} med. caus. in the meaning: he or she makes inquiries (of others).” However, this does not fit with the parallel usage at \href{https://suttacentral.net/pli-tv-bu-vb-pc29/en/brahmali\#3.1.6}{Bu Pc 29:3.1.6} where the text says that she tells (\textit{\textsanskrit{sā} \textsanskrit{vā} \textsanskrit{āroceti}}) him, presumably referring to the nun telling the monk. In this case \textit{\textsanskrit{āroceti}} cannot refer to the monk making inquiries. The commentaries are silent, and I therefore assume that a straightforward meaning is the most likely one. I would suggest, then, that it refers to the group of travelers (i.e. the thieves or one of them) telling the monk directly. } %
\item[A group of thieves: ] thieves who have done their deed or thieves who have not. They are stealing from the king or smuggling. %
\item[With: ] together. %
\item[By arrangement: ] if he makes an arrangement like this: the monk says, “Let’s go,” and they reply, “Yes, let’s go, venerable;” or they say, “Let’s go, venerable,” and the monk replies, “Yes, let’s go;” or the monk says, “Let’s go today,” “Let’s go tomorrow,” or “Let’s go the day after tomorrow,” then he commits an offense of wrong conduct. %
\item[Even just to the next inhabited area: ] when the inhabited areas are a chicken’s flight apart, then for every next inhabited area he commits an offense entailing confession. When it is an uninhabited area, a wilderness, then for every six kilometers he commits an offense entailing confession.\footnote{For a discussion of the \textit{yojana}, see \textit{sugata} in Appendix of Technical Terms. } %
\end{description}

\subsection*{Permutations }

If\marginnote{2.2.1} it is a group of traveling thieves, and he perceives it as such, and he travels by arrangement with them, even just to the next inhabited area, he commits an offense entailing confession. If it is a group of traveling thieves, but he is unsure of it, and he travels by arrangement with them, even just to the next inhabited area, he commits an offense of wrong conduct. If it is a group of traveling thieves, but he does not perceive it as such, and he travels by arrangement with them, even just to the next inhabited area, there is no offense. 

If\marginnote{2.2.4} the monk makes an arrangement, but the group does not express its agreement, he commits an offense of wrong conduct. If it is not a group of traveling thieves, but he perceives it as such, he commits an offense of wrong conduct. If it is not a group of traveling thieves, but he is unsure of it, he commits an offense of wrong conduct. If it is not a group of traveling thieves, and he does not perceive it as such, there is no offense. 

\subsection*{Non-offenses }

There\marginnote{2.3.1} is no offense: if he goes, but not by arrangement;  if the group has made an arrangement, but he has not expressed his agreement;  if he goes, but not according to the arrangement;  if there is an emergency;  if he is insane;  if he is the first offender. 

\scendsutta{The training rule on a group of traveling thieves, the sixth, is finished. }

%
\section*{{\suttatitleacronym Bu Pc 67}{\suttatitletranslation 67. The training rule on arrangements }{\suttatitleroot Saṁvidhāna}}
\addcontentsline{toc}{section}{\tocacronym{Bu Pc 67} \toctranslation{67. The training rule on arrangements } \tocroot{Saṁvidhāna}}
\markboth{67. The training rule on arrangements }{Saṁvidhāna}
\extramarks{Bu Pc 67}{Bu Pc 67}

\subsection*{Origin story }

At\marginnote{1.1} one time when the Buddha was staying at \textsanskrit{Sāvatthī} in \textsanskrit{Anāthapiṇḍika}’s Monastery, a monk who was traveling through the Kosalan country on his way to \textsanskrit{Sāvatthī} walked through the gateway of a certain village. A woman who had had an argument with her husband walked through the same gateway. When she saw that monk, she asked him, “Venerable, where are you going?” 

“I’m\marginnote{1.5} going to \textsanskrit{Sāvatthī}.” 

“May\marginnote{1.6} I go with you?” 

“Sure.”\marginnote{1.7} 

Soon\marginnote{1.8} afterwards that woman’s husband also left that village. He asked around, “Have you seen such-and-such a woman?” 

“She’s\marginnote{1.10} walking along with a monastic.” 

He\marginnote{1.11} then followed after them, caught that monk, and gave him a beating. The monk sat down fuming at the foot of a tree. And the woman said to her husband, “This monk didn’t make me go; I was the one who went with him. He’s innocent. Go and ask his forgiveness.” And he did so. 

That\marginnote{1.19} monk then went to \textsanskrit{Sāvatthī} where he told the monks what had happened. The monks of few desires complained and criticized him, “How could a monk travel by arrangement with a woman?” … “Is it true, monk, that you did this?” 

“It’s\marginnote{1.23} true, sir.” 

The\marginnote{1.24} Buddha rebuked him … “Foolish man, how could do this? This will affect people’s confidence …” … “And, monks, this training rule should be recited like this: 

\subsection*{Final ruling }

\scrule{‘If a monk travels by arrangement with a woman, even just to the next inhabited area, he commits an offense entailing confession.’” }

\subsection*{Definitions }

\begin{description}%
\item[A: ] whoever … %
\item[Monk: ] … The monk who has been given the full ordination by a unanimous Sangha through a legal procedure consisting of one motion and three announcements that is irreversible and fit to stand—this sort of monk is meant in this case. %
\item[A woman: ] a female human being, not a female spirit, not a female ghost, not a female animal. She understands and is capable of discerning bad speech and good speech, what is indecent and what is decent. %
\item[With: ] together. %
\item[By arrangement: ] if he makes an arrangement like this: he says, “Let’s go,” and she replies, “Yes, let’s go, venerable;” or she says, “Let’s go, venerable,” and he replies, “Yes, let’s go;” or he says, “Let’s go today,” “Let’s go tomorrow,” or “Let’s go the day after tomorrow,” then he commits an offense of wrong conduct. %
\item[Even just to the next inhabited area: ] when the inhabited areas are a chicken’s flight apart, then for every next inhabited area he commits an offense entailing confession. When it is an uninhabited area, a wilderness, then for every six kilometers he commits an offense entailing confession.\footnote{For a discussion of the \textit{yojana}, see \textit{sugata} in Appendix of Technical Terms. } %
\end{description}

\subsection*{Permutations }

If\marginnote{2.2.1} it is a woman, and he perceives her as such, and he travels by arrangement with her, even just to the next inhabited area, he commits an offense entailing confession. If it is a woman, but he is unsure of it, and he travels by arrangement with her, even just to the next inhabited area, he commits an offense entailing confession. If it is a woman, but he does not perceive her as such, and he travels by arrangement with her, even just to the next inhabited area, he commits an offense entailing confession. 

If\marginnote{2.2.4} the monk makes an arrangement, but the woman does not express her agreement, he commits an offense of wrong conduct. If he travels by arrangement with a female spirit, with a female ghost, with a \textit{\textsanskrit{paṇḍaka}}, with a female animal in the form of a woman, even just to the next inhabited area, he commits an offense of wrong conduct. 

If\marginnote{2.2.6} it is not a woman, but he perceives them as such, he commits an offense of wrong conduct. If it is not a woman, but he is unsure of it, he commits an offense of wrong conduct. If it is not a woman, and he does not perceive them as such, there is no offense. 

\subsection*{Non-offenses }

There\marginnote{2.3.1} is no offense: if he goes, but not by arrangement;  if the woman has made an arrangement, but he has not expressed his agreement;  if he goes, but not according to the arrangement;  if there is an emergency;  if he is insane;  if he is the first offender. 

\scendsutta{The training rule on arrangements, the seventh, is finished. }

%
\section*{{\suttatitleacronym Bu Pc 68}{\suttatitletranslation 68. The training rule on Ariṭṭha }{\suttatitleroot Antarāyika}}
\addcontentsline{toc}{section}{\tocacronym{Bu Pc 68} \toctranslation{68. The training rule on Ariṭṭha } \tocroot{Antarāyika}}
\markboth{68. The training rule on Ariṭṭha }{Antarāyika}
\extramarks{Bu Pc 68}{Bu Pc 68}

\subsection*{Origin story }

At\marginnote{1.1} one time the Buddha was staying at \textsanskrit{Sāvatthī} in the Jeta Grove, \textsanskrit{Anāthapiṇḍika}’s Monastery. At that time the monk \textsanskrit{Ariṭṭha}, an ex-vulture-killer, had the following bad and erroneous view: “As I understand the Teaching of the Buddha, the things he calls obstacles are unable to obstruct one who indulges in them.” 

A\marginnote{1.4} number of monks heard that \textsanskrit{Ariṭṭha} had that view. They went to him and asked, “Is it true, \textsanskrit{Ariṭṭha}, that you have such a view?” 

“Yes,\marginnote{1.11} indeed. As I understand the Buddha’s Teaching, the things he calls obstacles are unable to obstruct one who indulges in them.” 

“No,\marginnote{1.12} \textsanskrit{Ariṭṭha}, don’t misrepresent the Buddha, for it’s not good to misrepresent him. The Buddha would never say such a thing. The Buddha has given many discourses about the obstacles being obstructive and how they obstruct one who indulges in them. The Buddha has said that the enjoyment provided by worldly pleasures is small, whereas the suffering and trouble with them are huge, and so their drawbacks are greater. The Buddha has said that worldly pleasures are similar to a skeleton … similar to a piece of meat … similar to a grass torch … similar to a pit of coals … similar to a dream … similar to borrowed goods … similar to fruits on a tree … similar to a knife and chopping block … similar to swords and stakes … similar to a snake’s head; the suffering and trouble with them are huge, and so their drawbacks are greater.” 

But\marginnote{1.25} even though the monks corrected \textsanskrit{Ariṭṭha} like this, he stubbornly held on to that bad and erroneous view, and continued to insist on it. Since they were unable to make him give up that view, they went to the Buddha and told him what had happened. Soon afterwards the Buddha had the Sangha gathered and questioned \textsanskrit{Ariṭṭha}: “Is it true, \textsanskrit{Ariṭṭha}, that you have such a view?” 

“Yes\marginnote{1.31} indeed, sir.” 

“Foolish\marginnote{1.32} man, who do you think I have taught like this? Haven’t I given many discourses about the obstacles being obstructive and how they obstruct one who indulges in them? I have said that the enjoyment provided by worldly pleasures is small, whereas the suffering and trouble with them are huge, and so their drawbacks are greater. I have said that worldly pleasures are similar to a skeleton … similar to a piece of meat … similar to a grass torch … similar to a pit of coals … similar to a dream … similar to borrowed goods … similar to fruits on a tree … similar to a knife and chopping block … similar to swords and stakes … similar to a snake’s head; the suffering and trouble with them are huge, and so their drawbacks are greater. And yet, foolish man, by misunderstanding you have misrepresented me, hurt yourself, and made much demerit. This will be for your long-lasting harm and suffering. 

This\marginnote{1.47} will affect people’s confidence …” … “And, monks, this training rule should be recited like this: 

\subsection*{Final ruling }

\scrule{‘If a monk says,  “As I understand the Buddha’s Teaching, the things he calls obstacles are unable to obstruct one who indulges in them,”  then the monks should correct him like this:  “No, venerable, don’t misrepresent the Buddha, for it’s not good to misrepresent the Buddha. The Buddha would never say such a thing.  In many discourses the Buddha has declared the obstacles to be obstructive and how they obstruct one who indulges in them.” If that monk continues as before, the monks should press him up to three times to make him give up that view. If he then gives it up, all is well.  If he does not, he commits an offense entailing confession.’” }

\subsection*{Definitions }

\begin{description}%
\item[A: ] whoever … %
\item[Monk: ] … The monk who has been given the full ordination by a unanimous Sangha through a legal procedure consisting of one motion and three announcements that is irreversible and fit to stand—this sort of monk is meant in this case. %
\item[Says: ] “As I understand the Buddha’s Teaching, the things he calls obstacles are unable to obstruct one who indulges in them.” %
\item[Him: ] the monk who speaks in that way. %
\item[The monks: ] other\marginnote{2.1.10} monks, those who see it or hear about it. They should correct him,\footnote{See \href{https://suttacentral.net/pli-tv-bu-vb-ss13/en/brahmali\#2.16}{Bu Ss 13:2.16} for the meaning of \textit{\textsanskrit{suṇanti}} in this context. } “No, venerable, don’t misrepresent the Buddha, for it’s not good to misrepresent the Buddha. The Buddha would never say such a thing. In many discourses the Buddha has declared the obstacles to be obstructive and how they obstruct one who indulges in them.” And they should correct him a second and a third time. If he gives up that view, all is well. If he does not, he commits an offense of wrong conduct. If those who hear about it do not say anything, they commit an offense of wrong conduct. 

That\marginnote{2.1.18} monk, even if he has to be pulled into the Sangha, should be corrected like this: “No, venerable, don’t misrepresent the Buddha, for it’s not good to misrepresent the Buddha. The Buddha would never say such a thing. In many discourses the Buddha has declared the obstacles to be obstructive and how they obstruct one who indulges in them.” And they should correct him a second and a third time. If he gives up that view, all is well. If he does not, he commits an offense of wrong conduct. He should then be pressed. “And, monks, he should be pressed like this. A competent and capable monk should inform the Sangha: 

‘Please,\marginnote{2.1.28} venerables, I ask the Sangha to listen. Monk so-and-so has the following a bad and erroneous view: “As I understand the Buddha’s Teaching, the things he calls obstacles are unable to obstruct one who indulges in them.” He is not giving up that view. If the Sangha is ready, it should press him to make him give it up. This is the motion. 

Please,\marginnote{2.1.35} venerables, I ask the Sangha to listen. Monk so-and-so has the following bad and erroneous view: “As I understand the Buddha’s Teaching, the things he calls obstacles are unable to obstruct one who indulges in them.” He is not giving up that view. The Sangha presses him to make him give it up. Any monk who approves of pressing him to make him give it up should remain silent. Any monk who doesn’t approve should speak up. 

A\marginnote{2.1.42} second time … A third time I speak on this matter: Please, venerables, I ask the Sangha to listen. Monk so-and-so has the following bad and erroneous view: “As I understand the Buddha’s Teaching, the things he calls obstacles are unable to obstruct one who indulges in them.” He is not giving up that view. The Sangha presses him to make him give it up. Any monk who approves of pressing him to make him give it up should remain silent. Any monk who doesn’t approve should speak up. 

The\marginnote{2.1.51} Sangha has pressed this monk to give up that view. The Sangha approves and is therefore silent. I’ll remember it thus.’” 

After\marginnote{2.1.53} the motion, he commits an offense of wrong conduct.\footnote{The Pali just says \textit{\textsanskrit{dukkaṭa}}, without specifying that it is an \textit{\textsanskrit{āpatti}}, “an offense”. Yet elsewhere, such as at \href{https://suttacentral.net/pli-tv-bu-vb-ss10/en/brahmali\#2.65}{Bu Ss 10:2.65}, the \textit{\textsanskrit{dukkaṭa}} is annulled if you commit the full offense of \textit{\textsanskrit{saṅghādisesa}}. The implication is that in these contexts \textit{\textsanskrit{dukkaṭa}} should be read as \textit{\textsanskrit{āpatti} \textsanskrit{dukkaṭassa}}, “an offense of wrong conduct”. } After each of the first two announcements, he commits an offense of wrong conduct. When the last announcement is finished, he commits an offense entailing confession. 

%
\end{description}

\subsection*{Permutations }

If\marginnote{2.2.1} it is a legitimate legal procedure, and he perceives it as such, and he does not give up his view, he commits an offense entailing confession. If it is a legitimate legal procedure, but he is unsure of it, and he does not give up his view, he commits an offense entailing confession. If it is a legitimate legal procedure, but he perceives it as illegitimate, and he does not give up his view, he commits an offense entailing confession. 

If\marginnote{2.2.4} it is an illegitimate legal procedure, but he perceives it as legitimate, he commits an offense of wrong conduct. If it is an illegitimate legal procedure, but he is unsure of it, he commits an offense of wrong conduct. If it is an illegitimate legal procedure, and he perceives it as such, he commits an offense of wrong conduct. 

\subsection*{Non-offenses }

There\marginnote{2.3.1} is no offense: if he has not been pressed;  if he gives it up;  if he is insane;  if he is the first offender.\footnote{By editorial mistake the Pali is missing \textit{\textsanskrit{ādikammikassa}}, “the first offender”. This is a universal exception that applies to all the rules of the Vinaya. } 

\scendsutta{The training rule on \textsanskrit{Ariṭṭha}, the eighth, is finished. }

%
\section*{{\suttatitleacronym Bu Pc 69}{\suttatitletranslation 69. The training rule on living with one who has been ejected }{\suttatitleroot Ukkhittasambhoga}}
\addcontentsline{toc}{section}{\tocacronym{Bu Pc 69} \toctranslation{69. The training rule on living with one who has been ejected } \tocroot{Ukkhittasambhoga}}
\markboth{69. The training rule on living with one who has been ejected }{Ukkhittasambhoga}
\extramarks{Bu Pc 69}{Bu Pc 69}

\subsection*{Origin story }

At\marginnote{1.1} one time the Buddha was staying at \textsanskrit{Sāvatthī} in the Jeta Grove, \textsanskrit{Anāthapiṇḍika}’s Monastery. At that time the monks from the group of six lived with the monk \textsanskrit{Ariṭṭha}, and they did formal meetings and shared a sleeping place with him. Yet they knew that he was saying such things, that he had not made amends according to the rule, and that he had not given up that view.\footnote{“Such things” refers back to the previous rule. } 

The\marginnote{1.3} monks of few desires complained and criticized them, “How can the monks from the group of six live, do formal meetings, and share a sleeping place with the monk \textsanskrit{Ariṭṭha}, even though they know that he’s saying such things, that he hasn’t made amends according to the rule, and that he hasn’t given up that view?” … “Is it true, monks, that you do this?” 

“It’s\marginnote{1.6} true, sir.” 

The\marginnote{1.7} Buddha rebuked them … “Foolish men, how can you do this? This will affect people’s confidence …” … “And, monks, this training rule should be recited like this: 

\subsection*{Final ruling }

\scrule{‘If a monk lives, does formal meetings, or shares a sleeping place with a monk who he knows is saying such things, who has not made amends according to the rule, and who has not given up that view, he commits an offense entailing confession.’” }

\subsection*{Definitions }

\begin{description}%
\item[A: ] whoever … %
\item[Monk: ] … The monk who has been given the full ordination by a unanimous Sangha through a legal procedure consisting of one motion and three announcements that is irreversible and fit to stand—this sort of monk is meant in this case. %
\item[He knows: ] he knows by himself or others have told him or the offender has told him.\footnote{The meaning of the last of these three ways of knowing, \textit{so \textsanskrit{vā} \textsanskrit{āroceti}}, is not clear. CPD suggests: “\textit{sa (\textsanskrit{sā}) \textsanskrit{āroceti}} (?). Perhaps this last form is conformable to sa. \textit{\textsanskrit{ārocayate}} med. caus. in the meaning: he or she makes inquiries (of others).” However, this does not fit with the parallel usage at \href{https://suttacentral.net/pli-tv-bu-vb-pc29/en/brahmali\#3.1.6}{Bu Pc 29:3.1.6} where the text says that she tells (\textit{\textsanskrit{sā} \textsanskrit{vā} \textsanskrit{āroceti}}) him, presumably referring to the nun telling the monk. In this case \textit{\textsanskrit{āroceti}} cannot refer to the monk making inquiries. The commentaries are silent, and I therefore assume that a straightforward meaning is the most likely one. I would suggest, then, that it simply refers to the ejected monk telling him directly. } %
\item[Who is saying such things: ] one who says this: “As I understand the Buddha’s Teaching, the things he calls obstacles are unable to obstruct one who indulges in them.” %
\item[Who has not made amends according to the rule: ] who has been ejected and not readmitted. %
\item[With a monk who has not given up that view: ] with a monk who has not given up this view. %
\item[Lives with: ] there are two types of living together: material living together and spiritual living together. %
\item[Material living together: ] if he gives or receives material things, he commits an offense entailing confession. %
\item[Spiritual living together: ] he recites or has the other recite. If he recites or has the other recite by the line, then for every line he commits an offense entailing confession. If he recites or has the other recite by the syllable, then for every syllable he commits an offense entailing confession. %
\item[Does formal meetings with: ] if he does the observance-day ceremony, the invitation ceremony, or a legal procedure with one who has been ejected, he commits an offense entailing confession.\footnote{“Observance-day ceremony” renders \textit{uposatha}. See Appendix of Technical Terms for discussion. } %
\item[Shares a sleeping place with: ] under the same ceiling: if the monk lies down when the one who has been ejected is already lying down, he commits an offense entailing confession; if the monk is already lying down when the one who has been ejected lies down, he commits an offense entailing confession; if they both lie down together, he commits an offense entailing confession; every time they get up and then lie down again, he commits an offense entailing confession. %
\end{description}

\subsection*{Permutations }

If\marginnote{2.2.1} the other monk has been ejected, and he perceives him as such, and he lives or does formal meetings or shares a sleeping place with him, he commits an offense entailing confession. If the other monk has been ejected, but he is unsure of it, and he lives or does formal meetings or shares a sleeping place with him, he commits an offense of wrong conduct. If the other monk has been ejected, but he does not perceive him as such, and he lives or does formal meetings or shares a sleeping place with him, there is no offense. 

If\marginnote{2.2.4} the other monk has not been ejected, but he perceives him as such, he commits an offense of wrong conduct. If the other monk has not been ejected, but he is unsure of it, he commits an offense of wrong conduct. If the other monk has not been ejected, and he does not perceive him as such, there is no offense. 

\subsection*{Non-offenses }

There\marginnote{2.3.1} is no offense: if he knows that he has not been ejected;  if he knows that he has been readmitted after being ejected;  if he knows that he has given up that view;  if he is insane;  if he is the first offender. 

\scendsutta{The training rule on living with one who has been ejected, the ninth, is finished. }

%
\section*{{\suttatitleacronym Bu Pc 70}{\suttatitletranslation 70. The training rule on Kaṇṭaka }{\suttatitleroot Samaṇuddesaantarāyika}}
\addcontentsline{toc}{section}{\tocacronym{Bu Pc 70} \toctranslation{70. The training rule on Kaṇṭaka } \tocroot{Samaṇuddesaantarāyika}}
\markboth{70. The training rule on Kaṇṭaka }{Samaṇuddesaantarāyika}
\extramarks{Bu Pc 70}{Bu Pc 70}

\subsection*{Origin story }

At\marginnote{1.1} one time the Buddha was staying at \textsanskrit{Sāvatthī} in the Jeta Grove, \textsanskrit{Anāthapiṇḍika}’s Monastery. At that time the novice monastic \textsanskrit{Kaṇṭaka} had the following bad and erroneous view: “As I understand the teachings of the Buddha, the things he calls obstacles are unable to obstruct one who indulges in them.” 

A\marginnote{1.4} number of monks heard that the novice monastic \textsanskrit{Kaṇṭaka} had that view. They went to see him and asked, “Is it true, \textsanskrit{Kaṇṭaka}, that you have such a view?” 

“Yes,\marginnote{1.9} indeed. As I understand the Buddha’s Teaching, the things he calls obstacles are unable to obstruct one who indulges in them.” 

“No,\marginnote{1.10} \textsanskrit{Kaṇṭaka}, don’t misrepresent the Buddha, for it’s not good to misrepresent him. The Buddha would never say such a thing. The Buddha has given many discourses about the obstacles being obstructive and how they obstruct one who indulges in them. The Buddha has said that the enjoyment provided by worldly pleasures is small, whereas the suffering and trouble with them are huge, and so their drawbacks are greater. …” But even though the monks corrected \textsanskrit{Kaṇṭaka} like this, he stubbornly held on to that bad and erroneous view, and he continued to insist on it. 

Since\marginnote{1.15} those monks were unable to make him give up that view, they went to the Buddha and told him what had happened. Soon afterwards the Buddha had the Sangha gathered and questioned the \textsanskrit{Kaṇṭaka}: “Is it true, \textsanskrit{Kaṇṭaka}, that you have such a view?” 

“Yes\marginnote{1.19} indeed, venerable sir.” 

“Foolish\marginnote{1.20} man, who do you think I have taught like this? Haven’t I given many discourses about the obstacles being obstructive and how they obstruct one who indulges in them? I have said that the enjoyment provided by worldly pleasures is small, whereas the suffering and trouble with them are huge, and so their drawbacks are greater. I have said that worldly pleasures are similar to a skeleton … similar to a snake’s head; the suffering and trouble with them are huge, and so their drawbacks are greater. And yet, foolish man, by misunderstanding you have misrepresented me, hurt yourself, and made much demerit. This will be for your long-lasting harm and suffering. 

This\marginnote{1.27} will affect people’s confidence, and cause some to lose it.” 

Having\marginnote{1.29} rebuked him … the Buddha gave a teaching and addressed the monks: “Well then, monks, the Sangha should expel the novice monastic \textsanskrit{Kaṇṭaka}. And this is how he should be expelled: ‘From today, \textsanskrit{Kaṇṭaka}, you may not refer to the Buddha as your teacher. And, unlike other novices, you can no longer share a sleeping place with the monks for two or three nights. Go! Away with you!’” The Sangha then expelled \textsanskrit{Kaṇṭaka}. 

Soon\marginnote{1.37} afterwards the monks from the group of six befriended \textsanskrit{Kaṇṭaka}, and they were attended on by him, lived with him, and shared a sleeping place with him. Yet they knew that he had been expelled. The monks of few desires complained and criticized those monks, “How can the monks from the group of six befriend \textsanskrit{Kaṇṭaka}, and be attended on by him, live with him, and share a sleeping place with him, even though they know that he has been expelled?” … “Is it true, monks, that you do this?” 

“It’s\marginnote{1.41} true, sir.” 

The\marginnote{1.42} Buddha rebuked them … “Foolish men, how can you do this? This will affect people’s confidence …” … “And, monks, this training rule should be recited like this: 

\subsection*{Final ruling }

\scrule{‘Also if a novice monastic says, “As I understand the Buddha’s Teaching, the things he calls obstacles are unable to obstruct one who indulges in them,”  then the monks should correct him like this: “No, don’t misrepresent the Buddha, for it’s not good to misrepresent the Buddha. The Buddha would never say such a thing. In many discourses the Buddha has declared the obstacles to be obstructive and how they obstruct one who indulges in them.” If that novice monastic continues as before, he should be told: “From today on you may not refer to the Buddha as your teacher. And, unlike other novices, you can no longer share a sleeping place with the monks for two or three nights. Go! Away with you!” If a monk befriends that novice monastic, or he is attended on by him, lives with him, or shares a sleeping place with him, even though he knows that he has been expelled in this way, he commits an offense entailing confession.’” }

\subsection*{Definitions }

\begin{description}%
\item[A novice monastic: ] a novice monk is what is meant. %
\item[Says: ] “As I understand the Buddha’s Teaching, the things he calls obstacles are unable to obstruct one who indulges in them.” %
\item[Him: ] the novice monastic who speaks in that way. %
\item[The monks: ] other\marginnote{2.1.8} monks, those who see it or hear about it. They should correct him, “No, don’t misrepresent the Buddha, for it’s not good to misrepresent the Buddha. The Buddha would never say such a thing.\footnote{“Should correct” renders \textit{vattabba}, the future passive participle of \textit{vadati}. For a discussion of the verb \textit{vadati} used in the sense of “correct”, see Appendix of Technical Terms. } In many discourses the Buddha has declared the obstacles to be obstructive and how they obstruct one who indulges in them.” 

And\marginnote{2.1.10} they should correct him a second time … And they should correct him a third time … 

If\marginnote{2.1.12} he gives up that view, all is well. If he does not, he should be told: “From today on you may not refer to the Buddha as your teacher. And, unlike other novices, you can no longer share a sleeping place with the monks for two or three nights. Go! Away with you!” 

%
\item[A: ] whoever … %
\item[Monk: ] … The monk who has been given the full ordination by a unanimous Sangha through a legal procedure consisting of one motion and three announcements that is irreversible and fit to stand—this sort of monk is meant in this case. %
\item[He knows: ] he knows by himself or others have told him or the offending novice monastic has told him.\footnote{The meaning of the last of these three ways of knowing, \textit{so \textsanskrit{vā} \textsanskrit{āroceti}}, is not clear. CPD suggests: “\textit{sa (\textsanskrit{sā}) \textsanskrit{āroceti}} (?). Perhaps this last form is conformable to sa. \textit{\textsanskrit{ārocayate}} med. caus. in the meaning: he or she makes inquiries (of others).” However, this does not fit with the parallel usage at \href{https://suttacentral.net/pli-tv-bu-vb-pc29/en/brahmali\#3.1.6}{Bu Pc 29:3.1.6} where the text says that she tells (\textit{\textsanskrit{sā} \textsanskrit{vā} \textsanskrit{āroceti}}) him, presumably referring to the nun telling the monk. In this case \textit{\textsanskrit{āroceti}} cannot refer to the monk making inquiries. The commentaries are silent, and I therefore assume that a straightforward meaning is the most likely one. I would suggest, then, that it simply refers to the novice monk telling the monk directly. } %
\item[Expelled in this way: ] expelled like this. %
\item[Novice monastic: ] novice monk is what is meant. %
\item[Befriends: ] if he befriends him, saying, “I’ll give him a bowl,” “I’ll give him a robe,” “I’ll recite to him,” or “I’ll test him,” he commits an offense entailing confession. %
\item[Is attended on by: ] if he accepts bath powder, soap, a tooth cleaner, or water for rinsing the mouth from him, he commits an offense entailing confession.\footnote{For an explanation of the renderings “bath powder” and “soap” for \textit{\textsanskrit{cuṇṇa}} and \textit{mattika} respectively, see Appendix of Technical Terms. } %
\item[Lives with: ] there are two types of living together: material living together and spiritual living together. %
\item[Material living together: ] if he gives or receives material things, he commits an offense entailing confession. %
\item[Spiritual living together: ] he recites or has the other recite. If he recites or has the other recite by the line, then for every line he commits an offense entailing confession. If he recites or has the other recite by the syllable, then for every syllable he commits an offense entailing confession. %
\item[Shares a sleeping place with: ] under the same ceiling: if the monk lies down when the expelled novice monastic is already lying down, he commits an offense entailing confession; if the monk is already lying down when the expelled novice monastic lies down, he commits an offense entailing confession; if they both lie down together, he commits an offense entailing confession; every time they get up and then lie down again, he commits an offense entailing confession. %
\end{description}

\subsection*{Permutations }

If\marginnote{2.2.1} the novice monastic has been expelled, and he perceives him as such, and he befriends him or is attended on by him or lives with him or shares a sleeping place with him, he commits an offense entailing confession. If the novice monastic has been expelled, but he is unsure of it, and he befriends him or is attended on by him or lives with him or shares a sleeping place with him, he commits an offense of wrong conduct. If the novice monastic has been expelled, but he does not perceive him as such, and he befriends him or is attended on by him or lives with him or shares a sleeping place with him, there is no offense. 

If\marginnote{2.2.4} the novice monastic has not been expelled, but he perceives him such, he commits an offense of wrong conduct. If the novice monastic has not been expelled, but he is unsure of it, he commits an offense of wrong conduct. If the novice monastic has not been expelled, and he does not perceive him as such, there is no offense. 

\subsection*{Non-offenses }

There\marginnote{2.3.1} is no offense: if he knows that he has not been expelled;  if he knows that he has given up that view;  if he is insane;  if he is the first offender. 

\scendsutta{The training rule on \textsanskrit{Kaṇṭaka}, the tenth, is finished. }

\scendvagga{The seventh subchapter on containing living beings is finished. }

\scuddanaintro{This is the summary: }

\begin{scuddana}%
“Intentionally\marginnote{2.3.9} killing, containing living beings, \\
Agitation, concealing what is grave; \\
Less than twenty, and group of travelers, \\
Arrangement, on \textsanskrit{Ariṭṭha}; \\
Ejected, and \textsanskrit{Kaṇṭaka}: \\
These ten training rules.” 

%
\end{scuddana}

%
\section*{{\suttatitleacronym Bu Pc 71}{\suttatitletranslation 71. The training rule on legitimately }{\suttatitleroot Sahadhammika}}
\addcontentsline{toc}{section}{\tocacronym{Bu Pc 71} \toctranslation{71. The training rule on legitimately } \tocroot{Sahadhammika}}
\markboth{71. The training rule on legitimately }{Sahadhammika}
\extramarks{Bu Pc 71}{Bu Pc 71}

\subsection*{Origin story }

At\marginnote{1.1} one time when the Buddha was staying at \textsanskrit{Kosambī} in Ghosita’s Monastery, Venerable Channa was misbehaving. The monks told him, “Don’t do that, Channa, it’s not allowable,” and he would reply, “I won’t practice this training rule until I’ve questioned a monk who is an expert on the Monastic Law.” 

The\marginnote{1.8} monks of few desires complained and criticized him, “How could Venerable Channa say this when legitimately corrected by the monks?” … “Is it true, Channa, that you said this?” 

“It’s\marginnote{1.13} true, sir.” 

The\marginnote{1.14} Buddha rebuked him … “Foolish man, how could you say this when legitimately corrected by the monks? This will affect people’s confidence …” … “And, monks, this training rule should be recited like this: 

\subsection*{Final ruling }

\scrule{‘If a monk, when legitimately corrected by the monks, says, “I won’t practice this training rule until I’ve questioned a monk who is an expert on the Monastic Law,” he commits an offense entailing confession. A monk who is training should understand, should question, should enquire. This is the proper procedure.’” }

\subsection*{Definitions }

\begin{description}%
\item[A: ] whoever … %
\item[Monk: ] … The monk who has been given the full ordination by a unanimous Sangha through a legal procedure consisting of one motion and three announcements that is irreversible and fit to stand—this sort of monk is meant in this case. %
\item[By the monks: ] by other monks. %
\item[Legitimately: ] the training rules laid down by the Buddha—this is called “legitimately”. When corrected in regard to this, he says, “I won’t practice this training rule until I’ve questioned a monk who’s an expert on the Monastic Law.” If he says, “I’ll question one who’s wise,” “I’ll question one who’s competent,” “I’ll question one who’s intelligent,” “I’ll question one who’s learned,” “I’ll question an expounder of the Teaching,” he commits an offense entailing confession. %
\end{description}

\subsection*{Permutations }

If\marginnote{2.2.1} the one who corrects him is fully ordained, and he perceives him as such, and he says such a thing, he commits an offense entailing confession. If the one who corrects him is fully ordained, but he is unsure of it, and he says such a thing, he commits an offense entailing confession. If the one who corrects him is fully ordained, but he does not perceive him as such, and he says such a thing, he commits an offense entailing confession. 

If\marginnote{2.2.4} he is corrected about something that has not been laid down: “This isn’t conducive to self-effacement,” “This isn’t conducive to ascetic practices,” “This isn’t conducive to being inspiring,” “This isn’t conducive to a reduction in things,” “This isn’t conducive to being energetic,” and he says, “I won’t practice this training rule until I’ve questioned a monk who’s competent,” “… until I’ve questioned a monk who’s an expert on the Monastic Law,” “… until I’ve questioned a monk who’s wise,” “… until I’ve questioned a monk who’s intelligent,” “… until I’ve questioned a monk who’s learned,” “… until I’ve questioned a monk who’s an expounder of the Teaching,” he commits an offense of wrong conduct. 

If\marginnote{2.2.6} he is corrected by one who is not fully ordained about something that has or has not been laid down: “This isn’t conducive to self-effacement,” “This isn’t conducive to ascetic practices,” “This isn’t conducive to being inspiring,” “This isn’t conducive to a reduction in things,” “This isn’t conducive to being energetic,” and he says, “I won’t practice this training rule until I’ve questioned a monk who’s competent,” “… until I’ve questioned a monk who’s an expert on the Monastic Law,” “… until I’ve questioned a monk who’s wise,” “… until I’ve questioned a monk who’s intelligent,” “… until I’ve questioned a monk who’s learned,” “… until I’ve questioned a monk who’s an expounder of the Teaching,” he commits an offense of wrong conduct. 

If\marginnote{2.2.8} the one who corrects him is not fully ordained, but he perceives them as such, he commits an offense of wrong conduct. If the one who corrects him is not fully ordained, but he is unsure of it, he commits an offense of wrong conduct. If the one who corrects him is not fully ordained, and he does not perceive them as such, he commits an offense of wrong conduct. 

\subsection*{More Definitions }

\begin{description}%
\item[Who is training: ] who wants to train. %
\item[Should understand: ] should find out. %
\item[Should question: ] should ask, “Venerable, how is this? What’s the meaning of this?” %
\item[Should enquire: ] should reflect, should weigh up. %
\item[This is the proper procedure: ] this is the right method. %
\end{description}

\subsection*{Non-offenses }

There\marginnote{2.3.1} is no offense: if he says, “I’ll find out and I’ll train;”  if he is insane;  if he is the first offender. 

\scendsutta{The training rule on legitimately, the first, is finished. }

%
\section*{{\suttatitleacronym Bu Pc 72}{\suttatitletranslation 72. The training rule on annoyance }{\suttatitleroot Vilekhana}}
\addcontentsline{toc}{section}{\tocacronym{Bu Pc 72} \toctranslation{72. The training rule on annoyance } \tocroot{Vilekhana}}
\markboth{72. The training rule on annoyance }{Vilekhana}
\extramarks{Bu Pc 72}{Bu Pc 72}

\subsection*{Origin story }

At\marginnote{1.1} one time the Buddha was staying at \textsanskrit{Sāvatthī} in the Jeta Grove, \textsanskrit{Anāthapiṇḍika}’s Monastery. At that time the Buddha gave many talks about the Monastic Law, spoke in praise of the Monastic Law and of learning the Monastic Law, and repeatedly praised Venerable \textsanskrit{Upāli}. When the monks heard this, they thought, “Well then, let’s learn the Monastic Law from Venerable \textsanskrit{Upāli}.” And many monks, both senior and junior, as well as those of middle standing, learned the Monastic Law from \textsanskrit{Upāli}. 

The\marginnote{1.6} monks from the group of six considered this and thought, “If these monks become well-versed in the Monastic Law, they’ll boss us around as they like. So let’s disparage the Monastic Law.” 

They\marginnote{1.10} went to the other monks and said, “What’s the point of reciting these minor training rules, when they just lead to anxiety, oppression, and annoyance?” 

The\marginnote{1.12} monks of few desires complained and criticized them, “How can the monks from the group of six disparage the Monastic Law?” … “Is it true, monks, that you do this?” 

“It’s\marginnote{1.15} true, sir.” 

The\marginnote{1.16} Buddha rebuked them … “Foolish men, how can you do this? This will affect people’s confidence …” … “And, monks, this training rule should be recited like this: 

\subsection*{Final ruling }

\scrule{‘When the Monastic Code is being recited, if a monk says, “What’s the point of reciting these minor training rules, when they just lead to anxiety, oppression, and annoyance?” then in disparaging the training rules, he commits an offense entailing confession.’” }

\subsection*{Definitions }

\begin{description}%
\item[A: ] whoever … %
\item[Monk: ] … The monk who has been given the full ordination by a unanimous Sangha through a legal procedure consisting of one motion and three announcements that is irreversible and fit to stand—this sort of monk is meant in this case. %
\item[When the Monastic Code is being recited: ] when reciting it, when having it recited, or when practicing it. %
\item[Says: ] “What’s the point of reciting these minor training rules, when they just lead to anxiety, oppression, and annoyance?” If he disparages the Monastic Law to one who is fully ordained, saying, “Those who learn this will be anxious,” “They will feel oppressed,” “They will be annoyed;” “Those who don’t learn this won’t be anxious,” “They won’t feel oppressed,” “They won’t be annoyed;” “It’s better left unrecited,” “It’s better left unlearned,” “It’s better left unstudied,” “It’s better left unmastered;” “May the Monastic Law disappear, or may these monks remain ignorant,” then he commits an offense entailing confession. %
\end{description}

\subsection*{Permutations }

If\marginnote{2.2.1} he disparages the Monastic Law to one who is fully ordained, and he perceives them as fully ordained, he commits an offense entailing confession. If he disparages the Monastic Law to one who is fully ordained, but he is unsure of it, he commits an offense entailing confession. If he disparages the Monastic Law to one who is fully ordained, but he does not perceive them as fully ordained, he commits an offense entailing confession. 

If\marginnote{2.2.4} he disparages some other rule, he commits an offense of wrong conduct. If he disparages the Monastic Law or some other rule to one who is not fully ordained, he commits an offense of wrong conduct. 

If\marginnote{2.2.6} it is to one who is not fully ordained, but he perceives them as such, he commits an offense of wrong conduct. If it is to one who is not fully ordained, but he is unsure of it, he commits an offense of wrong conduct. If it is to one who is not fully ordained, and he does not perceive them as such, he commits an offense of wrong conduct. 

\subsection*{Non-offenses }

There\marginnote{2.3.1} is no offense: if, not desiring to disparage, he says, “Listen, learn discourses or verses or philosophy, and later you can learn the Monastic Law;”  if he is insane;  if he is the first offender. 

\scendsutta{The training rule on annoyance, the second, is finished. }

%
\section*{{\suttatitleacronym Bu Pc 73}{\suttatitletranslation 73. The training rule on deception }{\suttatitleroot Mohana}}
\addcontentsline{toc}{section}{\tocacronym{Bu Pc 73} \toctranslation{73. The training rule on deception } \tocroot{Mohana}}
\markboth{73. The training rule on deception }{Mohana}
\extramarks{Bu Pc 73}{Bu Pc 73}

\subsection*{Origin story }

At\marginnote{1.1} one time when the Buddha was staying at \textsanskrit{Sāvatthī} in \textsanskrit{Anāthapiṇḍika}’s Monastery, the monks from the group of six were misbehaving. They said to each other, “Let’s make the other monks think that we committed these offenses because we didn’t know the rules.” Then, during the recitation of the Monastic Code, they said, “Only now did we find out that this rule too is included in the Monastic Code and comes up for recitation every half-month.” 

The\marginnote{1.4} monks of few desires complained and criticized them, “How could the monks from the group of six say this during the recitation of the Monastic Code?” … “Is it true, monks, that you said this?” 

“It’s\marginnote{1.9} true, sir.” 

The\marginnote{1.10} Buddha rebuked them … “Foolish men, how could you say this during the recitation of the Monastic Code? This will affect people’s confidence …” … “And, monks, this training rule should be recited like this: 

\subsection*{Final ruling }

\scrule{‘During the half-monthly recitation of the Monastic Code, a monk might say, “Only now did I find out that this rule too has come down in the Monastic Code, is included in the Monastic Code, and comes up for recitation every half-month.” If other monks know that that monk has previously sat through at least two or three recitations of the Monastic Code, then that monk is not let off because of ignorance, and he is to be dealt with according to the rule. Further, he should be charged with deception: “It’s a loss for you that you don’t pay proper attention during the recitation of the Monastic Code.” And for the act of deception, he commits an offense entailing confession.’” }

\subsection*{Definitions }

\begin{description}%
\item[A: ] whoever … %
\item[Monk: ] … The monk who has been given the full ordination by a unanimous Sangha through a legal procedure consisting of one motion and three announcements that is irreversible and fit to stand—this sort of monk is meant in this case. %
\item[Half-monthly: ] on every observance day. %
\item[During the recitation of the Monastic Code: ] during the reciting. %
\item[Might say: ] after misbehaving, he thinks, “Let them think that I committed these offenses because I didn’t know the rules.” If, during the recitation of the Monastic Code, he then says, “Only now did I find out that this rule too has come down in the Monastic Code, is included in the Monastic Code, and comes up for recitation every half-month,” then he commits an offense of wrong conduct. %
\end{description}

If\marginnote{2.1.13} other monks know that the monk who wants to deceive has previously sat through at least two or three recitations of the Monastic Code, then that monk is not let off because of ignorance, and he is to be dealt with according to the rule. Further, he should be charged with deception. 

“And,\marginnote{2.1.14} monks, he is to be charged like this. A competent and capable monk should inform the Sangha: 

‘Please,\marginnote{2.1.16} venerables, I ask the Sangha to listen. Monk so-and-so did not pay proper attention during the recitation of the Monastic Code. If the Sangha is ready, it should charge him with deception. This is the motion. 

Please,\marginnote{2.1.20} venerables, I ask the Sangha to listen. Monk so-and-so did not pay proper attention during the recitation of the Monastic Code. The Sangha charges him with deception. Any monk who approves of charging him with deception should remain silent. Any monk who doesn’t approve should speak up. 

The\marginnote{2.1.25} Sangha has charged monk so-and-so with deception. The Sangha approves and is therefore silent. I’ll remember it thus.’” 

If\marginnote{2.1.27} he deceives, but he has not been charged with deception, he commits an offense of wrong conduct. If he deceives, and he has been charged with deception, he commits an offense entailing confession. 

\subsection*{Permutations }

If\marginnote{2.2.1} it is a legitimate legal procedure, and he perceives it as legitimate, and he deceives, he commits an offense entailing confession. If it is a legitimate legal procedure, but he is unsure of it, and he deceives, he commits an offense entailing confession. If it is a legitimate legal procedure, but he perceives it as illegitimate, and he deceives, he commits an offense entailing confession. 

If\marginnote{2.2.4} it is an illegitimate legal procedure, but he perceives it as legitimate, he commits an offense of wrong conduct. If it is an illegitimate legal procedure, but he is unsure of it, he commits an offense of wrong conduct. If it is an illegitimate legal procedure, and he perceives it as such, he commits an offense of wrong conduct. 

\subsection*{Non-offenses }

There\marginnote{2.3.1} is no offense: if he has not heard it in full;  if he has heard it fewer than two or three times in full;  if he does not want to deceive;  if he is insane;  if he is the first offender. 

\scendsutta{The training rule on deception, the third, is finished. }

%
\section*{{\suttatitleacronym Bu Pc 74}{\suttatitletranslation 74. The training rule on hitting }{\suttatitleroot Pahāra}}
\addcontentsline{toc}{section}{\tocacronym{Bu Pc 74} \toctranslation{74. The training rule on hitting } \tocroot{Pahāra}}
\markboth{74. The training rule on hitting }{Pahāra}
\extramarks{Bu Pc 74}{Bu Pc 74}

\subsection*{Origin story }

At\marginnote{1.1} one time the when Buddha was staying at \textsanskrit{Sāvatthī} in \textsanskrit{Anāthapiṇḍika}’s Monastery, the monks from the group of six hit the monks from the group of seventeen in anger. They cried. Other monks asked them why, and they told them what had happened. 

The\marginnote{1.7} monks of few desires complained and criticized them, “How could the monks from the group of six hit other monks in anger?” … “Is it true, monks, that you did this?” 

“It’s\marginnote{1.10} true, sir.” 

The\marginnote{1.11} Buddha rebuked them … “Foolish men, how could you do this? This will affect people’s confidence …” … “And, monks, this training rule should be recited like this: 

\subsection*{Final ruling }

\scrule{‘If a monk hits a monk in anger, he commits an offense entailing confession.’” }

\subsection*{Definitions }

\begin{description}%
\item[A: ] whoever … %
\item[Monk: ] … The monk who has been given the full ordination by a unanimous Sangha through a legal procedure consisting of one motion and three announcements that is irreversible and fit to stand—this sort of monk is meant in this case. %
\item[A monk: ] another monk. %
\item[In anger: ] discontent, having hatred, hostile. %
\item[Hits: ] if he hits with his body, with anything connected to his body, or with anything released, even if just with a lotus leaf, he commits an offense entailing confession. %
\end{description}

\subsection*{Permutations }

If\marginnote{2.2.1} it is one who is fully ordained, and he perceives him as such, and he hits him in anger, he commits an offense entailing confession. If it is one who is fully ordained, but he is unsure of it, and he hits him in anger, he commits an offense entailing confession. If it is one who is fully ordained, but he does not perceive him as such, and he hits him in anger, he commits an offense entailing confession. 

If\marginnote{2.2.4} he hits one who is not fully ordained in anger, he commits an offense of wrong conduct. If it is one who is not fully ordained, but he perceives them as such, he commits an offense of wrong conduct. If it is one who is not fully ordained, but he is unsure of it, he commits an offense of wrong conduct. If it is one who is not fully ordained, and he does not perceive them as such, he commits an offense of wrong conduct. 

\subsection*{Non-offenses }

There\marginnote{2.3.1} is no offense: if he hits in self-defense;  if he is insane;  if he is the first offender. 

\scendsutta{The training rule on hitting, the fourth, is finished. }

%
\section*{{\suttatitleacronym Bu Pc 75}{\suttatitletranslation 75. The training rule on raising a hand }{\suttatitleroot Talasattika}}
\addcontentsline{toc}{section}{\tocacronym{Bu Pc 75} \toctranslation{75. The training rule on raising a hand } \tocroot{Talasattika}}
\markboth{75. The training rule on raising a hand }{Talasattika}
\extramarks{Bu Pc 75}{Bu Pc 75}

\subsection*{Origin story }

At\marginnote{1.1} one time when the Buddha was staying at \textsanskrit{Sāvatthī} in \textsanskrit{Anāthapiṇḍika}’s Monastery, the monks from the group of six raised their hands in anger against the monks from the group of seventeen. Expecting to be hit, they cried. Other monks asked them why, and they told them what had happened. 

The\marginnote{1.7} monks of few desires complained and criticized them, “How could the monks from the group of six do this?” … “Is it true, monks, that you did this?” 

“It’s\marginnote{1.10} true, sir.” 

The\marginnote{1.11} Buddha rebuked them … “Foolish men, how could you do this? This will affect people’s confidence …” … “And, monks, this training rule should be recited like this: 

\subsection*{Final ruling }

\scrule{‘If a monk raises a hand in anger against a monk, he commits an offense entailing confession.’” }

\subsection*{Definitions }

\begin{description}%
\item[A: ] whoever … %
\item[Monk: ] … The monk who has been given the full ordination by a unanimous Sangha through a legal procedure consisting of one motion and three announcements that is irreversible and fit to stand—this sort of monk is meant in this case. %
\item[Against a monk: ] against another monk. %
\item[In anger: ] discontent, having hatred, hostile. %
\item[Raises a hand: ] if he raises any part of his body or anything connected to his body, even if just a lotus leaf, he commits an offense entailing confession.\footnote{\textit{\textsanskrit{Kāyaṁ} \textsanskrit{uccāreti}} literally means, “He raises the body,” but presumably it refers to any part of the body. } %
\end{description}

\subsection*{Permutations }

If\marginnote{2.2.1} it is one who is fully ordained, and he perceives him as such, and he raises a hand against him in anger, he commits an offense entailing confession. If it is one who is fully ordained, but he is unsure of it, and he raises a hand against him in anger, he commits an offense entailing confession. If it is one who is fully ordained, but he does not perceive him as such, and he raises a hand against him in anger, he commits an offense entailing confession. 

If\marginnote{2.2.4} he raises a hand in anger against one who is not fully ordained, he commits an offense of wrong conduct. If it is one who is not fully ordained, but he perceives them as such, he commits an offense of wrong conduct. If it is one who is not fully ordained, but he is unsure of it, he commits an offense of wrong conduct. If it is one who is not fully ordained, and he does not perceive them as such, he commits an offense of wrong conduct. 

\subsection*{Non-offenses }

There\marginnote{2.2.8.1} is no offense: if he raises his hand in self-defense;  if he is insane;  if he is the first offender. 

\scendsutta{The training rule on raising a hand, the fifth, is finished. }

%
\section*{{\suttatitleacronym Bu Pc 76}{\suttatitletranslation 76. The training rule on groundless }{\suttatitleroot Amūlaka}}
\addcontentsline{toc}{section}{\tocacronym{Bu Pc 76} \toctranslation{76. The training rule on groundless } \tocroot{Amūlaka}}
\markboth{76. The training rule on groundless }{Amūlaka}
\extramarks{Bu Pc 76}{Bu Pc 76}

\subsection*{Origin story }

At\marginnote{1.1} one time when the Buddha was staying at \textsanskrit{Sāvatthī} in \textsanskrit{Anāthapiṇḍika}’s Monastery, the monks from the group of six groundlessly charged a monk with an offense entailing suspension. 

The\marginnote{1.3} monks of few desires complained and criticized them, “How could the monks from the group of six do this?” … “Is it true, monks, that you did this?” 

“It’s\marginnote{1.6} true, sir.” 

The\marginnote{1.7} Buddha rebuked them … “Foolish men, how could you do this? This will affect people’s confidence …” … “And, monks, this training rule should be recited like this: 

\subsection*{Final ruling }

\scrule{‘If a monk groundlessly charges a monk with an offense entailing suspension, he commits an offense entailing confession.’” }

\subsection*{Definitions }

\begin{description}%
\item[A: ] whoever … %
\item[Monk: ] … The monk who has been given the full ordination by a unanimous Sangha through a legal procedure consisting of one motion and three announcements that is irreversible and fit to stand—this sort of monk is meant in this case. %
\item[A monk: ] another monk. %
\item[Groundless: ] not seen, not heard, not suspected. %
\item[An offense entailing suspension: ] one of the thirteen. %
\item[Charges: ] if he accuses him or has him accused, he commits an offense entailing confession. %
\end{description}

\subsection*{Permutations }

If\marginnote{2.2.1} it is one who is fully ordained, and he perceives him as such, and he groundlessly charges him with an offense entailing suspension, he commits an offense entailing confession. If it is one who is fully ordained, but he is unsure of it, and he groundlessly charges him with an offense entailing suspension, he commits an offense entailing confession. If it is one who is fully ordained, but he does not perceive him as such, and he groundlessly charges him with an offense entailing suspension, he commits an offense entailing confession. 

If\marginnote{2.2.4} he charges someone with failure in conduct or failure in view, he commits an offense of wrong conduct. If he charges one who is not fully ordained, he commits an offense of wrong conduct. 

If\marginnote{2.2.6} it is one who is not fully ordained, but he perceives them as such, he commits an offense of wrong conduct. If it is one who is not fully ordained, but he is unsure of it, he commits an offense of wrong conduct. If it is one who is not fully ordained, and he does not perceive them as such, he commits an offense of wrong conduct. 

\subsection*{Non-offenses }

There\marginnote{2.3.1} is no offense: if he accuses someone, or has someone accused, according to what he has perceived;  if he is insane;  if he is the first offender. 

\scendsutta{The training rule on groundless, the sixth, is finished. }

%
\section*{{\suttatitleacronym Bu Pc 77}{\suttatitletranslation 77. The training rule on intentionally }{\suttatitleroot Sañcicca}}
\addcontentsline{toc}{section}{\tocacronym{Bu Pc 77} \toctranslation{77. The training rule on intentionally } \tocroot{Sañcicca}}
\markboth{77. The training rule on intentionally }{Sañcicca}
\extramarks{Bu Pc 77}{Bu Pc 77}

\subsection*{Origin story }

At\marginnote{1.1} one time when the Buddha was staying at \textsanskrit{Sāvatthī} in \textsanskrit{Anāthapiṇḍika}’s Monastery, the monks from the group of six intentionally made the monks from the group of seventeen anxious. They said, “The Buddha has laid down a rule that a person who is less than twenty years old shouldn’t be given the full ordination. And you were less than twenty when you got the full ordination. Could it be that you’re not fully ordained?” They cried. Other monks asked them why, and they said, “The monks from the group of six intentionally make us anxious.” 

The\marginnote{1.11} monks of few desires complained and criticized them, “How can the monks from the group of six do this?” … “Is it true, monks, that you do this?” 

“It’s\marginnote{1.14} true, sir.” 

The\marginnote{1.15} Buddha rebuked them … “Foolish men, how can you do this? This will affect people’s confidence …” … “And, monks, this training rule should be recited like this: 

\subsection*{Final ruling }

\scrule{‘If a monk intentionally makes a monk anxious, thinking, “In this way he will be ill at ease for at least a moment,” and he does so only for this reason and no other, he commits an offense entailing confession.’” }

\subsection*{Definitions }

\begin{description}%
\item[A: ] whoever … %
\item[Monk: ] … The monk who has been given the full ordination by a unanimous Sangha through a legal procedure consisting of one motion and three announcements that is irreversible and fit to stand—this sort of monk is meant in this case. %
\item[A monk: ] another monk. %
\item[Intentionally: ] knowing, perceiving, having intended, having decided, he transgresses. %
\item[Makes anxious: ] if he makes him anxious, saying, “It would seem that you were less than twenty years old when you were given the full ordination,” “It would seem that you have eaten at the wrong time,” “It would seem that you have drunk alcohol,” “It would seem that you have been sitting in private with a woman,” he commits an offense entailing confession. %
\item[He does so only for this reason and no other: ] there is no other reason for making him anxious. %
\end{description}

\subsection*{Permutations }

If\marginnote{2.2.1} it is one who is fully ordained, and he perceives him as such, and he intentionally makes him anxious, he commits an offense entailing confession. If it is one who is fully ordained, but he is unsure of it, and he intentionally makes him anxious, he commits an offense entailing confession. If it is one who is fully ordained, but he does not perceive him as such, and he intentionally makes him anxious, he commits an offense entailing confession. 

If\marginnote{2.2.4} he intentionally makes one who is not fully ordained anxious, he commits an offense of wrong conduct. If it is one who is not fully ordained, but he perceives them as such, he commits an offense of wrong conduct. If it is one who is not fully ordained, but he is unsure of it, he commits an offense of wrong conduct. If it is one who is not fully ordained, and he does not perceive them as such, he commits an offense of wrong conduct. 

\subsection*{Non-offenses }

There\marginnote{2.3.1} is no offense: if, not wanting to make him anxious, he says, “It would seem that you were less than twenty years old when you were given the full ordination,” “It would seem that you have eaten at the wrong time,” “It would seem that you have drunk alcohol,” “It would seem that you have been sitting in private with a woman,” and then, “Find out about it, so that you don’t get anxious later;”  if he is insane;  if he is the first offender. 

\scendsutta{The training rule on intentionally, the seventh, is finished. }

%
\section*{{\suttatitleacronym Bu Pc 78}{\suttatitletranslation 78. The training rule on eavesdropping }{\suttatitleroot Upassuti}}
\addcontentsline{toc}{section}{\tocacronym{Bu Pc 78} \toctranslation{78. The training rule on eavesdropping } \tocroot{Upassuti}}
\markboth{78. The training rule on eavesdropping }{Upassuti}
\extramarks{Bu Pc 78}{Bu Pc 78}

\subsection*{Origin story }

At\marginnote{1.1} one time when the Buddha was staying at \textsanskrit{Sāvatthī} in \textsanskrit{Anāthapiṇḍika}’s Monastery, the monks from the group of six were arguing with the good monks. The good monks said, “These monks from the group of six are shameless; it’s not possible to argue with them.” 

And\marginnote{1.6} the monks from the group of six said, “Why are you slandering us by calling us shameless?” 

“How\marginnote{1.8} did you know?” 

“We\marginnote{1.9} were eavesdropping on you.” 

The\marginnote{1.10} monks of few desires complained and criticized them, “How could the monks from the group of six eavesdrop on monks they are arguing and disputing with?” …\footnote{The Pali might be translated more literally as: “How can the monks from the group of six eavesdrop on monks who are arguing and disputing?” This, however, would not fit the context. Thus my slightly altered rendering. } “Is it true, monks, that you did this?” 

“It’s\marginnote{1.13} true, sir.” 

The\marginnote{1.14} Buddha rebuked them … “Foolish men, how could you do this? This will affect people’s confidence …” … “And, monks, this training rule should be recited like this: 

\subsection*{Final ruling }

\scrule{‘If a monk eavesdrops on monks who are arguing and disputing, thinking, “I’ll hear what they say,” and he does so only for this reason and no other, he commits an offense entailing confession.’” }

\subsection*{Definitions }

\begin{description}%
\item[A: ] whoever … %
\item[Monk: ] … The monk who has been given the full ordination by a unanimous Sangha through a legal procedure consisting of one motion and three announcements that is irreversible and fit to stand—this sort of monk is meant in this case. %
\item[On monks: ] on other monks. %
\item[Who are arguing and disputing: ] who are engaged in a legal issue. %
\item[Eavesdrops: ] if he is on his way to eavesdrop, thinking, “After hearing what they say, I’ll accuse them,” “… I’ll remind them,” “… I’ll counter accuse them,” “… I’ll counter remind them,” “… I’ll humiliate them,” he commits an offense of wrong conduct. Wherever he stands to listen, he commits an offense entailing confession. If he is walking behind someone, and he speeds up with the intention to eavesdrop, he commits an offense of wrong conduct. Wherever he stands to listen, he commits an offense entailing confession. If he is walking in front of someone, and he slows down with the intention to eavesdrop, he commits an offense of wrong conduct. Wherever he stands to eavesdrop, he commits an offense entailing confession. If he goes to where a monk who is speaking privately is standing, sitting, or lying down, he should clear his throat or make his presence known. If he does not clear his throat or make his presence known, he commits an offense entailing confession. %
\item[He does so only for this reason and no other: ] there is no other reason for eavesdropping. %
\end{description}

\subsection*{Permutations }

If\marginnote{2.2.1} it is one who is fully ordained, and he perceives him as such, and he eavesdrops on him, he commits an offense entailing confession. If it is one who is fully ordained, but he is unsure of it, and he eavesdrops on him, he commits an offense entailing confession. If it is one who is fully ordained, but he does not perceive him as such, and he eavesdrops on him, he commits an offense entailing confession. 

If\marginnote{2.2.4} he eavesdrops on one who is not fully ordained, he commits an offense of wrong conduct. If it is one who is not fully ordained, but he perceives them as such, he commits an offense of wrong conduct. If it is one who is not fully ordained, but he is unsure of it, he commits an offense of wrong conduct. If it is one who is not fully ordained, and he does not perceive them as such, he commits an offense of wrong conduct. 

\subsection*{Non-offenses }

There\marginnote{2.3.1} is no offense: if he goes, thinking, “After hearing what they say, I’ll hold back,” “… I’ll abstain,” “… I’ll resolve it,” “… I’ll free myself;”  if he is insane;  if he is the first offender. 

\scendsutta{The training rule on eavesdropping, the eighth, is finished. }

%
\section*{{\suttatitleacronym Bu Pc 79}{\suttatitletranslation 79. The training rule on obstructing a legal procedure }{\suttatitleroot Kammapaṭibāhana}}
\addcontentsline{toc}{section}{\tocacronym{Bu Pc 79} \toctranslation{79. The training rule on obstructing a legal procedure } \tocroot{Kammapaṭibāhana}}
\markboth{79. The training rule on obstructing a legal procedure }{Kammapaṭibāhana}
\extramarks{Bu Pc 79}{Bu Pc 79}

\subsection*{Origin story }

At\marginnote{1.1} one time when the Buddha was staying at \textsanskrit{Sāvatthī} in \textsanskrit{Anāthapiṇḍika}’s Monastery, the monks from the group of six were misbehaving, but when a legal procedure was being done against any one of them, they would object. 

On\marginnote{1.3} one occasion the Sangha had gathered on some business. The monks from the group of six were busy making robes and so they gave their consent to one among them. When the monks saw that only one monk from the group of six had come, they did a legal procedure against him. When he returned to the monks from the group of six, they asked him, “What did the Sangha do?” 

“It\marginnote{1.10} did a legal procedure against me.” 

“We\marginnote{1.11} didn’t give our consent for that. If we had known that a procedure would be done against you, we wouldn’t have given our consent.” 

The\marginnote{1.14} monks of few desires complained and criticized them, “How could the monks from the group of six give their consent to legitimate legal procedures and then criticize them afterwards?” … “Is it true, monks, that you did this?” 

“It’s\marginnote{1.17} true, sir.” 

The\marginnote{1.18} Buddha rebuked them … “Foolish men, how could you do this? This will affect people’s confidence …” … “And, monks, this training rule should be recited like this: 

\subsection*{Final ruling }

\scrule{‘If a monk gives his consent to legitimate legal procedures, and then criticizes them afterwards, he commits an offense entailing confession.’” }

\subsection*{Definitions }

\begin{description}%
\item[A: ] whoever … %
\item[Monk: ] … The monk who has been given the full ordination by a unanimous Sangha through a legal procedure consisting of one motion and three announcements that is irreversible and fit to stand—this sort of monk is meant in this case. %
\item[A legitimate legal procedure: ] a legal procedure consisting of getting permission, a legal procedure consisting of one motion, a legal procedure consisting of one motion and one announcement, a legal procedure consisting of one motion and three announcements; done according to the Teaching, according to the Monastic Law, according to the Teacher’s instruction. This is called a “legitimate legal procedure”. If he gives his consent, and then criticizes the procedure, he commits an offense entailing confession. %
\end{description}

\subsection*{Permutations }

If\marginnote{2.2.1} it is a legitimate legal procedure, and he perceives it as such, and he criticizes it after giving his consent, he commits an offense entailing confession. If it is a legitimate legal procedure, but he is unsure of it, and he criticizes it after giving his consent, he commits an offense of wrong conduct. If it is a legitimate legal procedure, but he perceives it as illegitimate, and he criticizes it after giving his consent, there is no offense. 

If\marginnote{2.2.4} it is an illegitimate legal procedure, but he perceives it as legitimate, he commits an offense of wrong conduct. If it is an illegitimate legal procedure, but he is unsure of it, he commits an offense of wrong conduct. If it is an illegitimate legal procedure, and he perceives it as such, there is no offense. 

\subsection*{Non-offenses }

There\marginnote{2.3.1} is no offense: if he criticizes it because he knows that the legal procedure was illegitimate, done by an incomplete assembly, or done against one who did not deserve it;  if he is insane;  if he is the first offender. 

\scendsutta{The training rule on obstructing a legal procedure, the ninth, is finished. }

%
\section*{{\suttatitleacronym Bu Pc 80}{\suttatitletranslation 80. The training rule on leaving without giving consent }{\suttatitleroot Chandaṁadatvāgamana}}
\addcontentsline{toc}{section}{\tocacronym{Bu Pc 80} \toctranslation{80. The training rule on leaving without giving consent } \tocroot{Chandaṁadatvāgamana}}
\markboth{80. The training rule on leaving without giving consent }{Chandaṁadatvāgamana}
\extramarks{Bu Pc 80}{Bu Pc 80}

\subsection*{Origin story }

At\marginnote{1.1} one time when the Buddha was staying at \textsanskrit{Sāvatthī} in \textsanskrit{Anāthapiṇḍika}’s Monastery, the Sangha had gathered on some business. The monks from the group of six were busy making robes and so they gave their consent to one among them. 

When\marginnote{1.4} the Sangha was ready to do the legal procedure for which it had gathered, it put forward a motion. That monk from the group of six thought, “This is just how they do legal procedures against us one by one, but against who will you do this one?” and without giving his consent, he got up from his seat and left. 

The\marginnote{1.7} monks of few desires complained and criticized him, “When the Sangha is in the middle of a discussion, how could that monk get up from his seat and leave without giving his consent?” … “Is it true, monk, that you did this?” 

“It’s\marginnote{1.10} true, sir.” 

The\marginnote{1.11} Buddha rebuked him … “Foolish man, how could you do this? This will affect people’s confidence …” … “And, monks, this training rule should be recited like this: 

\subsection*{Final ruling }

\scrule{‘When the Sangha is in the middle of a discussion, if a monk gets up from his seat and leaves without first giving his consent, he commits an offense entailing confession.’” }

\subsection*{Definitions }

\begin{description}%
\item[A: ] whoever … %
\item[Monk: ] … The monk who has been given the full ordination by a unanimous Sangha through a legal procedure consisting of one motion and three announcements that is irreversible and fit to stand—this sort of monk is meant in this case. %
\item[When the Sangha is in the middle of a discussion: ] when the topic has been announced but the discussion has not yet been concluded, or when the motion has been put forward, or when the announcement is still under way. %
\item[Gets up from his seat and leaves without first giving his consent: ] if he leaves, thinking, “How may this legal procedure be disturbed?” or “How may this legal procedure be done by an incomplete assembly?” or “How may this legal procedure not be done?” then he commits an offense of wrong conduct.\footnote{I read the Pali as if \textit{\textsanskrit{kathaṁ} \textsanskrit{idaṁ} \textsanskrit{kammaṁ}} is to be distributed over the next three phrases: \textit{\textsanskrit{kathaṁ} \textsanskrit{idaṁ} \textsanskrit{kammaṁ} \textsanskrit{kuppaṁ} assa, \textsanskrit{kathaṁ} \textsanskrit{idaṁ} \textsanskrit{kammaṁ} \textsanskrit{vaggaṁ} assa, \textsanskrit{kathaṁ} \textsanskrit{idaṁ} \textsanskrit{kammaṁ}  na \textsanskrit{kareyyā}}. } If he is in the process of going beyond arm’s reach of the gathering, he commits an offense of wrong conduct. When he has gone beyond arm’s reach, he commits an offense entailing confession. %
\end{description}

\subsection*{Permutations }

If\marginnote{2.2.1} it is a legitimate legal procedure, and he perceives it as such, and he gets up from his seat and leaves without first giving his consent, he commits an offense entailing confession. If it is a legitimate legal procedure, but he is unsure of it, and he gets up from his seat and leaves without first giving his consent, he commits an offense of wrong conduct. If it is a legitimate legal procedure, but he perceives it as illegitimate, and he gets up from his seat and leaves without first giving his consent, there is no offense. 

If\marginnote{2.2.4} it is an illegitimate legal procedure, but he perceives it as legitimate, he commits an offense of wrong conduct. If it is an illegitimate legal procedure, but he is unsure of it, he commits an offense of wrong conduct. If it is an illegitimate legal procedure, and he perceives it as such, there is no offense. 

\subsection*{Non-offenses }

There\marginnote{2.3.1} is no offense: if he leaves because he thinks there will be quarrels or disputes in the Sangha;  if he leaves because he thinks there will be a fracture or schism in the Sangha;  if he leaves because he thinks the legal procedure will be illegitimate, done by an incomplete assembly, or done against one who does not deserve it;  if he leaves because he is sick;  if he leaves because he has to take care of someone who is sick;  if he leaves because he needs to relieve himself;  if he leaves with the intention to return, and not because he wants to invalidate the legal procedure;  if he is insane;  if he is the first offender. 

\scendsutta{The training rule on leaving without giving consent, the tenth, is finished. }

%
\section*{{\suttatitleacronym Bu Pc 81}{\suttatitletranslation 81. The training rule on what is worn out }{\suttatitleroot Dubbala}}
\addcontentsline{toc}{section}{\tocacronym{Bu Pc 81} \toctranslation{81. The training rule on what is worn out } \tocroot{Dubbala}}
\markboth{81. The training rule on what is worn out }{Dubbala}
\extramarks{Bu Pc 81}{Bu Pc 81}

\subsection*{Origin story }

At\marginnote{1.1} one time when the Buddha was staying at \textsanskrit{Rājagaha} in the Bamboo Grove, Venerable Dabba the Mallian, who was the assigner of dwellings and the designator of meals, had a robe that was worn out. Just then the Sangha had obtained a robe, which it gave to Dabba. The monks from the group of six complained and criticized it, “The monks are diverting the Sangha’s material things according to friendship.” 

The\marginnote{1.8} monks of few desires complained and criticized them, “How could the monks from the group of six give out a robe as part of a unanimous Sangha and then criticize it afterwards?” … “Is it true, monks, that you did this?” 

“It’s\marginnote{1.11} true, sir.” 

The\marginnote{1.12} Buddha rebuked them … “Foolish men, how could you do this? This will affect people’s confidence …” … “And, monks, this training rule should be recited like this: 

\subsection*{Final ruling }

\scrule{‘If a monk gives out a robe as part of a unanimous Sangha and then criticizes it afterwards, saying, “The monks are diverting the Sangha’s material things according to friendship,” he commits an offense entailing confession.’” }

\subsection*{Definitions }

\begin{description}%
\item[A: ] whoever … %
\item[Monk: ] … The monk who has been given the full ordination by a unanimous Sangha through a legal procedure consisting of one motion and three announcements that is irreversible and fit to stand—this sort of monk is meant in this case. %
\item[A unanimous Sangha: ] belonging to the same Buddhist sect and staying within the same monastery zone. %
\item[A robe: ] one of the six kinds of robe-cloth, but not smaller than what can be assigned to another.\footnote{The six are linen, cotton, silk, wool, sunn hemp, and hemp; see \href{https://suttacentral.net/pli-tv-kd8/en/brahmali\#3.1.6}{Kd 8:3.1.6}. According to \href{https://suttacentral.net/pli-tv-kd8/en/brahmali\#21.1.4}{Kd 8:21.1.4} the size referred to here is no smaller than 8 by 4 \textit{\textsanskrit{sugataṅgula}}, “standard fingerbreadths”. For an explanation of the idea of \textit{\textsanskrit{vikappanā}}, see Appendix of Technical Terms. } %
\item[Gives out: ] gives out himself. %
\item[According to friendship: ] according to friendship, according to companionship, according to who one is devoted to, according to being a co-student, according to being a co-pupil. %
\item[The Sangha’s: ] given to the Sangha, given up to the Sangha. %
\item[Material things: ] robes, almsfood, a dwelling, and medicinal supplies; even a bit of bath powder, a tooth cleaner, or a piece of string. %
\item[Criticizes it afterwards: ] when robe-cloth has been given to someone who is fully ordained and who is the assigner of dwellings or the designator of meals or the distributor of congee or the distributor of fruit or the distributor of fresh foods or the distributor of minor requisites, and he has been appointed by the Sangha as such, then if a monk criticizes the giving, he commits an offense entailing confession.\footnote{These are all formally appointed officers of the Sangha. | According to Sp 2.615 (commenting on the \textit{sekhiya} rules) \textit{khajjaka} refers to all fresh foods: \textit{ettha \textsanskrit{mūlakhādanīyādi} \textsanskrit{sabbaṁ} \textsanskrit{gahetabbaṁ}} , “Here the fresh foods which are roots, etc., may all be taken.” } %
\end{description}

\subsection*{Permutations }

If\marginnote{2.2.1} it is a legitimate legal procedure, and he perceives it as such, and he criticizes the giving of robe-cloth, he commits an offense entailing confession. If it is a legitimate legal procedure, but he is unsure of it, and he criticizes the giving of robe-cloth, he commits an offense entailing confession. If it is a legitimate legal procedure, but he perceives it as illegitimate, and he criticizes the giving of robe-cloth, he commits an offense entailing confession. 

If\marginnote{2.2.4} he criticizes the giving of another requisite, he commits an offense of wrong conduct. When robe-cloth or another requisite has been given to someone who is fully ordained and who is the assigner of dwellings or the designator of meals or the distributor of congee or the distributor of fruit or the distributor of fresh foods or the distributor of minor requisites, but he has not been appointed by the Sangha as such, then if a monk criticizes the giving, he commits an offense of wrong conduct. When robe-cloth or another requisite has been given to someone who is not fully ordained and who is the assigner of dwellings or the designator of meals or the distributor of congee or the distributor of fruit or the distributor of fresh foods or the distributor of minor requisites, whether he has been appointed by the Sangha as such or not, then if a monk criticizes the giving, he commits an offense of wrong conduct. 

If\marginnote{2.2.7} it is an illegitimate legal procedure, but he perceives it as legitimate, he commits an offense of wrong conduct. If it is an illegitimate legal procedure, but he is unsure of it, he commits an offense of wrong conduct. If it is an illegitimate legal procedure, and he perceives it as such, there is no offense. 

\subsection*{Non-offenses }

There\marginnote{2.3.1} is no offense: if he criticizes one who regularly acts out of favoritism, ill will, confusion, or fear, saying, “What’s the point of giving it to him—he’ll ruin it or use it inappropriately;”  if he is insane;  if he is the first offender. 

\scendsutta{The training rule on what is worn out, the eleventh, is finished. }

%
\section*{{\suttatitleacronym Bu Pc 82}{\suttatitletranslation 82. The training rule on diverting }{\suttatitleroot Pariṇāmana}}
\addcontentsline{toc}{section}{\tocacronym{Bu Pc 82} \toctranslation{82. The training rule on diverting } \tocroot{Pariṇāmana}}
\markboth{82. The training rule on diverting }{Pariṇāmana}
\extramarks{Bu Pc 82}{Bu Pc 82}

\subsection*{Origin story }

At\marginnote{1.1} one time when the Buddha was staying at \textsanskrit{Sāvatthī} in \textsanskrit{Anāthapiṇḍika}’s Monastery, an association had prepared a meal together with robe-cloth for the Sangha. They said, “After giving the food, we’ll offer the robe-cloth.” 

But\marginnote{1.4} the monks from the group of six went to that association and said, “Please give the robe-cloth to these monks.” 

“Venerables,\marginnote{1.7} we can’t do that. We’ve prepared our annual alms-offering together with robe-cloth for the Sangha.” 

“The\marginnote{1.9} Sangha has many donors and supporters. But since these monks are staying here, they are looking to you for support. If you don’t give to them, who will? So give them the robe-cloth.” Being pressured by the monks from the group of six, that association gave the prepared robe-cloth to those monks and served the food to the Sangha. 

The\marginnote{1.14} monks who knew that a meal together with robe-cloth had been prepared for the Sangha, but not that the robe-cloth had been given to those monks from the group of six, said, “You may offer the robe-cloth.” 

“There\marginnote{1.16} isn’t any. The robe-cloth we had prepared was diverted by the monks from the group of six.” 

The\marginnote{1.18} monks of few desires complained and criticized those monks, “How could the monks from the group of six divert to an individual things they knew were intended for the Sangha?” … “Is it true, monks, that you did this?” 

“It’s\marginnote{1.21} true, sir.” 

The\marginnote{1.22} Buddha rebuked them … “Foolish men, how could you do this? This will affect people’s confidence …” … “And, monks, this training rule should be recited like this: 

\subsection*{Final ruling }

\scrule{‘If a monk diverts to an individual material support that he knows was intended for the Sangha, he commits an offense entailing confession.’” }

\subsection*{Definitions }

\begin{description}%
\item[A: ] whoever … %
\item[Monk: ] … The monk who has been given the full ordination by a unanimous Sangha through a legal procedure consisting of one motion and three announcements that is irreversible and fit to stand—this sort of monk is meant in this case. %
\item[He knows: ] he knows by himself or others have told him or the donor has told him.\footnote{The meaning of the last of these three ways of knowing, \textit{so \textsanskrit{vā} \textsanskrit{āroceti}}, is not clear. CPD suggests: “\textit{sa (\textsanskrit{sā}) \textsanskrit{āroceti}} (?). Perhaps this last form is conformable to sa. \textit{\textsanskrit{ārocayate}} med. caus. in the meaning: he or she makes inquiries (of others).” However, this does not fit with the parallel usage at \href{https://suttacentral.net/pli-tv-bu-vb-pc29/en/brahmali\#3.1.6}{Bu Pc 29:3.1.6} where the text says that she tells (\textit{\textsanskrit{sā} \textsanskrit{vā} \textsanskrit{āroceti}}) him, presumably referring to the nun telling the monk. In this case \textit{\textsanskrit{āroceti}} cannot refer to the monk making inquiries. The commentaries are silent, and I therefore assume that a straightforward meaning is the most likely one. I would suggest, then, that it simply refers to the donor telling the monk directly. } %
\item[For the Sangha: ] given to the Sangha, given up to the Sangha. %
\item[Material support: ] robe-cloth, almsfood, a dwelling, and medicinal supplies; even a bit of bath powder, a tooth cleaner, or a piece of string. %
\item[Intended: ] if they have said, “We’ll give,” “We’ll prepare,” and he diverts it to an individual, he commits an offense entailing confession. %
\end{description}

\subsection*{Permutations }

If\marginnote{2.2.1} it was intended for the Sangha and he perceives it as such, and he diverts it to an individual, he commits an offense entailing confession. If it was intended for the Sangha, but he is unsure of it, and he diverts it to an individual, he commits an offense of wrong conduct. If it was intended for the Sangha, but he does not perceive it as such, and he diverts it to an individual, there is no offense. 

If\marginnote{2.2.4} it was intended for one Sangha and he diverts it to another Sangha or to a shrine, he commits an offense of wrong conduct. If it was intended for one shrine and he diverts it to another shrine or to a sangha or to an individual, he commits an offense of wrong conduct. If it was intended for an individual and he diverts it to another individual or to a sangha or to a shrine, he commits an offense of wrong conduct. 

If\marginnote{2.2.7} it was not intended for the Sangha, but he perceives it as such, he commits an offense of wrong conduct. If it was not intended for the Sangha, but he is unsure of it, he commits an offense of wrong conduct. If it was not intended for the Sangha and he does not perceive it as such, there is no offense. 

\subsection*{Non-offenses }

There\marginnote{2.3.1} is no offense: if being asked, “Where may we give?” he says, “Give where your gift will be useful,” “… where it goes toward repairs,” “… where it will last for a long time,” “… where you feel inspired;”  if he is insane;  if he is the first offender. 

\scendsutta{The training rule on diverting, the twelfth, is finished. }

\scendvagga{The eighth subchapter on legitimately is finished. }

\scuddanaintro{This is the summary: }

\begin{scuddana}%
“Legitimately,\marginnote{2.3.9} and disparaging, \\
Deception, hitting; \\
Raising a hand, and groundless, \\
And intentionally, eavesdropping; \\
Obstructing, and consent, \\
And Dabba, diverting.” 

%
\end{scuddana}

%
\section*{{\suttatitleacronym Bu Pc 83}{\suttatitletranslation 83. The training rule on royal compounds }{\suttatitleroot Antepura}}
\addcontentsline{toc}{section}{\tocacronym{Bu Pc 83} \toctranslation{83. The training rule on royal compounds } \tocroot{Antepura}}
\markboth{83. The training rule on royal compounds }{Antepura}
\extramarks{Bu Pc 83}{Bu Pc 83}

\subsection*{Origin story }

At\marginnote{1.1.1} one time when the Buddha was staying at \textsanskrit{Sāvatthī} in \textsanskrit{Anāthapiṇḍika}’s Monastery, King Pasenadi of Kosala told his park-keeper, “Go and clean up the park; I’ll be going there.” 

“Yes,\marginnote{1.1.3} sir.” While cleaning the park, he saw the Buddha seated at the foot of a tree. He then went to King Pasenadi and said, “The park is clean, sir, but the Buddha is sitting there.” “Marvelous! I’ll visit him.” 

The\marginnote{1.1.6} king went to the park and approached the Buddha, but just then a lay follower was seated there. When the king saw him, he became fearful and stopped. But he considered, “This man isn’t likely to be bad, since he’s visiting the Buddha,” and so he approached the Buddha, bowed, and sat down. But when that lay follower, because of his respect for the Buddha, neither bowed down to the king nor stood up for him, the king became annoyed. The Buddha realized what was happening and said to the king, “Great king, this lay follower is learned, a master of the tradition, and he’s free from sensual desire.” 

The\marginnote{1.1.11} king thought, “This lay follower doesn’t deserve to be in an inferior position, since even the Buddha praises him.” And he said to that lay follower, “Please say what you want.” 

“Thank\marginnote{1.1.13} you, sir.”\footnote{\textit{\textsanskrit{Suṭṭhu}, \textsanskrit{devāti}}, literally, “It is good, sir.” The exact contextual meaning is not clear to me. } 

The\marginnote{1.1.14} Buddha then instructed, inspired, and gladdened King Pasenadi with a teaching, after which the king got up from his seat, bowed down, circumambulated the Buddha with his right side toward him, and left. 

Soon\marginnote{1.1.15} afterwards King Pasenadi was up in his finest stilt house, when he saw that lay follower walking along the street, holding a sunshade. He summoned him and said, “You are a learned Buddhist, a master of the tradition. Please teach my harem.” 

“Whatever\marginnote{1.1.18} I know, sir, I know because of the monks. They should teach the harem.” 

Knowing\marginnote{1.2.1} that the lay follower was right, the king went to the Buddha, bowed, sat down, and said, “Sir, please ask a monk to teach my harem.” The Buddha then instructed, inspired, and gladdened King Pasenadi with a teaching, after which the king got up from his seat, paid his respects as before, and left. 

Soon\marginnote{1.2.4} afterwards, the Buddha said to Venerable Ānanda, “Well then, Ānanda, teach the king’s harem.” 

“Yes,\marginnote{1.2.5} sir.” And from time to time he would enter the harem and teach. 

Then,\marginnote{1.2.6} after robing up in the morning, Ānanda took his bowl and robe and went to King Pasenadi’s house. 

On\marginnote{1.2.7} that occasion the king was in bed with Queen \textsanskrit{Mallikā}. The queen saw Ānanda coming and she quickly got up, but her burnished golden dressing gown fell off. Ānanda turned around right there and returned to the monastery. And he told the monks what had happened. 

The\marginnote{1.2.10} monks of few desires complained and criticized him, “How could Venerable Ānanda enter the royal compound without first being announced?” … “Is it true, Ānanda, that you did this?” 

“It’s\marginnote{1.2.13} true, sir.” 

The\marginnote{1.2.14} Buddha rebuked him … “Ānanda, how could you do this? This will affect people’s confidence …” … Having rebuked him … the Buddha gave a teaching and addressed the monks: 

“Monks,\marginnote{1.3.1} there are these ten dangers of entering a royal compound.\footnote{For the rendering of \textit{antepura} as “royal compound”, see Appendix of Technical Terms. } What ten? 

It\marginnote{1.3.3} may be that a monk enters where the king is sitting with his queen. The queen smiles when she sees the monk or the monk smiles when he sees the queen. The king thinks, ‘Surely they’ve done it, or they’re going to.’ 

Again,\marginnote{1.3.9} because the king is very busy, he does not remember having slept with a certain woman, yet she becomes pregnant because of that. The king thinks, ‘Only the monk enters here. Is he responsible for this?’ 

Again,\marginnote{1.3.16} a gem disappears from the royal compound. The king thinks, ‘Only the monk enters here. Is he responsible for this?’ 

Again,\marginnote{1.3.21} secret discussions in the royal compound are spread outside. The king thinks, ‘Only the monk enters here. Is he responsible for this?’ 

Again,\marginnote{1.3.26} in the royal compound a father longs for his son, or a son longs for his father.\footnote{This may refer to illegitimate offspring, and the monk acting as an informant. } They think, ‘Only the monk enters here. Is he responsible for this?’ 

Again,\marginnote{1.3.31} the king promotes someone. Those who dislike this think, ‘The king is close to the monk. Is he responsible for this?’ 

Again,\marginnote{1.3.36} the king demotes someone. Those who dislike this think, ‘The king is close to the monk. Is he responsible for this?’ 

Again,\marginnote{1.3.41} the king sends out the army at an inappropriate time. Those who dislike this think, ‘The king is close to the monk. Is he responsible for this?’ 

Again,\marginnote{1.3.46} after sending out the army at an appropriate time, the king orders it to turn back while still en route. Those who dislike this think, ‘The king is close to the monk. Is he responsible for this?’ 

Again,\marginnote{1.3.51} a royal compound is crowded with elephants, horses, and chariots, as well as enticing sights, sounds, smells, tastes, and tangibles that are not suitable for a monk. 

Monks,\marginnote{1.3.53} these are the ten dangers of entering a royal compound.” 

Then,\marginnote{1.3.54} after rebuking Ānanda in many ways, the Buddha spoke in dispraise of being difficult to support … “And, monks, this training rule should be recited like this: 

\subsection*{Final ruling }

\scrule{‘If a monk, without first being announced, crosses the threshold to the bedroom of a consecrated aristocrat king, when both the king and the queen are present, he commits an offense entailing confession.’”\footnote{The Pali just says threshold: “The threshold of a consecrated king”. To give the rule a more definitive meaning, I have added “bedroom” from the word commentary below. } }

\subsection*{Definitions }

\begin{description}%
\item[A: ] whoever … %
\item[Monk: ] … The monk who has been given the full ordination by a unanimous Sangha through a legal procedure consisting of one motion and three announcements that is irreversible and fit to stand—this sort of monk is meant in this case. %
\item[Aristocrat: ] well-born on both his mother’s side and his father’s side, pure in descent, irreproachable and impeccable with respect to birth going back eight generations of male ancestors.\footnote{\textit{\textsanskrit{Pitāmashayuga}} refers to “grandfather generation”, thus a total of eight generations back. } %
\item[Consecrated: ] consecrated with the aristocratic consecration. %
\item[The king is present: ] the king has not left the bedroom. %
\item[The queen is present: ] the queen has not left the bedroom. Or both have not left. %
\item[Without first being announced: ] without first having informed. %
\item[The threshold: ] the threshold to the bedroom is what is meant. %
\item[The bedroom: ] wherever a king’s bed is prepared, even if just enclosed by a cloth screen. %
\item[Crosses the threshold to the bedroom: ] if he crosses the threshold with the first foot, he commits an offense of wrong conduct. If he crosses with the second foot, he commits an offense entailing confession. %
\end{description}

\subsection*{Permutations }

If\marginnote{2.2.1} he has not been announced, and he perceives that he has not, and he crosses the threshold to the bedroom, he commits an offense entailing confession. If he has not been announced, but he is unsure of it, and he crosses the threshold to the bedroom, he commits an offense entailing confession. If he has not been announced, but he perceives that he has, and he crosses the threshold to the bedroom, he commits an offense entailing confession. 

If\marginnote{2.2.4} he has been announced, but he perceives that he has not, he commits an offense of wrong conduct. If he has been announced, but he is unsure of it, he commits an offense of wrong conduct. If he has been announced, and he perceives that he has, there is no offense. 

\subsection*{Non-offenses }

There\marginnote{2.3.1} is no offense: if he has been announced;  if it is not an aristocrat king;  if the king has not been consecrated with the aristocratic consecration;  if the king has left the bedroom;  if the queen has left the bedroom;  if they both have left;  if it is not a bedroom;  if he is insane;  if he is the first offender. 

\scendsutta{The training rule on royal compounds, the first, is finished. }

%
\section*{{\suttatitleacronym Bu Pc 84}{\suttatitletranslation 84. The training rule on precious things }{\suttatitleroot Ratana}}
\addcontentsline{toc}{section}{\tocacronym{Bu Pc 84} \toctranslation{84. The training rule on precious things } \tocroot{Ratana}}
\markboth{84. The training rule on precious things }{Ratana}
\extramarks{Bu Pc 84}{Bu Pc 84}

\subsection*{Origin story }

\subsubsection*{First sub-story }

At\marginnote{1.1} one time when the Buddha was staying at \textsanskrit{Sāvatthī} in \textsanskrit{Anāthapiṇḍika}’s Monastery, a monk was bathing in the river \textsanskrit{Aciravatī}, when a brahmin, too, came there to bathe. He deposited a bag with five hundred coins on the ground, bathed, forgot about the bag, and left. The monk thought, “This is the bag belonging to that brahmin; it wouldn’t be nice if it got lost,” and he picked it up. 

Soon\marginnote{1.4} the brahmin remembered. He hurried back and asked that monk, “Sir, did you see my bag?” 

Saying,\marginnote{1.5} “I did indeed,” he gave it to him. 

The\marginnote{1.6} brahmin thought, “How can I avoid giving a reward to this monk?” And he said, “I didn’t have five hundred coins, I had a thousand!” And he seized hold of that monk. 

After\marginnote{1.8} being released, that monk went to the monastery and told the monks what had happened. The monks of few desires complained and criticized him, “How could a monk pick up precious things?” … “Is it true, monk, that you did this?” 

“It’s\marginnote{1.12} true, sir.” 

The\marginnote{1.13} Buddha rebuked him … “Foolish man, how could you do this? This will affect people’s confidence …” … “And, monks, this training rule should be recited like this: 

\subsubsection*{First preliminary ruling }

\scrule{‘If a monk picks up something precious or something regarded as precious, or he has it picked up, he commits an offense entailing confession.’” }

In\marginnote{1.18} this way the Buddha laid down this training rule for the monks. 

\subsubsection*{Second sub-story }

Soon\marginnote{2.1} afterwards there was a celebration in \textsanskrit{Sāvatthī}. People were going to the park all dressed up, as did \textsanskrit{Visākhā} \textsanskrit{Migāramātā}. As she was leaving her village, she thought, “What will I do when I get to the park? Why don’t I pay a visit to the Buddha!” She then removed all her ornaments, bound them in a bundle with her upper robe, and gave it to her slave girl, saying, “Listen, look after this bundle.” 

\textsanskrit{Visākhā}\marginnote{2.5} then went to the Buddha, bowed, and sat down. And the Buddha instructed, inspired, and gladdened her with a teaching, after which she got up from her seat, bowed down, circumambulated him with her right side toward him, and left. And the slave girl left too, forgetting the bundle. 

The\marginnote{2.8} monks saw it and told the Buddha. “Well then, monks, pick it up and put it aside.” Soon afterwards the Buddha gave a teaching and addressed the monks: “Within a monastery, monks, you should pick up precious things or what’s regarded as precious, or have it picked up, and then put it aside with the thought, ‘Whoever owns it will come and get it.’ And so, monks, this training rule should be recited like this: 

\subsubsection*{Second preliminary ruling }

\scrule{‘If a monk picks up something precious or something regarded as precious, or he has it picked up, except within a monastery, he commits an offense entailing confession.’” }

In\marginnote{2.13} this way the Buddha laid down this training rule for the monks. 

\subsubsection*{Third sub-story }

At\marginnote{3.1} that time the householder \textsanskrit{Anāthapiṇḍika} had a whole village working for him in the country of \textsanskrit{Kāsi}, and he had told an apprentice there that if the monks arrive he should make them a meal. Soon afterwards a number of monks were wandering in the country of \textsanskrit{Kāsi}, when they came to that village. When that man saw them coming, he approached them, bowed, and said, “Venerables, please accept a meal from \textsanskrit{Anāthapiṇḍika} tomorrow.” The monks accepted by remaining silent. 

The\marginnote{3.6} following morning, after having various kinds of fine foods prepared, he had the monks informed that it was time for the meal. He removed a ring from his finger and then offered the food to the monks, saying, “Venerables, please leave after you’ve eaten. I have to go to work.” And he left, forgetting his ring. 

The\marginnote{3.8} monks saw it and said, “If we just go, this ring will be lost,” and so they stayed right there. When that man returned from work, he saw the monks and said to them, “Why are you still here?” And the monks told him what had happened. 

Those\marginnote{3.11} monks then went to \textsanskrit{Sāvatthī} where they told the monks, who in turn told the Buddha. 

After\marginnote{3.13} giving a teaching, the Buddha addressed the monks: “Within a monastery, monks, or inside a lodging, you should pick up precious things or what’s regarded as precious, or have it picked up, and then put it aside with the thought, ‘Whoever owns it will come and get it.’ And so, monks, this training rule should be recited like this: 

\subsection*{Final ruling }

\scrule{‘If a monk picks up something precious or something regarded as precious, or he has it picked up, except within a monastery or inside a lodging, he commits an offense entailing confession. If he picks up something precious or something regarded as precious, or he has it picked up, within a monastery or inside a lodging, he should put it aside with the thought, “Whoever owns it will come and get it.” This is the proper procedure.’” }

\subsection*{Definitions }

\begin{description}%
\item[A: ] whoever … %
\item[Monk: ] … The monk who has been given the full ordination by a unanimous Sangha through a legal procedure consisting of one motion and three announcements that is irreversible and fit to stand—this sort of monk is meant in this case. %
\item[Something precious: ] a pearl, a gem, a beryl, mother-of-pearl, quartz, a coral, silver, gold, a ruby, a cat’s eye.\footnote{“Beryl” renders \textit{\textsanskrit{veḷuriya}}. Sp-\textsanskrit{ṭ} 1.281: \textit{\textsanskrit{Veḷuriyoti} \textsanskrit{vaṁsavaṇṇamaṇi}}, “The bamboo-colored gem is called \textit{\textsanskrit{veḷuriya}}.” According to PED \textit{\textsanskrit{veḷuriya}} is lapis lazuli, which cannot be correct because lapis lazuli is blue. } %
\item[Something regarded as precious: ] whatever people regard as valuable or useful—this is called “regarded as precious”. %
\item[Except within a monastery or inside a lodging: ] apart from within a monastery or inside a lodging. %
\item[Within a monastery: ] if the monastery is enclosed, then within the enclosure. If the monastery is unenclosed, then in the vicinity. %
\item[Inside a lodging: ] if the lodging is enclosed, then within the enclosure. If the lodging is unenclosed, then in the vicinity. %
\item[Picks up: ] if he takes hold of it himself, he commits an offense entailing confession. %
\item[Has picked up: ] if he gets someone else to take hold of it, he commits an offense entailing confession. %
\item[If he picks up something precious or something regarded as precious, or he has it picked up, within a monastery or inside a lodging, he should put it aside: ] after taking note of its appearance or distinguishing marks, he should put it aside, and then make an announcement: “Whoever has lost anything should come.” If anyone comes, they should be told, “Please describe it.” If they rightly describe its appearance or distinguishing marks, it should be given to them. If they do not, they should be told, “Keep looking.” If that monk is leaving that monastery, he should first place that item into the hands of suitable monks there. If there are no suitable monks, he should place it into the hands of suitable householders there. %
\item[This is the proper procedure: ] this is the right method. %
\end{description}

\subsection*{Non-offenses }

There\marginnote{4.2.1} is no offense: if he picks up something precious or something regarded as precious, or he has it picked up, within a monastery or inside a lodging, and then puts it aside with the thought, “Whoever owns it will come and get it;” if he takes something regarded as precious on trust or he borrows it or he perceives it as discarded;\footnote{“Taking on trust” refers to a situation where you have an agreement with a close friend that you may take their belongings on trust. The conditions for taking on trust are set out at \href{https://suttacentral.net/pli-tv-kd8/en/brahmali\#19.1.5}{Kd 8:19.1.5}. }  if he is insane;  if he is the first offender. 

\scendsutta{The training rule on precious things, the second, is finished. }

%
\section*{{\suttatitleacronym Bu Pc 85}{\suttatitletranslation 85. The training rule on entering an inhabited area at the wrong time }{\suttatitleroot Vikālagāmappavisana}}
\addcontentsline{toc}{section}{\tocacronym{Bu Pc 85} \toctranslation{85. The training rule on entering an inhabited area at the wrong time } \tocroot{Vikālagāmappavisana}}
\markboth{85. The training rule on entering an inhabited area at the wrong time }{Vikālagāmappavisana}
\extramarks{Bu Pc 85}{Bu Pc 85}

\subsection*{Origin story }

\subsubsection*{First sub-story }

At\marginnote{1.1} one time the Buddha was staying at \textsanskrit{Sāvatthī} in the Jeta Grove, \textsanskrit{Anāthapiṇḍika}’s Monastery. At that time the monks from the group of six entered the village at the wrong time, sat down in the public meeting hall, and talked about all sorts of worldly things: about kings, gangsters, and officials; about armies, dangers, and battles; about food, drink, clothes, and beds; about garlands and perfumes; about relatives, vehicles, villages, towns, and countries; about women and heroes; gossip; about the departed; about various trivialities; about the world and the sea; about the various kinds of existence. 

People\marginnote{1.4} complained and criticized them, “How can the Sakyan monastics enter the village at the wrong time, sit down in the public meeting hall, and talk about such worldly things? They’re just like householders who indulge in worldly pleasures!” 

The\marginnote{1.8} monks heard the complaints of those people, and the monks of few desires complained and criticized those monks, “How can the monks from the group of six do this?” … “Is it true, monks, that you do this?” 

“It’s\marginnote{1.16} true, sir.” 

The\marginnote{1.17} Buddha rebuked them … “Foolish men, how can you do this? This will affect people’s confidence …” … “And, monks, this training rule should be recited like this: 

\subsubsection*{First preliminary ruling }

\scrule{‘If a monk enters an inhabited area at the wrong time, he commits an offense entailing confession.’” }

In\marginnote{1.24} this way the Buddha laid down this training rule for the monks. 

\subsubsection*{Second sub-story }

Soon\marginnote{2.1} afterwards a number of monks were walking through the Kosalan country on their way to \textsanskrit{Sāvatthī}, when one evening they came to a certain village. People saw them and said, “Venerables, please enter the village.” But knowing that entering a village at the wrong time had been prohibited by the Buddha and being afraid of wrongdoing, they declined. And so thieves robbed them. 

They\marginnote{2.7} then went to \textsanskrit{Sāvatthī} and told the monks what had happened, who in turn told the Buddha. Soon afterwards the Buddha gave a teaching and addressed the monks: 

\scrule{“Monks, I allow you to enter an inhabited area at the wrong time after informing someone. }

And\marginnote{2.11} so, monks, this training rule should be recited like this: 

\subsubsection*{Second preliminary ruling }

\scrule{‘If a monk, without informing anyone, enters an inhabited area at the wrong time, he commits an offense entailing confession.’” }

In\marginnote{2.13} this way the Buddha laid down this training rule for the monks. 

\subsubsection*{Third sub-story }

Soon\marginnote{3.1} afterwards a monk was walking through the Kosalan country on his way to \textsanskrit{Sāvatthī}, when one evening he came to a certain village. People saw him and said, “Venerable, please enter the village.” But knowing that entering a village at the wrong time without informing anyone had been prohibited by the Buddha and being afraid of wrongdoing, he declined. And so thieves robbed him. 

He\marginnote{3.7} then went to \textsanskrit{Sāvatthī} and told the monks what had happened, who in turn told the Buddha. Soon afterwards the Buddha gave a teaching and addressed the monks: 

\scrule{“Monks, I allow you to enter an inhabited area at the wrong time after informing an available monk. }

And\marginnote{3.11} so, monks, this training rule should be recited like this: 

\subsubsection*{Third preliminary ruling }

\scrule{‘If a monk, when a monk is available, enters an inhabited area at the wrong time without informing him, he commits an offense entailing confession.’” }

In\marginnote{3.13} this way the Buddha laid down this training rule for the monks. 

\subsubsection*{Fourth sub-story }

Soon\marginnote{4.1} afterwards a monk was bitten by a snake. Another monk went to the village to get fire. He then remembered that the Buddha has prohibited entering a village at the wrong time without informing an available monk. And being afraid of wrongdoing, he did not enter.\footnote{The ellipses points are not found in the PTS version. In fact, the ellipses points do not fit, since they imply that the monk gets robbed (as above), which does not fit the context. I therefore follow the PTS punctuation. } They told the Buddha. Soon afterwards the Buddha gave a teaching and addressed the monks: 

\scrule{“Monks, I allow you to enter an inhabited area at the wrong time without informing an available monk if there is some appropriate urgent business. }

And\marginnote{4.8} so, monks, this training rule should be recited like this: 

\subsection*{Final ruling }

\scrule{‘If a monk, when another monk is available, enters an inhabited area at the wrong time without informing him, except if there is some appropriate urgent business, he commits an offense entailing confession.’” }

\subsection*{Definitions }

\begin{description}%
\item[A: ] whoever … %
\item[Monk: ] … The monk who has been given the full ordination by a unanimous Sangha through a legal procedure consisting of one motion and three announcements that is irreversible and fit to stand—this sort of monk is meant in this case. %
\item[An available monk: ] he is able to inform him and then enter. %
\item[No available monk: ] he is not able to inform anyone and then enter. %
\item[At the wrong time: ] when the middle of the day has passed, until dawn. %
\item[Enters an inhabited area: ] if he crosses the boundary of an enclosed inhabited area, he commits an offense entailing confession. If he enters the vicinity of an unenclosed inhabited area, he commits an offense entailing confession. %
\item[Except if there is some appropriate urgent business: ] unless there is some appropriate urgent business. %
\end{description}

\subsection*{Permutations }

If\marginnote{5.2.1} it is the wrong time, and he perceives it as such, and he enters an inhabited area without informing an available monk, except if there is some appropriate urgent business, he commits an offense entailing confession. If it is the wrong time, but he is unsure of it, and he enters an inhabited area without informing an available monk, except if there is some appropriate urgent business, he commits an offense entailing confession. If it is the wrong time, but he perceives it as the right time, and he enters an inhabited area without informing an available monk, except if there is some appropriate urgent business, he commits an offense entailing confession. 

If\marginnote{5.2.4} it is the right time, but he perceives it as the wrong time, he commits an offense of wrong conduct. If it is the right time, but he is unsure of it, he commits an offense of wrong conduct. If it is the right time, and he perceives it as such, there is no offense. 

\subsection*{Non-offenses }

There\marginnote{5.3.1} is no offense: if there is some appropriate urgent business;  if he enters after informing an available monk;  if he enters without informing anyone when there is no available monk;  if he is going between monasteries;  if he is going to the dwelling place of nuns;  if he is going to the dwelling place of the monastics of another religion;  if he is returning to the monastery;  if the road goes via an inhabited area;  if there is an emergency;  if he is insane;  if he is the first offender. 

\scendsutta{The training rule on entering an inhabited area at the wrong time, the third, is finished. }

%
\section*{{\suttatitleacronym Bu Pc 86}{\suttatitletranslation 86. The training rule on needle cases }{\suttatitleroot Sūcighara}}
\addcontentsline{toc}{section}{\tocacronym{Bu Pc 86} \toctranslation{86. The training rule on needle cases } \tocroot{Sūcighara}}
\markboth{86. The training rule on needle cases }{Sūcighara}
\extramarks{Bu Pc 86}{Bu Pc 86}

\subsection*{Origin story }

At\marginnote{1.1} one time the Buddha was staying in the Sakyan country in the Banyan Tree Monastery at Kapilavatthu. At that time an ivory-worker had invited the monks who needed needle cases to ask for one. And the monks asked for many needle cases. Those who had small needle cases asked for large ones, and those who had large needle cases asked for small ones. The ivory-worker was so busy making needle cases for the monks that he was unable to make goods for sale. He could not make a living for himself, and his wives and children suffered. People complained and criticized them, “How can the Sakyan monastics not have any sense of moderation and ask for many needle cases? This ivory-worker is so busy making needle cases for them that he’s unable to make goods for sale. He can’t make a living for himself, and his wives and children are suffering.” 

The\marginnote{1.11} monks heard the complaints of those people, and the monks of few desires complained and criticized those monks, “How can those monks not have a sense of moderation and ask for many needle cases?” … “Is it true, monks, that there are monks who do this?” 

“It’s\marginnote{1.15} true, sir.” 

The\marginnote{1.16} Buddha rebuked them … “How can those foolish men do this? This will affect people’s confidence …” … “And, monks, this training rule should be recited like this: 

\subsection*{Final ruling }

\scrule{‘If a monk has a needle case made from bone, ivory, or horn, it is to be destroyed, and he commits an offense entailing confession.’” }

\subsection*{Definitions }

\begin{description}%
\item[A: ] whoever … %
\item[Monk: ] … The monk who has been given the full ordination by a unanimous Sangha through a legal procedure consisting of one motion and three announcements that is irreversible and fit to stand—this sort of monk is meant in this case. %
\item[Bone: ] any kind of bone. %
\item[Ivory: ] elephant tooth is what is meant. %
\item[Horn: ] any kind of horn. %
\item[Has made: ] if he makes one or has one made, then for the effort there is an act of wrong conduct. When he gets it, it is to be destroyed, and he is then to confess an offense entailing confession. %
\end{description}

\subsection*{Permutations }

If\marginnote{2.2.1} he finishes what he began himself, he commits an offense entailing confession. If he has others finish what he began himself, he commits an offense entailing confession. If he finishes himself what was begun by others, he commits an offense entailing confession. If he has others finish what was begun by others, he commits an offense entailing confession. 

If\marginnote{2.2.5} he makes one or has one made for the benefit of someone else, he commits an offense of wrong conduct. If he gets one that was made by someone else and then uses it, he commits an offense of wrong conduct. 

\subsection*{Non-offenses }

There\marginnote{2.3.1} is no offense: if it is a toggle;  if it is a fire kindler;  if it is a buckle;  if it is an ointment box;  if it is an ointment stick;  if it is an adz handle;  if it is a water wiper;  if he is insane;  if he is the first offender. 

\scendsutta{The training rule on needle cases, the fourth, is finished. }

%
\section*{{\suttatitleacronym Bu Pc 87}{\suttatitletranslation 87. The training rule on beds and benches }{\suttatitleroot Mañcapīṭha}}
\addcontentsline{toc}{section}{\tocacronym{Bu Pc 87} \toctranslation{87. The training rule on beds and benches } \tocroot{Mañcapīṭha}}
\markboth{87. The training rule on beds and benches }{Mañcapīṭha}
\extramarks{Bu Pc 87}{Bu Pc 87}

\subsection*{Origin story }

At\marginnote{1.1} one time when the Buddha was staying at \textsanskrit{Sāvatthī} in \textsanskrit{Anāthapiṇḍika}’s Monastery, Venerable Upananda the Sakyan was sleeping on a high bed. On one occasion, as the Buddha and a number of monks were walking about the dwellings, they came to Upananda’s dwelling. Upananda saw the Buddha coming and said to him, “Sir, please come and see my bed.” The Buddha turned around on the spot and addressed the monks: “A fool, monks, can be recognized by his sleeping place.” 

After\marginnote{1.9} rebuking Upananda in many ways, the Buddha spoke in dispraise of being difficult to support … “And, monks, this training rule should be recited like this: 

\subsection*{Final ruling }

\scrule{‘If a monk is having a new bed or bench made, it is to have legs eight standard fingerbreadths long below the lowest frame. If the legs exceed that, they are to be cut down, and he commits an offense entailing confession.’”\footnote{For an explanation of rendering \textit{sugata} as “standard”, see Appendix of Technical Terms. } }

\subsection*{Definitions }

\begin{description}%
\item[New: ] newly made is what is meant. %
\item[A bed: ] there are four kinds of beds: one with legs and frame, called \textit{\textsanskrit{masāraka}}; one with legs and frame, called \textit{\textsanskrit{bundikābaddha}}; one with crooked legs; one with detachable legs.\footnote{For a discussion of these and those below, see Appendix of Furniture. } %
\item[A bench: ] there are four kinds of benches: one with legs and frame, called \textit{\textsanskrit{masāraka}}; one with legs and frame, called \textit{\textsanskrit{bundikābaddha}}; one with crooked legs; one with detachable legs. %
\item[Is having made: ] making it himself or having it made. %
\item[It is to have legs eight standard fingerbreadths long below the lowest frame: ] apart from the lowest frame. If he makes one, or has one made, that exceeds that, then for the effort there is an act of wrong conduct. When he gets it, the legs are to be cut down, and he is then to confess an offense entailing confession. %
\end{description}

\subsection*{Permutations }

If\marginnote{2.1.12.1} he finishes what he began himself, he commits an offense entailing confession. If he has others finish what he began himself, he commits an offense entailing confession. If he finishes himself what was begun by others, he commits an offense entailing confession. If he has others finish what was begun by others, he commits an offense entailing confession. 

If\marginnote{2.1.16} he makes one or has one made for the benefit of someone else, he commits an offense of wrong conduct. 

If\marginnote{2.1.17} he gets one that was made by someone else and then uses it, he commits an offense of wrong conduct. 

\subsection*{Non-offenses }

There\marginnote{2.2.1} is no offense: if he makes it the right height;  if he makes it lower than the right height;  if he gets one made by another that exceeds the right height and then cuts the legs down before using it;  if he is insane;  if he is the first offender. 

\scendsutta{The training rule on beds and benches, the fifth, is finished. }

%
\section*{{\suttatitleacronym Bu Pc 88}{\suttatitletranslation 88. The training rule on upholstered with cotton down }{\suttatitleroot Tūlonaddha}}
\addcontentsline{toc}{section}{\tocacronym{Bu Pc 88} \toctranslation{88. The training rule on upholstered with cotton down } \tocroot{Tūlonaddha}}
\markboth{88. The training rule on upholstered with cotton down }{Tūlonaddha}
\extramarks{Bu Pc 88}{Bu Pc 88}

\subsection*{Origin story }

At\marginnote{1.1} one time when the Buddha was staying at \textsanskrit{Sāvatthī} in \textsanskrit{Anāthapiṇḍika}’s Monastery, the monks from the group of six had beds and benches made upholstered with cotton down. When people walking about the dwellings saw this, they complained and criticized those monks, “How can the Sakyan monastics have beds and benches made upholstered with cotton down? They’re just like householders who indulge in worldly pleasures!” 

The\marginnote{1.5} monks heard the complaints of those people, and the monks of few desires complained and criticized those monks, “How could the monks from the group of six do this?” … “Is it true, monks, that you did this?” 

“It’s\marginnote{1.9} true, sir.” 

The\marginnote{1.10} Buddha rebuked them … “Foolish men, how could you do this? This will affect people’s confidence …” … “And, monks, this training rule should be recited like this: 

\subsection*{Final ruling }

\scrule{‘If a monk has a bed or a bench made upholstered with cotton down, it is to be stripped, and he commits an offense entailing confession.’” }

\subsection*{Definitions }

\begin{description}%
\item[A: ] whoever … %
\item[Monk: ] … The monk who has been given the full ordination by a unanimous Sangha through a legal procedure consisting of one motion and three announcements that is irreversible and fit to stand—this sort of monk is meant in this case. %
\item[A bed: ] there are four kinds of beds: one with legs and frame, called \textit{\textsanskrit{masāraka}}; one with legs and frame, called \textit{\textsanskrit{bundikābaddha}}; one with crooked legs; one with detachable legs.\footnote{For a discussion of these and those below, see Appendix of Furniture. } %
\item[A bench: ] there are four kinds of benches: one with legs and frame, called \textit{\textsanskrit{masāraka}}; one with legs and frame, called \textit{\textsanskrit{bundikābaddha}}; one with crooked legs; one with detachable legs. %
\item[Cotton down: ] there are three kinds of cotton down: cotton down from trees, cotton down from creepers, cotton down from grass. %
\item[Has made: ] if he makes one or has one made, then for the effort there is an act of wrong conduct. When he gets it, it is to be stripped, and he is then to confess an offense entailing confession. %
\end{description}

\subsection*{Permutations }

If\marginnote{2.1.14.1} he finishes what he began himself, he commits an offense entailing confession. If he has others finish what he began himself, he commits an offense entailing confession. If he finishes himself what was begun by others, he commits an offense entailing confession. If he has others finish what was begun by others, he commits an offense entailing confession. 

If\marginnote{2.1.18} he makes one or has one made for the benefit of someone else, he commits an offense of wrong conduct. If he gets one that was made by someone else and then uses it, he commits an offense of wrong conduct. 

\subsection*{Non-offenses }

There\marginnote{2.2.1} is no offense: if it is for a back-and-knee strap;\footnote{The \textit{\textsanskrit{āyoga}} is used as a support for the \textit{\textsanskrit{pallattikā}} sitting posture. See Bhikkhu Ñā\textsanskrit{ṇatusita}, “Analysis of the Bhikkhu Pātimokkha”, p. 259, (re. \href{https://suttacentral.net/pli-tv-bu-vb-sk26/en/brahmali\#1.3.1}{Bu Sk 26:1.3.1}). }  if it is for a belt;  if it is for a shoulder strap;  if it is for a bowl bag;  if it is for a water filter;  if he is making a pillow;  if he gets one made by another and then strips it before using it;  if he is insane;  if he is the first offender. 

\scendsutta{The training rule on upholstered with cotton down, the sixth, is finished. }

%
\section*{{\suttatitleacronym Bu Pc 89}{\suttatitletranslation 89. The training rule on sitting mats }{\suttatitleroot Nisīdana}}
\addcontentsline{toc}{section}{\tocacronym{Bu Pc 89} \toctranslation{89. The training rule on sitting mats } \tocroot{Nisīdana}}
\markboth{89. The training rule on sitting mats }{Nisīdana}
\extramarks{Bu Pc 89}{Bu Pc 89}

\subsection*{Origin story }

\subsubsection*{First sub-story }

At\marginnote{1.1} one time when the Buddha was staying at \textsanskrit{Sāvatthī} in \textsanskrit{Anāthapiṇḍika}’s Monastery, he allowed sitting mats for the monks. Knowing this, the monks from the group of six used sitting mats that were inappropriate in size. The sitting mats hung down from beds and benches, both in front and behind. 

The\marginnote{1.6} monks of few desires complained and criticized those monks, “How can the monks from the group of six use such sitting mats?” … “Is it true, monks, that you do this?” 

“It’s\marginnote{1.9} true, sir.” 

The\marginnote{1.10} Buddha rebuked them … “Foolish men, how can you do this? This will affect people’s confidence …” … “And, monks, this training rule should be recited like this: 

\subsubsection*{Preliminary ruling }

\scrule{‘If a monk is having a sitting mat made, it is to be made the right size. This is the right size: two standard handspans long and one-and-a-half wide. If it exceeds that, it is to be cut down, and he commits an offense entailing confession.’” }

In\marginnote{1.17} this way the Buddha laid down this training rule for the monks. 

\subsubsection*{Second sub-story }

At\marginnote{2.1} that time there was a big monk called Venerable \textsanskrit{Udāyī}. After putting out his sitting mat in front of the Buddha, he stretched it on all sides before sitting down. The Buddha asked him, “\textsanskrit{Udāyī}, why are you stretching the sitting mat on all sides, as if an old hide?” 

“Because,\marginnote{2.6} sir, the sitting mat you’ve allowed for the monks is too small.” 

Soon\marginnote{2.7} afterwards the Buddha gave a teaching and addressed the monks: 

\scrule{“Monks, I allow a border of one handspan for the sitting mat. }

And\marginnote{2.9} so, monks, this training rule should be recited like this: 

\subsection*{Final ruling }

\scrule{‘If a monk is having a sitting mat made, it should be made the right size. This is the right size: two standard handspans long and one-and-a-half wide, and a border of one handspan.\footnote{For an explanation of rendering \textit{sugata} as “standard”, see Appendix of Technical Terms. } If it exceeds that, it is to be cut down, and he commits an offense entailing confession.’” }

\subsection*{Definitions }

\begin{description}%
\item[A sitting mat: ] one with a border is what is meant. %
\item[Is having made: ] making it himself or having it made, it should be made the right size. This is the right size: two standard handspans long and one-and-a-half wide, and a border of one handspan. If he makes one, or has one made, that exceeds that, then for the effort there is an act of wrong conduct. When he gets it, it is to be cut down, and he is then to confess an offense entailing confession. %
\end{description}

\subsection*{Permutations }

If\marginnote{3.1.8.1} he finishes what he began himself, he commits an offense entailing confession. If he has others finish what he began himself, he commits an offense entailing confession. If he finishes himself what was begun by others, he commits an offense entailing confession. If he has others finish what was begun by others, he commits an offense entailing confession. 

If\marginnote{3.1.12} he makes one or has one made for the benefit of someone else, he commits an offense of wrong conduct. If he gets one that was made by someone else and then uses it, he commits an offense of wrong conduct. 

\subsection*{Non-offenses }

There\marginnote{3.2.1} is no offense: if he makes it the right size;  if he makes it smaller than the right size;  if he gets one made by another that exceeds the right size and then cuts it down before using it;  if he makes a canopy, a floor cover, a cloth screen, a mattress, or a pillow;  if he is insane;  if he is the first offender. 

\scendsutta{The training rule on sitting mats, the seventh, is finished. }

%
\section*{{\suttatitleacronym Bu Pc 90}{\suttatitletranslation 90. The training rule on itch covers }{\suttatitleroot Kaṇḍuppaṭicchādi}}
\addcontentsline{toc}{section}{\tocacronym{Bu Pc 90} \toctranslation{90. The training rule on itch covers } \tocroot{Kaṇḍuppaṭicchādi}}
\markboth{90. The training rule on itch covers }{Kaṇḍuppaṭicchādi}
\extramarks{Bu Pc 90}{Bu Pc 90}

\subsection*{Origin story }

At\marginnote{1.1} one time when the Buddha was staying at \textsanskrit{Sāvatthī} in \textsanskrit{Anāthapiṇḍika}’s Monastery, he allowed itch-covering cloths for the monks. Knowing this, the monks from the group of six wore itch covers that were inappropriate in size. As they were walking about, they were dragging them along, both in front and behind. 

The\marginnote{1.6} monks of few desires complained and criticized those monks, “How can the monks from the group of six wear such itch covers?” … “Is it true, monks, that you do this?” 

“It’s\marginnote{1.9} true, sir.” 

The\marginnote{1.10} Buddha rebuked them … “Foolish men, how can you do this? This will affect people’s confidence …” … “And, monks, this training rule should be recited like this: 

\subsection*{Final ruling }

\scrule{‘If a monk is having an itch-covering cloth made, it should be made the right size. This is the right size: four standard handspans long and two wide.\footnote{For an explanation of rendering \textit{sugata} as “standard”, see Appendix of Technical Terms. } If it exceeds that, it is to be cut down, and he commits an offense entailing confession.’” }

\subsection*{Definitions }

\begin{description}%
\item[An itch-covering cloth: ] it is for the purpose of covering an itch or a boil or a running sore or a carbuncle, below the navel and above the knees. %
\item[Is having made: ] making it himself or having it made, it should be made the right size. This is the right size: four standard handspans long and two wide. If he makes one or has one made that exceeds that, then for the effort there is an act of wrong conduct. When he gets it, it is to be cut down, and he is then to confess an offense entailing confession. %
\end{description}

\subsection*{Permutations }

If\marginnote{2.8.1} he finishes what he began himself, he commits an offense entailing confession. If he has others finish what he began himself, he commits an offense entailing confession. If he finishes himself what was begun by others, he commits an offense entailing confession. If he has others finish what was begun by others, he commits an offense entailing confession. 

If\marginnote{2.12} he makes one or has one made for the benefit of someone else, he commits an offense of wrong conduct. If he gets one that was made by someone else and then uses it, he commits an offense of wrong conduct. 

\subsection*{Non-offenses }

There\marginnote{2.14.1} is no offense: if he makes it the right size;  if he makes it smaller than the right size;  if he gets one made by another that exceeds the right size and then cuts it down before using it;  if he makes a canopy, a floor cover, a cloth screen, a mattress, or a pillow;  if he is insane;  if he is the first offender. 

\scendsutta{The training rule on itch covers, the eighth, is finished. }

%
\section*{{\suttatitleacronym Bu Pc 91}{\suttatitletranslation 91. The training rule on the rainy-season robe }{\suttatitleroot Vassikasāṭikā}}
\addcontentsline{toc}{section}{\tocacronym{Bu Pc 91} \toctranslation{91. The training rule on the rainy-season robe } \tocroot{Vassikasāṭikā}}
\markboth{91. The training rule on the rainy-season robe }{Vassikasāṭikā}
\extramarks{Bu Pc 91}{Bu Pc 91}

\subsection*{Origin story }

At\marginnote{1.1} one time when the Buddha was staying at \textsanskrit{Sāvatthī} in \textsanskrit{Anāthapiṇḍika}’s Monastery, he allowed rainy-season robes for the monks. Knowing this, the monks from the group of six wore rainy-season robes that were inappropriate in size. As they were walking about, they were dragging them along, both in front and behind. 

The\marginnote{1.6} monks of few desires complained and criticized those monks, “How can the monks from the group of six wear such rainy-season robes?” … “Is it true, monks, that you do this?” 

“It’s\marginnote{1.9} true, sir.” 

The\marginnote{1.10} Buddha rebuked them … “Foolish men, how can you do this? This will affect people’s confidence …” … “And, monks, this training rule should be recited like this: 

\subsection*{Final ruling }

\scrule{‘If a monk is having a rainy-season robe made, it should be made the right size. This is the right size: six standard handspans long and two-and-a-half wide.\footnote{For an explanation of rendering \textit{sugata} as “standard”, see Appendix of Technical Terms. } If it exceeds that, it is to be cut down, and he commits an offense entailing confession.’” }

\subsection*{Definitions }

\begin{description}%
\item[A rainy-season robe: ] it is for use during the four months of the rainy season. %
\item[Is having made: ] making it himself or having it made, it should be made to the right size. This is the right size: six standard handspans long and two and a half wide. If he makes one or has one made that exceeds that, then for the effort there is an act of wrong conduct. When he gets it, it is to be cut down, and he is then to confess an offense entailing confession. %
\end{description}

\subsection*{Permutations }

If\marginnote{2.8.1} he finishes what he began himself, he commits an offense entailing confession. If he has others finish what he began himself, he commits an offense entailing confession. If he finishes himself what was begun by others, he commits an offense entailing confession. If he has others finish what was begun by others, he commits an offense entailing confession. 

If\marginnote{2.12} he makes one or has one made for the benefit of someone else, he commits an offense of wrong conduct. If he gets one that was made by someone else and then uses it, he commits an offense of wrong conduct. 

\subsection*{Non-offenses }

There\marginnote{2.14.1} is no offense: if he makes it the right size;  if he makes it smaller than the right size;  if he gets one made by another that exceeds the right size and then cuts it down before using it;  if he makes a canopy, a floor cover, a cloth screen, a mattress, or a pillow;  if he is insane;  if he is the first offender. 

\scendsutta{The training rule on the rainy-season robe, the ninth, is finished. }

%
\section*{{\suttatitleacronym Bu Pc 92}{\suttatitletranslation 92. The training rule on Nanda }{\suttatitleroot Sugatacīvara}}
\addcontentsline{toc}{section}{\tocacronym{Bu Pc 92} \toctranslation{92. The training rule on Nanda } \tocroot{Sugatacīvara}}
\markboth{92. The training rule on Nanda }{Sugatacīvara}
\extramarks{Bu Pc 92}{Bu Pc 92}

\subsection*{Origin story }

At\marginnote{1.1} one time the Buddha was staying at \textsanskrit{Sāvatthī} in the Jeta Grove, \textsanskrit{Anāthapiṇḍika}’s Monastery. At that time there was a handsome and graceful monk called Venerable Nanda, the Buddha’s cousin. He was seven centimeters shorter than the Buddha,\footnote{That is, four fingerbreadths. For a discussion of the \textit{\textsanskrit{aṅgula}}, see \textit{sugata} in Appendix of Technical Terms. } but he wore a robe that was the same size as the Buddha’s.\footnote{Here I render \textit{sugata} as “Buddha”, which seems required by the narrative. I do not think, however, that this narrative usage ties our hands in the interpretation of this term. See below. } When the senior monks saw him coming, they thought it was the Buddha and got up from their seats. 

But\marginnote{1.7} when he came close, they realized who it was, and they complained and criticized him, “How can Venerable Nanda wear a robe the same size as the Buddha’s?” … “Is it true, Nanda, that you do this?” 

“It’s\marginnote{1.10} true, sir.” 

The\marginnote{1.11} Buddha rebuked him … “Nanda, how can you do this? This will affect people’s confidence …” … “And, monks, this training rule should be recited like this: 

\subsection*{Final ruling }

\scrule{‘If a monk has a robe made that is the standard robe size or larger, it is to be cut down, and he commits an offense entailing confession.\footnote{For an explanation of rendering \textit{sugata} as “standard”, see Appendix of Technical Terms. } This is the standard robe size: nine standard handspans long and six wide.’” }

\subsection*{Definitions }

\begin{description}%
\item[A: ] whoever … %
\item[Monk: ] … The monk who has been given the full ordination by a unanimous Sangha through a legal procedure consisting of one motion and three announcements that is irreversible and fit to stand—this sort of monk is meant in this case. %
\item[The standard robe size: ] nine standard handspans long and six wide. %
\item[Has made: ] if he makes one or has one made, then for the effort there is an act of wrong conduct. When he gets it, it is to be cut down, and he is then to confess an offense entailing confession. %
\end{description}

\subsection*{Permutations }

If\marginnote{2.1.10.1} he finishes what he began himself, he commits an offense entailing confession. If he has others finish what he began himself, he commits an offense entailing confession. If he finishes himself what was begun by others, he commits an offense entailing confession. If he has others finish what was begun by others, he commits an offense entailing confession. 

If\marginnote{2.1.14} he makes one or has one made for the benefit of someone else, he commits an offense of wrong conduct. If he gets one that was made by someone else and then uses it, he commits an offense of wrong conduct. 

\subsection*{Non-offenses }

There\marginnote{2.2.1} is no offense: if he makes it smaller than the standard robe;  if he gets one made by another that is too large and then cuts it down before using it;  if he makes a canopy, a floor cover, a cloth screen, a mattress, or a pillow;  if he is insane;  if he is the first offender. 

\scendsutta{The training rule on Nanda, the tenth, is finished. }

\scendvagga{The ninth subchapter on precious things is finished. }

\scuddanaintro{This is the summary: }

\begin{scuddana}%
“And\marginnote{2.2.10} a king’s, precious things, available, \\
Needle, and bed, cotton down; \\
And sitting mat, and itch, \\
Rainy-season, and by the standard.” 

%
\end{scuddana}

“Venerables,\marginnote{2.2.14} the ninety-two rules on confession have been recited. In regard to this I ask you, ‘Are you pure in this?’ A second time I ask, ‘Are you pure in this?’ A third time I ask, ‘Are you pure in this?’ You are pure in this and therefore silent. I’ll remember it thus.” 

\scend{The section on minor rules has been completed. }

\scendkanda{The chapter on offenses entailing confession is finished. }

%
\addtocontents{toc}{\let\protect\contentsline\protect\nopagecontentsline}
\chapter*{Acknowledgment }
\addcontentsline{toc}{chapter}{\tocchapterline{Acknowledgment }}
\addtocontents{toc}{\let\protect\contentsline\protect\oldcontentsline}

%
%
\section*{{\suttatitleacronym Bu Pd 1}{\suttatitletranslation The first training rule on acknowledgment }{\suttatitleroot Paṭhamapāṭidesanīya}}
\addcontentsline{toc}{section}{\tocacronym{Bu Pd 1} \toctranslation{The first training rule on acknowledgment } \tocroot{Paṭhamapāṭidesanīya}}
\markboth{The first training rule on acknowledgment }{Paṭhamapāṭidesanīya}
\extramarks{Bu Pd 1}{Bu Pd 1}

Venerables,\marginnote{0.5} these four rules on acknowledgment come up for recitation. 

\subsection*{Origin story }

At\marginnote{1.1} one time when the Buddha was staying at \textsanskrit{Sāvatthī} in \textsanskrit{Anāthapiṇḍika}’s Monastery, a certain nun was returning from almsround in \textsanskrit{Sāvatthī}. She saw a monk and said to him, “Here, venerable, please take some almsfood.” 

Saying,\marginnote{1.4} “Alright, Sister,” he took everything. But because the time for eating was coming to an end, she was not able to go for alms, and she missed her meal. 

The\marginnote{1.6} next day and the following day the same thing happened again. On the fourth day that nun was walking along a street, trembling. A wealthy merchant coming by carriage in the opposite direction saw her and shouted out, “Watch out, venerable!” As she stepped aside, she collapsed right there. 

The\marginnote{1.15} merchant asked her for forgiveness: “Forgive me, venerable, since you fell because of me.” 

“I\marginnote{1.17} didn’t fall because of you, but because I’m weak.” 

“But\marginnote{1.19} why are you so weak?” 

The\marginnote{1.20} nun told him what had happened. He then brought her to his house and gave her a meal. Afterwards he complained and criticized the monks, “How can the venerables receive food directly from a nun? It’s difficult for women to get material support.” 

The\marginnote{1.24} monks heard the complaints of that merchant, and the monks of few desires complained and criticized that monk, “How could a monk receive food directly from a nun?” … “Is it true, monk, that you did this?” 

“It’s\marginnote{1.28} true, sir.” 

“Is\marginnote{1.29} she a relative of yours?” 

“No,\marginnote{1.30} sir.” 

“Foolish\marginnote{1.31} man, a man and a woman who are unrelated don’t know what’s appropriate and inappropriate, what’s good and bad, in dealing with each other. So how could you do this? This will affect people’s confidence …” … “And, monks, this training rule should be recited like this: 

\subsection*{Final ruling }

\scrule{‘If a monk receives fresh or cooked food directly from an unrelated nun who has entered an inhabited area, and then eats it, he must acknowledge it: “I have done a blameworthy and unsuitable thing that is to be acknowledged. I acknowledge it.”’” }

\subsection*{Definitions }

\begin{description}%
\item[A: ] whoever … %
\item[Monk: ] … The monk who has been given the full ordination by a unanimous Sangha through a legal procedure consisting of one motion and three announcements that is irreversible and fit to stand—this sort of monk is meant in this case. %
\item[Unrelated: ] anyone who is not a descendant of one’s male ancestors going back eight generations, either on the mother’s side or on the father’s side.\footnote{Sp 1.505: \textit{Tattha \textsanskrit{yāva} \textsanskrit{sattamā} \textsanskrit{pitāmahayugāti} \textsanskrit{pitupitā} \textsanskrit{pitāmaho}, \textsanskrit{pitāmahassa} \textsanskrit{yugaṁ} \textsanskrit{pitāmahayugaṁ}}, “In this \textit{\textsanskrit{yāva} \textsanskrit{sattamā} \textsanskrit{pitāmahayuga}} means: the father of a father is a grandfather. The generation of a grandfather is called a \textit{\textsanskrit{pitāmahayuga}}.” So the PaIi phrase \textit{\textsanskrit{yāva} \textsanskrit{sattamā} \textsanskrit{pitāmahayuga}} means “as far as the seventh generation of grandfathers”, that is, eight generations back. This can be counted as follows: (1) one’s grandfather; (2) his father; (3) 2’s father; (4) 3’s father; (5) 4’s father; (6) 5’s father; and (7) 6’s father. This applies to both one’s paternal and maternal grandfathers. This gives a total of 14 ancestors. Anyone who is a descendent of these fourteen is considered a relative. Anyone who is not such a descendent is not regarded as a relative. } %
\item[A nun: ] she has been given the full ordination by both Sanghas. %
\item[An inhabited area: ] a street, a cul-de-sac, an intersection, a house. %
\item[Fresh food: ] apart from the five cooked foods, the post-midday tonics, the seven-day tonics, and the lifetime tonics, the rest is called “fresh food”. %
\item[Cooked food: ] there are five kinds of cooked food: cooked grain, porridge, flour products, fish, and meat.\footnote{“Flour products” renders \textit{sattu}. See discussion in Appendix of Technical Terms. } %
\end{description}

If\marginnote{2.1.15} he receives the food with the intention of eating it, he commits an offense of wrong conduct. For every mouthful swallowed, he commits an offense entailing acknowledgment. 

\subsection*{Permutations }

If\marginnote{2.2.1} she is unrelated and he perceives her as such, and he receives fresh or cooked food directly from her when she has entered an inhabited area, and then eats it, he commits an offense entailing acknowledgment. If she is unrelated, but he is unsure of it, and he receives fresh or cooked food directly from her when she has entered an inhabited area, and then eats it, he commits an offense entailing acknowledgment. If she is unrelated, but he perceives her as related, and he receives fresh or cooked food directly from her when she has entered an inhabited area, and then eats it, he commits an offense entailing acknowledgment. 

If\marginnote{2.2.4} he receives post-midday tonics, seven-day tonics, or lifetime tonics for the purpose of food, he commits an offense of wrong conduct. For every mouthful swallowed, he commits an offense of wrong conduct. If he receives fresh or cooked food, with the intention of eating it, directly from a nun who is fully ordained only on one side, he commits an offense of wrong conduct. For every mouthful swallowed, he commits an offense of wrong conduct. 

If\marginnote{2.2.9} she is related, but he perceives her as unrelated, he commits an offense of wrong conduct. If she is related, but he is unsure of it, he commits an offense of wrong conduct. If she is related, and he perceives her as such, there is no offense. 

\subsection*{Non-offenses }

There\marginnote{2.3.1} is no offense: if she is related;  if she gets someone else to give it and does not give it herself;  if she gives by placing it near;  if it is inside a monastery;  if it is at the dwelling place of nuns;  if it is at the dwelling place of the monastics of another religion;  if it is on returning to the monastery;  if she gives after carrying it out of the village;  if she gives post-midday tonics, seven-day tonics, or lifetime tonics, saying, “Use these when there’s a reason;”  if it is a trainee nun;  if it is a novice nun;  if he is insane;  if he is the first offender. 

\scendsutta{The first training rule on acknowledgment is finished. }

%
\section*{{\suttatitleacronym Bu Pd 2}{\suttatitletranslation The second training rule on acknowledgment }{\suttatitleroot Dutiyapāṭidesanīya}}
\addcontentsline{toc}{section}{\tocacronym{Bu Pd 2} \toctranslation{The second training rule on acknowledgment } \tocroot{Dutiyapāṭidesanīya}}
\markboth{The second training rule on acknowledgment }{Dutiyapāṭidesanīya}
\extramarks{Bu Pd 2}{Bu Pd 2}

\subsection*{Origin story }

At\marginnote{1.1} one time the Buddha was staying at \textsanskrit{Rājagaha} in the Bamboo Grove, the squirrel sanctuary. At that time, when families invited monks to meals, the nuns from the group of six were directing people toward the monks from the group of six, saying, “Give curry here; give rice there.” The monks from the group of six ate as much as they wanted, but not so the other monks. 

The\marginnote{1.7} monks of few desires complained and criticized them, “How could the monks from the group of six not restrain the nuns from giving directions?” … “Is it true, monks, that you didn’t do this?” 

“It’s\marginnote{1.10} true, sir.” 

The\marginnote{1.11} Buddha rebuked them … “Foolish men, how could you not do this? This will affect people’s confidence …” … “And, monks, this training rule should be recited like this: 

\subsection*{Final ruling }

\scrule{‘When monks eat by invitation to families, if a nun is there giving directions, saying, “Give bean curry here; give rice there,” then those monks should stop her: “Stop, Sister, while the monks are eating.” If not even a single monk addresses that nun in this way to stop her, they must acknowledge it: “We have done a blameworthy and unsuitable thing which is to be acknowledged. We acknowledge it.”’” }

\subsection*{Definitions }

\begin{description}%
\item[When monks eat by invitation to families: ] a family: there are four kinds of families: the aristocratic family, the brahmin family, the merchant family, the worker family. %
\item[Eat by invitation: ] eat any of the five cooked foods by invitation. %
\item[A nun: ] she has been given the full ordination by both Sanghas. %
\item[Giving directions: ] saying, “Give bean curry here; give rice there,” according to friendship, according to companionship, according to who one is devoted to, according to being a co-student, according to being a co-pupil—this is called “giving directions”. %
\item[Those monks: ] the monks who are eating. %
\item[Her: ] the nun who is giving directions. %
\item[Those monks should stop her: ‘Stop, Sister, while the monks are eating’: ] if she is not stopped by even one monk, and a monk then receives food with the intention of eating it, he commits an offense of wrong conduct.\footnote{\textit{\textsanskrit{Khādissāmi} \textsanskrit{bhuñjissāmi}}, which would normally refer to both fresh and cooked food. But since fresh food is explicitly excluded, both above and below, it would seem that in this context even \textit{\textsanskrit{khādissāmi}} refers to eating cooked food. } For every mouthful swallowed, he commits an offense entailing acknowledgment. %
\end{description}

\subsection*{Permutations }

If\marginnote{2.2.1} she is fully ordained, and he perceives her as such, and he does not restrain her from giving directions, he commits an offense entailing acknowledgment. If she is fully ordained, but he is unsure of it, and he does not restrain her from giving directions, he commits an offense entailing acknowledgment. If she is fully ordained, but he does not perceive her as such, and he does not restrain her from giving directions, he commits an offense entailing acknowledgment. 

If\marginnote{2.2.4} he does not restrain a nun who is fully ordained only on one side from giving directions, he commits an offense of wrong conduct. If the person is not fully ordained, but he perceives them as such, he commits an offense of wrong conduct. If the person is not fully ordained, but he is unsure of it, he commits an offense of wrong conduct. If the person is not fully ordained, and he does not perceive them as such, there is no offense. 

\subsection*{Non-offenses }

There\marginnote{2.3.1} is no offense: if a nun does not give it herself, but gets someone else to give her own food;  if a nun does not get someone else to give it, but she gives someone else’s food herself;  if a nun gets someone else to give what has not yet been given;  if a nun gets someone else to give to someone who has not yet received anything;  if a nun gets someone else to give equally to all;  if a trainee nun is giving directions;  if a novice nun is giving directions;  if it is anything apart from the five kinds of cooked food;  if he is insane;  if he is the first offender. 

\scendsutta{The second training rule on acknowledgment is finished. }

%
\section*{{\suttatitleacronym Bu Pd 3}{\suttatitletranslation The third training rule on acknowledgment }{\suttatitleroot Tatiyapāṭidesanīya}}
\addcontentsline{toc}{section}{\tocacronym{Bu Pd 3} \toctranslation{The third training rule on acknowledgment } \tocroot{Tatiyapāṭidesanīya}}
\markboth{The third training rule on acknowledgment }{Tatiyapāṭidesanīya}
\extramarks{Bu Pd 3}{Bu Pd 3}

\subsection*{Origin story }

\subsubsection*{First sub-story }

At\marginnote{1.1} one time the Buddha was staying at \textsanskrit{Sāvatthī} in the Jeta Grove, \textsanskrit{Anāthapiṇḍika}’s Monastery. At that time in \textsanskrit{Sāvatthī} there was a family where both the husband and the wife had confidence. They were growing in faith, but declining in wealth. Whatever food they had in the morning, they gave to the monks. Sometimes they went without food. 

People\marginnote{1.4} complained and criticized the monks, “How can the Sakyan monastics not have a sense of moderation in receiving offerings? After giving to them, these people sometimes go without.” 

The\marginnote{1.7} monks heard the complaints of those people and they told the Buddha. Soon afterwards the Buddha gave a teaching and addressed the monks: 

\scrule{“Monks, if a family’s faith is growing, but its wealth is declining, you should designate it as ‘in training’, through a legal procedure consisting of one motion and one announcement. }

And\marginnote{1.11} the designation should be given like this. A competent and capable monk should inform the Sangha: 

‘Please,\marginnote{1.13} venerables, I ask the Sangha to listen. Such-and-such a family is growing in faith, but declining in wealth. If the Sangha is ready, it should designate that family as “in training”. This is the motion. 

Please,\marginnote{1.17} venerables, I ask the Sangha to listen. Such-and-such a family is growing in faith, but declining in wealth. The Sangha designates that family as “in training”. Any monk who approves of designating that family as “in training” should remain silent. Any monk who doesn’t approve should speak up. 

The\marginnote{1.22} Sangha has designated such-and-such a family as “in training”. The Sangha approves and is therefore silent. I’ll remember it thus.’ 

“And,\marginnote{1.24} monks, this training rule should be recited like this: 

\subsubsection*{First preliminary ruling }

\scrule{‘There are families that are designated as “in training”. If a monk eats fresh or cooked food after personally receiving it from such a family, he must acknowledge it: “I have done a blameworthy and unsuitable thing that is to be acknowledged. I acknowledge it.”’” }

In\marginnote{1.27} this way the Buddha laid down this training rule for the monks. 

\subsubsection*{Second sub-story }

Soon\marginnote{2.1} afterwards there was a celebration in \textsanskrit{Sāvatthī} and people invited the monks for a meal. And so did the family that had been designated as in training. But knowing that the Buddha had prohibited eating fresh or cooked food after personally receiving it from such a family, and being afraid of wrongdoing, the monks did not accept. That family complained and criticized them, “What is it with us that they don’t receive from us?” 

The\marginnote{2.8} monks heard the complaints of those people and they told the Buddha. Soon afterwards the Buddha gave a teaching and addressed the monks: 

\scrule{“Monks, if you have been invited, I allow you to eat fresh or cooked food after personally receiving it from a family designated as in training. }

And\marginnote{2.12} so, monks, this training rule should be recited like this: 

\subsubsection*{Second preliminary ruling }

\scrule{‘There are families that are designated as “in training”. If a monk, without first being invited, eats fresh or cooked food after personally receiving it from such a family, he must acknowledge it: “I have done a blameworthy and unsuitable thing that is to be acknowledged. I acknowledge it.”’” }

In\marginnote{2.15} this way the Buddha laid down this training rule for the monks. 

\subsubsection*{Third sub-story }

Soon\marginnote{3.1} afterwards a certain monk was associating with that family. One morning he robed up, took his bowl and robe, went to them, and sat down on the prepared seat. Just then that monk was sick, and so they invited him to eat. But knowing that the Buddha had prohibited an uninvited monk from eating fresh or cooked food after personally receiving it from such a family, and being afraid of wrongdoing, he did not accept. And being unable to walk for alms, he missed his meal. 

He\marginnote{3.10} then returned to the monastery and told the monks what had happened, who in turn told the Buddha. Soon afterwards the Buddha gave a teaching and addressed the monks: 

\scrule{“Monks, I allow a sick monk to eat fresh or cooked food after personally receiving it from a family designated as in training. }

And\marginnote{3.14} so, monks, this training rule should be recited like this: 

\subsection*{Final ruling }

\scrule{‘There are families that are designated as “in training”. If a monk, without being sick and without first being invited, eats fresh or cooked food after personally receiving it from such a family, he must acknowledge it: “I have done a blameworthy and unsuitable thing that is to be acknowledged. I acknowledge it.”’” }

\subsection*{Definitions }

\begin{description}%
\item[There are families that are designated as “in training”: ] a family designated as in training: a family growing in faith but declining in wealth. Such a family is designated as “in training” through a legal procedure consisting of one motion and one announcement. %
\item[A: ] whoever … %
\item[Monk: ] … The monk who has been given the full ordination by a unanimous Sangha through a legal procedure consisting of one motion and three announcements that is irreversible and fit to stand—this sort of monk is meant in this case. %
\item[Such a family: ] that kind of family. %
\item[Without being invited: ] without being invited for the same or the following day. If the invitation is made when he has entered the vicinity of the house, this is called “without being invited”. %
\item[Invited: ] invited for the same or the following day. If the invitation is made when he has not entered the vicinity of the house, this is called “invited”. %
\item[Without being sick: ] who is able to walk for alms. %
\item[Sick: ] who is unable to walk for alms. %
\item[Fresh food: ] apart from the five cooked foods, the post-midday tonics, the seven-day tonics, and the lifetime tonics—the rest is called “fresh food”. %
\item[Cooked food: ] there are five kinds of cooked food: cooked grain, porridge, flour  products, fish, and meat. %
\end{description}

If,\marginnote{4.1.21} without being sick and without being invited, he receives fresh or cooked food with the intention of eating it, he commits an offense of wrong conduct. For every mouthful swallowed, he commits an offense entailing acknowledgment. 

\subsection*{Permutations }

If\marginnote{4.2.1} a family is designated as in training, and he perceives it as such, and he, without being sick or without being invited, eats fresh or cooked food after personally receiving it from that family, he commits an offense entailing acknowledgment. If a family is designated as in training, but he is unsure of it, and he, without being sick or without being invited, eats fresh or cooked food after personally receiving it from that family, he commits an offense entailing acknowledgment. If a family is designated as in training, but he does not perceive it as such, and he, without being sick or without being invited, eats fresh or cooked food after personally receiving it from that family, he commits an offense entailing acknowledgment. 

If\marginnote{4.2.4} he receives post-midday tonics, seven-day tonics, or lifetime tonics for the purpose of food, he commits an offense of wrong conduct. For every mouthful swallowed, he commits an offense of wrong conduct. If a family is not designated as in training, but he perceives it as such, he commits an offense of wrong conduct. If a family is not designated as in training, but he is unsure of it, he commits an offense of wrong conduct. If a family is not designated as in training, and he does not perceive it as such, there is no offense. 

\subsection*{Non-offenses }

There\marginnote{4.3.1} is no offense: if he has been invited;  if he is sick;  if he eats the leftovers from one who has been invited or who is sick;  if other people’s almsfood is prepared there;  if they give after coming out from the house;  if it is a regular meal invitation;  if it is a meal for which lots are drawn;  if it is a half-monthly meal;  if it is on the observance day;  if it is on the day after the observance day;  if the family gives post-midday tonics, seven-day tonics, or lifetime tonics, saying, “Use these when there’s a reason;” if he is insane;  if he is the first offender. 

\scendsutta{The third training rule on acknowledgment is finished. }

%
\section*{{\suttatitleacronym Bu Pd 4}{\suttatitletranslation The fourth training rule on acknowledgment }{\suttatitleroot Catutthapāṭidesanīya}}
\addcontentsline{toc}{section}{\tocacronym{Bu Pd 4} \toctranslation{The fourth training rule on acknowledgment } \tocroot{Catutthapāṭidesanīya}}
\markboth{The fourth training rule on acknowledgment }{Catutthapāṭidesanīya}
\extramarks{Bu Pd 4}{Bu Pd 4}

\subsection*{Origin story }

\subsubsection*{First sub-story }

At\marginnote{1.1} one time when the Buddha was staying in the Sakyan country in the Banyan Tree Monastery at Kapilavatthu, the slaves of the Sakyans were rebelling. The Sakyan women wished to prepare a meal at the wilderness dwellings, but the slaves heard about this and besieged the path. When the Sakyan women took various kinds of fine foods and set out for a wilderness dwelling, the slaves emerged, and they robbed and raped those Sakyan women. Soon afterwards the Sakyan men came out, and they got hold of those criminals together with their loot. They then complained and criticized the monks, “How could they not inform us that there are criminals staying near the monastery?” 

The\marginnote{1.11} monks heard the complaints of the Sakyans and they told the Buddha. Soon afterwards the Buddha gave a teaching and addressed the monks: “Well then, monks, I will lay down a training rule for the following ten reasons: for the well-being of the Sangha, for the comfort of the Sangha, for the restraint of bad people, for the ease of good monks, for the restraint of the corruptions relating to the present life, for the restraint of the corruptions relating to future lives, to give rise to confidence in those without it, to increase the confidence of those who have it, for the longevity of the true Teaching, and for supporting the training. And, monks, this training rule should be recited like this: 

\subsubsection*{Preliminary ruling }

\scrule{‘There are wilderness dwellings that are considered risky and dangerous. If a monk, without first making an announcement about those dwellings, eats fresh or cooked food after personally receiving it inside that monastery, he must acknowledge it: “I have done a blameworthy and unsuitable thing that is to be acknowledged. I acknowledge it.”’” }

In\marginnote{1.18} this way the Buddha laid down this training rule for the monks. 

\subsubsection*{Second sub-story }

Soon\marginnote{2.1} afterwards a monk in a wilderness dwelling was sick. People took fresh and cooked food and went to that wilderness dwelling, and they invited that monk to eat. But knowing that the Buddha had prohibited the eating of fresh or cooked food after personally receiving it at a wilderness dwelling, and being afraid of wrongdoing, he did not accept it. And being unable to walk for alms, he missed his meal. 

He\marginnote{2.7} then told the monks what had happened, and they in turn told the Buddha. Soon afterwards the Buddha gave a teaching and addressed the monks: 

\scrule{“Monks, I allow a sick monk, without first making an announcement, to eat fresh or cooked food after personally receiving it at a wilderness dwelling. }

And\marginnote{2.11} so, monks, this training rule should be recited like this: 

\subsection*{Final ruling }

\scrule{‘There are wilderness dwellings that are considered risky and dangerous. If a monk who is not sick, without first making an announcement about those dwellings, eats fresh or cooked food after personally receiving it inside that monastery, he must acknowledge it:\footnote{It is hard to make out from the Pali what the announcement is about. The crucial word \textit{\textsanskrit{appaṭisaṁviditaṁ}} can be read as an adjective qualifying \textit{\textsanskrit{khādanīyaṁ}} and \textit{\textsanskrit{bhojanīyaṁ}}, in which case it is the food that needs to be announced. Alternatively, it can be regarded as an independent sentence verb, in which case it would seem that it is the risky and dangerous nature of the dwellings that needs to be announced. The word commentary in part supports the former interpretation, whereas the origin story supports the latter. Given this uncertainty, I have opted to translate according to what seems to make the best sense. } “I have done a blameworthy and unsuitable thing that is to be acknowledged. I acknowledge it.”’” }

\subsection*{Definitions }

\begin{description}%
\item[There are wilderness dwellings: ] a wilderness dwelling: if it is at least 800 meters away from any inhabited area.\footnote{That is, 500 bow-lengths. For a discussion of the \textit{dhanu}, see \textit{sugata} in Appendix of Technical Terms. } %
\item[Risky: ] in the monastery, or in the vicinity of the monastery, criminals have been seen camping, eating, standing, sitting, or lying down. %
\item[Dangerous: ] in the monastery, or in the vicinity of the monastery, criminals have been seen injuring, robbing, or beating people. %
\item[A: ] whoever … %
\item[Monk: ] … The monk who has been given the full ordination by a unanimous Sangha through a legal procedure consisting of one motion and three announcements that is irreversible and fit to stand—this sort of monk is meant in this case. %
\item[About those dwellings: ] about such kinds of dwellings. %
\item[Without making an announcement: ] if an announcement is made to any of one’s five co-monastics, this is called “without making an announcement”.\footnote{Sp 2.573: \textit{\textsanskrit{Pañcannaṁ} \textsanskrit{paṭisaṁviditanti} \textsanskrit{pañcasu} sahadhammikesu}, “\textit{\textsanskrit{Pañcannaṁ} \textsanskrit{paṭisaṁviditaṁ}} means the five co-monastics.” That is, a monk, a nun, a trainee nun, a novice monk, or a novice nun. The point here is that the announcement needs to be made directly to the lay supporters to be valid. } If an announcement is made about anything apart from the monastery or its vicinity, this is called “without making an announcement”. %
\item[Making an announcement: ] if a woman or a man comes to the monastery or the vicinity of the monastery and says, “Venerable, people will be bringing so-and-so’s fresh or cooked food,” then, if it is risky, this should be declared, and if it is dangerous, that should be declared.\footnote{Presumably this refers to a servant or a messenger coming to the monastery to let the monks know of the forthcoming offering of food. The “people” would then be other servants bringing the food on behalf of the donor. } If the person says, “Never mind, it will be brought,” then the criminals are to be told, “People are coming here; go away.” When an announcement has been made in regard to congee, and accompanying food is brought, this is called “announced”. When an announcement has been made in regard to rice, and accompanying food is brought, this is called “announced”. When an announcement has been made in regard to fresh food, and accompanying food is brought, this is called “announced”. When an announcement has been made in regard to a particular family, then when any person from that family brings fresh or cooked food, this is called “announced”. When an announcement has been made in regard to a particular village, then when any person from that village brings fresh or cooked food, this is called “announced”. When an announcement has been made in regard to a particular association, then when any person from that association brings fresh or cooked food, this is called “announced”. %
\item[Fresh food: ] apart from the five cooked foods, the post-midday tonics, the seven-day tonics, and the lifetime tonics—the rest is called “fresh food”. %
\item[Cooked food: ] there are five kinds of cooked food: cooked grain, porridge, flour products, fish, and meat. %
\item[Inside that monastery: ] if the monastery is enclosed, then within the enclosure. If the monastery is unenclosed, then in the vicinity. %
\item[Who is not sick: ] who is able to walk for alms. %
\item[Who is sick: ] who is unable to walk for alms. %
\end{description}

If,\marginnote{3.1.36} without making an announcement, one who is not sick receives fresh or cooked food with the intention of eating it, he commits an offense of wrong conduct. For every mouthful swallowed, he commits an offense entailing acknowledgment. 

\subsection*{Permutations }

If\marginnote{3.1.38.1} there has been no announcement, and he does not perceive that there has, and he, not being sick, eats fresh or cooked food after personally receiving it inside that monastery, then he commits an offense entailing acknowledgment. If there has been no announcement, but he is unsure of it, and he, not being sick, eats fresh or cooked food after personally receiving it inside that monastery, then he commits an offense entailing acknowledgment. If there has been no announcement, but he perceives that there has, and he, not being sick, eats fresh or cooked food after personally receiving it inside that monastery, then he commits an offense entailing acknowledgment. 

If\marginnote{3.1.41} he receives post-midday tonics, seven-day tonics, or lifetime tonics for the purpose of food, he commits an offense of wrong conduct. For every mouthful swallowed, he commits an offense of wrong conduct. If there has been an announcement, but he does not perceive that there has, he commits an offense of wrong conduct. If there has been an announcement, but he is unsure of it, he commits an offense of wrong conduct. If there has been an announcement, and he perceives that there has, there is no offense. 

\subsection*{Non-offenses }

There\marginnote{3.2.1} is no offense: if there has been an announcement;  if he is sick;  if he eats the leftovers from where there has been an announcement or from one who is sick;  if he receives the food outside the monastery and then eats it inside;  if he eats a root, bark, a leaf, a flower, or a fruit originating in that monastery;  if, when there is a reason, he uses post-midday tonics, seven-day tonics, or lifetime tonics;  if he is insane;  if he is the first offender. 

\scendsutta{The fourth training rule entailing acknowledgment is finished. }

“Venerables,\marginnote{3.2.11} the four rules on acknowledgment have been recited. In regard to this I ask you, ‘Are you pure in this?’ A second time I ask, ‘Are you pure in this?’ A third time I ask, ‘Are you pure in this?’ You are pure in this and therefore silent. I’ll remember it thus.” 

\scendkanda{The chapter on offenses entailing acknowledgment is finished. }

%
\addtocontents{toc}{\let\protect\contentsline\protect\nopagecontentsline}
\chapter*{Rules for Training }
\addcontentsline{toc}{chapter}{\tocchapterline{Rules for Training }}
\addtocontents{toc}{\let\protect\contentsline\protect\oldcontentsline}

%
%
\section*{{\suttatitleacronym Bu Sk 1}{\suttatitletranslation 1. The training rule on evenly all around }{\suttatitleroot Parimaṇḍala}}
\addcontentsline{toc}{section}{\tocacronym{Bu Sk 1} \toctranslation{1. The training rule on evenly all around } \tocroot{Parimaṇḍala}}
\markboth{1. The training rule on evenly all around }{Parimaṇḍala}
\extramarks{Bu Sk 1}{Bu Sk 1}

Venerables,\marginnote{0.6} these rules to be trained in come up for recitation. 

\subsection*{Origin story }

At\marginnote{1.1} one time when the Buddha was staying at \textsanskrit{Sāvatthī} in \textsanskrit{Anāthapiṇḍika}’s Monastery, the monks from the group of six were wearing their sarongs hanging down in front and behind. People complained and criticized them, “How can the Sakyan monastics wear their sarongs hanging down in front and behind? They’re just like householders who indulge in worldly pleasures!” 

The\marginnote{1.5} monks heard the complaints of those people, and the monks of few desires complained and criticized those monks, “How can the monks from the group of six wear their sarongs hanging down in front and behind?” 

After\marginnote{1.8} rebuking those monks in many ways, they told the Buddha. Soon afterwards the Buddha had the Sangha gathered and questioned the monks from the group of six: “Is it true, monks, that you do this?” 

“It’s\marginnote{1.11} true, sir.” 

The\marginnote{1.12} Buddha rebuked them … “Foolish men, how can you do this? This will affect people’s confidence …” … “And, monks, this training rule should be recited like this: 

\subsection*{Final ruling }

\scrule{‘“I will wear my sarong evenly all around,” this is how you should train.’” }

One\marginnote{1.17} should wear one’s sarong evenly all around, covering the navel and the knees. If a monk, out of disrespect, wears his sarong hanging down in front or behind, he commits an offense of wrong conduct. 

\subsection*{Non-offenses }

There\marginnote{1.19.1} is no offense: if it is unintentional;  if he is not mindful;  if he does not know;  if he is sick;  if there is an emergency;  if he is insane;  if he is the first offender. 

\scendsutta{The first training rule is finished. }

%
\section*{{\suttatitleacronym Bu Sk 2}{\suttatitletranslation 2. The second training rule on evenly all around }{\suttatitleroot Dutiyaparimaṇḍala}}
\addcontentsline{toc}{section}{\tocacronym{Bu Sk 2} \toctranslation{2. The second training rule on evenly all around } \tocroot{Dutiyaparimaṇḍala}}
\markboth{2. The second training rule on evenly all around }{Dutiyaparimaṇḍala}
\extramarks{Bu Sk 2}{Bu Sk 2}

\subsection*{Origin story }

At\marginnote{1.1} one time the Buddha was staying at \textsanskrit{Sāvatthī} in the Jeta Grove, \textsanskrit{Anāthapiṇḍika}’s Monastery. At that time the monks from the group of six wore their upper robes hanging down in front and behind. … 

\subsection*{Final ruling }

\scrule{“‘I will wear my upper robe evenly all around,’ this is how you should train.” }

One\marginnote{1.4} should wear one’s upper robe evenly all around, making both corners even. If a monk, out of disrespect, wears his upper robe hanging down in front or behind, he commits an offense of wrong conduct. 

\subsection*{Non-offenses }

There\marginnote{1.6.1} is no offense: if it is unintentional;  if he is not mindful;  if he does not know;  if he is sick;  if there is an emergency;  if he is insane;  if he is the first offender. 

\scendsutta{The second training rule is finished. }

%
\section*{{\suttatitleacronym Bu Sk 3}{\suttatitletranslation 3. The training rule on well-covered }{\suttatitleroot Suppaṭicchanna}}
\addcontentsline{toc}{section}{\tocacronym{Bu Sk 3} \toctranslation{3. The training rule on well-covered } \tocroot{Suppaṭicchanna}}
\markboth{3. The training rule on well-covered }{Suppaṭicchanna}
\extramarks{Bu Sk 3}{Bu Sk 3}

\subsection*{Origin story }

At\marginnote{1.1} one time the Buddha was staying at \textsanskrit{Sāvatthī} in the Jeta Grove, \textsanskrit{Anāthapiṇḍika}’s Monastery. At that time the monks from the group of six did not cover their bodies while walking in inhabited areas. … 

\subsection*{Final ruling }

\scrule{“‘I will be well-covered while walking in inhabited areas,’ this is how you should train.” }

One\marginnote{1.4} should be well-covered while walking in an inhabited area. If a monk, out of disrespect, does not cover his body while walking in an inhabited area, he commits an offense of wrong conduct.\footnote{Sp 2.578: \textit{\textsanskrit{Kāyaṁ} \textsanskrit{vivaritvāti} jattumpi urampi \textsanskrit{vivaritvā}}, “Does not cover his body means shoulder or chest uncovered.” } 

\subsection*{Non-offenses }

There\marginnote{1.6.1} is no offense: if it is unintentional;  if he is not mindful;  if he does not know;  if he is sick;  if there is an emergency;  if he is insane;  if he is the first offender. 

\scendsutta{The third training rule is finished. }

%
\section*{{\suttatitleacronym Bu Sk 4}{\suttatitletranslation 4. The second training rule on well-covered }{\suttatitleroot Dutiyasuppaṭicchanna}}
\addcontentsline{toc}{section}{\tocacronym{Bu Sk 4} \toctranslation{4. The second training rule on well-covered } \tocroot{Dutiyasuppaṭicchanna}}
\markboth{4. The second training rule on well-covered }{Dutiyasuppaṭicchanna}
\extramarks{Bu Sk 4}{Bu Sk 4}

\subsection*{Origin story }

At\marginnote{1.1} one time the Buddha was staying at \textsanskrit{Sāvatthī} in the Jeta Grove, \textsanskrit{Anāthapiṇḍika}’s Monastery. At that time the monks from the group of six did not cover their bodies while sitting in inhabited areas. … 

\subsection*{Final ruling }

\scrule{“‘I will be well-covered while sitting in inhabited areas,’ this is how you should train.” }

One\marginnote{1.4} should be well-covered while sitting in an inhabited area. If a monk, out of disrespect, does not cover his body while sitting in an inhabited area, he commits an offense of wrong conduct.\footnote{Sp 2.578: \textit{\textsanskrit{Kāyaṁ} \textsanskrit{vivaritvāti} jattumpi urampi \textsanskrit{vivaritvā}}, “Does not cover his body means shoulder or chest uncovered.” } 

\subsection*{Non-offenses }

There\marginnote{1.6.1} is no offense: if it is unintentional;  if he is not mindful;  if he does not know;  if he is sick;  if he has entered his dwelling;  if there is an emergency;  if he is insane;  if he is the first offender. 

\scendsutta{The fourth training rule is finished. }

%
\section*{{\suttatitleacronym Bu Sk 5}{\suttatitletranslation 5. The training rule on well-restrained }{\suttatitleroot Susaṁvuta}}
\addcontentsline{toc}{section}{\tocacronym{Bu Sk 5} \toctranslation{5. The training rule on well-restrained } \tocroot{Susaṁvuta}}
\markboth{5. The training rule on well-restrained }{Susaṁvuta}
\extramarks{Bu Sk 5}{Bu Sk 5}

\subsection*{Origin story }

At\marginnote{1.1} one time the Buddha was staying at \textsanskrit{Sāvatthī} in the Jeta Grove, \textsanskrit{Anāthapiṇḍika}’s Monastery. At that time the monks from the group of six were playing with their hands and feet while walking in inhabited areas. … 

\subsection*{Final ruling }

\scrule{“‘I will be well-restrained while walking in inhabited areas,’ this is how you should train.” }

One\marginnote{1.4} should be well-restrained while walking in an inhabited area. If a monk, out of disrespect, plays with his hands or feet while walking in an inhabited area, he commits an offense of wrong conduct. 

\subsection*{Non-offenses }

There\marginnote{1.6.1} is no offense: if it is unintentional;  if he is not mindful;  if he does not know;  if he is sick;  if he is insane;  if he is the first offender. 

\scendsutta{The fifth training rule is finished. }

%
\section*{{\suttatitleacronym Bu Sk 6}{\suttatitletranslation 6. The second training rule on well-restrained }{\suttatitleroot Dutiyasusaṁvuta}}
\addcontentsline{toc}{section}{\tocacronym{Bu Sk 6} \toctranslation{6. The second training rule on well-restrained } \tocroot{Dutiyasusaṁvuta}}
\markboth{6. The second training rule on well-restrained }{Dutiyasusaṁvuta}
\extramarks{Bu Sk 6}{Bu Sk 6}

\subsection*{Origin story }

At\marginnote{1.1} one time the Buddha was staying at \textsanskrit{Sāvatthī} in the Jeta Grove, \textsanskrit{Anāthapiṇḍika}’s Monastery. At that time the monks from the group of six were playing with their hands and feet while sitting in inhabited areas. … 

\subsection*{Final ruling }

\scrule{“‘I will be well-restrained while sitting in inhabited areas,’ this is how you should train.” }

One\marginnote{1.4} should be well-restrained while sitting in an inhabited area. If a monk, out of disrespect, plays with his hands or feet while sitting in an inhabited area, he commits an offense of wrong conduct. 

\subsection*{Non-offenses }

There\marginnote{1.6.1} is no offense: if it is unintentional;  if he is not mindful;  if he does not know;  if he is sick;  if he is insane;  if he is the first offender. 

\scendsutta{The sixth training rule is finished. }

%
\section*{{\suttatitleacronym Bu Sk 7}{\suttatitletranslation 7. The training rule on lowered eyes }{\suttatitleroot Okkhittacakkhu}}
\addcontentsline{toc}{section}{\tocacronym{Bu Sk 7} \toctranslation{7. The training rule on lowered eyes } \tocroot{Okkhittacakkhu}}
\markboth{7. The training rule on lowered eyes }{Okkhittacakkhu}
\extramarks{Bu Sk 7}{Bu Sk 7}

\subsection*{Origin story }

At\marginnote{1.1} one time the Buddha was staying at \textsanskrit{Sāvatthī} in the Jeta Grove, \textsanskrit{Anāthapiṇḍika}’s Monastery. At that time the monks from the group of six were looking here and there while walking in inhabited areas. … 

\subsection*{Final ruling }

\scrule{“‘I will lower my eyes while walking in inhabited areas,’ this is how you should train.” }

One\marginnote{1.4} should lower one’s eyes while walking in an inhabited area, looking a plow’s length ahead. If a monk, out of disrespect, looks here and there while walking in an inhabited area, he commits an offense of wrong conduct. 

\subsection*{Non-offenses }

There\marginnote{1.6.1} is no offense: if it is unintentional;  if he is not mindful;  if he does not know;  if he is sick;  if there is an emergency;  if he is insane;  if he is the first offender. 

\scendsutta{The seventh training rule is finished. }

%
\section*{{\suttatitleacronym Bu Sk 8}{\suttatitletranslation 8. The second training rule on lowered eyes }{\suttatitleroot Dutiyaokkhittacakkhu}}
\addcontentsline{toc}{section}{\tocacronym{Bu Sk 8} \toctranslation{8. The second training rule on lowered eyes } \tocroot{Dutiyaokkhittacakkhu}}
\markboth{8. The second training rule on lowered eyes }{Dutiyaokkhittacakkhu}
\extramarks{Bu Sk 8}{Bu Sk 8}

\subsection*{Origin story }

At\marginnote{1.1} one time the Buddha was staying at \textsanskrit{Sāvatthī} in the Jeta Grove, \textsanskrit{Anāthapiṇḍika}’s Monastery. At that time the monks from the group of six were looking here and there while sitting in inhabited areas. … 

\subsection*{Final ruling }

\scrule{“‘I will lower my eyes while sitting in inhabited areas,’ this is how you should train.” }

One\marginnote{1.4} should lower one’s eyes while sitting in an inhabited area, looking a plow’s length ahead. If a monk, out of disrespect, looks here and there while sitting in an inhabited area, he commits an offense of wrong conduct. 

\subsection*{Non-offenses }

There\marginnote{1.6.1} is no offense: if it is unintentional;  if he is not mindful;  if he does not know;  if he is sick;  if there is an emergency;  if he is insane;  if he is the first offender. 

\scendsutta{The eighth training rule is finished. }

%
\section*{{\suttatitleacronym Bu Sk 9}{\suttatitletranslation 9. The training rule on lifted robes }{\suttatitleroot Ukkhittaka}}
\addcontentsline{toc}{section}{\tocacronym{Bu Sk 9} \toctranslation{9. The training rule on lifted robes } \tocroot{Ukkhittaka}}
\markboth{9. The training rule on lifted robes }{Ukkhittaka}
\extramarks{Bu Sk 9}{Bu Sk 9}

\subsection*{Origin story }

At\marginnote{1.1} one time the Buddha was staying at \textsanskrit{Sāvatthī} in the Jeta Grove, \textsanskrit{Anāthapiṇḍika}’s Monastery. At that time the monks from the group of six were lifting their robes while walking in inhabited areas. … 

\subsection*{Final ruling }

\scrule{“‘I will not lift my robe while walking in inhabited areas,’ this is how you should train.” }

One\marginnote{1.4} should not lift one’s robe while walking in an inhabited area. If a monk, out of disrespect, lifts his robe on one or both sides while walking in an inhabited area, he commits an offense of wrong conduct. 

\subsection*{Non-offenses }

There\marginnote{1.6.1} is no offense:: if it is unintentional;  if he is not mindful;  if he does not know;  if he is sick;  if there is an emergency;  if he is insane;  if he is the first offender. 

\scendsutta{The ninth training rule is finished. }

%
\section*{{\suttatitleacronym Bu Sk 10}{\suttatitletranslation 10. The second training rule on lifted robes }{\suttatitleroot Dutiyaukkhittaka}}
\addcontentsline{toc}{section}{\tocacronym{Bu Sk 10} \toctranslation{10. The second training rule on lifted robes } \tocroot{Dutiyaukkhittaka}}
\markboth{10. The second training rule on lifted robes }{Dutiyaukkhittaka}
\extramarks{Bu Sk 10}{Bu Sk 10}

\subsection*{Origin story }

At\marginnote{1.1} one time the Buddha was staying at \textsanskrit{Sāvatthī} in the Jeta Grove, \textsanskrit{Anāthapiṇḍika}’s Monastery. At that time the monks from the group of six were lifting their robes while sitting in inhabited areas. … 

\subsection*{Final ruling }

\scrule{“‘I will not lift my robe while sitting in inhabited areas,’ this is how you should train.” }

One\marginnote{1.4} should not lift one’s robe while sitting in an inhabited area. If a monk, out of disrespect, lifts his robe on one or both sides while sitting in an inhabited area, he commits an offense of wrong conduct. 

\subsection*{Non-offenses }

There\marginnote{1.6.1} is no offense: if it is unintentional;  if he is not mindful;  if he does not know;  if he is sick;  if he has entered his dwelling;  if there is an emergency;  if he is insane;  if he is the first offender. 

\scendsutta{The tenth training rule is finished. }

\scendvagga{The first subchapter on evenly all around is finished. }

%
\section*{{\suttatitleacronym Bu Sk 11}{\suttatitletranslation 11. The training rule on laughing loudly }{\suttatitleroot Ujjagghika}}
\addcontentsline{toc}{section}{\tocacronym{Bu Sk 11} \toctranslation{11. The training rule on laughing loudly } \tocroot{Ujjagghika}}
\markboth{11. The training rule on laughing loudly }{Ujjagghika}
\extramarks{Bu Sk 11}{Bu Sk 11}

\subsection*{Origin story }

At\marginnote{1.1} one time the Buddha was staying at \textsanskrit{Sāvatthī} in the Jeta Grove, \textsanskrit{Anāthapiṇḍika}’s Monastery. At that time the monks from the group of six were laughing loudly while walking in inhabited areas. … 

\subsection*{Final ruling }

\scrule{“‘I will not laugh loudly while walking in inhabited areas,’ this is how you should train.” }

One\marginnote{1.4} should not laugh loudly while walking in an inhabited area. If a monk, out of disrespect, laughs loudly while walking in an inhabited area, he commits an offense of wrong conduct. 

\subsection*{Non-offenses }

There\marginnote{1.6.1} is no offense: if it is unintentional;  if he is not mindful;  if he does not know;  if he is sick;  if he merely smiles when something is funny;  if there is an emergency;  if he is insane;  if he is the first offender. 

\scendsutta{The first training rule is finished. }

%
\section*{{\suttatitleacronym Bu Sk 12}{\suttatitletranslation 12. The second training rule on laughing loudly }{\suttatitleroot Dutiyaujjagghika}}
\addcontentsline{toc}{section}{\tocacronym{Bu Sk 12} \toctranslation{12. The second training rule on laughing loudly } \tocroot{Dutiyaujjagghika}}
\markboth{12. The second training rule on laughing loudly }{Dutiyaujjagghika}
\extramarks{Bu Sk 12}{Bu Sk 12}

\subsection*{Origin story }

At\marginnote{1.1} one time the Buddha was staying at \textsanskrit{Sāvatthī} in the Jeta Grove, \textsanskrit{Anāthapiṇḍika}’s Monastery. At that time the monks from the group of six were laughing loudly while sitting in inhabited areas. … 

\subsection*{Final ruling }

\scrule{“‘I will not laugh loudly while sitting in inhabited areas,’ this is how you should train.” }

One\marginnote{1.4} should not laugh loudly while sitting in an inhabited area. If a monk, out of disrespect, laughs loudly while sitting in an inhabited area, he commits an offense of wrong conduct. 

\subsection*{Non-offenses }

There\marginnote{1.6.1} is no offense: if it is unintentional;  if he is not mindful;  if he does not know;  if he is sick;  if he merely smiles when something is funny;  if there is an emergency;  if he is insane;  if he is the first offender. 

\scendsutta{The second training rule is finished. }

%
\section*{{\suttatitleacronym Bu Sk 13}{\suttatitletranslation 13. The training rule on being noisy }{\suttatitleroot Appasadda}}
\addcontentsline{toc}{section}{\tocacronym{Bu Sk 13} \toctranslation{13. The training rule on being noisy } \tocroot{Appasadda}}
\markboth{13. The training rule on being noisy }{Appasadda}
\extramarks{Bu Sk 13}{Bu Sk 13}

\subsection*{Origin story }

At\marginnote{1.1} one time the Buddha was staying at \textsanskrit{Sāvatthī} in the Jeta Grove, \textsanskrit{Anāthapiṇḍika}’s Monastery. At that time the monks from the group of six were noisy while walking in inhabited areas. … 

\subsection*{Final ruling }

\scrule{“‘I will not be noisy while walking in inhabited areas,’ this is how you should train.” }

One\marginnote{1.4} should not be noisy while walking in an inhabited area. If a monk, out of disrespect, is noisy while walking in an inhabited area, he commits an offense of wrong conduct. 

\subsection*{Non-offenses }

There\marginnote{1.6.1} is no offense: if it is unintentional;  if he is not mindful;  if he does not know;  if he is sick;  if there is an emergency;  if he is insane;  if he is the first offender. 

\scendsutta{The third training rule is finished. }

%
\section*{{\suttatitleacronym Bu Sk 14}{\suttatitletranslation 14. The second training rule on being noisy }{\suttatitleroot Dutiyaappasadda}}
\addcontentsline{toc}{section}{\tocacronym{Bu Sk 14} \toctranslation{14. The second training rule on being noisy } \tocroot{Dutiyaappasadda}}
\markboth{14. The second training rule on being noisy }{Dutiyaappasadda}
\extramarks{Bu Sk 14}{Bu Sk 14}

\subsection*{Origin story }

At\marginnote{1.1} one time the Buddha was staying at \textsanskrit{Sāvatthī} in the Jeta Grove, \textsanskrit{Anāthapiṇḍika}’s Monastery. At that time the monks from the group of six were noisy while sitting in inhabited areas. … 

\subsection*{Final ruling }

\scrule{“‘I will not be noisy while sitting in inhabited areas,’ this is how you should train.” }

One\marginnote{1.4} should not be noisy while sitting in an inhabited area. If a monk, out of disrespect, is noisy while sitting in an inhabited area, he commits an offense of wrong conduct. 

\subsection*{Non-offenses }

There\marginnote{1.6.1} is no offense: if it is unintentional;  if he is not mindful;  if he does not know;  if he is sick;  if there is an emergency;  if he is insane;  if he is the first offender. 

\scendsutta{The fourth training rule is finished. }

%
\section*{{\suttatitleacronym Bu Sk 15}{\suttatitletranslation 15. The training rule on swaying the body }{\suttatitleroot Kāyappacālaka}}
\addcontentsline{toc}{section}{\tocacronym{Bu Sk 15} \toctranslation{15. The training rule on swaying the body } \tocroot{Kāyappacālaka}}
\markboth{15. The training rule on swaying the body }{Kāyappacālaka}
\extramarks{Bu Sk 15}{Bu Sk 15}

\subsection*{Origin story }

At\marginnote{1.1} one time the Buddha was staying at \textsanskrit{Sāvatthī} in the Jeta Grove, \textsanskrit{Anāthapiṇḍika}’s Monastery. At that time the monks from the group of six were swaying and leaning their bodies while walking in inhabited areas. … 

\subsection*{Final ruling }

\scrule{“‘I will not sway my body while walking in inhabited areas,’ this is how you should train.” }

One\marginnote{1.4} should not sway one’s body while walking in an inhabited area; one should walk keeping one’s body straight.\footnote{Sp 2.590: \textit{\textsanskrit{Kāyaṁ} \textsanskrit{paggahetvāti} \textsanskrit{niccalaṁ} \textsanskrit{katvā} ujukena \textsanskrit{kāyena} samena \textsanskrit{iriyāpathena} \textsanskrit{gantabbañceva} \textsanskrit{nisīditabbañca}}, \textit{\textsanskrit{Kāyaṁ} \textsanskrit{paggahetvā}}: “one should walk and sit with a steady and straight body, and with upright physical behavior.” } If a monk, out of disrespect, sways and leans his body while walking in an inhabited area, he commits an offense of wrong conduct. 

\subsection*{Non-offenses }

There\marginnote{1.7.1} is no offense: if it is unintentional;  if he is not mindful;  if he does not know;  if he is sick;  if there is an emergency;  if he is insane;  if he is the first offender. 

\scendsutta{The fifth training rule is finished. }

%
\section*{{\suttatitleacronym Bu Sk 16}{\suttatitletranslation 16. The second training rule on swaying the body }{\suttatitleroot Dutiyakāyappacālaka}}
\addcontentsline{toc}{section}{\tocacronym{Bu Sk 16} \toctranslation{16. The second training rule on swaying the body } \tocroot{Dutiyakāyappacālaka}}
\markboth{16. The second training rule on swaying the body }{Dutiyakāyappacālaka}
\extramarks{Bu Sk 16}{Bu Sk 16}

\subsection*{Origin story }

At\marginnote{1.1} one time the Buddha was staying at \textsanskrit{Sāvatthī} in the Jeta Grove, \textsanskrit{Anāthapiṇḍika}’s Monastery. At that time the monks from the group of six were swaying and leaning their bodies while sitting in inhabited areas. … 

\subsection*{Final ruling }

\scrule{“‘I will not sway my body while sitting in inhabited areas,’ this is how you should train.” }

One\marginnote{1.4} should not sway one’s body while sitting in an inhabited area; one should sit keeping one’s body straight. If a monk, out of disrespect, sways and leans his body while sitting in an inhabited area, he commits an offense of wrong conduct. 

\subsection*{Non-offenses }

There\marginnote{1.7.1} is no offense: if it is unintentional;  if he is not mindful;  if he does not know;  if he is sick;  if he has entered his dwelling;  if there is an emergency;  if he is insane;  if he is the first offender. 

\scendsutta{The sixth training rule is finished. }

%
\section*{{\suttatitleacronym Bu Sk 17}{\suttatitletranslation 17. The training rule on swinging the arms }{\suttatitleroot Bāhuppacālaka}}
\addcontentsline{toc}{section}{\tocacronym{Bu Sk 17} \toctranslation{17. The training rule on swinging the arms } \tocroot{Bāhuppacālaka}}
\markboth{17. The training rule on swinging the arms }{Bāhuppacālaka}
\extramarks{Bu Sk 17}{Bu Sk 17}

\subsection*{Origin story }

At\marginnote{1.1} one time the Buddha was staying at \textsanskrit{Sāvatthī} in the Jeta Grove, \textsanskrit{Anāthapiṇḍika}’s Monastery. At that time the monks from the group of six were swinging and dangling their arms while walking in inhabited areas. … 

\subsection*{Final ruling }

\scrule{“‘I will not swing my arms while walking in inhabited areas,’ this is how you should train.” }

One\marginnote{1.4} should not swing one’s arms while walking in an inhabited area; one should walk keeping one’s arms steady. If a monk, out of disrespect, swings and dangles his arms while walking in an inhabited area, he commits an offense of wrong conduct. 

\subsection*{Non-offenses }

There\marginnote{1.7.1} is no offense: if it is unintentional;  if he is not mindful;  if he does not know;  if he is sick;  if there is an emergency;  if he is insane;  if he is the first offender. 

\scendsutta{The seventh training rule is finished. }

%
\section*{{\suttatitleacronym Bu Sk 18}{\suttatitletranslation 18. The second training rule on swinging the arms }{\suttatitleroot Dutiyabāhuppacālaka}}
\addcontentsline{toc}{section}{\tocacronym{Bu Sk 18} \toctranslation{18. The second training rule on swinging the arms } \tocroot{Dutiyabāhuppacālaka}}
\markboth{18. The second training rule on swinging the arms }{Dutiyabāhuppacālaka}
\extramarks{Bu Sk 18}{Bu Sk 18}

\subsection*{Origin story }

At\marginnote{1.1} one time the Buddha was staying at \textsanskrit{Sāvatthī} in the Jeta Grove, \textsanskrit{Anāthapiṇḍika}’s Monastery. At that time the monks from the group of six were swinging and dangling their arms while sitting in inhabited areas. … 

\subsection*{Final ruling }

\scrule{“‘I will not swing my arms while sitting in inhabited areas,’ this is how you should train.” }

One\marginnote{1.4} should not swing one’s arms while sitting in an inhabited area; one should sit keeping one’s arms steady. If a monk, out of disrespect, swings and dangles his arms while sitting in an inhabited area, he commits an offense of wrong conduct. 

\subsection*{Non-offenses }

There\marginnote{1.7.1} is no offense: if it is unintentional;  if he is not mindful;  if he does not know;  if he is sick;  if he has entered his dwelling;  if there is an emergency;  if he is insane;  if he is the first offender. 

\scendsutta{The eighth training rule is finished. }

%
\section*{{\suttatitleacronym Bu Sk 19}{\suttatitletranslation 19. The training rule on swaying the head }{\suttatitleroot Sīsappacālaka}}
\addcontentsline{toc}{section}{\tocacronym{Bu Sk 19} \toctranslation{19. The training rule on swaying the head } \tocroot{Sīsappacālaka}}
\markboth{19. The training rule on swaying the head }{Sīsappacālaka}
\extramarks{Bu Sk 19}{Bu Sk 19}

\subsection*{Origin story }

At\marginnote{1.1} one time the Buddha was staying at \textsanskrit{Sāvatthī} in the Jeta Grove, \textsanskrit{Anāthapiṇḍika}’s Monastery. At that time the monks from the group of six were swaying and tilting their heads while walking in inhabited areas. … 

\subsection*{Final ruling }

\scrule{“‘I will not sway my head while walking in inhabited areas,’ this is how you should train.” }

One\marginnote{1.4} should not sway one’s head while walking in an inhabited area; one should walk keeping one’s head straight. If a monk, out of disrespect, sways and tilts his head while walking in an inhabited area, he commits an offense of wrong conduct. 

\subsection*{Non-offenses }

There\marginnote{1.7.1} is no offense: if it is unintentional;  if he is not mindful;  if he does not know;  if he is sick;  if there is an emergency;  if he is insane;  if he is the first offender. 

\scendsutta{The ninth training rule is finished. }

%
\section*{{\suttatitleacronym Bu Sk 20}{\suttatitletranslation 20. The second training rule on swaying the head }{\suttatitleroot Dutiyasīsappacālaka}}
\addcontentsline{toc}{section}{\tocacronym{Bu Sk 20} \toctranslation{20. The second training rule on swaying the head } \tocroot{Dutiyasīsappacālaka}}
\markboth{20. The second training rule on swaying the head }{Dutiyasīsappacālaka}
\extramarks{Bu Sk 20}{Bu Sk 20}

\subsection*{Origin story }

At\marginnote{1.1} one time the Buddha was staying at \textsanskrit{Sāvatthī} in the Jeta Grove, \textsanskrit{Anāthapiṇḍika}’s Monastery. At that time the monks from the group of six were swaying and hanging their heads while sitting in inhabited areas. … 

\subsection*{Final ruling }

\scrule{“‘I will not sway my head while sitting in inhabited areas,’ this is how you should train.” }

One\marginnote{1.4} should not sway one’s head while sitting in an inhabited area; one should sit keeping one’s head straight. If a monk, out of disrespect, sways and hangs his head while sitting in an inhabited area, he commits an offense of wrong conduct. 

\subsection*{Non-offenses }

There\marginnote{1.7.1} is no offense: if it is unintentional;  if he is not mindful;  if he does not know;  if he is sick;  if he has entered his dwelling;  if there is an emergency;  if he is insane;  if he is the first offender. 

\scendsutta{The tenth training rule is finished. }

\scendvagga{The second subchapter on laughing loudly is finished. }

%
\section*{{\suttatitleacronym Bu Sk 21}{\suttatitletranslation 21. The training rule on hands on hips }{\suttatitleroot Khambhakata}}
\addcontentsline{toc}{section}{\tocacronym{Bu Sk 21} \toctranslation{21. The training rule on hands on hips } \tocroot{Khambhakata}}
\markboth{21. The training rule on hands on hips }{Khambhakata}
\extramarks{Bu Sk 21}{Bu Sk 21}

\subsection*{Origin story }

At\marginnote{1.1} one time the Buddha was staying at \textsanskrit{Sāvatthī} in the Jeta Grove, \textsanskrit{Anāthapiṇḍika}’s Monastery. At that time the monks from the group of six had their hands on their hips while walking in inhabited areas. … 

\subsection*{Final ruling }

\scrule{“‘I will not have my hands on my hips while walking in inhabited areas,’ this is how you should train.” }

One\marginnote{1.4} should not have one’s hands on one’s hips while walking in an inhabited area. If a monk, out of disrespect, has one or both hands on his hips while walking in an inhabited area, he commits an offense of wrong conduct. 

\subsection*{Non-offenses }

There\marginnote{1.6.1} is no offense: if it is unintentional;  if he is not mindful;  if he does not know;  if he is sick;  if there is an emergency;  if he is insane;  if he is the first offender. 

\scendsutta{The first training rule is finished. }

%
\section*{{\suttatitleacronym Bu Sk 22}{\suttatitletranslation 22. The second training rule on hands on hips }{\suttatitleroot Dutiyakhambhakata}}
\addcontentsline{toc}{section}{\tocacronym{Bu Sk 22} \toctranslation{22. The second training rule on hands on hips } \tocroot{Dutiyakhambhakata}}
\markboth{22. The second training rule on hands on hips }{Dutiyakhambhakata}
\extramarks{Bu Sk 22}{Bu Sk 22}

\subsection*{Origin story }

At\marginnote{1.1} one time the Buddha was staying at \textsanskrit{Sāvatthī} in the Jeta Grove, \textsanskrit{Anāthapiṇḍika}’s Monastery. At that time the monks from the group of six had their hands on their hips while sitting in inhabited areas. … 

\subsection*{Final ruling }

\scrule{“‘I will not have my hands on my hips while sitting in inhabited areas,’ this is how you should train.” }

One\marginnote{1.4} should not have one’s hands on one’s hips while sitting in an inhabited area. If a monk, out of disrespect, has one or both hands on his hips while sitting in an inhabited area, he commits an offense of wrong conduct. 

\subsection*{Non-offenses }

There\marginnote{1.6.1} is no offense: if it is unintentional;  if he is not mindful;  if he does not know;  if he is sick;  if he has entered his dwelling;  if there is an emergency;  if he is insane;  if he is the first offender. 

\scendsutta{The second training rule is finished. }

%
\section*{{\suttatitleacronym Bu Sk 23}{\suttatitletranslation 23. The training rule on covering the head }{\suttatitleroot Oguṇṭhita}}
\addcontentsline{toc}{section}{\tocacronym{Bu Sk 23} \toctranslation{23. The training rule on covering the head } \tocroot{Oguṇṭhita}}
\markboth{23. The training rule on covering the head }{Oguṇṭhita}
\extramarks{Bu Sk 23}{Bu Sk 23}

\subsection*{Origin story }

At\marginnote{1.1} one time the Buddha was staying at \textsanskrit{Sāvatthī} in the Jeta Grove, \textsanskrit{Anāthapiṇḍika}’s Monastery. At that time the monks from the group of six were covering their heads with their upper robes while walking in inhabited areas. … 

\subsection*{Final ruling }

\scrule{“‘I will not cover my head while walking in inhabited areas,’ this is how you should train.” }

One\marginnote{1.4} should not cover one’s head while walking in an inhabited area. If a monk, out of disrespect, covers his head with his upper robe while walking in an inhabited area, he commits an offense of wrong conduct. 

\subsection*{Non-offenses }

There\marginnote{1.6.1} is no offense: if it is unintentional;  if he is not mindful;  if he does not know;  if he is sick;  if there is an emergency;  if he is insane;  if he is the first offender. 

\scendsutta{The third training rule is finished. }

%
\section*{{\suttatitleacronym Bu Sk 24}{\suttatitletranslation 24. The second training rule on covering the head }{\suttatitleroot Dutiyaoguṇṭhita}}
\addcontentsline{toc}{section}{\tocacronym{Bu Sk 24} \toctranslation{24. The second training rule on covering the head } \tocroot{Dutiyaoguṇṭhita}}
\markboth{24. The second training rule on covering the head }{Dutiyaoguṇṭhita}
\extramarks{Bu Sk 24}{Bu Sk 24}

\subsection*{Origin story }

At\marginnote{1.1} one time the Buddha was staying at \textsanskrit{Sāvatthī} in the Jeta Grove, \textsanskrit{Anāthapiṇḍika}’s Monastery. At that time the monks from the group of six were covering their heads with their robes while sitting in inhabited areas. … 

\subsection*{Final ruling }

\scrule{“‘I will not cover my head while sitting in inhabited areas;’ this is how you should train.” }

One\marginnote{1.4} should not cover one’s head while sitting in an inhabited area. If a monk, out of disrespect, covers his head with his upper robe while sitting in an inhabited area, he commits an offense of wrong conduct. 

\subsection*{Non-offenses }

There\marginnote{1.6.1} is no offense: if it is unintentional;  if he is not mindful;  if he does not know;  if he is sick;  if he has entered his dwelling;  if there is an emergency;  if he is insane;  if he is the first offender. 

\scendsutta{The fourth training rule is finished. }

%
\section*{{\suttatitleacronym Bu Sk 25}{\suttatitletranslation 25. The training rule on squatting on the heels }{\suttatitleroot Ukkuṭika}}
\addcontentsline{toc}{section}{\tocacronym{Bu Sk 25} \toctranslation{25. The training rule on squatting on the heels } \tocroot{Ukkuṭika}}
\markboth{25. The training rule on squatting on the heels }{Ukkuṭika}
\extramarks{Bu Sk 25}{Bu Sk 25}

\subsection*{Origin story }

At\marginnote{1.1} one time the Buddha was staying at \textsanskrit{Sāvatthī} in the Jeta Grove, \textsanskrit{Anāthapiṇḍika}’s Monastery. At that time the monks from the group of six were moving about while squatting on their heels in inhabited areas. … 

\subsection*{Final ruling }

\scrule{“‘I will not move about while squatting on my heels in inhabited areas,’ this is how you should train.”\footnote{As at \href{https://suttacentral.net/pli-tv-bu-vb-sk15/en/brahmali\#1.3.1}{Bu Sk 15:1.3.1}, it seems that \textit{gacchati}, which normally means “to go”, here refers to any posture more upright than outright sitting down. Moreover, it seems to refer to any sort of movement from one place to another, however minor. } }

One\marginnote{1.4} should not move about while squatting on one’s heels in an inhabited area. If a monk, out of disrespect, moves about while squatting on his heels in an inhabited area, he commits an offense of wrong conduct. 

\subsection*{Non-offenses }

There\marginnote{1.6.1} is no offense: if it is unintentional;  if he is not mindful;  if he does not know;  if he is sick;  if there is an emergency;  if he is insane;  if he is the first offender. 

\scendsutta{The fifth training rule is finished. }

%
\section*{{\suttatitleacronym Bu Sk 26}{\suttatitletranslation 26. The training rule on sitting with clasped knees }{\suttatitleroot Pallatthika}}
\addcontentsline{toc}{section}{\tocacronym{Bu Sk 26} \toctranslation{26. The training rule on sitting with clasped knees } \tocroot{Pallatthika}}
\markboth{26. The training rule on sitting with clasped knees }{Pallatthika}
\extramarks{Bu Sk 26}{Bu Sk 26}

\subsection*{Origin story }

At\marginnote{1.1} one time the Buddha was staying at \textsanskrit{Sāvatthī} in the Jeta Grove, \textsanskrit{Anāthapiṇḍika}’s Monastery. At that time the monks from the group of six were clasping their knees while sitting in inhabited areas. … 

\subsection*{Final ruling }

\scrule{“‘I will not clasp my knees while sitting in inhabited areas,’ this is how you should train.” }

One\marginnote{1.4} should not clasp one’s knees while sitting in an inhabited area. If a monk, out of disrespect, clasps his knees with his hands or with a cloth while sitting in an inhabited area, he commits an offense of wrong conduct. 

\subsection*{Non-offenses }

There\marginnote{1.6.1} is no offense: if it is unintentional;  if he is not mindful;  if he does not know;  if he is sick;  if he has entered his dwelling;  if there is an emergency;  if he is insane;  if he is the first offender. 

\scendsutta{The sixth training rule is finished. }

%
\section*{{\suttatitleacronym Bu Sk 27}{\suttatitletranslation 27. The training rule on receiving respectfully }{\suttatitleroot Sakkaccapaṭiggahaṇa}}
\addcontentsline{toc}{section}{\tocacronym{Bu Sk 27} \toctranslation{27. The training rule on receiving respectfully } \tocroot{Sakkaccapaṭiggahaṇa}}
\markboth{27. The training rule on receiving respectfully }{Sakkaccapaṭiggahaṇa}
\extramarks{Bu Sk 27}{Bu Sk 27}

\subsection*{Origin story }

At\marginnote{1.1} one time the Buddha was staying at \textsanskrit{Sāvatthī} in the Jeta Grove, \textsanskrit{Anāthapiṇḍika}’s Monastery. At that time the monks from the group of six were receiving almsfood contemptuously, as if wanting to throw it away. … 

\subsection*{Final ruling }

\scrule{“‘I will receive almsfood respectfully,’ this is how you should train.” }

Almsfood\marginnote{1.4} is to be received respectfully. If a monk, out of disrespect, receives almsfood contemptuously, as if wanting to throw it away, he commits an offense of wrong conduct. 

\subsection*{Non-offenses }

There\marginnote{1.6.1} is no offense: if it is unintentional;  if he is not mindful;  if he does not know;  if he is sick;  if there is an emergency;  if he is insane;  if he is the first offender. 

\scendsutta{The seventh training rule is finished. }

%
\section*{{\suttatitleacronym Bu Sk 28}{\suttatitletranslation 28. The training rule on receiving with attention on the almsbowl }{\suttatitleroot Pattasaññīpaṭiggahaṇa}}
\addcontentsline{toc}{section}{\tocacronym{Bu Sk 28} \toctranslation{28. The training rule on receiving with attention on the almsbowl } \tocroot{Pattasaññīpaṭiggahaṇa}}
\markboth{28. The training rule on receiving with attention on the almsbowl }{Pattasaññīpaṭiggahaṇa}
\extramarks{Bu Sk 28}{Bu Sk 28}

\subsection*{Origin story }

At\marginnote{1.1} one time the Buddha was staying at \textsanskrit{Sāvatthī} in the Jeta Grove, \textsanskrit{Anāthapiṇḍika}’s Monastery. At that time the monks from the group of six were receiving almsfood while looking here and there, and they did not know whether people were still giving or whether they had received too much. … 

\subsection*{Final ruling }

\scrule{“‘I will receive almsfood with attention on the almsbowl,’ this is how you should train.” }

Almsfood\marginnote{1.4} is to be received with attention on the bowl. If a monk, out of disrespect, receives almsfood while looking here and there, he commits an offense of wrong conduct. 

\subsection*{Non-offenses }

There\marginnote{1.6.1} is no offense: if it is unintentional;  if he is not mindful;  if he does not know;  if he is sick;  if there is an emergency;  if he is insane;  if he is the first offender. 

\scendsutta{The eighth training rule is finished. }

%
\section*{{\suttatitleacronym Bu Sk 29}{\suttatitletranslation 29. The training rule on receiving the right proportion of bean curry }{\suttatitleroot Samasūpakapaṭiggahaṇa}}
\addcontentsline{toc}{section}{\tocacronym{Bu Sk 29} \toctranslation{29. The training rule on receiving the right proportion of bean curry } \tocroot{Samasūpakapaṭiggahaṇa}}
\markboth{29. The training rule on receiving the right proportion of bean curry }{Samasūpakapaṭiggahaṇa}
\extramarks{Bu Sk 29}{Bu Sk 29}

\subsection*{Origin story }

At\marginnote{1.1} one time the Buddha was staying at \textsanskrit{Sāvatthī} in the Jeta Grove, \textsanskrit{Anāthapiṇḍika}’s Monastery. At that time the monks from the group of six were receiving almsfood with large amounts of bean curry. … 

\subsection*{Final ruling }

\scrule{“‘I will receive almsfood with the right proportion of bean curry,’ this is how you should train.”\footnote{According to Sp 2.604 this means the bean curries should make up one quarter of the rice: \textit{\textsanskrit{Samasūpako} \textsanskrit{nāma} yattha bhattassa \textsanskrit{catutthabhāgappamāṇo} \textsanskrit{sūpo} hoti}, “Wherever the bean curry is one quarter of the rice is called \textit{\textsanskrit{samasūpaka}}.” } }

\subsection*{Definitions }

\begin{description}%
\item[Bean curry: ] there are two kinds of bean curry, to be taken with the hand: mung-bean curry and black-gram curry.\footnote{The punctuation of MS has been corrected here so as to agree with \href{https://suttacentral.net/pli-tv-bu-vb-sk34/en/brahmali\#1.6}{Bu Sk 34:1.6}. The text mistakenly had \textit{\textsanskrit{hatthahāriyo}} in the next sentence. } %
\end{description}

Almsfood\marginnote{1.6} should be received with the right proportion of bean curry. If a monk, out of disrespect, receives much bean curry, he commits an offense of wrong conduct. 

\subsection*{Non-offenses }

There\marginnote{1.8.1} is no offense: if it is unintentional;  if he is not mindful;  if he does not know;  if he is sick;  if it is any food apart from bean curry;\footnote{According to the \textsanskrit{Kaṅkhāvitaraṇī} commentary this means any curries apart from bean curry: \textit{\textsanskrit{Ṭhapetvā} pana \textsanskrit{sūpaṁ} \textsanskrit{avasesā} \textsanskrit{sabbā} pi \textsanskrit{sūpeyyabyañjanavikati} rasaraso \textsanskrit{nāma} hoti}, “But apart from bean curry, all kinds of curry are called \textit{rasarasa}.” My translation, however, is based on the fact that all other foods too are excluded. }  if it is from relatives;  if it is from those who have given an invitation;  if it is for the benefit of someone else;  if it is by means of his own property;  if there is an emergency;  if he is insane;  if he is the first offender. 

\scendsutta{The ninth training rule is finished. }

%
\section*{{\suttatitleacronym Bu Sk 30}{\suttatitletranslation 30. The training rule on even levels }{\suttatitleroot Samatittika}}
\addcontentsline{toc}{section}{\tocacronym{Bu Sk 30} \toctranslation{30. The training rule on even levels } \tocroot{Samatittika}}
\markboth{30. The training rule on even levels }{Samatittika}
\extramarks{Bu Sk 30}{Bu Sk 30}

\subsection*{Origin story }

At\marginnote{1.1} one time the Buddha was staying at \textsanskrit{Sāvatthī} in the Jeta Grove, \textsanskrit{Anāthapiṇḍika}’s Monastery. At that time the monks from the group of six received almsfood in a heap. … 

\subsection*{Final ruling }

\scrule{“‘I will receive an even level of almsfood,’ this is how you should train.” }

Almsfood\marginnote{1.4} is to be received at an even level. If a monk, out of disrespect, receives almsfood in a heap, he commits an offense of wrong conduct. 

\subsection*{Non-offenses }

There\marginnote{1.6.1} is no offense: if it is unintentional;  if he is not mindful;  if he does not know;  if there is an emergency;  if he is insane;  if he is the first offender. 

\scendsutta{The tenth training rule is finished. }

\scendvagga{The third subchapter on hands on hips is finished. }

%
\section*{{\suttatitleacronym Bu Sk 31}{\suttatitletranslation 31. The training rule on respectfully }{\suttatitleroot Sakkacca}}
\addcontentsline{toc}{section}{\tocacronym{Bu Sk 31} \toctranslation{31. The training rule on respectfully } \tocroot{Sakkacca}}
\markboth{31. The training rule on respectfully }{Sakkacca}
\extramarks{Bu Sk 31}{Bu Sk 31}

\subsection*{Origin story }

At\marginnote{1.1} one time the Buddha was staying at \textsanskrit{Sāvatthī} in the Jeta Grove, \textsanskrit{Anāthapiṇḍika}’s Monastery. At that time the monks from the group of six were eating almsfood contemptuously, as if not wanting to eat it. … 

\subsection*{Final ruling }

\scrule{“‘I will eat almsfood respectfully,’ this is how you should train.” }

Almsfood\marginnote{1.4} is to be eaten respectfully. If a monk, out of disrespect, eats almsfood contemptuously, he commits an offense of wrong conduct. 

\subsection*{Non-offenses }

There\marginnote{1.6.1} is no offense: if it is unintentional;  if he is not mindful;  if he does not know;  if he is sick;  if there is an emergency;  if he is insane;  if he is the first offender. 

\scendsutta{The first training rule is finished. }

%
\section*{{\suttatitleacronym Bu Sk 32}{\suttatitletranslation 32. The training rule on attention on the almsbowl }{\suttatitleroot Pattasaññī}}
\addcontentsline{toc}{section}{\tocacronym{Bu Sk 32} \toctranslation{32. The training rule on attention on the almsbowl } \tocroot{Pattasaññī}}
\markboth{32. The training rule on attention on the almsbowl }{Pattasaññī}
\extramarks{Bu Sk 32}{Bu Sk 32}

\subsection*{Origin story }

At\marginnote{1.1} one time the Buddha was staying at \textsanskrit{Sāvatthī} in the Jeta Grove, \textsanskrit{Anāthapiṇḍika}’s Monastery. At that time the monks from the group of six were eating almsfood while looking here and there, and they did not know whether people were still giving or whether they had received too much. … 

\subsection*{Final ruling }

\scrule{“‘I will eat almsfood with attention on the almsbowl,’ this is how you should train.” }

Almsfood\marginnote{1.4} is to be eaten with attention on the bowl. If a monk, out of disrespect, eats almsfood while looking here and there, he commits an offense of wrong conduct. 

\subsection*{Non-offenses }

There\marginnote{1.6.1} is no offense: if it is unintentional;  if he is not mindful;  if he does not know;  if he is sick;  if there is an emergency;  if he is insane;  if he is the first offender. 

\scendsutta{The second training rule is finished. }

%
\section*{{\suttatitleacronym Bu Sk 33}{\suttatitletranslation 33. The training rule on in order }{\suttatitleroot Sapadāna}}
\addcontentsline{toc}{section}{\tocacronym{Bu Sk 33} \toctranslation{33. The training rule on in order } \tocroot{Sapadāna}}
\markboth{33. The training rule on in order }{Sapadāna}
\extramarks{Bu Sk 33}{Bu Sk 33}

\subsection*{Origin story }

At\marginnote{1.1} one time the Buddha was staying at \textsanskrit{Sāvatthī} in the Jeta Grove, \textsanskrit{Anāthapiṇḍika}’s Monastery. At that time the monks from the group of six ate almsfood picking here and there. … 

\subsection*{Final ruling }

\scrule{“‘I will eat almsfood in order,’ this is how you should train.” }

Almsfood\marginnote{1.4} is to be eaten in order. If a monk, out of disrespect, eats almsfood picking here and there, he commits an offense of wrong conduct. 

\subsection*{Non-offenses }

There\marginnote{1.6.1} is no offense: if it is unintentional;  if he is not mindful;  if he does not know;  if he is sick;  if he picks things out to give to others;  if he picks things out to put it into someone else’s vessel;  if it is a non-bean curry;\footnote{According to Vin-vn-\textsanskrit{ṭ} 1907 \textit{\textsanskrit{uttaribhaṅga}} is just another word for \textit{\textsanskrit{byañjana}}: \textit{\textsanskrit{uttaribhaṅgaṁ} \textsanskrit{nāma} \textsanskrit{byañjanaṁ}}. \textit{\textsanskrit{Byañjana}} seems to refer to curries apart from the standard bean-curries, which are called \textit{\textsanskrit{sūpa}}. However, since \textit{\textsanskrit{uttaribhaṅga}} is a rare word, the basic meaning of which is something like “further condiments”, it is quite possible that it is here meant to include all foods apart from bean curries. }  if there is an emergency;  if he is insane;  if he is the first offender. 

\scendsutta{The third training rule is finished. }

%
\section*{{\suttatitleacronym Bu Sk 34}{\suttatitletranslation 34. The training rule on the right proportion of bean curry }{\suttatitleroot Samasūpaka}}
\addcontentsline{toc}{section}{\tocacronym{Bu Sk 34} \toctranslation{34. The training rule on the right proportion of bean curry } \tocroot{Samasūpaka}}
\markboth{34. The training rule on the right proportion of bean curry }{Samasūpaka}
\extramarks{Bu Sk 34}{Bu Sk 34}

\subsection*{Origin story }

At\marginnote{1.1} one time the Buddha was staying at \textsanskrit{Sāvatthī} in the Jeta Grove, \textsanskrit{Anāthapiṇḍika}’s Monastery. At that time the monks from the group of six were eating almsfood with large amounts of bean curry. … 

\subsection*{Final ruling }

\scrule{“‘I will eat almsfood with the right proportion of bean curry,’ this is how you should train.”\footnote{According to Sp 2.604 this means the bean curries should make up one quarter of the rice: \textit{\textsanskrit{Samasūpako} \textsanskrit{nāma} yattha bhattassa \textsanskrit{catutthabhāgappamāṇo} \textsanskrit{sūpo} hoti}, “Wherever the bean curry is one quarter of the rice is called \textit{\textsanskrit{samasūpaka}}.” } }

\subsection*{Definitions }

\begin{description}%
\item[Bean curry: ] there are two kinds of bean curry, to be taken with the hand: mung-bean curry and black-gram curry. %
\end{description}

Almsfood\marginnote{1.7} should be eaten with the right proportion of bean curry. If a monk, out of disrespect, eats much bean curry, he commits an offense of wrong conduct. 

\subsection*{Non-offenses }

There\marginnote{1.9.1} is no offense: if it is unintentional;  if he is not mindful;  if he does not know;  if he is sick;  if it is any food apart from bean curry;\footnote{According to Kkh this means any curries apart from bean curry: \textit{\textsanskrit{Ṭhapetvā} pana \textsanskrit{sūpaṁ} \textsanskrit{avasesā} \textsanskrit{sabbā} pi \textsanskrit{sūpeyyabyañjanavikati} rasaraso \textsanskrit{nāma} hoti}, “But apart from bean curry, all kinds of curry are called \textit{rasarasa}.” My translation, however, is based on the fact that all other foods too are excluded. }  if it is from relatives;  if it is from those who have given an invitation;  if it is by means of his own property;  if there is an emergency;  if he is insane;  if he is the first offender. 

\scendsutta{The fourth training rule is finished. }

%
\section*{{\suttatitleacronym Bu Sk 35}{\suttatitletranslation 35. The training rule on making a heap }{\suttatitleroot Thūpakata}}
\addcontentsline{toc}{section}{\tocacronym{Bu Sk 35} \toctranslation{35. The training rule on making a heap } \tocroot{Thūpakata}}
\markboth{35. The training rule on making a heap }{Thūpakata}
\extramarks{Bu Sk 35}{Bu Sk 35}

\subsection*{Origin story }

At\marginnote{1.1} one time the Buddha was staying at \textsanskrit{Sāvatthī} in the Jeta Grove, \textsanskrit{Anāthapiṇḍika}’s Monastery. At that time the monks from the group of six ate their almsfood after making a heap. … 

\subsection*{Final ruling }

\scrule{“‘I will not eat almsfood after making a heap,’ this is how you should train.” }

Almsfood\marginnote{1.4} should not be eaten after making a heap. If a monk, out of disrespect, eats almsfood after making heap, he commits an offense of wrong conduct. 

\subsection*{Non-offenses }

There\marginnote{1.6.1} is no offense: if it is unintentional;  if he is not mindful;  if he does not know;  if he is sick;  if only a little food remains and he eats it after collecting it in one place;  if there is an emergency;  if he is insane;  if he is the first offender. 

\scendsutta{The fifth training rule is finished. }

%
\section*{{\suttatitleacronym Bu Sk 36}{\suttatitletranslation 36. The training rule on covering with rice }{\suttatitleroot Odanappaṭicchādana}}
\addcontentsline{toc}{section}{\tocacronym{Bu Sk 36} \toctranslation{36. The training rule on covering with rice } \tocroot{Odanappaṭicchādana}}
\markboth{36. The training rule on covering with rice }{Odanappaṭicchādana}
\extramarks{Bu Sk 36}{Bu Sk 36}

\subsection*{Origin story }

At\marginnote{1.1} one time the Buddha was staying at \textsanskrit{Sāvatthī} in the Jeta Grove, \textsanskrit{Anāthapiṇḍika}’s Monastery. At that time the monks from the group of six were covering their curries with rice because they wanted more. … 

\subsection*{Final ruling }

\scrule{“‘I will not cover my curries with rice because I want more,’ this is how you should train.”\footnote{\textit{\textsanskrit{Sūpa}} is defined at \href{https://suttacentral.net/pli-tv-bu-vb-sk29/en/brahmali\#1.4.1}{Bu Sk 29:1.4.1} as bean curry. \textit{\textsanskrit{Byañjana}} is explained at \textsanskrit{Khuddasikkhā}-\textsanskrit{abhinavaṭīkā} 61 as \textit{\textsanskrit{macchamaṁsādibyañjanañca}}, “And \textit{\textsanskrit{byañjana}} of fish, meat, etc.” So \textit{\textsanskrit{byañjana}} seems to refer to fine curries, whereas \textit{\textsanskrit{sūpa}} is more ordinary. } }

One\marginnote{1.4} should not cover one’s curries with rice because one wants more. If a monk, out of disrespect, covers his curries with rice because he wants more, he commits an offense of wrong conduct. 

\subsection*{Non-offenses }

There\marginnote{1.6.1} is no offense: if it is unintentional;  if he is not mindful;  if he does not know;  if the owners cover it and then give;  if it is not out of a desire for more;  if there is an emergency;  if he is insane;  if he is the first offender. 

\scendsutta{The sixth training rule is finished. }

%
\section*{{\suttatitleacronym Bu Sk 37}{\suttatitletranslation 37. The training rule on requesting rice and bean curry }{\suttatitleroot Sūpodanaviññatti}}
\addcontentsline{toc}{section}{\tocacronym{Bu Sk 37} \toctranslation{37. The training rule on requesting rice and bean curry } \tocroot{Sūpodanaviññatti}}
\markboth{37. The training rule on requesting rice and bean curry }{Sūpodanaviññatti}
\extramarks{Bu Sk 37}{Bu Sk 37}

\subsection*{Origin story }

\subsubsection*{First sub-story }

At\marginnote{1.1} one time the Buddha was staying at \textsanskrit{Sāvatthī} in the Jeta Grove, \textsanskrit{Anāthapiṇḍika}’s Monastery. At that time the monks from the group of six ate rice and bean curry that they had requested themselves. People complained and criticized them, “How can the Sakyan monastics eat rice and bean curry that they have requested themselves? Who doesn’t like nice food? Who doesn’t prefer tasty food?” 

The\marginnote{1.7} monks heard the complaints of those people, and the monks of few desires complained and criticized those monks, “How can the monks from the group of six do this?” … “Is it true, monks, that you do this?” 

“It’s\marginnote{1.11} true, sir.” 

The\marginnote{1.12} Buddha rebuked them … “Foolish men, how can you do this? This will affect people’s confidence …” … “And, monks, this training rule should be recited like this: 

\subsubsection*{Preliminary ruling }

\scrule{‘“I will not request bean curry or rice for myself and then eat it,” this is how you should train.’” }

In\marginnote{1.17} this way the Buddha laid down this training rule for the monks. 

\subsubsection*{Second sub-story }

Soon\marginnote{2.1} afterwards a number of monks were sick. The monks who were looking after them asked, “I hope you’re bearing up? I hope you’re getting better?” 

“Previously\marginnote{2.4} we ate rice and bean curry that we had requested ourselves, and then we were comfortable. But now that the Buddha has prohibited this, we don’t ask because we’re afraid of wrongdoing. And because of that we’re not comfortable.” 

They\marginnote{2.7} told the Buddha. Soon afterwards he gave a teaching and addressed the monks: 

\scrule{“Monks, I allow a sick monk to eat rice and bean curry that he has requested for himself. }

And\marginnote{2.9} so, monks, this training rule should be recited like this: 

\subsection*{Final ruling }

\scrule{‘“When not sick, I will not request bean curry or rice for myself and then eat it,” this is how you should train.’” }

When\marginnote{2.11} one is not sick, one should not request bean curry or rice for oneself and then eat it. If a monk who is not sick, out of disrespect, eats bean curry or rice that he has requested for himself, he commits an offense of wrong conduct. 

\subsection*{Non-offenses }

There\marginnote{2.13.1} is no offense: if it is unintentional;  if he is not mindful;  if he does not know;  if he is sick;  if it is from relatives;  if it is from those who have given an invitation;  if it is for the benefit of someone else;  if it is by means of his own property;  if there is an emergency;  if he is insane;  if he is the first offender. 

\scendsutta{The seventh training rule is finished. }

%
\section*{{\suttatitleacronym Bu Sk 38}{\suttatitletranslation 38. The training rule on finding fault }{\suttatitleroot Ujjhānasaññī}}
\addcontentsline{toc}{section}{\tocacronym{Bu Sk 38} \toctranslation{38. The training rule on finding fault } \tocroot{Ujjhānasaññī}}
\markboth{38. The training rule on finding fault }{Ujjhānasaññī}
\extramarks{Bu Sk 38}{Bu Sk 38}

\subsection*{Origin story }

At\marginnote{1.1} one time the Buddha was staying at \textsanskrit{Sāvatthī} in the Jeta Grove, \textsanskrit{Anāthapiṇḍika}’s Monastery. At that time the monks from the group of six were looking at the bowls of others finding fault. … 

\subsection*{Final ruling }

\scrule{“‘I will not look at another’s almsbowl finding fault,’ this is how you should train.” }

One\marginnote{1.4} should not look at the bowl of another finding fault. If a monk, out of disrespect, looks at the bowl of another finding fault, he commits an offense of wrong conduct. 

\subsection*{Non-offenses }

There\marginnote{1.6.1} is no offense: if it is unintentional;  if he is not mindful;  if he does not know;  if he looks with the intention of giving or having someone give;  if he is not finding fault;  if there is an emergency;  if he is insane;  if he is the first offender. 

\scendsutta{The eighth training rule is finished. }

%
\section*{{\suttatitleacronym Bu Sk 39}{\suttatitletranslation 39. The training rule on mouthfuls }{\suttatitleroot Nātimahanta}}
\addcontentsline{toc}{section}{\tocacronym{Bu Sk 39} \toctranslation{39. The training rule on mouthfuls } \tocroot{Nātimahanta}}
\markboth{39. The training rule on mouthfuls }{Nātimahanta}
\extramarks{Bu Sk 39}{Bu Sk 39}

\subsection*{Origin story }

At\marginnote{1.1} one time the Buddha was staying at \textsanskrit{Sāvatthī} in the Jeta Grove, \textsanskrit{Anāthapiṇḍika}’s Monastery. At that time the monks from the group of six were making large mouthfuls. … 

\subsection*{Final ruling }

\scrule{“‘I will not make mouthfuls that are too large,’ this is how you should train.” }

One\marginnote{1.4} should not make mouthfuls that are too large. If a monk, out of disrespect, makes a large mouthful, he commits an offense of wrong conduct. 

\subsection*{Non-offenses }

There\marginnote{1.6.1} is no offense: if it is unintentional;  if he is not mindful;  if he does not know;  if he is sick;  if it is a fresh food;\footnote{According to Sp 2.615 \textit{khajjaka} refers to all fresh foods: \textit{Ettha \textsanskrit{mūlakhādanīyādi} \textsanskrit{sabbaṁ} \textsanskrit{gahetabbaṁ}}, “Here the fresh foods which are roots, etc., may all be taken.” }  if it is any kind of fruit;\footnote{For \textit{\textsanskrit{phalāphale}} see PED, sv. \textit{\textsanskrit{ā}}⁴. }  if it is a non-bean curry;\footnote{According to Vin-vn-\textsanskrit{ṭ} 1907 \textit{\textsanskrit{uttaribhaṅga}} is just another word for \textit{\textsanskrit{byañjana}}: \textit{\textsanskrit{uttaribhaṅgaṁ} \textsanskrit{nāma} \textsanskrit{byañjanaṁ}}. \textit{\textsanskrit{Byañjana}} seems to refer to curries apart from the standard bean-curries, which are called \textit{\textsanskrit{sūpa}}. }  if there is an emergency;  if he is insane;  if he is the first offender. 

\scendsutta{The ninth training rule is finished. }

%
\section*{{\suttatitleacronym Bu Sk 40}{\suttatitletranslation 40. The training rule on mouthfuls }{\suttatitleroot Parimaṇḍala}}
\addcontentsline{toc}{section}{\tocacronym{Bu Sk 40} \toctranslation{40. The training rule on mouthfuls } \tocroot{Parimaṇḍala}}
\markboth{40. The training rule on mouthfuls }{Parimaṇḍala}
\extramarks{Bu Sk 40}{Bu Sk 40}

\subsection*{Origin story }

At\marginnote{1.1} one time the Buddha was staying at \textsanskrit{Sāvatthī} in the Jeta Grove, \textsanskrit{Anāthapiṇḍika}’s Monastery. At that time the monks from the group of six were making elongated mouthfuls. … 

\subsection*{Final ruling }

\scrule{“‘I will make rounded mouthfuls,’ this is how you should train.” }

One\marginnote{1.4} should make rounded mouthfuls. If a monk, out of disrespect, makes an elongated mouthful, he commits an offense of wrong conduct. 

\subsection*{Non-offenses }

There\marginnote{1.6.1} is no offense: if it is unintentional;  if he is not mindful;  if he does not know;  if he is sick;  if it is a fresh food;\footnote{According to Sp 2.615 \textit{khajjaka} refers to all fresh foods: \textit{Ettha \textsanskrit{mūlakhādanīyādi} \textsanskrit{sabbaṁ} \textsanskrit{gahetabbaṁ}}, “Here the fresh foods which are roots, etc., may all be taken.” }  if it is any kind of fruit;\footnote{For \textit{\textsanskrit{phalāphale}} see PED, sv. \textit{\textsanskrit{ā}}⁴. }  if it is a non-bean curry;\footnote{According to Vin-vn-\textsanskrit{ṭ} 1907 \textit{\textsanskrit{uttaribhaṅga}} is just another word for \textit{\textsanskrit{byañjana}}: \textit{\textsanskrit{uttaribhaṅgaṁ} \textsanskrit{nāma} \textsanskrit{byañjanaṁ}}. \textit{\textsanskrit{Byañjana}} seems to refer to curries apart from the standard bean curries, which are called \textit{\textsanskrit{sūpa}}. }  if there is an emergency;  if he is insane;  if he is the first offender. 

\scendsutta{The tenth training rule is finished. }

\scendvagga{The fourth subchapter on respectfully is finished. }

%
\section*{{\suttatitleacronym Bu Sk 41}{\suttatitletranslation 41. The training rule on without bringing }{\suttatitleroot Anāhaṭa}}
\addcontentsline{toc}{section}{\tocacronym{Bu Sk 41} \toctranslation{41. The training rule on without bringing } \tocroot{Anāhaṭa}}
\markboth{41. The training rule on without bringing }{Anāhaṭa}
\extramarks{Bu Sk 41}{Bu Sk 41}

\subsection*{Origin story }

At\marginnote{1.1} one time the Buddha was staying at \textsanskrit{Sāvatthī} in the Jeta Grove, \textsanskrit{Anāthapiṇḍika}’s Monastery. At that time the monks from the group of six opened their mouths without bringing a mouthful to it. … 

\subsection*{Final ruling }

\scrule{“‘I will not open my mouth without bringing a mouthful to it,’ this is how you should train.” }

One\marginnote{1.4} should not open one’s mouth without bringing a mouthful to it. If a monk, out of disrespect, opens his mouth without bringing a mouthful to it, he commits an offense of wrong conduct. 

\subsection*{Non-offenses }

There\marginnote{1.6.1} is no offense: if it is unintentional;  if he is not mindful;  if he does not know;  if he is sick;  if there is an emergency;  if he is insane;  if he is the first offender. 

\scendsutta{The first training rule is finished. }

%
\section*{{\suttatitleacronym Bu Sk 42}{\suttatitletranslation 42. The second training rule on without bringing }{\suttatitleroot Sabbahattha}}
\addcontentsline{toc}{section}{\tocacronym{Bu Sk 42} \toctranslation{42. The second training rule on without bringing } \tocroot{Sabbahattha}}
\markboth{42. The second training rule on without bringing }{Sabbahattha}
\extramarks{Bu Sk 42}{Bu Sk 42}

\subsection*{Origin story }

At\marginnote{1.1} one time the Buddha was staying at \textsanskrit{Sāvatthī} in the Jeta Grove, \textsanskrit{Anāthapiṇḍika}’s Monastery. At that time the monks from the group of six put their whole hand in their mouths while eating. … 

\subsection*{Final ruling }

\scrule{“‘I will not put my whole hand in my mouth while eating,’ this is how you should train.”\footnote{According to the sub-commentary, Sp-yoj 2.618, the whole hand refers to the five fingers: \textit{\textsanskrit{pañcaṅguliṁ} \textsanskrit{sandhāya} \textsanskrit{vuttaṁ}}, “It is said with regard to the five fingers.” The same point is made at Kkh-\textsanskrit{pṭ}: \textit{\textsanskrit{Sakalaṁ} hatthanti \textsanskrit{sakalā} \textsanskrit{aṅguliyo}}, “The whole hand means all the fingers.” } }

One\marginnote{1.4} should not put one’s whole hand in one’s mouth while eating. If a monk, out of disrespect, puts his whole hand in his mouth while eating, he commits an offense of wrong conduct. 

\subsection*{Non-offenses }

There\marginnote{1.6.1} is no offense: if it is unintentional;  if he is not mindful;  if he does not know;  if he is sick;  if there is an emergency;  if he is insane;  if he is the first offender. 

\scendsutta{The second training rule is finished. }

%
\section*{{\suttatitleacronym Bu Sk 43}{\suttatitletranslation 43. The training rule on with a mouthful }{\suttatitleroot Sakabaḷa}}
\addcontentsline{toc}{section}{\tocacronym{Bu Sk 43} \toctranslation{43. The training rule on with a mouthful } \tocroot{Sakabaḷa}}
\markboth{43. The training rule on with a mouthful }{Sakabaḷa}
\extramarks{Bu Sk 43}{Bu Sk 43}

\subsection*{Origin story }

At\marginnote{1.1} one time the Buddha was staying at \textsanskrit{Sāvatthī} in the Jeta Grove, \textsanskrit{Anāthapiṇḍika}’s Monastery. At that time the monks from the group of six spoke with food in their mouths. … 

\subsection*{Final ruling }

\scrule{“‘I will not speak with food in my mouth,’ this is how you should train.” }

One\marginnote{1.4} should not speak with food in one’s mouth. If a monk, out of disrespect, speaks with food in his mouth, he commits an offense of wrong conduct. 

\subsection*{Non-offenses }

There\marginnote{1.6.1} is no offense: if it is unintentional;  if he is not mindful;  if he does not know;  if he is sick;  if there is an emergency;  if he is insane;  if he is the first offender. 

\scendsutta{The third training rule is finished. }

%
\section*{{\suttatitleacronym Bu Sk 44}{\suttatitletranslation 44. The training rule on lifted balls of food }{\suttatitleroot Piṇḍukkhepaka}}
\addcontentsline{toc}{section}{\tocacronym{Bu Sk 44} \toctranslation{44. The training rule on lifted balls of food } \tocroot{Piṇḍukkhepaka}}
\markboth{44. The training rule on lifted balls of food }{Piṇḍukkhepaka}
\extramarks{Bu Sk 44}{Bu Sk 44}

\subsection*{Origin story }

At\marginnote{1.1} one time the Buddha was staying at \textsanskrit{Sāvatthī} in the Jeta Grove, \textsanskrit{Anāthapiṇḍika}’s Monastery. At that time the monks from the group of six ate from lifted balls of food. … 

\subsection*{Final ruling }

\scrule{“‘I will not eat from a lifted ball of food,’ this is how you should train.”\footnote{According to Sp 2.620 this means repeatedly lifting a mouthful. Since the following rule refers to breaking pieces off a ball of food, this rule may refer to biting repeatedly into the same ball of food. } }

One\marginnote{1.4} should not eat from a lifted ball of food. If a monk, out of disrespect, eats from a lifted ball of food, he commits an offense of wrong conduct. 

\subsection*{Non-offenses }

There\marginnote{1.6.1} is no offense: if it is unintentional;  if he is not mindful;  if he does not know;  if he is sick;  if it is a fresh food;\footnote{According to Sp 2.615 \textit{khajjaka} refers to all fresh foods: \textit{Ettha \textsanskrit{mūlakhādanīyādi} \textsanskrit{sabbaṁ} \textsanskrit{gahetabbaṁ}}, “Here the fresh foods which are roots, etc., may all be taken.” }  if it is any kind of fruit;\footnote{For \textit{\textsanskrit{phalāphale}} see PED, sv. \textit{\textsanskrit{ā}}⁴. }  if there is an emergency;  if he is insane;  if he is the first offender. 

\scendsutta{The fourth training rule is finished. }

%
\section*{{\suttatitleacronym Bu Sk 45}{\suttatitletranslation 45. The training rule on breaking up mouthfuls }{\suttatitleroot Kabaḷāvacchedaka}}
\addcontentsline{toc}{section}{\tocacronym{Bu Sk 45} \toctranslation{45. The training rule on breaking up mouthfuls } \tocroot{Kabaḷāvacchedaka}}
\markboth{45. The training rule on breaking up mouthfuls }{Kabaḷāvacchedaka}
\extramarks{Bu Sk 45}{Bu Sk 45}

\subsection*{Origin story }

At\marginnote{1.1} one time the Buddha was staying at \textsanskrit{Sāvatthī} in the Jeta Grove, \textsanskrit{Anāthapiṇḍika}’s Monastery. At that time the monks from the group of six ate breaking up mouthfuls. … 

\subsection*{Final ruling }

\scrule{“‘I will not eat breaking up mouthfuls,’ this is how you should train.”\footnote{According to Sp 2.621 this means repeatedly breaking pieces off a ball of food. The word \textit{avacchedaka} is further explained at Vin-vn-\textsanskrit{ṭ} 1920 as \textit{\textsanskrit{chinditvā}}, “having broken off”. This, it seems, can only refer to breaking off pieces, not to biting. Each ball of food was a single mouthful, consisting of rice with suitable curry. } }

One\marginnote{1.4} should not eat breaking up mouthfuls. If a monk, out of disrespect, eats breaking up mouthfuls, he commits an offense of wrong conduct. 

\subsection*{Non-offenses }

There\marginnote{1.6.1} is no offense: if it is unintentional;  if he is not mindful;  if he does not know;  if he is sick;  if it is a fresh food;\footnote{According to Sp 2.615 \textit{khajjaka} refers to all fresh foods: \textit{Ettha \textsanskrit{mūlakhādanīyādi} \textsanskrit{sabbaṁ} \textsanskrit{gahetabbaṁ}}, “Here the fresh foods which are roots, etc., may all be taken.” }  if it is any kind of fruit;\footnote{For \textit{\textsanskrit{phalāphale}} see PED, sv. \textit{\textsanskrit{ā}}⁴. }  if it is a non-bean curry;\footnote{According to Vin-vn-\textsanskrit{ṭ} 1907 \textit{\textsanskrit{uttaribhaṅga}} is just another word for \textit{\textsanskrit{byañjana}}: \textit{\textsanskrit{uttaribhaṅgaṁ} \textsanskrit{nāma} \textsanskrit{byañjanaṁ}}. \textit{\textsanskrit{Byañjana}} seems to refer to curries apart from the standard bean-curries, which are called \textit{\textsanskrit{sūpa}}. }  if there is an emergency;  if he is insane;  if he is the first offender. 

\scendsutta{The fifth training rule is finished. }

%
\section*{{\suttatitleacronym Bu Sk 46}{\suttatitletranslation 46. The training rule on stuffing the cheeks }{\suttatitleroot Avagaṇḍakāraka}}
\addcontentsline{toc}{section}{\tocacronym{Bu Sk 46} \toctranslation{46. The training rule on stuffing the cheeks } \tocroot{Avagaṇḍakāraka}}
\markboth{46. The training rule on stuffing the cheeks }{Avagaṇḍakāraka}
\extramarks{Bu Sk 46}{Bu Sk 46}

\subsection*{Origin story }

At\marginnote{1.1} one time the Buddha was staying at \textsanskrit{Sāvatthī} in the Jeta Grove, \textsanskrit{Anāthapiṇḍika}’s Monastery. At that time the monks from the group of six ate stuffing their cheeks. … 

\subsection*{Final ruling }

\scrule{“‘I will not eat stuffing my cheeks,’ this is how you should train.” }

One\marginnote{1.4} should not eat stuffing one’s cheeks. If a monk, out of disrespect, eats stuffing one or both cheeks, he commits an offense of wrong conduct. 

\subsection*{Non-offenses }

There\marginnote{1.6.1} is no offense: if it is unintentional;  if he is not mindful;  if he does not know;  if he is sick;  if it is any kind of fruit;\footnote{For \textit{\textsanskrit{phalāphale}} see PED, sv. \textit{\textsanskrit{ā}}⁴. }  if there is an emergency;  if he is insane;  if he is the first offender. 

\scendsutta{The sixth training rule is finished. }

%
\section*{{\suttatitleacronym Bu Sk 47}{\suttatitletranslation 47. The training rule on shaking the hand }{\suttatitleroot Hatthaniddhunaka}}
\addcontentsline{toc}{section}{\tocacronym{Bu Sk 47} \toctranslation{47. The training rule on shaking the hand } \tocroot{Hatthaniddhunaka}}
\markboth{47. The training rule on shaking the hand }{Hatthaniddhunaka}
\extramarks{Bu Sk 47}{Bu Sk 47}

\subsection*{Origin story }

At\marginnote{1.1} one time the Buddha was staying at \textsanskrit{Sāvatthī} in the Jeta Grove, \textsanskrit{Anāthapiṇḍika}’s Monastery. At that time the monks from the group of six ate shaking their hands. … 

\subsection*{Final ruling }

\scrule{“‘I will not eat shaking my hand,’ this is how you should train.” }

One\marginnote{1.4} should not eat shaking one’s hand. If a monk, out of disrespect, eats shaking his hand, he commits an offense of wrong conduct. 

\subsection*{Non-offenses }

There\marginnote{1.6.1} is no offense: if it is unintentional;  if he is not mindful;  if he does not know;  if he is sick;  if he shakes the hand to discard trash;  if there is an emergency;  if he is insane;  if he is the first offender. 

\scendsutta{The seventh training rule is finished. }

%
\section*{{\suttatitleacronym Bu Sk 48}{\suttatitletranslation 48. The training rule on scattering rice }{\suttatitleroot Sitthāvakāraka}}
\addcontentsline{toc}{section}{\tocacronym{Bu Sk 48} \toctranslation{48. The training rule on scattering rice } \tocroot{Sitthāvakāraka}}
\markboth{48. The training rule on scattering rice }{Sitthāvakāraka}
\extramarks{Bu Sk 48}{Bu Sk 48}

\subsection*{Origin story }

At\marginnote{1.1} one time the Buddha was staying at \textsanskrit{Sāvatthī} in the Jeta Grove, \textsanskrit{Anāthapiṇḍika}’s Monastery. At that time the monks from the group of six scattered rice while eating. … 

\subsection*{Final ruling }

\scrule{“‘I will not eat scattering rice,’ this is how you should train.” }

One\marginnote{1.4} should not scatter rice while eating. If a monk, out of disrespect, eats scattering rice, he commits an offense of wrong conduct. 

\subsection*{Non-offenses }

There\marginnote{1.6.1} is no offense: if it is unintentional;  if he is not mindful;  if he does not know;  if he is sick;  if he discards rice while discarding trash;  if there is an emergency;  if he is insane;  if he is the first offender. 

\scendsutta{The eighth training rule is finished. }

%
\section*{{\suttatitleacronym Bu Sk 49}{\suttatitletranslation 49. The training rule on sticking out the tongue }{\suttatitleroot Jivhānicchāraka}}
\addcontentsline{toc}{section}{\tocacronym{Bu Sk 49} \toctranslation{49. The training rule on sticking out the tongue } \tocroot{Jivhānicchāraka}}
\markboth{49. The training rule on sticking out the tongue }{Jivhānicchāraka}
\extramarks{Bu Sk 49}{Bu Sk 49}

\subsection*{Origin story }

At\marginnote{1.1} one time the Buddha was staying at \textsanskrit{Sāvatthī} in the Jeta Grove, \textsanskrit{Anāthapiṇḍika}’s Monastery. At that time the monks from the group of six ate sticking out their tongues. … 

\subsection*{Final ruling }

\scrule{“‘I will not eat sticking out my tongue,’ this is how you should train.” }

One\marginnote{1.4} should not eat sticking out one’s tongue. If a monk, out of disrespect, eats sticking out his tongue, he commits an offense of wrong conduct. 

\subsection*{Non-offenses }

There\marginnote{1.6.1} is no offense: if it is unintentional;  if he is not mindful;  if he does not know;  if he is sick;  if there is an emergency;  if he is insane;  if he is the first offender. 

\scendsutta{The ninth training rule is finished. }

%
\section*{{\suttatitleacronym Bu Sk 50}{\suttatitletranslation 50. The training rule on chomping }{\suttatitleroot Capucapukāraka}}
\addcontentsline{toc}{section}{\tocacronym{Bu Sk 50} \toctranslation{50. The training rule on chomping } \tocroot{Capucapukāraka}}
\markboth{50. The training rule on chomping }{Capucapukāraka}
\extramarks{Bu Sk 50}{Bu Sk 50}

\subsection*{Origin story }

At\marginnote{1.1} one time the Buddha was staying at \textsanskrit{Sāvatthī} in the Jeta Grove, \textsanskrit{Anāthapiṇḍika}’s Monastery. At that time the monks from the group of six made chomping sounds while eating. … 

\subsection*{Final ruling }

\scrule{“‘I will not make a chomping sound while eating,’ this is how you should train.” }

One\marginnote{1.4} should not make a chomping sound while eating. If a monk, out of disrespect, makes a chomping sound while eating, he commits an offense of wrong conduct. 

\subsection*{Non-offenses }

There\marginnote{1.6.1} is no offense: if it is unintentional;  if he is not mindful;  if he does not know;  if he is sick;  if there is an emergency;  if he is insane;  if he is the first offender. 

\scendsutta{The tenth training rule is finished. }

\scendvagga{The fifth subchapter on mouthfuls is finished. }

%
\section*{{\suttatitleacronym Bu Sk 51}{\suttatitletranslation 51. The training rule on slurping }{\suttatitleroot Surusurukāraka}}
\addcontentsline{toc}{section}{\tocacronym{Bu Sk 51} \toctranslation{51. The training rule on slurping } \tocroot{Surusurukāraka}}
\markboth{51. The training rule on slurping }{Surusurukāraka}
\extramarks{Bu Sk 51}{Bu Sk 51}

\subsection*{Origin story }

At\marginnote{1.1} one time when the Buddha was staying at \textsanskrit{Kosambī} in Ghosita’s Monastery, a brahmin prepared a milk drink for the Sangha. The monks slurped while drinking the milk. A monk who was previously an entertainer made a joke of it, saying, “It’s as if the whole Sangha is cooled.”\footnote{This is a play on words in the Pali. The Pali word here translated as “cooled”, \textit{\textsanskrit{sītīkata}} (literally, “made cool”), is similar to \textit{\textsanskrit{sītibhūta}}, “become cool”, an epithet for \textit{arahants}, “perfected ones”. } 

The\marginnote{1.6} monks of few desires complained and criticized him, “How could a monk joke about the Sangha?” … “Is it true, monk, that you did this?” 

“It’s\marginnote{1.9} true, sir.” 

The\marginnote{1.10} Buddha rebuked him … “Foolish man, how could you do this? This will affect people’s confidence …” After rebuking him … he gave a teaching and addressed the monks: 

\scrule{“Monks, you should not joke about the Buddha, the Teaching, or the Sangha. If you do, you commit an offense of wrong conduct.” }

After\marginnote{1.17} rebuking that monk in many ways, the Buddha spoke in dispraise of being difficult to support … “And, monks, this training rule should be recited like this: 

\subsection*{Final ruling }

\scrule{‘“I will not slurp while eating,” this is how you should train.’” }

One\marginnote{1.20} should not slurp while eating. If a monk, out of disrespect, slurps while eating, he commits an offense of wrong conduct. 

\subsection*{Non-offenses }

There\marginnote{1.22.1} is no offense: if it is unintentional;  if he is not mindful;  if he does not know;  if he is sick;  if there is an emergency;  if he is insane;  if he is the first offender. 

\scendsutta{The first training rule is finished. }

%
\section*{{\suttatitleacronym Bu Sk 52}{\suttatitletranslation 52. The training rule on licking the hands }{\suttatitleroot Hatthanillehaka}}
\addcontentsline{toc}{section}{\tocacronym{Bu Sk 52} \toctranslation{52. The training rule on licking the hands } \tocroot{Hatthanillehaka}}
\markboth{52. The training rule on licking the hands }{Hatthanillehaka}
\extramarks{Bu Sk 52}{Bu Sk 52}

\subsection*{Origin story }

At\marginnote{1.1} one time the Buddha was staying at \textsanskrit{Sāvatthī} in the Jeta Grove, \textsanskrit{Anāthapiṇḍika}’s Monastery. At that time the monks from the group of six were licking their hands while eating. … 

\subsection*{Final ruling }

\scrule{“‘I will not lick my hands while eating,’ this is how you should train.” }

One\marginnote{1.4} should not lick one’s hands while eating. If a monk, out of disrespect, licks his hands while eating, he commits an offense of wrong conduct. 

\subsection*{Non-offenses }

There\marginnote{1.6.1} is no offense: if it is unintentional;  if he is not mindful;  if he does not know;  if he is sick;  if there is an emergency;  if he is insane;  if he is the first offender. 

\scendsutta{The second training rule is finished. }

%
\section*{{\suttatitleacronym Bu Sk 53}{\suttatitletranslation 53. The training rule on licking the almsbowl }{\suttatitleroot Pattanillehaka}}
\addcontentsline{toc}{section}{\tocacronym{Bu Sk 53} \toctranslation{53. The training rule on licking the almsbowl } \tocroot{Pattanillehaka}}
\markboth{53. The training rule on licking the almsbowl }{Pattanillehaka}
\extramarks{Bu Sk 53}{Bu Sk 53}

\subsection*{Origin story }

At\marginnote{1.1} one time the Buddha was staying at \textsanskrit{Sāvatthī} in the Jeta Grove, \textsanskrit{Anāthapiṇḍika}’s Monastery. At that time the monks from the group of six were licking their bowls while eating. … 

\subsection*{Final ruling }

\scrule{“‘I will not lick my almsbowl while eating,’ this is how you should train.”\footnote{According to Sp 2.628 “licking” includes using one’s finger: \textit{\textsanskrit{Ekaṅguliyāpi} patto na nillehitabbo}, “The bowl should not be ‘licked’ even with one finger.” It thus seems that the Pali word \textit{nillehaka} is broader than “licking” in English. } }

One\marginnote{1.4} should not lick one’s bowl while eating. If a monk, out of disrespect, licks his bowl while eating, he commits an offense of wrong conduct. 

\subsection*{Non-offenses }

There\marginnote{1.6.1} is no offense: if it is unintentional;  if he is not mindful;  if he does not know;  if he is sick;  if only a little food remains and he eats it after collecting it in one place and then licking it;  if there is an emergency;  if he is insane;  if he is the first offender. 

\scendsutta{The third training rule is finished. }

%
\section*{{\suttatitleacronym Bu Sk 54}{\suttatitletranslation 54. The training rule on licking the lips }{\suttatitleroot Oṭṭhanillehaka}}
\addcontentsline{toc}{section}{\tocacronym{Bu Sk 54} \toctranslation{54. The training rule on licking the lips } \tocroot{Oṭṭhanillehaka}}
\markboth{54. The training rule on licking the lips }{Oṭṭhanillehaka}
\extramarks{Bu Sk 54}{Bu Sk 54}

\subsection*{Origin story }

At\marginnote{1.1} one time the Buddha was staying at \textsanskrit{Sāvatthī} in the Jeta Grove, \textsanskrit{Anāthapiṇḍika}’s Monastery. At that time the monks from the group of six were licking their lips while eating. … 

\subsection*{Final ruling }

\scrule{“‘I will not lick my lips while eating,’ this is how you should train.” }

One\marginnote{1.4} should not lick one’s lips while eating. If a monk, out of disrespect, licks his lips while eating, he commits an offense of wrong conduct. 

\subsection*{Non-offenses }

There\marginnote{1.6.1} is no offense: if it is unintentional;  if he is not mindful;  if he does not know;  if he is sick;  if there is an emergency;  if he is insane;  if he is the first offender. 

\scendsutta{The fourth training rule is finished. }

%
\section*{{\suttatitleacronym Bu Sk 55}{\suttatitletranslation 55. The training rule on with food }{\suttatitleroot Sāmisa}}
\addcontentsline{toc}{section}{\tocacronym{Bu Sk 55} \toctranslation{55. The training rule on with food } \tocroot{Sāmisa}}
\markboth{55. The training rule on with food }{Sāmisa}
\extramarks{Bu Sk 55}{Bu Sk 55}

\subsection*{Origin story }

At\marginnote{1.1} one time when the Buddha was staying in the Bhagga country at \textsanskrit{Susumāragira} in the \textsanskrit{Bhesakaḷā} Grove, the monks in the Kokanada stilt house were receiving the drinking-water vessel with hands soiled with food. People complained and criticized them, “How can the Sakyan monastics receive drinking-water vessels with hands soiled with food? They’re just like householders who indulge in worldly pleasures!” 

The\marginnote{1.5} monks heard the complaints of those people, and the monks of few desires complained and criticized those monks, “How can those monks do this?” … “Is it true, monks, that they do this?” 

“It’s\marginnote{1.9} true, sir.” 

The\marginnote{1.10} Buddha rebuked them … “How can those foolish men do this? This will affect people’s confidence …” … “And, monks, this training rule should be recited like this: 

\subsection*{Final ruling }

\scrule{‘“I will not receive the drinking-water vessel with a hand soiled with food,” this is how you should train.’” }

One\marginnote{1.15} should not receive the drinking-water vessel with a hand soiled with food. If a monk, out of disrespect, receives the drinking-water vessel with a hand soiled with food, he commits an offense of wrong conduct. 

\subsection*{Non-offenses }

There\marginnote{1.17.1} is no offense: if it is unintentional;  if he is not mindful;  if he does not know;  if he is sick;  if he receives it with the intention of washing it or having it washed;  if there is an emergency;  if he is insane;  if he is the first offender. 

\scendsutta{The fifth training rule is finished. }

%
\section*{{\suttatitleacronym Bu Sk 56}{\suttatitletranslation 56. The training rule on containing rice }{\suttatitleroot Sasitthaka}}
\addcontentsline{toc}{section}{\tocacronym{Bu Sk 56} \toctranslation{56. The training rule on containing rice } \tocroot{Sasitthaka}}
\markboth{56. The training rule on containing rice }{Sasitthaka}
\extramarks{Bu Sk 56}{Bu Sk 56}

\subsection*{Origin story }

At\marginnote{1.1} one time when the Buddha was staying in the Bhagga country at \textsanskrit{Susumāragira} in the \textsanskrit{Bhesakaḷā} Grove, the monks in the Kokanada stilt house were discarding their bowl-washing water containing rice in inhabited areas. People complained and criticized them, “How can the Sakyan monastics discard their bowl-washing water containing rice in inhabited areas? They’re just like householders who indulge in worldly pleasures!” 

The\marginnote{1.5} monks heard the complaints of those people, and the monks of few desires complained and criticized those monks, “How can those monks do this?” … “Is it true, monks, that they do this?” 

“It’s\marginnote{1.9} true, sir.” 

The\marginnote{1.10} Buddha rebuked them … “How can those foolish men do this? This will affect people’s confidence …” … “And, monks, this training rule should be recited like this: 

\subsection*{Final ruling }

\scrule{‘“I will not discard bowl-washing water containing rice in inhabited areas,” this is how you should train.’” }

One\marginnote{1.15} should not discard bowl-washing water containing rice in an inhabited area. If a monk, out of disrespect, discards bowl-washing water containing rice in an inhabited area, he commits an offense of wrong conduct. 

\subsection*{Non-offenses }

There\marginnote{1.17.1} is no offense: if it is unintentional;  if he is not mindful;  if he does not know;  if he is sick;  if he discards it after removing the rice, after breaking it up, into a container, or after taking it outside;  if there is an emergency;  if he is insane;  if he is the first offender. 

\scendsutta{The sixth training rule is finished. }

%
\section*{{\suttatitleacronym Bu Sk 57}{\suttatitletranslation 57. The training rule on holding a sunshade }{\suttatitleroot Chattapāṇi}}
\addcontentsline{toc}{section}{\tocacronym{Bu Sk 57} \toctranslation{57. The training rule on holding a sunshade } \tocroot{Chattapāṇi}}
\markboth{57. The training rule on holding a sunshade }{Chattapāṇi}
\extramarks{Bu Sk 57}{Bu Sk 57}

\subsection*{Origin story }

\subsubsection*{First sub-story }

At\marginnote{1.1} one time when the Buddha was staying in the Bhagga country at \textsanskrit{Susumāragira} in the \textsanskrit{Bhesakaḷā} Grove, the monks from the group of six gave teachings to people holding a sunshade. 

The\marginnote{1.3} monks of few desires complained and criticized those monks, “How can the monks from the group of six give teachings to people holding a sunshade?” … “Is it true, monks, that you do this?” 

“It’s\marginnote{1.6} true, sir.” 

The\marginnote{1.7} Buddha rebuked them … “Foolish men, how can you do this? This will affect people’s confidence …” … “And, monks, this training rule should be recited like this: 

\subsubsection*{Preliminary ruling }

\scrule{‘“I will not give a teaching to anyone holding a sunshade,” this is how you should train.’” }

In\marginnote{1.12} this way the Buddha laid down this training rule for the monks. 

\subsubsection*{Second sub-story }

Soon\marginnote{2.1} afterwards, being afraid of wrongdoing, the monks did not give teachings to sick people holding a sunshade. People complained and criticized them, “How can the Sakyan monastics not give teachings to someone who’s sick holding a sunshade?” 

The\marginnote{2.4} monks heard the complaints of those people, and they told the Buddha. Soon afterwards he gave a teaching and addressed the monks: 

\scrule{“Monks, I allow you to give a teaching to someone who’s sick holding a sunshade. }

And\marginnote{2.8} so, monks, this training rule should be recited like this: 

\subsection*{Final ruling }

\scrule{‘“I will not give a teaching to anyone holding a sunshade who is not sick,” this is how you should train.’” }

\subsection*{Definitions }

\begin{description}%
\item[A sunshade: ] there are three kinds of sunshades: the white sunshade, the reed sunshade, the leaf sunshade. They are bound at the rim and bound at the ribs.\footnote{Sp 2.634: \textit{\textsanskrit{Maṇḍalabaddhaṁ} \textsanskrit{salākabaddhanti} \textsanskrit{idaṁ} pana \textsanskrit{tiṇṇampi} \textsanskrit{chattānaṁ} \textsanskrit{pañjaradassanatthaṁ} \textsanskrit{vuttaṁ}. \textsanskrit{Tāni} hi \textsanskrit{maṇḍalabaddhāni} ceva honti \textsanskrit{salākabaddhāni} ca.} “\textit{\textsanskrit{Maṇḍalabaddhaṁ} \textsanskrit{salākabaddhan}}: this is said for the purpose of showing the frame of the three sunshades. For they are bound at the rim (\textit{\textsanskrit{maṇḍalabaddha}}) and bound at the ribs (\textit{\textsanskrit{salākabaddha}}).” } %
\item[Teaching: ] what has been spoken by the Buddha, what has been spoken by disciples, what has been spoken by sages, what has been spoken by gods, what is connected with what is beneficial, what is connected with the Teaching. %
\item[Give: ] if he teaches by the line, then for every line he commits an offense of wrong conduct. If he teaches by the syllable, then for every syllable he commits an offense of wrong conduct. %
\end{description}

One\marginnote{2.17} should not give a teaching to anyone holding a sunshade who is not sick. If a monk, out of disrespect, gives a teaching to someone holding a sunshade who is not sick, he commits an offense of wrong conduct. 

\subsection*{Non-offenses }

There\marginnote{2.19.1} is no offense: if it is unintentional;  if he is not mindful;  if he does not know;  if he is sick;  if there is an emergency;  if he is insane;  if he is the first offender. 

\scendsutta{The seventh training rule is finished. }

%
\section*{{\suttatitleacronym Bu Sk 58}{\suttatitletranslation 58. The training rule on holding a staff }{\suttatitleroot Daṇḍapāṇi}}
\addcontentsline{toc}{section}{\tocacronym{Bu Sk 58} \toctranslation{58. The training rule on holding a staff } \tocroot{Daṇḍapāṇi}}
\markboth{58. The training rule on holding a staff }{Daṇḍapāṇi}
\extramarks{Bu Sk 58}{Bu Sk 58}

\subsection*{Origin story }

At\marginnote{1.1} one time the Buddha was staying at \textsanskrit{Sāvatthī} in the Jeta Grove, \textsanskrit{Anāthapiṇḍika}’s Monastery. At that time the monks from the group of six gave teachings to people holding staffs. … 

\subsection*{Final ruling }

\scrule{“‘I will not give a teaching to anyone holding a staff who is not sick,’ this is how you should train.” }

\subsection*{Definitions }

\begin{description}%
\item[A staff: ] a stick measuring 1.6 meters.\footnote{\textit{Majjhimassa purisassa catuhattho \textsanskrit{daṇḍo}}, “It is a stick of four forearms of a man of average height.” For a discussion of the \textit{hattha}, see \textit{sugata} in Appendix of Technical Terms. } What is longer than that is not a staff, nor what is shorter. %
\end{description}

One\marginnote{1.7} should not give a teaching to anyone holding a staff who is not sick. If a monk out, of disrespect, gives a teaching to someone holding a staff who is not sick, he commits an offense of wrong conduct. 

\subsection*{Non-offenses }

There\marginnote{1.9.1} is no offense: if it is unintentional;  if he is not mindful;  if he does not know;  if he is sick;  if there is an emergency;  if he is insane;  if he is the first offender. 

\scendsutta{The eighth training rule is finished. }

%
\section*{{\suttatitleacronym Bu Sk 59}{\suttatitletranslation 59. The training rule on holding a knife }{\suttatitleroot Satthapāṇi}}
\addcontentsline{toc}{section}{\tocacronym{Bu Sk 59} \toctranslation{59. The training rule on holding a knife } \tocroot{Satthapāṇi}}
\markboth{59. The training rule on holding a knife }{Satthapāṇi}
\extramarks{Bu Sk 59}{Bu Sk 59}

\subsection*{Origin story }

At\marginnote{1.1} one time the Buddha was staying at \textsanskrit{Sāvatthī} in the Jeta Grove, \textsanskrit{Anāthapiṇḍika}’s Monastery. At that time the monks from the group of six gave teachings to people holding knives. … 

\subsection*{Final ruling }

\scrule{“‘I will not give a teaching to anyone holding a knife who is not sick,’ this is how you should train.” }

\subsection*{Definitions }

\begin{description}%
\item[A knife: ] a weapon with a single-edged or double-edged blade.\footnote{For \textit{\textsanskrit{paharaṇaṁ}} see SED, sv. \textit{\textsanskrit{praharaṇa}}. } %
\end{description}

One\marginnote{1.6} should not give a teaching to anyone holding a knife who is not sick. If a monk, out of disrespect, gives a teaching to someone holding a knife who is not sick, he commits an offense of wrong conduct. 

\subsection*{Non-offenses }

There\marginnote{1.8.1} is no offense: if it is unintentional;  if he is not mindful;  if he does not know;  if he is sick;  if there is an emergency;  if he is insane;  if he is the first offender. 

\scendsutta{The ninth training rule is finished. }

%
\section*{{\suttatitleacronym Bu Sk 60}{\suttatitletranslation 60. The training rule on holding a weapon }{\suttatitleroot Āvudhapāṇi}}
\addcontentsline{toc}{section}{\tocacronym{Bu Sk 60} \toctranslation{60. The training rule on holding a weapon } \tocroot{Āvudhapāṇi}}
\markboth{60. The training rule on holding a weapon }{Āvudhapāṇi}
\extramarks{Bu Sk 60}{Bu Sk 60}

\subsection*{Origin story }

At\marginnote{1.1} one time the Buddha was staying at \textsanskrit{Sāvatthī} in the Jeta Grove, \textsanskrit{Anāthapiṇḍika}’s Monastery. At that time the monks from the group of six gave teachings to people holding weapons. … 

\subsection*{Final ruling }

\scrule{“‘I will not give a teaching to anyone holding a weapon who is not sick,’ this is how you should train.” }

\subsection*{Definitions }

\begin{description}%
\item[A weapon: ] any kind of bow. %
\end{description}

One\marginnote{1.6} should not give a teaching to anyone holding a weapon who is not sick. If a monk, out of disrespect, gives a teaching to someone holding a weapon who is not sick, he commits an offense of wrong conduct. 

\subsection*{Non-offenses }

There\marginnote{1.8.1} is no offense: if it is unintentional;  if he is not mindful;  if he does not know;  if he is sick;  if there is an emergency;  if he is insane;  if he is the first offender. 

\scendsutta{The tenth training rule is finished. }

\scendvagga{The sixth subchapter on slurping is finished. }

%
\section*{{\suttatitleacronym Bu Sk 61}{\suttatitletranslation 61. The training rule on shoes }{\suttatitleroot Pāduka}}
\addcontentsline{toc}{section}{\tocacronym{Bu Sk 61} \toctranslation{61. The training rule on shoes } \tocroot{Pāduka}}
\markboth{61. The training rule on shoes }{Pāduka}
\extramarks{Bu Sk 61}{Bu Sk 61}

\subsection*{Origin story }

At\marginnote{1.1} one time the Buddha was staying at \textsanskrit{Sāvatthī} in the Jeta Grove, \textsanskrit{Anāthapiṇḍika}’s Monastery. At that time the monks from the group of six gave teachings to people wearing shoes. … 

\subsection*{Final ruling }

\scrule{“‘I will not give a teaching to anyone wearing shoes who is not sick,’ this is how you should train.” }

One\marginnote{1.4} should not give a teaching to anyone wearing shoes who is not sick. If a monk, out of disrespect, gives a teaching to someone who is not sick and who is standing on his shoes, whose shoes are fastened, or whose shoes are loose, he commits an offense of wrong conduct.\footnote{Sp 2.638: \textit{\textsanskrit{Akkantassāti} \textsanskrit{chattadaṇḍake} \textsanskrit{aṅgulantaraṁ} \textsanskrit{appavesetvā} \textsanskrit{kevalaṁ} \textsanskrit{pādukaṁ} \textsanskrit{akkamitvā} \textsanskrit{ṭhitassa}. \textsanskrit{Paṭimukkassāti} \textsanskrit{paṭimuñcitvā} \textsanskrit{ṭhitassa}. \textsanskrit{Upāhanāyapi} eseva nayo. Omukkoti panettha \textsanskrit{paṇhikabaddhaṁ} \textsanskrit{omuñcitvā} \textsanskrit{ṭhito} vuccati}, “\textit{Akkantassa}: not having entered the toes into the \textit{\textsanskrit{chattadaṇḍaka}}, he stands on top of the entire shoe. \textit{\textsanskrit{Paṭimukkassa}}: he stands after fastening. But here, standing after loosening the heel strap, it is called \textit{omukka}.” It is not clear what a \textit{\textsanskrit{chattadaṇḍaka}}, literally, “a sunshade rib”, means in this context. Perhaps it refers to the thong that goes between the toes. } 

\subsection*{Non-offenses }

There\marginnote{1.6.1} is no offense: if it is unintentional;  if he is not mindful;  if he does not know;  if he is sick;  if there is an emergency;  if he is insane;  if he is the first offender. 

\scendsutta{The first training rule is finished. }

%
\section*{{\suttatitleacronym Bu Sk 62}{\suttatitletranslation 62. The training rule on sandals }{\suttatitleroot Upāhanāruḷha}}
\addcontentsline{toc}{section}{\tocacronym{Bu Sk 62} \toctranslation{62. The training rule on sandals } \tocroot{Upāhanāruḷha}}
\markboth{62. The training rule on sandals }{Upāhanāruḷha}
\extramarks{Bu Sk 62}{Bu Sk 62}

\subsection*{Origin story }

At\marginnote{1.1} one time the Buddha was staying at \textsanskrit{Sāvatthī} in the Jeta Grove, \textsanskrit{Anāthapiṇḍika}’s Monastery. At that time the monks from the group of six gave teachings to people wearing sandals. … 

\subsection*{Final ruling }

\scrule{“‘I will not give a teaching to anyone wearing sandals who is not sick,’ this is how you should train.” }

One\marginnote{1.4} should not give a teaching to anyone wearing sandals who is not sick. If a monk, out of disrespect, gives a teaching to someone who is not sick and who is standing on sandals, whose sandals are fastened, or whose sandals are loose, he commits an offense of wrong conduct. 

\subsection*{Non-offenses }

There\marginnote{1.6.1} is no offense: if it is unintentional;  if he is not mindful;  if he does not know;  if he is sick;  if there is an emergency;  if he is insane;  if he is the first offender. 

\scendsutta{The second training rule is finished. }

%
\section*{{\suttatitleacronym Bu Sk 63}{\suttatitletranslation 63. The training rule on vehicles }{\suttatitleroot Yāna}}
\addcontentsline{toc}{section}{\tocacronym{Bu Sk 63} \toctranslation{63. The training rule on vehicles } \tocroot{Yāna}}
\markboth{63. The training rule on vehicles }{Yāna}
\extramarks{Bu Sk 63}{Bu Sk 63}

\subsection*{Origin story }

At\marginnote{1.1} one time the Buddha was staying at \textsanskrit{Sāvatthī} in the Jeta Grove, \textsanskrit{Anāthapiṇḍika}’s Monastery. At that time the monks from the group of six gave teachings to people in vehicles. … 

\subsection*{Final ruling }

\scrule{“‘I will not give a teaching to anyone in a vehicle who is not sick,’ this is how you should train.” }

\subsection*{Definitions }

\begin{description}%
\item[A vehicle: ] a wagon, a carriage, a cart, a chariot, a palanquin, a litter. %
\end{description}

One\marginnote{1.6} should not give a teaching to anyone in a vehicle who is not sick. If a monk, out of disrespect, gives a teaching to someone in a vehicle who is not sick, he commits an offense of wrong conduct. 

\subsection*{Non-offenses }

There\marginnote{1.8.1} is no offense: if it is unintentional;  if he is not mindful;  if he does not know;  if he is sick;  if there is an emergency;  if he is insane;  if he is the first offender. 

\scendsutta{The third training rule is finished. }

%
\section*{{\suttatitleacronym Bu Sk 64}{\suttatitletranslation 64. The training rule on lying down }{\suttatitleroot Sayana}}
\addcontentsline{toc}{section}{\tocacronym{Bu Sk 64} \toctranslation{64. The training rule on lying down } \tocroot{Sayana}}
\markboth{64. The training rule on lying down }{Sayana}
\extramarks{Bu Sk 64}{Bu Sk 64}

\subsection*{Origin story }

At\marginnote{1.1} one time the Buddha was staying at \textsanskrit{Sāvatthī} in the Jeta Grove, \textsanskrit{Anāthapiṇḍika}’s Monastery. At that time the monks from the group of six gave teachings to people who were lying down. … 

\subsection*{Final ruling }

\scrule{“‘I will not give a teaching to anyone lying down who is not sick,’ this is how you should train.” }

One\marginnote{1.4} should not give a teaching to anyone lying down who is not sick. If a monk, out of disrespect, gives a teaching to someone who is lying down and who is not sick, even if they are just lying on the ground, he commits an offense of wrong conduct. 

\subsection*{Non-offenses }

There\marginnote{1.6.1} is no offense: if it is unintentional;  if he is not mindful;  if he does not know;  if he is sick;  if there is an emergency;  if he is insane;  if he is the first offender. 

\scendsutta{The fourth training rule is finished. }

%
\section*{{\suttatitleacronym Bu Sk 65}{\suttatitletranslation 65. The training rule on clasping the knees }{\suttatitleroot Pallatthika}}
\addcontentsline{toc}{section}{\tocacronym{Bu Sk 65} \toctranslation{65. The training rule on clasping the knees } \tocroot{Pallatthika}}
\markboth{65. The training rule on clasping the knees }{Pallatthika}
\extramarks{Bu Sk 65}{Bu Sk 65}

\subsection*{Origin story }

At\marginnote{1.1} one time the Buddha was staying at \textsanskrit{Sāvatthī} in the Jeta Grove, \textsanskrit{Anāthapiṇḍika}’s Monastery. At that time the monks from the group of six gave teachings to people who were seated clasping their knees. … 

\subsection*{Final ruling }

\scrule{“‘I will not give a teaching to anyone who is seated clasping their knees and who is not sick,’ this is how you should train.” }

One\marginnote{1.4} should not give a teaching to anyone who is seated clasping their knees and who is not sick. If a monk, out of disrespect, gives a teaching to someone who is clasping their knees with their hands or with a cloth and who is not sick, he commits an offense of wrong conduct. 

\subsection*{Non-offenses }

There\marginnote{1.6.1} is no offense: if it is unintentional;  if he is not mindful;  if he does not know;  if he is sick;  if there is an emergency;  if he is insane;  if he is the first offender. 

\scendsutta{The fifth training rule is finished. }

%
\section*{{\suttatitleacronym Bu Sk 66}{\suttatitletranslation 66. The training rule on headdresses }{\suttatitleroot Veṭhita}}
\addcontentsline{toc}{section}{\tocacronym{Bu Sk 66} \toctranslation{66. The training rule on headdresses } \tocroot{Veṭhita}}
\markboth{66. The training rule on headdresses }{Veṭhita}
\extramarks{Bu Sk 66}{Bu Sk 66}

\subsection*{Origin story }

At\marginnote{1.1} one time the Buddha was staying at \textsanskrit{Sāvatthī} in the Jeta Grove, \textsanskrit{Anāthapiṇḍika}’s Monastery. At that time the monks from the group of six gave teachings to people with headdresses. … 

\subsection*{Final ruling }

\scrule{“‘I will not give a teaching to anyone with a headdress who is not sick,’ this is how you should train.” }

\subsection*{Definitions }

\begin{description}%
\item[A headdress: ] if the ends of the hair are not showing it is a headdress. %
\end{description}

One\marginnote{1.6} should not give a teaching to anyone with a headdress who is not sick. If a monk, out of disrespect, gives a teaching to someone with a headdress who is not sick, he commits an offense of wrong conduct. 

\subsection*{Non-offenses }

There\marginnote{1.8.1} is no offense: if it is unintentional;  if he is not mindful;  if he does not know;  if he is sick;  if he teaches the person after having them uncover the ends of their hair;  if there is an emergency;  if he is insane;  if he is the first offender. 

\scendsutta{The sixth training rule is finished. }

%
\section*{{\suttatitleacronym Bu Sk 67}{\suttatitletranslation 67. The training rule on covered heads }{\suttatitleroot Oguṇṭhita}}
\addcontentsline{toc}{section}{\tocacronym{Bu Sk 67} \toctranslation{67. The training rule on covered heads } \tocroot{Oguṇṭhita}}
\markboth{67. The training rule on covered heads }{Oguṇṭhita}
\extramarks{Bu Sk 67}{Bu Sk 67}

\subsection*{Origin story }

At\marginnote{1.1} one time the Buddha was staying at \textsanskrit{Sāvatthī} in the Jeta Grove, \textsanskrit{Anāthapiṇḍika}’s Monastery. At that time the monks from the group of six gave teachings to people with covered heads. … 

\subsection*{Final ruling }

\scrule{“‘I will not give a teaching to anyone with a covered head who is not sick,’ this is how you should train.” }

\subsection*{Definitions }

\begin{description}%
\item[With a covered head: ] the upper robe covering the head is what is meant. %
\end{description}

One\marginnote{1.6} should not give a teaching to anyone with a covered head who is not sick. If a monk, out of disrespect, gives a teaching to someone with a covered head who is not sick, he commits an offense of wrong conduct. 

\subsection*{Non-offenses }

There\marginnote{1.8.1} is no offense: if it is unintentional;  if he is not mindful;  if he does not know;  if he is sick;  if he teaches the person after having them uncover their head;  if there is an emergency;  if he is insane;  if he is the first offender. 

\scendsutta{The seventh training rule is finished. }

%
\section*{{\suttatitleacronym Bu Sk 68}{\suttatitletranslation 68. The training rule on the ground }{\suttatitleroot Chamā}}
\addcontentsline{toc}{section}{\tocacronym{Bu Sk 68} \toctranslation{68. The training rule on the ground } \tocroot{Chamā}}
\markboth{68. The training rule on the ground }{Chamā}
\extramarks{Bu Sk 68}{Bu Sk 68}

\subsection*{Origin story }

At\marginnote{1.1} one time the Buddha was staying at \textsanskrit{Sāvatthī} in the Jeta Grove, \textsanskrit{Anāthapiṇḍika}’s Monastery. At that time the monks from the group of six were sitting on the ground giving teachings to people sitting on seats. … 

\subsection*{Final ruling }

\scrule{“‘I will not give a teaching while sitting on the ground to anyone sitting on a seat who is not sick,’ this is how you should train.” }

One\marginnote{1.4} should not give a teaching while sitting on the ground to anyone sitting on a seat who is not sick. If a monk, out of disrespect, gives a teaching while sitting on the ground to someone sitting on a seat who is not sick, he commits an offense of wrong conduct. 

\subsection*{Non-offenses }

There\marginnote{1.6.1} is no offense: if it is unintentional;  if he is not mindful;  if he does not know;  if he is sick;  if there is an emergency;  if he is insane;  if he is the first offender. 

\scendsutta{The eighth training rule is finished. }

%
\section*{{\suttatitleacronym Bu Sk 69}{\suttatitletranslation 69. The training rule on low seats }{\suttatitleroot Nīcāsana}}
\addcontentsline{toc}{section}{\tocacronym{Bu Sk 69} \toctranslation{69. The training rule on low seats } \tocroot{Nīcāsana}}
\markboth{69. The training rule on low seats }{Nīcāsana}
\extramarks{Bu Sk 69}{Bu Sk 69}

\subsection*{Origin story }

At\marginnote{1.1} one time the Buddha was staying at \textsanskrit{Sāvatthī} in the Jeta Grove, \textsanskrit{Anāthapiṇḍika}’s Monastery. At that time the monks from the group of six were sitting on low seats while giving teachings to people sitting on high seats. 

The\marginnote{1.3} monks of few desires complained and criticized them, “How can the monks from the group of six sit on low seats while giving teachings to people sitting on high seats?” … “Is it true, monks, that you do this?” 

“It’s\marginnote{1.6} true, sir.” 

The\marginnote{1.7} Buddha rebuked them … “Foolish men, how can you do this? This will affect people’s confidence …” After rebuking them … he gave a teaching and addressed the monks: 

\subparagraph*{Jataka }

“Once\marginnote{1.12.1} upon a time in \textsanskrit{Bārāṇasī}, monks, there was a low-caste man whose wife became pregnant. She said to him, ‘I’m pregnant. I crave mangoes.’ 

‘But\marginnote{1.16} there are no mangoes. It’s the wrong season.’ 

‘If\marginnote{1.17} I don’t get any, I’ll die.’ 

At\marginnote{1.18} that time the king had a mango tree that was always bearing fruit. Then that low-caste man went to that mango tree, climbed it, and hid himself. Just then the king and his brahmin counselor went to that same mango tree. There the king sat on a high seat while learning the Vedas. The low-caste man thought, ‘How wrong-headed this king is, sitting on a high seat while learning the Vedas. And the brahmin is wrong-headed too, sitting on a low seat while teaching the Vedas to one sitting on a high seat. And I’m wrong-headed too, stealing mangoes from the king because of a woman. This is all so low!’\footnote{Sp-\textsanskrit{ṭ} 2.647: \textit{\textsanskrit{Yaṁ} amhehi \textsanskrit{tīhi} janehi \textsanskrit{kataṁ}, \textsanskrit{sabbamidaṁ} \textsanskrit{kiccaṁ} \textsanskrit{lāmakaṁ} \textsanskrit{nimmariyādaṁ} \textsanskrit{adhammikaṁ}}, “What the three of us have done is all low, outside of acceptable bounds, and unrighteous.” }  And he fell out of the tree right there. 

\begin{verse}%
\scspeaker{The\marginnote{1.27.0} low-caste man says: }\\
‘Neither understands what is good,\footnote{The attribution of speakers is according to Sp 2.647. } \\
Neither sees the Truth: \\
Not he who teaches the Vedas, \\
Nor he who learns improperly.’ 

\scspeaker{The\marginnote{1.31.0} brahmin replies: }\\
‘I’ve eaten the finest rice, \\
With a curry of pure meat: \\
Therefore I’m not practicing the Teaching, \\
The Teaching praised by the Noble Ones.’ 

\scspeaker{The\marginnote{1.35.0} low-caste man replies in turn: }\\
‘A curse it is the obtaining of wealth, \\
And so is becoming popular, brahmin; \\
These things come with a low rebirth, \\
Or with wrong-headed behavior. 

‘Go\marginnote{1.39} forth, great brahmin, \\
Other beings will do the cooking; \\
Don’t go against the Teaching, \\
Because you’ll break like a jar.’ 

%
\end{verse}

Even\marginnote{1.43} at that time, monks, I was displeased by someone teaching the Vedas while sitting on a low seat to someone sitting on a high seat. How, then, could it not be displeasing now? This will affect people’s confidence …” … “And, monks, this training rule should be recited like this: 

\subsection*{Final ruling }

\scrule{‘“I will not give a teaching while sitting on a low seat to anyone sitting on a high seat who is not sick,” this is how you should train.’” }

One\marginnote{1.47} should not give a teaching while sitting on a low seat to anyone sitting on a high seat who is not sick. If a monk, out of disrespect, gives a teaching while sitting on a low seat to someone sitting on a high seat who is not sick, he commits an offense of wrong conduct. 

\subsection*{Non-offenses }

There\marginnote{1.49.1} is no offense: if it is unintentional;  if he is not mindful;  if he does not know;  if he is sick;  if there is an emergency;  if he is insane;  if he is the first offender. 

\scendsutta{The ninth training rule is finished. }

%
\section*{{\suttatitleacronym Bu Sk 70}{\suttatitletranslation 70. The training rule on standing }{\suttatitleroot Ṭhita}}
\addcontentsline{toc}{section}{\tocacronym{Bu Sk 70} \toctranslation{70. The training rule on standing } \tocroot{Ṭhita}}
\markboth{70. The training rule on standing }{Ṭhita}
\extramarks{Bu Sk 70}{Bu Sk 70}

\subsection*{Origin story }

At\marginnote{1.1} one time the Buddha was staying at \textsanskrit{Sāvatthī} in the Jeta Grove, \textsanskrit{Anāthapiṇḍika}’s Monastery. At that time the monks from the group of six were standing while giving teachings to people who were sitting. … 

\subsection*{Final ruling }

\scrule{“‘I will not give a teaching while standing to anyone sitting who is not sick,’ this is how you should train.” }

One\marginnote{1.4} should not give a teaching while standing to anyone sitting who is not sick. If a monk, out of disrespect, gives a teaching while standing to someone sitting who is not sick, he commits an offense of wrong conduct. 

\subsection*{Non-offenses }

There\marginnote{1.6.1} is no offense: if it is unintentional;  if he is not mindful;  if he does not know;  if he is sick;  if there is an emergency;  if he is insane;  if he is the first offender. 

\scendsutta{The tenth training rule is finished. }

%
\section*{{\suttatitleacronym Bu Sk 71}{\suttatitletranslation 71. The training rule on walking behind }{\suttatitleroot Pacchatogamana}}
\addcontentsline{toc}{section}{\tocacronym{Bu Sk 71} \toctranslation{71. The training rule on walking behind } \tocroot{Pacchatogamana}}
\markboth{71. The training rule on walking behind }{Pacchatogamana}
\extramarks{Bu Sk 71}{Bu Sk 71}

\subsection*{Origin story }

At\marginnote{1.1} one time the Buddha was staying at \textsanskrit{Sāvatthī} in the Jeta Grove, \textsanskrit{Anāthapiṇḍika}’s Monastery. At that time the monks from the group of six were giving teachings to people walking in front of them. … 

\subsection*{Final ruling }

\scrule{“‘I will not give a teaching to anyone walking in front of me who is not sick,’ this is how you should train.” }

One\marginnote{1.4} should not give a teaching to anyone walking in front of oneself who is not sick. If a monk, out of disrespect, gives a teaching to someone walking in front of him who is not sick, he commits an offense of wrong conduct. 

\subsection*{Non-offenses }

There\marginnote{1.6.1} is no offense: if it is unintentional;  if he is not mindful;  if he does not know;  if he is sick;  if there is an emergency;  if he is insane;  if he is the first offender. 

\scendsutta{The eleventh training rule is finished. }

%
\section*{{\suttatitleacronym Bu Sk 72}{\suttatitletranslation 72. The training rule on walking next to the path }{\suttatitleroot Uppathenagamana}}
\addcontentsline{toc}{section}{\tocacronym{Bu Sk 72} \toctranslation{72. The training rule on walking next to the path } \tocroot{Uppathenagamana}}
\markboth{72. The training rule on walking next to the path }{Uppathenagamana}
\extramarks{Bu Sk 72}{Bu Sk 72}

\subsection*{Origin story }

At\marginnote{1.1} one time the Buddha was staying at \textsanskrit{Sāvatthī} in the Jeta Grove, \textsanskrit{Anāthapiṇḍika}’s Monastery. At that time the monks from the group of six were walking next to the path while giving teachings to people walking on the path. … 

\subsection*{Final ruling }

\scrule{“‘I will not give a teaching while walking next to the path to anyone walking on the path who is not sick,’ this is how you should train.” }

One\marginnote{1.4} should not give a teaching while walking next to the path to anyone walking on the path who is not sick. If a monk, out of disrespect, gives a teaching while walking next to the path to someone walking on the path who is not sick, he commits an offense of wrong conduct. 

\subsection*{Non-offenses }

There\marginnote{1.6.1} is no offense: if it is unintentional;  if he is not mindful;  if he does not know;  if he is sick;  if there is an emergency;  if he is insane;  if he is the first offender. 

\scendsutta{The twelfth training rule is finished. }

%
\section*{{\suttatitleacronym Bu Sk 73}{\suttatitletranslation 73. The training rule on defecating while standing }{\suttatitleroot Ṭhitouccāra}}
\addcontentsline{toc}{section}{\tocacronym{Bu Sk 73} \toctranslation{73. The training rule on defecating while standing } \tocroot{Ṭhitouccāra}}
\markboth{73. The training rule on defecating while standing }{Ṭhitouccāra}
\extramarks{Bu Sk 73}{Bu Sk 73}

\subsection*{Origin story }

At\marginnote{1.1} one time the Buddha was staying at \textsanskrit{Sāvatthī} in the Jeta Grove, \textsanskrit{Anāthapiṇḍika}’s Monastery. At that time the monks from the group of six were defecating and urinating while standing. … 

\subsection*{Final ruling }

\scrule{“‘When not sick, I will not defecate or urinate while standing,’ this is how you should train.” }

If\marginnote{1.4} one is not sick, one should not defecate or urinate while standing. If, out of disrespect, a monk who is not sick defecates or urinates while standing, he commits an offense of wrong conduct. 

\subsection*{Non-offenses }

There\marginnote{1.6.1} is no offense: if it is unintentional;  if he is not mindful;  if he does not know;  if he is sick;  if there is an emergency;  if he is insane;  if he is the first offender. 

\scendsutta{The thirteenth training rule is finished. }

%
\section*{{\suttatitleacronym Bu Sk 74}{\suttatitletranslation 74. The training rule on defecating on cultivated plants }{\suttatitleroot Hariteuccāra}}
\addcontentsline{toc}{section}{\tocacronym{Bu Sk 74} \toctranslation{74. The training rule on defecating on cultivated plants } \tocroot{Hariteuccāra}}
\markboth{74. The training rule on defecating on cultivated plants }{Hariteuccāra}
\extramarks{Bu Sk 74}{Bu Sk 74}

\subsection*{Origin story }

At\marginnote{1.1} one time the Buddha was staying at \textsanskrit{Sāvatthī} in the Jeta Grove, \textsanskrit{Anāthapiṇḍika}’s Monastery. At that time the monks from the group of six were defecating, urinating, and spitting on cultivated plants. … 

\subsection*{Final ruling }

\scrule{“‘When not sick, I will not defecate, urinate, or spit on cultivated plants,’ this is how you should train.”\footnote{\textit{Harite} could in principle refer to all plants, but it is elsewhere defined as what is cultivated, see \href{https://suttacentral.net/pli-tv-bu-vb-pc19/en/brahmali\#2.1.14}{Bu Pc 19:2.1.14} and \href{https://suttacentral.net/pli-tv-bi-vb-pc9/en/brahmali\#2.1.14}{Bi Pc 9:2.1.14}. } }

If\marginnote{1.4} one is not sick, one should not defecate, urinate, or spit on cultivated plants. If, out of disrespect, a monk who is not sick defecates, urinates, or spits on cultivated plants, he commits an offense of wrong conduct. 

\subsection*{Non-offenses }

There\marginnote{1.6.1} is no offense: if it is unintentional;  if he is not mindful;  if he does not know;  if he is sick;  if he does it in a place with no cultivated plants, but it then spreads to cultivated plants;  if there is an emergency;  if he is insane;  if he is the first offender. 

\scendsutta{The fourteenth training rule is finished. }

%
\section*{{\suttatitleacronym Bu Sk 75}{\suttatitletranslation 75. The training rule on defecating in water }{\suttatitleroot Udakeuccāra}}
\addcontentsline{toc}{section}{\tocacronym{Bu Sk 75} \toctranslation{75. The training rule on defecating in water } \tocroot{Udakeuccāra}}
\markboth{75. The training rule on defecating in water }{Udakeuccāra}
\extramarks{Bu Sk 75}{Bu Sk 75}

\subsection*{Origin story }

\subsubsection*{First sub-story }

At\marginnote{1.1} one time the Buddha was staying at \textsanskrit{Sāvatthī} in the Jeta Grove, \textsanskrit{Anāthapiṇḍika}’s Monastery. At that time the monks from the group of six were defecating, urinating, and spitting in water. People complained and criticized them, “How can the Sakyan monastics defecate, urinate, and spit in water? They’re just like householders who indulge in worldly pleasures!” 

The\marginnote{1.5} monks heard the complaints of those people, and the monks of few desires complained and criticized those monks, “How can the monks from the group of six do this?” … “Is it true, monks, that you do this?” 

“It’s\marginnote{1.9} true, sir.” 

The\marginnote{1.10} Buddha rebuked them … “Foolish men, how can you do this? This will affect people’s confidence …” … “And, monks, this training rule should be recited like this: 

\subsubsection*{Preliminary ruling }

\scrule{‘“I will not defecate, urinate, or spit in water,” this is how you should train.’” }

In\marginnote{1.15} this way the Buddha laid down this training rule for the monks. 

\subsubsection*{Second sub-story }

Soon\marginnote{2.1} afterwards, being afraid of wrongdoing, sick monks did not defecate, urinate, or spit in water. They told the Buddha. Soon afterwards he gave a teaching and addressed the monks: 

\scrule{“Monks, I allow a sick monk to defecate, urinate, or spit in water. }

And\marginnote{2.5} so, monks, this training rule should be recited like this: 

\subsection*{Final ruling }

\scrule{‘“When not sick, I will not defecate, urinate, or spit in water,” this is how you should train.’” }

If\marginnote{2.7} one is not sick, one should not defecate, urinate, or spit in water. If, out of disrespect, a monk who is not sick defecates, urinates, or spits in water, he commits an offense of wrong conduct. 

\subsection*{Non-offenses }

There\marginnote{2.9.1} is no offense: if it is unintentional;  if he is not mindful;  if he does not know;  if he is sick;  if he does it on dry ground, but it then spreads to water;  if there is an emergency;  if he is insane;  if he is deranged;  if he is overwhelmed by pain;  if he is the first offender. 

\scendsutta{The fifteenth training rule is finished. }

\scendvagga{The seventh subchapter on shoes is finished. }

“Venerables,\marginnote{2.22} the rules to be trained in have been recited. In regard to this I ask you, ‘Are you pure in this?’ A second time I ask, ‘Are you pure in this?’ A third time I ask, ‘Are you pure in this?’ You are pure in this and therefore silent. I’ll remember it thus.” 

\scend{The rules to be trained in are finished. }

\scendkanda{The chapter on training is finished. }

%
\addtocontents{toc}{\let\protect\contentsline\protect\nopagecontentsline}
\chapter*{Settling Legal Issues }
\addcontentsline{toc}{chapter}{\tocchapterline{Settling Legal Issues }}
\addtocontents{toc}{\let\protect\contentsline\protect\oldcontentsline}

%
%
\section*{{\suttatitleacronym  Bu As 1–7}{\suttatitletranslation The settling of legal issues }{\suttatitleroot Adhikaraṇasamatha}}
\addcontentsline{toc}{section}{\tocacronym{ Bu As 1–7} \toctranslation{The settling of legal issues } \tocroot{Adhikaraṇasamatha}}
\markboth{The settling of legal issues }{Adhikaraṇasamatha}
\extramarks{ Bu As 1–7}{ Bu As 1–7}

“Venerables,\marginnote{1.1} these seven principles for the settling of legal issues come up for recitation. 

For\marginnote{2.1} the settling and resolving of legal issues whenever they arise there is: 

\scrule{Resolution face-to-face to be applied; }

\scrule{Resolution through recollection to be granted; }

\scrule{Resolution because of past insanity to be granted; }

\scrule{Acting according to what has been admitted; }

\scrule{Majority decision; }

\scrule{Further penalty; }

\scrule{Covering over as if with grass. }

Venerables,\marginnote{2.1} the seven principles for the settling of legal issues have been recited. In regard to this I ask you, ‘Are you pure in this?’ A second time I ask, ‘Are you pure in this?’ A third time I ask, ‘Are you pure in this?’ You are pure in this and therefore silent. I’ll remember it thus.” 

\scendsutta{The seven principles for the settling of legal issues are finished. }

“Venerables,\marginnote{4.1} the introduction has been recited; the four rules on expulsion have been recited; the thirteen rules on suspension have been recited; the two undetermined rules have been recited; the thirty rules on relinquishment and confession have been recited; the ninety-two rules on confession have been recited; the four rules on acknowledgment have been recited; the rules to be trained in have been recited; the seven principles for the settling of legal issues have been recited. This much has come down and is included in the Monastic Code of the Buddha and comes up for recitation every half-month. In regard to this everyone should train in unity, in concord, without dispute.” 

\scendbook{The Great Analysis is finished. }

%
\backmatter%
%
\chapter*{Appendices}
\addcontentsline{toc}{chapter}{Appendices}
\markboth{Appendices}{Appendices}

\emph{Appendices for all volumes may be found at the end of the first volume, The Great Analysis, part I.}

%
\chapter*{Colophon}
\addcontentsline{toc}{chapter}{Colophon}
\markboth{Colophon}{Colophon}

\section*{The Translator}

Bhikkhu Brahmali was born Norway in 1964. He first became interested in Buddhism and meditation in his early 20s after a visit to Japan. Having completed degrees in engineering and finance, he began his monastic training as an anagarika (keeping the eight precepts) in England at Amaravati and Chithurst Buddhist Monastery.

After hearing teachings from Ajahn Brahm he decided to travel to Australia to train at Bodhinyana Monastery. Bhikkhu Brahmali has lived at Bodhinyana Monastery since 1994, and was ordained as a Bhikkhu, with Ajahn Brahm as his preceptor, in 1996. In 2015 he entered his 20th Rains Retreat as a fully ordained monastic and received the title Maha Thera (Great Elder).

Bhikkhu Brahmali’s knowledge of the Pali language and of the Suttas is excellent. Bhikkhu Bodhi, who translated most of the Pali Canon into English for Wisdom Publications, called him one of his major helpers for the 2012 translation of \emph{The Numerical Discourses of the Buddha}. He has also published two essays on Dependent Origination and a book called \emph{The Authenticity of the Early Buddhist Texts} with the Buddhist Publication Society in collaboration with Bhante Sujato.

The monastics of the Buddhist Society of WA (BSWA) often turn to him to clarify Vinaya (monastic discipline) or Sutta questions. They also greatly appreciate his Sutta and Pali classes. Furthermore he has been instrumental in most of the building and maintenance projects at Bodhinyana Monastery and at the emerging Hermit Hill property in Serpentine.

\section*{Creation Process}

Translated from the Pali. The primary source was the \textsanskrit{Mahāsaṅgīti} edition, with occasional reference to other Pali editions, especially the \textsanskrit{Chaṭṭha} \textsanskrit{Saṅgāyana} edition and the Pali Text Society edition. I cross-checked with I.B. Horner’s English translation, “The Book of the Discipline”, as well as Bhikkhu \textsanskrit{Ñāṇatusita}’s “A Translation and Analysis of the \textsanskrit{Pātimokkha}” and Ajahn \textsanskrit{Ṭhānissaro}’s “Buddhist Monastic Code”.

\section*{The Translation}

This is the first complete translation of the Vinaya \textsanskrit{Piṭaka} in English. The aim has been to produce a translation that is easy to read, clear, and accurate, and also modern in vocabulary and style.

\section*{About SuttaCentral}

SuttaCentral publishes early Buddhist texts. Since 2005 we have provided root texts in Pali, Chinese, Sanskrit, Tibetan, and other languages, parallels between these texts, and translations in many modern languages. Building on the work of generations of scholars, we offer our contribution freely.

SuttaCentral is driven by volunteer contributions, and in addition we employ professional developers. We offer a sponsorship program for high quality translations from the original languages. Financial support for SuttaCentral is handled by the SuttaCentral Development Trust, a charitable trust registered in Australia.

\section*{About Bilara}

“Bilara” means “cat” in Pali, and it is the name of our Computer Assisted Translation (CAT) software. Bilara is a web app that enables translators to translate early Buddhist texts into their own language. These translations are published on SuttaCentral with the root text and translation side by side.

\section*{About SuttaCentral Editions}

The SuttaCentral Editions project makes high quality books from selected Bilara translations. These are published in formats including HTML, EPUB, PDF, and print.

You are welcome to print any of our Editions.

%
\end{document}