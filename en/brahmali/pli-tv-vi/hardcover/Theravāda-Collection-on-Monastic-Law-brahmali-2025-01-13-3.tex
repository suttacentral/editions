\documentclass[12pt,openany]{book}%
\usepackage{lastpage}%
%
\usepackage{ragged2e}
\usepackage{verse}
\usepackage[a-3u]{pdfx}
\usepackage[inner=1in, outer=1in, top=.7in, bottom=1in, papersize={6in,9in}, headheight=13pt]{geometry}
\usepackage{polyglossia}
\usepackage[12pt]{moresize}
\usepackage{soul}%
\usepackage{microtype}
\usepackage{tocbasic}
\usepackage{realscripts}
\usepackage{epigraph}%
\usepackage{setspace}%
\usepackage{sectsty}
\usepackage{fontspec}
\usepackage{marginnote}
\usepackage[bottom]{footmisc}
\usepackage{enumitem}
\usepackage{fancyhdr}
\usepackage{emptypage}
\usepackage{extramarks}
\usepackage{graphicx}
\usepackage{relsize}
\usepackage{etoolbox}

% improve ragged right headings by suppressing hyphenation and orphans. spaceskip plus and minus adjust interword spacing; increase rightskip stretch to make it want to push a word on the first line(s) to the next line; reduce parfillskip stretch to make line length more equal . spacefillskip and xspacefillskip can be deleted to use defaults.
\protected\def\BalancedRagged{
\leftskip     0pt
\rightskip    0pt plus 10em
\spaceskip=1\fontdimen2\font plus .5\fontdimen3\font minus 1.5\fontdimen4\font
\xspaceskip=1\fontdimen2\font plus 1\fontdimen3\font minus 1\fontdimen4\font
\parfillskip  0pt plus 15em
\relax
}

\hypersetup{
colorlinks=true,
urlcolor=black,
linkcolor=black,
citecolor=black,
allcolors=black
}

% use a small amount of tracking on small caps
\SetTracking[ spacing = {25*,166, } ]{ encoding = *, shape = sc }{ 25 }

% add a blank page
\newcommand{\blankpage}{
\newpage
\thispagestyle{empty}
\mbox{}
\newpage
}

% define languages
\setdefaultlanguage[]{english}
\setotherlanguage[script=Latin]{sanskrit}

%\usepackage{pagegrid}
%\pagegridsetup{top-left, step=.25in}

% define fonts
% use if arno sanskrit is unavailable
%\setmainfont{Gentium Plus}
%\newfontfamily\Marginalfont[]{Gentium Plus}
%\newfontfamily\Allsmallcapsfont[RawFeature=+c2sc]{Gentium Plus}
%\newfontfamily\Noligaturefont[Renderer=Basic]{Gentium Plus}
%\newfontfamily\Noligaturecaptionfont[Renderer=Basic]{Gentium Plus}
%\newfontfamily\Fleuronfont[Ornament=1]{Gentium Plus}

% use if arno sanskrit is available. display is applied to \chapter and \part, subhead to \section and \subsection.
\setmainfont[
  FontFace={sb}{n}{Font = {Arno Pro Semibold}},
  FontFace={sb}{it}{Font = {Arno  Pro Semibold Italic}}
]{Arno Pro}

% create commands for using semibold
\DeclareRobustCommand{\sbseries}{\fontseries{sb}\selectfont}
\DeclareTextFontCommand{\textsb}{\sbseries}

\newfontfamily\Marginalfont[RawFeature=+subs]{Arno Pro Regular}
\newfontfamily\Allsmallcapsfont[RawFeature=+c2sc]{Arno Pro}
\newfontfamily\Noligaturefont[Renderer=Basic]{Arno Pro}
\newfontfamily\Noligaturecaptionfont[Renderer=Basic]{Arno Pro Caption}

% chinese fonts
\newfontfamily\cjk{Noto Serif TC}
\newcommand*{\langlzh}[1]{\cjk{#1}\normalfont}%

% logo
\newfontfamily\Logofont{sclogo.ttf}
\newcommand*{\sclogo}[1]{\large\Logofont{#1}}

% use subscript numerals for margin notes
\renewcommand*{\marginfont}{\Marginalfont}

% ensure margin notes have consistent vertical alignment
\renewcommand*{\marginnotevadjust}{-.17em}

% use compact lists
\setitemize{noitemsep,leftmargin=1em}
\setenumerate{noitemsep,leftmargin=1em}
\setdescription{noitemsep, style=unboxed, leftmargin=1em}

% style ToC
\DeclareTOCStyleEntries[
  raggedentrytext,
  linefill=\hfill,
  pagenumberwidth=.5in,
  pagenumberformat=\normalfont,
  entryformat=\normalfont
]{tocline}{chapter,section}


  \setlength\topsep{0pt}%
  \setlength\parskip{0pt}%

% define new \centerpars command for use in ToC. This ensures centering, proper wrapping, and no page break after
\def\startcenter{%
  \par
  \begingroup
  \leftskip=0pt plus 1fil
  \rightskip=\leftskip
  \parindent=0pt
  \parfillskip=0pt
}
\def\stopcenter{%
  \par
  \endgroup
}
\long\def\centerpars#1{\startcenter#1\stopcenter}

% redefine part, so that it adds a toc entry without page number
\let\oldcontentsline\contentsline
\newcommand{\nopagecontentsline}[3]{\oldcontentsline{#1}{#2}{}}

    \makeatletter
\renewcommand*\l@part[2]{%
  \ifnum \c@tocdepth >-2\relax
    \addpenalty{-\@highpenalty}%
    \addvspace{0em \@plus\p@}%
    \setlength\@tempdima{3em}%
    \begingroup
      \parindent \z@ \rightskip \@pnumwidth
      \parfillskip -\@pnumwidth
      {\leavevmode
       \setstretch{.85}\large\scshape\centerpars{#1}\vspace*{-1em}\llap{#2}}\par
       \nobreak
         \global\@nobreaktrue
         \everypar{\global\@nobreakfalse\everypar{}}%
    \endgroup
  \fi}
\makeatother

\makeatletter
\def\@pnumwidth{2em}
\makeatother

% define new sectioning command, which is only used in volumes where the pannasa is found in some parts but not others, especially in an and sn

\newcommand*{\pannasa}[1]{\clearpage\thispagestyle{empty}\begin{center}\vspace*{14em}\setstretch{.85}\huge\itshape\scshape\MakeLowercase{#1}\end{center}}

    \makeatletter
\newcommand*\l@pannasa[2]{%
  \ifnum \c@tocdepth >-2\relax
    \addpenalty{-\@highpenalty}%
    \addvspace{.5em \@plus\p@}%
    \setlength\@tempdima{3em}%
    \begingroup
      \parindent \z@ \rightskip \@pnumwidth
      \parfillskip -\@pnumwidth
      {\leavevmode
       \setstretch{.85}\large\itshape\scshape\lowercase{\centerpars{#1}}\vspace*{-1em}\llap{#2}}\par
       \nobreak
         \global\@nobreaktrue
         \everypar{\global\@nobreakfalse\everypar{}}%
    \endgroup
  \fi}
\makeatother

% don't put page number on first page of toc (relies on etoolbox)
\patchcmd{\chapter}{plain}{empty}{}{}

% global line height
\setstretch{1.05}

% allow linebreak after em-dash
\catcode`\—=13
\protected\def—{\unskip\textemdash\allowbreak}

% style headings with secsty. chapter and section are defined per-edition
\partfont{\setstretch{.85}\normalfont\centering\textsc}
\subsectionfont{\setstretch{.95}\normalfont\BalancedRagged}%
\subsubsectionfont{\setstretch{1}\normalfont\itshape\BalancedRagged}

% style elements of suttatitle
\newcommand*{\suttatitleacronym}[1]{\smaller[2]{#1}\vspace*{.3em}}
\newcommand*{\suttatitletranslation}[1]{\linebreak{#1}}
\newcommand*{\suttatitleroot}[1]{\linebreak\smaller[2]\itshape{#1}}

\DeclareTOCStyleEntries[
  indent=3.3em,
  dynindent,
  beforeskip=.2em plus -2pt minus -1pt,
]{tocline}{section}

\DeclareTOCStyleEntries[
  indent=0em,
  dynindent,
  beforeskip=.4em plus -2pt minus -1pt,
]{tocline}{chapter}

\newcommand*{\tocacronym}[1]{\hspace*{-3.3em}{#1}\quad}
\newcommand*{\toctranslation}[1]{#1}
\newcommand*{\tocroot}[1]{(\textit{#1})}
\newcommand*{\tocchapterline}[1]{\bfseries\itshape{#1}}


% redefine paragraph and subparagraph headings to not be inline
\makeatletter
% Change the style of paragraph headings %
\renewcommand\paragraph{\@startsection{paragraph}{4}{\z@}%
            {-2.5ex\@plus -1ex \@minus -.25ex}%
            {1.25ex \@plus .25ex}%
            {\noindent\normalfont\itshape\small}}

% Change the style of subparagraph headings %
\renewcommand\subparagraph{\@startsection{subparagraph}{5}{\z@}%
            {-2.5ex\@plus -1ex \@minus -.25ex}%
            {1.25ex \@plus .25ex}%
            {\noindent\normalfont\itshape\footnotesize}}
\makeatother

% use etoolbox to suppress page numbers on \part
\patchcmd{\part}{\thispagestyle{plain}}{\thispagestyle{empty}}
  {}{\errmessage{Cannot patch \string\part}}

% and to reduce margins on quotation
\patchcmd{\quotation}{\rightmargin}{\leftmargin 1.2em \rightmargin}{}{}
\AtBeginEnvironment{quotation}{\small}

% titlepage
\newcommand*{\titlepageTranslationTitle}[1]{{\begin{center}\begin{large}{#1}\end{large}\end{center}}}
\newcommand*{\titlepageCreatorName}[1]{{\begin{center}\begin{normalsize}{#1}\end{normalsize}\end{center}}}

% halftitlepage
\newcommand*{\halftitlepageTranslationTitle}[1]{\setstretch{2.5}{\begin{Huge}\uppercase{\so{#1}}\end{Huge}}}
\newcommand*{\halftitlepageTranslationSubtitle}[1]{\setstretch{1.2}{\begin{large}{#1}\end{large}}}
\newcommand*{\halftitlepageFleuron}[1]{{\begin{large}\Fleuronfont{{#1}}\end{large}}}
\newcommand*{\halftitlepageByline}[1]{{\begin{normalsize}\textit{{#1}}\end{normalsize}}}
\newcommand*{\halftitlepageCreatorName}[1]{{\begin{LARGE}{\textsc{#1}}\end{LARGE}}}
\newcommand*{\halftitlepageVolumeNumber}[1]{{\begin{normalsize}{\Allsmallcapsfont{\textsc{#1}}}\end{normalsize}}}
\newcommand*{\halftitlepageVolumeAcronym}[1]{{\begin{normalsize}{#1}\end{normalsize}}}
\newcommand*{\halftitlepageVolumeTranslationTitle}[1]{{\begin{Large}{\textsc{#1}}\end{Large}}}
\newcommand*{\halftitlepageVolumeRootTitle}[1]{{\begin{normalsize}{\Allsmallcapsfont{\textsc{\itshape #1}}}\end{normalsize}}}
\newcommand*{\halftitlepagePublisher}[1]{{\begin{large}{\Noligaturecaptionfont\textsc{#1}}\end{large}}}

% epigraph
\renewcommand{\epigraphflush}{center}
\renewcommand*{\epigraphwidth}{.85\textwidth}
\newcommand*{\epigraphTranslatedTitle}[1]{\vspace*{.5em}\footnotesize\textsc{#1}\\}%
\newcommand*{\epigraphRootTitle}[1]{\footnotesize\textit{#1}\\}%
\newcommand*{\epigraphReference}[1]{\footnotesize{#1}}%

% map
\newsavebox\IBox

% custom commands for html styling classes
\newcommand*{\scnamo}[1]{\begin{Center}\textit{#1}\end{Center}\bigskip}
\newcommand*{\scendsection}[1]{\begin{Center}\begin{small}\textit{#1}\end{small}\end{Center}\addvspace{1em}}
\newcommand*{\scendsutta}[1]{\begin{Center}\textit{#1}\end{Center}\addvspace{1em}}
\newcommand*{\scendbook}[1]{\bigskip\begin{Center}\uppercase{#1}\end{Center}\addvspace{1em}}
\newcommand*{\scendkanda}[1]{\begin{Center}\textbf{#1}\end{Center}\addvspace{1em}} % use for ending vinaya rule sections and also samyuttas %
\newcommand*{\scend}[1]{\begin{Center}\begin{small}\textit{#1}\end{small}\end{Center}\addvspace{1em}}
\newcommand*{\scendvagga}[1]{\begin{Center}\textbf{#1}\end{Center}\addvspace{1em}}
\newcommand*{\scrule}[1]{\textsb{#1}}
\newcommand*{\scadd}[1]{\textit{#1}}
\newcommand*{\scevam}[1]{\textsc{#1}}
\newcommand*{\scspeaker}[1]{\hspace{2em}\textit{#1}}
\newcommand*{\scbyline}[1]{\begin{flushright}\textit{#1}\end{flushright}\bigskip}
\newcommand*{\scexpansioninstructions}[1]{\begin{small}\textit{#1}\end{small}}
\newcommand*{\scuddanaintro}[1]{\medskip\noindent\begin{footnotesize}\textit{#1}\end{footnotesize}\smallskip}

\newenvironment{scuddana}{%
\setlength{\stanzaskip}{.5\baselineskip}%
  \vspace{-1em}\begin{verse}\begin{footnotesize}%
}{%
\end{footnotesize}\end{verse}
}%

% custom command for thematic break = hr
\newcommand*{\thematicbreak}{\begin{center}\rule[.5ex]{6em}{.4pt}\begin{normalsize}\quad\Fleuronfont{•}\quad\end{normalsize}\rule[.5ex]{6em}{.4pt}\end{center}}

% manage and style page header and footer. "fancy" has header and footer, "plain" has footer only

\pagestyle{fancy}
\fancyhf{}
\fancyfoot[RE,LO]{\thepage}
\fancyfoot[LE,RO]{\footnotesize\lastleftxmark}
\fancyhead[CE]{\setstretch{.85}\Noligaturefont\MakeLowercase{\textsc{\firstrightmark}}}
\fancyhead[CO]{\setstretch{.85}\Noligaturefont\MakeLowercase{\textsc{\firstleftmark}}}
\renewcommand{\headrulewidth}{0pt}
\fancypagestyle{plain}{ %
\fancyhf{} % remove everything
\fancyfoot[RE,LO]{\thepage}
\fancyfoot[LE,RO]{\footnotesize\lastleftxmark}
\renewcommand{\headrulewidth}{0pt}
\renewcommand{\footrulewidth}{0pt}}
\fancypagestyle{plainer}{ %
\fancyhf{} % remove everything
\fancyfoot[RE,LO]{\thepage}
\renewcommand{\headrulewidth}{0pt}
\renewcommand{\footrulewidth}{0pt}}

% style footnotes
\setlength{\skip\footins}{1em}

\makeatletter
\newcommand{\@makefntextcustom}[1]{%
    \parindent 0em%
    \thefootnote.\enskip #1%
}
\renewcommand{\@makefntext}[1]{\@makefntextcustom{#1}}
\makeatother

% hang quotes (requires microtype)
\microtypesetup{
  protrusion = true,
  expansion  = true,
  tracking   = true,
  factor     = 1000,
  patch      = all,
  final
}

% Custom protrusion rules to allow hanging punctuation
\SetProtrusion
{ encoding = *}
{
% char   right left
  {-} = {    , 500 },
  % Double Quotes
  \textquotedblleft
      = {1000,     },
  \textquotedblright
      = {    , 1000},
  \quotedblbase
      = {1000,     },
  % Single Quotes
  \textquoteleft
      = {1000,     },
  \textquoteright
      = {    , 1000},
  \quotesinglbase
      = {1000,     }
}

% make latex use actual font em for parindent, not Computer Modern Roman
\AtBeginDocument{\setlength{\parindent}{1em}}%
%

% Default values; a bit sloppier than normal
\tolerance 1414
\hbadness 1414
\emergencystretch 1.5em
\hfuzz 0.3pt
\clubpenalty = 10000
\widowpenalty = 10000
\displaywidowpenalty = 10000
\hfuzz \vfuzz
 \raggedbottom%

\title{Theravāda Collection on Monastic Law}
\author{Bhikkhu Brahmali}
\date{}%
% define a different fleuron for each edition
\newfontfamily\Fleuronfont[Ornament=9]{Arno Pro}

% Define heading styles per edition for chapter and section. Suttatitle can be either of these, depending on the volume. 

\let\oldfrontmatter\frontmatter
\renewcommand{\frontmatter}{%
\chapterfont{\setstretch{.85}\normalfont\centering}%
\sectionfont{\setstretch{.85}\normalfont\BalancedRagged}%
\oldfrontmatter}

\let\oldmainmatter\mainmatter
\renewcommand{\mainmatter}{%
\chapterfont{\thispagestyle{empty}\setstretch{.85}\normalfont\centering}%
\sectionfont{\clearpage\thispagestyle{plain}\setstretch{.85}\normalfont\centering}%
\oldmainmatter}

\let\oldbackmatter\backmatter
\renewcommand{\backmatter}{%
\chapterfont{\setstretch{.85}\normalfont\centering}%
\sectionfont{\setstretch{.85}\normalfont\BalancedRagged}%
\pagestyle{plainer}%
\oldbackmatter}
%
%
\begin{document}%
\normalsize%
\frontmatter%
\setlength{\parindent}{0cm}

\pagestyle{empty}

\maketitle

\blankpage%
\begin{center}

\vspace*{2.2em}

\halftitlepageTranslationTitle{Theravāda Collection on Monastic Law}

\vspace*{1em}

\halftitlepageTranslationSubtitle{A translation of the Pali Vinaya Piṭaka into English}

\vspace*{2em}

\halftitlepageFleuron{•}

\vspace*{2em}

\halftitlepageByline{translated and introduced by}

\vspace*{.5em}

\halftitlepageCreatorName{Bhikkhu Brahmali}

\vspace*{4em}

\halftitlepageVolumeNumber{Volume 3}

\smallskip

\halftitlepageVolumeAcronym{Bi Vb}

\smallskip

\halftitlepageVolumeTranslationTitle{Analysis of Rules for Nuns}

\smallskip

\halftitlepageVolumeRootTitle{Bhikkhunī Vibhaṅga}

\vspace*{\fill}

\sclogo{0}
 \halftitlepagePublisher{SuttaCentral}

\end{center}

\newpage
%
\setstretch{1.05}

\begin{footnotesize}

\textit{Theravāda Collection on Monastic Law} is a translation of the Theravāda Vinayapiṭaka by Bhikkhu Brahmali.

\medskip

Creative Commons Zero (CC0)

To the extent possible under law, Bhikkhu Brahmali has waived all copyright and related or neighboring rights to \textit{Theravāda Collection on Monastic Law}.

\medskip

This work is published from Australia.

\begin{center}
\textit{This translation is an expression of an ancient spiritual text that has been passed down by the Buddhist tradition for the benefit of all sentient beings. It is dedicated to the public domain via Creative Commons Zero (CC0). You are encouraged to copy, reproduce, adapt, alter, or otherwise make use of this translation. The translator respectfully requests that any use be in accordance with the values and principles of the Buddhist community.}
\end{center}

\medskip

\begin{description}
    \item[Web publication date] 2021
    \item[This edition] 2025-01-13 01:01:43
    \item[Publication type] hardcover
    \item[Edition] ed3
    \item[Number of volumes] 6
    \item[Publication ISBN] 978-1-76132-006-4
    \item[Volume ISBN] 978-1-76132-009-5
    \item[Publication URL] \href{https://suttacentral.net/editions/pli-tv-vi/en/brahmali}{https://suttacentral.net/editions/pli-tv-vi/en/brahmali}
    \item[Source URL] \href{https://github.com/suttacentral/bilara-data/tree/published/translation/en/brahmali/vinaya}{https://github.com/suttacentral/bilara-data/tree/published/translation/en/brahmali/vinaya}
    \item[Publication number] scpub8
\end{description}

\medskip

Map of Jambudīpa is by Jonas David Mitja Lang, and is released by him under Creative Commons Zero (CC0).

\medskip

Published by SuttaCentral

\medskip

\textit{SuttaCentral,\\
c/o Alwis \& Alwis Pty Ltd\\
Kaurna Country,\\
Suite 12,\\
198 Greenhill Road,\\
Eastwood,\\
SA 5063,\\
Australia}

\end{footnotesize}

\newpage

\setlength{\parindent}{1em}%%
\tableofcontents
\newpage
\pagestyle{fancy}
%
\chapter*{The \textsanskrit{Bhikkhunī}-\textsanskrit{vibhaṅga}: the Nuns’ \textsanskrit{Pātimokkha} rules and their analysis}
\addcontentsline{toc}{chapter}{The \textsanskrit{Bhikkhunī}-\textsanskrit{vibhaṅga}: the Nuns’ \textsanskrit{Pātimokkha} rules and their analysis}
\markboth{The \textsanskrit{Bhikkhunī}-\textsanskrit{vibhaṅga}: the Nuns’ \textsanskrit{Pātimokkha} rules and their analysis}{The \textsanskrit{Bhikkhunī}-\textsanskrit{vibhaṅga}: the Nuns’ \textsanskrit{Pātimokkha} rules and their analysis}

\scbyline{Bhikkhu Brahmali, 2024}

The present volume is the third of six, the total of which constitutes a complete translation of the Vinaya \textsanskrit{Piṭaka}, the Monastic Law. This volume consists of the \textsanskrit{Bhikkhunī}-\textsanskrit{vibhaṅga}, the Nuns’ \textsanskrit{Pātimokkha} rules and their analysis, which I will call the Nuns’ Analysis for short. In the present introduction, I will survey the contents of volume 3, and also make observations of points of particular interest. For a general introduction to the Monastic Law, see volume 1.

The existence of the Nuns’ Analysis is a result of the \textit{\textsanskrit{bhikkhunīs}} having a separate \textsanskrit{Pātimokkha}, that is, they have different rules from the monks. Because they have different rules, they have to recite the \textsanskrit{Pātimokkha} separately, which in turn means that the legal procedures of the two Sanghas were also done apart. And so, given that the Buddha laid different rules for the nuns, he must have envisaged an autonomous nuns’ Sangha from the beginning. This is the basis for the nuns being largely independent of the monks. I will return to this important point just below.

The structure of the Nuns’ Analysis is the same as for the Monks’ Analysis, the Bhikkhu-\textsanskrit{vibhaṅga}, for which see the introduction to volume 1. The classes of rules are the same, with the exception of the \textit{aniyatas}, the “indeterminate offenses”, which do not occur in the \textsanskrit{Bhikkhunī}-\textsanskrit{vibhaṅga}. The sequence of the classes is also the same. Altogether the \textsanskrit{Bhikkhunī}-\textsanskrit{pātimokkha} consists of 311 rules around which the Nuns’ Analysis is structured. This means that the \textit{\textsanskrit{bhikkhunīs}} have 84 more \textsanskrit{Pātimokkha} rules than the \textit{bhikkhus}. The \textit{\textsanskrit{bhikkhunī}} rules are distributed as follows over the various classes of offenses:

\begin{enumerate}%
\item 8 offenses entailing expulsion, \textit{\textsanskrit{pārājikas}} (Pj)%
\item 17 offenses entailing suspension, \textit{\textsanskrit{saṅghādisesas}} (Ss)%
\item 30 offenses entailing relinquishment and confession, \textit{nissaggiya \textsanskrit{pācittiyas}} (NP)%
\item 166 offenses entailing confession, \textit{\textsanskrit{pācittiyas}} (Pc)%
\item 8 offenses entailing acknowledgment, \textit{\textsanskrit{pāṭidesanīyas}} (Pd)%
\item 75 rules of training, \textit{sekhiyas} (Sk)%
\item 7 principles for settling legal issues, \textit{\textsanskrit{adhikaraṇasamathadhammas}}, or just \textit{\textsanskrit{adhikaraṇasamathas}} (As).%
\end{enumerate}

These rules can be divided into those that are in common with the monks and those that are unique to the nuns. As to the rules held in common, the nuns are to practice them in the same way as the monks (\href{https://suttacentral.net/pli-tv-kd20/en/brahmali\#4.1.3}{Kd~20:4.1.3}). This means that there is a separate \textit{\textsanskrit{vibhaṅga}}, “analysis”, only for the rules that are unique to the nuns. The distribution of shared and unique rules is as follows:

\begin{description}%
\item[8 \textit{\textsanskrit{pārājikas}} (Pj)] 4 shared with monks%
\item[17 \textit{\textsanskrit{saṅghādisesas}} (Ss)] 4 unique to nuns%
\item[30 \textit{nissaggiya \textsanskrit{pācittiyas}} (NP)] 7 shared%
\item[166 \textit{\textsanskrit{pācittiyas}} (Pc)] 10 unique%
\item[8 \textit{\textsanskrit{pāṭidesanīyas}} (Pd)] 18 shared%
\item[75 \textit{sekhiyas} (Sk)] 12 unique%
\item[7 \textit{\textsanskrit{adhikaraṇasamathas}} (As)] 70 shared%
\end{description}

We see here that the \textit{sekhiyas} and the \textit{\textsanskrit{adhikaraṇasamathas}} are exactly the same for the monks and the nuns. That the \textit{sekhiyas} are the same may be because the social etiquette of monks and nuns was determined by their status as monastics rather than the gender differences as found in society at large.

More interesting are the \textit{\textsanskrit{adhikaraṇasamathas}}. As I discussed briefly in the introduction to volume 1, these are the overarching rules that govern the execution of Sangha business. That they are the same for monks and nuns suggests that the overall framework for the governance of the two Sanghas is the same. Given, as we have seen above, that the two Sanghas operate independently of each other, this means that the nuns have internal autonomy in their decision making, with no possibility for the monks to interfere. Thus, the \textsanskrit{Bhikkhunī}-sangha may well be the world’s first attested large-scale institution governed by women. There are a few exceptions to this autonomy, which I will return to in the introduction to volume 5.

\section*{Differences between the Nuns’ Analysis and the Monks’ Analysis}

Beyond the similarities sketched out above, there are some important differences between the two sets of rules. The first, as noted above, is the absence in the nuns’ rules of the class called \textit{aniyata}, of which the monks have two rules. As I have argued in the introduction to volume 1, these rules were laid down to give lay women a voice when monks acted in sexually inappropriate ways. Because of the imbalance in power between the genders in ancient India, there was presumably no need to have equivalent rules for the nuns.

The second obvious difference is in the number of rules, with the nuns having 84 more rules than the monks. One might expect this to be the result of discrimination against women in a patriarchal culture. Although this is likely to be part of the explanation, the reality is more complex.

The main reason why the nuns have more rules is simply that they inherited most of their rules from the monks, in total 181 out of 311, whereas there is no evidence that the monks inherited anything from the nuns. There is no discussion in the Canonical texts of how or by whom these rules were selected, but it seems reasonable to think that it was done to create a foundation for the nuns’ training. The chosen rules were presumably the ones considered most relevant.

A second reason is that the nuns’ rules were added to for a considerable period after the monks’ rules were fixed. This can be seen by comparing the rules across the various schools for which we still have a nuns’ \textsanskrit{Pātimokkha}.\footnote{Kabilsingh, 1998. } Whereas the monks’ rules, with the exception of the \textit{sekhiyas}, are very similar across the different schools,\footnote{Pachow, 2000. } the nuns’ rules vary significantly, especially among the \textit{\textsanskrit{pācittiyas}}.\footnote{Kabilsingh, 1984. } It may be that the monks’ position as teachers and instructors of the nuns disposed them to lay down new rules into the sectarian period, decades and even centuries after the Buddha had passed away.

There are further reasons why the nuns have more rules. To enable a proper discussion of this, however, I will first have to discuss the rules. I will then return to this question toward the end of this introduction. Before turning to the rules, there are a few more general issues worthy of attention.

Although the \textsanskrit{Bhikkhunī}-\textsanskrit{vibhaṅga} has the same basic structure as the Monks’ Analysis, it is much shorter and simpler. For instance, the four \textit{\textsanskrit{pārājikas}} that are common to the two Sanghas cover a total of 109 pages in the PTS edition, whereas the four \textit{\textsanskrit{pārājikas}} unique to the nuns cover just 11 pages. Although the difference is especially pronounced for the \textit{\textsanskrit{pārājika}} rules, it is symptomatic of the \textsanskrit{Vibhaṅga} as a whole.

The difference in the extent of the two \textit{\textsanskrit{vibhaṅgas}} is most obvious in two areas: the permutation series and the case studies. Whereas the monks’ rules often have long permutation series attached to them, this is only the case for the nuns’ fifth \textit{\textsanskrit{pārājika}} rule. As to the case studies, there is none in the Nuns’ Analysis. It is interesting, however, that as part of the case studies pertaining to the monks’ rules, there are a few cases that actually concern nuns, including two in \textit{\textsanskrit{pārājika}} 1, three in \textit{\textsanskrit{pārājika}} 2, and one in \textit{\textsanskrit{pārājika}} 3.\footnote{Respectively at \href{https://suttacentral.net/pli-tv-bu-vb-pj1/en/brahmali\#10.5.1}{Bu~Pj~1:10.5.1}, \href{https://suttacentral.net/pli-tv-bu-vb-pj1/en/brahmali\#10.6.6}{Bu~Pj~1:10.6.6}, \href{https://suttacentral.net/pli-tv-bu-vb-pj2/en/brahmali\#7.6.20}{Bu~Pj~2:7.6.20}, \href{https://suttacentral.net/pli-tv-bu-vb-pj2/en/brahmali\#7.45.1}{Bu~Pj~2:7.45.1}, \href{https://suttacentral.net/pli-tv-bu-vb-pj2/en/brahmali\#7.45.12}{Bu~Pj~2:7.45.12}, and \href{https://suttacentral.net/pli-tv-bu-vb-pj3/en/brahmali\#5.33.10}{Bu~Pj~3:5.33.10}. } These stories may originally have been part of a larger \textsanskrit{Bhikkhunī}-\textsanskrit{vibhaṅga} or perhaps it was considered expedient from the start to consolidate all the case studies in one place. Lastly, apart from the permutation series and the case studies, the 130 rules that are unique to the nuns are rarely amended, thus reducing the overall length of the origin stories. The nuns only have four amended unique rules as compared to a total of 39 for the monks.\footnote{This is not counting the eight \textit{\textsanskrit{pāṭidesanīyas}} for \textit{\textsanskrit{bhikkhunīs}}. Although these are normally included in the count of 130 unique rules for the nuns, for all practical purposes they are shared with Bu Pc 39. The four amended rules are \href{https://suttacentral.net/pli-tv-bi-vb-ss3/en/brahmali\#1.16.1}{Bi~Ss~3}, \href{https://suttacentral.net/pli-tv-bi-vb-pc51/en/brahmali\#1.12.1}{Bi~Pc~51}, \href{https://suttacentral.net/pli-tv-bi-vb-pc84/en/brahmali\#1.1.14.1}{Bi~Pc~84}, and \href{https://suttacentral.net/pli-tv-bi-vb-pc85/en/brahmali\#1.1.14.1}{Bi~Pc~85}. For the monks the equivalent is 39 out of 227. I include all the monks’ rules, not just the unique ones, because they all originated with the monks. In effect, we are comparing the \textit{\textsanskrit{vibhaṅga}} of the rules that originated with the nuns to the \textit{\textsanskrit{vibhaṅga}} of the rules that originated with the monks. }

There are probably a number of reasons for these differences between the two \textsanskrit{Vibhaṅgas}. An obvious one is that the nuns’ rules were generally laid down later than those of the monks. They would have had less time to evolve, especially during the Buddha’s lifetime. A second reason is that we have grounds to believe that the monks were the chief maintainers of the Vinaya. Both the first and the second Councils, \textit{\textsanskrit{saṅgītis}}, were only attended by monks. Moreover, the nuns are repeatedly depicted, especially in the Bhikkhuni-kkhandhaka, as depending on the monks for their understanding of the Vinaya.\footnote{See \href{https://suttacentral.net/pli-tv-kd20/en/brahmali\#6.1.1}{Kd~20:6.1.1}–8.1.13. } Third, the monks are likely to have had the most experts on the Monastic Law. They had better access to the Buddha, they started earlier, and there were more of them. The final reason is the patriarchal culture that would have regarded it as more important to analyze the monks’ rules. This would have been especially so after the Buddha passed away, which is when most of the \textit{\textsanskrit{vibhaṅga}} material was laid down.

\section*{The personalities of the \textsanskrit{Bhikkhunī}-\textsanskrit{vibhaṅga}}

In the introduction to the Bhikkhu-\textsanskrit{vibhaṅga} in volume 1, we looked at some of the characters that feature in the origin stories. As it happens, the nuns too have their fair share of notorious personalities. To start with, we find a group of six nuns, who seem to be modeled on the group of six monks.\footnote{Alternatively, they can be understood as the nuns belonging to the group of six, that is, they were affiliated with the group of six monks. } Sometimes the group of six nuns commit exactly the same misconduct as the group of six monks. They accumulate bowls and refuse to return bowls relinquished by others (\href{https://suttacentral.net/pli-tv-bi-vb-np1/en/brahmali\#1.2}{Bi~NP~1} and \href{https://suttacentral.net/pli-tv-bi-vb-np1/en/brahmali\#3.1}{Bi~NP~1:3.1}), which is exactly what the group of six monks do (\href{https://suttacentral.net/pli-tv-bu-vb-np21/en/brahmali\#1.2}{Bu~NP~21}). Then we have the fact that the group of six nuns use oversize bathing robes (\href{https://suttacentral.net/pli-tv-bi-vb-pc22/en/brahmali\#1.4}{Bi~Pc~22}), as do the group six monks (\href{https://suttacentral.net/pli-tv-bu-vb-pc91/en/brahmali\#1.4}{Bu~Pc~91}), and that the group of six nuns ask for fine foods (\href{https://suttacentral.net/pli-tv-bi-vb-pd1/en/brahmali\#1.1.1}{Bi~Pd~1–8}), just like their monastic brothers (\href{https://suttacentral.net/pli-tv-bu-vb-pc39/en/brahmali\#1.1}{Bu~Pc~39}). There are a number of further similarities that I will leave out in the name of brevity. Yet most telling of all is that the group of six nuns are said to be the originators of all, or virtually all, of the \textit{sekhiya} offenses for nuns,\footnote{The \textit{sekhiya} section for nuns is contracted, and so it is not obvious whether all the origin stories feature the group of six nuns, or whether they just feature in the same rules as do the group of six monks, that is, 72 out of 75. } just as the group of six monks are the originators of the same offenses for monks. These origin stories make little sense because the \textit{sekhiyas} are clearly a common set of rules that do not require separate origin stories for the nuns. So it seems as if the group of six nuns, as in the case of the group of six monks, often function as a convenient group upon which offenses were projected, especially in cases where the original perpetrators had been forgotten.

The group of six does not exhaust the rogues’ gallery of notorious nuns. The two most infamous nuns in the Vinaya \textsanskrit{Piṭaka} are \textsanskrit{Thullanandā} and, to a lesser extent, her unruly disciple \textsanskrit{Caṇḍakāḷī}. \textsanskrit{Thullanandā} is the original offender for a large number of the rules that are unique to the nuns: two out of four \textit{\textsanskrit{pārājikas}}, four of ten \textit{\textsanskrit{saṅghādisesas}}, seven of twelve \textit{nissaggiya \textsanskrit{pācittiyas}}, and lastly twenty-four of ninety-six \textit{\textsanskrit{pācittiyas}}. \textsanskrit{Thullanandā}’s student \textsanskrit{Caṇḍakāḷī} is the originator of a further two \textit{\textsanskrit{saṅghādisesas}}, as well as four \textit{\textsanskrit{pācittiyas}}.

Where \textsanskrit{Thullanandā} is not the originator of a rule, she is often involved in other ways. It seems \textsanskrit{Thullanandā} was one of four sisters, all of whom became nuns (\href{https://suttacentral.net/pli-tv-bi-vb-pj5/en/brahmali\#1.6}{Bi~Pj~5:1.6}). One of her sisters, \textsanskrit{Sundarīnandā}, was the originator of Bi Pj 5, which is concerned with lustful bodily contact. The origin story to the following rule, \href{https://suttacentral.net/pli-tv-bi-vb-pj6/en/brahmali\#1.2}{Bi~Pj~6}, carries on from the previous one, with \textsanskrit{Thullanandā} hiding the \textit{\textsanskrit{pārājika}} offense of the same sister, who has now become pregnant by her lover. In this way \textsanskrit{Thullanandā} herself becomes the originator of a \textit{\textsanskrit{pārājika}} offense. At \href{https://suttacentral.net/pli-tv-bi-vb-ss12/en/brahmali\#1.2}{Bi~Ss~12} it is \textsanskrit{Thullanandā}’s students who behave badly and then hide each other’s offenses. In the following rule, \href{https://suttacentral.net/pli-tv-bi-vb-ss13/en/brahmali\#1.2}{Bi~Ss~13}, \textsanskrit{Thullanandā} disregards the Sangha’s admonishment and encourages her students to carry on as before. And so, two \textit{\textsanskrit{saṅghādisesa}} offenses come into being.

\textsanskrit{Thullanandā} was well disposed toward some of the most notorious monks of the time. She praises Devadatta and his followers, comparing them favorably to some of the Buddha’s greatest disciples, including \textsanskrit{Sāriputta} and \textsanskrit{Mahāmoggallāna} (\href{https://suttacentral.net/pli-tv-bu-vb-pc29/en/brahmali\#1.9}{Bu~Pc~29:1.9}). At another time, she dismisses a number of senior monks so that she can ordain a trainee nun with the help of Devadatta and his friends (\href{https://suttacentral.net/pli-tv-bi-vb-pc81/en/brahmali\#1.1}{Bi~Pc~81:1.1}). She is also connected to other disreputable monks, such as \textsanskrit{Ariṭṭha} who was ejected from the Bhikkhu-sangha for his wrong views (\href{https://suttacentral.net/pli-tv-kd11/en/brahmali\#32.3.25}{Kd~11:32.3.25}). When she takes his side, she becomes the offender giving rise to \href{https://suttacentral.net/pli-tv-bi-vb-pj7/en/brahmali\#1.1}{Bi~Pj~7}. With such bad friends, it is hardly surprising \textsanskrit{Thullanandā} would be a difficult character herself, in turn also affecting her own students, particularly \textsanskrit{Caṇḍakāḷī}.

Indeed, \textsanskrit{Thullanandā} comes across as troublesome and difficult, rarely accepting the authority of the Sanghas. As we have seen, \textsanskrit{Thullanandā} disregards the Sangha’s admonishment (\href{https://suttacentral.net/pli-tv-bi-vb-ss13/en/brahmali\#1.2}{Bi~Ss~13:1.2}). Elsewhere, she readmits \textsanskrit{Caṇḍakāḷī} without consultation after the latter had been ejected by the Sangha, thus giving rise to a \textit{\textsanskrit{saṅghādisesa}} offense (\href{https://suttacentral.net/pli-tv-bi-vb-ss4/en/brahmali\#1.16}{Bi~Ss~4:1.16}). She also reviles the community of nuns out of anger (\href{https://suttacentral.net/pli-tv-bi-vb-pc53/en/brahmali\#1.18}{Bi~Pc~53}).

In a significant number of \textit{\textsanskrit{pācittiya}} rules \textsanskrit{Thullanandā} is portrayed as selfish and inconsiderate toward other nuns, including her own students (\href{https://suttacentral.net/pli-tv-bi-vb-pc33/en/brahmali\#1.1}{Bi~Pc~33}–\href{https://suttacentral.net/pli-tv-bi-vb-pc35/en/brahmali\#1.1}{35}, \href{https://suttacentral.net/pli-tv-bi-vb-pc45/en/brahmali\#1.1}{45}, \href{https://suttacentral.net/pli-tv-bi-vb-pc47/en/brahmali\#1.1}{47}–\href{https://suttacentral.net/pli-tv-bi-vb-pc48/en/brahmali\#1.1}{48}, \href{https://suttacentral.net/pli-tv-bi-vb-pc68/en/brahmali\#1.1}{68}, \href{https://suttacentral.net/pli-tv-bi-vb-pc70/en/brahmali\#1.1}{70}, and \href{https://suttacentral.net/pli-tv-bi-vb-pc77/en/brahmali\#1.1}{77}–\href{https://suttacentral.net/pli-tv-bi-vb-pc78/en/brahmali\#1.1}{78}). She was immoderate and greedy, often trying to get extra cloth at the expense of other nuns (\href{https://suttacentral.net/pli-tv-bi-vb-np2/en/brahmali\#1.1}{Bi~NP~2}, \href{https://suttacentral.net/pli-tv-bi-vb-pc26/en/brahmali\#1.1}{Bi~Pc~26}–\href{https://suttacentral.net/pli-tv-bi-vb-pc27/en/brahmali\#1.1}{27}, and \href{https://suttacentral.net/pli-tv-bi-vb-pc29/en/brahmali\#1.1}{29}–\href{https://suttacentral.net/pli-tv-bi-vb-pc30/en/brahmali\#1.1}{30}). She would often engage in trading, e.g. at \href{https://suttacentral.net/pli-tv-bi-vb-np3/en/brahmali\#1.1}{Bi~NP~3}–\href{https://suttacentral.net/pli-tv-bi-vb-np5/en/brahmali\#1.1}{5} and \href{https://suttacentral.net/pli-tv-bi-vb-np10/en/brahmali\#1.1}{10}–\href{https://suttacentral.net/pli-tv-bi-vb-np12/en/brahmali\#1.1}{12}, and she curried favors with householders (\href{https://suttacentral.net/pli-tv-bi-vb-pc28/en/brahmali\#1.1}{Bi~Pc~28} and \href{https://suttacentral.net/pli-tv-bi-vb-pc46/en/brahmali\#1.1}{46}). We even have a \textsanskrit{Jātaka} tale where we witness \textsanskrit{Thullanandā}’s greed also in a past life, in this case highlighting the negative consequences of excessive desires (\href{https://suttacentral.net/pli-tv-bi-vb-pc1/en/brahmali\#1.25.1}{Bi~Pc~1}).

Yet, as has been pointed out,\footnote{Ohnuma, 2013. } \textsanskrit{Thullanandā} is a more complex character than she might appear at first sight. According to the origin stories to several rules, especially \href{https://suttacentral.net/pli-tv-bi-vb-np10/en/brahmali\#1.2}{Bi~NP~10}, \href{https://suttacentral.net/pli-tv-bi-vb-np11/en/brahmali\#1.2}{11}, and \href{https://suttacentral.net/pli-tv-bi-vb-np12/en/brahmali\#1.2}{12}, she is a learned, confident, and gifted teacher of the Dhamma, with many people visiting her and making offerings. At Bi NP 11 and 12, she inspires King Pasenadi with a Dhamma talk to the point where he makes her the generous offer to ask for anything she wants. The background story to \href{https://suttacentral.net/pli-tv-bi-vb-ss4/en/brahmali\#1.1}{Bi~Ss~4} suggests she was an expert on the Monastic Law. According to \href{https://suttacentral.net/pli-tv-bu-vb-pc29/en/brahmali\#1.2}{Bu~Pc~29}, she was invited to receive regular meals from householders, showing that she was well respected. The origin story to \href{https://suttacentral.net/pli-tv-bi-vb-ss1/en/brahmali\#1.25}{Bi~Ss~1} shows her standing up for the rights of \textit{\textsanskrit{bhikkhunīs}}, hinting, perhaps, that she was an early feminist. The origin story to \href{https://suttacentral.net/pli-tv-bi-vb-ss2/en/brahmali\#1.1}{Bi~Ss~2} might be read in a similar light.\footnote{See Ohnuma, pp. 53–56. }

We are left with the portrait of a complex character. This, of course, is exactly what we would expect of a real person, for human beings are rarely one-dimensional. In fact, the number and nature of the details we have from \textsanskrit{Thullanandā}’s life suggest she is a historical figure. In addition to what we have seen already, these details include her involvement in plots that carry on over several rules and her association with monks who are well attested in the Pali tradition, such as Ānanda and \textsanskrit{Mahākassapa}. Significantly, she is encountered with similar frequency also in the other schools of Buddhism.\footnote{For instance, in the \textsanskrit{Mahāsaṅghika} Vinaya, where she is the originator of a large number of rules, including, Bi Ss 4, 9, 15, and 18; and Bi NP 11, 12, 13, 16, 18, and 29. }

Given these details of \textsanskrit{Thullanandā}’s character, it is perhaps not surprising that she is involved, directly and indirectly, in so many offenses. This is probably what we should expect from a strong, fiery, and sometimes problematic personality. Nevertheless, it seems likely that she was at times a convenient scapegoat, especially in the many cases where the true origin story had been forgotten. In the \textit{\textsanskrit{bhikkhunī} nissaggiya \textsanskrit{pācittiyas}}, for instance, a total of nine very similar rules are attributed to \textsanskrit{Thullanandā}’s misbehavior. It is hard not to suspect that her name was sometimes copied and pasted from one rule to the next.

\section*{The \textit{\textsanskrit{pārājikas}} (Pj)}

The nuns have eight \textit{\textsanskrit{pārājika}} rules, consisting of the same four as the monks and an additional four that are unique to the \textit{\textsanskrit{bhikkhunīs}}. I have discussed the general aspects of the \textit{\textsanskrit{pārājikas}} in the introduction to volume 1. Here I will focus on the aspects that are peculiar to the nuns.

Of the four \textit{\textsanskrit{pārājikas}} that the nuns have in common with the monks, the first one is in fact slightly different.\footnote{In other schools of Buddhism, such as the \textsanskrit{Mahāsaṅghikas}, the rules are identical. See the translations by \textsanskrit{Bhikkhunī} \textsanskrit{Vimalañāṇī} at https://vimalanyani.github.io/vinaya-lzh/mg/pm/. } First, the nuns’ version adds the word \textit{chandaso}, “willingly”, and second, it omits the phrase \textit{\textsanskrit{sikkhaṁ} \textsanskrit{appaccakkhāya} \textsanskrit{dubbalyaṁ} \textsanskrit{anāvikatvā}}, “without first renouncing the training and revealing his/her weakness”. Both of these differences are worthy of a brief discussion.

The purpose of adding the word “willingly” is presumably to acknowledge the problem of rape in ancient India and to ensure that the nuns were not penalized for being the victims of violence. There is no origin story, however, that explains the circumstances in which this addition was made, as is normally the case for the \textsanskrit{Pātimokkha} rules. It could be that this addition to the rule has its origin in the story of \textsanskrit{Uppalavaṇṇā}, found among the case stories to \textit{bhikkhu \textsanskrit{pārājika}} 1 (\href{https://suttacentral.net/pli-tv-bu-vb-pj1/en/brahmali\#10.5.1}{Bu~Pj~1:10.5.1}). This might in fact be the reason for the presence of this story in the Bhikkhu-\textsanskrit{vibhaṅga}. At some point the connection between this origin story and the reformulation of the rule was lost, probably because the rules that the nuns have in common with the monks were only preserved in the monks’ \textsanskrit{Pātimokkha}.\footnote{In the Pali tradition, the full nuns’ \textsanskrit{Pātimokkha} is now only found in the \textsanskrit{Dvemātikāpāḷi}, a sub-commentary. } What is indisputable, however, is that, according to the \textsanskrit{Vibhaṅga}, a monk too would not commit a \textit{\textsanskrit{pārājika}} if he were the victim of rape.\footnote{The non-offense clause of \textit{bhikkhu \textsanskrit{pārājika}} 1, at \href{https://suttacentral.net/pli-tv-bu-vb-pj1/en/brahmali\#9.7.25}{Bu~Pj~1:9.7.25}, shows that a monk can only commit a \textit{\textsanskrit{pārājika}} if he consents. } This makes it likely that the addition to the nuns’ rule must have happened before this portion of the \textsanskrit{Vibhaṅga} existed. Once again, we see that the \textsanskrit{Vibhaṅga} material is likely to be later than the majority of \textsanskrit{Pātimokkha} rules.

The reason for the absence of the phrase “without first renouncing the training and revealing her weakness” can be explained in two different ways. The straightforward explanation comes from a passage according to which the nuns, in contrast to the monks, are not allowed to disrobe by verbally rejecting the training (\href{https://suttacentral.net/pli-tv-kd20/en/brahmali\#26.1.4}{Kd~20:26.1.4}).\footnote{The referenced passage from the Bhikkhuni-kkhandhaka says that a nun disrobes by removing her robes and putting on lay clothes. } There is, however, an alternative and more intriguing possibility. Once the word \textit{chandaso} had been added to the nuns’ rule, it could no longer be considered as fully in common with the monks. A consequence of this might be that it became unnatural to update the nuns’ rule as a result of changes to the monks’ rule. It follows from this rather speculative premise that if “willingly” was added to the nuns’ rule before “without first renouncing the training and revealing his weakness” was added to the monks’ version, then the latter may never have made its way into the nuns’ version. My suggestion, then, is that the prohibition at Kd 20 against a nun verbally renouncing the training may have its origin in this phrase missing from the nuns’ version of \textit{\textsanskrit{pārājika}} 1, which in turn might be the result of an accident of history. If the Khandhakas are generally later than the \textsanskrit{Pātimokkha} rules, as seems to be the case, this would at least be a possible unfolding of events.

Coming to the \textit{\textsanskrit{pārājika}} rules that are unique to the nuns, \textit{\textsanskrit{pārājika}} 5 concerns lustful physical contact. The question arises why this is a more serious offense for the nuns than for the monks, for whom this is a \textit{\textsanskrit{saṅghādisesa}} offense at \href{https://suttacentral.net/pli-tv-bu-vb-ss2/en/brahmali\#1.2.15.1}{Bu~Ss~2}. In fact, such disparity between the monks and the nuns in the consequences of performing the same action is not unique to this case. According to \href{https://suttacentral.net/pli-tv-bi-vb-pj6/en/brahmali\#1.23.1}{Bi~Pj~6}, concealing another nun’s \textit{\textsanskrit{pārājika}} offense is itself a \textit{\textsanskrit{pārājika}}, yet for the monks this is no more than a \textit{\textsanskrit{pācittiya}} offense at \href{https://suttacentral.net/pli-tv-bu-vb-pc64/en/brahmali\#1.23.1}{Bu~Pc~64}. According to \href{https://suttacentral.net/pli-tv-bi-vb-pj7/en/brahmali\#1.11.1}{Bi~Pj~7}, a nun who sides with an ejected monk again commits a \textit{\textsanskrit{pārājika}} offense, whereas for a monk the even graver action of siding with a schismatic results in a \textit{\textsanskrit{saṅghādisesa}} at \href{https://suttacentral.net/pli-tv-bu-vb-ss11/en/brahmali\#1.19.1}{Bu~Ss~11}. Then there are a number of \textit{\textsanskrit{pācittiya}} rules for the \textit{\textsanskrit{bhikkhunīs}} that for the monks are minor rules found outside the \textsanskrit{Pātimokkha}.\footnote{Whether this is a true disparity depends on how one regards such non-\textit{\textsanskrit{pātimokkha}} rules. If they are regarded as confessable offenses, they are for all practical purposes the same as the \textit{\textsanskrit{pācittiyas}}. } These include the rule against eating garlic, respectively at \href{https://suttacentral.net/pli-tv-bi-vb-pc1/en/brahmali\#1.41.1}{Bi~Pc~1} and \href{https://suttacentral.net/pli-tv-kd15/en/brahmali\#34.1.15}{Kd~15:34.1.15}, the rule against entertainment at \href{https://suttacentral.net/pli-tv-bi-vb-pc10/en/brahmali\#1.15.1}{Bi~Pc~10} and \href{https://suttacentral.net/pli-tv-kd15/en/brahmali\#2.6.6}{Kd~15:2.6.6}, the rule against using luxurious furniture at \href{https://suttacentral.net/pli-tv-bi-vb-pc42/en/brahmali\#1.14.1}{Bi~Pc~42} and \href{https://suttacentral.net/pli-tv-kd5/en/brahmali\#10.5.2}{Kd~5:10.5.2}, and more.

Yet the disparity also goes the other way, with the monks sometimes being penalized more heavily for the same action. An obvious example is masturbation, which is a \textit{\textsanskrit{saṅghādisesa}} for monks at \href{https://suttacentral.net/pli-tv-bu-vb-ss1/en/brahmali\#2.1.13.1}{Bu~Ss~1}, whereas for the nuns it is a \textit{\textsanskrit{pācittiya}} at \href{https://suttacentral.net/pli-tv-bi-vb-pc3/en/brahmali\#1.14.1}{Bi~Pc~3} and \href{https://suttacentral.net/pli-tv-bi-vb-pc4/en/brahmali\#1.21.1}{4}. Other important examples are several \textit{\textsanskrit{saṅghādisesa}} offenses for the monks, in particular the offenses for indecent speech and for building dwellings that are too large—at \href{https://suttacentral.net/pli-tv-bu-vb-ss3/en/brahmali\#1.2.14.1}{Bu~Ss~3}, \href{https://suttacentral.net/pli-tv-bu-vb-ss4/en/brahmali\#1.2.19.1}{4}, \href{https://suttacentral.net/pli-tv-bu-vb-ss6/en/brahmali\#1.6.6.1}{6}, and \href{https://suttacentral.net/pli-tv-bu-vb-ss7/en/brahmali\#1.19.1}{7}—which are not offenses at all for the nuns.\footnote{It might be objected that the nuns too have a large number of rules that the monks do not have. We will return to this question at the end of this introduction. } There is also the interesting case of homosexual sex being a \textit{\textsanskrit{pārājika}} for monks, but generally no more than a \textit{\textsanskrit{pācittiya}} for nuns. There are a number of other examples of lesser importance.

Why this disparity? A traditional explanation might be that the Buddha understood the psychological differences between men and women and so tailored the rules to each gender. Yet there is little evidence for this. What we do know, however, is that the Buddha often lays down rules because of complaints from lay people. Such complaints would have been colored by social expectations, causing gender differences in society to make their way into the monastic rules, at least partially. Indeed, it seems reasonable to assume that the Buddha himself would have taken such societal expectations into account when laying down rules, whether intentionally or through default reaction. This might explain, for instance, why the nuns incur a \textit{\textsanskrit{pārājika}} offense for lustful physical contact, whereas the monks incur a \textit{\textsanskrit{saṅghādisesa}} offense. Unfair as this may seem from a contemporary point of view, society may well have judged it as coarser and more serious for a female to initiate such contact.

Another reason for the disparity is the difference in historical evolution of the two \textsanskrit{Pātimokkhas}. As we have seen, the monks’ rules were closed to addition earlier than the nuns’ rules. The same inappropriate action may then have led to a new \textit{\textsanskrit{pācittiya}} rule for the nuns, while for the monks it may have led to the laying down of an act of wrong conduct in the Khandhakas. And indeed, we see a number of such cases.\footnote{The equivalents are listed in the long footnote in the last section of this introduction. }

Finally, it may be the case that the Buddha occasionally did lay down rules based on what he perceived as psychological differences between the two genders. Clearly the Buddha had exceptional insight into human psychology, the Dhamma essentially being a manual of the path to psychological well-being. Still, we should not overestimate such a motivation in the absence of evidence. In the area of Monastic Law, the Buddha is generally depicted as a pragmatist who reacted to external demands, and rarely if ever as a visionary who worked from more fundamental principles.

Let’s briefly consider the remaining \textit{\textsanskrit{pārājikas}}. As we have seen, \textit{\textsanskrit{pārājika}} 6 concerns hiding another nun’s \textit{\textsanskrit{pārājika}} offense, whereas \textit{\textsanskrit{pārājika}} 7 is about siding with an ejected monk. \textit{\textsanskrit{Pārājika}} 8 again concerns inappropriate association with the opposite gender. This rule is unique in that it requires the offending nun to do a series of eight separate actions before the offense is committed.\footnote{According to the \textsanskrit{Vibhaṅga}, each of these actions is a \textit{thullaccaya}, “a serious offense”, in its own right. } The likelihood that someone would now commit exactly the same eight actions may seem small. At the same time, if one gets trapped in infatuation, it is surprising how unskillful actions can accumulate, eventually leading to the sort of scenario we see in this rule. This rule also presents us with some interesting interpretative challenges, for which see Appendix II: Technical Discussion of Individual \textsanskrit{Bhikkhunī} Rules.

\section*{The \textit{\textsanskrit{saṅghādisesas}} (Ss)}

The nuns have seventeen \textit{\textsanskrit{saṅghādisesa}} offenses, of which seven are in common with the monks. This means the nuns have ten unique \textit{\textsanskrit{saṅghādisesas}}, while the monks have six. Moreover, the nuns have a total of nine \textit{\textsanskrit{saṅghādisesas}} that are immediate offenses, whereas eight are offenses after the performance of a legal procedure of one motion and three announcements. In other words, the nuns must make greater use of \textit{\textsanskrit{saṅghakamma}} in the lead up to a \textit{\textsanskrit{saṅghādisesa}} offense than must the monks. As with the monks, this \textit{\textsanskrit{saṅghakamma}} functions, in effect, as an extended admonishment, giving the offender extra time to reconsider their actions.

Although this class of offenses is common to the two Sanghas, there are some important differences in the practical application. Where the monks have a six-day trial period, \textit{\textsanskrit{mānatta}}, the nuns have half a month. On the other hand, the nuns have no probation period, \textit{\textsanskrit{parivāsa}}, for hiding their offenses. Most likely this difference is due to the fact that a \textit{\textsanskrit{bhikkhunī}} cannot live alone and thus requires another \textit{\textsanskrit{bhikkhunī}} to stay with her (\href{https://suttacentral.net/pli-tv-kd20/en/brahmali\#25.3.1}{Kd~20:25.3.1}). If the nuns had to undergo a sometimes-lengthy \textit{\textsanskrit{parivāsa}} for hiding their offenses, this would require a designated chaperone for extended periods of time, which would be an unreasonable burden on other nuns. An additional possible explanation for the difference is that some of the monks’ \textit{\textsanskrit{saṅghādisesas}} may have been regarded as more embarrassing and thus more likely to be hidden. It is also the case that the period of penance sometimes made the process of emerging from a \textit{\textsanskrit{saṅghādisesa}} offense especially cumbersome, as can be seen from \href{https://suttacentral.net/pli-tv-kd13/en/brahmali}{Kd~13}. Penance may have been scrapped for the nuns because it sometimes made rehabilitation unnecessarily complicated.

Another point worthy of brief comment is the fact that the half-monthly trial period for nuns is laid down twice, both at the end of the nuns’ \textit{\textsanskrit{saṅghādisesa}} offenses and in the \textit{garudhammas}.\footnote{The \textit{garudhammas} are a set of rules that, according to the tradition, were laid down when the Buddha’s foster mother \textsanskrit{Mahāpajāpati} \textsanskrit{Gotamī} was ordained as the first \textit{\textsanskrit{bhikkhunī}} (\href{https://suttacentral.net/pli-tv-kd20/en/brahmali\#1.4.1}{Kd~20:1.4.1}–1.5.23). The status of these rules, including whether they are binding on \textit{\textsanskrit{bhikkhunīs}}, is controversial. See discussion in see Bhikkhu Sujato, “Bhikkhuni Vinaya Studies”, pp. 46–74. } This leads to the obvious question of which is prior. Without going into a detailed discussion, it seems to fit better at the end of the \textit{\textsanskrit{saṅghādisesas}} where it parallels the equivalent section for the monks. This is one among a number of cases where the \textit{garudhammas} are at odds with other parts of the Vinaya \textsanskrit{Piṭaka}, suggesting that they may not go back to the earliest period.

Each of the nuns’ \textit{\textsanskrit{saṅghādisesa}} offenses includes the qualifier \textit{\textsanskrit{nissāraṇīya}}, “entailing sending away”, not shared with the monks. This relates to the trial period, and is explained in the \textsanskrit{Vibhaṅga} as sending away from the Sangha, which parallels a similar rule for monks at \href{https://suttacentral.net/pli-tv-kd12/en/brahmali\#2.1.6}{Kd~12:2.1.6}. The problem with such sending away is that it seems to clash with \href{https://suttacentral.net/pli-tv-bi-vb-ss3/en/brahmali\#4.14.1}{Bi~Ss~3}, which says that a \textit{\textsanskrit{bhikkhunī}} cannot stay by herself. As we have seen, this is resolved by the Sangha appointing a companion to the nun observing the trial period (\href{https://suttacentral.net/pli-tv-kd20/en/brahmali\#25.3.5}{Kd~20:25.3.5}). Alternatively, given that the Khandhakas are generally later than the \textsanskrit{Pātimokkha} rules, we may wonder whether in the earliest period it was acceptable for nuns to live alone.

Returning to the comparison of the nuns’ rules with those of the monks, the monks’ six \textit{\textsanskrit{saṅghādisesa}} offenses not shared with the nuns are \textit{\textsanskrit{saṅghādisesas}} 1–4 and 6–7. Although these rules are not shared within the same class of rules, two of them have rough equivalents elsewhere in \textsanskrit{Bhikkhunī}-\textsanskrit{pātimokkha}. We have seen that \href{https://suttacentral.net/pli-tv-bu-vb-ss1/en/brahmali\#2.1.13.1}{Bu~Ss~1} finds an approximate equivalent in \href{https://suttacentral.net/pli-tv-bi-vb-pc3/en/brahmali\#1.14.1}{Bi~Pc~3} and \href{https://suttacentral.net/pli-tv-bi-vb-pc4/en/brahmali\#1.21.1}{4}, whereas \href{https://suttacentral.net/pli-tv-bu-vb-ss2/en/brahmali\#1.2.15.1}{Bu~Ss~2} resembles \href{https://suttacentral.net/pli-tv-bi-vb-pj5/en/brahmali\#1.54.1}{Bi~Pj~5}. Of the remaining four rules—\href{https://suttacentral.net/pli-tv-bu-vb-ss3/en/brahmali\#1.2.14.1}{Bu~Ss~3}, \href{https://suttacentral.net/pli-tv-bu-vb-ss4/en/brahmali\#1.2.19.1}{4}, \href{https://suttacentral.net/pli-tv-bu-vb-ss6/en/brahmali\#1.6.6.1}{6}, and \href{https://suttacentral.net/pli-tv-bu-vb-ss7/en/brahmali\#1.19.1}{7}—there is no equivalent for the nuns.

Let’s now consider the \textit{\textsanskrit{saṅghādisesa}} offenses that are unique to the nuns. Some of these rules have understandably caused concern among modern \textit{\textsanskrit{bhikkhunīs}}. These rules were perhaps reasonable in ancient India, but not so much in the modern context. For instance, \href{https://suttacentral.net/pli-tv-bi-vb-ss1/en/brahmali\#1.56.1}{Bi~Ss~1} prohibits a nun from taking legal action, which from a modern perspective hampers nuns in seeking redress for injustices and is therefore nothing short of discriminatory. Fortunately, the non-offense clause allows for legal action in cases where the nuns need protection. A broad understanding of this makes it possible to justify legal action whenever a nun or the nuns’ community has been treated unfairly, effectively restricting the rule to malicious legal action.

This leads us to the important question of how to interpret the rules, especially those for the nuns. Because the Vinaya \textsanskrit{Piṭaka} rules were formulated to fit a society that in many ways is quite different from our own, especially when it comes to the discrimination against women, we need to look for principles of interpretation that make the \textsanskrit{Bhikkhunī}-sangha sustainable. We have seen that the \textsanskrit{Pātimokkha} rules are generally older than the \textsanskrit{Vibhaṅga} material, which means we should give priority to the rules over the explanatory material. At the same time, we have just seen that for Bi Ss 1 the \textsanskrit{Vibhaṅga} is more reasonable than the rule. It seems, therefore, that whenever the \textsanskrit{Vibhaṅga} interprets a rule in a lenient fashion—and, importantly, this happened despite the conservatism of that society—we, that is, the \textit{\textsanskrit{bhikkhunīs}}, have the right to follow suit. What I am suggesting, then, is that it is acceptable with any particular rule to choose whether one wishes to follow the rule or the \textsanskrit{Vibhaṅga}, and that there is no need for a consistent approach across the entire \textsanskrit{Pātimokkha}.

Such a lenient interpretative framework—which still falls within the wording of either the rule or the \textsanskrit{Vibhaṅga}—becomes especially important when we come to rules, such as \href{https://suttacentral.net/pli-tv-bi-vb-ss3/en/brahmali\#4.14.1}{Bi~Ss~3}, that seem particularly restrictive from a modern point of view. This rule, if interpreted strictly, makes it impossible for a \textit{\textsanskrit{bhikkhunī}} to travel or live independently. This may have been necessary to protect \textit{\textsanskrit{bhikkhunīs}} in ancient India, but is incompatible with modern sensibilities. With a degree of good will, however, it is possible to interpret this rule such that it is not unreasonably restrictive, even making it possible for a modern \textit{\textsanskrit{bhikkhunī}} to keep it.

The origin story to \href{https://suttacentral.net/pli-tv-bi-vb-ss2/en/brahmali\#1.1}{Bi~Ss~2} is particularly shocking as to the level of sexism in ancient India. A man whose wife has been serially unfaithful goes to a meeting of his clan to get permission to kill her. They agree. The woman then runs away and finds refuge by ordaining as a \textit{\textsanskrit{bhikkhunī}}, setting the stage for a rule against ordaining criminals. The prohibition against ordaining a criminal who is seeking to escape justice is reasonable, and we do in fact have such rules elsewhere, but applicable to both monks and nuns (\href{https://suttacentral.net/pli-tv-kd1/en/brahmali\#43.1.14}{Kd~1:43.1.14}). What is problematic is the treatment of women by society as essentially the property of men, who are then able to mete out punishment largely as they see fit. This goes to show that it is sensible, even necessary, to interpret the nuns’ rules leniently.

Of the remaining \textit{\textsanskrit{saṅghādisesa}} offenses that are unique to the nuns, three in particular stand out as different from anything that the monks have. \href{https://suttacentral.net/pli-tv-bi-vb-ss10/en/brahmali\#1.19.1}{Bi~Ss~10} prohibits a nun from verbally rejecting the triple gem in a fit of anger. This rule gets its significance from the fact that nuns, as opposed to monks, cannot renounce the training verbally (\href{https://suttacentral.net/pli-tv-kd20/en/brahmali\#26.1.4}{Kd~20:26.1.4}). \href{https://suttacentral.net/pli-tv-bi-vb-ss12/en/brahmali\#1.11.1}{Bi~Ss~12} stops nuns from socializing too much, including bad behavior that results from such socializing, whereas \href{https://suttacentral.net/pli-tv-bi-vb-ss13/en/brahmali\#1.23.1}{Bi~Ss~13} stops a nun from encouraging those who socialize inappropriately to continue their bad behavior. We do not know why only the nuns have these rules, but we can perhaps make some educated guesses. One obvious reason is that the nuns were not allowed to live in the wilderness (\href{https://suttacentral.net/pli-tv-kd20/en/brahmali\#23.1.4}{Kd~20:23.1.4}), which would have compelled them to live close to general society. This in turn would have made inappropriate socializing more likely. It is also possible that gender stereotypes—whether based on real or imaginary differences—would have played a role.

\section*{The \textit{nissaggiya \textsanskrit{pācittiyas}} (NP)}

The \textit{\textsanskrit{bhikkhunīs}} have the same number of \textit{nissaggiya \textsanskrit{pācittiya}} offenses as the monks, that is, a total of thirty, eighteen being in common and twelve unique to the nuns, which means there are also twelve \textit{nissaggiyas} unique to the monks.

As with \textit{\textsanskrit{pārājika}} 1, there is in fact one shared offense that is not quite shared, namely \href{https://suttacentral.net/pli-tv-bu-vb-np2/en/brahmali\#2.39.1}{Bu~NP~2}/Bi NP 14. This rule concerns staying apart from one’s robes for a period of more than 24 hours. Because the nuns have five robes, compared to three for the monks, we might expect this difference to be reflected in the rule. Nevertheless, in the Sixth Council edition of the \textsanskrit{Tipiṭaka} we find only three robes mentioned both for the monks and the nuns.\footnote{This reading is found in the \textsanskrit{Chaṭṭhasaṅgāyana} version of the sub-commentary \textsanskrit{Dvemātikāpāli}, which preserves both \textsanskrit{Pātimokkhas}. } According to Bhikkhu \textsanskrit{Ñāṇatusita}’s \textit{\textsanskrit{Bhikkhunī} \textsanskrit{Pātimokkha} \textsanskrit{Pāḷi}}, however, there are several known Pali manuscripts that mention five robes.\footnote{Nyanatusita, 2010. } Moreover, it seems that all the other schools for which we have a \textsanskrit{Bhikkhunī}-\textsanskrit{pātimokkha} mention five robes for \textit{\textsanskrit{bhikkhunīs}} in connection with this rule.\footnote{Private communication from Ven. \textsanskrit{Vimalañāṇī}. } It seems likely, then, that five is the original number.

Of the twelve \textit{nissaggiyas} that are unique to the monks, three concern monks’ inappropriate dealings with nuns.\footnote{\href{https://suttacentral.net/pli-tv-bu-vb-np4/en/brahmali\#1.31.1}{Bu~NP~4}, \href{https://suttacentral.net/pli-tv-bu-vb-np5/en/brahmali\#1.2.31.1}{Bu~NP~5}, and \href{https://suttacentral.net/pli-tv-bu-vb-np17/en/brahmali\#1.20.1}{Bu~NP~17}. } These rules do not exist for nuns, suggesting that the monks were more likely to treat the nuns unfairly than the other way around. Five of the twelve concern blankets, the so-called \textit{santhatas}.\footnote{\href{https://suttacentral.net/pli-tv-bu-vb-np11/en/brahmali\#1.23.1}{Bu~NP~11}–\href{https://suttacentral.net/pli-tv-bu-vb-np15/en/brahmali\#1.3.10.1}{15}. } The fact that these rules were not inherited by the nuns may mean either that the \textit{santhata} was not considered an important requisite or, perhaps, that its use was discontinued after the earliest period. In the present day, the \textit{santhata} is all but unknown. Of the four remaining rules, \href{https://suttacentral.net/pli-tv-bu-vb-np24/en/brahmali\#1.18.1}{Bu~NP~24} and \href{https://suttacentral.net/pli-tv-bu-vb-np29/en/brahmali\#1.2.16.1}{29} concern situations that are specific to the monks; one, \href{https://suttacentral.net/pli-tv-bu-vb-np16/en/brahmali\#1.23.1}{Bu~NP~16}, is about the rather particular case of carrying unspun wool over long distances; and the final one, \href{https://suttacentral.net/pli-tv-bu-vb-np21/en/brahmali\#2.17.1}{Bu~NP~21}, is similar to \href{https://suttacentral.net/pli-tv-bi-vb-np1/en/brahmali\#1.14.1}{Bi~NP~1}.

There is little new in the \textit{nissaggiya} offenses that are unique to the nuns. As mentioned, Bi NP 1 is essentially the same as Bu NP 21. \href{https://suttacentral.net/pli-tv-bi-vb-np3/en/brahmali\#1.20.1}{Bi~NP~3}, which prohibits giving a robe and then taking it back, is similar to \href{https://suttacentral.net/pli-tv-bu-vb-np25/en/brahmali\#1.27.1}{Bu~NP~25}. \href{https://suttacentral.net/pli-tv-bi-vb-np4/en/brahmali\#1.27.1}{Bi~NP~4}–12 are in reality little more than special cases of \href{https://suttacentral.net/pli-tv-bu-vb-np20/en/brahmali\#1.35.1}{Bu~NP~20} = Bi NP 23. That leaves only one \textit{nissaggiya} offense that is properly unique to the nuns, namely \href{https://suttacentral.net/pli-tv-bi-vb-np2/en/brahmali\#1.25.1}{Bi~NP~2}, which concerns inappropriately distributing robe-cloth to one’s own monastic followers.

\section*{The \textit{\textsanskrit{pācittiyas}} (Pc)}

The \textit{\textsanskrit{bhikkhunīs}} have a total of 166 \textit{\textsanskrit{pācittiya}} offenses, of which seventy are in common with the monks. This means the nuns have 96 unique \textit{\textsanskrit{pācittiyas}}, whereas the monks have 22.

As mentioned above, there is substantial variation in the number of \textit{\textsanskrit{pācittiya}} rules for the nuns among the various schools. This contrasts with the other rules, which are largely the same in number. In her comparative study of the \textsanskrit{Bhikkhunī}-\textsanskrit{pātimokkha},\footnote{Kabilsingh, 1991. } Kabilsingh shows that the number of \textit{\textsanskrit{pācittiya}} rules varies from 141 to 210, which is comparable to the variation among the \textit{sekhiya} rules. This means that rules were added to the nuns’ \textit{\textsanskrit{pācittiyas}} after the various schools started to form, probably within the first couple of centuries after the Buddha’s passing. We thus see an interesting difference between the monks and the nuns: the \textit{\textsanskrit{pācittiyas}} of the former were kept largely unaltered from the earliest period, whereas the nuns’ \textit{\textsanskrit{pācittiyas}} were added to. What might be the reason for this?

We see in the Bhikkhuni-kkhandhaka at \href{https://suttacentral.net/pli-tv-kd20/en/brahmali\#6.1.1}{Kd~20:6.1.1}–8.1.13 that the monks were entrusted with teaching the nuns in matters related to the Vinaya. It could be the case, then, that the monks considered themselves authorized to lay down new rules for the nuns, especially in the category of minor rules, that is, the \textit{\textsanskrit{pācittiyas}}.\footnote{The \textit{\textsanskrit{pācittiyas}} are called minor, \textit{khuddaka}, at the end of the relevant section both in the Bhikkhu- and \textsanskrit{Bhikkhunī}-\textsanskrit{vibhaṅgas}. } The fact that all the early schools seem to have laid down such rules supports this thesis. Moreover, this is parallel to what happened to the \textit{sekhiya} rules, which were also added to in the sectarian period. Although the \textit{sekhiyas} are binding on both Sanghas, they were probably added to the Bhikkhu-\textsanskrit{pātimokkha}, with the nuns then inheriting the rules. As such, they were primarily considered rules for the monks. My suggestion, then, is that when rules were added to the \textsanskrit{Pātimokkhas} after the Buddha’s passing, they were added to the \textit{sekhiyas} for the monks, but to the \textit{\textsanskrit{pācittiyas}} for the nuns. In this way, there was a parallel development between the two Sanghas.

Of the 22 \textit{\textsanskrit{pācittiyas}} that are unique to the monks, ten govern the proper relationship between monks and nuns. The first four of these, \href{https://suttacentral.net/pli-tv-bu-vb-pc21/en/brahmali\#1.44.1}{Bu~Pc~21}–24, concern the \textit{\textsanskrit{ovāda}}, the monks’ fortnightly instruction to the nuns. Obviously, the nuns do not have these rules. The next two, \href{https://suttacentral.net/pli-tv-bu-vb-pc25/en/brahmali\#2.11.1}{Bu~Pc~25}–26, prohibit monks from giving robes to the nuns. It is not immediately clear why the nuns did not inherit these, that is, nuns being prohibited from giving robes to the monks, but it could be that they were redundant because of the nuns’ difficulties in obtaining requisites. \href{https://suttacentral.net/pli-tv-bu-vb-pc29/en/brahmali\#2.13.1}{Bu~Pc~29} concerns nuns ordering lay people to give food to their favorite monks. It is unsurprising that such a specific rule was not inherited by the nuns. The last three rules, \href{https://suttacentral.net/pli-tv-bu-vb-pc27/en/brahmali\#2.15.1}{Bu~Pc~27}, \href{https://suttacentral.net/pli-tv-bu-vb-pc28/en/brahmali\#2.16.1}{28}, and \href{https://suttacentral.net/pli-tv-bu-vb-pc30/en/brahmali\#1.13.1}{30}, relate to monks and nuns associating inappropriately with each other. The fact that the nuns do not have these could be because of the strict requirements of \href{https://suttacentral.net/pli-tv-bi-vb-ss3/en/brahmali\#4.14.1}{Bi~Ss~3}, which may have been regarded as sufficient to cover such situations.

A further five rules are not relevant to the nuns because they have other rules covering approximately the same areas.\footnote{\href{https://suttacentral.net/pli-tv-bu-vb-pc39/en/brahmali\#2.10.1}{Bu~Pc~39} is the same as \href{https://suttacentral.net/pli-tv-bi-vb-pd1/en/brahmali\#1.2.9.1}{Bi~Pd~1}–8; \href{https://suttacentral.net/pli-tv-bu-vb-pc41/en/brahmali\#1.2.15.1}{Bu~Pc~41} is similar to \href{https://suttacentral.net/pli-tv-bi-vb-pc46/en/brahmali\#1.15.1}{Bi~Pc~46}; \href{https://suttacentral.net/pli-tv-bu-vb-pc64/en/brahmali\#1.23.1}{Bu~Pc~64} has roughly the same scope as \href{https://suttacentral.net/pli-tv-bi-vb-pj6/en/brahmali\#1.23.1}{Bi~Pj~6} and \href{https://suttacentral.net/pli-tv-bi-vb-ss12/en/brahmali\#1.11.1}{Bi~Ss~12}; \href{https://suttacentral.net/pli-tv-bu-vb-pc65/en/brahmali\#1.53.1}{Bu~Pc~65} is covered by \href{https://suttacentral.net/pli-tv-bi-vb-pc65/en/brahmali\#1.17.1}{Bi~Pc~65} and \href{https://suttacentral.net/pli-tv-bi-vb-pc71/en/brahmali\#1.17.1}{71}; and \href{https://suttacentral.net/pli-tv-bu-vb-pc91/en/brahmali\#1.14.1}{Bu~Pc~91} is the same as \href{https://suttacentral.net/pli-tv-bi-vb-pc22/en/brahmali\#1.14.1}{Bi~Pc~22}. } Then there are \href{https://suttacentral.net/pli-tv-bu-vb-pc33/en/brahmali\#3.15.1}{Bu~Pc~33}, \href{https://suttacentral.net/pli-tv-bu-vb-pc35/en/brahmali\#2.15.1}{35}, and \href{https://suttacentral.net/pli-tv-bu-vb-pc36/en/brahmali\#1.28.1}{36}, all concerned with eating meals in succession, the combined effect of which is no more than a slight expansion of \href{https://suttacentral.net/pli-tv-bi-vb-pc54/en/brahmali\#1.20.1}{Bi~Pc~54}. That the nuns do not have an equivalent of \href{https://suttacentral.net/pli-tv-bu-vb-pc67/en/brahmali\#1.28.1}{Bu~Pc~67}, which concerns monks traveling by arrangement with women, is, again, probably because of \href{https://suttacentral.net/pli-tv-bi-vb-ss3/en/brahmali\#4.14.1}{Bi~Ss~3}, which compels a nun always to travel in the company of another nun. Next, because \textit{\textsanskrit{bhikkhunīs}} do not live in the wilderness, they have neither the equivalent of \href{https://suttacentral.net/pli-tv-bu-vb-pc85/en/brahmali\#4.9.1}{Bu~Pc~85}, which concerns forest monks entering an inhabited area at the wrong time, nor \href{https://suttacentral.net/pli-tv-bu-vb-pc89/en/brahmali\#2.10.1}{Bu~Pc~89}, which sets limits on the size of the sitting mat, a requisite that was used mostly in the wilderness. Moreover, given that monks are allowed to be without the sitting mat for up to four months (\href{https://suttacentral.net/pli-tv-kd15/en/brahmali\#18.1.3}{Kd~15:18.1.3}), it would not have been regarded as a particularly important requisite. As to the final rule that the nuns do not share with the monks, that is \href{https://suttacentral.net/pli-tv-bu-vb-pc83/en/brahmali\#1.3.56.1}{Bu~Pc~83}, which concerns entering a royal compound without being announced, it is less clear why the nuns do not have it. Perhaps the reason is simply that nuns were not considered as potential competitors to the king for the attention of his wives.

The overall impression is that there are sufficiently good reasons why the nuns did not inherit certain of the monks’ \textit{\textsanskrit{pācittiya}} rules. It is not always equally clear-cut, however, why the nuns were given ninety-six additional \textit{\textsanskrit{pācittiyas}}. Let’s have a closer look at them.

In many cases the unique rules for the nuns are found elsewhere for the monks, sometimes with a different wording or belonging to a different class of offense. Without getting distracted by the details and bearing in mind that the boundaries are often blurry, I count 22 such rules for the nuns, some of which are significant, including rules concerned with sexuality (\href{https://suttacentral.net/pli-tv-bi-vb-pc3/en/brahmali\#1.14.1}{Bi~Pc~3}, \href{https://suttacentral.net/pli-tv-bi-vb-pc4/en/brahmali\#1.21.1}{4}, and \href{https://suttacentral.net/pli-tv-bi-vb-pc5/en/brahmali\#1.2.12.1}{5}), entertainment (\href{https://suttacentral.net/pli-tv-bi-vb-pc10/en/brahmali\#1.15.1}{Bi~Pc~10}), and the invitation ceremony (\href{https://suttacentral.net/pli-tv-bi-vb-pc57/en/brahmali\#1.15.1}{Bi~Pc~57}).\footnote{Other rules I count as belonging to this category are as follows: Bi Pc 1–2, 22, 31–33, 35, 39, 42, 49–50, 54, 68–69, 74, and 84–85. } On top of these, I count an additional 14 rules that are effectively just minor expansions of the monks’ rules.\footnote{Bi Pc 7, 11–14, 46, 76, 80, 86–87, and 90–93. } This leaves a total of 60 \textit{\textsanskrit{pācittiyas}} that are genuinely unique to the nuns and worthy of special attention.\footnote{Bi Pc 6, 8–9, 15–21, 23–30, 34, 36–38, 40–41, 43–45, 47–48, 51–53, 55–56, 58–67, 70–73, 75, 77–79, 81–83, 88–89, and 94–96. }

Especially noteworthy among these latter rules are the large number concerned with ordination and related issues, in total 18, compared to only one such \textit{\textsanskrit{pācittiya}} rule for the monks.\footnote{Bi Pc 61–83, that is, 23 rules in all. Five of these, Bi Pc 68, 69, 74, 76, and 80 are included in the rules already mentioned, thus a total of 18 of the remaining 60 rules. The single such rule for monks is \href{https://suttacentral.net/pli-tv-bu-vb-pc65/en/brahmali\#1.53.1}{Bu~Pc~65}. } Some of these rules concern special circumstances for women, such as \href{https://suttacentral.net/pli-tv-bi-vb-pc61/en/brahmali\#1.17.1}{Bi~Pc~61} and \href{https://suttacentral.net/pli-tv-bi-vb-pc62/en/brahmali\#1.17.1}{62} that prohibit the ordination of women who are pregnant or breastfeeding. Other rules concern the more complicated leadup to the ordination of women, with rules mandating a period of two years as a trainee nun, a \textit{\textsanskrit{sikkhamānā}}, and rules requiring a special legal procedure to approve a trainee nun for ordination. This is further complicated by the fact that there are three classes of women who may seek ordination: a general class of trainee nuns (\textit{\textsanskrit{sikkhamānā}}), a class of married girls (\textit{gihigata}), and a class of unmarried women (\textit{\textsanskrit{kumāribhūta}}), which leads to a total of eight rules.\footnote{Bi Pc 63–64, 65–67, and 71–73. } It is peculiar that we first find two general rules for trainee nuns and then two subclasses, one for married and one for unmarried women, each with three rules. There is an obvious redundancy here since either the two general rules or the six more specific rules would have been sufficient on their own. One is left with the impression that the general rules were laid down first, possibly by the Buddha, and that the more specific rules were added at a later stage after the Buddha had passed away. We may speculate that the redundancy was caused by a conservative Sangha not being willing to update or abolish rules they regarded as coming from the Master himself.

The existence of different ordination rules for married and unmarried women is interesting. Unmarried women are treated much in the same way as men, with a minimum ordination age of twenty (\href{https://suttacentral.net/pli-tv-bi-vb-pc71/en/brahmali\#1.17.1}{Bi~Pc~71}). For married girls, however, the minimum ordination age is set at twelve (\href{https://suttacentral.net/pli-tv-bi-vb-pc65/en/brahmali\#1.17.1}{Bi~Pc~65}). This shows, first of all, that the custom of child brides is ancient. The reason why they were allowed to ordain at such a young age seems to be that they were regarded as adults once married. The origin stories to the rules that deal with the minimum age for ordination all have to do with the ability of the ordinand to deal with the hardships of monasticism. In other words, it was probably assumed that the nature of married life was such that it forced you to grow up quickly, enabling you to deal with the difficult realities of life. Since most countries now have a minimum marriage age of eighteen, these rules are largely redundant.

The above ordination rules do not place any restrictions on nuns that are fundamentally different from those of the monks. If anything, the rules are slightly more liberal for the nuns, such as the lower ordination age and the duty to follow your preceptor for only two years, as compared to five years for monks.\footnote{The nuns also have to live as \textit{\textsanskrit{sikkhamānas}} for two years, effectively increasing the number of years of dependence on one’s preceptor to four. Thus the nuns’ rules are only slightly more liberal. } The nuns do, however, have at least two rules that restrict them quite severely in matters of ordination. The first, \href{https://suttacentral.net/pli-tv-bi-vb-pc82/en/brahmali\#1.14.1}{Bi~Pc~82}, prohibits a nun from performing ordinations in consecutive years, whereas the other, \href{https://suttacentral.net/pli-tv-bi-vb-pc83/en/brahmali\#1.16.1}{Bi~Pc~83}, prohibits her from ordaining more than one person per year. The combined effect of these two rules is to dramatically reduce the growth potential of the \textsanskrit{Bhikkhunī}-sangha. This is particularly problematic at a time when the Sangha of nuns has only recently been reestablished in Theravada Buddhism.

Still, the situation is not quite as dire as it may seem. Looking closer at these rules, it becomes clear that only the first of them has any real claim to authenticity. From the comparative information on SuttaCentral,\footnote{See https://suttacentral.net/pli-tv-bi-vb-pc82. } we see that the first rule is found in all six schools, whereas the second one is only found in the Pali tradition. This means that \href{https://suttacentral.net/pli-tv-bi-vb-pc83/en/brahmali\#1.16.1}{Bi~Pc~83} almost certainly originated in the sectarian period. Nonetheless, depending on how it is interpreted, even \href{https://suttacentral.net/pli-tv-bi-vb-pc82/en/brahmali\#1.14.1}{Bi~Pc~82} may severely constrain the growth of the \textsanskrit{Bhikkhunī}-sangha.

Besides the rules on ordination, there are a number of small groups of affiliated rules that are unique to the nuns. One such group concerns protecting nuns from falling back into the worldly ways of a householder. The Vinaya has several examples of monks, and possibly even lay people, pressuring nuns to do what might be considered domestic chores, including \href{https://suttacentral.net/pli-tv-bu-vb-np4/en/brahmali\#1.31.1}{Bu~NP~4} and \href{https://suttacentral.net/pli-tv-bu-vb-np17/en/brahmali\#1.20.1}{Bu~NP~17}. Such pressure was no doubt more likely put on the nuns than on the monks. In other rules it is not clear whether the nuns were pressured or themselves chose to do such tasks. For instance, \href{https://suttacentral.net/pli-tv-bi-vb-pc6/en/brahmali\#1.19.1}{Bi~Pc~6} prohibits a nun from doing certain services for a monk while he is eating, and \href{https://suttacentral.net/pli-tv-bi-vb-pc43/en/brahmali\#1.14.1}{Bi~Pc~43} prohibits a nun from spinning yarn, whereas \href{https://suttacentral.net/pli-tv-bi-vb-pc44/en/brahmali\#1.11.1}{Bi~Pc~44} stops her from doing chores for householders. Finally, \href{https://suttacentral.net/pli-tv-bi-vb-pc36/en/brahmali\#1.11.1}{Bi~Pc~36} bans a nun from improper socializing with men.

Another group of related rules forbids nuns from being verbally abusive. \href{https://suttacentral.net/pli-tv-bi-vb-pc19/en/brahmali\#1.17.1}{Bi~Pc~19} bars a nun from cursing herself or others. \href{https://suttacentral.net/pli-tv-bi-vb-pc52/en/brahmali\#1.29.1}{Bi~Pc~52} and \href{https://suttacentral.net/pli-tv-bi-vb-pc53/en/brahmali\#1.27.1}{53} stop a \textit{\textsanskrit{bhikkhunī}} from abusing a monk and a community of nuns respectively. The first two of these are variations of \href{https://suttacentral.net/pli-tv-bu-vb-pc2/en/brahmali\#1.2.33.1}{Bu~Pc~2}/Bi Pc 98, whereas the last one, which according to the \textsanskrit{Vibhaṅga} concerns \textit{sanghakamma}, is essentially an elaboration on \href{https://suttacentral.net/pli-tv-bu-vb-pc79/en/brahmali\#1.22.1}{Bu~Pc~79}/Bi Pc 157 and \href{https://suttacentral.net/pli-tv-bu-vb-pc81/en/brahmali\#1.16.1}{Bu~Pc~81}/Bi Pc 159, which forbid criticizing a properly performed \textit{sanghakamma}. It is interesting that the first of the three, Bi Pc 19, also prohibits self-harm, a prohibition taken further in \href{https://suttacentral.net/pli-tv-bi-vb-pc20/en/brahmali\#1.11.1}{Bi~Pc~20}, which prohibits a nun from beating herself and then crying. Self-harm may have been a significant issue in a society where women often experienced discrimination and violence.

Yet another group of rules was laid down to protect nuns from harm, including \href{https://suttacentral.net/pli-tv-bi-vb-pc21/en/brahmali\#1.16.1}{Bi~Pc~21}, which prohibits a nun from bathing naked, and \href{https://suttacentral.net/pli-tv-bi-vb-pc37/en/brahmali\#1.12.1}{Bi~Pc~37} and \href{https://suttacentral.net/pli-tv-bi-vb-pc38/en/brahmali\#1.12.1}{38}, which prohibit traveling in dangerous places without company. The protection of nuns from harm is a recurring theme in the nuns’ rules, including in \href{https://suttacentral.net/pli-tv-bi-vb-ss3/en/brahmali\#1.1}{Bi~Ss~3} and in several minor rules in the Bhikkhuni-kkhandhaka.

Then there is a group of eight rules concerned with the proper conduct in relation to robes.\footnote{Bi Pc 23–29 and 47. } These rules focus on two main issues, treating one’s fellow nuns considerately and the proper looking after of one’s requisites. An important consideration in many of these is no doubt, once again, the difficulty nuns had in obtaining material support.

There is also a group of seven \textit{\textsanskrit{bhikkhunī} \textsanskrit{pācittiyas}} concerned broadly with etiquette.\footnote{Bi Pc 8–9, 15–17, and 94–95. } Among these, \href{https://suttacentral.net/pli-tv-bi-vb-pc15/en/brahmali\#1.19.1}{Bi~Pc~15}–17 concern rude behavior toward householders and \href{https://suttacentral.net/pli-tv-bi-vb-pc94/en/brahmali\#1.11.1}{Bi~Pc~94}–95 are about inappropriate behavior toward monks. These latter two rules, together with \href{https://suttacentral.net/pli-tv-bi-vb-pc52/en/brahmali\#1.29.1}{Bi~Pc~52} discussed above, are a reminder of the gender hierarchy that existed in ancient India. Still, among the six schools mentioned earlier, Bi Pc 94 is only found in the Pali and the \textsanskrit{Sarvāstivāda} Vinaya,\footnote{\textsanskrit{Sarvāstivāda} \textit{\textsanskrit{bhikkhunī} \textsanskrit{pācittiya}} 104. } whereas Bi Pc 52 is found neither in the \textsanskrit{Mahiśāsaka} nor the \textsanskrit{Mūlasarvāstivāda} Vinaya, suggesting that both of these rules are sectarian in origin, thus arguably not binding on \textit{\textsanskrit{bhikkhunīs}}. Moreover, the ruling at \href{https://suttacentral.net/pli-tv-bi-vb-pc95/en/brahmali\#1.11.1}{Bi~Pc~95} that nuns must ask for permission before asking a question, which happens to be shared with four other schools,\footnote{The four are \textsanskrit{Mahiśāsaka} \textit{\textsanskrit{bhikkhunī} \textsanskrit{pācittiya}} 186, Dharmaguptaka \textit{\textsanskrit{bhikkhunī} \textsanskrit{pācittiya}} 172, \textsanskrit{Mūlasarvāstivāda} \textit{\textsanskrit{bhikkhunī} \textsanskrit{pācittiya}} 169, and \textsanskrit{Sarvāstivāda} \textit{\textsanskrit{bhikkhunī} \textsanskrit{pācittiya}} 158. } partly overlaps with a similar rule for the monks at \href{https://suttacentral.net/pli-tv-kd2/en/brahmali\#15.6.3}{Kd~2:15.6.3}.

The remaining two rules in this group, \href{https://suttacentral.net/pli-tv-bi-vb-pc8/en/brahmali\#1.26.1}{Bi~Pc~8} and \href{https://suttacentral.net/pli-tv-bi-vb-pc9/en/brahmali\#1.15.1}{9}, are concerned with the disposal of waste products. The origin story to the first of these is particularly entertaining. Early one morning a nun empties a chamber pot by throwing the contents over a wall, all of it landing on the head of a brahmin who happens to be on his way to work. In a fury, the brahmin decides to burn down the nuns’ residence. Just as he is about to enter the building with a firebrand, a lay supporter sees him. The brahmin tells him what has happened, upon which the lay supporter tells him how lucky he is to receive such a blessing from the nuns! The brahmin cools down and departs. Too good to be true, you say? Not when you know of the many bizarre things still happening in the Buddhist world. Finally, the last rule in this group, \href{https://suttacentral.net/pli-tv-bi-vb-pc9/en/brahmali\#1.15.1}{Bi~Pc~9}, concerns spoiling a field with waste products. This rule is effectively no more than a slight expansion of \href{https://suttacentral.net/pli-tv-bu-vb-sk74/en/brahmali\#1.3.1}{Sk~74}.

Moving on to the next group, the nuns have three unique rules against excessive indulgence, that is, \href{https://suttacentral.net/pli-tv-bi-vb-pc41/en/brahmali\#1.16.1}{Bi~Pc~41}, \href{https://suttacentral.net/pli-tv-bi-vb-pc88/en/brahmali\#1.14.1}{88}, and \href{https://suttacentral.net/pli-tv-bi-vb-pc89/en/brahmali\#1.14.1}{89}, the first of which concerns the visiting of pleasure houses and parks, whereas the last two are about bathing in scented water. The problem of indulgence is an important theme throughout the Vinaya \textsanskrit{Piṭaka}, and especially so in the Khandhakas.

There are a further four rules that regulate the important relationship between the nuns and the Bhikkhu-sangha: \href{https://suttacentral.net/pli-tv-bi-vb-pc51/en/brahmali\#3.9.1}{Bi~Pc~51}, \href{https://suttacentral.net/pli-tv-bi-vb-pc56/en/brahmali\#1.16.1}{56}, \href{https://suttacentral.net/pli-tv-bi-vb-pc58/en/brahmali\#1.14.1}{58}, and \href{https://suttacentral.net/pli-tv-bi-vb-pc59/en/brahmali\#1.11.1}{59}. Bi Pc 58 and 59 require \textit{\textsanskrit{bhikkhunīs}} to ask for and take part in the half-monthly instruction, the \textit{\textsanskrit{ovāda}}, whereas Bi Pc 56 is a practical consequence of this, thus prohibiting a \textit{\textsanskrit{bhikkhunī}} from spending the rainy season in a monastery without monks. In the early years of the Sangha, the \textit{\textsanskrit{bhikkhunīs}} were the junior partners, both in terms of seniority and numbers, and it was therefore natural for the monks to support them with teachings. In the present day this is no longer always the case, as a result of which these rules may seem discriminatory. In practice, however, the nuns are often grateful for such teachings, especially if they come from a senior and well-respected member of the Bhikkhu-sangha.

Finally, there are eight remaining miscellaneous rules,\footnote{Bi Pc 18, 34, 40, 45, 48, 55, 60, and 96. } the majority of which concern inconsiderate conduct toward fellow nuns. Then there is \href{https://suttacentral.net/pli-tv-bi-vb-pc40/en/brahmali\#1.13.1}{Bi~Pc~40}, which requires the nuns to go wandering after the rainy season residence. The purpose of this rule, according to the origin story, is to ensure the nuns go out to meet lay supporters, a part of which would have been teaching engagements. It seems, then, that the nuns were encouraged from the very beginning to take an active part in inspiring faith and teaching the Dhamma. The last of the eight, \href{https://suttacentral.net/pli-tv-bi-vb-pc96/en/brahmali\#1.16.1}{Bi~Pc~96}, requires a nun to wear a “chest wrap”, a \textit{\textsanskrit{saṅkaccikā}}, when entering an inhabited area. This cloth is a special robe worn only by \textit{\textsanskrit{bhikkhunīs}}. Its purpose is to protect a nun’s modesty, especially among lay people.

\section*{Once again, why do the nuns have more rules than the monks?}

Having surveyed the content of the \textsanskrit{Bhikkhunī}-\textsanskrit{vibhaṅga}, we are now in a position to return to this unavoidable question. As we have seen, the main reason for the discrepancy is that the nuns inherited a large number of rules from the monks, but not vice versa. This continued into the sectarian period, with the nuns inheriting all the \textit{sekhiyas} that were laid down at this time. (See introduction to volume 1.) Altogether they inherited 181 rules, leaving only 46 unique rules for the monks, compared to 130 for the nuns. In addition, the nuns’ \textit{\textsanskrit{pācittiya}} rules were added to long after the monks’ \textit{\textsanskrit{pācittiyas}} were fixed. These two reasons probably account for most of the difference in the number of rules, perhaps even all of it. Still, we are now in a position to look more closely at other contributing factors.

To begin with, the nuns sometimes have several rules where the monks only have one. For instance, the nuns have eight \textit{\textsanskrit{pāṭidesanīya}} rules, \href{https://suttacentral.net/pli-tv-bi-vb-pd1/en/brahmali\#1.2.9.1}{Bi~Pd~1}–8, which together correspond to a single rule for the monks, that is, \href{https://suttacentral.net/pli-tv-bu-vb-pc39/en/brahmali\#2.10.1}{Bu~Pc~39}. Similarly, the nuns have nine \textit{nissaggiya \textsanskrit{pācittiya}} rules, \href{https://suttacentral.net/pli-tv-bi-vb-np4/en/brahmali\#1.27.1}{Bi~NP~4}–12, that are effectively reducible to the shared rule on bartering, \href{https://suttacentral.net/pli-tv-bu-vb-np20/en/brahmali\#1.35.1}{Bu~NP~20}/Bi NP 23.\footnote{It is not clear to me why the nuns have these extra rules when they seem to be covered by \href{https://suttacentral.net/pli-tv-bu-vb-np20/en/brahmali\#1.35.1}{Bu~NP~20}/Bi NP 23. } This means that the nuns have 17 rules that are collectively covered by two rules in the Bhikkhu-\textsanskrit{pātimokkha}. And so, none of these is truly unique.

Second, we have seen above that 36 of the nuns’ unique \textit{\textsanskrit{pācittiya}} rules are also rules for the monks, but elsewhere in the Vinaya \textsanskrit{Piṭaka}, often in the Khandhakas. There are another three such rules among the nuns’ \textit{\textsanskrit{saṅghādisesas}} and \textit{nissaggiya \textsanskrit{pācittiyas}}, adding up to 39.

\begin{enumerate}%
\item Bi Ss 2 ≈ \href{https://suttacentral.net/pli-tv-kd1/en/brahmali\#43.1.14}{Kd~1:43.1.14}, which prohibit ordaining a criminal.%
\item Bi NP 1 = \href{https://suttacentral.net/pli-tv-bu-vb-np21/en/brahmali\#2.17.1}{Bu~NP~21}, which prohibit having more than one almsbowl.%
\item Bi NP 3 ≈ \href{https://suttacentral.net/pli-tv-bu-vb-np25/en/brahmali\#1.27.1}{Bu~NP~25}/Bi NP 26, which prohibit taking back a robe.%
\item Bi Pc 1 = \href{https://suttacentral.net/pli-tv-kd15/en/brahmali\#34.1.15}{Kd~15:34.1.15}, which prohibit the eating of garlic.%
\item Bi Pc 2 = \href{https://suttacentral.net/pli-tv-kd15/en/brahmali\#27.4.19}{Kd~15:27.4.19}, which prohibit the removal of pubic hair.%
\item Bi Pc 3–5 ≈ \href{https://suttacentral.net/pli-tv-bu-vb-ss1/en/brahmali\#2.1.13.1}{Bu~Ss~1}, which prohibit masturbation.%
\item Bi Pc 7 ≈ \href{https://suttacentral.net/pli-tv-kd6/en/brahmali\#17.4.1}{Kd~6:17.4.1}, which prohibit cooking.%
\item Bi Pc 10 = \href{https://suttacentral.net/pli-tv-kd15/en/brahmali\#2.6.6}{Kd~15:2.6.6}, which prohibit entertainment.%
\item Bi Pc 11–14 ≈ \href{https://suttacentral.net/pli-tv-bu-vb-pc44/en/brahmali\#1.14.1}{Bu~Pc~44}/Bi Pc 125 and \href{https://suttacentral.net/pli-tv-bu-vb-pc45/en/brahmali\#1.14.1}{Bu~Pc~45}/Bi Pc 126, which prohibit being alone with a person of the opposite gender.%
\item Bi Pc 22 = \href{https://suttacentral.net/pli-tv-bu-vb-pc91/en/brahmali\#1.14.1}{Bu~Pc~91:1.14.1}, which prohibit an oversize bathing cloth.%
\item Bi Pc 31–32 = \href{https://suttacentral.net/pli-tv-kd15/en/brahmali\#19.2.6}{Kd~15:19.2.6}–19.2.12, which prohibit sleeping together.%
\item Bi Pc 33 ≈ \href{https://suttacentral.net/pli-tv-bu-vb-pc77/en/brahmali\#1.19.1}{Bu~Pc~77}/Bi Pc 155, which prohibit making a monastic feel ill at ease.%
\item Bi Pc 35 = \href{https://suttacentral.net/pli-tv-bu-vb-pc17/en/brahmali\#1.31.1}{Bu~Pc~17}/Bi Pc 113, which prohibit evicting a monastic from a dwelling.%
\item Bi Pc 39 = \href{https://suttacentral.net/pli-tv-kd3/en/brahmali\#3.2.7}{Kd~3:3.2.7}, which prohibit traveling during the rainy season residence.%
\item Bi Pc 42 = \href{https://suttacentral.net/pli-tv-kd5/en/brahmali\#10.5.1}{Kd~5:10.5.1}, which prohibit luxurious furniture.%
\item Bi Pc 46 ≈ \href{https://suttacentral.net/pli-tv-bu-vb-pc41/en/brahmali\#1.2.15.1}{Bu~Pc~41}, which prohibit giving food to a non-monastic.%
\item Bi Pc 49 = \href{https://suttacentral.net/pli-tv-kd15/en/brahmali\#33.2.22}{Kd~15:33.2.22}, which prohibit studying worldly subjects.%
\item Bi Pc 50 = \href{https://suttacentral.net/pli-tv-kd15/en/brahmali\#33.2.28}{Kd~15:33.2.28}, which prohibit teaching worldly subjects.%
\item Bi Pc 54 ≈ \href{https://suttacentral.net/pli-tv-bu-vb-pc35/en/brahmali\#2.15.1}{Bu~Pc~35}, which prohibit eating another meal after an invitation.%
\item Bi Pc 57 ≈ \href{https://suttacentral.net/pli-tv-kd4/en/brahmali\#1.13.5}{Kd~4:1.13.5}, which require doing the invitation ceremony.%
\item Bi Pc 68 ≈ \href{https://suttacentral.net/pli-tv-kd1/en/brahmali\#26.1.1}{Kd~1:26.1.1}, which require supporting a student.%
\item Bi Pc 69 ≈ \href{https://suttacentral.net/pli-tv-kd1/en/brahmali\#53.4.7}{Kd~1:53.4.7}, which require staying with one’s preceptor for a given length of time.%
\item Bi Pc 74 ≈ \href{https://suttacentral.net/pli-tv-kd1/en/brahmali\#31.5.14}{Kd~1:31.5.14}, which prohibit inexperienced preceptors.%
\item Bi Pc 76 ≈ \href{https://suttacentral.net/pli-tv-bu-vb-pc79/en/brahmali\#1.22.1}{Bu~Pc~79}/Bi Pc 157 + \href{https://suttacentral.net/pli-tv-bu-vb-pc81/en/brahmali\#1.16.1}{Bu~Pc~81}/Bi Pc 159, which prohibit criticizing a valid legal procedure.%
\item Bi Pc 80 ≈ \href{https://suttacentral.net/pli-tv-kd1/en/brahmali\#54.6.4}{Kd~1:54.6.4}, which prohibit ordination without permission of interested parties.%
\item Bi Pc 84 = \href{https://suttacentral.net/pli-tv-kd15/en/brahmali\#23.2.19}{Kd~15:23.2.19} and \href{https://suttacentral.net/pli-tv-kd5/en/brahmali\#12.1.5}{Kd~5:12.1.5}, which respectively prohibit sunshades and sandals.%
\item Bi Pc 85 = \href{https://suttacentral.net/pli-tv-kd5/en/brahmali\#9.4.0}{Kd~5:9.4.0}, which prohibit traveling in a vehicle.%
\item Bi Pc 86–87 ≈ \href{https://suttacentral.net/pli-tv-kd15/en/brahmali\#2.1.18}{Kd~15:2.1.18–2.1.26}, which prohibit ornaments.%
\item Bi Pc 90–93 ≈ \href{https://suttacentral.net/pli-tv-kd15/en/brahmali\#1.4.5}{Kd~15:1.4.5}, which prohibit massage.%
\end{enumerate}

In fact, it is hard to give a precise number of such rules because sometimes the equivalence is not exact. Arguably there are even more such rules. Nonetheless, adding the 17 rules from above, we have 56 unique nuns’ rules that turn out not to be truly unique after all. Instead of 130 unique rules for the nuns, we are down to 74. On top of this, it is reasonable to regard some of the large number of nuns’ rules concerned with ordination as an elaboration on \href{https://suttacentral.net/pli-tv-bu-vb-pc65/en/brahmali\#1.53.1}{Bu~Pc~65},\footnote{Especially \href{https://suttacentral.net/pli-tv-bi-vb-pc63/en/brahmali\#1.41.1}{Bi~Pc~63}–67 and \href{https://suttacentral.net/pli-tv-bi-vb-pc71/en/brahmali\#1.17.1}{Bi~Pc~71}–73, altogether eight rules being roughly equivalent to the monks’ \href{https://suttacentral.net/pli-tv-bu-vb-pc65/en/brahmali\#1.53.1}{Bu~Pc~65}. } which brings us down to perhaps 66 rules for the nuns that are unique in a meaningful sense.

These findings substantially change our picture of the difference in the number of rules between the two Sanghas. Given that the monks effectively have 39 unique rules and the nuns 66,\footnote{I have reduced the number of unique monks’ rules from the 46 given earlier to 39, because seven of them are mentioned in the footnote on 39 equivalents as being effective equivalents to certain \textit{\textsanskrit{bhikkhunī}} rules. These are \href{https://suttacentral.net/pli-tv-bu-vb-ss1/en/brahmali\#2.1.13.1}{Bu~Ss~1}, \href{https://suttacentral.net/pli-tv-bu-vb-np21/en/brahmali\#2.17.1}{Bu~NP~21}, \href{https://suttacentral.net/pli-tv-bu-vb-pc35/en/brahmali\#2.15.1}{Bu~Pc~35}, \href{https://suttacentral.net/pli-tv-bu-vb-pc39/en/brahmali\#2.10.1}{Bu~Pc~39}, \href{https://suttacentral.net/pli-tv-bu-vb-pc41/en/brahmali\#1.2.15.1}{Bu~Pc~41}, \href{https://suttacentral.net/pli-tv-bu-vb-pc65/en/brahmali\#1.53.1}{Bu~Pc~65} and \href{https://suttacentral.net/pli-tv-bu-vb-pc91/en/brahmali\#1.14.1}{Bu~Pc~91}. } the difference is down from 84 to 27. Suddenly the difference seems less of a problem, amounting as it does to less than 10% of the nuns’ overall number of \textsanskrit{Pātimokkha} rules. Such a relatively small difference might just be a result of the contingencies of history. It does not define the nuns’ \textsanskrit{Pātimokkha} as fundamentally different from that of the monks.

Yet, even this is not the complete picture. We need to return to the fact that many of the nuns’ \textit{\textsanskrit{pācittiya}} rules evidently were laid down in the sectarian period, which means they are arguably not binding on the nuns.\footnote{This is, in fact, also true of the \textit{\textsanskrit{bhikkhunī} nissaggiya \textsanskrit{pācittiya}} rules. There are considerable variations between the different schools in the rules that correspond to the Theravadin \href{https://suttacentral.net/pli-tv-bi-vb-np4/en/brahmali\#1.27.1}{Bi~NP~4}–10. However, I have already discounted these rules as roughly equivalent to \href{https://suttacentral.net/pli-tv-bu-vb-np20/en/brahmali\#1.35.1}{Bu~NP~20}/Bi NP 23. I cannot discount the same rules a second time. } Using the comparative tables provided by SuttaCentral,\footnote{See https://suttacentral.net/pli-tv-bi-vb-pc, and click the parallels button for each rule. } we find that a large number of \textit{\textsanskrit{pācittiya}} rules are not shared by all the six schools. If we take the standard that any rule not shared by all schools is sectarian, that is, not stemming from the earliest period of Buddhism, we discover that this is so for 31 of the 66 remaining rules.\footnote{Taking the Pali Vinaya as a baseline and using the Theravada numbering system, the 31 unique \textit{\textsanskrit{pācittiya}} rules that are missing from one or more of the other schools are as follows: Bi Pc 15, 21, 23–27, 29–30, 34, 36, 41, 43, 45, 47–48, 51–53, 56, 61–62, 75, 77, 79, 81, 83, 89, 94–96. } This means that only 35 unique nuns’ rules may reasonably be considered as non-sectarian, compared to 39 for the monks.\footnote{The 35 are the following: Bi Pj 5–8; Bi Ss 1, 3–6, and 10–13; Bi NP 2; Bi Pc 6, 8–9, 16–20, 28, 37–38, 40, 44, 55, 58–60, 70, 78, 82, and 88. } We have arrived at virtual parity.

Although this is a remarkable result, we need to keep in mind that this is no more than a rough estimate. The uncertainties are especially pronounced in the number of rules that are likely to be sectarian. Nevertheless, this is a much more plausible picture than any direct comparison between the number of rules in the \textsanskrit{Pātimokkhas} of the two Sanghas.

We arrive at a rather unexpected and welcome conclusion. When we exclude any rules that we can infer were not laid down by the Buddha, the number of rules that is truly mandatory for the \textit{\textsanskrit{bhikkhunīs}} is essentially the same as for the monks.\footnote{As to the related question of whether the nuns’ rules are generally more stringent than those of the monks, see the discussion in the section on the \textit{\textsanskrit{bhikkhunī} \textsanskrit{pārājika}} offenses above. }

%
\chapter*{Abbreviations}
\addcontentsline{toc}{chapter}{Abbreviations}
\markboth{Abbreviations}{Abbreviations}

\begin{description}%
\item[AN] \textsanskrit{Aṅguttara} Nikāya (references are to Nipāta and \textit{sutta} numbers)%
\item[AN-a] \textsanskrit{Aṅguttara} Nikāya \textsanskrit{aṭṭhakathā}, the commentary on the \textsanskrit{Aṅguttara} Nikāya%
\item[As] \textit{\textsanskrit{adhikaraṇasamathadhamma}}%
\item[Ay] \textit{aniyata}%
\item[Bi] \textit{\textsanskrit{bhikkhunī}}%
\item[Bu] \textit{bhikkhu}%
\item[CPD] Critical Pali Dictionary%
\item[DN] \textsanskrit{Dīgha} \textsanskrit{Nikāya} (references are to \textit{sutta} numbers)%
\item[DN-a] \textsanskrit{Dīgha} \textsanskrit{Nikāya} \textsanskrit{aṭṭhakathā}, the commentary on the \textsanskrit{Dīgha} \textsanskrit{Nikāya}%
\item[DOP] Dictionary of Pali%
\item[f, ff] and the following page, pages%
\item[Iti] Itivuttaka (references are to verse numbers)%
\item[Ja] \textsanskrit{Jātaka} and \textsanskrit{Jātaka} \textsanskrit{aṭṭhakathā}%
\item[Kd] Khandhaka%
\item[Khuddas-\textsanskrit{pṭ}] \textsanskrit{Khuddasikkhā}-\textsanskrit{purāṇaṭīkā} (references are to paragraph numbers)%
\item[Khuddas-\textsanskrit{nṭ}] \textsanskrit{Khuddasikkhā}-\textsanskrit{abhinavaṭīkā} (references are to paragraph numbers)%
\item[Kkh] \textsanskrit{Kaṅkha}̄\textsanskrit{vitaraṇi}̄%
\item[Kkh-\textsanskrit{pṭ}] \textsanskrit{Kaṅkhāvitaraṇīpurāṇa}-\textsanskrit{ṭīkā}%
\item[MN] Majjhima \textsanskrit{Nikāya} (references are to \textit{sutta} numbers)%
\item[MN-a] Majjhima \textsanskrit{Nikāya} \textsanskrit{aṭṭhakathā}, the commentary on the Majjhima \textsanskrit{Nikāya}%
\item[MS] \textsanskrit{Mahāsaṅgīti} \textsanskrit{Tipiṭaka} (the version of the \textsanskrit{Tipiṭaka} found on SuttaCentral)%
\item[N\&E] “Nature and the Environment in Early Buddhism”, Bhante Dhammika%
\item[Nidd-a] \textsanskrit{Mahāniddesa} \textsanskrit{aṭṭhakathā} (references are to VRI edition paragraph numbers)%
\item[NP] \textit{nissaggiya \textsanskrit{pācittiya}}%
\item[p., pp.] page, pages%
\item[Pc] \textit{\textsanskrit{pācittiya}}%
\item[Pd] \textit{\textsanskrit{pāṭidesanīya}}%
\item[PED] Pali English Dictionary%
\item[Pj] \textit{\textsanskrit{pārājika}}%
\item[PTS] Pali Text Society%
\item[Pvr] \textsanskrit{Parivāra}%
\item[SAF] “South Asian Flora as reflected in the twelfth-century Pali lexicon \textsanskrit{Abhidhānapadīpikā}”, J. Liyanaratne%
\item[SED] Sanskrit English Dictionary%
\item[Sk] \textit{sekhiya}%
\item[SN] \textsanskrit{Saṁyutta} \textsanskrit{Nikāya} (references are to \textsanskrit{Saṁyutta} and \textit{sutta} numbers)%
\item[SN-a] \textsanskrit{Saṁyutta} \textsanskrit{Nikāya} \textsanskrit{aṭṭhakathā}, the commentary on the \textsanskrit{Saṁyutta} \textsanskrit{Nikāya} (references are to volume number and paragraph numbers of the VRI version)%
\item[Sp] Samantapāsādikā, the commentary on the Vinaya \textsanskrit{Piṭaka} (references are to volume and paragraph numbers of the VRI version)%
\item[Sp‑ṭ] Sāratthadīpanī-ṭīkā (references follow the division into five volumes of the Canonical text and then add the paragraph number of the VRI version of the sub-commentary)%
\item[Sp‑yoj] \textsanskrit{Pācityādiyojanā} (volume numbers match those of Sp of the online VRI version, which, given that Sp‑yoj starts with the \textit{bhikkhu \textsanskrit{pācittiyas}}, means that Sp‑yoj is divided into four volumes, starting at volume 2; paragraph numbers are those of the VRI version)%
\item[SRT] Siamrath \textsanskrit{Tipiṭaka}, official edition of the \textsanskrit{Tipiṭaka} published in Thailand%
\item[Ss] \textit{\textsanskrit{saṅghādisesa}}%
\item[sv.] \textit{sub voce}, see under%
\item[\textsanskrit{Thīg}] \textsanskrit{Therīgāthā}%
\item[Ud-a] \textsanskrit{Udāna} \textsanskrit{aṭṭhakathā}, the commentary on the \textsanskrit{Udāna} (references are to \textit{sutta} number)%
\item[Vb] \textsanskrit{Vibhaṅga}, the second book of the Abhidhamma \textsanskrit{Piṭaka}%
\item[Vin-\textsanskrit{ālaṅ}-\textsanskrit{ṭ}] \textsanskrit{Vinayālaṅkāra}-\textsanskrit{ṭīkā} (references are to chapter number and paragraph numbers of the VRI version)%
\item[Vin-vn-\textsanskrit{ṭ}] \textsanskrit{Vinayavinicchayaṭīkā} (references are to paragraph numbers of the VRI version)%
\item[Vjb] \textsanskrit{Vajirabuddhiṭīkā} (references are to volume and paragraph numbers of the VRI version)%
\item[Vmv] \textsanskrit{Vimativinodanī}-\textsanskrit{ṭīkā} (references are to volume and paragraph numbers of the VRI version)%
\item[VRI] Vipassana Research Institute, the publisher of the online version of the Sixth Council edition of the Pali Canon at https://www.tipitaka.org%
\item[Vv-a] \textsanskrit{Vimānavatthu} \textsanskrit{aṭṭhakathā}, the commentary on the \textsanskrit{Vimānavatthu} (references are to paragraph numbers of the VRI edition).%
\end{description}

%
\mainmatter%
\pagestyle{fancy}%
\addtocontents{toc}{\let\protect\contentsline\protect\nopagecontentsline}
\part*{Nuns’ Rules and Their Analysis }
\addcontentsline{toc}{part}{Nuns’ Rules and Their Analysis }
\markboth{}{}
\addtocontents{toc}{\let\protect\contentsline\protect\oldcontentsline}

%
%
\addtocontents{toc}{\let\protect\contentsline\protect\nopagecontentsline}
\chapter*{Expulsion}
\addcontentsline{toc}{chapter}{\tocchapterline{Expulsion}}
\addtocontents{toc}{\let\protect\contentsline\protect\oldcontentsline}

%
\section*{{\suttatitleacronym  Bi Pj 1–4}{\suttatitletranslation Rules shared in common with monks }{\suttatitleroot Sādhāraṇapārājika}}
\addcontentsline{toc}{section}{\tocacronym{ Bi Pj 1–4} \toctranslation{Rules shared in common with monks } \tocroot{Sādhāraṇapārājika}}
\markboth{Rules shared in common with monks }{Sādhāraṇapārājika}
\extramarks{ Bi Pj 1–4}{ Bi Pj 1–4}

\scadd{The \textit{\textsanskrit{pārājika}} rules 1–4 for nuns are not found in any manuscript. Tradition says they are similar to the corresponding rules for monks. }

%
\section*{{\suttatitleacronym Bi Pj 5}{\suttatitletranslation The training rule on above the knees }{\suttatitleroot Ubbhajāṇumaṇḍalikā}}
\addcontentsline{toc}{section}{\tocacronym{Bi Pj 5} \toctranslation{The training rule on above the knees } \tocroot{Ubbhajāṇumaṇḍalikā}}
\markboth{The training rule on above the knees }{Ubbhajāṇumaṇḍalikā}
\extramarks{Bi Pj 5}{Bi Pj 5}

\subsection*{Origin story }

\scnamo{Homage to the Buddha, the Perfected One, the fully Awakened One }

At\marginnote{1.1} one time when the Buddha was staying at \textsanskrit{Sāvatthī} in the Jeta Grove, \textsanskrit{Anāthapiṇḍika}’s Monastery,\footnote{The numbering begins with five because the first four rules have been omitted. This is because these rules are almost identical to the rules entailing expulsion for the monks, the only difference being the first rule, which is worded slightly differently. } \textsanskrit{Migāra}’s grandson \textsanskrit{Sāḷha} wanted to build a dwelling for the Sangha of nuns. He went to the nuns and said, “Venerables, I wish to build a dwelling for the Sangha of nuns. Please get me the nun in charge of building work.” 

At\marginnote{1.6} that time four sisters had gone forth as nuns: \textsanskrit{Nandā}, \textsanskrit{Nandavatī}, \textsanskrit{Sundarīnandā}, and \textsanskrit{Thullanandā}. Of these, the nun \textsanskrit{Sundarīnandā} had gone forth when she was young, and she was beautiful, intelligent, skilled, and diligent, and she possessed good judgment in regard to doing and arranging things well. After appointing \textsanskrit{Sundarīnandā} to be in charge of building work, the Sangha made her work with \textsanskrit{Sāḷha}. As a consequence of this, \textsanskrit{Sundarīnandā} would often go to \textsanskrit{Sāḷha}’s house to ask for tools, whether an adz, a hatchet, an ax, a spade, or a chisel. And \textsanskrit{Sāḷha} would often go to the nuns’ dwelling place to find out about the progress of the building work. And because they saw each other frequently, they fell in love. 

But\marginnote{1.14} because \textsanskrit{Sāḷha} could not find any opportunity to be intimate with \textsanskrit{Sundarīnandā}, he invited the Sangha of nuns for a meal. When preparing the seats in the dining hall, he counted the number of nuns senior to Venerable \textsanskrit{Sundarīnandā} and placed their seats to one side, and he counted the number of nuns junior to her and placed their seats on the other side. He then placed \textsanskrit{Sundarīnandā}’s seat in a concealed spot in a corner. In this way the senior nuns would think she was seated close to the junior ones and the junior nuns would think she was seated close to the senior ones. Soon afterwards he informed the Sangha that the meal was ready. 

\textsanskrit{Sundarīnandā}\marginnote{1.18} thought, “\textsanskrit{Sāḷha} hasn’t prepared a meal for the Sangha as a service, but because he wants to be intimate with me. If I go, I will get into trouble.” She then told a nun who was her pupil, “Go and bring back almsfood for me. If anyone asks about me, tell them I’m sick.” 

“Yes,\marginnote{1.21} Venerable,” she replied. 

Soon\marginnote{1.22} afterwards \textsanskrit{Sāḷha} was standing outside his gatehouse repeatedly inquiring after \textsanskrit{Sundarīnandā}:\footnote{For a discussion of the rendering “gatehouse” for \textit{\textsanskrit{koṭṭhaka}}, see Appendix of Technical Terms. } “Venerables, where’s Venerable \textsanskrit{Sundarīnandā}?” The nun who was \textsanskrit{Sundarīnandā}’s pupil told him: “She’s sick. I’m bringing her almsfood.” \textsanskrit{Sāḷha} thought, “I invited the Sangha of nuns because of \textsanskrit{Sundarīnandā},” and after telling the people there to serve the meal to the Sangha of nuns, he left for the nuns’ dwelling place. 

Just\marginnote{1.27} then \textsanskrit{Sundarīnandā} was standing outside the monastery gatehouse longing for \textsanskrit{Sāḷha}. When she saw him coming, she entered the dwelling, put on her upper robe so that it covered her head, and lay down on her bed. \textsanskrit{Sāḷha} approached her and said, “Venerable, what’s wrong with you? Why are you lying down?” 

“That’s\marginnote{1.30} what happens when you desire someone who doesn’t desire you in return.” 

“What\marginnote{1.31} do you mean I don’t desire you? I just couldn’t find an opportunity to be intimate with you.”\footnote{“Intimate with” renders \textit{\textsanskrit{dūsetuṁ}}. For a discussion of the meaning of \textit{\textsanskrit{dūseti}}, see Appendix of Technical Terms. } And both having lust, he made physical contact with \textsanskrit{Sundarīnandā}. 

Just\marginnote{1.34} then a nun who was weak from old age and had problems with her feet was lying down not far from \textsanskrit{Sundarīnandā}. She saw how \textsanskrit{Sāḷha} made physical contact with \textsanskrit{Sundarīnandā} while both of them had lust. She complained and criticized her, “How could Venerable \textsanskrit{Sundarīnandā} consent to a man making physical contact with her, when they both had lust?” That nun then told the nuns what had happened. The nuns of few desires, who had a sense of conscience, and who were contented, afraid of wrongdoing, and fond of the training, complained and criticized her, “How could Venerable \textsanskrit{Sundarīnandā} consent to a man making physical contact with her, when they both had lust?” Those nuns then told the monks. And the monks of few desires, who had a sense of conscience, and who were contented, afraid of wrongdoing, and fond of the training, criticized her in the same way. 

After\marginnote{1.41} rebuking the nun \textsanskrit{Sundarīnandā} in many ways, they told the Buddha. Soon afterwards the Buddha had the Sangha gathered and questioned the monks: “Is it true, monks, that the nun \textsanskrit{Sundarīnandā} acted like this?” 

“It’s\marginnote{1.44} true, Sir.” 

The\marginnote{1.45} Buddha rebuked her, “It’s not suitable, monks, it’s not proper for the nun \textsanskrit{Sundarīnandā}, it’s not worthy of a monastic, it’s not allowable, it’s not to be done. How could \textsanskrit{Sundarīnandā} consent to a man making physical contact with her, when they both had lust? This will affect people’s confidence, and cause some to lose it.” And the Buddha spoke in many ways in dispraise of being difficult to support and maintain, in dispraise of great desires, discontent, socializing, and laziness; but he spoke in many ways in praise of being easy to support and maintain, of fewness of wishes, contentment, self-effacement, ascetic practices, serenity, reduction in things, and being energetic. After giving a teaching on what is right and proper, he addressed the monks: 

“Well\marginnote{1.51} then, monks, I will lay down a training rule for the following ten reasons: for the well-being of the Sangha, for the comfort of the Sangha, for the restraint of bad nuns, for the ease of good nuns, for the restraint of the corruptions relating to the present life, for the restraint of the corruptions relating to future lives, to give rise to confidence in those without it, to increase the confidence of those who have it, for the longevity of the true Teaching, and for supporting the training.\footnote{“Training” renders \textit{vinaya}. For a discussion of this word, see Appendix of Technical Terms. } 

And,\marginnote{1.53} monks, the nuns should recite this training rule like this: 

\subsection*{Final ruling }

\scrule{‘If a lustful nun consents to a lustful man making physical contact with her, to touching her, to taking hold of her, to contacting her, or to squeezing her, anywhere below the collar bone but above the knees, she too is expelled and excluded from the community. The training rule on above the knees.’”\footnote{For a detailed discussion of this rule, see Appendix on Individual \textsanskrit{Bhikkhunī} Rules. } }

\subsection*{Definitions }

\begin{description}%
\item[A: ] whoever, of such a kind, of such activity, of such caste, of such name, of such family, of such conduct, of such behavior, of such association, who is senior, who is junior, or who is of middle standing—this is called “a”. %
\item[Nun: ] she is a nun because she lives on alms; a nun because she has gone over to living on alms; a nun because she wears a patchwork cloth; a nun by convention; a nun on account of her claim; a “come, nun” nun; a nun given the full ordination by taking the three refuges; a good nun; a nun of substance; a trainee nun; a fully trained nun; a nun who has been given the full ordination in unanimity by both Sanghas through a legal procedure consisting of one motion and three announcements that is irreversible and fit to stand. The nun who has been given the full ordination in unanimity by both Sanghas through a legal procedure consisting of one motion and three announcements that is irreversible and fit to stand—this sort of nun is meant in this case. %
\item[Lustful: ] having lust, longing for, in love with. %
\item[Lustful: ] having lust, longing for, in love with. %
\item[Man: ] a human male, not a male spirit, not a male ghost, not a male animal. He understands and is capable of making physical contact. %
\item[Below the collar bone: ] down from the collar bone. %
\item[Above the knees: ] up from the knees. %
\item[Making physical contact: ] making mere physical contact. %
\item[Touching: ] touching here and there. %
\item[Taking hold of: ] the mere taking hold of. %
\item[Contacting: ] mere contacting. %
\item[Consents to squeezing: ] consents to the taking hold of a bodily part and then pressing. %
\item[She too: ] this is said with reference to the preceding offenses entailing expulsion. %
\item[Is expelled: ] just as a man with his head cut off is unable to continue living by reconnecting it to the body, so is a lustful nun who consents to a lustful man making physical contact with her, to touching her, to taking hold of her, to contacting her, or to squeezing her, anywhere below the collar bone but above the knees, not a monastic, not a daughter of the Sakyan. Therefore it is said “she is expelled.” %
\item[Excluded from the community: ] Community: joint legal procedures, a joint recitation, the same training—this is called “community”. She does not take part in this—therefore it is called “excluded from the community”. %
\end{description}

\subsection*{Permutations }

\subparagraph*{Both having lust: contact below the collar bone but above the knees }

If\marginnote{2.2.1} both have lust and either of them makes physical contact, below the collar bone but above the knees, body to body, she commits an offense entailing expulsion. If either of them, with their own body, makes physical contact with something connected to the other’s body, she commits a serious offense.\footnote{Something connected to the body means any contact that is not skin to skin, such as contact through clothes. That touching through clothes does not count as touching body to body is supported by two of the Vinayas in Chinese translation, namely, the Dharmaguptaska Vinaya and the \textsanskrit{Sarvāstivāda} Vinaya. } If either of them, with something connected to their own body, makes physical contact with the other’s body, she commits a serious offense. If either of them, with something connected to their own body, makes physical contact with something connected to the other’s body, she commits an offense of wrong conduct. 

If\marginnote{2.2.5} either of them, with something released by them, makes physical contact with the other’s body, she commits an offense of wrong conduct. If either of them, with something released by them, makes physical contact with something connected to the other’s body, she commits an offense of wrong conduct. If either of them, with something released by them, makes physical contact with something released by the other, she commits an offense of wrong conduct. 

\subparagraph*{Both having lust: contact above the collar bone or below the knees }

If\marginnote{2.2.8.1} either of them makes physical contact, above the collar bone or below the knees, body to body, she commits a serious offense. If either of them, with their own body, makes physical contact with something connected to the other’s body, she commits an offense of wrong conduct. If either of them, with something connected to their own body, makes physical contact with the other’s body, she commits an offense of wrong conduct. If either of them, with something connected to their own body, makes physical contact with something connected to the other’s body, she commits an offense of wrong conduct. 

If\marginnote{2.2.12} either of them, with something released by them, makes physical contact with the other’s body, she commits an offense of wrong conduct. If either of them, with something released by them, makes physical contact with something connected to the other’s body, she commits an offense of wrong conduct. If either of them, with something released by them, makes physical contact with something released by the other, she commits an offense of wrong conduct. 

\subparagraph*{Only the nun having lust: contact below the collar bone but above the knees }

If\marginnote{2.2.15.1} only the nun has lust and either of them makes physical contact, below the collar bone but above the knees, body to body, she commits a serious offense.\footnote{The Pali just says that, “One of them has lust”, \textit{ekatoavassute}, without specifying who. However, it seems reasonable that this should refer to the nun, since it is hard to imagine she would incur an offense if her mind were pure. This supposition is supported by Sp 2.662: \textit{Ekatoavassuteti ettha \textsanskrit{kiñcāpi} ekatoti avisesena \textsanskrit{vuttaṁ}, \textsanskrit{tathāpi} \textsanskrit{bhikkhuniyā} eva avassute sati \textsanskrit{ayaṁ} \textsanskrit{āpattibhedo} vuttoti veditabbo}, “‘One of them has lust’: here, although it is said ‘one of them’ without distinction, still it is to be understood that this offense is said to be incurred only when the nun has lust.” } If either of them, with their own body, makes physical contact with something connected to the other’s body, she commits an offense of wrong conduct. If either of them, with something connected to their own body, makes physical contact with the other’s body, she commits an offense of wrong conduct. If either of them, with something connected to their own body, makes physical contact with something connected to the other’s body, she commits an offense of wrong conduct. 

If\marginnote{2.2.19} either of them, with something released by them, makes physical contact with the other’s body, she commits an offense of wrong conduct. If either of them, with something released by them, makes physical contact with something connected to the other’s body, she commits an offense of wrong conduct. If either of them, with something released by them, makes physical contact with something released by the other, she commits an offense of wrong conduct. 

\subparagraph*{Only the nun having lust: contact above the collar bone or below the knees }

If\marginnote{2.2.22.1} either of them makes physical contact, above the collar bone or below the knees, body to body, she commits an offense of wrong conduct. If either of them, with their own body, makes physical contact with something connected to the other’s body, she commits an offense of wrong conduct. If either of them, with something connected to their own body, makes physical contact with the other’s body, she commits an offense of wrong conduct. If either of them, with something connected to their own body, makes physical contact with something connected to the other’s body, she commits an offense of wrong conduct. 

If\marginnote{2.2.26} either of them, with something released by them, makes physical contact with the other’s body, she commits an offense of wrong conduct. If either of them, with something released by them, makes physical contact with something connected to the other’s body, she commits an offense of wrong conduct. If either of them, with something released by them, makes physical contact with something released by the other, she commits an offense of wrong conduct. 

\subparagraph*{Both having lust: contact with other beings below the collar bone but above the knees }

If\marginnote{2.2.29.1} both have lust and she makes physical contact with a spirit, a ghost, a \textit{\textsanskrit{paṇḍaka}}, or an animal in human form, below the collar bone but above the knees, body to body, she commits a serious offense.\footnote{In this case, it seems implied by the genitive case ending for the various kinds of beings that it is only the nun who is making the contact. That the being in question is male is implied by the fact that there is no offense for touching a female human being. For a discussion of \textit{\textsanskrit{paṇḍaka}}, see Appendix of Technical Terms. } If she, with her own body, makes physical contact with something connected to their body, she commits an offense of wrong conduct. If she, with something connected to her own body, makes physical contact with their body, she commits an offense of wrong conduct. If she, with something connected to her own body, makes physical contact with something connected to their body, she commits an offense of wrong conduct. 

If\marginnote{2.2.33} she, with something released by her, makes physical contact with their body, she commits an offense of wrong conduct. If she, with something released by her, makes physical contact with something connected to their body, she commits an offense of wrong conduct. If she, with something released by her, makes physical contact with something released by them, she commits an offense of wrong conduct. 

\subparagraph*{Both having lust: contact with other beings above the collar bone or below the knees }

If\marginnote{2.2.36.1} she makes physical contact with them, above the collar bone or below the knees, body to body, she commits an offense of wrong conduct. If she, with her own body, makes physical contact with something connected to their body, she commits an offense of wrong conduct. If she, with something connected to her own body, makes physical contact with their body, she commits an offense of wrong conduct. If she, with something connected to her own body, makes physical contact with something connected to their body, she commits an offense of wrong conduct. 

If\marginnote{2.2.40} she, with something released by her, makes physical contact with their body, she commits an offense of wrong conduct. If she, with something released by her, makes physical contact with something connected to their body, she commits an offense of wrong conduct. If she, with something released by her, makes physical contact with something released by them, she commits an offense of wrong conduct. 

\subparagraph*{Only the nun having lust: contact with other beings below the collar bone but above the knees }

If\marginnote{2.2.43.1} only the nun has lust and she makes physical contact with them, below the collar bone but above the knees, body to body, she commits an offense of wrong conduct. If she, with her own body, makes physical contact with something connected to their body, she commits an offense of wrong conduct. If she, with something connected to her own body, makes physical contact with their body, she commits an offense of wrong conduct. If she, with something connected to her own body, makes physical contact with something connected to their body, she commits an offense of wrong conduct. 

If\marginnote{2.2.47} she, with something released by her, makes physical contact with their body, she commits an offense of wrong conduct. If she, with something released by her, makes physical contact with something connected to their body, she commits an offense of wrong conduct. If she, with something released by her, makes physical contact with something released by them, she commits an offense of wrong conduct. 

\subparagraph*{Only the nun having lust: contact with other beings above the collar bone or below the knees }

If\marginnote{2.2.50.1} she makes physical contact with them, above the collar bone or below the knees, body to body, she commits an offense of wrong conduct. If she, with her own body, makes physical contact with something connected to their body, she commits an offense of wrong conduct. If she, with something connected to her own body, makes physical contact with their body, she commits an offense of wrong conduct. If she, with something connected to her own body, makes physical contact with something connected to their body, she commits an offense of wrong conduct. 

If\marginnote{2.2.54} she, with something released by her, makes physical contact with their body, she commits an offense of wrong conduct. If she, with something released by her, makes physical contact with something connected to their body, she commits an offense of wrong conduct. If she, with something released by her, makes physical contact with something released by them, she commits an offense of wrong conduct. 

\subsection*{Non-offenses }

There\marginnote{2.3.1} is no offense: if it is unintentional;  if she is not mindful;  if she does not know;  if she does not consent;  if she is insane;  if she is deranged;  if she is overwhelmed by pain;  if she is the first offender. 

\scendsutta{The fifth offense entailing expulsion is finished.\footnote{The Pali says the “first offense”, but since I am including the four rules entailing expulsion that the nuns have in common with the monks, I get “fifth offense” instead. That this is the correct way of counting is confirmed by the word-commentary of the present rule, which states that the \textit{pi}, “too”, of \textit{ayampi} refers to the preceding \textit{\textsanskrit{pārājika}} rules. The equivalent adjustment is required for the next three rules entailing expulsion. } }

%
\section*{{\suttatitleacronym Bi Pj 6}{\suttatitletranslation The training rule on those who conceal offenses }{\suttatitleroot Vajjappaṭicchādikā}}
\addcontentsline{toc}{section}{\tocacronym{Bi Pj 6} \toctranslation{The training rule on those who conceal offenses } \tocroot{Vajjappaṭicchādikā}}
\markboth{The training rule on those who conceal offenses }{Vajjappaṭicchādikā}
\extramarks{Bi Pj 6}{Bi Pj 6}

\subsection*{Origin story }

At\marginnote{1.1} one time the Buddha was staying at \textsanskrit{Sāvatthī} in the Jeta Grove, \textsanskrit{Anāthapiṇḍika}’s Monastery. At that time the nun \textsanskrit{Sundarīnandā} was pregnant by \textsanskrit{Migāra}’s grandson \textsanskrit{Sāḷha}. When the fetus got large, she concealed her condition. And when the fetus was fully grown, she disrobed and gave birth. 

The\marginnote{1.5} nuns said to the nun \textsanskrit{Thullanandā}, “Venerable, \textsanskrit{Sundarīnandā} gave birth shortly after disrobing.” “Could it be that she was pregnant while she was still a nun?” 

“Yes,\marginnote{1.8} Venerables.” 

“But,\marginnote{1.9} Venerable, when you knew that a nun had committed an offense entailing expulsion, why didn’t you either confront her yourself or tell the community?”\footnote{“Confront” renders \textit{\textsanskrit{paṭicodesi}}. For a discussion of the closely related term \textit{codeti}, see Appendix of Technical Terms. } 

“Her\marginnote{1.10} disrepute is my disrepute, her infamy is my infamy, her notoriety is my notoriety, her loss is my loss. Why would I tell others of my own disrepute, infamy, notoriety, and loss?” 

The\marginnote{1.12} nuns of few desires complained and criticized her, “How could Venerable \textsanskrit{Thullanandā}, knowing that a nun had committed an offense entailing expulsion, neither confront her herself nor tell the community?” 

Then\marginnote{1.14} those nuns told the monks what had happened, and the monks in turn told the Buddha. Soon afterwards the Buddha had the Sangha gathered and questioned the monks: “Is it true, monks, that the nun \textsanskrit{Thullanandā} acted like this?” 

“It’s\marginnote{1.18} true, Sir.” 

The\marginnote{1.19} Buddha rebuked her … “How could the nun \textsanskrit{Thullanandā}, knowing that a nun had committed an offense entailing expulsion, neither confront her herself nor tell the community? This will affect people’s confidence …” … “And, monks, the nuns should recite this training rule like this: 

\subsection*{Final ruling }

\scrule{‘If a nun knows that a nun has committed an offense entailing expulsion, but she neither confronts her herself nor tells the community, and afterward—whether that nun remains or has died or has been expelled or has converted—she says, “Venerables, although I previously knew that this nun was like this, I thought, ‘I will neither confront her myself nor tell the community,’” she too is expelled and excluded from the community. The training rule on those who conceal offenses.’” }

\subsection*{Definitions }

\begin{description}%
\item[A: ] whoever … %
\item[Nun: ] … The nun who has been given the full ordination in unanimity by both Sanghas through a legal procedure consisting of one motion and three announcements that is irreversible and fit to stand—this sort of nun is meant in this case. %
\item[Knows: ] she knows by herself or others have told her or she has told her.\footnote{“She has told her” presumably means that the nun who has committed the offense has spoken about it. } %
\item[Has committed an offense entailing expulsion: ] she has committed any one of the eight offenses entailing expulsion. %
\item[She neither confronts her herself: ] she does not herself accuse her. %
\item[Nor tells the community: ] she does not tell other nuns. %
\item[Whether that nun remains or has died: ] Remains: what is meant is that she remains as a nun.\footnote{Vin-vn-\textsanskrit{ṭ} 1989: \textit{\textsanskrit{Saliṅge} tu \textsanskrit{ṭhitāyāti} \textsanskrit{pabbajjāliṅgeyeva} \textsanskrit{ṭhitāya}}, “For one remaining in the characteristic means: for one remaining in the characteristic of being gone forth.” } Has died: what is meant is that she has passed away. %
\item[Has been expelled: ] she has either disrobed herself or been expelled by others.\footnote{“Disrobed” renders \textit{\textsanskrit{vibbhantā}}. For a discussion of this word, see Appendix of Technical Terms. } %
\item[Has converted: ] what is meant is that she has joined another religious community. %
\item[Afterward she says, “Venerables, although I previously knew that this nun was like this, I thought, ‘I will neither confront her myself’”: ] “I won’t accuse her myself.” %
\item[“Nor tell the community”: ] “Nor tell other nuns.” %
\item[She too: ] this is said with reference to the preceding offenses entailing expulsion. %
\item[Is expelled: ] just as a fallen, withered leaf is incapable of becoming green again, so is a nun who knows that a nun has committed an offense entailing expulsion, but who thinks, “I will neither confront her myself nor tell the community,” by the mere fact of abandoning her duty, not a monastic, not a daughter of the Sakyan. Therefore it is said “she is expelled.” %
\item[Excluded from the community: ] Community: joint legal procedures, a joint recitation, the same training—this is called “community”. She does not take part in this—therefore it is called “excluded from the community”. %
\end{description}

\subsection*{Non-offenses }

There\marginnote{2.2.1} is no offense: if she does not tell because she thinks there will be quarrels or disputes in the Sangha;  if she does not tell because she thinks there will be a schism or fracture in the Sangha;  if she does not tell because she thinks the person she is telling about is cruel and harsh and that she might become a threat to life or the monastic life;  if she does not tell because she does not see any suitable nuns;  if she does not tell, but not because she wants to conceal;  if she does not tell because she thinks the other person will be known through her own actions;  if she is insane;  if she is deranged;  if she is overwhelmed by pain;  if she is the first offender. 

\scendsutta{The sixth offense entailing expulsion is finished. }

%
\section*{{\suttatitleacronym Bi Pj 7}{\suttatitletranslation The training rule on taking sides with one who has been ejected }{\suttatitleroot Ukkhittānuvattikā}}
\addcontentsline{toc}{section}{\tocacronym{Bi Pj 7} \toctranslation{The training rule on taking sides with one who has been ejected } \tocroot{Ukkhittānuvattikā}}
\markboth{The training rule on taking sides with one who has been ejected }{Ukkhittānuvattikā}
\extramarks{Bi Pj 7}{Bi Pj 7}

\subsection*{Origin story }

At\marginnote{1.1} one time when the Buddha was staying at \textsanskrit{Sāvatthī} in the Jeta Grove, \textsanskrit{Anāthapiṇḍika}’s Monastery, the nun \textsanskrit{Thullanandā} was taking sides with the monk \textsanskrit{Ariṭṭha}, an ex-vulture-killer, who had been ejected by a unanimous Sangha. 

The\marginnote{1.3} nuns of few desires complained and criticized her, “How can Venerable \textsanskrit{Thullanandā} take sides with the monk \textsanskrit{Ariṭṭha} who has been ejected by a unanimous Sangha?” … “Is it true, monks, that the nun \textsanskrit{Thullanandā} is doing this?” 

“It’s\marginnote{1.6} true, Sir.” 

The\marginnote{1.7} Buddha rebuked her … “How can the nun \textsanskrit{Thullanandā} take sides with the monk \textsanskrit{Ariṭṭha} who has been ejected by a unanimous Sangha? This will affect people’s confidence …” … “And, monks, the nuns should recite this training rule like this: 

\subsection*{Final ruling }

\scrule{‘If a nun takes sides with a monk who has been ejected by a unanimous Sangha—in accordance with the Teaching, the Monastic Law, and the Teacher’s instruction—and who is disrespectful, who has not made amends, and who has not made friends, the nuns should correct her like this: “Venerable, this monk has been ejected by a unanimous Sangha in accordance with the Teaching, the Monastic Law, and the Teacher’s instruction. He’s disrespectful, hasn’t made amends, and hasn’t made friends. Venerable, don’t take sides with this monk.” If that nun continues as before, the nuns should press her up to three times to make her stop. If she then stops, all is well. If she does not stop, she too is expelled and excluded from the community. The training rule on taking sides with one who has been ejected.’” }

\subsection*{Definitions }

\begin{description}%
\item[A: ] whoever … %
\item[Nun: ] … The nun who has been given the full ordination in unanimity by both Sanghas through a legal procedure consisting of one motion and three announcements that is irreversible and fit to stand—this sort of nun is meant in this case. %
\item[A unanimous Sangha: ] those belonging to the same Buddhist sect and staying within the same monastery zone.\footnote{For a discussion of the rendering “monastery zone” for \textit{\textsanskrit{sīmā}}, see Appendix of Technical Terms. } %
\item[Who has been ejected: ] who has been ejected for not recognizing an offense, for not making amends, or for not giving up a bad view. %
\item[In accordance with the Teaching, the Monastic Law: ] in accordance with that Teaching, in accordance with that Monastic Law.\footnote{Sp 2.669: \textit{\textsanskrit{Dhammenāti} \textsanskrit{bhūtena} \textsanskrit{vatthunā}. \textsanskrit{Vinayenāti} \textsanskrit{codetvā} \textsanskrit{sāretvā}. \textsanskrit{Padabhājanaṁ} panassa “yena dhammena yena vinayena ukkhitto suukkhitto \textsanskrit{hotī}”ti \textsanskrit{imaṁadhippāyamattaṁ} \textsanskrit{dassetuṁ} \textsanskrit{vuttaṁ}}, “\textit{Dhammena}: according to truth, according to the rule. \textit{Vinayena}: having accused, having reminded. But the word analysis is spoken to show just this meaning: ‘\textit{Yena dhammena yena vinayena} means ejected, properly ejected.’” } %
\item[In accordance with the Teacher’s instruction: ] in accordance with the Victor’s instruction, in accordance with the Buddha’s instruction. %
\item[Who is disrespectful: ] he does not heed the Sangha, groups of monks, individual monks, or legal procedures. %
\item[Who has not made amends: ] he has been ejected and not readmitted. %
\item[Who has not made friends: ] monks belonging to the same Buddhist sect is what is meant by “friends”. He is not together with them—therefore it is called “who has not made friends”. %
\item[Takes sides with: ] she has the same view, the same belief, the same persuasion as he does. %
\item[Her: ] that nun who supports one who has been ejected. %
\item[The nuns: ] other\marginnote{2.1.25} nuns who see it or hear about it. They should correct her like this: 

“Venerable,\marginnote{2.1.26} this monk has been ejected by a unanimous Sangha in accordance with the Teaching, the Monastic Law, and the Teacher’s instruction. He’s disrespectful, hasn’t made amends, and hasn’t made friends. Venerable, don’t take sides with this monk.” And they should correct her a second and a third time. 

If\marginnote{2.1.30} she stops, all is well. If she does not stop, she commits an offense of wrong conduct. If those who hear about it do not say anything, they commit an offense of wrong conduct. 

That\marginnote{2.1.33} nun, even if she has to be pulled into the midst of the Sangha, should be corrected like this: 

“Venerable,\marginnote{2.1.34} this monk has been ejected by a unanimous Sangha in accordance with the Teaching, the Monastic Law, and the Teacher’s instruction. He’s disrespectful, hasn’t made amends, and hasn’t made friends. Venerable, don’t take sides with this monk.” They should correct her a second and a third time. 

If\marginnote{2.1.38} she stops, all is well. If she does not stop, she commits an offense of wrong conduct. 

%
\item[Should press her: ] “And,\marginnote{2.1.41} monks, she should be pressed like this. A competent and capable nun should inform the Sangha: 

‘Please,\marginnote{2.1.43} Venerables, I ask the Sangha to listen. The nun so-and-so is taking sides with a monk who has been ejected by a unanimous Sangha—in accordance with the Teaching, the Monastic Law, and the Teacher’s instruction—and who is disrespectful, who has not made amends, and who has not made friends. And she keeps on doing it. If the Sangha is ready, it should press her to make her stop. This is the motion. 

Please,\marginnote{2.1.47} Venerables, I ask the Sangha to listen. The nun so-and-so is taking sides with a monk who has been ejected by a unanimous Sangha—in accordance with the Teaching, the Monastic Law, and the Teacher’s instruction—and who is disrespectful, who has not made amends, and who has not made friends. And she keeps on doing it. The Sangha presses her to make her stop. Any nun who approves of pressing her to make her stop should remain silent. Any nun who doesn’t approve should speak up. 

For\marginnote{2.1.53} the second time I speak on this matter … For the third time I speak on this matter … 

The\marginnote{2.1.55} Sangha has pressed nun so-and-so to stop. The Sangha approves and is therefore silent. I’ll remember it thus.’” 

After\marginnote{2.1.57} the motion, she commits an offense of wrong conduct.\footnote{The Pali just says \textit{\textsanskrit{dukkaṭa}}, without specifying that it is an \textit{\textsanskrit{āpatti}}, “an offense”. Yet elsewhere, such as at \href{https://suttacentral.net/pli-tv-bu-vb-ss10/en/brahmali\#2.65}{Bu Ss 10:2.65}, the \textit{\textsanskrit{dukkaṭa}} is annulled if you commit the full offense of \textit{\textsanskrit{saṅghādisesa}}. The implication is that in these contexts \textit{\textsanskrit{dukkaṭa}} should be read as \textit{\textsanskrit{āpatti} \textsanskrit{dukkaṭassa}}, “an offense of wrong conduct”. }  After each of the first two announcements, she commits a serious offense.  When the last announcement is finished, she commits an offense entailing expulsion. 

%
\item[She too: ] this is said with reference to the preceding offenses entailing expulsion. %
\item[Is expelled: ] just as an ordinary stone that has broken in half cannot be put together again, so is a nun who does not stop when pressed three times not a monastic, not a daughter of the Sakyan. Therefore it is said “she is expelled.” %
\item[Excluded from the community: ] Community: joint legal procedures, a joint recitation, the same training—this is called “community”. She does not take part in this—therefore it is called “excluded from the community”. %
\end{description}

\subsection*{Permutations }

If\marginnote{2.2.1} it is a legitimate legal procedure, and she perceives it as such, but she does not stop, she commits an offense entailing expulsion. If it is a legitimate legal procedure, but she is unsure of it, and she does not stop, she commits an offense entailing expulsion. If it is a legitimate legal procedure, but she perceives it as illegitimate, and she does not stop, she commits an offense entailing expulsion. 

If\marginnote{2.2.4} it is an illegitimate legal procedure, but she perceives it as legitimate, she commits an offense of wrong conduct. If it is an illegitimate legal procedure, but she is unsure of it, she commits an offense of wrong conduct. If it is an illegitimate legal procedure, and she perceives it as such, she commits an offense of wrong conduct. 

\subsection*{Non-offenses }

There\marginnote{2.3.1} is no offense: if she has not been pressed;  if she stops;  if she is insane; if she is deranged;  if she is overwhelmed by pain; if she is the first offender. 

\scendsutta{The seventh offense entailing expulsion is finished. }

%
\section*{{\suttatitleacronym Bi Pj 8}{\suttatitletranslation The training rule having eight parts }{\suttatitleroot Aṭṭhavatthukā}}
\addcontentsline{toc}{section}{\tocacronym{Bi Pj 8} \toctranslation{The training rule having eight parts } \tocroot{Aṭṭhavatthukā}}
\markboth{The training rule having eight parts }{Aṭṭhavatthukā}
\extramarks{Bi Pj 8}{Bi Pj 8}

\subsection*{Origin story }

At\marginnote{1.1} one time the Buddha was staying at \textsanskrit{Sāvatthī} in the Jeta Grove, \textsanskrit{Anāthapiṇḍika}’s Monastery. At that time the nuns from the group of six, being lustful and aiming to indulge in inappropriate sexual conduct, consented to lustful men holding their hands and the edge of their robes, and they stood with them, chatted with them, went to rendezvous with them, consented to men coming to them, entered covered places with them, and disposed their bodies for that purpose. 

The\marginnote{1.3} nuns of few desires complained and criticized them, “How can the nuns from the group of six do such things?” … “Is it true, monks, that those nuns do these things?” 

“It’s\marginnote{1.6} true, Sir.” 

The\marginnote{1.7} Buddha rebuked them … “How can the nuns from the group of six, being lustful and aiming to indulge in inappropriate sexual conduct, consent to lustful men holding their hands and the edge of their robes, and how can they stand with them, chat with them, go to rendezvous with them, consent to men coming to them, enter covered places with them, and dispose their bodies for that purpose? This will affect people’s confidence …” … “And, monks, the nuns should recite this training rule like this: 

\subsection*{Final ruling }

\scrule{‘If, for the purpose of indulging in inappropriate sexual conduct, a lustful nun consents to a lustful man holding her hand and the edge of her robe, and she stands with him and chats with him and goes to a rendezvous with him and consents to him coming to her and enters a covered place with him and disposes her body for him for that purpose, she too is expelled and excluded from the community. The training rule having eight parts.’”\footnote{For a discussion of my rendering of \textit{\textsanskrit{vā}} as “and”, see Appendix on Individual \textsanskrit{Bhikkhunī} Rules. For the rendering of \textit{\textsanskrit{saṅghāṭi}} as “robe”, see Appendix of Technical Terms. } }

\subsection*{Definitions }

\begin{description}%
\item[A: ] whoever … %
\item[Nun: ] … The nun who has been given the full ordination in unanimity by both Sanghas through a legal procedure consisting of one motion and three announcements that is irreversible and fit to stand—this sort of nun is meant in this case. %
\item[Lustful: ] having lust, longing for, in love with. %
\item[Lustful: ] having lust, longing for, in love with. %
\item[Man: ] a human male, not a male spirit, not a male ghost, not a male animal. He understands and is capable of making physical contact. %
\item[Consents to holding her hand: ] hand: from the elbow to the tip of the nails. If, for the purpose of indulging in inappropriate sexual conduct, she consents to him holding her above the collar bone or below the knees, she commits a serious offense. %
\item[And the edge of her robe: ] if, for the purpose of indulging in inappropriate sexual conduct, she consents to him holding her sarong or upper robe, she commits a serious offense. %
\item[And stands with him: ] if, for the purpose of indulging in inappropriate sexual conduct, she stands within arm’s reach of a man, she commits a serious offense. %
\item[And chats with him: ] if, for the purpose of indulging in inappropriate sexual conduct, she stands within arm’s reach of a man, chatting with him, she commits a serious offense. %
\item[And goes to a rendezvous with him: ] if, for the purpose of indulging in inappropriate sexual conduct, she goes to such-and-such a place when told by a man to do so, then for every step, she commits an offense of wrong conduct. For entering within arm’s reach of the man, she commits a serious offense. %
\item[And consents to him coming to her: ] if, for the purpose of indulging in inappropriate sexual conduct, she consents to a man coming to her, she commits an offense of wrong conduct. When he enters within arm’s reach, she commits a serious offense. %
\item[And enters a covered place with him: ] if, for the purpose of indulging in inappropriate sexual conduct, she enters a concealed place with any man, she commits a serious offense. %
\item[And disposes her body for him for that purpose: ] if, for the purpose of indulging in inappropriate sexual conduct, she disposes her body for a man while standing within arm’s reach of him, she commits a serious offense. %
\item[She too: ] this is said with reference to the preceding offenses entailing expulsion. %
\item[Is expelled: ] just as a palm tree with its crown cut off is incapable of further growth, so is a nun who fulfills the eight parts not a monastic, not a daughter of the Sakyan. Therefore it is said “she is expelled.”\footnote{The Pali actually reads “the eighth part”, rather than “the eight parts”. This, however, does not mean number eight in the list, but the last of the eight to be completed, whichever that is, and thus it implies the fulfillment of all eight. } %
\item[Excluded from the community: ] Community: joint legal procedures, a joint recitation, the same training—this is called “community”. She does not take part in this—therefore it is called “excluded from the community”. %
\end{description}

\subsection*{Non-offenses }

There\marginnote{2.2.1} is no offense: if it is unintentional;  if she is not mindful;  if she does not know;  if she does not consent;  if she is insane;  if she is deranged;  if she is overwhelmed by pain;  if she is the first offender. 

\scendsutta{The eighth offense entailing expulsion is finished. }

“Venerables,\marginnote{2.2.11} the eight rules on expulsion have been recited. If a nun commits any one of them, she no longer belongs to the community of nuns. As before, so after, she is expelled and excluded from the community. In regard to this I ask you, ‘Are you pure in this?’ A second time I ask, ‘Are you pure in this?’ A third time I ask, ‘Are you pure in this?’ You are pure in this and therefore silent. I’ll remember it thus.” 

\scendkanda{The chapter on offenses entailing expulsion in the Nuns’ Analysis is finished. }

%
\addtocontents{toc}{\let\protect\contentsline\protect\nopagecontentsline}
\chapter*{Suspension}
\addcontentsline{toc}{chapter}{\tocchapterline{Suspension}}
\addtocontents{toc}{\let\protect\contentsline\protect\oldcontentsline}

%
\section*{{\suttatitleacronym Bi Ss 1}{\suttatitletranslation The training rule on taking legal action }{\suttatitleroot Ussayavādikā}}
\addcontentsline{toc}{section}{\tocacronym{Bi Ss 1} \toctranslation{The training rule on taking legal action } \tocroot{Ussayavādikā}}
\markboth{The training rule on taking legal action }{Ussayavādikā}
\extramarks{Bi Ss 1}{Bi Ss 1}

Venerables,\marginnote{0.5} these seventeen rules on suspension come up for recitation. 

\subsection*{Origin story }

At\marginnote{1.1} one time the Buddha was staying at \textsanskrit{Sāvatthī} in the Jeta Grove, \textsanskrit{Anāthapiṇḍika}’s monastery. At that time a lay follower who had given a storehouse to the Sangha of nuns had died. He had two sons, one with and one without faith and confidence, and they divided their father’s property between them. Then the one without faith said to the other, “The storehouse is ours; let’s allocate it to one of us.” But the one with faith responded, “No, our father gave it to the Sangha of nuns.” 

A\marginnote{1.11} second time they both said the same thing, and a third time the one without faith repeated his proposal. The one with faith then thought, “If I get it, I too would give it to the Sangha of nuns,” and he said, “Alright, let’s allocate it.” 

But\marginnote{1.22} when they allocated it, it fell to the one without faith. He then went to the nuns and said, “Please leave, Venerables, this storehouse is mine.” 

The\marginnote{1.25} nun \textsanskrit{Thullanandā} said to him, “No, your father gave it to the Sangha of nuns.” 

Because\marginnote{1.27} they were unable to agree, they asked judges to decide on the matter. They said, “Venerable, who knows that it was given to the Sangha of nuns?” \textsanskrit{Thullanandā} replied, “But Sirs, didn’t you appoint a witness who saw or heard the giving of the gift?” Saying, “It’s true what the Venerable says,” the judges made the storehouse the property of the Sangha of nuns. 

The\marginnote{1.34} defeated man complained and criticized the nuns, “They’re not monastics these shaven-headed sluts. How could they take my storehouse?” \textsanskrit{Thullanandā} told the judges of this and they punished him. That man then made a dwelling place for \textsanskrit{Ājīvaka} ascetics not far from the nuns, inciting them to abuse the nuns. 

Once\marginnote{1.41} again \textsanskrit{Thullanandā} told the judges and the judges jailed him. People then complained and criticized those nuns, “First the nuns take his storehouse, then they have him punished, and then they have him jailed. Next they’ll have him executed!” 

Nuns\marginnote{1.46} heard the complaints of those people, and the nuns of few desires complained and criticized her, “How could Venerable \textsanskrit{Thullanandā} take legal action?” 

Then\marginnote{1.49} those nuns told the monks … “Is it true, monks, that the nun \textsanskrit{Thullanandā} is taking legal action?” 

“It’s\marginnote{1.51} true, Sir.” 

The\marginnote{1.52} Buddha rebuked her … “How could the nun \textsanskrit{Thullanandā} take legal action? This will affect people’s confidence …” … “And, monks, the nuns should recite this training rule like this: 

\subsection*{Final ruling }

\scrule{‘If a nun takes legal action against a householder or a householder’s offspring or a slave or a worker or even toward a monastic or a wanderer, then that nun has committed an immediate offense entailing sending away and suspension.’” }

\subsection*{Definitions }

\begin{description}%
\item[A: ] whoever … %
\item[Nun: ] … The nun who has been given the full ordination in unanimity by both Sanghas through a legal procedure consisting of one motion and three announcements that is irreversible and fit to stand—this sort of nun is meant in this case. %
\item[Takes legal action: ] what is meant is that she is the initiator of a lawsuit. %
\item[A householder: ] anyone who lives at home.\footnote{\textit{\textsanskrit{Agāraṁ}} is typically rendered as “in a house”. The problem with this is that it is not unallowable for a monastic to live in a building that is the equivalent of a house. What a monastic should not do is own a home and then live there. } %
\item[A householder’s offspring: ] whoever is an offspring or a sibling.\footnote{“Offspring” renders \textit{putta/\textsanskrit{ā}}, whereas “sibling” renders \textit{\textsanskrit{bhātaro}}. In Pali the male gender takes precedent if a group contains people of both sexes. For instance, the plural \textit{\textsanskrit{puttā}}, “sons”, may mean “children” or “offsping”, depending on the context. In the same way, the plural \textit{\textsanskrit{bhātāro}}, “brothers”, can mean “siblings”. This way of understanding male-gender nouns is confirmed in the introduction to the Pali lexical work the \textsanskrit{Abhidhānappadīpikāṭīkā}: \textit{Ettha hi \textsanskrit{mātā} ca \textsanskrit{pitā} ca pitaro, putto ca \textsanskrit{dhītā} ca \textsanskrit{puttā}, sassu ca sasuro ca \textsanskrit{sasurā}, \textsanskrit{bhātā} ca \textsanskrit{bhaginī} ca \textsanskrit{bhātaroti} \textsanskrit{bhinnaliṅgānampi} ekaseso dassitoti}, “Mother and father are fathers; son and daughter are sons; mother-in-law and father-in-law are fathers-in-law; brother and sister are brothers;’ in this case the split gender is shown with only one gender remaining.” The \textsanskrit{Abhidhānappadīpikāṭīkā} is available online at tipitaka.org. } %
\item[A slave: ] one born in the household, one who has been bought, one who has been brought back as a captive. %
\item[A worker: ] a paid worker, a servant. %
\item[A monastic or a wanderer: ] anyone who is a wanderer apart from Buddhist monks, nuns, trainee nuns, novice monks, and novice nuns. If, thinking, “I’ll initiate a lawsuit,” she looks for a companion or just goes there herself, she commits an offense of wrong conduct. If she tells one other person, she commits an offense of wrong conduct. If she tells a second person, she commits a serious offense. At the end of the lawsuit, she commits an offense entailing suspension. %
\item[An immediate offense: ] there is an offense as soon as the misconduct is committed, and no pressing is required. %
\item[Entailing sending away: ] she is sent away from the Sangha. %
\item[Suspension: ] only the Sangha gives the trial period for that offense, sends back to the beginning, and rehabilitates—not several nuns, not an individual nun. Therefore it is called an offense entailing suspension.\footnote{For a discussion of the rendering “several” for \textit{sambahula}, see Appendix of Technical Terms. } This is the name and designation of this class of offense. Therefore, too, it is called an offense entailing suspension. %
\end{description}

\subsection*{Non-offenses }

There\marginnote{2.2.1} is no offense: if she goes there because people pull her;  if she is asking for protection;  if she tells without specifying a person;  if she is insane; if she is deranged;  if she is overwhelmed by pain; if she is the first offender. 

\scendsutta{The first offense entailing suspension is finished. }

%
\section*{{\suttatitleacronym Bi Ss 2}{\suttatitletranslation The training rule on one who gives the full admission to a female criminal }{\suttatitleroot Corīvuṭṭhāpikā}}
\addcontentsline{toc}{section}{\tocacronym{Bi Ss 2} \toctranslation{The training rule on one who gives the full admission to a female criminal } \tocroot{Corīvuṭṭhāpikā}}
\markboth{The training rule on one who gives the full admission to a female criminal }{Corīvuṭṭhāpikā}
\extramarks{Bi Ss 2}{Bi Ss 2}

\subsection*{Origin story }

At\marginnote{1.1} one time the Buddha was staying at \textsanskrit{Sāvatthī} in the Jeta Grove, \textsanskrit{Anāthapiṇḍika}’s Monastery. At that time in \textsanskrit{Vesālī} the wife of a certain \textsanskrit{Licchavī} man was unfaithful. He said to her, “Please stop. If you don’t, I’ll punish you.” But she did not listen. 

Just\marginnote{1.6} then in \textsanskrit{Vesālī} the \textsanskrit{Licchavī} clan had gathered on some business. That \textsanskrit{Licchavī} man said to them, “Sirs, please give me permission in regard to one of my wives.” 

“What\marginnote{1.9} is it with her?” 

“She’s\marginnote{1.10} unfaithful. I wish to kill her.” 

“You\marginnote{1.11} may go ahead.”\footnote{\textit{\textsanskrit{Jānāhi}}, literally, “You know.” The implied meaning is not clear. However, in commenting on a different passage, one of the sub-commentaries, Sp-\textsanskrit{ṭ} 4.330, defines the word as follows: \textit{\textsanskrit{Jānāhīti} cettha \textsanskrit{paṭipajjāti} attho veditabbo}, “And here the meaning of ‘you know’ is to be understood as ‘you undertake’.” } 

When\marginnote{1.12} his wife heard that her husband wanted to kill her, she took their most valuable possessions and went to \textsanskrit{Sāvatthī}. There she went to the monastics of other religions and asked for the going forth, but they refused. She then went to the Buddhist nuns and again asked for the going forth, but they too refused. She then went to the nun \textsanskrit{Thullanandā}, showed her the goods, and once again asked for the going forth. \textsanskrit{Thullanandā} took the goods and gave her the going forth. 

That\marginnote{1.20} \textsanskrit{Licchavī} man then went to \textsanskrit{Sāvatthī} in search of his wife. When he saw that she had been given the going forth as a nun, he went to King Pasenadi of Kosala and said, “Sir, my wife took my most valuable possessions and came to \textsanskrit{Sāvatthī}. Please permit me to deal with her.” 

“Well\marginnote{1.24} then, find her and then inform me.” 

“I’ve\marginnote{1.25} seen her. She’s gone forth as a nun.” 

“If\marginnote{1.26} she’s gone forth as a nun, there’s nothing that can be done. The Teaching of the Buddha is well-proclaimed. Let her practice the spiritual life for the full ending of suffering.” 

Then\marginnote{1.28} that \textsanskrit{Licchavī} man complained and criticized the nuns, “How could the nuns give the going forth to a criminal?” 

The\marginnote{1.30} nuns heard the complaints of that \textsanskrit{Licchavī} man, and the nuns of few desires complained and criticized her, “How could Venerable \textsanskrit{Thullanandā} give the going forth to a criminal?” The nuns told the monks. … “Is it true, monks, that the nun \textsanskrit{Thullanandā} did this?” 

“It’s\marginnote{1.35} true, Sir.” 

The\marginnote{1.36} Buddha rebuked her … “How could the nun \textsanskrit{Thullanandā} give the going forth to a criminal? This will affect people’s confidence …” … “And, monks, the nuns should recite this training rule like this: 

\subsection*{Final ruling }

\scrule{‘If a nun, without getting permission from the king or the Sangha or a community or an association or a society, knowingly gives the full admission to a female criminal who is known as sentenced to death, then, except when it is allowable, that nun too has committed an immediate offense entailing sending away and suspension.’” }

\subsection*{Definitions }

\begin{description}%
\item[A: ] whoever … %
\item[Nun: ] … The nun who has been given the full ordination in unanimity by both Sanghas through a legal procedure consisting of one motion and three announcements that is irreversible and fit to stand—this sort of nun is meant in this case. %
\item[She knows: ] she knows by herself or others have told her or she has told her. %
\item[A female criminal: ] any female who has stolen anything worth five \textit{\textsanskrit{māsaka}} coins or more is called “a female criminal”. %
\item[Sentenced to death: ] she has been sentenced to death because of her action. %
\item[Is known: ] it is known to other people that she has been sentenced to death. %
\item[Without getting permission from: ] without having asked permission. %
\item[The king: ] where a king reigns, permission should be obtained from the king. %
\item[The Sangha: ] what is meant is the Sangha of nuns, and permission should be obtained from that Sangha. %
\item[A community: ] where a community governs, permission should be obtained from that community. %
\item[An association: ] where an association governs, permission should be obtained from that association. %
\item[A society: ] where a society governs, permission should be obtained from that society. %
\item[Except when it is allowable: ] unless it is allowable. %
\item[Allowable: ] there are two allowable situations: she has gone forth with monastics of another religion or she has gone forth with other Buddhist nuns. If, intending to give the full admission, she searches for a group, a teacher, a bowl, or a robe, or she establishes a monastery zone, then, except when it is allowable, she commits an offense of wrong conduct. After the motion, she commits an offense of wrong conduct.\footnote{The Pali just says \textit{\textsanskrit{dukkaṭa}}, without specifying that it is an \textit{\textsanskrit{āpatti}}, “an offense”. Yet elsewhere, such as at \href{https://suttacentral.net/pli-tv-bu-vb-ss10/en/brahmali\#2.65}{Bu Ss 10:2.65}, the \textit{\textsanskrit{dukkaṭa}} is annulled if you commit the full offense of \textit{\textsanskrit{saṅghādisesa}}. The implication is that in these contexts \textit{\textsanskrit{dukkaṭa}} should be read as \textit{\textsanskrit{āpatti} \textsanskrit{dukkaṭassa}}, “an offense of wrong conduct”. } After each of the first two announcements, she commits a serious offense. When the last announcement is finished, the preceptor commits an offense entailing suspension, and the group and the teacher commit an offense of wrong conduct. %
\item[That too: ] this is said with reference to the preceding offense. %
\item[An immediate offense: ] there is an offense as soon as the misconduct is committed, and no pressing is required. %
\item[Entailing sending away: ] she is sent away from the Sangha. %
\item[Suspension: ] … Therefore, too, it is called an offense entailing suspension. %
\end{description}

\subsection*{Permutations }

If\marginnote{2.2.1} she is a criminal, and the nun perceives her as such, and she gives her the full admission, except when it is allowable, she commits an offense entailing suspension. If she is a criminal, but the nun is unsure of it, and she gives her the full admission, except when it is allowable, she commits an offense of wrong conduct. If she is a criminal, but the nun does not perceive her as such, and she gives her the full admission, except when it is allowable, there is no offense. 

If\marginnote{2.2.4} she is not a criminal, but the nun perceives her as such, she commits an offense of wrong conduct. If she is not a criminal, but the nun is unsure of it, she commits an offense of wrong conduct. If she is not a criminal, and the nun does not perceive her as such, there is no offense. 

\subsection*{Non-offenses }

There\marginnote{2.3.1} is no offense: if she gives her the full admission without knowing that she is a criminal;  if she gives her the full admission after getting permission;  if she gives her the full admission when it is allowable;  if she is insane;  if she is the first offender. 

\scendsutta{The second offense entailing suspension is finished. }

%
\section*{{\suttatitleacronym Bi Ss 3}{\suttatitletranslation The training rule on walking alone to the next village }{\suttatitleroot Ekagāmantara}}
\addcontentsline{toc}{section}{\tocacronym{Bi Ss 3} \toctranslation{The training rule on walking alone to the next village } \tocroot{Ekagāmantara}}
\markboth{The training rule on walking alone to the next village }{Ekagāmantara}
\extramarks{Bi Ss 3}{Bi Ss 3}

\subsection*{Origin story }

\subsubsection*{First sub-story }

At\marginnote{1.1} one time when the Buddha was staying at \textsanskrit{Sāvatthī} in \textsanskrit{Anāthapiṇḍika}’s Monastery, a nun who was a pupil of \textsanskrit{Bhaddā} \textsanskrit{Kāpilānī} had an argument with the nuns and then went to her relatives’ village. Not seeing her pupil anywhere, \textsanskrit{Bhaddā} \textsanskrit{Kāpilānī} asked the nuns, “Where’s so-and-so? She’s disappeared.” 

“She\marginnote{1.4} disappeared, Venerable, after arguing with the nuns.” 

“My\marginnote{1.5} dears, her relatives live in such-and-such a village. Go there and look for her.” 

The\marginnote{1.6} nuns went there, and when they saw her, they said to her, “Why did you go alone, Venerable? We hope you weren’t assaulted?” 

“I\marginnote{1.7} wasn’t.” 

The\marginnote{1.8} nuns of few desires complained and criticized her, “How could a nun walk to the next village by herself?” … “Is it true, monks, that a nun did this?” 

“It’s\marginnote{1.11} true, Sir.” 

The\marginnote{1.12} Buddha rebuked her … “How could a nun do this? This will affect people’s confidence …” … “And, monks, the nuns should recite this training rule like this: 

\subsubsection*{First preliminary ruling }

\scrule{‘If a nun walks to the next inhabited area by herself, then that nun too has committed an immediate offense entailing sending away and suspension.’” }

In\marginnote{1.17} this way the Buddha laid down this training rule for the nuns. 

\subsubsection*{Second sub-story }

On\marginnote{2.1} one occasion two nuns were traveling from \textsanskrit{Sāketa} to \textsanskrit{Sāvatthī}. On the way they had to cross a river. They went to a boatman and said, “Please take us across.” 

“I’m\marginnote{2.4} not able, Venerables, to take both of you across at the same time.” And so they crossed individually, alone with the boatman. When he had crossed with the first nun, he raped her.\footnote{“Raped” renders \textit{\textsanskrit{dūsesi}}. For a discussion of this word, see Appendix of Technical Terms. } And after returning to the first bank, he raped the other nun as well. Later, when they were reunited, they asked each other, “Venerable, I hope you weren’t assaulted?” 

“I\marginnote{2.10} was. And you, Venerable, were you assaulted?” 

“I\marginnote{2.12} was, too.” 

They\marginnote{2.13} then continued on to \textsanskrit{Sāvatthī} and told the nuns there what had happened. The nuns of few desires complained and criticized them, “How could a nun cross a river by herself?” They told the monks, who in turn told the Buddha. Soon afterwards he had the Sangha gathered and questioned the monks: “Is it true, monks, that a nun did this?” 

“It’s\marginnote{2.19} true, Sir.” 

The\marginnote{2.20} Buddha rebuked them … “How could a nun do this? This will affect people’s confidence …” … “And, monks, the nuns should recite this training rule like this: 

\subsubsection*{Second preliminary ruling }

\scrule{‘If a nun walks to the next inhabited area by herself or crosses a river by herself, then that nun too has committed an immediate offense entailing sending away and suspension.’” }

In\marginnote{2.25} this way the Buddha laid down this training rule for the nuns. 

\subsubsection*{Third sub-story }

On\marginnote{3.1} one occasion a number of nuns were walking through the Kosalan country on their way to \textsanskrit{Sāvatthī}, when one evening they arrived at a certain village. One of the nuns was beautiful and graceful, and a certain man fell in love with her as soon as he saw her. Then, as he was preparing sleeping places for those nuns, he prepared hers to one side. And that nun thought, “This man is obsessed with me. If I go there for the night, I’ll get into trouble.” Then, without informing the nuns, she went to a certain family and slept there. 

When\marginnote{3.6} night arrived, that man went searching for that nun, and as he did so he bumped into the other nuns. Not seeing that nun anywhere, the nuns said, “No doubt she has left with a man.” 

The\marginnote{3.8} following morning that nun returned to the nuns, and they said to her, “Venerable, why did you leave with a man?” 

“I\marginnote{3.9} didn’t leave with a man, Venerables.” 

She\marginnote{3.10} then told the nuns what had happened. The nuns of few desires complained and criticized her, “How could a nun spend the night apart by herself?” … “Is it true, monks, that a nun did this?” 

“It’s\marginnote{3.14} true, Sir.” 

The\marginnote{3.15} Buddha rebuked her … “How could a nun do this? This will affect people’s confidence …” … “And, monks, the nuns should recite this training rule like this: 

\subsubsection*{Third preliminary ruling }

\scrule{‘If a nun walks to the next inhabited area by herself or crosses a river by herself or spends the night apart by herself, then that nun too has committed an immediate offense entailing sending away and suspension.’” }

In\marginnote{3.20} this way the Buddha laid down this training rule for the nuns. 

\subsubsection*{Fourth sub-story }

On\marginnote{4.1} one occasion a number of nuns were traveling through the Kosalan country on their way to \textsanskrit{Sāvatthī}. One of the nuns, needing to defecate, stayed behind by herself, and then followed behind the others. People saw her and raped her. She then went to the other nuns, and they said to her, “Why did you stay behind by yourself, Venerable? We hope you weren’t assaulted?” 

“I\marginnote{4.5} was.” 

The\marginnote{4.6} nuns of few desires complained and criticized her, “How could a nun lag behind her companions by herself?” … “Is it true, monks, that a nun did this?” 

“It’s\marginnote{4.9} true, Sir.” 

The\marginnote{4.10} Buddha rebuked her … “How could a nun do this? This will affect people’s confidence …” … “And, monks, the nuns should recite this training rule like this: 

\subsection*{Final ruling }

\scrule{‘If a nun walks to the next inhabited area by herself or crosses a river by herself or spends the night apart by herself or lags behind her companions by herself, then that nun too has committed an immediate offense entailing sending away and suspension.’” }

\subsection*{Definitions }

\begin{description}%
\item[A: ] whoever … %
\item[Nun: ] … The nun who has been given the full ordination in unanimity by both Sanghas through a legal procedure consisting of one motion and three announcements that is irreversible and fit to stand—this sort of nun is meant in this case. %
\item[Walks to the next inhabited area by herself: ] if she crosses the boundary of an enclosed inhabited area with her first foot, she commits a serious offense.\footnote{For a discussion of the rendering “inhabited area” for \textit{\textsanskrit{gāma}}, see Appendix of Technical Terms. }  If she then crosses it with her second foot, she commits an offense entailing suspension. If she enters the vicinity of an unenclosed inhabited area with her first foot, she commits a serious offense.\footnote{For a discussion of the rendering “vicinity” for \textit{\textsanskrit{upacāra}}, see Appendix of Technical Terms. } If she then enters it with her second foot, she commits an offense entailing suspension. %
\item[Or crosses a river by herself: ] A river: wherever, after covering the three circles, the sarong gets wet when the nun is crossing. When she has crossed with the first foot, she commits a serious offense. When she has crossed with the second foot, she commits an offense entailing suspension. %
\item[Or spends the night apart by herself: ] if, at dawn, she is in the process of going beyond arm’s reach of her companion nun, she commits a serious offense. When she has gone beyond, she commits an offense entailing suspension. %
\item[Or lags behind her companions by herself: ] if, in an uninhabited area, in the wilderness, she is in the process of going beyond the range of sight or the range of hearing of her companion nun, she commits a serious offense. When she has gone beyond, she commits an offense entailing suspension. %
\item[That too: ] this is said with reference to the preceding offenses. %
\item[An immediate offense: ] there is an offense as soon as the misconduct is committed, and no pressing is required. %
\item[Entailing sending away: ] she is sent away from the Sangha. %
\item[Suspension: ] … Therefore, too, it is called an offense entailing suspension. %
\end{description}

\subsection*{Non-offenses }

There\marginnote{5.2.1} is no offense: if her companion nun has left or disrobed or died or joined another group;\footnote{Sp 2.693 defines \textit{\textsanskrit{pakkhasaṅkantā}} as joining another religion: \textit{\textsanskrit{Pakkhasaṅkantā} \textsanskrit{vāti} \textsanskrit{titthāyatanaṁ} \textsanskrit{saṅkantā}}, “\textit{\textsanskrit{Pakkhasaṅkantā} \textsanskrit{vā}} means one who has joined the ascetics of another religion.” Yet the idea of \textit{pakkha} also refers to groups or factions within the Sangha, for instance, when the Sangha is split into different communities (\textit{\textsanskrit{nānāsaṁvāsa}}) that no longer perform legal procedures together. As such, it is a term for a separate sect of Buddhism. }  if there is an emergency;  if she is insane;  if she is the first offender. 

\scendsutta{The third offense entailing suspension is finished. }

%
\section*{{\suttatitleacronym Bi Ss 4}{\suttatitletranslation The training rule on readmitting one who has been ejected }{\suttatitleroot Ukkhittakaosāraṇa}}
\addcontentsline{toc}{section}{\tocacronym{Bi Ss 4} \toctranslation{The training rule on readmitting one who has been ejected } \tocroot{Ukkhittakaosāraṇa}}
\markboth{The training rule on readmitting one who has been ejected }{Ukkhittakaosāraṇa}
\extramarks{Bi Ss 4}{Bi Ss 4}

\subsection*{Origin story }

At\marginnote{1.1} one time the Buddha was staying at \textsanskrit{Sāvatthī} in the Jeta Grove, \textsanskrit{Anāthapiṇḍika}’s Monastery. At that time the nun \textsanskrit{Caṇḍakāḷī} was quarrelsome and argumentative, and she created legal issues in the Sangha. But when a legal procedure was being done against her, the nun \textsanskrit{Thullanandā} objected. 

Soon\marginnote{1.4} afterwards \textsanskrit{Thullanandā} went to a village on some business. The Sangha of nuns took the opportunity to eject \textsanskrit{Caṇḍakāḷī} for not recognizing an offense. When \textsanskrit{Thullanandā} had finished her business in that village, she returned to \textsanskrit{Sāvatthī}. On her return, \textsanskrit{Caṇḍakāḷī} neither prepared a seat for her, nor set out a foot stool, a foot scraper, or water for washing the feet; and she did not go out to meet her to receive her bowl and robe, nor ask whether she wanted water to drink. \textsanskrit{Thullanandā} asked her why she was acting like this. She replied, “That’s how it is, Venerable, when you don’t have a protector.” 

“But\marginnote{1.12} how is it, Venerable, that you don’t have a protector?” 

“When\marginnote{1.13} the nuns knew that no one would speak up for me because I am not esteemed by them and I didn’t have a protector, they ejected me for not recognizing an offense.” 

“They\marginnote{1.15} are incompetent fools. They don’t know about legal procedures or their flaws, nor what makes them fail or succeed. But we know all these things. We can get legal procedures done that haven’t been done, and we can get procedures that have been done overturned.” And she quickly gathered a sangha of nuns and readmitted the nun \textsanskrit{Caṇḍakāḷī}. 

The\marginnote{1.19} nuns of few desires complained and criticized her, “How could Venerable \textsanskrit{Thullanandā} readmit a nun who had been ejected by a unanimous Sangha in accordance with the Teaching, the Monastic Law, and the Teacher’s instruction, without first getting permission from the Sangha that did the legal procedure and without the consent of the community?” … “Is it true, monks, that the nun \textsanskrit{Thullanandā} did this?” 

“It’s\marginnote{1.22} true, Sir.” 

The\marginnote{1.23} Buddha rebuked her … “How could the nun \textsanskrit{Thullanandā} readmit a nun who had been ejected by a unanimous Sangha in accordance with the Teaching, the Monastic Law, and the Teacher’s instruction, without first getting permission from the Sangha that did the legal procedure and without the consent of the community? This will affect people’s confidence …” … “And, monks, the nuns should recite this training rule like this: 

\subsection*{Final ruling }

\scrule{‘If a nun, without getting permission from the Sangha that did the legal procedure and without the consent of the community, readmits a nun who has been ejected by a unanimous Sangha in accordance with the Teaching and the Monastic Law and the Teacher’s instruction, then that nun too has committed an immediate offense entailing sending away and suspension.’” }

\subsection*{Definitions }

\begin{description}%
\item[A: ] whoever … %
\item[Nun: ] … The nun who has been given the full ordination in unanimity by both Sanghas through a legal procedure consisting of one motion and three announcements that is irreversible and fit to stand—this sort of nun is meant in this case. %
\item[A unanimous Sangha: ] those belonging to the same Buddhist sect and staying within the same monastery zone. %
\item[Who has been ejected: ] who has been ejected for not recognizing an offense, for not making amends, or for not giving up a bad view. %
\item[In accordance with the Teaching and the Monastic Law: ] in accordance with that Teaching, in accordance with that Monastic Law.\footnote{Sp 2.669: \textit{\textsanskrit{Dhammenāti} \textsanskrit{bhūtena} \textsanskrit{vatthunā}. \textsanskrit{Vinayenāti} \textsanskrit{codetvā} \textsanskrit{sāretvā}. \textsanskrit{Padabhājanaṁ} panassa “yena dhammena yena vinayena ukkhitto suukkhitto \textsanskrit{hotī}”ti \textsanskrit{imaṁadhippāyamattaṁ} \textsanskrit{dassetuṁ} \textsanskrit{vuttaṁ}}, “\textit{Dhammena}: according to truth, according to the rule. \textit{Vinayena}: having accused, having reminded. But the word analysis is spoken to show just this meaning: ‘\textit{Yena dhammena yena vinayena} means ejected, properly ejected.’” } %
\item[In accordance with the Teacher’s instruction: ] in accordance with the Victor’s instruction, in accordance with the Buddha’s instruction. %
\item[Without getting permission from the Sangha that did the legal procedure: ] without having asked permission of the Sangha that did the legal procedure of ejection. %
\item[Without the consent of the community: ] without being aware of any consent from the community. If, intending to readmit her, she searches for a group or establishes a monastery zone, she commits an offense of wrong conduct. After the motion, she commits an offense of wrong conduct. After each of the first two announcements, she commits a serious offense. When the last announcement is finished, she commits an offense entailing suspension.\footnote{The Pali just says \textit{\textsanskrit{dukkaṭa}}, without specifying that it is an \textit{\textsanskrit{āpatti}}, “an offense”. Yet elsewhere, such as at \href{https://suttacentral.net/pli-tv-bu-vb-ss10/en/brahmali\#2.65}{Bu Ss 10:2.65}, the \textit{\textsanskrit{dukkaṭa}} is annulled if you commit the full offense of \textit{\textsanskrit{saṅghādisesa}}. The implication is that in these contexts \textit{\textsanskrit{dukkaṭa}} should be read as \textit{\textsanskrit{āpatti} \textsanskrit{dukkaṭassa}}, “an offense of wrong conduct”. } %
\item[That too: ] this is said with reference to the preceding offenses. %
\item[An immediate offense: ] there is an offense as soon as the misconduct is committed, and no pressing is required. %
\item[Entailing sending away: ] she is sent away from the Sangha. %
\item[Suspension: ] … Therefore, too, it is called an offense entailing suspension. %
\end{description}

\subsection*{Permutations }

If\marginnote{2.2.1} it is a legitimate legal procedure, and she perceives it as such, and she readmits her, she commits an offense entailing suspension. If it is a legitimate legal procedure, but she is unsure of it, and she readmits her, she commits an offense entailing suspension. If it is a legitimate legal procedure, but she perceives it as illegitimate, and she readmits her, she commits an offense entailing suspension. 

If\marginnote{2.2.4} it is an illegitimate legal procedure, but she perceives it as legitimate, she commits an offense of wrong conduct. If it is an illegitimate legal procedure, but she is unsure of it, she commits an offense of wrong conduct. If it is an illegitimate legal procedure, and she perceives it as such, she commits an offense of wrong conduct. 

\subsection*{Non-offenses }

There\marginnote{2.3.1} is no offense: if she readmits her after getting permission from the Sangha that did the procedure;  if she readmits her with the consent of the community;  if she readmits one who is behaving properly;  if she readmits her when the Sangha that did the procedure is unavailable;\footnote{\textit{Asanta} can mean “non-existent”, but in this kind of context it often means “unavailable”. See for instance \href{https://suttacentral.net/pli-tv-bu-vb-pc46/en/brahmali\#6.1.11}{Bu Pc 46:6.1.11}, \href{https://suttacentral.net/pli-tv-bu-vb-pc85/en/brahmali\#5.1.7}{Bu Pc 85:5.1.7} and \href{https://suttacentral.net/pli-tv-bi-vb-pc51/en/brahmali\#4.3.3}{Bi Pc 51:4.3.3}. }  if she is insane;  if she is the first offender. 

\scendsutta{The fourth offense entailing suspension is finished. }

%
\section*{{\suttatitleacronym Bi Ss 5}{\suttatitletranslation The training rule on receiving food }{\suttatitleroot Avassutāavassutassa}}
\addcontentsline{toc}{section}{\tocacronym{Bi Ss 5} \toctranslation{The training rule on receiving food } \tocroot{Avassutāavassutassa}}
\markboth{The training rule on receiving food }{Avassutāavassutassa}
\extramarks{Bi Ss 5}{Bi Ss 5}

\subsection*{Origin story }

At\marginnote{1.1} one time the Buddha was staying at \textsanskrit{Sāvatthī} in the Jeta Grove, \textsanskrit{Anāthapiṇḍika}’s Monastery. At that time there was a beautiful nun called \textsanskrit{Sundarīnandā}. When people saw her in the dining hall, they gave her the best food, both the donors and the recipient having lust. \textsanskrit{Sundarīnandā} ate as much as she liked, but not so the other nuns. 

The\marginnote{1.6} nuns of few desires complained and criticized her, “How could Venerable \textsanskrit{Sundarīnandā}, being lustful, eat either fresh or cooked food after receiving it directly from a lustful man?” … “Is it true, monks, that the nun \textsanskrit{Sundarīnandā} did this?” 

“It’s\marginnote{1.9} true, Sir.” 

The\marginnote{1.10} Buddha rebuked her … “How could \textsanskrit{Sundarīnandā}, being lustful, eat either fresh or cooked food after receiving it directly from a lustful man? This will affect people’s confidence …” … “And, monks, the nuns should recite this training rule like this: 

\subsection*{Final ruling }

\scrule{‘If a lustful nun eats fresh or cooked food after receiving it directly from a lustful man, then that nun too has committed an immediate offense entailing sending away and suspension.’” }

\subsection*{Definitions }

\begin{description}%
\item[A: ] whoever … %
\item[Nun: ] … The nun who has been given the full ordination in unanimity by both Sanghas through a legal procedure consisting of one motion and three announcements that is irreversible and fit to stand—this sort of nun is meant in this case. %
\item[Lustful: ] having lust, longing for, in love with. %
\item[Lustful: ] having lust, longing for, in love with. %
\item[Man: ] a human male, not a male spirit, not a male ghost, not a male animal. He understands and is capable of having lust. %
\item[Fresh food: ] apart from the five cooked foods, water, and tooth cleaners, the rest is called “fresh food”.\footnote{For a discussion of the rendering “fresh food” for \textit{\textsanskrit{khādanīya}}, see Appendix of Technical Terms. } %
\item[Cooked food: ] there are five kinds of cooked food: cooked grain, porridge, flour products, fish, and meat.\footnote{For a discussion of the renderings “cooked food” for \textit{\textsanskrit{bhojanīya}} and “flour products” for \textit{sattu}, see Appendix of Technical Terms. } If she receives fresh or cooked food with the intention of eating it, she commits a serious offense. For every mouthful, she commits an offense entailing suspension. %
\item[That too: ] this is said with reference to the preceding offenses. %
\item[An immediate offense: ] there is an offense as soon as the misconduct is committed, and no pressing is required. %
\item[Entailing sending away: ] she is sent away from the Sangha. %
\item[Suspension: ] … Therefore, too, it is called an offense entailing suspension. %
\end{description}

\subsection*{Permutations }

If\marginnote{2.2.1} she receives water or a tooth cleaner, she commits an offense of wrong conduct. 

If\marginnote{2.2.2} only the man has lust, and she receives fresh or cooked food with the intention of eating it, she commits an offense of wrong conduct.\footnote{As with  \href{https://suttacentral.net/pli-tv-bi-vb-pj5/en/brahmali\#2.2.15.1}{Bi Pj 5:2.2.15.1} above, the Pali just says that, “One of them has lust”, \textit{ekatoavassute}, without specifying who. In this rule, however, the non-offense clause specifically says that there is no offense if the nun knows that the man does not have lust, and so the state of mind of the nun does not seem to be an issue. It follows that “one of them has lust” must refer to the man. This understanding is supported by Sp 2.701: \textit{Ekato avassuteti ettha “\textsanskrit{bhikkhuniyā} \textsanskrit{avassutabhāvo} \textsanskrit{daṭṭhabbo}”ti \textsanskrit{mahāpaccariyaṁ} \textsanskrit{vuttaṁ}. \textsanskrit{Mahāaṭṭhakathāyaṁ} \textsanskrit{panetaṁ} na \textsanskrit{vuttaṁ}, \textsanskrit{taṁ} \textsanskrit{pāḷiyā} sameti}, “‘One of them has lust’: In regard to this it is said in the \textsanskrit{Mahāpaccarī}, ‘There being lust in the nun is to be understood.’ But this is not said in the \textsanskrit{Mahāaṭṭhakathā}, and this agrees with the Canonical text.” } 

For\marginnote{2.2.3} every mouthful, she commits a serious offense. If she receives water or a tooth cleaner, she commits an offense of wrong conduct. 

If\marginnote{2.2.5} both of them have lust, and she receives fresh or cooked food with the intention of eating it directly from a male spirit, a male ghost, a \textit{\textsanskrit{paṇḍaka}}, or a male animal in human form, she commits an offense of wrong conduct. For every mouthful, she commits a serious offense. If she receives water or a tooth cleaner, she commits an offense of wrong conduct. 

If\marginnote{2.2.8} only the male being has lust, and she receives fresh or cooked food with the intention of eating it, she commits an offense of wrong conduct. For every mouthful, she commits an offense of wrong conduct. If she receives water or a tooth cleaner, she commits an offense of wrong conduct. 

\subsection*{Non-offenses }

There\marginnote{2.3.1} is no offense: if both are without lust;  if she receives, knowing that the man has no lust;  if she is insane;  if she is the first offender. 

\scendsutta{The fifth offense entailing suspension is finished. }

%
\section*{{\suttatitleacronym Bi Ss 6}{\suttatitletranslation The second training rule on receiving food }{\suttatitleroot Kiṁteavassutovāanavassutovā}}
\addcontentsline{toc}{section}{\tocacronym{Bi Ss 6} \toctranslation{The second training rule on receiving food } \tocroot{Kiṁteavassutovāanavassutovā}}
\markboth{The second training rule on receiving food }{Kiṁteavassutovāanavassutovā}
\extramarks{Bi Ss 6}{Bi Ss 6}

\subsection*{Origin story }

At\marginnote{1.1} one time the Buddha was staying at \textsanskrit{Sāvatthī} in the Jeta Grove, \textsanskrit{Anāthapiṇḍika}’s Monastery. At that time there was a beautiful nun called \textsanskrit{Sundarīnandā}. When people saw her in the dining hall, they were affected by lust, and they gave her the best food. Being afraid of wrongdoing, \textsanskrit{Sundarīnandā} did not receive it. The nun next to her said, “Why didn’t you receive it, Venerable?” 

“Because\marginnote{1.7} they have lust.” 

“But\marginnote{1.8} do you have lust?” 

“No.”\marginnote{1.9} 

“What\marginnote{1.10} can this man do to you, whether he has lust or not, when you’re without it? Go on, Venerable, receive it with your own hands, and eat whatever fresh or cooked food he gives you.” 

The\marginnote{1.12} nuns of few desires complained and criticized her, “How could a nun say, ‘What can this man do to you, whether he has lust or not, when you’re without it? Go on, Venerable, receive it with your own hands, and eat whatever fresh or cooked food he gives you’?” … “Is it true, monks, that a nun said this?” 

“It’s\marginnote{1.19} true, Sir.” 

The\marginnote{1.20} Buddha rebuked her … “How could a nun say this?” This will affect people’s confidence …” … “And, monks, the nuns should recite this training rule like this: 

\subsection*{Final ruling }

\scrule{‘If a nun says, “Venerable, what can this man do to you, whether he has lust or not, if you’re without? Go on, Venerable, receive it with your own hands and then eat whatever fresh or cooked food he gives you,” then that nun too has committed an immediate offense entailing sending away and suspension.’” }

\subsection*{Definitions }

\begin{description}%
\item[A: ] whoever … %
\item[Nun: ] … The nun who has been given the full ordination in unanimity by both Sanghas through a legal procedure consisting of one motion and three announcements that is irreversible and fit to stand—this sort of nun is meant in this case. %
\item[Says: ] if she urges her on, saying, “Venerable, what can this man do to you, whether he has lust or not, if you’re without? Go on, Venerable, receive it with your own hands and then eat whatever fresh or cooked food he gives you,” then she commits an offense of wrong conduct. If, because of her statement, the other nun receives fresh or cooked food with the intention of eating it, she commits an offense of wrong conduct. For every mouthful, she commits a serious offense. At the end of the meal, she commits an offense entailing suspension. %
\item[That too: ] this is said with reference to the preceding offenses. %
\item[An immediate offense: ] there is an offense as soon as the misconduct is committed, and no pressing is required. %
\item[Entailing sending away: ] she is sent away from the Sangha. %
\item[Suspension: ] … Therefore, too, it is called an offense entailing suspension. %
\end{description}

\subsection*{Permutations }

If\marginnote{2.2.1} she urges her on to receive water or a tooth cleaner, she commits an offense of wrong conduct. If, because of her statement, the other nun receives it with the intention of eating it, she commits an offense of wrong conduct. 

If,\marginnote{2.2.3} when only the male being has lust, she urges her on to eat fresh or cooked food received directly from a male spirit, a male ghost, a \textit{\textsanskrit{paṇḍaka}}, or a male animal in human form, she commits an offense of wrong conduct.\footnote{As with \href{https://suttacentral.net/pli-tv-bi-vb-pj5/en/brahmali\#2.2.15.1}{Bi Pj 5:2.2.15.1} and \href{https://suttacentral.net/pli-tv-bi-vb-ss5/en/brahmali\#2.2.2}{Bi Ss 5:2.2.2}, the Pali just says that “One of them has lust”, \textit{ekatoavassute}, without specifying who. Yet as with Bi Ss 5, the non-offense clause specifically says that there is no offense if the nun knows that the man or the male being does not have lust, and so the state of mind of the nun does not seem to be an issue. It follows that “one of them has lust” must refer to the male being. This understanding is supported by Sp 2.701: \textit{Ekato avassuteti ettha “\textsanskrit{bhikkhuniyā} \textsanskrit{avassutabhāvo} \textsanskrit{daṭṭhabbo}”ti \textsanskrit{mahāpaccariyaṁ} \textsanskrit{vuttaṁ}. \textsanskrit{Mahāaṭṭhakathāyaṁ} \textsanskrit{panetaṁ} na \textsanskrit{vuttaṁ}, \textsanskrit{taṁ} \textsanskrit{pāḷiyā} sameti}, “‘One of them has lust’: In regard to this it is said in the \textsanskrit{Mahāpaccarī}, ‘There being lust in the nun is to be understood.’ But this is not said in the \textsanskrit{Mahāaṭṭhakathā}, and this agrees with the Canonical text.” } If, because of her statement, the other nun receives it with the intention of eating it, she commits an offense of wrong conduct. For every mouthful, she commits an offense of wrong conduct. At the end of the meal, she commits a serious offense. If she urges her on to receive water or a tooth cleaner, she commits an offense of wrong conduct. If, because of her statement, the other nun receives it with the intention of eating it, she commits an offense of wrong conduct. 

\subsection*{Non-offenses }

There\marginnote{2.3.1} is no offense: if she urges her on, knowing that he has no lust;  if she urges her on, thinking that she is not receiving because of anger;  if she urges her on, thinking that she is not receiving because of compassion for the family;  if she is insane;  if she is the first offender. 

\scendsutta{The sixth offense entailing suspension is finished.\footnote{The nuns’ offenses entailing suspension 7, 8, and 9 are identical to \href{https://suttacentral.net/pli-tv-bu-vb-ss5/en/brahmali\#2.2.13.1}{Bu Ss 5:2.2.13.1}, \href{https://suttacentral.net/pli-tv-bu-vb-ss8/en/brahmali\#1.9.32.1}{Bu Ss 8:1.9.32.1}, and \href{https://suttacentral.net/pli-tv-bu-vb-ss9/en/brahmali\#1.2.14.1}{Bu Ss 9:1.2.14.1} respectively. } }

%
\section*{{\suttatitleacronym Bi Ss 10}{\suttatitletranslation The training rule on renouncing the training }{\suttatitleroot Sikkhaṁpaccācikkhaṇa}}
\addcontentsline{toc}{section}{\tocacronym{Bi Ss 10} \toctranslation{The training rule on renouncing the training } \tocroot{Sikkhaṁpaccācikkhaṇa}}
\markboth{The training rule on renouncing the training }{Sikkhaṁpaccācikkhaṇa}
\extramarks{Bi Ss 10}{Bi Ss 10}

\subsection*{Origin story }

At\marginnote{1.1} one time when the Buddha was staying at \textsanskrit{Sāvatthī} in \textsanskrit{Anāthapiṇḍika}’s Monastery,\footnote{The nuns’ offenses entailing suspension 7, 8, and 9 are identical to the monks’ suspension 5 \href{https://suttacentral.net/pli-tv-bu-vb-ss5/en/brahmali\#2.2.13.1}{Bu Ss 5:2.2.13.1}, 8 \href{https://suttacentral.net/pli-tv-bu-vb-ss8/en/brahmali\#1.9.32.1}{Bu Ss 8:1.9.32.1}, and 9 \href{https://suttacentral.net/pli-tv-bu-vb-ss9/en/brahmali\#1.2.14.1}{Bu Ss 9:1.2.14.1} respectively. } the nun \textsanskrit{Caṇḍakāḷī} had argued with the nuns. In anger she said this: “I renounce the Buddha, I renounce the Teaching, I renounce the Sangha, I renounce the training! The Sakyan daughters are not the only monastics. There are other monastics who have a sense of conscience, who are afraid of wrongdoing and fond of the training. I’ll practice the spiritual life with them.”\footnote{According to Sp-\textsanskrit{ṭ} 2.709 \textit{\textsanskrit{kinnumāva}} should be read as \textit{\textsanskrit{kiṁ} nu \textsanskrit{imā} eva}. \textit{\textsanskrit{Kinnumāva} \textsanskrit{samaṇiyo} \textsanskrit{yā} \textsanskrit{samaṇiyo} \textsanskrit{sakyadhītaro}} might then be rendered quite literally as, “Those ascetics who are daughters of the Sakyan (\textit{\textsanskrit{yā} \textsanskrit{samaṇiyo} \textsanskrit{sakyadhītaro}}), why are just these ones ascetics (\textit{\textsanskrit{kinnumāva} \textsanskrit{samaṇiyo}})?” } 

The\marginnote{1.5} nuns of few desires complained and criticized her, “How could the nun \textsanskrit{Caṇḍakāḷī} say this in anger?” … “Is it true, monks, that the nun \textsanskrit{Caṇḍakāḷī} said this in anger?” 

“It’s\marginnote{1.12} true, Sir.” 

The\marginnote{1.13} Buddha rebuked her … “How could the nun \textsanskrit{Caṇḍakāḷī} say this in anger? This will affect people’s confidence …” … “And, monks, the nuns should recite this training rule like this: 

\subsection*{Final ruling }

\scrule{‘If a nun says in anger, “I renounce the Buddha, I renounce the Teaching, I renounce the Sangha, I renounce the training! The Sakyan daughters are not the only monastics. There are other monastics who have a sense of conscience, who are afraid of wrongdoing and fond of the training. I’ll practice the spiritual life with them,” then the nuns should correct her like this: “Venerable, don’t say such things in anger, ‘I renounce the Buddha, I renounce the Teaching, I renounce the Sangha, I renounce the training! The Sakyan daughters are not the only monastics. There are other monastics who have a sense of conscience, who are afraid of wrongdoing and fond of the training. I’ll practice the spiritual life with them.’ Take delight, Venerable; the Teaching is well proclaimed. Practice the spiritual life for the complete ending of suffering.” If that nun continues as before, the nuns should press her up to three times to make her stop. If she then stops, all is well. If she does not stop, then after the third announcement that nun too has committed an offense entailing sending away and suspension.’” }

\subsection*{Definitions }

\begin{description}%
\item[A: ] whoever … %
\item[Nun: ] … The nun who has been given the full ordination in unanimity by both Sanghas through a legal procedure consisting of one motion and three announcements that is irreversible and fit to stand—this sort of nun is meant in this case. %
\item[In anger: ] discontent, having hatred, hostile. %
\item[Says: ] “I renounce the Buddha, I renounce the Teaching, I renounce the Sangha, I renounce the training! The Sakyan daughters are not the only monastics. There are other monastics who have a sense of conscience, who are afraid of wrongdoing and fond of the training. I’ll practice the spiritual life with them.” %
\item[Her: ] that nun who speaks thus. %
\item[The nuns: ] other\marginnote{2.1.12} nuns who see it or hear about it. They should correct her like this: 

“Venerable,\marginnote{2.1.13} don’t say such things in anger: ‘I renounce the Buddha, I renounce the Teaching, I renounce the Sangha, I renounce the training! The Sakyan daughters are not the only monastics. There are other monastics who have a sense of conscience, who are afraid of wrongdoing and fond of the training. I’ll practice the spiritual life with them.’ Take delight, Venerable; the Teaching is well proclaimed. Practice the spiritual life for the complete ending of suffering.” 

And\marginnote{2.1.19} they should correct her a second and a third time. If she stops, all is well.  If she does not stop, she commits an offense of wrong conduct.  If those who hear about it do not say anything, they commit an offense of wrong conduct. 

That\marginnote{2.1.24} nun, even if she has to be pulled into the Sangha, should be corrected like this: 

“Venerable,\marginnote{2.1.25} don’t say such things in anger: ‘I renounce the Buddha, I renounce the Teaching, I renounce the Sangha, I renounce the training! The Sakyan daughters are not the only monastics. There are other monastics who have a sense of conscience, who are afraid of wrongdoing and fond of the training. I’ll practice the spiritual life with them.’ Take delight, Venerable; the Teaching is well proclaimed. Practice the spiritual life for the complete ending of suffering.” 

They\marginnote{2.1.31} should correct her a second and a third time. If she stops, all is well.  If she does not stop, she commits an offense of wrong conduct. 

%
\item[Should press her: ] “And,\marginnote{2.1.36} monks, she should be pressed like this. A competent and capable nun should inform the Sangha: 

‘Please,\marginnote{2.1.38} Venerables, I ask the Sangha to listen. The nun so-and-so says this in anger: “I renounce the Buddha, I renounce the Teaching, I renounce the Sangha, I renounce the training! The Sakyan daughters are not the only monastics. There are other monastics who have a sense of conscience, who are afraid of wrongdoing and fond of the training. I’ll practice the spiritual life with them.” And she keeps on saying it. If the Sangha is ready, it should press her to make her stop. This is the motion. 

Please,\marginnote{2.1.46} Venerables, I ask the Sangha to listen. The nun so-and-so says this in anger: “I renounce the Buddha, I renounce the Teaching, I renounce the Sangha, I renounce the training! The Sakyan daughters are not the only monastics. There are other monastics who have a sense of conscience, who are afraid of wrongdoing and fond of the training. I’ll practice the spiritual life with them.” And she keeps on saying it. The Sangha presses her to make her stop. Any nun who approves of pressing her to make her stop should remain silent. Any nun who doesn’t approve should speak up. 

For\marginnote{2.1.55} the second time I speak on this matter … For the third time I speak on this matter … 

The\marginnote{2.1.57} Sangha has pressed nun so-and-so to stop. The Sangha approves and is therefore silent. I’ll remember it thus.’” 

After\marginnote{2.1.59} the motion, she commits an offense of wrong conduct.\footnote{The Pali just says \textit{\textsanskrit{dukkaṭa}}, without specifying that it is an \textit{\textsanskrit{āpatti}}, an offense. Yet just below the text says that the \textit{\textsanskrit{dukkaṭa}} is annulled if you commit the full offense of \textit{\textsanskrit{saṅghādisesa}}. The implication is that \textit{\textsanskrit{dukkaṭa}} should be read as \textit{\textsanskrit{āpatti} \textsanskrit{dukkaṭassa}}, “an offense of wrong conduct”. } After each of the first two announcements, she commits a serious offense. When the last announcement is finished, she commits an offense entailing suspension. For one who commits the offense entailing suspension, the offense of wrong conduct and the serious offenses are annulled. 

%
\item[That too: ] this is said with reference to the preceding offenses. %
\item[After the third announcement: ] there is an offense when she has been pressed for the third time, not as soon as the misconduct has been committed. %
\item[Entailing sending away: ] she is sent away from the Sangha. %
\item[Suspension: ] … Therefore, too, it is called an offense entailing suspension. %
\end{description}

\subsection*{Permutations }

If\marginnote{2.2.1} it is a legitimate legal procedure, and she perceives it as such, and she does not stop, she commits an offense entailing suspension. If it is a legitimate legal procedure, but she is unsure of it, and she does not stop, she commits an offense entailing suspension. If it is a legitimate legal procedure, but she perceives it as illegitimate, and she does not stop, she commits an offense entailing suspension. 

If\marginnote{2.2.4} it is an illegitimate legal procedure, but she perceives it as legitimate, she commits an offense of wrong conduct. If it is an illegitimate legal procedure, but she is unsure of it, she commits an offense of wrong conduct. If it is an illegitimate legal procedure, and she perceives it as such, she commits an offense of wrong conduct. 

\subsection*{Non-offenses }

There\marginnote{2.3.1} is no offense: if she has not been pressed;  if she stops;  if she is insane;  if she is the first offender. 

\scendsutta{The tenth offense entailing suspension is finished.\footnote{The Pali says the “seventh offense”, but since I am including the rules that the nuns have in common with the monks in the count, I get “tenth offense” instead. The same adjustment is required for the next three rules. } }

%
\section*{{\suttatitleacronym Bi Ss 11}{\suttatitletranslation The training rule on being angry about a legal issue }{\suttatitleroot Adhikaraṇakupita}}
\addcontentsline{toc}{section}{\tocacronym{Bi Ss 11} \toctranslation{The training rule on being angry about a legal issue } \tocroot{Adhikaraṇakupita}}
\markboth{The training rule on being angry about a legal issue }{Adhikaraṇakupita}
\extramarks{Bi Ss 11}{Bi Ss 11}

\subsection*{Origin story }

At\marginnote{1.1} one time the Buddha was staying at \textsanskrit{Sāvatthī} in the Jeta Grove, \textsanskrit{Anāthapiṇḍika}’s Monastery. At that time the nun \textsanskrit{Caṇḍakāḷī} was angry that she had lost a legal case, saying, “The nuns are acting out of favoritism, ill will, confusion, and fear.” 

The\marginnote{1.4} nuns of few desires complained and criticized her, “How can Venerable \textsanskrit{Caṇḍakāḷī} say such things just because she’s angry that she has lost a legal case?” “Is it true, monks, that the nun \textsanskrit{Caṇḍakāḷī} says this because she’s angry?” 

“It’s\marginnote{1.9} true, Sir.” 

The\marginnote{1.10} Buddha rebuked her … “How can the nun \textsanskrit{Caṇḍakāḷī} say such things just because she’s angry that she has lost a legal case? This will affect people’s confidence …” … “And, monks, the nuns should recite this training rule like this: 

\subsection*{Final ruling }

\scrule{‘If a nun is angry because she has lost a legal case, saying, “The nuns are acting out of favoritism, ill will, confusion, and fear,”  then the nuns should correct her like this: “Venerable, just because you’re angry that you’ve lost a legal case, don’t say, ‘The nuns are acting out of favoritism, ill will, confusion, and fear.’ Perhaps it’s you who are acting out of favoritism, ill will, confusion, and fear.” If that nun continues as before, the nuns should press her up to three times to make her stop. If she then stops, all is well. If she does not stop, then after the third announcement that nun too has committed an offense entailing sending away and suspension.’” }

\subsection*{Definitions }

\begin{description}%
\item[A: ] whoever … %
\item[Nun: ] … The nun who has been given the full ordination in unanimity by both Sanghas through a legal procedure consisting of one motion and three announcements that is irreversible and fit to stand—this sort of nun is meant in this case. %
\item[A legal issue: ] there are four kinds of legal issues: legal issues arising from disputes, legal issues arising from accusations, legal issues arising from offenses, legal issues arising from business. %
\item[Has lost: ] what is meant is that she has been defeated. %
\item[Angry: ] discontent, having hatred, hostile. %
\item[Saying: ] “The nuns are acting out of favoritism … and fear.” %
\item[Her: ] that nun who speaks thus. %
\item[The nuns: ] other\marginnote{2.1.16} nuns who see it or hear about it. They should correct her like this: 

“Venerable,\marginnote{2.1.17} just because you’re angry that you’ve lost a legal case, don’t say, ‘The nuns are acting out of favoritism, ill will, confusion, and fear.’ Perhaps it’s you who are acting out of favoritism, ill will, confusion, and fear.” 

And\marginnote{2.1.20} they should correct her a second and a third time. If she stops, all is well. If she does not stop, she commits an offense of wrong conduct. If those who hear about it do not say anything, they commit an offense of wrong conduct. 

That\marginnote{2.1.25} nun, even if she has to be pulled into the Sangha, should be corrected like this: 

“Venerable,\marginnote{2.1.26} just because you’re angry that you’ve lost a legal case, don’t say, ‘The nuns are acting out of favoritism, ill will, confusion, and fear.’ Perhaps it’s you who are acting out of favoritism, ill will, confusion, and fear.” 

They\marginnote{2.1.29} should correct her a second and a third time. If she stops, all is well. If she does not stop, she commits an offense of wrong conduct. 

%
\item[Should press her: ] “And,\marginnote{2.1.34} monks, she should be pressed like this. A competent and capable nun should inform the Sangha: 

‘Please,\marginnote{2.1.36} Venerables, I ask the Sangha to listen. The nun so-and-so, because she’s angry that she has lost a legal case, says this:  “The nuns are acting out of favoritism, ill will, confusion, and fear.” And she keeps on saying it. If the Sangha is ready, it should press her to make her stop. This is the motion. 

Please,\marginnote{2.1.41} Venerables, I ask the Sangha to listen. The nun so-and-so, because she’s angry that she has lost a legal case, says this: “The nuns are acting out of favoritism, ill will, confusion, and fear.” And she keeps on saying it. The Sangha presses her to make her stop. Any nun who approves of pressing her to make her stop should remain silent. Any nun who doesn’t approve should speak up. 

For\marginnote{2.1.48} the second time I speak on this matter … For the third time I speak on this matter … 

The\marginnote{2.1.50} Sangha has pressed nun so-and-so to stop. The Sangha approves and is therefore silent. I’ll remember it thus.’” 

After\marginnote{2.1.52} the motion, she commits an offense of wrong conduct.\footnote{The Pali just says \textit{\textsanskrit{dukkaṭa}}, without specifying that it is an \textit{\textsanskrit{āpatti}}, an offense. Yet just below the text says that the \textit{\textsanskrit{dukkaṭa}} is annulled if you commit the full offense of \textit{\textsanskrit{saṅghādisesa}}. The implication is that \textit{\textsanskrit{dukkaṭa}} should be read as \textit{\textsanskrit{āpatti} \textsanskrit{dukkaṭassa}}, “an offense of wrong conduct”. } After each of the first two announcements, she commits a serious offense. When the last announcement is finished, she commits an offense entailing suspension. For one who commits the offense entailing suspension, the offense of wrong conduct and the serious offenses are annulled. 

%
\item[That too: ] this is said with reference to the preceding offenses. %
\item[After the third announcement: ] there is an offense when she has been pressed for the third time, not as soon as the misconduct has been committed. %
\item[Entailing sending away: ] she is sent away from the Sangha. %
\item[Suspension: ] … Therefore, too, it is called an offense entailing suspension. %
\end{description}

\subsection*{Permutations }

If\marginnote{2.2.1} it is a legitimate legal procedure, and she perceives it as such, and she does not stop, she commits an offense entailing suspension. If it is a legitimate legal procedure, but she is unsure of it, and she does not stop, she commits an offense entailing suspension. If it is a legitimate legal procedure, but she perceives it as illegitimate, and she does not stop, she commits an offense entailing suspension. 

If\marginnote{2.2.4} it is an illegitimate legal procedure, but she perceives it as legitimate, she commits an offense of wrong conduct. If it is an illegitimate legal procedure, but she is unsure of it, she commits an offense of wrong conduct. If it is an illegitimate legal procedure, and she perceives it as such, she commits an offense of wrong conduct. 

\subsection*{Non-offenses }

There\marginnote{2.3.1} is no offense: if she has not been pressed;  if she stops;  if she is insane;  if she is the first offender. 

\scendsutta{The eleventh offense entailing suspension is finished. }

%
\section*{{\suttatitleacronym Bi Ss 12}{\suttatitletranslation The training rule on bad behavior }{\suttatitleroot Saṁsaṭṭhā}}
\addcontentsline{toc}{section}{\tocacronym{Bi Ss 12} \toctranslation{The training rule on bad behavior } \tocroot{Saṁsaṭṭhā}}
\markboth{The training rule on bad behavior }{Saṁsaṭṭhā}
\extramarks{Bi Ss 12}{Bi Ss 12}

\subsection*{Origin story }

At\marginnote{1.1} one time the Buddha was staying at \textsanskrit{Sāvatthī} in the Jeta Grove, \textsanskrit{Anāthapiṇḍika}’s Monastery. At that time the nuns who were the pupils of the nun \textsanskrit{Thullanandā} were socializing and behaving badly, had a bad reputation, and were harassing the Sangha of nuns and hiding each other’s offenses. 

The\marginnote{1.3} nuns of few desires complained and criticized them, “How can nuns behave in this way?” … “Is it true, monks, that nuns are behaving like this?” 

“It’s\marginnote{1.6} true, Sir.” 

The\marginnote{1.7} Buddha rebuked them … “How can nuns behave in this way? This will affect people’s confidence …” … “And, monks, the nuns should recite this training rule like this: 

\subsection*{Final ruling }

\scrule{‘If nuns socialize, behave badly, have a bad reputation, are notorious, harass the Sangha of nuns, and hide each other’s offenses, then the nuns should correct them like this: “Sisters, you socialize, behave badly, have a bad reputation, are notorious, harass the Sangha of nuns, and hide each other’s offenses. Be secluded, Venerables. The Sangha praises seclusion for the Sisters.” If those nuns still continue as before, the nuns should press them up to three times to make them stop. If they then stop, all is well. If they do not stop, then after the third announcement those nuns too have committed an offense entailing sending away and suspension.’” }

\subsection*{Definitions }

\begin{description}%
\item[Nuns: ] what is meant is that they are fully ordained. %
\item[Socialize: ] they socialize with improper bodily and verbal action. %
\item[Behave badly: ] having bad behavior. %
\item[Have a bad reputation: ] a bad reputation has spread about them. %
\item[Are notorious: ] they make a living by means of a bad and wrong livelihood. %
\item[Harass the Sangha of nuns: ] they object on each other’s behalf when a legal procedure is being carried out against them. %
\item[Hide each other’s offenses: ] they hide one another’s offenses. %
\item[Them: ] those nuns who socialize. %
\item[The nuns: ] other\marginnote{2.1.18} nuns who see it or hear about it. They should correct them like this: 

“Sisters,\marginnote{2.1.19} you socialize, behave badly, have a bad reputation, are notorious, harass the Sangha of nuns, and hide each other’s offenses. Be secluded, Venerables. The Sangha praises seclusion for the Sisters.” 

And\marginnote{2.1.21} they should correct them a second and a third time. If they stop, all is well. If they do not stop, they commit an offense of wrong conduct. If those who hear about it do not say anything, they commit an offense of wrong conduct. 

Those\marginnote{2.1.26} nuns, even if they have to be pulled into the Sangha, should be corrected like this: 

“Sisters,\marginnote{2.1.27} you socialize, behave badly, have a bad reputation, are notorious, harass the Sangha of nuns, and hide each other’s offenses. Be secluded, Venerables. The Sangha praises seclusion for the Sisters.” 

They\marginnote{2.1.29} should correct them a second and a third time. If they stop, all is well. If they do not stop, they commit an offense of wrong conduct. 

%
\item[Should press them: ] “And,\marginnote{2.1.34} monks, they should be pressed like this. A competent and capable nun should inform the Sangha: 

‘Please,\marginnote{2.1.36} Venerables, I ask the Sangha to listen. The nuns so-and-so and so-and-so socialize, behave badly, have a bad reputation, are notorious, harass the Sangha of nuns, and hide each other’s offenses. And they keep on doing it. If the Sangha is ready, it should press them to make them stop. This is the motion. 

Please,\marginnote{2.1.41} Venerables, I ask the Sangha to listen. The nuns so-and-so and so-and-so socialize, behave badly, have a bad reputation, are notorious, harass the Sangha of nuns, and hide each other’s offenses. And they keep on doing it. The Sangha presses them to make them stop. Any nun who approves of pressing the nuns so-and-so and so-and-so to make them stop should remain silent. Any nun who doesn’t approve should speak up. 

For\marginnote{2.1.47} the second time I speak on this matter … For the third time I speak on this matter … 

The\marginnote{2.1.49} Sangha has pressed the nuns so-and-so and so-and-so to stop. The Sangha approves and is therefore silent. I’ll remember it thus.’” 

After\marginnote{2.1.51} the motion, they commit an offense of wrong conduct.\footnote{The Pali just says \textit{\textsanskrit{dukkaṭa}}, without specifying that it is an \textit{\textsanskrit{āpatti}}, an offense. Yet just below the text says that the \textit{\textsanskrit{dukkaṭa}} is annulled if you commit the full offense of \textit{\textsanskrit{saṅghādisesa}}. The implication is that \textit{\textsanskrit{dukkaṭa}} should be read as \textit{\textsanskrit{āpatti} \textsanskrit{dukkaṭassa}}, “an offense of wrong conduct”. }  After each of the first two announcements, they commit a serious offense.  When the last announcement is finished, they commit an offense entailing suspension.  If they commit the offense entailing suspension, the offense of wrong conduct and the serious offenses are annulled. Two or three may be pressed together, but not more than that. 

%
\item[Those nuns too: ] this is said with reference to the preceding offenses. %
\item[After the third announcement: ] there is an offense when they have been pressed for the third time, not as soon as the misconduct has been committed. %
\item[Entailing sending away: ] they are sent away from the Sangha. %
\item[Suspension: ] … Therefore, too, it is called an offense entailing suspension. %
\end{description}

\subsection*{Permutations }

If\marginnote{2.2.1} it is a legitimate legal procedure, and they perceive it as such, and they do not stop, they commit an offense entailing suspension. If it is a legitimate legal procedure, but they are unsure of it, and they do not stop, they commit an offense entailing suspension. If it is a legitimate legal procedure, but they perceive it as illegitimate, and they do not stop, they commit an offense entailing suspension. 

If\marginnote{2.2.4} it is an illegitimate legal procedure, but they perceive it as legitimate, they commit an offense of wrong conduct. If it is an illegitimate legal procedure, but they are unsure of it, they commit an offense of wrong conduct. If it is an illegitimate legal procedure, and they perceive it as such, they commit an offense of wrong conduct. 

\subsection*{Non-offenses }

There\marginnote{2.3.1} is no offense: if they have not been pressed;  if they stop;  if they are insane;  if they are the first offenders. 

\scendsutta{The twelfth offense entailing suspension is finished. }

%
\section*{{\suttatitleacronym Bi Ss 13}{\suttatitletranslation The second training rule on bad behavior }{\suttatitleroot Saṁsaṭṭhānuvattaka}}
\addcontentsline{toc}{section}{\tocacronym{Bi Ss 13} \toctranslation{The second training rule on bad behavior } \tocroot{Saṁsaṭṭhānuvattaka}}
\markboth{The second training rule on bad behavior }{Saṁsaṭṭhānuvattaka}
\extramarks{Bi Ss 13}{Bi Ss 13}

\subsection*{Origin story }

At\marginnote{1.1} one time the Buddha was staying at \textsanskrit{Sāvatthī} in the Jeta Grove, \textsanskrit{Anāthapiṇḍika}’s Monastery. At that time the nun \textsanskrit{Thullanandā} had been pressed by the Sangha. She then said to the nuns, “Venerables, you should socialize. Don’t live separately. There are other nuns in the Sangha who have such behavior, reputation, and notoriety,  and who harass the Sangha of nuns and hide each other’s offenses. The Sangha says nothing to them. It’s because of disrespect, contempt, impatience, and slander, and because you are weak that the Sangha says to you, ‘Sisters, you socialize, behave badly, have a bad reputation, are notorious, harass the Sangha of nuns, and hide each other’s offenses. Be secluded, Venerables. The Sangha praises seclusion for the Sisters.’”\footnote{In the sequence \textit{\textsanskrit{uññāya} paribhavena \textsanskrit{akkhantiyā} \textsanskrit{vebhassiyā} \textsanskrit{dubbalyā}} every word seems to be in the instrumental case except \textit{\textsanskrit{dubbalyā}}, which presumably is an ablative. This is unusual. The commentary, at Sp 2.727, says: \textit{\textsanskrit{Dubbalyāti} \textsanskrit{tumhākaṁ} \textsanskrit{dubbalabhāvena}}, “\textit{\textsanskrit{Dubbalyā}}: because of your weakness.” This implies that \textit{\textsanskrit{dubbalyā}} is the only word in the sequence that does not refer to the Sangha, but rather to the misbehaving nuns, which would be odd. Yet the commentarial interpretation is supported by the word commentary below, which defines \textit{\textsanskrit{dubbalyā}} as \textit{\textsanskrit{apakkhatā}} , “because of lack of supporters”. I therefore feel compelled to follow the commentarial interpretation. } 

The\marginnote{1.6} nuns of few desires complained and criticized her, “How could Venerable \textsanskrit{Thullanandā}, after being pressed by the Sangha, say this to the nuns?” … “Is it true, monks, that after being pressed by the Sangha, the nun \textsanskrit{Thullanandā} said this to the nuns?” 

“It’s\marginnote{1.15} true, Sir.” 

The\marginnote{1.16} Buddha rebuked her … “How could the nun \textsanskrit{Thullanandā}, after being pressed by the Sangha, say this to the nuns? This will affect people’s confidence …” … “And, monks, the nuns should recite this training rule like this: 

\subsection*{Final ruling }

\scrule{‘If a nun says, “Venerables, you should socialize. Don’t live separately. There are other nuns in the Sangha who have such behavior, reputation, and notoriety, and who harass the Sangha of nuns and hide each other’s offenses. The Sangha says nothing to them. It’s because of disrespect, contempt, impatience, and slander, and because you are weak that the Sangha says to you, ‘Sisters, you socialize, behave badly, have a bad reputation, are notorious, harass the Sangha of nuns, and hide each other’s offenses. Be secluded, Venerables. The Sangha praises seclusion for the Sisters,’” then the nuns should correct her like this: “Venerable, don’t say such things: ‘Venerables, you should socialize. Don’t live separately. There are other nuns in the Sangha who have such behavior, reputation, and notoriety, and who harass the Sangha of nuns and hide each other’s offenses. The Sangha says nothing to them. It’s because of disrespect, contempt, impatience, and slander, and because you are weak that the Sangha says to you, “Sisters, you socialize, behave badly, have a bad reputation, are notorious, harass the Sangha of nuns, and hide each other’s offenses. Be secluded, Venerables. The Sangha praises seclusion for the Sisters.”’” If that nun continues as before, the nuns should press her up to three times to make her stop. If she then stops, all is well. If she does not stop, then after the third announcement that nun too has committed an offense entailing sending away and suspension.’” }

\subsection*{Definitions }

\begin{description}%
\item[A: ] whoever … %
\item[Nun: ] … The nun who has been given the full ordination in unanimity by both Sanghas through a legal procedure consisting of one motion and three announcements that is irreversible and fit to stand—this sort of nun is meant in this case. %
\item[Says: ] “Venerables, you should socialize. Don’t live separately. There are other nuns in the Sangha who have such behavior, reputation, and notoriety, and who harass the Sangha of nuns and hide each other’s offenses. The Sangha says nothing to them.” %
\item[It’s because of disrespect that the Sangha … to you: ] because of despising. %
\item[Because of contempt: ] because of contemptuousness. %
\item[Because of impatience: ] because of irritation. %
\item[Because of slander: ] because of slandering. %
\item[Because … weak: ] because of lack of supporters. %
\item[Says: ] “Sisters, you socialize, behave badly, have a bad reputation, are notorious, harass the Sangha of nuns, and hide each other’s offenses. Be secluded, Venerables. The Sangha praises seclusion for the Sisters.” %
\item[Her: ] that nun who speaks thus. %
\item[The nuns: ] other\marginnote{2.24} nuns who see it or hear about it. They should correct her like this: 

“Venerable,\marginnote{2.25} don’t say such things: ‘Venerables, you should socialize. Don’t live separately. There are other nuns in the Sangha … “… Be secluded, Venerables. The Sangha praises seclusion for the Sisters.”’” 

And\marginnote{2.29} they should correct her a second and a third time. If she stops, all is well. If she does not stop, she commits an offense of wrong conduct. If those who hear about it do not say anything, they commit an offense of wrong conduct. 

That\marginnote{2.34} nun, even if she has to be pulled into the Sangha, should be corrected like this: 

“Venerable,\marginnote{2.35} don’t say such things: ‘Venerables, you should socialize. Don’t live separately. There are other nuns in the Sangha … “… Be secluded, Venerables. The Sangha praises seclusion for the Sisters.”’” 

They\marginnote{2.39} should correct her a second and a third time. If she stops, all is well. If she does not stop, she commits an offense of wrong conduct. 

%
\item[Should press her: ] “And,\marginnote{2.44} monks, she should be pressed like this. A competent and capable nun should inform the Sangha: 

‘Please,\marginnote{2.46} Venerables, I ask the Sangha to listen. The nun so-and-so, after being pressed by the Sangha, says this to the nuns, “Venerables, you should socialize. Don’t live separately. There are other nuns in the Sangha who have such behavior, reputation, and notoriety, and who harass the Sangha of nuns and hide each other’s offenses. The Sangha says nothing to them. It’s because of disrespect, contempt, impatience, and slander, and because you are weak that the Sangha says to you, ‘Sisters, you socialize, behave badly, have a bad reputation, are notorious, harass the Sangha of nuns, and hide each other’s offenses. Be secluded, Venerables. The Sangha praises seclusion for the Sisters.’” And she keeps on saying it. If the Sangha is ready, it should press her to make her stop. This is the motion. 

Please,\marginnote{2.59} Venerables, I ask the Sangha to listen. The nun so-and-so, after being pressed by the Sangha, says this to the nuns, “Venerables, you should socialize. Don’t live separately. There are other nuns in the Sangha who have such behavior, reputation, and notoriety, and who harass the Sangha of nuns and hide each other’s offenses. The Sangha says nothing to them. It’s because of disrespect, contempt, impatience, and slander, and because you are weak that the Sangha says to you, ‘Sisters, you socialize, behave badly, have a bad reputation, are notorious, harass the Sangha of nuns, and hide each other’s offenses. Be secluded, Venerables. The Sangha praises seclusion for the Sisters.’” And she keeps on saying it. The Sangha presses her to make her stop. Any nun who approves of pressing her to make her stop should remain silent. Any nun who doesn’t approve should speak up. 

For\marginnote{2.73} the second time I speak on this matter … For the third time I speak on this matter … 

The\marginnote{2.75} Sangha has pressed nun so-and-so to stop. The Sangha approves and is therefore silent. I’ll remember it thus.’” 

After\marginnote{2.77} the motion, she commits an offense of wrong conduct.\footnote{The Pali just says \textit{\textsanskrit{dukkaṭa}}, without specifying that it is an \textit{\textsanskrit{āpatti}}, an offense. Yet just below the text says that the \textit{\textsanskrit{dukkaṭa}} is annulled if you commit the full offense of \textit{\textsanskrit{saṅghādisesa}}. The implication is that \textit{\textsanskrit{dukkaṭa}} should be read as \textit{\textsanskrit{āpatti} \textsanskrit{dukkaṭassa}}, “an offense of wrong conduct”. } After each of the first two announcements, she commits a serious offense. When the last announcement is finished, she commits an offense entailing suspension. For one who commits the offense entailing suspension, the offense of wrong conduct and the serious offenses are annulled. 

%
\item[That too: ] this is said with reference to the preceding offenses. %
\item[After the third announcement: ] there is an offense when she has been pressed for the third time, not as soon as the misconduct has been committed. %
\item[Entailing sending away: ] she is sent away from the Sangha. %
\item[Suspension: ] only the Sangha gives the trial period for that offense, sends back to the beginning, and rehabilitates—not several nuns, not an individual nun. Therefore it is called an offense entailing suspension. This is the name and designation of this class of offense. Therefore, too, it is called an offense entailing suspension. %
\end{description}

\subsection*{Permutations }

If\marginnote{3.1} it is a legitimate legal procedure, and she perceives it as such, and she does not stop, she commits an offense entailing suspension. If it is a legitimate legal procedure, but she is unsure of it, and she does not stop, she commits an offense entailing suspension. If it is a legitimate legal procedure, but she perceives it as illegitimate, and she does not stop, she commits an offense entailing suspension. 

If\marginnote{3.4} it is an illegitimate legal procedure, but she perceives it as legitimate, she commits an offense of wrong conduct. If it is an illegitimate legal procedure, but she is unsure of it, she commits an offense of wrong conduct. If it is an illegitimate legal procedure, and she perceives it as such, she commits an offense of wrong conduct. 

\subsection*{Non-offenses }

There\marginnote{3.7.1} is no offense: if she has not been pressed;  if she stops;  if she is insane;  if she is the first offender. 

\scendsutta{The thirteenth offense entailing suspension is finished.\footnote{The nuns’ offenses entailing suspension 14, 15, 16, and 17 are respectively identical to the monks’ suspension 10 at \href{https://suttacentral.net/pli-tv-bu-vb-ss10/en/brahmali\#1.3.16.1}{Bu Ss 10:1.3.16.1}, 11 at \href{https://suttacentral.net/pli-tv-bu-vb-ss11/en/brahmali\#1.19.1}{Bu Ss 11:1.19.1}, 12 at \href{https://suttacentral.net/pli-tv-bu-vb-ss12/en/brahmali\#1.26.1}{Bu Ss 12:1.26.1}, and 13 at \href{https://suttacentral.net/pli-tv-bu-vb-ss13/en/brahmali\#1.8.10.1}{Bu Ss 13:1.8.10.1}, with appropriate gender changes. } }

“Venerables,\marginnote{3.14} the seventeen rules on suspension have been recited, nine being immediate offenses, eight after the third announcement. If a nun commits any one of them, she must undertake a trial period for a half-month toward both Sanghas. When this is completed, she is to be rehabilitated wherever there is a sangha of at least twenty nuns. If that nun is rehabilitated by a sangha of nuns of even one less than twenty, then that nun is not rehabilitated and those nuns are at fault. This is proper procedure. 

In\marginnote{3.20} regard to this I ask you, ‘Are you pure in this?’ A second time I ask, ‘Are you pure in this?’ A third time I ask, ‘Are you pure in this?’ You are pure in this and therefore silent. I’ll remember it thus.” 

\scend{The group of seventeen is finished. }

\scendkanda{The chapter on offenses entailing suspension in the Nuns’ Analysis is finished. }

%
\addtocontents{toc}{\let\protect\contentsline\protect\nopagecontentsline}
\chapter*{Relinquishment With Confession}
\addcontentsline{toc}{chapter}{\tocchapterline{Relinquishment With Confession}}
\addtocontents{toc}{\let\protect\contentsline\protect\oldcontentsline}

%
\section*{{\suttatitleacronym Bi Np 1}{\suttatitletranslation The training rule on collections of almsbowls }{\suttatitleroot Pattasannicaya}}
\addcontentsline{toc}{section}{\tocacronym{Bi Np 1} \toctranslation{The training rule on collections of almsbowls } \tocroot{Pattasannicaya}}
\markboth{The training rule on collections of almsbowls }{Pattasannicaya}
\extramarks{Bi Np 1}{Bi Np 1}

Venerables,\marginnote{0.6} these thirty rules on relinquishment and confession come up for recitation. 

\subsection*{Origin story }

At\marginnote{1.1} one time the Buddha was staying at \textsanskrit{Sāvatthī} in the Jeta Grove, \textsanskrit{Anāthapiṇḍika}’s Monastery. At that time the nuns from the group of six had collected a large number of almsbowls. When people walking about the dwellings saw this, they complained and criticized those nuns, “How can the nuns collect a large number of bowls? Will they start up as bowl merchants or set up a bowl shop?” 

The\marginnote{1.5} nuns heard the complaints of those people, and the nuns of few desires complained and criticized those nuns, “How can the nuns from the group of six collect bowls?” … “Is it true, monks, that the nuns from the group of six do this?” 

“It’s\marginnote{1.9} true, Sir.” 

The\marginnote{1.10} Buddha rebuked them … “How can the nuns from the group of six collect bowls? This will affect people’s confidence …” … “And, monks, the nuns should recite this training rule like this: 

\subsection*{Final ruling }

\scrule{‘If a nun collects almsbowls, she commits an offense entailing relinquishment and confession.’” }

\subsection*{Definitions }

\begin{description}%
\item[A: ] whoever … %
\item[Nun: ] … The nun who has been given the full ordination in unanimity by both Sanghas through a legal procedure consisting of one motion and three announcements that is irreversible and fit to stand—this sort of nun is meant in this case. %
\item[An almsbowl: ] there are two kinds of bowls: the iron bowl and the ceramic bowl. And there are three sizes of bowls: the large bowl, the medium bowl, and the small bowl. %
\item[The large bowl: ] it takes half an \textit{\textsanskrit{āḷhaka}} measure of boiled rice, a fourth part of fresh food, and a suitable amount of curry. %
\item[The medium bowl: ] it takes a \textit{\textsanskrit{nāḷika}} measure of boiled rice, a fourth part of fresh food, and a suitable amount of curry. %
\item[The small bowl: ] it takes a \textit{pattha} measure of boiled rice, a fourth part of fresh food, and a suitable amount of curry. Anything larger than this is not a bowl, nor anything smaller. %
\item[Collects: ] almsbowls that have neither been determined nor assigned to another.\footnote{For an explanation of the idea of \textit{\textsanskrit{vikappanā}}, see Appendix of Technical Terms. } %
\item[Entailing relinquishment: ] entailing relinquishment at dawn. %
\end{description}

The\marginnote{2.1.19} bowl should be relinquished to a sangha, a group, or an individual nun. “And, monks, it should relinquished like this. After approaching the Sangha, that nun should arrange her upper robe over one shoulder and pay respect at the feet of the senior nuns. She should then squat on her heels, raise her joined palms, and say: 

‘Venerables,\marginnote{2.1.22} this almsbowl, which I have kept for more than one day, is to be relinquished. I relinquish it to the Sangha.’ 

After\marginnote{2.1.23} relinquishing it, she is to confess the offense. The confession should be received by a competent and capable nun. The relinquished bowl is then to be given back: 

‘Please,\marginnote{2.1.26} Venerables, I ask the Sangha to listen. This almsbowl, which was to be relinquished by the nun so-and-so, has been relinquished to the Sangha. If the Sangha is ready, it should give this bowl back to nun so-and-so.’ 

Or:\marginnote{2.1.29} after approaching several nuns, that nun should arrange her upper robe over one shoulder and pay respect at the feet of the senior nuns. She should then squat on her heels, raise her joined palms, and say: 

‘Venerables,\marginnote{2.1.30} this almsbowl, which I have kept for more than one day, is to be relinquished. I relinquish it to you.’ 

After\marginnote{2.1.31} relinquishing it, she is to confess the offense. The confession should be received by a competent and capable nun. The relinquished bowl is then to be given back: 

‘Please,\marginnote{2.1.34} Venerables, I ask you to listen. This almsbowl, which was to be relinquished by the nun so-and-so, has been relinquished to you. If the Venerables are ready, you should give this bowl back to nun so-and-so.’ 

Or:\marginnote{2.1.37} after approaching a single nun, that nun should arrange her upper robe over one shoulder, squat on her heels, raise her joined palms, and say: 

‘This\marginnote{2.1.38} almsbowl, which I have kept for more than one day, is to be relinquished. I relinquish it to you.’ 

After\marginnote{2.1.40} relinquishing it, she is to confess the offense. The confession should be received by that nun. The relinquished bowl is then to be given back: 

‘I\marginnote{2.1.43} give this almsbowl back to you.’” 

\subsection*{Permutations }

If\marginnote{2.2.1} it is more than one day and she perceives it as more, she commits an offense entailing relinquishment and confession. If it is more than one day, but she is unsure of it, she commits an offense entailing relinquishment and confession. If it is more than one day, but she perceives it as less, she commits an offense entailing relinquishment and confession. If it has not been determined, but she perceives that it has, she commits an offense entailing relinquishment and confession. If it has not been assigned to another, but she perceives that it has, she commits an offense entailing relinquishment and confession. If it has not been given away, but she perceives that it has, she commits an offense entailing relinquishment and confession. If it has not been not lost, but she perceives that it has … If it has not been destroyed, but she perceives that it has … If it has not been broken, but she perceives that it has … If it has not been stolen, but she perceives that it has, she commits an offense entailing relinquishment and confession. 

If\marginnote{2.2.11} she uses an almsbowl that should be relinquished without first relinquishing it, she commits an offense of wrong conduct. If it is less than one day, but she perceives it as more, she commits an offense of wrong conduct. If it is less than one day, but she is unsure of it, she commits an offense of wrong conduct. If it is less than one day and she perceives it as less, there is no offense. 

\subsection*{Non-offenses }

There\marginnote{2.3.1} is no offense: if, before dawn, it has been determined, assigned to another, given away, lost, destroyed, broken, stolen, or taken on trust;  if she is insane;  if she is the first offender. 

Soon\marginnote{3.1} afterwards the nuns from the group of six did not give back a relinquished bowl. They told the Buddha. 

\scrule{“Monks, a relinquished almsbowl should be given back. If a nun doesn’t give it back, she commits an offense of wrong conduct.” }

\scendsutta{The first training rule is finished. }

%
\section*{{\suttatitleacronym Bi Np 2}{\suttatitletranslation The training rule on distributing out-of-season robe-cloth }{\suttatitleroot Akālacīvara}}
\addcontentsline{toc}{section}{\tocacronym{Bi Np 2} \toctranslation{The training rule on distributing out-of-season robe-cloth } \tocroot{Akālacīvara}}
\markboth{The training rule on distributing out-of-season robe-cloth }{Akālacīvara}
\extramarks{Bi Np 2}{Bi Np 2}

\subsection*{Origin story }

At\marginnote{1.1} one time the Buddha was staying at \textsanskrit{Sāvatthī} in the Jeta Grove, \textsanskrit{Anāthapiṇḍika}’s Monastery. At that time, after completing the rainy-season residence in a village monastery, a number of nuns were traveling to \textsanskrit{Sāvatthī}. They were perfect in conduct and deportment, but poorly dressed in shabby robes. Some lay followers who saw them thought, “These nuns are perfect in conduct and deportment, but poorly dressed in shabby robes; they must’ve been robbed,” and they gave out-of-season robe-cloth to the Sangha of nuns. Because they had performed the robe-making ceremony, the nun \textsanskrit{Thullanandā} determined it as in-season robe-cloth, and then distributed it. The lay followers asked those nuns whether they had obtained any robe-cloth. They replied that they hadn’t and told them what had happened.\footnote{The point here is that because \textsanskrit{Thullanandā} (illegitimately) determines the robe-cloth as in-season, it can only be distributed to those nuns who have spent the rainy season in the monastery where the robe-cloth is given. The traveling nuns were therefore excluded from receiving it. } Those lay followers then complained and criticized her, “How could Venerable \textsanskrit{Thullanandā} determine out-of-season robe-cloth as ‘in-season’, and then distribute it?” 

The\marginnote{1.14} nuns heard the complaints of those lay followers, and the nuns of few desires complained and criticized her, “How could Venerable \textsanskrit{Thullanandā} do this?” Those nuns then told the monks, who in turn told the Buddha. Soon afterwards he had the Sangha gathered and questioned the monks: “Is it true, monks, that the nun \textsanskrit{Thullanandā} did this?” 

“It’s\marginnote{1.20} true, Sir.” 

The\marginnote{1.21} Buddha rebuked her … “How could the nun \textsanskrit{Thullanandā} do this? This will affect people’s confidence …” … “And, monks, the nuns should recite this training rule like this: 

\subsection*{Final ruling }

\scrule{‘If a nun determines out-of-season robe-cloth as “in-season”, and then distributes it, she commits an offense entailing relinquishment and confession.’” }

\subsection*{Definitions }

\begin{description}%
\item[A: ] whoever … %
\item[Nun: ] … The nun who has been given the full ordination in unanimity by both Sanghas through a legal procedure consisting of one motion and three announcements that is irreversible and fit to stand—this sort of nun is meant in this case. %
\item[Out-of-season robe-cloth: ] for one who has not participated in the robe-making ceremony, it is robe-cloth given during the eleven months. For one who has participated in the robe-making ceremony, it is robe-cloth given during the seven months. Also, if it is given in the robe season, but the cloth is designated, it is called “out-of-season robe-cloth”.\footnote{That is, in-season robe-cloth is cloth obtained during the last month of the rainy season, while out-of-season robe-cloth is cloth obtained during the remaining eleven months of the year. See the \textsanskrit{Kaṅkhāvitaraṇī} commentary. | According to the commentary to Bu Np 3, Sp 1.499: \textit{\textsanskrit{Kālepi} \textsanskrit{ādissa} dinnanti \textsanskrit{saṅghassa} \textsanskrit{vā} “\textsanskrit{idaṁ} \textsanskrit{akālacīvara}”nti \textsanskrit{uddisitvā} \textsanskrit{dinnaṁ}, ekapuggalassa \textsanskrit{vā} “\textsanskrit{idaṁ} \textsanskrit{tuyhaṁ} \textsanskrit{dammī}”ti \textsanskrit{dinnaṁ}}, “‘Also, if it is given in the robe season, but it is designated’ means: it is given to the Sangha after designating it by saying, ‘This is out-of-season robe-cloth’, or it is given to an individual by saying, ‘I give this to you.’” In other words, it is designated as out-of-season cloth or designated to an individual. The commentary to the present rule, at Sp 2.740, adds that designating to a group is included in designated cloth: \textit{Ādissa dinnanti \textsanskrit{sampattā} \textsanskrit{bhājentūti} \textsanskrit{vatvāpi} \textsanskrit{idaṁ} \textsanskrit{gaṇassa} \textsanskrit{idaṁ} \textsanskrit{tumhākaṁ} \textsanskrit{dammīti} \textsanskrit{vatvā} \textsanskrit{vā} \textsanskrit{dātukamyatāya} \textsanskrit{pādamūle} \textsanskrit{ṭhapetvā} \textsanskrit{vā} dinnampi \textsanskrit{ādissa} \textsanskrit{dinnaṁ} \textsanskrit{nāma} hoti; \textsanskrit{etaṁ} sabbampi \textsanskrit{akālacīvaraṁ}}, “‘Given after designating’ means: if they give after saying, ‘Let those who are present share it out’, or after saying, ‘I give this to the group, to you’, or they place it at the feet (of the recipient) wishing to give, this is called ‘Given after designating’. All this is called out-of-season robe-cloth.” See also \textit{kathina} in Appendix of Technical Terms. } %
\end{description}

If\marginnote{2.1.7} she distributes it after determining the out-of-season robe-cloth as “in-season”, then for the effort there is an act of wrong conduct. When she gets it, it becomes subject to relinquishment. The robe-cloth should be relinquished to a sangha, a group, or an individual nun. 

“And,\marginnote{2.1.10} monks, it’s to be relinquished like this. (To be expanded as in \href{https://suttacentral.net/pli-tv-bi-vb-np1/en/brahmali\#2.1.21}{Bi Np 1:2.1.21}–Bi Np 1:2.1.43, with appropriate substitutions.) 

…\marginnote{2.1.12} ‘Venerables, this out-of-season robe-cloth, which I distributed after determining it as “in-season”, is to be relinquished. I relinquish it to the Sangha.’ … the Sangha should give … you should give … ‘I give this robe-cloth back to you.’” 

\subsection*{Permutations }

If\marginnote{2.2.1} it is out-of-season robe-cloth and she perceives it as such, and she distributes it after determining it as “in-season”, she commits an offense entailing relinquishment and confession. If it is out-of-season robe-cloth, but she is unsure of it, and she distributes it after determining it as “in-season”, she commits an offense of wrong conduct. If it is out-of-season robe-cloth, but she perceives it as in-season robe-cloth, and she distributes it after determining it as “in-season”, there is no offense. 

If\marginnote{2.2.4} it is in-season robe-cloth, but she perceives it as out-of-season robe-cloth, she commits an offense of wrong conduct. If it is in-season robe-cloth, but she is unsure of it, she commits an offense of wrong conduct. If it is in-season robe-cloth and she perceives it as such, there is no offense. 

\subsection*{Non-offenses }

There\marginnote{2.3.1} is no offense: if she distributes out-of-season robe-cloth that she perceives as in-season;  if she distributes in-season robe-cloth that she perceives as in-season;  if she is insane;  if she is the first offender. 

\scendsutta{The second training rule is finished. }

%
\section*{{\suttatitleacronym Bi Np 3}{\suttatitletranslation The training rule on trading robes }{\suttatitleroot Cīvaraparivattana}}
\addcontentsline{toc}{section}{\tocacronym{Bi Np 3} \toctranslation{The training rule on trading robes } \tocroot{Cīvaraparivattana}}
\markboth{The training rule on trading robes }{Cīvaraparivattana}
\extramarks{Bi Np 3}{Bi Np 3}

\subsection*{Origin story }

At\marginnote{1.1} one time the Buddha was staying at \textsanskrit{Sāvatthī} in the Jeta Grove, \textsanskrit{Anāthapiṇḍika}’s Monastery. At that time the nun \textsanskrit{Thullanandā} was using a robe she had received after trading with another nun. But the other nun folded up the robe she had received and put it aside. \textsanskrit{Thullanandā} then said to her, “Venerable, where’s that robe that I traded with you?” She brought out the robe and showed it to her, and \textsanskrit{Thullanandā} said, “Here’s your robe and give me that robe of mine. That which is yours is yours and that which is mine is mine. Give me that and take back what’s yours.” And she just took it. 

That\marginnote{1.9} nun then told the nuns what had happened. The nuns of few desires complained and criticized her, “How could Venerable \textsanskrit{Thullanandā} trade a robe with a nun and then take it back?” Those nuns then told the monks, who in turn told the Buddha. Soon afterwards he had the Sangha gathered and questioned the monks: “Is it true, monks, that the nun \textsanskrit{Thullanandā} did this?” 

“It’s\marginnote{1.15} true, Sir.” 

The\marginnote{1.16} Buddha rebuked her … “How could the nun \textsanskrit{Thullanandā} trade a robe with a nun and then take it back? This will affect people’s confidence …” … “And, monks, the nuns should recite this training rule like this: 

\subsection*{Final ruling }

\scrule{‘If a nun trades robes with a nun and then says, “Here’s your robe; give me that robe of mine. That which is yours is yours, and that which is mine is mine. Give me that, and take back what’s yours;” and she just takes it or has it taken, she commits an offense entailing relinquishment and confession.’” }

\subsection*{Definitions }

\begin{description}%
\item[A: ] whoever … %
\item[Nun: ] … The nun who has been given the full ordination in unanimity by both Sanghas through a legal procedure consisting of one motion and three announcements that is irreversible and fit to stand—this sort of nun is meant in this case. %
\item[With a nun: ] with another nun. %
\item[Robes: ] one of the six kinds of robe-cloth, but not smaller than what can be assigned to another.\footnote{The six are linen, cotton, silk, wool, sunn hemp, and hemp; see \href{https://suttacentral.net/pli-tv-kd8/en/brahmali\#3.1.6}{Kd 8:3.1.6}. According to \href{https://suttacentral.net/pli-tv-kd8/en/brahmali\#21.1.4}{Kd 8:21.1.4} the size referred to here is no smaller than 8 by 4 \textit{\textsanskrit{sugataṅgula}}, “standard fingerbreadths”. For an explanation of \textit{sugata} as “standard” and the idea of \textit{\textsanskrit{vikappanā}}, see Appendix of Technical Terms. } %
\item[Trades: ] much with little or little with much. %
\item[Just takes it: ] if she just takes it herself, she commits an offense entailing relinquishment and confession. %
\item[Has it taken: ] if she asks another, she commits an offense of wrong conduct. If she only asks once, then even if the other takes back many, it becomes subject to relinquishment.\footnote{“Many” renders \textit{\textsanskrit{bahukaṁ}}. This is based on the commentary to Bu Np 25, Sp 1.633: \textit{\textsanskrit{Āṇatto} \textsanskrit{bahūni} \textsanskrit{gaṇhāti}, \textsanskrit{ekaṁ} \textsanskrit{pācittiyaṁ}}, “If the one who is asked takes many, there is (only) one offense entailing confession.” } %
\end{description}

The\marginnote{2.1.16} robe-cloth should be relinquished to a sangha, a group, or an individual nun. “And, monks, it’s to be relinquished like this. (To be expanded as in \href{https://suttacentral.net/pli-tv-bi-vb-np1/en/brahmali\#2.1.21}{Bi Np 1:2.1.21}–Bi Np 1:2.1.43, with appropriate substitutions.) 

…\marginnote{2.1.19} ‘Venerables, this robe-cloth, which I took back after trading it with a nun, is to be relinquished. I relinquish it to the Sangha.’ … the Sangha should give … you should give … ‘I give this robe-cloth back to you.’” 

\subsection*{Permutations }

If\marginnote{2.2.1} the other person is fully ordained and she perceives her as such, and after trading robe-cloth with her she takes it back or has it taken back, she commits an offense entailing relinquishment and confession. If the other person is fully ordained, but she is unsure of it, and after trading robe-cloth with her she takes it back or has it taken back, she commits an offense entailing relinquishment and confession. If the other person is fully ordained, but she does not perceive her as such, and after trading robe-cloth with her she takes it back or has it taken back, she commits an offense entailing relinquishment and confession. 

If\marginnote{2.2.4} she trades another requisite, and then takes it back or has it taken back, she commits an offense of wrong conduct. If she trades robe-cloth or another requisite with someone who is not fully ordained, and then takes it back or has it taken back, she commits an offense of wrong conduct. 

If\marginnote{2.2.6} the other person is not fully ordained, but she perceives her as such, she commits an offense of wrong conduct. If the other person is not fully ordained, but she is unsure of it, she commits an offense of wrong conduct. If the other person is not fully ordained, and she does not perceive her as such, she commits an offense of wrong conduct. 

\subsection*{Non-offenses }

There\marginnote{2.3.1} is no offense: if the other nun gives it back;  if she takes it on trust from her;  if she is insane;  if she is the first offender. 

\scendsutta{The third training rule is finished. }

%
\section*{{\suttatitleacronym Bi Np 4}{\suttatitletranslation The training rule on asking for something else }{\suttatitleroot Aññaviññāpana}}
\addcontentsline{toc}{section}{\tocacronym{Bi Np 4} \toctranslation{The training rule on asking for something else } \tocroot{Aññaviññāpana}}
\markboth{The training rule on asking for something else }{Aññaviññāpana}
\extramarks{Bi Np 4}{Bi Np 4}

\subsection*{Origin story }

At\marginnote{1.1} one time when the Buddha was staying at \textsanskrit{Sāvatthī} in \textsanskrit{Anāthapiṇḍika}’s Monastery, the nun \textsanskrit{Thullanandā} was sick. A lay follower went to her and asked, “Venerable, what’s wrong with you? What may I get you?” 

“I\marginnote{1.5} need ghee.” 

That\marginnote{1.6} lay follower then brought back from a shop a \textit{\textsanskrit{kahāpaṇa}}’s worth of ghee and gave it to \textsanskrit{Thullanandā}. \textsanskrit{Thullanandā} said, “I don’t need ghee; I need oil.” The lay follower returned to the shopkeeper and said, “It seems the nun doesn’t need ghee, but oil. Here’s your ghee; please give me oil.” 

“Sir,\marginnote{1.13} if we were to take back goods that have been sold, when would our goods be sold? When ghee is bought, ghee is taken away. Buying oil, you receive that, and you’ll take that away.” 

That\marginnote{1.15} lay follower then complained and criticized her, “How could Venerable \textsanskrit{Thullanandā} ask for one thing and then for something else?” 

The\marginnote{1.17} nuns heard the complaints of that lay follower, and the nuns of few desires complained and criticized her … Those nuns then told the monks, who in turn told the Buddha. Soon afterwards he had the Sangha gathered and questioned the monks: “Is it true, monks, that the nun \textsanskrit{Thullanandā} did this?” 

“It’s\marginnote{1.22} true, Sir.” 

The\marginnote{1.23} Buddha rebuked her … “How could the nun \textsanskrit{Thullanandā} ask for one thing and then for something else? This will affect people’s confidence …” … “And, monks, the nuns should recite this training rule like this: 

\subsection*{Final ruling }

\scrule{‘If a nun asks for one thing and then for something else, she commits an offense entailing relinquishment and confession.’” }

\subsection*{Definitions }

\begin{description}%
\item[A: ] whoever … %
\item[Nun: ] … The nun who has been given the full ordination in unanimity by both Sanghas through a legal procedure consisting of one motion and three announcements that is irreversible and fit to stand—this sort of nun is meant in this case. %
\item[Asks for one thing: ] whatever she asks for. %
\item[Then for something else: ] apart from that thing, if she asks for something else, then for the effort there is an act of wrong conduct. When she gets it, it becomes subject to relinquishment. %
\end{description}

It\marginnote{2.1.10} should be relinquished to a sangha, a group, or an individual nun. “And, monks, it’s to be relinquished like this. (To be expanded as in \href{https://suttacentral.net/pli-tv-bi-vb-np1/en/brahmali\#2.1.21}{Bi Np 1:2.1.21}–Bi Np 1:2.1.43, with appropriate substitutions.) 

…\marginnote{2.1.13} ‘Venerables, this thing, which I asked for after asking for something else, is to be relinquished. I relinquish it to the Sangha.’ … the Sangha should give … you should give … ‘I give this back to you.’” 

\subsection*{Permutations }

If\marginnote{2.2.1} it is something else and she perceives it as such, and she asks for that, she commits an offense entailing relinquishment and confession. If it is something else, but she is unsure of it, and she asks for that, she commits an offense entailing relinquishment and confession. If it is something else, but she does not perceive it as such, and she asks for that, she commits an offense entailing relinquishment and confession. 

If\marginnote{2.2.4} it is not something else, but she perceives it as such, and she asks for that, she commits an offense of wrong conduct. If it is not something else, but she is unsure of it, and she asks for that, she commits an offense of wrong conduct. If it is not something else, and she does not perceive it as such, and she asks for that, there is no offense. 

\subsection*{Non-offenses }

There\marginnote{2.3.1} is no offense: if she asks for both at the same time;  if she can show a benefit in asking;  if she is insane;  if she is the first offender. 

\scendsutta{The fourth training rule is finished. }

%
\section*{{\suttatitleacronym Bi Np 5}{\suttatitletranslation The training rule on exchanging for something else }{\suttatitleroot Aññacetāpana}}
\addcontentsline{toc}{section}{\tocacronym{Bi Np 5} \toctranslation{The training rule on exchanging for something else } \tocroot{Aññacetāpana}}
\markboth{The training rule on exchanging for something else }{Aññacetāpana}
\extramarks{Bi Np 5}{Bi Np 5}

\subsection*{Origin story }

At\marginnote{1.1} one time when the Buddha was staying at \textsanskrit{Sāvatthī} in \textsanskrit{Anāthapiṇḍika}’s Monastery, the nun \textsanskrit{Thullanandā} was sick. A lay follower went to her and asked, “I hope you’re bearing up, Venerable, I hope you’re getting better?” 

“I’m\marginnote{1.5} not bearing up, and I’m not getting better.” 

“We’ll\marginnote{1.6} deposit a \textit{\textsanskrit{kahāpaṇa}} coin in such-and-such a shop. Please get whatever you wish from there.” 

\textsanskrit{Thullanandā}\marginnote{1.7} then said to a trainee nun, “Go to such-and-such a shop and bring back a \textit{\textsanskrit{kahāpaṇa}}’s worth of oil.” That trainee nun did just that and gave it to \textsanskrit{Thullanandā}. \textsanskrit{Thullanandā} said, “I don’t need oil; I need ghee.” The trainee nun returned to the shopkeeper and said, “It seems the nun doesn’t need oil, but ghee. Here’s your oil; please give me ghee.” 

“Venerable,\marginnote{1.14} if we were to take back goods that have been sold, when would our goods be sold? When oil is bought, oil is taken away. Buying ghee, you receive that, and you’ll take that away.” 

The\marginnote{1.16} trainee nun started to cry. The nuns asked her why, 

and\marginnote{1.19} she told them what had happened. 

The\marginnote{1.20} nuns of few desires complained and criticized her, “How could Venerable \textsanskrit{Thullanandā} get one thing in exchange and then something else?” … “Is it true, monks, that the nun \textsanskrit{Thullanandā} did this?” 

“It’s\marginnote{1.23} true, Sir.” 

The\marginnote{1.24} Buddha rebuked her … “How could the nun \textsanskrit{Thullanandā} get one thing in exchange and then something else? This will affect people’s confidence …” … “And, monks, the nuns should recite this training rule like this: 

\subsection*{Final ruling }

\scrule{‘If a nun gets one thing in exchange and then something else, she commits an offense entailing relinquishment and confession.’” }

\subsection*{Definitions }

\begin{description}%
\item[A: ] whoever … %
\item[Nun: ] … The nun who has been given the full ordination in unanimity by both Sanghas through a legal procedure consisting of one motion and three announcements that is irreversible and fit to stand—this sort of nun is meant in this case. %
\item[Gets one thing in exchange: ] whatever she gets in exchange. %
\item[Then something else: ] apart from that thing, if she gets something else in exchange, then for the effort there is an act of wrong conduct. When she gets it, it becomes subject to relinquishment. %
\end{description}

It\marginnote{2.10} should be relinquished to a sangha, a group, or an individual nun. “And, monks, it’s to be relinquished like this. (To be expanded as in \href{https://suttacentral.net/pli-tv-bi-vb-np1/en/brahmali\#2.1.21}{Bi Np 1:2.1.21}–Bi Np 1:2.1.43, with appropriate substitutions.) 

…\marginnote{2.13} ‘Venerables, this thing, which I got in exchange, having first gotten something else in exchange, is to be relinquished. I relinquish it to the Sangha.’ … the Sangha should give … you should give … ‘I give this back to you.’” 

\subsection*{Permutations }

If\marginnote{2.17.1} it is something else and she perceives it as such, and she gets that in exchange, she commits an offense entailing relinquishment and confession. If it is something else, but she is unsure of it, and she gets that in exchange, she commits an offense entailing relinquishment and confession. If it is something else, but she does not perceive it as such, and she gets that in exchange, she commits an offense entailing relinquishment and confession. 

If\marginnote{2.20} it is not something else, but she perceives it as such, and she gets that in exchange, she commits an offense of wrong conduct. If it is not something else, but she is unsure of it, and she gets that in exchange, she commits an offense of wrong conduct. If it is not something else, and she does not perceive it as such, there is no offense. 

\subsection*{Non-offenses }

There\marginnote{2.23.1} is no offense: if she gets both in exchange at the same time;  if she can show a benefit in doing the exchange;  if she is insane;  if she is the first offender. 

\scendsutta{The fifth training rule is finished. }

%
\section*{{\suttatitleacronym Bi Np 6}{\suttatitletranslation The training rule on exchanging what belongs to the Sangha }{\suttatitleroot Saṁghikacetapana}}
\addcontentsline{toc}{section}{\tocacronym{Bi Np 6} \toctranslation{The training rule on exchanging what belongs to the Sangha } \tocroot{Saṁghikacetapana}}
\markboth{The training rule on exchanging what belongs to the Sangha }{Saṁghikacetapana}
\extramarks{Bi Np 6}{Bi Np 6}

\subsection*{Origin story }

At\marginnote{1.1} one time the Buddha was staying at \textsanskrit{Sāvatthī} in the Jeta Grove, \textsanskrit{Anāthapiṇḍika}’s Monastery. At that time the lay followers collected voluntary contributions to supply the Sangha of nuns with robes. They stored the requisites in a cloth merchant’s shop, and then went to the nuns and said,\footnote{“Requisites” renders \textit{\textsanskrit{parikkhāra}}. For a discussion of this word, see Appendix of Technical Terms. } “Venerables, requisites to be used for robes are stored in such-and-such a cloth merchant’s shop. Please have someone get cloth from there and share it out.” But the nuns exchanged those requisites for tonics, which they then used.\footnote{“Tonics” renders \textit{bhesajja}. See discussion of this word in Appendix of Technical Terms. } When the lay followers found out about this, they complained and criticized them, “When requisites belonging to the Sangha are designated for a specific purpose, how could the nuns exchange them for something else?” 

The\marginnote{1.7} nuns heard the complaints of those lay followers, and the nuns of few desires complained and criticized them, “When requisites belonging to the Sangha are designated for a specific purpose, how could the nuns exchange them for something else?” … “Is it true, monks, that the nuns did this?” 

“It’s\marginnote{1.11} true, Sir.” 

The\marginnote{1.12} Buddha rebuked them … “When requisites belonging to the Sangha are designated for a specific purpose, how could the nuns exchange them for something else? This will affect people’s confidence …” … “And, monks, the nuns should recite this training rule like this: 

\subsection*{Final ruling }

\scrule{‘When a requisite belonging to the Sangha is designated for a specific purpose, if a nun exchanges it for something else, she commits an offense entailing relinquishment and confession.’” }

\subsection*{Definitions }

\begin{description}%
\item[A: ] whoever … %
\item[Nun: ] … The nun who has been given the full ordination in unanimity by both Sanghas through a legal procedure consisting of one motion and three announcements that is irreversible and fit to stand—this sort of nun is meant in this case. %
\item[A requisite is designated for a specific purpose: ] it was given for a specific purpose. %
\item[Belonging to the Sangha: ] belonging to the Sangha, not to a group, not to an individual nun. %
\item[Exchanges it for something else: ] if, apart from the purpose for which it was given, she exchanges it for something else, then for the effort there is an act of wrong conduct. When she gets it, it becomes subject to relinquishment. %
\end{description}

It\marginnote{2.1.12} should be relinquished to a sangha, a group, or an individual nun. “And, monks, it’s to be relinquished like this. (To be expanded as in \href{https://suttacentral.net/pli-tv-bi-vb-np1/en/brahmali\#2.1.21}{Bi Np 1:2.1.21}–Bi Np 1:2.1.43, with appropriate substitutions.) 

‘Venerables,\marginnote{2.1.15} this thing, which I got in exchange for a requisite belonging to the Sangha that was designated for a specific purpose, is to be relinquished. I relinquish it to the Sangha.’ … the Sangha should give … you should give … ‘I give this back to you.’” 

\subsection*{Permutations }

If\marginnote{2.2.1} it is for a specific purpose and she perceives that it is, and she exchanges it for something else, she commits an offense entailing relinquishment and confession. If it is for a specific purpose, but she is unsure of it, and she exchanges it for something else, she commits an offense entailing relinquishment and confession. If it is for a specific purpose, but she does not perceive that it is, and she exchanges it for something else, she commits an offense entailing relinquishment and confession. When she receives in return what had been relinquished, it is to be used in accordance with the intention of the donors.\footnote{\textit{\textsanskrit{Yathādāne} \textsanskrit{upanetabbaṁ}}. This is explained at Vmv.2.740: \textit{\textsanskrit{Nissaṭṭhaṁ} \textsanskrit{paṭilabhitvāpi} \textsanskrit{yaṁ} uddissa \textsanskrit{dāyakehi} \textsanskrit{dinnaṁ}, tattheva \textsanskrit{dātabbaṁ}. \textsanskrit{Tenāha} “\textsanskrit{yathādāneyeva} upanetabba”nti}, “Having obtained what was relinquished, it is to be given in accordance with the designation of the donors. \textit{\textsanskrit{Yathādāneyeva} \textsanskrit{upanetabbaṁ}} is said in regard to this.” } 

If\marginnote{2.2.5} it is not for a specific purpose, but she perceives that it is, she commits an offense of wrong conduct. If it is not for a specific purpose, but she is unsure of it, she commits an offense of wrong conduct. If it is not for a specific purpose, and she does not perceive that it is, there is no offense. 

\subsection*{Non-offenses }

There\marginnote{2.3.1} is no offense: if she uses the remainder;\footnote{This seems to mean that if there is a remainder after the requisites have been used as intended, then this may be exchanged for something other than what it was specified for. Sp 2.762: \textit{\textsanskrit{Sesakaṁ} \textsanskrit{upanetīti} \textsanskrit{yadatthāya} dinno, \textsanskrit{taṁ} \textsanskrit{cetāpetvā} \textsanskrit{avasesaṁ} \textsanskrit{aññassatthāya} upaneti}, “She uses the remainder means: after exchanging it for the purpose for which it was given, she uses the remainder for another purpose.” }  if she uses it after getting permission from the owners;\footnote{That is, if she makes use of it in another way than what was intended by the owners. Sp 2.762: \textit{\textsanskrit{Sāmike} \textsanskrit{apaloketvāti} “tumhehi \textsanskrit{cīvaratthāya} dinno, \textsanskrit{amhākañca} \textsanskrit{cīvaraṁ} atthi, \textsanskrit{telādīhi} pana attho”ti \textsanskrit{evaṁ} \textsanskrit{āpucchitvā} upaneti}, “After getting permission from the owners means: she makes use of it after asking, ‘It was given by you for the purpose of robes, but we have robes and we need oil, etc.’” }  if there is an emergency;\footnote{For a discussion of the rendering “emergency” for \textit{\textsanskrit{āpadāsu}}, see Appendix of Technical Terms. }  if she is insane;  if she is the first offender. 

\scendsutta{The sixth training rule is finished. }

%
\section*{{\suttatitleacronym Bi Np 7}{\suttatitletranslation The second training rule on exchanging what belongs to the Sangha }{\suttatitleroot Dutiyasaṁghikacetāpana}}
\addcontentsline{toc}{section}{\tocacronym{Bi Np 7} \toctranslation{The second training rule on exchanging what belongs to the Sangha } \tocroot{Dutiyasaṁghikacetāpana}}
\markboth{The second training rule on exchanging what belongs to the Sangha }{Dutiyasaṁghikacetāpana}
\extramarks{Bi Np 7}{Bi Np 7}

\subsection*{Origin story }

At\marginnote{1.1} one time the Buddha was staying at \textsanskrit{Sāvatthī} in the Jeta Grove, \textsanskrit{Anāthapiṇḍika}’s Monastery. At that time the lay followers collected voluntary contributions to supply the Sangha of nuns with robes. They stored the requisites in a cloth merchant’s shop, and then went to the nuns and said, “Venerables, requisites to be used for robes are stored in such-and-such a cloth merchant’s shop. Please have someone get cloth from there and share it out.” But even though they had asked for them, the nuns exchanged those requisites for tonics, which they then used.\footnote{See \href{https://suttacentral.net/pli-tv-bi-vb-np10/en/brahmali\#1.9}{Bi NP 10:1.9} for the use of \textit{sayampi \textsanskrit{yācitvā}}. } 

When\marginnote{1.5} the lay followers found out about this, they complained and criticized them, “When requisites belonging to the Sangha are designated for a specific purpose and were asked for, how could the nuns exchange them for something else?” … “Is it true, monks, that the nuns did this?” 

“It’s\marginnote{1.8} true, Sir.” 

The\marginnote{1.9} Buddha rebuked them … “When requisites belonging to the Sangha are designated for a specific purpose and were asked for, how could the nuns exchange them for something else? This will affect people’s confidence …” … “And, monks, the nuns should recite this training rule like this: 

\subsection*{Final ruling }

\scrule{‘When a requisite belonging to the Sangha is designated for a specific purpose and was asked for, if a nun exchanges it for something else, she commits an offense entailing relinquishment and confession.’” }

\subsection*{Definitions }

\begin{description}%
\item[A: ] whoever … %
\item[Nun: ] … The nun who has been given the full ordination in unanimity by both Sanghas through a legal procedure consisting of one motion and three announcements that is irreversible and fit to stand—this sort of nun is meant in this case. %
\item[A requisite is designated for a specific purpose: ] it was given for a specific purpose. %
\item[Belonging to the Sangha: ] belonging to the Sangha, not to a group, not to an individual nun. %
\item[Was asked for: ] that she herself had asked for. %
\item[Exchanges it for something else: ] if, apart from the purpose for which it was given, she exchanges it for something else, then for the effort there is an act of wrong conduct. When she gets it, it becomes subject to relinquishment. %
\end{description}

It\marginnote{2.14} should be relinquished to a sangha, a group, or an individual nun. “And, monks, it’s to be relinquished like this.  (To be expanded as in \href{https://suttacentral.net/pli-tv-bi-vb-np1/en/brahmali\#2.1.21}{Bi Np 1:2.1.21}–Bi Np 1:2.1.43, with appropriate substitutions.) 

‘Venerables,\marginnote{2.17} this thing, which I got in exchange for a requisite belonging to the Sangha that was designated for a specific purpose and had been asked for, is to be relinquished. I relinquish it to the Sangha.’ … the Sangha should give … you should give … ‘I give this back to you.’” 

\subsection*{Permutations }

If\marginnote{2.21.1} it is for a specific purpose and she perceives that it is, and she exchanges it for something else, she commits an offense entailing relinquishment and confession. If it is for a specific purpose, but she is unsure of it, and she exchanges it for something else, she commits an offense entailing relinquishment and confession. If it is for a specific purpose, but she does not perceive that it is, and she exchanges it for something else, she commits an offense entailing relinquishment and confession. When she receives in return what had been relinquished, it is to be used in accordance with the intention of the donors.\footnote{Vmv 2.740: \textit{\textsanskrit{nissaṭṭhaṁ} \textsanskrit{paṭilabhitvāpi} \textsanskrit{yaṁ} uddissa \textsanskrit{dāyakehi} \textsanskrit{dinnaṁ}, tattheva \textsanskrit{dātabbaṁ}. \textsanskrit{Tenāha} “\textsanskrit{yathādāneyeva} upanetabba”nti}, “Having obtained what was relinquished, it is to be given in accordance with the designation of the donors. \textit{\textsanskrit{Yathādāneyeva} \textsanskrit{upanetabbaṁ}} is said in regard to this.” } 

If\marginnote{2.25} it is not for a specific purpose, but she perceives that it is, she commits an offense of wrong conduct. If it is not for a specific purpose, but she is unsure of it, she commits an offense of wrong conduct. If it is not for a specific purpose, and she does not perceive that it is, there is no offense. 

\subsection*{Non-offenses }

There\marginnote{2.28.1} is no offense: if she uses the remainder;\footnote{This seems to mean that if there is a remainder after the article has been used as intended, then this may be exchanged for something other than what was specified. Sp 2.762: \textit{\textsanskrit{Sesakaṁ} \textsanskrit{upanetīti} \textsanskrit{yadatthāya} dinno, \textsanskrit{taṁ} \textsanskrit{cetāpetvā} \textsanskrit{avasesaṁ} \textsanskrit{aññassatthāya} upaneti}, “She uses the remainder means: after exchanging it for the purpose for which it was given, she uses the remainder for another purpose.” }  if she uses it after getting permission from the owners;\footnote{That is, if she makes use of it in another way than what was intended by the owners. Sp 2.762: \textit{\textsanskrit{Sāmike} \textsanskrit{apaloketvāti} “tumhehi \textsanskrit{cīvaratthāya} dinno, \textsanskrit{amhākañca} \textsanskrit{cīvaraṁ} atthi, \textsanskrit{telādīhi} pana attho”ti \textsanskrit{evaṁ} \textsanskrit{āpucchitvā} upaneti}, “After getting permission from the owners means: she makes use of it after asking, ‘It was given by you for the purpose of robes, but we have robes and we need oil, etc.’” }  if there is an emergency;  if she is insane;  if she is the first offender. 

\scendsutta{The seventh training rule is finished. }

%
\section*{{\suttatitleacronym Bi Np 8}{\suttatitletranslation The training rule on exchanging what belongs to a group }{\suttatitleroot Gaṇikacetāpana}}
\addcontentsline{toc}{section}{\tocacronym{Bi Np 8} \toctranslation{The training rule on exchanging what belongs to a group } \tocroot{Gaṇikacetāpana}}
\markboth{The training rule on exchanging what belongs to a group }{Gaṇikacetāpana}
\extramarks{Bi Np 8}{Bi Np 8}

\subsection*{Origin story }

At\marginnote{1.1} one time the Buddha was staying at \textsanskrit{Sāvatthī} in the Jeta Grove, \textsanskrit{Anāthapiṇḍika}’s Monastery. At that time the nuns staying in the yard belonging to a certain association were weak because of a lack of congee.\footnote{“Yard” renders \textit{\textsanskrit{pariveṇa}}. For a discussion of this word, see Appendix of Technical Terms. } Then, after collecting voluntary contributions to supply the nuns with congee, that association stored the ingredients in a shop. They then went to the nuns and said,\footnote{For an explanation of the rendering “ingredients” for \textit{\textsanskrit{parikkhāra}}, see Appendix of Technical Terms. } “Venerables, congee ingredients are stored in such-and-such a shop. Please have someone get rice from there, have congee cooked, and then eat it.” But the nuns exchanged those ingredients for tonics, which they then used. 

When\marginnote{1.6} that association found out about this, they complained and criticized them, “When collective ingredients are designated for a specific purpose, how could the nuns exchange them for something else?” … “Is it true, monks, that the nuns did this?” 

“It’s\marginnote{1.9} true, Sir.” 

The\marginnote{1.10} Buddha rebuked them … “When collective ingredients are designated for a specific purpose, how could the nuns exchange them for something else? This will affect people’s confidence …” … “And, monks, the nuns should recite this training rule like this: 

\subsection*{Final ruling }

\scrule{‘When a collective requisite is designated for a specific purpose, if a nun exchanges it for something else, she commits an offense entailing relinquishment and confession.’” }

\subsection*{Definitions }

\begin{description}%
\item[A: ] whoever … %
\item[Nun: ] … The nun who has been given the full ordination in unanimity by both Sanghas through a legal procedure consisting of one motion and three announcements that is irreversible and fit to stand—this sort of nun is meant in this case. %
\item[A requisite is designated for a specific purpose: ] it was given for a specific purpose. %
\item[Collective: ] belonging to a group, not to the Sangha, not to an individual nun. %
\item[Exchanges it for something else: ] if, apart from the purpose for which it was given, she exchanges it for something else, then for the effort there is an act of wrong conduct. When she gets it, it becomes subject to relinquishment. %
\end{description}

It\marginnote{2.12} should be relinquished to a sangha, a group, or an individual nun. “And, monks, it’s to be relinquished like this.  (To be expanded as in \href{https://suttacentral.net/pli-tv-bi-vb-np1/en/brahmali\#2.1.21}{Bi Np 1:2.1.21}–Bi Np 1:2.1.43, with appropriate substitutions.) 

‘Venerables,\marginnote{2.15} this thing, which I got in exchange for a collective requisite that was designated for a specific purpose, is to be relinquished. I relinquish it to the Sangha.’ … the Sangha should give … you should give … ‘I give this back to you.’” 

\subsection*{Permutations }

If\marginnote{2.19.1} it is for a specific purpose and she perceives that it is, and she exchanges it for something else, she commits an offense entailing relinquishment and confession. If it is for a specific purpose, but she is unsure of it, and she exchanges it for something else, she commits an offense entailing relinquishment and confession. If it is for a specific purpose, but she does not perceive that it is, and she exchanges it for something else, she commits an offense entailing relinquishment and confession. When she receives in return what had been relinquished, it is to be used in accordance with the intention of the donors.\footnote{Vmv 2.740: \textit{\textsanskrit{nissaṭṭhaṁ} \textsanskrit{paṭilabhitvāpi} \textsanskrit{yaṁ} uddissa \textsanskrit{dāyakehi} \textsanskrit{dinnaṁ}, tattheva \textsanskrit{dātabbaṁ}. \textsanskrit{Tenāha} “\textsanskrit{yathādāneyeva} upanetabba”nti}, “Having obtained what was relinquished, it is to be given in accordance with the designation of the donors. \textit{\textsanskrit{Yathādāneyeva} \textsanskrit{upanetabbaṁ}} is said in regard to this.” } 

If\marginnote{2.23} it is not for a specific purpose, but she perceives that it is, she commits an offense of wrong conduct. If it is not for a specific purpose, but she is unsure of it, she commits an offense of wrong conduct. If it is not for a specific purpose, and she does not perceive that it is, there is no offense. 

\subsection*{Non-offenses }

There\marginnote{2.26.1} is no offense: if she uses the remainder;\footnote{This seems to mean that if there is a remainder after the requisites have been used as intended, then this may be exchanged for something other than what was specified. Sp 2.762: \textit{\textsanskrit{Sesakaṁ} \textsanskrit{upanetīti} \textsanskrit{yadatthāya} dinno, \textsanskrit{taṁ} \textsanskrit{cetāpetvā} \textsanskrit{avasesaṁ} \textsanskrit{aññassatthāya} upaneti}, “She uses the remainder means: after exchanging it for the purpose for which it was given, she uses the remainder for another purpose.” }  if she uses it after getting permission from the owners;\footnote{That is, if she makes use of it in another way than what was intended by the owners. Sp 2.762: \textit{\textsanskrit{Sāmike} \textsanskrit{apaloketvāti} “tumhehi \textsanskrit{cīvaratthāya} dinno, \textsanskrit{amhākañca} \textsanskrit{cīvaraṁ} atthi, \textsanskrit{telādīhi} pana attho”ti \textsanskrit{evaṁ} \textsanskrit{āpucchitvā} upaneti}, “After getting permission from the owners means: she makes use of it after asking, ‘It was given by you for the purpose of robes, but we have robes and we need oil, etc.’” }  if there is an emergency;  if she is insane;  if she is the first offender. 

\scendsutta{The eighth training rule is finished. }

%
\section*{{\suttatitleacronym Bi Np 9}{\suttatitletranslation The second training rule on exchanging what belongs to a group }{\suttatitleroot Dutiyagaṇikacetāpana}}
\addcontentsline{toc}{section}{\tocacronym{Bi Np 9} \toctranslation{The second training rule on exchanging what belongs to a group } \tocroot{Dutiyagaṇikacetāpana}}
\markboth{The second training rule on exchanging what belongs to a group }{Dutiyagaṇikacetāpana}
\extramarks{Bi Np 9}{Bi Np 9}

\subsection*{Origin story }

At\marginnote{1.1} one time the Buddha was staying at \textsanskrit{Sāvatthī} in the Jeta Grove, \textsanskrit{Anāthapiṇḍika}’s Monastery. At that time the nuns staying in the yard belonging to a certain association were weak because of a lack of congee. Then, after collecting voluntary contributions to supply the nuns with congee, that association stored the ingredients in a shop. They then went to the nuns and said, “Venerables, congee ingredients are stored in such-and-such a shop. Please have someone get rice from there, have congee cooked, and then eat it.” But even though they had asked for them, the nuns exchanged those ingredients for tonics, which they then used.\footnote{See \href{https://suttacentral.net/pli-tv-bi-vb-np10/en/brahmali\#1.9}{Bi NP 10:1.9} for the use of \textit{sayampi \textsanskrit{yācitvā}}. } 

When\marginnote{1.7} that association found out about this, they complained and criticized them, “When collective ingredients are designated for a specific purpose and were asked for, how could the nuns exchange them for something else?” … “Is it true, monks, that the nuns did this?” 

“It’s\marginnote{1.10} true, Sir.” 

The\marginnote{1.11} Buddha rebuked them … “When collective ingredients are designated for a specific purpose and were asked for, how could the nuns exchange them for something else? This will affect people’s confidence …” … “And, monks, the nuns should recite this training rule like this: 

\subsection*{Final ruling }

\scrule{‘When a collective requisite is designated for a specific purpose and was asked for, if a nun exchanges it for something else, she commits an offense entailing relinquishment and confession.’” }

\subsection*{Definitions }

\begin{description}%
\item[A: ] whoever … %
\item[Nun: ] … The nun who has been given the full ordination in unanimity by both Sanghas through a legal procedure consisting of one motion and three announcements that is irreversible and fit to stand—this sort of nun is meant in this case. %
\item[A requisite is designated for a specific purpose: ] it was given for a specific purpose. %
\item[Collective: ] belonging to a group, not to the Sangha, not to an individual nun. %
\item[Was asked for: ] that she herself had asked for. %
\item[Exchanges it for something else: ] if, apart from the purpose for which it was given, she exchanges it for something else, then for the effort there is an act of wrong conduct. When she gets it, it becomes subject to relinquishment. %
\end{description}

It\marginnote{2.14} should be relinquished to a sangha, a group, or an individual nun. “And, monks, it’s to be relinquished like this.  (To be expanded as in \href{https://suttacentral.net/pli-tv-bi-vb-np1/en/brahmali\#2.1.21}{Bi Np 1:2.1.21}–Bi Np 1:2.1.43, with appropriate substitutions.) 

‘Venerables,\marginnote{2.17} this thing, which I got in exchange for a collective requisite that was designated for a specific purpose and had been asked for, is to be relinquished. I relinquish it to the Sangha.’ … the Sangha should give … you should give … ‘I give this back to you.’” 

\subsection*{Permutations }

If\marginnote{2.22.1} it is for a specific purpose and she perceives that it is, and she exchanges it for something else, she commits an offense entailing relinquishment and confession. If it is for a specific purpose, but she is unsure of it, and she exchanges it for something else, she commits an offense entailing relinquishment and confession. If it is for a specific purpose, but she does not perceive that it is, and she exchanges it for something else, she commits an offense entailing relinquishment and confession. When she receives in return what had been relinquished, it is to be used in accordance with the intention of the donors.\footnote{Vmv 2.740: \textit{\textsanskrit{nissaṭṭhaṁ} \textsanskrit{paṭilabhitvāpi} \textsanskrit{yaṁ} uddissa \textsanskrit{dāyakehi} \textsanskrit{dinnaṁ}, tattheva \textsanskrit{dātabbaṁ}. \textsanskrit{Tenāha} “\textsanskrit{yathādāneyeva} upanetabba”nti}, “Having obtained what was relinquished, it is to be given in accordance with the designation of the donors. \textit{\textsanskrit{Yathādāneyeva} \textsanskrit{upanetabbaṁ}} is said in regard to this.” } 

If\marginnote{2.26} it is not for a specific purpose, but she perceives that it is, she commits an offense of wrong conduct. If it is not for a specific purpose, but she is unsure of it, she commits an offense of wrong conduct. If it is not for a specific purpose, and she does not perceive that it is, there is no offense. 

\subsection*{Non-offenses }

There\marginnote{2.29.1} is no offense: if she uses the remainder;\footnote{This seems to mean that if there is a remainder after the article has been used as intended, then this may be exchanged for something other than what was specified. Sp 2.762: \textit{\textsanskrit{Sesakaṁ} \textsanskrit{upanetīti} \textsanskrit{yadatthāya} dinno, \textsanskrit{taṁ} \textsanskrit{cetāpetvā} \textsanskrit{avasesaṁ} \textsanskrit{aññassatthāya} upaneti}, “She uses the remainder means: after exchanging it for the purpose for which it was given, she uses the remainder for another purpose.” }  if she uses it after getting permission from the owners;\footnote{That is, if she makes use of it in another way than what was intended by the owners. Sp 2.762: \textit{\textsanskrit{Sāmike} \textsanskrit{apaloketvāti} “tumhehi \textsanskrit{cīvaratthāya} dinno, \textsanskrit{amhākañca} \textsanskrit{cīvaraṁ} atthi, \textsanskrit{telādīhi} pana attho”ti \textsanskrit{evaṁ} \textsanskrit{āpucchitvā} upaneti}, “After getting permission from the owners means: she makes use of it after asking, ‘It was given by you for the purpose of robes, but we have robes and we need oil, etc.’” }  if there is an emergency;  if she is insane;  if she is the first offender. 

\scendsutta{The ninth training rule is finished. }

%
\section*{{\suttatitleacronym Bi Np 10}{\suttatitletranslation The training rule on exchanging what belongs to an individual }{\suttatitleroot Puggalikacetāpana}}
\addcontentsline{toc}{section}{\tocacronym{Bi Np 10} \toctranslation{The training rule on exchanging what belongs to an individual } \tocroot{Puggalikacetāpana}}
\markboth{The training rule on exchanging what belongs to an individual }{Puggalikacetāpana}
\extramarks{Bi Np 10}{Bi Np 10}

\subsection*{Origin story }

At\marginnote{1.1} one time the Buddha was staying at \textsanskrit{Sāvatthī} in the Jeta Grove, \textsanskrit{Anāthapiṇḍika}’s Monastery. At that time the nun \textsanskrit{Thullanandā} was a learned reciter, and she was confident and skilled at giving teachings. Many people visited her. Just then the yard of \textsanskrit{Thullanandā}’s dwelling was deteriorating. People asked her why, and she replied, “I have neither donor, nor worker.” Then, after collecting voluntary contributions for the yard of \textsanskrit{Thullanandā}’s dwelling, the people gave the collected requisites to \textsanskrit{Thullanandā}. But even though she had asked for them, \textsanskrit{Thullanandā} exchanged those requisites for tonics, which she then used. 

When\marginnote{1.10} the people found out about this, they complained and criticized her, “When personal requisites are designated for a specific purpose and were asked for, how could the nun \textsanskrit{Thullanandā} exchange them for something else?” … “Is it true, monks, that the nun \textsanskrit{Thullanandā} did this?” 

“It’s\marginnote{1.13} true, Sir.” 

The\marginnote{1.14} Buddha rebuked her … “When personal requisites are designated for a specific purpose and were asked for, how could the nun \textsanskrit{Thullanandā} exchange them for something else? This will affect people’s confidence …” … “And, monks, the nuns should recite this training rule like this: 

\subsection*{Final ruling }

\scrule{‘When a personal requisite is designated for a specific purpose and was asked for, if a nun exchanges it for something else, she commits an offense entailing relinquishment and confession.’” }

\subsection*{Definitions }

\begin{description}%
\item[A: ] whoever … %
\item[Nun : ] The nun who has been given the full ordination in unanimity by both Sanghas through a legal procedure consisting of one motion and three announcements that is irreversible and fit to stand—this sort of nun is meant in this case. %
\item[A requisite is designated for a specific purpose: ] it was given for a specific purpose. %
\item[Personal: ] belonging to an individual nun, not to the Sangha, not to a group. %
\item[Was asked for: ] that she herself had asked for. %
\item[Exchanges it for something else: ] if, apart from the purpose for which it was given, she exchanges it for something else, then for the effort there is an act of wrong conduct. When she gets it, it becomes subject to relinquishment. %
\end{description}

It\marginnote{2.14} should be relinquished to a sangha, a group, or an individual nun. “And, monks, it’s to be relinquished like this.  (To be expanded as in \href{https://suttacentral.net/pli-tv-bi-vb-np1/en/brahmali\#2.1.21}{Bi Np 1:2.1.21}–Bi Np 1:2.1.43, with appropriate substitutions.) 

‘Venerables,\marginnote{2.17} this thing, which I got in exchange for a personal requisite that was designated for a specific purpose and had been asked for, is to be relinquished. I relinquish it to the Sangha.’ … the Sangha should give … you should give … ‘I give this back to you.’” 

\subsection*{Permutations }

If\marginnote{2.21.1} it is for a specific purpose and she perceives that it is, and she exchanges it for something else, she commits an offense entailing relinquishment and confession. If it is for a specific purpose, but she is unsure of it, and she exchanges it for something else, she commits an offense entailing relinquishment and confession. If it is for a specific purpose, but she does not perceive that it is, and she exchanges it for something else, she commits an offense entailing relinquishment and confession. When she receives in return what had been relinquished, it is to be used in accordance with the intention of the donors.\footnote{Vmv 2.740: \textit{\textsanskrit{nissaṭṭhaṁ} \textsanskrit{paṭilabhitvāpi} \textsanskrit{yaṁ} uddissa \textsanskrit{dāyakehi} \textsanskrit{dinnaṁ}, tattheva \textsanskrit{dātabbaṁ}. \textsanskrit{Tenāha} “\textsanskrit{yathādāneyeva} upanetabba”nti}, “Having obtained what was relinquished, it is to be given in accordance with the designation of the donors. \textit{\textsanskrit{Yathādāneyeva} \textsanskrit{upanetabbaṁ}} is said in regard to this.” } 

If\marginnote{2.25} it is not for a specific purpose, but she perceives that it is, she commits an offense of wrong conduct. If it is not for a specific purpose, but she is unsure of it, she commits an offense of wrong conduct. If it is not for a specific purpose, and she does not perceive that it is, there is no offense. 

\subsection*{Non-offenses }

There\marginnote{2.28.1} is no offense: if she uses the remainder;\footnote{This seems to mean that if there is a remainder after the article has been used as intended, then this may be exchanged for something other than what was specified. Sp 2.762: \textit{\textsanskrit{Sesakaṁ} \textsanskrit{upanetīti} \textsanskrit{yadatthāya} dinno, \textsanskrit{taṁ} \textsanskrit{cetāpetvā} \textsanskrit{avasesaṁ} \textsanskrit{aññassatthāya} upaneti}, “She uses the remainder means: after exchanging it for the purpose for which it was given, she uses the remainder for another purpose.” }  if she uses it after getting permission from the owners;\footnote{That is, if she makes use of it in another way than what was intended by the owners. Sp 2.762: \textit{\textsanskrit{Sāmike} \textsanskrit{apaloketvāti} “tumhehi \textsanskrit{cīvaratthāya} dinno, \textsanskrit{amhākañca} \textsanskrit{cīvaraṁ} atthi, \textsanskrit{telādīhi} pana attho”ti \textsanskrit{evaṁ} \textsanskrit{āpucchitvā} upaneti}, “After getting permission from the owners means: she makes use of it after asking, ‘It was given by you for the purpose of robes, but we have robes and we need oil, etc.’” }  if there is an emergency;  if she is insane;  if she is the first offender. 

\scendsutta{The tenth training rule is finished. }

%
\section*{{\suttatitleacronym Bi Np 11}{\suttatitletranslation The training rule on heavy cloaks }{\suttatitleroot Garupāvuraṇa}}
\addcontentsline{toc}{section}{\tocacronym{Bi Np 11} \toctranslation{The training rule on heavy cloaks } \tocroot{Garupāvuraṇa}}
\markboth{The training rule on heavy cloaks }{Garupāvuraṇa}
\extramarks{Bi Np 11}{Bi Np 11}

\subsection*{Origin story }

At\marginnote{1.1} one time the Buddha was staying at \textsanskrit{Sāvatthī} in the Jeta Grove, \textsanskrit{Anāthapiṇḍika}’s Monastery. At that time the nun \textsanskrit{Thullanandā} was a learned reciter, and she was confident and skilled at giving teachings. On one occasion when the weather was cold, King Pasenadi of Kosala put on an expensive woolen cloak and went to \textsanskrit{Thullanandā}. He bowed and sat down. And \textsanskrit{Thullanandā} instructed, inspired, and gladdened him with a teaching. He then said, “Venerable, please say what you need.” 

“Great\marginnote{1.7} king, if you wish to give me something, then give me this woolen cloak.” 

The\marginnote{1.8} king gave her his cloak. He then got up from his seat, bowed down, circumambulated her with his right side toward her, and left. People complained and criticized her, “These nuns have great desires; they are not content. How can they ask the king for his woolen cloak?” 

The\marginnote{1.12} nuns heard the complaints of those people, and the nuns of few desires complained and criticized her, “How could Venerable \textsanskrit{Thullanandā} ask the king for his woolen cloak?” … “Is it true, monks, that the nun \textsanskrit{Thullanandā} asked for this?” 

“It’s\marginnote{1.16} true, Sir.” 

The\marginnote{1.17} Buddha rebuked her … “How could the nun \textsanskrit{Thullanandā} ask the king for his woolen cloak? This will affect people’s confidence …” … “And, monks, the nuns should recite this training rule like this: 

\subsection*{Final ruling }

\scrule{‘If a nun carries out an exchange to get a heavy cloak, it is to be worth at most four \textit{\textsanskrit{kaṁsa}} coins. If she gets one in exchange that is worth more than that, she commits an offense entailing relinquishment and confession.’” }

\subsection*{Definitions }

\begin{description}%
\item[A heavy cloak:\footnote{The \textit{\textsanskrit{pāvuraṇa}} was a garment for lay people, used in much the same way as the upper robe was used by monastic. When you put it on, you \textit{\textsanskrit{pārupati}} it, which is what you do with an upper robe. This again points to the small difference between monastic robes and lay people’s clothes at the time of the Buddha. This is presumably why \textsanskrit{Thullanandā} could use such a cloak for herself, thereby effectively transforming it into a robe. } ] whatever cloak is used in cold weather. %
\item[Carries out an exchange to get: ] asks for. %
\item[It is to be worth at most four \textit{\textsanskrit{kaṁsa}} coins: ] it may be worth sixteen \textit{\textsanskrit{kahāpaṇa}} coins. %
\item[If she gets one in exchange that is worth more than that: ] if she asks for one worth more than that, then for the effort there is an act of wrong conduct. When she gets it, it becomes subject to relinquishment. %
\end{description}

It\marginnote{2.1.10} should be relinquished to a sangha, a group, or an individual nun. “And, monks, it’s to be relinquished like this.  (To be expanded as in \href{https://suttacentral.net/pli-tv-bi-vb-np1/en/brahmali\#2.1.21}{Bi Np 1:2.1.21}–Bi Np 1:2.1.43, with appropriate substitutions.) 

‘Venerables,\marginnote{2.1.13} this heavy cloak worth more than four \textit{\textsanskrit{kaṁsa}} coins, which I got in exchange, is to be relinquished. I relinquish it to the Sangha.’ … the Sangha should give … you should give … ‘I give this back to you.’” 

\subsection*{Permutations }

If\marginnote{2.2.1} it is worth more than four \textit{\textsanskrit{kaṁsa}} coins, and she perceives it as such, and she gets it in exchange, she commits an offense entailing relinquishment and confession. If it is worth more than four \textit{\textsanskrit{kaṁsa}} coins, but she is unsure of it, and she gets it in exchange, she commits an offense entailing relinquishment and confession. If it is worth more than four \textit{\textsanskrit{kaṁsa}} coins, but she perceives it as being worth less, and she gets it in exchange, she commits an offense entailing relinquishment and confession. 

If\marginnote{2.2.4} it is worth less than four \textit{\textsanskrit{kaṁsa}} coins, but she perceives it as being worth more, she commits an offense of wrong conduct. If it is worth less than four \textit{\textsanskrit{kaṁsa}} coins, but she is unsure of it, she commits an offense of wrong conduct. If it is worth less than four \textit{\textsanskrit{kaṁsa}} coins, and she perceives it as such, there is no offense. 

\subsection*{Non-offenses }

There\marginnote{2.3.1} is no offense: if she carries out an exchange for one worth at most four \textit{\textsanskrit{kaṁsa}} coins;  if she carries out an exchange for one worth less than four \textit{\textsanskrit{kaṁsa}} coins;  if it is from relatives;  if it is from those who have given an invitation;  if it is for the benefit of someone else;  if it is by means of her own property;  if she gets one in exchange that has little value from someone who wants to exchange one of great value;  if she is insane;  if she is the first offender. 

\scendsutta{The eleventh training rule is finished. }

%
\section*{{\suttatitleacronym Bi Np 12}{\suttatitletranslation The training rule on light cloaks }{\suttatitleroot Lahupāvuraṇa}}
\addcontentsline{toc}{section}{\tocacronym{Bi Np 12} \toctranslation{The training rule on light cloaks } \tocroot{Lahupāvuraṇa}}
\markboth{The training rule on light cloaks }{Lahupāvuraṇa}
\extramarks{Bi Np 12}{Bi Np 12}

\subsection*{Origin story }

At\marginnote{1.1} one time the Buddha was staying at \textsanskrit{Sāvatthī} in the Jeta Grove, \textsanskrit{Anāthapiṇḍika}’s Monastery. At that time the nun \textsanskrit{Thullanandā} was a learned reciter, and she was confident and skilled at giving teachings. On one occasion when the weather was warm, King Pasenadi of Kosala put on an expensive linen cloak and went to \textsanskrit{Thullanandā}. He bowed and sat down. And \textsanskrit{Thullanandā} instructed, inspired, and gladdened him with a teaching. He then said, “Venerable, please say what you need.” 

“Great\marginnote{1.7} king, if you wish to give me something, then give me this linen cloak.” 

The\marginnote{1.8} king gave her his cloak. He then got up from his seat, bowed down, circumambulated her with his right side toward her, and left. People complained and criticized her, “These nuns have great desires; they are not content. How can they ask the king for his linen cloak?” 

The\marginnote{1.12} nuns heard the complaints of those people, and the nuns of few desires complained and criticized her, “How could Venerable \textsanskrit{Thullanandā} ask the king for his linen cloak?” … “Is it true, monks, that the nun \textsanskrit{Thullanandā} asked for this?” 

“It’s\marginnote{1.16} true, Sir.” 

The\marginnote{1.17} Buddha rebuked her … “How could the nun \textsanskrit{Thullanandā} ask the king for his linen cloak? This will affect people’s confidence …” … “And, monks, the nuns should recite this training rule like this: 

\subsection*{Final ruling }

\scrule{‘If a nun carries out an exchange to get a light cloak, it is to be worth at most two-and-a-half \textit{\textsanskrit{kaṁsa}} coins. If she gets one in exchange that is worth more than that, she commits an offense entailing relinquishment and confession.’” }

\subsection*{Definitions }

\begin{description}%
\item[A light cloak: ] whatever cloak is used in warm weather. %
\item[Carries out an exchange to get: ] asks for. %
\item[It is to be worth at most two-and-a-half \textit{\textsanskrit{kaṁsa}} coins: ] it may be worth ten \textit{\textsanskrit{kahāpaṇa}} coins. %
\item[If she gets one in exchange that is worth more than that: ] if she asks for one worth more than that, then for the effort there is an act of wrong conduct. When she gets it, it becomes subject to relinquishment. %
\end{description}

It\marginnote{2.1.10} should be relinquished to a sangha, a group, or an individual nun. “And, monks, it’s to be relinquished like this.  (To be expanded as in \href{https://suttacentral.net/pli-tv-bi-vb-np1/en/brahmali\#2.1.21}{Bi Np 1:2.1.21}–Bi Np 1:2.1.43, with appropriate substitutions.) 

‘Venerables,\marginnote{2.1.13} this light cloak worth more than two-and-a-half \textit{\textsanskrit{kaṁsa}} coins, which I got in exchange, is to be relinquished. I relinquish it to the Sangha.’ … the Sangha should give … you should give … ‘I give this back to you.’” 

\subsection*{Permutations }

If\marginnote{2.1.17.1} it is worth more than two-and-a-half \textit{\textsanskrit{kaṁsa}} coins, and she perceives it as such, and she gets it in exchange, she commits an offense entailing relinquishment and confession. If it is worth more than two-and-a-half \textit{\textsanskrit{kaṁsa}} coins, but she is unsure of it, and she gets it in exchange, she commits an offense entailing relinquishment and confession. If it is worth more than two-and-a-half \textit{\textsanskrit{kaṁsa}} coins, but she perceives it as being worth less, and she gets it in exchange, she commits an offense entailing relinquishment and confession. 

If\marginnote{2.1.20} it is worth less than two-and-a-half \textit{\textsanskrit{kaṁsa}} coins, but she perceives it as being worth more, she commits an offense of wrong conduct. If it is worth less than two-and-a-half \textit{\textsanskrit{kaṁsa}} coins, but she is unsure of it, she commits an offense of wrong conduct. If it is worth less than two-and-a-half \textit{\textsanskrit{kaṁsa}} coins, and she perceives it as such, there is no offense. 

\subsection*{Non-offenses }

There\marginnote{2.2.1} is no offense: if she carries out an exchange for one worth at most two-and-a-half \textit{\textsanskrit{kaṁsa}} coins;  if she carries out an exchange for one worth less than two-and-a-half \textit{\textsanskrit{kaṁsa}} coins;  if it is from relatives;  if it is from those who have given an invitation;  if it is for the benefit of someone else;  if it is by means of her own property;  if she gets one in exchange that has little value from someone who wants to exchange one of great value;  if she is insane;  if she is the first offender. 

\scendsutta{The twelfth training rule is finished.\footnote{Following this, the Pali reads \textit{(Dutiya) Pattavagga} to indicate the last chapter of \textit{nissaggiya \textsanskrit{pācittiya}} rules for the nuns, and then  \textit{\textsanskrit{Pariṇatasikkhāpada}} to indicate the last rule in the chapter of \textit{nissaggiya \textsanskrit{pācittiya}} rules. These indications are redundant in the present translation, since I have pointed to all the missing rules, including direct references to their counterparts in the chapter on \textit{nissaggiya \textsanskrit{pācittiya}} rules for the \textit{bhikkhus}. } }

“Venerables,\marginnote{2.2.14} the thirty rules on relinquishment and confession have been recited. In regard to this I ask you, ‘Are you pure in this?’ A second time I ask, ‘Are you pure in this?’ A third time I ask, ‘Are you pure in this?’ You are pure in this and therefore silent. I’ll remember it thus.” 

\scendkanda{The chapter on offenses entailing relinquishment in the Nuns’ Analysis is finished. }

%
\addtocontents{toc}{\let\protect\contentsline\protect\nopagecontentsline}
\chapter*{Confession}
\addcontentsline{toc}{chapter}{\tocchapterline{Confession}}
\addtocontents{toc}{\let\protect\contentsline\protect\oldcontentsline}

%
\section*{{\suttatitleacronym Bi Pc 1}{\suttatitletranslation The training rule on garlic }{\suttatitleroot Lasuṇa}}
\addcontentsline{toc}{section}{\tocacronym{Bi Pc 1} \toctranslation{The training rule on garlic } \tocroot{Lasuṇa}}
\markboth{The training rule on garlic }{Lasuṇa}
\extramarks{Bi Pc 1}{Bi Pc 1}

Venerables,\marginnote{0.6} these one hundred and sixty-six rules on offenses entailing confession come up for recitation. 

\subsection*{Origin story }

At\marginnote{1.1} one time the Buddha was staying at \textsanskrit{Sāvatthī} in the Jeta Grove, \textsanskrit{Anāthapiṇḍika}’s Monastery. At that time a lay follower had invited the Sangha of nuns to ask for garlic: “If any of the nuns need garlic, I’ll supply it.” And he had told his field-keeper, “If the nuns come, give two or three bulbs to each nun.”\footnote{“Bulb” renders \textit{\textsanskrit{bhaṇḍika}}. Normally I render this word as “bundle”, but here the intended meaning seems to be a “bundle” of cloves, that is, a single bulb of garlic. Sp 2.793: \textit{Dve tayo \textsanskrit{bhaṇḍiketi} dve \textsanskrit{vā} tayo \textsanskrit{vā} \textsanskrit{poṭṭalike}; \textsanskrit{sampuṇṇamiñjānametaṁ} \textsanskrit{adhivacanaṁ}}, “Two or three \textit{\textsanskrit{bhaṇḍika}}: two or three bundles; this is an expression for being complete in cloves.” } 

On\marginnote{1.6} that occasion they were holding a celebration in \textsanskrit{Sāvatthī}, and the garlic was used up as soon as it arrived there. Just then the nuns went to that lay follower and said, “We need garlic.” 

“Venerables,\marginnote{1.10} there isn’t any. As soon as the garlic arrives, it’s used up. Please go to the field.” 

The\marginnote{1.13} nun \textsanskrit{Thullanandā} went to the field, and not having any sense of moderation she took a large amount of garlic. The field-keeper complained and criticized them, “How can the nuns not have any sense of moderation and take a large amount of garlic?” 

The\marginnote{1.16} nuns heard the complaints of that field-keeper, and the nuns of few desires complained and criticized her, “How could Venerable \textsanskrit{Thullanandā} not have any sense of moderation and take a large amount of garlic?” … “Is it true, monks, that the nun \textsanskrit{Thullanandā} did this?” 

“It’s\marginnote{1.20} true, Sir.” 

The\marginnote{1.21} Buddha rebuked her, “How could the nun \textsanskrit{Thullanandā} not have any sense of moderation and take a large amount of garlic? This will affect people’s confidence …” … and after giving a teaching he addressed the monks: 

\subparagraph*{Jataka }

“Once\marginnote{1.25.1} upon a time, monks, \textsanskrit{Thullanandā} was the wife of a brahmin. They had three daughters, \textsanskrit{Nandā}, \textsanskrit{Nandavatī}, and \textsanskrit{Sundarīnandā}. When that brahmin eventually died, he was reborn as a goose, whose feathers were all made of gold. And he gave his former family members one feather at the time. \textsanskrit{Thullanandā} considered this. She then grabbed hold of that king of geese and plucked him. But when his feathers regrew they were white. At that time too, monks, \textsanskrit{Thullanandā} lost her gold because she was too greedy. Now she will lose her garlic.” 

\begin{verse}%
“One\marginnote{1.35} should be content with what one gets, \\
Because excessive greed is bad. \\
After grabbing hold of the king of geese, \\
The gold came to an end.” 

%
\end{verse}

Then,\marginnote{1.39} after rebuking the nun \textsanskrit{Thullanandā} in many ways, the Buddha spoke in dispraise of being difficult to support … “And, monks, the nuns should recite this training rule like this: 

\subsection*{Final ruling }

\scrule{‘If a nun eats garlic, she commits an offense entailing confession.’” }

\subsection*{Definitions }

\begin{description}%
\item[A: ] whoever … %
\item[Nun: ] … The nun who has been given the full ordination in unanimity by both Sanghas through a legal procedure consisting of one motion and three announcements that is irreversible and fit to stand—this sort of nun is meant in this case. %
\item[Garlic: ] from Magadha is what is meant. %
\end{description}

If\marginnote{2.1.7} she receives it with the intention of eating it, she commits an offense of wrong conduct.\footnote{The punctuation of the Pali should presumably be amended to \textit{‘\textsanskrit{Khādissāmī}”ti \textsanskrit{paṭiggaṇhāti}, \textsanskrit{āpatti} \textsanskrit{dukkaṭassa}}. } For every mouthful, she commits an offense entailing confession. 

\subsection*{Permutations }

If\marginnote{2.2.1} it is garlic and she perceives it as such, and she eats it, she commits an offense entailing confession. If it is garlic, but she is unsure of it, and she eats it, she commits an offense entailing confession. If it is garlic, but she does not perceive it as such, and she eats it, she commits an offense entailing confession. 

If\marginnote{2.2.4} it is not garlic, but she perceives it as such, and she eats it, she commits an offense of wrong conduct. If it is not garlic, but she is unsure of it, and she eats it, she commits an offense of wrong conduct. If it is not garlic and she does not perceive it as such, and she eats it, there is no offense. 

\subsection*{Non-offenses }

There\marginnote{2.3.1} is no offense: if it is an onion;  if it is a shallot;\footnote{DOP says of the \textit{\textsanskrit{bhañjanaka}} that it is “a kind of onion or similar vegetable”. The commentarial description, however, is an almost perfect fit for a shallot. Sp 2.797: \textsanskrit{Bhañjanako} \textsanskrit{lohitavaṇṇo}. … \textsanskrit{Miñjāya} pana … \textsanskrit{bhañjanakassa} dve, “The \textsanskrit{bhañjanaka} is red. … But in regard to cloves … the \textsanskrit{bhañjanaka} has two.” }  if it is chebulic myrobalan;\footnote{The botanical name is \textit{Terminalia chebula}, see SED and SAF, p. 57. It is also known as “black myrobalan”. }  if it is a spring onion;\footnote{DOP suggests “spring onion” for \textit{\textsanskrit{cāpalasuṇa}} with a question mark. Its identity therefore remains uncertain. Sp 2.797: \textit{\textsanskrit{Cāpalasuṇo} \textsanskrit{amiñjako}, \textsanskrit{aṅkuramattameva} hi tassa hoti}, “The \textit{\textsanskrit{cāpalasuṇa}} does not have cloves. It is just a sprout.” }  if it is cooked in a bean curry;  if it is cooked with meat;  if it is cooked with oil;  if it is in sweets;\footnote{“Sweets” renders \textit{\textsanskrit{sāḷava}}. SED, sv. \textit{\textsanskrit{shāḍava}}, says: “confectionery, sweetmeats”. }  if it is a special curry;  if she is insane;  if she is the first offender. 

\scendsutta{The training rule on garlic, the first, is finished. }

%
\section*{{\suttatitleacronym Bi Pc 2}{\suttatitletranslation The training rule on the hair of the private parts }{\suttatitleroot Sambādhaloma}}
\addcontentsline{toc}{section}{\tocacronym{Bi Pc 2} \toctranslation{The training rule on the hair of the private parts } \tocroot{Sambādhaloma}}
\markboth{The training rule on the hair of the private parts }{Sambādhaloma}
\extramarks{Bi Pc 2}{Bi Pc 2}

\subsection*{Origin story }

At\marginnote{1.1} one time the Buddha was staying at \textsanskrit{Sāvatthī} in the Jeta Grove, \textsanskrit{Anāthapiṇḍika}’s Monastery. At that time the nuns from the group of six removed the hair from their private parts. They then bathed naked at a ford in the river \textsanskrit{Aciravatī} together with sex workers. The sex workers complained and criticized them, “How can the nuns remove the hair from their private parts? They’re just like householders who indulge in worldly pleasures!” 

The\marginnote{1.5} nuns heard the complaints of those sex workers, and the nuns of few desires complained and criticized them, “How can the nuns from the group of six remove the hair from their private parts?” … “Is it true, monks, that those nuns do that?” 

“It’s\marginnote{1.9} true, Sir.” 

The\marginnote{1.10} Buddha rebuked them, “How can the nuns from the group of six do that? This will affect people’s confidence …” … “And, monks, the nuns should recite this training rule like this: 

\subsection*{Final ruling }

\scrule{‘If a nun removes hair from her private parts, she commits an offense entailing confession.’” }

\subsection*{Definitions }

\begin{description}%
\item[A: ] whoever … %
\item[Nun: ] … The nun who has been given the full ordination in unanimity by both Sanghas through a legal procedure consisting of one motion and three announcements that is irreversible and fit to stand—this sort of nun is meant in this case. %
\item[The private parts: ] both armpits and the groin. %
\item[Removes: ] if she removes one hair, she commits an offense entailing confession. Even if she removes many hairs, she commits one offense entailing confession. %
\end{description}

\subsection*{Non-offenses }

There\marginnote{2.10.1} is no offense: if she does it because she is sick;  if she is insane;  if she is the first offender. 

\scendsutta{The second training rule is finished. }

%
\section*{{\suttatitleacronym Bi Pc 3}{\suttatitletranslation The training rule on slapping with the palm of the hand }{\suttatitleroot Talaghātaka}}
\addcontentsline{toc}{section}{\tocacronym{Bi Pc 3} \toctranslation{The training rule on slapping with the palm of the hand } \tocroot{Talaghātaka}}
\markboth{The training rule on slapping with the palm of the hand }{Talaghātaka}
\extramarks{Bi Pc 3}{Bi Pc 3}

\subsection*{Origin story }

At\marginnote{1.1} one time when the Buddha was staying at \textsanskrit{Sāvatthī} in \textsanskrit{Anāthapiṇḍika}’s Monastery, two nuns who were plagued by lust entered their room and slapped their genitals with the palms of their hands. Hearing the sound, the nuns rushed up and asked them, “Venerables, are you having sex with a man?” 

“No,\marginnote{1.5} we’re not,” and they told them what had happened. 

The\marginnote{1.6} nuns of few desires complained and criticized them, “How can nuns slap their genitals with their hands?” … “Is it true, monks, that nuns did that?” 

“It’s\marginnote{1.9} true, Sir.” 

The\marginnote{1.10} Buddha rebuked them, “How could nuns do that? This will affect people’s confidence …” … “And, monks, the nuns should recite this training rule like this: 

\subsection*{Final ruling }

\scrule{‘If a nun slaps her genitals with the palm of her hand, she commits an offense entailing confession.’”\footnote{“Slaps her genitals with the palm of her hand” renders \textit{\textsanskrit{talaghātaka}}, literally, “Hits with the palm.” It seems from the origin story, however, that this was an indirect expression referring to the genitals. Sp 2.803 supports this interpretation: \textit{\textsanskrit{Talaghātaketi} \textsanskrit{muttakaraṇatalaghātane}}, “\textit{\textsanskrit{Talaghātake}}: hitting the genitals with the palm of the hand.” } }

\subsection*{Definitions }

\begin{description}%
\item[Slaps her genitals with the palm of her hand: ] if, consenting to the contact, she hits her genitals, even with a lotus leaf, she commits an offense entailing confession. %
\end{description}

\subsection*{Non-offenses }

There\marginnote{2.3.1} is no offense: if she does it because she is sick;  if she is insane;  if she is the first offender. 

\scendsutta{The third training rule is finished. }

%
\section*{{\suttatitleacronym Bi Pc 4}{\suttatitletranslation The training rule on dildos }{\suttatitleroot Jatumaṭṭhaka}}
\addcontentsline{toc}{section}{\tocacronym{Bi Pc 4} \toctranslation{The training rule on dildos } \tocroot{Jatumaṭṭhaka}}
\markboth{The training rule on dildos }{Jatumaṭṭhaka}
\extramarks{Bi Pc 4}{Bi Pc 4}

\subsection*{Origin story }

At\marginnote{1.1} one time the Buddha was staying at \textsanskrit{Sāvatthī} in the Jeta Grove, \textsanskrit{Anāthapiṇḍika}’s Monastery. At that time a woman who had previously belonged to the king’s harem had gone forth as a nun. Another nun who was plagued by lust went to that nun and said, “Venerable, the king only came to you at long intervals. How did you cope?” 

“With\marginnote{1.6} a dildo.” 

“What’s\marginnote{1.7} a dildo?” 

That\marginnote{1.8} nun described a dildo to her. The other nun then used a dildo. But she forgot to wash it before disposing of it in a certain place. The nuns saw it covered with flies, and they said, “Who did this?” 

“I\marginnote{1.12} did it,” she replied. 

The\marginnote{1.13} nuns of few desires complained and criticized her, “How could a nun use a dildo?” … “Is it true, monks, that a nun did this?” 

“It’s\marginnote{1.16} true, Sir.” 

The\marginnote{1.17} Buddha rebuked her, “How could a nun use a dildo? This will affect people’s confidence …” … “And, monks, the nuns should recite this training rule like this: 

\subsection*{Final ruling }

\scrule{‘If a nun uses a dildo, she commits an offense entailing confession.’”\footnote{Sp 2.807: \textit{\textsanskrit{Jatumaṭṭhaketi} \textsanskrit{jatunā} kate \textsanskrit{maṭṭhadaṇḍake}}, “\textit{\textsanskrit{Jatumaṭṭhake}}: a polished rod made of resin.” } }

\subsection*{Definitions }

\begin{description}%
\item[A dildo: ] made of resin, made of wood, made of flour, made of clay. %
\item[Uses: ] if she consents to the contact and inserts it into her vagina, even if it is just a lotus leaf, she commits an offense entailing confession. %
\end{description}

\subsection*{Non-offenses }

There\marginnote{2.5.1} is no offense: if she does it because she is sick;  if she is insane;  if she is the first offender. 

\scendsutta{The fourth training rule is finished. }

%
\section*{{\suttatitleacronym Bi Pc 5}{\suttatitletranslation The training rule on cleaning with water }{\suttatitleroot Udakasuddhika}}
\addcontentsline{toc}{section}{\tocacronym{Bi Pc 5} \toctranslation{The training rule on cleaning with water } \tocroot{Udakasuddhika}}
\markboth{The training rule on cleaning with water }{Udakasuddhika}
\extramarks{Bi Pc 5}{Bi Pc 5}

\subsection*{Origin story }

At\marginnote{1.1.1} one time when the Buddha was staying in the Sakyan country in the Banyan Tree Monastery at Kapilavatthu, \textsanskrit{Mahāpajāpati} \textsanskrit{Gotamī} went to him and bowed. Standing downwind from him, she said, “Sir, women smell.” 

“Well\marginnote{1.1.4} then, the nuns should clean themselves with water.” And the Buddha instructed, inspired, and gladdened her with a teaching, after which she bowed down, circumambulated him with her right side toward him, and left. Soon afterwards the Buddha gave a teaching and addressed the monks: 

\scrule{“Monks, I allow the nuns to clean themselves with water.” }

Being\marginnote{1.2.1} aware that the Buddha had allowed cleaning with water, a nun did it too deeply, causing a sore in her vagina. 

She\marginnote{1.2.3} told the nuns what had happened. The nuns of few desires complained and criticized her, “How could a nun clean herself too deeply with water?” … “Is it true, monks, that a nun did this?” 

“It’s\marginnote{1.2.7} true, Sir.” 

The\marginnote{1.2.8} Buddha rebuked her, “How could a nun clean herself too deeply with water? This will affect people’s confidence …” … “And, monks, the nuns should recite this training rule like this: 

\subsection*{Final ruling }

\scrule{‘If a nun is cleaning herself with water, she may insert two finger joints at the most. If she goes further than that, she commits an offense entailing confession.’” }

\subsection*{Definitions }

\begin{description}%
\item[Is cleaning herself with water: ] rinsing the vagina is what is meant. %
\item[Cleaning: ] rinses. %
\item[She may insert two finger joints at the most: ] she may insert two joints of two fingers at the most. %
\item[If she goes further than that: ] if she consents to the contact and goes further even by a hair’s breadth, she commits an offense entailing confession. %
\end{description}

\subsection*{Permutations }

If\marginnote{2.2.1} it is more than two finger joints, and she perceives it as more, and she inserts them, she commits an offense entailing confession. If it is more than two finger joints, but she is unsure of it, and she inserts them, she commits an offense entailing confession. If it is more than two finger joints, but she perceives it as less, and she inserts them, she commits an offense entailing confession. 

If\marginnote{2.2.4} it is less than two finger joints, but she perceives it as more, she commits an offense of wrong conduct. If it is less than two finger joints, but she is unsure of it, she commits an offense of wrong conduct. If it is less than two finger joints, and she perceives it as less, there is no offense. 

\subsection*{Non-offenses }

There\marginnote{2.3.1} is no offense: if she inserts two finger joints;  if she inserts less than two finger joints;  if she does it because she is sick;  if she is insane;  if she is the first offender. 

\scendsutta{The fifth training rule is finished. }

%
\section*{{\suttatitleacronym Bi Pc 6}{\suttatitletranslation The training rule on attending on }{\suttatitleroot Upatiṭṭhana}}
\addcontentsline{toc}{section}{\tocacronym{Bi Pc 6} \toctranslation{The training rule on attending on } \tocroot{Upatiṭṭhana}}
\markboth{The training rule on attending on }{Upatiṭṭhana}
\extramarks{Bi Pc 6}{Bi Pc 6}

\subsection*{Origin story }

At\marginnote{1.1} one time the Buddha was staying at \textsanskrit{Sāvatthī} in \textsanskrit{Anāthapiṇḍika}’s Monastery, a government official called Ārohanta became a monk and his ex-wife a nun. On one occasion that monk was having his meal in the presence of that nun. While he was eating, she attended on him with drinking water and a fan, and she flirted with him. But he dismissed her, saying, “Don’t do that; it’s not allowable.” 

“Before\marginnote{1.9} you did such and such to me, but now you can’t even take this much.” And she dropped the water vessel on its head and struck him with the fan.\footnote{It is not clear whether \textit{matthake}, “on the head”, refers to the water vessel or the monk. I have assumed it is the latter since it seems more meaningful. } 

The\marginnote{1.11} nuns of few desires complained and criticized her, “How could a nun hit a monk?” … “Is it true, monks, that a nun did this?” 

“It’s\marginnote{1.14} true, Sir.” 

The\marginnote{1.15} Buddha rebuked her, “How could a nun hit a monk? This will affect people’s confidence …” … “And, monks, the nuns should recite this training rule like this: 

\subsection*{Final ruling }

\scrule{‘If, when a monk is eating, a nun attends on him with drinking water or a fan, she commits an offense entailing confession.’” }

\subsection*{Definitions }

\begin{description}%
\item[A: ] whoever … %
\item[Nun: ] … The nun who has been given the full ordination in unanimity by both Sanghas through a legal procedure consisting of one motion and three announcements that is irreversible and fit to stand—this sort of nun is meant in this case. %
\item[A monk: ] fully ordained. %
\item[Is eating: ] is eating any of the five cooked foods. %
\item[Drinking water: ] any kind of drink. %
\item[A fan: ] any kind of fan. %
\item[Attends on: ] if she stands within arm’s reach, she commits an offense entailing confession. %
\end{description}

\subsection*{Permutations }

If\marginnote{2.2.1} he is fully ordained, and she perceives him as such, and she attends on him with drinking water or a fan, she commits an offense entailing confession. If he is fully ordained, but she is unsure of it, and she attends on him with drinking water or a fan, she commits an offense entailing confession. If he is fully ordained, but she does not perceive him as such, and she attends on him with drinking water or a fan, she commits an offense entailing confession. 

If\marginnote{2.2.4} she attends on him from beyond arm’s reach, she commits an offense of wrong conduct. If she attends on him when he is eating fresh food, she commits an offense of wrong conduct. If she attends on one who is not fully ordained, she commits an offense of wrong conduct. 

If\marginnote{2.2.7} he is not fully ordained, but she perceives him as such, she commits an offense of wrong conduct. If he is not fully ordained, but she is unsure of it, she commits an offense of wrong conduct. If he is not fully ordained, and she does not perceive him as such, she commits an offense of wrong conduct. 

\subsection*{Non-offenses }

There\marginnote{2.3.1} is no offense: if she gives something;  if she has someone else give something;  if she asks someone who is not fully ordained to do it;  if she is insane;  if she is the first offender. 

\scendsutta{The sixth training rule is finished. }

%
\section*{{\suttatitleacronym Bi Pc 7}{\suttatitletranslation The training rule on raw grain }{\suttatitleroot Āmakadhañña}}
\addcontentsline{toc}{section}{\tocacronym{Bi Pc 7} \toctranslation{The training rule on raw grain } \tocroot{Āmakadhañña}}
\markboth{The training rule on raw grain }{Āmakadhañña}
\extramarks{Bi Pc 7}{Bi Pc 7}

\subsection*{Origin story }

At\marginnote{1.1} one time when the Buddha was staying at \textsanskrit{Sāvatthī} in \textsanskrit{Anāthapiṇḍika}’s Monastery, it was the harvest season. At that time the nuns had asked for raw grain, which they then carried to town. At the town gate they were detained and told, “Venerables, give a share.” 

After\marginnote{1.4} being released, they went to the nuns’ dwelling place and told the nuns what had happened. The nuns of few desires complained and criticized them, “How could nuns ask for raw grain?” … “Is it true, monks, that nuns did this?” 

“It’s\marginnote{1.9} true, Sir.” 

The\marginnote{1.10} Buddha rebuked them, “How could nuns do this? This will affect people’s confidence …” … “And, monks, the nuns should recite this training rule like this: 

\subsection*{Final ruling }

\scrule{‘If a nun asks for or has someone else ask for raw grain, or she roasts it or has it roasted, or she pounds it or has it pounded, or she cooks it or has it cooked, and she then eats it, she commits an offense entailing confession.’” }

\subsection*{Definitions }

\begin{description}%
\item[A: ] whoever … %
\item[Nun: ] … The nun who has been given the full ordination in unanimity by both Sanghas through a legal procedure consisting of one motion and three announcements that is irreversible and fit to stand—this sort of nun is meant in this case. %
\item[Raw grain: ] rice, barley, wheat, millet, wild gram, kodo millet.\footnote{I render \textit{\textsanskrit{sāli}} and \textit{\textsanskrit{vīhi}} with the single word “rice”. In total there are seven grains. } %
\item[Asks for: ] she asks herself. %
\item[Has someone ask for: ] she gets someone else to ask. %
\item[Roasts: ] she roasts it herself. %
\item[Has it roasted: ] she gets someone else to roast it. %
\item[Pounds: ] she pounds it herself. %
\item[Has it pounded: ] she gets someone else to pound it. %
\item[Cooks: ] she cooks it herself. %
\item[Has it cooked: ] she gets someone else to cook it. %
\end{description}

If\marginnote{2.1.23} she receives it with the intention of eating it, she commits an offense of wrong conduct. For every mouthful, she commits an offense entailing confession. 

\subsection*{Non-offenses }

There\marginnote{2.2.1} is no offense: if she does it because she is sick;  if she asks for vegetables;  if she is insane;  if she is the first offender. 

\scendsutta{The seventh training rule is finished. }

%
\section*{{\suttatitleacronym Bi Pc 8}{\suttatitletranslation The training rule on disposing of feces }{\suttatitleroot Tirokuṭṭuccārachaḍḍana}}
\addcontentsline{toc}{section}{\tocacronym{Bi Pc 8} \toctranslation{The training rule on disposing of feces } \tocroot{Tirokuṭṭuccārachaḍḍana}}
\markboth{The training rule on disposing of feces }{Tirokuṭṭuccārachaḍḍana}
\extramarks{Bi Pc 8}{Bi Pc 8}

\subsection*{Origin story }

At\marginnote{1.1} one time when the Buddha was staying at \textsanskrit{Sāvatthī} in \textsanskrit{Anāthapiṇḍika}’s Monastery, a brahmin who had earned money by working for the king thought, “I’ll ask for my wages.” After washing his hair, he walked past the nuns’ dwelling place on his way to the king’s residence. Just then, after defecating in a pot, a nun disposed of the feces over a wall, and it landed on the head of that brahmin. He complained and criticized the nuns, “They’re not monastics, these shaven-headed sluts! How can they dump shit on my head? I’m gonna burn their place down!” And he got hold of a firebrand and entered their dwelling place. Just then a lay follower who was coming out from the nuns’ dwelling place saw that brahmin with a firebrand on his way in, and he said to him, “Sir, why are you entering the nuns’ dwelling place with a firebrand?” 

“These\marginnote{1.12} shaven-headed sluts dumped shit on my head. I’m gonna burn their place down!” 

“But\marginnote{1.14} this is auspicious, brahmin! You’ll get your wages and a thousand coins in addition.” 

That\marginnote{1.16} brahmin then washed his hair, went to the king’s residence, and he got his wages and a thousand coins in addition. 

But\marginnote{1.17} that lay follower returned to the nuns’ dwelling place, told them what had happened, and then scolded them. The nuns of few desires complained and criticized them, “How can nuns dispose of feces over a wall?” … “Is it true, monks, that nuns do this?” 

“It’s\marginnote{1.21} true, Sir.” 

The\marginnote{1.22} Buddha rebuked them, “How can nuns dispose of feces over a wall? This will affect people’s confidence …” … “And, monks, the nuns should recite this training rule like this: 

\subsection*{Final ruling }

\scrule{‘If a nun disposes of feces or urine or trash or food scraps over a wall or over an encircling wall, or she has it disposed of in this way, she commits an offense entailing confession.’” }

\subsection*{Definitions }

\begin{description}%
\item[A: ] whoever … %
\item[Nun: ] … The nun who has been given the full ordination in unanimity by both Sanghas through a legal procedure consisting of one motion and three announcements that is irreversible and fit to stand—this sort of nun is meant in this case. %
\item[Feces: ] excrement is what is meant. %
\item[Urine: ] pee is what is meant. %
\item[Trash: ] refuse is what is meant. %
\item[Food scraps: ] food remnants or bones or used water. %
\item[A wall: ] there are three kinds of walls: walls made of bricks, walls made of stone, walls made of wood. %
\item[An encircling wall: ] there are three kinds of encircling walls: encircling walls made of bricks, encircling walls made of stone, encircling walls made of wood. %
\item[Over a wall: ] to the other side of the wall. %
\item[Over an encircling wall: ] to the other side of the encircling wall. %
\item[Disposes of: ] if she disposes of it herself, she commits an offense entailing confession. %
\item[Has it disposed of: ] in asking another, she commits an offense of wrong conduct. If she only asks once, then even if the other disposes of such things many times, she commits one offense entailing confession. %
\end{description}

\subsection*{Non-offenses }

There\marginnote{2.2.1} is no offense: if she disposes of it after having looked;  if she disposes of it at a place where no one passes by;  if she is insane;  if she is the first offender. 

\scendsutta{The eighth training rule is finished. }

%
\section*{{\suttatitleacronym Bi Pc 9}{\suttatitletranslation The second training rule on disposing of feces }{\suttatitleroot Harituccārachaḍḍana}}
\addcontentsline{toc}{section}{\tocacronym{Bi Pc 9} \toctranslation{The second training rule on disposing of feces } \tocroot{Harituccārachaḍḍana}}
\markboth{The second training rule on disposing of feces }{Harituccārachaḍḍana}
\extramarks{Bi Pc 9}{Bi Pc 9}

\subsection*{Origin story }

At\marginnote{1.1} one time the Buddha was staying at \textsanskrit{Sāvatthī} in the Jeta Grove, \textsanskrit{Anāthapiṇḍika}’s Monastery. At that time a brahmin had a barley field next to the nuns’ dwelling place. The nuns disposed of feces, urine, trash, and food scraps in that field. The brahmin complained and criticized them, “How could the nuns spoil my barley field?”\footnote{“Spoil” renders \textit{\textsanskrit{dūsessanti}}. For a discussion of this word, see Appendix of Technical Terms. } 

The\marginnote{1.6} nuns heard the complaints of that brahmin, and the nuns of few desires complained and criticized them, “How could nuns dispose of feces, urine, trash, and food scraps on cultivated plants?” … “Is it true, monks, that nuns did this?” 

“It’s\marginnote{1.10} true, Sir.” 

The\marginnote{1.11} Buddha rebuked them, “How could nuns do this? This will affect people’s confidence …” … “And, monks, the nuns should recite this training rule like this: 

\subsection*{Final ruling }

\scrule{‘If a nun disposes of feces or urine or trash or food scraps on cultivated plants, or she has it disposed of in this way, she commits an offense entailing confession.’” }

\subsection*{Definitions }

\begin{description}%
\item[A: ] whoever … %
\item[Nun: ] … The nun who has been given the full ordination in unanimity by both Sanghas through a legal procedure consisting of one motion and three announcements that is irreversible and fit to stand—this sort of nun is meant in this case. %
\item[Feces: ] excrement is what is meant. %
\item[Urine: ] pee is what is meant. %
\item[Trash: ] refuse is what is meant. %
\item[Food scraps: ] food remnants or bones or used water. %
\item[Cultivated plants: ] grain, vegetables, whatever cultivated plants people consider valuable or useful. %
\item[Disposes of: ] if she disposes of it herself, she commits an offense entailing confession. %
\item[Has it disposed of: ] in asking another, she commits an offense of wrong conduct. If she only asks once, then even if the other disposes of such things many times, she commits one offense entailing confession. %
\end{description}

\subsection*{Permutations }

If\marginnote{2.2.1} they are cultivated plants, and she perceives them as such, and she disposes of it or has it disposed of, she commits an offense entailing confession. If they are cultivated plants, but she is unsure of it, and she disposes of it or has it disposed of, she commits an offense entailing confession. If they are cultivated plants, but she perceives them as uncultivated, and she disposes of it or has it disposed of, she commits an offense entailing confession. 

If\marginnote{2.2.4} they are uncultivated plants, but she perceives them as cultivated, she commits an offense of wrong conduct. If they are uncultivated plants, but she is unsure of it, she commits an offense of wrong conduct. If they are uncultivated plants, and she perceives them as such, there is no offense. 

\subsection*{Non-offenses }

There\marginnote{2.3.1} is no offense: if she disposes of it after having looked;  if she disposes of it at the edge of the field;  if she disposes of it after asking and getting permission from the owners;  if she is insane;  if she is the first offender. 

\scendsutta{The ninth training rule is finished. }

%
\section*{{\suttatitleacronym Bi Pc 10}{\suttatitletranslation The training rule on dancing and singing }{\suttatitleroot Naccagīta}}
\addcontentsline{toc}{section}{\tocacronym{Bi Pc 10} \toctranslation{The training rule on dancing and singing } \tocroot{Naccagīta}}
\markboth{The training rule on dancing and singing }{Naccagīta}
\extramarks{Bi Pc 10}{Bi Pc 10}

\subsection*{Origin story }

At\marginnote{1.1} one time the Buddha was staying at \textsanskrit{Rājagaha} in the Bamboo Grove, the squirrel sanctuary. At that time in \textsanskrit{Rājagaha} there was a hilltop fair, and the nuns from the group of six went to see it. People complained and criticized them, “How can nuns go to see dancing, singing, and music? They’re just like householders who indulge in worldly pleasures!” 

The\marginnote{1.6} nuns heard the complaints of those people, and the nuns of few desires complained and criticized them, “How could the nuns from the group of six go to see dancing, singing, and music?” … “Is it true, monks, that those nuns did that?” 

“It’s\marginnote{1.10} true, Sir.” 

The\marginnote{1.11} Buddha rebuked them, “How could the nuns from the group of six do this? This will affect people’s confidence …” … “And, monks, the nuns should recite this training rule like this: 

\subsection*{Final ruling }

\scrule{‘If a nun goes to see dancing or singing or music, she commits an offense entailing confession.’” }

\subsection*{Definitions }

\begin{description}%
\item[A: ] whoever … %
\item[Nun: ] … The nun who has been given the full ordination in unanimity by both Sanghas through a legal procedure consisting of one motion and three announcements that is irreversible and fit to stand—this sort of nun is meant in this case. %
\item[Dancing: ] any kind of dancing. %
\item[Singing: ] any kind of singing. %
\item[Music: ] any kind of music. %
\end{description}

If\marginnote{2.1.11} she is on her way to see it, she commits an offense of wrong conduct. Wherever she stands to see it or hear it, she commits an offense entailing confession. Every time she goes beyond the range of sight and then sees it or hears it again, she commits an offense entailing confession. 

If\marginnote{2.1.14} she is on her way to see any one of the three, she commits an offense of wrong conduct. Wherever she stands to see it or hear it, she commits an offense entailing confession. Every time she goes beyond the range of sight and then sees it or hears it again, she commits an offense entailing confession. 

\subsection*{Non-offenses }

There\marginnote{2.2.1} is no offense: if she sees it or hears it while remaining in the monastery;  if the dancing, singing, or music comes to the place where the nun is standing, sitting, or lying down;  if she sees it or hears it while walking in the opposite direction;  if she goes there because there is something to be done and she then sees it or hears it;  if there is an emergency;  if she is insane;  if she is the first offender. 

\scendsutta{The tenth training rule is finished. }

\scendvagga{The first subchapter on garlic is finished. }

%
\section*{{\suttatitleacronym Bi Pc 11}{\suttatitletranslation The training rule on the dark of the night }{\suttatitleroot Rattandhakāra}}
\addcontentsline{toc}{section}{\tocacronym{Bi Pc 11} \toctranslation{The training rule on the dark of the night } \tocroot{Rattandhakāra}}
\markboth{The training rule on the dark of the night }{Rattandhakāra}
\extramarks{Bi Pc 11}{Bi Pc 11}

\subsection*{Origin story }

At\marginnote{1.1} one time when the Buddha was staying at \textsanskrit{Sāvatthī} in \textsanskrit{Anāthapiṇḍika}’s Monastery, a male relative of a nun who was a pupil of \textsanskrit{Bhaddā} \textsanskrit{Kāpilānī} went from his own village to \textsanskrit{Sāvatthī} on some business. Then, in the dark of the night and without a lamp, that nun stood and talked alone with that man. 

The\marginnote{1.4} nuns of few desires complained and criticized her, “How could a nun do such a thing?” … “Is it true, monks, that a nun did this?” 

“It’s\marginnote{1.7} true, Sir.” 

The\marginnote{1.8} Buddha rebuked her … “How could a nun do such a thing? This will affect people’s confidence …” … “And, monks, the nuns should recite this training rule like this: 

\subsection*{Final ruling }

\scrule{‘If, in the dark of the night without a lamp, a nun stands or talks alone with a man, she commits an offense entailing confession.’” }

\subsection*{Definitions }

\begin{description}%
\item[A: ] whoever … %
\item[Nun: ] … The nun who has been given the full ordination in unanimity by both Sanghas through a legal procedure consisting of one motion and three announcements that is irreversible and fit to stand—this sort of nun is meant in this case. %
\item[In the dark of the night: ] when the sun has set. %
\item[Without a lamp: ] without light. %
\item[A man: ] a human male, not a male spirit, not a male ghost, not a male animal. He understands and is capable of standing together and talking. %
\item[With: ] together. %
\item[Alone: ] just the man and the nun. %
\item[Stands with: ] if she stands within arm’s reach of the man, she commits an offense entailing confession. %
\item[Talks with: ] if she stands talking within arm’s reach of the man, she commits an offense entailing confession. %
\end{description}

If\marginnote{2.1.19} she stands or talks outside of arm’s reach, she commits an offense of wrong conduct. If she stands or talks with a male spirit, a male ghost, a \textit{\textsanskrit{paṇḍaka}}, or a male animal in human form, she commits an offense of wrong conduct.\footnote{For a discussion of \textit{\textsanskrit{paṇḍaka}}, see Appendix of Technical Terms. } 

\subsection*{Non-offenses }

There\marginnote{2.2.1} is no offense: if she has a companion who understands;  if she is not seeking privacy;  if she stands or talks thinking of something else;\footnote{Sp 5.467: \textit{\textsanskrit{Aññavihitoti} \textsanskrit{aññaṁ} \textsanskrit{cintayamāno}}, “\textit{\textsanskrit{Aññavihita}}: thinking of something else.” }  if she is insane;  if she is the first offender. 

\scendsutta{The first training rule is finished. }

%
\section*{{\suttatitleacronym Bi Pc 12}{\suttatitletranslation The training rule on concealed places }{\suttatitleroot Paṭicchannokāsa}}
\addcontentsline{toc}{section}{\tocacronym{Bi Pc 12} \toctranslation{The training rule on concealed places } \tocroot{Paṭicchannokāsa}}
\markboth{The training rule on concealed places }{Paṭicchannokāsa}
\extramarks{Bi Pc 12}{Bi Pc 12}

\subsection*{Origin story }

At\marginnote{1.1} one time when the Buddha was staying at \textsanskrit{Sāvatthī} in \textsanskrit{Anāthapiṇḍika}’s Monastery, a male relative of a nun who was a pupil of \textsanskrit{Bhaddā} \textsanskrit{Kāpilānī} went from his own village to \textsanskrit{Sāvatthī} on some business. Then, knowing that the Buddha had prohibited standing or talking alone with a man in the dark of the night without a lamp, she instead stood and talked alone with that man in a concealed place. 

The\marginnote{1.5} nuns of few desires complained and criticized her, “How could a nun do such a thing?” … “Is it true, monks, that a nun did this?” 

“It’s\marginnote{1.8} true, Sir.” 

The\marginnote{1.9} Buddha rebuked her … “How could a nun do such a thing? This will affect people’s confidence …” … “And, monks, the nuns should recite this training rule like this: 

\subsection*{Final ruling }

\scrule{‘If a nun stands or talks alone with a man in a concealed place, she commits an offense entailing confession.’” }

\subsection*{Definitions }

\begin{description}%
\item[A: ] whoever … %
\item[Nun: ] … The nun who has been given the full ordination in unanimity by both Sanghas through a legal procedure consisting of one motion and three announcements that is irreversible and fit to stand—this sort of nun is meant in this case. %
\item[In a concealed place: ] it is concealed by a wall, a door, a screen, a cloth screen, a tree, a pillar, a grain container, or anything else. %
\item[A man: ] a human male, not a male spirit, not a male ghost, not a male animal. He understands and is capable of standing together and talking. %
\item[With: ] together. %
\item[Alone: ] just the man and the nun. %
\item[Stands with: ] if she stands within arm’s reach of the man, she commits an offense entailing confession. %
\item[Talks with: ] if she stands talking within arm’s reach of the man, she commits an offense entailing confession. %
\end{description}

If\marginnote{2.17} she stands or talks outside of arm’s reach, she commits an offense of wrong conduct. If she stands or talks with a male spirit, a male ghost, a \textit{\textsanskrit{paṇḍaka}}, or a male animal in human form, she commits an offense of wrong conduct. 

\subsection*{Non-offenses }

There\marginnote{2.19.1} is no offense: if she has a companion who understands;  if she is not seeking privacy;  if she stands or talks thinking of something else;\footnote{Sp 5.467: \textit{\textsanskrit{Aññavihitoti} \textsanskrit{aññaṁ} \textsanskrit{cintayamāno}}, “\textit{\textsanskrit{Aññavihita}}: thinking of something else.” }  if she is insane;  if she is the first offender. 

\scendsutta{The second training rule is finished. }

%
\section*{{\suttatitleacronym Bi Pc 13}{\suttatitletranslation The training rule on talking out in the open }{\suttatitleroot Ajjhokāsasallapana}}
\addcontentsline{toc}{section}{\tocacronym{Bi Pc 13} \toctranslation{The training rule on talking out in the open } \tocroot{Ajjhokāsasallapana}}
\markboth{The training rule on talking out in the open }{Ajjhokāsasallapana}
\extramarks{Bi Pc 13}{Bi Pc 13}

\subsection*{Origin story }

At\marginnote{1.1} one time when the Buddha was staying at \textsanskrit{Sāvatthī} in \textsanskrit{Anāthapiṇḍika}’s Monastery, a male relative of a nun who was a pupil of \textsanskrit{Bhaddā} \textsanskrit{Kāpilānī} went from his own village to \textsanskrit{Sāvatthī} on some business. Then, knowing that the Buddha had prohibited standing or talking alone with a man in a concealed place, she instead stood and talked alone with that man out in the open. 

The\marginnote{1.5} nuns of few desires complained and criticized her, “How could a nun do such a thing?” … “Is it true, monks, that a nun did this?” 

“It’s\marginnote{1.8} true, Sir.” 

The\marginnote{1.9} Buddha rebuked her … “How could a nun do such a thing? This will affect people’s confidence …” … “And, monks, the nuns should recite this training rule like this: 

\subsection*{Final ruling }

\scrule{‘If a nun stands or talks alone with a man out in the open, she commits an offense entailing confession.’” }

\subsection*{Definitions }

\begin{description}%
\item[A: ] whoever … %
\item[Nun: ] … The nun who has been given the full ordination in unanimity by both Sanghas through a legal procedure consisting of one motion and three announcements that is irreversible and fit to stand—this sort of nun is meant in this case. %
\item[Out in the open:\footnote{For a discussion of the rendering “out in the open” for \textit{\textsanskrit{ajjhokāsa}}, see Appendix of Technical Terms. } ] not concealed by a wall, a door, a screen, a cloth screen, a tree, a pillar, a grain container, or anything else. %
\item[A man: ] a human male, not a male spirit, not a male ghost, not a male animal. He understands and is capable of standing together and talking. %
\item[With: ] together. %
\item[Alone: ] just the man and the nun. %
\item[Stands with: ] if she stands within arm’s reach of the man, she commits an offense entailing confession. %
\item[Talks with: ] if she stands talking within arm’s reach of the man, she commits an offense entailing confession. %
\end{description}

If\marginnote{2.17} she stands or talks outside of arm’s reach, she commits an offense of wrong conduct. If she stands or talks with a male spirit, a male ghost, a \textit{\textsanskrit{paṇḍaka}}, or a male animal in human form, she commits an offense of wrong conduct. 

\subsection*{Non-offenses }

There\marginnote{2.19.1} is no offense: if she has a companion who understands;  if she is not seeking privacy;  if she stands or talks thinking of something else;\footnote{Sp 5.467: \textit{\textsanskrit{Aññavihitoti} \textsanskrit{aññaṁ} \textsanskrit{cintayamāno}}, “\textit{\textsanskrit{Aññavihita}}: thinking of something else.” }  if she is insane;  if she is the first offender. 

\scendsutta{The third training rule is finished. }

%
\section*{{\suttatitleacronym Bi Pc 14}{\suttatitletranslation The training rule on dismissing a companion }{\suttatitleroot Dutiyikauyyojana}}
\addcontentsline{toc}{section}{\tocacronym{Bi Pc 14} \toctranslation{The training rule on dismissing a companion } \tocroot{Dutiyikauyyojana}}
\markboth{The training rule on dismissing a companion }{Dutiyikauyyojana}
\extramarks{Bi Pc 14}{Bi Pc 14}

\subsection*{Origin story }

At\marginnote{1.1} one time the Buddha was staying at \textsanskrit{Sāvatthī} in the Jeta Grove, \textsanskrit{Anāthapiṇḍika}’s Monastery. At that time the nun \textsanskrit{Thullanandā} stood and talked alone with men on streets, in cul-de-sacs, and at intersections, and she whispered in their ears and dismissed her companion nun. 

The\marginnote{1.3} nuns of few desires complained and criticized her, “How can Venerable \textsanskrit{Thullanandā} do such things?” … “Is it true, monks, that the nun \textsanskrit{Thullanandā} does this?” 

“It’s\marginnote{1.6} true, Sir.” 

The\marginnote{1.7} Buddha rebuked her … “How can the nun \textsanskrit{Thullanandā} do such things? This will affect people’s confidence …” … “And, monks, the nuns should recite this training rule like this: 

\subsection*{Final ruling }

\scrule{‘If a nun stands or talks alone with a man on a street or in a cul-de-sac or at an intersection, or she whispers in his ear or dismisses her companion nun, she commits an offense entailing confession.’” }

\subsection*{Definitions }

\begin{description}%
\item[A: ] whoever … %
\item[Nun: ] … The nun who has been given the full ordination in unanimity by both Sanghas through a legal procedure consisting of one motion and three announcements that is irreversible and fit to stand—this sort of nun is meant in this case. %
\item[A street: ] a carriage road is what is meant. %
\item[A cul-de-sac: ] one departs the same way one enters. %
\item[An intersection: ] a crossroads is what is meant. %
\item[A man: ] a human male, not a male spirit, not a male ghost, not a male animal. He understands and is capable of standing together and talking. %
\item[With: ] together. %
\item[Alone: ] just the man and the nun. %
\item[Stands with: ] if she stands within arm’s reach of the man, she commits an offense entailing confession. %
\item[Talks with: ] if she stands talking within arm’s reach of the man, she commits an offense entailing confession. %
\item[Whispers in his ear: ] if she speaks into the ear of a man, she commits an offense entailing confession. %
\item[Dismisses her companion nun: ] if, wanting to misbehave, she dismisses her companion nun, she commits an offense of wrong conduct. If the companion nun is in the process of going beyond the range of sight or the range of hearing, she commits an offense of wrong conduct. When the companion nun has gone beyond, she commits an offense entailing confession. %
\end{description}

If\marginnote{2.2.1} she stands or talks outside of arm’s reach, she commits an offense of wrong conduct. If she stands or talks with a male spirit, a male ghost, a \textit{\textsanskrit{paṇḍaka}}, or a male animal in human form, she commits an offense of wrong conduct. 

\subsection*{Non-offenses }

There\marginnote{2.2.3.1} is no offense: if she has a companion who understands;\footnote{This non-offense clause is noteworthy in that it seems to contradict the “or” structure of the rule. The rule suggests that there is an offense if any of the five sub-clauses are fulfilled, whereas this non-offense clause says that there is no offense if the last clause is not fulfilled. }  if she is not seeking privacy;  if she stands or talks thinking of something else;\footnote{Sp 5.467: \textit{\textsanskrit{Aññavihitoti} \textsanskrit{aññaṁ} \textsanskrit{cintayamāno}}, “\textit{\textsanskrit{Aññavihita}}: thinking of something else.” }  if she does not want to misbehave;  if she dismisses her companion nun when there is something to be done;  if she is insane;  if she is the first offender. 

\scendsutta{The fourth training rule is finished. }

%
\section*{{\suttatitleacronym Bi Pc 15}{\suttatitletranslation The training rule on departing without informing }{\suttatitleroot Anāpucchāpakkamana}}
\addcontentsline{toc}{section}{\tocacronym{Bi Pc 15} \toctranslation{The training rule on departing without informing } \tocroot{Anāpucchāpakkamana}}
\markboth{The training rule on departing without informing }{Anāpucchāpakkamana}
\extramarks{Bi Pc 15}{Bi Pc 15}

\subsection*{Origin story }

At\marginnote{1.1} one time the Buddha was staying at \textsanskrit{Sāvatthī} in the Jeta Grove, \textsanskrit{Anāthapiṇḍika}’s Monastery. At that time a certain nun was associating with a family from which she received a regular meal. Then, after robing up one morning, she took her bowl and robe and went to that family where she sat down on a seat. She then departed without informing the owners. A slave-woman who was sweeping the house put that seat in between some vessels. Not seeing the seat, soon afterwards the people there asked that nun, “Venerable, where’s that seat?” 

“I\marginnote{1.7} don’t know.” 

“Give\marginnote{1.8} back the seat, Venerable.” And after scolding her, they made an end of her regular meal. Then, while those people were cleaning the house, they saw that seat in between those vessels. They asked that nun for forgiveness and restored her regular meal. 

That\marginnote{1.10} nun then told the nuns what had happened. The nuns of few desires complained and criticized her, “How could a nun visit a family before the meal, sit down on a seat, and then depart without informing the owners?” … “Is it true, monks, that a nun did this?” 

“It’s\marginnote{1.14} true, Sir.” 

The\marginnote{1.15} Buddha rebuked her … “How could a nun act like this? This will affect people’s confidence …” … “And, monks, the nuns should recite this training rule like this: 

\subsection*{Final ruling }

\scrule{‘If a nun visits families before the meal, sits down on a seat, and then departs without informing the owners, she commits an offense entailing confession.’” }

\subsection*{Definitions }

\begin{description}%
\item[A: ] whoever … %
\item[Nun: ] … The nun who has been given the full ordination in unanimity by both Sanghas through a legal procedure consisting of one motion and three announcements that is irreversible and fit to stand—this sort of nun is meant in this case. %
\item[Before the meal: ] from dawn until midday. %
\item[A family: ] there are four kinds of families: the aristocratic family, the brahmin family, the merchant family, the worker family. %
\item[Visits: ] goes there. %
\item[A seat: ] a place for sitting cross-legged is what is meant. %
\item[Sits down: ] sits down on that seat. %
\item[Departs without informing the owners: ] if, without informing a person who understands in that family, she goes beyond the roof cover of that house, she commits an offense entailing confession. If it is out in the open and she goes beyond the vicinity of the seat, she commits an offense entailing confession. %
\end{description}

\subsection*{Permutations }

If\marginnote{2.2.1} she has not informed, and she does not perceive that she has, and she departs, she commits an offense entailing confession. If she has not informed, but she is unsure of it, and she departs, she commits an offense entailing confession. If she has not informed, but she perceives that she has, and she departs, she commits an offense entailing confession. 

If\marginnote{2.2.4} it is not a place for sitting cross-legged, she commits an offense of wrong conduct. If she has informed, but she does not perceive that she has, she commits an offense of wrong conduct. If she has informed, but she is unsure of it, she commits an offense of wrong conduct. If she has informed, and she perceives that she has, there is no offense. 

\subsection*{Non-offenses }

There\marginnote{2.3.1} is no offense: if she departs after informing someone;  if the seat is not movable;  if she is sick;  if there is an emergency;  if she is insane;  if she is the first offender. 

\scendsutta{The fifth training rule is finished. }

%
\section*{{\suttatitleacronym Bi Pc 16}{\suttatitletranslation The training rule on sitting down without asking permission }{\suttatitleroot Anāpucchāabhinisīdana}}
\addcontentsline{toc}{section}{\tocacronym{Bi Pc 16} \toctranslation{The training rule on sitting down without asking permission } \tocroot{Anāpucchāabhinisīdana}}
\markboth{The training rule on sitting down without asking permission }{Anāpucchāabhinisīdana}
\extramarks{Bi Pc 16}{Bi Pc 16}

\subsection*{Origin story }

At\marginnote{1.1} one time the Buddha was staying at \textsanskrit{Sāvatthī} in the Jeta Grove, \textsanskrit{Anāthapiṇḍika}’s Monastery. At that time the nun \textsanskrit{Thullanandā} visited families after the meal, and she sat down and lay down on the seats without asking permission of the owners. Because of \textsanskrit{Thullanandā}, the people there had qualms,  neither sitting nor lying down. They then complained and criticized her, “How could Venerable \textsanskrit{Thullanandā} visit families after the meal, and then sit down and lie down on the seats without asking the owners for permission?” 

The\marginnote{1.6} nuns heard the complaints of those people. The nuns of few desires complained and criticized her, “How could Venerable \textsanskrit{Thullanandā} act in this way?” … “Is it true, monks, that the nun \textsanskrit{Thullanandā} did this?” 

“It’s\marginnote{1.10} true, Sir.” 

The\marginnote{1.11} Buddha rebuked her … “How could the nun \textsanskrit{Thullanandā} act in this way? This will affect people’s confidence …” … “And, monks, the nuns should recite this training rule like this: 

\subsection*{Final ruling }

\scrule{‘If a nun visits families after the meal, and then sits down or lies down on a seat without asking permission of the owners, she commits an offense entailing confession.’” }

\subsection*{Definitions }

\begin{description}%
\item[A: ] whoever … %
\item[Nun: ] … The nun who has been given the full ordination in unanimity by both Sanghas through a legal procedure consisting of one motion and three announcements that is irreversible and fit to stand—this sort of nun is meant in this case. %
\item[After the meal: ] when the middle of the day has passed, until sunset. %
\item[A family: ] there are four kinds of families: the aristocratic family, the brahmin family, the merchant family, the worker family. %
\item[Visits: ] goes there. %
\item[Without asking permission of the owners: ] not having asked permission of a person in that family who is an owner and who has the authority to give. %
\item[A seat: ] a place for sitting cross-legged is what is meant. %
\item[Sits down: ] if she sits down on that seat, she commits an offense entailing confession. %
\item[Lies down: ] if she lies down on that seat, she commits an offense entailing confession. %
\end{description}

\subsection*{Permutations }

If\marginnote{2.2.1} she has not asked permission, and she does not perceive that she has, and she sits down or lies down on a seat, she commits an offense entailing confession. If she has not asked permission, but she is unsure of it, and she sits down or lies down on a seat, she commits an offense entailing confession. If she has not asked permission, but she perceives that she has, and she sits down or lies down on a seat, she commits an offense entailing confession. 

If\marginnote{2.2.4} it is not a place for sitting cross-legged, she commits an offense of wrong conduct. If she has asked permission, but she does not perceive that she has, she commits an offense of wrong conduct. If she has asked permission, but she is unsure of it, she commits an offense of wrong conduct. If she has asked permission, and she perceives that she has, there is no offense. 

\subsection*{Non-offenses }

There\marginnote{2.3.1} is no offense: if she sits down or lies down after asking permission;  if a dedicated seat is permanently ready for her;  if she is sick;  if there is an emergency;  if she is insane;  if she is the first offender. 

\scendsutta{The sixth training rule is finished. }

%
\section*{{\suttatitleacronym Bi Pc 17}{\suttatitletranslation The training rule on spreading out without asking permission }{\suttatitleroot Anāpucchāsantharaṇa}}
\addcontentsline{toc}{section}{\tocacronym{Bi Pc 17} \toctranslation{The training rule on spreading out without asking permission } \tocroot{Anāpucchāsantharaṇa}}
\markboth{The training rule on spreading out without asking permission }{Anāpucchāsantharaṇa}
\extramarks{Bi Pc 17}{Bi Pc 17}

\subsection*{Origin story }

At\marginnote{1.1} one time the Buddha was staying at \textsanskrit{Sāvatthī} in the Jeta Grove, \textsanskrit{Anāthapiṇḍika}’s Monastery. At that time a number of nuns were traveling through the Kosalan country on their way to \textsanskrit{Sāvatthī}, when one evening they arrived at a certain village. There they went to a brahmin family and asked for a place to stay. The brahmin woman told them, “Please wait, Venerables, until my husband returns.” While they were waiting, the nuns put out bedding, and some sat down on it while others lay down. 

When\marginnote{1.7} the husband returned at night, he said to his wife, “Who are they?” 

“They\marginnote{1.9} are nuns.” 

“Throw\marginnote{1.10} out these shaven-headed sluts!”, and he had them thrown out of the house. 

Those\marginnote{1.11} nuns then went to \textsanskrit{Sāvatthī}, where they told the nuns what had happened. The nuns of few desires complained and criticized them, “How can nuns visit families at the wrong time, put out bedding without asking permission of the owners, and then sit down and lie down on it?” … “Is it true, monks, that nuns did this?” 

“It’s\marginnote{1.15} true, Sir.” 

The\marginnote{1.16} Buddha rebuked them … “How could nuns act like this? This will affect people’s confidence …” … “And, monks, the nuns should recite this training rule like this: 

\subsection*{Final ruling }

\scrule{‘If a nun visits families at the wrong time, puts out bedding without asking permission of the owners, or has it put out, and then sits down or lies down on it, she commits an offense entailing confession.’” }

\subsection*{Definitions }

\begin{description}%
\item[A: ] whoever … %
\item[Nun: ] … The nun who has been given the full ordination in unanimity by both Sanghas through a legal procedure consisting of one motion and three announcements that is irreversible and fit to stand—this sort of nun is meant in this case. %
\item[The wrong time: ] from sunset until dawn. %
\item[A family: ] there are four kinds of families: the aristocratic family, the brahmin family, the merchant family, the worker family. %
\item[Visits: ] goes there. %
\item[Without asking permission of the owners: ] not having asked permission of a person in that family who is an owner and who has the authority to give. %
\item[Bedding: ] even a spread of leaves. %
\item[Puts out: ] she puts it out herself. %
\item[Has it put out: ] she has someone else to put it out. %
\item[Sits down: ] if she sits down on it, she commits an offense entailing confession. %
\item[Lies down: ] if she lies down on it, she commits an offense entailing confession. %
\end{description}

\subsection*{Permutations }

If\marginnote{2.2.1} she has not asked permission, and she does not perceive that she has, and she sits down or lies down on bedding after putting it out or having it put out, she commits an offense entailing confession. If she has not asked permission, but she is unsure of it, and she sits down or lies down on bedding after putting it out or having it put out, she commits an offense entailing confession. If she has not asked permission, but she perceives that she has, and she sits down or lies down on bedding after putting it out or having it put out, she commits an offense entailing confession. 

If\marginnote{2.2.4} she has asked permission, but she does not perceive that she has, she commits an offense of wrong conduct. If she has asked permission, but she is unsure of it, she commits an offense of wrong conduct. If she has asked permission, and she perceives that she has, there is no offense. 

\subsection*{Non-offenses }

There\marginnote{2.3.1} is no offense: if she first asks permission, and then, after putting out bedding or having it put out, sits down or lies down on it;  if she is sick;  if there is an emergency;  if she is insane;  if she is the first offender. 

\scendsutta{The seventh training rule is finished. }

%
\section*{{\suttatitleacronym Bi Pc 18}{\suttatitletranslation The training rule on complaining about others }{\suttatitleroot Paraujjhāpanaka}}
\addcontentsline{toc}{section}{\tocacronym{Bi Pc 18} \toctranslation{The training rule on complaining about others } \tocroot{Paraujjhāpanaka}}
\markboth{The training rule on complaining about others }{Paraujjhāpanaka}
\extramarks{Bi Pc 18}{Bi Pc 18}

\subsection*{Origin story }

At\marginnote{1.1} one time the Buddha was staying at \textsanskrit{Sāvatthī} in the Jeta Grove, \textsanskrit{Anāthapiṇḍika}’s Monastery. At that time a nun who was a pupil of \textsanskrit{Bhaddā} \textsanskrit{Kāpilānī} attended on her with care. \textsanskrit{Bhaddā} \textsanskrit{Kāpilānī} said to the nuns, “Venerables, this nun is attending on me with care. I’ll give her a robe.” Then, because of misunderstanding and a lack of proper reflection, that nun complained about \textsanskrit{Bhaddā} \textsanskrit{Kāpilānī}, “Venerables, if I didn’t attend on her with care, she wouldn’t give me a robe.” 

The\marginnote{1.7} nuns of few desires complained and criticized her, “How could a nun complain about someone else because of misunderstanding and a lack of proper reflection?” … “Is it true, monks, that a nun did this?” 

“It’s\marginnote{1.10} true, Sir.” 

The\marginnote{1.11} Buddha rebuked her … “How could a nun act in this way? This will affect people’s confidence …” … “And, monks, the nuns should recite this training rule like this: 

\subsection*{Final ruling }

\scrule{‘If a nun complains about someone else because of misunderstanding and a lack of proper reflection, she commits an offense entailing confession.’” }

\subsection*{Definitions }

\begin{description}%
\item[A: ] whoever … %
\item[Nun: ] … The nun who has been given the full ordination in unanimity by both Sanghas through a legal procedure consisting of one motion and three announcements that is irreversible and fit to stand—this sort of nun is meant in this case. %
\item[Because of misunderstanding: ] because of wrong understanding. %
\item[Because of a lack of proper reflection: ] because of wrong reflection. %
\item[Someone else: ] if she complains about one who is fully ordained, she commits an offense entailing confession. %
\end{description}

\subsection*{Permutations }

If\marginnote{2.2.1} the other person is fully ordained, and she perceives them as such, and she complains about them, she commits an offense entailing confession. If the other person is fully ordained, but she is unsure of it, and she complains about them, she commits an offense entailing confession. If the other person is fully ordained, but she does not perceive them as such, and she complains about them, she commits an offense entailing confession. 

If\marginnote{2.2.4} she complains about someone who is not fully ordained, she commits an offense of wrong conduct. If the other person is not fully ordained, but she perceives them as such, she commits an offense of wrong conduct. If the other person is not fully ordained, but she is unsure of it, she commits an offense of wrong conduct. If the other person is not fully ordained, and she does not perceive them as such, she commits an offense of wrong conduct. 

\subsection*{Non-offenses }

There\marginnote{2.3.1} is no offense: if she is insane;  if she is the first offender. 

\scendsutta{The eighth training rule is finished. }

%
\section*{{\suttatitleacronym Bi Pc 19}{\suttatitletranslation The training rule on cursing another }{\suttatitleroot Paraabhisapana}}
\addcontentsline{toc}{section}{\tocacronym{Bi Pc 19} \toctranslation{The training rule on cursing another } \tocroot{Paraabhisapana}}
\markboth{The training rule on cursing another }{Paraabhisapana}
\extramarks{Bi Pc 19}{Bi Pc 19}

\subsection*{Origin story }

At\marginnote{1.1} one time when the Buddha was staying at \textsanskrit{Sāvatthī} in \textsanskrit{Anāthapiṇḍika}’s Monastery, some nuns were unable to find their possessions. They said to the nun \textsanskrit{Caṇḍakāḷī}, “Venerable, have you seen our things?” 

\textsanskrit{Caṇḍakāḷī}\marginnote{1.4} complained and criticized them, “Why are you asking me if I’ve seen your things? Am I a thief? Am I shameless? Venerables, if I took your things I wouldn’t be a monastic anymore. I would fall from the spiritual life and be reborn in hell. May anyone who speaks such an untruth about me depart from monasticism, fall from the spiritual life, and be reborn in hell!” 

The\marginnote{1.9} nuns of few desires complained and criticized her, “How could Venerable \textsanskrit{Caṇḍakāḷī} curse herself and others, referring to hell and the spiritual life?” … “Is it true, monks, that the nun \textsanskrit{Caṇḍakāḷī} did this?” 

“It’s\marginnote{1.12} true, Sir.” 

The\marginnote{1.13} Buddha rebuked her … “How could the nun \textsanskrit{Caṇḍakāḷī} act like this? This will affect people’s confidence …” … “And, monks, the nuns should recite this training rule like this: 

\subsection*{Final ruling }

\scrule{‘If a nun curses herself or someone else, referring to hell or the spiritual life, she commits an offense entailing confession.’” }

\subsection*{Definitions }

\begin{description}%
\item[A: ] whoever … %
\item[Nun: ] … The nun who has been given the full ordination in unanimity by both Sanghas through a legal procedure consisting of one motion and three announcements that is irreversible and fit to stand—this sort of nun is meant in this case. %
\item[Herself: ] oneself. %
\item[Someone else: ] one who is fully ordained. If she curses, referring to hell or the spiritual life, she commits an offense entailing confession. %
\end{description}

\subsection*{Permutations }

If\marginnote{2.2.1} the other person is fully ordained, and she perceives them as such, and she curses them, referring to hell or the spiritual life, she commits an offense entailing confession. If the other person is fully ordained, but she is unsure of it, she curses them, referring to hell or the spiritual life, she commits an offense entailing confession. If the other person is fully ordained, but she does not perceive them as such, she curses them, referring to hell or the spiritual life, she commits an offense entailing confession. 

If\marginnote{2.2.4} she curses, referring to the animal realm, the ghost realm, or human misfortune, she commits an offense of wrong conduct. If she curses someone who is not fully ordained, she commits an offense of wrong conduct. If the other person is not fully ordained, but she perceives them as such, she commits an offense of wrong conduct. If the other person is not fully ordained, but she is unsure of it, she commits an offense of wrong conduct. If the other person is not fully ordained, and she does not perceive them as such, she commits an offense of wrong conduct. 

\subsection*{Non-offenses }

There\marginnote{2.3.1} is no offense: if she is aiming at something beneficial;  if she is aiming at giving a teaching;  if she is aiming at giving an instruction;  if she is insane;  if she is the first offender. 

\scendsutta{The ninth training rule is finished. }

%
\section*{{\suttatitleacronym Bi Pc 20}{\suttatitletranslation The training rule on crying }{\suttatitleroot Rodana}}
\addcontentsline{toc}{section}{\tocacronym{Bi Pc 20} \toctranslation{The training rule on crying } \tocroot{Rodana}}
\markboth{The training rule on crying }{Rodana}
\extramarks{Bi Pc 20}{Bi Pc 20}

\subsection*{Origin story }

At\marginnote{1.1} one time when the Buddha was staying at \textsanskrit{Sāvatthī} in the \textsanskrit{Anāthapiṇḍika}’s Monastery, the nun \textsanskrit{Caṇḍakāḷī} quarreled with the nuns and then cried after repeatedly beating herself. The nuns of few desires complained and criticized her, “How could Venerable \textsanskrit{Caṇḍakāḷī} cry after repeatedly beating herself?” … “Is it true, monks, that the nun \textsanskrit{Caṇḍakāḷī} did this?” 

“It’s\marginnote{1.6} true, Sir.” 

The\marginnote{1.7} Buddha rebuked her … “How could the nun \textsanskrit{Caṇḍakāḷī} act like this? This will affect people’s confidence …” … “And, monks, the nuns should recite this training rule like this: 

\subsection*{Final ruling }

\scrule{‘If a nun cries after repeatedly beating herself, she commits an offense entailing confession.’” }

\subsection*{Definitions }

\begin{description}%
\item[A: ] whoever … %
\item[Nun: ] … The nun who has been given the full ordination in unanimity by both Sanghas through a legal procedure consisting of one motion and three announcements that is irreversible and fit to stand—this sort of nun is meant in this case. %
\item[Herself: ] oneself. %
\end{description}

If\marginnote{2.1.7} she cries after repeatedly beating herself, she commits an offense entailing confession.  If she beats herself, but does not cry, she commits an offense of wrong conduct.  If she cries, but does not beat herself, she commits an offense of wrong conduct. 

\subsection*{Non-offenses }

There\marginnote{2.2.1} is no offense: if she cries, but does not beat herself, because of loss of relatives, loss of property, or loss of health;  if she is insane;  if she is the first offender. 

\scendsutta{The tenth training rule is finished. }

\scendvagga{The second subchapter on the dark of the night is finished. }

%
\section*{{\suttatitleacronym Bi Pc 21}{\suttatitletranslation The training rule on nakedness }{\suttatitleroot Nagga}}
\addcontentsline{toc}{section}{\tocacronym{Bi Pc 21} \toctranslation{The training rule on nakedness } \tocroot{Nagga}}
\markboth{The training rule on nakedness }{Nagga}
\extramarks{Bi Pc 21}{Bi Pc 21}

\subsection*{Origin story }

At\marginnote{1.1} one time when the Buddha was staying at \textsanskrit{Sāvatthī} in \textsanskrit{Anāthapiṇḍika}’s Monastery, a number of nuns were bathing naked at a ford in the river \textsanskrit{Aciravatī} together with sex workers. The sex workers teased the nuns, “Venerables, why practice the spiritual life when you’re still young? Why not enjoy worldly pleasures? You can practice the spiritual life when you’re old. In this way you’ll get the benefit of both.” The nuns felt humiliated. 

They\marginnote{1.8} then went to the nuns’ dwelling place and told the nuns what had happened. The nuns told the monks, who in turn told the Buddha. Soon afterwards the Buddha gave a teaching and addressed the monks: “Well then, monks, I will lay down a training rule for the nuns for the following ten reasons: for the well-being of the Sangha … for supporting the training. And, monks, the nuns should recite this training rule like this: 

\subsection*{Final ruling }

\scrule{‘If a nun bathes naked, she commits an offense entailing confession.’”\footnote{Sp 2.886: \textit{Sacepi \textsanskrit{udakasāṭikacīvaraṁ} \textsanskrit{mahagghaṁ} hoti, na \textsanskrit{sakkā} \textsanskrit{nivāsetvā} bahi \textsanskrit{gantuṁ}, evampi \textsanskrit{naggāya} \textsanskrit{nhāyituṁ} \textsanskrit{vaṭṭati}}; “Also, if her bathing robe is valuable, or she is unable to put on a lower robe before going outside, then it is allowable to bathe naked.” The implication of this is that this rule concerns outdoor bathing, which fits the origin story. } }

\subsection*{Definitions }

\begin{description}%
\item[A: ] whoever … %
\item[Nun: ] … The nun who has been given the full ordination in unanimity by both Sanghas through a legal procedure consisting of one motion and three announcements that is irreversible and fit to stand—this sort of nun is meant in this case. %
\item[Bathes naked: ] if she bathes without wearing a sarong or an upper robe, then for the effort there is an act of wrong conduct. At the end of the bath, she commits an offense entailing confession. %
\end{description}

\subsection*{Non-offenses }

There\marginnote{2.2.1} is no offense: if her robes have been stolen or lost;  if there is an emergency;  if she is insane;  if she is the first offender. 

\scendsutta{The first training rule is finished. }

%
\section*{{\suttatitleacronym Bi Pc 22}{\suttatitletranslation The training rule on bathing robes }{\suttatitleroot Udakasāṭika}}
\addcontentsline{toc}{section}{\tocacronym{Bi Pc 22} \toctranslation{The training rule on bathing robes } \tocroot{Udakasāṭika}}
\markboth{The training rule on bathing robes }{Udakasāṭika}
\extramarks{Bi Pc 22}{Bi Pc 22}

\subsection*{Origin story }

At\marginnote{1.1} one time the Buddha was staying at \textsanskrit{Sāvatthī} in the Jeta Grove, \textsanskrit{Anāthapiṇḍika}’s Monastery. At that time the Buddha had allowed bathing robes for the nuns. Knowing this, the nuns from the group of six wore bathing robes of inappropriate size. As they were walking about, they were dragging them along, both in front and behind. 

The\marginnote{1.6} nuns of few desires complained and criticized them, “How can the nuns from the group of six wear such bathing robes?” … “Is it true, monks, that the nuns from the group of six do this?” 

“It’s\marginnote{1.9} true, Sir.” 

The\marginnote{1.10} Buddha rebuked them … “How can the nuns from the group of six wear such bathing robes? This will affect people’s confidence …” … “And, monks, the nuns should recite this training rule like this: 

\subsection*{Final ruling }

\scrule{‘If a nun is having a bathing robe made, it should be made the right size. This is the right size: four standard handspans long and two wide. If it exceeds that, it is to be cut down, and she commits an offense entailing confession.’”\footnote{For an explanation of the rendering of \textit{sugata} as “standard” and of \textit{vidatthi} as “handspan”, see Appendix of Technical Terms. } }

\subsection*{Definitions }

\begin{description}%
\item[A bathing robe: ] wearing it as a sarong, she bathes. %
\item[Is having made: ] making it herself or having someone else make it, it should be made the right size. This is the right size: four standard handspans long and two wide. If she makes one or has one made that exceeds that, then for the effort there is an act of wrong conduct. When she gets it, it is to be cut down, and she is then to confess an offense entailing confession. %
\end{description}

\subsection*{Permutations }

If\marginnote{2.1.8.1} she finishes what she began herself, she commits an offense entailing confession. If she has others finish what she began herself, she commits an offense entailing confession. If she finishes herself what was begun by others, she commits an offense entailing confession. If she has others finish what was begun by others, she commits an offense entailing confession. 

If\marginnote{2.1.12} she makes one or has one made for the benefit of someone else, she commits an offense of wrong conduct. If she gets one that was made by someone else and then uses it, she commits an offense of wrong conduct. 

\subsection*{Non-offenses }

There\marginnote{2.2.1} is no offense: if she makes it the right size;  if she makes it smaller than the right size;  if she gets one made by someone else that exceeds the right size and she cuts it down before using it;  if she makes a canopy, a floor cover, a cloth screen, a mattress, or a pillow;  if she is insane;  if she is the first offender. 

\scendsutta{The second training rule is finished. }

%
\section*{{\suttatitleacronym Bi Pc 23}{\suttatitletranslation The training rule on sewing robes }{\suttatitleroot Cīvarasibbana}}
\addcontentsline{toc}{section}{\tocacronym{Bi Pc 23} \toctranslation{The training rule on sewing robes } \tocroot{Cīvarasibbana}}
\markboth{The training rule on sewing robes }{Cīvarasibbana}
\extramarks{Bi Pc 23}{Bi Pc 23}

\subsection*{Origin story }

At\marginnote{1.1} one time the Buddha was staying at \textsanskrit{Sāvatthī} in the Jeta Grove, \textsanskrit{Anāthapiṇḍika}’s Monastery. At that time a robe belonging to a certain nun had been badly made and badly sewn from expensive robe-cloth. The nun \textsanskrit{Thullanandā} said to her, “Venerable, this robe-cloth is beautiful, but the robe has been badly made and badly sewn.” 

“If\marginnote{1.6} I unstitch it, will you sew it back together?” 

“Sure.”\marginnote{1.7} 

Then\marginnote{1.8} that nun unstitched the robe and gave it to \textsanskrit{Thullanandā}. Yet although \textsanskrit{Thullanandā} repeatedly said she would sew it, she neither sewed it herself nor did she make any effort to have someone else do it. 

That\marginnote{1.11} nun then told the nuns what had happened. The nuns of few desires complained and criticized \textsanskrit{Thullanandā}, “How could Venerable \textsanskrit{Thullanandā} have a nun’s robe unstitched, and then neither sew it herself nor make any effort to have someone else do it?” … “Is it true, monks, that the nun \textsanskrit{Thullanandā} did this?” 

“It’s\marginnote{1.15} true, Sir.” 

The\marginnote{1.16} Buddha rebuked her … “How could the nun \textsanskrit{Thullanandā} act like this? This will affect people’s confidence …” … “And, monks, the nuns should recite this training rule like this: 

\subsection*{Final ruling }

\scrule{‘If a nun, after unstitching a nun’s robe or having it unstitched, neither sews it herself nor makes any effort to have someone else sew it, and there were no obstacles, then, except if it was no more than four or five days, she commits an offense entailing confession.’” }

\subsection*{Definitions }

\begin{description}%
\item[A: ] whoever … %
\item[Nun: ] … The nun who has been given the full ordination in unanimity by both Sanghas through a legal procedure consisting of one motion and three announcements that is irreversible and fit to stand—this sort of nun is meant in this case. %
\item[A nun’s: ] another nun’s. %
\item[Robe: ] one of the six kinds of robes. %
\item[Unstitching: ] she unstitches it herself. %
\item[Having it unstitched: ] she has someone else unstitch it. %
\item[And there were no obstacles: ] when there is no obstacle. %
\item[She neither sews: ] she does not sew it herself. %
\item[Nor makes any effort to have someone else sew it: ] she does not ask anyone else. %
\item[Except if it was no more than four or five days: ] unless it was no more than four or five days. %
\end{description}

If\marginnote{2.1.21} she thinks, “I’ll neither sew it nor make any effort to have someone else sew it,” then by the mere fact of abandoning her duty, she commits an offense entailing confession. 

\subsection*{Permutations }

If\marginnote{2.2.1} the other person is fully ordained, and she perceives her as such, and she unstitches her robe or has it unstitched, and then neither sews it herself nor makes any effort to have someone else sew it, and there were no obstacles, then, except if it was no more than four or five days, she commits an offense entailing confession. If the other person is fully ordained, but she is unsure of it, and she unstitches her robe or has it unstitched, and then neither sews it herself nor makes any effort to have someone else sew it, and there were no obstacles, then, except if it was no more than four or five days, she commits an offense entailing confession. If the other person is fully ordained, but she does not perceive her as such, and she unstitches her robe or has it unstitched, and then neither sews it herself nor makes any effort to have someone else sew it, and there were no obstacles, then, except if it was no more than four or five days, she commits an offense entailing confession. 

If\marginnote{2.2.4} she unstitches another requisite or has it unstitched, and then neither sews it herself nor makes any effort to have someone else sew it, and there were no obstacles, then, except if it was no more than four or five days, she commits an offense of wrong conduct. If the other person is not fully ordained, and she unstitches her robe or another requisite, or has it unstitched, and then neither sews it herself nor makes any effort to have someone else sew it, and there were no obstacles, then, except if it was no more than four or five days, she commits an offense of wrong conduct. 

If\marginnote{2.2.6} the other person is not fully ordained, but she perceives her as such, she commits an offense of wrong conduct. If the other person is not fully ordained, but she is unsure of it, she commits an offense of wrong conduct. If the other person is not fully ordained, and she does not perceive her as such, she commits an offense of wrong conduct. 

\subsection*{Non-offenses }

There\marginnote{2.3.1} is no offense: if there is an obstacle;  if she searches for someone to sew, but does not find anyone;  if she is doing it, but she takes longer than four or five days;  if she is sick;  if there is an emergency;  if she is insane;  if she is the first offender. 

\scendsutta{The third training rule is finished. }

%
\section*{{\suttatitleacronym Bi Pc 24}{\suttatitletranslation The training rule on moving the robes }{\suttatitleroot Saṅghāṭicāra}}
\addcontentsline{toc}{section}{\tocacronym{Bi Pc 24} \toctranslation{The training rule on moving the robes } \tocroot{Saṅghāṭicāra}}
\markboth{The training rule on moving the robes }{Saṅghāṭicāra}
\extramarks{Bi Pc 24}{Bi Pc 24}

\subsection*{Origin story }

At\marginnote{1.1} one time the Buddha was staying at \textsanskrit{Sāvatthī} in the Jeta Grove, \textsanskrit{Anāthapiṇḍika}’s Monastery. At that time the nuns stored one of their robes with other nuns and then left to wander the country in a sarong and an upper robe. Because they were stored for a long time, the robes became moldy. When the nuns put them out to sun them, other nuns asked them, “Whose moldy robes are these?” 

And\marginnote{1.7} they told them what had happened. 

The\marginnote{1.8} nuns of few desires complained and criticized them, “How can nuns store one of their robes with other nuns and then go wandering the country in a sarong and an upper robe?” … “Is it true, monks, that nuns do this?” 

“It’s\marginnote{1.11} true, Sir.” 

The\marginnote{1.12} Buddha rebuked them … “How can nuns do this? This will affect people’s confidence …” … “And, monks, the nuns should recite this training rule like this: 

\subsection*{Final ruling }

\scrule{‘If a nun does not move her robes for more than five days, she commits an offense entailing confession.’”\footnote{“Move (her) robes” renders \textit{\textsanskrit{saṅghāṭicāra}}. Sp 2.898: \textit{\textsanskrit{Saṅghāṭīnaṁ} \textsanskrit{cāro} \textsanskrit{saṅghāṭicāro}}, “\textit{\textsanskrit{Saṅghāṭicāra}} is the moving of the robes.” For the meaning of \textit{\textsanskrit{saṅghāṭi}}, see Appendix of Technical Terms. } }

\subsection*{Definitions }

\begin{description}%
\item[A: ] whoever … %
\item[Nun: ] … The nun who has been given the full ordination in unanimity by both Sanghas through a legal procedure consisting of one motion and three announcements that is irreversible and fit to stand—this sort of nun is meant in this case. %
\item[Does not move her robes for more than five days: ] if she does not wear or sun her five robes on the fifth day, then, when the fifth day has passed, she commits an offense entailing confession. %
\end{description}

\subsection*{Permutations }

If\marginnote{2.2.1} it is more than five days, and she perceives it as more, she commits an offense entailing confession. If it is more than five days, but she is unsure of it, she commits an offense entailing confession. If it is more than five days, but she perceives it as less, she commits an offense entailing confession. 

If\marginnote{2.2.4} it is less than five days, but she perceives it as more, she commits an offense of wrong conduct. If it is less than five days, but she is unsure of it, she commits an offense of wrong conduct. If it is less than five days and she perceives it as less, there is no offense. 

\subsection*{Non-offenses }

There\marginnote{2.3.1} is no offense: if she wears or suns the five robes on the fifth day;  if she is sick;  if there is an emergency;  if she is insane;  if she is the first offender. 

\scendsutta{The fourth training rule is finished. }

%
\section*{{\suttatitleacronym Bi Pc 25}{\suttatitletranslation The training rule on borrowed robes }{\suttatitleroot Cīvarasaṅkamanīya}}
\addcontentsline{toc}{section}{\tocacronym{Bi Pc 25} \toctranslation{The training rule on borrowed robes } \tocroot{Cīvarasaṅkamanīya}}
\markboth{The training rule on borrowed robes }{Cīvarasaṅkamanīya}
\extramarks{Bi Pc 25}{Bi Pc 25}

\subsection*{Origin story }

At\marginnote{1.1} one time when the Buddha was staying at \textsanskrit{Sāvatthī} in \textsanskrit{Anāthapiṇḍika}’s Monastery, there was a nun who, after walking for almsfood, spread out her damp robe and entered her dwelling. A second nun put on that robe and went to the village for alms. Soon afterwards the first nun came out and asked the nuns, “Venerables, have you seen my robe?” The nuns told her what had happened. She then complained and criticized the second nun, “How could a nun put on my robe without asking permission?” 

She\marginnote{1.9} told the nuns what had happened, and the nuns of few desires complained and criticized her, “How could a nun put on another nun’s robe without asking permission?” … “Is it true, monks, that a nun did this?” 

“It’s\marginnote{1.13} true, Sir.” 

The\marginnote{1.14} Buddha rebuked her … “How could a nun do this? This will affect people’s confidence …” … “And, monks, the nuns should recite this training rule like this: 

\subsection*{Final ruling }

\scrule{‘If a nun wears a robe taken on loan, she commits an offense entailing confession.’”\footnote{“A robe taken on loan” renders \textit{\textsanskrit{cīvarasaṅkamanīya}}. Sp 1.16: \textit{\textsanskrit{Cīvarasaṅkamanīyanti} \textsanskrit{saṅkametabbaṁ} \textsanskrit{cīvaraṁ}; \textsanskrit{aññissā} \textsanskrit{santakaṁ} \textsanskrit{anāpucchā} \textsanskrit{gahitaṁ} puna \textsanskrit{paṭidātabbacīvaranti} attho}, “\textit{\textsanskrit{Cīvarasaṅkamanīya}} means a robe to be returned; a robe belonging to another, taken without permission, and to be given back.” } }

\subsection*{Definitions }

\begin{description}%
\item[A: ] whoever … %
\item[Nun: ] … The nun who has been given the full ordination in unanimity by both Sanghas through a legal procedure consisting of one motion and three announcements that is irreversible and fit to stand—this sort of nun is meant in this case. %
\item[A robe taken on loan: ] if she wears any of the five robes belonging to a fully ordained nun, and it has not been given to her, nor has she asked permission to use it, she commits an offense entailing confession. %
\end{description}

\subsection*{Permutations }

If\marginnote{2.2.1} the other person is fully ordained, and she perceives her as such, and she takes on loan a robe belonging to her and then wears it, she commits an offense entailing confession. If the other person is fully ordained, but she is unsure of it, and she takes on loan a robe belonging to her and then wears it, she commits an offense entailing confession. If the other person is fully ordained, but she does not perceive her as such, and she takes on loan a robe belonging to her and then wears it, she commits an offense entailing confession. 

If\marginnote{2.2.4} the other person is not fully ordained, and she takes on loan a robe belonging to her and then wears it, she commits an offense of wrong conduct. If the other person is not fully ordained, but she perceives her as such, she commits an offense of wrong conduct. If the other person is not fully ordained, but she is unsure of it, she commits an offense of wrong conduct. If the other person is not fully ordained, and she does not perceive her as such, she commits an offense of wrong conduct. 

\subsection*{Non-offenses }

There\marginnote{2.3.1} is no offense: if the robe has been given to her;\footnote{The commentaries are silent, but \textit{\textsanskrit{sā}} presumably refers to the owner of the robe. }  if she wears it after asking permission;  if her own robe has been stolen;  if her own robe has been lost;  if there is an emergency;  if she is insane;  if she is the first offender. 

\scendsutta{The fifth training rule is finished. }

%
\section*{{\suttatitleacronym Bi Pc 26}{\suttatitletranslation The training rule on robe-cloth for the community }{\suttatitleroot Gaṇacīvara}}
\addcontentsline{toc}{section}{\tocacronym{Bi Pc 26} \toctranslation{The training rule on robe-cloth for the community } \tocroot{Gaṇacīvara}}
\markboth{The training rule on robe-cloth for the community }{Gaṇacīvara}
\extramarks{Bi Pc 26}{Bi Pc 26}

\subsection*{Origin story }

At\marginnote{1.1} one time when the Buddha was staying at \textsanskrit{Sāvatthī} in \textsanskrit{Anāthapiṇḍika}’s Monastery, a family that was supporting the nun \textsanskrit{Thullanandā} said to her, “Venerable, we’re going to give robe-cloth to the Sangha of nuns.” 

She\marginnote{1.4} replied, “You’re very busy,” and she created an obstacle for them. 

Soon\marginnote{1.6} afterwards the house of that family burned down. They then complained and criticized her, “How could Venerable \textsanskrit{Thullanandā} create an obstacle for our gift? Now we have neither possessions nor merit.” 

The\marginnote{1.10} nuns heard the complaints of those people, and the nuns of few desires complained and criticized her, “How could Venerable \textsanskrit{Thullanandā} create an obstacle for the community to get robe-cloth?” … “Is it true, monks, that the nun \textsanskrit{Thullanandā} did this?” 

“It’s\marginnote{1.14} true, Sir.” 

The\marginnote{1.15} Buddha rebuked her … “How could the nun \textsanskrit{Thullanandā} do this? This will affect people’s confidence …” … “And, monks, the nuns should recite this training rule like this: 

\subsection*{Final ruling }

\scrule{‘If a nun creates an obstacle for the community to get robe-cloth, she commits an offense entailing confession.’” }

\subsection*{Definitions }

\begin{description}%
\item[A: ] whoever … %
\item[Nun: ] … The nun who has been given the full ordination in unanimity by both Sanghas through a legal procedure consisting of one motion and three announcements that is irreversible and fit to stand—this sort of nun is meant in this case. %
\item[The community: ] the Sangha of nuns is what is meant. %
\item[Robe-cloth: ] one of the six kinds of robe-cloth, but not smaller than what can be assigned to another.\footnote{The six are linen, cotton, silk, wool, sunn hemp, and hemp, see \href{https://suttacentral.net/pli-tv-kd8/en/brahmali\#3.1.6}{Kd 8:3.1.6}. According to \href{https://suttacentral.net/pli-tv-kd8/en/brahmali\#21.1.4}{Kd 8:21.1.4} the size referred to here is no smaller than 8 by 4 \textit{\textsanskrit{sugataṅgula}}, “standard fingerbreadths”. For an explanation of \textit{sugata} as “standard” and the idea of \textit{\textsanskrit{vikappanā}}, see Appendix of Technical Terms. } %
\item[Creates an obstacle: ] if she creates an obstacle with the thought, “What can be done so that these people don’t give robe-cloth?”, she commits an offense entailing confession. %
\end{description}

If\marginnote{2.1.11} she creates an obstacle for another requisite, she commits an offense of wrong conduct. If she creates an obstacle for a number of nuns or for a single nun or for someone who is not fully ordained to get robe-cloth or another requisite, she commits an offense of wrong conduct. 

\subsection*{Non-offenses }

There\marginnote{2.2.1} is no offense: if she can show a benefit in obstructing them;  if she is insane;  if she is the first offender. 

\scendsutta{The sixth training rule is finished. }

%
\section*{{\suttatitleacronym Bi Pc 27}{\suttatitletranslation The training rule on blocking }{\suttatitleroot Paṭibāhana}}
\addcontentsline{toc}{section}{\tocacronym{Bi Pc 27} \toctranslation{The training rule on blocking } \tocroot{Paṭibāhana}}
\markboth{The training rule on blocking }{Paṭibāhana}
\extramarks{Bi Pc 27}{Bi Pc 27}

\subsection*{Origin story }

At\marginnote{1.1} one time the Buddha was staying at \textsanskrit{Sāvatthī} in the Jeta Grove, \textsanskrit{Anāthapiṇḍika}’s Monastery. At that time out-of-season robe-cloth had been given to the Sangha of nuns, and the Sangha gathered to share it out. Just then the nuns who were pupils of the nun \textsanskrit{Thullanandā} were away. \textsanskrit{Thullanandā} said to the nuns: “Venerables, there are nuns who are away. We cannot distribute the robe-cloth until they get back.” And she blocked the distribution of robe-cloth. Since the robe-cloth could not be distributed until those nuns returned, other nuns left. Then, when her pupils returned, \textsanskrit{Thullanandā} had that robe-cloth distributed. 

The\marginnote{1.10} nuns of few desires complained and criticized her, “How could Venerable \textsanskrit{Thullanandā} block a legitimate distribution of robe-cloth?” … “Is it true, monks, that the nun \textsanskrit{Thullanandā} did this?” 

“It’s\marginnote{1.13} true, Sir.” 

The\marginnote{1.14} Buddha rebuked her … “How could the nun \textsanskrit{Thullanandā} do this? This will affect people’s confidence …” … “And, monks, the nuns should recite this training rule like this: 

\subsection*{Final ruling }

\scrule{‘If a nun blocks a legitimate distribution of robe-cloth, she commits an offense entailing confession.’” }

\subsection*{Definitions }

\begin{description}%
\item[A: ] whoever … %
\item[Nun: ] … The nun who has been given the full ordination in unanimity by both Sanghas through a legal procedure consisting of one motion and three announcements that is irreversible and fit to stand—this sort of nun is meant in this case. %
\item[A legitimate distribution of robe-cloth: ] it is distributed by a unanimous Sangha of nuns. %
\item[Blocks: ] if she blocks it with the thought, “What can be done so that this robe-cloth isn’t distributed?”, she commits an offense entailing confession. %
\end{description}

\subsection*{Permutations }

If\marginnote{2.2.1} it is a legitimate legal procedure, and she perceives it as such, and she blocks it, she commits an offense entailing confession. If it is a legitimate legal procedure, but she is unsure of it, and she blocks it, she commits an offense of wrong conduct. If it is a legitimate legal procedure, but she perceives it as illegitimate, and she blocks it, there is no offense. 

If\marginnote{2.2.4} it is an illegitimate legal procedure, but she perceives it as legitimate, she commits an offense of wrong conduct. If it is an illegitimate legal procedure, but she is unsure of it, she commits an offense of wrong conduct. If it is an illegitimate legal procedure, and she perceives it as such, there is no offense. 

\subsection*{Non-offenses }

There\marginnote{2.3.1} is no offense: if she blocks it after demonstrating a benefit;  if she is insane;  if she is the first offender. 

\scendsutta{The seventh training rule is finished. }

%
\section*{{\suttatitleacronym Bi Pc 28}{\suttatitletranslation The training rule on giving robes }{\suttatitleroot Cīvaradāna}}
\addcontentsline{toc}{section}{\tocacronym{Bi Pc 28} \toctranslation{The training rule on giving robes } \tocroot{Cīvaradāna}}
\markboth{The training rule on giving robes }{Cīvaradāna}
\extramarks{Bi Pc 28}{Bi Pc 28}

\subsection*{Origin story }

At\marginnote{1.1} one time the Buddha was staying at \textsanskrit{Sāvatthī} in the Jeta Grove, \textsanskrit{Anāthapiṇḍika}’s Monastery. At that time the nun \textsanskrit{Thullanandā} was giving monastic robes to actors, dancers, acrobats, magicians, and musicians, saying, “Please praise me in public.” And they did: “Venerable \textsanskrit{Thullanandā} is a learned reciter; she’s confident and skilled at giving teachings. Give to her; work for her!” 

The\marginnote{1.7} nuns of few desires complained and criticized her, “How can Venerable \textsanskrit{Thullanandā} give monastic robes to householders?” … “Is it true, monks, that the nun \textsanskrit{Thullanandā} does this?” 

“It’s\marginnote{1.10} true, Sir.” 

The\marginnote{1.11} Buddha rebuked her … “How can the nun \textsanskrit{Thullanandā} do this? This will affect people’s confidence …” … “And, monks, the nuns should recite this training rule like this: 

\subsection*{Final ruling }

\scrule{‘If a nun gives a monastic robe to a householder or a male wanderer or a female wanderer, she commits an offense entailing confession.’” }

\subsection*{Definitions }

\begin{description}%
\item[A: ] whoever … %
\item[Nun: ] … The nun who has been given the full ordination in unanimity by both Sanghas through a legal procedure consisting of one motion and three announcements that is irreversible and fit to stand—this sort of nun is meant in this case. %
\item[A householder: ] anyone who lives at home. %
\item[A male wanderer: ] any male wanderer apart from Buddhist monks and novice monks. %
\item[A female wanderer: ] any female wanderer apart from Buddhist nuns, trainee nuns, and novice nuns. %
\item[A monastic robe: ] a mark has been made is what is meant. If she gives it away, she commits an offense entailing confession. %
\end{description}

\subsection*{Non-offenses }

There\marginnote{2.2.1} is no offense: if she gives one to her mother or father;  if she lends one out;  if she is insane;  if she is the first offender. 

\scendsutta{The eighth training rule is finished. }

%
\section*{{\suttatitleacronym Bi Pc 29}{\suttatitletranslation The training rule on letting the time expire }{\suttatitleroot Kālaatikkamana}}
\addcontentsline{toc}{section}{\tocacronym{Bi Pc 29} \toctranslation{The training rule on letting the time expire } \tocroot{Kālaatikkamana}}
\markboth{The training rule on letting the time expire }{Kālaatikkamana}
\extramarks{Bi Pc 29}{Bi Pc 29}

\subsection*{Origin story }

At\marginnote{1.1} one time when the Buddha was staying at \textsanskrit{Sāvatthī} in \textsanskrit{Anāthapiṇḍika}’s Monastery, a family that was supporting the nun \textsanskrit{Thullanandā} said to her, “If we’re able, Venerable, we’ll give robe-cloth to the Sangha of nuns.” 

Just\marginnote{1.4} then the nuns who had completed the rainy-season residence had gathered to distribute robe-cloth. But \textsanskrit{Thullanandā} said to them, “Please wait, Venerables. The Sangha is expecting more cloth.” The nuns said to her, “Go and find out what’s happening.” She then went to that family and said to them, “Please give the robe-cloth to the Sangha of nuns.” 

“We’re\marginnote{1.11} sorry, but we’re not able to give robe-cloth to the Sangha.” 

\textsanskrit{Thullanandā}\marginnote{1.12} told the nuns what had happened. The nuns of few desires complained and criticized her, “How could Venerable \textsanskrit{Thullanandā} allow the robe season to expire because of an uncertain expectation of robe-cloth?” … “Is it true, monks, that the nun \textsanskrit{Thullanandā} did this?” 

“It’s\marginnote{1.16} true, Sir.” 

The\marginnote{1.17} Buddha rebuked her … “How could the nun \textsanskrit{Thullanandā} do this? This will affect people’s confidence …” … “And, monks, the nuns should recite this training rule like this: 

\subsection*{Final ruling }

\scrule{‘If a nun lets the robe season expire because of an uncertain expectation of robe-cloth, she commits an offense entailing confession.’” }

\subsection*{Definitions }

\begin{description}%
\item[A: ] whoever … %
\item[Nun: ] … The nun who has been given the full ordination in unanimity by both Sanghas through a legal procedure consisting of one motion and three announcements that is irreversible and fit to stand—this sort of nun is meant in this case. %
\item[An uncertain expectation of robe-cloth: ] they have said, “If we’re able, then we’ll give, then we’ll act.” %
\item[Robe season: ] for one who has not participated in the robe-making ceremony, it is the last month of the rainy season; for one who has participated in the robe-making ceremony, it is the five month period.\footnote{“Robe-making ceremony” refers to the \textit{kathina \textsanskrit{saṅghakamma}}, the making of the \textit{kathina} robe, and the rejoicing in the process, all three together represented by the words \textit{(an)atthate kathine }. “The five month period” is the last month of the rainy season plus the four months of the cold season. See also \textit{kathina} in Appendix of Technical Terms. } %
\item[Lets the robe season expire: ] for one who has not participated in the robe-making ceremony, if she goes beyond the last day of the rainy season, she commits an offense entailing confession. For one who has participated in the robe-making ceremony, if she goes beyond the day on which the robe season ends, she commits an offense entailing confession.\footnote{The robe season ends if the Sangha decides to forgo the robe-season privileges, or if the nun leaves the monastery where she spent the rains residence and gives up any intention of making a robe before the end of the cold season, see \href{https://suttacentral.net/pli-tv-bu-vb-np1/en/brahmali\#3.1.4}{Bu NP 1:3.1.4} and \href{https://suttacentral.net/pli-tv-kd7/en/brahmali\#13.2.1}{Kd 7:13.2.1}. } %
\end{description}

\subsection*{Permutations }

If\marginnote{2.2.1} it is an uncertain expectation of robe-cloth, and she perceives it as such, and she lets the robe season expire, she commits an offense entailing confession.\footnote{I take \textit{\textsanskrit{dubbalacīvara}} as shorthand for \textit{\textsanskrit{dubbalacīvarapaccāsā}}, and I translate accordingly. } If it is an uncertain expectation of robe-cloth, but she is unsure of it, and she lets the robe season expire, she commits an offense of wrong conduct. If it is an uncertain expectation of robe-cloth, but she does not perceive it as such, and she lets the robe season expire, there is no offense. 

If\marginnote{2.2.4} it is not an uncertain expectation of robe-cloth, but she perceives it as such, she commits an offense of wrong conduct. If it is not an uncertain expectation of robe-cloth, but she is unsure of it, she commits an offense of wrong conduct. If it is not an uncertain expectation of robe-cloth, and she does not perceive it as such, there is no offense. 

\subsection*{Non-offenses }

There\marginnote{2.3.1} is no offense: if she can show a benefit in hindering it;  if she is insane;  if she is the first offender. 

\scendsutta{The ninth training rule is finished. }

%
\section*{{\suttatitleacronym Bi Pc 30}{\suttatitletranslation The training rule on the ending of the robe season }{\suttatitleroot Kathinuddhāra}}
\addcontentsline{toc}{section}{\tocacronym{Bi Pc 30} \toctranslation{The training rule on the ending of the robe season } \tocroot{Kathinuddhāra}}
\markboth{The training rule on the ending of the robe season }{Kathinuddhāra}
\extramarks{Bi Pc 30}{Bi Pc 30}

\subsection*{Origin story }

At\marginnote{1.1.1} one time the Buddha was staying at \textsanskrit{Sāvatthī} in the Jeta Grove, \textsanskrit{Anāthapiṇḍika}’s Monastery. At that time a lay follower had built a dwelling for the Sangha. He wanted to give out-of-season robe-cloth to both Sanghas at the presentation ceremony for that dwelling, but both Sanghas had already performed the robe-making ceremony. That lay follower then went to the Sangha and asked it to make an end of the robe season. 

They\marginnote{1.1.6} told the Buddha. Soon afterwards the Buddha gave a teaching and addressed the monks: 

\scrule{“Monks, I allow you to end the robe season. }

And\marginnote{1.1.9} it should be done like this. A competent and capable monk should inform the Sangha: 

‘Please,\marginnote{1.1.11} Venerables, I ask the Sangha to listen. If the Sangha is ready, it should end the robe season. This is the motion. 

Please,\marginnote{1.1.14} Venerables, I ask the Sangha to listen. The Sangha ends the robe season. Any monk who approves of ending the robe season should remain silent. Any monk who doesn’t approve should speak up. 

The\marginnote{1.1.18} Sangha has made an end of the robe season. The Sangha approves and is therefore silent. I’ll remember it thus.’” 

That\marginnote{1.2.1} lay follower then went to the Sangha of nuns and asked them to end the robe season. But the nun \textsanskrit{Thullanandā}, aiming to get robe-cloth for herself, blocked the Sangha from doing it. That lay follower complained and criticized them, “How could the nuns not end the robe season for us?” 

The\marginnote{1.2.6} nuns heard the complaints of that lay follower. The nuns of few desires complained and criticized her, “How could Venerable \textsanskrit{Thullanandā} block a legitimate ending of the robe season?” … “Is it true, monks, that the nun \textsanskrit{Thullanandā} did this?” 

“It’s\marginnote{1.2.10} true, Sir.” 

The\marginnote{1.2.11} Buddha rebuked her … “How could the nun \textsanskrit{Thullanandā} do this? This will affect people’s confidence …” … “And, monks, the nuns should recite this training rule like this: 

\subsection*{Final ruling }

\scrule{‘If a nun blocks a legitimate ending of the robe season, she commits an offense entailing confession.’” }

\subsection*{Definitions }

\begin{description}%
\item[A: ] whoever … %
\item[Nun: ] … The nun who has been given the full ordination in unanimity by both Sanghas through a legal procedure consisting of one motion and three announcements that is irreversible and fit to stand—this sort of nun is meant in this case. %
\item[A legitimate ending of the robe season: ] a unanimous Sangha of nuns brings it to an end. %
\item[Blocks: ] if she blocks it with the thought, “What can be done so that the robe season doesn’t end?”, she commits an offense entailing confession. %
\end{description}

\subsection*{Permutations }

If\marginnote{2.1.9.1} it is a legitimate legal procedure, and she perceives it as such, and she blocks it, she commits an offense entailing confession. If it is a legitimate legal procedure, but she is unsure of it, and she blocks it, she commits an offense of wrong conduct. If it is a legitimate legal procedure, but she perceives it as illegitimate, and she blocks it, there is no offense. 

If\marginnote{2.1.12} it is an illegitimate legal procedure, but she perceives it as legitimate, she commits an offense of wrong conduct. If it is an illegitimate legal procedure, but she is unsure of it, she commits an offense of wrong conduct. If it is an illegitimate legal procedure, and she perceives it as such, there is no offense. 

\subsection*{Non-offenses }

There\marginnote{2.2.1} is no offense: if she can show a benefit in blocking it;  if she is insane;  if she is the first offender. 

\scendsutta{The tenth training rule is finished. }

\scendvagga{The third subchapter on nakedness is finished. }

%
\section*{{\suttatitleacronym Bi Pc 31}{\suttatitletranslation The training rule on lying down on the same bed }{\suttatitleroot Ekamañcatuvaṭṭana}}
\addcontentsline{toc}{section}{\tocacronym{Bi Pc 31} \toctranslation{The training rule on lying down on the same bed } \tocroot{Ekamañcatuvaṭṭana}}
\markboth{The training rule on lying down on the same bed }{Ekamañcatuvaṭṭana}
\extramarks{Bi Pc 31}{Bi Pc 31}

\subsection*{Origin story }

At\marginnote{1.1} one time when the Buddha was staying at \textsanskrit{Sāvatthī} in \textsanskrit{Anāthapiṇḍika}’s Monastery, two nuns were lying down on the same bed. When people walking about the dwellings saw this, they complained and criticized them, “How can two nuns lie down on the same bed? They’re just like householders who indulge in worldly pleasures!” 

The\marginnote{1.5} nuns heard the complaints of those people, and the nuns of few desires complained and criticized them, “How can nuns do this?” … “Is it true, monks, that nuns do this?” 

“It’s\marginnote{1.9} true, Sir.” 

The\marginnote{1.10} Buddha rebuked them … “How can nuns do this? This will affect people’s confidence …” … “And, monks, the nuns should recite this training rule like this: 

\subsection*{Final ruling }

\scrule{‘If two nuns lie down on the same bed, they commit an offense entailing confession.’” }

\subsection*{Definitions }

\begin{description}%
\item[Two: ] whoever … %
\item[Nuns: ] fully ordained is what is meant. %
\item[If two lie down on the same bed: ] if, when one is lying down, the other lies down, they commit an offense entailing confession. If both lie down together, they commit an offense entailing confession. Every time they get up and then lie down again, they commit an offense entailing confession. %
\end{description}

\subsection*{Non-offenses }

There\marginnote{2.2.1} is no offense: if, when one is lying down, the other sits down;  if both sit down together;  if they are insane;  if they are the first offenders. 

\scendsutta{The first training rule is finished. }

%
\section*{{\suttatitleacronym Bi Pc 32}{\suttatitletranslation The training rule on lying down on the same sheet }{\suttatitleroot Ekattharaṇatuvaṭṭana}}
\addcontentsline{toc}{section}{\tocacronym{Bi Pc 32} \toctranslation{The training rule on lying down on the same sheet } \tocroot{Ekattharaṇatuvaṭṭana}}
\markboth{The training rule on lying down on the same sheet }{Ekattharaṇatuvaṭṭana}
\extramarks{Bi Pc 32}{Bi Pc 32}

\subsection*{Origin story }

At\marginnote{1.1} one time when the Buddha was staying at \textsanskrit{Sāvatthī} in \textsanskrit{Anāthapiṇḍika}’s Monastery, two nuns were lying down on the same sheet and under the same cover. When people walking about the dwellings saw this, they complained and criticized them, “How can two nuns lie down on the same sheet and under the same cover? They’re just like householders who indulge in worldly pleasures!” 

The\marginnote{1.5} nuns heard the complaints of those people, and the nuns of few desires complained and criticized them, “How can nuns do this?” … “Is it true, monks, that nuns do this?” 

“It’s\marginnote{1.9} true, Sir.” 

The\marginnote{1.10} Buddha rebuked them … “How can nuns do this? This will affect people’s confidence …” … “And, monks, the nuns should recite this training rule like this: 

\subsection*{Final ruling }

\scrule{‘If two nuns lie down on the same sheet and under the same cover, they commit an offense entailing confession.’” }

\subsection*{Definitions }

\begin{description}%
\item[Two: ] whoever … %
\item[Nuns: ] fully ordained is what is meant. %
\item[If two lie down on the same sheet and under the same cover: ] if they spread out just the one and cover themselves with just the one, they commit an offense entailing confession. %
\end{description}

\subsection*{Permutations }

If\marginnote{2.2.1} it is the same sheet and the same cover, and they perceive them as such, and they lie down, they commit an offense entailing confession. If it is the same sheet and the same cover, but they are unsure of it, and they lie down, they commit an offense entailing confession. If it is the same sheet and the same cover, but they perceive them as different, and they lie down, they commit an offense entailing confession. 

If\marginnote{2.2.4} it is the same sheet but different covers, they commit an offense of wrong conduct. If it is different sheets but the same cover, they commit an offense of wrong conduct. 

If\marginnote{2.2.6} it is different sheets and different covers, but they perceive them as the same, they commit an offense of wrong conduct. If it is different sheets and different covers, but they are unsure of it, they commit an offense of wrong conduct. If it is different sheets and different covers, and they perceive them as such, there is no offense. 

\subsection*{Non-offenses }

There\marginnote{2.3.1} is no offense: if they make a partition and then lie down;\footnote{Sp 2.940: \textit{\textsanskrit{Vavatthānaṁ} \textsanskrit{dassetvāti} majjhe \textsanskrit{kāsāvaṁ} \textsanskrit{vā} \textsanskrit{kattarayaṭṭhiṁ} \textsanskrit{vā} antamaso \textsanskrit{kāyabandhanampi} \textsanskrit{ṭhapetvā} \textsanskrit{nipajjantīnaṁ} \textsanskrit{anāpattīti} attho}, “\textit{\textsanskrit{Vavatthānaṁ} \textsanskrit{dassetvā}} means there is no offense for those who lie down after putting an ocher cloth, a staff, or even a belt in the middle.” }  if they are insane;  if they are the first offenders. 

\scendsutta{The second training rule is finished. }

%
\section*{{\suttatitleacronym Bi Pc 33}{\suttatitletranslation The training rule on making ill at ease }{\suttatitleroot Aphāsukaraṇa}}
\addcontentsline{toc}{section}{\tocacronym{Bi Pc 33} \toctranslation{The training rule on making ill at ease } \tocroot{Aphāsukaraṇa}}
\markboth{The training rule on making ill at ease }{Aphāsukaraṇa}
\extramarks{Bi Pc 33}{Bi Pc 33}

\subsection*{Origin story }

At\marginnote{1.1} one time the Buddha was staying at \textsanskrit{Sāvatthī} in the Jeta Grove, \textsanskrit{Anāthapiṇḍika}’s Monastery. At that time the nun \textsanskrit{Thullanandā} was a learned reciter, and she was confident and skilled at giving teachings. \textsanskrit{Bhaddā} \textsanskrit{Kāpilānī}, too, was a learned reciter who was confident and skilled at giving teachings, and she was respected for her excellence. Because of this, people visited \textsanskrit{Bhaddā} \textsanskrit{Kāpilānī} first and then \textsanskrit{Thullanandā}. Overcome by jealousy, \textsanskrit{Thullanandā} thought, “These ones, who are supposedly contented and have few desires, who are supposedly secluded and not socializing, are always persuading and convincing people.” And in front of \textsanskrit{Bhaddā} \textsanskrit{Kāpilānī}, she walked back and forth, stood, sat down, and lay down, and she recited and had others recite, and she rehearsed.\footnote{Sp 2.941: \textit{\textsanskrit{Saññattibahulā}; \textsanskrit{divasaṁ} \textsanskrit{mahājanaṁ} \textsanskrit{saññāpayamānāti} attho. … \textsanskrit{Viññattīti} \textsanskrit{hetūdāharaṇādīhi} vividhehi nayehi \textsanskrit{ñāpanā} \textsanskrit{veditabbā}, na \textsanskrit{yācanā}.} “\textit{\textsanskrit{Saññattibahulā}}; the meaning is they are trying to persuade crowds of people during the day. … \textit{\textsanskrit{Viññatti}}: not to be understood as asking for things, but as making known by various methods consisting of reasons, examples, etc.” } 

The\marginnote{1.8} nuns of few desires complained and criticized her, “How could Venerable \textsanskrit{Thullanandā} intentionally make \textsanskrit{Bhaddā} \textsanskrit{Kāpilānī} ill at ease?” … “Is it true, monks, that the nun \textsanskrit{Thullanandā} did this?” 

“It’s\marginnote{1.11} true, Sir.” 

The\marginnote{1.12} Buddha rebuked her … “How could the nun \textsanskrit{Thullanandā} do this? This will affect people’s confidence …” … “And, monks, the nuns should recite this training rule like this: 

\subsection*{Final ruling }

\scrule{‘If a nun intentionally makes a nun ill at ease, she commits an offense entailing confession.’” }

\subsection*{Definitions }

\begin{description}%
\item[A: ] whoever … %
\item[Nun: ] … The nun who has been given the full ordination in unanimity by both Sanghas through a legal procedure consisting of one motion and three announcements that is irreversible and fit to stand—this sort of nun is meant in this case. %
\item[A nun: ] another nun. %
\item[Intentionally: ] knowing, perceiving, having intended, having decided, she transgresses. %
\item[Makes ill at ease: ] if, without asking permission, but thinking, “In this way she will be ill at ease,” she walks back and forth in front of her, or she stands, sits down, or lies down in front of her, or she recites, has others recite, or rehearses in front of her, she commits an offense entailing confession. %
\end{description}

\subsection*{Permutations }

If\marginnote{2.2.1} the other person is fully ordained, and she perceives her as such, and she intentionally makes her ill at ease, she commits an offense entailing confession. If the other person is fully ordained, but she is unsure of it, and she intentionally makes her ill at ease, she commits an offense entailing confession. If the other person is fully ordained, but she does not perceive her as such, and she intentionally makes her ill at ease, she commits an offense entailing confession. 

If\marginnote{2.2.4} the other person is not fully ordained, and she intentionally makes her ill at ease, she commits an offense of wrong conduct. If the other person is not fully ordained, but she perceives her as such, she commits an offense of wrong conduct. If the other person is not fully ordained, but she is unsure of it, she commits an offense of wrong conduct. If the other person is not fully ordained, and she does not perceive her as such, she commits an offense of wrong conduct. 

\subsection*{Non-offenses }

There\marginnote{2.3.1} is no offense: if, after asking permission and not desiring to make her ill at ease, she walks back and forth in front of her, or she stands, sits down, or lies down in front of her, or she recites, has others recite, or rehearses in front of her;  if she is insane;  if she is the first offender. 

\scendsutta{The third training rule is finished. }

%
\section*{{\suttatitleacronym Bi Pc 34}{\suttatitletranslation The training rule on not having someone nursed }{\suttatitleroot Naupaṭṭhāpana}}
\addcontentsline{toc}{section}{\tocacronym{Bi Pc 34} \toctranslation{The training rule on not having someone nursed } \tocroot{Naupaṭṭhāpana}}
\markboth{The training rule on not having someone nursed }{Naupaṭṭhāpana}
\extramarks{Bi Pc 34}{Bi Pc 34}

\subsection*{Origin story }

At\marginnote{1.1} one time the Buddha was staying at \textsanskrit{Sāvatthī} in the Jeta Grove, \textsanskrit{Anāthapiṇḍika}’s Monastery. At that time a disciple of the nun \textsanskrit{Thullanandā} was suffering, but \textsanskrit{Thullanandā} neither nursed her nor made any effort to have someone else nurse her. 

The\marginnote{1.3} nuns of few desires complained and criticized her, “How could Venerable \textsanskrit{Thullanandā} not nurse a suffering disciple, nor make any effort to have someone else nurse her?” … “Is it true, monks, that the nun \textsanskrit{Thullanandā} didn’t do this?” 

“It’s\marginnote{1.6} true, Sir.” 

The\marginnote{1.7} Buddha rebuked her … “How could the nun \textsanskrit{Thullanandā} not do this? This will affect people’s confidence …” … “And, monks, the nuns should recite this training rule like this: 

\subsection*{Final ruling }

\scrule{‘If a nun neither nurses a suffering disciple, nor makes any effort to have someone else nurse her, she commits an offense entailing confession.’” }

\subsection*{Definitions }

\begin{description}%
\item[A: ] whoever … %
\item[Nun: ] … The nun who has been given the full ordination in unanimity by both Sanghas through a legal procedure consisting of one motion and three announcements that is irreversible and fit to stand—this sort of nun is meant in this case. %
\item[Suffering: ] sick is what is meant. %
\item[Disciple: ] a student is what is meant. %
\item[Neither nurses: ] does not herself nurse her. %
\item[Nor makes any effort to have someone else nurse her: ] she does not ask anyone else. %
\end{description}

If\marginnote{2.1.13} she thinks, “I will neither nurse her nor make any effort to have someone else nurse her,” then by the mere fact of abandoning her duty, she commits an offense entailing confession. If she neither nurses nor makes any effort to have someone else nurse a pupil or one who is not fully ordained, she commits an offense of wrong conduct.\footnote{The reason a pupil (\textit{\textsanskrit{antevāsī}}) is treated differently from a disciple/student (\textit{\textsanskrit{sahajīvinī}/\textsanskrit{saddhivihārinī}}) is presumably because the former refers to the pupil of an \textit{\textsanskrit{ācarinī}}, a teacher, whereas the latter to the student of a \textit{\textsanskrit{pavattinī}}, a preceptor. } 

\subsection*{Non-offenses }

There\marginnote{2.2.1} is no offense: if there is an obstacle;  if she searches but does not find a nurse;  if she is sick;  if there is an emergency;  if she is insane;  if she is the first offender. 

\scendsutta{The fourth training rule is finished. }

%
\section*{{\suttatitleacronym Bi Pc 35}{\suttatitletranslation The training rule on throwing out }{\suttatitleroot Nikkaḍḍhana}}
\addcontentsline{toc}{section}{\tocacronym{Bi Pc 35} \toctranslation{The training rule on throwing out } \tocroot{Nikkaḍḍhana}}
\markboth{The training rule on throwing out }{Nikkaḍḍhana}
\extramarks{Bi Pc 35}{Bi Pc 35}

\subsection*{Origin story }

At\marginnote{1.1} one time the Buddha was staying at \textsanskrit{Sāvatthī} in the Jeta Grove, \textsanskrit{Anāthapiṇḍika}’s Monastery. At that time \textsanskrit{Bhaddā} \textsanskrit{Kāpilānī} had entered the rainy-season residence at \textsanskrit{Sāketa}. But because she was disturbed by a certain matter, she sent a message to the nun \textsanskrit{Thullanandā}: “If you would give me a dwelling place, I would come to \textsanskrit{Sāvatthī}.” \textsanskrit{Thullanandā} replied, “Please come; I’ll give you one.” 

\textsanskrit{Bhaddā}\marginnote{1.7} \textsanskrit{Kāpilānī} then traveled from \textsanskrit{Sāketa} to \textsanskrit{Sāvatthī}, and \textsanskrit{Thullanandā} gave her a dwelling place. At that time \textsanskrit{Thullanandā} was a learned reciter, and she was confident and skilled at giving teachings. \textsanskrit{Bhaddā} \textsanskrit{Kāpilānī}, too, was a learned reciter who was confident and skilled at giving teachings, and she was respected for her excellence. Because of this, people visited \textsanskrit{Bhaddā} \textsanskrit{Kāpilānī} first and then \textsanskrit{Thullanandā}. Overcome by jealousy, \textsanskrit{Thullanandā} thought, “These ones, who are supposedly contented and have few desires, who are supposedly secluded and not socializing, are always persuading and convincing people.” And in anger she threw \textsanskrit{Bhaddā} \textsanskrit{Kāpilānī} out of that dwelling place.\footnote{Sp 2.941: \textit{\textsanskrit{Saññattibahulā}; \textsanskrit{divasaṁ} \textsanskrit{mahājanaṁ} \textsanskrit{saññāpayamānāti} attho. … \textsanskrit{Viññattīti} \textsanskrit{hetūdāharaṇādīhi} vividhehi nayehi \textsanskrit{ñāpanā} \textsanskrit{veditabbā}, na \textsanskrit{yācanā}.} “\textit{\textsanskrit{Saññattibahulā}}; the meaning is they are trying to persuade crowds of people during the day. … \textit{\textsanskrit{Viññatti}}: not to be understood as asking for things, but as making known by various methods consisting of reasons, examples, etc.” } 

The\marginnote{1.15} nuns of few desires complained and criticized her, “How could Venerable \textsanskrit{Thullanandā} give a dwelling place to Venerable \textsanskrit{Bhaddā} \textsanskrit{Kāpilānī} and then throw her out in anger?” … “Is it true, monks, that the nun \textsanskrit{Thullanandā} did this?” 

“It’s\marginnote{1.18} true, Sir.” 

The\marginnote{1.19} Buddha rebuked her … “How could the nun \textsanskrit{Thullanandā} do this? This will affect people’s confidence …” … “And, monks, the nuns should recite this training rule like this: 

\subsection*{Final ruling }

\scrule{‘If a nun gives a dwelling place to a nun, and then, in anger, throws her out or has her thrown out, she commits an offense entailing confession.’” }

\subsection*{Definitions }

\begin{description}%
\item[A: ] whoever … %
\item[Nun: ] … The nun who has been given the full ordination in unanimity by both Sanghas through a legal procedure consisting of one motion and three announcements that is irreversible and fit to stand—this sort of nun is meant in this case. %
\item[To a nun: ] to another nun. %
\item[A dwelling place: ] one that has a door is what is meant. %
\item[Gives: ] she gives it herself. %
\item[In anger: ] discontent, having hatred, hostile. %
\item[Throws out: ] if she takes hold of her in a room and throws her out to the entryway, she commits an offense entailing confession. If she takes hold of her in the entryway and throws her outside, she commits an offense entailing confession. Even if she makes her go through many doors with a single effort, she commits one offense entailing confession. %
\item[Has thrown out: ] if she asks another, she commits an offense entailing confession. If she only asks once, then even if the other makes her go through many doors, she commits one offense entailing confession. %
\end{description}

\subsection*{Permutations }

If\marginnote{2.2.1} the other person is fully ordained, and she perceives her as such, and she gives her a dwelling place, and she then throws her out in anger or has her thrown out, she commits an offense entailing confession. If the other person is fully ordained, but she is unsure of it, and she gives her a dwelling place, and she then throws her out in anger or has her thrown out, she commits an offense entailing confession. If the other person is fully ordained, but she does not perceive her as such, and she gives her a dwelling place, and she then throws her out in anger or has her thrown out, she commits an offense entailing confession. 

If\marginnote{2.2.4} she throws out one of her requisites, or she has it thrown out, she commits an offense of wrong conduct.\footnote{“Requisites” renders \textit{\textsanskrit{parikkhāra}}. For a discussion of this word, see Appendix of Technical Terms. } If she throws her out or has her thrown out from a dwelling place without a door, she commits an offense of wrong conduct. If she throws out one of her requisites from a dwelling place without a door, or she has it thrown out, she commits an offense of wrong conduct. 

If\marginnote{2.2.7} she throws out one who is not fully ordained, or she has her thrown out, from a dwelling place with or without a door, she commits an offense of wrong conduct.\footnote{\textit{\textsanskrit{Anupasampannaṁ}} could in theory be either male or female, but since the rest of this permutation series uses the feminine gender, I take it that the feminine is to be understood here as well. } If she throws out one of her requisites from that place, or she has it thrown out, she commits an offense of wrong conduct. 

If\marginnote{2.2.9} the other person is not fully ordained, but she perceives her as such, she commits an offense of wrong conduct. If the other person is not fully ordained, but she is unsure of it, she commits an offense of wrong conduct. If the other person is not fully ordained, and she does not perceive her as such, she commits an offense of wrong conduct. 

\subsection*{Non-offenses }

There\marginnote{2.3.1} is no offense: if she throws out, or has thrown out, one who is shameless;  if she throws out, or has thrown out, a requisite belonging to that person;  if she throws out, or has thrown out, one who is insane;  if she throws out, or has thrown out, a requisite belonging to that person;  if she throws out, or has thrown out, one who is quarrelsome and argumentative, and who creates legal issues in the Sangha;  if she throws out, or has thrown out, a requisite belonging to that person;  if she throws out, or has thrown out, a pupil or student who is not conducting herself properly;  if she throws out, or has thrown out, a requisite belonging to that person;  if she is insane;  if she is the first offender. 

\scendsutta{The fifth training rule is finished. }

%
\section*{{\suttatitleacronym Bi Pc 36}{\suttatitletranslation The training rule on socializing }{\suttatitleroot Saṁsaṭṭha}}
\addcontentsline{toc}{section}{\tocacronym{Bi Pc 36} \toctranslation{The training rule on socializing } \tocroot{Saṁsaṭṭha}}
\markboth{The training rule on socializing }{Saṁsaṭṭha}
\extramarks{Bi Pc 36}{Bi Pc 36}

\subsection*{Origin story }

At\marginnote{1.1} one time when the Buddha was staying at \textsanskrit{Sāvatthī} in \textsanskrit{Anāthapiṇḍika}’s Monastery, the nun \textsanskrit{Caṇḍakāḷī} was socializing with householders and their offspring. 

The\marginnote{1.3} nuns of few desires complained and criticized her, “How can Venerable \textsanskrit{Caṇḍakāḷī} socialize with householders and their offspring?” … “Is it true, monks, that the nun \textsanskrit{Caṇḍakāḷī} does this?” 

“It’s\marginnote{1.6} true, Sir.” 

The\marginnote{1.7} Buddha rebuked her … “How can the nun \textsanskrit{Caṇḍakāḷī} do this? This will affect people’s confidence …” … “And, monks, the nuns should recite this training rule like this: 

\subsection*{Final ruling }

\scrule{‘If a nun is socializing with a householder or a householder’s offspring, the nuns should correct her like this: “Venerable, don’t socialize with householders or householders’ offspring. Be secluded, Venerable. The Sangha praises seclusion for the Sisters.”\footnote{The Pali has the singular \textit{\textsanskrit{bhaginiyā}}, “for a sister”, but I render it in the plural to fit better with English idiom. } If that nun continues as before, the nuns should press her up to three times to make her stop. If she then stops, all is well. If she does not stop, she commits an offense entailing confession.’” }

\subsection*{Definitions }

\begin{description}%
\item[A: ] whoever … %
\item[Nun: ] … The nun who has been given the full ordination in unanimity by both Sanghas through a legal procedure consisting of one motion and three announcements that is irreversible and fit to stand—this sort of nun is meant in this case. %
\item[Socializing: ] she socializes with improper bodily and verbal actions. %
\item[A householder: ] anyone who lives at home. %
\item[A householder’s offspring: ] whoever is an offspring or a sibling.\footnote{“Offspring” renders \textit{putta/\textsanskrit{ā}}, whereas “sibling” renders \textit{\textsanskrit{bhātaro}}. In Pali the male gender takes precedent if a group contains people of both sexes. For instance, the plural \textit{\textsanskrit{puttā}}, “sons”, may mean “children” or “offsping”, depending on the context. In the same way, the plural \textit{\textsanskrit{bhātāro}}, “brothers”, can mean “siblings”. This way of understanding male-gender nouns is confirmed in the introduction to the Pali lexical work the \textsanskrit{Abhidhānappadīpikāṭīkā}: \textit{Ettha hi \textsanskrit{mātā} ca \textsanskrit{pitā} ca pitaro, putto ca \textsanskrit{dhītā} ca \textsanskrit{puttā}, sassu ca sasuro ca \textsanskrit{sasurā}, \textsanskrit{bhātā} ca \textsanskrit{bhaginī} ca \textsanskrit{bhātaroti} \textsanskrit{bhinnaliṅgānampi} ekaseso dassitoti}, “Mother and father are fathers; son and daughter are sons; mother-in-law and father-in-law are fathers-in-law; brother and sister are brothers;’ in this case the split gender is shown with only one gender remaining.” The \textsanskrit{Abhidhānappadīpikāṭīkā} is available online at tipitaka.org. } %
\item[Her: ] the nun who is socializing. %
\item[The nuns: ] other\marginnote{2.1.14} nuns who see it or hear about it. They should correct her like this: 

“Venerable,\marginnote{2.1.15} don’t socialize with householders or householders’ offspring. Be secluded, Venerable. The Sangha praises seclusion for the Sisters.” 

And\marginnote{2.1.18} they should correct her a second and a third time. If she stops, all is well. If she does not stop, she commits an offense of wrong conduct. If those who hear about it do not say anything, they commit an offense of wrong conduct. 

That\marginnote{2.1.23} nun, even if she has to be pulled into the Sangha, should be corrected like this: 

“Venerable,\marginnote{2.1.24} don’t socialize with householders or householders’ offspring. Be secluded, Venerable. The Sangha praises seclusion for the Sisters.” 

They\marginnote{2.1.27} should correct her a second and a third time. If she stops, all is well. If she does not stop, she commits an offense of wrong conduct. 

%
\item[Should press her: ] “And,\marginnote{2.1.32} monks, she should be pressed like this. A competent and capable nun should inform the Sangha: 

‘Please,\marginnote{2.1.34} Venerables, I ask the Sangha to listen. The nun so-and-so is socializing with householders and their offspring. And she keeps on doing it. If the Sangha is ready, it should press her to make her stop. This is the motion. 

Please,\marginnote{2.1.39} Venerables, I ask the Sangha to listen. The nun so-and-so is socializing with householders and their offspring. And she keeps on doing it. The Sangha presses her to make her stop. Any nun who approves of pressing nun so-and-so to make her stop should remain silent. Any nun who doesn’t approve should speak up. 

For\marginnote{2.1.45} the second time I speak on this matter … For the third time I speak on this matter … 

The\marginnote{2.1.47} Sangha has pressed nun so-and-so to stop. The Sangha approves and is therefore silent. I’ll remember it thus.’” 

After\marginnote{2.1.49} the motion, she commits an offense of wrong conduct.\footnote{The Pali just says \textit{\textsanskrit{dukkaṭa}}, without specifying that it is an \textit{\textsanskrit{āpatti}}, “an offense”. Yet elsewhere, such as at \href{https://suttacentral.net/pli-tv-bu-vb-ss10/en/brahmali\#2.65}{Bu Ss 10:2.65}, the \textit{\textsanskrit{dukkaṭa}} is annulled if you commit the full offense of \textit{\textsanskrit{saṅghādisesa}}. The implication is that in these contexts \textit{\textsanskrit{dukkaṭa}} should be read as \textit{\textsanskrit{āpatti} \textsanskrit{dukkaṭassa}}, “an offense of wrong conduct”. } After each of the first two announcements, she commits an offense of wrong conduct. When the last announcement is finished, she commits an offense entailing confession. 

%
\end{description}

\subsection*{Permutations }

If\marginnote{2.2.1} it is a legitimate legal procedure, and she perceives it as such, and she does not stop, she commits an offense entailing confession. If it is a legitimate legal procedure, but she is unsure of it, and she does not stop, she commits an offense entailing confession. If it is a legitimate legal procedure, but she perceives it as illegitimate, and she does not stop, she commits an offense entailing confession. 

If\marginnote{2.2.4} it is an illegitimate legal procedure, but she perceives it as legitimate, she commits an offense of wrong conduct. If it is an illegitimate legal procedure, but she is unsure of it, she commits an offense of wrong conduct. If it is an illegitimate legal procedure, and she perceives it as such, she commits an offense of wrong conduct. 

\subsection*{Non-offenses }

There\marginnote{2.3.1} is no offense: if she has not been pressed;  if she stops;  if she is insane;  if she is the first offender. 

\scendsutta{The sixth training rule is finished. }

%
\section*{{\suttatitleacronym Bi Pc 37}{\suttatitletranslation The training rule on within their own country }{\suttatitleroot Antoraṭṭha}}
\addcontentsline{toc}{section}{\tocacronym{Bi Pc 37} \toctranslation{The training rule on within their own country } \tocroot{Antoraṭṭha}}
\markboth{The training rule on within their own country }{Antoraṭṭha}
\extramarks{Bi Pc 37}{Bi Pc 37}

\subsection*{Origin story }

At\marginnote{1.1} one time the Buddha was staying at \textsanskrit{Sāvatthī} in the Jeta Grove, \textsanskrit{Anāthapiṇḍika}’s Monastery. At that time the nuns went wandering without a group of travelers where it was considered risky and dangerous within their own country. Scoundrels raped them.\footnote{“Raped” renders \textit{\textsanskrit{dūsenti}}. For a discussion of this word, see Appendix of Technical Terms. } 

The\marginnote{1.4} nuns of few desires complained and criticized them, “How can nuns go wandering without a group of travelers where it’s considered risky and dangerous within their own country?” … “Is it true, monks, that nuns do this?” 

“It’s\marginnote{1.7} true, Sir.” 

The\marginnote{1.8} Buddha rebuked them … “How can nuns do this? This will affect people’s confidence …” … “And, monks, the nuns should recite this training rule like this: 

\subsection*{Final ruling }

\scrule{‘If a nun goes wandering without a group of travelers where it is considered risky and dangerous within her own country, she commits an offense entailing confession.’” }

\subsection*{Definitions }

\begin{description}%
\item[A: ] whoever … %
\item[Nun: ] … The nun who has been given the full ordination in unanimity by both Sanghas through a legal procedure consisting of one motion and three announcements that is irreversible and fit to stand—this sort of nun is meant in this case. %
\item[Within her own country: ] in the country where she is living. %
\item[Risky: ] a place has been seen along that road where criminals are camping, eating, standing, sitting, or lying down. %
\item[Dangerous: ] criminals have been seen along that road, injuring, robbing, or beating people. %
\item[Without a group of travelers: ] in the absence of a group of travelers. %
\item[Goes wandering: ] when the villages are a chicken’s flight apart, then for every next village she commits an offense entailing confession.\footnote{For a discussion of the rendering “inhabited area” for \textit{\textsanskrit{gāma}}, see Appendix of Technical Terms. } When it is an uninhabited area, a wilderness, then for every six kilometers she commits an offense entailing confession.\footnote{“Six kilometers” renders \textit{addhayojana}, “half a \textit{yojana}”. For further discussion of the \textit{yojana}, see \textit{sugata} in Appendix of Technical Terms. } %
\end{description}

\subsection*{Non-offenses }

There\marginnote{2.2.1} is no offense: if she travels with a group;  if she travels where it is safe and free from danger;  if there is an emergency;  if she is insane;  if she is the first offender. 

\scendsutta{The seventh training rule is finished. }

%
\section*{{\suttatitleacronym Bi Pc 38}{\suttatitletranslation The training rule on outside their own country }{\suttatitleroot Tiroraṭṭha}}
\addcontentsline{toc}{section}{\tocacronym{Bi Pc 38} \toctranslation{The training rule on outside their own country } \tocroot{Tiroraṭṭha}}
\markboth{The training rule on outside their own country }{Tiroraṭṭha}
\extramarks{Bi Pc 38}{Bi Pc 38}

\subsection*{Origin story }

At\marginnote{1.1} one time the Buddha was staying at \textsanskrit{Sāvatthī} in the Jeta Grove, \textsanskrit{Anāthapiṇḍika}’s Monastery. At that time the nuns went wandering without a group of travelers where it was considered risky and dangerous outside their own country. Scoundrels raped them. 

The\marginnote{1.4} nuns of few desires complained and criticized them, “How can nuns go wandering without a group of travelers where it’s considered risky and dangerous outside their own country?” … “Is it true, monks, that nuns do this?” 

“It’s\marginnote{1.7} true, Sir.” 

The\marginnote{1.8} Buddha rebuked them … “How can nuns do this? This will affect people’s confidence …” … “And, monks, the nuns should recite this training rule like this: 

\subsection*{Final ruling }

\scrule{‘If a nun goes wandering without a group of travelers where it is considered risky and dangerous outside her own country, she commits an offense entailing confession.’” }

\subsection*{Definitions }

\begin{description}%
\item[A: ] whoever … %
\item[Nun: ] … The nun who has been given the full ordination in unanimity by both Sanghas through a legal procedure consisting of one motion and three announcements that is irreversible and fit to stand—this sort of nun is meant in this case. %
\item[Outside her own country: ] in any country apart from the one where she is living. %
\item[Risky: ] a place has been seen along that road where criminals are camping, eating, standing, sitting, or lying down. %
\item[Dangerous: ] criminals have been seen along that road, injuring, robbing, or beating people. %
\item[Without a group of travelers: ] in the absence of a group of travelers. %
\item[Goes wandering: ] when the villages are a chicken’s flight apart, then for every next village she commits an offense entailing confession. When it is an uninhabited area, a wilderness, then for every six kilometers she commits an offense entailing confession.\footnote{“Six kilometers” renders \textit{addhayojana}, “half a \textit{yojana}”. For further discussion of the \textit{yojana}, see \textit{sugata} in Appendix of Technical Terms. } %
\end{description}

\subsection*{Non-offenses }

There\marginnote{2.16.1} is no offense: if she travels with a group;  if she travels where it is safe and free from danger;  if there is an emergency;  if she is insane;  if she is the first offender. 

\scendsutta{The eighth training rule is finished. }

%
\section*{{\suttatitleacronym Bi Pc 39}{\suttatitletranslation The training rule on during the rainy season }{\suttatitleroot Antovassa}}
\addcontentsline{toc}{section}{\tocacronym{Bi Pc 39} \toctranslation{The training rule on during the rainy season } \tocroot{Antovassa}}
\markboth{The training rule on during the rainy season }{Antovassa}
\extramarks{Bi Pc 39}{Bi Pc 39}

\subsection*{Origin story }

At\marginnote{1.1} one time when the Buddha was staying at \textsanskrit{Rājagaha} in the Bamboo Grove, the squirrel sanctuary, the nuns went wandering during the rainy season. People complained and criticized them, “How can the nuns go wandering during the rainy season? They are trampling down the green grass, harming one-sensed life, and destroying many small beings.” 

The\marginnote{1.5} nuns heard the complaints of those people, and the nuns of few desires complained and criticized them, “How can nuns go wandering during the rainy season?” … “Is it true, monks, that nuns do this?” 

“It’s\marginnote{1.9} true, Sir.” 

The\marginnote{1.10} Buddha rebuked them … “How can nuns do this? This will affect people’s confidence …” … “And, monks, the nuns should recite this training rule like this: 

\subsection*{Final ruling }

\scrule{‘If a nun goes wandering during the rainy season, she commits an offense entailing confession.’” }

\subsection*{Definitions }

\begin{description}%
\item[A: ] whoever … %
\item[Nun: ] … The nun who has been given the full ordination in unanimity by both Sanghas through a legal procedure consisting of one motion and three announcements that is irreversible and fit to stand—this sort of nun is meant in this case. %
\item[During the rainy season: ] not having stayed put for the first three or the last three months of the rainy season. %
\item[Goes wandering: ] when the villages are a chicken’s flight apart, then for every next village she commits an offense entailing confession. When it is an uninhabited area, a wilderness, then for every six kilometers she commits an offense entailing confession.\footnote{“Six kilometers” renders \textit{addhayojana}, “half a \textit{yojana}”. For further discussion of the \textit{yojana}, see \textit{sugata} in Appendix of Technical Terms. } %
\end{description}

\subsection*{Non-offenses }

There\marginnote{2.2.1} is no offense: if she goes on seven-day business;  if she goes because something is disturbing her;  if there is an emergency;  if she is insane;  if she is the first offender. 

\scendsutta{The ninth training rule is finished. }

%
\section*{{\suttatitleacronym Bi Pc 40}{\suttatitletranslation The training rule on going wandering }{\suttatitleroot Cārikanapakkamana}}
\addcontentsline{toc}{section}{\tocacronym{Bi Pc 40} \toctranslation{The training rule on going wandering } \tocroot{Cārikanapakkamana}}
\markboth{The training rule on going wandering }{Cārikanapakkamana}
\extramarks{Bi Pc 40}{Bi Pc 40}

\subsection*{Origin story }

At\marginnote{1.1} one time the Buddha was staying at \textsanskrit{Rājagaha} in the Bamboo Grove, the squirrel sanctuary. At that time the nuns were staying right there at \textsanskrit{Rājagaha} for the rainy season, the winter, and the summer. People complained and criticized them, “The nuns are leaving the districts in darkness and obscurity. They don’t brighten them up by their presence.” 

The\marginnote{1.5} nuns heard the complaints of those people. They then told the monks, who in turn told the Buddha. Soon afterwards the Buddha gave a teaching and addressed the monks: “Well then, monks, I will lay down a training rule for the nuns for the following ten reasons: for the well-being of the Sangha … for the longevity of the true Teaching, and for supporting the training. And, monks, the nuns should recite this training rule like this: 

\subsection*{Final ruling }

\scrule{‘If a nun who has completed the rainy-season residence does not go wandering at least 65 to 80 kilometers, she commits an offense entailing confession.’”\footnote{That is, five or six \textit{yojanas}. For further discussion of the \textit{yojana}, see \textit{sugata} in Appendix of Technical Terms. } }

\subsection*{Definitions }

\begin{description}%
\item[A: ] whoever … %
\item[Nun: ] … The nun who has been given the full ordination in unanimity by both Sanghas through a legal procedure consisting of one motion and three announcements that is irreversible and fit to stand—this sort of nun is meant in this case. %
\item[Who has completed the rainy-season residence: ] who has completed the first three or the last three months of the rainy-season residence. %
\end{description}

If\marginnote{2.1.7} she thinks, “I won’t go wandering, not even 65 to 80 kilometers,” then by the mere fact of abandoning her duty, she commits an offense entailing confession. 

\subsection*{Non-offenses }

There\marginnote{2.2.1} is no offense: if there is an obstacle;  if she searches for a companion nun, but is unable to find one;  if she is sick;  if there is an emergency;  if she is insane;  if she is the first offender. 

\scendsutta{The tenth training rule is finished. }

\scendvagga{The fourth subchapter on lying down is finished. }

%
\section*{{\suttatitleacronym Bi Pc 41}{\suttatitletranslation The training rule on royal houses }{\suttatitleroot Rājāgāra}}
\addcontentsline{toc}{section}{\tocacronym{Bi Pc 41} \toctranslation{The training rule on royal houses } \tocroot{Rājāgāra}}
\markboth{The training rule on royal houses }{Rājāgāra}
\extramarks{Bi Pc 41}{Bi Pc 41}

\subsection*{Origin story }

At\marginnote{1.1} one time when the Buddha was staying at \textsanskrit{Sāvatthī} in \textsanskrit{Anāthapiṇḍika}’s Monastery, artwork had been installed in the pleasure house in King Pasenadi of Kosala’s park. Many people visited the pleasure house, as did the nuns from the group of six. People complained and criticized them, “How can nuns visit a pleasure house? They’re just like householders who indulge in worldly pleasures!” 

The\marginnote{1.7} nuns heard the complaints of those people, and the nuns of few desires complained and criticized them, “How could the nuns from the group of six do this?” … “Is it true, monks, that those nuns did this?” 

“It’s\marginnote{1.11} true, Sir.” 

The\marginnote{1.12} Buddha rebuked them … “How could the nuns from the group of six do this? This will affect people’s confidence …” … “And, monks, the nuns should recite this training rule like this: 

\subsection*{Final ruling }

\scrule{‘If a nun visits a royal house or a pleasure house or a park or a garden or a lotus pond, she commits an offense entailing confession.’” }

\subsection*{Definitions }

\begin{description}%
\item[A: ] whoever … %
\item[Nun: ] … The nun who has been given the full ordination in unanimity by both Sanghas through a legal procedure consisting of one motion and three announcements that is irreversible and fit to stand—this sort of nun is meant in this case. %
\item[A royal house: ] wherever one has been built for a king to entertain and enjoy himself. %
\item[A pleasure house: ] wherever one has been built for a people to entertain and enjoy themselves. %
\item[A park: ] wherever one has been made for a people to entertain and enjoy themselves. %
\item[A garden: ] wherever one has been made for a people to entertain and enjoy themselves. %
\item[A lotus pond: ] wherever one has been made for a people to entertain and enjoy themselves.\footnote{The usual meaning of \textit{\textsanskrit{pokkharaṇī}} as “lotus pond” is well established in the suttas. Much of the time they seem to have been decorative, but there are few instances in the Vinaya where they are used for washing, such as in \textsanskrit{Mahā}-khandhaka where the Buddha washes a cloth in a \textit{\textsanskrit{pokkharaṇī}} (\href{https://suttacentral.net/pli-tv-kd1/en/brahmali\#20.1.4}{Kd 1:20.1.4}). In fact, this distinction in use is reflected in the commentaries, which speak of \textit{\textsanskrit{nahāna}-\textsanskrit{pokkharaṇī}}, “\textit{\textsanskrit{pokkharaṇi}} for bathing”, and \textit{\textsanskrit{kīḷana}-\textsanskrit{pokkharaṇi}}, “\textit{\textsanskrit{pokkharaṇi}} for playing”. Because \textit{\textsanskrit{pokkharaṇis}} were given to monasteries “for the benefit of the Sangha” (\href{https://suttacentral.net/pli-tv-kd15/en/brahmali\#17.2.1}{Kd 15:17.2.1}), they were probably meant for bathing and washing, not just for decoration. Moreover, it seems the \textit{\textsanskrit{pokkharaṇis}} could be quite elaborate structures with foundations, staircases, and rails (\href{https://suttacentral.net/pli-tv-kd15/en/brahmali\#17.2.6}{Kd 15:17.2.1}). For these reasons I vary my translation according to context, sometimes using “pond”, at other times “tank”, and sometimes adding the qualifier “lotus”. } %
\end{description}

If\marginnote{2.1.15} she is on her way to visit them, she commits an offense of wrong conduct. Wherever she stands to see them, she commits an offense entailing confession. Every time she goes beyond the range of sight and then sees them again, she commits an offense entailing confession. 

If\marginnote{2.1.18} she is on her way to visit any one of them, she commits an offense of wrong conduct. Wherever she stands to see it, she commits an offense entailing confession. Every time she goes beyond the range of sight and then sees it again, she commits an offense entailing confession. 

\subsection*{Non-offenses }

There\marginnote{2.2.1} is no offense: if she sees it while remaining in a monastery;  if she sees it while coming or going;  if she goes when there is something to be done and then sees it;  if there is an emergency;  if she is insane;  if she is the first offender. 

\scendsutta{The first training rule is finished. }

%
\section*{{\suttatitleacronym Bi Pc 42}{\suttatitletranslation The training rule on using high couches }{\suttatitleroot Āsandiparibhuñjana}}
\addcontentsline{toc}{section}{\tocacronym{Bi Pc 42} \toctranslation{The training rule on using high couches } \tocroot{Āsandiparibhuñjana}}
\markboth{The training rule on using high couches }{Āsandiparibhuñjana}
\extramarks{Bi Pc 42}{Bi Pc 42}

\subsection*{Origin story }

At\marginnote{1.1} one time when the Buddha was staying at \textsanskrit{Sāvatthī} in \textsanskrit{Anāthapiṇḍika}’s Monastery, the nuns were using high and luxurious couches. When people walking about the dwellings saw this, they complained and criticized them, “How can nuns use high and luxurious couches? They’re just like householders who indulge in worldly pleasures!” 

The\marginnote{1.5} nuns heard the complaints of those people, and the nuns of few desires complained and criticized them, “How can nuns do this?” … “Is it true, monks, that nuns do this?” 

“It’s\marginnote{1.9} true, Sir.” 

The\marginnote{1.10} Buddha rebuked them … “How can nuns do this? This will affect people’s confidence …” … “And, monks, the nuns should recite this training rule like this: 

\subsection*{Final ruling }

\scrule{‘If a nun uses a high or luxurious couch, she commits an offense entailing confession.’” }

\subsection*{Definitions }

\begin{description}%
\item[A: ] whoever … %
\item[Nun: ] … The nun who has been given the full ordination in unanimity by both Sanghas through a legal procedure consisting of one motion and three announcements that is irreversible and fit to stand—this sort of nun is meant in this case. %
\item[A high couch: ] one that is oversize is what is meant. %
\item[A luxurious couch: ] one decorated with images of predatory animals. %
\item[Uses: ] if she sits down or lies down on it, she commits an offense entailing confession. %
\end{description}

\subsection*{Non-offenses }

There\marginnote{2.2.1} is no offense: if she uses a high couch after cutting off the legs;  if she uses a luxurious couch after removing the decorations with images of predatory animals;  if she is insane;  if she is the first offender. 

\scendsutta{The second training rule is finished. }

%
\section*{{\suttatitleacronym Bi Pc 43}{\suttatitletranslation The training rule on spinning yarn }{\suttatitleroot Suttakantana}}
\addcontentsline{toc}{section}{\tocacronym{Bi Pc 43} \toctranslation{The training rule on spinning yarn } \tocroot{Suttakantana}}
\markboth{The training rule on spinning yarn }{Suttakantana}
\extramarks{Bi Pc 43}{Bi Pc 43}

\subsection*{Origin story }

At\marginnote{1.1} one time when the Buddha was staying at \textsanskrit{Sāvatthī} in \textsanskrit{Anāthapiṇḍika}’s Monastery, the nuns from the group of six were spinning yarn. When people walking about the dwellings saw this, they complained and criticized them, “How can the nuns spin yarn? They’re just like householders who indulge in worldly pleasures!” 

The\marginnote{1.5} nuns heard the complaints of those people, and the nuns of few desires complained and criticized them, “How can the nuns from the group of six spin yarn?” … “Is it true, monks, that those nuns do this?” 

“It’s\marginnote{1.9} true, Sir.” 

The\marginnote{1.10} Buddha rebuked them … “How can the nuns from the group of six do this? This will affect people’s confidence …” … “And, monks, the nuns should recite this training rule like this: 

\subsection*{Final ruling }

\scrule{‘If a nun spins yarn, she commits an offense entailing confession.’” }

\subsection*{Definitions }

\begin{description}%
\item[A: ] whoever … %
\item[Nun: ] … The nun who has been given the full ordination in unanimity by both Sanghas through a legal procedure consisting of one motion and three announcements that is irreversible and fit to stand—this sort of nun is meant in this case. %
\item[Yarn: ] there are six kinds of yarn: linen, cotton, silk, wool, sunn hemp, and hemp.\footnote{For further discussion of \textit{\textsanskrit{kappāsika}}, “cotton”, and \textit{\textsanskrit{bhaṅga}}, “hemp”, see Appendix of Plants, in volume 2 of this series. } %
\item[Spins: ] if she spins it herself, then for the effort there is an act of wrong conduct. For every pull, she commits an offense entailing confession.\footnote{Sp 2.988: \textit{Ujjavujjaveti \textsanskrit{yattakaṁ} hatthena \textsanskrit{añchitaṁ} hoti, \textsanskrit{tasmiṁ} takkamhi \textsanskrit{veṭhite} \textsanskrit{ekā} \textsanskrit{āpatti}}, “\textit{Ujjavujjava}: however much is pulled by hand, there is one offense when the spindle is turned.” } %
\end{description}

\subsection*{Non-offenses }

There\marginnote{2.2.1} is no offense: if she spins yarn that has already been spun;  if she is insane;  if she is the first offender. 

\scendsutta{The third training rule is finished. }

%
\section*{{\suttatitleacronym Bi Pc 44}{\suttatitletranslation The training rule on providing services for householders }{\suttatitleroot Gihiveyyāvacca}}
\addcontentsline{toc}{section}{\tocacronym{Bi Pc 44} \toctranslation{The training rule on providing services for householders } \tocroot{Gihiveyyāvacca}}
\markboth{The training rule on providing services for householders }{Gihiveyyāvacca}
\extramarks{Bi Pc 44}{Bi Pc 44}

\subsection*{Origin story }

At\marginnote{1.1} one time when the Buddha was staying at \textsanskrit{Sāvatthī} in \textsanskrit{Anāthapiṇḍika}’s Monastery, the nuns were providing services for householders. The nuns of few desires complained and criticized them, “How can nuns provide services for householders?” … “Is it true, monks, that nuns do this?” 

“It’s\marginnote{1.6} true, Sir.” 

The\marginnote{1.7} Buddha rebuked them … “How can nuns do this? This will affect people’s confidence …” … “And, monks, the nuns should recite this training rule like this: 

\subsection*{Final ruling }

\scrule{‘If a nun provides services for a householder, she commits an offense entailing confession.’” }

\subsection*{Definitions }

\begin{description}%
\item[A: ] whoever … %
\item[Nun: ] … The nun who has been given the full ordination in unanimity by both Sanghas through a legal procedure consisting of one motion and three announcements that is irreversible and fit to stand—this sort of nun is meant in this case. %
\item[Services for a householder: ] if she cooks congee, a meal, or fresh food for a householder, or she washes a wrap garment or a turban for them, she commits an offense entailing confession. %
\end{description}

\subsection*{Non-offenses }

There\marginnote{2.2.1} is no offense: if it is a congee drink;  if it is a meal for the Sangha;  if it is to venerate a shrine;  if she cooks congee, a meal, or fresh food for her own service-provider, or she washes a wrap garment or a turban for them;  if she is insane;  if she is the first offender. 

\scendsutta{The fourth training rule is finished. }

%
\section*{{\suttatitleacronym Bi Pc 45}{\suttatitletranslation The training rule on legal issues }{\suttatitleroot Adhikaraṇa}}
\addcontentsline{toc}{section}{\tocacronym{Bi Pc 45} \toctranslation{The training rule on legal issues } \tocroot{Adhikaraṇa}}
\markboth{The training rule on legal issues }{Adhikaraṇa}
\extramarks{Bi Pc 45}{Bi Pc 45}

\subsection*{Origin story }

At\marginnote{1.1} one time when the Buddha was staying at \textsanskrit{Sāvatthī} in \textsanskrit{Anāthapiṇḍika}’s Monastery, a certain nun went to the nun \textsanskrit{Thullanandā} and said, “Please come, Venerable, and resolve this legal issue.” \textsanskrit{Thullanandā} agreed, but then neither resolved it nor made any effort to do so. 

That\marginnote{1.6} nun told the nuns what had happened. The nuns of few desires complained and criticized her, “How could Venerable \textsanskrit{Thullanandā} agree to resolve a legal issue, but then neither resolve it nor make any effort to do so?” … “Is it true, monks, that the nun \textsanskrit{Thullanandā} acted like this?” 

“It’s\marginnote{1.14} true, Sir.” 

The\marginnote{1.15} Buddha rebuked her … “How could the nun \textsanskrit{Thullanandā} act like this? This will affect people’s confidence …” … “And, monks, the nuns should recite this training rule like this: 

\subsection*{Final ruling }

\scrule{‘If, when a nun is requested by a nun to resolve a legal issue, she agrees, but then neither resolves it nor makes any effort to resolve it, then, if there were no obstacles, she commits an offense entailing confession.’” }

\subsection*{Definitions }

\begin{description}%
\item[A: ] whoever … %
\item[Nun: ] … The nun who has been given the full ordination in unanimity by both Sanghas through a legal procedure consisting of one motion and three announcements that is irreversible and fit to stand—this sort of nun is meant in this case. %
\item[By a nun: ] by another nun. %
\item[A legal issue: ] there are four kinds of legal issues: legal issues arising from disputes, legal issues arising from accusations, legal issues arising from offenses, legal issues arising from business. %
\item[To resolve a legal issue: ] to make a decision on a legal issue. %
\item[Then, if there were no obstacles: ] when there is no obstacle. %
\item[Neither resolves it: ] she does not resolve it herself. %
\item[Nor makes any effort to resolve it: ] she does not ask anyone else. %
\end{description}

If,\marginnote{2.1.17} thinking, “I’ll neither resolve it nor make any effort to resolve it,” then, by the mere fact of abandoning her duty, she commits an offense entailing confession. 

\subsection*{Permutations }

If\marginnote{2.2.1} the other person is fully ordained, and she perceives her as such, and she neither resolves the legal issue nor makes any effort to resolve it, she commits an offense entailing confession. If the other person is fully ordained, but she is unsure of it, and she neither resolves the legal issue nor makes any effort to resolve it, she commits an offense entailing confession. If the other person is fully ordained, but she does not perceive her as such, and she neither resolves the legal issue nor makes any effort to resolve it, she commits an offense entailing confession. 

If\marginnote{2.2.4} the other person is not fully ordained, and she neither resolves the legal issue nor makes any effort to resolve it, she commits an offense of wrong conduct. If the other person is not fully ordained, but she perceives her as such, she commits an offense of wrong conduct. If the other person is not fully ordained, but she is unsure of it, she commits an offense of wrong conduct. If the other person is not fully ordained, and she does not perceive her as such, she commits an offense of wrong conduct. 

\subsection*{Non-offenses }

There\marginnote{2.3.1} is no offense: if there is an obstacle;  if she searches, but is unable to find anyone to settle it;  if she is sick;  if there is an emergency;  if she is insane;  if she is the first offender. 

\scendsutta{The fifth training rule is finished. }

%
\section*{{\suttatitleacronym Bi Pc 46}{\suttatitletranslation The training rule on giving food }{\suttatitleroot Bhojanadāna}}
\addcontentsline{toc}{section}{\tocacronym{Bi Pc 46} \toctranslation{The training rule on giving food } \tocroot{Bhojanadāna}}
\markboth{The training rule on giving food }{Bhojanadāna}
\extramarks{Bi Pc 46}{Bi Pc 46}

\subsection*{Origin story }

At\marginnote{1.1} one time the Buddha was staying at \textsanskrit{Sāvatthī} in the Jeta Grove, \textsanskrit{Anāthapiṇḍika}’s Monastery. At that time the nun \textsanskrit{Thullanandā} was personally giving food to actors, dancers, acrobats, magicians, and musicians, saying, “Please praise me in public.” And they did: “Venerable \textsanskrit{Thullanandā} is a learned reciter; she’s confident and skilled at giving teachings. Give to her; work for her!” 

The\marginnote{1.7} nuns of few desires complained and criticized her, “How could Venerable \textsanskrit{Thullanandā} personally give food to householders?” … “Is it true, monks, that the nun \textsanskrit{Thullanandā} does this?” 

“It’s\marginnote{1.10} true, Sir.” 

The\marginnote{1.11} Buddha rebuked her … “How could the nun \textsanskrit{Thullanandā} do this? This will affect people’s confidence …” … “And, monks, the nuns should recite this training rule like this: 

\subsection*{Final ruling }

\scrule{‘If a nun personally gives fresh or cooked food to a householder or a male wanderer or a female wanderer, she commits an offense entailing confession.’” }

\subsection*{Definitions }

\begin{description}%
\item[A: ] whoever … %
\item[Nun: ] … The nun who has been given the full ordination in unanimity by both Sanghas through a legal procedure consisting of one motion and three announcements that is irreversible and fit to stand—this sort of nun is meant in this case. %
\item[A householder: ] anyone who lives at home.\footnote{\textit{\textsanskrit{Agāraṁ}} is typically rendered as “in a house”. The problem with this is that it is not unallowable for a monastic to live in a building that is the equivalent of a house. What a monastic should not do is own a home and then live there. } %
\item[A male wanderer: ] any male wanderer apart from Buddhist monks and novice monks. %
\item[A female wanderer: ] any female wanderer apart from Buddhist nuns, trainee nuns, and novice nuns. %
\item[Fresh food: ] apart from the five cooked foods and water and tooth cleaners, the rest is called “fresh food”.\footnote{For a discussion of the rendering “fresh food” for \textit{\textsanskrit{khādanīya}}, see Appendix of Technical Terms. } %
\item[Cooked food: ] there are five kinds of cooked food: cooked grain, porridge, flour products, fish, and meat.\footnote{For a discussion of the rendering “flour products” for \textit{sattu}, see Appendix of Technical Terms. } %
\item[Gives: ] if she gives by body or by what is connected to the body or by releasing, she commits an offense entailing confession. If she gives water or tooth cleaners, she commits an offense of wrong conduct. %
\end{description}

\subsection*{Non-offenses }

There\marginnote{2.2.1} is no offense: if she does not give, but has it given;  if she gives by placing it near the person;  if she gives ointments for external use;  if she is insane;  if she is the first offender. 

\scendsutta{The sixth training rule is finished. }

%
\section*{{\suttatitleacronym Bi Pc 47}{\suttatitletranslation The training rule on communal robes }{\suttatitleroot Āvasathacīvara}}
\addcontentsline{toc}{section}{\tocacronym{Bi Pc 47} \toctranslation{The training rule on communal robes } \tocroot{Āvasathacīvara}}
\markboth{The training rule on communal robes }{Āvasathacīvara}
\extramarks{Bi Pc 47}{Bi Pc 47}

\subsection*{Origin story }

At\marginnote{1.1} one time the Buddha was staying at \textsanskrit{Sāvatthī} in the Jeta Grove, \textsanskrit{Anāthapiṇḍika}’s Monastery. At that time the nun \textsanskrit{Thullanandā} did not relinquish the communal robe, but continued using it. Other menstruating nuns did not get to use it. 

The\marginnote{1.4} nuns of few desires complained and criticized her, “How could Venerable \textsanskrit{Thullanandā} not relinquish the communal robe, but continue using it?” … “Is it true, monks, that the nun \textsanskrit{Thullanandā} did this?” 

“It’s\marginnote{1.7} true, Sir.” 

The\marginnote{1.8} Buddha rebuked her … “How could the nun \textsanskrit{Thullanandā} do this? This will affect people’s confidence …” … “And, monks, the nuns should recite this training rule like this: 

\subsection*{Final ruling }

\scrule{‘If a nun does not relinquish a communal robe, but continues to use it, she commits an offense entailing confession.’”\footnote{“A communal robe” renders \textit{\textsanskrit{āvasathacīvara}}, literally, “a lodging robe”. Since “lodging robe” is awkward in English, and because these robes were used in common among the nuns, I prefer the given rendering. } }

\subsection*{Definitions }

\begin{description}%
\item[A: ] whoever … %
\item[Nun: ] … The nun who has been given the full ordination in unanimity by both Sanghas through a legal procedure consisting of one motion and three announcements that is irreversible and fit to stand—this sort of nun is meant in this case. %
\item[A communal robe: ] it is given specifically for the use of menstruating nuns. %
\item[Does not relinquish, but continues to use it: ] if she uses it for two or three days, washes it on the fourth day, and then uses it again without relinquishing it to a nun or a trainee nun or a novice nun, she commits an offense entailing confession. %
\end{description}

\subsection*{Permutations }

If\marginnote{2.2.1} it has not been relinquished, and she perceives that it has not, and she uses it, she commits an offense entailing confession. If it has not been relinquished, but she is unsure of it, and she uses it, she commits an offense entailing confession. If it has not been relinquished, but she perceives that it has, and she uses it, she commits an offense entailing confession. 

If\marginnote{2.2.4} it has been relinquished, but she perceives that it has not, she commits an offense of wrong conduct. If it has been relinquished, but she is unsure of it, she commits an offense of wrong conduct. If it has been relinquished, and she perceives that it has, there is no offense. 

\subsection*{Non-offenses }

There\marginnote{2.3.1} is no offense: if she relinquishes it and then uses it;  if she uses it again at the next turn;  if there are no other menstruating nuns;  if her robe has been stolen;  if her robe has been lost;  if there is an emergency;  if she is insane;  if she is the first offender. 

\scendsutta{The seventh training rule is finished. }

%
\section*{{\suttatitleacronym Bi Pc 48}{\suttatitletranslation The training rule on lodgings }{\suttatitleroot Āvasathavihāra}}
\addcontentsline{toc}{section}{\tocacronym{Bi Pc 48} \toctranslation{The training rule on lodgings } \tocroot{Āvasathavihāra}}
\markboth{The training rule on lodgings }{Āvasathavihāra}
\extramarks{Bi Pc 48}{Bi Pc 48}

\subsection*{Origin story }

At\marginnote{1.1} one time when the Buddha was staying at \textsanskrit{Sāvatthī} in \textsanskrit{Anāthapiṇḍika}’s Monastery, the nun \textsanskrit{Thullanandā} went wandering without first relinquishing her lodging. Soon afterwards her lodging caught fire. Some nuns said, “Come, Venerables, let’s remove her things.” But others replied, “No, let’s not. She’ll just make us responsible for anything that gets lost.” 

When\marginnote{1.9} \textsanskrit{Thullanandā} returned to her lodging, she asked the nuns, “Venerables, I hope you removed my things?” 

“No,\marginnote{1.11} we didn’t.” 

\textsanskrit{Thullanandā}\marginnote{1.12} complained and criticized them, “How could the nuns not remove the contents when a lodging is burning?” 

But\marginnote{1.14} the nuns of few desires complained and criticized her, “How could Venerable \textsanskrit{Thullanandā} go wandering without relinquishing her lodging?” … “Is it true, monks, that the nun \textsanskrit{Thullanandā} did this?” 

“It’s\marginnote{1.17} true, Sir.” 

The\marginnote{1.18} Buddha rebuked her … “How could the nun \textsanskrit{Thullanandā} do this? This will affect people’s confidence …” … “And, monks, the nuns should recite this training rule like this: 

\subsection*{Final ruling }

\scrule{‘If a nun goes wandering without relinquishing her lodging, she commits an offense entailing confession.’” }

\subsection*{Definitions }

\begin{description}%
\item[A: ] whoever … %
\item[Nun: ] … The nun who has been given the full ordination in unanimity by both Sanghas through a legal procedure consisting of one motion and three announcements that is irreversible and fit to stand—this sort of nun is meant in this case. %
\item[Lodging: ] one with a door is what is meant. %
\item[Goes wandering without relinquishing: ] if she crosses the boundary of an enclosed lodging without first relinquishing it to a nun, a trainee nun, or a novice nun, she commits an offense entailing confession. If she goes beyond the vicinity of an unenclosed lodging, she commits an offense entailing confession. %
\end{description}

\subsection*{Permutations }

If\marginnote{2.2.1} it has not been relinquished, and she perceives that it has not, and she goes, she commits an offense entailing confession. If it has not been relinquished, but she is unsure of it, and she goes, she commits an offense entailing confession. If it has not been relinquished, but she perceives that it has, and she goes, she commits an offense entailing confession. 

If\marginnote{2.2.4} she does not relinquish a lodging without a door, and then goes, she commits an offense of wrong conduct. If it has been relinquished, but she perceives that it has not, she commits an offense of wrong conduct. If it has been relinquished, but she is unsure of it, she commits an offense of wrong conduct. If it has been relinquished, and she perceives that it has, there is no offense. 

\subsection*{Non-offenses }

There\marginnote{2.3.1} is no offense: if she relinquishes it and then goes;  if there is an obstacle;  if she searches, but is unable to find anyone to relinquish it to;  if she is sick;  if there is an emergency;  if she is insane;  if she is the first offender. 

\scendsutta{The eighth training rule is finished. }

%
\section*{{\suttatitleacronym Bi Pc 49}{\suttatitletranslation The training rule on studying worldly subjects }{\suttatitleroot Tiracchānavijjāpariyāpuṇana}}
\addcontentsline{toc}{section}{\tocacronym{Bi Pc 49} \toctranslation{The training rule on studying worldly subjects } \tocroot{Tiracchānavijjāpariyāpuṇana}}
\markboth{The training rule on studying worldly subjects }{Tiracchānavijjāpariyāpuṇana}
\extramarks{Bi Pc 49}{Bi Pc 49}

\subsection*{Origin story }

At\marginnote{1.1} one time the Buddha was staying at \textsanskrit{Sāvatthī} in the Jeta Grove, \textsanskrit{Anāthapiṇḍika}’s Monastery. At that time the nuns from the group of six were studying worldly subjects. People complained and criticized them, “How can the nuns study worldly subjects? They’re just like householders who indulge in worldly pleasures!” 

The\marginnote{1.5} nuns heard the complaints of those people, and the nuns of few desires complained and criticized them, “How can the nuns from the group of six do this?” … “Is it true, monks, that those nuns do this?” 

“It’s\marginnote{1.9} true, Sir.” 

The\marginnote{1.10} Buddha rebuked them … “How can the nuns from the group of six do this? This will affect people’s confidence …” … “And, monks, the nuns should recite this training rule like this: 

\subsection*{Final ruling }

\scrule{‘If a nun studies worldly subjects, she commits an offense entailing confession.’” }

\subsection*{Definitions }

\begin{description}%
\item[A: ] whoever … %
\item[Nun: ] … The nun who has been given the full ordination in unanimity by both Sanghas through a legal procedure consisting of one motion and three announcements that is irreversible and fit to stand—this sort of nun is meant in this case. %
\item[Worldly subjects: ] whatever is external to the Buddha’s Teaching, not connected with the goal. %
\item[Learns: ] if she learns by the line, then for every line she commits an offense entailing confession. If she learns by the syllable, then for every syllable she commits an offense entailing confession. %
\end{description}

\subsection*{Non-offenses }

There\marginnote{2.2.1} is no offense: if she learns writing;  if she learns protective verses;\footnote{\textit{\textsanskrit{Dhāraṇa}} normally means “remembering”, but in the present context this does not fit. Here it is probably used in the sense of \textit{\textsanskrit{dhāraṇī}}, a verse, charm, or prayer used for protection, see SED. In this sense it is a near synonym for \textit{paritta}. }  if she learns verses for the purpose of protection;  if she is insane;  if she is the first offender. 

\scendsutta{The ninth training rule is finished. }

%
\section*{{\suttatitleacronym Bi Pc 50}{\suttatitletranslation The training rule on teaching worldly subjects }{\suttatitleroot Tiracchānavijjāvācana}}
\addcontentsline{toc}{section}{\tocacronym{Bi Pc 50} \toctranslation{The training rule on teaching worldly subjects } \tocroot{Tiracchānavijjāvācana}}
\markboth{The training rule on teaching worldly subjects }{Tiracchānavijjāvācana}
\extramarks{Bi Pc 50}{Bi Pc 50}

\subsection*{Origin story }

At\marginnote{1.1} one time the Buddha was staying at \textsanskrit{Sāvatthī} in the Jeta Grove, \textsanskrit{Anāthapiṇḍika}’s Monastery. At that time the nuns from the group of six were teaching worldly subjects. People complained and criticized them, “How can the nuns teach worldly subjects? They’re just like householders who indulge in worldly pleasures!” 

The\marginnote{1.5} nuns heard the complaints of those people, and the nuns of few desires complained and criticized them, “How can the nuns from the group of six do this?” … “Is it true, monks, that those nuns do this?” 

“It’s\marginnote{1.9} true, Sir.” 

The\marginnote{1.10} Buddha rebuked them … “How can the nuns from the group of six do this? This will affect people’s confidence …” … “And, monks, the nuns should recite this training rule like this: 

\subsection*{Final ruling }

\scrule{‘If a nun teaches worldly subjects, she commits an offense entailing confession.’” }

\subsection*{Definitions }

\begin{description}%
\item[A: ] whoever … %
\item[Nun: ] … The nun who has been given the full ordination in unanimity by both Sanghas through a legal procedure consisting of one motion and three announcements that is irreversible and fit to stand—this sort of nun is meant in this case. %
\item[Worldly subjects: ] whatever is external to the Buddha’s Teaching, not connected with the goal. %
\item[Teaches: ] if she teaches by the line, then for every line she commits an offense entailing confession. If she teaches by the syllable, then for every syllable she commits an offense entailing confession. %
\end{description}

\subsection*{Non-offenses }

There\marginnote{2.10.1} is no offense: if she teaches writing;  if she teaches protective verses;\footnote{\textit{\textsanskrit{Dhāraṇa}} normally means “remembering”, but in the present context this does not fit. Here it is probably used in the sense of \textit{\textsanskrit{dhāraṇī}}, a verse, charm, or prayer used for protection, see SED. In this sense it is a near synonym for \textit{paritta}. }  if she teaches verses for the purpose of protection;  if she is insane;  if she is the first offender. 

\scendsutta{The tenth training rule is finished. }

\scendvagga{The fifth subchapter on pleasure houses is finished. }

%
\section*{{\suttatitleacronym Bi Pc 51}{\suttatitletranslation The training rule on entering monasteries }{\suttatitleroot Ārāmapavisana}}
\addcontentsline{toc}{section}{\tocacronym{Bi Pc 51} \toctranslation{The training rule on entering monasteries } \tocroot{Ārāmapavisana}}
\markboth{The training rule on entering monasteries }{Ārāmapavisana}
\extramarks{Bi Pc 51}{Bi Pc 51}

\subsection*{Origin story }

\subsubsection*{First sub-story }

At\marginnote{1.1} one time when the Buddha was staying at \textsanskrit{Sāvatthī} in \textsanskrit{Anāthapiṇḍika}’s Monastery, a number of monks dressed only in sarongs were making robes in a certain village monastery. Nuns entered that monastery without asking permission and approached those monks. The monks complained and criticized them, “How can nuns enter a monastery without asking permission?” … “Is it true, monks, that nuns did that?” 

“It’s\marginnote{1.7} true, Sir.” 

The\marginnote{1.8} Buddha rebuked them … “How can nuns do that? This will affect people’s confidence …” … “And, monks, the nuns should recite this training rule like this: 

\subsubsection*{First preliminary ruling }

\scrule{‘If a nun enters a monastery without asking permission, she commits an offense entailing confession.’” }

In\marginnote{1.13} this way the Buddha laid down this training rule for the nuns. 

\subsubsection*{Second sub-story }

Soon\marginnote{2.1} afterwards those monks left that monastery. But even though the nuns had heard that the monks had left, they did not go there. Soon the monks returned. When the nuns heard that this was the case, they asked permission and then entered that monastery. They approached the monks and bowed down to them. The monks then said to them, “Sisters, why didn’t you sweep the monastery, or put out water for drinking or water for washing?” 

“The\marginnote{2.7} Buddha has laid down a training rule that we may not enter a monastery without asking permission. That’s why we didn’t do it.” 

They\marginnote{2.9} told the Buddha. Soon afterwards the Buddha had the Sangha gathered and addressed the monks: “Monks, when a monk is available, a nun should enter a monastery only after asking permission. And so, monks, the nuns should recite this training rule like this: 

\subsubsection*{Second preliminary ruling }

\scrule{‘If a nun, when a monk is available, enters a monastery without asking permission, she commits an offense entailing confession.’” }

In\marginnote{2.13} this way the Buddha laid down this training rule for the nuns. 

\subsubsection*{Third sub-story }

After\marginnote{3.1} leaving that monastery once more, the monks returned yet again. Thinking that the monks were still away, the nuns entered the monastery without asking permission. The nuns became anxious, thinking, “We have entered a monastery without asking permission, even though a monk was available. And the Buddha has laid down a training rule against this. Could it be that we have committed an offense entailing confession?” 

They\marginnote{3.6} told the Buddha. Soon afterwards he gave a teaching and addressed the monks: “And so, monks, the nuns should recite this training rule like this: 

\subsection*{Final ruling }

\scrule{‘If a nun, knowing that there are monks in a monastery, enters it without asking permission, she commits an offense entailing confession.’” }

\subsection*{Definitions }

\begin{description}%
\item[A: ] whoever … %
\item[Nun: ] … The nun who has been given the full ordination in unanimity by both Sanghas through a legal procedure consisting of one motion and three announcements that is irreversible and fit to stand—this sort of nun is meant in this case. %
\item[Knowing: ] she knows it by herself or others have told her or they have told her.\footnote{The last of these three ways of knowing presumably refers to the monks themselves having told the nun. } %
\item[There are monks in a monastery: ] even where monks stay at the foot of a tree. %
\item[Enters it without asking permission: ] if she crosses the boundary of an enclosed monastery without asking permission of a monk or a novice monk or a monastery worker, she commits an offense entailing confession. If she enters the vicinity of an unenclosed monastery, she commits an offense entailing confession. %
\end{description}

\subsection*{Permutations }

If\marginnote{4.2.1} there are monks in a monastery, and she perceives that there are, and she enters it without asking permission of an available monk, she commits an offense entailing confession. If there are monks in a monastery, but she is unsure of it, and she enters it without asking permission of an available monk, she commits an offense of wrong conduct. If there are monks in a monastery, but she does not perceive that there are, and she enters it without asking permission of an available monk, there is no offense. 

If\marginnote{4.2.4} there are no monks in a monastery, but she perceives that there are, she commits an offense of wrong conduct. If there are no monks in a monastery, but she is unsure of it, she commits an offense of wrong conduct. If there are no monks in a monastery, and she does not perceive that there are, there is no offense. 

\subsection*{Non-offenses }

There\marginnote{4.3.1} is no offense: if she enters after asking permission of an available monk;  if, when there is no available monk, she enters without asking permission;  if she goes while looking at the head of a nun in front of her;\footnote{Sp 2.1027: \textit{\textsanskrit{Sīsānulokikāti} \textsanskrit{paṭhamaṁ} \textsanskrit{pavisantīnaṁ} \textsanskrit{bhikkhunīnaṁ} \textsanskrit{sīsaṁ} \textsanskrit{anulokentī} pavisati, \textsanskrit{anāpatti}}, “\textit{\textsanskrit{Sīsānulokika}} means there is no offense if she enters while looking at the head of the nuns entering first.” The point, perhaps, is that she may assume that the nun in front has asked for permission. }  if she is going to where the nuns have gathered;  if a path goes through the monastery;  if she is sick;  if there is an emergency;  if she is insane;  if she is the first offender. 

\scendsutta{The first training rule is finished. }

%
\section*{{\suttatitleacronym Bi Pc 52}{\suttatitletranslation The training rule on abusing monks }{\suttatitleroot Bhikkhuakkosana}}
\addcontentsline{toc}{section}{\tocacronym{Bi Pc 52} \toctranslation{The training rule on abusing monks } \tocroot{Bhikkhuakkosana}}
\markboth{The training rule on abusing monks }{Bhikkhuakkosana}
\extramarks{Bi Pc 52}{Bi Pc 52}

\subsection*{Origin story }

At\marginnote{1.1} one time when the Buddha was staying  in the hall with the peaked roof in the Great Wood near \textsanskrit{Vesālī}, Venerable \textsanskrit{Upāli}’s preceptor, Venerable Kappita, was staying in a charnel ground. At that time the leader of the nuns from the group of six had just died. They took her to that charnel ground, cremated her near Venerable Kappita’s dwelling, and made a stupa. They then went there to cry. 

Venerable\marginnote{1.5} Kappita was disturbed by the noise, and so he demolished the stupa and scattered the rubble. The nuns from the group of six thought, “Kappita has demolished our Venerable’s stupa; let’s kill him,” and they laid a plan. A certain nun told Venerable \textsanskrit{Upāli} what was happening, and he in turn told Venerable Kappita. Venerable Kappita then left his dwelling and went into hiding. Soon afterwards the nuns from the group of six went to his dwelling and crushed it with rocks and lumps of earth. When they thought he was dead, they left. 

The\marginnote{1.13} following morning Venerable Kappita robed up, took his bowl and robe, and entered \textsanskrit{Vesālī} for almsfood. When the nuns from the group of six saw him, they said, “Kappita is alive! Who spoiled our plan?” 

When\marginnote{1.17} they heard it was Venerable \textsanskrit{Upāli}, they abused him, “How could this barber, this dirt remover of inferior caste, spoil our plan?” 

The\marginnote{1.21} nuns of few desires complained and criticized them, “How could the nuns from the group of six abuse Venerable \textsanskrit{Upāli}?” … “Is it true, monks, that those nuns did this?” 

“It’s\marginnote{1.24} true, Sir.” 

The\marginnote{1.25} Buddha rebuked them … “How could the nuns from the group of six do this? This will affect people’s confidence …” … “And, monks, the nuns should recite this training rule like this: 

\subsection*{Final ruling }

\scrule{‘If a nun abuses or reviles a monk, she commits an offense entailing confession.’” }

\subsection*{Definitions }

\begin{description}%
\item[A: ] whoever … %
\item[Nun: ] … The nun who has been given the full ordination in unanimity by both Sanghas through a legal procedure consisting of one motion and three announcements that is irreversible and fit to stand—this sort of nun is meant in this case. %
\item[A monk: ] one who is fully ordained. %
\item[Abuses: ] if she abuses with the ten kinds of abuse or with any one of them, she commits an offense entailing confession.\footnote{See \href{https://suttacentral.net/pli-tv-bu-vb-pc2/en/brahmali\#1.2.33.1}{Bu Pc 2:1.2.33.1}. } %
\item[Reviles: ] if she induces fear, she commits an offense entailing confession. %
\end{description}

\subsection*{Permutations }

If\marginnote{2.2.1} he is fully ordained, and she perceives him as such, and she abuses or reviles him, she commits an offense entailing confession. If he is fully ordained, but she is unsure of it, and she abuses or reviles him, she commits an offense entailing confession. If he is fully ordained, but she does not perceive him as such, and she abuses or reviles him, she commits an offense entailing confession. 

If\marginnote{2.2.4} she abuses or reviles someone who is not fully ordained, she commits an offense of wrong conduct. If he is not fully ordained, but she perceives him as such, she commits an offense of wrong conduct. If he is not fully ordained, but she is unsure of it, she commits an offense of wrong conduct. If he is not fully ordained, and she does not perceive him as such, she commits an offense of wrong conduct. 

\subsection*{Non-offenses }

There\marginnote{2.3.1} is no offense: if she is aiming at something beneficial;  if she is aiming at giving a teaching;  if she is aiming at giving an instruction;  if she is insane;  if she is the first offender. 

\scendsutta{The second training rule is finished. }

%
\section*{{\suttatitleacronym Bi Pc 53}{\suttatitletranslation The training rule on reviling the community }{\suttatitleroot Gaṇaparibhāsana}}
\addcontentsline{toc}{section}{\tocacronym{Bi Pc 53} \toctranslation{The training rule on reviling the community } \tocroot{Gaṇaparibhāsana}}
\markboth{The training rule on reviling the community }{Gaṇaparibhāsana}
\extramarks{Bi Pc 53}{Bi Pc 53}

\subsection*{Origin story }

At\marginnote{1.1} one time the Buddha was staying at \textsanskrit{Sāvatthī} in the Jeta Grove, \textsanskrit{Anāthapiṇḍika}’s Monastery. At that time the nun \textsanskrit{Caṇḍakāḷī} was quarrelsome and argumentative, and she created legal issues in the Sangha. But when a legal procedure was being done against her, the nun \textsanskrit{Thullanandā} objected. 

Soon\marginnote{1.4} afterwards when \textsanskrit{Thullanandā} went to a village on some business, the Sangha of nuns took the opportunity to eject \textsanskrit{Caṇḍakāḷī} for not recognizing an offense. When \textsanskrit{Thullanandā} had finished her business, she returned to \textsanskrit{Sāvatthī}. As she was approaching, \textsanskrit{Caṇḍakāḷī} neither prepared a seat for her, nor set out a foot stool, a foot scraper, or water for washing the feet; and she did not go out to meet her to receive her bowl and robe or to ask whether she wanted water to drink. \textsanskrit{Thullanandā} asked her why she was acting like this. She replied, “That’s how it is, Venerable, when one doesn’t have a protector.” 

“But\marginnote{1.14} how is it, Venerable, that you don’t have a protector?” 

“When\marginnote{1.15} these nuns knew that no one would speak up for me because I am not esteemed by them and I didn’t have a protector, they ejected me for not recognizing an offense.” 

“They\marginnote{1.17} are incompetent fools! They don’t understand legal procedures, nor what makes them fail or succeed.” And being furious, she reviled the community.\footnote{In this case \textit{\textsanskrit{gaṇa}} seems to stand for the Sangha that did the legal procedure against \textsanskrit{Caṇḍakāḷī}. See also the definition of \textit{\textsanskrit{gaṇa}} below. } 

The\marginnote{1.19} nuns of few desires complained and criticized her, “How could Venerable \textsanskrit{Thullanandā} revile the community because she is furious?” … “Is it true, monks, that the nun \textsanskrit{Thullanandā} did this?” 

“It’s\marginnote{1.22} true, Sir.” 

The\marginnote{1.23} Buddha rebuked her … “How could the nun \textsanskrit{Thullanandā} do this? This will affect people’s confidence …” … “And, monks, the nuns should recite this training rule like this: 

\subsection*{Final ruling }

\scrule{‘If a furious nun reviles the community, she commits an offense entailing confession.’” }

\subsection*{Definitions }

\begin{description}%
\item[A: ] whoever … %
\item[Nun: ] … The nun who has been given the full ordination in unanimity by both Sanghas through a legal procedure consisting of one motion and three announcements that is irreversible and fit to stand—this sort of nun is meant in this case. %
\item[Furious: ] angry is what is meant. %
\item[The community: ] the Sangha of nuns is what is meant. %
\item[Reviles: ] if she reviles them, saying, “They are incompetent fools. They don’t understand legal procedures, nor what makes them fail or succeed,” she commits an offense entailing confession. If she reviles several nuns, a single nun, or someone who is not fully ordained, she commits an offense of wrong conduct.\footnote{For a discussion of the rendering “several” for \textit{sambahula}, see Appendix of Technical Terms. } %
\end{description}

\subsection*{Non-offenses }

There\marginnote{2.12.1} is no offense: if she is aiming at something beneficial;  if she is aiming at giving a teaching;  if she is aiming at giving an instruction;  if she is insane;  if she is the first offender. 

\scendsutta{The third training rule is finished. }

%
\section*{{\suttatitleacronym Bi Pc 54}{\suttatitletranslation The training rule on inviting }{\suttatitleroot Pavārita}}
\addcontentsline{toc}{section}{\tocacronym{Bi Pc 54} \toctranslation{The training rule on inviting } \tocroot{Pavārita}}
\markboth{The training rule on inviting }{Pavārita}
\extramarks{Bi Pc 54}{Bi Pc 54}

\subsection*{Origin story }

At\marginnote{1.1} one time when the Buddha was staying at \textsanskrit{Sāvatthī} in \textsanskrit{Anāthapiṇḍika}’s Monastery, a certain brahmin had invited the nuns for a meal. When the nuns had finished and refused an invitation to eat more, they went to their respective families, where some ate and others got almsfood and left. 

Soon\marginnote{1.4} afterwards that brahmin said this to his neighbors, “I have satisfied the nuns. Come, and I’ll satisfy you, too.” 

“How\marginnote{1.6} could you satisfy us? Those nuns who were invited by you came to our houses, where some ate and others got almsfood and left.” 

That\marginnote{1.9} brahmin complained and criticized those nuns, “How could they eat in our house and afterwards eat elsewhere? Am I not able to give them as much as they want?” 

The\marginnote{1.11} nuns heard the complaints of that brahmin, and the nuns of few desires complained and criticized them, “How could nuns finish their meal, refuse an invitation to eat more, and then eat elsewhere?” … “Is it true, monks, that nuns did this?” 

“It’s\marginnote{1.15} true, Sir.” 

The\marginnote{1.16} Buddha rebuked them … “How could nuns act in this way? This will affect people’s confidence …” … “And, monks, the nuns should recite this training rule like this: 

\subsection*{Final ruling }

\scrule{‘If a nun, who has been invited to a meal, refuses an invitation to eat more, and then eats fresh or cooked food, she commits an offense entailing confession.’”\footnote{“Refuses an offer to eat more” renders \textit{\textsanskrit{pavārita}}. I normally translate both \textit{nimantita} and \textit{\textsanskrit{pavārita}} as “invited”. Yet this becomes awkward on the rare occasions, such as here, when the two words occur together, and so in this instance I instead use the verb “offer” for \textit{\textsanskrit{pavārita}}. Moreover, in the present case, the contextual meaning of \textit{\textsanskrit{pavārita}} is that the nun has expressed her satisfaction after being invited to take more, thus my rendering. See the discussion of this rule in Appendix on Individual \textsanskrit{Bhikkhunī} Rules. For further discussion of \textit{\textsanskrit{pavārita}}/\textit{\textsanskrit{pavāraṇā}}, see Appendix of Technical Terms. } }

\subsection*{Definitions }

\begin{description}%
\item[A: ] whoever … %
\item[Nun: ] … The nun who has been given the full ordination in unanimity by both Sanghas through a legal procedure consisting of one motion and three announcements that is irreversible and fit to stand—this sort of nun is meant in this case. %
\item[Refuses an offer to eat more: ] refuses an offer to eat any of the five cooked foods. %
\item[Refuses an invitation to eat more: ] eating is seen, cooked food is seen, it is brought forward within arm’s reach, a refusal is seen.\footnote{As with \href{https://suttacentral.net/pli-tv-bu-vb-pc35/en/brahmali\#3.1.8}{Bu Pc 35:3.1.8} and \href{https://suttacentral.net/pli-tv-bu-vb-pc36/en/brahmali\#2.1.10}{Bu Pc 36:2.1.10}, the punctuation of the Pali is wrong. An additional comma is required between \textit{\textsanskrit{ṭhito}} and \textit{abhiharati}. “They” refers to the donor, whether male or female. } %
\item[Fresh food: ] apart from the five cooked foods, congee, the post-midday tonics, the seven-day tonics, and the lifetime tonics, the rest is called “fresh food”. %
\item[Cooked food: ] there are five kinds of cooked food: cooked grain, porridge, flour products, fish, and meat.\footnote{“Cooked food” renders \textit{\textsanskrit{bhojanīya}}. See discussion of this word in Appendix of Technical Terms. } If she receives it with the intention of eating it, she commits an offense of wrong conduct. For every mouthful, she commits an offense entailing confession. %
\end{description}

\subsection*{Permutations }

If\marginnote{2.2.1} she has refused an offer, and she perceives that she has, and she eats fresh or cooked food, she commits an offense entailing confession.\footnote{The non-offense clause below states that there is no offense if one has been invited but has not refused an offer to eat more. In other words, there is only an offense if both of these factors are fulfilled. I therefore take invited here to be a shorthand for invited plus refused an offer to eat more. } If she has refused an offer, but she is unsure of it, and she eats fresh or cooked food, she commits an offense entailing confession. If she has refused an offer, but she does not perceive that she has, and she eats fresh or cooked food, she commits an offense entailing confession. 

If\marginnote{2.2.4} she receives post-midday tonics, seven-day tonics, or lifetime tonics for the purpose of food, she commits an offense of wrong conduct. For every mouthful, she commits an offense of wrong conduct. …\footnote{The Pali has ellipses points, which seems to be a mistake. } 

\subsection*{Non-offenses }

There\marginnote{2.2.6.1} is no offense: if she has refused an offer to eat more, but not an invitation;  if she drinks congee;  if she eats more after getting permission from the person who invited her;  if, when there is a reason, she uses post-midday tonics, seven-day tonics, or lifetime tonics;  if she is insane;  if she is the first offender. 

\scendsutta{The fourth training rule is finished. }

%
\section*{{\suttatitleacronym Bi Pc 55}{\suttatitletranslation The training rule on keeping families for oneself }{\suttatitleroot Kulamaccharinī}}
\addcontentsline{toc}{section}{\tocacronym{Bi Pc 55} \toctranslation{The training rule on keeping families for oneself } \tocroot{Kulamaccharinī}}
\markboth{The training rule on keeping families for oneself }{Kulamaccharinī}
\extramarks{Bi Pc 55}{Bi Pc 55}

\subsection*{Origin story }

At\marginnote{1.1} one time when the Buddha was staying at \textsanskrit{Sāvatthī} in \textsanskrit{Anāthapiṇḍika}’s Monastery, a certain nun there was walking for almsfood along a street. She then went to a certain family where she sat down on the prepared seat. The people there gave her a meal and said, “Venerable, other nuns may come too.” Then that nun thought, “What can I do so that other nuns don’t come?” And she went up to other nuns and said, “Venerables, in such-and-such a place there are malicious dogs, a fierce ox, and it’s muddy. Don’t go there.” 

Another\marginnote{1.9} nun, too, went to that family while walking for almsfood along that street. She sat down on the prepared seat and the people there gave her a meal. They then asked her, “Venerable, why don’t the nuns come here?” 

And\marginnote{1.12} she told them what had happened. 

People\marginnote{1.13} complained and criticized her, “How can a nun keep a family to herself?” … “Is it true, monks, that a nun acted like this?” 

“It’s\marginnote{1.16} true, Sir.” 

The\marginnote{1.17} Buddha rebuked her … “How could a nun act like this? This will affect people’s confidence …” … “And, monks, the nuns should recite this training rule like this: 

\subsection*{Final ruling }

\scrule{‘If a nun keeps a family to herself, she commits an offense entailing confession.’” }

\subsection*{Definitions }

\begin{description}%
\item[A: ] whoever … %
\item[Nun: ] … The nun who has been given the full ordination in unanimity by both Sanghas through a legal procedure consisting of one motion and three announcements that is irreversible and fit to stand—this sort of nun is meant in this case. %
\item[A family: ] there are four kinds of families: the aristocratic family, the brahmin family, the merchant family, the worker family. %
\item[Keeps for herself: ] if she thinks, “What can I do so that other nuns don’t come?” and she then speaks disparagingly about that family to the nuns, she commits an offense entailing confession. If she speaks disparagingly about that family to the nuns, she commits an offense entailing confession. %
\end{description}

\subsection*{Non-offenses }

There\marginnote{2.2.1} is no offense: if she is not keeping a family to herself, but describes the drawbacks as they are;  if she is insane;  if she is the first offender. 

\scendsutta{The fifth training rule is finished. }

%
\section*{{\suttatitleacronym Bi Pc 56}{\suttatitletranslation The training rule on monasteries without monks }{\suttatitleroot Abhikkhukāvāsa}}
\addcontentsline{toc}{section}{\tocacronym{Bi Pc 56} \toctranslation{The training rule on monasteries without monks } \tocroot{Abhikkhukāvāsa}}
\markboth{The training rule on monasteries without monks }{Abhikkhukāvāsa}
\extramarks{Bi Pc 56}{Bi Pc 56}

\subsection*{Origin story }

At\marginnote{1.1} one time the Buddha was staying at \textsanskrit{Sāvatthī} in the Jeta Grove, \textsanskrit{Anāthapiṇḍika}’s Monastery. At that time a number of nuns who had completed the rainy-season residence in a village monastery went to \textsanskrit{Sāvatthī}. The nuns there asked them, “Venerables, where did you spend the rains residence? We hope the instruction was effective?” 

“There\marginnote{1.6} were no monks there. So how could the instruction be effective?” 

The\marginnote{1.8} nuns of few desires complained and criticized them, “How could nuns spend the rains residence in a monastery without monks?” … “Is it true, monks, that nuns did this?” 

“It’s\marginnote{1.11} true, Sir.” 

The\marginnote{1.12} Buddha rebuked them … “How could nuns do this? This will affect people’s confidence …” … “And, monks, the nuns should recite this training rule like this: 

\subsection*{Final ruling }

\scrule{‘If a nun spends the rainy-season residence in a monastery without monks, she commits an offense entailing confession.’” }

\subsection*{Definitions }

\begin{description}%
\item[A: ] whoever … %
\item[Nun: ] … The nun who has been given the full ordination in unanimity by both Sanghas through a legal procedure consisting of one motion and three announcements that is irreversible and fit to stand—this sort of nun is meant in this case. %
\item[A monastery without monks: ] she is not able to go to the instruction or to a formal meeting of the community.\footnote{According to \href{https://suttacentral.net/pli-tv-bu-vb-pc69/en/brahmali\#2.1.21}{Bu Pc 69:2.1.21} “community”, \textit{\textsanskrit{saṁvāsa}}, refers to formal meetings of the community. The present rule refers to formal meetings where both \textit{bhikkhus} and \textit{\textsanskrit{bhikkhunīs}} are present. } If she thinks, “I’ll stay here for the rainy-season residence,” and she prepares a dwelling, sets out water for drinking and water for washing, and sweeps the yard, she commits an offense of wrong conduct.\footnote{“Yard” renders \textit{\textsanskrit{pariveṇa}}. For a discussion of this word, see Appendix of Technical Terms. } At dawn, she commits an offense entailing confession. %
\end{description}

\subsection*{Non-offenses }

There\marginnote{2.2.1} is no offense: if the monks who entered the rainy-season residence there depart or disrobe or die or join another group;\footnote{Sp 2.693 defines \textit{\textsanskrit{pakkhasaṅkantā}} as joining another religion: \textit{\textsanskrit{Pakkhasaṅkantā} \textsanskrit{vāti} \textsanskrit{titthāyatanaṁ} \textsanskrit{saṅkantā}}, “\textit{\textsanskrit{Pakkhasaṅkantā} \textsanskrit{vā}} means one who has joined the ascetics of another religion.” Yet the idea of \textit{pakkha} also refers to groups or factions within the Sangha, for instance, when the Sangha is split into different communities (\textit{\textsanskrit{nānāsaṁvāsa}}) that no longer perform legal procedures together. As such, it is a term for a separate sect of Buddhism. For a discussion of the word \textit{\textsanskrit{vibbhantā}}, “disrobed”, see Appendix of Technical Terms. }  if there is an emergency;  if she is insane;  if she is the first offender. 

\scendsutta{The sixth training rule is finished. }

%
\section*{{\suttatitleacronym Bi Pc 57}{\suttatitletranslation The training rule on not inviting }{\suttatitleroot Apavāraṇā}}
\addcontentsline{toc}{section}{\tocacronym{Bi Pc 57} \toctranslation{The training rule on not inviting } \tocroot{Apavāraṇā}}
\markboth{The training rule on not inviting }{Apavāraṇā}
\extramarks{Bi Pc 57}{Bi Pc 57}

\subsection*{Origin story }

At\marginnote{1.1} one time the Buddha was staying at \textsanskrit{Sāvatthī} in the Jeta Grove, \textsanskrit{Anāthapiṇḍika}’s Monastery. At that time a number of nuns who had completed the rainy-season residence in a village monastery went to \textsanskrit{Sāvatthī}. The nuns there asked them, “Venerables, where did you spend the rains residence? Where did you invite the Sangha of monks for correction?” 

“We\marginnote{1.6} didn’t invite the Sangha of monks for correction.” 

The\marginnote{1.7} nuns of few desires complained and criticized them, “How could nuns who have completed the rains residence not invite the Sangha of monks for correction?” … “Is it true, monks, that nuns didn’t do this?” 

“It’s\marginnote{1.10} true, Sir.” 

The\marginnote{1.11} Buddha rebuked them … “How could nuns not do this? This will affect people’s confidence …” … “And, monks, the nuns should recite this training rule like this: 

\subsection*{Final ruling }

\scrule{‘If a nun who has completed the rainy-season residence does not invite correction from both Sanghas in regard to three things—what has been seen, heard, or suspected—she commits an offense entailing confession.’” }

\subsection*{Definitions }

\begin{description}%
\item[A: ] whoever … %
\item[Nun: ] … The nun who has been given the full ordination in unanimity by both Sanghas through a legal procedure consisting of one motion and three announcements that is irreversible and fit to stand—this sort of nun is meant in this case. %
\item[Who has completed the rainy-season residence: ] who has completed the first three or the last three months of the rainy-season residence. If she thinks, “I won’t invite correction from both Sanghas in regard to three things—what has been seen, heard, or suspected,” then by the mere fact of abandoning her duty, she commits an offense entailing confession. %
\end{description}

\subsection*{Non-offenses }

There\marginnote{2.2.1} is no offense: if there is an obstacle;  if she searches, but does not find anyone to invite for correction;  if she is sick;  if there is an emergency;  if she is insane;  if she is the first offender. 

\scendsutta{The seventh training rule is finished. }

%
\section*{{\suttatitleacronym Bi Pc 58}{\suttatitletranslation The training rule on the instruction }{\suttatitleroot Ovāda}}
\addcontentsline{toc}{section}{\tocacronym{Bi Pc 58} \toctranslation{The training rule on the instruction } \tocroot{Ovāda}}
\markboth{The training rule on the instruction }{Ovāda}
\extramarks{Bi Pc 58}{Bi Pc 58}

\subsection*{Origin story }

At\marginnote{1.1} one time when the Buddha was staying in the Sakyan country in the Banyan Tree Monastery at Kapilavatthu, the monks from the group of six went to the nuns’ dwelling place to instruct the nuns from the group of six. Soon afterwards other nuns said to those nuns, “Come, Venerables, let’s go to the instruction.” 

“There’s\marginnote{1.5} no need. The monks from the group of six came and instructed us right here.” 

The\marginnote{1.6} nuns of few desires complained and criticized them, “How could the nuns from the group of six not go to the instruction?” … “Is it true, monks, that those nuns didn’t do this?” 

“It’s\marginnote{1.9} true, Sir.” 

The\marginnote{1.10} Buddha rebuked them … “How could the nuns from the group of six not do this? This will affect people’s confidence …” … “And, monks, the nuns should recite this training rule like this: 

\subsection*{Final ruling }

\scrule{‘If a nun does not go to the instruction or to a formal meeting of the community, she commits an offense entailing confession.’” }

\subsection*{Definitions }

\begin{description}%
\item[A: ] whoever … %
\item[Nun: ] … The nun who has been given the full ordination in unanimity by both Sanghas through a legal procedure consisting of one motion and three announcements that is irreversible and fit to stand—this sort of nun is meant in this case. %
\item[The instruction: ] the eight important principles. %
\item[The community: ] joint legal procedures, a joint recitation, the same training.\footnote{According to \href{https://suttacentral.net/pli-tv-bu-vb-pc69/en/brahmali\#2.1.21}{Bu Pc 69:2.1.21}, “community”, \textit{\textsanskrit{saṁvāsa}}, refers to formal meetings of the community. The present rule would seem to refer to formal meetings where both \textit{bhikkhus} and \textit{\textsanskrit{bhikkhunīs}} are present. } If she thinks, “I won’t go to the instruction or to formal meetings of the community,” then by the mere fact of abandoning her duty, she commits an offense entailing confession. %
\end{description}

\subsection*{Non-offenses }

There\marginnote{2.2.1} is no offense: if there is an obstacle;  if she searches for a companion nun, but does not find one;  if she is sick;  if there is an emergency;  if she is insane;  if she is the first offender. 

\scendsutta{The eighth training rule is finished. }

%
\section*{{\suttatitleacronym Bi Pc 59}{\suttatitletranslation The training rule on going to the instruction }{\suttatitleroot Ovādūpasaṅkamana}}
\addcontentsline{toc}{section}{\tocacronym{Bi Pc 59} \toctranslation{The training rule on going to the instruction } \tocroot{Ovādūpasaṅkamana}}
\markboth{The training rule on going to the instruction }{Ovādūpasaṅkamana}
\extramarks{Bi Pc 59}{Bi Pc 59}

\subsection*{Origin story }

At\marginnote{1.1} one time the Buddha was staying at \textsanskrit{Sāvatthī} in the Jeta Grove, \textsanskrit{Anāthapiṇḍika}’s Monastery. At that time the nuns did not enquire about the observance day nor ask for the instruction. The monks complained and criticized them, “How can the nuns not enquire about the observance day nor ask for the instruction?” … “Is it true, monks, that the nuns don’t do this?” 

“It’s\marginnote{1.6} true, Sir.” 

The\marginnote{1.7} Buddha rebuked them … “How can the nuns not do this? This will affect people’s confidence …” … “And, monks, the nuns should recite this training rule like this: 

\subsection*{Final ruling }

\scrule{‘Every half-month a nun should seek two things from the Sangha of monks: asking it about the observance day and going to it for the instruction. If she lets the half-month pass, she commits an offense entailing confession.’”\footnote{\href{https://suttacentral.net/pli-tv-kd20/en/brahmali\#9.4.18}{Kd 20:9.4.18} makes it clear how this is supposed to happen. A group of two or three nuns should approach the Sangha of monks and ask for the date of the observance day and for when to come for the instruction, if at all. The nuns are essentially finding out who is available for the instruction and when. } }

\subsection*{Definitions }

\begin{description}%
\item[Every half-month: ] every observance day. %
\item[The observance day: ] there are two observance days: the fourteenth and the fifteenth day of the lunar half-month.\footnote{For a discussion of the rendering “observance day (ceremony)” for \textit{uposatha}, see Appendix of Technical Terms. } %
\item[The instruction: ] the eight important principles. If she thinks, “I won’t enquire about the observance day, nor ask for the instruction,” then by the mere fact of abandoning her duty, she commits an offense entailing confession. %
\end{description}

\subsection*{Non-offenses }

There\marginnote{2.8.1} is no offense: if there is an obstacle;  if she searches for a companion nun, but does not find one;  if she is sick;  if there is an emergency;  if she is insane;  if she is the first offender. 

\scendsutta{The ninth training rule is finished. }

%
\section*{{\suttatitleacronym Bi Pc 60}{\suttatitletranslation The training rule on what is growing on the lower part of the body }{\suttatitleroot Pasākhejāta}}
\addcontentsline{toc}{section}{\tocacronym{Bi Pc 60} \toctranslation{The training rule on what is growing on the lower part of the body } \tocroot{Pasākhejāta}}
\markboth{The training rule on what is growing on the lower part of the body }{Pasākhejāta}
\extramarks{Bi Pc 60}{Bi Pc 60}

\subsection*{Origin story }

At\marginnote{1.1} one time when the Buddha was staying at \textsanskrit{Sāvatthī} in \textsanskrit{Anāthapiṇḍika}’s Monastery, a certain nun was alone with a man, when she had him rupture an abscess growing on the lower part of her body. When he tried to rape her, she cried out. The nuns rushed up and asked her why. 

And\marginnote{1.7} she told them what had happened. 

The\marginnote{1.8} nuns of few desires complained and criticized her, “How could a nun be alone with a man and have him rupture an abscess growing on the lower part of her body?” … “Is it true, monks, that a nun did this?” 

“It’s\marginnote{1.11} true, Sir.” 

The\marginnote{1.12} Buddha rebuked her … “How could a nun do this? This will affect people’s confidence …” … “And, monks, the nuns should recite this training rule like this: 

\subsection*{Final ruling }

\scrule{‘If a nun, alone with a man, without getting permission from the Sangha or a group, has an abscess or a wound situated on the lower part of her body ruptured by him, or split open, washed, anointed, bandaged, or unwrapped by him, she commits an offense entailing confession.’” }

\subsection*{Definitions }

\begin{description}%
\item[A: ] whoever … %
\item[Nun: ] … The nun who has been given the full ordination in unanimity by both Sanghas through a legal procedure consisting of one motion and three announcements that is irreversible and fit to stand—this sort of nun is meant in this case. %
\item[The lower part of the body: ] below the navel and above the knees. %
\item[Situated: ] situated there. %
\item[An abscess: ] any kind of abscess.\footnote{For a discussion of the rendering “abscess” for \textit{\textsanskrit{gaṇḍa}}, see Appendix of Technical Terms. } %
\item[A wound: ] any kind of sore. %
\item[Without getting permission: ] without having asked permission. %
\item[The Sangha: ] the Sangha of nuns is what is meant. %
\item[A group: ] several nuns is what is meant. %
\item[A man: ] a human male, not a male spirit, not a male ghost, not a male animal. He understands and is capable of raping. %
\item[With: ] together. %
\item[Alone: ] just the man and the nun. %
\end{description}

If\marginnote{2.1.25} she tells him, “Rupture it,” she commits an offense of wrong conduct. When it has been ruptured, she commits an offense entailing confession. If she tells him, “Split it open,” she commits an offense of wrong conduct. When it has been split open, she commits an offense entailing confession. If she tells him, “Wash it,” she commits an offense of wrong conduct. When it has been washed, she commits an offense entailing confession. If she tells him, “Anoint it,” she commits an offense of wrong conduct. When it has been anointed, she commits an offense entailing confession. If she tells him, “Bandage it,” she commits an offense of wrong conduct. When it has been bandaged, she commits an offense entailing confession. If she tells him, “Unwrap it,” she commits an offense of wrong conduct. When it has been unwrapped, she commits an offense entailing confession. 

\subsection*{Non-offenses }

There\marginnote{2.2.1} is no offense: if she gets permission and then has it ruptured, split open, washed, anointed, bandaged, or unwrapped;  if she has a female companion who understands;  if she is insane;  if she is the first offender. 

\scendsutta{The tenth training rule is finished. }

\scendvagga{The sixth subchapter on monasteries is finished. }

%
\section*{{\suttatitleacronym Bi Pc 61}{\suttatitletranslation The training rule on pregnant women }{\suttatitleroot Gabbhinī}}
\addcontentsline{toc}{section}{\tocacronym{Bi Pc 61} \toctranslation{The training rule on pregnant women } \tocroot{Gabbhinī}}
\markboth{The training rule on pregnant women }{Gabbhinī}
\extramarks{Bi Pc 61}{Bi Pc 61}

\subsection*{Origin story }

At\marginnote{1.1} one time when the Buddha was staying at \textsanskrit{Sāvatthī} in \textsanskrit{Anāthapiṇḍika}’s Monastery, the nuns gave the full admission to a pregnant woman. When she walked for alms, people said, “Give almsfood to the Venerable. She’s carrying a heavy burden.” 

People\marginnote{1.6} complained and criticized them, “How can the nuns give the full admission to a pregnant woman?” 

The\marginnote{1.8} nuns heard the complaints of those people and the nuns of few desires complained and criticized them, “How could nuns do this?” … “Is it true, monks, that nuns did this?” 

“It’s\marginnote{1.12} true, Sir.” 

The\marginnote{1.13} Buddha rebuked them … “How could nuns do this? This will affect people’s confidence …” … “And, monks, the nuns should recite this training rule like this: 

\subsection*{Final ruling }

\scrule{‘If a nun gives the full admission to a pregnant woman, she commits an offense entailing confession.’” }

\subsection*{Definitions }

\begin{description}%
\item[A: ] whoever … %
\item[Nun: ] … The nun who has been given the full ordination in unanimity by both Sanghas through a legal procedure consisting of one motion and three announcements that is irreversible and fit to stand—this sort of nun is meant in this case. %
\item[A pregnant woman: ] a woman with child is what is meant. %
\item[Gives the full admission: ] gives the full ordination. If, intending to give the full admission, she searches for a group, a teacher, a bowl, or a robe, or she establishes a monastery zone, she commits an offense of wrong conduct.\footnote{For a discussion of the rendering “monastery zone” for \textit{\textsanskrit{sīmā}}, see Appendix of Technical Terms. } After the motion, she commits an offense of wrong conduct.\footnote{The Pali just says \textit{\textsanskrit{dukkaṭa}}, without specifying that it is an \textit{\textsanskrit{āpatti}}, “an offense”. Yet elsewhere, such as at \href{https://suttacentral.net/pli-tv-bu-vb-ss10/en/brahmali\#2.65}{Bu Ss 10:2.65}, the \textit{\textsanskrit{dukkaṭa}} is annulled if you commit the full offense of \textit{\textsanskrit{saṅghādisesa}}. The implication is that in these contexts \textit{\textsanskrit{dukkaṭa}} should be read as \textit{\textsanskrit{āpatti} \textsanskrit{dukkaṭassa}}, “an offense of wrong conduct”. } After each of the first two announcements, she commits an offense of wrong conduct. When the last announcement is finished, the preceptor commits an offense entailing confession, and the group and the teacher commit an offense of wrong conduct. %
\end{description}

\subsection*{Permutations }

If\marginnote{2.2.1} the woman is pregnant, and the nun perceives her as such, yet she gives her the full admission, she commits an offense entailing confession. If the woman is pregnant, but the nun is unsure of it, yet she gives her the full admission, she commits an offense of wrong conduct. If the woman is pregnant, but the nun does not perceive her as such, and she gives her the full admission, there is no offense. 

If\marginnote{2.2.4} the woman is not pregnant, but the nun perceives her as such, she commits an offense of wrong conduct. If the woman is not pregnant, but the nun is unsure of it, she commits an offense of wrong conduct. If the woman is not pregnant, and the nun does not perceive her as such, there is no offense. 

\subsection*{Non-offenses }

There\marginnote{2.3.1} is no offense: if she gives the full admission to a woman who is pregnant, but she  does not perceive her as such;  if she gives the full admission to a woman who is not pregnant, and she does not perceive her as such;  if she is insane;  if she is the first offender. 

\scendsutta{The first training rule is finished. }

%
\section*{{\suttatitleacronym Bi Pc 62}{\suttatitletranslation The training rule on women who are breastfeeding }{\suttatitleroot Pāyantī}}
\addcontentsline{toc}{section}{\tocacronym{Bi Pc 62} \toctranslation{The training rule on women who are breastfeeding } \tocroot{Pāyantī}}
\markboth{The training rule on women who are breastfeeding }{Pāyantī}
\extramarks{Bi Pc 62}{Bi Pc 62}

\subsection*{Origin story }

At\marginnote{1.1} one time when the Buddha was staying at \textsanskrit{Sāvatthī} in \textsanskrit{Anāthapiṇḍika}’s Monastery, the nuns gave the full admission to a woman who was breastfeeding. When she walked for alms, people said, “Give almsfood to the Venerable. She has a companion.” People complained and criticized them, “How can the nuns give the full admission to a woman who is breastfeeding?” 

The\marginnote{1.8} nuns heard the complaints of those people and the nuns of few desires complained and criticized them, “How could nuns do this?” … “Is it true, monks, that nuns did this?” 

“It’s\marginnote{1.12} true, Sir.” 

The\marginnote{1.13} Buddha rebuked them … “How could nuns do this? This will affect people’s confidence …” … “And, monks, the nuns should recite this training rule like this: 

\subsection*{Final ruling }

\scrule{‘If a nun gives the full admission to a woman who is breastfeeding, she commits an offense entailing confession.’” }

\subsection*{Definitions }

\begin{description}%
\item[A: ] whoever … %
\item[Nun: ] … The nun who has been given the full ordination in unanimity by both Sanghas through a legal procedure consisting of one motion and three announcements that is irreversible and fit to stand—this sort of nun is meant in this case. %
\item[A woman who is breastfeeding: ] she is a mother or a wet-nurse. %
\item[Gives the full admission: ] gives the full ordination. If, intending to give the full admission, she searches for a group, a teacher, a bowl, or a robe, or she establishes a monastery zone, she commits an offense of wrong conduct. After the motion, she commits an offense of wrong conduct.\footnote{The Pali just says \textit{\textsanskrit{dukkaṭa}}, without specifying that it is an \textit{\textsanskrit{āpatti}}, “an offense”. Yet elsewhere, such as at \href{https://suttacentral.net/pli-tv-bu-vb-ss10/en/brahmali\#2.65}{Bu Ss 10:2.65}, the \textit{\textsanskrit{dukkaṭa}} is annulled if you commit the full offense of \textit{\textsanskrit{saṅghādisesa}}. The implication is that in these contexts \textit{\textsanskrit{dukkaṭa}} should be read as \textit{\textsanskrit{āpatti} \textsanskrit{dukkaṭassa}}, “an offense of wrong conduct”. } After each of the first two announcements, she commits an offense of wrong conduct. When the last announcement is finished, the preceptor commits an offense entailing confession, and the group and the teacher commit an offense of wrong conduct. %
\end{description}

\subsection*{Permutations }

If\marginnote{2.14.1} the woman is breastfeeding, and the nun perceives her as such, yet she gives her the full admission, she commits an offense entailing confession. If the woman is breastfeeding, but the nun is unsure of it, yet she gives her the full admission, she commits an offense of wrong conduct. If the woman is breastfeeding, but the nun does not perceive her as such, and she gives her the full admission, there is no offense. 

If\marginnote{2.17} the woman is not breastfeeding, but the nun perceives her as such, she commits an offense of wrong conduct. If the woman is not breastfeeding, but the nun is unsure of it, she commits an offense of wrong conduct. If the woman is not breastfeeding, and the nun does not perceive her as such, there is no offense. 

\subsection*{Non-offenses }

There\marginnote{2.20.1} is no offense: if she gives the full admission to a woman who is breastfeeding, but she does not perceive her as such;  if she gives the full admission to a woman who is not breastfeeding, and she does not perceive her as such;  if she is insane;  if she is the first offender. 

\scendsutta{The second training rule is finished. }

%
\section*{{\suttatitleacronym Bi Pc 63}{\suttatitletranslation The training rule on trainee nuns }{\suttatitleroot Asikkhita-sikkhamānā}}
\addcontentsline{toc}{section}{\tocacronym{Bi Pc 63} \toctranslation{The training rule on trainee nuns } \tocroot{Asikkhita-sikkhamānā}}
\markboth{The training rule on trainee nuns }{Asikkhita-sikkhamānā}
\extramarks{Bi Pc 63}{Bi Pc 63}

\subsection*{Origin story }

At\marginnote{1.1} one time the Buddha was staying at \textsanskrit{Sāvatthī} in the Jeta Grove, \textsanskrit{Anāthapiṇḍika}’s Monastery. At that time the nuns were giving the full admission to trainee nuns who had not trained for two years in the six rules. They were ignorant and incompetent, and did not know what was allowable and what was not. 

The\marginnote{1.4} nuns of few desires complained and criticized them, “How can nuns give the full admission to trainee nuns who haven’t trained for two years in the six rules?” … “Is it true, monks, that nuns do this?” 

“It’s\marginnote{1.7} true, Sir.” 

The\marginnote{1.8} Buddha rebuked them … “How can nuns do this? This will affect people’s confidence …” After rebuking them … he gave a teaching and addressed the monks: 

“Monks,\marginnote{1.13} approval is required for a trainee nun to train in the six rules for two years.\footnote{For the contextual meaning of the Pali, see \href{https://suttacentral.net/pli-tv-bi-vb-pc64/en/brahmali\#1.36.1}{Bi Pc 64:1.36.1}. } 

And\marginnote{1.14} it should be given like this. 

After\marginnote{1.15} approaching the Sangha of nuns, that trainee nun should arrange her upper robe over one shoulder and pay respect at the feet of the nuns. She should then squat on her heels, raise her joined palms, and say: 

‘Venerables,\marginnote{1.16} I, so-and-so, am a trainee nun under Venerable so-and-so. I ask the Sangha for approval to train for two years in the six rules.’ And she should ask a second and a third time. 

A\marginnote{1.20} competent and capable nun should then inform the Sangha: 

‘Please,\marginnote{1.21} Venerables, I ask the Sangha to listen. So-and-so, who is a trainee nun under Venerable so-and-so, is asking the Sangha for approval to train in the six rules for two years. If the Sangha is ready, it should give approval to trainee nun so-and-so to train in the six rules for two years. This is the motion. 

Please,\marginnote{1.25} Venerables, I ask the Sangha to listen. So-and-so, who is a trainee nun under Venerable so-and-so, is asking the Sangha for approval to train in the six rules for two years. The Sangha gives approval to trainee nun so-and-so to train in the six rules for two years. Any nun who approves of giving approval to trainee nun so-and-so to train in the six rules for two years should remain silent. Any nun who doesn’t approve should speak up. 

The\marginnote{1.30} Sangha has given approval to trainee nun so-and-so to train in the six rules for two years. The Sangha approves and is therefore silent. I’ll remember it thus.’ 

That\marginnote{1.32} trainee nun should then be told to say this: ‘I undertake to abstain from killing living beings for two years without transgression. I undertake to abstain from stealing for two years without transgression. I undertake to abstain from sexual activity for two years without transgression. I undertake to abstain from lying for two years without transgression. I undertake to abstain from alcohol, which causes heedlessness, for two years without transgression. I undertake to abstain from eating at the wrong time for two years without transgression.’” 

Then,\marginnote{1.39} after rebuking those nuns in many ways, the Buddha spoke in dispraise of being difficult to support … “And, monks, the nuns should recite this training rule like this: 

\subsection*{Final ruling }

\scrule{‘If a nun gives the full admission to a trainee nun who has not trained in the six rules for two years, she commits an offense entailing confession.’” }

\subsection*{Definitions }

\begin{description}%
\item[A: ] whoever … %
\item[Nun: ] … The nun who has been given the full ordination in unanimity by both Sanghas through a legal procedure consisting of one motion and three announcements that is irreversible and fit to stand—this sort of nun is meant in this case. %
\item[Two years: ] two twelve-month periods. %
\item[Who has not trained: ] the training has not been given to her, or the training has been given to her, but she has failed in it. %
\item[Gives the full admission: ] gives the full ordination. If, intending to give the full admission, she searches for a group, a teacher, a bowl, or a robe, or she establishes a monastery zone, she commits an offense of wrong conduct. After the motion, she commits an offense of wrong conduct.\footnote{The Pali just says \textit{\textsanskrit{dukkaṭa}}, without specifying that it is an \textit{\textsanskrit{āpatti}}, “an offense”. Yet elsewhere, such as at \href{https://suttacentral.net/pli-tv-bu-vb-ss10/en/brahmali\#2.65}{Bu Ss 10:2.65}, the \textit{\textsanskrit{dukkaṭa}} is annulled if you commit the full offense of \textit{\textsanskrit{saṅghādisesa}}. The implication is that in these contexts \textit{\textsanskrit{dukkaṭa}} should be read as \textit{\textsanskrit{āpatti} \textsanskrit{dukkaṭassa}}, “an offense of wrong conduct”. } After each of the first two announcements, she commits an offense of wrong conduct. When the last announcement is finished, the preceptor commits an offense entailing confession, and the group and the teacher commit an offense of wrong conduct. %
\end{description}

\subsection*{Permutations }

If\marginnote{2.2.1} it is a legitimate legal procedure, and she perceives it as such, and she gives the full admission, she commits an offense entailing confession. If it is a legitimate legal procedure, but she is unsure of it, and she gives the full admission, she commits an offense entailing confession. If it is a legitimate legal procedure, but she perceives it as illegitimate, and she gives the full admission, she commits an offense entailing confession. 

If\marginnote{2.2.4} it is an illegitimate legal procedure, but she perceives it as legitimate, she commits an offense of wrong conduct. If it is an illegitimate legal procedure, but she is unsure of it, she commits an offense of wrong conduct. If it is an illegitimate legal procedure, and she perceives it as such, she commits an offense of wrong conduct. 

\subsection*{Non-offenses }

There\marginnote{2.3.1} is no offense: if she gives the full admission to a trainee nun who has trained in the six rules for two years;  if she is insane;  if she is the first offender. 

\scendsutta{The third training rule is finished. }

%
\section*{{\suttatitleacronym Bi Pc 64}{\suttatitletranslation The second training rule on trainee nuns }{\suttatitleroot Sikkhita-sikkhamānā-asammata}}
\addcontentsline{toc}{section}{\tocacronym{Bi Pc 64} \toctranslation{The second training rule on trainee nuns } \tocroot{Sikkhita-sikkhamānā-asammata}}
\markboth{The second training rule on trainee nuns }{Sikkhita-sikkhamānā-asammata}
\extramarks{Bi Pc 64}{Bi Pc 64}

\subsection*{Origin story }

At\marginnote{1.1} one time the Buddha was staying at \textsanskrit{Sāvatthī} in the Jeta Grove, \textsanskrit{Anāthapiṇḍika}’s Monastery. At that time the nuns were giving the full admission to trainee nuns who had trained in the six rules for two years, but who had not been approved by the Sangha. The nuns said, “Come, trainee nuns, find out about this,” “Give this,” “Bring this,” “There’s need for this,” or “Make this allowable.” But they replied, “Venerables, we’re not trainee nuns. We’re nuns.” 

The\marginnote{1.7} nuns of few desires complained and criticized them, “How can nuns give the full admission to trainee nuns who have trained for two years in the six rules, but who haven’t been approved by the Sangha?” … “Is it true, monks, that nuns do this?” 

“It’s\marginnote{1.10} true, Sir.” 

The\marginnote{1.11} Buddha rebuked them … “How can nuns do this? This will affect people’s confidence …” After rebuking them … he gave a teaching and addressed the monks: 

“Monks,\marginnote{1.16} approval is required for the full admission of a trainee nun who has trained for two years in the six rules. 

And\marginnote{1.17} it should be given like this. 

After\marginnote{1.18} approaching the Sangha of nuns, that trainee nun should arrange her upper robe over one shoulder and pay respect at the feet of the nuns. She should then squat on her heels, raise her joined palms, and say: 

‘Venerables,\marginnote{1.19} I, the trainee nun so-and-so, who has trained for two years in the six rules under Venerable so-and-so, ask the Sangha for approval to be fully admitted.’ 

And\marginnote{1.20} she should ask a second and a third time. 

A\marginnote{1.22} competent and capable nun should then inform the Sangha: 

‘Please,\marginnote{1.23} Venerables, I ask the Sangha to listen. This trainee nun so-and-so, who has trained under Venerable so-and-so for two years in the six rules, is asking the Sangha for approval to be fully admitted. If the Sangha is ready, it should give approval for the trainee nun so-and-so, who has trained for two years in the six rules, to be fully admitted. This is the motion. 

Please,\marginnote{1.27} Venerables, I ask the Sangha to listen. This trainee nun so-and-so, who has trained under Venerable so-and-so for two years in the six rules, is asking the Sangha for approval to be fully admitted. The Sangha gives approval for the trainee nun so-and-so, who has trained for two years in the six rules, to be fully admitted. Any nun who approves of giving approval for the trainee nun so-and-so, who has trained for two years in the six rules, to be fully admitted should remain silent. Any nun who doesn’t approve should speak up. 

The\marginnote{1.32} Sangha has given approval for the trainee nun so-and-so, who has trained for two years in the six rules, to be fully admitted. The Sangha approves and is therefore silent. I’ll remember it thus.’” 

Then,\marginnote{1.34} after rebuking those nuns in many ways, the Buddha spoke in dispraise of being difficult to support … “And, monks, the nuns should recite this training rule like this: 

\subsection*{Final ruling }

\scrule{‘If a nun gives the full admission to a trainee nun who has trained in the six rules for two years, but who has not been approved by the Sangha, she commits an offense entailing confession.’” }

\subsection*{Definitions }

\begin{description}%
\item[A: ] whoever … %
\item[Nun: ] … The nun who has been given the full ordination in unanimity by both Sanghas through a legal procedure consisting of one motion and three announcements that is irreversible and fit to stand—this sort of nun is meant in this case. %
\item[Two years: ] two twelve-month periods. %
\item[Who has trained: ] who has trained in the six rules. %
\item[Who has not been approved: ] approval to be fully admitted has not been given though a legal procedure consisting of one motion and one announcement. %
\item[Gives the full admission: ] gives the full ordination. If, intending to give the full admission, she searches for a group, a teacher, a bowl, or a robe, or she establishes a monastery zone, she commits an offense of wrong conduct. After the motion, she commits an offense of wrong conduct.\footnote{The Pali just says \textit{\textsanskrit{dukkaṭa}}, without specifying that it is an \textit{\textsanskrit{āpatti}}, “an offense”. Yet elsewhere, such as at \href{https://suttacentral.net/pli-tv-bu-vb-ss10/en/brahmali\#2.65}{Bu Ss 10:2.65}, the \textit{\textsanskrit{dukkaṭa}} is annulled if you commit the full offense of \textit{\textsanskrit{saṅghādisesa}}. The implication is that in these contexts \textit{\textsanskrit{dukkaṭa}} should be read as \textit{\textsanskrit{āpatti} \textsanskrit{dukkaṭassa}}, “an offense of wrong conduct”. } After each of the first two announcements, she commits an offense of wrong conduct. When the last announcement is finished, the preceptor commits an offense entailing confession, and the group and the teacher commit an offense of wrong conduct. %
\end{description}

\subsection*{Permutations }

If\marginnote{2.18.1} it is a legitimate legal procedure, and she perceives it as such, and she gives the full admission, she commits an offense entailing confession. If it is a legitimate legal procedure, but she is unsure of it, and she gives the full admission, she commits an offense entailing confession. If it is a legitimate legal procedure, but she perceives it as illegitimate, and she gives the full admission, she commits an offense entailing confession. 

If\marginnote{2.21} it is an illegitimate legal procedure, but she perceives it as legitimate, she commits an offense of wrong conduct. If it is an illegitimate legal procedure, but she is unsure of it, she commits an offense of wrong conduct. If it is an illegitimate legal procedure, and she perceives it as such, she commits an offense of wrong conduct. 

\subsection*{Non-offenses }

There\marginnote{2.24.1} is no offense: if she gives the full admission to a trainee nun who has trained in the six rules for two years and who has been approved by the Sangha;  if she is insane;  if she is the first offender. 

\scendsutta{The fourth training rule is finished. }

%
\section*{{\suttatitleacronym Bi Pc 65}{\suttatitletranslation The training rule on married girls }{\suttatitleroot Ūnadvādasavassa-gihigata}}
\addcontentsline{toc}{section}{\tocacronym{Bi Pc 65} \toctranslation{The training rule on married girls } \tocroot{Ūnadvādasavassa-gihigata}}
\markboth{The training rule on married girls }{Ūnadvādasavassa-gihigata}
\extramarks{Bi Pc 65}{Bi Pc 65}

\subsection*{Origin story }

At\marginnote{1.1} one time when the Buddha was staying at \textsanskrit{Sāvatthī} in \textsanskrit{Anāthapiṇḍika}’s Monastery, the nuns were giving the full admission to married girls less than twelve years old. They were unable to endure cold and heat; hunger and thirst; contact with horseflies, mosquitoes, wind, the burning sun, and creeping animals and insects; rude and unwelcome speech; and they were unable to bear up with bodily feelings that are painful, severe, sharp, and destructive of life. 

The\marginnote{1.5} nuns of few desires complained and criticized them, “How can nuns give the full admission to married girls less than twelve years old?” … “Is it true, monks, that nuns do this?” 

“It’s\marginnote{1.8} true, Sir.” 

The\marginnote{1.9} Buddha rebuked them … “How can nuns do this? A married girl less than twelve years old is unable to endure cold and heat; hunger and thirst; contact with horseflies, mosquitoes, wind, the burning sun, and creeping animals and insects; rude and unwelcome speech; and she’s unable to bear up with bodily feelings that are painful, severe, sharp, and destructive of life. But a married girl who is twelve years old is able to endure these things. This will affect people’s confidence …” … “And, monks, the nuns should recite this training rule like this: 

\subsection*{Final ruling }

\scrule{‘If a nun gives the full admission to a married girl who is less than twelve years old, she commits an offense entailing confession.’” }

\subsection*{Definitions }

\begin{description}%
\item[A: ] whoever … %
\item[Nun: ] … The nun who has been given the full ordination in unanimity by both Sanghas through a legal procedure consisting of one motion and three announcements that is irreversible and fit to stand—this sort of nun is meant in this case. %
\item[Who is less than twelve years old: ] who has not reached twelve years of age. %
\item[A married girl:\footnote{For the rendering of \textit{\textsanskrit{gihigatā}} as “married girl”,  see Appendix of Technical Terms. } ] one who has gone to the place of a man is what is meant. %
\item[Gives the full admission: ] gives the full ordination. If, intending to give the full admission, she searches for a group, a teacher, a bowl, or a robe, or she establishes a monastery zone, she commits an offense of wrong conduct. After the motion, she commits an offense of wrong conduct.\footnote{The Pali just says \textit{\textsanskrit{dukkaṭa}}, without specifying that it is an \textit{\textsanskrit{āpatti}}, “an offense”. Yet elsewhere, such as at \href{https://suttacentral.net/pli-tv-bu-vb-ss10/en/brahmali\#2.65}{Bu Ss 10:2.65}, the \textit{\textsanskrit{dukkaṭa}} is annulled if you commit the full offense of \textit{\textsanskrit{saṅghādisesa}}. The implication is that in these contexts \textit{\textsanskrit{dukkaṭa}} should be read as \textit{\textsanskrit{āpatti} \textsanskrit{dukkaṭassa}}, “an offense of wrong conduct”. } After each of the first two announcements, she commits an offense of wrong conduct. When the last announcement is finished, the preceptor commits an offense entailing confession, and the group and the teacher commit an offense of wrong conduct. %
\end{description}

\subsection*{Permutations }

If\marginnote{2.2.1} the girl is less than twelve years old, but the nun perceives her as less, yet she gives her the full admission, she commits an offense entailing confession. If the girl is less than twelve years old, but the nun is unsure of it, yet she gives her the full admission, she commits an offense of wrong conduct. If the girl is less than twelve years old, but the nun perceives her as more, and she gives her the full admission, there is no offense. 

If\marginnote{2.2.4} the girl is more than twelve years old, but the nun perceives her as less, she commits an offense of wrong conduct. If the girl is more than twelve years old, but the nun is unsure of it, she commits an offense of wrong conduct. If the girl is more than twelve years old, and the nun perceives her as more, there is no offense. 

\subsection*{Non-offenses }

There\marginnote{2.3.1} is no offense: if she gives the full admission to a girl less than twelve years old, but she perceives her as more;  if she gives the full admission to a girl more than twelve years old, and she perceives her as more;  if she is insane;  if she is the first offender. 

\scendsutta{The fifth training rule is finished. }

%
\section*{{\suttatitleacronym Bi Pc 66}{\suttatitletranslation The second training rule on married girls }{\suttatitleroot Paripuṇṇadvādasavassa-asikkhita-gihigata}}
\addcontentsline{toc}{section}{\tocacronym{Bi Pc 66} \toctranslation{The second training rule on married girls } \tocroot{Paripuṇṇadvādasavassa-asikkhita-gihigata}}
\markboth{The second training rule on married girls }{Paripuṇṇadvādasavassa-asikkhita-gihigata}
\extramarks{Bi Pc 66}{Bi Pc 66}

\subsection*{Origin story }

At\marginnote{1.1} one time the Buddha was staying at \textsanskrit{Sāvatthī} in the Jeta Grove, \textsanskrit{Anāthapiṇḍika}’s Monastery. At that time the nuns were giving the full admission to married girls who were more than twelve years old, but who had not trained for two years in the six rules. They were ignorant and incompetent, and they did not know what was allowable and what was not. 

The\marginnote{1.4} nuns of few desires complained and criticized them, “How can nuns give the full admission to married girls who are more than twelve years old, but who haven’t trained for two years in the six rules?” … “Is it true, monks, that nuns do this?” 

“It’s\marginnote{1.7} true, Sir.” 

The\marginnote{1.8} Buddha rebuked them … “How can nuns do this? This will affect people’s confidence …” After rebuking them … he gave a teaching and addressed the monks: 

“Monks,\marginnote{1.12} approval is required for a married girl who is more than twelve years old to train in the six rules for two years. 

And\marginnote{1.13} the approval is to be given like this. 

After\marginnote{1.14} approaching the Sangha of nuns, that married girl who is more than twelve years old should arrange her upper robe over one shoulder and pay respect at the feet of the nuns. She should then squat on her heels, raise her joined palms, and say: 

‘Venerables,\marginnote{1.15} I, so-and-so, a married girl who is more than twelve years old, am training under Venerable so-and-so. I ask the Sangha for approval to train for two years in the six rules.’ 

And\marginnote{1.16} she should ask a second and a third time. 

A\marginnote{1.18} competent and capable nun should then inform the Sangha: 

‘Please,\marginnote{1.19} Venerables, I ask the Sangha to listen. The married girl so-and-so, who is more than twelve years old and training under Venerable so-and-so, is asking the Sangha for approval to train in the six rules for two years. If the Sangha is ready, it should give approval to the married girl so-and-so, who is more than twelve years old, to train in the six rules for two years. This is the motion. 

Please,\marginnote{1.23} Venerables, I ask the Sangha to listen. The married girl so-and-so, who is more than twelve years old and training under Venerable so-and-so, is asking the Sangha for approval to train in the six rules for two years. The Sangha gives approval to the married girl so-and-so, who is more than twelve years old, to train in the six rules for two years. Any nun who approves of giving approval to the married girl so-and-so, who is more than twelve years old, to train in the six rules for two years should remain silent. Any nun who doesn’t approve should speak up. 

The\marginnote{1.28} Sangha has given approval to the married girl so-and-so, who is more than twelve years old, to train in the six rules for two years. The Sangha approves and is therefore silent. I’ll remember it thus.’ 

That\marginnote{1.30} married girl, who is more than twelve years old, should be told to say this: ‘I undertake to abstain from killing living beings for two years without transgression. … I undertake to abstain from eating at the wrong time for two years without transgression.’” 

Then,\marginnote{1.33} after rebuking those nuns in many ways, the Buddha spoke in dispraise of being difficult to support … “And, monks, the nuns should recite this training rule like this: 

\subsection*{Final ruling }

\scrule{‘If a nun gives the full admission to a married girl who is more than twelve years old, but who has not trained in the six rules for two years, she commits an offense entailing confession.’” }

\subsection*{Definitions }

\begin{description}%
\item[A: ] whoever … %
\item[Nun: ] … The nun who has been given the full ordination in unanimity by both Sanghas through a legal procedure consisting of one motion and three announcements that is irreversible and fit to stand—this sort of nun is meant in this case. %
\item[Who is more than twelve years old: ] who has reached twelve years of age. %
\item[A married girl: ] one who has gone to the place of a man is what is meant. %
\item[Two years: ] two twelve-month periods. %
\item[Who has not trained: ] the training has not been given to her, or the training has been given to her, but she has failed in it. %
\item[Gives the full admission: ] gives the full ordination. If, intending to give the full admission, she searches for a group, a teacher, a bowl, or a robe, or she establishes a monastery zone, she commits an offense of wrong conduct. After the motion, she commits an offense of wrong conduct.\footnote{The Pali just says \textit{\textsanskrit{dukkaṭa}}, without specifying that it is an \textit{\textsanskrit{āpatti}}, “an offense”. Yet elsewhere, such as at \href{https://suttacentral.net/pli-tv-bu-vb-ss10/en/brahmali\#2.65}{Bu Ss 10:2.65}, the \textit{\textsanskrit{dukkaṭa}} is annulled if you commit the full offense of \textit{\textsanskrit{saṅghādisesa}}. The implication is that in these contexts \textit{\textsanskrit{dukkaṭa}} should be read as \textit{\textsanskrit{āpatti} \textsanskrit{dukkaṭassa}}, “an offense of wrong conduct”. } After each of the first two announcements, she commits an offense of wrong conduct. When the last announcement is finished, the preceptor commits an offense entailing confession, and the group and the teacher commit an offense of wrong conduct. %
\end{description}

\subsection*{Permutations }

If\marginnote{2.20.1} it is a legitimate legal procedure, and she perceives it as such, and she gives the full admission, she commits an offense entailing confession. If it is a legitimate legal procedure, but she is unsure of it, and she gives the full admission, she commits an offense entailing confession. If it is a legitimate legal procedure, but she perceives it as illegitimate, and she gives the full admission, she commits an offense entailing confession. 

If\marginnote{2.23} it is an illegitimate legal procedure, but she perceives it as legitimate, and she gives the full admission, she commits an offense of wrong conduct. If it is an illegitimate legal procedure, but she is unsure of it, and she gives the full admission, she commits an offense of wrong conduct. If it is an illegitimate legal procedure, and she perceives it as such, and she gives the full admission, she commits an offense of wrong conduct. 

\subsection*{Non-offenses }

There\marginnote{2.26.1} is no offense: if she gives the full admission to a married girl who is more than twelve years old and who has trained in the six rules for two years;  if she is insane;  if she is the first offender. 

\scendsutta{The sixth training rule is finished. }

%
\section*{{\suttatitleacronym Bi Pc 67}{\suttatitletranslation The third training rule on married girls }{\suttatitleroot Paripuṇṇadvādasavassa-sikkhita-gihigata-asammata}}
\addcontentsline{toc}{section}{\tocacronym{Bi Pc 67} \toctranslation{The third training rule on married girls } \tocroot{Paripuṇṇadvādasavassa-sikkhita-gihigata-asammata}}
\markboth{The third training rule on married girls }{Paripuṇṇadvādasavassa-sikkhita-gihigata-asammata}
\extramarks{Bi Pc 67}{Bi Pc 67}

\subsection*{Origin story }

At\marginnote{1.1} one time the Buddha was staying at \textsanskrit{Sāvatthī} in the Jeta Grove, \textsanskrit{Anāthapiṇḍika}’s Monastery. At that time the nuns were giving the full admission to married girls who were more than twelve years old and who had trained in the six rules for two years, but who had not been approved by the Sangha. The nuns said, “Come, trainee nuns, find out about this,” “Give this,” “Bring this,” “There’s need for this,” or “Make this allowable.” But they replied, “Venerables, we’re not trainee nuns. We’re nuns.” 

The\marginnote{1.7} nuns of few desires complained and criticized them, “How can nuns give the full admission to married girls who are more than twelve years old and who have trained for two years in the six rules, but who haven’t been approved by the Sangha?” … “Is it true, monks, that nuns do this?” 

“It’s\marginnote{1.10} true, Sir.” 

The\marginnote{1.11} Buddha rebuked them … “How can nuns do this? This will affect people’s confidence …” After rebuking them … he gave a teaching and addressed the monks: 

“Monks,\marginnote{1.16} approval should be given for the full admission of a married girl who is more than twelve years old and who has trained for two years in the six rules. 

And\marginnote{1.17} it should be given like this. 

After\marginnote{1.18} approaching the Sangha of nuns, that married girl who is more than twelve years old should arrange her upper robe over one shoulder and pay respect at the feet of the nuns. She should then squat on her heels, raise her joined palms, and say: 

‘Venerables,\marginnote{1.19} I, the married girl so-and-so, who is more than twelve years old and who has trained for two years in the six rules under Venerable so-and-so, ask the Sangha for approval to be fully admitted.’ And she should ask a second and a third time. 

A\marginnote{1.22} competent and capable nun should then inform the Sangha: 

‘Please,\marginnote{1.23} Venerables, I ask the Sangha to listen. This married girl so-and-so, who is more than twelve years old and who has trained for two years in the six rules under Venerable so-and-so, is asking the Sangha for approval to be fully admitted. If the Sangha is ready, it should give approval for the married girl so-and-so, who is more than twelve years old and who has trained for two years in the six rules under Venerable so-and-so, to be fully admitted. This is the motion. 

Please,\marginnote{1.27} Venerables, I ask the Sangha to listen. This married girl so-and-so, who is more than twelve years old and who has trained for two years in the six rules under Venerable so-and-so, is asking the Sangha for approval to be fully admitted. The Sangha gives approval for the married girl so-and-so, who is more than twelve years old and who has trained for two years in the six rules under Venerable so-and-so, to be fully admitted. Any nun who approves of giving approval for the married girl so-and-so, who is more than twelve years old and who has trained for two years in the six rules under Venerable so-and-so, to be fully admitted should remain silent. Any nun who doesn’t approve should speak up. 

The\marginnote{1.32} Sangha has given approval for the married girl so-and-so, who is more than twelve years old and who has trained for two years in the six rules under Venerable so-and-so, to be fully admitted. The Sangha approves and is therefore silent. I’ll remember it thus.’” 

Then,\marginnote{1.34} after rebuking those nuns in many ways, the Buddha spoke in dispraise of being difficult to support … “And, monks, the nuns should recite this training rule like this: 

\subsection*{Final ruling }

\scrule{‘If a nun gives the full admission to a married girl who is more than twelve years old and who has trained for two years in the six rules, but who has not been approved by the Sangha, she commits an offense entailing confession.’” }

\subsection*{Definitions }

\begin{description}%
\item[A: ] whoever … %
\item[Nun: ] … The nun who has been given the full ordination in unanimity by both Sanghas through a legal procedure consisting of one motion and three announcements that is irreversible and fit to stand—this sort of nun is meant in this case. %
\item[Who is more than twelve years old: ] who has reached twelve years of age. %
\item[A married girl: ] one who has gone to the place of a man is what is meant. %
\item[Two years: ] two twelve-month periods. %
\item[Who has trained: ] who has trained in the six rules. %
\item[Who has not been approved: ] approval to be fully admitted has not been given though a legal procedure consisting of one motion and one announcement. %
\item[Gives the full admission: ] gives the full ordination. If, intending to give the full admission, she searches for a group, a teacher, a bowl, or a robe, or she establishes a monastery zone, she commits an offense of wrong conduct. After the motion, she commits an offense of wrong conduct.\footnote{The Pali just says \textit{\textsanskrit{dukkaṭa}}, without specifying that it is an \textit{\textsanskrit{āpatti}}, “an offense”. Yet elsewhere, such as at \href{https://suttacentral.net/pli-tv-bu-vb-ss10/en/brahmali\#2.65}{Bu Ss 10:2.65}, the \textit{\textsanskrit{dukkaṭa}} is annulled if you commit the full offense of \textit{\textsanskrit{saṅghādisesa}}. The implication is that in these contexts \textit{\textsanskrit{dukkaṭa}} should be read as \textit{\textsanskrit{āpatti} \textsanskrit{dukkaṭassa}}, “an offense of wrong conduct”. } After each of the first two announcements, she commits an offense of wrong conduct. When the last announcement is finished, the preceptor commits an offense entailing confession, and the group and the teacher commit an offense of wrong conduct. %
\end{description}

\subsection*{Permutations }

If\marginnote{2.22.1} it is a legitimate legal procedure, and she perceives it as such, and she gives the full admission, she commits an offense entailing confession. If it is a legitimate legal procedure, but she is unsure of it, and she gives the full admission, she commits an offense entailing confession. If it is a legitimate legal procedure, but she perceives it as illegitimate, and she gives the full admission, she commits an offense entailing confession. 

If\marginnote{2.25} it is an illegitimate legal procedure, but she perceives it as legitimate, she commits an offense of wrong conduct. If it is an illegitimate legal procedure, but she is unsure of it, she commits an offense of wrong conduct. If it is an illegitimate legal procedure, and she perceives it as such, she commits an offense of wrong conduct. 

\subsection*{Non-offenses }

There\marginnote{2.28.1} is no offense: if she gives the full admission to a married girl who is more than twelve years old and who has trained in the six rules for two years and who has been approved by the Sangha;  if she is insane;  if she is the first offender. 

\scendsutta{The seventh training rule is finished. }

%
\section*{{\suttatitleacronym Bi Pc 68}{\suttatitletranslation The training rule on disciples }{\suttatitleroot Sahajīvinī-ananuggahaṇa}}
\addcontentsline{toc}{section}{\tocacronym{Bi Pc 68} \toctranslation{The training rule on disciples } \tocroot{Sahajīvinī-ananuggahaṇa}}
\markboth{The training rule on disciples }{Sahajīvinī-ananuggahaṇa}
\extramarks{Bi Pc 68}{Bi Pc 68}

\subsection*{Origin story }

At\marginnote{1.1} one time the Buddha was staying at \textsanskrit{Sāvatthī} in the Jeta Grove, \textsanskrit{Anāthapiṇḍika}’s Monastery. At that time the nun \textsanskrit{Thullanandā} gave the full admission to a disciple, but then, for the next two years, neither guided her nor had her guided. They were ignorant and incompetent, and they did not know what was allowable and what was not. 

The\marginnote{1.5} nuns of few desires complained and criticized her, “How can Venerable \textsanskrit{Thullanandā} give the full admission to a disciple, and then neither guide her nor have her guided for two years?” … “Is it true, monks, that the nun \textsanskrit{Thullanandā} does this?” 

“It’s\marginnote{1.8} true, Sir.” 

The\marginnote{1.9} Buddha rebuked her … “How can the nun \textsanskrit{Thullanandā} do this? This will affect people’s confidence …” … “And, monks, the nuns should recite this training rule like this: 

\subsection*{Final ruling }

\scrule{‘If a nun gives the full admission to a disciple, and then, for the next two years, neither guides her nor has her guided, she commits an offense entailing confession.’” }

\subsection*{Definitions }

\begin{description}%
\item[A: ] whoever … %
\item[Nun: ] … The nun who has been given the full ordination in unanimity by both Sanghas through a legal procedure consisting of one motion and three announcements that is irreversible and fit to stand—this sort of nun is meant in this case. %
\item[A disciple: ] a student is what is meant. %
\item[Gives the full admission: ] gives the full ordination. %
\item[Two years: ] two twelve-month periods. %
\item[Neither guides her: ] she does not herself guide her—through recitation, through questioning, through instruction, though teaching. %
\item[Nor has her guided: ] she does not ask anyone else.  If she thinks, “I’ll neither guide her nor have her guided for two years,” then by the mere fact of abandoning her duty, she commits an offense entailing confession. %
\end{description}

\subsection*{Non-offenses }

There\marginnote{2.2.1} is no offense: if there is an obstacle;  if she searches for someone to guide her, but cannot find anyone;  if she is sick;  if there is an emergency;  if she is insane;  if she is the first offender. 

\scendsutta{The eighth training rule is finished. }

%
\section*{{\suttatitleacronym Bi Pc 69}{\suttatitletranslation The training rule on not following one’s mentor }{\suttatitleroot Pavattinī-nānubandhana}}
\addcontentsline{toc}{section}{\tocacronym{Bi Pc 69} \toctranslation{The training rule on not following one’s mentor } \tocroot{Pavattinī-nānubandhana}}
\markboth{The training rule on not following one’s mentor }{Pavattinī-nānubandhana}
\extramarks{Bi Pc 69}{Bi Pc 69}

\subsection*{Origin story }

At\marginnote{1.1} one time the Buddha was staying at \textsanskrit{Sāvatthī} in the Jeta Grove, \textsanskrit{Anāthapiṇḍika}’s Monastery. At that time nuns who had received the full admission did not follow their mentors for two years. They were ignorant and incompetent, and they did not know what was allowable and what was not. 

The\marginnote{1.5} nuns of few desires complained and criticized them, “How can those nuns receive the full admission and then not follow their mentors for two years?” … “Is it true, monks, that nuns don’t do this?” 

“It’s\marginnote{1.8} true, Sir.” 

The\marginnote{1.9} Buddha rebuked them … “How can nuns not do this? This will affect people’s confidence …” … “And, monks, the nuns should recite this training rule like this: 

\subsection*{Final ruling }

\scrule{‘If a nun does not follow the mentor who gave her the full admission for two years, she commits an offense entailing confession.’” }

\subsection*{Definitions }

\begin{description}%
\item[A: ] whoever … %
\item[Nun: ] … The nun who has been given the full ordination in unanimity by both Sanghas through a legal procedure consisting of one motion and three announcements that is irreversible and fit to stand—this sort of nun is meant in this case. %
\item[Who gave her the full admission: ] who gave her the full ordination. %
\item[The mentor: ] the preceptor is what is meant. %
\item[Two years: ] two twelve-month periods. %
\item[Does not follow: ] does not herself attend on her. If she thinks, “I won’t follow her for two years,” then by the mere fact of abandoning her duty, she commits an offense entailing confession. %
\end{description}

\subsection*{Non-offenses }

There\marginnote{2.2.1} is no offense: if the preceptor is ignorant or shameless;  if she is sick;  if there is an emergency;  if she is insane;  if she is the first offender. 

\scendsutta{The ninth training rule is finished. }

%
\section*{{\suttatitleacronym Bi Pc 70}{\suttatitletranslation The second training rule on disciples }{\suttatitleroot Sahajīvinī-avūpakāsana}}
\addcontentsline{toc}{section}{\tocacronym{Bi Pc 70} \toctranslation{The second training rule on disciples } \tocroot{Sahajīvinī-avūpakāsana}}
\markboth{The second training rule on disciples }{Sahajīvinī-avūpakāsana}
\extramarks{Bi Pc 70}{Bi Pc 70}

\subsection*{Origin story }

At\marginnote{1.1} one time the Buddha was staying at \textsanskrit{Sāvatthī} in the Jeta Grove, \textsanskrit{Anāthapiṇḍika}’s Monastery. At that time the nun \textsanskrit{Thullanandā} gave the full admission to a disciple, but then neither sent her away nor had her sent away. And so her husband got hold of her. 

The\marginnote{1.4} nuns of few desires complained and criticized her, “How could Venerable \textsanskrit{Thullanandā} give the full admission to a disciple, and then neither send her away nor have her sent away? If this nun had gone away, her husband would not have gotten hold of her.” … “Is it true, monks, that the nun \textsanskrit{Thullanandā} didn’t do this?” 

“It’s\marginnote{1.8} true, Sir.” 

The\marginnote{1.9} Buddha rebuked her … “How could the nun \textsanskrit{Thullanandā} not do this? This will affect people’s confidence …” … “And, monks, the nuns should recite this training rule like this: 

\subsection*{Final ruling }

\scrule{‘If a nun gives the full admission to a disciple, and then neither sends her away nor has her sent away at least 65 to 80 kilometers, she commits an offense entailing confession.’” }

\subsection*{Definitions }

\begin{description}%
\item[A: ] whoever … %
\item[Nun: ] … The nun who has been given the full ordination in unanimity by both Sanghas through a legal procedure consisting of one motion and three announcements that is irreversible and fit to stand—this sort of nun is meant in this case. %
\item[Disciple: ] student is what is meant. %
\item[Gives the full admission: ] gives the full ordination. %
\item[Neither sends her away: ] she does not herself send her away.\footnote{According to the commentary, this means actually grabbing the newly ordained person and personally taking her away. Sp 2.1116: \textit{Neva \textsanskrit{vūpakāseyyāti} na \textsanskrit{gahetvā} gaccheyya}, “\textit{Neva \textsanskrit{vūpakāseyya}} means one does not take hold of her and leave.” The verb \textit{\textsanskrit{vūpakāseti}}, however, just means to cause separation; it does not imply such personal involvement. } %
\item[Nor has her sent away: ] she does not ask anyone else to send her away. If she thinks, “I’ll neither send her away nor have her sent away, not even 65 to 80 kilometers,” then by the mere fact of abandoning her duty, she commits an offense entailing confession.\footnote{For a discussion of the \textit{yojana}, see \textit{sugata} in Appendix of Technical Terms. } %
\end{description}

\subsection*{Non-offenses }

There\marginnote{2.2.1} is no offense: if there is an obstacle;  if she searches for a companion nun for her, but cannot find anyone;  if she is sick;  if there is an emergency;  if she is insane;  if she is the first offender. 

\scendsutta{The tenth training rule is finished. }

\scendvagga{The seventh subchapter on pregnant women is finished. }

%
\section*{{\suttatitleacronym Bi Pc 71}{\suttatitletranslation The training rule on unmarried girls }{\suttatitleroot Ūnavīsativassa-kumāribhūta}}
\addcontentsline{toc}{section}{\tocacronym{Bi Pc 71} \toctranslation{The training rule on unmarried girls } \tocroot{Ūnavīsativassa-kumāribhūta}}
\markboth{The training rule on unmarried girls }{Ūnavīsativassa-kumāribhūta}
\extramarks{Bi Pc 71}{Bi Pc 71}

\subsection*{Origin story }

At\marginnote{1.1} one time the Buddha was staying at \textsanskrit{Sāvatthī} in the Jeta Grove, \textsanskrit{Anāthapiṇḍika}’s Monastery. At that time the nuns were giving the full admission to unmarried girls less than twenty years old. They were unable to endure cold and heat; hunger and thirst; contact with horseflies, mosquitoes, wind, the burning sun, and creeping animals and insects; rude and unwelcome speech; and they were unable to bear up with bodily feelings that are painful, severe, sharp, and destructive of life. 

The\marginnote{1.5} nuns of few desires complained and criticized them, “How can nuns give the full admission to unmarried girls less than twelve years old?” … “Is it true, monks, that nuns do this?” 

“It’s\marginnote{1.8} true, Sir.” 

The\marginnote{1.9} Buddha rebuked them … “How can nuns do this? An unmarried girl less than twenty years old is unable to endure cold and heat; hunger and thirst; contact with horseflies, mosquitoes, wind, the burning sun, and creeping animals and insects; rude and unwelcome speech; and she’s unable to bear up with bodily feelings that are painful, severe, sharp, and destructive of life. But an unmarried girl who is twenty years old is able to endure these things. This will affect people’s confidence …” … “And, monks, the nuns should recite this training rule like this: 

\subsection*{Final ruling }

\scrule{‘If a nun gives the full admission to an unmarried girl who is less than twenty years old, she commits an offense entailing confession.’” }

\subsection*{Definitions }

\begin{description}%
\item[A: ] whoever … %
\item[Nun: ] … The nun who has been given the full ordination in unanimity by both Sanghas through a legal procedure consisting of one motion and three announcements that is irreversible and fit to stand—this sort of nun is meant in this case. %
\item[Who is less than twenty years old: ] who has not reached twenty years of age. %
\item[An unmarried girl: ] a novice nun is what is meant. %
\item[Gives the full admission: ] gives the full ordination. If, intending to give the full admission, she searches for a group, a teacher, a bowl, or a robe, or she establishes a monastery zone, she commits an offense of wrong conduct. After the motion, she commits an offense of wrong conduct.\footnote{The Pali just says \textit{\textsanskrit{dukkaṭa}}, without specifying that it is an \textit{\textsanskrit{āpatti}}, “an offense”. Yet elsewhere, such as at \href{https://suttacentral.net/pli-tv-bu-vb-ss10/en/brahmali\#2.65}{Bu Ss 10:2.65}, the \textit{\textsanskrit{dukkaṭa}} is annulled if you commit the full offense of \textit{\textsanskrit{saṅghādisesa}}. The implication is that in these contexts \textit{\textsanskrit{dukkaṭa}} should be read as \textit{\textsanskrit{āpatti} \textsanskrit{dukkaṭassa}}, “an offense of wrong conduct”. } After each of the first two announcements, she commits an offense of wrong conduct. When the last announcement is finished, the preceptor commits an offense entailing confession, and the group and the teacher commit an offense of wrong conduct. %
\end{description}

\subsection*{Permutations }

If\marginnote{2.16.1} the girl is less than twenty years old, and the nun perceives her as less, and she gives her the full admission, she commits an offense entailing confession. If the girl is less than twenty years old, but the nun is unsure of it, and she gives her the full admission, she commits an offense of wrong conduct. If the girl is less than twenty years old, but the nun perceives her as more, and she gives her the full admission, there is no offense. 

If\marginnote{2.19} the girl is more than twenty years old, but the nun perceives her as less, she commits an offense of wrong conduct. If the girl is more than twenty years old, but the nun is unsure of it, she commits an offense of wrong conduct. If the girl is more than twenty years old, and the nun perceives her as more, there is no offense. 

\subsection*{Non-offenses }

There\marginnote{2.22.1} is no offense: if she gives the full admission to a girl less than twenty years old, but she perceives her as more;  if she gives the full admission to a girl more than twenty years old, and she perceives her as more;  if she is insane;  if she is the first offender. 

\scendsutta{The first training rule is finished. }

%
\section*{{\suttatitleacronym Bi Pc 72}{\suttatitletranslation The second training rule on unmarried girls }{\suttatitleroot Paripuṇṇavīsativassa-asikkhita-kumāribhūta}}
\addcontentsline{toc}{section}{\tocacronym{Bi Pc 72} \toctranslation{The second training rule on unmarried girls } \tocroot{Paripuṇṇavīsativassa-asikkhita-kumāribhūta}}
\markboth{The second training rule on unmarried girls }{Paripuṇṇavīsativassa-asikkhita-kumāribhūta}
\extramarks{Bi Pc 72}{Bi Pc 72}

\subsection*{Origin story }

At\marginnote{1.1} one time the Buddha was staying at \textsanskrit{Sāvatthī} in the Jeta Grove, \textsanskrit{Anāthapiṇḍika}’s Monastery. At that time the nuns were giving the full admission to unmarried girls who were more than twenty years old, but who had not trained for two years in the six rules. They were ignorant and incompetent, and they did not know what was allowable and what was not. 

The\marginnote{1.5} nuns of few desires complained and criticized them, “How can nuns give the full admission to unmarried girls who are more than twenty years old, but who haven’t trained for two years in the six rules?” … “Is it true, monks, that nuns do this?” 

“It’s\marginnote{1.8} true, Sir.” 

The\marginnote{1.9} Buddha rebuked them … “How can nuns do this? This will affect people’s confidence …” After rebuking them … he gave a teaching and addressed the monks: 

“Monks,\marginnote{1.14} approval is required for an unmarried girl who is eighteen years old to train in the six rules for two years. 

And\marginnote{1.15} the approval is to be given like this. 

After\marginnote{1.16} approaching the Sangha of nuns, that unmarried girl who is eighteen years old should arrange her upper robe over one shoulder and pay respect at the feet of the nuns. She should then squat on her heels, raise her joined palms, and say: 

‘Venerables,\marginnote{1.17} I, so-and-so, an unmarried girl who is eighteen years old, am training under Venerable so-and-so. I ask the Sangha for approval to train for two years in the six rules.’\footnote{\textit{\textsanskrit{Itthannāmāya} \textsanskrit{ayyāya}} is probably the genitive case, which would literally mean that the trainee nun “belongs” to her teacher. I take this to be an indirect reference to being her student. } 

And\marginnote{1.18} she should ask a second and a third time. 

A\marginnote{1.20} competent and capable nun should then inform the Sangha: 

‘Please,\marginnote{1.21} Venerables, I ask the Sangha to listen. This unmarried girl so-and-so, who is eighteen years old and training under Venerable so-and-so, is asking the Sangha for approval to train in the six rules for two years. If the Sangha is ready, it should give approval to the unmarried girl so-and-so who is eighteen years old to train in the six rules for two years. This is the motion. 

Please,\marginnote{1.25} Venerables, I ask the Sangha to listen. This unmarried girl so-and-so, who is eighteen years old and training under Venerable so-and-so, is asking the Sangha for approval to train in the six rules for two years. The Sangha gives approval to the unmarried girl so-and-so who is eighteen years old to train in the six rules for two years. Any nun who approves of giving approval to the unmarried girl so-and-so who is eighteen years old to train in the six rules for two years should remain silent. Any nun who doesn’t approve should speak up. 

The\marginnote{1.30} Sangha has given approval to the unmarried girl so-and-so who is eighteen years old to train in the six rules for two years. The Sangha approves and is therefore silent. I’ll remember it thus.’ 

That\marginnote{1.32} unmarried girl who is eighteen years old should be told to say this: ‘I undertake to abstain from killing living beings for two years without transgression. … I undertake to abstain from eating at the wrong time for two years without transgression.’” 

Then,\marginnote{1.35} after rebuking those nuns in many ways, the Buddha spoke in dispraise of being difficult to support … “And, monks, the nuns should recite this training rule like this: 

\subsection*{Final ruling }

\scrule{‘If a nun gives the full admission to an unmarried girl who is more than twenty years old, but who has not trained in the six rules for two years, she commits an offense entailing confession.’” }

\subsection*{Definitions }

\begin{description}%
\item[A: ] whoever … %
\item[Nun: ] … The nun who has been given the full ordination in unanimity by both Sanghas through a legal procedure consisting of one motion and three announcements that is irreversible and fit to stand—this sort of nun is meant in this case. %
\item[Who is more than twenty years old: ] who has reached twenty years of age. %
\item[An unmarried girl: ] a novice nun is what is meant. %
\item[Two years: ] two twelve-month periods. %
\item[Who has not trained: ] the training has not been given to her, or the training has been given to her, but she has failed in it. %
\item[Gives the full admission: ] gives the full ordination. If, intending to give the full admission, she searches for a group, a teacher, a bowl, or a robe, or she establishes a monastery zone, she commits an offense of wrong conduct. After the motion, she commits an offense of wrong conduct.\footnote{The Pali just says \textit{\textsanskrit{dukkaṭa}}, without specifying that it is an \textit{\textsanskrit{āpatti}}, “an offense”. Yet elsewhere, such as at \href{https://suttacentral.net/pli-tv-bu-vb-ss10/en/brahmali\#2.65}{Bu Ss 10:2.65}, the \textit{\textsanskrit{dukkaṭa}} is annulled if you commit the full offense of \textit{\textsanskrit{saṅghādisesa}}. The implication is that in these contexts \textit{\textsanskrit{dukkaṭa}} should be read as \textit{\textsanskrit{āpatti} \textsanskrit{dukkaṭassa}}, “an offense of wrong conduct”. } After each of the first two announcements, she commits an offense of wrong conduct. When the last announcement is finished, the preceptor commits an offense entailing confession, and the group and the teacher commit an offense of wrong conduct. %
\end{description}

\subsection*{Permutations }

If\marginnote{2.20.1} it is a legitimate legal procedure, and she perceives it as such, and she gives the full admission, she commits an offense entailing confession. If it is a legitimate legal procedure, but she is unsure of it, and she gives the full admission, she commits an offense entailing confession. If it is a legitimate legal procedure, but she perceives it as illegitimate, and she gives the full admission, she commits an offense entailing confession. 

If\marginnote{2.23} it is an illegitimate legal procedure, but she perceives it as legitimate, she commits an offense of wrong conduct. If it is an illegitimate legal procedure, but she is unsure of it, she commits an offense of wrong conduct. If it is an illegitimate legal procedure, and she perceives it as such, she commits an offense of wrong conduct. 

\subsection*{Non-offenses }

There\marginnote{2.26.1} is no offense: if she gives the full admission to an unmarried girl who is more than twenty years old, and who has trained in the six rules for two years;  if she is insane;  if she is the first offender. 

\scendsutta{The second training rule is finished. }

%
\section*{{\suttatitleacronym Bi Pc 73}{\suttatitletranslation The third training rule on unmarried girls }{\suttatitleroot Paripuṇṇavīsativassa-sikkhita-kumāribhūta-asammata}}
\addcontentsline{toc}{section}{\tocacronym{Bi Pc 73} \toctranslation{The third training rule on unmarried girls } \tocroot{Paripuṇṇavīsativassa-sikkhita-kumāribhūta-asammata}}
\markboth{The third training rule on unmarried girls }{Paripuṇṇavīsativassa-sikkhita-kumāribhūta-asammata}
\extramarks{Bi Pc 73}{Bi Pc 73}

\subsection*{Origin story }

At\marginnote{1.1} one time the Buddha was staying at \textsanskrit{Sāvatthī} in the Jeta Grove, \textsanskrit{Anāthapiṇḍika}’s Monastery. At that time the nuns were giving the full admission to unmarried girls who were more than twenty years old and who had trained in the six rules for two years, but who had not been approved by the Sangha. The nuns said this: “Come, trainee nuns, find out about this,” “Give this,” “Bring this,” “There’s need for this,” or “Make this allowable.” But they replied, “Venerables, we’re not trainee nuns. We’re nuns.” 

The\marginnote{1.7} nuns of few desires complained and criticized them, “How can nuns give the full admission to unmarried girls who are more than twenty years old and who have trained for two years in the six rules, but who haven’t been approved by the Sangha?” … “Is it true, monks, that nuns do this?” 

“It’s\marginnote{1.10} true, Sir.” 

The\marginnote{1.11} Buddha rebuked them … “How can nuns do this? This will affect people’s confidence …” After rebuking them … he gave a teaching and addressed the monks: 

“Monks,\marginnote{1.16} approval is required for the full admission of an unmarried girl who is more than twenty years old and who has trained for two years in the six rules. 

And\marginnote{1.17} the approval is to be given like this. 

After\marginnote{1.18} approaching the Sangha of nuns, that unmarried girl who is more than twenty years old should arrange her upper robe over one shoulder and pay respect at the feet of the nuns. She should then squat on her heels, raise her joined palms, and say: 

‘Venerables,\marginnote{1.19} I, the unmarried girl so-and-so, who is more than twenty years old and who has trained for two years in the six rules under Venerable so-and-so, ask the Sangha for approval to be fully admitted.’ 

And\marginnote{1.20} she should ask a second and a third time. 

A\marginnote{1.22} competent and capable nun should then inform the Sangha: 

‘Please,\marginnote{1.23} Venerables, I ask the Sangha to listen. This unmarried girl so-and-so, who is more than twenty years old and who has trained for two years in the six rules under Venerable so-and-so, is asking the Sangha for approval to be fully admitted. If the Sangha is ready, it should give approval for the unmarried girl so-and-so, who is more than twenty years old and who has trained for two years in the six rules under Venerable so-and-so, to be fully admitted. This is the motion. 

Please,\marginnote{1.27} Venerables, I ask the Sangha to listen. This unmarried girl so-and-so, who is more than twenty years old and who has trained for two years in the six rules under Venerable so-and-so, is asking the Sangha for approval to be fully admitted. The Sangha gives approval for the unmarried girl so-and-so, who is more than twenty years old and who has trained for two years in the six rules under Venerable so-and-so, to be fully admitted. Any nun who approves of giving approval for the unmarried girl so-and-so, who is more than twenty years old and who has trained for two years in the six rules under Venerable so-and-so, to be fully admitted should remain silent. Any nun who doesn’t approve should speak up. 

The\marginnote{1.32} Sangha has given approval for the unmarried girl so-and-so, who is more than twenty years old and who has trained for two years in the six rules under Venerable so-and-so, to be fully admitted. The Sangha approves and is therefore silent. I’ll remember it thus.’” 

Then,\marginnote{1.34} after rebuking those nuns in many ways, the Buddha spoke in dispraise of being difficult to support … “And, monks, the nuns should recite this training rule like this: 

\subsection*{Final ruling }

\scrule{‘If a nun gives the full admission to an unmarried girl who is more than twenty years old and who has trained for two years in the six rules, but who has not been approved by the Sangha, she commits an offense entailing confession.’” }

\subsection*{Definitions }

\begin{description}%
\item[A: ] whoever … %
\item[Nun: ] … The nun who has been given the full ordination in unanimity by both Sanghas through a legal procedure consisting of one motion and three announcements that is irreversible and fit to stand—this sort of nun is meant in this case. %
\item[Who is more than twenty years old: ] who has reached twenty years of age. %
\item[An unmarried girl: ] a novice nun is what is meant. %
\item[Two years: ] two twelve-month periods. %
\item[Who has trained: ] who has trained in the six rules. %
\item[Who has not been approved: ] approval to be fully admitted has not been given though a legal procedure consisting of one motion and one announcement. %
\item[Gives the full admission: ] gives the full ordination. If, intending to give the full admission, she searches for a group, a teacher, a bowl, or a robe, or she establishes a monastery zone, she commits an offense of wrong conduct. After the motion, she commits an offense of wrong conduct.\footnote{The Pali just says \textit{\textsanskrit{dukkaṭa}}, without specifying that it is an \textit{\textsanskrit{āpatti}}, “an offense”. Yet elsewhere, such as at \href{https://suttacentral.net/pli-tv-bu-vb-ss10/en/brahmali\#2.65}{Bu Ss 10:2.65}, the \textit{\textsanskrit{dukkaṭa}} is annulled if you commit the full offense of \textit{\textsanskrit{saṅghādisesa}}. The implication is that in these contexts \textit{\textsanskrit{dukkaṭa}} should be read as \textit{\textsanskrit{āpatti} \textsanskrit{dukkaṭassa}}, “an offense of wrong conduct”. } After each of the first two announcements, she commits an offense of wrong conduct. When the last announcement is finished, the preceptor commits an offense entailing confession, and the group and the teacher commit an offense of wrong conduct. %
\end{description}

\subsection*{Permutations }

If\marginnote{2.22.1} it is a legitimate legal procedure, and she perceives it as such, and she gives the full admission, she commits an offense entailing confession. If it is a legitimate legal procedure, but she is unsure of it, and she gives the full admission, she commits an offense entailing confession. If it is a legitimate legal procedure, but she perceives it as illegitimate, and she gives the full admission, she commits an offense entailing confession. 

If\marginnote{2.25} it is an illegitimate legal procedure, but she perceives it as legitimate, she commits an offense of wrong conduct. If it is an illegitimate legal procedure, but she is unsure of it, she commits an offense of wrong conduct. If it is an illegitimate legal procedure, and she perceives it as such, she commits an offense of wrong conduct. 

\subsection*{Non-offenses }

There\marginnote{2.28.1} is no offense: if she gives the full admission to an unmarried girl who is more than twenty years old, who has trained in the six rules for two years, and who has been approved by the Sangha;  if she is insane;  if she is the first offender. 

\scendsutta{The third training rule is finished. }

%
\section*{{\suttatitleacronym Bi Pc 74}{\suttatitletranslation The training rule on less than twelve years }{\suttatitleroot Ūnadvādasavassa}}
\addcontentsline{toc}{section}{\tocacronym{Bi Pc 74} \toctranslation{The training rule on less than twelve years } \tocroot{Ūnadvādasavassa}}
\markboth{The training rule on less than twelve years }{Ūnadvādasavassa}
\extramarks{Bi Pc 74}{Bi Pc 74}

\subsection*{Origin story }

At\marginnote{1.1} one time the Buddha was staying at \textsanskrit{Sāvatthī} in the Jeta Grove, \textsanskrit{Anāthapiṇḍika}’s Monastery. At that time nuns who had less than twelve years of seniority were giving the full admission. They were ignorant and incompetent, and they did not know what was allowable and what was not. And their students too were ignorant and incompetent, and they too did not know what was allowable and what was not. 

The\marginnote{1.6} nuns of few desires complained and criticized them, “How can nuns who have less than twelve years of seniority give the full admission?” … “Is it true, monks, that nuns do this?” 

“It’s\marginnote{1.9} true, Sir.” 

The\marginnote{1.10} Buddha rebuked them … “How can nuns do this? This will affect people’s confidence …” … “And, monks, the nuns should recite this training rule like this: 

\subsection*{Final ruling }

\scrule{‘If a nun who has less than twelve years of seniority gives the full admission, she commits an offense entailing confession.’” }

\subsection*{Definitions }

\begin{description}%
\item[A: ] whoever … %
\item[Nun: ] … The nun who has been given the full ordination in unanimity by both Sanghas through a legal procedure consisting of one motion and three announcements that is irreversible and fit to stand—this sort of nun is meant in this case. %
\item[Who has less than twelve years of seniority: ] who has not reached twelve years of seniority. %
\item[Gives the full admission: ] gives the full ordination. If, intending to give the full admission, she searches for a group, a teacher, a bowl, or a robe, or she establishes a monastery zone, she commits an offense of wrong conduct. After the motion, she commits an offense of wrong conduct.\footnote{The Pali just says \textit{\textsanskrit{dukkaṭa}}, without specifying that it is an \textit{\textsanskrit{āpatti}}, “an offense”. Yet elsewhere, such as at \href{https://suttacentral.net/pli-tv-bu-vb-ss10/en/brahmali\#2.65}{Bu Ss 10:2.65}, the \textit{\textsanskrit{dukkaṭa}} is annulled if you commit the full offense of \textit{\textsanskrit{saṅghādisesa}}. The implication is that in these contexts \textit{\textsanskrit{dukkaṭa}} should be read as \textit{\textsanskrit{āpatti} \textsanskrit{dukkaṭassa}}, “an offense of wrong conduct”. } After each of the first two announcements, she commits an offense of wrong conduct. When the last announcement is finished, the preceptor commits an offense entailing confession, and the group and the teacher commit an offense of wrong conduct. %
\end{description}

\subsection*{Non-offenses }

There\marginnote{2.2.1} is no offense: if one who has twelve years of seniority gives the full admission;  if she is insane;  if she is the first offender. 

\scendsutta{The fourth training rule is finished. }

%
\section*{{\suttatitleacronym Bi Pc 75}{\suttatitletranslation The training rule on twelve years of seniority }{\suttatitleroot Paripuṇṇadvādasavassa-asammata}}
\addcontentsline{toc}{section}{\tocacronym{Bi Pc 75} \toctranslation{The training rule on twelve years of seniority } \tocroot{Paripuṇṇadvādasavassa-asammata}}
\markboth{The training rule on twelve years of seniority }{Paripuṇṇadvādasavassa-asammata}
\extramarks{Bi Pc 75}{Bi Pc 75}

\subsection*{Origin story }

At\marginnote{1.1} one time the Buddha was staying at \textsanskrit{Sāvatthī} in the Jeta Grove, \textsanskrit{Anāthapiṇḍika}’s Monastery. At that time nuns who had twelve years of seniority were giving the full admission without being approved by the Sangha. They were ignorant and incompetent, and they did not know what was allowable and what was not. And their students too were ignorant and incompetent, and they too did not know what was allowable and what was not. 

The\marginnote{1.7} nuns of few desires complained and criticized them, “How can nuns who have twelve years of seniority give the full admission without being approved by the Sangha?” … “Is it true, monks, that nuns do this?” 

“It’s\marginnote{1.10} true, Sir.” 

The\marginnote{1.11} Buddha rebuked them … “How can nuns do this? This will affect people’s confidence …” After rebuking them … he gave a teaching and addressed the monks: 

“Monks,\marginnote{1.16} approval is required for a nun who has twelve years of seniority to give the full admission. 

And\marginnote{1.17} the approval is to be given like this. 

After\marginnote{1.18} approaching the Sangha of nuns, that nun who has twelve years of seniority should arrange her upper robe over one shoulder and pay respect at the feet of the senior nuns. She should then squat on her heels, raise her joined palms, and say: 

‘Venerables,\marginnote{1.19} I, the nun so-and-so who has twelve years of seniority, ask the Sangha for approval to give the full admission.’ 

And\marginnote{1.20} she should ask a second and a third time. 

The\marginnote{1.22} Sangha should then decide whether that nun is competent and has a sense of conscience. 

\begin{itemize}%
\item If she is ignorant and shameless, approval should not be given. %
\item If she is ignorant but has a sense of conscience, approval should not be given. %
\item If she is competent but shameless, approval should not be given. %
\item If she is competent and has a sense of conscience, approval should be given. %
\end{itemize}

And\marginnote{1.27} it is to be given like this. A competent and capable nun should inform the Sangha: 

‘Please,\marginnote{1.29} Venerables, I ask the Sangha to listen. The nun so-and-so, who has twelve years of seniority, is asking the Sangha for approval to give the full admission. If the Sangha is ready, it should give approval to nun so-and-so, who has twelve years of seniority, to give the full admission. This is the motion. 

Please,\marginnote{1.33} Venerables, I ask the Sangha to listen. The nun so-and-so, who has twelve years of seniority, is asking the Sangha for approval to give the full admission. The Sangha gives approval to nun so-and-so, who has twelve years of seniority, to give the full admission. Any nun who approves of giving approval to nun so-and-so, who has twelve years of seniority, to give the full admission should remain silent. Any nun who doesn’t approve should speak up. 

The\marginnote{1.38} Sangha has given approval to nun so-and-so, who has twelve years of seniority, to give the full admission. The Sangha approves and is therefore silent. I’ll remember it thus.’” 

Then,\marginnote{1.40} after rebuking those nuns in many ways, the Buddha spoke in dispraise of being difficult to support … “And, monks, the nuns should recite this training rule like this: 

\subsection*{Final ruling }

\scrule{‘If a nun who has twelve years of seniority gives the full admission without approval from the Sangha, she commits an offense entailing confession.’” }

\subsection*{Definitions }

\begin{description}%
\item[A: ] whoever … %
\item[Nun: ] … The nun who has been given the full ordination in unanimity by both Sanghas through a legal procedure consisting of one motion and three announcements that is irreversible and fit to stand—this sort of nun is meant in this case. %
\item[Who has twelve years of seniority: ] who has reached twelve years of seniority. %
\item[Without approval: ] approval to give the full admission has not been given though a legal procedure consisting of one motion and one announcement. %
\item[Gives the full admission: ] gives the full ordination. If, intending to give the full admission, she searches for a group, a teacher, a bowl, or a robe, or she establishes a monastery zone, she commits an offense of wrong conduct. After the motion, she commits an offense of wrong conduct.\footnote{The Pali just says \textit{\textsanskrit{dukkaṭa}}, without specifying that it is an \textit{\textsanskrit{āpatti}}, “an offense”. Yet elsewhere, such as at \href{https://suttacentral.net/pli-tv-bu-vb-ss10/en/brahmali\#2.65}{Bu Ss 10:2.65}, the \textit{\textsanskrit{dukkaṭa}} is annulled if you commit the full offense of \textit{\textsanskrit{saṅghādisesa}}. The implication is that in these contexts \textit{\textsanskrit{dukkaṭa}} should be read as \textit{\textsanskrit{āpatti} \textsanskrit{dukkaṭassa}}, “an offense of wrong conduct”. } After each of the first two announcements, she commits an offense of wrong conduct. When the last announcement is finished, the preceptor commits an offense entailing confession, and the group and the teacher commit an offense of wrong conduct. %
\end{description}

\subsection*{Permutations }

If\marginnote{2.16.1} it is a legitimate legal procedure, and she perceives it as such, and she gives the full admission, she commits an offense entailing confession. If it is a legitimate legal procedure, but she is unsure of it, and she gives the full admission, she commits an offense entailing confession. If it is a legitimate legal procedure, but she perceives it as illegitimate, and she gives the full admission, she commits an offense entailing confession. 

If\marginnote{2.19} it is an illegitimate legal procedure, but she perceives it as legitimate, she commits an offense of wrong conduct. If it is an illegitimate legal procedure, but she is unsure of it, she commits an offense of wrong conduct. If it is an illegitimate legal procedure, and she perceives it as such, she commits an offense of wrong conduct. 

\subsection*{Non-offenses }

There\marginnote{2.22.1} is no offense: if she has twelve years of seniority, has been approved by the Sangha, and then gives the full admission;  if she is insane;  if she is the first offender. 

\scendsutta{The fifth training rule is finished. }

%
\section*{{\suttatitleacronym Bi Pc 76}{\suttatitletranslation The training rule on criticizing }{\suttatitleroot Khiyyanadhamma}}
\addcontentsline{toc}{section}{\tocacronym{Bi Pc 76} \toctranslation{The training rule on criticizing } \tocroot{Khiyyanadhamma}}
\markboth{The training rule on criticizing }{Khiyyanadhamma}
\extramarks{Bi Pc 76}{Bi Pc 76}

\subsection*{Origin story }

At\marginnote{1.1} one time when the Buddha was staying at \textsanskrit{Sāvatthī} in \textsanskrit{Anāthapiṇḍika}’s Monastery, the nun \textsanskrit{Caṇḍakāḷī} approached the Sangha of nuns and asked for approval to give the full admission. The Sangha of nuns decided that she should not, and \textsanskrit{Caṇḍakāḷī} consented. 

Soon\marginnote{1.6} afterwards the Sangha of nuns gave approval to other nuns to give the full admission. \textsanskrit{Caṇḍakāḷī} complained and criticized it, “So it seems I’m ignorant and shameless, since the Sangha gives approval to other nuns, but not to me.” 

The\marginnote{1.10} nuns of few desires complained and criticized her, “How could Venerable \textsanskrit{Caṇḍakāḷī} consent to not being approved to give the full admission, and then criticize it afterwards?” … “Is it true, monks, that the nun \textsanskrit{Caṇḍakāḷī} did this?” 

“It’s\marginnote{1.14} true, Sir.” 

The\marginnote{1.15} Buddha rebuked her … “How could the nun \textsanskrit{Caṇḍakāḷī} do this? This will affect people’s confidence …” … “And, monks, the nuns should recite this training rule like this: 

\subsection*{Final ruling }

\scrule{‘If a nun is told, “Venerable, you’ve given enough full admissions for now,” and she consents, saying, “Fine,” but then criticizes it afterwards, she commits an offense entailing confession.’” }

\subsection*{Definitions }

\begin{description}%
\item[A: ] whoever … %
\item[Nun: ] … The nun who has been given the full ordination in unanimity by both Sanghas through a legal procedure consisting of one motion and three announcements that is irreversible and fit to stand—this sort of nun is meant in this case. %
\item[“Venerable, you’ve given enough full admissions for now”: ] “Venerable, you’ve given enough full ordinations for now.” If she consents, saying, “Fine,” but then criticizes it afterwards, she commits an offense entailing confession. %
\end{description}

\subsection*{Non-offenses }

There\marginnote{2.2.1} is no offense: if she criticizes one who regularly acts out of favoritism, ill will, confusion, or fear;  if she is insane;  if she is the first offender. 

\scendsutta{The sixth training rule is finished. }

%
\section*{{\suttatitleacronym Bi Pc 77}{\suttatitletranslation The training rule on not giving the full admission to trainee nuns }{\suttatitleroot Cīvaradāna-sikkhamānana-vuṭṭhāpana}}
\addcontentsline{toc}{section}{\tocacronym{Bi Pc 77} \toctranslation{The training rule on not giving the full admission to trainee nuns } \tocroot{Cīvaradāna-sikkhamānana-vuṭṭhāpana}}
\markboth{The training rule on not giving the full admission to trainee nuns }{Cīvaradāna-sikkhamānana-vuṭṭhāpana}
\extramarks{Bi Pc 77}{Bi Pc 77}

\subsection*{Origin story }

At\marginnote{1.1} one time when the Buddha was staying at \textsanskrit{Sāvatthī} in \textsanskrit{Anāthapiṇḍika}’s Monastery, a trainee nun went to the nun \textsanskrit{Thullanandā} and asked her for the full ordination. \textsanskrit{Thullanandā} told her, “If you give me a robe, Venerable, I’ll give you the full admission.” But she neither gave her the full admission nor made any effort to have her fully admitted. 

That\marginnote{1.5} trainee nun told the nuns what had happened. The nuns of few desires complained and criticized her, “How could Venerable \textsanskrit{Thullanandā} say that to a trainee nun and then neither give her the full admission nor make any effort to have her fully admitted?” … “Is it true, monks, that the nun \textsanskrit{Thullanandā} did this?” 

“It’s\marginnote{1.11} true, Sir.” 

The\marginnote{1.12} Buddha rebuked her … “How could the nun \textsanskrit{Thullanandā} act in this way? This will affect people’s confidence …” … “And, monks, the nuns should recite this training rule like this: 

\subsection*{Final ruling }

\scrule{‘If a nun tells a trainee nun, “If you give me a robe, Venerable, I’ll give you the full admission,” but she then neither gives her the full admission nor makes any effort to have her fully admitted, then, if there were no obstacles, she commits an offense entailing confession.’” }

\subsection*{Definitions }

\begin{description}%
\item[A: ] whoever … %
\item[Nun: ] … The nun who has been given the full ordination in unanimity by both Sanghas through a legal procedure consisting of one motion and three announcements that is irreversible and fit to stand—this sort of nun is meant in this case. %
\item[A trainee nun: ] one who has trained for two years in the six rules. %
\item[If you give me a robe, Venerable, I’ll give you the full admission: ] I’ll give you the full ordination. %
\item[Then, if there were no obstacles: ] when there is no obstacle. %
\item[She neither gives her the full admission: ] does not herself give her the full admission. %
\item[Nor makes any effort to have her fully admitted: ] does not ask anyone else to give her the full ordination. %
\end{description}

If\marginnote{2.1.15} she thinks, “I’ll neither give her the full admission nor make any effort to have her fully admitted,” then by the mere fact of abandoning her duty, she commits an offense entailing confession. 

\subsection*{Non-offenses }

There\marginnote{2.2.1} is no offense: if there is an obstacle;  if she searches for someone to give her the full ordination, but is unable to find anyone;  if she is sick;  if there is an emergency;  if she is insane;  if she is the first offender. 

\scendsutta{The seventh training rule is finished. }

%
\section*{{\suttatitleacronym Bi Pc 78}{\suttatitletranslation The second training rule on not giving the full admission to trainee nuns }{\suttatitleroot Anubandha-sikkhamānana-vuṭṭhāpana}}
\addcontentsline{toc}{section}{\tocacronym{Bi Pc 78} \toctranslation{The second training rule on not giving the full admission to trainee nuns } \tocroot{Anubandha-sikkhamānana-vuṭṭhāpana}}
\markboth{The second training rule on not giving the full admission to trainee nuns }{Anubandha-sikkhamānana-vuṭṭhāpana}
\extramarks{Bi Pc 78}{Bi Pc 78}

\subsection*{Origin story }

At\marginnote{1.1} one time when the Buddha was staying at \textsanskrit{Sāvatthī} in \textsanskrit{Anāthapiṇḍika}’s Monastery, a trainee nun went to the nun \textsanskrit{Thullanandā} and asked for the full ordination. \textsanskrit{Thullanandā} told her, “If you follow me for two years, Venerable, I’ll give you the full admission.” But she neither gave her the full admission nor made any effort to have her fully admitted. 

That\marginnote{1.4} trainee nun told the nuns what had happened. The nuns of few desires complained and criticized her, “How could Venerable \textsanskrit{Thullanandā} say that to a trainee nun and then neither give her the full admission nor make any effort to have her fully admitted?” … “Is it true, monks, that the nun \textsanskrit{Thullanandā} did this?” 

“It’s\marginnote{1.10} true, Sir.” 

The\marginnote{1.11} Buddha rebuked her … “How could the nun \textsanskrit{Thullanandā} act in this way? This will affect people’s confidence …” … “And, monks, the nuns should recite this training rule like this: 

\subsection*{Final ruling }

\scrule{‘If a nun tells a trainee nun, “If you follow me for two years, Venerable, I’ll give you the full admission,” but she then neither gives her the full admission nor makes any effort to have her fully admitted, then, if there were no obstacles, she commits an offense entailing confession.’” }

\subsection*{Definitions }

\begin{description}%
\item[A: ] whoever … %
\item[Nun: ] … The nun who has been given the full ordination in unanimity by both Sanghas through a legal procedure consisting of one motion and three announcements that is irreversible and fit to stand—this sort of nun is meant in this case. %
\item[A trainee nun: ] one who has trained for two years in the six rules. %
\item[“If you follow me for two years, Venerable”: ] if you attend on me for two twelve-month periods. %
\item[“I’ll give you the full admission”: ] I’ll give you the full ordination. %
\item[Then, if there were no obstacles: ] when there is no obstacle. %
\item[She neither gives her the full admission: ] she does not herself give her the full admission. %
\item[Nor makes any effort to have her fully admitted: ] she does not ask anyone else to give her the full admission. %
\end{description}

If\marginnote{2.17} she thinks, “I’ll neither give her the full admission nor make any effort to have her fully admitted,” then by the mere fact of abandoning her duty, she commits an offense entailing confession. 

\subsection*{Non-offenses }

There\marginnote{2.18.1} is no offense: if there is an obstacle;  if she searches for someone to give her the full ordination, but is unable to find anyone;  if she is sick;  if there is an emergency;  if she is insane;  if she is the first offender. 

\scendsutta{The eighth training rule is finished. }

%
\section*{{\suttatitleacronym Bi Pc 79}{\suttatitletranslation The training rule on one who is difficult to live with }{\suttatitleroot Sokāvāsa}}
\addcontentsline{toc}{section}{\tocacronym{Bi Pc 79} \toctranslation{The training rule on one who is difficult to live with } \tocroot{Sokāvāsa}}
\markboth{The training rule on one who is difficult to live with }{Sokāvāsa}
\extramarks{Bi Pc 79}{Bi Pc 79}

\subsection*{Origin story }

At\marginnote{1.1} one time the Buddha was staying at \textsanskrit{Sāvatthī} in the Jeta Grove, \textsanskrit{Anāthapiṇḍika}’s Monastery. At that time the nun \textsanskrit{Thullanandā} gave the full admission to the trainee nun \textsanskrit{Caṇḍakāḷī}, who was socializing with men and boys, and who was temperamental and difficult to live with. 

The\marginnote{1.3} nuns of few desires complained and criticized her, “How could Venerable \textsanskrit{Thullanandā} give the full admission to the trainee nun \textsanskrit{Caṇḍakāḷī}, who is socializing with men and boys, and who is temperamental and difficult to live with?” … “Is it true, monks, that the nun \textsanskrit{Thullanandā} did this?” 

“It’s\marginnote{1.6} true, Sir.” 

The\marginnote{1.7} Buddha rebuked her … “How could the nun \textsanskrit{Thullanandā} do this? This will affect people’s confidence …” … “And, monks, the nuns should recite this training rule like this: 

\subsection*{Final ruling }

\scrule{‘If a nun gives the full admission to a trainee nun who is socializing with men and boys and who is temperamental and difficult to live with, she commits an offense entailing confession.’” }

\subsection*{Definitions }

\begin{description}%
\item[A: ] whoever … %
\item[Nun: ] … The nun who has been given the full ordination in unanimity by both Sanghas through a legal procedure consisting of one motion and three announcements that is irreversible and fit to stand—this sort of nun is meant in this case. %
\item[Men: ] those who have reached twenty years of age. %
\item[Boys: ] those who have not reached twenty years of age. %
\item[Socializing: ] she socializes with improper bodily and verbal actions. %
\item[Temperamental: ] angry is what is meant. %
\item[Difficult to live with: ] she causes suffering to others, and grieves herself. %
\item[A trainee nun: ] one who has trained for two years in the six rules. %
\item[Gives the full admission: ] gives the full ordination. If, intending to give the full admission, she searches for a group, a teacher, a bowl, or a robe, or she establishes a monastery zone, she commits an offense of wrong conduct. After the motion, she commits an offense of wrong conduct.\footnote{The Pali just says \textit{\textsanskrit{dukkaṭa}}, without specifying that it is an \textit{\textsanskrit{āpatti}}, “an offense”. Yet elsewhere, such as at \href{https://suttacentral.net/pli-tv-bu-vb-ss10/en/brahmali\#2.65}{Bu Ss 10:2.65}, the \textit{\textsanskrit{dukkaṭa}} is annulled if you commit the full offense of \textit{\textsanskrit{saṅghādisesa}}. The implication is that in these contexts \textit{\textsanskrit{dukkaṭa}} should be read as \textit{\textsanskrit{āpatti} \textsanskrit{dukkaṭassa}}, “an offense of wrong conduct”. } After each of the first two announcements, she commits an offense of wrong conduct. When the last announcement is finished, the preceptor commits an offense entailing confession, and the group and the teacher commit an offense of wrong conduct. %
\end{description}

\subsection*{Non-offenses }

There\marginnote{2.2.1} is no offense: if she gives her the full admission without knowing what she is like;  if she is insane;  if she is the first offender. 

\scendsutta{The ninth training rule is finished. }

%
\section*{{\suttatitleacronym Bi Pc 80}{\suttatitletranslation The training rule on lack of permission }{\suttatitleroot Ananuññāta}}
\addcontentsline{toc}{section}{\tocacronym{Bi Pc 80} \toctranslation{The training rule on lack of permission } \tocroot{Ananuññāta}}
\markboth{The training rule on lack of permission }{Ananuññāta}
\extramarks{Bi Pc 80}{Bi Pc 80}

\subsection*{Origin story }

At\marginnote{1.1} one time the Buddha was staying at \textsanskrit{Sāvatthī} in the Jeta Grove, \textsanskrit{Anāthapiṇḍika}’s Monastery. At that time the nun \textsanskrit{Thullanandā} gave the full admission to a trainee nun who had not been given permission by her parents and her husband.\footnote{According to DOP, the combination \textit{na} + \textit{pi} should be understood as “nor”. In the present case the \textit{na} is represented by the negation \textit{an} in \textit{\textsanskrit{ananuññāta}}. We then have a neither-nor sentence, which makes it fit with the rule below. Reading the \textit{pi} as “and” creates a discrepancy with the rule. } They complained and criticized her, “How could Venerable \textsanskrit{Thullanandā} give the full admission to that trainee nun without our permission?” 

The\marginnote{1.5} nuns heard the complaints of the parents and the husband. The nuns of few desires complained and criticized her, “How could Venerable \textsanskrit{Thullanandā} give the full admission to a trainee nun who doesn’t have permission from her parents and her husband?” … “Is it true, monks, that the nun \textsanskrit{Thullanandā} did this?” 

“It’s\marginnote{1.9} true, Sir.” 

The\marginnote{1.10} Buddha rebuked her … “How could the nun \textsanskrit{Thullanandā} do this? This will affect people’s confidence …” … “And, monks, the nuns should recite this training rule like this: 

\subsection*{Final ruling }

\scrule{‘If a nun gives the full admission to a trainee nun who has not been given permission by her parents or her husband, she commits an offense entailing confession.’”\footnote{It is often understood that a woman needs permission from both her parents and her husband to ordain as a \textit{\textsanskrit{bhikkhunī}}. Yet the rule uses the conjunction \textit{\textsanskrit{vā}}, which can only reasonably be rendered as “or”. } }

\subsection*{Definitions }

\begin{description}%
\item[A: ] whoever … %
\item[Nun: ] … The nun who has been given the full ordination in unanimity by both Sanghas through a legal procedure consisting of one motion and three announcements that is irreversible and fit to stand—this sort of nun is meant in this case. %
\item[Parents: ] the biological parents is what is meant. %
\item[Husband: ] he who possesses her. %
\item[Who has not been given permission: ] who has not asked permission. %
\item[A trainee nun: ] one who has trained for two years in the six rules. %
\item[Gives the full admission: ] gives the full ordination. If, intending to give the full admission, she searches for a group, a teacher, a bowl, or a robe, or she establishes a monastery zone, she commits an offense of wrong conduct. After the motion, she commits an offense of wrong conduct.\footnote{The Pali just says \textit{\textsanskrit{dukkaṭa}}, without specifying that it is an \textit{\textsanskrit{āpatti}}, “an offense”. Yet elsewhere, such as at \href{https://suttacentral.net/pli-tv-bu-vb-ss10/en/brahmali\#2.65}{Bu Ss 10:2.65}, the \textit{\textsanskrit{dukkaṭa}} is annulled if you commit the full offense of \textit{\textsanskrit{saṅghādisesa}}. The implication is that in these contexts \textit{\textsanskrit{dukkaṭa}} should be read as \textit{\textsanskrit{āpatti} \textsanskrit{dukkaṭassa}}, “an offense of wrong conduct”. } After each of the first two announcements, she commits an offense of wrong conduct. When the last announcement is finished, the preceptor commits an offense entailing confession, and the group and the teacher commit an offense of wrong conduct. %
\end{description}

\subsection*{Non-offenses }

There\marginnote{2.2.1} is no offense: if she gives her the full admission without knowing;  if she gives her the full admission after permission has been given;  if she is insane;  if she is the first offender. 

\scendsutta{The tenth training rule is finished. }

%
\section*{{\suttatitleacronym Bi Pc 81}{\suttatitletranslation The training rule on what is expired }{\suttatitleroot Pārivāsika}}
\addcontentsline{toc}{section}{\tocacronym{Bi Pc 81} \toctranslation{The training rule on what is expired } \tocroot{Pārivāsika}}
\markboth{The training rule on what is expired }{Pārivāsika}
\extramarks{Bi Pc 81}{Bi Pc 81}

\subsection*{Origin story }

At\marginnote{1.1} one time when the Buddha was staying at \textsanskrit{Sāvatthī} in \textsanskrit{Anāthapiṇḍika}’s Monastery, the nun \textsanskrit{Thullanandā} had gathered a group of senior monks, intending to give the full admission to a trainee nun. But after seeing much fresh and cooked food, she dismissed the senior monks, saying, “Venerables, I won’t give the full admission to the trainee nun just yet.” She then gathered Devadatta, \textsanskrit{Kokālika}, \textsanskrit{Kaṭamodakatissaka}, \textsanskrit{Khaṇḍadeviyāputta}, and Samuddadatta, and gave the full admission to that trainee nun. 

The\marginnote{1.5} nuns of few desires complained and criticized her, “How could Venerable \textsanskrit{Thullanandā} give the full admission to a trainee nun when the given consent had expired?” … “Is it true, monks, that the nun \textsanskrit{Thullanandā} did this?” 

“It’s\marginnote{1.8} true, Sir.” 

The\marginnote{1.9} Buddha rebuked her … “How could the nun \textsanskrit{Thullanandā} do this? This will affect people’s confidence …” … “And, monks, the nuns should recite this training rule like this: 

\subsection*{Final ruling }

\scrule{‘If, when a given consent has expired, a nun gives the full admission to a trainee nun, she commits an offense entailing confession.’” }

\subsection*{Definitions }

\begin{description}%
\item[A: ] whoever … %
\item[Nun: ] … The nun who has been given the full ordination in unanimity by both Sanghas through a legal procedure consisting of one motion and three announcements that is irreversible and fit to stand—this sort of nun is meant in this case. %
\item[When a given consent has expired: ] when the gathering has left.\footnote{In other words, consent is given for a specific meeting of the Sangha. When the meeting is over—that is, the gathering has left—the consent is no longer valid. See the discussion in Appendix on Individual \textsanskrit{Bhikkhunī} Rules. } %
\item[A trainee nun: ] one who has trained for two years in the six rules. %
\item[Gives the full admission: ] gives the full ordination. If, intending to give the full admission, she searches for a group, a teacher, a bowl, or a robe, or she establishes a monastery zone, she commits an offense of wrong conduct. After the motion, she commits an offense of wrong conduct.\footnote{The Pali just says \textit{\textsanskrit{dukkaṭa}}, without specifying that it is an \textit{\textsanskrit{āpatti}}, “an offense”. Yet elsewhere, such as at \href{https://suttacentral.net/pli-tv-bu-vb-ss10/en/brahmali\#2.65}{Bu Ss 10:2.65}, the \textit{\textsanskrit{dukkaṭa}} is annulled if you commit the full offense of \textit{\textsanskrit{saṅghādisesa}}. The implication is that in these contexts \textit{\textsanskrit{dukkaṭa}} should be read as \textit{\textsanskrit{āpatti} \textsanskrit{dukkaṭassa}}, “an offense of wrong conduct”. } After each of the first two announcements, she commits an offense of wrong conduct. When the last announcement is finished, the preceptor commits an offense entailing confession, and the group and the teacher commit an offense of wrong conduct. %
\end{description}

\subsection*{Non-offenses }

There\marginnote{2.2.1} is no offense: if she gives her the full admission while the gathering has not yet left;  if she is insane;  if she is the first offender. 

\scendsutta{The eleventh training rule is finished. }

%
\section*{{\suttatitleacronym Bi Pc 82}{\suttatitletranslation The training rule on every year }{\suttatitleroot Anuvassa}}
\addcontentsline{toc}{section}{\tocacronym{Bi Pc 82} \toctranslation{The training rule on every year } \tocroot{Anuvassa}}
\markboth{The training rule on every year }{Anuvassa}
\extramarks{Bi Pc 82}{Bi Pc 82}

\subsection*{Origin story }

At\marginnote{1.1} one time the Buddha was staying at \textsanskrit{Sāvatthī} in the Jeta Grove, \textsanskrit{Anāthapiṇḍika}’s Monastery. At that time the nuns were giving full admission every year, and the nuns’ dwelling place did not have sufficient capacity. People complained and criticized them, “How can the nuns give full admission every year, when the nuns’ dwelling place doesn’t have sufficient capacity?” 

The\marginnote{1.5} nuns heard the complaints of those people. The nuns of few desires complained and criticized them, “How can the nuns give full admission every year?” … “Is it true, monks, that the nuns do this?” 

“It’s\marginnote{1.9} true, Sir.” 

The\marginnote{1.10} Buddha rebuked them … “How can the nuns do this? This will affect people’s confidence …” … “And, monks, the nuns should recite this training rule like this: 

\subsection*{Final ruling }

\scrule{‘If a nun gives full admission every year, she commits an offense entailing confession.’” }

\subsection*{Definitions }

\begin{description}%
\item[A: ] whoever … %
\item[Nun: ] … The nun who has been given the full ordination in unanimity by both Sanghas through a legal procedure consisting of one motion and three announcements that is irreversible and fit to stand—this sort of nun is meant in this case. %
\item[Every year: ] every twelve-month period. %
\item[Gives full admission: ] gives full ordination. If, intending to give full admission, she searches for a group, a teacher, a bowl, or a robe, or she establishes a monastery zone, she commits an offense of wrong conduct. After the motion, she commits an offense of wrong conduct.\footnote{The Pali just says \textit{\textsanskrit{dukkaṭa}}, without specifying that it is an \textit{\textsanskrit{āpatti}}, “an offense”. Yet elsewhere, such as at \href{https://suttacentral.net/pli-tv-bu-vb-ss10/en/brahmali\#2.65}{Bu Ss 10:2.65}, the \textit{\textsanskrit{dukkaṭa}} is annulled if you commit the full offense of \textit{\textsanskrit{saṅghādisesa}}. The implication is that in these contexts \textit{\textsanskrit{dukkaṭa}} should be read as \textit{\textsanskrit{āpatti} \textsanskrit{dukkaṭassa}}, “an offense of wrong conduct”. } After each of the first two announcements, she commits an offense of wrong conduct. When the last announcement is finished, the preceptor commits an offense entailing confession, and the group and the teacher commit an offense of wrong conduct. %
\end{description}

\subsection*{Non-offenses }

There\marginnote{2.2.1} is no offense: if she gives full admission every other year;  if she is insane;  if she is the first offender. 

\scendsutta{The twelfth training rule is finished. }

%
\section*{{\suttatitleacronym Bi Pc 83}{\suttatitletranslation The training rule on one year }{\suttatitleroot Ekavassa}}
\addcontentsline{toc}{section}{\tocacronym{Bi Pc 83} \toctranslation{The training rule on one year } \tocroot{Ekavassa}}
\markboth{The training rule on one year }{Ekavassa}
\extramarks{Bi Pc 83}{Bi Pc 83}

\subsection*{Origin story }

At\marginnote{1.1} one time the Buddha was staying at \textsanskrit{Sāvatthī} in the Jeta Grove, \textsanskrit{Anāthapiṇḍika}’s Monastery. At that time the nuns were giving the full admission to two women per year, and the nuns’ dwelling place still did not have sufficient capacity. People complained and criticized them, “How can the nuns give the full admission to two women per year, when the nuns’ dwelling place still doesn’t have sufficient capacity?” 

The\marginnote{1.7} nuns heard the complaints of those people. The nuns of few desires complained and criticized them, “How can nuns give the full admission to two women per year?” … “Is it true, monks, that the nuns do this?” 

“It’s\marginnote{1.11} true, Sir.” 

The\marginnote{1.12} Buddha rebuked them … “How can nuns do this? This will affect people’s confidence …” … “And, monks, the nuns should recite this training rule like this: 

\subsection*{Final ruling }

\scrule{‘If a nun gives the full admission to two women in one year, she commits an offense entailing confession.’” }

\subsection*{Definitions }

\begin{description}%
\item[A: ] whoever … %
\item[Nun: ] … The nun who has been given the full ordination in unanimity by both Sanghas through a legal procedure consisting of one motion and three announcements that is irreversible and fit to stand—this sort of nun is meant in this case. %
\item[In one year: ] in one twelve-month period. %
\item[Gives the full admission to two women: ] gives the full ordination to two women. If, intending to give the full admission to two women, she searches for a group, a teacher, a bowl, or a robe, or she establishes a monastery zone, she commits an offense of wrong conduct. After the motion, she commits an offense of wrong conduct.\footnote{The Pali just says \textit{\textsanskrit{dukkaṭa}}, without specifying that it is an \textit{\textsanskrit{āpatti}}, “an offense”. Yet elsewhere, such as at \href{https://suttacentral.net/pli-tv-bu-vb-ss10/en/brahmali\#2.65}{Bu Ss 10:2.65}, the \textit{\textsanskrit{dukkaṭa}} is annulled if you commit the full offense of \textit{\textsanskrit{saṅghādisesa}}. The implication is that in these contexts \textit{\textsanskrit{dukkaṭa}} should be read as \textit{\textsanskrit{āpatti} \textsanskrit{dukkaṭassa}}, “an offense of wrong conduct”. } After each of the first two announcements, she commits an offense of wrong conduct. When the last announcement is finished, the preceptor commits an offense entailing confession, and the group and the teacher commit an offense of wrong conduct. %
\end{description}

\subsection*{Non-offenses }

There\marginnote{2.2.1} is no offense: if she gives the full admission to one woman every other year;  if she is insane;  if she is the first offender. 

\scendsutta{The thirteenth training rule is finished. }

\scendvagga{The eighth subchapter on unmarried girls is finished. }

%
\section*{{\suttatitleacronym Bi Pc 84}{\suttatitletranslation The training rule on sunshades and sandals }{\suttatitleroot Chattupāhana}}
\addcontentsline{toc}{section}{\tocacronym{Bi Pc 84} \toctranslation{The training rule on sunshades and sandals } \tocroot{Chattupāhana}}
\markboth{The training rule on sunshades and sandals }{Chattupāhana}
\extramarks{Bi Pc 84}{Bi Pc 84}

\subsection*{Origin story }

\subsubsection*{First sub-story }

At\marginnote{1.1.1} one time when the Buddha was staying at \textsanskrit{Sāvatthī} in \textsanskrit{Anāthapiṇḍika}’s Monastery, the nuns from the group of six used sunshades and sandals. People complained and criticized them, “How can nuns use sunshades and sandals? They’re just like householders who indulge in worldly pleasures!” 

The\marginnote{1.1.5} nuns heard the complaints of those people. The nuns of few desires complained and criticized them, “How can the nuns from the group of six use sunshades and sandals?” … “Is it true, monks, that those nuns do this?” 

“It’s\marginnote{1.1.9} true, Sir.” 

The\marginnote{1.1.10} Buddha rebuked them … “How can the nuns from the group of six do this? This will affect people’s confidence …” … “And, monks, the nuns should recite this training rule like this: 

\subsubsection*{First preliminary ruling }

\scrule{‘If a nun uses a sunshade and sandals, she commits an offense entailing confession.’” }

In\marginnote{1.1.15} this way the Buddha laid down this training rule for the nuns. 

\subsubsection*{Second sub-story }

Soon\marginnote{1.2.1} afterwards there was a sick nun who was not comfortable without sandals and a sunshade. … They told the Buddha. The Buddha then had the Sangha gathered and addressed the monks: 

\scrule{“Monks, I allow a sick nun to use a sunshade and sandals. }

And\marginnote{1.2.5} so, monks, the nuns should recite this training rule like this: 

\subsection*{Final ruling }

\scrule{‘If a nun who is not sick uses a sunshade and sandals, she commits an offense entailing confession.’” }

\subsection*{Definitions }

\begin{description}%
\item[A: ] whoever … %
\item[Nun: ] … The nun who has been given the full ordination in unanimity by both Sanghas through a legal procedure consisting of one motion and three announcements that is irreversible and fit to stand—this sort of nun is meant in this case. %
\item[Who is not sick: ] who is comfortable without a sunshade and sandals. %
\item[Who is sick: ] who is not comfortable without a sunshade and sandals. %
\item[A sunshade: ] there are three kinds of sunshades: the white sunshade, the reed sunshade, the leaf sunshade. They are bound at the rim and bound at the ribs.\footnote{Sp 2.634: \textit{\textsanskrit{Maṇḍalabaddhaṁ} \textsanskrit{salākabaddhanti} \textsanskrit{idaṁ} pana \textsanskrit{tiṇṇampi} \textsanskrit{chattānaṁ} \textsanskrit{pañjaradassanatthaṁ} \textsanskrit{vuttaṁ}. \textsanskrit{Tāni} hi \textsanskrit{maṇḍalabaddhāni} ceva honti \textsanskrit{salākabaddhāni} ca.} “\textit{\textsanskrit{Maṇḍalabaddhaṁ} \textsanskrit{salākabaddhan}}: this is said for the purpose of showing the frame of the three sunshades. For they are bound at the rim (\textit{\textsanskrit{maṇḍalabaddha}}) and bound at the ribs (\textit{\textsanskrit{salākabaddha}}).” } %
\item[Uses: ] if she uses them even once, she commits an offense entailing confession. %
\end{description}

\subsection*{Permutations }

If\marginnote{2.2.1} she is not sick, and she does not perceive herself as sick, and she uses a sunshade and sandals, she commits an offense entailing confession. If she is not sick, but she is unsure of it, and she uses a sunshade and sandals, she commits an offense entailing confession. If she is not sick, but she perceives herself as sick, and she uses a sunshade and sandals, she commits an offense entailing confession. 

If\marginnote{2.2.4} she uses a sunshade, but not sandals, she commits an offense of wrong conduct. If she uses sandals, but not a sunshade, she commits an offense of wrong conduct. If she is sick, but she does not perceive herself as sick, she commits an offense of wrong conduct. If she is sick, but she is unsure of it, she commits an offense of wrong conduct. If she is sick, and she perceives herself as sick, there is no offense. 

\subsection*{Non-offenses }

There\marginnote{2.3.1} is no offense: if she is sick;  if she uses them in a monastery or in the vicinity of a monastery;  if there is an emergency;  if she is insane;  if she is the first offender. 

\scendsutta{The first training rule is finished. }

%
\section*{{\suttatitleacronym Bi Pc 85}{\suttatitletranslation The training rule on vehicles }{\suttatitleroot Yāna}}
\addcontentsline{toc}{section}{\tocacronym{Bi Pc 85} \toctranslation{The training rule on vehicles } \tocroot{Yāna}}
\markboth{The training rule on vehicles }{Yāna}
\extramarks{Bi Pc 85}{Bi Pc 85}

\subsection*{Origin story }

\subsubsection*{First sub-story }

At\marginnote{1.1.1} one time when the Buddha was staying at \textsanskrit{Sāvatthī} in \textsanskrit{Anāthapiṇḍika}’s Monastery, the nuns from the group of six were traveling in vehicles. People complained and criticized them, “How can the nuns travel in a vehicle? They’re just like householders who indulge in worldly pleasures!” 

The\marginnote{1.1.5} nuns heard the complaints of those people. The nuns of few desires complained and criticized them, “How can the nuns from the group of six travel in a vehicle?” … “Is it true, monks, that those nuns do this?” 

“It’s\marginnote{1.1.9} true, Sir.” 

The\marginnote{1.1.10} Buddha rebuked them … “How can the nuns from the group of six do this? This will affect people’s confidence …” … “And, monks, the nuns should recite this training rule like this: 

\subsubsection*{First preliminary ruling }

\scrule{‘If a nun travels in a vehicle, she commits an offense entailing confession.’” }

In\marginnote{1.1.15} this way the Buddha laid down this training rule for the nuns. 

\subsubsection*{Second sub-story }

Soon\marginnote{1.2.1} afterwards there was a sick nun who was not able to travel on foot. … They told the Buddha. The Buddha then had the Sangha gathered and addressed the monks: 

\scrule{“Monks, I allow a sick nun to use a vehicle. }

And\marginnote{1.2.4} so, monks, the nuns should recite this training rule like this: 

\subsection*{Final ruling }

\scrule{‘If a nun who is not sick travels in a vehicle, she commits an offense entailing confession.’” }

\subsection*{Definitions }

\begin{description}%
\item[A: ] whoever … %
\item[Nun: ] … The nun who has been given the full ordination in unanimity by both Sanghas through a legal procedure consisting of one motion and three announcements that is irreversible and fit to stand—this sort of nun is meant in this case. %
\item[Who is not sick: ] who is able to travel on foot. %
\item[Who is sick: ] who is unable to travel on foot. %
\item[A vehicle: ] a wagon, a carriage, a cart, a chariot, a palanquin, a litter. %
\item[Travels: ] if she travels in a vehicle even once, she commits an offense entailing confession. %
\end{description}

\subsection*{Permutations }

If\marginnote{2.2.1} she is not sick, and she does not perceive herself as sick, and she travels in a vehicle, she commits an offense entailing confession. If she is not sick, but she is unsure of it, and she travels in a vehicle, she commits an offense entailing confession. If she is not sick, but she perceives herself as sick, and she travels in a vehicle, she commits an offense entailing confession. 

If\marginnote{2.2.4} she is sick, but she does not perceive herself as sick, she commits an offense of wrong conduct. If she is sick, but she is unsure of it, she commits an offense of wrong conduct. If she is sick, and she perceives herself as sick, there is no offense. 

\subsection*{Non-offenses }

There\marginnote{2.3.1} is no offense: if she is sick;  if there is an emergency;  if she is insane;  if she is the first offender. 

\scendsutta{The second training rule is finished. }

%
\section*{{\suttatitleacronym Bi Pc 86}{\suttatitletranslation The training rule on ornamentations of the hip }{\suttatitleroot Saṅghāṇi}}
\addcontentsline{toc}{section}{\tocacronym{Bi Pc 86} \toctranslation{The training rule on ornamentations of the hip } \tocroot{Saṅghāṇi}}
\markboth{The training rule on ornamentations of the hip }{Saṅghāṇi}
\extramarks{Bi Pc 86}{Bi Pc 86}

\subsection*{Origin story }

At\marginnote{1.1} one time when the Buddha was staying at \textsanskrit{Sāvatthī} in \textsanskrit{Anāthapiṇḍika}’s Monastery, there was a nun was associating with the family of a certain woman. That woman said to that nun, “Venerable, please give this hip ornament to such-and-such a woman.” The nun thought, “If I carry it in my almsbowl, I’ll get into trouble,” and so she put it on and then left. While she was walking along a street the threads snapped and were scattered all over. People complained and criticized her, “How can nuns wear hip ornaments? They’re just like householders who indulge in worldly pleasures!” 

The\marginnote{1.10} nuns heard the complaints of those people. The nuns of few desires complained and criticized her, “How could a nun wear a hip ornament?” … “Is it true, monks, that a nun did this?” 

“It’s\marginnote{1.14} true, Sir.” 

The\marginnote{1.15} Buddha rebuked her … “How could a nun do this? This will affect people’s confidence …” … “And, monks, the nuns should recite this training rule like this: 

\subsection*{Final ruling }

\scrule{‘If a nun wears a hip ornament, she commits an offense entailing confession.’” }

\subsection*{Definitions }

\begin{description}%
\item[A: ] whoever … %
\item[Nun: ] … The nun who has been given the full ordination in unanimity by both Sanghas through a legal procedure consisting of one motion and three announcements that is irreversible and fit to stand—this sort of nun is meant in this case. %
\item[A hip ornament: ] whatever goes on the hip. %
\item[Wears: ] if she wears it even once, she commits an offense entailing confession. %
\end{description}

\subsection*{Non-offenses }

There\marginnote{2.2.1} is no offense: if she wears it because she is sick;  if she wears a girdle;  if she is insane;  if she is the first offender. 

\scendsutta{The third training rule is finished. }

%
\section*{{\suttatitleacronym Bi Pc 87}{\suttatitletranslation The training rule on jewellery }{\suttatitleroot Itthālaṅkāra}}
\addcontentsline{toc}{section}{\tocacronym{Bi Pc 87} \toctranslation{The training rule on jewellery } \tocroot{Itthālaṅkāra}}
\markboth{The training rule on jewellery }{Itthālaṅkāra}
\extramarks{Bi Pc 87}{Bi Pc 87}

\subsection*{Origin story }

At\marginnote{1.1} one time when the Buddha was staying at \textsanskrit{Sāvatthī} in \textsanskrit{Anāthapiṇḍika}’s Monastery, the nuns from the group of six were wearing jewellery. People complained and criticized them, “How can nuns wear jewellery? They’re just like householders who indulge in worldly pleasures!” 

The\marginnote{1.5} nuns heard the complaints of those people. The nuns of few desires complained and criticized them, “How can the nuns from the group of six wear jewellery?” … “Is it true, monks, that those nuns do this?” 

“It’s\marginnote{1.9} true, Sir.” 

The\marginnote{1.10} Buddha rebuked them … “How can the nuns from the group of six do this? This will affect people’s confidence …” … “And, monks, the nuns should recite this training rule like this: 

\subsection*{Final ruling }

\scrule{‘If a nun wears jewellery, she commits an offense entailing confession.’” }

\subsection*{Definitions }

\begin{description}%
\item[A: ] whoever … %
\item[Nun: ] … The nun who has been given the full ordination in unanimity by both Sanghas through a legal procedure consisting of one motion and three announcements that is irreversible and fit to stand—this sort of nun is meant in this case. %
\item[Jewellery: ] what goes on the head, what goes around the neck, what goes on the hands, what goes on the feet, what goes around the hips. %
\item[Wears: ] if she wears it even once, she commits an offense entailing confession. %
\end{description}

\subsection*{Non-offenses }

There\marginnote{2.2.1} is no offense: if she does it because she is sick;  if she is insane;  if she is the first offender. 

\scendsutta{The fourth training rule is finished. }

%
\section*{{\suttatitleacronym Bi Pc 88}{\suttatitletranslation The training rule on scents and colors }{\suttatitleroot Gandhavaṇṇaka}}
\addcontentsline{toc}{section}{\tocacronym{Bi Pc 88} \toctranslation{The training rule on scents and colors } \tocroot{Gandhavaṇṇaka}}
\markboth{The training rule on scents and colors }{Gandhavaṇṇaka}
\extramarks{Bi Pc 88}{Bi Pc 88}

\subsection*{Origin story }

At\marginnote{1.1} one time when the Buddha was staying at \textsanskrit{Sāvatthī} in \textsanskrit{Anāthapiṇḍika}’s Monastery, the nuns from the group of six were bathing with scents and colors. People complained and criticized them, “How can nuns bathe with scents and colors? They’re just like householders who indulge in worldly pleasures!” 

The\marginnote{1.5} nuns heard the complaints of those people. The nuns of few desires complained and criticized them, “How can the nuns from the group of six bathe with scents and colors?” … “Is it true, monks, that those nuns do this?” 

“It’s\marginnote{1.9} true, Sir.” 

The\marginnote{1.10} Buddha rebuked them … “How can the nuns from the group of six do this? This will affect people’s confidence …” … “And, monks, the nuns should recite this training rule like this: 

\subsection*{Final ruling }

\scrule{‘If a nun bathes with scents and colors, she commits an offense entailing confession.’” }

\subsection*{Definitions }

\begin{description}%
\item[A: ] whoever … %
\item[Nun: ] … The nun who has been given the full ordination in unanimity by both Sanghas through a legal procedure consisting of one motion and three announcements that is irreversible and fit to stand—this sort of nun is meant in this case. %
\item[Scents: ] any kind of scent. %
\item[Colors: ] any kind of color. %
\item[Bathes: ] is bathing. For the effort there is an act of wrong conduct. At the end of the bath, she commits an offense entailing confession. %
\end{description}

\subsection*{Non-offenses }

There\marginnote{2.2.1} is no offense: if she does it because she is sick;  if she is insane;  if she is the first offender. 

\scendsutta{The fifth training rule is finished. }

%
\section*{{\suttatitleacronym Bi Pc 89}{\suttatitletranslation The training rule on what is scented }{\suttatitleroot Vāsitaka}}
\addcontentsline{toc}{section}{\tocacronym{Bi Pc 89} \toctranslation{The training rule on what is scented } \tocroot{Vāsitaka}}
\markboth{The training rule on what is scented }{Vāsitaka}
\extramarks{Bi Pc 89}{Bi Pc 89}

\subsection*{Origin story }

At\marginnote{1.1} one time when the Buddha was staying at \textsanskrit{Sāvatthī} in \textsanskrit{Anāthapiṇḍika}’s Monastery, the nuns from the group of six were bathing with scents and oilseed flour. People complained and criticized them, “How can nuns bathe with scents and oilseed flour? They’re just like householders who indulge in worldly pleasures!” 

The\marginnote{1.5} nuns heard the complaints of those people. The nuns of few desires complained and criticized them, “How can the nuns from the group of six bathe with scents and oilseed flour?” … “Is it true, monks, that those nuns do this?” 

“It’s\marginnote{1.9} true, Sir.” 

The\marginnote{1.10} Buddha rebuked them … “How can the nuns from the group of six do this? This will affect people’s confidence …” … “And, monks, the nuns should recite this training rule like this: 

\subsection*{Final ruling }

\scrule{‘If a nun bathes with scents and oilseed flour, she commits an offense entailing confession.’” }

\subsection*{Definitions }

\begin{description}%
\item[A: ] whoever … %
\item[Nun: ] … The nun who has been given the full ordination in unanimity by both Sanghas through a legal procedure consisting of one motion and three announcements that is irreversible and fit to stand—this sort of nun is meant in this case. %
\item[Scents: ] any kind of scent. %
\item[Oilseed flour: ] ground sesame is what is meant. %
\item[Bathes: ] is bathing. For the effort there is an act of wrong conduct. At the end of the bath, she commits an offense entailing confession. %
\end{description}

\subsection*{Non-offenses }

There\marginnote{2.2.1} is no offense: if she does it because she is sick;  if she bathes with ordinary oilseed flour;  if she is insane;  if she is the first offender. 

\scendsutta{The sixth training rule is finished. }

%
\section*{{\suttatitleacronym Bi Pc 90}{\suttatitletranslation The training rule on having a nun massage }{\suttatitleroot Bhikkhunī-ummaddāpana}}
\addcontentsline{toc}{section}{\tocacronym{Bi Pc 90} \toctranslation{The training rule on having a nun massage } \tocroot{Bhikkhunī-ummaddāpana}}
\markboth{The training rule on having a nun massage }{Bhikkhunī-ummaddāpana}
\extramarks{Bi Pc 90}{Bi Pc 90}

\subsection*{Origin story }

At\marginnote{1.1} one time when the Buddha was staying at \textsanskrit{Sāvatthī} in \textsanskrit{Anāthapiṇḍika}’s Monastery, the nuns were having a nun massage and rub them. When people walking about the dwellings saw this, they complained and criticized them, “How can nuns get a nun to massage and rub them? They’re just like householders who indulge in worldly pleasures!” 

The\marginnote{1.5} nuns heard the complaints of those people. The nuns of few desires complained and criticized them, “How can nuns get a nun to massage and rub them?” … “Is it true, monks, that nuns do this?” 

“It’s\marginnote{1.9} true, Sir.” 

The\marginnote{1.10} Buddha rebuked them … “How can nuns do this? This will affect people’s confidence …” … “And, monks, the nuns should recite this training rule like this: 

\subsection*{Final ruling }

\scrule{‘If a nun has a nun massage her or rub her, she commits an offense entailing confession.’” }

\subsection*{Definitions }

\begin{description}%
\item[A: ] whoever … %
\item[Nun: ] … The nun who has been given the full ordination in unanimity by both Sanghas through a legal procedure consisting of one motion and three announcements that is irreversible and fit to stand—this sort of nun is meant in this case. %
\item[A nun: ] another nun. %
\item[If she has a nun massage her: ] if she gets her to massage her, she commits an offense entailing confession. %
\item[Or if she has a nun rub her: ] if she gets her to rub her, she commits an offense entailing confession. %
\end{description}

\subsection*{Non-offenses }

There\marginnote{2.2.1} is no offense: if she is sick;  if there is an emergency;  if she is insane;  if she is the first offender. 

\scendsutta{The seventh training rule is finished. }

%
\section*{{\suttatitleacronym  Bi Pc 91–93}{\suttatitletranslation The training rules on having a trainee nun … a novice nun … a female householder massage }{\suttatitleroot Sikkhamānā-sāmaṇerī-gihinī-ummaddāpana}}
\addcontentsline{toc}{section}{\tocacronym{ Bi Pc 91–93} \toctranslation{The training rules on having a trainee nun … a novice nun … a female householder massage } \tocroot{Sikkhamānā-sāmaṇerī-gihinī-ummaddāpana}}
\markboth{The training rules on having a trainee nun … a novice nun … a female householder massage }{Sikkhamānā-sāmaṇerī-gihinī-ummaddāpana}
\extramarks{ Bi Pc 91–93}{ Bi Pc 91–93}

\subsection*{Origin story }

At\marginnote{1.1} one time when the Buddha was staying at \textsanskrit{Sāvatthī} in \textsanskrit{Anāthapiṇḍika}’s Monastery, the nuns were having a trainee nun … 

…\marginnote{1.1} a novice nun … 

…\marginnote{1.1} a female householder massage and rub them. 

When\marginnote{1.2} people walking about the dwellings saw this, they complained and criticized them, “How can the nuns get a female householder to massage and rub them? They’re just like householders who indulge in worldly pleasures!” 

The\marginnote{1.4} nuns heard the complaints of those people. 

The\marginnote{1.5} nuns of few desires complained and criticized them, “How can nuns get a female householder to massage and rub them?” 

“Is\marginnote{1.7} it true, monks, that the nuns do this?” 

“It’s\marginnote{1.8} true, Sir.” 

The\marginnote{1.9} Buddha rebuked them … “How can nuns do this? This will affect people’s confidence …” … “And, monks, the nuns should recite this training rule like this: 

\subsection*{Final ruling }

\scrule{‘If a nun has (a trainee nun … a novice nun …) a female householder massage her or rub her, she commits an offense entailing confession.’” }

\subsection*{Definitions }

\begin{description}%
\item[A: ] whoever … %
\item[Nun: ] … The nun who has been given the full ordination in unanimity by both Sanghas through a legal procedure consisting of one motion and three announcements that is irreversible and fit to stand—this sort of nun is meant in this case. %
\item[A trainee nun: ] a female who is training for two years in the six rules.\footnote{It seems the past participle \textit{sikkhita} here needs to be read as a present participle. } %
\item[A novice nun: ] a female training in the ten training rules. %
\item[A female householder: ] a female who lives in a house is what is meant. %
\item[If she has her massage her: ] if she gets her to massage her, she commits an offense entailing confession. %
\item[If she has her rub her: ] if she gets her to rub her, she commits an offense entailing confession. %
\end{description}

\subsection*{Non-offenses }

There\marginnote{2.2.1} is no offense: if she is sick;  if there is an emergency;  if she is insane;  if she is the first offender. 

\scendsutta{The tenth training rule is finished. }

%
\section*{{\suttatitleacronym Bi Pc 94}{\suttatitletranslation The training rule on not asking permission }{\suttatitleroot Anāpucchā}}
\addcontentsline{toc}{section}{\tocacronym{Bi Pc 94} \toctranslation{The training rule on not asking permission } \tocroot{Anāpucchā}}
\markboth{The training rule on not asking permission }{Anāpucchā}
\extramarks{Bi Pc 94}{Bi Pc 94}

\subsection*{Origin story }

At\marginnote{1.1} one time when the Buddha was staying at \textsanskrit{Sāvatthī} in \textsanskrit{Anāthapiṇḍika}’s Monastery, nuns sat down on seats in front of a monk without asking permission. The monks complained and criticized them, “How can nuns sit down on seats in front of a monk without asking permission?” … “Is it true, monks, that nuns do this?” 

“It’s\marginnote{1.6} true, Sir.” 

The\marginnote{1.7} Buddha rebuked them … “How can nuns do this? This will affect people’s confidence …” … “And, monks, the nuns should recite this training rule like this: 

\subsection*{Final ruling }

\scrule{‘If a nun sits down on a seat in front of a monk without asking permission, she commits an offense entailing confession.’” }

\subsection*{Definitions }

\begin{description}%
\item[A: ] whoever … %
\item[Nun: ] … The nun who has been given the full ordination in unanimity by both Sanghas through a legal procedure consisting of one motion and three announcements that is irreversible and fit to stand—this sort of nun is meant in this case. %
\item[In front of a monk: ] in front of one who is fully ordained. %
\item[Without asking permission: ] without getting permission. %
\item[Sits down on a seat: ] even if she sits down on the ground, she commits an offense entailing confession. %
\end{description}

\subsection*{Permutations }

If\marginnote{2.2.1} she has not asked permission, and she does not perceive that she has, and she sits down on a seat, she commits an offense entailing confession. If she has not asked permission, but she is unsure of it, and she sits down on a seat, she commits an offense entailing confession. If she has not asked permission, but she perceives that she has, and she sits down on a seat, she commits an offense entailing confession. 

If\marginnote{2.2.4} she has asked permission, but she does not perceive that she has, she commits an offense of wrong conduct. If she has asked permission, but she is unsure of it, she commits an offense of wrong conduct. If she has asked permission, and she perceives that she has, there is no offense. 

\subsection*{Non-offenses }

There\marginnote{2.3.1} is no offense: if she sits down on the seat after asking permission;  if she is sick;  if there is an emergency;  if she is insane;  if she is the first offender. 

\scendsutta{The eleventh training rule is finished. }

%
\section*{{\suttatitleacronym Bi Pc 95}{\suttatitletranslation The training rule on asking questions }{\suttatitleroot Pañhāpucchana}}
\addcontentsline{toc}{section}{\tocacronym{Bi Pc 95} \toctranslation{The training rule on asking questions } \tocroot{Pañhāpucchana}}
\markboth{The training rule on asking questions }{Pañhāpucchana}
\extramarks{Bi Pc 95}{Bi Pc 95}

\subsection*{Origin story }

At\marginnote{1.1} one time when the Buddha was staying at \textsanskrit{Sāvatthī} in \textsanskrit{Anāthapiṇḍika}’s Monastery, the nuns asked questions of a monk who had not given them permission. The monks complained and criticized them, “How can nuns ask questions of a monk who hasn’t given them permission?” … “Is it true, monks, that nuns do this?” 

“It’s\marginnote{1.6} true, Sir.” 

The\marginnote{1.7} Buddha rebuked them … “How can nuns do this? This will affect people’s confidence …” … “And, monks, the nuns should recite this training rule like this: 

\subsection*{Final ruling }

\scrule{‘If a nun asks a question of a monk who has not given her permission, she commits an offense entailing confession.’” }

\subsection*{Definitions }

\begin{description}%
\item[A: ] whoever … %
\item[Nun: ] … The nun who has been given the full ordination in unanimity by both Sanghas through a legal procedure consisting of one motion and three announcements that is irreversible and fit to stand—this sort of nun is meant in this case. %
\item[Who has not given her permission: ] without asking permission. %
\item[A monk: ] one who is fully ordained. %
\item[Asks a question: ] if she gets permission to ask about the discourses, but she asks about the Monastic Law or philosophy, she commits an offense entailing confession. If she gets permission to ask about the Monastic Law, but she asks about the discourses or philosophy, she commits an offense entailing confession. If she gets permission to ask about philosophy, but she asks about the discourses or the Monastic Law, she commits an offense entailing confession. %
\end{description}

\subsection*{Permutations }

If\marginnote{2.2.1} she has not asked permission, and she does not perceive that she has, and she asks a question, she commits an offense entailing confession. If she has not asked permission, but she is unsure of it, and she asks a question, she commits an offense entailing confession. If she has not asked permission, but she perceives that she has, and she asks a question, she commits an offense entailing confession. 

If\marginnote{2.2.4} she has asked permission, but she does not perceive that she has, she commits an offense of wrong conduct. If she has asked permission, but she is unsure of it, she commits an offense of wrong conduct. If she has asked permission, and she perceives that she has, there is no offense. 

\subsection*{Non-offenses }

There\marginnote{2.3.1} is no offense: if she asks after getting permission;  if she gets permission, but not in regard to a specific subject, and she then asks about any subject;  if she is insane;  if she is the first offender. 

\scendsutta{The twelfth training rule is finished. }

%
\section*{{\suttatitleacronym Bi Pc 96}{\suttatitletranslation The training rule on not wearing a chest wrap }{\suttatitleroot Saṁkakṣikā}}
\addcontentsline{toc}{section}{\tocacronym{Bi Pc 96} \toctranslation{The training rule on not wearing a chest wrap } \tocroot{Saṁkakṣikā}}
\markboth{The training rule on not wearing a chest wrap }{Saṁkakṣikā}
\extramarks{Bi Pc 96}{Bi Pc 96}

\subsection*{Origin story }

At\marginnote{1.1} one time when the Buddha was staying at \textsanskrit{Sāvatthī} in \textsanskrit{Anāthapiṇḍika}’s Monastery, a nun had gone to the village for alms without wearing her chest wrap. While she was walking along a street, a whirlwind lifted up her upper robes. People shouted out, “She has beautiful breasts and belly!” Because she was teased by those people, she felt humiliated. 

After\marginnote{1.7} returning to the nuns’ dwelling place, she told the nuns what had happened. The nuns of few desires complained and criticized her, “How could a nun enter an inhabited area without wearing her chest wrap?” … “Is it true, monks, that a nun did this?” 

“It’s\marginnote{1.11} true, Sir.” 

The\marginnote{1.12} Buddha rebuked her … “How could a nun do this? This will affect people’s confidence …” … “And, monks, the nuns should recite this training rule like this: 

\subsection*{Final ruling }

\scrule{‘If a nun enters an inhabited area without wearing her chest wrap, she commits an offense entailing confession.’”\footnote{I translate in accordance with the findings of Oskar von Hinüber and Bhikkhu \textsanskrit{Anālayo} in their paper “The Robes of a \textit{\textsanskrit{Bhikkhunī}}”. } }

\subsection*{Definitions }

\begin{description}%
\item[A: ] whoever … %
\item[Nun: ] … The nun who has been given the full ordination in unanimity by both Sanghas through a legal procedure consisting of one motion and three announcements that is irreversible and fit to stand—this sort of nun is meant in this case. %
\item[Without wearing her chest wrap: ] not wearing her chest wrap. %
\item[Chest wrap: ] it is for the purpose of concealing the body below the collar bone and above the navel. %
\item[Enters an inhabited area: ] if she crosses the boundary of an enclosed inhabited area, she commits an offense entailing confession. If she enters the vicinity of an unenclosed inhabited area, she commits an offense entailing confession. %
\end{description}

\subsection*{Non-offenses }

There\marginnote{2.2.1} is no offense: if her robe is stolen;\footnote{Sp 4.1227: \textit{\textsanskrit{Acchinnacīvarikāyātiādīsu} \textsanskrit{saṅkaccikacīvarameva} \textsanskrit{cīvaranti} \textsanskrit{veditabbaṁ}}, “Her robe is stolen, etc.: just the robe which is the chest cover is to be understood.” The “etc.”, \textit{\textsanskrit{ādisu}}, refers to the following non-offense, “If her robe is lost”. }  if her robe is lost;  if she is sick;  if she is not mindful;  if she does not know;  if there is an emergency;  if she is insane;  if she is the first offender. 

\scendsutta{The thirteenth training rule is finished. }

\scendvagga{The ninth subchapter on sunshades and sandals is finished. }

\subsection*{\footnote{At this point come the remaining seventy \textit{\textsanskrit{pācittiya}} rules that the nuns have in common with the monks. By \textit{Dhammikavagga} and \textit{\textsanskrit{Nandasikkhāpada}} the Pali text indicates the last chapter and the last rule of the \textit{\textsanskrit{pācittiyas}}. In summary these seventy rues are as follows. Bi Pc 97–116 are equivalent to Bu Pc 1–20. Then 117–118 = 31–32, 119 = 34, 120–121 = 37–38, 122=40, 123–144 = 42–63, 145 = 66, 146–160 = 68–82, 161 = 84, 162–164 = 86–88, 165 = 90, 166 = 92. } }

“Venerables,\marginnote{2.2.14} the one hundred and sixty-six rules on confession have been recited. In regard to this I ask you, ‘Are you pure in this?’ A second time I ask, ‘Are you pure in this?’ A third time I ask, ‘Are you pure in this?’ You are pure in this and therefore silent. I’ll remember it thus.” 

\scend{The section on minor rules has been completed. }

\scendkanda{The chapter on offenses entailing confession in the Nuns’ Analysis is finished. }

%
\addtocontents{toc}{\let\protect\contentsline\protect\nopagecontentsline}
\chapter*{Acknowledgment}
\addcontentsline{toc}{chapter}{\tocchapterline{Acknowledgment}}
\addtocontents{toc}{\let\protect\contentsline\protect\oldcontentsline}

%
\section*{{\suttatitleacronym Bi Pd 1}{\suttatitletranslation The training rule on asking for ghee }{\suttatitleroot Sappiviññāpana}}
\addcontentsline{toc}{section}{\tocacronym{Bi Pd 1} \toctranslation{The training rule on asking for ghee } \tocroot{Sappiviññāpana}}
\markboth{The training rule on asking for ghee }{Sappiviññāpana}
\extramarks{Bi Pd 1}{Bi Pd 1}

Venerables,\marginnote{0.5} these eight rules on acknowledgment come up for recitation. 

\subsection*{Origin story }

\subsubsection*{First sub-story }

At\marginnote{1.1.1} one time when the Buddha was staying at \textsanskrit{Sāvatthī} in \textsanskrit{Anāthapiṇḍika}’s Monastery, the nuns from the group of six were eating ghee that they had asked for. People complained and criticized them, “How can nuns eat ghee that they have asked for? Who doesn’t like nice food? Who doesn’t prefer tasty food?” 

The\marginnote{1.1.6} nuns heard the complaints of those people. The nuns of few desires complained and criticized them, “How can the nuns from the group of six eat ghee that they have asked for?” … “Is it true, monks, that those nuns do this?” 

“It’s\marginnote{1.1.10} true, Sir.” 

The\marginnote{1.1.11} Buddha rebuked them … “How can the nuns from the group of six do this? This will affect people’s confidence …” … “And, monks, the nuns should recite this training rule like this: 

\subsubsection*{First preliminary ruling }

\scrule{‘If a nun asks for ghee and then eats it, she must acknowledge it: “I have done a blameworthy and unsuitable thing that is to be acknowledged. I acknowledge it.”’” }

In\marginnote{1.1.17} this way the Buddha laid down this training rule for the nuns. 

\subsubsection*{Second sub-story }

Soon\marginnote{1.2.1} afterwards a number of nuns were sick. The nuns who were looking after them asked, “I hope you’re bearing up? I hope you’re getting better?” 

“Previously\marginnote{1.2.4} we ate ghee that we had asked for, and then we were comfortable. But now that the Buddha has prohibited this, we don’t ask because we’re afraid of wrongdoing. And because of that we’re not comfortable.” … 

They\marginnote{1.2.6} told the Buddha. Soon afterwards the Buddha had the Sangha gathered and addressed the monks: 

\scrule{“Monks, I allow a sick nun to eat ghee that she has asked for. }

And\marginnote{1.2.8} so, monks, the nuns should recite this training rule like this: 

\subsection*{Final ruling }

\scrule{‘If a nun who is not sick asks for ghee and then eats it, she must acknowledge it: “I have done a blameworthy and unsuitable thing that is to be acknowledged. I acknowledge it.”’” }

\subsection*{Definitions }

\begin{description}%
\item[A: ] whoever … %
\item[Nun: ] … The nun who has been given the full ordination in unanimity by both Sanghas through a legal procedure consisting of one motion and three announcements that is irreversible and fit to stand—this sort of nun is meant in this case. %
\item[Who is not sick: ] who is comfortable without ghee. %
\item[Who is sick: ] who is not comfortable without ghee. %
\item[Ghee: ] ghee from cows, ghee from goats, ghee from buffaloes, or ghee from whatever animal whose meat is allowable. %
\end{description}

If\marginnote{2.1.11} she is not sick and she asks for herself, then for the effort there is an act of wrong conduct.  When she receives it with the intention of eating it, she commits an offense of wrong conduct.  For every mouthful, she commits an offense entailing acknowledgment. 

\subsection*{Permutations }

If\marginnote{2.2.1} she is not sick, and she does not perceive herself as sick, and she eats ghee that she has asked for, she commits an offense entailing acknowledgment. If she is not sick, but she is unsure of it, and she eats ghee that she has asked for, she commits an offense entailing acknowledgment. If she is not sick, but she perceives herself as sick, and she eats ghee that she has asked for, she commits an offense entailing acknowledgment. 

If\marginnote{2.2.4} she is sick, but she does not perceive herself as sick, she commits an offense of wrong conduct. If she is sick, but she is unsure of it, she commits an offense of wrong conduct. If she is sick, and she perceives herself as sick, there is no offense. 

\subsection*{Non-offenses }

There\marginnote{2.3.1} is no offense: if she is sick;  if she asked for it when she was sick, but eats it when she is no longer sick;  if she eats the leftovers from one who is sick;  if it is from relatives;  if it is from those who have given an invitation;  if it is for the benefit of someone else;  if it is by means of her own property;  if she is insane;  if she is the first offender. 

\scendsutta{The first training rule on acknowledgment is finished. }

%
\section*{{\suttatitleacronym  Bi Pd 2–8}{\suttatitletranslation The training rules on asking for oil … honey … syrup … fish … meat … milk … curd }{\suttatitleroot Telādi}}
\addcontentsline{toc}{section}{\tocacronym{ Bi Pd 2–8} \toctranslation{The training rules on asking for oil … honey … syrup … fish … meat … milk … curd } \tocroot{Telādi}}
\markboth{The training rules on asking for oil … honey … syrup … fish … meat … milk … curd }{Telādi}
\extramarks{ Bi Pd 2–8}{ Bi Pd 2–8}

\subsection*{Origin story }

\subsubsection*{First sub-story }

At\marginnote{1.1.1} one time when the Buddha was staying at \textsanskrit{Sāvatthī} in \textsanskrit{Anāthapiṇḍika}’s Monastery,  the nuns from the group of six were eating oil that they had asked for.  … were eating honey that they had asked for.  … were eating syrup that they had asked for.  … were eating fish that they had asked for.  … were eating meat that they had asked for.  … were drinking milk that they had asked for.  … were eating curd that they had asked for. 

People\marginnote{1.1.9} complained and criticized them,  “How can nuns eat curd that they have asked for?  Who doesn’t like nice food? Who doesn’t prefer tasty food?” 

The\marginnote{1.1.12} nuns heard the complaints of those people.  The nuns of few desires complained and criticized them,  “How can the nuns from the group of six eat curd that they have asked for?” …  “Is it true, monks, that those nuns do this?” 

“It’s\marginnote{1.1.16} true, Sir.” 

The\marginnote{1.1.17} Buddha rebuked them …  “How can the nuns from the group of six do this?  This will affect people’s confidence …” …  “And, monks, the nuns should recite this training rule like this: 

\subsubsection*{First preliminary ruling }

\scrule{‘If a nun asks for curd and then eats it, she must acknowledge it:  “I have done a blameworthy and unsuitable thing that is to be acknowledged. I acknowledge it.”’” }

In\marginnote{1.1.23} this way the Buddha laid down this training rule for the nuns. 

\subsubsection*{Second sub-story }

Soon\marginnote{1.2.1} afterwards a number of nuns were sick.  The nuns who were looking after them asked,  “I hope you’re bearing up? I hope you’re getting better?” 

“Previously\marginnote{1.2.4} we ate curd that we had asked for, and then we were comfortable.  But now that the Buddha has prohibited this, we don’t ask because we’re afraid of wrongdoing. And because of that we’re not comfortable.” … 

They\marginnote{1.2.6} told the Buddha. Soon afterwards the Buddha had the Sangha gathered and addressed the monks: 

\scrule{“Monks, I allow a sick nun to eat curd that she has asked for. }

And\marginnote{1.2.8} so, monks, the nuns should recite this training rule like this: 

\subsection*{Final ruling }

\scrule{‘If a nun who is not sick asks for (oil … }

\scrule{honey … }

\scrule{syrup … }

\scrule{fish … }

\scrule{meat … }

\scrule{milk … ) }

\scrule{curd and then eats it, she must acknowledge it: “I have done a blameworthy and unsuitable thing that is to be acknowledged. I acknowledge it.”’” }

\subsection*{Definitions }

\begin{description}%
\item[A: ] whoever … %
\item[Nun: ] … The nun who has been given the full ordination in unanimity by both Sanghas through a legal procedure consisting of one motion and three announcements that is irreversible and fit to stand—this sort of nun is meant in this case. %
\item[Who is not sick: ] who is comfortable without curd. %
\item[Who is sick: ] who is not comfortable without curd. %
\item[Oil: ] sesame oil, mustard seed oil, honey tree oil, castor oil, oil from tallow. %
\item[Honey: ] honey from bees. %
\item[Syrup: ] from sugar cane. %
\item[Fish: ] what lives in water is what is meant. %
\item[Meat: ] the meat of those animals whose meat is allowable. %
\item[Milk: ] milk from cows, milk from goats, milk from buffaloes, or milk from whatever animal whose meat is allowable. %
\item[Curd: ] curd from those same animals. %
\end{description}

If\marginnote{2.1.23} she is not sick and she asks for herself, then for the effort there is an act of wrong conduct.  When she receives it with the intention of eating it, she commits an offense of wrong conduct.  For every mouthful, she commits an offense entailing acknowledgment. 

\subsection*{Permutations }

If\marginnote{2.2.1} she is not sick, and she does not perceive herself as sick, and she eats curd that she has asked for, she commits an offense entailing acknowledgment.  If she is not sick, but she is unsure of it, and she eats curd that she has asked for, she commits an offense entailing acknowledgment.  If she is not sick, but she perceives herself as sick, and she eats curd that she has asked for, she commits an offense entailing acknowledgment. 

If\marginnote{2.2.4} she is sick, but she does not perceive herself as sick, she commits an offense of wrong conduct.  If she is sick, but she is unsure of it, she commits an offense of wrong conduct.  If she is sick, and she perceives herself as sick, there is no offense. 

\subsection*{Non-offenses }

There\marginnote{2.3.1} is no offense:  if she is sick;  if she asked for it when she was sick, but eats it when she is no longer sick;  if she eats the leftovers from one who is sick;  if it is from relatives;  if it is from those who have given an invitation;  if it is for the benefit of someone else;  if it is by means of her own property;  if she is insane;  if she is the first offender. 

\scendsutta{The eighth training rule entailing acknowledgment is finished. }

“Venerables,\marginnote{2.3.12} the eight rules on acknowledgment have been recited.  In regard to this I ask you,  ‘Are you pure in this?’  A second time I ask,  ‘Are you pure in this?’  A third time I ask,  ‘Are you pure in this?’  You are pure in this and therefore silent. I’ll remember it thus.” 

\scendkanda{The chapter on offenses entailing acknowledgment in the Nuns’ Analysis is finished. }

%
\addtocontents{toc}{\let\protect\contentsline\protect\nopagecontentsline}
\chapter*{Rules for Training}
\addcontentsline{toc}{chapter}{\tocchapterline{Rules for Training}}
\addtocontents{toc}{\let\protect\contentsline\protect\oldcontentsline}

%
\section*{{\suttatitleacronym Bi Sk 1}{\suttatitletranslation The training rule on evenly all around }{\suttatitleroot Parimaṇḍala}}
\addcontentsline{toc}{section}{\tocacronym{Bi Sk 1} \toctranslation{The training rule on evenly all around } \tocroot{Parimaṇḍala}}
\markboth{The training rule on evenly all around }{Parimaṇḍala}
\extramarks{Bi Sk 1}{Bi Sk 1}

Venerables,\marginnote{0.6} these rules to be trained in come up for recitation. 

\subsection*{Origin story }

At\marginnote{1.1} one time when the Buddha was staying at \textsanskrit{Sāvatthī} in \textsanskrit{Anāthapiṇḍika}’s Monastery, the nuns from the group of six wore their sarongs hanging down in front and behind. People complained and criticized them, “How can nuns wear their sarongs hanging down in front and behind? They’re just like householders who indulge in worldly pleasures!” 

The\marginnote{1.5} nuns heard the complaints of those people, and the nuns of few desires complained and criticized them, “How can the nuns from the group of six wear their sarongs hanging down in front and behind?” … “Is it true, monks, that those nuns do this?” 

“It’s\marginnote{1.9} true, Sir.” 

The\marginnote{1.10} Buddha rebuked them … “How can the nuns from the group of six do this? This will affect people’s confidence …” … “And, monks, the nuns should recite this training rule like this: 

\subsection*{Final ruling }

\scrule{‘“I will wear my sarong evenly all around,” this is how you should train.’” }

One\marginnote{1.15} should wear one’s sarong evenly all around, covering the navel and the knees. If a nun, out of disrespect, wears her sarong hanging down in front or behind, she commits an offense of wrong conduct. 

\subsection*{Non-offenses }

There\marginnote{1.17.1} is no offense: if it is unintentional;  if she is not mindful;  if she does not know;  if she is sick;  if there is an emergency;  if she is insane;  if she is the first offender. 

(Contracted.)\marginnote{1.25} 

%
\section*{{\suttatitleacronym Bi Sk 75}{\suttatitletranslation the training rule on defecating in water }{\suttatitleroot Udakeuccāra}}
\addcontentsline{toc}{section}{\tocacronym{Bi Sk 75} \toctranslation{the training rule on defecating in water } \tocroot{Udakeuccāra}}
\markboth{the training rule on defecating in water }{Udakeuccāra}
\extramarks{Bi Sk 75}{Bi Sk 75}

\subsection*{Origin story }

\subsubsection*{First sub-story }

At\marginnote{1.1} one time when the Buddha was staying at \textsanskrit{Sāvatthī} in \textsanskrit{Anāthapiṇḍika}’s Monastery, the nuns from the group of six were defecating, urinating, and spitting in water. People complained and criticized them, “How can nuns defecate, urinate, and spit in water? They’re just like householders who indulge in worldly pleasures!” 

The\marginnote{1.5} nuns heard the complaints of those people, and the nuns of few desires complained and criticized them, “How can the nuns from the group of six defecate, urinate, and spit in water?” The nuns told the monks, who in turn told the Buddha. Soon afterwards the Buddha had the Sangha gathered and questioned the monks: “Is it true, monks, that those nuns do this?” 

“It’s\marginnote{1.13} true, Sir.” 

The\marginnote{1.14} Buddha rebuked them … “How can the nuns from the group of six do this? This will affect people’s confidence …” … “And, monks, the nuns should recite this training rule like this: 

\subsubsection*{Preliminary ruling }

\scrule{‘“I will not defecate, urinate, or spit in water,” this is how you should train.’” }

In\marginnote{1.19} this way the Buddha laid down this training rule for the nuns. 

\subsubsection*{Second sub-story }

Soon\marginnote{2.1} afterwards, being afraid of wrongdoing, sick nuns did not defecate, urinate, or spit in water. They told the Buddha. He then had the Sangha gathered and addressed the monks: 

\scrule{“Monks, I allow a sick nun to defecate, urinate, and spit in water. }

And\marginnote{2.4} so, monks, the nuns should recite this training rule like this: 

\subsection*{Final ruling }

\scrule{‘“When not sick, I will not defecate, urinate, or spit in water,” this is how you should train.’” }

If\marginnote{2.6} one is not sick, one should not defecate, urinate, or spit in water. If, out of disrespect, a nun who is not sick defecates, urinates, or spits in water, she commits an offense of wrong conduct. 

\subsection*{Non-offenses }

There\marginnote{2.8.1} is no offense: if it is unintentional;  if she is not mindful;  if she does not know;  if she is sick;  if she does it on dry ground, but it then spreads to water;  if there is an emergency;  if she is insane;  if she is deranged;  if she is overwhelmed by pain;  if she is the first offender. 

\scendsutta{The fifteenth training rule is finished. }

\scendvagga{The seventh subchapter on shoes is finished. }

“Venerables,\marginnote{2.21} the rules to be trained in have been recited. In regard to this I ask you, ‘Are you pure in this?’ A second time I ask, ‘Are you pure in this?’ A third time I ask, ‘Are you pure in this?’ You are pure in this and therefore silent. I’ll remember it thus.” 

\scendkanda{The chapter on training is finished. }

%
\addtocontents{toc}{\let\protect\contentsline\protect\nopagecontentsline}
\chapter*{Settling Legal Issues}
\addcontentsline{toc}{chapter}{\tocchapterline{Settling Legal Issues}}
\addtocontents{toc}{\let\protect\contentsline\protect\oldcontentsline}

%
\section*{{\suttatitleacronym  Bi As 1–7}{\suttatitletranslation The settling of legal issues }{\suttatitleroot Adhikaraṇasamatha}}
\addcontentsline{toc}{section}{\tocacronym{ Bi As 1–7} \toctranslation{The settling of legal issues } \tocroot{Adhikaraṇasamatha}}
\markboth{The settling of legal issues }{Adhikaraṇasamatha}
\extramarks{ Bi As 1–7}{ Bi As 1–7}

“Venerables,\marginnote{1.1} these seven principles for the settling of legal issues come up for recitation. 

For\marginnote{2.1} the settling and resolving of legal issues whenever they arise there is: 

\scrule{Resolution face-to-face to be applied; }

\scrule{Resolution through recollection to be granted; }

\scrule{Resolution because of past insanity to be granted; }

\scrule{Acting according to what has been admitted; }

\scrule{Majority decision; }

\scrule{Further penalty; }

\scrule{Covering over as if with grass. }

Venerables,\marginnote{2.1} the seven principles for the settling of legal issues have been recited. In regard to this I ask you, ‘Are you pure in this?’ A second time I ask, ‘Are you pure in this?’ A third time I ask, ‘Are you pure in this?’ You are pure in this and therefore silent. I’ll remember it thus.” 

\scendsutta{The seven principles for the settling of legal issues are finished. }

“Venerables,\marginnote{4.1} the introduction has been recited; the eight rules on expulsion have been recited; the seventeen rules on suspension have been recited; the thirty rules on relinquishment and confession have been recited; the one hundred and sixty-six rules on confession have been recited; the eight rules on acknowledgment have been recited; the rules to be trained in have been recited; the seven principles for the settling of legal issues have been recited. This much has come down and is included in the Monastic Code of the Buddha and comes up for recitation every half-month. In regard to this everyone should train in unity, in concord, without dispute.” The Nuns’ Analysis is finished. 

\scendbook{The canonical text beginning with offenses entailing confession is finished. }

%
\backmatter%
%
\chapter*{Appendix: Technical Discussion of Individual \textsanskrit{Bhikkhunī} Rules}
\addcontentsline{toc}{chapter}{Appendix: Technical Discussion of Individual \textsanskrit{Bhikkhunī} Rules}
\markboth{Appendix: Technical Discussion of Individual \textsanskrit{Bhikkhunī} Rules}{Appendix: Technical Discussion of Individual \textsanskrit{Bhikkhunī} Rules}

\subsection*{\textit{\textsanskrit{Bhikkhunī} \textsanskrit{pārājika}} 1}

This rule is largely identical with \textit{bhikkhu \textsanskrit{pārājika}} 1, except for a slight change of wording. For reference, here are the monks’ and the nuns’ versions of this rule:

\begin{quotation}%
“If a monk, after taking on the monks’ training and way of life, without first renouncing the training and revealing his weakness, has sexual intercourse, even with a female animal, he is expelled and excluded from the community.” (\href{https://suttacentral.net/pli-tv-bu-vb-pj1/en/brahmali\#7.1.16.1}{Bu~Pj~1:7.1.16.1})

%
\end{quotation}

\begin{quotation}%
“If a nun willingly has sexual intercourse, even with a male animal, she is expelled and excluded from the community.” (\href{https://suttacentral.net/pli-tv-bi-pm/en/brahmali\#9.1}{Bi~Pm:9.1})

%
\end{quotation}

There are two main differences. The nuns’ rule adds the word “willingly”, \textit{chandaso}, while it omits the phrase “after taking on the monks’ training and way of life, without first renouncing the training and revealing his weakness”.\footnote{\textit{\textsanskrit{Bhikkhūnaṁ} \textsanskrit{sikkhāsājīvasamāpanno} \textsanskrit{sikkhaṁ} \textsanskrit{appaccakkhāya} \textsanskrit{dubbalyaṁ} \textsanskrit{anāvikatvā}.} } I’ve discussed both at length in the introduction to this volume and have nothing to add here.

There are also two minor gender changes, which do warrant commenting on. The agent of the rule is now a \textit{\textsanskrit{bhikkhunī}} rather than a \textit{bhikkhu}, which affects the grammatical gender of any words that agree with the agent, of which there are three in this particular rule (\textit{\textsanskrit{yā}}, \textit{\textsanskrit{pārājikā}}, and \textit{\textsanskrit{asaṁvāsā}}). This is merely a mechanical change that affects all the rules that the monks and the nuns have in common. It has no bearing on the nature or scope of the offense.

The other minor change is marginally more interesting. Towards the end of the rule we find the clause \textit{antamaso \textsanskrit{tiracchānagatāyapi}}, “even with a male animal”, which means that the offense is incurred even for having sex with an animal. For the monks we find the feminine ending, -\textit{\textsanskrit{āya}}, whereas for the nuns we have the masculine ending, -\textit{ena}. This is as expected, since it reasonable to assume that even in the exceptional case of sex with an animal it is more likely to happen with an animal of the opposite sex. This might seem to suggest that one would not incur an offense for having sex with an animal of one’s own gender. However, we know from the detailed exposition of \textit{\textsanskrit{pārājika}} 1 that such sex, too, incurs the full penalty under this rule (\href{https://suttacentral.net/pli-tv-bu-vb-pj1/en/brahmali\#9.1.24}{Bu~Pj~1:9.1.24}).

How then do we explain the gender difference in the animal? The most likely explanation is that the grammatical ending reflects the fact that an animal is either male or female, which means there can be no neutral grammatical ending for animals. And given the lack of a neutral ending, it is only natural to pair the feminine ending with the monks’ rule and the masculine ending with the one for the nuns. The fact that one is forced to make a grammatical gender choice, however, does not mean that the rule is restricted to animals of the equivalent gender.

\subsection*{\textit{\textsanskrit{Bhikkhunī} \textsanskrit{pārājika}} 5}

\textit{\textsanskrit{Bhikkhunī} \textsanskrit{pārājika}} 5 presents us with the seemingly strange case of a \textit{\textsanskrit{bhikkhunī}} committing a \textit{\textsanskrit{pārājika}} offense by consenting to the touch of a man, but no offense at all if she herself touches a man. The rule reads as follows:

\begin{quotation}%
“If a lustful nun consents to a lustful man making physical contact with her, to touching her, to taking hold of her, to contacting her, or to squeezing her, anywhere below the collar bone but above the knees, she too is expelled and excluded from the community. The training rule on above the knees.” (\href{https://suttacentral.nethttps://suttacentral.net/pli-tv-bi-vb-pj5/en/brahmali/en/brahmali\#1.54.1}{Bi~Pj~5:1.54.1})

%
\end{quotation}

To get our bearings, let’s start by investigating the \textsanskrit{Vibhaṅga}. The word-definition section has no relevant information, but the permutation series does. Here is the first permutation, which is typical of the entire series:

\begin{quotation}%
If both have lust and he/she makes physical contact, below the collar bone but above the knees, body to body, she commits an offense entailing expulsion. (\href{https://suttacentral.net/pli-tv-bi-vb-pj5/en/brahmali\#2.2.1}{Bi~Pj~5:2.2.1})

%
\end{quotation}

“Makes physical contact” renders \textit{\textsanskrit{āmasati}}. No agent is specified, and thus my provisional rendering “he/she”. Going by the wording of the rule, it would seem the agent must be the man. The commentary, however, states that the agent may be either the man or the \textit{\textsanskrit{bhikkhunī}}:

\begin{quotation}%
Makes physical contact, body with body: the \textit{\textsanskrit{bhikkhunī}} touches whatever (part of the) body of a man with her body as delimited, or the man touches the body of the \textit{\textsanskrit{bhikkhunī}} as delimited with whatever (part of) his body. In both cases, there is a \textit{\textsanskrit{pārājika}} offense for the \textit{\textsanskrit{bhikkhunī}}.\footnote{Sp 2.659: \textit{\textsanskrit{Kāyena} \textsanskrit{kāyaṁ} \textsanskrit{āmasatīti} \textsanskrit{bhikkhunī} \textsanskrit{yathāparicchinnena} \textsanskrit{kāyena} purisassa \textsanskrit{yaṁkiñci} \textsanskrit{kāyaṁ} puriso \textsanskrit{vā} yena kenaci \textsanskrit{kāyena} \textsanskrit{bhikkhuniyā} \textsanskrit{yathāparicchinnaṁ} \textsanskrit{kāyaṁ} \textsanskrit{āmasati}, \textsanskrit{ubhayathāpi} \textsanskrit{bhikkhuniyā} \textsanskrit{pārājikaṁ}}. \textit{\textsanskrit{Yathāparicchinnena}}, “as delimited”, presumably refers to the fact that the touching must be below the collar bone and above the knees. }

%
\end{quotation}

Is there anything in the Canonical text that supports the commentary? As it happens, there is one unusual case in the permutation series that deserves closer attention. The ambiguity in the agent continues for the entire permutation series, except in one case:

\begin{quotation}%
If both have lust and she makes physical contact with a spirit, a ghost, a \textit{\textsanskrit{paṇḍaka}}, or an animal in human form, below the collar bone but above the knees, body to body, she commits a serious offense. (\href{https://suttacentral.net/pli-tv-bi-vb-pj5/en/brahmali\#2.2.29.1}{Bi~Pj~5:2.2.29.1})

%
\end{quotation}

Contrary to the formulation of the rule, in this sentence it seems required that the \textit{\textsanskrit{bhikkhunī}} is the agent. The various beings that are mentioned here are all in the genitive/dative case, which suggests they are not doing the touching, but are being touched. If the agent of \textit{\textsanskrit{āmasati}} in this case is the \textit{\textsanskrit{bhikkhunī}}, then it would be natural to conclude that the same must be true for the entire permutation series. The parallels in wording are just too close. This might then be taken as lending support to the commentarial position, which would mean that the agent may be either the man or the \textit{\textsanskrit{bhikkhunī}}. This would resolve the ambiguity.

Still, there are some immediate problems with the above suggestion. As we have just seen, the above sentence suggests that the \textit{\textsanskrit{bhikkhunī}} is the only agent. Yet taking the \textit{\textsanskrit{bhikkhunī}} as the sole agent for the entire permutation series goes too far, because it would fly in the face of the wording of the rule. What we have, in fact, are two different and irreconcilable positions: (i) the position of the rule, which says the man is the agent, and (ii) the position of the one sentence from the \textsanskrit{Vibhaṅga}, which says the \textit{\textsanskrit{bhikkhunī}} is the agent. Only in the Commentary is this combined to suggest that either the man or the \textit{\textsanskrit{bhikkhunī}} can be the agent in all cases, including in the rule. Yet it is by no means obvious that the commentary is right about this.

Let’s return to the sentence above that apparently concerns a \textit{\textsanskrit{bhikkhunī}} touching various kinds of beings. The first problem with understanding this sentence as suggested is that it leaves out the case of these beings touching the \textit{\textsanskrit{bhikkhunī}}. This is a rather major problem since the \textit{\textsanskrit{bhikkhunī}} being touched, not the \textit{\textsanskrit{bhikkhunī}} touching, is the concern of the main rule. It would mean that there is no statement anywhere about what the offense is if, for instance, a \textit{\textsanskrit{bhikkhunī}} consents to being touched by a \textit{\textsanskrit{paṇḍaka}}. In this way, a secondary development of a \textit{\textsanskrit{bhikkhunī}} doing the touching would be mentioned, whereas the main concern of the rule would not.

Another problem is that the permutation series would then mention the case of a \textit{\textsanskrit{bhikkhunī}} touching a number of beings of various kinds, including those belonging to non-human realms, but would not mention the case of a \textit{\textsanskrit{bhikkhunī}} touching a man. Thus, the most important case, by far, would be left out.

Given these obvious problems, I do not think it is acceptable to read the entire permutation series in light of this one unusual sentence. Instead, I think we need to follow the reading of the rule, which is only concerned with the case of a \textit{\textsanskrit{bhikkhunī}} consenting to a man touching her. The case of a \textit{\textsanskrit{bhikkhunī}} touching a man would not be covered.

How, then, might we understand the one sentence on other beings in the permutation series? I would suggest this sentence has been corrupted and originally had a structure closer to the one in the main rule. In the main rule the \textit{\textsanskrit{bhikkhunī}} consents to the touch of a man, the man being expressed in the genitive case, \textit{purisapuggalassa}. The syntax of the sentence is such that this makes the man the agent. I would suggest the beings in the permutation series that are expressed in the genitive case should be understood in the same way. They are meant to be the actual agents of the sentence. The syntax no longer supports this, and thus my proposal that there has been a corruption of the text at some point in history. Originally the syntax may have been similar to what we find in the main rule.

Once we look at the text in this way, the problems mentioned above disappear. The case of various beings touching the \textit{\textsanskrit{bhikkhunī}} is then no more than a subsidiary rule under the main one. And the case of a \textit{\textsanskrit{bhikkhunī}} touching a man would not be included.

This suggestion is supported by the parallels to this rule in Chinese translation. Here are the translations of \textit{\textsanskrit{bhikkhunī} \textsanskrit{pārājika}} 5, as found in five different Vinayas, all translated from the Chinese by \textsanskrit{Bhikkhunī} \textsanskrit{Vimalañāṇī}:

\begin{quotation}%
“If a \textit{\textsanskrit{bhikkhunī}}, with a defiled mind, allows a man with a defiled mind, having removed her robes, to rub her up and down below the hairline and above the knees and the wrists, to pull, push, press, and squeeze her, to lift her up and set her down, that \textit{\textsanskrit{bhikkhunī}} commits a \textit{\textsanskrit{pārājika}}, and doesn’t live in community.”\footnote{\textsanskrit{Sarvāstivāda} Vinaya, fascicle 42 (Part 1 of the seventh recitation chapter): \langlzh{若比丘尼有漏心,聽漏心男子髮際以下至腕膝以上却衣,順摩、逆摩、牽推、按掐、抱上、抱下,是比丘尼得波羅夷不共住。」} }

%
\end{quotation}

Here we see the use of the word “allows”, which presumably functions in the same way as “consents to” in the Pali version. The \textsanskrit{Vibhaṅga} to this rule then uses the same term throughout. This means that this is all about the man being the agent, not the \textit{\textsanskrit{bhikkhunī}}.

\begin{quotation}%
“If a \textit{\textsanskrit{bhikkhunī}} with a defiled mind and a man with defiled mind have their bodies touch each other with the thought to feel pleasure, below the eyes and above the knees, even with the slightest touch in this way, that \textit{\textsanskrit{bhikkhunī}} incurs a \textit{\textsanskrit{pārājika}} and shouldn’t live in the community.”\footnote{\textsanskrit{Mūlasarvāstivāda} \textsanskrit{bhikkhunī} vinaya, fascicle 5: \langlzh{「若復苾芻尼自有染心,共染心男子,從目已下、膝已上,作受樂心身相摩觸、若極摩觸。於如是事,此苾芻尼亦得波羅市迦,不應共住。」} }

%
\end{quotation}

Here both the \textit{\textsanskrit{bhikkhunī}} and the man are agents. This would seem to mean that there is only the full offense if both are involved in the touching.

\begin{quotation}%
“If a \textit{\textsanskrit{bhikkhunī}} in a mind-state altered by abundant desire, lets a man rub and touch her in all kinds of ways below the hairline, above the knees, and above the elbows, that \textit{\textsanskrit{bhikkhunī}} incurs a \textit{\textsanskrit{pārājika}}, and doesn’t live in community.”\footnote{\textsanskrit{Mahīśāsaka} \textsanskrit{Vibhaṅga}, pj 5: \langlzh{「若比丘尼,欲盛變心,受男子種種摩觸:髮際已下,膝已上,肘已後。是比丘尼得波羅夷,不共住。」} }

%
\end{quotation}

“Lets (rub and touch)” once again suggests that this has to do only with consent. No agency on the part of the \textit{\textsanskrit{bhikkhunī}} is mentioned.

\begin{quotation}%
“If a \textit{\textsanskrit{bhikkhunī}} with a defiled mind, feels pleasure from being rubbed and touched below the shoulders and above the knees by a man with a defiled mind, that \textit{\textsanskrit{bhikkhunī}} is \textit{\textsanskrit{pārājika}}, and shouldn’t live in community.”\footnote{\textsanskrit{Mahāsaṅghika} \textsanskrit{Vibhaṅga}, pj 5: \langlzh{若比丘尼漏心。漏心男子邊肩以下膝以上摩觸受樂者。是比丘尼波羅夷不應共住。} }

%
\end{quotation}

Here the term “feels pleasure” is likely to be equivalent to “consents to”. The Pali word behind “to consent” is \textit{\textsanskrit{sādiyati}}, which can also be rendered as “to feel pleasure”. This, then, is a third case of only the man being the agent.

\begin{quotation}%
“If a \textit{\textsanskrit{bhikkhunī}} with a defiled mind and a man with a defiled mind have their bodies touch each other below the armpits and above the knees; if they grasp, pull, push, rub upwards and downwards, lift up, set down, hold, and press each other, that \textit{\textsanskrit{bhikkhunī}} is \textit{\textsanskrit{pārājika}}, and doesn’t live in community, because she’s ‘one with bodily contact’.”\footnote{Dharmaguptaka \textsanskrit{Vibhaṅga}, pj 5: \langlzh{若比丘尼染污心,共染污心男子,從腋已下膝已上身相觸,若捉摩、若牽、若推、若上摩、若下摩、若舉、若下、若捉、若捺,是比丘尼波羅夷,不共住。是身相觸也。」} }

%
\end{quotation}

This is a second case of both the \textit{\textsanskrit{bhikkhunī}} and the man being agents. Yet even here the \textsanskrit{Vibhaṅga} focuses on the man as the agent.

If we include the Pali version, we have four cases where the man is the only agent, and two cases where also the \textit{\textsanskrit{bhikkhunī}} is an agent. This suggests that only the man being the agent is the earlier reading. Moreover, in the two rules where both are agents, the natural interpretation is that the rule is only broken if \textit{both} take part in the touching. And so again, if the \textit{\textsanskrit{bhikkhunī}} is the sole agent, there would be no offense. We have a situation where the man touching is always required for the committing of a \textit{\textsanskrit{pārājika}} offense, whereas the \textit{\textsanskrit{bhikkhunī}} touching is never sufficient in itself.

My overall conclusion from this brief comparative study is that there is no \textit{\textsanskrit{pārājika}} offense for a \textit{\textsanskrit{bhikkhunī}} who touches a man, whether she is lustful or not, assuming the man does not touch her. I would suggest the Pali should be interpreted accordingly.

\subsection*{\textit{\textsanskrit{Bhikkhunī} \textsanskrit{pārājika}} 8}

In \textit{\textsanskrit{bhikkhunī} \textsanskrit{pārājika}} 8 there is a series of eight actions. The question is whether a \textit{\textsanskrit{bhikkhunī}} has to do all eight or whether doing a single one is enough to commit the offense. Here is the rule for reference:

\begin{quotation}%
“If, for the purpose of indulging in inappropriate sexual conduct, a lustful nun consents to a lustful man holding her hand and the edge of her robe, and she stands with him and chats with him and goes to a rendezvous with him and consents to him coming to her and enters a covered place with him and disposes her body for him for that purpose, she too is expelled and excluded from the community. The training rule having eight parts.” (\href{https://suttacentral.net/pli-tv-bi-vb-pj8/en/brahmali\#1.11.1}{Bi~Pj~8:1.11.1})

%
\end{quotation}

On the surface, it appears as if doing any one of the eight is sufficient. The eight actions are connected with the disjunctive \textit{\textsanskrit{vā}}, which is almost universally translated as “or”. However, in the word definitions that immediately follow the rule formulation, it is equally clear that doing a single action is not enough to commit a \textit{\textsanskrit{pārājika}}. Doing any of the sub-actions is stated to be a serious offense, a \textit{thullaccaya}. Going by this, all eight actions need to be done to incur the full offense of \textit{\textsanskrit{pārājika}}.

In fact, the interpretation of the word-definitions section is the only tenable one. Most of the sub-actions described in Bi Pj 8 are found as lesser rules elsewhere in the \textsanskrit{Bhikkhunī}-\textsanskrit{pātimokkha}. For instance, if “a lustful nun consents to a lustful man holding her hand”, it is a serious offense at \href{https://suttacentral.net/pli-tv-bi-vb-pj8/en/brahmali\#2.1.11}{Bi~Pj~8:2.1.11}. Or if “a lustful nun consents to a lustful man holding the edge of her robe”, it is again a serious offense under the same rule at \href{https://suttacentral.net/pli-tv-bi-vb-pj8/en/brahmali\#2.1.14}{Bi~Pj~8:2.1.14}. Or if “(a lustful nun) stands with him” or “she chats with him”, it is an offense entailing confession, a \textit{\textsanskrit{pācittiya}}, at \href{https://suttacentral.net/pli-tv-bi-vb-pc11/en/brahmali\#1.12.1}{Bi~Pc~11}. I could go on, but that should be sufficient to make the point. Making each of the actions at Bi Pj 8 an offense of \textit{\textsanskrit{pārājika}} would clash with how these actions are treated elsewhere. Again, we are compelled to interpret this rule as all eight actions needing to be fulfilled.

This is also the position of the commentary:

\begin{quotation}%
Even if she fulfills one of the eight or seven of the eight a hundred times, she is not expelled.\footnote{Sp 2.676: \textit{\textsanskrit{Yā} pana \textsanskrit{ekaṁ} \textsanskrit{vā} \textsanskrit{vatthuṁ} satta \textsanskrit{vā} \textsanskrit{vatthūni} satakkhattumpi \textsanskrit{pūreti}, neva \textsanskrit{assamaṇī} hoti}. }

%
\end{quotation}

Moreover, according to the \textsanskrit{Kaṅkhāvitaraṇī} commentary, if, before committing the eighth and final action that would result in a \textit{\textsanskrit{pārājika}} offense, a nun confesses any of the previous seven partial transgressions, she does not incur a \textit{\textsanskrit{pārājika}}:

\begin{quotation}%
Therefore, if she has committed one (of the actions), but then makes a pledge of obligation (to practice the rules) and then confesses, if she then commits an offense again because of defilements, and confesses again, then even if she fulfills the eighth factor, she does not commit an offense entailing expulsion.\footnote{\textit{\textsanskrit{Tasmā} \textsanskrit{yā} \textsanskrit{ekaṁ} \textsanskrit{āpannā} \textsanskrit{dhuranikkhepaṁ} \textsanskrit{katvā} \textsanskrit{desetvā} puna kilesavasena \textsanskrit{āpajjati}, puna pi deseti, evam \textsanskrit{aṭṭhamaṁ} \textsanskrit{paripūrentī} pi \textsanskrit{pārājika} na hoti.} }

%
\end{quotation}

The evidence, then, strongly suggests that the \textit{\textsanskrit{vā}} in the rule must be understood as “and” rather than “or”. Taking a closer look at Pali literature, we discover that \textit{\textsanskrit{vā}} is quite commonly used as a conjunctive instead of a disjunctive. The same, it seems, is true for Sanskrit literature. Starting with the Sanskrit, SED says this:

\begin{quotation}%
\textit{\textsanskrit{Vā}} is sometimes interchangeable with \textit{ca} and \textit{api} … (sv. \textit{\textsanskrit{vā}})

%
\end{quotation}

Closer to home, the Pali commentaries frequently gloss \textit{\textsanskrit{vā}} as a conjunction, defining it with the terms \textit{samuccaya} and \textit{\textsanskrit{sampiṇḍana}}, both of which refer to coming together, rather than disjunction. A particularly instructive case is the following:

\begin{quotation}%
“But there will be three dangers for \textsanskrit{Pāṭaliputta}: fire, water, and (\textit{\textsanskrit{vā}}) internal dissension.”\footnote{\textit{\textsanskrit{Pāṭaliputtassa} kho, \textsanskrit{ānanda}, tayo \textsanskrit{antarāyā} bhavissanti – aggito \textsanskrit{vā} udakato \textsanskrit{vā} abbhantarato \textsanskrit{vā} \textsanskrit{mithubhedā}.} } (\href{https://suttacentral.net/pli-tv-kd6/en/brahmali\#28.8.9}{Kd~6:28.8.9})

%
\end{quotation}

Here the text is specific that there will be three dangers and so the \textit{\textsanskrit{vā}}, “or”, cannot mean that only one of the three alternatives will occur. At the same time, it seems unlikely that the three should be seen as happening simultaneously, which may explain why \textit{ca}, “and”, is not used. It follows that \textit{\textsanskrit{vā}} here should most likely be understood conjunctively in the sense of all three occurring, but not conjunctively in the sense of happening simultaneously.

Another striking example of the same phenomenon is found in the \textsanskrit{Ānāpānassati} Sutta. Here is the relevant extract in Pali, followed by a translation, rendering \textit{\textsanskrit{vā}} as “or”:

\begin{quotation}%
\textit{\textsanskrit{Dīghaṁ} \textbf{\textsanskrit{vā}} assasanto \textsanskrit{dīghaṁ} \textsanskrit{assasāmīti} \textsanskrit{pajānāti}, \textsanskrit{dīghaṁ} \textbf{\textsanskrit{vā}} passasanto \textsanskrit{dīghaṁ} \textsanskrit{passasāmīti} \textsanskrit{pajānāti}; \textsanskrit{rassaṁ} \textbf{\textsanskrit{vā}} assasanto \textsanskrit{rassaṁ} \textsanskrit{assasāmīti} \textsanskrit{pajānāti}, \textsanskrit{rassaṁ} \textbf{\textsanskrit{vā}} passasanto \textsanskrit{rassaṁ} \textsanskrit{passasāmīti} \textsanskrit{pajānāti}.}

%
\end{quotation}

\begin{quotation}%
“When breathing in heavily they know: ‘I’m breathing in heavily.’ \textbf{Or} when breathing out heavily they know: ‘I’m breathing out heavily.’ \textbf{Or} when breathing in lightly they know: ‘I’m breathing in lightly.’ \textbf{Or} when breathing out lightly they know: ‘I’m breathing out lightly.’” (\href{https://suttacentral.net/mn118/en/sujato\#18.1}{MN~118:18.1})

%
\end{quotation}

Clearly, one does not either know the breathing in or the breathing out. Rather, one first knows one, then the other, alternating between the two. Once again, the use of the \textit{\textsanskrit{vā}} makes the point that the two actions do not happen simultaneously. To make the same point in English one would have to render the \textit{\textsanskrit{vā}} as “and”, or one could simply drop it altogether:

\begin{quotation}%
“When breathing in heavily they know: ‘I’m breathing in heavily.’ When breathing out heavily they know: ‘I’m breathing out heavily.’ When breathing in lightly they know: ‘I’m breathing in lightly.’ When breathing out lightly they know: ‘I’m breathing out lightly.’”

%
\end{quotation}

Both of the above examples are exact parallels to the situation in the present \textit{\textsanskrit{pārājika}} rule. In this rule we have eight factors, all of which need to be fulfilled, but not at the same time. Based on this precedent, and the explanation found in the \textsanskrit{Vibhaṅga}, I believe it is only reasonable to render \textit{\textsanskrit{vā}} as “and” also in Bi Pj 8.

Finally, I am told that \textit{ca} and \textit{\textsanskrit{vā}} are so similar in Sinhala characters that they are often confused when manuscripts are copied. In fact, this phenomenon is known to the commentaries:

\begin{quotation}%
For recently, in some places in the written book the word \textit{\textsanskrit{vā}} is seen, but also the word \textit{ca}.\footnote{\textsanskrit{Sīlakkhandhavagga}-\textsanskrit{Abhinavaṭīkā}, 166: \textit{\textsanskrit{Adhunā} hi katthaci potthake \textsanskrit{vā}-saddo, ca-saddopi dissati}. My thanks to Ven. \textsanskrit{Dhammānando} who kindly provided me with this information. }

%
\end{quotation}

\subsection*{\textit{\textsanskrit{Bhikkhunī} \textsanskrit{pācittiya}} 54}

This rule is very similar to \textit{bhikkhu \textsanskrit{pācittiya}} 35, except for two things: (1) it uses the word \textit{\textsanskrit{nimantitā}} (“invited”) in place of \textit{\textsanskrit{bhuttāvī}} (“eaten”); and (2), it adds a \textit{\textsanskrit{vā}}, “or”, which at first glance seems out of place. Here are the two rules for easy reference:

\begin{quotation}%
“If a nun, who has been invited to a meal (\textit{\textsanskrit{nimantitā}}), refuses an offer to eat more (\textit{\textsanskrit{pavāritā}}), and then eats fresh or cooked food, she commits an offense entailing confession.”\footnote{The two relevant \textit{\textsanskrit{vās}}, “ors”, can be seen here: \textit{\textsanskrit{Yā} pana \textsanskrit{bhikkhunī} \textsanskrit{nimantitā} \textbf{\textsanskrit{vā}} \textsanskrit{pavāritā} \textbf{\textsanskrit{vā}} \textsanskrit{khādanīyaṁ} \textsanskrit{vā} \textsanskrit{bhojanīyaṁ} \textsanskrit{vā} \textsanskrit{khādeyya} \textsanskrit{vā} \textsanskrit{bhuñjeyya} \textsanskrit{vā}, \textsanskrit{pācittiyanti}}. } (\href{https://suttacentral.net/pli-tv-bi-vb-pc54/en/brahmali\#1.20.1}{Bi~Pc~54})

%
\end{quotation}

\begin{quotation}%
“If a monk has finished his meal (\textit{\textsanskrit{bhuttāvī}}) and refused an invitation to eat more (\textit{\textsanskrit{pavāritā}}), and then eats fresh or cooked food that is not left over, he commits an offense entailing confession.” (\href{https://suttacentral.net/pli-tv-bu-vb-pc35/en/brahmali\#2.15.1}{Bu~Pc~35})

%
\end{quotation}

The origin story to \href{https://suttacentral.net/pli-tv-bi-vb-pc54/en/brahmali\#1.1}{Bi~Pc~54} is practically identical to the one at \href{https://suttacentral.net/pli-tv-bu-vb-pc35/en/brahmali\#1.1}{Bu~Pc~35}, and in both stories the word \textit{\textsanskrit{bhuttāvī}} is used together with \textit{\textsanskrit{pavārita}}. In the \textit{bhikkhu} rule, \textit{\textsanskrit{bhuttāvī}} is then taken into the rule, but not in the \textit{\textsanskrit{bhikkhunī}} rule, where it is replaced by \textit{\textsanskrit{nimantitā}}. The significance of this difference, even whether there is one, is by no means clear.

One solution is to regard \textit{\textsanskrit{nimantitā}} and \textit{\textsanskrit{bhuttāvī}} as synonyms or near-synonyms. If so, then instead of two different situations in which the present rule is breached (\textit{\textsanskrit{nimantitā}} or \textit{\textsanskrit{pavāritā}}), there is only one, which is what we have in the \textit{bhikkhu} rule. Yet this is not quite satisfactory given the non-offense clause of Bi Pc 54, according to which there is no offense if you have been \textit{\textsanskrit{nimantitā}} but not \textit{\textsanskrit{pavāritā}}.

An alternative solution is to understand \textit{\textsanskrit{vā}} as “and”, as it must in \textit{\textsanskrit{bhikkhunī} \textsanskrit{pārājika}} 8, instead of as “or”. In fact, the \textsanskrit{Kaṅkhāvitaraṇī} commentary has this to say:

\begin{quotation}%
In the fourth rule (that is, the present rule) \textit{\textsanskrit{nimantitā}} is to be understood by the method spoken of in the rule on group meals (\href{https://suttacentral.net/pli-tv-bu-vb-pc32/en/brahmali\#8.15.1}{Bu~Pc~32}); and \textit{\textsanskrit{pavāritā}} is to be understood by the method spoken of in the rule on invitation (\href{https://suttacentral.net/pli-tv-bu-vb-pc35/en/brahmali\#2.15.1}{Bu~Pc~35}).\footnote{\textit{Catutthe \textsanskrit{gaṇabhojane} vuttanayena \textsanskrit{nimantitā}, \textsanskrit{pavāraṇāsikkhāpade} vuttanayena \textsanskrit{pavāritā} \textsanskrit{veditabbā}.} }

%
\end{quotation}

At Bu Pc 32, \textit{nimantita} is used to signify that one has been invited to a meal. In contrast to \textit{\textsanskrit{pavārita}}, it is not used to indicate an offer to eat specific foods. \textit{Nimantita} is thus something that happens before one arrives to a meal offering, whereas \textit{\textsanskrit{pavārita}}, as used in Bu Pc 35, indicates an offering of food into one’s hands, specifically a refusal of such food.

On this understanding the two words are not synonymous, but instead a sequence of actions. This fits with the non-offense clause of the present rule, where there is no offense if one has simply been invited, but \emph{not} refused an offer to eat more (\href{https://suttacentral.net/pli-tv-bi-vb-pc54/en/brahmali\#2.2.7}{Bi~Pc~54:2.2.7}). Here the two actions are clearly separate. To bring this out in translation, however, necessitates rendering \textit{\textsanskrit{vā}} as “and”, as in Bi Pj 8, for which see the discussion above. To my mind, this is the best solution to the present conundrum, and I translate accordingly.

It is not at all implausible that the difficulty with interpreting the present rule is a result of corruption. According to private communication from Ven. \textsanskrit{Vimalañāṇī}, a number of the other schools of Buddhism have this rule as shared between the \textit{bhikkhus} and \textit{\textsanskrit{bhikkhunīs}}, which means that the wording of the rules is the same. Specifically, the \textit{\textsanskrit{bhikkhunī}} rule has the equivalent of the Pali \textit{\textsanskrit{bhuttāvī}} and \textit{\textsanskrit{pavāritā}}, rather than \textit{\textsanskrit{nimantitā}} and \textit{\textsanskrit{pavāritā}}.

Here are the details for the individual schools as told by Ven. \textsanskrit{Vimalañāṇī}. The \textsanskrit{Lokuttaravādins} (extant text in Sanskrit), the \textsanskrit{Mahīśāsakas}, and the \textsanskrit{Mūlasarvāstivādins} (both with extant texts in Chinese) all have the equivalent of \textit{\textsanskrit{bhuttāvī}} instead of \textit{\textsanskrit{pavāritā}}. For the \textsanskrit{Mahāsaṅghikas} (extant text in Chinese), the \textit{bhikkhus} and \textit{\textsanskrit{bhikkhunīs}} have a shared rule that is closer to \href{https://suttacentral.net/pli-tv-bu-vb-pc36/en/brahmali\#1.28.1}{Bi~Pc~36} of the Pali, which again means that its reading would have been \textit{\textsanskrit{bhuttāvī}}, not \textit{\textsanskrit{nimantitā}}. For the Dharmaguptakas, the \textit{\textsanskrit{bhikkhunīs}} have a rule that is not held in common with the \textit{bhikkhus}, but the reading of the Chinese once again suggests the original was \textit{\textsanskrit{bhuttāvī}}. Given this strong preference for \textit{\textsanskrit{bhuttāvī}} in all schools apart from the Pali, it seems quite possible that the Pali text has been corrupted.

\subsection*{\textit{\textsanskrit{Bhikkhunī} \textsanskrit{pācittiya}} 81}

Here is the rule:

\begin{quotation}%
“If, when a given consent has expired (\textit{\textsanskrit{pārivāsikachandadānena}}), a nun gives the full admission to a trainee nun, she commits an offense entailing confession.” (\href{https://suttacentral.net/pli-tv-bi-vb-pc81/en/brahmali\#1.13.1}{Bi~Pc~81})

%
\end{quotation}

The main problem here is the meaning of \textit{\textsanskrit{pārivāsikachandadānena}}. The latter part of the compound, \textit{\textsanskrit{chandadānena}}, most likely means, “by the giving of consent”, or “by the passing on of consent”, referring to the consent passed on by monastics who are not present at a legal procedure of the Sangha (see Bhikkhu Sujato, “Bhikkhuni Vinaya Studies”, pp. 205–210). The first part of the compound, \textit{\textsanskrit{pārivāsika}}, is trickier. This is what the commentaries say:

\begin{quotation}%
Again, the monks are seated, thinking, “We will do a legal procedure, such as a rehabilitation, etc.,” but then a monk who is an expert in constellations says, “Today the constellations are inauspicious; don’t do this procedure.” Because of his statement, they withdraw their consent (\textit{\textsanskrit{chandaṁ} \textsanskrit{vissajjetvā}}), but remain seated. Then another monk arrives and says: “What is beneficial is lost for the fool who honors constellations. Why worry about constellations?” This is expired consent and expired intention. With this sort of outdatedness, if one does not convey consent and purity anew, it is not allowable to do the procedure.\footnote{Sp 2.1167: \textit{Puna \textsanskrit{bhikkhū} “\textsanskrit{kiñcideva} \textsanskrit{abbhānādisaṅghakammaṁ} \textsanskrit{karissāmā}”ti \textsanskrit{nisinnā} honti, tatreko \textsanskrit{nakkhattapāṭhako} bhikkhu \textsanskrit{evaṁ} vadati – “ajja \textsanskrit{nakkhattaṁ} \textsanskrit{dāruṇaṁ}, \textsanskrit{mā} \textsanskrit{imaṁ} \textsanskrit{kammaṁ} \textsanskrit{karothā}”ti. Te tassa vacanena \textsanskrit{chandaṁ} \textsanskrit{vissajjetvā} tattheva \textsanskrit{nisinnā} honti. \textsanskrit{Athañño} \textsanskrit{āgantvā} “\textsanskrit{nakkhattaṁ} \textsanskrit{paṭimānentaṁ} attho \textsanskrit{bālaṁ} \textsanskrit{upaccagā}”ti \textsanskrit{vatvā} “\textsanskrit{kiṁ} nakkhattena \textsanskrit{karothā}”ti vadati. \textsanskrit{Idaṁ} \textsanskrit{chandapārivāsiyañceva} \textsanskrit{ajjhāsayapārivāsiyañca}. \textsanskrit{Etasmiṁ} \textsanskrit{pārivāsiye} puna \textsanskrit{chandapārisuddhiṁ} \textsanskrit{anānetvā} \textsanskrit{kammaṁ} \textsanskrit{kātuṁ} na \textsanskrit{vaṭṭati}.} }

%
\end{quotation}

\begin{quotation}%
\textit{\textsanskrit{Chandaṁ} \textsanskrit{vissajjetvā}}: about this, it is said in the \textsanskrit{Anugaṇṭhipada}: “This \textit{\textsanskrit{saṅghakamma}} is not to be done today. Having said, ‘(Do it) at your convenience’, (the consent) is given …”\footnote{Vjb 2.1166: \textit{\textsanskrit{Chandaṁ} \textsanskrit{vissajjetvāti} ettha \textsanskrit{anugaṇṭhipade} \textsanskrit{evaṁ} \textsanskrit{vuttaṁ} “\textsanskrit{idaṁ} \textsanskrit{kammaṁ} ajja na \textsanskrit{kattabbaṁ}. ‘\textsanskrit{Yathāsukha}’nti \textsanskrit{vatvā} \textsanskrit{vissajjitaṁ} hoti …}”. }

%
\end{quotation}

This interpretation, however, seems strained and artificial. Consent is normally given by those who cannot be present at a particular meeting, not as suggested here, by an entire group of people who may or may not be present at an unspecified future meeting. I prefer to interpret this rule to refer to the consent that has been given prior to a particular ordination ceremony. If the ceremony is postponed—which is defined in the \textsanskrit{Vibhaṅga} as the “assembly having risen” —then that consent is no longer valid for any new ceremony happening later. Renewed consent is required for the postponed ordination.

It would seem strange, however, to limit this regulation to ordination ceremonies. It is reasonable to think it should apply to all \textit{\textsanskrit{saṅghakamma}}. We do, in fact, see indications of this elsewhere in the Vinaya \textsanskrit{Piṭaka}. At Kd 2 and Kd 4 we find the following regulations:

\begin{quotation}%
“You shouldn’t do the observance-day ceremony with a passed-on purity that has expired (\textit{\textsanskrit{pārivāsikapārisuddhidānena}}), except if the gathering is still seated together.” (\href{https://suttacentral.net/pli-tv-kd2/en/brahmali\#36.4.1}{Kd~2:36.4.1})

%
\end{quotation}

\begin{quotation}%
“You shouldn’t do the invitation ceremony with a passed-on invitation that has expired (\textit{\textsanskrit{pārivāsikapavāraṇādānena}}), except if the gathering is still seated together.” (\href{https://suttacentral.net/pli-tv-kd4/en/brahmali\#14.4.1}{Kd~4:14.4.1})

%
\end{quotation}

The two compounds here given in Pali are the same as the one we have discussed above, except that \textit{chanda} is replaced by \textit{\textsanskrit{pārisuddhi}} and by \textit{\textsanskrit{pavāraṇā}} respectively. Both of these situations parallel what we have seen earlier, namely, that neither the observance-day ceremony not the invitation ceremony should be done if a previously given consent has expired, as explained above. Given three instances of the same principle applied to a variety of \textit{\textsanskrit{saṅghakamma}}, we can infer it pertains to all legal procedures.

%
\chapter*{Appendices}
\addcontentsline{toc}{chapter}{Appendices}
\markboth{Appendices}{Appendices}

\emph{Appendices for all volumes may be found at the end of the first volume, The Great Analysis, part I.}

%
\chapter*{Colophon}
\addcontentsline{toc}{chapter}{Colophon}
\markboth{Colophon}{Colophon}

\section*{The Translator}

Bhikkhu Brahmali was born Norway in 1964. He first became interested in Buddhism and meditation in his early 20s after a visit to Japan. Having completed degrees in engineering and finance, he began his monastic training as an anagarika (keeping the eight precepts) in England at Amaravati and Chithurst Buddhist Monastery.

After hearing teachings from Ajahn Brahm he decided to travel to Australia to train at Bodhinyana Monastery. Bhikkhu Brahmali has lived at Bodhinyana Monastery since 1994, and was ordained as a Bhikkhu, with Ajahn Brahm as his preceptor, in 1996. In 2015 he entered his 20th Rains Retreat as a fully ordained monastic and received the title Maha Thera (Great Elder).

Bhikkhu Brahmali’s knowledge of the Pali language and of the Suttas is excellent. Bhikkhu Bodhi, who translated most of the Pali Canon into English for Wisdom Publications, called him one of his major helpers for the 2012 translation of \emph{The Numerical Discourses of the Buddha}. He has also published two essays on Dependent Origination and a book called \emph{The Authenticity of the Early Buddhist Texts} with the Buddhist Publication Society in collaboration with Bhante Sujato.

The monastics of the Buddhist Society of WA (BSWA) often turn to him to clarify Vinaya (monastic discipline) or Sutta questions. They also greatly appreciate his Sutta and Pali classes. Furthermore he has been instrumental in most of the building and maintenance projects at Bodhinyana Monastery and at the emerging Hermit Hill property in Serpentine.

\section*{Creation Process}

Translated from the Pali. The primary source was the \textsanskrit{Mahāsaṅgīti} edition, with occasional reference to other Pali editions, especially the \textsanskrit{Chaṭṭha} \textsanskrit{Saṅgāyana} edition and the Pali Text Society edition. I cross-checked with I.B. Horner’s English translation, “The Book of the Discipline”, as well as Bhikkhu \textsanskrit{Ñāṇatusita}’s “A Translation and Analysis of the \textsanskrit{Pātimokkha}” and Ajahn \textsanskrit{Ṭhānissaro}’s “Buddhist Monastic Code”.

\section*{The Translation}

This is the first complete translation of the Vinaya \textsanskrit{Piṭaka} in English. The aim has been to produce a translation that is easy to read, clear, and accurate, and also modern in vocabulary and style.

\section*{About SuttaCentral}

SuttaCentral publishes early Buddhist texts. Since 2005 we have provided root texts in Pali, Chinese, Sanskrit, Tibetan, and other languages, parallels between these texts, and translations in many modern languages. Building on the work of generations of scholars, we offer our contribution freely.

SuttaCentral is driven by volunteer contributions, and in addition we employ professional developers. We offer a sponsorship program for high quality translations from the original languages. Financial support for SuttaCentral is handled by the SuttaCentral Development Trust, a charitable trust registered in Australia.

\section*{About Bilara}

“Bilara” means “cat” in Pali, and it is the name of our Computer Assisted Translation (CAT) software. Bilara is a web app that enables translators to translate early Buddhist texts into their own language. These translations are published on SuttaCentral with the root text and translation side by side.

\section*{About SuttaCentral Editions}

The SuttaCentral Editions project makes high quality books from selected Bilara translations. These are published in formats including HTML, EPUB, PDF, and print.

You are welcome to print any of our Editions.

%
\end{document}