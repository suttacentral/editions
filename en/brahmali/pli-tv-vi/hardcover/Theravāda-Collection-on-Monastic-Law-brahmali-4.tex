\documentclass[12pt,openany]{book}%
\usepackage{lastpage}%
%
\usepackage{ragged2e}
\usepackage{verse}
\usepackage[a-3u]{pdfx}
\usepackage[inner=1in, outer=1in, top=.7in, bottom=1in, papersize={6in,9in}, headheight=13pt]{geometry}
\usepackage{polyglossia}
\usepackage[12pt]{moresize}
\usepackage{soul}%
\usepackage{microtype}
\usepackage{tocbasic}
\usepackage{realscripts}
\usepackage{epigraph}%
\usepackage{setspace}%
\usepackage{sectsty}
\usepackage{fontspec}
\usepackage{marginnote}
\usepackage[bottom]{footmisc}
\usepackage{enumitem}
\usepackage{fancyhdr}
\usepackage{emptypage}
\usepackage{extramarks}
\usepackage{graphicx}
\usepackage{relsize}
\usepackage{etoolbox}

% improve ragged right headings by suppressing hyphenation and orphans. spaceskip plus and minus adjust interword spacing; increase rightskip stretch to make it want to push a word on the first line(s) to the next line; reduce parfillskip stretch to make line length more equal . spacefillskip and xspacefillskip can be deleted to use defaults.
\protected\def\BalancedRagged{
\leftskip     0pt
\rightskip    0pt plus 10em
\spaceskip=1\fontdimen2\font plus .5\fontdimen3\font minus 1.5\fontdimen4\font
\xspaceskip=1\fontdimen2\font plus 1\fontdimen3\font minus 1\fontdimen4\font
\parfillskip  0pt plus 15em
\relax
}

\hypersetup{
colorlinks=true,
urlcolor=black,
linkcolor=black,
citecolor=black,
allcolors=black
}

% use a small amount of tracking on small caps
\SetTracking[ spacing = {25*,166, } ]{ encoding = *, shape = sc }{ 25 }

% add a blank page
\newcommand{\blankpage}{
\newpage
\thispagestyle{empty}
\mbox{}
\newpage
}

% define languages
\setdefaultlanguage[]{english}
\setotherlanguage[script=Latin]{sanskrit}

%\usepackage{pagegrid}
%\pagegridsetup{top-left, step=.25in}

% define fonts
% use if arno sanskrit is unavailable
%\setmainfont{Gentium Plus}
%\newfontfamily\Marginalfont[]{Gentium Plus}
%\newfontfamily\Allsmallcapsfont[RawFeature=+c2sc]{Gentium Plus}
%\newfontfamily\Noligaturefont[Renderer=Basic]{Gentium Plus}
%\newfontfamily\Noligaturecaptionfont[Renderer=Basic]{Gentium Plus}
%\newfontfamily\Fleuronfont[Ornament=1]{Gentium Plus}

% use if arno sanskrit is available. display is applied to \chapter and \part, subhead to \section and \subsection.
\setmainfont[
  FontFace={sb}{n}{Font = {Arno Pro Semibold}},
  FontFace={sb}{it}{Font = {Arno  Pro Semibold Italic}}
]{Arno Pro}

% create commands for using semibold
\DeclareRobustCommand{\sbseries}{\fontseries{sb}\selectfont}
\DeclareTextFontCommand{\textsb}{\sbseries}

\newfontfamily\Marginalfont[RawFeature=+subs]{Arno Pro Regular}
\newfontfamily\Allsmallcapsfont[RawFeature=+c2sc]{Arno Pro}
\newfontfamily\Noligaturefont[Renderer=Basic]{Arno Pro}
\newfontfamily\Noligaturecaptionfont[Renderer=Basic]{Arno Pro Caption}

% chinese fonts
\newfontfamily\cjk{Noto Serif TC}
\newcommand*{\langlzh}[1]{\cjk{#1}\normalfont}%

% logo
\newfontfamily\Logofont{sclogo.ttf}
\newcommand*{\sclogo}[1]{\large\Logofont{#1}}

% use subscript numerals for margin notes
\renewcommand*{\marginfont}{\Marginalfont}

% ensure margin notes have consistent vertical alignment
\renewcommand*{\marginnotevadjust}{-.17em}

% use compact lists
\setitemize{noitemsep,leftmargin=1em}
\setenumerate{noitemsep,leftmargin=1em}
\setdescription{noitemsep, style=unboxed, leftmargin=1em}

% style ToC
\DeclareTOCStyleEntries[
  raggedentrytext,
  linefill=\hfill,
  pagenumberwidth=.5in,
  pagenumberformat=\normalfont,
  entryformat=\normalfont
]{tocline}{chapter,section}


  \setlength\topsep{0pt}%
  \setlength\parskip{0pt}%

% define new \centerpars command for use in ToC. This ensures centering, proper wrapping, and no page break after
\def\startcenter{%
  \par
  \begingroup
  \leftskip=0pt plus 1fil
  \rightskip=\leftskip
  \parindent=0pt
  \parfillskip=0pt
}
\def\stopcenter{%
  \par
  \endgroup
}
\long\def\centerpars#1{\startcenter#1\stopcenter}

% redefine part, so that it adds a toc entry without page number
\let\oldcontentsline\contentsline
\newcommand{\nopagecontentsline}[3]{\oldcontentsline{#1}{#2}{}}

    \makeatletter
\renewcommand*\l@part[2]{%
  \ifnum \c@tocdepth >-2\relax
    \addpenalty{-\@highpenalty}%
    \addvspace{0em \@plus\p@}%
    \setlength\@tempdima{3em}%
    \begingroup
      \parindent \z@ \rightskip \@pnumwidth
      \parfillskip -\@pnumwidth
      {\leavevmode
       \setstretch{.85}\large\scshape\centerpars{#1}\vspace*{-1em}\llap{#2}}\par
       \nobreak
         \global\@nobreaktrue
         \everypar{\global\@nobreakfalse\everypar{}}%
    \endgroup
  \fi}
\makeatother

\makeatletter
\def\@pnumwidth{2em}
\makeatother

% define new sectioning command, which is only used in volumes where the pannasa is found in some parts but not others, especially in an and sn

\newcommand*{\pannasa}[1]{\clearpage\thispagestyle{empty}\begin{center}\vspace*{14em}\setstretch{.85}\huge\itshape\scshape\MakeLowercase{#1}\end{center}}

    \makeatletter
\newcommand*\l@pannasa[2]{%
  \ifnum \c@tocdepth >-2\relax
    \addpenalty{-\@highpenalty}%
    \addvspace{.5em \@plus\p@}%
    \setlength\@tempdima{3em}%
    \begingroup
      \parindent \z@ \rightskip \@pnumwidth
      \parfillskip -\@pnumwidth
      {\leavevmode
       \setstretch{.85}\large\itshape\scshape\lowercase{\centerpars{#1}}\vspace*{-1em}\llap{#2}}\par
       \nobreak
         \global\@nobreaktrue
         \everypar{\global\@nobreakfalse\everypar{}}%
    \endgroup
  \fi}
\makeatother

% don't put page number on first page of toc (relies on etoolbox)
\patchcmd{\chapter}{plain}{empty}{}{}

% global line height
\setstretch{1.05}

% allow linebreak after em-dash
\catcode`\—=13
\protected\def—{\unskip\textemdash\allowbreak}

% style headings with secsty. chapter and section are defined per-edition
\partfont{\setstretch{.85}\normalfont\centering\textsc}
\subsectionfont{\setstretch{.95}\normalfont\BalancedRagged}%
\subsubsectionfont{\setstretch{1}\normalfont\itshape\BalancedRagged}

% style elements of suttatitle
\newcommand*{\suttatitleacronym}[1]{\smaller[2]{#1}\vspace*{.3em}}
\newcommand*{\suttatitletranslation}[1]{\linebreak{#1}}
\newcommand*{\suttatitleroot}[1]{\linebreak\smaller[2]\itshape{#1}}

\DeclareTOCStyleEntries[
  indent=3.3em,
  dynindent,
  beforeskip=.2em plus -2pt minus -1pt,
]{tocline}{section}

\DeclareTOCStyleEntries[
  indent=0em,
  dynindent,
  beforeskip=.4em plus -2pt minus -1pt,
]{tocline}{chapter}

\newcommand*{\tocacronym}[1]{\hspace*{-3.3em}{#1}\quad}
\newcommand*{\toctranslation}[1]{#1}
\newcommand*{\tocroot}[1]{(\textit{#1})}
\newcommand*{\tocchapterline}[1]{\bfseries\itshape{#1}}


% redefine paragraph and subparagraph headings to not be inline
\makeatletter
% Change the style of paragraph headings %
\renewcommand\paragraph{\@startsection{paragraph}{4}{\z@}%
            {-2.5ex\@plus -1ex \@minus -.25ex}%
            {1.25ex \@plus .25ex}%
            {\noindent\normalfont\itshape\small}}

% Change the style of subparagraph headings %
\renewcommand\subparagraph{\@startsection{subparagraph}{5}{\z@}%
            {-2.5ex\@plus -1ex \@minus -.25ex}%
            {1.25ex \@plus .25ex}%
            {\noindent\normalfont\itshape\footnotesize}}
\makeatother

% use etoolbox to suppress page numbers on \part
\patchcmd{\part}{\thispagestyle{plain}}{\thispagestyle{empty}}
  {}{\errmessage{Cannot patch \string\part}}

% and to reduce margins on quotation
\patchcmd{\quotation}{\rightmargin}{\leftmargin 1.2em \rightmargin}{}{}
\AtBeginEnvironment{quotation}{\small}

% titlepage
\newcommand*{\titlepageTranslationTitle}[1]{{\begin{center}\begin{large}{#1}\end{large}\end{center}}}
\newcommand*{\titlepageCreatorName}[1]{{\begin{center}\begin{normalsize}{#1}\end{normalsize}\end{center}}}

% halftitlepage
\newcommand*{\halftitlepageTranslationTitle}[1]{\setstretch{2.5}{\begin{Huge}\uppercase{\so{#1}}\end{Huge}}}
\newcommand*{\halftitlepageTranslationSubtitle}[1]{\setstretch{1.2}{\begin{large}{#1}\end{large}}}
\newcommand*{\halftitlepageFleuron}[1]{{\begin{large}\Fleuronfont{{#1}}\end{large}}}
\newcommand*{\halftitlepageByline}[1]{{\begin{normalsize}\textit{{#1}}\end{normalsize}}}
\newcommand*{\halftitlepageCreatorName}[1]{{\begin{LARGE}{\textsc{#1}}\end{LARGE}}}
\newcommand*{\halftitlepageVolumeNumber}[1]{{\begin{normalsize}{\Allsmallcapsfont{\textsc{#1}}}\end{normalsize}}}
\newcommand*{\halftitlepageVolumeAcronym}[1]{{\begin{normalsize}{#1}\end{normalsize}}}
\newcommand*{\halftitlepageVolumeTranslationTitle}[1]{{\begin{Large}{\textsc{#1}}\end{Large}}}
\newcommand*{\halftitlepageVolumeRootTitle}[1]{{\begin{normalsize}{\Allsmallcapsfont{\textsc{\itshape #1}}}\end{normalsize}}}
\newcommand*{\halftitlepagePublisher}[1]{{\begin{large}{\Noligaturecaptionfont\textsc{#1}}\end{large}}}

% epigraph
\renewcommand{\epigraphflush}{center}
\renewcommand*{\epigraphwidth}{.85\textwidth}
\newcommand*{\epigraphTranslatedTitle}[1]{\vspace*{.5em}\footnotesize\textsc{#1}\\}%
\newcommand*{\epigraphRootTitle}[1]{\footnotesize\textit{#1}\\}%
\newcommand*{\epigraphReference}[1]{\footnotesize{#1}}%

% map
\newsavebox\IBox

% custom commands for html styling classes
\newcommand*{\scnamo}[1]{\begin{Center}\textit{#1}\end{Center}\bigskip}
\newcommand*{\scendsection}[1]{\begin{Center}\begin{small}\textit{#1}\end{small}\end{Center}\addvspace{1em}}
\newcommand*{\scendsutta}[1]{\begin{Center}\textit{#1}\end{Center}\addvspace{1em}}
\newcommand*{\scendbook}[1]{\bigskip\begin{Center}\uppercase{#1}\end{Center}\addvspace{1em}}
\newcommand*{\scendkanda}[1]{\begin{Center}\textbf{#1}\end{Center}\addvspace{1em}} % use for ending vinaya rule sections and also samyuttas %
\newcommand*{\scend}[1]{\begin{Center}\begin{small}\textit{#1}\end{small}\end{Center}\addvspace{1em}}
\newcommand*{\scendvagga}[1]{\begin{Center}\textbf{#1}\end{Center}\addvspace{1em}}
\newcommand*{\scrule}[1]{\textsb{#1}}
\newcommand*{\scadd}[1]{\textit{#1}}
\newcommand*{\scevam}[1]{\textsc{#1}}
\newcommand*{\scspeaker}[1]{\hspace{2em}\textit{#1}}
\newcommand*{\scbyline}[1]{\begin{flushright}\textit{#1}\end{flushright}\bigskip}
\newcommand*{\scexpansioninstructions}[1]{\begin{small}\textit{#1}\end{small}}
\newcommand*{\scuddanaintro}[1]{\medskip\noindent\begin{footnotesize}\textit{#1}\end{footnotesize}\smallskip}

\newenvironment{scuddana}{%
\setlength{\stanzaskip}{.5\baselineskip}%
  \vspace{-1em}\begin{verse}\begin{footnotesize}%
}{%
\end{footnotesize}\end{verse}
}%

% custom command for thematic break = hr
\newcommand*{\thematicbreak}{\begin{center}\rule[.5ex]{6em}{.4pt}\begin{normalsize}\quad\Fleuronfont{•}\quad\end{normalsize}\rule[.5ex]{6em}{.4pt}\end{center}}

% manage and style page header and footer. "fancy" has header and footer, "plain" has footer only

\pagestyle{fancy}
\fancyhf{}
\fancyfoot[RE,LO]{\thepage}
\fancyfoot[LE,RO]{\footnotesize\lastleftxmark}
\fancyhead[CE]{\setstretch{.85}\Noligaturefont\MakeLowercase{\textsc{\firstrightmark}}}
\fancyhead[CO]{\setstretch{.85}\Noligaturefont\MakeLowercase{\textsc{\firstleftmark}}}
\renewcommand{\headrulewidth}{0pt}
\fancypagestyle{plain}{ %
\fancyhf{} % remove everything
\fancyfoot[RE,LO]{\thepage}
\fancyfoot[LE,RO]{\footnotesize\lastleftxmark}
\renewcommand{\headrulewidth}{0pt}
\renewcommand{\footrulewidth}{0pt}}
\fancypagestyle{plainer}{ %
\fancyhf{} % remove everything
\fancyfoot[RE,LO]{\thepage}
\renewcommand{\headrulewidth}{0pt}
\renewcommand{\footrulewidth}{0pt}}

% style footnotes
\setlength{\skip\footins}{1em}

\makeatletter
\newcommand{\@makefntextcustom}[1]{%
    \parindent 0em%
    \thefootnote.\enskip #1%
}
\renewcommand{\@makefntext}[1]{\@makefntextcustom{#1}}
\makeatother

% hang quotes (requires microtype)
\microtypesetup{
  protrusion = true,
  expansion  = true,
  tracking   = true,
  factor     = 1000,
  patch      = all,
  final
}

% Custom protrusion rules to allow hanging punctuation
\SetProtrusion
{ encoding = *}
{
% char   right left
  {-} = {    , 500 },
  % Double Quotes
  \textquotedblleft
      = {1000,     },
  \textquotedblright
      = {    , 1000},
  \quotedblbase
      = {1000,     },
  % Single Quotes
  \textquoteleft
      = {1000,     },
  \textquoteright
      = {    , 1000},
  \quotesinglbase
      = {1000,     }
}

% make latex use actual font em for parindent, not Computer Modern Roman
\AtBeginDocument{\setlength{\parindent}{1em}}%
%

% Default values; a bit sloppier than normal
\tolerance 1414
\hbadness 1414
\emergencystretch 1.5em
\hfuzz 0.3pt
\clubpenalty = 10000
\widowpenalty = 10000
\displaywidowpenalty = 10000
\hfuzz \vfuzz
 \raggedbottom%

\title{Theravāda Collection on Monastic Law}
\author{Bhikkhu Brahmali}
\date{}%
% define a different fleuron for each edition
\newfontfamily\Fleuronfont[Ornament=9]{Arno Pro}

% Define heading styles per edition for chapter and section. Suttatitle can be either of these, depending on the volume. 

\let\oldfrontmatter\frontmatter
\renewcommand{\frontmatter}{%
\chapterfont{\setstretch{.85}\normalfont\centering}%
\sectionfont{\setstretch{.85}\normalfont\BalancedRagged}%
\oldfrontmatter}

\let\oldmainmatter\mainmatter
\renewcommand{\mainmatter}{%
\chapterfont{\setstretch{.85}\normalfont\centering}%
\sectionfont{\setstretch{.85}\normalfont\BalancedRagged}%
\oldmainmatter}

\let\oldbackmatter\backmatter
\renewcommand{\backmatter}{%
\chapterfont{\setstretch{.85}\normalfont\centering}%
\sectionfont{\setstretch{.85}\normalfont\BalancedRagged}%
\pagestyle{plainer}%
\oldbackmatter}

% for reasons, flat texts align too far in the margin in ToC, this fixes it. 
\renewcommand*{\tocacronym}[1]{\hspace*{0em}{#1}\quad}%
%
\begin{document}%
\normalsize%
\frontmatter%
\setlength{\parindent}{0cm}

\pagestyle{empty}

\maketitle

\blankpage%
\begin{center}

\vspace*{2.2em}

\halftitlepageTranslationTitle{Theravāda Collection on Monastic Law}

\vspace*{1em}

\halftitlepageTranslationSubtitle{A translation of the Pali Vinaya Piṭaka into English}

\vspace*{2em}

\halftitlepageFleuron{•}

\vspace*{2em}

\halftitlepageByline{translated and introduced by}

\vspace*{.5em}

\halftitlepageCreatorName{Bhikkhu Brahmali}

\vspace*{4em}

\halftitlepageVolumeNumber{Volume 4}

\smallskip

\halftitlepageVolumeAcronym{Kd 1–10}

\smallskip

\halftitlepageVolumeTranslationTitle{The Great Division}

\smallskip

\halftitlepageVolumeRootTitle{Mahāvagga}

\vspace*{\fill}

\sclogo{0}
 \halftitlepagePublisher{SuttaCentral}

\end{center}

\newpage
%
\setstretch{1.05}

\begin{footnotesize}

\textit{Theravāda Collection on Monastic Law} is a translation of the Theravāda Vinayapiṭaka by Bhikkhu Brahmali.

\medskip

Creative Commons Zero (CC0)

To the extent possible under law, Bhikkhu Brahmali has waived all copyright and related or neighboring rights to \textit{Theravāda Collection on Monastic Law}.

\medskip

This work is published from Australia.

\begin{center}
\textit{This translation is an expression of an ancient spiritual text that has been passed down by the Buddhist tradition for the benefit of all sentient beings. It is dedicated to the public domain via Creative Commons Zero (CC0). You are encouraged to copy, reproduce, adapt, alter, or otherwise make use of this translation. The translator respectfully requests that any use be in accordance with the values and principles of the Buddhist community.}
\end{center}

\medskip

\begin{description}
    \item[Web publication date] 2021
    \item[This edition] 2025-01-13 01:01:43
    \item[Publication type] hardcover
    \item[Edition] ed3
    \item[Number of volumes] 6
    \item[Publication ISBN] 978-1-76132-006-4
    \item[Volume ISBN] 978-1-76132-010-1
    \item[Publication URL] \href{https://suttacentral.net/editions/pli-tv-vi/en/brahmali}{https://suttacentral.net/editions/pli-tv-vi/en/brahmali}
    \item[Source URL] \href{https://github.com/suttacentral/bilara-data/tree/published/translation/en/brahmali/vinaya}{https://github.com/suttacentral/bilara-data/tree/published/translation/en/brahmali/vinaya}
    \item[Publication number] scpub8
\end{description}

\medskip

Map of Jambudīpa is by Jonas David Mitja Lang, and is released by him under Creative Commons Zero (CC0).

\medskip

Published by SuttaCentral

\medskip

\textit{SuttaCentral,\\
c/o Alwis \& Alwis Pty Ltd\\
Kaurna Country,\\
Suite 12,\\
198 Greenhill Road,\\
Eastwood,\\
SA 5063,\\
Australia}

\end{footnotesize}

\newpage

\setlength{\parindent}{1em}%%
\tableofcontents
\newpage
\pagestyle{fancy}
%
\chapter*{Introduction to the Khandhakas, “The Chapters”, part I, Kd 1–10}
\addcontentsline{toc}{chapter}{Introduction to the Khandhakas, “The Chapters”, part I, Kd 1–10}
\markboth{Introduction to the Khandhakas, “The Chapters”, part I, Kd 1–10}{Introduction to the Khandhakas, “The Chapters”, part I, Kd 1–10}

\scbyline{Bhikkhu Brahmali, 2024}

The present volume is the fourth of six, the total of which constitutes a complete translation of the Vinaya \textsanskrit{Piṭaka}, the Monastic Law. This volume consists of the first part of the Khandhakas, also known as the \textsanskrit{Mahāvagga}, “the Great Division”, comprising the first 10 of altogether 22 chapters. The remaining 12 chapters, which will make up volume 5, are collectively known as the Cullavagga, “the Small Division”. In the present introduction, I will survey the contents of volume 4 and make observations of points of particular interest. For a general introduction to the Monastic Law, see volume 1.

The word \textit{khandhaka} is derived from the word \textit{khandha}, “an aggregate” or “a collection”, which is a core doctrinal term familiar from the Suttas, as in the phrase “the five aggregates”. \textit{Khandhaka} is thus “a collection”. To refer to these collections as a group, I either use the Pali term Khandhakas or I render it into English as “the Chapters”. Other schools of Buddhism sometimes use the terms \textit{vastu} or \textit{\textsanskrit{pakiṇṇaka}}, “subject matter”/“account” or “miscellaneous matters”, instead of \textit{khandhaka}.\footnote{Especially the \textsanskrit{Mūlasarvāstivādins}. See Frauwallner, p. 3. }

The splitting of the Khandhakas into two divisions, the \textsanskrit{Mahāvagga} and the Cullavagga, is peculiar to the Pali tradition and encountered mostly in the sub-commentaries. It is not part of the common Buddhist heritage. As with the Sutta-\textsanskrit{vibhaṅga}, this split is probably an artifact of the manuscript tradition, which needed to divide the text into chunks suitable for binding.

While the Sutta-\textsanskrit{vibhaṅga}, contained in the first three volumes of this series, is divided into one part for monks and another for nuns, the Khandhakas are the same for both Sanghas, except for certain areas where the nuns have their own rules, especially as reflected in the \textsanskrit{Bhikkhunī}-kkhandhaka.\footnote{For instance, the \textit{\textsanskrit{bhikkhunīs}} have their own ordination ceremony. Then there is the fact that some of the rules in the Khandhakas are \textit{\textsanskrit{pācittiya}} rules for the \textit{\textsanskrit{bhikkhunīs}}. By and large, however, it seems that the Khandhakas are binding on both Sanghas. } We will take a closer look at the Bhikkhuni-kkhandhaka, Kd 20, in the introduction to volume 5.

Before we discuss some general features of the Khandhakas, here is a brief overview of their main contents. Whereas the Sutta-\textsanskrit{vibhaṅga} concerns the \textsanskrit{Pātimokkha} rules and their analysis, at the core of the Khandhakas are the regulations that govern Sangha meetings and decision making. And while the Sutta-\textsanskrit{vibhaṅga} is quite homogenous, with a regular and predictable structure, the Khandhakas are more diverse. Apart from the regulations that govern Sangha meetings, the Khandhakas include a large number of stories, many of them featuring some of the best-loved personalities of early Buddhism, and a great number of minor rules dealing with everything from building regulations to personal grooming. Another significant part of the Khandhakas is the extended Buddha biography and the related accounts of the first two Councils. I shall return to this latter point shortly.

The ten chapters of the \textsanskrit{Mahāvagga} are as follows:

\begin{enumerate}%
\item The Great Chapter, \textsanskrit{Mahā}-khandhaka (\href{https://suttacentral.net/pli-tv-kd1/en/brahmali}{Kd~1})%
\item The Chapter on the Observance Day, Uposatha-kkhandhaka (\href{https://suttacentral.net/pli-tv-kd2/en/brahmali}{Kd~2})%
\item The Chapter on Entering the Rainy-season Residence, \textsanskrit{Vassūpanāyika}-kkhandhaka (\href{https://suttacentral.net/pli-tv-kd3/en/brahmali}{Kd~3})%
\item The Chapter on the Invitation Ceremony, \textsanskrit{Pavāraṇā}-kkhandhaka (\href{https://suttacentral.net/pli-tv-kd4/en/brahmali}{Kd~4})%
\item The Chapter on Skins, Camma-kkhandhaka (\href{https://suttacentral.net/pli-tv-kd5/en/brahmali}{Kd~5})%
\item The Chapter on Medicines, Bhesajja-kkhandhaka (\href{https://suttacentral.net/pli-tv-kd6/en/brahmali}{Kd~6})%
\item The Chapter on the Robe-making Ceremony, Kathina-kkhandhaka (\href{https://suttacentral.net/pli-tv-kd7/en/brahmali}{Kd~7})%
\item The Chapter on Robes, \textsanskrit{Cīvara}-kkhandhaka (\href{https://suttacentral.net/pli-tv-kd8/en/brahmali}{Kd~8})%
\item The Chapter Connected with \textsanskrit{Campā}, Campeyya-kkhandhaka (\href{https://suttacentral.net/pli-tv-kd9/en/brahmali}{Kd~9}))%
\item The Chapter Connected with \textsanskrit{Kosambī}, Kosambaka-kkhandhaka (\href{https://suttacentral.net/pli-tv-kd10/en/brahmali}{Kd~10}).%
\end{enumerate}

\section*{The origin of the Khandhakas}

The Khandhakas did not exist as a separate part of the Vinaya \textsanskrit{Piṭaka} from the earliest period. I have argued in the introduction to volume 1 that initially there was only a \textsanskrit{Pātimokkha}, whereas the Vinaya \textsanskrit{Piṭaka} as a class of literature only arose later. This is evident from the fact that the \textsanskrit{Pātimokkha} is mentioned with relative frequency in the Suttas, while the Khandhakas are not mentioned at all.\footnote{The earliest reference to the Khandhakas, or an early version of them, is probably found at \href{https://suttacentral.net/pli-tv-kd22/en/brahmali\#2.8.45}{Kd~22:2.8.45}, where it is called the Uposatha-\textsanskrit{saṁyutta}. } In the four main \textsanskrit{Nikāyas}, even the word \textit{vinaya} mostly refers to training in a general sense, not to a separate collection of scriptures, a \textit{\textsanskrit{piṭaka}}.

This suggestion is reinforced by the story of the first \textit{\textsanskrit{saṅgīti}} or Council. According to the account at Kd 21, \textsanskrit{Mahākassapa}, who presides at the meeting, first asks \textsanskrit{Upāli} about the Vinaya and then Ānanda about the Suttas. After asking \textsanskrit{Upāli} about the four \textit{\textsanskrit{pārājika}} offences for monks, the rest of the recitation is abbreviated with the following phrase: “In this way he asked about the analyses of both Monastic Codes.”\footnote{\textit{Eteneva \textsanskrit{upāyena} \textsanskrit{ubhatovibhaṅge} pucchi}. } There is no mention of the Khandhakas.

Despite this absence of the Khandhakas as a collection, some of its content must have existed from the earliest period of Buddhism. The \textsanskrit{Pātimokkha} rules themselves imply further rules and regulations. These include confession formulas for the \textit{\textsanskrit{pācittiya}} offenses, details about the process of emerging from \textit{\textsanskrit{saṅghādisesa}} offenses, false accusations (Bu Ss 8 and 9), dealing with monastics holding wrong views (Bu Pc 68–70), and resolving disagreements about the Dhamma. Beyond this are such important and fundamental rituals as the ordination ceremony and the observance day, the \textit{uposatha}, which are mentioned throughout the Sutta \textsanskrit{Piṭaka}. Much of this must have come into existence soon after the Buddha founded the monastic community. But if this is so, where were these laid down? Among which scriptures were these procedures that form the backbone of a functioning Sangha kept? I believe the answer, for the most part, is the \textsanskrit{Pātimokkha} itself.

We have seen in the introduction to volume 2 that the \textsanskrit{Pātimokkha}, both of the \textit{bhikkhus} and the \textit{\textsanskrit{bhikkhunīs}}, includes a section of seven rules called the \textit{\textsanskrit{adhikaraṇasamathadhammas}}, “the principles for settling legal issues”. I pointed out that these seven principles were used by the community as a whole in resolving problems and dealing with business, rather than being rules to be followed by individual members of the Sangha. What is especially striking about them is that they are presented without any analysis or explanation. This is in contrast to all the other rules included in the \textsanskrit{Pātimokkha}. In fact, the way they are now listed in the \textsanskrit{Pātimokkha}, it is impossible to know what they refer to or how they should be applied. An analysis, a \textit{\textsanskrit{vibhaṅga}}, must have existed at some point. My proposal is that this missing analysis was removed from the \textsanskrit{Pātimokkha}, either to be included in the Khandhakas or to form the kernel around which the Khandhakas grew.\footnote{I will consider this proposal in greater detail in the introduction to volume 5 in connection with my discussion of Kd 14. }

If the core of the Khandhakas as we have them now originated as a \textit{\textsanskrit{vibhaṅga}} to the seven principles for settling legal issues, what might this original analysis have looked like? Fortunately, we have a \textit{sutta}, MN 104, that gives a brief description of the seven principles. It seems reasonable to assume that an early \textit{\textsanskrit{vibhaṅga}} would have looked much like this, with either the Sutta being a precursor to the \textit{\textsanskrit{vibhaṅga}} or the two being roughly contemporaneous. If so, what does this tell us?

The description at MN 104 focuses largely on how to deal with offenses, with five of the seven being concerned with this.\footnote{\textit{Sativinaya}, “resolution through recollection”, concerns pure monastics being accused of an offense; \textit{\textsanskrit{amūḷhavinaya}}, “resolution because of past insanity”, is about offenses committed while insane; \textit{\textsanskrit{paṭiññātakaraṇa}}, “acting according to what has been admitted”, sets out the formulas of confession; \textit{\textsanskrit{tassapāpiyasikā}}, “further penalty”, discusses the further penalty for someone trying in various ways to wriggle out of an offense; and \textit{\textsanskrit{tiṇavatthāraka}}, “covering over as if with grass”, concerns the cumulative confession of offenses after a large number have been committed during the course of an argument. } They are thus closely tied to the \textsanskrit{Pātimokkha} rules, and it makes sense that they would be found as an addition to them. The remaining two, however, concern disputes about the Dhamma and how these should be resolved harmoniously.\footnote{\textit{\textsanskrit{Sammukhāvinaya}}, “resolution face-to-face”, requires all parties concerned with an issue to be present when it is decided; and \textit{\textsanskrit{yebhuyyasikā}}, “majority decision”, is a democratic way of deciding, but only when the majority is line with the Dhamma. In fact, it is interesting that in MN 104 these two principles are grouped together at the top of the list, whereas in the \textsanskrit{Pātimokkha} they are separated. It is possible that the order in MN 104 is earlier than what is found in the Vinaya. } It is not immediately obvious why these would be part of the \textsanskrit{Pātimokkha}.

It is these latter two rules above all, in my opinion, that show that the \textsanskrit{Pātimokkha} is more than a set of rules binding on individuals. We are now moving into the sphere of regulating the Sangha as a functioning community. Dealing with disagreements about the Dhamma is an obvious example of this.

As I have mentioned above, however, there is more, including the important functions of the Sangha such as ordination and the observance-day ceremony. Where were the blueprints for these and other rituals kept? One possibility is that they were part of the first of the seven principles, “resolution face-to-face”. This principle is essentially about the Sangha coming together and doing whatever business is on the agenda. In other words, the most basic meaning of “face-to-face” is being in the presence of the Sangha.\footnote{Cf. \href{https://suttacentral.net/pli-tv-kd14/en/brahmali\#14.16.13}{Kd~14:14.16.13}. } And so, the process for carrying out these important functions may have been set out under this principle.\footnote{Alternatively, it may be that these processes were kept as separate “documents”, which then became the kernel around which the Khandhakas were formed. There is, in fact, a tradition in Theravada Buddhism of keeping formal procedures of the Sangha, so-called \textit{\textsanskrit{kammavācas}}, as separate manuscripts. (See, for instance, the Journal of the Pali Text Society, 1993, pp. 1–41.) It is conceivable that this tradition goes back to the earliest period. } If so, then the \textsanskrit{Pātimokkha} is beginning to look like much more than a simple set of rules.

Given such a development, it is easy to see how the first of the seven principles would have become bulky and unwieldy from early on.\footnote{It is possible that the connection between “resolution face-to-face” and “majority decision” was lost as a consequence of this expansion. It might then have been natural to move “majority decision” down the list of principles as an aid to memorization, in this case, keeping all the principles ending in \textit{vinaya} together. This may have been the origin for the difference in sequence found in the Vinaya compared to that of MN 104. } An ordination ceremony would have been required soon after the Sangha was formed, as can be seen from the exposition at Kd 1. Over time this ceremony developed into a long procedure. The same is true of other legal procedures, \textit{\textsanskrit{saṅghakammas}}, that are adumbrated in the \textsanskrit{Pātimokkha} rules, such as the procedures related to the emergence from \textit{\textsanskrit{saṅghādisesa}} offenses. The observance-day ceremony required rules for its implementation, as did the invitation ceremony, the \textit{\textsanskrit{pavāraṇā}}, and much more. As all this was added to the \textsanskrit{Pātimokkha}, specifically to the first of the seven principles, it would not be long before this section became overloaded and disjointed because of its varied content. It would then be natural to create a separate section for all the new and evolving Vinaya material. It is in this way that I envisage the creation of the Khandhakas.

What about the structure of the Khandhakas? It was no doubt natural to divide the diverse content into fairly homogenous topics. Moreover, as new material was added, it would either be an expansion of an existing topic, in which case it would go into the relevant section, usually at the end of it, or it would be added as a new chapter at the end of the Khandhakas. In this way, the material would have a roughly chronological sequence.

If we consider the Khandhakas as we have them now, this is essentially what we find. They start with the \textit{khandhaka} on ordination, for ordination would have been required from the very beginning. Without ordination, there is no Sangha to which the rules in the Khandhakas can apply. Once there was a Sangha, there would be a need for occasional meetings and also a connection with the lay supporters. Such meetings were to be held on the observance day, the \textit{uposatha}, an ancient Indian institution.\footnote{The pre-Buddhist \textit{uposatha} is described in Śatapatha-\textsanskrit{brāhmaṇa} 1.1.1, see https://www.wisdomlib.org/hinduism/book/satapatha-brahmana-english/d/doc63113.html. } These meetings were the seed from which the second \textit{khandhaka} developed. Then there was the need to remain in one place during the rainy season, a practice already undertaken by ascetics of other religions. There would have been pressure from society on the evolving Buddhist Sangha to follow such precedents. The rainy-season residence is the topic of the third \textit{khandhaka}. This is followed by the \textit{khandhaka} on the invitation ceremony, the \textit{\textsanskrit{pavāraṇā}}, which was to be performed at the end of the rainy season. The invitation ceremony fulfills the important function of opening oneself up for correction by one’s fellow monastics.

We could carry on this exercise for all the 22 \textit{khandhakas}. For some of them, it is not obvious that they have a particular temporal position, yet the idea of a chronological sequence does seem to form the overarching principle on which they were laid down. The last of the 22 concerns the second Council, an event that is also historically the last.

The content of the individual \textit{khandhakas} suggests a similar temporal development. They often start with the laying down of a fundamental principle, followed by the most basic rules and regulations that relate to it. Then we have one or more stories, sometimes none, which usually serve as the background for further rules. Each \textit{khandhaka} often ends with a rather theoretical series of permutations, often of limited practical significance. We will consider each \textit{khandhaka} in more detail, including their temporal position in the series, as we discuss them individually below.

To complete the picture of the origin of the Khandhakas, there is one more issue that needs to be considered. As you start to read the Khandhakas, it is striking that they do not start with the laying down of rules, but with an extended biography of the Buddha’s post-awakening experiences. At the conclusion of the Khandhakas, in Kd 21 and 22, we find material of a similar sort, namely, aspects of the history of the Sangha after the Buddha had passed away. How can we explain the inclusion of this material?

We know that Buddha biographies started to appear in earnest after the Buddha passed away. All the schools of Buddhism whose scriptures we still possess had them.\footnote{See Frauwallner, p. 50. } In the Pali tradition, the most developed and well-known of these stories is found in the introduction to the \textsanskrit{Jātaka} stories, the so-called \textsanskrit{Jātaka}-\textsanskrit{nidāna}. As these stories started to form, the question would have arisen where to keep them. Being narratives created by an unknown person or community, they do not fit with the Suttas, which are generally the words of the Buddha. In fact, the biographical narrative in Kd 1 is not too different from the short narratives that introduce each Sutta, that is, narratives that give basic information about where the Buddha was staying and who he was speaking to. It might have been natural, then, to use the new biography in the same way. This is how I suppose it became the introduction to the entire Khandhaka literature.

There is another important reason this framing makes good sense. The early biography of the Buddha would have coincided with the early development of the Sangha. As the Buddha started teaching and getting a monastic following, rules and regulations for the Sangha would gradually be required. Throughout his life, the Buddha laid down such rules and regulations, which makes the Buddha biography a natural container for the entire Vinaya \textsanskrit{Piṭaka}.

At an early point in the creation of the Khandhakas, it is plausible that the Buddha’s biography would have formed a continuous narrative into which all the rules and regulations of the Sangha found their natural place. As the material expanded, however, this natural structure broke down. Still, even now we see glimpses of the Buddha’s life throughout the Khandhakas. Eventually the Buddha dies. The last two \textit{khandhakas} are therefore concerned with the preservation of the Vinaya in the period after his demise.

Yet there is a curious gap in this biography, especially as we have it in the Pali tradition. The important events surrounding the Buddha’s death, and the time leading up to it, are not mentioned. This is in spite of the fact that these events contain important guidelines for how the Vinaya should be regarded and practiced after the Buddha is gone.

It has been pointed out by the likes of Frauwallner that the \textsanskrit{Mahāparinibbāna} Sutta (DN 16) fits quite naturally with the biographical and historical material of the Khandhakas. A curious detail about Kd 21, which begins immediately after the death of the Buddha, is that it starts quite abruptly: “Then Venerable \textsanskrit{Mahākassapa} addressed the monks …”, followed by \textsanskrit{Mahākassapa} telling of the events that are narrated toward the end of DN 16. Nowhere else in the Suttas or the Vinaya does a separate section or \textit{sutta} begin with the words \textit{atha kho}, “then”. The Pali \textit{atha kho} functions in a similar way to “then” in English, in that it connects the narrative to some previous event. It is unnatural to start a new section in this way without any relation to a preceding narrative. The sense one gets is that this originally was part of DN 16, or vice versa, forming an extended narrative. In fact, this is exactly what we find in the Vinayas of the \textsanskrit{Mūlasarvāstivādins} and the \textsanskrit{Mahāsaṅghikas}.\footnote{Frauwallner, p. 44. He then concludes as follows: “We can now sum up our results thus: The story of the death of the Buddha and the account of the two earliest councils formed originally one single narrative. This narrative, according to the evidence of the great majority of the sources, was a fixed component of the Vinaya. It belonged to the Vinaya already in its earliest form recognizable to us, and had its place at the end of the Skandhaka.” (p. 46) }

DN 16 is, in fact, quite an anomalous Sutta. The Suttas are almost universally presented as the word of the Buddha with a short narrative framework. DN 16, by contrast, is essentially the opposite, that is, a narrative incorporating the word of the Buddha at various points, sometimes very briefly. Also, it is much longer than any other \textit{sutta} in the four main \textsanskrit{Nikāyas}, being almost twice the length of the next longest \textit{sutta}.\footnote{The Pali word count of DN 16 is in excess of 15,200, whereas the \textsanskrit{Mahāpadāna} Sutta, DN 14, which is the second longest, has about 8,700 words. There are several other \textit{suttas} with over 7,000 words, including DN 1, DN 2, DN 3, and DN 33, which means that DN 16 stands out as anomalous. } Finally, DN 16 includes material that was composed after the Buddha’s passing, making it approximately contemporaneous with Kd 21. These considerations, combined with its close affiliation with Kd 21, suggests it was originally not part of the Suttas, but existed separately as an evolving biography of the Buddha. In some schools of early Buddhism this biography was broken up, with parts of it becoming DN 16 or its equivalent, whereas the remainder became the framework for the Khandhakas.\footnote{For the \textsanskrit{Sarvāstivādins}, the part of the Buddha biography that is equivalent to the story found in Kd 1, became a separate \textit{sutta} in their \textsanskrit{Dīrghāgama}, known as the \textsanskrit{Catuṣpariṣatsūtra}. See Frauwallner pp. 48–49. } For other schools the story was kept entirely within the Khandhakas.

In sum, it seems the Khandhakas were created in the period soon after the Buddha’s passing away, incorporating various elements from an evolving tradition. These elements included the \textsanskrit{Vibhaṅga} material from the seven principles for settling legal issues, the various Sangha procedures and ceremonies that had been established either as separate documents or as part the seven principles, and finally the evolving Buddha biography. This core would then have evolved as new material was added, culminating with the story of the second \textit{\textsanskrit{saṅgīti}} one hundred years after the Buddha’s death.

This brings us to the interesting question of when the Khandhakas, and also the Sutta-\textsanskrit{vibhaṅga}, were closed to new material. One of the issues we have not explained satisfactorily is the considerable sectarian differences between the Vinayas of the various schools. The emergence of separate schools only started around the time of Ashoka, maybe 150–200 years after the Buddha. It would seem, then, that we need to assume that changes were made to the Vinaya as long as 200–300 years after the Buddha.

Perhaps, but not necessarily so. The Sangha was spreading out over significant parts of India already during the lifetime of the Buddha. By the time of the second \textit{\textsanskrit{saṅgīti}}, the Sangha had spread over a large geographical area. We know this from the geographical information given in Kd 22 and elsewhere.\footnote{In the introduction to the Sutta-\textsanskrit{vibhaṅga} we find a number of place names that are geographically west of where the Buddha had stayed, in particular \textsanskrit{Payāgapatiṭṭhāna}, Soreyya, \textsanskrit{Saṅkassa}, and \textsanskrit{Kaṇṇakujja}. Ven. Dhammika of Australia tells me (private communication) that he believes he has located most of these places, see \href{https://suttacentral.net/pli-tv-bu-vb-pj1/en/brahmali\#4.18}{Bu~Pj~1:4.18}. The introduction to the \textsanskrit{Pārāyanavagga} at \href{https://suttacentral.net/snp5.1/en/sujato\#36.1}{Snp 5.1:36.1}, likewise, mentions names to the south, which may mean that Buddhism was spreading to this area. } It would have been a difficult or even impossible task to efficiently disseminate new rules and regulations over such a large area. At the same time, it is likely that the Sangha was already splitting into groups. We see in Kd 22 that some monks were apparently following the Vinaya regulations closely, whereas others less so. We can surmise that differences in the interpretation of the Suttas would have created similar divisions. There were no doubt groups forming around charismatic teachers, an early example of which might be the monk \textsanskrit{Purāṇa} of Kd 21 who refused to accept the authority of the Suttas and Vinaya as recited at the first \textit{\textsanskrit{saṅgīti}}. Finally, we have the brute fact of physical distances which would have complicated the spreading of new material further.

Given this state of affairs, it seems likely to me that proto-schools started to form long before Ashoka and probably soon after the Buddha’s passing. With the arising of different group identities, it would no longer be natural to uncritically receive updates to the Vinaya, or indeed to the Suttas, from a group with another identity. If a rule was regarded as coming from the Buddha, everyone would presumably embrace it, but not so if its origin lay elsewhere. The literature would have started to diverge. When independent schools arose properly in the post-Ashokan period, some would have inherited one version of this Vinaya, whereas others would have inherited other versions. It is in this way, I propose, that the differences we now observe between the different schools started to take shape soon after the Buddha had passed away. It is conceivable, yet by no means certain, that the various Vinayas as we have them today, including the Khandhakas, were in large part fixed by the time of the second \textit{\textsanskrit{saṅgīti}}.

\section*{The mnemonic verses of the Khandhakas}

We have seen that the Khandhakas probably developed over a long period of time, with the core of it being laid down by the Buddha and the final version coming into existence at the earliest at the second \textit{\textsanskrit{saṅgīti}}. The mnemonic verses at the end of each chapter point to a similar conclusion. Let us take a closer look at their content.

In the Sutta-\textsanskrit{vibhaṅga} and the four main \textsanskrit{Nikāyas}, the mnemonic verses at the end of chapters, called \textit{\textsanskrit{uddānas}}, serve merely as aids to memorization, giving a series of key words that relate to the content of the preceding material. With the Khandhakas, however, the \textit{\textsanskrit{uddānas}} take on a new function. Apart from being aids to memory, they here incorporate verses that extol the Buddha and the Vinaya, and even speak of the process of composing the Khandhakas.

The \textit{\textsanskrit{uddāna}} to Kd 1 starts with a series of seven verses in praise of the Vinaya. Especially notable is the inclusion here of the \textsanskrit{Parivāra}, a text that is clearly sectarian and peculiar to the Pali tradition. The Abhidhamma, which is also sectarian, is mentioned too. Then there is the claim that Buddhism, the \textit{\textsanskrit{sāsana}}, remains so long as the Vinaya persists, even if the Suttas and the Abhidhamma are forgotten. These opening verses set the scene for the expounding of the mnemonic verses, whose purpose it is to preserve the Vinaya. It is also noteworthy that the text speaks of the Chapters \textit{and} the Monastic Law, apparently viewing them as separate entities. It seems possible that the Khandhakas for a long time were regarded as a separate class of literature before eventually being incorporated into the Vinaya \textsanskrit{Piṭaka}.

The verses referred to above must have been added a long time after the second \textit{\textsanskrit{saṅgīti}}, and probably after the Canon had arrived in Sri Lanka. The idea that the Vinaya is what matters for the persistence of Buddhism echoes a similar saying in the Vinaya commentary, the \textsanskrit{Samantapāsādikā}, that “the Vinaya is called the life of the Buddha’s dispensation; while the Vinaya persists, so does Buddhism.”\footnote{Sp 1.0: \textit{vinayo \textsanskrit{nāma} \textsanskrit{buddhasāsanassa} \textsanskrit{āyu}, vinaye \textsanskrit{ṭhite} \textsanskrit{sāsanaṁ} \textsanskrit{ṭhitaṁ} hot}i. } This is pretty much the opposite of what we would expect. At MN 104 we see the Buddha not being concerned about a dispute about the Vinaya, whereas he considers a dispute about the Dhamma as potentially very destructive.\footnote{\href{https://suttacentral.net/mn104/en/sujato\#5.8}{MN~104:5.8}: “Ānanda, a dispute about livelihood or the monastic code is a minor matter. But should a dispute arise in the Sangha concerning the path or the practice, that would be for the detriment, suffering, and harm of the people, for the detriment and suffering of gods and humans.” } The sentiment expressed in the \textit{\textsanskrit{uddāna}} aligns better with the commentaries than the Suttas, again suggesting a late composition.

We find similar issues in the summary verses of other \textit{khandhakas}. At Kd 15 it is said that “A well-trained expert in the Monastic Law … is a learned one worthy of homage.” At Kd 18, which is concerned with etiquette, we find the following: “If you do not fulfill the proper conduct … you are not released from suffering.” Whereas morality, \textit{\textsanskrit{sīla}}, is normally said to be the foundation of the path, this is here replaced by etiquette. Both these examples show a similar bias toward the Vinaya as what we see in the commentaries, as referenced above.

The most obvious case of the \textit{\textsanskrit{uddānas}} being late, however, is found in Kd 13. Here we find the following verse:

\begin{verse}%
“The teachers of analytical statements,\\

Who are the inspiration of Sri Lanka,\\

The residents of the \textsanskrit{Mahāvihāra} Monastery—\\

These were their words for the longevity of the true Teachings.”\footnote{At \href{https://suttacentral.net/pli-tv-kd13/en/brahmali\#36.4.65}{Kd~13:36.4.65}. }

%
\end{verse}

Here there is no doubt that we are deeply into the sectarian period. Not only were the summary verses composed in Sri Lanka, but so, apparently, was much of Kd 13.\footnote{I say “much of” because, according to Frauwallner, pp. 109–110, Kd 13 does have parallels in the other schools. }

Another point worthy of consideration is that the \textit{\textsanskrit{uddānas}} themselves seem to suggest a process of gradual accretion, with no definitive cut-off point. The summary verses at Kd 1 say that “It’s hard to complete without remainder—you should know it from the method.” And at Kd 3 we find that “Because of the gaps in the summary of topics, one should attend carefully to the way of the passages of the Canonical text.” It is not quite clear why this is stated, but a reasonable suggestion is that material was accumulating even as the verses were composed. It follows that the verses themselves soon became outdated and inadequate.

As it happens, there are plenty of such “gaps” where the summary verses do not capture certain passages in the main text. Sometimes the summary verses are very detailed and capture every minor rule or regulation, as at the beginning of Kd 15 where every rule is listed in the \textit{\textsanskrit{uddāna}}. At other times, as we shall see, whole passages are omitted.

I have not done a systematic survey of these omissions, but I have noted a few obvious examples:

\begin{description}%
\item[Kd 1] \begin{itemize}%
\item Section 3 on the story of Mucalinda is not mentioned in the \textit{\textsanskrit{uddāna}}, whereas the other stories are.%
\item The long sections on the duties to preceptors, students, teachers, and pupils are missing from the \textit{\textsanskrit{uddāna}} (\href{https://suttacentral.net/pli-tv-kd1/en/brahmali\#25.8.1}{Kd~1:25.8.1}–26.11.12 and \href{https://suttacentral.net/pli-tv-kd1/en/brahmali\#32.3.1}{Kd~1:32.3.1}–33.1.111).%
\item The several \textsanskrit{Aṅguttara}-style lists are left out.\footnote{\href{https://suttacentral.net/pli-tv-kd1/en/brahmali\#27.6.1}{Kd~1:27.6.1}–27.8.5 and \href{https://suttacentral.net/pli-tv-kd1/en/brahmali\#34.1.33}{Kd~1:34.1.33}–34.1.49, which concern the kind of student/pupil who deserves to be dismissed. }%
\end{itemize}

%
\item[Kd 2] \begin{itemize}%
\item Section 32, 36, and 37 are not in the \textit{\textsanskrit{uddāna}}.%
\item Except for the last three cases, the rest of section 38 is missing in the \textit{\textsanskrit{uddāna}}.%
\end{itemize}

%
\item[Kd 3] \begin{itemize}%
\item The \textit{\textsanskrit{uddāna}} says eight on schism, but the main text has 10.%
\item The two subsections on “observance-day within monastery” are missing in the \textit{\textsanskrit{uddāna}} (\href{https://suttacentral.net/pli-tv-kd3/en/brahmali\#14.8.1}{Kd~3:14.8.1}–14.10.6 and \href{https://suttacentral.net/pli-tv-kd3/en/brahmali\#14.11.31.1}{Kd~3:14.11.31.1}–14.11.49).%
\end{itemize}

%
\item[Kd 4] \begin{itemize}%
\item Although section 9 consists of 15 elements, it is only mentioned with three words in the \textit{\textsanskrit{uddāna}}, “greater, and equal, smaller”, which contrasts with the parallel section in Kd 2 where 12 out of 15 elements are mentioned.%
\item Sections 10–13 are not in the \textit{\textsanskrit{uddāna}}.%
\item In the \textit{\textsanskrit{uddāna}}, \textit{\textsanskrit{chandadāne}}, “about giving consent”, occurs after section 20, whereas in the main text it occurs much earlier at the end of section 3 where it fits naturally.\footnote{It may be that all the sections in between, 4–20, were copied over from Kd 2 where they occur in same way, with the only difference being that the word \textit{uposatha} had been replaced by \textit{\textsanskrit{pavāraṇā}}, thus interrupting the natural order of the \textit{\textsanskrit{uddāna}} in Kd 4. }%
\end{itemize}

%
\item[Kd 8] \begin{itemize}%
\item Neither the appointment of “the receiver of robes” nor “the keeper of robe-cloth” is found in the \textit{\textsanskrit{uddāna}}.%
\end{itemize}

%
\item[Kd 9] \begin{itemize}%
\item In regard to section 5, the \textit{\textsanskrit{uddāna}} has 18 kinds of people, whereas the text has 24. Of these, only 15 are in common.\footnote{It may be that the main text has copied the list of people from the previous section (section 4), whereas the \textit{\textsanskrit{uddāna}} may have preserved an older version. }%
\end{itemize}

%
\item[Kd 10] \begin{itemize}%
\item The entire story of \textsanskrit{Dīghāvu} is missing in the \textit{\textsanskrit{uddāna}}.%
\end{itemize}

%
\item[Kd 13] \begin{itemize}%
\item Sections 5 and 6 are not mentioned in the \textit{\textsanskrit{uddāna}}.%
\end{itemize}

%
\end{description}

It seems likely that these passages are missing because they were added after the \textit{\textsanskrit{uddānas}} were considered complete. It may have been difficult, for reasons of style and meter, to alter the \textit{\textsanskrit{uddānas}} with every addition of a new passage to the main text. It may also be that the use of \textit{\textsanskrit{uddānas}} slowly went out of fashion as the texts were written down. Regardless, this supports our contention that the Khandhakas were added to for a long time into the sectarian period, even after they had arrived in Sri Lanka.

A final point from the \textit{\textsanskrit{uddānas}} is the interesting fact that there is none for Kd 14. It is tempting to conclude that this \textit{khandhaka} is therefore particularly late. It happens to be the case, however, that Kd 14 has parallels in all other schools for which we have an extant Vinaya. This suggests that Kd 14 is not particularly late, especially since Kd 13, which we have shown as being late, does have summary verses. The absence of verses in Kd 14 requires a different explanation, a matter we shall return to when we discuss this chapter in the introduction to volume 5.

In sum, the mnemonic verses suggest that the Khandhakas were finalized in the sectarian period proper. This does not mean that most of their content comes from this period. Rather, I think there are good grounds for believing, based on the overall correspondence between the early schools,\footnote{Frauwallner, pp. 68–129. } that the main content of the Khandhakas was fixed already at the second \textit{\textsanskrit{saṅgīti}}.

Now let us look at the content of each of the first ten \textit{khandhakas} in turn.

\section*{The Great Chapter, \textsanskrit{Mahā}-khandhaka, Kd 1}

The \textsanskrit{Mahā}-khandhaka, Kd 1, is the longest chapter of the Khandhakas. It is chiefly concerned with ordination. In other schools of Buddhism, it is called “the Account of Going Forth”, \textsanskrit{Pravrajyā}-vastu, or its equivalent in Chinese and Tibetan.\footnote{Frauwallner, p. 70. } This chapter presumably comes first in the collection because ordination is the most basic of all Buddhist ceremonies in the sense that the Sangha could not exist without it.

Kd 1 begins with the biography of the Buddha, starting immediately after his awakening. After reflecting on his discovery, Kd 1 shows the Buddha meeting various beings, among them the two merchants Tapussa and Bhallika who become his first lay followers. The text turns to the famous request by Brahma Sahampati to teach the Dhamma and the Buddha’s response that “the doors to the freedom from death are open”. We then have the same narrative sequence as found in MN 26, “the Noble Search”, followed by the full version of the Dhammacakkappavattana Sutta,\footnote{Also found at \href{https://suttacentral.net/sn56.11/en/sujato}{SN~56.11}. } at the end of which \textsanskrit{Koṇḍañña} becomes a stream-enterer. Then comes the ordination of the group of five monks through the earliest ordination procedure, the so-called “come, monk” formula,\footnote{This formula is also occasionally used for nuns. } followed by the \textsanskrit{Anattalakkhaṇa} Sutta,\footnote{Also found at \href{https://suttacentral.net/sn22.59/en/sujato}{SN~22.59}. } at the end of which all five achieve perfection, \textit{arahantship}. We see that the narrative in Kd 1 is more complete and continuous than anything we find in the Suttas. We are dealing with a new sort of Buddha biography, which fills in gaps and adds details not found anywhere else.

The narrative continues with the story of the young man Yasa, whose going forth is partly modelled on that of the Buddha-to-be, but adding a number of supernormal events. Such additions deviate from the down-to-earth accounts found in the Suttas,\footnote{There are occasional descriptions in the Suttas of the Buddha performing supernormal feats, but these tend to be later additions. For instance, at \href{https://suttacentral.net/dn24/en/sujato\#2.13.1}{DN~24:2.13.1} we find an example of the Buddha supposedly levitating, of which Analayo 2016, p. 12, concludes: “In sum, the departure by levitation reported in the \textit{\textsanskrit{Pāṭika}-sutta} and its \textit{\textsanskrit{Dīrgha}-\textsanskrit{āgama}} parallel seems to be a later addition to the discourse.” In relation to fire “miracles” Analayo 2015, p. 33, has this to say: “The selected examples of fire miracles performed by the Buddha surveyed above seem to be for the most part identifiable as later developments, probably the result of literal interpretations of metaphorical usages of the fire motif attested in text and art.” } and are akin to later Buddha biographies,\footnote{As mentioned above, for the Pali tradition this means especially the \textsanskrit{Jātaka}-\textsanskrit{nidāna}, which forms the beginning of the commentary on the \textsanskrit{Jātaka} verses. } which take this tendency even further. Examples from the current narrative include gates being opened by spirits and the Buddha making Yasa invisible to his own father. There is a sense that history is turning into myth.

The Sangha gradually grows until we reach a well-known passage where the Buddha tells his monks to go wandering to spread his teaching. This has sometimes been taken to mean that Buddhism is a missionary religion. Of course, the Buddha knew he had an important message for the world, a message that would benefit humanity. Yet proselytizing does not have the central role in the Dhamma that it has in other religions, such a Christianity. One reason for this is that Buddhism is not about conversion as such, but about sustained practice, which means that understanding the teaching and following up with study and reflection are fundamental to make it work. Buddhism is more about making the teaching available to anyone interested than it is about actively seeking converts. This is reflected in the way the Buddha teaches. People generally approach him to hear what he has to say, not the other way around.

A natural consequence of the monks spreading out over a larger geographical area was a dispersed demand for ordination. It became impractical for the Buddha to ordain all aspirants. The Buddha then lays down a new ordination formula, by way of taking the three refuges, and allows the monks to perform ordinations.

Although the threefold formula of going for refuge at some point becomes the standard way of declaring yourself a lay follower of the Buddha, it is found only rarely in the \textsanskrit{Tipiṭaka}, and is entirely missing from the four main \textsanskrit{Nikāyas}.\footnote{The closest to this formula is found in DN 5, where the going for refuge to the Buddha, Dhamma, and Sangha is stated once. } It is possible, then, that since this formula was initially an ordination ceremony for monks, it was considered unsuitable for lay people. Only when the ordination ceremony evolved further, did the triple refuge become freed up for use by lay people. As it happens, we see that it becomes much more common in later Pali literature, especially the commentaries. In the Suttas and Vinaya the lay people use a related but simpler formula: “I go for refuge to the \textit{\textsanskrit{bhagavā}} (or “Sir Gotama”), and the Teaching, and the community of monks.”\footnote{\textit{\textsanskrit{Bhagavantaṁ} (}or \textit{\textsanskrit{bhavantaṁ} \textsanskrit{gotamaṁ}) \textsanskrit{saraṇaṁ} \textsanskrit{gacchāmi} \textsanskrit{dhammañca} \textsanskrit{bhikkhusaṅghañca}}. }

Next, we find several conversion accounts, the longest of which tells the story of how the Buddha persuaded one thousand fire worshippers to become his followers. Again, we see the tendency of mythologizing the life of the Buddha. These stories are full of wonders, psychic powers, and improbable events, setting them apart from the more grounded autobiographical material of the four main \textsanskrit{Nikāyas}.\footnote{Such as MN 4, MN 12, MN 14, MN 19, MN 26, MN 36, MN 85, MN 128, and AN 3.39. An exception to this general tendency is MN 49. DN 16 is another exception. Yet, as I have argued, this is not a \textit{sutta} in the ordinary sense and it better fits with the Khandhaka material. Bhante Sujato comments as follows on the wonders found in the Vinaya: “On one level, it’s obviously a co-opting of Brahmanical prestige. But at the same time, it doesn’t just \emph{dismiss} the miracles, it tells us that the rules to follow should be taken seriously because they were laid down by this person. It is a mode of establishing authority and meaning.” (Private communication.) } The story ends with the Buddha giving the well-known Fire Discourse, at the end of which all the one thousand ascetics reach full awakening.\footnote{Also found at \href{https://suttacentral.net/sn35.28/en/sujato}{SN~35.28}. }

Another of these conversion accounts concerns King \textsanskrit{Bimbisāra} of Magadha, the kingdom that was the precursor to King Ashoka’s empire a couple of centuries later. According to the \textsanskrit{Pabbajjā} Sutta of the Sutta \textsanskrit{Nipāta} at \href{https://suttacentral.net/snp3.1/en/sujato}{Snp 3.1}, \textsanskrit{Bimbisāra} had met the Buddha before his awakening, which may explain the king’s eagerness to see him again. The Buddha gives the king a teaching, upon which he becomes a stream-enterer together with 110,000 brahmin householders. The mythologizing tendency is once again on display.

Moreover, despite the prominence of King \textsanskrit{Bimbisāra} in the consciousness of many Buddhists, he is a marginal figure in the four main \textsanskrit{Nikāyas}, only mentioned in five separate \textit{suttas}, and he is never personally present in the narrated events. In DN 4 and 5, he is talked about in the third person as someone who respects certain brahmins, as a consequence of which he has granted them land (\href{https://suttacentral.net/dn4/en/sujato\#5.17}{DN~4:5.17} and \href{https://suttacentral.net/dn5/en/sujato\#6.17}{DN~5:6.17}). In DN  18, he has already passed away (\href{https://suttacentral.net/dn18/en/sujato\#4.12}{DN~18:4.12}). In MN 14 and 86, he is again spoken of in the third person (\href{https://suttacentral.net/mn14/en/sujato\#20.2}{MN~14:20.2} and \href{https://suttacentral.net/mn86/en/sujato\#9.4}{MN~86:9.4}). Already in the Suttas, \textsanskrit{Bimbisāra} has a certain mythical quality to him. He becomes the legendary ideal king against whom other kings, especially his own son \textsanskrit{Ajātasattu}, are measured. I believe there are good grounds to doubt whether the story of his meeting with the Buddha is authentic.

However this may be, the story continues with \textsanskrit{Bimbisāra} giving his Bamboo Grove park, situated just outside of \textsanskrit{Rājagaha}, to the Sangha. As part of the dedication, he pours water from a golden ceremonial vessel, a \textit{\textsanskrit{bhiṅkāra}}.\footnote{In the Suttas, we find the \textit{\textsanskrit{bhiṅkāra}} used by the wheel-turning monarch to sprinkle the wheel gem, e.g. at \href{https://suttacentral.net/dn17/en/sujato}{DN~17}. } This was presumably an ancient Indian custom,\footnote{See, for instance, the description of purification by water in Śatapatha-\textsanskrit{brāhmaṇa} 1.1.1 at \href{https://www.wisdomlib.org/hinduism/book/satapatha-brahmana-english/d/doc63113.html}{↩}} which you will see performed to the present day in Buddhist monasteries around the world. A remarkably large part of present-day Buddhist culture has its source in the Suttas and the Vinaya \textsanskrit{Piṭaka}.

The account of King \textsanskrit{Bimbisāra} is followed by a final and most consequential conversion story, namely that of \textsanskrit{Sāriputta} and \textsanskrit{Moggallāna}. One morning \textsanskrit{Sāriputta} observes Assaji, one of the first five monks, walking for alms. He is inspired by his demeanor and asks him who his teacher is. Assaji replies that it is the Buddha and then gives \textsanskrit{Sāriputta} the following brief teaching:

\begin{verse}%
“Of causally arisen things,\\

The Buddha has declared their cause,\\

As well as their ending.\\

This is the teaching of the Great Ascetic.”\footnote{At \href{https://suttacentral.net/pli-tv-kd1/en/brahmali\#23.5.2}{Kd~1:23.5.2}. }

%
\end{verse}

\textsanskrit{Sāriputta} immediately becomes a stream-enterer, as does his friend \textsanskrit{Moggallāna} when he is told soon afterwards. They go to the Buddha who gives them the full ordination, declaring that they will become his chief disciples. This concludes most of the biographical narrative of this chapter.

Kd 1 continues with a detailed discussion of the proper relationship between teachers and students.\footnote{\href{https://suttacentral.net/pli-tv-kd1/en/brahmali\#25.8.1}{Kd~1:25.8.1}–26.11.12 and \href{https://suttacentral.net/pli-tv-kd1/en/brahmali\#32.3.1}{Kd~1:32.3.1}–33.1.111. This section is not mentioned in the \textit{\textsanskrit{uddāna}}, the summary verses at the end of the chapter. Moreover, it is repeated verbatim at Kd 18. This suggests that this section did not originally belong to this chapter. } To ensure that newly ordained monastics get proper training, the Buddha lays down the role of the preceptor, the \textit{\textsanskrit{upajjhāya}} (\href{https://suttacentral.net/pli-tv-kd1/en/brahmali\#25.6.2}{Kd~1:25.6.2}). The preceptor is named during the ordination ceremony (\href{https://suttacentral.net/pli-tv-kd1/en/brahmali\#76.7.17}{Kd~1:76.7.17}). The newly ordained monastic must then live supported by their preceptor or another teacher for five years in the case of monks and for two years in the case of nuns.\footnote{See respectively \href{https://suttacentral.net/pli-tv-kd1/en/brahmali\#53.4.7}{Kd~1:53.4.7} and \href{https://suttacentral.net/pli-tv-bi-vb-pc69/en/brahmali\#1.13.1}{Bi~Pc~69}. From here on I will use the word teacher as a reference to both the preceptor and any other teacher who may take the place of the preceptor. } It is a relationship where compassion should be the focus, not severity of discipline. The teacher should look upon their student as their son or daughter, and a student upon their teacher as their father or mother (\href{https://suttacentral.net/pli-tv-kd1/en/brahmali\#25.6.3}{Kd~1:25.6.3}). Many of the details of proper conduct given in this section are still normative for how good monastics behave in the present day.

The relationship between teacher and student is surprisingly two-sided. As one would expect, the preceptor should guide the student if the student loses their way. But the reverse is also true. If the preceptor speaks in a way that borders on an offense, the student should stop them (\href{https://suttacentral.net/pli-tv-kd1/en/brahmali\#25.10.3}{Kd~1:25.10.3}). If the preceptor becomes discontent with the spiritual life, becomes anxious,\footnote{Presumably because he thinks he may have committed an offense. } has wrong view, or has committed a serious offense, the student should help them out of their predicament (\href{https://suttacentral.net/pli-tv-kd1/en/brahmali\#25.20.1}{Kd~1:25.20.1}–25.22.4). Ideally the preceptor should be an inspiring role model for his or her students. Reality, unfortunately, does not always measure up to such ideals.

The discussion on teachers and students continues with the issue of wrong conduct, especially on the part of the student. The student should ask for forgiveness and the teacher should grant it, not doing either of which is an offense of wrong conduct (\href{https://suttacentral.net/pli-tv-kd1/en/brahmali\#27.3.1}{Kd~1:27.3.1}–27.4.8). We then come to a list of five qualities that a good student should have in regard to his or her teacher: affection, confidence, conscience, and respect, and their mind should develop under the guidance of their teacher (\href{https://suttacentral.net/pli-tv-kd1/en/brahmali\#27.6.4}{Kd~1:27.6.4}).

Among a large number of origin stories for various rules, Kd 1 tells the touching account of a brahmin who wishes to ordain, but who is rejected by the Sangha. The Buddha then asks if anyone can remember any good actions of that brahmin, in response to which \textsanskrit{Sāriputta} says that he recalls him giving a ladleful of food. The Buddha tells \textsanskrit{Sāriputta} to ordain him. When \textsanskrit{Sāriputta} asks the Buddha how, the Buddha lays down a new ordination procedure, which constitutes the core of the one we still use in the present day. It is performed by the Sangha through a legal procedure of one motion and three announcements, a so-called \textit{\textsanskrit{saṅghakamma}}. I will look at \textit{\textsanskrit{saṅghakamma}} in more detail when I discuss Kd 9 below.

Many more rules concerned with ordination are laid down in the remainder of Kd 1. Many of these have the effect of expanding the ordination ceremony further,\footnote{Here are the main additions: “should be asked” and four supports at \href{https://suttacentral.net/pli-tv-kd1/en/brahmali\#29.1.1}{Kd~1:29.1.1}–30.4.13; illnesses at \href{https://suttacentral.net/pli-tv-kd1/en/brahmali\#39.6.3}{Kd~1:39.6.3}; those employed by the king at \href{https://suttacentral.net/pli-tv-kd1/en/brahmali\#40.4.1}{Kd~1:40.4.1}; debt and slavery at \href{https://suttacentral.net/pli-tv-kd1/en/brahmali\#46.1.1}{Kd~1:46.1.1}–47.1.12; 20 years minimum age at \href{https://suttacentral.net/pli-tv-kd1/en/brahmali\#49.5.7}{Kd~1:49.5.7}; parental permission at \href{https://suttacentral.net/pli-tv-kd1/en/brahmali\#54.6.1}{Kd~1:54.6.1}; must be human at \href{https://suttacentral.net/pli-tv-kd1/en/brahmali\#63.5.2}{Kd~1:63.5.2}; must have preceptor at \href{https://suttacentral.net/pli-tv-kd1/en/brahmali\#69.1.1}{Kd~1:69.1.1}; and must have bowl and robe at \href{https://suttacentral.net/pli-tv-kd1/en/brahmali\#70.1.1}{Kd~1:70.1.1}–70.3.7. } until it reaches its final form toward the end of the chapter (\href{https://suttacentral.net/pli-tv-kd1/en/brahmali\#76.5.6}{Kd~1:76.5.6}–76.12.17). Apart from the rules that directly relate to the ordination ceremony, there are other rules that concern ordination in a broader sense. There is a section that lists the qualities required of one who wishes to give ordinations (\href{https://suttacentral.net/pli-tv-kd1/en/brahmali\#36.2.1}{Kd~1:36.2.1}–37.14.2). Then there are rules on the ordination and training of novices (\href{https://suttacentral.net/pli-tv-kd1/en/brahmali\#50.1.1}{Kd~1:50.1.1}–52.1.8 and \href{https://suttacentral.net/pli-tv-kd1/en/brahmali\#54.1.1}{Kd~1:54.1.1}–60.1.15), and a discussion on the ordination of monastics of other religions (\href{https://suttacentral.net/pli-tv-kd1/en/brahmali\#38.1.1}{Kd~1:38.1.1}–38.11.7). The latter need to be on probation for four months to show that their true faith lies with the Buddha. We see this requirement mentioned quite regularly in the Suttas.

Kd 1 also contains rules on \textit{nissaya}, often rendered as “dependence”, but here given as “formal support” (\href{https://suttacentral.net/pli-tv-kd1/en/brahmali\#35.1.1}{Kd~1:35.1.1}–36.1.7, \href{https://suttacentral.net/pli-tv-kd1/en/brahmali\#53.4.2.1}{Kd~1:53.4.2.1}–53.13.5, and \href{https://suttacentral.net/pli-tv-kd1/en/brahmali\#72.1.1}{Kd~1:72.1.1}–73.4.9). \textit{Nissaya} concerns the relationship between teacher and student. The basic idea is that a newly ordained monk should stay with his teacher for a minimum of five years or until he is sufficiently knowledgeable to live independently, whichever is the longest. There are many more rules, mostly minor, that I will not discuss here.

\section*{The Chapter on the Observance Day, Uposatha-kkhandhaka, Kd 2}

The chapter on the observance day, the \textit{uposatha}, is a natural continuation of the chapter on ordination. Once the Sangha had come into existence, it would have required certain basic functions to make it a cohesive entity. The most fundamental of these functions is the half-monthly observance day when the Sangha meets to give Dhamma talks and recite the \textsanskrit{Pātimokkha}.

The origin story to this chapter points to the \textit{uposatha} being an ancient religious observance that goes back to the time before the Buddha. It was held on the full moon, the new moon, and the quarter moons. The monastics of the various religions would gather and give teachings, thus creating a following among householders.

The Buddha often adopted the customs of contemporary society. As we shall see in the next chapter, he accepted the established practice of staying put during the rainy season. He took on practices such as putting one’s palms together as a sign of respect, known as \textit{\textsanskrit{añjali}}, and bowing. He had his monastics wear robes that were hardly distinguishable from the robes of monastics of other religions.\footnote{See \href{https://suttacentral.net/pli-tv-bu-vb-np20/en/brahmali\#1.4}{Bu~NP~20:1.4}. } Most important of all, he often adopted the religious vocabulary of the time, keeping much of the existing meaning while often adding a slant of his own.\footnote{Examples include words such as \textit{kamma}, \textit{\textsanskrit{samādhi}}, \textit{\textsanskrit{jhāna}}, \textit{\textsanskrit{brahmaṇa}}, and many more. } The Buddha was a pragmatist. Although the essence of his message was revolutionary, he only broke with convention when necessary.

When the Buddha lays down that the \textsanskrit{Pātimokkha} should be recited on the \textit{uposatha}, he provides a preamble, a \textit{\textsanskrit{nidāna}}, to the recitation of the rules (\href{https://suttacentral.net/pli-tv-kd2/en/brahmali\#3.2.4}{Kd~2:3.2.4}). This is followed by a word commentary, the only such commentary in the Khandhakas (\href{https://suttacentral.net/pli-tv-kd2/en/brahmali\#3.4.1}{Kd~2:3.4.1}–3.8.7). Neither the \textit{\textsanskrit{nidāna}} nor the word commentary is found in the Sutta-\textsanskrit{vibhaṅga}, where it would seem to belong. Instead, it is part of the \textsanskrit{Pātimokkha} as preserved in commentaries.\footnote{In the sub-commentary known as the \textsanskrit{Dvemātikāpāḷi}. } It seems likely to me that the \textit{\textsanskrit{nidāna}} together with its commentary originally found its home in the Sutta-\textsanskrit{vibhaṅga}, but was moved to the Khandhakas once these had been created. We see the same process at play that we have discussed above with reference to the seven principles for settling legal issues.

The purpose of the recitation the \textsanskrit{Pātimokkha} is to remind the monks and the nuns of the rules they are meant to follow. To emphasize this point, the Buddha lays down a rule that one should not listen to the \textsanskrit{Pātimokkha} with unconfessed offenses, followed by a confession formula (\href{https://suttacentral.net/pli-tv-kd2/en/brahmali\#27.1.3}{Kd~2:27.1.3}). The question sometimes arises of which offenses one needs to confess: the \textsanskrit{Pātimokkha} offenses or all offenses laid down in the Vinaya \textsanskrit{Piṭaka}? On the answer to this question hinges the important principle of how the confession is to be done. There are so many offenses in the Vinaya as a whole that it is impossible to remember them all. From this arises the modern habit of doing general confessions of entire classes of offenses, just in case one has forgotten an offense.

Yet general confessions are a problem. Part of the confession formula is to undertake restraint for the future, which is impossible if one does a general confession. One needs to be clear about which offense one has committed to be able to take on restraint. And so the confession formula degenerates into a ritual, its original purpose no longer fulfilled.

Fortunately, we have good indications that only the \textsanskrit{Pātimokkha} offenses need to be confessed.\footnote{It is perfectly fine to also confess other offenses. The point is that it is not necessary for listening to the \textsanskrit{Pātimokkha} recitation. } For instance, we encounter a monk who remembers an offense while listening to the \textsanskrit{Pātimokkha}, presumably because he has just been reminded (\href{https://suttacentral.net/pli-tv-kd2/en/brahmali\#27.4.1}{Kd~2:27.4.1}). Moreover, we know from \href{https://suttacentral.net/an4/en/brahmali\#244}{AN~4.244} that in the earliest period there were only four classes of offenses, that is, the main offenses of the \textsanskrit{Pātimokkha}, excluding the \textit{sekhiyas}. The introduction to Kd 2 speaks of five or seven classes of offenses, which is curious given that the five are included in the seven. It seems likely that five is the earlier reading, with the seven added at some point. The five would have been the four classes mentioned at AN 4.244, plus the \textit{dukkata} offenses in the \textit{sekhiya} rules, and so we are again limited to the \textsanskrit{Pātimokkha} offenses. Then there is \href{https://suttacentral.net/pli-tv-bu-vb-pc72/en/brahmali\#1.20.1}{Bu~Pc~72} and \href{https://suttacentral.net/pli-tv-bu-vb-pc73/en/brahmali\#1.15.1}{Bu~Pc~73}, both of which concern monks who become aware, or pretend to become aware, of an offense as they hear it mentioned during the \textsanskrit{Pātimokkha} recitation. There is nothing about offenses outside the \textsanskrit{Pātimokkha}.

Moreover, as we have seen, monks sometimes remembered offenses during the \textsanskrit{Pātimokkha} recitation, in which case they did the confession after the recitation (\href{https://suttacentral.net/pli-tv-kd2/en/brahmali\#27.5.1}{Kd~2:27.5.1}). Had they done a general confession beforehand, this would not be required, and so it is clear that the monastics did not do general confessions at this time. If they had forgotten an offense, they would presumably either not confess it at all or wait until they remembered it. On top of this, we have the fact that the name of the offense is always specified in the confession formulas given in the Vinaya \textsanskrit{Piṭaka}.\footnote{The formula is found at \href{https://suttacentral.net/pli-tv-kd2/en/brahmali\#27.1.10}{Kd~2:27.1.10}, \href{https://suttacentral.net/pli-tv-kd4/en/brahmali\#6.1.10}{Kd~4:6.1.10}, and \href{https://suttacentral.net/pli-tv-kd14/en/brahmali\#14.30.11}{Kd~14:14.30.11}–14.32.12. } I conclude that confession is about the clearing of specific offenses, and not a generalized ritual.

We now come to Monastery zones, \textit{\textsanskrit{sīmās}}, another fundamental Vinaya topic discussed in this chapter (\href{https://suttacentral.net/pli-tv-kd2/en/brahmali\#6.1.1}{Kd~2:6.1.1}–7.2.6). Once the \textit{uposatha} ceremony is laid down, the Sangha needs to know who should attend. As the monastics gradually disperse over a large area, it becomes impossible to assemble them all for the twice-monthly recitation of the \textsanskrit{Pātimokkha}. The Buddha then lays down the creation of monastery zones, areas within which all monastics must come together to perform the observance-day ceremony or to carry out other official Sangha business. The number of rules concerned with monastery zones in the Vinaya \textsanskrit{Piṭaka} is relatively small, but in later Pali literature this becomes a major issue, with entire tracts dedicated to the analysis of a variety of mostly marginal circumstances.\footnote{CPD lists altogether eight such \textit{\textsanskrit{sīmā}} tracts. }

It soon became necessary to build observance-day halls (\href{https://suttacentral.net/pli-tv-kd2/en/brahmali\#8.1.1}{Kd~2:8.1.1}–8.4.11). It seems reasonable to infer that monasteries would have developed around such core infrastructure. In later \textit{khandhakas}, especially Kd 16, we shall see how a variety of buildings are allowed by the Buddha, presumably leading to quite extensive monastic institutions. While the early ideal of the independent monastic no doubt still existed, a significant portion of the monastic community would have settled within highly developed monasteries. This is to be expected. As Buddhism attracted an ever-greater number of monastics, only a decreasing proportion would have been able to cope with the solitude and simplicity of a more independent lifestyle.

Most of the remainder of Kd 2 concerns details of the \textit{uposatha} ceremony. As part of this, there is a brief discussion of \textit{\textsanskrit{saṅghakamma}}, the legal procedures of the Sangha (\href{https://suttacentral.net/pli-tv-kd2/en/brahmali\#16.4.1}{Kd~2:16.4.1}–16.5.9). According to the \textsanskrit{Parivāra}, the \textit{uposatha} ceremony is a kind of \textit{\textsanskrit{saṅghakamma}}.\footnote{A so-called \textit{\textsanskrit{ñattikamma}}, “a legal procedure consisting of one motion”. See \href{https://suttacentral.net/pli-tv-pvr21/en/brahmali\#15.5}{Pvr~21:15.5}. } In the rest of the Vinaya, however, it seems to be regarded as separate from \textit{\textsanskrit{saṅghakamma}}. In a number of places, we find \textit{\textsanskrit{saṅghakamma}} and \textit{uposatha}, and often \textit{\textsanskrit{pavāraṇā}} (“invitation”) as well, listed as separate items, indicating that they were not regarded as equivalent.\footnote{See \href{https://suttacentral.net/pli-tv-bu-vb-ss8/en/brahmali\#3.1.4}{Bu~Ss~8:3.1.4}, \href{https://suttacentral.net/pli-tv-bu-vb-ss9/en/brahmali\#2.3.9}{Bu~Ss~9:2.3.9}, \href{https://suttacentral.net/pli-tv-bu-vb-pc69/en/brahmali\#2.1.21}{Bu~Pc~69:2.1.21}, \href{https://suttacentral.net/pli-tv-kd2/en/brahmali\#5.3.3}{Kd~2:5.3.3}, \href{https://suttacentral.net/pli-tv-kd10/en/brahmali\#1.6.4}{Kd~10:1.6.4}, and \href{https://suttacentral.net/pli-tv-kd17/en/brahmali\#5.2.21}{Kd~17:5.2.21}. } I conclude that, in the earliest period, neither the \textit{uposatha} ceremony nor the \textit{\textsanskrit{pavāraṇā}} ceremony were regarded as \textit{\textsanskrit{saṅghakammas}} proper. This is an interesting point which I will return to when I discuss \textit{\textsanskrit{saṅghakamma}} in greater detail in relation to Kd 9 below.

Nevertheless, it is clear that many of the rules that govern \textit{\textsanskrit{saṅghakamma}} are binding on the \textit{uposatha} ceremony, including the requirement that the assembly be complete (\href{https://suttacentral.net/pli-tv-kd2/en/brahmali\#23.1.1}{Kd~2:23.1.1}–24.3.14 and \href{https://suttacentral.net/pli-tv-kd2/en/brahmali\#28.1.1}{Kd~2:28.1.1}–34.13.5). Because these rules are the same for the two circumstances, and because the \textit{uposatha} ceremony is described in greater detail than the exposition of any specific \textit{\textsanskrit{saṅghakamma}}, it seems reasonable to take the rules for the \textit{uposatha} ceremony as normative for \textit{\textsanskrit{saṅghakamma}}. This matters, as I will now show.

The validity of ordinations is a perennial issue, often discussed in monastic circles. In brief, the question is how we can know the validity of all ordinations going all the way back to the time of the Buddha. The straightforward answer is that this is impossible. There are many ways in which \textit{\textsanskrit{saṅghakammas}} fail, any of which would invalidate an ordination.\footnote{Examples include the quorum not being met, for instance because one or more monks in the assembly have committed a \textit{\textsanskrit{pārājika}} or are otherwise non-\textit{bhikkhus}. Another example is a monk entering the monastery zone while the ordination is being performed, thus making the assembly incomplete. } Given the history of the Sangha and its periodic corruption, one could then reasonably doubt whether the current crop of \textit{bhikkhus} are real monastics.

It is in this context that the following passage is particularly interesting:

\begin{quotation}%
“On the observance day, four or more resident monks may have gathered together in a certain monastery. They don’t know there are other resident monks who haven’t arrived. Perceiving that they’re acting according to the Teaching and the Monastic Law, perceiving that the assembly is complete although it’s not, they do the observance-day ceremony and recite the Monastic Code. When they’ve just finished, and the entire gathering has left, a smaller number of resident monks arrive. In such a case, what has been recited is valid, and the late arrivals should announce their purity in the presence of the others. There’s no offense for the reciters.” (\href{https://suttacentral.net/pli-tv-kd2/en/brahmali\#28.7.15}{Kd~2:28.7.15}–28.7.21)

%
\end{quotation}

Here the \textit{uposatha} ceremony has been performed with an incomplete assembly. Had they known that the assembly was incomplete, the ceremony would have been invalid and the monks would have committed an offense, as the subsequent section makes clear. But because they perceive the assembly as complete, the ceremony is valid and there is no offense for the monks taking part. The important point here is that it is their \textit{perception} that matters. That is, if they perceive the assembly as complete, then for all practical purposes it is complete.

This is different from how \textit{\textsanskrit{saṅghakamma}} is generally understood. Most monastics will assume that an ordination is invalid if a monastic happens to pass through the monastery zone while the ordination is being performed.\footnote{See for instance Bhikkhu \textsanskrit{Ṭhānissaro}, “The Buddhist Monastic Code II”, p. 174: “However, large territories create their own difficulties. To begin with, there is the difficulty in ensuring that, during a meeting, no unknown bhikkhus have wandered into the territory, invalidating any transaction carried out at the meeting.” } This passage proves the opposite. Yet one of the drivers of the tradition of creating small monastery zones known as \textit{\textsanskrit{khaṇḍa}-\textsanskrit{sīmās}}, often within the walls of a building, is precisely to avoid anyone entering the zone while a \textit{\textsanskrit{saṅghakamma}} is carried out. But given this passage in Kd 2, this seems unnecessary. Worse, the whole tradition of small monastery zones voids the purpose of such zones, which is to ensure that the whole monastic community is present when important decisions are made. With a proper monastery zone that extends over an entire monastery, all the residents must be present for the decision to be valid. With a small monastery zone, any group of four monastics can make whatever decision they wish without consultation. The democratic system of \textit{\textsanskrit{saṅghakamma}} effectively breaks down.

If, however, as suggested in Kd 2, perception is a factor in determining the validity of \textit{\textsanskrit{saṅghakamma}}, there are two important benefits. First, so long as you are not aware of any specific ordination in the past that was invalid, you can conclude that \textit{bhikkhus} today are monks in the true sense of the word.\footnote{The same basic idea applies for nuns, but the situation is more complicated due to the disappearance of the ordination lineage in Theravadin countries. I will briefly return to this topic when I discuss the \textsanskrit{Bhikkhunī}-kkhandhaka in the introduction to volume 5. } Second, there is no longer any good reason to create small monastery zones. In fact, to ensure that monastery zones fulfill their original purpose, it would make sense to return to the ancient practice of creating monastery zones that cover meaningful areas, such as complete monasteries.

\section*{The Chapter on Entering the Rainy-season Residence, \textsanskrit{Vassūpanāyika}-kkhandhaka, Kd 3}

Kd 3 concerns the annual three-month rainy-season residence, which is compulsory for all monastics. During this period, which coincides with the Indian monsoon season, monastics must stay put in one place. According to Kd 3, this is because travel would result in the destruction of life, but presumably it was also because travel was difficult and even hazardous at this time. Interestingly, it seems householders too would sometimes stay put during the rainy season (\href{https://suttacentral.net/an11.13/en/sujato\#2.1}{AN~11.13:2.1}).

As with the \textit{uposatha}, the Buddha adopted this tradition from the preexisting norms for monastics. As such, it is reasonable to think that this was instituted soon after the Sangha reached a certain size, probably quite early in the Buddha’s teaching career. The placement of this chapter immediately after the chapter on the \textit{uposatha} is thus natural and may reflect the chronological sequence in which these things were laid down. The sequence of the first four chapters of the Khandhakas is in fact the same for all early schools except the \textsanskrit{Mahāsaṅghikas}, who reorganized their Khandhakas (known as \textsanskrit{Pakiṇṇakas}) away from the general structure used by the other schools.

Most rules in this chapter are directly related to the rainy-season residence. To begin with, monastics are allowed to travel for seven days if there is important business to be undertaken, such as looking after a sick monastic or a family member (\href{https://suttacentral.net/pli-tv-kd3/en/brahmali\#5.1.1}{Kd~3:5.1.1}–8.1.7). Then follows a section which lays down that the place of residence must be properly covered—and have a door, says the commentary\footnote{Sp 3.204. }—of which a simple hut, a \textit{\textsanskrit{kuṭi}}, would presumably be the most obvious choice (\href{https://suttacentral.net/pli-tv-kd3/en/brahmali\#12.1.1}{Kd~3:12.1.1}–12.9.6). Again, we see that a settled form of Buddhism must have existed virtually from the beginning.

This chapter appears to allow a monastic to spend the rainy season in more than one place, a fact that is rarely commented on. The monk Upananda is shown to spend the rainy season in two different monasteries, but is neither penalized nor criticized for this (\href{https://suttacentral.net/pli-tv-kd3/en/brahmali\#14.1.1}{Kd~3:14.1.1}). In the Chapter on Robes, we even find the Buddha laying down a rule on the appropriate distribution of cloth for monastics who spend the rains in two different monasteries (\href{https://suttacentral.net/pli-tv-kd8/en/brahmali\#25.4.1}{Kd~8:25.4.1}). This makes it clear that such an arrangement was considered acceptable.

Tangentially to the main content, monastics are told to comply with “the wishes of kings”.\footnote{\textit{\textsanskrit{Rājūnaṁ} \textsanskrit{anuvattituṁ}}, literally, “(You should) behave according to the kings,” at \href{https://suttacentral.net/pli-tv-kd3/en/brahmali\#4.3.}{Kd~3:4.3.1}. } One implication of this is presumably that laws laid down by kings must be adhered to. And so, despite occasional indications to the contrary,\footnote{See for instance the origin story to Bu Pj 2, where the monk Dhaniya avoids punishment because of his status as a monk (\href{https://suttacentral.net/pli-tv-bu-vb-pj2/en/brahmali\#1.5.16}{Bu~Pj~2:1.5.16}). } it seems monastics too are bound by the law of the land.

\section*{The Chapter on the Invitation Ceremony, \textsanskrit{Pavāraṇā}-kkhandhaka, Kd 4}

This chapter lays down an annual invitation ceremony at which monks and nuns invite admonition from their fellow monastics. This happens at the end of the rainy-season residence, at which point most monastics will have spent three months living in the company of co-monastics, which would put them in a good position to give constructive feedback. According to \href{https://suttacentral.net/dhp76%E2%80%9389/en/brahmali\#1}{Dhp 76}, the pointing out of real flaws in one’s character is equivalent to the revealing of a treasure. Again, with the invitation ceremony coming straight after the chapter on the rainy-season residence, we see a natural chronological evolution of these rules and regulations.

Kd 4 begins with a story of monastics spending the three months of the rainy-season residence without talking to each other. Perhaps surprisingly, the Buddha admonishes them for this practice, calling it a living in discomfort, \textit{\textsanskrit{aphāsuṁ} \textsanskrit{vuṭṭhā}} (\href{https://suttacentral.net/pli-tv-kd4/en/brahmali\#1.12.2}{Kd~4:1.12.2}). Right speech in the Dhamma is about saying what is necessary in the right way and at the right time, with an emphasis on being quiet, but taking a vow of silence is going too far.

As with the observance-day ceremony, the invitation ceremony does not seem to have been regarded as a \textit{\textsanskrit{saṅghakamma}} in the earliest period of Buddhism, for which see the discussion in Kd 2 above. Still, as with the \textit{uposatha} ceremony, the rules for \textit{\textsanskrit{saṅghakamma}}, especially those concerning legitimate assemblies, are applicable here too. A large part of this chapter is taken up with such rules.

\section*{The Chapter on Skins, Camma-kkhandhaka, Kd 5}

An important part of Kd 5 is its treatment of allowable and unallowable leather goods, hence its name. It also contains two interesting stories and a host of minor rules, many of which concern footwear.

Kd 5 begins with the story of \textsanskrit{Soṇa} \textsanskrit{Koḷivisa} who had been raised in such comfort that he had hairs growing on the soles of his feet.\footnote{The full story is at \href{https://suttacentral.net/pli-tv-kd5/en/brahmali\#1.1.1}{Kd~5:1.1.1}–1.27.20. The latter part of this story has a parallel at \href{https://suttacentral.net/an6.55/en/sujato}{AN~6.55}. } King \textsanskrit{Bimbisāra} of Magadha demands to see this, and \textsanskrit{Soṇa} is sent to meet the king. When the king has been duly satisfied, \textsanskrit{Soṇa} joins a group of 80,000 village chiefs to see the Buddha. The Buddha’s attendant, Ven. \textsanskrit{Sāgata}, displays numerous psychic powers, after which the Buddha teaches the Dhamma, leading all 80,000 to stream-entry. \textsanskrit{Soṇa}, however, asks for the going forth.

After going forth, \textsanskrit{Soṇa} exerts himself to such an extent that, according to the list of the Buddha’s most prominent disciples at \href{https://suttacentral.net/an1.198-208/en/sujato\#an1.205_1.1}{AN~1.205}, he is foremost in putting forth energy. Due to his sheltered upbringing, however, his feet are unable to cope with his long hours on the walking path. He sheds so much blood that the path looks “like a slaughterhouse”! When he thinks of returning to lay life, the Buddha visits him and teaches him the well-known simile of the lute: just as a lute is melodious only when the strings have the right tension, so the practice only ripens in good results when the energy has the right balance. \textsanskrit{Soṇa} follows the Buddha’s instructions, eventually becoming an \textit{arahant}. He declares his achievement to the Buddha with stirring words, concluding with a beautiful set of verses (\href{https://suttacentral.net/pli-tv-kd5/en/brahmali\#1.20.1}{Kd~5:1.20.1}–1.27.20). Then, as a rather abrupt anticlimax to a remarkable story, the Buddha lays down an allowance for monastics to use sandals.

After a number of further rules mostly concerned with footwear, we come to a prohibition against monastics using luxurious furniture, including beds (\href{https://suttacentral.net/pli-tv-kd5/en/brahmali\#10.5.1}{Kd~5:10.5.1}–10.5.3). Minor as it may seem, this rule fills an important gap in the \textsanskrit{Pātimokkha}. Looking at the \textsanskrit{Pātimokkha} in isolation, one might conclude that fully ordained monks do not need to keep all the precepts of a novice. Yet those of the ten precepts that are not found among the \textsanskrit{Pātimokkha} rules are all covered in the Khandhakas. High and luxurious beds are prohibited here, whereas entertainment and personal beautification are banned at \href{https://suttacentral.net/pli-tv-kd15/en/brahmali\#2.6.6}{Kd~15:2.6.6} and \href{https://suttacentral.net/pli-tv-kd15/en/brahmali\#2.1.1}{Kd~15:2.1.1}–2.5.11.

Kd 5 ends with the story of \textsanskrit{Soṇa} \textsanskrit{Kuṭikaṇṇa},\footnote{The full story is at \href{https://suttacentral.net/pli-tv-kd5/en/brahmali\#13.1.1}{Kd~5:13.1.1}–13.11.4. It has a parallel at \href{https://suttacentral.net/ud5.6/en/sujato}{Ud~5.6}. } at the end of which the Buddha loosens some of the Vinaya rules for areas beyond the Ganges plain. The story begins with \textsanskrit{Soṇa}, who lives in a distant country, seeking ordination. But he is unable to obtain it because of the difficulty in assembling ten monks to perform the ceremony. He eventually gets ordained after waiting for three years. Soon, he decides to visit the Buddha. When he arrives, he is put up in the Buddha’s dwelling, a sign that he is regarded as special.

The following morning the Buddha asks him to recite Dhamma, upon which he chants the Chapter of Eights, the \textsanskrit{Aṭṭhakavagga}, which is now included in the Sutta \textsanskrit{Nipāta}. The Buddha praises him and then asks why it took him so long to get ordained. Seeing that there is a problem, the Buddha agrees to relax some of the rules for distant countries. Most importantly, he reduces the number of monastics required to perform an ordination ceremony from ten to five.

It is possible that the position of Kd 5 after the Chapter on the Invitation Ceremony is a result of this charming story. As the Sangha grew, it would have gradually spread out over a large area. The need to reform certain rules to accommodate this spread would no doubt have been felt from early on.

\section*{The Chapter on Medicines, Bhesajja-kkhandhaka, Kd 6}

The core concerns of Kd 6 are medicines and almsfood. These are two of the four requisites of a monastic, the other two being robes and dwellings. Robes are dealt with in Kd 8, whereas dwellings and other buildings feature in Kd 16. As we shall see, Kd 6 also contains many interesting and entertaining stories, featuring some of the most beloved characters from the Suttas.

The idea of medicines is quite broad in early Buddhism. It includes certain foodstuffs that can provide a boost of energy without being classed as substantial foods. Kd 6 opens with the Buddha allowing such “tonics” for sick monastics—even outside the regular meal time from dawn to noon. The discussion moves on to medicines proper, medical equipment, and medical treatments and procedures.

After an entertaining story with Ven. Pilindavaccha, which I will return to shortly, we find a series of rules on food. Noteworthy regulations include the prohibition against cooking (\href{https://suttacentral.net/pli-tv-kd6/en/brahmali\#17.3.9}{Kd~6:17.3.9}); the relaxation of certain rules at times of famine (\href{https://suttacentral.net/pli-tv-kd6/en/brahmali\#17.7.1}{Kd~6:17.7.1}–20.4.4); and, further on, the prohibition against human meat and the meat of animals considered noble, disgusting, or dangerous (\href{https://suttacentral.net/pli-tv-kd6/en/brahmali\#23.1.1}{Kd~6:23.1.1}–23.15.9).

Now let us turn to the stories, starting with the extraordinary Pilindavaccha (\href{https://suttacentral.net/pli-tv-kd6/en/brahmali\#15.1.1}{Kd~6:15.1.1}–15.10.8). Pilindavaccha is building a shelter when King \textsanskrit{Bimbisāra} offers to support him with monastery workers. Eventually there is a whole village with such workers, all closely affiliated with Pilindavaccha. One day when Pilindavaccha arrives at the house of a poor family, the daughter is crying because her parents cannot afford ornaments for her. Pilindavaccha uses his psychic powers to create a golden garland, upon which the whole family is arrested and charged with theft. Pilindavaccha goes to the king, turns his house into gold, and asks where he has gotten so much gold from. The king realizes that Pilindavaccha was using special powers all along, and he releases the family. The downside for Pilindavaccha is that he now becomes famous. He is given large amounts of tonics, to the point where the Buddha lays down a rule against storing tonics for more than seven days.

Pilindavaccha is also encountered at Bu Pj 2 where he saves two young boys from kidnappers, perhaps the earliest kidnapping story in the history of literature (\href{https://suttacentral.net/pli-tv-bu-vb-pj2/en/brahmali\#7.47.1}{Bu~Pj~2:7.47.1}). Again, he uses psychic powers. Yet despite his special abilities, Pilindavaccha was also sickly, as can be seen from the early parts of Kd 6. Perhaps this combination of strength and vulnerability made him especially beloved. At \href{https://suttacentral.net/an1.215/en/sujato}{AN~1.215} he is said to be the Buddha’s foremost disciple in being dear to the gods.

Then there is the striking story of \textsanskrit{Suppiyā} who has so much faith in the Sangha that she cuts a piece of meat from her own thigh to support a sick monk (\href{https://suttacentral.net/pli-tv-kd6/en/brahmali\#23.1.4}{Kd~6:23.1.4}–23.9.10). The Buddha uses his special powers to instantly heal her thigh. He then criticizes the monks for not being more circumspect in receiving offerings from lay supporters. The theme of the importance of sensitivity to lay supporters is continued in the story of the government minister with weak faith (\href{https://suttacentral.net/pli-tv-kd6/en/brahmali\#25.1.1}{Kd~6:25.1.1}–25.7.9). The minister invites a large sangha for a meal, but is hurt when the monks only eat a little. It turns out they have eaten elsewhere beforehand. The Buddha criticizes them and lays down a rule prohibiting such behavior.

The story of the General \textsanskrit{Sīha} , also found at \href{https://suttacentral.net/an8.12/en/sujato}{AN~8.12}, tells of how he converted from Jainism after a meeting with the Buddha (\href{https://suttacentral.net/pli-tv-kd6/en/brahmali\#31.1.1}{Kd 31.1.1}–31.14.6). The story ends with \textsanskrit{Sīha} inviting the Buddha and the Sangha for a meal that includes meat, showing that the Buddha ate meat. Still, the Buddha lays down a rule that a monastic may only eat meat when they have no reason to believe that the animal was killed especially for them.

Another story concerns the remarkable layman \textsanskrit{Meṇḍaka} (\href{https://suttacentral.net/pli-tv-kd6/en/brahmali\#34.1.1}{Kd~6:34.1.1}–34.16.3). He and many of his family members have supernormal powers, as does his slave. After a long story in which \textsanskrit{Meṇḍaka} and his family show their powers to a government minister, \textsanskrit{Meṇḍaka} becomes a follower of the Buddha. The Buddha grants \textsanskrit{Meṇḍaka} his wish of supplying provisions for monastics who are traveling in the wilderness. He then lays down a rule that one may look for provisions before traveling in such places. He also lays down a rule that an attendant may keep money on behalf of a monastic. This allowance, known as the \textsanskrit{Meṇḍaka} allowance, becomes the basis for \href{https://suttacentral.net/pli-tv-bu-vb-np10/en/brahmali\#1.3.1}{Bu~NP~10}, which explains the procedure for making use of such funds.

In most of these stories we see the prominent use of psychic powers. Again, this suggests that these stories are slightly later than the four main \textsanskrit{Nikāyas} and the core material of the Vinaya \textsanskrit{Piṭaka}. There are many more stories in this chapter, including the account of the Magadhan ministers Sunidha and \textsanskrit{Vassakāra} and the account of \textsanskrit{Ambapālī} offering her mango grove. These are essentially the same as the parallel versions found in the \textsanskrit{Mahāparinibbāna} Sutta (at \href{https://suttacentral.net/dn16/en/sujato\#1.19.1}{DN~16:1.19.1}–2.3.10 and \href{https://suttacentral.net/dn16/en/sujato\#2.14.1}{DN~16:2.14.1}–2.19.9). These remnants of DN 16 in Kd 6 could be a further indication that the whole Sutta originally was part of the Vinaya, as discussed earlier.

Kd 6 ends with an important set of rules sometimes called “the four great standards”, the \textit{\textsanskrit{catumahāpadesa}}, which essentially state that new circumstances are to be compared to existing ones and adjudicated according to the ones they resemble the most (\href{https://suttacentral.net/pli-tv-kd6/en/brahmali\#40.1.4}{Kd~6:40.1.4}). These rules are a response to the reality that it is impossible to cover all conceivable situations with a fixed set of rules. Moreover, such standards become particularly important after the Buddha’s passing and as society evolves in unpredictable ways. The Buddha makes it clear in the \textsanskrit{Mahāparinibbāna} Sutta that the monastics should not create new rules after he is gone (\href{https://suttacentral.net/dn16/en/sujato\#1.6.13}{DN~16:1.6.13}). This regulation, then, is a way of creating the necessary flexibility in the Vinaya so that it can deal with new and unforeseen circumstances.

\section*{The Chapter on the Robe-making Ceremony, Kathina-kkhandhaka, Kd 7}

The robe-making ceremony was laid down to help monastics obtain new robes before they set out wandering at the end of the rainy-season residence. Having spent three months in one place, the monastics would have built up a relationship with the local lay Buddhists. The robe-making ceremony was an opportunity for the lay followers to express their gratitude by offering robe-cloth to the monastics.

In brief, this is how it works. The lay supporters designate a special cloth, which they give for the purpose of performing the \textit{kathina} ceremony. The Sangha decides on an individual monastic to receive the \textit{kathina} robe, which they then proceed to sew. When the robe is finished, it is given to the designated recipient, with everyone expressing their approval. All the monastics who take part in this process gain so-called \textit{kathina} privileges. The most important of these is that one may continue collecting robe-cloth until the end of the cold season, four months after the end of the rainy season. This is a significant extension on the normal ten-day limit for storing robe-cloth at \href{https://suttacentral.net/pli-tv-bu-vb-np1/en/brahmali\#2.17.1}{Bu~NP~1}, giving the monastics involved a reasonable opportunity to collect enough cloth to make a robe. The remaining four privileges, which are set out at \href{https://suttacentral.net/pli-tv-kd7/en/brahmali\#1.3.2}{Kd~7:1.3.2}–1.3.4, are concerned with making it easier for monastics to acquire robe-cloth.

The description of the \textit{kathina} ceremony in Kd 7 is short and lacking in certain detail. According to Frauwallner, p. 185, “… the description of the Kathina procedure itself is so mutilated, that without comparing the other Vinaya it is impossible to get a clear idea of it.” To gain a full understanding of the process, it is necessary to consult the “Subdivision on the Robe-making Ceremony”, the \textit{kathinabheda}, in the \textsanskrit{Parivāra} at \href{https://suttacentral.net/pli-tv-pvr16/en/brahmali}{Pvr~16}. This is one of the few occasions where the \textsanskrit{Parivāra} contains critical information for a proper understanding of the Vinaya.

The remainder of Kd 7 sets out a long permutation series on when the \textit{kathina} privileges come to an end. This, in turn, is summarized as the coming together of two factors: (1) one leaves the monastery at which one stayed for the rains residence with no intention to return; and (2) one either has made a robe or gives up one’s intention to make one (\href{https://suttacentral.net/pli-tv-kd7/en/brahmali\#13.1.1}{Kd~7:13.1.1}–13.2.7).

The \textit{kathina} ceremony is the last of the main Sangha procedures that all or most monastics are expected to take part in. There are further legal procedures laid down in the remaining \textit{khandhakas}, but they concern special circumstances.

\section*{The Chapter on Robes, \textsanskrit{Cīvara}-kkhandhaka, Kd 8}

As we have seen, in Kd 6 the focus is on medicines and almsfood, whereas here the focus is on robes, the third of the four traditional requisites of a monastic. The fourth requisite, dwellings, is the subject of Kd 16. Apart from rules relating to robes, this chapter has three interesting and inspiring stories that enliven an otherwise dry exposition.

Monastic robes are heavily regulated. Allowable materials and colors are specified, as is the pattern into which the robe may be sewn (respectively at \href{https://suttacentral.net/pli-tv-kd8/en/brahmali\#3.1.5}{Kd~8:3.1.5}, \href{https://suttacentral.net/pli-tv-kd8/en/brahmali\#29.1.18}{Kd~8:29.1.18}, and \href{https://suttacentral.net/pli-tv-kd8/en/brahmali\#12.1.1}{Kd~8:12.1.1}). The number of robes is restricted to three (\href{https://suttacentral.net/pli-tv-kd8/en/brahmali\#13.5.8}{Kd~8:13.5.8}). The common practice of keeping extra robes is only made possible by exploiting loopholes in the rules. The maximum size of the robes is not specified here but at \href{https://suttacentral.net/pli-tv-bu-vb-pc92/en/brahmali}{Bu~Pc~92}–9.4.4. There is even a correct procedure for the distribution of robe-cloth (\href{https://suttacentral.net/pli-tv-kd8/en/brahmali\#9.1.1}{Kd~8:9.1.1}–9.4.4).

As robe-cloth accrued to monasteries, the cloth needed to be received, stored, and distributed. This required Sangha officials to perform these functions. An important new regulation allows for the appointment of Sangha officials through a legal procedure, a \textit{\textsanskrit{saṅghakamma}}, of one motion and one announcement (\href{https://suttacentral.net/pli-tv-kd8/en/brahmali\#5.1.1}{Kd~8:5.1.1}–6.2.12). This is important for at least two reasons. First, it shows that \textit{\textsanskrit{saṅghakamma}} is the primary tool of the Sangha for making all sorts of internal decisions. We shall see at Kd 21 that \textit{\textsanskrit{saṅghakamma}} is the appropriate mechanism whenever a decision is made that concerns the whole community. This contrasts with the modern tendency of decisions being made by abbots or sometimes groups of senior monastics. Second, it shows the importance of delegation in Sangha affairs. Instead of the whole Sangha being involved in minor decision making, any office or job can be delegated to individual monastics. This makes the running of a monastic community more efficient.

Apart from robes, many other cloth requisites are mentioned in this chapter. One of these is the \textit{\textsanskrit{nisīdana}}, “the sitting mat”, often rendered as “sitting cloth” (\href{https://suttacentral.net/pli-tv-kd8/en/brahmali\#16.1.1}{Kd~8:16.1.1}–16.4.3). The latter rendering, however, seems to be a result of the current practice of using the \textit{\textsanskrit{nisīdana}} indoors as opposed to outdoors, which is how we see it used in the Suttas. I discuss the \textit{\textsanskrit{nisīdana}} further in Appendix I: Technical Terms.

The first of the three significant stories is that of \textsanskrit{Jīvaka}, who becomes the Buddha’s personal physician (\href{https://suttacentral.net/pli-tv-kd8/en/brahmali\#1.4.1}{Kd~8:1.4.1}–1.34.14). \textsanskrit{Jīvaka} is the unwanted child of a high-class sex worker. Soon after he is born, she puts him in a basket and has him thrown on a trash heap, where he was found and then brought up by King \textsanskrit{Bimbisāra}’s son Prince Abhaya. Not being satisfied with the life of a royal, \textsanskrit{Jīvaka} secretly departs for \textsanskrit{Takkasilā} to become a physician. He studies for seven years. His final exam consists of traveling all around \textsanskrit{Takkasilā} to a distance of a \textit{yojana}—approximately 13 kilometers\footnote{See Appendix I: Technical Terms for an estimate of the length of the \textit{yojana}. }—and bringing back any plant that is not medicinal. \textsanskrit{Jīvaka} comes back empty handed, upon which he is declared fully educated. We have much to learn from the ancients in the art of conducting exams!

On his way back to \textsanskrit{Rājagaha}, \textsanskrit{Jīvaka} cures an apparently incurable wealthy lady and earns a fortune. Upon his return to \textsanskrit{Rājagaha}, he cures King \textsanskrit{Bimbisāra} of hemorrhoids. Next, he performs what may be the world’s first recorded brain surgery, during which he removes two insects from the brain of a wealthy merchant. Next up is the son of a wealthy merchant. \textsanskrit{Jīvaka} ties him to a pillar, cuts open his belly, and then unravels his twisted gut. The most daring of \textsanskrit{Jīvaka}’s medical adventures, however, was his cure of King Pajjota who was infamous for his hot temper. \textsanskrit{Jīvaka} treats him with a medicine that does not agree with him, upon which the king unleashes his fearsome temper. When the king discovers that \textsanskrit{Jīvaka} has already fled, he dispatches his best man to capture him, but \textsanskrit{Jīvaka} cleverly escapes. Soon King Pajjota realizes he is cured. He sends two exquisite cloths to \textsanskrit{Jīvaka} in gratitude, which \textsanskrit{Jīvaka} gives to the Buddha. The Buddha uses the occasion to allow monastics to accept robes from householders. As a side issue, it is noteworthy that the Buddha does not just accept the cloths, granting himself a special exemption from the rules. Finally, \textsanskrit{Jīvaka} becomes the Buddha’s doctor, and he also looks after the Sangha.

Where the story of \textsanskrit{Jīvaka} is entertaining, the story of \textsanskrit{Visākhā} is inspiring (\href{https://suttacentral.net/pli-tv-kd8/en/brahmali\#15.1.1}{Kd~8:15.1.1}–15.14.13). When \textsanskrit{Visākhā} invites the Sangha headed by the Buddha for a meal, she learns that the monks are bathing naked in the rain. After serving the meal, she asks the Buddha to grant her a favor. The Buddha initially refuses, but she persists. She tells him that she wishes to give rainy-season bathing cloths to the monks. She also wants to give meals to newly arrived and departing monastics, to sick monastics and those who look after the sick, and more. The Buddha asks why he should grant her such a special privilege. She responds that she will get so much joy and happiness from this that she will gain \textit{\textsanskrit{samādhi}} as a result. The Buddha is so impressed with her understanding of the Dhamma that he assents to what he normally would not.

In the final story, the Buddha himself is the focus (\href{https://suttacentral.net/pli-tv-kd8/en/brahmali\#26.1.1}{Kd~8:26.1.1}–26.4.14). While the Buddha is walking around the monastery, he comes across a monk who is suffering from dysentery, yet he has no nurse. The Buddha, together with Ānanda, cleans him up and lifts him onto a bed. The Buddha then admonishes the monks for not looking after the sick. He points out that there is no-one else to look after them. He then says that whoever would look after him, that is, the Buddha, should look after the sick.

It is easy to think of the Buddha as a rather distant and mystical figure who is above the fray. The reality, however, as we see here and elsewhere, is that the Buddha was very human in his interactions with the monks. By forgetting this side of the Buddha, we risk losing our connection to him. It is precisely because the Buddha was a human being that his teachings and example are so relevant. By noticing the little details in the Suttas and the Vinaya, we get a more realistic and down-to-earth appreciation of the Buddha as a historical personality.

This may be the right place to comment briefly on the occasional use of later vocabulary in the Khandhakas. In the story of \textsanskrit{Jīvaka} we find the word \textit{\textsanskrit{matthaluṅga}}, “brain”, which is not encountered in the four main \textsanskrit{Nikāyas} or the remainder of the Vinaya \textsanskrit{Piṭaka}. In fact, in the four main \textsanskrit{Nikāyas}, this word is conspicuously absent from the standard list of body parts used in \textit{asubha}, “ugliness”, contemplation, whereas it is included in the same list in later literature, such as the Khuddaka-\textsanskrit{pāṭha}. As a rule, the presence or prevalence of certain vocabulary is a good way to establish the relative age of Pali literature. Much good research is waiting to be done in this area.

\section*{The Chapter Connected with \textsanskrit{Campā}, Campeyya-kkhandhaka, Kd 9}

Kd 9 is about \textit{\textsanskrit{saṅghakamma}}, “the legal procedures of the Sangha”. We have seen how \textit{\textsanskrit{saṅghakamma}} is used for Sangha decision making to make the process democratic and transparent. For such decisions to have proper authority, however, \textit{\textsanskrit{saṅghakamma}} needs to be clearly defined. The main purpose of this chapter, then, is to set out what constitutes valid and invalid \textit{\textsanskrit{saṅghakamma}}. As such, it is quite technical in nature.

The chapter begins with the story of the monk Kassapagotta who is wrongly ejected from the Sangha by a group of visiting monks. This spurs the Buddha to lay down rules for the proper execution of \textit{\textsanskrit{saṅghakamma}}.

A \textit{\textsanskrit{saṅghakamma}} is valid only when:

\begin{enumerate}%
\item The assembly is complete (\href{https://suttacentral.net/pli-tv-kd9/en/brahmali\#3.6.2}{Kd~9:3.6.2}–3.6.3). This means that the \textit{\textsanskrit{saṅghakamma}} is valid only if:
\begin{itemize}%
\item All monks/nuns within the monastery zone (\textit{\textsanskrit{sīmā}}) are present at the meeting%
\item Anyone who is not present, for whatever reason, has sent their consent%
\item No-one present objects to the decision.%
\end{itemize}

%
\item The quorum requirement is met. Most \textit{\textsanskrit{saṅghakammas}} require a quorum of four monastics. Some important \textit{\textsanskrit{saṅghakammas}} require a quorum of five, ten, or twenty monastics (\href{https://suttacentral.net/pli-tv-kd9/en/brahmali\#4.1.1}{Kd~9:4.1.1}–4.1.10).%
\item The object is valid, which means that:
\begin{itemize}%
\item A \textit{\textsanskrit{saṅghakamma}} must be done against a maximum of three monastics at a time (\href{https://suttacentral.net/pli-tv-kd9/en/brahmali\#2.3.18}{Kd~9:2.3.18})%
\item If the object is a person, they must generally be present for the \textit{\textsanskrit{saṅghakamma}} to be valid (\href{https://suttacentral.net/pli-tv-kd14/en/brahmali\#1.1.18}{Kd~14:1.1.18}–1.1.20)%
\item The object toward which the \textit{\textsanskrit{saṅghakamma}} is directed must meet the requirements laid down in the Vinaya \textsanskrit{Piṭaka}. For instance, a man receiving ordination must be twenty years old. If he is less, the ordination is invalid (\href{https://suttacentral.net/pli-tv-bu-vb-pc65/en/brahmali\#1.53.1}{Bu~Pc~65:1.53.1}).%
\end{itemize}

%
\item The proclamation is performed correctly. This means that:
\begin{itemize}%
\item There must be one motion and either one or three announcements (\href{https://suttacentral.net/pli-tv-kd9/en/brahmali\#3.3.3}{Kd~9:3.3.3}–3.4.9)\footnote{That is, the \textit{\textsanskrit{saṅghakamma}} of one motion and one announcement, or the \textit{\textsanskrit{saṅghakamma}} of one motion and three announcements. For reasons I give immediately below, I have not included the \textit{\textsanskrit{saṅghakamma}} of one motion or the \textit{\textsanskrit{saṅghakamma}} that consist of getting permission. }%
\item The motion and the announcements must be in the right order (\href{https://suttacentral.net/pli-tv-kd9/en/brahmali\#3.9.2}{Kd~9:3.9.2}–3.9.3)%
\item The \textsanskrit{Parivāra} adds that the wording must include certain critical elements (\href{https://suttacentral.net/pli-tv-pvr21/en/brahmali\#3.1}{Pvr~21:3.1}–4.3).%
\end{itemize}

%
\end{enumerate}

I have not yet discussed the four kinds of \textit{\textsanskrit{saṅghakamma}} occasionally mentioned in the Vinaya \textsanskrit{Piṭaka}. Remarkably, the current chapter only mentions two of the four, that is, legal procedures consisting of one motion and one announcement and legal procedures consisting of one motion and three announcements. Kd 9 does not mention legal procedures consisting of one motion or legal procedures consisting of getting permission. Given that Kd 9 is the main exposition of \textit{\textsanskrit{saṅghakamma}} in the Vinaya \textsanskrit{Piṭaka}, how can this be?

It turns out that the two legal procedures not mentioned in Kd 9 are only rarely encountered anywhere in the Vinaya \textsanskrit{Piṭaka}. They are found in the word commentary to \href{https://suttacentral.net/pli-tv-bu-vb-pc79/en/brahmali\#2.1.6}{Bu~Pc~79}, and five times in the Samatha-kkhandhaka, “The chapter on the settling of legal issues” (\href{https://suttacentral.net/pli-tv-kd14/en/brahmali\#14.2.15}{Kd~14:14.2.15}, \href{https://suttacentral.net/pli-tv-kd14/en/brahmali\#14.11.5}{Kd~14:14.11.5}–14.11.13, and \href{https://suttacentral.net/pli-tv-kd14/en/brahmali\#14.15.4}{Kd~14:14.15.4}). All these instances can reasonably be considered late in the evolution of the \textsanskrit{Tipiṭaka}.\footnote{Word commentaries are often late, and certainly later than the rules they comment on. The section of Kd 14 that mentions these two legal procedures uses Abhidhamma terminology, which again suggests lateness. }

As we have seen above in the discussion to Kd 2, neither the \textit{uposatha} ceremony nor the \textit{\textsanskrit{pavāraṇā}} ceremony were regarded as \textit{\textsanskrit{saṅghakamma}} proper in the earliest period. This makes sense if \textit{\textsanskrit{saṅghakamma}} was restricted to those legal procedures that had either one or three announcements. Neither the \textit{uposatha} ceremony nor the \textit{\textsanskrit{pavāraṇā}} ceremony has this structure, being limited to the equivalent of a motion. At some point it was decided that both the \textit{uposatha} ceremony and the \textit{\textsanskrit{pavāraṇā}} ceremony were so similar to \textit{\textsanskrit{saṅghakamma}} that they needed to be included in this category. This necessitated the creation of legal procedures with a single motion. The legal procedure consisting of getting permission would have come into being in a similar way. We are left with the impression that both these latter procedures did not exist in the earliest period and are late additions to the Vinaya \textsanskrit{Piṭaka}.

This matters because it illuminates how \textit{\textsanskrit{saṅghakamma}} should ideally be performed in the modern era. The question arises of what form a \textit{\textsanskrit{saṅghakamma}} should take if the Sangha is making a decision for which there is no prescription in the Vinaya \textsanskrit{Piṭaka}. In Kd 21 we find an example of such a legal procedure that is external to the Vinaya (\href{https://suttacentral.net/pli-tv-kd21/en/brahmali\#1.4.1}{Kd~21:1.4.1}–1.4.14). Soon after the Buddha’s passing, the Sangha needed to appoint 500 monks to take part in the Council at \textsanskrit{Rājagaha}. This was done through a procedure consisting of one motion and one announcement. This, arguably, sets a precedent for how \textit{\textsanskrit{saṅghakamma}} should be performed in the absence of a prescribed formula. One motion and one announcement is the proper structure. A single motion or getting permission should not be used since they are unlikely to stem from the Buddha.

\section*{The Chapter Connected with \textsanskrit{Kosambī}, Kosambaka-kkhandhaka, Kd 10}

Kd 10, the last chapter of the \textsanskrit{Mahāvagga}, is concerned with disputes in the Sangha and their resolution. Disputes can potentially lead to schism, which the Buddha regarded as a very serious matter. But even if a dispute does not lead to schism, it can potentially have negative consequences. This is what the Buddha has to say at \href{https://suttacentral.net/mn104/en/sujato\#5.8}{MN~104:5.8}:

\begin{quotation}%
“Ānanda, a dispute about livelihood or the monastic code is a minor matter. But should a dispute arise in the Sangha concerning the path or the practice, that would be for the detriment, suffering, and harm of the people, for the detriment and suffering of gods and humans.”

%
\end{quotation}

Kd 10 begins with the well-known story of the dispute at \textsanskrit{Kosambī}. Part of this story is also told in MN 128 and in the Kosambiya \textsanskrit{Jātaka}, \href{https://suttacentral.net/ja428}{Ja~428}, and MN 48, the Kosambiya Sutta, is based on the same incident. In addition, many of the verses spoken by the Buddha in connection with it are found in the Dhammapada at \href{https://suttacentral.net/dhp3/en/brahmali}{Dhp 3}–6. One gets the impression that this event had a major impact on the Sangha.

The story begins with the Sangha disputing whether a certain monk has committed an offense. The Sangha decides to eject the monk concerned, but he still refuses to acknowledge any fault. Both sides of the conflict build up a group of supporters, causing the Sangha to split into factions. Eventually someone asks the Buddha to intervene, but to no avail. At this point the Buddha tells the story of \textsanskrit{Dīghāvu}, bits of which are found at \href{https://suttacentral.net/ja371}{Ja~371} and \href{https://suttacentral.net/ja428}{Ja~428}. After this long and beautiful tale on the power of forgiveness and gentleness, the Buddha asks his monks to act accordingly, but again they will not listen. At this point the Buddha realizes he can do no more. He recites a series of powerful verses in the midst of the Sangha, and then departs. These verses are among the most famous and beloved in Buddhism (\href{https://suttacentral.net/pli-tv-kd10/en/brahmali\#3.1.3}{Kd~10:3.1.3}–3.1.43).

The Buddha walks to the village \textsanskrit{Bālakaloṇaka} where he visits the monk Bhagu, before proceeding to the Eastern Bamboo Park where he meets the three monks Anuruddha, Nandiya, and Kimila. They are shining examples of how to live in harmony, and are clearly meant to provide an edifying contrast to the monks in \textsanskrit{Kosambī}.\footnote{The three monks are also met with at \href{https://suttacentral.net/mn128/en/sujato}{MN~128} and \href{https://suttacentral.net/mn31/en/sujato}{MN~31}. } The Buddha carries on to \textsanskrit{Pālileyyaka}, where he stays in solitude, only attended on by an elephant. This is another famous story, much told in the Buddhist world. Versions of it are found at \href{https://suttacentral.net/ud4.5/en/sujato}{Ud~4.5} and in the commentary to the Dhammapada. In the latter version a monkey offers honey to the Buddha, because of which he gets so excited that he falls out of the tree, dies, and is reborn straight in the heaven of the thirty-three. The Buddha eventually leaves and goes to \textsanskrit{Sāvatthī}. The entire story is narrated at \href{https://suttacentral.net/pli-tv-kd10/en/brahmali\#4.1.1}{Kd~10:4.1.1}–5.1.3.

In the meantime, after pressure from the lay people, the monks at \textsanskrit{Kosambī} have come to their senses. It is interesting to note in passing the potential power lay Buddhists have over monastics who misbehave. Given the number of scandals involving monastics in the Buddhist world, it is a power they probably should exercise more often.

In any case, the monks set out for \textsanskrit{Sāvatthī}. The monk who was at the center of the dispute has realized he actually did commit an offense. He asks to be readmitted, which he is. The Sangha then does a \textit{\textsanskrit{saṅghakamma}} to unify the community, a so-called \textit{\textsanskrit{sāmaggi}-uposatha}, followed by the recitation of the Monastic Code.

This long story forms the main content of this chapter. There are only a few mentions of rules and regulations. One of these is a list of eighteen grounds, all connected with the Dhamma and Vinaya, that form the basis of disputes. This ties this chapter to Kd 17, where schism in the Sangha is discussed in much more detail. I will return to this theme in the introduction to the Cullavagga in volume 5.

%
\chapter*{Abbreviations}
\addcontentsline{toc}{chapter}{Abbreviations}
\markboth{Abbreviations}{Abbreviations}

\begin{description}%
\item[AN] \textsanskrit{Aṅguttara} Nikāya (references are to Nipāta and \textit{sutta} numbers)%
\item[AN-a] \textsanskrit{Aṅguttara} Nikāya \textsanskrit{aṭṭhakathā}, the commentary on the \textsanskrit{Aṅguttara} Nikāya%
\item[As] \textit{\textsanskrit{adhikaraṇasamathadhamma}}%
\item[Ay] \textit{aniyata}%
\item[Bi] \textit{\textsanskrit{bhikkhunī}}%
\item[Bu] \textit{bhikkhu}%
\item[CPD] Critical Pali Dictionary%
\item[DN] \textsanskrit{Dīgha} \textsanskrit{Nikāya} (references are to \textit{sutta} numbers)%
\item[DN-a] \textsanskrit{Dīgha} \textsanskrit{Nikāya} \textsanskrit{aṭṭhakathā}, the commentary on the \textsanskrit{Dīgha} \textsanskrit{Nikāya}%
\item[DOP] Dictionary of Pali%
\item[f, ff] and the following page, pages%
\item[Iti] Itivuttaka (references are to verse numbers)%
\item[Ja] \textsanskrit{Jātaka} and \textsanskrit{Jātaka} \textsanskrit{aṭṭhakathā}%
\item[Kd] Khandhaka%
\item[Khuddas-\textsanskrit{pṭ}] \textsanskrit{Khuddasikkhā}-\textsanskrit{purāṇaṭīkā} (references are to paragraph numbers)%
\item[Khuddas-\textsanskrit{nṭ}] \textsanskrit{Khuddasikkhā}-\textsanskrit{abhinavaṭīkā} (references are to paragraph numbers)%
\item[Kkh] \textsanskrit{Kaṅkha}̄\textsanskrit{vitaraṇi}̄%
\item[Kkh-\textsanskrit{pṭ}] \textsanskrit{Kaṅkhāvitaraṇīpurāṇa}-\textsanskrit{ṭīkā}%
\item[MN] Majjhima \textsanskrit{Nikāya} (references are to \textit{sutta} numbers)%
\item[MN-a] Majjhima \textsanskrit{Nikāya} \textsanskrit{aṭṭhakathā}, the commentary on the Majjhima \textsanskrit{Nikāya}%
\item[MS] \textsanskrit{Mahāsaṅgīti} \textsanskrit{Tipiṭaka} (the version of the \textsanskrit{Tipiṭaka} found on SuttaCentral)%
\item[N\&E] “Nature and the Environment in Early Buddhism”, Bhante Dhammika%
\item[Nidd-a] \textsanskrit{Mahāniddesa} \textsanskrit{aṭṭhakathā} (references are to VRI edition paragraph numbers)%
\item[NP] \textit{nissaggiya \textsanskrit{pācittiya}}%
\item[p., pp.] page, pages%
\item[Pc] \textit{\textsanskrit{pācittiya}}%
\item[Pd] \textit{\textsanskrit{pāṭidesanīya}}%
\item[PED] Pali English Dictionary%
\item[Pj] \textit{\textsanskrit{pārājika}}%
\item[PTS] Pali Text Society%
\item[Pvr] \textsanskrit{Parivāra}%
\item[SAF] “South Asian Flora as reflected in the twelfth-century Pali lexicon \textsanskrit{Abhidhānapadīpikā}”, J. Liyanaratne%
\item[SED] Sanskrit English Dictionary%
\item[Sk] \textit{sekhiya}%
\item[SN] \textsanskrit{Saṁyutta} \textsanskrit{Nikāya} (references are to \textsanskrit{Saṁyutta} and \textit{sutta} numbers)%
\item[SN-a] \textsanskrit{Saṁyutta} \textsanskrit{Nikāya} \textsanskrit{aṭṭhakathā}, the commentary on the \textsanskrit{Saṁyutta} \textsanskrit{Nikāya} (references are to volume number and paragraph numbers of the VRI version)%
\item[Sp] Samantapāsādikā, the commentary on the Vinaya \textsanskrit{Piṭaka} (references are to volume and paragraph numbers of the VRI version)%
\item[Sp‑ṭ] Sāratthadīpanī-ṭīkā (references follow the division into five volumes of the Canonical text and then add the paragraph number of the VRI version of the sub-commentary)%
\item[Sp‑yoj] \textsanskrit{Pācityādiyojanā} (volume numbers match those of Sp of the online VRI version, which, given that Sp‑yoj starts with the \textit{bhikkhu \textsanskrit{pācittiyas}}, means that Sp‑yoj is divided into four volumes, starting at volume 2; paragraph numbers are those of the VRI version)%
\item[SRT] Siamrath \textsanskrit{Tipiṭaka}, official edition of the \textsanskrit{Tipiṭaka} published in Thailand%
\item[Ss] \textit{\textsanskrit{saṅghādisesa}}%
\item[sv.] \textit{sub voce}, see under%
\item[\textsanskrit{Thīg}] \textsanskrit{Therīgāthā}%
\item[Ud-a] \textsanskrit{Udāna} \textsanskrit{aṭṭhakathā}, the commentary on the \textsanskrit{Udāna} (references are to \textit{sutta} number)%
\item[Vb] \textsanskrit{Vibhaṅga}, the second book of the Abhidhamma \textsanskrit{Piṭaka}%
\item[Vin-\textsanskrit{ālaṅ}-\textsanskrit{ṭ}] \textsanskrit{Vinayālaṅkāra}-\textsanskrit{ṭīkā} (references are to chapter number and paragraph numbers of the VRI version)%
\item[Vin-vn-\textsanskrit{ṭ}] \textsanskrit{Vinayavinicchayaṭīkā} (references are to paragraph numbers of the VRI version)%
\item[Vjb] \textsanskrit{Vajirabuddhiṭīkā} (references are to volume and paragraph numbers of the VRI version)%
\item[Vmv] \textsanskrit{Vimativinodanī}-\textsanskrit{ṭīkā} (references are to volume and paragraph numbers of the VRI version)%
\item[VRI] Vipassana Research Institute, the publisher of the online version of the Sixth Council edition of the Pali Canon at https://www.tipitaka.org%
\item[Vv-a] \textsanskrit{Vimānavatthu} \textsanskrit{aṭṭhakathā}, the commentary on the \textsanskrit{Vimānavatthu} (references are to paragraph numbers of the VRI edition).%
\end{description}

%
\mainmatter%
\pagestyle{fancy}%
\addtocontents{toc}{\let\protect\contentsline\protect\nopagecontentsline}
\part*{The Great Division}
\addcontentsline{toc}{part}{The Great Division}
\markboth{}{}
\addtocontents{toc}{\let\protect\contentsline\protect\oldcontentsline}

%
\chapter*{{\suttatitleacronym Kd 1}{\suttatitletranslation The great chapter }{\suttatitleroot Mahākhandhaka}}
\addcontentsline{toc}{chapter}{\tocacronym{Kd 1} \toctranslation{The great chapter } \tocroot{Mahākhandhaka}}
\markboth{The great chapter }{Mahākhandhaka}
\extramarks{Kd 1}{Kd 1}

\section*{1. The account with the Bodhi tree }

\scnamo{Homage to the Buddha, the Perfected One, the fully Awakened One }

Soon\marginnote{1.1.1} after his awakening, the Buddha was staying at \textsanskrit{Uruvelā} on the bank of the river \textsanskrit{Nerañjara} at the foot of a Bodhi tree. There the Buddha sat cross-legged for seven days without moving, experiencing the bliss of freedom. Then, in the first part of the night, the Buddha reflected on dependent origination in forward and reverse order: 

“Ignorance\marginnote{1.2.2} is the condition for intentional activities; intentional activities are the condition for consciousness; consciousness is the condition for name and form; name and form are the condition for the six sense spheres; the six sense spheres are the condition for contact; contact is the condition for feeling; feeling is the condition for craving; craving is the condition for grasping; grasping is the condition for existence; existence is the condition for birth; birth is the condition for old age and death, for grief, sorrow, pain, aversion, and distress to come to be. This is how there is the origin of this whole mass of suffering. 

But\marginnote{1.2.4} with the complete fading away and end of ignorance comes the end of intentional activities; with the end of intentional activities comes the end of consciousness; with the end of consciousness comes the end of name and form; with the end of name and form comes the end of the six sense spheres; with the end of the six sense spheres comes the end of contact; with the end of contact comes the end of feeling; with the end of feeling comes the end of craving; with the end of craving comes the end of grasping; with the end of grasping comes the end of existence; with the end of existence comes the end of birth; with the end of birth comes the end of old age and death, and the end of sorrow, lamentation, pain, aversion, and distress. This is how there is the end of this whole mass of suffering.” 

Seeing\marginnote{1.3.1} the significance of this, the Buddha uttered a heartfelt exclamation: 

\begin{verse}%
“When\marginnote{1.3.2} things become clear \\
To the energetic brahmin who practices absorption, \\
Then all his doubts are dispelled, \\
Since he understands the natural order and its conditions.” 

%
\end{verse}

In\marginnote{1.4.1} the middle part of the night, the Buddha again reflected on dependent origination in forward and reverse order: 

“Ignorance\marginnote{1.4.2} is the condition for intentional activities; intentional activities are the condition for consciousness; consciousness is the condition for name and form … This is how there is the origin of this whole mass of suffering. … This is how there is the end of this whole mass of suffering.” 

Seeing\marginnote{1.5.1} the significance of this, the Buddha uttered a heartfelt exclamation: 

\begin{verse}%
“When\marginnote{1.5.2} things become clear \\
To the energetic brahmin who practices absorption, \\
Then all his doubts are dispelled, \\
Since he’s understood the end of the conditions.” 

%
\end{verse}

In\marginnote{1.6.1} the last part of the night, the Buddha again reflected on dependent origination in forward and reverse order: 

“Ignorance\marginnote{1.6.2} is the condition for intentional activities; intentional activities are the condition for consciousness; consciousness is the condition for name and form … This is how there is the origin of this whole mass of suffering. … This is how there is the end of this whole mass of suffering.” 

Seeing\marginnote{1.7.1} the significance of this, the Buddha uttered a heartfelt exclamation: 

\begin{verse}%
“When\marginnote{1.7.2} things become clear \\
To the energetic brahmin who practices absorption, \\
He defeats the army of the Lord of Death, \\
Like the sun beaming in the sky.” 

%
\end{verse}

\scend{The account with the Bodhi tree is finished. }

\section*{2. The account with the goatherd’s banyan tree }

After\marginnote{2.1.1} seven days, the Buddha came out from that stillness and went from the Bodhi tree to a goatherd’s banyan tree. There too he sat cross-legged for seven days without moving, experiencing the bliss of freedom. 

Then\marginnote{2.2.1} a brahmin devoted to mystical mantras went up to the Buddha,\footnote{According to Bhikkhu Sujato’s notes to the parallel verse at \href{https://suttacentral.net/ud1.4/en/brahmali\#4.1}{Ud 1.4}, \textit{\textsanskrit{huṁhuṅkajātika} \textsanskrit{brāhmaṇa}} refers to a brahmin who utters the syllable \textit{\textsanskrit{huṁ}}. In the \textsanskrit{Chāndogya} \textsanskrit{Upaniṣa}, this syllable, like the syllable \textit{om}, had a ritualistic purpose with mystical connotations. In the verse below, “the brahmin … who does not murmur mystical mantras”, \textit{\textsanskrit{brāhmaṇa} … \textsanskrit{nihuṁhuṅka}}, is intended to capture the idea that a true brahmin, a perfected individual, does not utter such syllables. See https://discourse.suttacentral.net/t/on-the-brahmin-who-said-hu. } exchanged pleasantries with him, and said, “Good Gotama, how is one a brahmin? What are the qualities that make one a brahmin?” 

Seeing\marginnote{2.3.1} the significance of this, the Buddha uttered a heartfelt exclamation: 

\begin{verse}%
“The\marginnote{2.3.2} brahmin who has shut out bad qualities, \\
Who does not murmur mystical mantras, but is free from flaws and self-controlled, \\
Who has reached final knowledge and has fulfilled the spiritual life—\\
He may rightly proclaim the highest doctrine,\footnote{Bhikkhu Sujato notes that the phrase \textit{\textsanskrit{brahmavāda}} only occurs at \href{https://suttacentral.net/ud1.4/en/brahmali\#4.4}{Ud 1.4} and here in the entire Pali Canon. It does occur, however, in pre-Buddhist Vedic texts in the meaning “orthodox Vedic doctrine” (Atharva Veda 11.3.26a, 15.1.8a; \textsanskrit{Bṛhadāraṇyaka} \textsanskrit{Upaniṣad} 14.7.3.1; \textsanskrit{Chāndogya} \textsanskrit{Upaniṣad} 2.24.1). To accord with the Buddhist teachings, the Buddha normally reinterprets the word \textit{brahma} as “the highest”, and I expect this is the case also here. } \\
Having no pride about anything in the world.” 

%
\end{verse}

\scend{The account with the goatherd’s banyan tree is finished. }

\section*{3. The account with the powderpuff tree }

After\marginnote{3.1.1} seven days, the Buddha came out from that stillness and went from the goatherd’s banyan tree to a powderpuff tree. There too he sat cross-legged for seven days without moving, experiencing the bliss of freedom.\footnote{According to SAF, pp. 85-86, \textit{mucalinda} is a \textit{Barringtonia racemosa}, sometimes known as a “powder-puff tree”. } 

Just\marginnote{3.2.1} then an unseasonal storm was approaching, bringing seven days of rain, cold winds, and clouds. Mucalinda, the dragon king, came out from his abode. He encircled the body of the Buddha with seven coils and spread his large hood over his head, thinking, “May the Buddha not be hot or cold, nor be bothered by horseflies or mosquitoes, by the wind or the burning sun, or by creeping animals or insects.” 

After\marginnote{3.3.1} seven days, when he knew the sky was clear, Mucalinda unraveled his coils from the Buddha’s body and transformed himself into a young brahmin. He then stood in front of the Buddha, raising his joined palms in veneration. 

Seeing\marginnote{3.3.2} the significance of this, the Buddha uttered a heartfelt exclamation: 

\begin{verse}%
“Seclusion\marginnote{3.4.1} is bliss for the contented \\
Who sees the Teaching that they have learned. \\
Kindness to the world is happiness, \\
For one who’s harmless to living beings. 

Dispassion\marginnote{3.4.5} for the world is happiness, \\
For one who overcomes worldly pleasures. \\
But removing the conceit ‘I am’, \\
This, indeed, is the highest bliss.” 

%
\end{verse}

\scend{The account with the powderpuff tree is finished. }

\section*{4. The account with the ape-flower tree }

After\marginnote{4.1.1} seven days, the Buddha came out from that stillness and went from the powderpuff tree to an ape-flower tree. There too he sat cross-legged for seven days without moving, experiencing the bliss of freedom.\footnote{According to SAF, p. 72, \textit{\textsanskrit{rājāyatana}} is a \textit{Buchanania axillaris}, sometimes known as an “ape-flower tree”. } 

Just\marginnote{4.2.1} then the merchants Tapussa and Bhallika were traveling from \textsanskrit{Ukkalā} to that area. Then a god who was a former relative of theirs said to them, “Sirs, a Buddha who has just attained awakening is staying at the foot of an ape-flower tree. Go to that Buddha and offer him crackers and honey.\footnote{“Crackers” renders \textit{mantha}. See discussion in Appendix of Technical Terms. } That will be for your benefit and happiness for a long time.” 

And\marginnote{4.3.1} they took crackers and honey and went to the Buddha. They bowed down and said, “Sir, please accept the crackers and honey from us. That will be for our benefit and happiness for a long time.” 

The\marginnote{4.4.1} Buddha thought, “Buddhas don’t receive with their hands. In what should I receive the crackers and honey?” 

Then,\marginnote{4.4.4} reading the mind of the Buddha, the four great kings offered him four crystal bowls from the four directions, saying, “Here, sir, please receive the crackers and honey in these.” After receiving the crackers and honey in one of the valuable crystal bowls, the Buddha ate them. 

When\marginnote{4.5.1} Tapussa and Bhallika knew that the Buddha had finished his meal, they bowed down with their heads at his feet, and said, “Sir, we go for refuge to the Buddha and the Teaching. Please accept us as lay followers who have gone for refuge for life.” By means of the double refuge, they became the first lay followers in the world. 

\scend{The account with the ape-flower tree is finished. }

\section*{5. The account of the supreme being’s request }

After\marginnote{5.1.1} seven days, the Buddha came out from that stillness and went from the ape-flower tree to a goatherd’s banyan tree, and he stayed there. Then, while reflecting in private, the Buddha thought this: 

“I\marginnote{5.2.2} have discovered this profound truth, so hard to see, so hard to comprehend. It’s peaceful and sublime, subtle, beyond the intellect, and knowable only to the wise. But human beings delight in holding on, find pleasure in holding on, rejoice in holding on, and because of that it’s hard for them to see causal relationships, dependent origination. This too is very hard for them to see: the stilling of all intentional activities, the giving up of all ownership, the stopping of craving, fading away, ending, extinguishment. If I were to teach this truth, others would not understand, and that would be wearying and troublesome for me.” 

And\marginnote{5.3.1} spontaneously, these verses never heard before occurred to the Buddha: 

\begin{verse}%
“What\marginnote{5.3.2} I’ve discovered with difficulty, \\
There’s no point in making it known. \\
For those overcome by sensual desire and ill will, \\
This truth is hard to understand. 

Those\marginnote{5.3.6} who are excited by sensual desire, \\
Obstructed by a mass of darkness, \\
Won’t see what goes against the stream, \\
What’s subtle and refined, profound and hard to see.” 

%
\end{verse}

When\marginnote{5.4.1} the Buddha reflected like this, he inclined to inactivity, not to teaching. 

Just\marginnote{5.4.2} then the supreme being Sahampati read the mind of the Buddha. He thought, “The world is lost; it’s perished!—for the Buddha, perfected and fully awakened, inclines to inaction, not to teaching.” 

Then,\marginnote{5.5.1} just as a strong man might bend or stretch his arm, Sahampati disappeared from the world of supreme beings and appeared in front of the Buddha. He arranged his upper robe over one shoulder, placed his right knee on the ground, raised his joined palms, and said, “Please teach, sir, please teach! There are beings with little dust in their eyes who are ruined because of not hearing the Teaching. There will be those who understand.” 

This\marginnote{5.7.1} is what Sahampati said, and he added: 

\begin{verse}%
“Earlier,\marginnote{5.7.2} among the Magadhans, \\
An impure teaching appeared, conceived by defiled people. \\
Open this door to the freedom from death! \\
Let them hear the Teaching, discovered by the Pure One. 

Just\marginnote{5.7.6} as one standing on a rocky mountain top \\
Would see the people all around, \\
Just so, All-seeing Wise One, \\
Ascend the temple of the Truth. \\
Being rid of sorrow, look upon the people, \\
Sunk in grief, overcome by birth and old age. 

Stand\marginnote{5.7.12} up, Victorious Hero! \\
Leader of travelers, wander the world without obligation. \\
Sir, proclaim the Teaching; \\
There will be those who understand.” 

%
\end{verse}

Twice\marginnote{5.8.1} the Buddha repeated to Sahampati what he had thought, and on both occasions Sahampati repeated his request. 

The\marginnote{5.10.1} Buddha understood the request of that supreme being. Then, with the eye of a Buddha, he surveyed the world out of compassion for sentient beings. He saw beings with little dust in their eyes and with much dust in their eyes, with sharp faculties and with dull faculties, with good qualities and with bad qualities, easy to teach and difficult to teach. He even saw some who regarded the next world as dangerous and to be avoided, while others did not.\footnote{I follow the reading in the PTS edition of the Pali, which omits the phrase \textit{appekacce na \textsanskrit{paralokavajjabhayadassāvine} viharante}. } It was just like blue, red, and white lotuses, sprouted and grown in a lotus pond: some remain submerged in the water without rising out of it, others reach the surface of the water, while others still rise out of the water without being touched by it. When he had seen this, the Buddha replied to Sahampati in verse: 

\begin{verse}%
“The\marginnote{5.12.2} doors to the freedom from death are open! \\
May those who hear release their faith. \\
Seeing trouble, supreme being, \\
I did not speak the sublime and subtle Truth.” 

%
\end{verse}

Sahampati\marginnote{5.13.1} thought, “The Buddha has consented to teach.” He bowed down, circumambulated the Buddha with his right side toward him, and disappeared right there. 

\scend{The account of the supreme being’s request is finished. }

\section*{6. The account of the group of five }

The\marginnote{6.1.1} Buddha thought, “Who should I teach first? Who will understand this Teaching quickly?” And it occurred to him, “\textsanskrit{Ālāra} \textsanskrit{Kālāma} is wise and competent, and has for a long time had little dust in his eyes. Let me teach him first. He will understand it quickly.” 

But\marginnote{6.2.1} an invisible god informed the Buddha, “Sir, \textsanskrit{Ālāra} \textsanskrit{Kālāma} died seven days ago,” and the Buddha also knew this for himself. He thought, “\textsanskrit{Ālāra} \textsanskrit{Kālāma}’s loss is great. If he had heard this Teaching, he would have understood it quickly.” 

Again\marginnote{6.3.1} the Buddha thought, “Who should I teach first? Who will understand this Teaching quickly?” And it occurred to him, “Udaka \textsanskrit{Rāmaputta} is wise and competent, and has for a long time had little dust in his eyes. Let me teach him first. He will understand it quickly.” 

But\marginnote{6.4.1} an invisible god informed the Buddha, “Sir, Udaka \textsanskrit{Rāmaputta} died last night,” and the Buddha also knew this for himself. He thought, “Udaka \textsanskrit{Rāmaputta}’s loss is great. If he had heard this Teaching, he would have understood it quickly.” 

Once\marginnote{6.5.1} again the Buddha thought, “Who should I teach first? Who will understand this Teaching quickly?” And it occurred to him, “The group of five monks who supported me while I was striving were of great service to me. Let me teach them first. But where are they staying now?” 

With\marginnote{6.6.4} his superhuman and purified clairvoyance, the Buddha saw that the group of five monks were staying near Benares, in the deer park at Isipatana. Then, after staying at \textsanskrit{Uruvelā} for as long as he liked, he set out wandering toward Benares. 

The\marginnote{6.7.1} \textsanskrit{Ājīvaka} ascetic Upaka saw the Buddha traveling between \textsanskrit{Gayā} and the place of awakening. He said to the Buddha, “Sir, your senses are clear and your skin is pure and bright. In whose name have you gone forth? Who is your teacher or whose teaching do you follow?” 

The\marginnote{6.8.1} Buddha replied to Upaka in verse: 

\begin{verse}%
“I’m\marginnote{6.8.2} the victor, the knower of all. \\
Abandoning all, I’m not soiled by anything. \\
Through my own insight, I’m freed by the ending of craving—\\
So who should I refer to as a teacher? 

I\marginnote{6.8.6} have no teacher; \\
No-one like me exists. \\
In the world with its gods, \\
I have no equal. 

For\marginnote{6.8.10} I’m the Perfected One, \\
The supreme teacher. \\
I alone am fully awakened; \\
I’m cool and extinguished. 

I’m\marginnote{6.8.14} going to the city of \textsanskrit{Kāsi}, \\
To set rolling the wheel of the Teaching. \\
In this world immersed in darkness, \\
I’ll beat the drum of freedom from death.” 

%
\end{verse}

“According\marginnote{6.9.1} to your own claim you must be a universal Victor.” 

\begin{verse}%
“Indeed,\marginnote{6.9.2} those like me are victors, \\
Those who have ended the corruptions. \\
I have conquered all bad traits—\\
Therefore, Upaka, I’m a Victor.” 

%
\end{verse}

Saying,\marginnote{6.9.6} “May it be so,” Upaka shook his head, chose the wrong path, and left. 

The\marginnote{6.10.1} Buddha continued wandering toward the deer park at Isipatana near Benares. When he eventually arrived, he went to the group of five monks. 

Seeing\marginnote{6.10.2} him coming, the group of five made an agreement with one another: “Here comes the ascetic Gotama, who has given up his striving and returned to a life of indulgence. We shouldn’t bow down to him, stand up for him, or receive his bowl and robe, but we should prepare a seat. If he wishes, he may sit down.” But as the Buddha approached, the group of five monks was unable to keep the agreement. One went to meet him to receive his bowl and robe, another prepared a seat, another set out water for washing the feet, yet another set out a foot stool, and the last one put out a foot scraper.\footnote{Sp-yoj 2.694: \textit{\textsanskrit{Pādassa} \textsanskrit{ṭhapanakaṁ} \textsanskrit{pīṭhaṁ} \textsanskrit{pādapīṭhaṁ}}, “A \textit{\textsanskrit{pādapīṭha}} is a bench for placing the feet.” Vmv 2.112: \textit{\textsanskrit{Pādakathalikanti} \textsanskrit{adhotapādaṁ} \textsanskrit{yasmiṁ} \textsanskrit{ghaṁsantā} dhovanti, \textsanskrit{taṁ} \textsanskrit{dāruphalakādi}}, “\textit{\textsanskrit{Pādakathalika}} means the wooden plank, etc., with which they wash the dirty feet by rubbing.” } The Buddha sat down on the prepared seat and washed his feet. But they still addressed him by name and as “friend”. 

The\marginnote{6.12.1} Buddha said to the group of five monks, “Monks, don’t address the Buddha by name or as ‘friend’. Listen, I’m perfected and fully awakened. I have discovered the freedom from death. I will instruct you and teach you the Truth. When you practice as instructed, in this very life you will soon realize with your own insight the supreme goal of the spiritual life for which gentlemen rightly go forth into homelessness.” 

They\marginnote{6.13.1} replied, “Friend Gotama, by practicing extreme austerities you didn’t gain any superhuman quality, any distinction in knowledge and vision worthy of noble ones. Since you have given up your striving and returned to a life of indulgence, how could you now have achieved any of this?” 

The\marginnote{6.14.1} Buddha said, “I haven’t given up striving and returned to a life of indulgence,” and he repeated what he had said before. 

A\marginnote{6.15.1} second time the group of five monks repeated their question and a second time the Buddha repeated his reply. A third time they repeated their question, and the Buddha then said, “Have you ever heard me speak like this?” 

“No,\marginnote{6.16.3} sir.” 

“Then\marginnote{6.16.4} listen. I’m perfected and fully awakened. I have discovered the freedom from death. I will instruct you and teach you the Truth. When you practice as instructed, in this very life you will soon realize with your own insight the supreme goal of the spiritual life for which gentlemen rightly go forth into homelessness.” 

The\marginnote{6.16.8} Buddha was able to persuade the group of five monks. They then listened to the Buddha, paid careful attention, and applied their minds to understand. 

And\marginnote{6.17.1} the Buddha addressed them: 

“There\marginnote{6.17.2} are these two opposites that should not be pursued by one who has gone forth. One is the devotion to worldly pleasures, which is inferior, crude, common, ignoble, and unbeneficial. The other is the devotion to self-torment, which is painful, ignoble, and unbeneficial. By avoiding these opposites, I have awakened to the middle path, which produces vision and knowledge, which leads to peace, insight, awakening, and extinguishment. 

And\marginnote{6.18.1} what, monks, is that middle path? It’s just this noble eightfold path, that is, right view, right aim, right speech, right action, right livelihood, right effort, right mindfulness, and right stillness. 

And\marginnote{6.19.1} this is noble truth of suffering: birth is suffering, old age is suffering, sickness is suffering, death is suffering, association with what is disliked is suffering, separation from what is liked is suffering, not getting what you want is suffering. In brief, the five aspects of existence affected by grasping are suffering. 

And\marginnote{6.20.1} this is noble truth of the origin of suffering: the craving that leads to rebirth, that comes with delight and sensual desire, ever delighting in this and that, that is, craving for worldly pleasures, craving for existence, and craving for non-existence. 

And\marginnote{6.21.1} this is noble truth of the end of suffering: the full fading away and ending of that very craving; giving it up, relinquishing it, releasing it, letting it go. 

And\marginnote{6.22.1} this is noble truth of the path leading to the end of suffering: just this noble eightfold path, that is, right view, right aim, right speech, right action, right livelihood, right effort, right mindfulness, and right stillness. 

I\marginnote{6.23.1} knew that this is the noble truth of suffering. Vision, knowledge, wisdom, understanding, and light arose in me regarding things I had never heard before. I knew that this noble truth of suffering should be fully understood. Vision, knowledge, wisdom, understanding, and light arose in me regarding things I had never heard before. I knew that this noble truth of suffering had been fully understood. Vision, knowledge, wisdom, understanding, and light arose in me regarding things I had never heard before. 

I\marginnote{6.24.1} knew that this is the noble truth of the origin of suffering. Vision, knowledge, wisdom, understanding, and light arose in me regarding things I had never heard before. I knew that this noble truth of the origin of suffering should be fully abandoned. Vision, knowledge, wisdom, understanding, and light arose in me regarding things I had never heard before. I knew that this noble truth of the origin of suffering had been fully abandoned. Vision, knowledge, wisdom, understanding, and light arose in me regarding things I had never heard before. 

I\marginnote{6.25.1} knew that this is the noble truth of the end of suffering. Vision, knowledge, wisdom, understanding, and light arose in me regarding things I had never heard before. I knew that this noble truth of the end of suffering should be fully experienced. Vision, knowledge, wisdom, understanding, and light arose in me regarding things I had never heard before. I knew that this noble truth of the end of suffering had been fully experienced. Vision, knowledge, wisdom, understanding, and light arose in me regarding things I had never heard before. 

I\marginnote{6.26.1} knew that this is the noble truth of the path leading to the end of suffering. Vision, knowledge, wisdom, understanding, and light arose in me regarding things I had never heard before. I knew that this noble truth of the path leading to the end of suffering should be fully developed. Vision, knowledge, wisdom, understanding, and light arose in me regarding things I had never heard before. I knew that this noble truth of the path leading to the end of suffering had been fully developed. Vision, knowledge, wisdom, understanding, and light arose in me regarding things I had never heard before. 

So\marginnote{6.27.1} long as I had not fully purified my knowledge and vision according to reality of these four noble truths with their three stages and twelve characteristics, I didn’t claim the supreme full awakening in this world with its gods, lords of death, and supreme beings, in this society with its monastics and brahmins, its gods and humans. 

But\marginnote{6.28.1} when I had fully purified my knowledge and vision according to reality of these four noble truths with their three stages and twelve characteristics, then I did claim the supreme full awakening in this world with its gods, lords of death, and supreme beings, in this society with its monastics and brahmins, its gods and humans. And knowledge and vision arose in me: ‘My freedom is unshakable, this is my last birth, now there is no further rebirth.’” 

This\marginnote{6.29.3} is what the Buddha said. The monks from the group of five were pleased and they rejoiced in the Buddha’s exposition. 

And\marginnote{6.29.4} while this exposition was being spoken, Venerable \textsanskrit{Koṇḍañña} experienced the stainless vision of the Truth: “Anything that has a beginning has an end.” 

When\marginnote{6.30.1} the Buddha had set rolling the wheel of the Teaching, the earth gods cried out, “At Benares, in the deer park at Isipatana, the Buddha has set rolling the supreme wheel of the Teaching. It can’t be stopped by any monastic, brahmin, god, lord of death, supreme being, or anyone in the world.” Hearing the earth gods, the gods of the four great kings cried out … Hearing the gods of the four great kings, the gods of the Thirty-three cried out … the \textsanskrit{Yāma} gods … the contented gods … the gods who delight in creation … the gods who control the creations of others … the gods of the realm of the supreme beings cried out, “At Benares, in the deer park at Isipatana, the Buddha has set rolling the supreme wheel of the Teaching. It can’t be stopped by any monastic, brahmin, god, lord of death, supreme being, or anyone in the world.” 

In\marginnote{6.31.1} that instant the news spread as far as the world of the supreme beings. Ten thousand solar systems shook and trembled. And there appeared in the world an immeasurable and glorious radiance, surpassing the splendor of the gods. 

Then\marginnote{6.31.4} the Buddha uttered a heartfelt exclamation: “\textsanskrit{Koṇḍañña} has understood! Indeed, \textsanskrit{Koṇḍañña} has understood!” That’s how \textsanskrit{Koṇḍañña} got the name “\textsanskrit{Aññāsikoṇḍañña}”, “\textsanskrit{Koṇḍañña} who has understood.” 

\textsanskrit{Aññāsikoṇḍañña}\marginnote{6.32.1} had seen the Truth, had reached, understood, and penetrated it. He had gone beyond doubt and uncertainty, had attained to confidence, and had become independent of others in the Teacher’s instruction. He then said to the Buddha, “Sir, I wish to receive the going forth in your presence. I wish to receive the full ordination.” The Buddha replied, “Come, monk. The Teaching is well-proclaimed. Practice the spiritual life to make a complete end of suffering.” That was the full ordination of that venerable. 

The\marginnote{6.33.1} Buddha then instructed and taught the rest of the monks. While they were being instructed and taught, Venerable Vappa and Venerable Bhaddiya experienced the stainless vision of the Truth: “Anything that has a beginning has an end.” They had seen the Truth, had reached, understood, and penetrated it. They had gone beyond doubt and uncertainty, had attained to confidence, and had become independent of others in the Teacher’s instruction. They then said to the Buddha, “Sir, we wish to receive the going forth in your presence. We wish to receive the full ordination.” The Buddha replied, “Come, monks. The Teaching is well-proclaimed. Practice the spiritual life to make a complete end of suffering.” That was the full ordination of those venerables. 

Living\marginnote{6.35.1} on the food brought to him, the Buddha then instructed and taught the remaining monks. The six of them lived on the almsfood brought by three. While they were being instructed and taught, Venerable \textsanskrit{Mahānāma} and Venerable Assaji experienced the stainless vision of the Truth: “Anything that has a beginning has an end.” They had seen the Truth, had reached, understood, and penetrated it; they had gone beyond doubt and uncertainty, had attained to confidence, and had become independent of others in the Teacher’s instruction. They then said to the Buddha, “Sir, we wish to receive the going forth in your presence. We wish to receive the full ordination.” The Buddha replied, “Come, monks. The Teaching is well-proclaimed. Practice the spiritual life to make a complete end of suffering.” That was the full ordination of those venerables. 

Then\marginnote{6.38.1} the Buddha addressed the group of five: 

“Form\marginnote{6.38.2} is not your essence. For if form were your essence, it would not lead to suffering, and you could make it be like this and not be like that. But because form is not your essence, it leads to suffering, and you can’t make it be like this and not be like that. 

Feeling\marginnote{6.39.1} is not your essence. For if feeling were your essence, it would not lead to suffering, and you could make it be like this and not be like that. But because feeling is not your essence, it leads to suffering, and you can’t make it be like this and not be like that. 

Perception\marginnote{6.40.1} is not your essence. For if perception were your essence, it would not lead to suffering, and you could make it be like this and not be like that. But because perception is not your essence, it leads to suffering, and you can’t make it be like this and not be like that. 

Intentional\marginnote{6.40.6} activities are not your essence. For if intentional activities were your essence, they would not lead to suffering, and you could make them be like this and not be like that. But because intentional activities are not your essence, they lead to suffering, and you can’t make them be like this and not be like that. 

Consciousness\marginnote{6.41.1} is not your essence. For if consciousness were your essence, it would not lead to suffering, and you could make it be like this and not be like that. But because consciousness is not your essence, it leads to suffering, and you can’t make it be like this and not be like that. 

What\marginnote{6.42.1} do you think, monks: is form permanent or impermanent?”—“Impermanent, sir.”—“Is what is impermanent suffering or happiness?”—“Suffering.”—“And that which is impermanent, suffering, and changeable by nature, is it proper to regard it like this: ‘This is mine, I am this, this is my essence?’”—“Definitely not.” 

“What\marginnote{6.43.1} do you think: is feeling permanent or impermanent?”—“Impermanent.”—“Is what is impermanent suffering or happiness?”—“Suffering.”—“And that which is impermanent, suffering, and changeable by nature, is it proper to regard it like this: ‘This is mine, I am this, this is my essence?’”—“Definitely not.” 

“What\marginnote{6.43.8} do you think: is perception permanent or impermanent?”—“Impermanent.”—“Is what is impermanent suffering or happiness?”—“Suffering.”—“And that which is impermanent, suffering, and changeable by nature, is it proper to regard it like this: ‘This is mine, I am this, this is my essence?’”—“Definitely not.” 

“What\marginnote{6.43.15} do you think: are intentional activities permanent or impermanent?”—“Impermanent.”—“Is what is impermanent suffering or happiness?”—“Suffering.”—“And that which is impermanent, suffering, and changeable by nature, is it proper to regard it like this: ‘This is mine, I am this, this is my essence?’”—“Definitely not.” 

“What\marginnote{6.43.22} do you think: is consciousness permanent or impermanent?”—“Impermanent.”—“Is what is impermanent suffering or happiness?”—“Suffering.”—“And that which is impermanent, suffering, and changeable by nature, is it proper to regard it like this: ‘This is mine, I am this, this is my essence?’”—“Definitely not.” 

“So,\marginnote{6.44.1} whatever form there is—whether past, present, or future, internal or external, gross or subtle, inferior or superior, far or near—it should all be seen with right wisdom according to reality: ‘This is not mine, I am not this, this is not my essence.’ 

Whatever\marginnote{6.45.1} feeling there is—whether past, present, or future, internal or external, gross or subtle, inferior or superior, far or near—it should all be seen with right wisdom according to reality: ‘This is not mine, I am not this, this is not my essence.’ 

Whatever\marginnote{6.45.4} perception there is—whether past, present, or future, internal or external, gross or subtle, inferior or superior, far or near—it should all be seen with right wisdom according to reality: ‘This is not mine, I am not this, this is not my essence.’ 

Whatever\marginnote{6.45.7} intentional activities there are—whether past, present, or future, internal or external, gross or subtle, inferior or superior, far or near—they should all be seen with right wisdom according to reality: ‘This is not mine, I am not this, this is not my essence.’ 

Whatever\marginnote{6.45.10} consciousness there is—whether past, present, or future, internal or external, gross or subtle, inferior or superior, far or near—it should all be seen with right wisdom according to reality: ‘This is not mine, I am not this, this is not my essence.’ 

A\marginnote{6.46.1} learned noble disciple who sees this is repelled by form, repelled by feeling, repelled by perception, repelled by intentional activities, and repelled by consciousness. Being repelled, they become desireless. Because they are desireless, they are freed. When they are freed, they know they are freed. They understand that birth has come to an end, that the spiritual life has been fulfilled, that the job has been done, that there is no further state of existence.” 

This\marginnote{6.47.1} is what the Buddha said. The monks from the group of five were pleased and they rejoiced in the Buddha’s exposition. And while this exposition was being spoken to the monks from the group of five, their minds were freed from the corruptions through letting go. 

Then\marginnote{6.47.4} there were six perfected ones in the world. 

\scend{The account of the group of five is finished. }

\scend{The first section for recitation is finished. }

\section*{7. The account of the going forth }

At\marginnote{7.1.1} that time in Benares there was a gentleman called Yasa, the son of a wealthy merchant, who had been brought up in great comfort. He had three stilt houses: one for the winter, one for the summer, and one for the rainy season. 

While\marginnote{7.1.4} Yasa was spending the four months of the rainy season in the rainy-season house, he was attended on by female musicians, and he did not come down from that house. On one occasion, while he was enjoying himself with worldly pleasures, he fell asleep before his attendants. He then woke up first, while the oil lamp was still burning. He saw his attendants sleeping: one with a lute in her armpit, another with a tabor on her neck, still another with a drum in her armpit; one with hair disheveled, another drooling, still another talking in her sleep. It was like a charnel ground before his very eyes. When he saw this, the downside became clear, and a feeling of repulsion stayed with him. He uttered a heartfelt exclamation: “Oh the oppression! Oh the affliction!” 

He\marginnote{7.3.1} then put on his golden shoes and went to the entrance door. Spirits opened the door, thinking, “No-one should create any obstacle for Yasa going forth into homelessness.” He went to the town gate, and again it was opened by spirits. He then went to the deer park at Isipatana. 

Just\marginnote{7.4.1} then, after getting up early in the morning, the Buddha was doing walking meditation outside. When the Buddha saw Yasa coming, he stepped down from his walking path and sat down on the prepared seat. 

As\marginnote{7.4.3} he was getting close to the Buddha, Yasa uttered the same heartfelt exclamation: “Oh the oppression! Oh the affliction!” 

The\marginnote{7.4.5} Buddha said, “This isn’t oppressive, Yasa, this isn’t afflictive. Come and sit down. I’ll give you a teaching.” 

Thinking,\marginnote{7.5.1} “Apparently this isn’t oppressive, apparently it’s not afflictive!” excited and joyful, Yasa removed his shoes, approached the Buddha, bowed, and sat down. 

The\marginnote{7.5.3} Buddha then gave Yasa a progressive talk—on generosity, morality, and heaven; on the downside, degradation, and defilement of worldly pleasures; and he revealed the benefits of renunciation. When the Buddha knew that Yasa’s mind was ready, supple, without hindrances, joyful, and confident, he revealed the teaching unique to the Buddhas: suffering, its origin, its end, and the path. Just as a clean and stainless cloth absorbs dye properly, so too, while he was sitting right there, Yasa experienced the stainless vision of the Truth: “Anything that has a beginning has an end.” 

Soon\marginnote{7.7.1} afterwards Yasa’s mother went up to his stilt house. Not seeing him, she went to her husband and said, “I can’t find your son Yasa.” The merchant then dispatched horsemen to the four directions, while he himself went to the deer park at Isipatana. He saw the imprints of the golden shoes on the ground and he followed along. 

When\marginnote{7.8.1} the Buddha saw the wealthy merchant coming, he thought, “Why don’t I use my supernormal powers so that the merchant, when he sits down, doesn’t see Yasa seated next to him?” And he did just that. 

The\marginnote{7.9.1} merchant approached the Buddha and said, “Sir, have you seen Yasa by any chance?” 

“Please\marginnote{7.9.3} sit down, householder. Perhaps you’ll get to see Yasa.” 

When\marginnote{7.9.4} the merchant heard this, he was elated and joyful. And he bowed and sat down. 

The\marginnote{7.10.1} Buddha then gave him a progressive talk—on generosity, morality, and heaven; on the downside, degradation, and defilement of worldly pleasures; and he revealed the benefits of renunciation. When the Buddha knew that his mind was ready, supple, without hindrances, joyful, and confident, he revealed the teaching unique to the Buddhas: suffering, its origin, its end, and the path. And just as a clean and stainless cloth absorbs dye properly, so too, while he was sitting right there, the merchant experienced the stainless vision of the Truth: “Anything that has a beginning has an end.” 

He\marginnote{7.10.8} had seen the Truth, had reached, understood, and penetrated it. He had gone beyond doubt and uncertainty, had attained to confidence, and had become independent of others in the Teacher’s instruction. And he said to the Buddha, “Wonderful, sir, wonderful! Just as one might set upright what’s overturned, or reveal what’s hidden, or show the way to one who’s lost, or bring a lamp into the darkness so that one with eyes might see what’s there—just so has the Buddha made the Teaching clear in many ways. I go for refuge to the Buddha, the Teaching, and the Sangha of monks. Please accept me as a lay follower who’s gone for refuge for life.” He was the first person in the world to become a lay follower by means of the triple refuge. 

While\marginnote{7.11.1} his father was given this teaching, Yasa reviewed what he had already seen and understood, and his mind was freed from the corruptions through letting go. Realizing what had happened, the Buddha thought, “Yasa is incapable of returning to the lower life to enjoy worldly pleasures as he did while still a householder. Let me stop using my supernormal powers.” And he did. 

The\marginnote{7.12.1} merchant saw Yasa sitting there and he said to him, “Dear Yasa, your mother is grieving and lamenting. Please give her back her life.” Yasa looked to the Buddha, and the Buddha said to the merchant, “What do you think, householder: suppose the mind of one such as you—who has seen and understood the Truth with the trainee’s knowledge and vision—while he was reviewing what he had already seen and understood, was freed from the corruptions through letting go. Would he be able to return to the lower life to enjoy worldly pleasures as he did while still a householder?” 

“Definitely\marginnote{7.13.5} not.” 

“But\marginnote{7.13.6} this is what has happened to Yasa. He is now unable to return to the lower life.” 

“It’s\marginnote{7.14.1} a great gain for Yasa that his mind has been freed from the corruptions through letting go! Sir, please accept today’s meal from me with Yasa as your attendant.” The Buddha consented by remaining silent. 

Knowing\marginnote{7.14.4} that the Buddha had consented, the merchant got up from his seat, bowed down, circumambulated the Buddha with his right side toward him, and left. Soon after the merchant had left, Yasa said to the Buddha, “Sir, I wish to receive the going forth in your presence. I wish to receive the full ordination.” The Buddha said, “Come, monk. The Teaching is well-proclaimed. Practice the spiritual life to make a complete end of suffering.” That was the full ordination of that venerable. Then there were seven perfected ones in the world. 

\scend{The going forth of Yasa is finished. }

The\marginnote{8.1.1} following morning the Buddha robed up, took his bowl and robe, and, with Venerable Yasa as his attendant, went to the house of that merchant where he sat down on the prepared seat. Yasa’s mother and ex-wife approached the Buddha, bowed, and sat down. 

The\marginnote{8.2.1} Buddha gave them a progressive talk—on generosity, morality, and heaven; on the downside, degradation, and defilement of worldly pleasures; and he revealed the benefits of renunciation. When the Buddha knew that their minds were ready, supple, without hindrances, joyful, and confident, he revealed the teaching unique to the Buddhas: suffering, its origin, its end, and the path. And just as a clean and stainless cloth absorbs dye properly, so too, while they were sitting right there, they experienced the stainless vision of the Truth: “Anything that has a beginning has an end.” 

They\marginnote{8.3.1} had seen the Truth, had reached, understood, and penetrated it. They had gone beyond doubt and uncertainty, had attained to confidence, and had become independent of others in the Teacher’s instruction. And they said to the Buddha, “Wonderful, sir, wonderful! … We go for refuge to the Buddha, the Teaching, and the Sangha of monks. Please accept us as lay followers who have gone for refuge for life.” And they were the first women in the world to become lay followers by means of the triple refuge. 

Yasa’s\marginnote{8.4.1} mother, father, and ex-wife personally served various kinds of fine foods to the Buddha and Yasa. When the Buddha had finished his meal, they sat down. The Buddha then instructed, inspired, and gladdened them with a teaching, before getting up from his seat and leaving. 

Now\marginnote{9.1.1} Yasa had four friends—Vimala, \textsanskrit{Subāhu}, \textsanskrit{Puṇṇaji}, and Gavampati—who were from the wealthiest merchant families in Benares. When they heard that Yasa had shaved off his hair and beard, put on ocher robes, and gone forth into homelessness, they said to one another, “This must be an extraordinary spiritual path, an extraordinary going forth, for Yasa to have done this.”\footnote{“Spiritual path” renders \textit{dhammavinaya}. See Appendix of Technical Terms. } And they went to Yasa and bowed down to him. 

Yasa\marginnote{9.2.2} then took his four friends to the Buddha. He bowed, sat down, and said, “Sir, these four friends of mine—Vimala, \textsanskrit{Subāhu}, \textsanskrit{Puṇṇaji}, and Gavampati—are from the wealthiest merchant families in Benares. Please instruct them.” 

The\marginnote{9.3.1} Buddha gave them a progressive talk: on generosity, morality, and heaven; on the downside, degradation, and defilement of worldly pleasures; and he revealed the benefits of renunciation. When the Buddha knew that their minds were ready, supple, without hindrances, joyful, and confident, he revealed the teaching unique to the Buddhas: suffering, its origin, its end, and the path. And just as a clean and stainless cloth absorbs dye properly, so too, while they were sitting right there, they experienced the stainless vision of the Truth: “Anything that has a beginning has an end.” 

They\marginnote{9.4.1} had seen the Truth, had reached, understood, and penetrated it. They had gone beyond doubt and uncertainty, had attained to confidence, and had become independent of others in the Teacher’s instruction. And they said to the Buddha, “Sir, we wish to receive the going forth in your presence. We wish to receive the full ordination.” The Buddha said, “Come, monks. The Teaching is well-proclaimed. Practice the spiritual life to make a complete end of suffering.” That was the full ordination of those venerables. Then, as the Buddha instructed those monks in the Teaching, their minds were freed from the corruptions through letting go. And there were eleven perfected ones in the world. 

\scend{The going forth of the four friends is finished. }

Fifty\marginnote{10.1.1} of Yasa’s friends from leading families in the countryside also heard that Yasa had shaved off his hair and beard, put on ocher robes, and gone forth into homelessness. They too said to one another, “This must be an extraordinary spiritual path, an extraordinary going forth, for Yasa to have done this.” And they went to Yasa and bowed down to him. 

Yasa\marginnote{10.2.2} then took his fifty friends to the Buddha. He bowed, sat down, and said, “Sir, these fifty friends of mine are from leading families in the countryside. Please instruct them.” 

The\marginnote{10.3.1} Buddha then gave them a progressive talk—on generosity, morality, and heaven; on the downside, degradation, and defilement of worldly pleasures; and he revealed the benefits of renunciation. When the Buddha knew that their minds were ready, supple, without hindrances, joyful, and confident, he revealed the teaching unique to the Buddhas: suffering, its origin, its end, and the path. And just as a clean and stainless cloth absorbs dye properly, so too, while they were sitting right there, they experienced the stainless vision of the Truth: “Anything that has a beginning has an end.” 

They\marginnote{10.4.1} had seen the Truth, had reached, understood, and penetrated it. They had gone beyond doubt and uncertainty, had attained to confidence, and had become independent of others in the Teacher’s instruction. And they said to the Buddha, “Sir, we wish to receive the going forth in your presence. We wish to receive the full ordination.” The Buddha said, “Come, monks. The Teaching is well-proclaimed. Practice the spiritual life to make a complete end of suffering.” That was the full ordination of those venerables. Then, as the Buddha instructed those monks in the Teaching, their minds were freed from the corruptions through letting go. And there were sixty-one perfected ones in the world. 

\scend{The going forth of the fifty friends is finished. }

\section*{8. The account of the Lord of Death }

Then\marginnote{11.1.1} the Buddha addressed those monks: “I’m free from all snares, both human and divine. You, too, are free from all snares, both human and divine. Go wandering, monks, for the benefit and happiness of humanity, out of compassion for the world, for the good, benefit, and happiness of gods and humans. You should each go a different way. Proclaim the Teaching that is good in the beginning, good in the middle, and good in the end, that has a true goal and is well articulated. Set out the perfectly complete and pure spiritual life. There are beings with little dust in their eyes who are ruined because of not hearing the Teaching. There will be those who understand. I too will go to \textsanskrit{Uruvelā}, to \textsanskrit{Senānigama}, to proclaim the Teaching.” 

Then\marginnote{11.2.1} the Lord of Death, the Evil One, went up to the Buddha and spoke to him in verse: 

\begin{verse}%
“You’re\marginnote{11.2.2} bound by all snares, \\
Both human and divine. \\
You’re bound by the great bond: \\
You’re not free from me, monastic.” 

“I’m\marginnote{11.2.6} free from all snares, \\
Both human and divine. \\
I’m free from the great bond: \\
Terminator, you’re defeated!” 

“The\marginnote{11.2.10} snare is ethereal, \\
And it comes from the mind. \\
With that I’ll bind you: \\
You’re not free from me, monastic.” 

“Sights,\marginnote{11.2.14} sounds, tastes, smells, \\
And tangibles, the mind’s delights—\\
For these I have no desire: \\
Terminator, you’re defeated!” 

%
\end{verse}

Then\marginnote{11.2.18} the Lord of Death, the Evil One, thought, “The Buddha knows me, the Happy One knows me,” and, sad and miserable, he disappeared right there. 

\scend{The account of the Lord of Death is finished. }

\section*{9. Discussion of the going forth and the full ordination }

Soon\marginnote{12.1.1} afterwards, the monks were bringing back, from various regions and countries, people desiring the going forth and the full ordination, thinking, “The Buddha will ordain them.” The monks became tired, as did those seeking ordination. 

Then,\marginnote{12.1.4} while reflecting in private, the Buddha thought, “Why don’t I allow the monks to give the going forth and the full ordination right there in those various regions and countries?” 

In\marginnote{12.2.1} the evening, the Buddha came out from seclusion, gave a teaching, and told the monks what he had thought, adding: 

\scrule{“I allow you to give the going forth and the full ordination in those various regions and countries. }

And,\marginnote{12.3.2} monks, it should be done like this. First the candidate should shave off his hair and beard and put on ocher robes. He should then arrange his upper robe over one shoulder, pay respect at the feet of the monks, squat on his heels, and raise his joined palms. He should then be told to say this: 

\begin{verse}%
‘I\marginnote{12.4.1} go for refuge to the Buddha, \\
I go for refuge to the Teaching, \\
I go for refuge to the Sangha. 

For\marginnote{12.4.4} the second time, I go for refuge to the Buddha, \\
For the second time, I go for refuge to the Teaching, \\
For the second time, I go for refuge to the Sangha. 

For\marginnote{12.4.7} the third time, I go for refuge to the Buddha, \\
For the third time, I go for refuge to the Teaching, \\
For the third time, I go for refuge to the Sangha.’ 

%
\end{verse}

\scrule{You should give the going forth and the full ordination through the taking of the three refuges.”\footnote{“Should” renders \textit{\textsanskrit{anujānāmi}}. For a discussion of this word, see Appendix of Technical Terms. } }

\scend{The discussion of the full ordination through the taking of the three refuges is finished. }

\section*{10. The second account of the Lord of Death }

When\marginnote{13.1.1} the Buddha had completed the rainy-season residence, he said to the monks, “Through wise attention and wise right effort, I have reached the supreme freedom, realized the supreme freedom. And you, monks, have done the same.” 

Then\marginnote{13.2.1} the Lord of Death, the Evil One, went up to the Buddha and spoke to him in verse: 

\begin{verse}%
“You’re\marginnote{13.2.2} bound by the snares of the Lord of Death, \\
Both human and divine. \\
You’re bound by the great bond: \\
You’re not free from me, monastic.” 

“I’m\marginnote{13.2.6} free from the snares of the Lord of Death, \\
Both human and divine. \\
I’m free from the great bond: \\
Terminator, you’re defeated!” 

%
\end{verse}

Then\marginnote{13.2.10} the Lord of Death, the Evil One, thought, “The Buddha knows me, the Happy One knows me,” and sad and miserable he disappeared right there. 

\scend{The second account of the Lord of Death is finished. }

\section*{11. The account of the fine group of people }

When\marginnote{14.1.1} the Buddha had stayed at Benares for as long as he liked, he set out wandering toward \textsanskrit{Uruvelā}. At a certain point he left the road, entered a forest grove, and sat down at the foot of a tree. 

Just\marginnote{14.1.3} then a fine group of thirty friends and their wives were enjoying themselves in that forest grove. Because one of them did not have a wife, they had brought him a sex worker. While they were all carelessly enjoying themselves, that sex worker took that man’s possessions and ran away. To help their friend, they all went searching for that woman. And as they walked about that forest grove, they saw the Buddha seated at the foot of a tree. They approached him and said, “Sir, have you seen a woman by any chance?” 

“But,\marginnote{14.2.4} young men, why look for a woman?”\footnote{This rendering is elliptical. The combination of \textit{\textsanskrit{kiṁ}} + \textit{\textsanskrit{itthiyā}}, an interrogative particle together with what is probably an instrumental case, normally means, “What use is a woman?” (See DOP, sv. \textit{ka}.) The context, however, makes it clear that the problem is not women as such, but rather the pursuit of sensuality when one is better off looking for a higher spiritual happiness. Thus my indirect translation. } 

They\marginnote{14.2.5} told him what had happened. 

“What\marginnote{14.3.1} do you think is better for you: that you search for a woman, or that you search for yourselves?” 

“It’s\marginnote{14.3.3} better that we search for ourselves.” 

“Well\marginnote{14.3.4} then, sit down, and I’ll give you a teaching.” 

Saying,\marginnote{14.3.5} “Yes, sir,” they bowed to the Buddha and sat down. 

The\marginnote{14.4.1} Buddha then gave them a progressive talk—on generosity, morality, and heaven; on the downside, degradation, and defilement of worldly pleasures; and he revealed the benefits of renunciation. When the Buddha knew that their minds were ready, supple, without hindrances, joyful, and confident, he revealed the teaching unique to the Buddhas: suffering, its origin, its end, and the path. And just as a clean and stainless cloth absorbs dye properly, so too, while they were sitting right there, they experienced the stainless vision of the Truth: “Anything that has a beginning has an end.” 

They\marginnote{14.5.1} had seen the Truth, had reached, understood, and penetrated it. They had gone beyond doubt and uncertainty, had attained to confidence, and had become independent of others in the Teacher’s instruction. And they said to the Buddha, “Sir, we wish to receive the going forth in your presence. We wish to receive the full ordination.” The Buddha said, “Come, monks. The Teaching is well-proclaimed. Practice the spiritual life to make a complete end of suffering.” That was the full ordination of those venerables. 

\scend{The account of the fine group of friends is finished. }

\scend{The second section for recitation is finished. }

\section*{12. The account of the wonders at \textsanskrit{Uruvelā} }

The\marginnote{15.1.1} Buddha continued his wandering and eventually arrived at \textsanskrit{Uruvelā}. At that time there were three dreadlocked ascetics living there: \textsanskrit{Uruvelā} Kassapa, \textsanskrit{Nadī} Kassapa, and \textsanskrit{Gayā} Kassapa. \textsanskrit{Uruvelā} Kassapa was the leader and chief of five hundred dreadlocked ascetics, \textsanskrit{Nadī} Kassapa of three hundred, and \textsanskrit{Gayā} Kassapa of two hundred. 

The\marginnote{15.2.1} Buddha went to the hermitage of \textsanskrit{Uruvelā} Kassapa and said to him, “If it’s not inconvenient for you, Kassapa, may I stay for one night in your fire hut?” 

“It’s\marginnote{15.2.3} not inconvenient for me, Great Ascetic, but there’s a fierce and highly venomous dragon king with supernormal powers there. I don’t want it to harm you.” 

The\marginnote{15.2.4} Buddha asked a second and a third time, and on both occasions \textsanskrit{Uruvelā} Kassapa replied as before. 

The\marginnote{15.2.7} Buddha then said, “Perhaps it won’t harm me. Come on, Kassapa, let me to stay in the fire hut.” 

“Well\marginnote{15.2.11} then, do as you like.” 

The\marginnote{15.3.1} Buddha entered the fire hut and prepared a spread of grass. He sat down, crossed his legs, straightened his body, and established mindfulness in front of him. 

When\marginnote{15.3.2} the dragon saw that the Buddha had entered, he was displeased and emitted smoke. The Buddha thought, “Let me overpower this dragon, using fire against fire, but without harming it in the slightest way.” 

The\marginnote{15.4.1} Buddha then used his supernormal powers so that he, too, emitted smoke. The dragon, being unable to contain his rage, emitted flames. The Buddha entered the fire element and he, too, emitted flames. With both of them emitting flames, it was as if the fire hut was ablaze and burning. Those dreadlocked ascetics gathered around the fire hut, saying, “The Great Ascetic is handsome, but the dragon is harming him.” 

The\marginnote{15.5.1} next morning the Buddha had overcome that dragon, using fire against fire, but without harming it in the slightest way. He put it in his almsbowl and showed it to \textsanskrit{Uruvelā} Kassapa: “Here is your dragon, Kassapa, his fire overpowered by fire.” 

\textsanskrit{Uruvelā}\marginnote{15.5.3} Kassapa thought, “The Great Ascetic is powerful and mighty. Using fire against fire, he has overcome that fierce and highly venomous dragon king with its supernormal powers. But he’s not a perfected one like me.” 

\begin{verse}%
At\marginnote{15.6.1} the \textsanskrit{Nerañjara} the Buddha said \\
To the dreadlocked ascetic \textsanskrit{Uruvelā} Kassapa, \\
“If it’s convenient for you, Kassapa, \\
May I stay for a night in your fire hut?” 

“It’s\marginnote{15.6.5} convenient for me, Great Ascetic, \\
But for your own good, I bar you. \\
A fierce dragon king is there, \\
Highly venomous, with supernormal powers: \\
I don’t want it to harm you.” 

“Perhaps\marginnote{15.6.10} it won’t harm me. Come on, Kassapa, \\
Let me stay in the fire hut.” \\
When he knew the answer was “Yes,” \\
He entered without fear. 

Seeing\marginnote{15.6.14} the sage who had entered, \\
The angry dragon emitted smoke. \\
With a mind of good will, \\
The Great Man, too, emitted smoke. 

Unable\marginnote{15.6.18} to contain his rage, \\
The dragon emitted fire. \\
Well-skilled in the fire element, \\
The Great Man, too, emitted fire. 

With\marginnote{15.6.22} both of them emitting flames, \\
The fire hut was glowing and blazing. \\
Looking on, the dreadlocked ascetics said, \\
“He’s handsome, the Great Ascetic, \\
But the dragon is harming him.” 

Yet\marginnote{15.7.1} the following morning \\
The dragon’s flames were extinguished, \\
While the One with supernormal powers \\
Had flames of various colors. 

Blue,\marginnote{15.7.5} red, and magenta,\footnote{“Red” renders \textit{lohitaka}, whereas “magenta” is for \textit{\textsanskrit{mañjiṭṭhā}}. \href{https://suttacentral.net/dn16/en/brahmali\#3.31.2}{DN 16:3.31.2} says that \textit{lohitaka} is the color of the \textit{\textsanskrit{bandhujīvakapuppha}}, which according to PED is the flower of \textit{Pentapetes phaenicea}, which is red. Vv-a 689: \textit{\textsanskrit{Sinduvārakaṇavīramakulasadisavaṇṇañhi} “\textsanskrit{mañjiṭṭhakan}”ti vuccati}, “For a color like the bud of the \textit{\textsanskrit{sinduvāra}} and the \textit{\textsanskrit{kaṇavīra}} is called \textit{\textsanskrit{mañjiṭṭhaka}}.” The \textit{\textsanskrit{sinduvāra}} (\textit{Vitex negundo}) flower is a variety of shades from white to blue, including purple, whereas the \textit{\textsanskrit{kaṇavīra}} (\textit{Pentapetes phaenicea}) flower is mostly pink. I have settled for “magenta” as an approximate description for this range of colors. } \\
Yellow, and the color of crystal: \\
Flames of various colors remained \\
In the body of \textsanskrit{Aṅgīrasa}. 

Putting\marginnote{15.7.9} the dragon in his bowl, \\
He showed it to the brahmin: \\
“Here is your dragon, Kassapa, \\
His fire overpowered by fire.” 

%
\end{verse}

Because\marginnote{15.7.13} of this wonder of supernormal power, \textsanskrit{Uruvelā} Kassapa gained confidence in the Buddha and said to him, “Great Ascetic, please stay right here. I’ll supply you with food.” 

\scend{The first wonder is finished. }

Soon\marginnote{16.1.1} afterwards the Buddha stayed in a forest grove not far from \textsanskrit{Uruvelā} Kassapa’s hermitage. Then, when the night was well advanced, the magnificent four great kings approached the Buddha, illuminating the whole forest grove. They bowed down to the Buddha and stood at the four cardinal points, appearing like great bonfires. 

The\marginnote{16.2.1} next morning \textsanskrit{Uruvelā} Kassapa went to the Buddha and said, “It’s time, Great Ascetic, the meal is ready. And who was it that visited you last night?” 

“That\marginnote{16.2.4} was the four great kings. They came to me to hear the Teaching.” 

\textsanskrit{Uruvelā}\marginnote{16.2.5} Kassapa thought, “The Great Ascetic is powerful and mighty, in that even the four great kings go to him to hear the Teaching. But he’s not a perfected one like me.” 

The\marginnote{16.2.7} Buddha ate his meal and continued to stay in the same forest grove. 

\scend{The second wonder is finished. }

Once\marginnote{17.1.1} again when the night was well advanced, Sakka, the magnificent ruler of the gods, approached the Buddha, illuminating the whole forest grove. He bowed down to the Buddha and stood up, appearing just like a great bonfire. But it was more splendid and sublime than the previous ones. 

The\marginnote{17.2.1} next morning \textsanskrit{Uruvelā} Kassapa went to the Buddha and said, “It’s time, Great Ascetic, the meal is ready. And who was it that visited you last night?” 

“That\marginnote{17.2.4} was Sakka, the ruler of the gods. He came to me to hear the Teaching.” 

\textsanskrit{Uruvelā}\marginnote{17.2.5} Kassapa thought, “The Great Ascetic is powerful and mighty, in that even Sakka, the ruler of gods, goes to him to hear the Teaching. But he’s not a perfected one like me.” 

The\marginnote{17.2.7} Buddha ate his meal and continued to stay in the same forest grove. 

\scend{The third wonder is finished. }

Once\marginnote{18.1.1} again when the night was well advanced, Sahampati, the magnificent supreme being, approached the Buddha, illuminating the whole forest grove. He bowed down to the Buddha and stood up, appearing just like a great bonfire. But it was even more splendid and sublime than the previous ones. 

The\marginnote{18.2.1} next morning \textsanskrit{Uruvelā} Kassapa went to the Buddha and said, “It’s time, Great Ascetic, the meal is ready. And who was it that visited you last night?” 

“That\marginnote{18.2.4} was Sahampati, the supreme being. He came to me to hear the Teaching.” 

\textsanskrit{Uruvelā}\marginnote{18.2.5} Kassapa thought, “The Great Ascetic is powerful and mighty, in that even Sahampati, the supreme being, goes to him to hear the Teaching. But he’s not a perfected one like me.” 

The\marginnote{18.2.7} Buddha ate his meal and continued to stay in the same forest grove. 

\scend{The fourth wonder is finished. }

At\marginnote{19.1.1} this time \textsanskrit{Uruvelā} Kassapa was holding a great sacrifice, and the whole of \textsanskrit{Aṅga} and Magadha wanted to attend with much food of various kinds. \textsanskrit{Uruvelā} Kassapa considered this and thought, “If the Great Ascetic performs a wonder of supernormal power for the great crowd, he’ll get more material support and honor, whereas I’ll get less. I hope he doesn’t come tomorrow.” 

The\marginnote{19.2.1} Buddha read the mind of \textsanskrit{Uruvelā} Kassapa. He then went to Uttarakuru, collected almsfood there, ate it at the Anotatta lake, and stayed there for the day’s meditation. 

The\marginnote{19.2.2} next morning \textsanskrit{Uruvelā} Kassapa went to the Buddha and said, “It’s time, Great Ascetic, the meal is ready. And why didn’t you come yesterday? We did think of you and set aside a share of various kinds of food.” 

“But,\marginnote{19.3.1} Kassapa, didn’t you think, ‘I hope he doesn’t come tomorrow’? Because I read your mind, I went to Uttarakuru, collected almsfood there, ate it at the Anotatta lake, and stayed there for the day’s meditation.” 

\textsanskrit{Uruvelā}\marginnote{19.4.2} Kassapa thought, “The Great Ascetic is powerful and mighty, in that he can read the minds of others. But he’s not a perfected one like me.” 

The\marginnote{19.4.4} Buddha ate his meal and continued to stay in the same forest grove. 

\scend{The fifth wonder is finished. }

Soon\marginnote{20.1.1} afterwards the Buddha got a rag and he thought, “Where can I wash it?” Reading the Buddha’s mind, Sakka dug a pond with his hand. And he said to the Buddha, “Sir, please wash it here.” 

The\marginnote{20.1.6} Buddha thought, “Where can I beat it?” Reading the Buddha’s mind once again, Sakka placed a boulder there. And he said to the Buddha, “Sir, please beat it here.” 

The\marginnote{20.2.1} Buddha thought, “What can I hold onto to get out of this pond?” A god living in an arjun tree read the Buddha’s mind. She then bent down a branch and said to the Buddha, “Sir, please come out by holding onto this.” 

The\marginnote{20.2.5} Buddha thought, “Where can I dry this rag?” Reading the Buddha’s mind yet again, Sakka placed another boulder there. And he said to the Buddha, “Sir, please dry it here.” 

The\marginnote{20.3.1} next morning \textsanskrit{Uruvelā} Kassapa went to the Buddha and said, “It’s time, Great Ascetic, the meal is ready. But what’s going on? There was no pond here before, but now there is. These boulders were not here before. Who placed them here? And this arjun tree didn’t have a bent branch, but now it does.” 

When\marginnote{20.6.1} the Buddha told him what had happened, \textsanskrit{Uruvelā} Kassapa thought, “The Great Ascetic is powerful and mighty, in that even Sakka, the ruler of the gods, performs services for him. But he’s not a perfected one like me.” 

The\marginnote{20.6.3} Buddha ate his meal and continued to stay in the same forest grove. 

The\marginnote{20.7.1} next morning \textsanskrit{Uruvelā} Kassapa went to the Buddha and said, “It’s time, Great Ascetic, the meal is ready.” 

“You\marginnote{20.7.3} just go ahead, Kassapa, I’ll come.” After dismissing him, he took a fruit from a rose-apple tree—the tree after which the Rose-apple Land of India is named—and then arrived first in the fire hut where he sat down. 

When\marginnote{20.8.1} \textsanskrit{Uruvelā} Kassapa saw the Buddha sitting there, he said to him, “Which path did you take? I left first, but you’re already here.” 

The\marginnote{20.9.1} Buddha told him what he had done and added, “This rose apple has a good color, and it’s fragrant and delicious, too. You can have it, if you wish.” 

“There’s\marginnote{20.9.4} no need. You deserve it and you should have it.” 

\textsanskrit{Uruvelā}\marginnote{20.9.5} Kassapa thought, “The Great Ascetic is powerful and mighty, in that he dismissed me, then took a fruit from a rose-apple tree, and still arrived first in the fire hut. But he’s not a perfected one like me.” 

The\marginnote{20.9.7} Buddha ate his meal and continued to stay in the same forest grove. 

The\marginnote{20.10.1} next morning \textsanskrit{Uruvelā} Kassapa went to the Buddha and said, “It’s time, Great Ascetic, the meal is ready.” 

“You\marginnote{20.10.3} just go ahead, Kassapa, I’ll come.” After dismissing him, he took a fruit from a mango tree not far from the rose-apple tree … he took a fruit from an emblic myrobalan tree not far from the mango tree … he took a fruit from a chebulic myrobalan tree not far from the emblic myrobalan tree … he went to \textsanskrit{Tāvatiṁsa} heaven, took a flower from an orchid tree, and then arrived first in the fire hut where he sat down.\footnote{“Orchid tree” renders \textit{\textsanskrit{pāricchattaka}}. According to PED the \textit{\textsanskrit{pāricchattaka}} tree is equivalent to the \textit{\textsanskrit{kovilāra}} tree, which DOP identifies as the \textit{Bauhinia variegata}, the orchid tree. } 

When\marginnote{20.10.7} \textsanskrit{Uruvelā} Kassapa saw the Buddha sitting there, he said to him, “Which path did you take? I left first, but you’re already here.” 

The\marginnote{20.11.1} Buddha told him what he had done, and added, “This orchid tree flower is colorful and fragrant. You can have it, if you wish.” 

“There’s\marginnote{20.11.4} no need. You deserve it and you should have it.” 

\textsanskrit{Uruvelā}\marginnote{20.11.5} Kassapa thought, “The Great Ascetic is powerful and mighty, in that he dismissed me, then went to \textsanskrit{Tāvatiṁsa} heaven, took an orchid tree flower, and still arrived first in the fire hut. But he’s not a perfected one like me.” 

Soon\marginnote{20.12.1} afterwards those dreadlocked ascetics wanted to tend the sacred fire, but were unable to split the logs. They thought, “This must be because of the supernormal powers of the Great Ascetic.” 

The\marginnote{20.12.4} Buddha said to \textsanskrit{Uruvelā} Kassapa, “May the logs be split, Kassapa.” 

“Yes,\marginnote{20.12.6} may they,” he replied. And five hundred logs were split all at once. 

\textsanskrit{Uruvelā}\marginnote{20.12.8} Kassapa thought, “The Great Ascetic is powerful and mighty, in that he can split logs just like that. But he’s not a perfected one like me.” 

Those\marginnote{20.13.1} ascetics still wanted to tend the sacred fire, but were unable to light it. They thought, “This must be because of the supernormal powers of the Great Ascetic.” 

The\marginnote{20.13.4} Buddha said to \textsanskrit{Uruvelā} Kassapa, “May the fires be lit, Kassapa.” 

“Yes,\marginnote{20.13.6} may they,” he replied. And five hundred fires were lit all at once. 

\textsanskrit{Uruvelā}\marginnote{20.13.8} Kassapa thought, “The Great Ascetic is powerful and mighty, in that he can light fires just like that. But he’s not a perfected one like me.” 

When\marginnote{20.14.1} those ascetics had tended the sacred fires, they were unable to extinguish them. They thought, “This must be because of the supernormal powers of the Great Ascetic.” 

The\marginnote{20.14.4} Buddha said to \textsanskrit{Uruvelā} Kassapa, “May the fires be extinguished, Kassapa.” 

“Yes,\marginnote{20.14.6} may they,” he replied. And the five hundred fires were extinguished all at once. 

\textsanskrit{Uruvelā}\marginnote{20.14.8} Kassapa thought, “The Great Ascetic is powerful and mighty, in that he can extinguish fires just like that. But he’s not a perfected one like me.” 

At\marginnote{20.15.1} that time it was midwinter, with cold days and snow. During this period those ascetics emerged from the \textsanskrit{Nerañjara} river, immersed themselves in it, and repeatedly emerged and immersed themselves. 

Then\marginnote{20.15.2} the Buddha manifested five hundred pans with hot coals, where those ascetics could warm themselves after coming out of the water. They thought, “These were no doubt created by the supernormal powers of the Great Ascetic.” 

\textsanskrit{Uruvelā}\marginnote{20.15.5} Kassapa thought, “The Great Ascetic is powerful and mighty, in that he can manifest so many pans with hot coals. But he’s not a perfected one like me.” 

Soon\marginnote{20.16.1} afterwards an unseasonal storm poured down, producing a great flood. The spot where the Buddha was staying was inundated.\footnote{Here I do not follow MS, which has \textit{na \textsanskrit{otthaṭo}}, but instead the reading \textit{\textsanskrit{anuotthaṭo}} or \textit{\textsanskrit{otthaṭo}} found in other editions. } The Buddha thought, “Why don’t I drive back the water on all sides and walk on the dry ground in the middle?” And he did. 

\textsanskrit{Uruvelā}\marginnote{20.16.6} Kassapa thought, “I hope the Great Ascetic hasn’t been swept away by the water.” Together with a number of ascetics he went by boat to where the Buddha was staying. He saw that the Buddha had driven back the water on all sides and was walking on the dry ground in the middle. And he said to the Buddha, “Is that you, Great Ascetic?” 

“It’s\marginnote{20.16.10} me, Kassapa.” 

The\marginnote{20.16.11} Buddha rose up into the air and landed in the boat.\footnote{“Landed” renders \textit{\textsanskrit{paccuṭṭhāsi}}. This verb usually means “to stand up” or “to get up”, as in getting up from one’s seat or getting up from bed. In the current context, however, the meaning must be slightly different. A fairly unambiguous context is found at \href{https://suttacentral.net/dn21/en/brahmali\#1.2.6}{DN 21:1.2.6}, in the \textsanskrit{Sakkapañhā} Sutta, where Sakka, the king of gods, is said to disappear in heaven and then \textit{\textsanskrit{paccuṭṭhāsi}} in Magadha. Here it would seem to be used synonymously with \textit{\textsanskrit{pātubhavati}}, “to reappear”. I take the meaning to be the same in the present context. } \textsanskrit{Uruvelā} Kassapa thought, “The Great Ascetic is powerful and mighty, in that he can displace the water. But he’s not a perfected one like me.” 

Then\marginnote{20.17.1} the Buddha thought, “For a long time this foolish man has thought, ‘The Great Ascetic is powerful and mighty, but he’s not a perfected one like me.’ Let me stir him up.” And he said to \textsanskrit{Uruvelā} Kassapa, “Kassapa, you’re not a perfected one or on the path to perfection. You don’t have the practice that might make you a perfected one or one on the path to perfection.” 

At\marginnote{20.17.8} that \textsanskrit{Uruvelā} Kassapa bowed down with his head at the Buddha’s feet and said, “Sir, I wish to receive the going forth in your presence. I wish to receive the full ordination.” 

“Kassapa,\marginnote{20.18.1} you’re the leader and chief of five hundred dreadlocked ascetics. Tell them first, so that they may take appropriate action.” 

\textsanskrit{Uruvelā}\marginnote{20.18.3} Kassapa then went to those ascetics and said, “I wish to practice the spiritual life under the Great Ascetic. Please do whatever you think is appropriate.” 

“Sir,\marginnote{20.18.5} we’ve had confidence in the Great Ascetic for a long time. If you are to practice the spiritual life under him, so will all of us.” 

Then,\marginnote{20.19.1} after letting their hair and dreadlocks, their carrying poles and bundles, and their fire-worship implements be carried away by the water, they went to the Buddha. They bowed down with their heads at his feet and said,\footnote{Following the commentary I render \textit{\textsanskrit{kesamissaṁ} \textsanskrit{jaṭāmissaṁ} \textsanskrit{khārikājamissaṁ} \textsanskrit{aggihutamissaṁ}} as if \textit{\textsanskrit{missaṁ}} were not there. Sp 3.52: \textit{\textsanskrit{Kesamissantiādīsu} \textsanskrit{kesā} eva \textsanskrit{kesamissaṁ}. Esa nayo sabbattha}, “In regard to \textit{kesamissa} etc., \textit{kesamissa} is just \textit{kesa}. This method applies to all (four).” Of \textit{aggihut(t)a} CPD says: “an instrument for the fire-worship.” } “Sir, we wish to receive the going forth in your presence. We wish to receive the full ordination.” 

The\marginnote{20.19.3} Buddha said, “Come, monks. The Teaching is well-proclaimed. Practice the spiritual life to make a complete end of suffering.” And that was the full ordination of those venerables. 

\textsanskrit{Nadī}\marginnote{20.20.1} Kassapa saw those things being carried away by the water, and he thought, “I hope my brother is okay.” He dispatched his ascetics, saying, “Go and check on my brother.” Together with the three hundred ascetics he then went to \textsanskrit{Uruvelā} Kassapa and said, “Is this better, Kassapa?” 

“Yes,\marginnote{20.20.7} this is better.” 

Then,\marginnote{20.21.1} after letting their hair and dreadlocks, their carrying poles and bundles, and their fire-worship implements be carried away by the water, they went to the Buddha. They bowed down with their heads at his feet and said, “Sir, we wish to receive the going forth in your presence. We wish to receive the full ordination.” 

The\marginnote{20.21.3} Buddha said, “Come, monks. The Teaching is well-proclaimed. Practice the spiritual life to make a complete end of suffering.” And that was the full ordination of those venerables. 

\textsanskrit{Gayā}\marginnote{20.22.1} Kassapa saw those things being carried away by the water, and he thought, “I hope my brothers are okay.” He dispatched his ascetics, saying, “Go and check on my brothers.” Together with the two hundred ascetics he then went to \textsanskrit{Uruvelā} Kassapa and said, “Is this better, Kassapa?” 

“Yes,\marginnote{20.22.7} this is better.” 

Then,\marginnote{20.23.1} after letting their hair and dreadlocks, their carrying poles and bundles, and their fire-worship implements be carried away by the water, they went to the Buddha. They bowed down with their heads at his feet and said, “Sir, we wish to receive the going forth in your presence. We wish to receive the full ordination.” 

The\marginnote{20.23.3} Buddha said, “Come, monks. The Teaching is well-proclaimed. Practice the spiritual life to make a complete end of suffering.” And that was the full ordination of those venerables. 

By\marginnote{20.24.1} an act of supernormal determination the Buddha stopped five hundred logs from being split before splitting them; he stopped fires from being lit before lighting them; he stopped them from being extinguished before extinguishing them; and he manifested five hundred pans with hot coals. In this way, there were three and a half thousand wonders. 

After\marginnote{21.1.1} staying at \textsanskrit{Uruvelā} for as long as he liked, the Buddha went to \textsanskrit{Gayāsīsa} together with that large sangha of one thousand monks, all of them previously dreadlocked ascetics, and they stayed there. 

Then\marginnote{21.2.1} the Buddha addressed the monks: 

“Everything\marginnote{21.2.2} is burning. What is that everything that is burning? The eye is burning. Sights are burning. Eye consciousness is burning. Eye contact is burning. Whatever feeling arises because of eye contact—whether pleasant, painful, or neither-pleasant-nor-painful—that too is burning. Burning with what? Burning with the fire of sensual desire, the fire of ill will, and the fire of confusion; burning with birth, old age, and death; burning with grief, sorrow, pain, aversion, and distress, I say. 

The\marginnote{21.3.1} ear is burning. Sounds are burning. Ear consciousness is burning. Ear contact is burning. Whatever feeling arises because of ear contact—whether pleasant, painful, or neither-pleasant-nor-painful—that too is burning. Burning with what? Burning with the fire of sensual desire, the fire of ill will, and the fire of confusion; burning with birth, old age, and death; burning with grief, sorrow, pain, aversion, and distress, I say. 

The\marginnote{21.3.4} nose is burning. Smells are burning. Nose consciousness is burning. Nose contact is burning. Whatever feeling arises because of nose contact—whether pleasant, painful, or neither-pleasant-nor-painful—that too is burning. Burning with what? Burning with the fire of sensual desire, the fire of ill will, and the fire of confusion; burning with birth, old age, and death; burning with grief, sorrow, pain, aversion, and distress, I say. 

The\marginnote{21.3.7} tongue is burning. Tastes are burning. Tongue consciousness is burning. Tongue contact is burning. Whatever feeling arises because of tongue contact—whether pleasant, painful, or neither-pleasant-nor-painful—that too is burning. Burning with what? Burning with the fire of sensual desire, the fire of ill will, and the fire of confusion; burning with birth, old age, and death; burning with grief, sorrow, pain, aversion, and distress, I say. 

The\marginnote{21.3.10} body is burning. Touches are burning. Body consciousness is burning. Body contact is burning. Whatever feeling arises because of body contact—whether pleasant, painful, or neither-pleasant-nor-painful—that too is burning. Burning with what? Burning with the fire of sensual desire, the fire of ill will, and the fire of confusion; burning with birth, old age, and death; burning with grief, sorrow, pain, aversion, and distress, I say. 

The\marginnote{21.3.13} mind is burning. Mental phenomena are burning. Mind consciousness is burning. Mind contact is burning. Whatever feeling arises because of mind contact—whether pleasant, painful, or neither-pleasant-nor-painful—that too is burning. Burning with what? Burning with the fire of sensual desire, the fire of ill will, and the fire of confusion; burning with birth, old age, and death; burning with grief, sorrow, pain, aversion, and distress, I say. 

When\marginnote{21.4.1} they see this, the learned noble disciple is repelled by the eye, repelled by sights, repelled by eye consciousness, repelled by eye contact, and repelled by whatever pleasant, painful, or neither-pleasant-nor-painful feeling that arises because of eye contact. 

They\marginnote{21.4.2} are repelled by the ear, repelled by sounds, repelled by ear consciousness, repelled by ear contact, and repelled by whatever pleasant, painful, or neither-pleasant-nor-painful feeling that arises because of ear contact. 

They\marginnote{21.4.3} are repelled by the nose, repelled by smells, repelled by nose consciousness, repelled by nose contact, and repelled by whatever pleasant, painful, or neither-pleasant-nor-painful feeling that arises because of nose contact. 

They\marginnote{21.4.4} are repelled by the tongue, repelled by tastes, repelled by tongue consciousness, repelled by tongue contact, and repelled by whatever pleasant, painful, or neither-pleasant-nor-painful feeling that arises because of tongue contact. 

They\marginnote{21.4.5} are repelled by the body, repelled by touches, repelled by body consciousness, repelled by body contact, and repelled by whatever pleasant, painful, or neither-pleasant-nor-painful feeling that arises because of body contact. 

They\marginnote{21.4.6} are repelled by the mind, repelled by mental phenomena, repelled by mind consciousness, repelled by mind contact, and repelled by whatever pleasant, painful, or neither-pleasant-nor-painful feeling that arises because of mind contact. 

Being\marginnote{21.4.7} repelled, they become desireless. Because they are desireless, they are freed. When they are freed, they know they are freed. They understand that birth has come to an end, that the spiritual life has been fulfilled, that the job has been done, that there is no further state of existence.” 

And\marginnote{21.4.9} while this exposition was being spoken, the minds of those one thousand monks were freed from the corruptions through letting go. 

\scend{The discourse on burning is finished. }

\scend{The third section for recitation on the wonders at \textsanskrit{Uruvelā} is finished. }

\section*{13. The account of the meeting with \textsanskrit{Bimbisāra} }

After\marginnote{22.1.1} staying at \textsanskrit{Gayāsīsa} for as long as he liked, the Buddha set out wandering toward \textsanskrit{Rājagaha} with that large sangha of one thousand monks, all of them previously dreadlocked ascetics. When he eventually arrived, he stayed in Cane Grove at the \textsanskrit{Suppatiṭṭha} Shrine. 

King\marginnote{22.2.1} Seniya \textsanskrit{Bimbisāra} of Magadha was told: “Sir, the ascetic Gotama, the Sakyan, who has gone forth from the Sakyan clan, has arrived at \textsanskrit{Rājagaha} and is staying in the Cane Grove at the \textsanskrit{Suppatiṭṭha} Shrine. That good Gotama has a fine reputation: 

‘He\marginnote{22.2.4} is a Buddha, perfected and fully awakened, complete in insight and conduct, happy, knower of the world, supreme leader of trainable people, teacher of gods and humans, awakened, a Buddha. With his own insight he has seen this world with its gods, its lords of death, and its supreme beings, this society with its monastics and brahmins, its gods and humans, and he makes it known to others. He has a Teaching that’s good in the beginning, good in the middle, and good in the end. It has a true goal and is well articulated. He sets out a perfectly complete and pure spiritual life.’ It’s good to see such perfected ones.” 

Then,\marginnote{22.3.1} accompanied by one hundred and twenty thousand brahmin householders from Magadha, King \textsanskrit{Bimbisāra} went to the Buddha, bowed, and sat down. Among those brahmins, some bowed to the Buddha and then sat down, some exchanged pleasantries with him and then sat down, some raised their joined palms and then sat down, some announced their name and family and then sat down, and some sat down in silence. They thought, “Is the Great Ascetic practicing the spiritual life under \textsanskrit{Uruvelā} Kassapa, or is \textsanskrit{Uruvelā} Kassapa practicing the spiritual life under the Great Ascetic?” 

Reading\marginnote{22.4.3} their minds, the Buddha spoke to Venerable \textsanskrit{Uruvelā} Kassapa in verse: 

\begin{verse}%
“The\marginnote{22.4.4} resident of \textsanskrit{Uruvelā}, known as The Emaciated One—\\
What did he see that he abandoned the fire? \\
Kassapa, I ask you this: \\
Why did you abandon the fire worship?” 

“As\marginnote{22.4.8} a reward for the sacrifice, they promise worldly pleasures: \\
Sights, sounds, and tastes, and women, too.\footnote{Sp 3.55: \textit{\textsanskrit{Dutiyagāthāya} ayamattho – ete \textsanskrit{rūpādike} \textsanskrit{kāme} itthiyo ca \textsanskrit{yaññā} abhivadanti}, “This is the meaning of the second verse: they promise that the sacrifice will give these worldly pleasures, starting with forms, and also women.” Sp-\textsanskrit{ṭ} 3.55 adds: \textit{\textsanskrit{Yaññā} \textsanskrit{abhivadantīti} \textsanskrit{yāgahetu} \textsanskrit{ijjhantīti} vadanti}, “\textit{\textsanskrit{Yaññā} abhivadanti} means: they say, ‘They get good results because of the sacrifice.’” } \\
But knowing the stain of ownership, \\
I found no delight in worship and sacrifice.” 

“So\marginnote{22.5.1} your mind didn’t delight there—\\
In sights, sounds, or tastes. \\
What then, in the world of gods and humans, \\
Does your mind delight in? Tell me this, Kassapa.” 

“I\marginnote{22.5.6} saw the state of peace that is detached from sensual existence, \\
Where there is nothing and no ownership; \\
It doesn’t change, and can’t be found through another.\footnote{Sp 3.55: \textit{\textsanskrit{Jātijarāmaraṇānaṁ} \textsanskrit{abhāvena} \textsanskrit{anaññathābhāviṁ}. \textsanskrit{Attanā} \textsanskrit{bhāvitena} maggeneva \textsanskrit{adhigantabbaṁ}, na \textsanskrit{aññena} kenaci adhigametabbanti \textsanskrit{anaññaneyyaṁ}}. “One is \textit{\textsanskrit{anaññathābhāvi}} by not being born, becoming old, or dying. \textit{\textsanskrit{Anaññaneyyaṁ}}: it is to be obtained by a path developed by oneself; it is not to be obtained by anyone else.” } \\
That’s why I found no delight in worship and sacrifice.” 

%
\end{verse}

\textsanskrit{Uruvelā}\marginnote{22.6.1} Kassapa got up from his seat, arranged his upper robe over one shoulder, bowed down with his head at the Buddha’s feet, and said, “Sir, you’re my teacher, I’m your disciple; you’re my teacher, I’m your disciple.” 

The\marginnote{22.6.4} one hundred and twenty thousand brahmin householders from Magadha thought, “So \textsanskrit{Uruvelā} Kassapa is practicing the spiritual life under the Great Ascetic.” Reading their minds, the Buddha gave them a progressive talk—on generosity, morality, and heaven; on the downside, degradation, and defilement of worldly pleasures; and he revealed the benefits of renunciation. When the Buddha knew that their minds were ready, supple, without hindrances, joyful, and confident, he revealed the teaching unique to the Buddhas: suffering, its origin, its end, and the path. And just as a clean and stainless cloth absorbs dye properly, so too, while they were sitting right there, one hundred and ten thousand of those brahmin householders headed by \textsanskrit{Bimbisāra} experienced the stainless vision of the Truth: “Anything that has a beginning has an end.” The remaining ten thousand declared themselves as lay followers. 

King\marginnote{22.9.1} \textsanskrit{Bimbisāra} had seen the Truth, had reached, understood, and penetrated it. He had gone beyond doubt and uncertainty, had attained to confidence, and had become independent of others in the Teacher’s instruction. He then said to the Buddha, “Sir, when I was a prince, I had five wishes, and they have now been fulfilled. When I was a prince, I thought, ‘Oh, I wish they would anoint me as the king!’ That was my first wish, which has now been fulfilled. ‘May one who is perfected and fully awakened come to my kingdom!’ That was my second wish, which has now been fulfilled. ‘May I get to visit that Buddha!’ That was my third wish, which has now been fulfilled. ‘May that Buddha give me a teaching!’ That was my fourth wish, which has now been fulfilled. ‘May I understand the Teaching of that Buddha!’ That was my fifth wish, which has now been fulfilled. Wonderful, sir, wonderful! Just as one might set upright what’s overturned, or reveal what’s hidden, or show the way to one who’s lost, or bring a lamp into the darkness so that one with eyes might see what’s there—just so has the Buddha made the Teaching clear in many ways. I go for refuge to the Buddha, the Teaching, and the Sangha of monks. Please accept me as a lay follower who’s gone for refuge for life. And please accept tomorrow’s meal from me together with the Sangha of monks.” The Buddha consented by remaining silent. Knowing that the Buddha had consented, the king got up from his seat, bowed down, circumambulated the Buddha with his right side toward him, and left. 

The\marginnote{22.12.2} following morning King \textsanskrit{Bimbisāra} had various kinds of fine foods prepared. He then had the Buddha informed that the meal was ready. 

The\marginnote{22.12.4} Buddha robed up, took his bowl and robe, and together with a large sangha of a thousand monks, all previously dreadlocked ascetics, he entered \textsanskrit{Rājagaha}. Just then Sakka, the ruler of the gods, had transformed himself into a young brahmin. He walked in front of the Sangha of monks headed by the Buddha, chanting these verses: 

\begin{verse}%
“The\marginnote{22.13.2} Tamed One with the tamed ones, previously dreadlocked; \\
The Liberated One with the liberated ones: \\
Golden in color, \\
The Buddha entered \textsanskrit{Rājagaha}. 

The\marginnote{22.13.6} Freed One with the freed ones, previously dreadlocked; \\
The Liberated One with the liberated ones: \\
Golden in color, \\
The Buddha entered \textsanskrit{Rājagaha}. 

The\marginnote{22.13.10} One Crossed Over with the ones crossed over, previously dreadlocked; \\
The Liberated One with the liberated ones: \\
Golden in color, \\
The Buddha entered \textsanskrit{Rājagaha}. 

The\marginnote{22.13.14} Peaceful One with the peaceful ones, previously dreadlocked; \\
The Liberated One with the liberated ones: \\
Golden in color, \\
The Buddha entered \textsanskrit{Rājagaha}. 

He\marginnote{22.13.18} has ten abidings and ten powers; \\
He knows ten truths and has ten qualities—\\
With a following of ten times one hundred, \\
The Buddha entered \textsanskrit{Rājagaha}.” 

%
\end{verse}

People\marginnote{22.14.1} saw Sakka, and they said, “This young brahmin is handsome and graceful. Who is he?” Sakka replied to them in verse: 

\begin{verse}%
“Unwavering\marginnote{22.14.5} and tamed in all respects, \\
Purified, perfected, and without equal; \\
The one in the world who is happy—\\
I’m his servant.” 

%
\end{verse}

The\marginnote{22.15.1} Buddha then went to King \textsanskrit{Bimbisāra}’s house where he sat down on the prepared seat, together with the Sangha of monks. The king personally served various kinds of fine foods to the Sangha of monks headed by the Buddha. When the Buddha had finished his meal, the king sat down to one side. And he thought, “Where will the Buddha stay that’s neither too far from habitation nor too close, that has good access roads and is easily accessible for people who seek him, that has few people during the day and is quiet at night, that’s free from chatter and offers solitude, a private resting place suitable for seclusion?” And it occurred to him, “My Bamboo Grove park has all these qualities. Why don’t I give it to the Sangha of monks headed by the Buddha?” 

The\marginnote{22.18.1} king then took hold of a golden ceremonial vessel and dedicated the park to the Buddha, saying, “I give this park, the Bamboo Grove, to the Sangha of monks headed by the Buddha.” The Buddha accepted the park. After instructing, inspiring, and gladdening the king with a teaching, he got up from his seat and left. Soon afterwards the Buddha gave a teaching and addressed the monks: 

\scrule{“I allow monasteries.”\footnote{“Monastery” renders \textit{\textsanskrit{ārāma}}. \textit{\textsanskrit{Ārāma}} could be rendered as “park”, which is the more fundamental meaning of the word. However, since such parks were sometimes given to the Sangha to serve as monasteries, the monasteries, too, became known by the same name. It is the latter meaning which predominates in the Vinaya \textsanskrit{Piṭaka}. } }

\scend{The account of the meeting with \textsanskrit{Bimbisāra} is finished. }

\section*{14. The account of the going forth of \textsanskrit{Sāriputta} and \textsanskrit{Moggallāna} }

At\marginnote{23.1.1} that time the wanderer \textsanskrit{Sañcaya} was staying at \textsanskrit{Rājagaha} with a large group of two hundred and fifty wanderers, including \textsanskrit{Sāriputta} and \textsanskrit{Moggallāna}. The two of them had made an agreement that whoever reached freedom from death first would inform the other. 

Just\marginnote{23.2.1} then, Venerable Assaji robed up in the morning, took his bowl and robe, and entered \textsanskrit{Rājagaha} for almsfood. He was pleasing in his conduct: in going out and coming back, in looking ahead and looking aside, in bending and stretching his arms. His eyes were lowered, and he was perfect in deportment. The wanderer \textsanskrit{Sāriputta} observed all this and thought, “This monk is one of those in the world who are perfected or on the path to perfection. Why don’t I go up to him and ask in whose name he has gone forth, and who his teacher is or whose teachings he follows?” But it occurred to him, “It’s the wrong time to ask him while he’s walking for almsfood among the houses. Let me follow behind him, for one who seeks the path will find it.” 

After\marginnote{23.3.4} walking for alms in \textsanskrit{Rājagaha}, Assaji turned back with his almsfood. \textsanskrit{Sāriputta} then went up to him and exchanged pleasantries with him. And he asked, “Venerable, your senses are clear and your skin is pure and bright. In whose name have you gone forth? Who is your teacher or whose teaching do you follow?” 

“There’s\marginnote{23.4.1} a great ascetic, a Sakyan who has gone forth from the Sakyan clan. I’ve gone forth in his name, he’s my teacher, and I follow his teaching.” 

“But\marginnote{23.4.2} what does he teach?” 

“I’ve\marginnote{23.4.3} only recently gone forth; I’m new to this spiritual path. I’m not able to give you the Teaching in full, but I can tell you the meaning in brief.” 

\textsanskrit{Sāriputta}\marginnote{23.4.4} replied, “Yes, please,” and he added: 

\begin{verse}%
“Speak\marginnote{23.4.6} little or much, \\
But do tell me the meaning. \\
I just want the meaning, \\
For what’s the point of a detailed exposition?” 

%
\end{verse}

And\marginnote{23.5.1} Assaji gave this teaching to the wanderer \textsanskrit{Sāriputta}: 

\begin{verse}%
“Of\marginnote{23.5.2} causally arisen things, \\
The Buddha has declared their cause, \\
As well as their ending. \\
This is the teaching of the Great Ascetic.” 

%
\end{verse}

When\marginnote{23.5.6} he had heard this teaching, \textsanskrit{Sāriputta} experienced the stainless vision of the Truth: “Anything that has a beginning has an end.” 

\begin{verse}%
“Now\marginnote{23.5.8} this is the truth, even just this much—\\
The sorrowless state that you have penetrated,\footnote{Sp 3.59 explains \textit{paccabyattha} as \textit{\textsanskrit{paṭividdhāttha} tumhe}, “You have penetrated.” } \\
Unseen and neglected \\
For innumerable eons.” 

%
\end{verse}

Then\marginnote{23.6.1} the wanderer \textsanskrit{Sāriputta} went to the wanderer \textsanskrit{Moggallāna}. When \textsanskrit{Moggallāna} saw him coming, he said to \textsanskrit{Sāriputta}, “Your senses are clear and your skin is pure and bright. You haven’t attained freedom from death, have you?” 

“I\marginnote{23.6.5} have.” 

“But\marginnote{23.6.6} how did it happen?” 

\textsanskrit{Sāriputta}\marginnote{23.7.1} told him everything up to and including the teaching given by Assaji. When he had heard this teaching, \textsanskrit{Moggallāna} experienced the stainless vision of the Truth: 

“Anything\marginnote{23.10.6} that has a beginning has an end.” 

\begin{verse}%
“Now\marginnote{23.10.8} this is the truth, even just this much—\\
The sorrowless state that you have penetrated, \\
Unseen and neglected \\
For innumerable eons.” 

%
\end{verse}

\textsanskrit{Moggallāna}\marginnote{24.1.1} said to \textsanskrit{Sāriputta}, “Let’s go to the Buddha. He’s our teacher.” 

“But\marginnote{24.1.3} these two hundred and fifty wanderers look to us for support. We must tell them first, so that they may take appropriate action.” And they went to those wanderers and said, “We’re going over to the Buddha. He’s our teacher.” 

“But\marginnote{24.1.7} we look to you for support. If you are to practice the spiritual life under the Great Ascetic, so will all of us.” 

Then\marginnote{24.2.1} \textsanskrit{Sāriputta} and \textsanskrit{Moggallāna} went to \textsanskrit{Sañcaya} and said, “We’re going over to the Buddha. He’s our teacher.” 

“Don’t\marginnote{24.2.3} go! The three of us can look after this community together.” 

\textsanskrit{Sāriputta}\marginnote{24.2.4} and \textsanskrit{Moggallāna} said the same thing a second time and a third time, and they got the same reply. They then took those two hundred and fifty wanderers and went to the Bamboo Grove. But the wanderer \textsanskrit{Sañcaya} vomited hot blood right there. 

When\marginnote{24.3.3} the Buddha saw \textsanskrit{Sāriputta} and \textsanskrit{Moggallāna} coming, he said to the monks, “The two friends Kolita and Upatissa are coming. They will become my most eminent disciples, an excellent pair.” 

\begin{verse}%
They\marginnote{24.3.6} had not even reached the Bamboo Grove, \\
Yet had a profound range of knowledge, \\
About the supreme end of ownership, about freedom. \\
And the Teacher said of them: 

“These\marginnote{24.3.10} two friends are coming, \\
Kolita and Upatissa. \\
They will be an excellent pair, \\
My most eminent disciples.” 

%
\end{verse}

\textsanskrit{Sāriputta}\marginnote{24.4.1} and \textsanskrit{Moggallāna} approached the Buddha, bowed down with their heads at his feet, and said, “Sir, we wish to receive the going forth in your presence. We wish to receive the full ordination.” The Buddha said, “Come, monks. The Teaching is well-proclaimed. Practice the spiritual life to make a complete end of suffering.” That was the full ordination of those venerables. 

\subsection*{The going forth of the well-known }

At\marginnote{24.5.1} that time many well-known gentlemen from Magadha were practicing the spiritual life under the Buddha. People complained and criticized him, “The ascetic Gotama is making us childless; he’s making us widows. He’s breaking up good families! A thousand dreadlocked ascetics have now gone forth because of him, and also these two hundred and fifty wanderers who were disciples of \textsanskrit{Sañcaya}. All these well-known gentlemen from Magadha are practicing the spiritual life under the ascetic Gotama.” And when they saw monks, they confronted them with this verse:\footnote{“Confronted” renders \textit{codenti}. See Appendix of Technical Terms. } 

\begin{verse}%
“The\marginnote{24.5.6} Great Ascetic has arrived \\
At Giribbaja in Magadha. \\
After leading away all of \textsanskrit{Sañcaya}’s disciples, \\
Who will he lead away next?” 

%
\end{verse}

The\marginnote{24.6.1} monks heard the complaints of those people and they told the Buddha. … “The complaining will soon stop. It will only go on for seven days. Still, when people confront you like this, you can confront them in return with this verse: 

\begin{verse}%
‘Indeed,\marginnote{24.6.10} the Great Heroes, the Buddhas, \\
Lead by means of a good teaching. \\
When you understand this, what indignation can there be \\
Toward those who lead legitimately?’” 

%
\end{verse}

Soon,\marginnote{24.7.1} when they saw monks, people confronted them with the same verse: 

\begin{verse}%
“The\marginnote{24.7.2} Great Ascetic has arrived \\
At Giribbaja in Magadha. \\
After leading away all of \textsanskrit{Sañcaya}’s disciples, \\
Who will he lead away next?” 

%
\end{verse}

And\marginnote{24.7.6} the monks confronted them in return with this verse: 

\begin{verse}%
“Indeed,\marginnote{24.7.7} the Great Heroes, the Buddhas, \\
Lead by means of a good teaching. \\
When you understand this, what indignation can there be \\
Toward those who lead legitimately?” 

%
\end{verse}

People\marginnote{24.7.11} thought, “So it seems the Sakyan monastics lead legitimately, not illegitimately.” The complaining went on for seven days and then stopped. 

\scend{The account of the going forth of \textsanskrit{Sāriputta} and \textsanskrit{Moggallāna} is finished. }

\scend{The fourth section for recitation is finished. }

\section*{15. Discussion of the proper conduct toward the preceptor }

At\marginnote{25.1.1} that time the monks did not have preceptors or teachers, and as a result they were not being instructed. When walking for almsfood, they were shabbily dressed and improper in appearance. While people were eating, they held out their almsbowls to receive leftovers, even right over their food, whether it was cooked or fresh food, delicacies or drinks. They ate bean curry and rice that they themselves had asked for, and they were noisy in the dining hall.\footnote{\textit{Bhattagga} is literally “a meal house”. The name suggests that the \textit{bhattagga} was a separate building for eating. They were found both in private houses and in monasteries (\href{https://suttacentral.net/pli-tv-kd10/en/brahmali\#4.5.7}{Kd 10:4.5.7}). Since they were part of houses or a compound of private buildings, “refectory” is not a satisfactory rendering. The fact that kitchens are not mentioned separately may mean that they were part of the \textit{bhattagga}, except in monasteries. This is supported by a passage \href{https://suttacentral.net/pli-tv-bu-vb-pj3/en/brahmali\#5.3.1}{Bu Pj 3:5.3.1} that mentions a cooking implement, a pestle, being stored in a village \textit{bhattagga}. } People complained and criticized them, “How can the Sakyan monastics act like this? They are just like brahmins at a brahminical meal!” 

The\marginnote{25.3.1} monks heard the complaints of those people. The monks of few desires, who had a sense of conscience, and who were contented, afraid of wrongdoing, and fond of the training, complained and criticized them, “How can monks act like this?” They then told the Buddha. … 

Soon\marginnote{25.4.3} afterwards the Buddha had the Sangha gathered and questioned the monks: “Is it true, monks, that monks act like this?” 

“It’s\marginnote{25.4.5} true, sir.” 

The\marginnote{25.5.1} Buddha rebuked them, “It’s not suitable for those foolish men, it’s not proper, it’s not worthy of a monastic, it’s not allowable, it’s not to be done. How can they act like this? This will affect people’s confidence, and cause some to lose it.” 

Then\marginnote{25.6.1} the Buddha spoke in many ways in dispraise of being difficult to support and maintain, in dispraise of great desires, discontent, socializing, and laziness; but he spoke in many ways in praise of being easy to support and maintain, of fewness of wishes, contentment, self-effacement, ascetic practices, serenity, reduction in things, and being energetic. After giving a teaching on what is right and proper, he addressed the monks: 

\scrule{“There should be a preceptor. }

The\marginnote{25.6.3} preceptor should think of his student as a son and the student his preceptor as a father. In this way they will respect, esteem, and be considerate toward each other, and they will grow and reach greatness on this spiritual path. 

A\marginnote{25.7.1} preceptor should be chosen like this. After arranging his upper robe over one shoulder, a student should pay respect at the feet of the potential preceptor. He should then squat on his heels, raise his joined palms, and say, ‘Venerable, please be my preceptor.’ And he should repeat this a second and a third time. If the other conveys the following by body, by speech, or by body and speech: ‘Yes;’ ‘No problem;’ ‘It’s suitable;’ ‘It’s appropriate;’ or, ‘Carry on with inspiration’—then a preceptor has been chosen. If the other doesn’t convey this by body, by speech, or by body and speech, then a preceptor hasn’t been chosen. 

“A\marginnote{25.8.1} student should conduct himself properly toward his preceptor. This is the proper conduct: 

\subsection*{Meals and almsround}

Having\marginnote{25.8.3.1} gotten up at the appropriate time, the student should remove his sandals and arrange his upper robe over one shoulder. He should then give his preceptor a tooth cleaner and water for rinsing the mouth, and he should prepare a seat for him. If there is congee, he should rinse a vessel and bring the congee to his preceptor. When he has drunk the congee, the student should give him water and receive the vessel. Holding it low, he should wash it carefully without scratching it and then put it away. When the preceptor has gotten up, the student should put away the seat. If the place is dirty, he should sweep it. 

If\marginnote{25.9.1} the preceptor wants to enter the village, the student should give him a sarong and receive the one he’s wearing in return. He should give him a belt. He should put the upper robes together, overlapping each other edge-to-edge, and then give them to him. He should rinse his preceptor’s bowl and give it to him while wet.\footnote{“He should put the upper robes together, overlapping each other edge-to-edge” renders \textit{\textsanskrit{saguṇaṁ} \textsanskrit{katvā} \textsanskrit{saṅghāṭiyo}}. Sp 3.66: \textit{\textsanskrit{Saguṇaṁ} \textsanskrit{katvāti} dve \textsanskrit{cīvarāni} ekato \textsanskrit{katvā}, \textsanskrit{tā} ekato \textsanskrit{katā} dvepi \textsanskrit{saṅghāṭiyo} \textsanskrit{dātabbā}. \textsanskrit{Sabbañhi} \textsanskrit{cīvaraṁ} \textsanskrit{saṅghaṭitattā} “\textsanskrit{saṅghāṭī}”ti vuccati}, “\textit{\textsanskrit{Saguṇaṁ} \textsanskrit{katvā}}: having made two robes into one, even those two upper robes made into one are to be given. All robes are called \textit{\textsanskrit{saṅghāṭi}} because of being pieced together.” See Appendix of Technical Terms for this rendering of \textit{\textsanskrit{saṅghāṭi}}. } If the preceptor wants an attendant, the student should put on his sarong evenly all around, covering the navel and the knees. He should put on a belt. Putting the upper robes together, overlapping each other edge-to-edge, he should put them on and fasten the toggle. He should rinse his bowl, bring it along, and be his preceptor’s attendant. 

He\marginnote{25.10.1} shouldn’t walk too far behind his preceptor or too close to him. He should receive the contents of his bowl. He shouldn’t interrupt his preceptor when he’s speaking. But if the preceptor’s speech is bordering on an offense, he should stop him. 

When\marginnote{25.10.4} returning, the student should go first to prepare a seat and to set out a foot stool, a foot scraper, and water for washing the feet. He should go out to meet the preceptor and receive his bowl and robe. He should give him a sarong and receive the one he’s wearing in return. If the robe is damp, he should sun it for a short while, but shouldn’t leave it in the heat. He should fold the robe, offsetting the edges by seven centimeters,\footnote{That is, four fingerbreadths. For a discussion of the \textit{\textsanskrit{aṅgula}}, see \textit{sugata} in Appendix of Technical Terms. } so that the fold doesn’t become worn. He should place the belt in the fold. 

If\marginnote{25.10.9} there is almsfood and his preceptor wants to eat, the student should give him water and then the almsfood. He should ask his preceptor if he wants water to drink. When the preceptor has eaten, the student should give him water and receive his bowl. Holding it low, he should wash it carefully without scratching it. He should then dry it and sun it for a short while, but shouldn’t leave it in the heat. 

The\marginnote{25.11.3} student should put away the robe and bowl. When putting away the bowl, he should hold the bowl in one hand, feel under the bed or the bench with the other, and then put it away. He shouldn’t put the bowl away on the bare floor. When putting away the robe, he should hold the robe in one hand, wipe the bamboo robe rack or the clothesline with the other, and then put it away by folding the robe over it, making the ends face the wall and the fold face out. When the preceptor has gotten up, the student should put away the seat and also the foot stool, the foot scraper, and the water for washing the feet. If the place is dirty, he should sweep it. 

\subsection*{Bathing}

If\marginnote{25.12.1} the preceptor wants to bathe, the student should prepare a bath. If he wants a cold bath, he should prepare that; if he wants a hot bath, he should prepare that. 

If\marginnote{25.12.4} the preceptor wants to take a sauna, the student should knead bath powder, moisten the clay, take a sauna bench, and follow behind his preceptor. After giving the preceptor the sauna bench, receiving his robe, and putting it aside, he should give him the bath powder and the clay. If he’s able, he should enter the sauna. When entering the sauna, he should smear his face with clay, cover himself front and back, and then enter. He shouldn’t sit encroaching on the senior monks, and he shouldn’t block the junior monks from getting a seat. While in the sauna, he should provide assistance to his preceptor. When leaving the sauna, he should take the sauna bench, cover himself front and back, and then leave. 

He\marginnote{25.13.5} should also provide assistance to his preceptor in the water. When he has bathed, he should be the first to come out. He should dry himself and put on his sarong. He should then wipe the water off his preceptor’s body, and he should give him his sarong and then his upper robe. Taking the sauna bench, he should be the first to return. He should prepare a seat, and also set out a foot stool, a foot scraper, and water for washing the feet. He should ask his preceptor if he wants water to drink. If the preceptor wants him to recite, he should do so. If the preceptor wants to question him, he should be questioned. 

\subsection*{The dwelling}

If\marginnote{25.14.3.1} the dwelling where the preceptor is staying is dirty, the student should clean it if he’s able. When he’s cleaning the dwelling, he should first take out the bowl and robe and put them aside. He should take out the sitting mat and the sheet and put them aside.\footnote{“Sitting mat” renders \textit{\textsanskrit{nisīdana}}. See Appendix of Technical Terms. } He should take out the mattress and the pillow and put them aside. Holding the bed low, he should carefully take it out without scratching it or knocking it against the door or the door frame, and he should put it aside. Holding the bench low, he should carefully take it out without scratching it or knocking it against the door or the door frame, and he should put it aside. He should take out the bed supports and put them aside. He should take out the spittoon and put it aside. He should take out the leaning board and put it aside. After taking note of its position, he should take out the floor cover and put it aside. If the dwelling has cobwebs, he should first remove them from the ceiling cloth, and he should then wipe the windows and the corners of the room.\footnote{“The windows and the corners of the room” renders \textit{\textsanskrit{ālokasandhikaṇṇabhāga}}. Sp 3.66: \textit{\textsanskrit{Ālokasandhikaṇṇabhāgāti} \textsanskrit{ālokasandhibhāgā} ca \textsanskrit{kaṇṇabhāgā} ca \textsanskrit{antarabāhiravātapānakavāṭakāni} ca gabbhassa ca \textsanskrit{cattāro} \textsanskrit{koṇā} \textsanskrit{pamajjitabbāti} attho}, “\textit{\textsanskrit{Ālokasandhikaṇṇabhāga}} means the windows and the corners. The meaning is that he should sweep inside and outside the windows and the door and the four corners of the room.” } If the walls have been treated with red ocher and they’re moldy, he should moisten a cloth, wring it out, and wipe the walls. If the floor has been treated with a black finish and it’s moldy, he should moisten a cloth, wring it out, and wipe the floor. If the floor is untreated, he should sprinkle it with water and then sweep it, trying to avoid stirring up dust. He should look out for any trash and discard it. 

He\marginnote{25.16.1} should sun the floor cover, clean it, beat it, bring it back inside, and put it back as before. He should sun the bed supports, wipe them, bring them back inside, and put them back where they were. He should sun the bed, clean it, and beat it. Holding it low, he should carefully bring it back inside without scratching it or knocking it against the door or the door frame, and he should put it back as before. He should sun the bench, clean it, and beat it. Holding it low, he should carefully bring it back inside without scratching it or knocking it against the door or the door frame, and he should put it back as before. He should sun the mattress and the pillow, clean them, beat them, bring them back inside, and put them back as before. He should sun the sitting mat and the sheet, clean them, beat them, bring them back inside, and put them back as before. He should sun the spittoon, wipe it, bring it back inside, and put it back where it was. He should sun the leaning board, wipe it, bring it back inside, and put it back where it was. He should put away the bowl and robe. When putting away the bowl, he should hold the bowl in one hand, feel under the bed or the bench with the other, and then put it away. He shouldn’t put the bowl away on the bare floor. When putting away the robe, he should hold the robe in one hand, wipe the bamboo robe rack or the clothesline with the other, and then put it away by folding the robe over it, making the ends face the wall and the fold face out. 

If\marginnote{25.18.1} dusty winds are blowing from the east, he should close the windows on the eastern side. If dusty winds are blowing from the west, he should close the windows on the western side. If dusty winds are blowing from the north, he should close the windows on the northern side. If dusty winds are blowing from the south, he should close the windows on the southern side. If the weather is cold, he should open the windows during the day and close them at night. If the weather is hot, he should close the windows during the day and open them at night. 

If\marginnote{25.19.1} the yard is dirty, he should sweep it.\footnote{“Yard” renders \textit{\textsanskrit{pariveṇa}}. See Appendix of Technical Terms. } If the gatehouse is dirty, he should sweep it.\footnote{“Gatehouse” renders \textit{\textsanskrit{koṭṭhaka}}. See Appendix of Technical Terms. } If the assembly hall is dirty, he should sweep it. If the water-boiling shed is dirty, he should sweep it. If the restroom is dirty, he should sweep it. If there is no water for drinking, he should get some. If there is no water for washing, he should get some. If there is no water in the restroom ablutions pot, he should fill it. 

\subsection*{Spiritual support, etc.}

If\marginnote{25.20.1} the preceptor becomes discontent with the spiritual life, the student should send him away or have him sent away, or he should give him a teaching. If the preceptor becomes anxious, the student should dispel it or have it dispelled, or he should give him a teaching. If the preceptor has wrong view, the student should make him give it up or have someone else do it, or he should give him a teaching. If the preceptor has committed a heavy offense and deserves probation, the student should try to get the Sangha to give it to him. If the preceptor has committed a heavy offense and deserves to be sent back to the beginning, the student should try to get the Sangha to do it. If the preceptor has committed a heavy offense and deserves the trial period, the student should try to get the Sangha to give it to him. If the preceptor has committed a heavy offense and deserves rehabilitation, the student should try to get the Sangha to give it to him. 

If\marginnote{25.22.1} the Sangha wants to do a legal procedure against his preceptor—whether a procedure of condemnation, demotion, banishment, reconciliation, or ejection—\footnote{“Demotion” renders \textit{niyassa}. See Appendix of Technical Terms. } the student should make an effort to stop it or to reduce the penalty. But if the Sangha has already done a legal procedure against his preceptor—whether a procedure of condemnation, demotion, banishment, reconciliation, or ejection—the student should help the preceptor conduct himself properly and suitably so as to deserve to be released, and try to get the Sangha to lift that procedure.\footnote{The meaning of the first of these phrases, \textit{\textsanskrit{sammā} vattati}, is straightforward, but the last two, \textit{\textsanskrit{lomaṁ} \textsanskrit{pāteti}} and \textit{\textsanskrit{netthāraṁ} vattati}, are more difficult. Commenting on Bu Ss 13, Sp 1.435 says: \textit{Na \textsanskrit{lomaṁ} \textsanskrit{pātentīti} \textsanskrit{anulomapaṭipadaṁ} \textsanskrit{appaṭipajjanatāya} na \textsanskrit{pannalomā} honti. Na \textsanskrit{netthāraṁ} \textsanskrit{vattantīti} attano \textsanskrit{nittharaṇamaggaṁ} na \textsanskrit{paṭipajjanti}}, “\textit{Na \textsanskrit{lomaṁ} \textsanskrit{pātenti}}: because of their non-practicing in conformity with the path, their bodily hairs are not flat. \textit{Na \textsanskrit{netthāraṁ} vattanti}: they are not practicing the path for their own getting out (of the offense).” My rendering attempts to capture the meaning in a non-literal way. } 

If\marginnote{25.23.1} the preceptor’s robe needs washing, the student should do it himself, or he should make an effort to get it done. If the preceptor needs a robe, the student should make one himself, or he should make an effort to get one made. If the preceptor needs dye, the student should make it himself, or he should make an effort to get it made. If the preceptor’s robe needs dyeing, the student should do it himself, or he should make an effort to get it done. When he’s dyeing the robe, he should carefully and repeatedly turn it over, and shouldn’t go away while it’s still dripping. 

Without\marginnote{25.24.1} asking his preceptor for permission, he shouldn’t do any of the following: give away or receive a bowl; give away or receive a robe; give away or receive a requisite; cut anyone’s hair or have it cut; provide assistance to anyone or have assistance provided by anyone; do a service for anyone or get a service done by anyone; be the attendant monk for anyone or take anyone as his attendant monk; bring back almsfood for anyone or get almsfood brought back by anyone; enter the village, go to the charnel ground, or leave for another region. If his preceptor is sick, he should nurse him for as long as he lives, or he should wait until he’s recovered.” 

\scend{The proper conduct toward the preceptor is finished. }

\section*{16. Discussion of the proper conduct toward a student }

“And\marginnote{26.1.1} a preceptor should conduct himself properly toward his student. This is the proper conduct: 

A\marginnote{26.1.3} preceptor should help and take care of his student through recitation, questioning, and instruction. If the preceptor has a bowl, but not the student, the preceptor should give it to him,\footnote{Sp 3.67: \textit{Sace \textsanskrit{upajjhāyassa} patto \textsanskrit{hotīti} sace atirekapatto hoti. Esa nayo sabbattha}, “‘If the preceptor has a bowl’ means if the preceptor has an extra bowl. This method applies to everything (below).” } or he should make an effort to get him one. If the preceptor has a robe, but not the student, the preceptor should give it to him, or he should make an effort to get him one. If the preceptor has a requisite, but not the student, the preceptor should give it to him, or he should make an effort to get him one. 

\subsection*{Meals and almsround}

If\marginnote{26.2.1} the student is sick, the preceptor should get up at the appropriate time and give his student a tooth cleaner and water for rinsing the mouth, and he should prepare a seat for him. If there is congee, he should rinse a vessel and bring the congee to his student. When he has drunk the congee, the preceptor should give him water and receive the vessel. Holding it low, he should wash it carefully without scratching it and then put it away. When the student has gotten up, the preceptor should put away the seat. If the place is dirty, he should sweep it. 

If\marginnote{26.3.1} the student wants to enter the village, the preceptor should give him a sarong and receive the one he’s wearing in return. He should give him a belt. He should put the upper robes together, overlapping each other edge-to-edge, and then give them to him. He should rinse his student’s bowl and give it to him while wet. Before he’s due back, the preceptor should prepare a seat and set out a foot stool, a foot scraper, and water for washing the feet. He should go out to meet the student and receive his bowl and robe. He should give him a sarong and receive the one he’s wearing in return. If the robe is damp, he should sun it for a short while, but shouldn’t leave it in the heat. He should fold the robe, offsetting the edges by seven centimeters, so that the fold doesn’t become worn. He should place the belt in the fold. 

If\marginnote{26.3.7} there is almsfood and his student wants to eat, the preceptor should give him water and then the almsfood. He should ask his student if he wants water to drink. When the student has eaten, the preceptor should give him water and receive his bowl. Holding it low, he should wash it carefully without scratching it. He should then dry it and sun it for a short while, but shouldn’t leave it in the heat. The preceptor should put away the robe and bowl. When putting away the bowl, he should hold the bowl in one hand, feel under the bed or the bench with the other, and then put it away. He shouldn’t put the bowl away on the bare floor. When putting away the robe, he should hold the robe in one hand, wipe the bamboo robe rack or the clothesline with the other, and then put it away by folding the robe over it, making the ends face the wall and the fold face out. When the student has gotten up, the preceptor should put away the seat and also the foot stool, the foot scraper, and the water for washing the feet. If the place is dirty, he should sweep it. 

\subsection*{Bathing}

If\marginnote{26.5.1} the student wants to bathe, the preceptor should prepare a bath. If he wants a cold bath, he should prepare that; if he wants a hot bath, he should prepare that. 

If\marginnote{26.5.4} the student wants to take a sauna, the preceptor should knead bath powder, moisten the clay, take a sauna bench, and go to the sauna. After giving the student the sauna bench, receiving his robe, and putting it aside, he should give him the bath powder and the clay. If he’s able, he should enter the sauna. When entering the sauna, he should smear his face with clay, cover himself front and back, and then enter. He shouldn’t sit encroaching on the senior monks, and he shouldn’t block the junior monks from getting a seat. While in the sauna, he should provide assistance to his student. When leaving the sauna, he should take the sauna bench, cover himself front and back, and then leave. 

The\marginnote{26.6.5} preceptor should also provide assistance to his student in the water. When the preceptor has bathed, he should be the first to come out. He should dry himself and put on his sarong. He should then wipe the water off his student’s body, and he should give him his sarong and then his upper robe. Taking the sauna bench, he should be the first to return. He should prepare a seat, and also set out a foot stool, a foot scraper, and water for washing the feet. He should ask his student if he wants water to drink. 

\subsection*{The dwelling}

If\marginnote{26.7.1} the dwelling where the student is staying is dirty, the preceptor should clean it if he’s able. When he’s cleaning the dwelling, he should first take out the bowl and robe and put them aside. He should take out the sitting mat and the sheet and put them aside. He should take out the mattress and the pillow and put them aside. Holding the bed low, he should carefully take it out without scratching it or knocking it against the door or the door frame, and he should put it aside. Holding the bench low, he should carefully take it out without scratching it or knocking it against the door or the door frame, and he should put it aside. He should take out the bed supports and put them aside. He should take out the spittoon and put it aside. He should take out the leaning board and put it aside. After taking note of its position, he should take out the floor cover and put it aside. If the dwelling has cobwebs, he should first remove them from the ceiling cloth, and he should then wipe the windows and the corners of the room. If the walls have been treated with red ocher and they’re moldy, he should moisten a cloth, wring it out, and wipe the walls. If the floor has been treated with a black finish and it’s moldy, he should moisten a cloth, wring it out, and wipe the floor. If the floor is untreated, he should sprinkle it with water and then sweep it, trying to avoid stirring up dust. He should look out for any trash and discard it. 

He\marginnote{26.7.17} should sun the floor cover, clean it, beat it, bring it back inside, and put it back as before. He should sun the bed supports, wipe them, bring them back inside, and put them back where they were. He should sun the bed, clean it, and beat it. Holding it low, he should carefully bring it back inside without scratching it or knocking it against the door or the door frame, and he should put it back as before. He should sun the bench, clean it, and beat it. Holding it low, he should carefully bring it back inside without scratching it or knocking it against the door or the door frame, and he should put it back as before. He should sun the mattress and the pillow, clean them, beat them, bring them back inside, and put them back as before. He should sun the sitting mat and the sheet, clean them, beat them, bring them back inside, and put them back as before. He should sun the spittoon, wipe it, bring it back inside, and put it back where it was. He should sun the leaning board, wipe it, bring it back inside, and put it back where it was. He should put away the bowl and robe. When putting away the bowl, he should hold the bowl in one hand, feel under the bed or the bench with the other, and then put it away. He shouldn’t put the bowl away on the bare floor. When putting away the robe, he should hold the robe in one hand, wipe the bamboo robe rack or the clothesline with the other, and then put it away by folding the robe over it, making the ends face the wall and the fold face out. 

If\marginnote{26.7.29} dusty winds are blowing from the east, he should close the windows on the eastern side. If dusty winds are blowing from the west, he should close the windows on the western side. If dusty winds are blowing from the north, he should close the windows on the northern side. If dusty winds are blowing from the south, he should close the windows on the southern side. If the weather is cold, he should open the windows during the day and close them at night. If the weather is hot, he should close the windows during the day and open them at night. 

If\marginnote{26.7.35} the yard is dirty, he should sweep it. If the gatehouse is dirty, he should sweep it. If the assembly hall is dirty, he should sweep it. If the water-boiling shed is dirty, he should sweep it. If the restroom is dirty, he should sweep it. If there is no water for drinking, he should get some. If there is no water for washing, he should get some. If there is no water in the restroom ablutions pot, he should fill it. 

\subsection*{Spiritual support, etc.}

If\marginnote{26.8.1} the student becomes discontent with the spiritual life, the preceptor should send him away or have him sent away, or he should give him a teaching. If the student becomes anxious, the preceptor should dispel it or have it dispelled, or he should give him a teaching. If the student has wrong view, the preceptor should make him give it up or have someone else do it, or he should give him a teaching. If the student has committed a heavy offense and deserves probation, the preceptor should try to get the Sangha to give it to him. If the student has committed a heavy offense and deserves to be sent back to the beginning, the preceptor should try to get the Sangha to do it. If the student has committed a heavy offense and deserves the trial period, the preceptor should try to get the Sangha to give it to him. If the student has committed a heavy offense and deserves rehabilitation, the preceptor should try to get the Sangha to give it to him. 

If\marginnote{26.10.1} the Sangha wants to do a legal procedure against his student—whether a procedure of condemnation, demotion, banishment, reconciliation, or ejection—the preceptor should make an effort to stop it or to reduce the penalty. But if the Sangha has already done a legal procedure against his student—whether a procedure of condemnation, demotion, banishment, reconciliation, or ejection—the preceptor should help the student conduct himself properly and suitably so as to deserve to be released, and try to get the Sangha to lift that procedure. 

If\marginnote{26.11.1} the student’s robe needs washing, the preceptor should show him how to do it, or he should make an effort to get it done. If the student needs a robe, the preceptor should show him how to make one, or he should make an effort to get one made. If the student needs dye, the preceptor should show him how to make it, or he should make an effort to get it made. If the student’s robe needs dyeing, the preceptor should show him how to do it, or he should make an effort to get it done. When he’s dyeing the robe, he should carefully and repeatedly turn it over, and shouldn’t go away while it’s still dripping. If his student is sick, he should nurse him for as long as he lives, or he should wait until he’s recovered.” 

\scend{The proper conduct toward a student is finished. }

\section*{17. Discussion on dismissal }

On\marginnote{27.1.1} a later occasion the students did not conduct themselves properly toward their preceptors. The monks of few desires complained and criticized them, “How can students not conduct themselves properly toward their preceptors?” They told the Buddha. … “Is it true, monks, that students are acting like this?” 

“It’s\marginnote{27.1.6} true, sir.” 

The\marginnote{27.1.7} Buddha rebuked them … “How can students not conduct themselves properly toward their preceptors?” … After rebuking them … he gave a teaching and addressed the monks: 

\scrule{“A student should conduct himself properly toward his preceptor. If he doesn’t, he commits an offense of wrong conduct.” }

They\marginnote{27.2.1} still did not conduct themselves properly. They told the Buddha. 

\scrule{“You should dismiss one who doesn’t conduct himself properly. }

And\marginnote{27.2.4} this is how he should be dismissed. If the preceptor conveys the following by body, by speech, or by body and speech: ‘I dismiss you;’ ‘Don’t come back here;’ ‘Remove your bowl and robe;’ or, ‘You shouldn’t attend on me’—then the student has been dismissed. If he doesn’t convey this by body, by speech, or by body and speech, then the student hasn’t been dismissed.” 

Students\marginnote{27.3.1} who had been dismissed did not ask for forgiveness. They told the Buddha. 

\scrule{“You should ask for forgiveness.” }

They\marginnote{27.3.4} still did not ask for forgiveness. They told the Buddha. 

\scrule{“One who has been dismissed should ask for forgiveness. If he doesn’t, he commits an offense of wrong conduct.” }

Preceptors\marginnote{27.4.1} who were asked for forgiveness did not forgive. They told the Buddha. 

\scrule{“You should forgive.” }

They\marginnote{27.4.4} still did not forgive. The students left, disrobed, and joined the monastics of other religions.\footnote{“Disrobed” renders \textit{vibbhamanti}. See Appendix of Technical Terms. } They told the Buddha. 

\scrule{“When asked for forgiveness, you should forgive. If you don’t, you commit an offense of wrong conduct.” }

Preceptors\marginnote{27.5.1} dismissed students who were conducting themselves properly and did not dismiss those who were not. They told the Buddha. 

\scrule{“You shouldn’t dismiss someone who is conducting himself properly. If you do, you commit an offense of wrong conduct. }

\scrule{And you should dismiss someone who isn’t conducting himself properly. If you don’t, you commit an offense of wrong conduct. }

If\marginnote{27.6.1} a student has five qualities, he should be dismissed: he doesn’t have much affection for his preceptor; he doesn’t have much confidence in his preceptor; he doesn’t have much conscience in regard to his preceptor; he doesn’t have much respect for his preceptor; he hasn’t developed his mind much under his preceptor. 

If\marginnote{27.6.4} a student has five qualities, he shouldn’t be dismissed: he has much affection for his preceptor; he has much confidence in his preceptor; he has much conscience in regard to his preceptor; he has much respect for his preceptor; he has developed his mind much under his preceptor. 

If\marginnote{27.7.1} a student has five qualities, he deserves to be dismissed: he doesn’t have much affection for his preceptor; he doesn’t have much confidence in his preceptor; he doesn’t have much conscience in regard to his preceptor; he doesn’t have much respect for his preceptor; he hasn’t developed his mind much under his preceptor. 

If\marginnote{27.7.4} a student has five qualities, he doesn’t deserve to be dismissed: he has much affection for his preceptor; he has much confidence in his preceptor; he has much conscience in regard to his preceptor; he has much respect for his preceptor; he has developed his mind much under his preceptor. 

If\marginnote{27.8.1} a student has five qualities, the preceptor is at fault if he doesn’t dismiss him, but not if he does: the student doesn’t have much affection for his preceptor; he doesn’t have much confidence in his preceptor; he doesn’t have much conscience in regard to his preceptor; he doesn’t have much respect for his preceptor; he hasn’t developed his mind much under his preceptor. 

If\marginnote{27.8.4} a student has five qualities, the preceptor is at fault if he dismisses him, but not if he doesn’t: the student has much affection for his preceptor; he has much confidence in his preceptor; he has much conscience in regard to his preceptor; he has much respect for his preceptor; he has developed his mind much under his preceptor.” 

On\marginnote{28.1.1} one occasion a brahmin went to the monks and asked for the going forth, but the monks declined. As a result, he became thin, haggard, and pale, with veins protruding all over his body. The Buddha saw him, and he asked the monks, “Why is that brahmin looking so sickly?” They told him what had happened. 

The\marginnote{28.2.1} Buddha said, “Does anyone remember any act of service from that brahmin?” 

Venerable\marginnote{28.2.3} \textsanskrit{Sāriputta} replied, “I do, sir.” 

“What\marginnote{28.2.5} service do you remember, \textsanskrit{Sāriputta}?” 

“When\marginnote{28.2.6} I was walking for almsfood here in \textsanskrit{Rājagaha}, that brahmin gave a ladleful of food.” 

“Good,\marginnote{28.3.1} good, \textsanskrit{Sāriputta}, superior people have gratitude. Well then, \textsanskrit{Sāriputta}, give that brahmin the going forth and the full ordination.” 

“But\marginnote{28.3.3} how should I do it?” 

The\marginnote{28.3.4} Buddha then gave a teaching and addressed the monks: 

\scrule{“From today I rescind the full ordination through the taking of the three refuges. Instead you should give the full ordination through a legal procedure consisting of one motion and three announcements. }

And\marginnote{28.4.2} the ordination should be done like this. A competent and capable monk should inform the Sangha: 

‘Please,\marginnote{28.4.4} venerables, I ask the Sangha to listen. So-and-so wants the full ordination with venerable so-and-so.\footnote{The Pali reads: \textit{\textsanskrit{Ayaṁ} \textsanskrit{itthannāmo} \textsanskrit{itthannāmassa} \textsanskrit{āyasmato} \textsanskrit{upasampadāpekkho}}. Taking the genitive case here to be the agent genitive, which seems to be the most obvious reading, this would mean, “So-and-so who is seeking to be fully ordained \emph{by} venerable so-and-so.” But it is the Sangha that ordains, not individuals, and so this translation does not seem quite right. According to Vmv 3.126 this phrase should be understood by means of this example: \textit{\textsanskrit{Ayaṁ} buddharakkhito \textsanskrit{āyasmato} dhammarakkhitassa \textsanskrit{saddhivihārikabhūto} \textsanskrit{upasampadāpekkho}}, “This Buddharakkhita, who is seeking the full ordination, is the student of Venerable Dhammarakkhita.” I have followed this interpretation, and thus my translation “with venerable so-and-so”. } If the Sangha is ready, it should give the full ordination to so-and-so with so-and-so as his preceptor. This is the motion. 

Please,\marginnote{28.5.1} venerables, I ask the Sangha to listen. So-and-so wants the full ordination with venerable so-and-so. The Sangha gives the full ordination to so-and-so with so-and-so as his preceptor. Any monk who approves of giving the full ordination to so-and-so with so-and-so as his preceptor should remain silent. Any monk who doesn’t approve should speak up. 

For\marginnote{28.5.6} the second time, I speak on this matter. Please, venerables, I ask the Sangha to listen. So-and-so wants the full ordination with venerable so-and-so. The Sangha gives the full ordination to so-and-so with so-and-so as his preceptor. Any monk who approves of giving the full ordination to so-and-so with so-and-so as his preceptor should remain silent. Any monk who doesn’t approve should speak up. 

For\marginnote{28.6.1} the third time, I speak on this matter. Please, venerables, I ask the Sangha to listen. So-and-so wants the full ordination with venerable so-and-so. The Sangha gives the full ordination to so-and-so with so-and-so as his preceptor. Any monk who approves of giving the full ordination to so-and-so with so-and-so as his preceptor should remain silent. Any monk who doesn’t approve should speak up. 

The\marginnote{28.6.7} Sangha has given the full ordination to so-and-so with so-and-so as his preceptor. The Sangha approves and is therefore silent. I’ll remember it thus.’” 

On\marginnote{29.1.1} a later occasion, a monk misbehaved immediately after his full ordination. The monks told him, “Don’t do that. It’s not allowable.” 

“But\marginnote{29.1.5} I didn’t ask you to ordain me. Why did you ordain me without being asked?” They told the Buddha. 

\scrule{“You shouldn’t give the full ordination to someone who hasn’t asked. If you do, you commit an offense of wrong conduct. I allow you to give the full ordination to someone who has asked. }

And\marginnote{29.2.1} this is how they should ask. After approaching the Sangha, the one who wants the full ordination should arrange his upper robe over one shoulder and pay respect at the feet of the monks. He should then squat on his heels, raise his joined palms, and say: ‘Venerables, I ask the Sangha for the full ordination. Please lift me up out of compassion.’ And he should ask a second and a third time. A competent and capable monk should then inform the Sangha: 

‘Please,\marginnote{29.3.2} venerables, I ask the Sangha to listen. So-and-so wants the full ordination with venerable so-and-so. So-and-so is asking the Sangha for the full ordination with so-and-so as his preceptor. If the Sangha is ready, it should give the full ordination to so-and-so with so-and-so as his preceptor. This is the motion. 

Please,\marginnote{29.4.1} venerables, I ask the Sangha to listen. So-and-so wants the full ordination with venerable so-and-so. So-and-so is asking the Sangha for the full ordination with so-and-so as his preceptor. The Sangha gives the full ordination to so-and-so with so-and-so as his preceptor. Any monk who approves of giving the full ordination to so-and-so with so-and-so as his preceptor should remain silent. Any monk who doesn’t approve should speak up. 

For\marginnote{29.4.7} the second time, I speak on this matter. … For the third time, I speak on this matter. … 

The\marginnote{29.4.9} Sangha has given the full ordination to so-and-so with so-and-so as his preceptor. The Sangha approves and is therefore silent. I’ll remember it thus.’” 

At\marginnote{30.1.1} that time in \textsanskrit{Rājagaha}, there was a succession of fine meals. A certain brahmin thought, “These Sakyan monastics have pleasant habits and a happy life. They eat nice food and sleep in beds sheltered from the wind. Why don’t I go forth with the Sakyan monastics?” 

Then\marginnote{30.1.5} that brahmin went to the monks and asked for the going forth. The monks gave him the going forth and the full ordination. When he had gone forth, that succession of meals came to a stop. The monks said to him, “Come, let’s walk for alms.” 

“I\marginnote{30.2.5} didn’t go forth to walk for alms. If you give me some, I’ll eat it. If not, I’ll disrobe.” 

“But\marginnote{30.2.7} did you go forth for the sake of your stomach?” 

“Yes.”\marginnote{30.2.8} 

The\marginnote{30.3.1} monks of few desires complained and criticized him, “How could a monk go forth on this well-proclaimed spiritual path for the sake of his stomach?” 

They\marginnote{30.3.3} told the Buddha. … “Is it true, monk, that you did this?” 

“It’s\marginnote{30.3.5} true, sir.” 

The\marginnote{30.3.6} Buddha rebuked him … “Foolish man, how could you go forth on this well-proclaimed spiritual path for the sake of your stomach? This will affect people’s confidence …” After rebuking him … he gave a teaching and addressed the monks: 

\scrule{“When you are giving the full ordination, you should point out the four supports: }

\begin{enumerate}%
\item One gone forth is supported by almsfood. You should persevere with this for life. There are these additional allowances: a meal for the Sangha, a meal for designated monks, an invitational meal, a meal for which lots are drawn, a half-monthly meal, a meal on the observance day, and a meal on the day after the observance day. %
\item One gone forth is supported by rag-robes. You should persevere with this for life. There are these additional allowances: linen, cotton, silk, wool, sunn hemp, and hemp. %
\item One gone forth is supported by the foot of a tree as a resting place. You should persevere with this for life.\footnote{“Resting place” renders \textit{\textsanskrit{senāsana}}. See Appendix of Technical Terms. } There are these additional allowances: a dwelling, a stilt house, and a cave.\footnote{Apart from the \textit{\textsanskrit{vihāra}}, “a dwelling”, and the \textit{\textsanskrit{guhā}}, “a cave”, the Pali mentions three kinds of buildings, the \textit{\textsanskrit{aḍḍhayoga}}, the \textit{\textsanskrit{pāsāda}}, and the \textit{hammiya}, all of which, according to the commentaries, are different kinds of \textit{\textsanskrit{pāsāda}}, “stilt houses”. Rather than try to differentiate between these buildings, which is unlikely to be useful from a practical perspective, I have instead grouped them together as “stilt house”. Here is what the commentaries have to say. Sp 4.294: \textit{\textsanskrit{Aḍḍhayogoti} \textsanskrit{supaṇṇavaṅkagehaṁ}}, “An \textit{\textsanskrit{aḍḍhayoga}} is a house bent like a \textit{\textsanskrit{supaṇṇa}}.” Sp-\textsanskrit{ṭ} 4.294 clarifies: \textit{\textsanskrit{Supaṇṇavaṅkagehanti} \textsanskrit{garuḷapakkhasaṇṭhānena} \textsanskrit{katagehaṁ}}, “\textit{\textsanskrit{Supaṇṇavaṅkageha}}: a house made in the shape of the wings of a \textit{\textsanskrit{garuḷa}}.” A \textit{\textsanskrit{garuḷa}}, better known in its Sanskrit form \textit{\textsanskrit{garuḍa}}, is a mythological bird. Sp 4.294 continues: \textit{\textsanskrit{Pāsādoti} \textsanskrit{dīghapāsādo}. Hammiyanti \textsanskrit{upariākāsatale} \textsanskrit{patiṭṭhitakūṭāgāro} \textsanskrit{pāsādoyeva}}, “A \textit{\textsanskrit{pāsāda}} is a long stilt house. A \textit{hammiya} is just a \textit{\textsanskrit{pāsāda}} that has an upper room on top of its flat roof.” At Sp-\textsanskrit{ṭ} 3.74, however, we find slightly different explanations. It seems clear, however, that all three are stilt houses and that they are distinguished according to their shape and the kind of roof they possess. } %
\item One gone forth is supported by medicine of fermented urine. You should persevere with this for life. There are these additional allowances: ghee, butter, oil, honey, and syrup.”\footnote{I. B. Horner translates \textit{\textsanskrit{phāṇita}} as “molasses”, which doesn’t quite hit the mark. SED defines \textit{\textsanskrit{phāṇita}} as “the inspissated juice of the sugar cane or other plants”, in other words, “cane syrup”. According to the commentary at Sp 1.623 it can be either cooked or uncooked, the difference presumably being whether the juice is raw or concentrated. “Syrup” seems closer to the mark than “molasses”. } %
\end{enumerate}

\scend{The fifth section for recitation on the proper conduct toward the preceptor is finished. }

\section*{18. Discussion of the proper conduct toward a teacher }

Soon\marginnote{31.1.1} afterwards a young brahmin went to the monks and asked for the going forth. The monks told him about the four supports. He said, “Venerables, if you had told me about this after my going forth, I would have been fine. But now I won’t go forth, for these supports are disgusting and repulsive to me.” They told the Buddha. 

\scrule{“You shouldn’t point out the supports beforehand. If you do, you commit an offense of wrong conduct. You should point out the supports immediately after the full ordination.” }

At\marginnote{31.2.1} that time, monks in groups of two and three gave the full ordination. They told the Buddha. 

\scrule{“You shouldn’t give the full ordination in groups of less than ten. If you do, you commit an offense of wrong conduct. You should give the full ordination in groups of ten or more than ten.” }

At\marginnote{31.3.1} that time monks who only had one or two years of seniority gave the full ordination, among them Venerable Upasena of \textsanskrit{Vaṅganta}. 

After\marginnote{31.3.3} completing the rainy-season residence, he had two years of seniority and his student had one. The two of them went to the Buddha, bowed, and sat down. Since it is the custom for Buddhas to greet newly-arrived monks, the Buddha said to Upasena, “I hope you’re keeping well, monk, I hope you’re getting by?  I hope you’re not tired from traveling?” 

“I’m\marginnote{31.4.4} keeping well, sir, I’m getting by. I’m not tired from traveling.” 

When\marginnote{31.4.6} Buddhas know what is going on, sometimes they ask and sometimes not. They know the right time to ask and when not to ask. Buddhas ask when it is beneficial, otherwise not, for Buddhas are incapable of doing what is unbeneficial.\footnote{“Incapable of doing” renders \textit{\textsanskrit{setughāta}}, literally, “destroyed the bridge”. Sp 1.16: \textit{Setu vuccati maggo, maggeneva \textsanskrit{tādisassa} vacanassa \textsanskrit{ghāto}, samucchedoti \textsanskrit{vuttaṁ} hoti}, “The path is called the bridge. What is said is that there is the destruction and cutting off of such speech by the path.” The commentary seems to take \textit{setu}, “bridge”, as a reference to the eightfold path. I prefer to understand “bridge” as a metaphor for access, that is, the Buddhas no longer have the possibility of doing what is unbeneficial. } Buddhas question the monks for two reasons: to give a teaching or to lay down a training rule. 

The\marginnote{31.5.1} Buddha said to Upasena, “How long have you been a monk?” 

“Two\marginnote{31.5.2} years, sir.” 

“And\marginnote{31.5.3} this monk?” 

“One\marginnote{31.5.4} year.” 

“And\marginnote{31.5.5} what’s his relationship to you?” 

“He’s\marginnote{31.5.6} my student.” 

The\marginnote{31.5.7} Buddha rebuked him, “It’s not suitable, foolish man, it’s not proper, it’s not worthy of a monastic, it’s not allowable, it’s not to be done. You ought to be taught and instructed by others. What, then, makes you think that you should teach and instruct another person? You have turned to indulgence too readily, that is, by forming a group. This will affect people’s confidence …” After rebuking him … he gave a teaching and addressed the monks: 

\scrule{“You shouldn’t give the full ordination if you have less than ten years of seniority. If you do, you commit an offense of wrong conduct. I allow you to give the full ordination if you have ten or more years of seniority.” }

Then,\marginnote{31.6.1} once they had ten years of seniority, ignorant and incompetent monks gave the full ordination. As a result there were ignorant preceptors with knowledgeable students, incompetent preceptors with competent students, uneducated preceptors with learned students, and foolish preceptors with wise students. A monk who had been a monastic in another religion even refuted his preceptor, despite being legitimately corrected by him. He then returned to that religious community.\footnote{“Correct” renders \textit{\textsanskrit{vuccamāno}}. See \textit{vadati} in Appendix of Technical Terms. } 

The\marginnote{31.7.1} monks of few desires complained and criticized them, “How can ignorant and incompetent monks give the full ordination just because they have ten years of seniority? There are ignorant preceptors with knowledgeable students, incompetent preceptors with competent students, uneducated preceptors with learned students, and foolish preceptors with wise students.” 

They\marginnote{31.7.4} told the Buddha. He said, “Is it true, monks, that this is happening?” 

“It’s\marginnote{31.7.9} true, sir.” 

The\marginnote{31.8.1} Buddha rebuked them … “How can those foolish men give the full ordination just because they have ten years of seniority? The consequences are evident. This will affect people’s confidence …” After rebuking them … he gave a teaching and addressed the monks: 

\scrule{“An ignorant and incompetent monk shouldn’t give the full ordination. If he does, he commits an offense of wrong conduct. I allow a competent and capable monk who has ten or more years of seniority to give the full ordination.” }

At\marginnote{32.1.1} that time there were preceptors who went away, disrobed, died, or joined another religion or sect, and as a result their students were not being instructed.\footnote{Sp 3.77: \textit{\textsanskrit{Pakkhasaṅkantesūti} \textsanskrit{titthiyapakkhasaṅkantesu}}, “\textit{\textsanskrit{Pakkhasaṅkantesu}} means those who have joined a group of monastics of another religion.” Yet the idea of \textit{pakkha} also refers to groups or factions within the Sangha, for instance, when the Sangha is split into different communities (\textit{\textsanskrit{nānāsaṁvāsa}}) that no longer perform legal procedures together. As such, it is a term for a separate sect of Buddhism. }  When walking for almsfood, they were shabbily dressed and improper in appearance.  While people were eating, they held out their almsbowls to receive leftovers, even right over their food, whether it was cooked or fresh food, delicacies or drinks. They ate bean curry and rice that they themselves had asked for, and they were noisy in the dining hall. 

People\marginnote{32.1.6} complained and criticized them, “How can the Sakyan monastics act like this? They are just like brahmins at a brahminical meal.” 

The\marginnote{32.1.11} monks heard the complaints of those people. … They then told the Buddha. “Is it true, monks … ?” 

“It’s\marginnote{32.1.14} true, sir.” … 

After\marginnote{32.1.15} rebuking them, the Buddha gave a teaching and addressed the monks: 

\scrule{“There should be a teacher. }

The\marginnote{32.1.17} teacher should think of his pupil as a son and the pupil his teacher as a father. In this way they will respect, esteem, and be considerate toward each other, and they will grow and reach greatness on this spiritual path. 

\scrule{You should live with formal support for ten years. And I allow a monk of ten years’ seniority to give such support.\footnote{“Formal support” renders \textit{\textsanskrit{nissāya}}. See Appendix of Technical Terms. } }

A\marginnote{32.2.1} teacher should be chosen like this. After arranging his upper robe over one shoulder, a pupil should pay respect at the feet of a potential teacher. He should then squat on his heels, raise his joined palms, and say, ‘Venerable, please be my teacher. I wish to live with formal support from you.’ And he should repeat this a second and a third time. If the other conveys the following by body, by speech, or by body and speech: ‘Yes;’ ‘No problem;’ ‘It’s suitable;’ ‘It’s appropriate;’ or, ‘Carry on with inspiration’—then a teacher has been chosen. If the other doesn’t convey this by body, by speech, or by body and speech, then a teacher hasn’t been chosen. 

“A\marginnote{32.3.1} pupil should conduct himself properly toward his teacher. This is the proper conduct: 

\subsection*{Meals and almsround}

Having\marginnote{32.3.3.1} gotten up at the appropriate time, the pupil should remove his sandals, and arrange his upper robe over one shoulder. He should then give his teacher a tooth cleaner and water for rinsing the mouth, and he should prepare a seat for him. If there is congee, he should rinse a vessel and bring the congee to his teacher. When he has drunk the congee, the pupil should give him water and receive the vessel. Holding it low, he should wash it carefully without scratching it and then put it away. When the teacher has gotten up, the pupil should put away the seat. If the place is dirty, he should sweep it. 

If\marginnote{32.3.8} the teacher wants to enter the village, the pupil should give him a sarong and receive the one he’s wearing in return. He should give him a belt. He should put the upper robes together, overlapping each other edge-to-edge, and then give them to him. He should rinse his teacher’s bowl and give it to him while wet. If the teacher wants an attendant, the pupil should put on his sarong evenly all around, covering the navel and the knees. He should put on a belt. Putting the upper robes together, overlapping each other edge-to-edge, he should put them on and fasten the toggle. He should rinse his bowl, bring it along, and be his teacher’s attendant. 

He\marginnote{32.3.10} shouldn’t walk too far behind his teacher or too close to him. He should receive the contents of his bowl. He shouldn’t interrupt his teacher when he’s speaking. But if the teacher’s speech is bordering on an offense, he should stop him. 

When\marginnote{32.3.13} returning, the pupil should go first to prepare a seat and to set out a foot stool, a foot scraper, and water for washing the feet. He should go out to meet the teacher and receive his bowl and robe. He should give him a sarong and receive the one he’s wearing in return. If the robe is damp, he should sun it for a short while, but shouldn’t leave it in the heat. He should fold the robe, offsetting the edges by seven centimeters, so that the fold doesn’t become worn. He should place the belt in the fold. 

If\marginnote{32.3.19} there is almsfood and his teacher wants to eat, the pupil should give him water and then the almsfood. He should ask his teacher if he wants water to drink. When the teacher has eaten, the pupil should give him water and receive his bowl. Holding it low, he should wash it carefully without scratching it. He should then dry it and sun it for a short while, but shouldn’t leave it in the heat. 

The\marginnote{32.3.22} pupil should put away the robe and bowl. When putting away the bowl, he should hold the bowl in one hand, feel under the bed or the bench with the other, and then put it away. He shouldn’t put the bowl away on the bare floor. When putting away the robe, he should hold the robe in one hand, wipe the bamboo robe rack or the clothesline with the other, and then put it away by folding the robe over it, making the ends face the wall and the fold face out. When the teacher has gotten up, the pupil should put away the seat and also the foot stool, the foot scraper, and the water for washing the feet. If the place is dirty, he should sweep it. 

\subsection*{Bathing}

If\marginnote{32.3.28.1} the teacher wants to bathe, the pupil should prepare a bath. If he wants a cold bath, he should prepare that; if he wants a hot bath, he should prepare that. 

If\marginnote{32.3.31} the teacher wants to take a sauna, the pupil should knead bath powder, moisten the clay, take a sauna bench, and follow behind his teacher. After giving the teacher the sauna bench, receiving his robe, and putting it aside, he should give him the bath powder and the clay. If he’s able, he should enter the sauna. When entering the sauna, he should smear his face with clay, cover himself front and back, and then enter. He shouldn’t sit encroaching on the senior monks, and he shouldn’t block the junior monks from getting a seat. While in the sauna, he should provide assistance to his teacher. When leaving the sauna, he should take the sauna bench, cover himself front and back, and then leave. 

He\marginnote{32.3.38} should also provide assistance to his teacher in the water. When he has bathed, he should be the first to come out. He should dry himself and put on his sarong. He should then wipe the water off his teacher’s body, and he should give him his sarong and then his upper robe. Taking the sauna bench, he should be first to return. He should prepare a seat, and also set out a foot stool, a foot scraper, and water for washing the feet. He should ask his teacher if he wants water to drink. If the teacher wants him to recite, he should do so. If the teacher wants to question him, he should be questioned. 

\subsection*{The dwelling}

If\marginnote{32.3.43.1} the dwelling where the teacher is staying is dirty, the pupil should clean it if he’s able. When he’s cleaning the dwelling, he should first take out the bowl and robe and put them aside. He should take out the sitting mat and the sheet and put them aside. He should take out the mattress and the pillow and put them aside. Holding the bed low, he should carefully take it out without scratching it or knocking it against the door or the door frame, and he should put it aside. Holding the bench low, he should carefully take it out without scratching it or knocking it against the door or the door frame, and he should put it aside. He should take out the bed supports and put them aside. He should take out the spittoon and put it aside. He should take out the leaning board and put it aside. After taking note of its position, he should take out the floor cover and put it aside. If the dwelling has cobwebs, he should first remove them from the ceiling cloth, and he should then wipe the windows and the corners of the room. If the walls have been treated with red ocher and they’re moldy, he should moisten a cloth, wring it out, and wipe the walls. If the floor has been treated with a black finish and it’s moldy, he should moisten a cloth, wring it out, and wipe the floor. If the floor is untreated, he should sprinkle it with water and then sweep it, trying to avoid stirring up dust. He should look out for any trash and discard it. 

He\marginnote{32.3.59} should sun the floor cover, clean it, beat it, bring it back inside, and put it back as before. He should sun the bed supports, wipe them, bring them back inside, and put them back where they were. He should sun the bed, clean it, and beat it. Holding it low, he should carefully bring it back inside without scratching it or knocking it against the door or the door frame, and he should put it back as before. He should sun the bench, clean it, and beat it. Holding it low, he should carefully bring it back inside without scratching it or knocking it against the door or the door frame, and he should put it back as before. He should sun the mattress and the pillow, clean them, beat them, bring them back inside, and put them back as before. He should sun the sitting mat and the sheet, clean them, beat them, bring them back inside, and put them back as before. He should sun the spittoon, wipe it, bring it back inside, and put it back where it was. He should sun the leaning board, wipe it, bring it back inside, and put it back where it was. He should put away the bowl and robe. When putting away the bowl, he should hold the bowl in one hand, feel under the bed or the bench with the other, and then put it away. He shouldn’t put the bowl away on the bare floor. When putting away the robe, he should hold the robe in one hand, wipe the bamboo robe rack or the clothesline with the other, and then put it away by folding the robe over it, making the ends face the wall and the fold face out. 

If\marginnote{32.3.71} dusty winds are blowing from the east, he should close the windows on the eastern side. If dusty winds are blowing from the west, he should close the windows on the western side. If dusty winds are blowing from the north, he should close the windows on the northern side. If dusty winds are blowing from the south, he should close the windows on the southern side. If the weather is cold, he should open the windows during the day and close them at night. If the weather is hot, he should close the windows during the day and open them at night. 

If\marginnote{32.3.77} the yard is dirty, he should sweep it. If the gatehouse is dirty, he should sweep it. If the assembly hall is dirty, he should sweep it. If the water-boiling shed is dirty, he should sweep it. If the restroom is dirty, he should sweep it. If there is no water for drinking, he should get some. If there is no water for washing, he should get some. If there is no water in the restroom ablutions pot, he should fill it. 

\subsection*{Spiritual support, etc.}

If\marginnote{32.3.85.1} the teacher becomes discontent with the spiritual life, the pupil should send him away or have him sent away, or he should give him a teaching. If the teacher becomes anxious, the pupil should dispel it or have it dispelled, or he should give him a teaching. If the teacher has wrong view, the pupil should make him give it up or have someone else do it, or he should give him a teaching. If the teacher has committed a heavy offense and deserves probation, the pupil should try to get the Sangha to give it to him. If the teacher has committed a heavy offense and deserves to be sent back to the beginning, the pupil should try to get the Sangha to do it. If the teacher has committed a heavy offense and deserves the trial period, the pupil should try to get the Sangha to give it to him. If the teacher has committed a heavy offense and deserves rehabilitation, the pupil should try to get the Sangha to give it to him. 

If\marginnote{32.3.96} the Sangha wants to do a legal procedure against his teacher—whether a procedure of condemnation, demotion, banishment, reconciliation, or ejection—the pupil should make an effort to stop it or to reduce the penalty. But if the Sangha has already done a legal procedure against his teacher—whether a procedure of condemnation, demotion, banishment, reconciliation, or ejection—the pupil should help the teacher conduct himself properly and suitably so as to deserve to be released, and try to get the Sangha to lift that procedure. 

If\marginnote{32.3.100} the teacher’s robe needs washing, the pupil should do it himself, or he should make an effort to get it done. If the teacher needs a robe, the pupil should make one himself, or he should make an effort to get one made. If the teacher needs dye, the pupil should make it himself, or he should make an effort to get it made. If the teacher’s robe needs dyeing, the pupil should do it himself, or he should make an effort to get it done. When he’s dyeing the robe, he should carefully and repeatedly turn it over, and shouldn’t go away while it’s still dripping. 

Without\marginnote{32.3.109} asking his teacher for permission, he shouldn’t do any of the following: give away or receive a bowl; give away or receive a robe; give away or receive a requisite; cut anyone’s hair or have it cut; provide assistance to anyone or have assistance provided by anyone; do a service for anyone or get a service done by anyone; be the attendant monk for anyone or take anyone as his attendant monk; bring back almsfood for anyone or get almsfood brought back by anyone; enter the village, go to the charnel ground, or leave for another region. If his teacher is sick, he should nurse him for as long as he lives, or he should wait until he’s recovered.” 

\scend{The proper conduct toward a teacher is finished. }

\section*{19. Discussion of the proper conduct toward a pupil }

“And\marginnote{33.1.1} a teacher should conduct himself properly toward his pupil. This is the proper conduct: 

A\marginnote{33.1.3} teacher should help and take care of his pupil through recitation, questioning, and instruction. If the teacher has a bowl, but not the pupil, the teacher should give it to him,\footnote{According to the commentary, Sp 3.77, this should be understood in the same way as with the preceptor, for which see comment at \href{https://suttacentral.net/pli-tv-kd1/en/brahmali\#26.1.4}{Kd 1:26.1.4}. } or he should make an effort to get him one. If the teacher has a robe, but not the pupil, the teacher should give it to him, or he should make an effort to get him one. If the teacher has a requisite, but not the pupil, the teacher should give it to him, or he should make an effort to get him one. 

\subsection*{Meals and almsround}

If\marginnote{33.1.10.1} the pupil is sick, the teacher should get up at the appropriate time and give his pupil a tooth cleaner and water for rinsing the mouth, and he should prepare a seat for him. If there is congee, he should rinse a vessel and bring the congee to his pupil. When he has drunk the congee, the teacher should give him water and receive the vessel. Holding it low, he should wash it carefully without scratching it and then put it away. When the pupil has gotten up, the teacher should put away the seat. If the place is dirty, he should sweep it. 

If\marginnote{33.1.15} the pupil wants to enter the village, the teacher should give him a sarong and receive the one he’s wearing in return. He should give him a belt. He should put the upper robes together, overlapping each other edge-to-edge, and then give them to him. He should rinse his pupil’s bowl and give it to him while wet. 

Before\marginnote{33.1.16} he’s due back, the teacher should prepare a seat and set out a foot stool, a foot scraper, and water for washing the feet. He should go out to meet the pupil and receive his bowl and robe. He should give him a sarong and receive the one he’s wearing in return. If the robe is damp, he should sun it for a short while, but shouldn’t leave it in the heat. He should fold the robe, offsetting the edges by seven centimeters, so that the fold doesn’t become worn. He should place the belt in the fold. 

If\marginnote{33.1.22} there is almsfood and his pupil wants to eat, the teacher should give him water and then the almsfood. He should ask his pupil if he wants water to drink. When the pupil has eaten, the teacher should give him water and receive his bowl. Holding it low, he should wash it carefully without scratching it. He should then dry it and sun it for a short while, but shouldn’t leave it in the heat. The teacher should put away the robe and bowl. When putting away the bowl, he should hold the bowl in one hand, feel under the bed or the bench with the other, and then put it away. He shouldn’t put the bowl away on the bare floor. When putting away the robe, he should hold the robe in one hand, wipe the bamboo robe rack or the clothesline with the other, and then put it away by folding the robe over it, making the ends face the wall and the fold face out. When the pupil has gotten up, the teacher should put away the seat and also the foot stool, the foot scraper, and the water for washing the feet. If the place is dirty, he should sweep it. 

\subsection*{Bathing}

If\marginnote{33.1.31.1} the pupil wants to bathe, the teacher should prepare a bath. If he wants a cold bath, he should prepare that; if he wants a hot bath, he should prepare that. 

If\marginnote{33.1.34} the pupil wants to take a sauna, the teacher should knead bath powder, moisten the clay, take a sauna bench, and go to the sauna. After giving the pupil the sauna bench, receiving his robe, and putting it aside, he should give him the bath powder and the clay. If he’s able, he should enter the sauna. When entering the sauna, he should smear his face with clay, cover himself front and back, and then enter. He shouldn’t sit encroaching on the senior monks, and he shouldn’t block the junior monks from getting a seat. While in the sauna, he should provide assistance to his pupil. When leaving the sauna, he should take the sauna bench, cover himself front and back, and then leave. 

The\marginnote{33.1.40} teacher should also provide assistance to his pupil in the water. When the teacher has bathed, he should be the first to come out. He should dry himself and put on his sarong. He should then wipe the water off his pupil’s body, and he should give him his sarong and then his upper robe. Taking the sauna bench, he should be first to return. He should prepare a seat, and also set out a foot stool, a foot scraper, and water for washing the feet. He should ask his pupil if he wants water to drink. 

\subsection*{The dwelling}

If\marginnote{33.1.43.1} the dwelling where the pupil is staying is dirty, the teacher should clean it if he’s able. When he’s cleaning the dwelling, he should first take out the bowl and robe and put them aside. He should take out the sitting mat and the sheet and put them aside. He should take out the mattress and the pillow and put them aside. Holding the bed low, he should carefully take it out without scratching it or knocking it against the door or the door frame, and he should put it aside. Holding the bench low, he should carefully take it out without scratching it or knocking it against the door or the door frame, and he should put it aside. He should take out the bed supports and put them aside. He should take out the spittoon and put it aside. He should take out the leaning board and put it aside. After taking note of its position, he should take out the floor cover and put it aside. If the dwelling has cobwebs, he should first remove them from the ceiling cloth, and he should then wipe the windows and the corners of the room. If the walls have been treated with red ocher and they’re moldy, he should moisten a cloth, wring it out, and wipe the walls. If the floor has been treated with a black finish and it’s moldy, he should moisten a cloth, wring it out, and wipe the floor. If the floor is untreated, he should sprinkle it with water and then sweep it, trying to avoid stirring up dust. He should look out for any trash and discard it. 

He\marginnote{33.1.59} should sun the floor cover, clean it, beat it, bring it back inside, and put it back as before. He should sun the bed supports, wipe them, bring them back inside, and put them back where they were. He should sun the bed, clean it, and beat it. Holding it low, he should carefully bring it back inside without scratching it or knocking it against the door or the door frame, and he should put it back as before. He should sun the bench, clean it, and beat it. Holding it low, he should carefully bring it back inside without scratching it or knocking it against the door or the door frame, and he should put it back as before. He should sun the mattress and the pillow, clean them, beat them, bring them back inside, and put them back as before. He should sun the sitting mat and the sheet, clean them, beat them, bring them back inside, and put them back as before. He should sun the spittoon, wipe it, bring it back inside, and put it back where it was. He should sun the leaning board, wipe it, bring it back inside, and put it back where it was. He should put away the bowl and robe. When putting away the bowl, he should hold the bowl in one hand, feel under the bed or the bench with the other, and then put it away. He shouldn’t put the bowl away on the bare floor. When putting away the robe, he should hold the robe in one hand, wipe the bamboo robe rack or the clothesline with the other, and then put it away by folding the robe over it, making the ends face the wall and the fold face out. 

If\marginnote{33.1.71} dusty winds are blowing from the east, he should close the windows on the eastern side. If dusty winds are blowing from the west, he should close the windows on the western side. If dusty winds are blowing from the north, he should close the windows on the northern side. If dusty winds are blowing from the south, he should close the windows on the southern side. If the weather is cold, he should open the windows during the day and close them at night. If the weather is hot, he should close the windows during the day and open them at night. 

If\marginnote{33.1.77} the yard is dirty, he should sweep it. If the gatehouse is dirty, he should sweep it. If the assembly hall is dirty, he should sweep it. If the water-boiling shed is dirty, he should sweep it. If the restroom is dirty, he should sweep it. If there is no water for drinking, he should get some. If there is no water for washing, he should get some. If there is no water in the restroom ablutions pot, he should fill it. 

\subsection*{Spiritual support, etc.}

If\marginnote{33.1.85.1} the pupil becomes discontent with the spiritual life, the teacher should send him away or have him sent away, or he should give him a teaching. If the pupil becomes anxious, the teacher should dispel it or have it dispelled, or he should give him a teaching. If the pupil has wrong view, the teacher should make him give it up or have someone else do it, or he should give him a teaching. If the pupil has committed a heavy offense and deserves probation, the teacher should try to get the Sangha to give it to him. If the pupil has committed a heavy offense and deserves to be sent back to the beginning, the teacher should try to get the Sangha to do it. If the pupil has committed a heavy offense and deserves the trial period, the teacher should try to get the Sangha to give it to him. If the pupil has committed a heavy offense and deserves rehabilitation, the teacher should try to get the Sangha to give it to him. 

If\marginnote{33.1.96} the Sangha wants to do a legal procedure against his pupil—whether a procedure of condemnation, demotion, banishment, reconciliation, or ejection—the teacher should make an effort to stop it or to reduce the penalty. But if the Sangha has already done a legal procedure against his pupil—whether a procedure of condemnation, demotion, banishment, reconciliation, or ejection—the teacher should help the pupil conduct himself properly and suitably so as to deserve to be released, and try to get the Sangha to lift that procedure. 

If\marginnote{33.1.100} the pupil’s robe needs washing, the teacher should show him how to do it, or he should make an effort to get it done. If the pupil needs a robe, the teacher should show him how to make one, or he should make an effort to get one made. If the pupil needs dye, the teacher should show him how to make it, or he should make an effort to get it made. If the pupil’s robe needs dyeing, the teacher should show him how to do it, or he should make an effort to get it done. When he’s dyeing the robe, he should carefully and repeatedly turn it over, and shouldn’t go away while it’s still dripping. If his pupil is sick, he should nurse him for as long as he lives, or he should wait until he’s recovered.” 

\scend{The proper conduct toward a pupil is finished. }

\scend{The sixth section for recitation is finished. }

\section*{20. Asking for forgiveness when dismissed }

On\marginnote{34.1.1} a later occasion the pupils did not conduct themselves properly toward their teachers. … They told the Buddha. … 

\scrule{“A pupil should conduct himself properly toward his teacher. If he doesn’t, he commits an offense of wrong conduct.” }

They\marginnote{34.1.5} still did not conduct themselves properly. They told the Buddha. … 

\scrule{“You should dismiss one who doesn’t conduct himself properly. }

And\marginnote{34.1.8} this is how he should be dismissed. If the teacher conveys the following by body, by speech, or by body and speech: ‘I dismiss you;’ ‘Don’t come back here;’ ‘Remove your bowl and robe;’ or, ‘You shouldn’t attend on me’—then the pupil has been dismissed. If he doesn’t convey this by body, by speech, or by body and speech, then the pupil hasn’t been dismissed.” 

Pupils\marginnote{34.1.12} who had been dismissed did not ask for forgiveness. They told the Buddha. 

\scrule{“You should ask for forgiveness.” }

They\marginnote{34.1.15} still did not ask for forgiveness. They told the Buddha. 

\scrule{“One who has been dismissed should ask for forgiveness. If he doesn’t, he commits an offense of wrong conduct.” }

Teachers\marginnote{34.1.19} who were asked for forgiveness did not forgive. They told the Buddha. 

\scrule{“You should forgive.” }

They\marginnote{34.1.22} still did not forgive. The pupils left, disrobed, and joined the monastics of other religions. They told the Buddha. 

\scrule{“When asked for forgiveness, you should forgive. If you don’t, you commit an offense of wrong conduct.” }

Teachers\marginnote{34.1.27} dismissed pupils who were conducting themselves properly and did not dismiss those who were not. They told the Buddha. 

\scrule{“You shouldn’t dismiss someone who is conducting himself properly. If you do, you commit an offense of wrong conduct. }

\scrule{And you should dismiss someone who isn’t conducting himself properly. If you don’t, you commit an offense of wrong conduct. }

If\marginnote{34.1.33} a pupil has five qualities, he should be dismissed: he doesn’t have much affection for his teacher; he doesn’t have much confidence in his teacher; he doesn’t have much conscience in regard to his teacher; he doesn’t have much respect for his teacher; he hasn’t developed his mind much under his teacher. 

If\marginnote{34.1.36} a pupil has five qualities, he shouldn’t be dismissed: he has much affection for his teacher; he has much confidence in his teacher; he has much conscience in regard to his teacher; he has much respect for his teacher; he has developed his mind much under his teacher. 

If\marginnote{34.1.39} a pupil has five qualities, he deserves to be dismissed: he doesn’t have much affection for his teacher; he doesn’t have much confidence in his teacher; he doesn’t have much conscience in regard to his teacher; he doesn’t have much respect for his teacher; he hasn’t developed his mind much under his teacher. 

If\marginnote{34.1.42} a pupil has five qualities, he doesn’t deserve to be dismissed: he has much affection for his teacher; he has much confidence in his teacher; he has much conscience in regard to his teacher; he has much respect for his teacher; he has developed his mind much under his teacher. 

If\marginnote{34.1.45} a pupil has five qualities, the teacher is at fault if he doesn’t dismiss him, but not if he does: the pupil doesn’t have much affection for his teacher; he doesn’t have much confidence in his teacher; he doesn’t have much conscience in regard to his teacher; he doesn’t have much respect for his teacher; he hasn’t developed his mind much under his teacher. 

If\marginnote{34.1.48} a pupil has five qualities, the teacher is at fault if he dismisses him, but not if he doesn’t: the pupil has much affection for his teacher; he has much confidence in his teacher; he has much conscience in regard to his teacher; he has much respect for his teacher; he has developed his mind much under his teacher.” 

\scend{Asking for forgiveness when dismissed is finished. }

\section*{21. The ignorant and incompetent }

Then,\marginnote{35.1.1} once they had ten years of seniority, ignorant and incompetent monks gave formal support. As a result there were ignorant teachers with knowledgeable pupils, incompetent teachers with competent pupils, uneducated teachers with learned pupils, and foolish teachers with wise pupils. 

The\marginnote{35.1.6} monks of few desires complained and criticized them, “How can ignorant and incompetent monks give formal support, just because they have ten years of seniority? There are ignorant teachers with knowledgeable pupils, incompetent teachers with competent pupils, uneducated teachers with learned pupils, and foolish teachers with wise pupils.” 

They\marginnote{35.2.1} told the Buddha. … “Is it true, monks, that this is happening?” 

“It’s\marginnote{35.2.5} true, sir.” 

The\marginnote{35.2.6} Buddha rebuked them … He then gave a teaching and addressed the monks: 

\scrule{“An ignorant and incompetent monk shouldn’t give formal support. If he does, he commits an offense of wrong conduct. I allow a competent and capable monk who has ten or more years of seniority to give formal support.” }

\scend{The section on the ignorant and incompetent is finished. }

\section*{22. Discussion of the ending of formal support }

At\marginnote{36.1.1} that time there were preceptors and teachers who went away, disrobed, died, or joined another religion or sect, but their pupils did not know about the ending of support. They told the Buddha. 

“There\marginnote{36.1.3} are these five reasons why the formal support from a preceptor comes to an end: the preceptor goes away; the preceptor disrobes; the preceptor dies; the preceptor joins another religion or sect; or the preceptor orders it. 

There\marginnote{36.1.6} are these six reasons why the formal support from a teacher comes to an end: the teacher goes away; the teacher disrobes; the teacher dies; the teacher joins another religion or sect; the teacher orders it; or one is reunited with one’s preceptor.” 

\scend{The discussion of the ending of formal support is finished. }

\section*{23. The five requirements for giving the full ordination }

“A\marginnote{36.2.1} monk who has five qualities shouldn’t give the full ordination, give formal support, or have a novice monk attend on him. He doesn’t have the virtue, stillness, wisdom, freedom, or knowledge and vision of freedom of one who is fully trained. 

But\marginnote{36.3.1} a monk who has five qualities may give the full ordination, give formal support, and have a novice monk attend on him. He has the virtue, stillness, wisdom, freedom, and knowledge and vision of freedom of one who is fully trained. 

“A\marginnote{36.4.1} monk who has another five qualities shouldn’t give the full ordination, give formal support, or have a novice monk attend on him. He neither has it himself nor encourages others in the virtue, stillness, wisdom, freedom, or knowledge and vision of freedom of one who is fully trained. 

But\marginnote{36.5.1} a monk who has five qualities may give the full ordination, give formal support, and have a novice monk attend on him. He both has it himself and encourages others in the virtue, stillness, wisdom, freedom, and knowledge and vision of freedom of one who is fully trained. 

“A\marginnote{36.6.1} monk who has another five qualities shouldn’t give the full ordination, give formal support, or have a novice monk attend on him. He has no faith, conscience, or moral prudence; and he is lazy and absentminded. 

But\marginnote{36.7.1} a monk who has five qualities may give the full ordination, give formal support, and have a novice monk attend on him. He has faith, conscience, moral prudence, energy, and mindfulness. 

“A\marginnote{36.8.1} monk who has another five qualities shouldn’t give the full ordination, give formal support, or have a novice monk attend on him. He has failed in the higher morality, in conduct, and in view; and he’s ignorant and foolish. 

But\marginnote{36.9.1} a monk who has five qualities may give the full ordination, give formal support, and have a novice monk attend on him. He hasn’t failed in the higher morality, in conduct, or in view; he’s learned and wise. 

“A\marginnote{36.10.1} monk who has another five qualities shouldn’t give the full ordination, give formal support, or have a novice monk attend on him. He’s not capable of three things in regard to a student: to nurse him or have him nursed when he’s sick; to send him away or have him sent away when he’s discontent with the spiritual life; and to use the Teaching to dispel anxiety. And he doesn’t know the offenses; and he doesn’t know how offenses are cleared. 

But\marginnote{36.11.1} a monk who has five qualities may give the full ordination, give formal support, and have a novice monk attend on him. He’s capable of three things in regard to a student: to nurse him or have him nursed when he’s sick; to send him away or have him sent away when he’s discontent with the spiritual life; and to use the Teaching to dispel anxiety. And he knows the offenses; and he knows how offenses are cleared. 

“A\marginnote{36.12.1} monk who has another five qualities shouldn’t give the full ordination, give formal support, or have a novice monk attend on him. He’s not capable of five things in regard to a student: to train him in good conduct; to train him in the basics of the spiritual life; to train him in the Teaching; to train him in the Monastic Law; to use the Teaching to make him give up wrong views. 

But\marginnote{36.13.1} a monk who has five qualities may give the full ordination, give formal support, and have a novice monk attend on him. He’s capable of five things in regard to a student: to train him in good conduct; to train him in the basics of the spiritual life; to train him in the Teaching; to train him in the Monastic Law; to use the Teaching to make him give up wrong views. 

“A\marginnote{36.14.1} monk who has another five qualities shouldn’t give the full ordination, give formal support, or have a novice monk attend on him. He doesn’t know the offenses; he doesn’t know the non-offenses; he doesn’t know which offenses are light; he doesn’t know which offenses are heavy; neither Monastic Code has been properly learned by him in detail, and he hasn’t analyzed them well, thoroughly mastered them, or investigated them well, either in terms of the rules or their detailed exposition. 

But\marginnote{36.15.1} a monk who has five qualities may give the full ordination, give formal support, and have a novice monk attend on him. He knows the offenses; he knows the non-offenses; he knows which offenses are light; he knows which offenses are heavy; he has properly learned both Monastic Codes in detail, and he has analyzed them well, thoroughly mastered them, and investigated them well, both in terms of the rules and their detailed exposition. 

“A\marginnote{36.16.1} monk who has another five qualities shouldn’t give the full ordination, give formal support, or have a novice monk attend on him. He doesn’t know the offenses; he doesn’t know the non-offenses; he doesn’t know which offenses are light; he doesn’t know which offenses are heavy; he has less than ten years of seniority. 

But\marginnote{36.17.1} a monk who has five qualities may give the full ordination, give formal support, and have a novice monk attend on him. He knows the offenses; he knows the non-offenses; he knows which offenses are light; he knows which offenses are heavy; he has ten or more years of seniority.” 

\scend{The section consisting of sixteen groups of five requirements for giving the full ordination is finished. }

\section*{24. The six requirements for giving the full ordination }

“A\marginnote{37.1.1} monk who has six qualities shouldn’t give the full ordination, give formal support, or have a novice monk attend on him. He doesn’t have the virtue, stillness, wisdom, freedom, or knowledge and vision of freedom of one who is fully trained, and he has less than ten years of seniority. 

But\marginnote{37.2.1} a monk who has six qualities may give the full ordination, give formal support, and have a novice monk attend on him. He has the virtue, stillness, wisdom, freedom, and knowledge and vision of freedom of one who is fully trained, and he has ten or more years of seniority. 

“A\marginnote{37.3.1} monk who has another six qualities shouldn’t give the full ordination, give formal support, or have a novice monk attend on him. He neither has it himself nor encourages others in the virtue, stillness, wisdom, freedom, or knowledge and vision of freedom of one who is fully trained, and he has less than ten years of seniority. 

But\marginnote{37.4.1} a monk who has six qualities may give the full ordination, give formal support, and have a novice monk attend on him. He both has it himself and encourages others in the virtue, stillness, wisdom, freedom, and knowledge and vision of freedom of one who is fully trained, and he has ten or more years of seniority. 

“A\marginnote{37.5.1} monk who has another six qualities shouldn’t give the full ordination, give formal support, or have a novice monk attend on him. He has no faith, conscience, or moral prudence; he is lazy and absentminded; and he has less than ten years of seniority. 

But\marginnote{37.6.1} a monk who has six qualities may give the full ordination, give formal support, and have a novice monk attend on him. He has faith, conscience, moral prudence, energy, mindfulness, and ten or more years of seniority. 

“A\marginnote{37.7.1} monk who has another six qualities shouldn’t give the full ordination, give formal support, or have a novice monk attend on him. He has failed in the higher morality, in conduct, and in view; he’s ignorant and foolish; and he has less than ten years of seniority. 

But\marginnote{37.8.1} a monk who has six qualities may give the full ordination, give formal support, and have a novice monk attend on him. He hasn’t failed in the higher morality, in conduct, or in view; he’s learned and wise; and he has ten or more years of seniority. 

“A\marginnote{37.9.1} monk who has another six qualities shouldn’t give the full ordination, give formal support, or have a novice monk attend on him. He’s not capable of three things in regard to a student: to nurse him or have him nursed when he’s sick; to send him away or have him sent away when he’s discontent with the spiritual life; to use the Teaching to dispel anxiety. And he doesn’t know the offenses; he doesn’t know how offenses are cleared; and he has less than ten years of seniority. 

But\marginnote{37.10.1} a monk who has six qualities may give the full ordination, give formal support, and have a novice monk attend on him. He’s capable of three things in regard to a student: to nurse him or have him nursed when he’s sick; to send him away or have him sent away when he’s discontent with the spiritual life; to use the Teaching to dispel anxiety. And he knows the offenses; he knows how offenses are cleared; and he has ten or more years of seniority. 

“A\marginnote{37.11.1} monk who has another six qualities shouldn’t give the full ordination, give formal support, or have a novice monk attend on him. He’s not capable of five things in regard to a student: to train him in good conduct; to train him in the basics of the spiritual life; to train him in the Teaching; to train him in the Monastic Law; or to use the Teaching to make him give up wrong views. And he has less than ten years of seniority. 

But\marginnote{37.12.1} a monk who has six qualities may give the full ordination, give formal support, and have a novice monk attend on him. He’s capable of five things in regard to a student: to train him in good conduct; to train him in the basics of the spiritual life; to train him in the Teaching; to train him in the Monastic Law; and to use the Teaching to make him give up wrong views. And he has ten or more years of seniority. 

“A\marginnote{37.13.1} monk who has another six qualities shouldn’t give the full ordination, give formal support, or have a novice monk attend on him. He doesn’t know the offenses; he doesn’t know the non-offenses; he doesn’t know which offenses are light; he doesn’t know which offenses are heavy; neither Monastic Code has been properly learned by him in detail, and he hasn’t analyzed them well, thoroughly mastered them, or investigated them well, either in terms of the rules or their detailed exposition; he has less than ten years of seniority. 

But\marginnote{37.14.1} a monk who has six qualities may give the full ordination, give formal support, and have a novice monk attend on him. He knows the offenses; he knows the non-offenses; he knows which offenses are light; he knows which offenses are heavy; he has properly learned both Monastic Codes in detail, and he has analyzed them well, thoroughly mastered them, and investigated them well, both in terms of the rules and their detailed exposition; he has ten or more years of seniority.” 

\scend{The section consisting of fourteen groups of six requirements for giving the full ordination is finished. }

\section*{25. Discussion on those who have been monastics of another religion }

Soon\marginnote{38.1.1} afterwards that monk who had been a monastic of another religion, and who had returned to that religious community after refuting his preceptor, came back to the monks and asked for the full ordination. The monks told the Buddha. 

\scrule{“Monks, when someone who has been a monastic of another religion refutes his preceptor after being legitimately corrected by him and then returns to that religion, but then comes back from that religious community once more, he shouldn’t be given the full ordination. }

\scrule{Anyone else who has been a monastic of another religion, and who wants the going forth and the full ordination on this spiritual path, should be given four months of probation. }

And\marginnote{38.2.1} it should be given like this. First he should shave off his hair and beard and put on the ocher robes. He should then arrange his upper robe over one shoulder, pay respect at the feet of the monks, squat on his heels, and raise his joined palms. He should then be told to say this: 

\begin{verse}%
‘I\marginnote{38.2.3} go for refuge to the Buddha, \\
I go for refuge to the Teaching, \\
I go for refuge to the Sangha. 

For\marginnote{38.2.6} the second time, I go for refuge to the Buddha, \\
For the second time, I go for refuge to the Teaching, \\
For the second time, I go for refuge to the Sangha. 

For\marginnote{38.2.9} the third time, I go for refuge to the Buddha, \\
For the third time, I go for refuge to the Teaching, \\
For the third time, I go for refuge to the Sangha.’ 

%
\end{verse}

Then,\marginnote{38.3.1} after approaching the Sangha, he who had been a monastic of another religion should arrange his upper robe over one shoulder, pay respect at the feet of the monks, squat on his heels, and raise his joined palms. He should then say this: ‘Venerables, I have been a monastic of another religion, and I wish for the full ordination on this spiritual path. I ask the Sangha for four months of probation.’ And he should ask a second and a third time. A competent and capable monk should then inform the Sangha: 

‘Please,\marginnote{38.3.7} venerables, I ask the Sangha to listen. So-and-so, who has been a monastic of another religion, wants the full ordination on this spiritual path. He is asking the Sangha for four months of probation. If the Sangha is ready, it should give four months of probation to so-and-so, who has been a monastic of another religion. This is the motion. 

Please,\marginnote{38.4.1} venerables, I ask the Sangha to listen. So-and-so, who has been a monastic of another religion, wants the full ordination on this spiritual path. He is asking the Sangha for four months of probation. The Sangha gives four months of probation to so-and-so, who has been a monastic of another religion. Any monk who approves of giving four months of probation to so-and-so, who has been a monastic of another religion, should remain silent. Any monk who doesn’t approve should speak up. 

The\marginnote{38.4.7} Sangha has given so-and-so, who has been a monastic of another religion, four months of probation. The Sangha approves and is therefore silent. I’ll remember it thus.’ 

And\marginnote{38.5.1} this is how someone who has been a monastic of another religion fails his probation: 

\begin{itemize}%
\item He enters the village too early and returns too late in the day. %
\item He regularly associates with sex workers, widows, single women, \textit{\textsanskrit{paṇḍakas}}, and nuns.\footnote{Sp-\textsanskrit{ṭ} 3.87: \textit{\textsanskrit{Vesiyā} gocaro mittasanthavavasena \textsanskrit{upasaṅkamitabbaṭṭhānaṁ} \textsanskrit{assāti} \textsanskrit{vesiyāgocaro}. Esa nayo sabbattha}, “\textit{\textsanskrit{Vesiyāgocaro}}: association with sex workers; the place to be approached by him for intimacy or friendship. This method applies to all (five groups).” } %
\item He’s not skilled or diligent in the various duties of his fellow monastics, and he lacks the proper judgment to organize and perform them well. %
\item He doesn’t have a keen desire for recitation, for questioning, for the higher morality, for the higher mind, or for the higher wisdom. %
\item He’s displeased when anyone disparages the teacher, the views, the beliefs, the persuasion, or the opinions of the religious community he’s left; but he’s pleased when anyone disparages the Buddha, the Teaching, or the Sangha. He’s pleased when anyone praises the teacher, the views, the beliefs, the persuasion, or the opinions of the religious community he’s left; but he’s displeased when anyone praises the Buddha, the Teaching, or the Sangha. This last one is the critical factor for someone who has been a monastic of another religion to fail his probation. %
\end{itemize}

\scrule{When he fails in this way, he shouldn’t be given the full ordination. }

And\marginnote{38.8.1} this is how someone who has been a monastic of another religion passes his probation: 

\begin{itemize}%
\item He doesn’t enter the village too early or return too late in the day. %
\item He doesn’t regularly associate with sex workers, widows, single women, \textit{\textsanskrit{paṇḍakas}}, or nuns. %
\item He’s skilled and diligent in the various duties of his fellow monastics, and he has the proper judgment to organize and perform them well. %
\item He has a keen desire for recitation, for questioning, for the higher morality, for the higher mind, and for the higher wisdom. %
\item He’s pleased when anyone disparages the teacher, the views, the beliefs, the persuasion, or the opinions of the religious community he’s left; but he’s displeased when anyone disparages the Buddha, the Teaching, or the Sangha. He’s displeased when anyone praises the teacher, the views, the beliefs, the persuasion, or the opinions of the religious community he’s left; but he’s pleased when anyone praises the Buddha, the Teaching, or the Sangha. This last one is the critical factor for someone who has been a monastic of another religion to pass his probation. %
\end{itemize}

\scrule{When he passes in this way, he should be given the full ordination. }

If\marginnote{38.11.1} someone who has been a monastic of another religion arrives naked, a robe should be sought through his preceptor. If he arrives with hair, he should get permission from the Sangha to shave. But any dreadlocked, fire-worshiping ascetic who comes to be ordained should be given the full ordination without probation. Why is that? Because they believe that deeds and actions have results. And if someone comes to be ordained who has been a monastic of another religion but is a Sakyan by birth, he should be given the full ordination without probation. I give this special privilege to my relatives.” 

\scend{The discussion on those who have been monastics of another religion is finished. }

\scend{The seventh section for recitation is finished. }

\section*{26. The five diseases }

At\marginnote{39.1.1} that time in Magadha, there were five common diseases: leprosy, abscesses, mild leprosy, tuberculosis, and epilepsy.\footnote{For an explanation of these, see Appendix of Medical Terminology. } When people were sick with any of these, they went to \textsanskrit{Jīvaka} \textsanskrit{Komārabhacca} and said, “Doctor, please treat us.” 

He\marginnote{39.1.5} replied, “I’m very busy. I look after King Seniya \textsanskrit{Bimbisāra} of Magadha and his harem. I also look after the Sangha of monks headed by the Buddha. I’m not able to treat you.” 

“We’ll\marginnote{39.1.8} give you everything we own, and we’ll be your slave, too. Please treat us, doctor.” 

\textsanskrit{Jīvaka}\marginnote{39.2.1} repeated what he had already said. And those people thought, “These Sakyan monastics have pleasant habits and a happy life. They eat nice food and sleep in beds sheltered from the wind. Why don’t we go forth with the Sakyan monastics? If we do, the monks will nurse us and \textsanskrit{Jīvaka} \textsanskrit{Komārabhacca} will treat us.” 

They\marginnote{39.2.7} then went to the monks and asked for the going forth. The monks gave them the going forth and the full ordination. And the monks nursed them, and \textsanskrit{Jīvaka} treated them. 

At\marginnote{39.3.1} one time the monks were nursing many sick monks. As a result, they kept on asking, “Please give a meal for the sick and for those nursing the sick. Please give medicines for the sick.” And because \textsanskrit{Jīvaka} was treating many sick monks, he was unable to fulfill his duty to King \textsanskrit{Bimbisāra}. 

Then\marginnote{39.4.1} a certain man who was afflicted with one of the five diseases went to \textsanskrit{Jīvaka} and said, “Doctor, please treat me.” 

He\marginnote{39.4.3} replied, “I’m very busy. I look after the king of Magadha and his harem. I also look after the Sangha of monks headed by the Buddha. I’m not able to treat you.” 

“I\marginnote{39.4.5} will give you everything I own, and I’ll be your slave, too. Please treat me, doctor.” 

\textsanskrit{Jīvaka}\marginnote{39.4.7} repeated what he had already said. That man thought, “These Sakyan monastics have pleasant habits and a happy life. They eat nice food and sleep in beds sheltered from the wind. Why don’t I go forth with the Sakyan monastics? If I do, the monks will nurse me, and \textsanskrit{Jīvaka} \textsanskrit{Komārabhacca} will treat me. And when I’m healthy, I’ll disrobe.” 

He\marginnote{39.5.6} then went to the monks and asked for the going forth. The monks gave him the going forth and the full ordination, after which they nursed him and \textsanskrit{Jīvaka} treated him. When he was healthy again, he disrobed. 

\textsanskrit{Jīvaka}\marginnote{39.5.10} saw that man after he had disrobed, and he asked him, “Didn’t you go forth with the monks?” 

“Yes,\marginnote{39.5.12} doctor.” 

“And\marginnote{39.5.13} why did you do it?” 

When\marginnote{39.5.14} that man had told him what had happened, \textsanskrit{Jīvaka} complained and criticized the monks, “How could the venerables allow one with the five diseases to go forth?” 

He\marginnote{39.6.3} went to the Buddha, bowed, sat down, and said, “Please, sir, may the venerables not allow those with the five diseases to go forth.” The Buddha instructed, inspired, and gladdened him with a teaching. \textsanskrit{Jīvaka} then got up from his seat, bowed down, circumambulated the Buddha with his right side toward him, and left. Soon afterwards the Buddha gave a teaching and addressed the monks: 

\scrule{“You shouldn’t give the going forth to anyone afflicted with any of the five diseases. If you do, you commit an offense of wrong conduct.” }

\section*{27. Those employed by the king }

On\marginnote{40.1.1} one occasion unrest erupted in the outlying districts governed by King \textsanskrit{Bimbisāra}. The king told his generals, “Go and sort out those districts.” 

“Yes,\marginnote{40.1.4} sir.” 

But\marginnote{40.2.1} the most distinguished soldiers thought, “If we go and enjoy the battle, we’ll do what’s bad and make much demerit. How can we avoid what’s bad and do what’s good instead?” 

It\marginnote{40.2.4} occurred to them, “These Sakyan monastics have integrity. They’re celibate and their conduct is good, and they’re truthful, moral, and have a good character. If we go forth with them, we’ll avoid what’s bad and do what’s good.” Those soldiers then went to the monks and asked for the going forth. And the monks gave them the going forth and the full ordination. 

Soon\marginnote{40.3.1} afterwards the generals asked among the king’s employees, “Where are the soldiers so-and-so and so-and-so?” 

“They’ve\marginnote{40.3.3} gone forth with the monks.” 

The\marginnote{40.3.4} generals complained and criticized the monks, “How could the Sakyan monastics give the going forth to those who are employed by the king?” They told King \textsanskrit{Bimbisāra}. 

The\marginnote{40.3.7} king then asked the judges, “What’s the penalty for one who gives the going forth to someone employed by the king?” 

“The\marginnote{40.3.9} preceptor should have his head cut off, the one who does the formal proclamation should have his tongue cut out, and the participating group should have half their ribs broken.” 

The\marginnote{40.4.1} king went to the Buddha, bowed, sat down, and said, “Sir, there are kings with little faith and confidence. They would give the monks a hard time even over small matters. Please, may the venerables not give the going forth to those employed by a king.” The Buddha instructed, inspired, and gladdened him with a teaching. The king then got up from his seat, bowed down, circumambulated the Buddha with his right side toward him, and left. Soon afterwards the Buddha gave a teaching and addressed the monks: 

\scrule{“You shouldn’t give the going forth to anyone employed by a king. If you do, you commit an offense of wrong conduct.” }

\section*{28. The criminal \textsanskrit{Aṅgulimāla} }

At\marginnote{41.1.1} that time the criminal \textsanskrit{Aṅgulimāla} had gone forth with the monks. When people saw him, they became alarmed and fearful. They turned away, took a different path, ran off, and closed their doors. People complained and criticized the monks, “How could the Sakyan monastics give the going forth to a notorious criminal?” The monks heard the complaints of those people. They then told the Buddha. … 

\scrule{“You shouldn’t give the going forth to a notorious criminal. If you do, you commit an offense of wrong conduct.” }

\section*{29. The escaped criminal }

At\marginnote{42.1.1} that time King \textsanskrit{Bimbisāra} had made the following declaration: “Nothing should be done to anyone who has gone forth with the Sakyan monastics. The Teaching is well-proclaimed. Allow them to practice the spiritual life to make a complete end of suffering.” 

Soon\marginnote{42.1.4} afterwards a certain thief was put in prison. But he escaped, ran away, and went forth with the monks. When people saw him, they said, “There’s that criminal who escaped from prison. Let’s get him!” But some said, “No, the king has declared that nothing should be done to anyone gone forth with the Sakyan monastics.” 

People\marginnote{42.2.9} complained and criticized the monks, “These Sakyan monastics are untouchable; you can’t do anything to them. So how could they give the going forth to an escaped criminal?” They told the Buddha. 

\scrule{“You shouldn’t give the going forth to an escaped criminal. If you do, you commit an offense of wrong conduct.” }

\section*{30. The wanted criminal }

On\marginnote{43.1.1} one occasion a certain man stole something, ran away, and then went forth with the monks. Yet the king’s court had issued a statement:\footnote{“Court” renders \textit{antepura}. See Appendix of Technical Terms. } “He should be executed wherever he’s seen.” 

When\marginnote{43.1.4} people saw him, they said, “There’s that wanted criminal.\footnote{\textit{Likhitaka}, literally, “One who has been written about”. Sp 3.93: \textit{Atha kho yo koci \textsanskrit{corikaṁ} \textsanskrit{vā} \textsanskrit{aññaṁ} \textsanskrit{vā} \textsanskrit{garuṁ} \textsanskrit{rājāparādhaṁ} \textsanskrit{katvā} \textsanskrit{palāto}, \textsanskrit{rājā} ca \textsanskrit{naṁ} \textsanskrit{paṇṇe} \textsanskrit{vā} potthake \textsanskrit{vā} “\textsanskrit{itthannāmo} yattha dissati, tattha \textsanskrit{gahetvā} \textsanskrit{māretabbo}”ti \textsanskrit{vā} “\textsanskrit{hatthapādānissa} \textsanskrit{chinditabbānī}”ti \textsanskrit{vā} “\textsanskrit{ettakaṁ} \textsanskrit{nāma} \textsanskrit{daṇḍaṁ} \textsanskrit{āharāpetabbo}”ti \textsanskrit{vā} \textsanskrit{likhāpeti}, \textsanskrit{ayaṁ} likhitako \textsanskrit{nāma}}, “When someone has run away after stealing or committing another heavy offense against the king, and the king causes the writing about him on a leaf or in a book that ‘wherever so-and-so is seen, he should be seized and executed’ or ‘his hands and feet are to be cut off’ or ‘this penalty is to be imposed’, this is called a wanted criminal.” } Let’s execute him!” 

But\marginnote{43.1.7} some said, “No, King \textsanskrit{Bimbisāra} has declared that nothing should be done to anyone gone forth with the Sakyan monastics.” 

People\marginnote{43.1.10} complained and criticized the monks, “These Sakyan monastics are untouchable; you can’t do anything to them. So how could they give the going forth to a wanted criminal?” They told the Buddha. 

\scrule{“You shouldn’t give the going forth to a wanted criminal. If you do, you commit an offense of wrong conduct.” }

\section*{31. The one who had been whipped }

At\marginnote{44.1.1} one time a certain man who had been whipped as a penalty went forth with the monks. People complained and criticized the monks, “How could the Sakyan monastics give the going forth to one who has been whipped as a penalty?” They told the Buddha. 

\scrule{“You shouldn’t give the going forth to one who has been whipped as a penalty. If you do, you commit an offense of wrong conduct.” }

\section*{32. The one who had been branded }

At\marginnote{45.1.1} one time a certain man who had been branded as a penalty went forth with the monks. People complained and criticized the monks, “How could the Sakyan monastics give the going forth to one who has been branded as a penalty?” They told the Buddha. 

\scrule{“You shouldn’t give the going forth to one who has been branded as a penalty. If you do, you commit an offense of wrong conduct.” }

\section*{33. The one in debt }

On\marginnote{46.1.1} one occasion a certain indebted man ran away and went forth with the monks. Soon afterwards the creditors saw him and said, “There’s that man who owes us. Let’s get him!” 

But\marginnote{46.1.5} some said, “No, King \textsanskrit{Bimbisāra} has declared that nothing should be done to anyone gone forth with the Sakyan monastics.” 

People\marginnote{46.1.10} complained and criticized the monks, “These Sakyan monastics are untouchable; you can’t do anything to them. So how could they give the going forth to an indebted person?” They told the Buddha. 

\scrule{“You shouldn’t give the going forth to one who is indebted. If you do, you commit an offense of wrong conduct.” }

\section*{34. The slave }

On\marginnote{47.1.1} one occasion a certain slave ran away and went forth with the monks. Soon afterwards the owners saw him and said, “There’s our slave. Let’s get him!” 

But\marginnote{47.1.5} some said, “No, King \textsanskrit{Bimbisāra} has declared that nothing should be done to anyone gone forth with the Sakyan monastics.” 

People\marginnote{47.1.7} complained and criticized the monks, “These Sakyan monastics are untouchable; you can’t do anything to them. So how could they give the going forth to a slave?” They told the Buddha. 

\scrule{“You shouldn’t give the going forth to a slave. If you do, you commit an offense of wrong conduct.” }

\section*{35. The shaven-headed smith }

At\marginnote{48.1.1} that time a certain shaven-headed smith had quarreled with his parents. He then went to the monastery and went forth with the monks. While looking for their son, the parents came to that monastery. They asked the monks, “Venerables, have you by any chance seen such-and-such a boy?” Because they had not, they said, “No.” 

Soon\marginnote{48.2.1} afterwards those parents saw that their son had gone forth as a monk. They then complained and criticized the monks, “These Sakyan monastics are shameless and immoral liars. They deny knowing what they know and having seen what they’ve seen. Our boy has gone forth as a monk.” The monks heard the complaints of those parents. They told the Buddha. 

\scrule{“You should get permission from the Sangha to shave someone’s head.” }

\section*{36. The boy \textsanskrit{Upāli} }

At\marginnote{49.1.1} that time in \textsanskrit{Rājagaha}, there was a group of seventeen boys who were friends and had \textsanskrit{Upāli} as their leader. 

On\marginnote{49.1.3} one occasion \textsanskrit{Upāli}’s parents thought, “How can we make sure that \textsanskrit{Upāli} is able to live happily without exhausting himself after we’ve passed away? He could become a clerk, but then his fingers will hurt. Or he could become an accountant, but then his chest will hurt. Or he could become a banker, but then his eyes will hurt. These Sakyan monastics, however, have pleasant habits and a happy life. They eat nice food and sleep in beds sheltered from the wind. If \textsanskrit{Upāli} goes forth with them, he’ll be able to live happily without exhausting himself after we’ve passed away.” 

\textsanskrit{Upāli}\marginnote{49.3.1} overheard this conversation between his parents. He then went to the other boys and said, “Come, let’s go forth with the Sakyan monastics.” 

“If\marginnote{49.3.4} you go forth, so will we.” 

The\marginnote{49.3.5} boys went each to his own parents and said, “Please allow me to go forth into homelessness.” Because the parents knew that all the boys had the same desire and good intentions, they gave their approval. The boys then went to the monks and asked them for the going forth, and the monks gave them the going forth and the full ordination. 

Soon\marginnote{49.4.1} afterwards they got up early in the morning and cried, “Give us congee, give us a meal, give us fresh food!”\footnote{“Fresh food” renders \textit{\textsanskrit{khādanīya}}. See Appendix of Technical Terms. } 

The\marginnote{49.4.3} monks said, “Wait until it gets light. If any of that becomes available then, you can have it. If not, you’ll eat after walking for alms.” But they carried on as before. And they defecated and urinated on the furniture.\footnote{“Furniture” renders \textit{\textsanskrit{senāsana}}. See Appendix of Technical Terms. } 

After\marginnote{49.5.1} rising early in the morning, the Buddha heard the sound of those boys. He asked Venerable Ānanda, who told him what was happening. Soon afterwards he had the Sangha gathered and questioned the monks: “Is it true, monks, that the monks give the full ordination to people they know are less than twenty years old?” 

“It’s\marginnote{49.5.6} true, sir.” 

The\marginnote{49.5.7} Buddha rebuked them … “How can those foolish men do this? A person who’s less than twenty years old is unable to endure cold and heat; hunger and thirst; horseflies, mosquitoes, wind, and the burning sun; creeping animals and insects; and rude and unwelcome speech. And they’re unable to bear up with bodily feelings that are painful, severe, sharp, and destructive of life.\footnote{Sp-\textsanskrit{ṭ} 4.295: \textit{\textsanskrit{Sarīsapeti} ye keci sarante gacchante \textsanskrit{dīghajātike}}, “\textit{\textsanskrit{Sarīsape}}: whatever long creatures are moving by flowing.” } But a person who’s twenty is able to endure these things. This will affect people’s confidence …” After rebuking them … he gave a teaching and addressed the monks: 

\scrule{“You shouldn’t give the full ordination to a person you know is less than twenty years old. If you do, you should be dealt with according to the rule.”\footnote{See \href{https://suttacentral.net/pli-tv-bu-vb-pc65/en/brahmali\#1.53.1}{Bu Pc 65:1.53.1}. } }

\section*{37. The deadly and contagious disease }

At\marginnote{50.1.1} one time most of the members of a particular family had died from a deadly and contagious disease. Only a father and son were left. After going forth as monks, they walked together for alms. Then, when the boy had handed over his almsfood to his father, he said, “Give to me too, daddy!” 

People\marginnote{50.1.7} complained and criticized the monks, “These Sakyan monastics are not celibate. This boy was born to a nun!” The monks heard the complaints of those people and they told the Buddha. 

\scrule{“You shouldn’t give the going forth to a boy less than fifteen years old. If you do, you commit an offense of wrong conduct.” }

At\marginnote{51.1.1} that time there was a family with faith and confidence that was supporting Venerable Ānanda. Then most of its members died from a deadly and contagious disease, and only two boys were left behind. When they saw the monks, they ran up to them, as they had done before. When the monks dismissed them, they cried. 

Ānanda\marginnote{51.1.5} thought, “The Buddha has laid down a rule that a boy less than fifteen years old shouldn’t be given the going forth, which applies to these boys. How then can I make sure that these boys don’t perish?” He told the Buddha. 

“Are\marginnote{51.1.11} they able, Ānanda, to scare away crows?” 

“Yes.”\marginnote{51.1.12} The Buddha then gave a teaching and addressed the monks: 

\scrule{“I allow you to give the going forth to a boy less than fifteen years old if he’s able to scare away crows.” }

\section*{38. \textsanskrit{Kaṇṭaka} }

At\marginnote{52.1.1} one time Venerable Upananda the Sakyan had two novice monks, \textsanskrit{Kaṇṭaka} and Mahaka. They had sex with each other. The monks complained and criticized them, “How could novice monks misbehave like this?” They told the Buddha. 

\scrule{“A single monk shouldn’t have two novice monks attend on him. If he does, he commits an offense of wrong conduct.” }

\section*{39. The obscure }

At\marginnote{53.1.1} one time the Buddha was staying right there at \textsanskrit{Rājagaha} during the rainy season, the winter, and the summer. People complained, “The districts are left in darkness and obscurity by the Sakyan monastics. They don’t brighten them up by their presence.” 

The\marginnote{53.1.4} monks heard the complaints of those people and told the Buddha. He said to Venerable Ānanda, “Take a key, Ānanda, and go around the yards, informing the monks that the Buddha wishes to go wandering in the Southern Hills. Anyone is welcome to join him.” 

Saying,\marginnote{53.2.5} “Yes, sir,” he did just that. 

The\marginnote{53.3.1} monks said, “Ānanda, the Buddha has laid down a rule that one must live with formal support for ten years and that one who has ten years’ seniority can give such support. If we were to go, we would have to obtain support for a short time, and when we returned, we would have to obtain support once again. So, if our preceptors and teachers go, we’ll go too. If they don’t, neither will we. We don’t want the burden.” 

As\marginnote{53.4.1} a result, the Buddha went wandering in the Southern Hills with a small group of monks. 

\section*{40. Discussion of release from formal support }

After\marginnote{53.4.2.1} staying in the Southern Hills for as long as he liked, the Buddha returned to \textsanskrit{Rājagaha}. He then asked Ānanda, “Why was it so small, Ānanda, the group of monks that came wandering with me in the Southern Hills?” Ānanda told him what had happened. Soon afterwards the Buddha gave a teaching and addressed the monks: 

\scrule{“A competent and capable monk should live with formal support for five years, but one who is incompetent should live with formal support for life. }

A\marginnote{53.5.1} monk who has five qualities should live with formal support: he doesn’t have the virtue, stillness, wisdom, freedom, or knowledge and vision of freedom of one who is fully trained. 

But\marginnote{53.5.4} a monk who has five qualities may live without formal support: he has the virtue, stillness, wisdom, freedom, and knowledge and vision of freedom of one who is fully trained. 

“A\marginnote{53.6.1} monk who has another five qualities should live with formal support: he has no faith, conscience, or moral prudence, and is lazy and absentminded. 

But\marginnote{53.6.4} a monk who has five qualities may live without formal support: he has faith, conscience, moral prudence, energy, and mindfulness. 

“A\marginnote{53.7.1} monk who has another five qualities should live with formal support: he has failed in the higher morality, in conduct, and in view; he’s ignorant and foolish. 

But\marginnote{53.7.4} a monk who has five qualities may live without formal support: he hasn’t failed in the higher morality, in conduct, or in view; he’s learned and wise. 

“A\marginnote{53.8.1} monk who has another five qualities should live with formal support: he doesn’t know the offenses; he doesn’t know the non-offenses; he doesn’t know which offenses are light; he doesn’t know which offenses are heavy; neither Monastic Code has been properly learned by him in detail, and he hasn’t analyzed them well, thoroughly mastered them, or investigated them well, either in terms of the rules or their detailed exposition. 

But\marginnote{53.8.4} a monk who has five qualities may live without formal support: he knows the offenses; he knows the non-offenses; he knows which offenses are light; he knows which offenses are heavy; he has properly learned both Monastic Codes in detail, and he has analyzed them well, thoroughly mastered them, and investigated them well, both in terms of the rules and their detailed exposition. 

“A\marginnote{53.9.1} monk who has another five qualities should live with formal support: he doesn’t know the offenses; he doesn’t know the non-offenses; he doesn’t know which offenses are light; he doesn’t know which offenses are heavy; he has less than five years of seniority. 

But\marginnote{53.9.4} a monk who has five qualities may live without formal support: he knows the offenses; he knows the non-offenses; he knows which offenses are light; he knows which offenses are heavy; he has five or more years of seniority.” 

\scend{The section consisting of ten groups of five is finished. }

“A\marginnote{53.10.1} monk who has six qualities should live with formal support: he doesn’t have the virtue, stillness, wisdom, freedom, or knowledge and vision of freedom of one who is fully trained, and he has less than five years of seniority. 

But\marginnote{53.10.4} a monk who has six qualities may live without formal support: he has the virtue, stillness, wisdom, freedom, and knowledge and vision of freedom of one who is fully trained, and he has five or more years of seniority. 

“A\marginnote{53.11.1} monk who has another six qualities should live with formal support: he has no faith, conscience, or moral prudence; he is lazy and absentminded; and he has less than five years of seniority. 

But\marginnote{53.11.4} a monk who has six qualities may live without formal support: he has faith, conscience, moral prudence, energy, mindfulness, and five or more years of seniority. 

“A\marginnote{53.12.1} monk who has another six qualities should live with formal support: he has failed in the higher morality, in conduct, and in view; he’s ignorant and foolish; he has less than five years of seniority. 

But\marginnote{53.12.4} a monk who has six qualities may live without formal support: he hasn’t failed in the higher morality, in conduct, or in view; he’s learned and wise; he has five or more years of seniority. 

“A\marginnote{53.13.1} monk who has another six qualities should live with formal support: he doesn’t know the offenses; he doesn’t know the non-offenses; he doesn’t know which offenses are light; he doesn’t know which offenses are heavy; neither Monastic Code has been properly learned by him in detail, and he hasn’t analyzed them well, thoroughly mastered them, or investigated them well, either in terms of the rules or their detailed exposition; he has less than five years of seniority. 

But\marginnote{53.13.4} a monk who has six qualities may live without formal support: he knows the offenses; he knows the non-offenses; he knows which offenses are light; he knows which offenses are heavy; he has properly learned both Monastic Codes in detail, and he has analyzed them well, thoroughly mastered them, and investigated them well, both in terms of the rules and their detailed exposition; he has five or more years of seniority.” 

\scend{The eighth section for recitation on untouchable is finished. }

\section*{41. \textsanskrit{Rāhula} }

After\marginnote{54.1.1} staying at \textsanskrit{Rājagaha} for as long as he liked, the Buddha set out wandering toward Kapilavatthu in the Sakyan country. When he eventually arrived, he stayed in the Banyan Tree Monastery. 

In\marginnote{54.1.4} the morning the Buddha robed up, took his bowl and robe, and went to Suddhodana the Sakyan’s house where he sat down on the prepared seat. The queen, the mother of \textsanskrit{Rāhula}, said to the boy, “This is your father, \textsanskrit{Rāhula}. Go and ask for your inheritance.” \textsanskrit{Rāhula} went up to the Buddha, stood in front of him, and said, “Ascetic, your shadow is pleasant.” When the Buddha got up from his seat and left, \textsanskrit{Rāhula} followed behind, saying “Give me my inheritance! Give me my inheritance!” The Buddha said to Venerable \textsanskrit{Sāriputta}, “Well then, \textsanskrit{Sāriputta}, give \textsanskrit{Rāhula} the going forth.” 

“But\marginnote{54.2.9} how, sir?” 

The\marginnote{54.3.1} Buddha then gave a teaching and addressed the monks: 

\scrule{“The going forth as a novice monk should be given through the taking of the three refuges. }

It\marginnote{54.3.3} should be done like this. First the candidate should shave off his hair and beard and put on ocher robes. He should then arrange his upper robe over one shoulder, pay respect at the feet of the monks, squat on his heels, and raise his joined palms. He should then be told to say this: 

\begin{verse}%
‘I\marginnote{54.3.5} go for refuge to the Buddha, \\
I go for refuge to the Teaching, \\
I go for refuge to the Sangha. 

For\marginnote{54.3.8} the second time, I go for refuge to the Buddha, \\
For the second time, I go for refuge to the Teaching, \\
For the second time, I go for refuge to the Sangha. 

For\marginnote{54.3.11} the third time, I go for refuge to the Buddha, \\
For the third time, I go for refuge to the Teaching, \\
For the third time, I go for refuge to the Sangha.’” 

%
\end{verse}

And\marginnote{54.4.1} \textsanskrit{Sāriputta} gave \textsanskrit{Rāhula} the going forth. 

Soon\marginnote{54.4.2} afterwards Suddhodana went to the Buddha, bowed, sat down, and said, “Sir, I want to ask for a favor.” 

“Buddhas\marginnote{54.4.5} don’t grant favors, Gotama.” 

“It’s\marginnote{54.4.6} allowable and blameless.” 

“Well\marginnote{54.4.7} then, say what it is.” 

“When\marginnote{54.5.1} the Buddha went forth, it was very painful for me, and the same when Nanda went forth. With \textsanskrit{Rāhula}, it’s even worse. Affection for a child cuts deep. It cuts through the outer and inner skin; it cuts through the flesh, the sinews, and the bones, and it reaches all the way to the bone-marrow. Please, may the venerables not give the going forth to a child without the parents’ permission.” 

The\marginnote{54.6.1} Buddha then instructed, inspired, and gladdened him with a teaching, after which Suddhodana got up from his seat, bowed down, circumambulated the Buddha with his right side toward him, and left. Soon afterwards the Buddha gave a teaching and addressed the monks: 

\scrule{“You shouldn’t give the going forth to a child without the parents’ permission. If you do, you commit an offense of wrong conduct.” }

After\marginnote{55.1.1} staying at Kapilavatthu for as long as he liked, the Buddha set out wandering toward \textsanskrit{Sāvatthī}. When he eventually arrived, he stayed in the Jeta Grove, \textsanskrit{Anāthapiṇḍika}’s Monastery. 

At\marginnote{55.1.4} this time a family that was supporting \textsanskrit{Sāriputta} sent him a boy with this message: “Please give the going forth to this boy.” 

\textsanskrit{Sāriputta}\marginnote{55.1.6} thought, “The Buddha has laid down a rule that a monk shouldn’t have two novices attend on him. I already have the novice \textsanskrit{Rāhula}. So what should I do now?” He told the Buddha. 

\scrule{“I allow a competent and capable monk to have two novice monks attend on him, or however many he’s able to teach and instruct.” }

\section*{42. Discussion of the training rules }

Soon\marginnote{56.1.1} afterwards the novices thought, “How many training rules do we have that we should train in?” They told the Buddha. … 

“There\marginnote{56.1.4} are ten training rules for the novice monks: 

\begin{enumerate}%
\item Abstention from killing living beings %
\item Abstention from stealing %
\item Abstention from sexual activity %
\item Abstention from lying %
\item Abstention from alcoholic drinks that cause heedlessness %
\item Abstention from eating at the wrong time %
\item Abstention from dancing, singing, music, and seeing shows %
\item Abstention from wearing garlands and using scents and cosmetics %
\item Abstention from high and luxurious resting places\footnote{“Resting place” renders \textit{sayana}, often translated as “bed”. As can be seen from \href{https://suttacentral.net/pli-tv-kd16/en/brahmali\#8.1}{Kd 16:8.1}–8.11, the \textit{sayana} was used for both sitting and lying down. } %
\item Abstention from receiving gold, silver, and money.”\footnote{“Gold, silver, and money” renders \textit{\textsanskrit{jātarūparajata}}. For a discussion of this compound, see Appendix of Technical Terms. } %
\end{enumerate}

\section*{43. Penalties }

Soon\marginnote{57.1.1} the novice monks were being disrespectful, undeferential, and rude toward the monks. The monks complained and criticized them, “How can the novices behave like this?” They told the Buddha. … 

\scrule{“I allow you to penalize a novice monk who has five qualities: }

\begin{enumerate}%
\item He’s trying to stop monks from getting material support %
\item He’s trying to harm monks %
\item He’s trying to make monks lose their place of residence %
\item He abuses and reviles monks %
\item He causes division between monks.” %
\end{enumerate}

The\marginnote{57.2.1} monks didn’t know which penalty to impose. They told the Buddha. 

\scrule{“I allow you to place restrictions on the novice monks.” }

The\marginnote{57.2.5} monks restricted the novices from the whole monastery. Because they were unable to enter the monastery, the novices left, disrobed, and joined the monastics of other religions. They told the Buddha. 

\scrule{“You shouldn’t restrict anyone from a whole monastery. If you do, you commit an offense of wrong conduct. I allow you to make restrictions for the place you’re staying and its access areas.”\footnote{Sp 3.107: \textit{Yattha \textsanskrit{vā} vasati yattha \textsanskrit{vā} \textsanskrit{paṭikkamatīti} yattha vasati \textsanskrit{vā} pavisati \textsanskrit{vā}}, “\textit{Yattha \textsanskrit{vā} vasati yattha \textsanskrit{vā} \textsanskrit{paṭikkamati}}: where one lives or enters.” } }

The\marginnote{57.3.1} monks placed restrictions on the novices’ food. People making congee and meals for the Sangha said to the novices, “Come, venerables, and drink congee. Come and eat a meal.” 

The\marginnote{57.3.5} novices replied, “We can’t. The monks have placed a restriction on us.” 

People\marginnote{57.3.8} complained and criticized them, “How can the venerables restrict the novices’ food?” They told the Buddha. 

\scrule{“You shouldn’t place restrictions on food. If you do, you commit an offense of wrong conduct.” }

\scend{The account of penalties is finished. }

\section*{44. Prohibiting without asking permission }

On\marginnote{58.1.1} one occasion the monks from the group of six placed restrictions on novices without asking their preceptors for permission. The preceptors could not find their novices. When other monks told them what had happened, the preceptors complained and criticized those monks, “How could the monks from the group of six place restrictions on our novices without asking us for permission?” They told the Buddha. 

\scrule{“You shouldn’t place a restriction without asking permission from the preceptor. If you do, you commit an offense of wrong conduct.” }

\section*{45. Luring away }

At\marginnote{59.1.1} one time the monks from the group of six were luring away the novices of the senior monks. The senior monks had to get their own tooth cleaners and water for rinsing the mouth. As a result, they became tired. They told the Buddha. 

\scrule{“You shouldn’t lure away another’s followers. If you do, you commit an offense of wrong conduct.” }

\section*{46. The novice \textsanskrit{Kaṇṭaka} }

At\marginnote{60.1.1} one time Venerable Upananda the Sakyan had a novice monk called \textsanskrit{Kaṇṭaka} who raped a nun called \textsanskrit{Kaṇṭakī}.\footnote{“Raped” renders \textit{\textsanskrit{dūsesi}}. See Appendix of Technical Terms. } The monks complained and criticized him, “How could a novice monk misbehave in this way?” They told the Buddha. 

\scrule{“I allow you to expel a novice monk who has ten qualities:\footnote{“To expel” renders \textit{\textsanskrit{nāsetuṁ}}. See Appendix of Technical Terms. } }

\begin{enumerate}%
\item He kills living beings %
\item He steals %
\item He’s not celibate %
\item He lies %
\item He drinks alcoholic drinks %
\item He disparages the Buddha %
\item He disparages the Teaching %
\item He disparages the Sangha %
\item He has wrong view %
\item He has raped a nun.”\footnote{Sp 3.115: \textit{\textsanskrit{Bhikkhunidūsako} bhikkhaveti ettha yo \textsanskrit{pakatattaṁ} \textsanskrit{bhikkhuniṁ} \textsanskrit{tiṇṇaṁ} \textsanskrit{maggānaṁ} \textsanskrit{aññatarasmiṁ} \textsanskrit{dūseti}, \textsanskrit{ayaṁ} \textsanskrit{bhikkhunidūsako} \textsanskrit{nāma}}, “\textit{\textsanskrit{Bhikkhunidūsako} bhikkhave}: in this context it means whoever violates an ordinary nun through one of three orifices (vagina, anus, or mouth) is called a \textit{\textsanskrit{bhikkhunidūsaka}}.” } %
\end{enumerate}

\section*{47. \textit{\textsanskrit{Paṇḍakas}} }

At\marginnote{61.1.1} one time a certain \textit{\textsanskrit{paṇḍaka}} had gone forth as a monk. He went to the young monks and said, “Come, venerables, have sex with me.” 

The\marginnote{61.1.4} monks dismissed him, “Go away, \textit{\textsanskrit{paṇḍaka}}. We don’t want you.” 

He\marginnote{61.1.6} went to the big and fat novices, said the same thing, and got the same response. He then went to the elephant keepers and the horse keepers and once again said the same thing. And they had sex with him. 

They\marginnote{61.1.13} complained and criticized him, “These Sakyan monastics are \textit{\textsanskrit{paṇḍakas}}. And those who are not have sex with them. None of them is celibate.” 

The\marginnote{61.1.17} monks heard their complaints and told the Buddha. 

\scrule{“A \textit{\textsanskrit{paṇḍaka}} shouldn’t be given the full ordination. If it has been given, he should be expelled.”\footnote{For the meaning of \textit{\textsanskrit{paṇḍaka}}, see Appendix of Technical Terms. } }

\section*{48. Fake monks }

At\marginnote{62.1.1} one time there was a gentleman who had been brought up in comfort, but whose entire family had died. He thought, “I’ve been brought up in comfort and I’m incapable of making money. How can I live happily without exhausting myself?” It occurred to him, “These Sakyan monastics have pleasant habits and a happy life. They eat nice food and sleep in beds sheltered from the wind. Why don’t I just get myself a bowl and robes, shave off my hair and beard, put on ocher robes, and then go to the monastery and live with the monks?” And he did just that. 

When\marginnote{62.2.1} he came to the monastery, he bowed down to the monks. The monks asked him, “How many rains do you have?” 

“What\marginnote{62.2.4} does ‘How many rains’ mean?” 

“Who’s\marginnote{62.2.5} your preceptor?” 

“What’s\marginnote{62.2.6} a preceptor?” 

The\marginnote{62.2.7} monks said to Venerable \textsanskrit{Upāli}, “\textsanskrit{Upāli}, please examine this person.” 

He\marginnote{62.3.1} then told \textsanskrit{Upāli} what had happened. \textsanskrit{Upāli} told the monks, who in turn told the Buddha. 

\scrule{“A fake monk shouldn’t be given the full ordination. If it has been given, they should be expelled. }

\scrule{Anyone who has previously left to join the monastics of another religion shouldn’t be given the full ordination. If it has been given, they should be expelled.”\footnote{“Who has previously left to join the monastics of another religion” renders \textit{titthiyapakkantaka}, literally, “one who has left for another religion”. Sp 3.110: \textit{Ettha pana titthiyesu pakkanto \textsanskrit{paviṭṭhoti} titthiyapakkantako. … upasampanno bhikkhu titthiyo \textsanskrit{bhavissāmīti} \textsanskrit{saliṅgeneva} \textsanskrit{tesaṁ} \textsanskrit{upassayaṁ} gacchati, \textsanskrit{padavāre} \textsanskrit{padavāre} \textsanskrit{dukkaṭaṁ}. \textsanskrit{Tesaṁ} \textsanskrit{liṅge} \textsanskrit{ādinnamatte} titthiyapakkantako hoti}, “Here \textit{titthiyapakkantaka} means one who has left and entered among the monastics of another religion. … If a fully ordained monk thinks, ‘I will become a monastic of another religion’, and he goes to their dwelling place while looking like a Buddhist monk, then each step is an instance of wrong conduct. Then, merely by taking on their characteristics, he is a \textit{titthiyapakkantaka}.” } }

\section*{49. Animals }

At\marginnote{63.1.1} one time there was a dragon who was troubled, ashamed, and disgusted with his existence as a dragon. He thought, “How can I get released from this existence and quickly become human?” It occurred to him, “These Sakyan monastics have integrity. They’re celibate and their conduct is good, and they’re truthful, moral, and have a good character. If I were to go forth with them, I would be released from this existence as a dragon and quickly become human.” 

Then,\marginnote{63.2.1} taking on the appearance of a young brahmin, that dragon went to the monks and asked for the going forth. The monks gave him the going forth and the full ordination. 

Soon\marginnote{63.2.3} afterwards that dragon was sharing a remote dwelling with a certain monk. After getting up early one morning, that monk did walking meditation outside. When the monk had left, the dragon relaxed and fell asleep. As a result, the serpent filled the whole dwelling, its coils even protruding from the windows. Just then that monk decided to go back inside. When he opened the door, he saw the serpent filling the whole dwelling. Terrified, he screamed. Monks came running to and asked him why he was screaming. And he told them what had seen. 

The\marginnote{63.3.5} dragon woke up from the noise and sat down on his seat. The monks asked him who he was. He replied, “I’m a dragon.” 

“Why\marginnote{63.3.9} did you do this?” The dragon told them what had happened, and they told the Buddha. 

He\marginnote{63.4.1} then had the Sangha of monks gathered and said to the dragon, “Dragons are unable to make progress on this spiritual path. Go, dragon, and keep the observance days of the fourteenth, the fifteenth, and the eighth of the lunar half-month. In this way you’ll be released from existence as a dragon and quickly become human.” 

When\marginnote{63.4.4} he heard this, the dragon wept. Sad and miserable, he cried out in distress and left. And the Buddha addressed the monks: 

“There\marginnote{63.5.2} are two occasions when dragons appear in their own form: when they have sexual intercourse with each other, and when they relax and fall asleep. 

\scrule{Monks, an animal shouldn’t be given the full ordination. If it has been given, it should be expelled.” }

\section*{50. Matricides }

At\marginnote{64.1.1} one time there was a young brahmin who had murdered his mother. He was troubled, ashamed, and disgusted by what he had done, and he thought, “How can I escape from this terrible action?” It occurred to him, “These Sakyan monastics have integrity. They’re celibate and their conduct is good, and they’re truthful, moral, and have a good character. If I were to go forth with them, I might be released from this bad deed.” 

He\marginnote{64.2.1} then went to the monks and asked for the going forth. The monks said to \textsanskrit{Upāli}, “Previously a dragon appearing as a young brahmin asked for the going forth. So, please examine this young brahmin, \textsanskrit{Upāli}.” 

The\marginnote{64.2.5} young brahmin told \textsanskrit{Upāli} what had happened. \textsanskrit{Upāli} told the monks, who in turn told the Buddha. 

\scrule{“A matricide shouldn’t be given the full ordination. If it has been given, he should be expelled.” }

\section*{51. Patricides }

At\marginnote{65.1.1} one time there was a young brahmin who had murdered his father. He was troubled, ashamed, and disgusted by what he had done, and he thought, “How can I escape from this terrible action?” It occurred to him, “These Sakyan monastics have integrity. They’re celibate and their conduct is good, and they’re truthful, moral, and have a good character. If I were to go forth with them, I might be released from this bad action.” 

He\marginnote{65.1.7} then went to the monks and asked for the going forth. The monks said to \textsanskrit{Upāli}, “Previously a dragon appearing as a young brahmin asked for the going forth. So, please examine this young brahmin, \textsanskrit{Upāli}.” 

The\marginnote{65.1.11} young brahmin told \textsanskrit{Upāli} what had happened. \textsanskrit{Upāli} told the monks, who in turn told the Buddha. 

\scrule{“A patricide shouldn’t be given the full ordination. If it has been given, he should be expelled.” }

\section*{52. Murderers of perfected ones }

On\marginnote{66.1.1} one occasion a number of monks were traveling from \textsanskrit{Sāketa} to \textsanskrit{Sāvatthī}. While on their way, they were attacked by gangsters. Some of the monks were robbed and some were killed. 

The\marginnote{66.1.3} king’s men came out from \textsanskrit{Sāvatthī}. They caught some of the criminals, while others escaped. Those who escaped went forth with the monks, but those who were caught were taken away for execution. Those who had gone forth saw the others being taken away for execution. They said, “It’s good that we escaped. Had we been caught, we would’ve been executed, too.” 

The\marginnote{66.2.3} monks asked, “But what have you done?” They told the monks what had happened, and the monks told the Buddha. 

“Those\marginnote{66.2.7} monks were perfected ones. 

\scrule{A murderer of a perfected one shouldn’t be given the full ordination. If it has been given, he should be expelled.” }

\section*{53. Rapists of nuns }

On\marginnote{67.1.1} one occasion a number of nuns were traveling from \textsanskrit{Sāketa} to \textsanskrit{Sāvatthī}. While on their way, they were attacked by gangsters. Some of the nuns were robbed and some were raped. 

The\marginnote{67.1.3} king’s men came out from \textsanskrit{Sāvatthī}. They caught some of the criminals, while others escaped. Those who escaped went forth with the monks, but those who were caught were taken away for execution. Those who had gone forth saw the others being taken away for execution. They said, “It’s good that we escaped. Had we been caught, we would’ve been executed, too.” 

The\marginnote{67.1.8} monks asked, “But what have you done?” They told the monks what had happened, and the monks told the Buddha. 

\scrule{“One who has raped a nun shouldn’t be given the full ordination. If it has been given, he should be expelled. }

\scrule{One who has caused a schism in the Sangha shouldn’t be given the full ordination. If it has been given, he should be expelled. }

\scrule{One who has caused the Buddha to bleed shouldn’t be given the full ordination. If it has been given, he should be expelled.” }

\section*{54. Hermaphrodites }

At\marginnote{68.1.1} one time a hermaphrodite had gone forth as a monk. He had sex and made others have it.\footnote{Sp 3.116: \textit{\textsanskrit{Karotīti} purisanimittena \textsanskrit{itthīsu} \textsanskrit{methunavītikkamaṁ} karoti. \textsanskrit{Kārāpetīti} \textsanskrit{paraṁ} \textsanskrit{samādapetvā} attano itthinimitte \textsanskrit{kārāpeti}}, “\textit{Karoti}: with the male characteristic he acts to transgress through sexual intercourse with women. \textit{\textsanskrit{Kārāpeti}}: having encouraged another, he causes action in his own female characteristic.” The meaning of the causative \textit{\textsanskrit{kārāpeti}}, however, is usually to make someone else act, not specifically to cause someone to act towards oneself. If this is correct, then the meaning here would be that one has sex oneself and generally causes others to have sex, not that the same person takes on different roles. I translate accordingly. } They told the Buddha. 

\scrule{“A hermaphrodite shouldn’t be given the full ordination. If it has been given, he should be expelled.”\footnote{For the meaning of \textit{\textsanskrit{ubhatobyañjanka}}, see Appendix of Technical Terms. } }

\section*{55. Those without a preceptor, etc. }

On\marginnote{69.1.1} one occasion the monks gave the full ordination to someone without a preceptor. They told the Buddha. 

\scrule{“You shouldn’t give the full ordination to someone without a preceptor. If you do, you commit an offense of wrong conduct.” }

On\marginnote{69.2.1} one occasion the monks gave the full ordination to someone with the Sangha as preceptor. They told the Buddha. 

\scrule{“You shouldn’t give the full ordination with the Sangha as preceptor. If you do, you commit an offense of wrong conduct.” }

On\marginnote{69.3.1} one occasion the monks gave the full ordination to someone with a group as preceptor. They told the Buddha. 

\scrule{“You shouldn’t give the full ordination with a group as preceptor. If you do, you commit an offense of wrong conduct.” }

On\marginnote{69.4.1} one occasion the monks gave the full ordination with a \textit{\textsanskrit{paṇḍaka}} as preceptor … with a fake monk as preceptor … with one who has previously left to join the monastics of another religion as preceptor … with an animal as preceptor … with a matricide as preceptor … with a patricide as preceptor … with a murderer of a perfected one as preceptor … with one who had raped a nun as preceptor … with one who had caused a schism in the Sangha as preceptor … with one who had caused the Buddha to bleed as preceptor … with a hermaphrodite as preceptor. They told the Buddha. 

\scrule{“You shouldn’t give the full ordination with a \textit{\textsanskrit{paṇḍaka}} as preceptor, with a fake monk as preceptor, with one who has previously left to join the monastics of another religion as preceptor, with an animal as preceptor, with a matricide as preceptor, with a patricide as preceptor, with a murderer of a perfected one as preceptor, with one who has raped a nun as preceptor, with one who has caused a schism in the Sangha as preceptor, with one who has caused the Buddha to bleed as preceptor, or with a hermaphrodite as preceptor. If you do, you commit an offense of wrong conduct.” }

\section*{56. Those without an almsbowl, etc. }

On\marginnote{70.1.1} one occasion the monks gave the full ordination to someone without an almsbowl. When walking for alms, he received it in his hands. People complained and criticized him, “He’s just like the monastics of other religions.” They told the Buddha. 

\scrule{“You shouldn’t give the full ordination to someone without an almsbowl. If you do, you commit an offense of wrong conduct.” }

On\marginnote{70.2.1} one occasion the monks gave the full ordination to someone without robes. He walked naked for alms. People complained and criticized him, “He’s just like the monastics of other religions.” They told the Buddha. 

\scrule{“You shouldn’t give the full ordination to someone without robes. If you do, you commit an offense of wrong conduct.” }

On\marginnote{70.3.1} one occasion the monks gave the full ordination to someone with neither almsbowl nor robes. He walked naked for alms and received it in his hands. People complained and criticized him, “He’s just like the monastics of other religions.” They told the Buddha. 

\scrule{“You shouldn’t give the full ordination to someone with neither almsbowl nor robes. If you do, you commit an offense of wrong conduct.” }

On\marginnote{70.4.1} one occasion the monks gave the full ordination to someone with a borrowed almsbowl. When he had been ordained, they took back the bowl. Then, when walking for alms, he received it in his hands. People complained and criticized him, “He’s just like the monastics of other religions.” They told the Buddha. 

\scrule{“You shouldn’t give the full ordination to someone with a borrowed almsbowl. If you do, you commit an offense of wrong conduct.” }

On\marginnote{70.5.1} one occasion the monks gave the full ordination to someone with borrowed robes. When he had been ordained, they took back the robes. He then walked naked for alms. People complained and criticized him, “He’s just like the monastics of other religions.” They told the Buddha. 

\scrule{“You shouldn’t give the full ordination to someone with borrowed robes. If you do, you commit an offense of wrong conduct.” }

On\marginnote{70.6.1} one occasion the monks gave the full ordination to someone with a borrowed almsbowl and borrowed robes. When he had been ordained, they took back the bowl and the robes. He then walked naked for alms and received it in his hands. People complained and criticized him, “He’s just like the monastics of other religions.” They told the Buddha. 

\scrule{“You shouldn’t give the full ordination to someone with a borrowed almsbowl and borrowed robes. If you do, you commit an offense of wrong conduct.” }

\scend{The section consisting of twenty-one cases when the full ordination is not to be given is finished. }

\section*{57. The section consisting of thirty-two cases when the going forth is not to be given }

On\marginnote{71.1.1} one occasion the monks gave the going forth to someone without a hand … to someone without a foot … to someone without a hand and foot … to someone without an ear … to someone without a nose … to someone without an ear and nose … to someone without a finger or toe … to someone with a cut tendon … to someone with joined fingers … to a hunchback … to a dwarf … to someone with goiter … to someone who had been branded … to someone who had been whipped … to a wanted criminal … to someone with elephantiasis … to someone with a serious sickness … to someone with abnormal appearance … to someone blind in one eye … to someone with a crooked limb … to someone lame … to someone paralyzed on one side … to someone crippled … to someone weak from old age … to someone blind … to a mute … to someone deaf … to someone blind and mute … to someone blind and deaf … to someone mute and deaf … to someone blind, mute, and deaf. They told the Buddha. … 

\scrule{“You shouldn’t give the going forth to someone without a hand, to someone without a foot, to someone without a hand and foot, to someone without an ear, to someone without a nose, to someone without an ear and nose, to someone without a finger or toe,\footnote{This single phrase combines two Pali terms, \textit{\textsanskrit{aṅgulicchinna}} and \textit{\textsanskrit{aḷacchinna}}. The latter refers to a thumb or a big toe, whereas the former refers to any of the remaining eight digits. } to someone with a cut tendon, to someone with joined fingers,\footnote{\textit{\textsanskrit{Phaṇahatthaka}}, literally, “One who has hands like a snake’s hood”. Sp 3.119: \textit{\textsanskrit{Phaṇahatthakoti} yassa \textsanskrit{vaggulipakkhakā} viya \textsanskrit{aṅguliyo} \textsanskrit{sambaddhā} honti}, “\textit{\textsanskrit{Phaṇahatthako}}: one whose fingers are connected like the wings of a bat.” } to a hunchback, to a dwarf, to someone with goiter, to someone who has been branded, to someone who has been whipped, to a wanted criminal, to someone with elephantiasis, to someone with a serious sickness, to someone with abnormal appearance,\footnote{\textit{\textsanskrit{Parisadūsaka}}, literally, “One who defiles an assembly”. Sp 3.119: \textit{\textsanskrit{Parisadūsakoti} yo attano \textsanskrit{virūpatāya} \textsanskrit{parisaṁ} \textsanskrit{dūseti}; \textsanskrit{atidīgho} \textsanskrit{vā} hoti \textsanskrit{aññesaṁ} \textsanskrit{sīsappamāṇanābhippadeso}, atirasso \textsanskrit{vā} …}, “\textit{\textsanskrit{Parisadūsaka}}: whoever defiles an assembly through his own bad appearance. He is too tall, a head taller than others, or he is too short …” } to someone blind in one eye, to someone with a crooked limb, to someone lame, to someone paralyzed on one side,\footnote{Sp 3.119: \textit{Pakkhahatoti yassa eko hattho \textsanskrit{vā} \textsanskrit{pādo} \textsanskrit{vā} \textsanskrit{aḍḍhasarīraṁ} \textsanskrit{vā} \textsanskrit{sukhaṁ} na vahati}, “\textit{Pakkhahata}: for whom one hand or one foot or half the body does not work properly.” } to someone crippled,\footnote{\textit{\textsanskrit{Chinniriyāpatha}}, literally, “The ways of movement have been cut off”. Sp 3.119: \textit{\textsanskrit{Chinniriyāpathoti} \textsanskrit{pīṭhasappi} vuccati}, “One who crawls is called \textit{\textsanskrit{chinniriyāpatha}}.” The exact meaning is not clear. } to someone weak from old age, to someone blind, to a mute, to someone deaf, to someone blind and mute, to someone blind and deaf, to someone mute and deaf, or to someone blind, mute, and deaf. If you do, you commit an offense of wrong conduct.” }

\scend{The section consisting of thirty-two cases when the going forth is not to be given is finished. }

\scend{The ninth section for recitation on inheritance is finished. }

\section*{58. Formal support for shameless monks }

At\marginnote{72.1.1} that time the monks from the group of six gave formal support to shameless monks. They told the Buddha. 

\scrule{“You shouldn’t give formal support to shameless monks. If you do, you commit an offense of wrong conduct.” }

At\marginnote{72.1.5} that time monks lived with formal support from shameless monks. Soon they too became shameless and bad. They told the Buddha. 

\scrule{“You shouldn’t live with formal support from shameless monks. If you do, you commit an offense of wrong conduct.” }

The\marginnote{72.1.10} monks thought, “The Buddha has laid down a rule that one should neither give formal support to shameless monks nor live with formal support from them. But how do we know who is shameless and who is not?” They told the Buddha. 

\scrule{“I allow you to wait for four or five days to find out if he is keeping the same standard as the monks.”\footnote{Sp 3.120: \textit{\textsanskrit{Yāva} \textsanskrit{bhikkhusabhāgataṁ} \textsanskrit{jānāmīti} \textsanskrit{nissayadāyakassa} bhikkhuno \textsanskrit{bhikkhūhi} \textsanskrit{sabhāgataṁ} \textsanskrit{lajjibhāvaṁ} \textsanskrit{yāva} \textsanskrit{jānāmīti} attho. \textsanskrit{Tasmā} \textsanskrit{navaṁ} \textsanskrit{ṭhānaṁ} gatena “ehi bhikkhu, \textsanskrit{nissayaṁ} \textsanskrit{gaṇhāhī}”ti \textsanskrit{vuccamānenāpi} \textsanskrit{catūhapañcāhaṁ} \textsanskrit{nissayadāyakassa} \textsanskrit{lajjibhāvaṁ} \textsanskrit{upaparikkhitvā} nissayo gahetabbo}, “The meaning of ‘to find out if the other person is compatible with the monks’ is: until I find out the compatibility with the monks in terms of conscientiousness of the monk giving support. Therefore, when a monk who has gone to a new place is being told to obtain support, he should observe the conscientiousness of the support giver for four or five days, and then obtain support.” } }

\section*{59. Formal support for those who are traveling, etc. }

On\marginnote{73.1.1} one occasion a certain monk was traveling through the Kosalan country. He thought, “The Buddha has laid down a rule that a monk like me shouldn’t live without formal support. But I’m traveling. So what should I do?” They told the Buddha. 

\scrule{“If you are traveling and unable to obtain formal support, I allow you to live without.” }

On\marginnote{73.2.1} one occasion two monks were traveling through the country of Kosala, when they arrived at a certain monastery. Just then one of them got sick. He thought, “The Buddha has laid down a rule that a monk like me shouldn’t live without formal support. But I’m sick. So what should I do?” They told the Buddha. 

\scrule{“If you are sick and unable to obtain formal support, I allow you to live without.” }

Then\marginnote{73.3.1} the monk who was nursing him thought, “The Buddha has laid down a rule that a monk like me shouldn’t live without formal support. But this monk is sick. So what should I do?” They told the Buddha. 

\scrule{“If you have been asked to nurse someone who is sick and you are unable to obtain formal support, I allow you to live without.” }

At\marginnote{73.4.1} one time there was a certain monk who was staying in the wilderness. He was enjoying his dwelling. He thought, “The Buddha has laid down a rule that a monk like me shouldn’t live without formal support. Yet I’m enjoying my dwelling in the wilderness. So what should I do?” They told the Buddha. 

\scrule{“If you notice that you are enjoying your stay in the wilderness, but unable to obtain formal support, I allow you to live without. When a suitable support-giver comes, you should live with formal support from him.” }

\section*{60. The allowance to make proclamations using the family name }

At\marginnote{74.1.1} one time a certain person wanted the full ordination with Venerable \textsanskrit{Mahākassapa}. \textsanskrit{Mahākassapa} sent a message to Venerable Ānanda: “Please come, Ānanda, and do the proclamation.” Ānanda thought, “Because I respect the elder so much, I can’t say his name.” They told the Buddha. 

\scrule{“I allow you to do the proclamation also using the family name.” }

\section*{61. The two people seeking the full ordination, etc. }

At\marginnote{74.2.1} one time there were two people who wanted the full ordination with Venerable \textsanskrit{Mahākassapa}. They argued about who should be ordained first. They told the Buddha. 

\scrule{“I allow you to give the full ordination to two people with a single proclamation.” }

At\marginnote{74.3.1} one time there were a number of people who wanted the full ordination with several senior monks.\footnote{“Several” renders \textit{sambahula}. See Appendix of Technical Terms. } They argued with one another about who should be ordained first. The senior monks said, “Well then, let’s ordain all of them with a single proclamation.” They told the Buddha. 

\scrule{“I allow you to give the full ordination to two or three people with a single proclamation, but only with a single preceptor, not with many.” }

\section*{62. The allowance to be fully ordained when one is twenty years old since appearing in the womb }

At\marginnote{75.1.1} that time Venerable \textsanskrit{Kumārakassapa} had been given the full ordination twenty years after he appeared in the womb. He thought, “The Buddha has laid down a rule that a person less than twenty years old shouldn’t be given the full ordination. I was ordained twenty years after appearing in the womb. I wonder, have I been ordained or not?” They told the Buddha. 

\scrule{“When the mind first appears in the mother’s womb, when the consciousness first manifests, that’s a person’s birth. I allow you to give the full ordination to someone who is twenty years old since appearing in the womb.” }

\section*{63. The process of full ordination }

At\marginnote{76.1.1} that time the full ordination had been given to people who had leprosy, abscesses, mild leprosy, tuberculosis, and epilepsy.\footnote{For an explanation of these, see Appendix of Medical Terminology. } They told the Buddha. 

\scrule{“The one who is giving the full ordination should ask about thirteen obstacles. }

It\marginnote{76.1.4} should be done like this: ‘Do you have any of these diseases: leprosy, abscesses, mild leprosy, tuberculosis, or epilepsy? Are you human? Are you a man? Are you free from slavery? Are you free from debt? Are you employed by the king? Do you have your parents’ permission? Are you twenty years old? Do you have a full set of bowl and robes? What’s your name? What’s the name of your preceptor?’” 

Soon\marginnote{76.2.1} afterwards they asked those seeking the full ordination about the obstacles without first instructing them. They were embarrassed, humiliated, and unable to respond. They told the Buddha. 

\scrule{“You should instruct first and then ask about the obstacles. }

They\marginnote{76.3.1} instructed them right there in the midst of the Sangha. Once more those seeking the full ordination were embarrassed, humiliated, and unable to respond. They told the Buddha. 

\scrule{“You should instruct them at a distance and then ask about the obstacles in the midst of the Sangha. }

And\marginnote{76.3.5} it should be done like this. First they should be told to choose a preceptor. Their bowls and robes should then be pointed out to them: ‘This is your bowl, this your outer robe, this your upper robe, and this your sarong. Now please go and stand over there.’” 

Then\marginnote{76.4.1} they were instructed by monks who were ignorant and incompetent. And because they were badly instructed, they were once again embarrassed, humiliated, and unable to respond. They told the Buddha. 

\scrule{“A monk who is ignorant and incompetent shouldn’t instruct. If he does, he commits an offense of wrong conduct. A monk who is competent and capable should instruct.” }

They\marginnote{76.5.1} instructed without having been appointed. They told the Buddha. 

\scrule{“A monk shouldn’t instruct if he hasn’t been appointed. If he does, he commits an offense of wrong conduct. I allow a monk to instruct if he’s been appointed to do so. }

And\marginnote{76.5.6} it should be done like this. One is either appointed through oneself or through someone else. How is one appointed through oneself? A competent and capable monk should inform the Sangha: 

‘Please,\marginnote{76.5.10} venerables, I ask the Sangha to listen. So-and-so is seeking the full ordination with venerable so-and-so. If the Sangha is ready, I will instruct so-and-so.’ 

And\marginnote{76.6.1} how is one appointed through someone else? A competent and capable monk should inform the Sangha: 

‘Please,\marginnote{76.6.3} venerables, I ask the Sangha to listen. So-and-so is seeking the full ordination with venerable so-and-so. If the Sangha is ready, so-and-so will instruct so-and-so.’ 

The\marginnote{76.7.1} appointed monk should go to the one who is seeking the full ordination and say this: 

‘Listen,\marginnote{76.7.2} so-and-so. Now is the time for you to tell the truth. You will be asked in the midst of the Sangha about various matters. If something is true, you should say, “Yes,” and if it’s not, you should say, “No.” Don’t be embarrassed or humiliated. This is what they’ll ask you: “Do you have any of these diseases: leprosy, abscesses, mild leprosy, tuberculosis, or epilepsy? Are you human? Are you a man? Are you free from slavery? Are you free from debt? Are you employed by the king? Do you have your parents’ permission? Are you twenty years old? Do you have a full set of bowl and robes? What’s your name? What’s the name of your preceptor?”’” 

They\marginnote{76.8.1} then returned to the Sangha together. 

The\marginnote{76.8.2} Buddha said, “They shouldn’t return together. The instructor should come first and inform the Sangha: 

‘Please,\marginnote{76.8.4} venerables, I ask the Sangha to listen. So-and-so is seeking the full ordination with venerable so-and-so. He’s been instructed by me. If the Sangha is ready, so-and-so should come.’ 

And\marginnote{76.8.8} he should be told to come. He should then arrange his upper robe over one shoulder, pay respect at the feet of the monks, squat on his heels, and raise his joined palms. He should then ask for the full ordination: 

‘Venerables,\marginnote{76.8.10} I ask the Sangha for the full ordination. Please lift me up out of compassion. For the second time, venerables, I ask the Sangha for the full ordination. Please lift me up out of compassion. For the third time, venerables, I ask the Sangha for the full ordination. Please lift me up out of compassion.’ A competent and capable monk should then inform the Sangha: 

‘Please,\marginnote{76.9.2} venerables, I ask the Sangha to listen. So-and-so is seeking the full ordination with venerable so-and-so. If the Sangha is ready, I will ask so-and-so about the obstacles. 

Listen,\marginnote{76.9.5} so-and-so. Now is the time for you to tell the truth. I will ask you about various matters. If something is true, you should say, “Yes,” and if it’s not, you should say, “No.” Do you have any of these diseases: leprosy, abscesses, mild leprosy, tuberculosis, or epilepsy? Are you human? Are you a man? Are you free from slavery? Are you free from debt? Are you employed by the king? Do you have your parents’ permission? Are you twenty years old? Do you have a full set of bowl and robes? What’s your name? What’s the name of your preceptor?’ 

A\marginnote{76.10.1} competent and capable monk should then inform the Sangha: 

‘Please,\marginnote{76.10.2} venerables, I ask the Sangha to listen. So-and-so is seeking the full ordination with venerable so-and-so. He is free from obstacles and his bowl and robes are complete.\footnote{The Pali reads: \textit{\textsanskrit{Ayaṁ} \textsanskrit{itthannāmo} \textsanskrit{itthannāmassa} \textsanskrit{āyasmato} \textsanskrit{upasampadāpekkho}}. Taking the genitive case here to be the agent genitive, which seems to be the most obvious reading, this would mean, “So-and-so who is seeking to be fully ordained \emph{by} venerable so-and-so.” But it is the Sangha that ordains, not individuals, and so this translation does not seem quite right. Vmv 3.126: \textit{\textsanskrit{Ayaṁ} buddharakkhito \textsanskrit{āyasmato} dhammarakkhitassa \textsanskrit{saddhivihārikabhūto} \textsanskrit{upasampadāpekkho}}, “This Buddharakkhita, who is seeking the full ordination, is the student of Venerable Dhammarakkhita.” I have followed this interpretation, and thus my translation “with venerable so-and-so”. } So-and-so is asking the Sangha for the full ordination with so-and-so as his preceptor. If the Sangha is ready, it should give the full ordination to so-and-so with so-and-so as his preceptor. This is the motion. 

Please,\marginnote{76.11.1} venerables, I ask the Sangha to listen. So-and-so is seeking the full ordination with venerable so-and-so. He is free from obstacles and his bowl and robes are complete. So-and-so is asking the Sangha for the full ordination with so-and-so as his preceptor. The Sangha gives the full ordination to so-and-so with so-and-so as his preceptor. Any monk who approves of giving the full ordination to so-and-so with so-and-so as his preceptor should remain silent. Any monk who doesn’t approve should speak up. 

For\marginnote{76.12.1} the second time, I speak on this matter. Please, venerables, I ask the Sangha to listen. So-and-so is seeking the full ordination with venerable so-and-so. He is free from obstacles and his bowl and robes are complete. So-and-so is asking the Sangha for the full ordination with so-and-so as his preceptor. The Sangha gives the full ordination to so-and-so with so-and-so as his preceptor. Any monk who approves of giving the full ordination to so-and-so with so-and-so as his preceptor should remain silent. Any monk who doesn’t approve should speak up. 

For\marginnote{76.12.8} the third time, I speak on this matter. Please, venerables, I ask the Sangha to listen. So-and-so is seeking the full ordination with venerable so-and-so. He is free from obstacles and his bowl and robes are complete. So-and-so is asking the Sangha for the full ordination with so-and-so as his preceptor. The Sangha gives the full ordination to so-and-so with so-and-so as his preceptor. Any monk who approves of giving the full ordination to so-and-so with so-and-so as his preceptor should remain silent. Any monk who doesn’t approve should speak up. 

The\marginnote{76.12.15} Sangha has given the full ordination to so-and-so with so-and-so as his preceptor. The Sangha approves and is therefore silent. I’ll remember it thus.’” 

\scend{The procedure of full ordination is finished. }

\section*{64. The four supports }

“Straightaway\marginnote{77.1.1} the time should be noted and the date should be pointed out. These should be declared jointly to everyone. And the four supports should be pointed out to him:\footnote{Sp 3.128: \textit{\textsanskrit{Chāyā} \textsanskrit{metabbāti} \textsanskrit{ekaporisā} \textsanskrit{vā} \textsanskrit{dviporisā} \textsanskrit{vāti} \textsanskrit{chāyā} \textsanskrit{metabbā}. \textsanskrit{Utuppamāṇaṁ} \textsanskrit{ācikkhitabbanti} “\textsanskrit{vassāno} hemanto gimho”ti \textsanskrit{evaṁ} \textsanskrit{utuppamāṇaṁ} \textsanskrit{ācikkhitabbaṁ}. Ettha ca utuyeva \textsanskrit{utuppamāṇaṁ}. Sace \textsanskrit{vassānādayo} \textsanskrit{aparipuṇṇā} honti, yattakehi divasehi yassa yo utu \textsanskrit{aparipuṇṇo}, te divase \textsanskrit{sallakkhetvā} so \textsanskrit{divasabhāgo} \textsanskrit{ācikkhitabbo}. Atha \textsanskrit{vā} “\textsanskrit{ayaṁ} \textsanskrit{nāma} utu, so ca kho \textsanskrit{paripuṇṇo} \textsanskrit{vā} \textsanskrit{aparipuṇṇo} \textsanskrit{vā}”ti \textsanskrit{evaṁ} \textsanskrit{utuppamāṇaṁ} \textsanskrit{ācikkhitabbaṁ}. “\textsanskrit{Pubbaṇho} \textsanskrit{vā} \textsanskrit{sāyanho} \textsanskrit{vā}”ti \textsanskrit{evaṁ} \textsanskrit{divasabhāgo} \textsanskrit{ācikkhitabbo}}, “\textit{\textsanskrit{Chāyā} \textsanskrit{metabbā}}: ‘It is the height of one man or the height of two men’, the length of the shade is to be measured. ‘The measure of the season (\textit{\textsanskrit{utuppamāṇa}}) should be pointed out’: ‘It is the rainy season, the cold season, the hot season’, in this way the measure of the season should be pointed out. In this case the measure of the season is just the season. If the rainy season, etc., is not complete, one should calculate the days until the completion of the season; that share of days (\textit{\textsanskrit{divasabhāga}}) is to be pointed out. Alternatively, ‘This is the name of the season, and it is complete or incomplete (by so many days)’, in this way the date is to be pointed out. ‘It is morning or evening’, in this way the part of the day is to be pointed out.” I have followed the latter of these two ways of understanding these terms, see the CPD. \textit{\textsanskrit{Saṅgīti} \textsanskrit{ācikkhitabbā}}, literally, “A joint recitation is to be declared”, which is rather cryptic. Sp 3.128: \textit{\textsanskrit{Saṅgītīti} idameva \textsanskrit{sabbaṁ} ekato \textsanskrit{katvā} “\textsanskrit{tvaṁ} \textsanskrit{kiṁ} labhasi, \textsanskrit{kā} te \textsanskrit{chāyā}, \textsanskrit{kiṁ} \textsanskrit{utuppamāṇaṁ}, ko \textsanskrit{divasabhāgo}”ti \textsanskrit{puṭṭho} “\textsanskrit{idaṁ} \textsanskrit{nāma} \textsanskrit{labhāmi} – \textsanskrit{vassaṁ} \textsanskrit{vā} \textsanskrit{hemantaṁ} \textsanskrit{vā} \textsanskrit{gimhaṁ} \textsanskrit{vā}, \textsanskrit{ayaṁ} me \textsanskrit{chāyā}, \textsanskrit{idaṁ} \textsanskrit{utuppamāṇaṁ}, \textsanskrit{ayaṁ} \textsanskrit{divasabhāgoti} \textsanskrit{vadeyyāsī}”ti \textsanskrit{evaṁ} \textsanskrit{ācikkhitabbaṁ}}, “\textit{\textsanskrit{Saṅgīti}}: here it means: having brought everyone together, it should pointed out: “When you are asked, ‘What did you have? What time did you have? What date did you have?’ you should reply, ‘I had this: it was the rainy season/the cold season/the hot season; I had this time; I had this date.’” The point seems to be that a newly ordained monk should remember the time and date of his ordination so that he may respond to questions about it in future. Vmv 3.128 clarifies: \textit{\textsanskrit{Chāyādikameva} \textsanskrit{sabbaṁ} \textsanskrit{saṅgahetvā} \textsanskrit{gāyitabbato} kathetabbato \textsanskrit{saṅgītīti} \textsanskrit{āha} “\textsanskrit{idamevā}”\textsanskrit{tiādi}. Tattha ekato \textsanskrit{katvā} \textsanskrit{ācikkhitabbaṁ}. \textsanskrit{Tvaṁ} \textsanskrit{kiṁ} \textsanskrit{labhasīti} \textsanskrit{tvaṁ} \textsanskrit{upasampādanakāle} \textsanskrit{kataravassaṁ}, \textsanskrit{katarautuñca} labhasi, \textsanskrit{katarasmiṁ} te \textsanskrit{upasampadā} \textsanskrit{laddhāti} attho}, “‘Here’, etc., means: having collected all—that is the time, etc.—\textit{\textsanskrit{saṅgīti}} is said because it is to be chanted, because it is to be declared. In regard to this, having brought (everyone) together, it is to be pointed out. ‘What did you have’ means: at the time of the ordination, which year did you have, which season; your ordination was obtained in which one?” } 

\scrule{‘One gone forth is supported by almsfood. You should persevere with this for life. There are these additional allowances: a meal for the Sangha, a meal for designated monks, an invitational meal, a meal for which lots are drawn, a half-monthly meal, a meal on the observance day, and a meal on the day after the observance day. }

\scrule{One gone forth is supported by rag-robes. You should persevere with this for life. There are these additional allowances: linen, cotton, silk, wool, sunn hemp, and hemp. }

\scrule{One gone forth is supported by the foot of a tree as a resting place. You should persevere with this for life. There are these additional allowances: a dwelling, a stilt house, and a cave.\footnote{For an explanation of the renderings “stilt house” and “cave” respectively for \textit{\textsanskrit{pāsāda}} and \textit{\textsanskrit{guhā}}, see Appendix of Technical Terms. Apart from the \textit{\textsanskrit{vihāra}}, “a dwelling”, and the \textit{\textsanskrit{guhā}}, “a cave”, the Pali mentions three kinds of buildings, the \textit{\textsanskrit{aḍḍhayoga}}, the \textit{\textsanskrit{pāsāda}}, and the \textit{hammiya}, all of which, according to the commentaries, are different kinds of \textit{\textsanskrit{pāsāda}}, “stilt houses”. Rather than try to differentiate between these buildings, which is unlikely to be useful from a practical perspective, I have instead grouped them together as “stilt house”. Here is what the commentaries have to say. Sp 4.294: \textit{\textsanskrit{Aḍḍhayogoti} \textsanskrit{supaṇṇavaṅkagehaṁ}}, “An \textit{\textsanskrit{aḍḍhayoga}} is a house bent like a \textit{\textsanskrit{supaṇṇa}}.” Sp-\textsanskrit{ṭ} 4.294 clarifies: \textit{\textsanskrit{Supaṇṇavaṅkagehanti} \textsanskrit{garuḷapakkhasaṇṭhānena} \textsanskrit{katagehaṁ}}, “\textit{\textsanskrit{Supaṇṇavaṅkageha}}: a house made in the shape of the wings of a \textit{\textsanskrit{garuḷa}}.” A \textit{\textsanskrit{garuḷa}}, better known in its Sanskrit form \textit{\textsanskrit{garuḍa}}, is a mythological bird. Sp 4.294 continues: \textit{\textsanskrit{Pāsādoti} \textsanskrit{dīghapāsādo}. Hammiyanti \textsanskrit{upariākāsatale} \textsanskrit{patiṭṭhitakūṭāgāro} \textsanskrit{pāsādoyeva}}, “A \textit{\textsanskrit{pāsāda}} is a long stilt house. A \textit{hammiya} is just a \textit{\textsanskrit{pāsāda}} that has an upper room on top of its flat roof.” At Sp-\textsanskrit{ṭ} 3.74, however, we find slightly different explanations. It seems clear, however, that all three are stilt houses and that they are distinguished according to their shape and the kind of roof they possess. } }

\scrule{One gone forth is supported by medicine of fermented urine. You should persevere with this for life. There are these additional allowances: ghee, butter, oil, honey, and syrup.’” }

\scend{The four supports are finished. }

\section*{65. The four things not to be done }

On\marginnote{78.1.1} one occasion, the monks gave the full ordination to someone and then departed. The newly ordained monk lagged behind, walking by himself. On the way he met his old wife. She said, “Have you now gone forth?” 

“Yes.”\marginnote{78.1.5} 

“It’s\marginnote{78.1.6} difficult for those gone forth to get sex. Come, let’s have intercourse.” He had intercourse with her. 

When\marginnote{78.1.9} he caught up with the monks, they asked him what had taken him so long. He told them what had happened, and they told the Buddha. 

\scrule{“When you have given the full ordination to someone, you should give him a companion and point out the four things not to be done: }

\scrule{A monk who’s fully ordained shouldn’t have sexual intercourse, not even with an animal. If he has sexual intercourse, he’s not an ascetic, not a Sakyan monastic. Just as a man with his head cut off is unable to continue living by reconnecting it to the body, so too is a monk who has had sexual intercourse not an ascetic, not a Sakyan monastic. You shouldn’t do this for as long as you live. }

\scrule{A monk who’s fully ordained shouldn’t steal, not even a straw. If he steals a \textit{\textsanskrit{pāda}} coin, the value of a \textit{\textsanskrit{pāda}}, or more than a \textit{\textsanskrit{pāda}}, he’s not an ascetic, not a Sakyan monastic. Just as a fallen, withered leaf is incapable of becoming green again, so too is a monk who, intending to steal, takes an ungiven \textit{\textsanskrit{pāda}} coin, the value of a \textit{\textsanskrit{pāda}}, or more than a \textit{\textsanskrit{pāda}} not an ascetic, not a Sakyan monastic. You shouldn’t do this for as long as you live. }

\scrule{A monk who’s fully ordained shouldn’t intentionally kill a living being, not even a small insect. If he intentionally kills a human being, even causing an abortion, he’s not an ascetic, not a Sakyan monastic. Just as an ordinary stone that has broken in half cannot be put back together again, so too is a monk who has intentionally killed a human being not an ascetic, not a Sakyan monastic. You shouldn’t do this for as long as you live. }

\scrule{A monk who’s fully ordained shouldn’t claim a superhuman quality, not even just saying, ‘I delight in solitude.’ If, because he has bad desires and is overcome by desire, he claims to have a non-existent superhuman quality—whether absorption, release, stillness, attainment, path, or fruit—he’s not an ascetic, not a Sakyan monastic. Just as a palm tree with its crown cut off is incapable of further growth, so too is a monk with bad desires, overcome by desire, who claims to have non-existent superhuman quality not an ascetic, not a Sakyan monastic. You shouldn’t do this for as long as you live.” }

\scend{The four things not to be done are finished. }

\section*{66. The one ejected for not recognizing an offense }

At\marginnote{79.1.1} one time a certain monk disrobed after being ejected for not recognizing an offense. He then returned and asked the monks for the full ordination. They told the Buddha. 

“When\marginnote{79.1.4} a monk disrobes after being ejected for not recognizing an offense, but then returns and asks the monks for the full ordination, he should be asked, ‘Will you recognize that offense?’ If he says, ‘I will,’ he should be given the going forth. If he says, ‘I won’t,’ he should not. 

When\marginnote{79.1.10} he’s been given the going forth, he should be asked again, ‘Will you recognize that offense?’ If he says, ‘I will,’ he should be given the full ordination. If he says, ‘I won’t,’ he should not. 

When\marginnote{79.2.1} he’s been given the full ordination, he should be asked again, ‘Will you recognize that offense?’ If he says, ‘I will,’ he should be readmitted. If he says, ‘I won’t,’ he should not. 

When\marginnote{79.2.5} he’s been readmitted, he should be asked again, ‘Do you recognize that offense?’ If he recognizes it, it’s good. If he doesn’t recognize it and you’re unanimous, he should be ejected once more. If you’re not unanimous, there’s no offense in living with him or in doing formal meetings of the community together.\footnote{See definitions of \textit{sambhoga} and \textit{\textsanskrit{saṁvāsa}} at \href{https://suttacentral.net/pli-tv-bu-vb-pc69/en/brahmali\#2.1.14}{Bu Pc 69:2.1.14} and \href{https://suttacentral.net/pli-tv-bu-vb-pc69/en/brahmali\#2.1.21}{Bu Pc 69:2.1.21}. } 

“When\marginnote{79.3.1} a monk disrobes after being ejected for not making amends for an offense, but then returns and asks the monks for the full ordination, he should be asked, ‘Will you make amends for that offense?’ If he says, ‘I will,’ he should be given the going forth. If he says, ‘I won’t,’ he should not. 

When\marginnote{79.3.7} he’s been given the going forth, he should be asked again, ‘Will you make amends for that offense?’ If he says, ‘I will,’ he should be given the full ordination. If he says, ‘I won’t,’ he should not. 

When\marginnote{79.3.11} he’s been given the full ordination, he should be asked again, ‘Will you make amends for that offense?’ If he says, ‘I will,’ he should be readmitted. If he says, ‘I won’t,’ he should not. 

When\marginnote{79.3.15} he’s been readmitted, he should be told, ‘Make amends for that offense.’ If he does, it’s good. If he doesn’t and you’re unanimous, he should be ejected once more. If you’re not unanimous, there’s no offense in living with him or in doing formal meetings of the community together. 

“When\marginnote{79.4.1} a monk disrobes after being ejected for not giving up a bad view, but then returns and asks the monks for the full ordination, he should be asked, ‘Will you give up that bad view?’ If he says, ‘I will,’ he should be given the going forth. If he says, ‘I won’t,’ he should not. 

When\marginnote{79.4.7} he’s been given the going forth, he should be asked again, ‘Will you give up that bad view?’ If he says, ‘I will,’ he should be given the full ordination. If he says, ‘I won’t,’ he should not. 

When\marginnote{79.4.11} he’s been given the full ordination, he should be asked again, ‘Will you give up that bad view?’ If he says, ‘I will,’ he should be readmitted. If he says, ‘I won’t,’ he should not. 

When\marginnote{79.4.15} he’s been readmitted, he should be told, ‘Give up that bad view.’ If he does, it’s good. If he doesn’t and you’re unanimous, he should be ejected once more. If you’re not unanimous, there’s no offense in living with him or in doing formal meetings of the community together.” 

\scendsutta{The great chapter, the first, is finished. }

\scuddanaintro{This is the summary:\footnote{Vmv 3.131: \textit{\textsanskrit{Vinayamhītiādigāthāsu} \textsanskrit{niggahānanti} \textsanskrit{niggahakaraṇesu}. \textsanskrit{Pāpiccheti} \textsanskrit{pāpapuggalānaṁ} \textsanskrit{niggahakaraṇesu}, \textsanskrit{lajjīnaṁ} paggahesu ca \textsanskrit{pesalānaṁ} \textsanskrit{sukhāvahe} mahante vinayamhi \textsanskrit{yathā} \textsanskrit{atthakārī} \textsanskrit{atthānuguṇaṁ} karontova \textsanskrit{yasmā} yoniso \textsanskrit{paṭipajjati} \textsanskrit{nāma} hoti, \textsanskrit{tasmā} \textsanskrit{uddānaṁ} \textsanskrit{pavakkhāmīti} \textsanskrit{sambandhayojanā} \textsanskrit{daṭṭhabbā}}; “In regard to the verses beginning with \textit{vinayamhi}: \textit{\textsanskrit{niggahānaṁ}} means concerning the production of restraint. \textit{\textsanskrit{Pāpicche}} means in regard to the production of restraint of bad people and in regard to helping those who have a sense of conscience and those who are good, in the great Monastic Law which brings happiness, concerning one making and helping what is beneficial, wherefore it is called one practicing wisely, therefore I speak this summary. It is to be seen as connected together.” } }

\begin{scuddana}%
“In\marginnote{79.4.22} the great Monastic Law,\footnote{Vmv 3.131 reads \textit{mahante vinayamhi}, “In the great Monastic Law”, which I follow. } \\
Which brings happiness to those who are good, \\
Restrains those who have bad desires,\footnote{Reading \textit{niggahe ca \textsanskrit{pāpicchānaṁ}} with the PTS edition. } \\
And helps those with a sense of conscience;\footnote{I am not clear here on the function of the locative plural \textit{esu}, but I am assuming the construction is parallel to the previous two lines. } 

And\marginnote{79.4.26} which is for the upkeep of Buddhism, \\
In the sphere of the Omniscient Victor, \\
Not within range of anyone else; \\
Which is safe, carefully laid down, without doubt—

That\marginnote{79.4.30} is, the Chapters and the Monastic Law, \\
The Compendium and the Key Terms—\footnote{In this sort of context, and perhaps elsewhere too, Key Terms is usually a reference to the two \textsanskrit{Pātimokkhas}. Sp 5.325: \textit{\textsanskrit{Pātimokkhanti} dve \textsanskrit{mātikā} na \textsanskrit{jānāti}}, “\textit{\textsanskrit{Pātimokkha}}: he does not know the two (collections of) Key Terms.” } \\
In this the skillful who does what’s beneficial, \\
Practices wisely. 

One\marginnote{79.4.34} who doesn’t understand cattle, \\
Doesn’t guard the herd; \\
In the same way, not understanding virtue, \\
How would one guard restraint? 

When\marginnote{79.4.38} the discourses are forgotten, \\
And the same for philosophy, \\
But the Monastic Law isn’t lost, \\
Then Buddhism still remains. 

Therefore,\marginnote{79.4.42} for the purpose of making a collection, \\
I’ll expound the summary, successively, \\
According to the right method. \\
Listen to me speak: 

Topic,\marginnote{79.4.46} origin story, offense, \\
Method, and repetition. \\
It’s hard to complete without remainder—\footnote{\textit{\textsanskrit{Asesetuṁ}} is presumably a denominative form of \textit{asesa}, “without remainder”. See CPD, sv. \textit{asesita}. } \\
You should know it from the method.” 

“Bodhi\marginnote{79.4.50} tree, and ape-flower tree, \\
The goatherd’s tree, Sahamapati \\
The supreme being, \textsanskrit{Ālāra}, Udaka, \\
And monk, the sage Upaka. 

\textsanskrit{Koṇḍañña},\marginnote{79.4.54} Vappa, Bhaddiya, \\
And \textsanskrit{Mahānāma}, Assaji; \\
Yasa, four, fifty, \\
He sent all to the districts. 

Topic,\marginnote{79.4.58} with the lords of death, and thirty, \\
\textsanskrit{Uruvelā}, three dreadlocked ascetics; \\
Fire hut, great kings, \\
Sakka, and the supreme being, the whole. 

Rag,\marginnote{79.4.62} pond, \\
And boulder, arjun tree, boulder; \\
Rose-apple tree, and mango tree, emblic myrobalan tree, \\
And he brought an orchid tree flower. 

May\marginnote{79.4.66} they split, may they be lit, \\
And may they be extinguished, Kassapa; \\
They immersed themselves, coal pans, cloud, \\
\textsanskrit{Gayā}, and Cane, of Magadha. 

Upatissa\marginnote{79.4.70} and Kolita, \\
And the well-known went forth; \\
Shabbily dressed, dismissal, \\
The thin and haggard brahmin. 

He\marginnote{79.4.74} misbehaved, \\
Stomach, young brahmin, group; \\
Seniority, by those who are ignorant, went away, \\
Ten years of formal support. 

They\marginnote{79.4.78} did not conduct themselves, to dismiss, \\
The ignorant, ending, five, six; \\
He who was from another religion, and naked, \\
Uncut, dreadlocked ascetic, and Sakyan. 

The\marginnote{79.4.82} five diseases in Magadha, \\
And one king, finger;\footnote{It is not clear what \textit{eko} refers to. } \\
And (the king) of Magadha declared, \\
Prison, wanted, whipped. 

Branded,\marginnote{79.4.86} debt, and slave, \\
Shaven, \textsanskrit{Upāli}, deadly disease; \\
Family with faith, and \textsanskrit{Kaṇṭaka}, \\
And the obscure. 

To\marginnote{79.4.90} live, the boy, the training, \\
And they were, which; \\
The whole, the mouth, the preceptors, \\
Luring away, \textsanskrit{Kaṇṭaka}. 

\textit{\textsanskrit{Paṇḍakas}},\marginnote{79.4.94} theft, and left, \\
And serpent, about mother, father; \\
Perfected one, nun, and schism, \\
And with blood, hermaphrodite. 

Without\marginnote{79.4.98} preceptor, and with the Sangha, \\
Group, \textit{\textsanskrit{paṇḍaka}}, and one without almsbowl; \\
Without robe, both of them, \\
Also the same three with borrowed. 

Hand,\marginnote{79.4.102} foot, hand and foot, \\
Ear, nose, both of them; \\
Finger, toe, and tendon, \\
Joined, and hunchback, dwarf. 

Goiter,\marginnote{79.4.106} and branded, \\
Whipped, wanted, and elephantiasis; \\
Serious, and abnormal appearance, \\
Blind in one eye, and so crooked limb. 

Lame,\marginnote{79.4.110} and paralyzed on one side, \\
Who is crippled; \\
Old age, blind, mute, and deaf, \\
And in regard to the blind and mute. 

What\marginnote{79.4.114} is called blind and deaf, \\
And mute and deaf; \\
And blind, mute, and deaf, \\
And formal support for the shameless. 

And\marginnote{79.4.118} should live, so traveling, \\
By one who is asked, notice;\footnote{Reading \textit{\textsanskrit{pekkhanā}} with the PTS version, instead of \textit{\textsanskrit{lakkhaṇā}}. } \\
Please come, they argued, \\
With one preceptor, Kassapa. 

And\marginnote{79.4.122} ordained people were seen \\
Oppressed by sicknesses; \\
The uninstructed were embarrassed, \\
Instructing just there. 

And\marginnote{79.4.126} so in the Sangha, then the ignorant, \\
And not appointed, together; \\
Please lift up, full ordination, \\
Support, by himself, three.” 

%
\end{scuddana}

\scend{In this chapter there are one hundred and seventy-two topics. }

\scendsutta{The great chapter is finished. }

%
\chapter*{{\suttatitleacronym Kd 2}{\suttatitletranslation The chapter on the observance day }{\suttatitleroot Uposathakkhandhaka}}
\addcontentsline{toc}{chapter}{\tocacronym{Kd 2} \toctranslation{The chapter on the observance day } \tocroot{Uposathakkhandhaka}}
\markboth{The chapter on the observance day }{Uposathakkhandhaka}
\extramarks{Kd 2}{Kd 2}

\section*{1. The instruction to gather together }

At\marginnote{1.1.1} one time the Buddha was staying on the Vulture Peak at \textsanskrit{Rājagaha}. At that time, on the fourteenth, fifteenth, and eighth day of the lunar half-month, the wanderers of other religions gathered and gave teachings. People went to listen to those teachings, and they acquired affection for and confidence in those wanderers. And the wanderers gained supporters. 

Then,\marginnote{1.2.1} when King Seniya \textsanskrit{Bimbisāra} of Magadha was reflecting in private, he considered this and thought, “Why don’t the venerables, too, gather on the fourteenth, fifteenth, and eighth day of the half-month?” 

He\marginnote{1.3.1} then went to the Buddha, bowed, sat down, and told him what he had thought, adding, “It would be good, sir, if the venerables, too, gathered on the fourteenth, fifteenth, and eighth day of the half-month.” The Buddha then instructed, inspired, and gladdened him with a teaching. When the Buddha had finished, the king got up from his seat, bowed, circumambulated the Buddha with his right side toward him, and left. Soon afterwards the Buddha gave a teaching and addressed the monks: 

\scrule{“You should gather together on the fourteenth, the fifteenth, and the eighth day of the lunar half-month.” }

When\marginnote{2.1.1} the monks heard about the Buddha’s instruction, they started gathering on those days. People came to hear a teaching, but the monks sat in silence. The people complained and criticized them, “How can the Sakyan monastics gather on the fourteenth, fifteenth, and eighth day of the half-month, but then sit in silence like dumb pigs? Shouldn’t they give a teaching when they gather together?” The monks heard the complaints of those people and they told the Buddha. Soon afterwards the Buddha gave a teaching and addressed the monks: 

\scrule{“When you gather together on the fourteenth, the fifteenth, and the eighth day of the lunar half-month, you should give a teaching.” }

\section*{2. The instruction to recite the Monastic Code }

While\marginnote{3.1.1} the Buddha was reflecting in private, he thought, “Why don’t I instruct the monks to recite a monastic code, consisting of those training rules that I have laid down for them? That would be their procedure for the observance day.” In the evening, when the Buddha had come out from seclusion, he gave a teaching and addressed the monks. He told them what he had thought, adding: 

\scrule{“You should recite the Monastic Code. }

And\marginnote{3.3.1} you should do it like this. A competent and capable monk should inform the Sangha: 

‘Please,\marginnote{3.3.3} venerables, I ask the Sangha to listen. If the Sangha is ready, it should do the observance-day ceremony, it should recite the Monastic Code.\footnote{“Observance-day ceremony” renders \textit{uposatha}. See Appendix of Technical Terms. } What is the preliminary duty of the Sangha? The venerables should declare their purity. I will recite the Monastic Code. Everyone present should listen to it and attend carefully. Anyone who has committed an offense should reveal it. If you haven’t committed any offense, you should remain silent. If you are silent, I will regard you as pure. Just as one responds when asked individually, so too, an announcement is made three times in this kind of gathering. If a monk remembers an offense while the announcement is being made up to the third time, but doesn’t reveal it, he is lying in full awareness. Lying in full awareness is called an obstacle by the Buddha. A monk who remembers an offense and is seeking purification should therefore reveal it. When it’s revealed, he will be at ease.’” 

\subsection*{Definitions }

\begin{description}%
\item[Monastic Code: ] this is the beginning, this is the front, this is at the head of wholesome qualities—therefore it is called “Monastic Code”.\footnote{This definition is a play on the two unrelated words \textit{mukha} and \textit{mokkha}, respectively meaning “front” and “freedom”. } %
\item[Venerables: ] this is a term of affection, a term of respect; it is an expression of respect and deference, that is, “venerables”. %
\item[I will recite: ] I will set forth, I will teach, I will declare, I will set out, I will reveal, I will analyze, I will make plain, I will manifest. %
\item[It: ] The Monastic Code is what is meant. %
\item[Everyone present: ] to whatever extent there are senior monks, junior monks, and monks of middle standing in that gathering—these are called “everyone present”. %
\item[Should listen carefully: ] should be attentive, should pay attention, should apply their whole mind. %
\item[Should attend: ] should listen with a one-pointed mind, with an undistracted mind, with a non-wandering mind. %
\item[Anyone who has committed an offense: ] a senior monk, a junior monk, or a monk of middle standing who has committed a particular offense among the five or seven classes of offenses. %
\item[Should reveal it: ] should confess it, should disclose it, should make it plain, should make it known—either in the midst of the Sangha, in the midst of a group, or to an individual. %
\item[If you haven’t committed any offense: ] if you have not committed any offense or you have cleared yourself after committing one. %
\item[You should remain silent: ] you should be patient; you shouldn’t say anything. %
\item[I will regard you as pure: ] I will know; I will remember. %
\item[Just as one responds when asked individually: ] just as one would respond when asked privately, so too, one should know of that gathering, “It’s asking me.” %
\item[This kind of gathering: ] a gathering of monks is what is meant. %
\item[When the announcement is made three times: ] when the announcement is made once, when the announcement is made for the second time, and also when the announcement is made for the third time. %
\item[Remembers: ] Knows, perceives. %
\item[An offense: ] one that has been committed, or one that has not been cleared after being committed.\footnote{The phrasing here is a bit curious, but according to the commentary at Sp 3.135 it is to be understood as the opposite of segment \href{https://suttacentral.net/pli-tv-kd2/en/brahmali\#3.5.6}{Kd 2:3.5.6} above. } %
\item[But doesn’t reveal it: ] does not confess it, disclose it, make it plain, make it known—either in the midst of the Sangha, in the midst of a group, or to an individual. %
\item[He is lying in full awareness: ] what is there for lying in full awareness? There is an act of wrong conduct.\footnote{Sp 3.135: \textit{\textsanskrit{Dukkaṭaṁ} \textsanskrit{hotīti} \textsanskrit{dukkaṭāpatti} hoti; \textsanskrit{sā} ca kho na \textsanskrit{musāvādalakkhaṇena}; bhagavato pana vacanena \textsanskrit{vacīdvāre} \textsanskrit{akiriyasamuṭṭhānā} \textsanskrit{āpatti} \textsanskrit{hotīti} \textsanskrit{veditabbā}}, “‘There is an act of wrong conduct’: there is an offense of wrong conduct. It does not have the characteristics of lying. But according to the statement by the Buddha, it is to be understood that there is an offense originating through non-action at the speech door.” } %
\item[Is called an obstacle by the Buddha: ] an obstacle for what? It is an obstacle for reaching the first absorption, the second absorption, the third absorption, the fourth absorption; an obstacle for reaching the wholesome qualities of absorption, release, stillness, attainment, renunciation, escape, seclusion. %
\item[Therefore: ] for that reason. %
\item[Who remembers: ] who knows, who perceives. %
\item[Is seeking purification: ] is desiring to be cleared, is desiring purity. %
\item[An offense: ] one that has been committed, or one that has not been cleared after being committed. %
\item[Should reveal it: ] should reveal it either in the midst of the Sangha, in the midst of a group, or to an individual. %
\item[When it’s revealed, he will be at ease: ] at ease for what? He will be at ease for reaching the first absorption, the second absorption, the third absorption, the forth absorption; at ease for reaching the wholesome qualities of absorption, release, stillness, attainment, renunciation, escape, seclusion. %
\end{description}

When\marginnote{4.1.1} they heard that the Buddha required the recitation of the Monastic Code, some monks recited it daily. They told the Buddha. 

\scrule{“You shouldn’t recite the Monastic Code every day. If you do, you commit an offense of wrong conduct. You should recite the Monastic Code on the observance day.” }

When\marginnote{4.2.1} they heard that the Buddha required the recitation of the Monastic Code on the observance day, some monks recited it three times per half-month: on the fourteenth, fifteenth, and eighth day. 

\scrule{“You shouldn’t recite the Monastic Code three times per lunar half-month. If you do, you commit an offense of wrong conduct. You should recite the Monastic Code once every lunar half-month: on the fourteenth or the fifteenth day.” }

On\marginnote{5.1.1} one occasion the monks from the group of six recited the Monastic Code separately, each to his own followers. 

\scrule{“You shouldn’t recite the Monastic Code separately, each to your own followers. If you do, you commit an offense of wrong conduct. You should do the observance-day procedure in a complete assembly.” }

When\marginnote{5.2.1} they knew that the Buddha had laid down a rule that the observance-day procedure should be done in a complete assembly, the monks thought, “How far does a complete assembly extend? As far as one monastery or as far as the entire earth?” 

\scrule{“A complete assembly extends as far as one monastery.” }

\section*{3. \textsanskrit{Mahākappina} }

At\marginnote{5.3.1} that time Venerable \textsanskrit{Mahākappina} was staying at \textsanskrit{Rājagaha} in the deer park at Maddakucchi. On one occasion, while reflecting in private, he thought, “Should I go to the observance-day ceremony? Should I go to the legal procedures of the Sangha? Regardless, I’ve reached the highest purity.” 

The\marginnote{5.4.1} Buddha read his mind. Then, just as a strong man might bend or stretch his arm, the Buddha disappeared from the Vulture Peak and reappeared in front of \textsanskrit{Mahākappina}, where he sat down on the prepared seat. \textsanskrit{Mahākappina} bowed and sat down, and the Buddha said to him: 

“Isn’t\marginnote{5.5.2} it the case, Kappina, that you were wondering whether or not you should go to the observance day and the legal procedures of the Sangha?” 

“Yes,\marginnote{5.5.4} venerable sir.” 

“If\marginnote{5.5.5} you brahmins don’t honor and revere the observance day, then who will? Go to the observance day, brahmin, and go to the legal procedures of the Sangha.” 

“Yes.”\marginnote{5.5.8} 

The\marginnote{5.6.1} Buddha instructed, inspired, and gladdened him with a teaching. Then, just as a strong man might bend or stretch his arm, the Buddha disappeared from \textsanskrit{Mahākappina}’s presence and reappeared on the Vulture Peak. 

\section*{4. The allowance for monastery zones }

When\marginnote{6.1.1} they knew that the Buddha had laid down a rule that a complete assembly extends as far as one monastery, the monks thought, “How far does a single monastery extend?” They told the Buddha. 

\scrule{“I allow you to establish a monastery zone.\footnote{“Monastery zone” renders \textit{\textsanskrit{sīmā}}. See Appendix of Technical Terms. } }

And\marginnote{6.1.6} it should be established like this. First you should announce the zone markers: a hill, a rock, a forest grove, a tree, a path, an anthill, a river, a lake. Then a competent and capable monk should inform the Sangha: 

‘Please,\marginnote{6.1.10} venerables, I ask the Sangha to listen. If the Sangha is ready, it should establish a monastery zone based on the announced markers, defining who belongs to the same community and who should do the observance-day ceremony together.\footnote{\textit{\textsanskrit{Nānāsaṁvāsa}} (and \textit{\textsanskrit{samānasaṁvāsa}}) need to be carefully distinguished from \textit{\textsanskrit{nānāsaṁvāsaka}} (and \textit{\textsanskrit{samānasaṁvāsaka}}). The former means “belonging to a different community”, as decided by \textit{\textsanskrit{sīmās}}. The latter means “one belonging to a different Buddhist sect”. } This is the motion. 

Please,\marginnote{6.2.1} venerables, I ask the Sangha to listen. The Sangha establishes a monastery zone based on the announced markers, defining who belongs to the same community and who should do the observance-day ceremony together. Any monk who approves of establishing a monastery zone based on these markers, defining who belongs to the same community and who should do the observance-day ceremony together, should remain silent. Any monk who doesn’t approve should speak up. 

The\marginnote{6.2.6} Sangha has established a monastery zone based on these markers, defining who belongs to the same community and who should do the observance-day ceremony together. The Sangha approves and is therefore silent. I’ll remember it thus.’” 

When\marginnote{7.1.1} they heard that the Buddha had made an allowance to establish a monastery zone, the monks from the group of six established zones that were too large: 50, 65, and even 80 kilometers across. Monks coming to the observance-day ceremony arrived while the Monastic Code was being recited or just after, and they had to stop overnight while on the way. They told the Buddha. 

\scrule{“You shouldn’t establish a monastery zone that is too large, whether 50, 65, or 80 kilometers across.\footnote{“Across” is not in the Canonical text, but is supplied from the commentary. Sp 3.140: \textit{Tiyojanaparamanti ettha \textsanskrit{tiyojanaṁ} \textsanskrit{paramaṁ} \textsanskrit{pamāṇametissāti} \textsanskrit{tiyojanaparamā}; \textsanskrit{taṁ} \textsanskrit{tiyojanaparamaṁ}. Sammannantena pana majjhe \textsanskrit{ṭhatvā} \textsanskrit{yathā} \textsanskrit{catūsupi} \textsanskrit{disāsu} \textsanskrit{diyaḍḍhadiyaḍḍhayojanaṁ} hoti, \textsanskrit{evaṁ} \textsanskrit{sammannitabbā}. Sace pana majjhe \textsanskrit{ṭhatvā} ekekadisato \textsanskrit{tiyojanaṁ} karonti, \textsanskrit{chayojanaṁ} \textsanskrit{hotīti} na \textsanskrit{vaṭṭati}. \textsanskrit{Caturassaṁ} \textsanskrit{vā} \textsanskrit{tikoṇaṁ} \textsanskrit{vā} sammannantena \textsanskrit{yathā} \textsanskrit{koṇato} \textsanskrit{koṇaṁ} \textsanskrit{tiyojanaṁ} hoti, \textsanskrit{evaṁ} \textsanskrit{sammannitabbā}. Sace hi yena kenaci pariyantena kesaggamattampi \textsanskrit{tiyojanaṁ} \textsanskrit{atikkāmeti}, \textsanskrit{āpattiñca} \textsanskrit{āpajjati} \textsanskrit{sīmā} ca \textsanskrit{asīmā} hoti}; “40 kilometers at the most: here 40 kilometers at the most is its measure. This is 40 kilometers at the most. One who is establishing (a monastery zone), standing in the middle, should establish (a zone) that is 20 kilometers in the four directions. If, standing in the middle, he makes it 40 kilometers in each direction, it will be 80 kilometers, which is not allowable. One who is establishing (a zone) that is quadrangular or triangular should establish (a zone) that is 40 kilometers corner to corner. If he exceeds the 40 kilometers even by a hair’s breadth on any side, he commits an offense, and the zone is not actually a monastery zone.” } If you do, you commit an offense of wrong conduct. You should establish a monastery zone that is 40 kilometers across at the most.”\footnote{The Pali for 50, 65, 80, and 40 kilometers is 4, 5, 6, and 3 \textit{yojanas} respectively. For a discussion of the \textit{yojana}, see \textit{sugata} in Appendix of Technical Terms. } }

At\marginnote{7.2.1} one time the monks from the group of six had established a zone that crossed a river. Monks on their way to the observance-day ceremony were swept away by the current, as were their bowls and robes. 

\scrule{“You shouldn’t establish a monastery zone that crosses a river. If you do, you commit an offense of wrong conduct. I allow you to establish a monastery zone that crosses a river only if there is a permanent bridge or ferry connection.” }

\section*{5. Discussion of the observance-day hall }

At\marginnote{8.1.1} that time the monks recited the Monastic Code in one yard after another without making a prior arrangement.\footnote{The point seems to be that they recited the \textit{\textsanskrit{pātimokkha}} in a different place every lunar half-month. Sp 3.141: \textit{\textsanskrit{Anupariveṇiyanti} \textsanskrit{ekasīmamahāvihāre} \textsanskrit{tasmiṁ} \textsanskrit{tasmiṁ} \textsanskrit{pariveṇe}}, “‘In one yard after another’: in this or that yard within a large monastery inside a single monastery zone.” } Newly-arrived monks did not know where the observance-day ceremony was to be held. They told the Buddha. 

\scrule{“You shouldn’t recite the Monastic Code in one yard after another without making a prior arrangement. If you do, you commit an offense of wrong conduct. I allow you to designate an observance-day hall—whether a dwelling, a stilt house, or a cave—for the observance-day ceremony.\footnote{For an explanation of the renderings “stilt house” and “cave” for \textit{\textsanskrit{pāsāda}} and \textit{\textsanskrit{guhā}} respectively, see Appendix of Technical Terms. Apart from \textit{\textsanskrit{vihāra}}, “a dwelling”, and \textit{\textsanskrit{guhā}}, “a cave”, the Pali mentions three kinds of buildings, the \textit{\textsanskrit{aḍḍhayoga}}, the \textit{\textsanskrit{pāsāda}}, and the \textit{hammiya}, all of which, according to the commentaries, are different kinds of \textit{\textsanskrit{pāsāda}}, “stilt houses”. Rather than try to differentiate between these buildings, which is unlikely to be useful from a practical perspective, I have instead grouped them together as “stilt houses”. Here is what the commentaries have to say. Sp 4.294: \textit{\textsanskrit{Aḍḍhayogoti} \textsanskrit{supaṇṇavaṅkagehaṁ}}, “An \textit{\textsanskrit{aḍḍhayoga}} is a house bent like a \textit{\textsanskrit{supaṇṇa}}.” Sp-\textsanskrit{ṭ} 4.294 clarifies: \textit{\textsanskrit{Supaṇṇavaṅkagehanti} \textsanskrit{garuḷapakkhasaṇṭhānena} \textsanskrit{katagehaṁ}}, “\textit{\textsanskrit{Supaṇṇavaṅkageha}}: a house made in the shape of the wings of a \textit{\textsanskrit{garuḷa}}.” A \textit{\textsanskrit{garuḷa}}, better known in its Sanskrit form \textit{\textsanskrit{garuḍa}}, is a mythological bird. Sp 4.294 continues: \textit{\textsanskrit{Pāsādoti} \textsanskrit{dīghapāsādo}. Hammiyanti \textsanskrit{upariākāsatale} \textsanskrit{patiṭṭhitakūṭāgāro} \textsanskrit{pāsādoyeva}}, “A \textit{\textsanskrit{pāsāda}} is a long stilt house. A \textit{hammiya} is just a \textit{\textsanskrit{pāsāda}} that has an upper room on top of its flat roof.” At Sp-\textsanskrit{ṭ} 3.74, however, we find slightly different explanations. It seems clear, however, that all three are stilt houses and that they are distinguished according to their shape and the kind of roof they possess. } }

And\marginnote{8.1.8} it should be designated like this. A competent and capable monk should inform the Sangha: 

‘Please,\marginnote{8.2.2} venerables, I ask the Sangha to listen. If the Sangha is ready, it should designate such-and-such a dwelling as the observance-day hall. This is the motion. 

Please,\marginnote{8.2.5} venerables, I ask the Sangha to listen. The Sangha designates such-and-such a dwelling as the observance-day hall. Any monk who approves of designating such-and-such a dwelling as the observance-day hall should remain silent. Any monk who doesn’t approve should speak up. 

The\marginnote{8.2.9} Sangha has designated such-and-such a dwelling as the observance-day hall. The Sangha approves and is therefore silent. I’ll remember it thus.’” 

Soon\marginnote{8.3.1} afterwards in a certain monastery, they designated two different observance-day halls. Monks gathered in both places, each group thinking, “The observance-day ceremony will be done here.” They told the Buddha. 

\scrule{“You shouldn’t designate two different observance-day halls within the same monastery. If you do, you commit an offense of wrong conduct. You should abolish one of them and do the observance-day ceremony in one place. }

And\marginnote{8.4.1} it should be abolished like this. A competent and capable monk should inform the Sangha: 

‘Please,\marginnote{8.4.3} venerables, I ask the Sangha to listen. If the Sangha is ready, it should abolish such-and-such an observance-day hall. This is the motion. 

Please,\marginnote{8.4.6} venerables, I ask the Sangha to listen. The Sangha abolishes such-and-such an observance-day hall. Any monk who approves of abolishing such-and-such an observance-day hall should remain silent. Any monk who doesn’t approve should speak up. 

The\marginnote{8.4.10} Sangha has abolished such-and-such an observance-day hall. The Sangha approves and is therefore silent. I’ll remember it thus.’” 

\section*{6. The allowance for an observance-day forecourt }

At\marginnote{9.1.1} one time in a certain monastery, they had designated an observance-day hall that was too small. On the observance day a large sangha of monks gathered there. Some monks listened to the recitation of the Monastic Code while sitting outside the designated area. Knowing that the Buddha had laid down a rule that the observance-day ceremony should be done after designating an observance-day hall, they wondered, “Have we done the observance-day ceremony or not?” They told the Buddha. 

\scrule{“Whether you listen to the recitation of the Monastic Code while seated within or outside the designated area, in either case you have done the observance-day ceremony. }

\scrule{Still, the Sangha may designate an observance-day forecourt as large as it likes.\footnote{“Forecourt” renders \textit{pamukha}. Sp-\textsanskrit{ṭ} 3.142: \textit{\textsanskrit{Uposathappamukhaṁ} \textsanskrit{nāma} \textsanskrit{uposathāgārassa} \textsanskrit{sammukhaṭṭhānaṁ}}, “The place which is face-to-face with the observance-day hall is called \textit{\textsanskrit{uposathappamukhaṁ}}.” } }

And\marginnote{9.2.2} it should be designated like this. First the markers should be announced. Then a competent and capable monk should inform the Sangha: 

‘Please,\marginnote{9.2.5} venerables, I ask the Sangha to listen. If the Sangha is ready, it should designate an observance-day forecourt based on the announced markers. This is the motion. 

Please,\marginnote{9.2.9} venerables, I ask the Sangha to listen. The Sangha designates an observance-day forecourt based on the announced markers. Any monk who approves of designating an observance-day forecourt based on these markers should remain silent. Any monk who doesn’t approve should speak up. 

The\marginnote{9.2.14} Sangha has designated an observance-day forecourt based on these markers. The Sangha approves and is therefore silent. I’ll remember it thus.’” 

On\marginnote{10.1.1} one occasion, on the observance day in a certain monastery, the junior monks had gathered first. Thinking, “There’s no point in being here before the senior monks arrive,” they left. As a consequence, the observance-day ceremony was done at the wrong time. 

\scrule{“On the observance day, the senior monks should gather first.” }

At\marginnote{11.1.1} that time at \textsanskrit{Rājagaha}, there was a number of monasteries within the same monastery zone. The monks argued about where the observance-day ceremony should be done. 

\scrule{“When there are a number of monasteries within the same zone and the monks are arguing about where the observance-day ceremony should be done, they should all gather in one place and do the observance-day ceremony there. Or they should gather wherever the most senior monk is staying.  You shouldn’t do the observance-day ceremony with an incomplete sangha. If you do, you commit an offense of wrong conduct.” }

\section*{7. The allowance for a may-stay-apart zone }

On\marginnote{12.1.1} one occasion Venerable \textsanskrit{Mahākassapa} was coming from Andhakavinda to \textsanskrit{Rājagaha} for the observance-day ceremony. As he was crossing a river on the way, he briefly got carried away by the current and his robes got wet. The monks asked him why his robes were wet, and he told them what had happened. They told the Buddha. 

\scrule{“When the Sangha has established a monastery zone, defining who belongs to the same community and who should do the observance-day ceremony together, the Sangha may designate this same zone as a may-stay-apart-from-the-three-robes area. }

And\marginnote{12.2.1} it should be designated like this. A competent and capable monk should inform the Sangha: 

‘Please,\marginnote{12.2.3} venerables, I ask the Sangha to listen. The Sangha has established a monastery zone, defining who belongs to the same community and who should do the observance-day ceremony together. If the Sangha is ready, it should designate this same zone as a may-stay-apart-from-the-three-robes area. This is the motion. 

Please,\marginnote{12.2.6} venerables, I ask the Sangha to listen. The Sangha has established a monastery zone, defining who belongs to the same community and who should do the observance-day ceremony together. The Sangha designates this same zone as a may-stay-apart-from-the-three-robes area. Any monk who approves of designating this monastery zone as a may-stay-apart-from-the-three-robes area should remain silent. Any monk who doesn’t approve should speak up. 

The\marginnote{12.2.10} Sangha has designated this monastery zone as a may-stay-apart-from-the-three-robes area. The Sangha approves and is therefore silent. I’ll remember it thus.’” 

When\marginnote{12.3.1} they heard that the Buddha had allowed the designation of a may-stay-apart-from-the-three-robes area, monks stored their robes in inhabited areas. Their robes were lost, burned, and eaten by rats. As a consequence, they had shabby robes. Other monks asked them why, and they told them what had happened. They told the Buddha. 

\scrule{“When the Sangha has established a monastery zone, defining who belongs to the same community and who should do the observance-day ceremony together, the Sangha may designate this same zone as a may-stay-apart-from-the-three-robes area, leaving out inhabited areas and the vicinity of inhabited areas.\footnote{“Vicinity” renders \textit{\textsanskrit{upacāra}}, while “inhabited area” is for \textit{\textsanskrit{gāma}}. See Appendix of Technical Terms. } }

And\marginnote{12.4.1} it should be designated like this. A competent and capable monk should inform the Sangha: 

‘Please,\marginnote{12.4.3} venerables, I ask the Sangha to listen. The Sangha has established a monastery zone, defining who belongs to the same community and who should do the observance-day ceremony together. If the Sangha is ready, it should designate this same zone as a may-stay-apart-from-the-three-robes area, leaving out inhabited areas and the vicinity of inhabited areas. This is the motion. 

Please,\marginnote{12.4.6} venerables, I ask the Sangha to listen. The Sangha has established a monastery zone, defining who belongs to the same community and who should do the observance-day ceremony together. The Sangha designates this same zone as a may-stay-apart-from-the-three-robes area, leaving out inhabited areas and the vicinity of inhabited areas. Any monk who approves of designating this monastery zone as a may-stay-apart-from-the-three-robes area, leaving out inhabited areas and the vicinity of inhabited areas, should remain silent. Any monk who doesn’t approve should speak up. 

The\marginnote{12.4.10} Sangha has designated this monastery zone as a may-stay-apart-from-the-three-robes area, leaving out inhabited areas and the vicinity of inhabited areas. The Sangha approves and is therefore silent. I’ll remember it thus.’ 

\section*{8. The abolishing of monastery zones }

\scrule{“Monks, when you’re establishing a monastery zone, the zone that defines who belongs to the same community should be established first. Afterwards you may designate the may-stay-apart-from-the-three-robes area. And when you’re abolishing a monastery zone, the may-stay-apart-from-the-three-robes area should be abolished first. Afterwards you may abolish the zone that defines who belongs to the same community. }

And\marginnote{12.5.3} this how a may-stay-apart-from-the-three-robes area should be abolished. A competent and capable monk should inform the Sangha: 

‘Please,\marginnote{12.5.5} venerables, I ask the Sangha to listen. If the Sangha is ready, it should abolish this may-stay-apart-from-the-three-robes area. This is the motion. 

Please,\marginnote{12.5.8} venerables, I ask the Sangha to listen. The Sangha abolishes this may-stay-apart-from-the-three-robes area. Any monk who approves of abolishing this may-stay-apart-from-the-three-robes area should remain silent. Any monk who doesn’t approve should speak up. 

The\marginnote{12.5.12} Sangha has abolished this may-stay-apart-from-the-three-robes area. The Sangha approves and is therefore silent. I’ll remember it thus.’ 

And\marginnote{12.6.1} a monastery zone should be abolished like this. A competent and capable monk should inform the Sangha: 

‘Please,\marginnote{12.6.3} venerables, I ask the Sangha to listen. If the Sangha is ready, it should abolish this monastery zone, defining who belongs to the same community and who should do the observance-day ceremony together. This is the motion. 

Please,\marginnote{12.6.6} venerables, I ask the Sangha to listen. The Sangha abolishes this monastery zone, defining who belongs to the same community and who should do the observance-day ceremony together. Any monk who approves of abolishing this monastery zone, defining who belongs to the same community and who should do the observance-day ceremony together, should remain silent. Any monk who doesn’t approve should speak up. 

The\marginnote{12.6.10} Sangha has abolished this monastery zone, defining who belongs to the same community and who should do the observance-day ceremony together. The Sangha approves and is therefore silent. I’ll remember it thus.’ 

\section*{9. Zones of inhabited areas, etc. }

\scrule{“There are monks who live supported by inhabited areas where no monastery zone has been established. In these cases, the zone of the inhabited area defines who belongs to the same community and who should do the observance-day ceremony together. If it is an uninhabited area in the wilderness, a distance of 80 meters on all sides defines who belongs to the same community and who should do the observance-day ceremony together.\footnote{That is, seven \textit{abbhantaras}. For a discussion of the \textit{abbhantara}, see \textit{sugata} in Appendix of Technical Terms. } A whole river, a whole ocean, or a whole lake cannot be a monastery zone in its own right. In a river, in the ocean, and in a lake, the zone that defines who belongs to the same community and who should do the observance-day ceremony together is the distance an average man can splash water in all directions.” }

At\marginnote{13.1.1} one time the monks from the group of six established a monastery zone that overlapped with an existing monastery zone. 

\scrule{“The establishment of the first zone is a legitimate legal procedure that is irreversible and fit to stand. The establishment of the subsequent zone is an illegitimate legal procedure that is reversible and unfit to stand. You shouldn’t establish a monastery zone that overlaps with an existing monastery zone. If you do, you commit an offense of wrong conduct.” }

At\marginnote{13.2.1} one time the monks from the group of six established a monastery zone that enclosed one existing monastery zone within it. 

\scrule{“The establishment of the first zone is a legitimate legal procedure that is irreversible and fit to stand. The establishment of the subsequent zone is an illegitimate legal procedure that is reversible and unfit to stand. You shouldn’t establish a monastery zone that encloses an existing monastery zone. If you do, you commit an offense of wrong conduct. }

\scrule{When you establish a monastery zone, you should leave a gap to any existing monastery zone.” }

\section*{10. Breach of the observance-day ceremony, etc. }

The\marginnote{14.1.1} monks thought, “How many observance days are there?” They told the Buddha. 

\scrule{“There are two observance days: the fourteenth and the fifteenth day of the lunar half-month.” }

The\marginnote{14.2.1} monks thought, “How many kinds of observance-day procedures are there?” 

“There\marginnote{14.2.4} are these four kinds: 

\begin{enumerate}%
\item The observance-day procedure that is illegitimate and has an incomplete assembly. %
\item The observance-day procedure that is illegitimate but has a complete assembly. %
\item The observance-day procedure that is legitimate but has an incomplete assembly. %
\item The observance-day procedure that is legitimate and has a complete assembly. %
\end{enumerate}

\scrule{The first, second, and third of these shouldn’t be done; I haven’t allowed such procedures. The fourth should be done; I have allowed such procedures. Therefore, monks, you should train like this: ‘We will do observance-day procedures that are legitimate and have a complete assembly.’” }

\section*{11. The recitation of the Monastic Code in brief, etc. }

The\marginnote{15.1.1} monks thought, “How many ways are there of reciting the Monastic Code?” They told the Buddha. 

“There\marginnote{15.1.4} are these five ways of reciting the Monastic Code: 

\begin{enumerate}%
\item After reciting the introduction, the rest is announced as if heard. This is the first way. %
\item After reciting the introduction and the four rules entailing expulsion, the rest is announced as if heard. This is the second way. %
\item After reciting the introduction, the four rules entailing expulsion, and the thirteen rules entailing suspension, the rest is announced as if heard. This is the third way. %
\item After reciting the introduction, the four rules entailing expulsion, the thirteen rules entailing suspension, and the two undetermined rules, the rest is announced as if heard. This is the fourth way. %
\item In full is the fifth.” %
\end{enumerate}

When\marginnote{15.2.1} they heard that the Buddha had allowed the recitation of the Monastic Code in brief, some monks recited it in brief all the time. 

\scrule{“You shouldn’t recite the Monastic Code in brief. If you do, you commit an offense of wrong conduct.” }

At\marginnote{15.3.1} that time, on the observance day in a certain monastery in the Kosalan country, there was a threat from primitive tribes.\footnote{Sp 3.150: \textit{Savarabhayanti \textsanskrit{aṭavimanussabhayaṁ}}, “\textit{\textsanskrit{Savarabhayaṁ}}: threat from forest people.” } The monks were unable to recite the Monastic Code in full. 

\scrule{“I allow you to recite the Monastic Code in brief when there are threats.” }

The\marginnote{15.4.1} monks from the group of six recited the Monastic Code in brief even when there were no threats. 

\scrule{“You shouldn’t recite the Monastic Code in brief when there are no threats. If you do, you commit an offense of wrong conduct. I allow you to recite the Monastic Code in brief when there are any of these threats: a threat from kings, bandits, fire, floods, people, spirits, predatory animals, or creeping animals, or a threat to life, or a threat to the monastic life.” }

On\marginnote{15.5.1} one occasion the monks from the group of six gave a teaching in the midst of the Sangha without being asked. 

\scrule{“You shouldn’t give a teaching in the midst of the Sangha without being asked. If you do, you commit an offense of wrong conduct. I allow the most senior monk either to give a teaching himself or to ask someone else.” }

\section*{12. Discussion of questioning on the Monastic Law }

On\marginnote{15.6.1} one occasion the monks from the group of six questioned others on the Monastic Law in the midst of the Sangha without being approved. 

\scrule{“You shouldn’t question others on the Monastic Law in the midst of the Sangha without being approved. If you do, you commit an offense of wrong conduct. I allow you to question others on the Monastic Law in the midst of the Sangha after being approved. }

And\marginnote{15.6.6} it should be done like this. One is either approved through oneself or through someone else. How is one approved through oneself? A competent and capable monk should inform the Sangha: 

‘Please,\marginnote{15.7.3} venerables, I ask the Sangha to listen. If the Sangha is ready, I will question so-and-so on the Monastic Law.’ 

And\marginnote{15.7.6} how is one approved through someone else? A competent and capable monk should inform the Sangha: 

‘Please,\marginnote{15.7.8} venerables, I ask the Sangha to listen. If the Sangha is ready, so-and-so will question so-and-so on the Monastic Law.’ 

Soon\marginnote{15.8.1} good monks asked questions on the Monastic Law in the midst of the Sangha after being approved. The monks from the group of six became angry and bitter, and they made threats of violence. 

\scrule{“The monk who has been approved should first survey the gathering and evaluate the individuals, and then ask questions on the Monastic Law in the midst of the Sangha.” }

\section*{13. Discussion of replying to questions on the Monastic Law }

On\marginnote{15.9.1} one occasion the monks from the group of six replied to questions on the Monastic Law in the midst of the Sangha without being approved. 

\scrule{“You shouldn’t reply to questions on the Monastic Law in the midst of the Sangha without being approved. If you do, you commit an offense of wrong conduct. I allow you to reply to questions on the Monastic Law in the midst of the Sangha after being approved. }

And\marginnote{15.9.6} it should be done like this. One is either approved through oneself or through someone else. 

How\marginnote{15.10.1} is one approved through oneself? A competent and capable monk should inform the Sangha: 

‘Please,\marginnote{15.10.3} venerables, I ask the Sangha to listen. If the Sangha is ready, I will reply when asked by so-and-so on the Monastic Law.’ 

And\marginnote{15.10.6} how is one approved through someone else? A competent and capable monk should inform the Sangha: 

‘Please,\marginnote{15.10.8} venerables, I ask the Sangha to listen. If the Sangha is ready, so-and-so will reply when asked by so-and-so on the Monastic Law.’ 

Soon\marginnote{15.11.1} good monks replied to questions on the Monastic Law in the midst of the Sangha after being approved. The monks from the group of six became angry and bitter, and they made threats of violence. 

\scrule{“The monk who has been approved should first survey the gathering and evaluate the individuals, and then reply to questions on the Monastic Law in the midst of the Sangha.” }

\section*{14. Discussion of accusing }

At\marginnote{16.1.1} one time the monks from the group of six accused a monk of an offense without first getting his permission to do so. 

\scrule{“You shouldn’t accuse a monk of an offense without first getting his permission. If you do, you commit an offense of wrong conduct. You should only accuse someone of an offense after getting their permission: ‘I wish to speak to you, venerable, please give me permission.’” }

Soon,\marginnote{16.2.1} after getting their permission, good monks accused the monks from the group of six of an offense. The monks from the group of six became angry and bitter, and they made threats of violence. 

\scrule{“Even when you have their permission, you should first evaluate the individual and then accuse them of an offense.” }

At\marginnote{16.3.1} this time the monks from the group of six—thinking to act before the good monks asked them for permission, but having no reason for doing so—got permission from pure monks who had not committed any offenses. 

\scrule{“When there is no reason for doing so, you shouldn’t get permission from pure monks who haven’t committed any offenses. If you do, you commit an offense of wrong conduct. And you should give permission only after evaluating the individual.” }

\section*{15. Objecting to an illegitimate legal procedure, etc. }

On\marginnote{16.4.1} one occasion the monks from the group of six did an illegitimate legal procedure in the midst of the Sangha. 

\scrule{“You shouldn’t do illegitimate legal procedures. If you do, you commit an offense of wrong conduct.” }

They\marginnote{16.4.5} still did illegitimate procedures. 

\scrule{“You should object when an illegitimate legal procedure is being done.” }

Soon\marginnote{16.5.1} afterwards good monks objected when the monks from the group of six did an illegitimate procedure. The monks from the group of six became angry and bitter, and they made threats of violence. 

\scrule{“I also allow you to state your view.” }

They\marginnote{16.5.5} did. Once again the monks from the group of six became angry and bitter, making threats of violence. 

\scrule{“A group of four or five should object, a group of two or three may state their view, and a single person may make a silent determination: ‘I don’t approve of this.’” }

On\marginnote{16.6.1} one occasion when the monks from the group of six were reciting the Monastic Code in the midst of the Sangha, they deliberately made themselves inaudible. 

\scrule{“When reciting the Monastic Code, you shouldn’t deliberately make yourselves inaudible. If you do, you commit an offense of wrong conduct.” }

At\marginnote{16.7.1} one time Venerable \textsanskrit{Udāyī} was the Sangha’s reciter of the Monastic Code, but he had a hoarse voice. He knew that the Buddha had laid down a rule that the reciters of the Monastic Code should make themselves heard, and he thought, ‘I have a hoarse voice. What should I do?’ 

\scrule{“The reciter of the Monastic Code should make an effort to be heard. If you make an effort, there’s no offense.” }

On\marginnote{16.8.1} one occasion Devadatta recited the Monastic Code in a gathering that included lay people. 

\scrule{“You shouldn’t recite the Monastic Code in a gathering that includes lay people. If you do, you commit an offense of wrong conduct.” }

On\marginnote{16.9.1} one occasion the monks from the group of six recited the Monastic Code in the midst of the Sangha without being asked. 

\scrule{“You shouldn’t recite the Monastic Code in the midst of the Sangha without first being asked to do so. If you do, you commit an offense of wrong conduct. The most senior monk should be in charge of the recitation of the Monastic Code.” }

\scend{The first section for recitation on monastics of other religions is finished. }

\section*{16. Requesting the recitation of the Monastic Code, etc. }

When\marginnote{17.1.1} the Buddha had stayed at \textsanskrit{Rājagaha} for as long as he liked, he set out wandering toward \textsanskrit{Codanāvatthu}. When he eventually arrived, he stayed there. 

At\marginnote{17.1.3} that time a number of monks were staying in a certain monastery where the most senior monk was ignorant and incompetent. He did not know about the observance-day ceremony or the observance-day procedure, nor about the Monastic Code or its recitation. The other monks knew that the Buddha had laid down a rule that the most senior monk should be in charge of the recitation of the Monastic Code, and so they wondered what to do. They told the Buddha. 

\scrule{“In such a case, a competent and capable monk there should be in charge of the recitation of the Monastic Code.” }

On\marginnote{17.3.1} one occasion on the observance day, a number of ignorant and incompetent monks were staying in a certain monastery. They did not know about the observance-day ceremony or the observance-day procedure, nor about the Monastic Code or its recitation. They requested the most senior monk to recite the Monastic Code, but he replied that he was incapable. They made the same request of the second-most and third-most senior monks, and on both occasions received the same reply. They then requested each monk in turn until they reached the most junior monk. And they all gave the same reply. 

\scrule{“When all the monks in a monastery are ignorant and incompetent, and none of them is able to recite the Monastic Code, they should straightaway send a monk to a neighboring monastery to learn the Monastic Code, either in brief or in full.” }

The\marginnote{17.6.1} monks thought, “Who is responsible for sending someone?” 

\scrule{“The most senior monk should tell a junior monk to go.” }

Although\marginnote{17.6.5} told by the senior monk, the junior monks did not go. 

\scrule{“If a monk isn’t sick and he’s told by the most senior monk to go, he should go. If he doesn’t, he commits an offense of wrong conduct.” }

\section*{17. The instruction to learn the number of the lunar half-month, etc. }

When\marginnote{18.1.1} he had stayed at \textsanskrit{Codanāvatthu} for as long as he liked, the Buddha returned to \textsanskrit{Rājagaha}. 

Then,\marginnote{18.1.2} while the monks were walking for almsfood, people asked them which half-month it was. They replied that they did not know. People complained and criticized them, “These Sakyan monastics don’t even know the number of the lunar half-month. So how could they possibly know anything truly useful?” They told the Buddha. 

\scrule{“You should learn the counting of the lunar half-months.” }

The\marginnote{18.2.1} monks thought, “Who should learn the counting of the lunar half-months?” 

\scrule{“You should all learn the counting of the lunar half-months.” }

On\marginnote{18.3.1} another occasion, while the monks were walking for almsfood, people asked them how many monks there were. They replied that they did not know. People complained and criticized them, “These Sakyan monastics don’t even know about one another. So how could they possibly know anything truly useful?” 

\scrule{“You should count the monks.” }

The\marginnote{18.4.1} monks thought, “When should we count the monks?” 

\scrule{“You should count the monks on the observance day, either by name or by distributing tickets.”\footnote{“By name” renders the obscure compound \textit{\textsanskrit{nāmaggena}}. The commentaries are silent. Alternative readings include \textit{\textsanskrit{nāmamattena}}, \textit{\textsanskrit{gaṇamaggena}}, and \textit{nasamaggena}, none of which is an obvious fit for the current context. } }

On\marginnote{19.1.1} one occasion, monks walked for almsfood in a faraway village, not knowing that it was the observance day. They arrived back while the Monastic Code was being recited or even just after. 

\scrule{“You should announce, ‘Today is the observance day.’” }

The\marginnote{19.1.5} monks thought, “Who should make the announcement?” 

\scrule{“The most senior monk should make the announcement early in the morning.” }

Soon\marginnote{19.1.9} afterwards a certain senior monk forgot to make the announcement early in the morning. 

\scrule{“I allow you to make the announcement at the mealtime too.” }

He\marginnote{19.1.12} forgot to make the announcement at the mealtime too. 

\scrule{“I allow you to make the announcement whenever you remember.” }

\section*{18. The instruction to do the prior duties }

On\marginnote{20.1.1} one occasion in a certain monastery, the observance-day hall was dirty. Newly-arrived monks complained, “Why don’t the resident monks sweep the hall?” They told the Buddha. 

\scrule{“You should sweep the observance-day hall.” }

The\marginnote{20.2.1} monks thought, “Who should sweep it?” 

\scrule{“The most senior monk should tell a junior monk.” }

Although\marginnote{20.2.5} told by the senior monk, the junior monks did not sweep. 

\scrule{“If a monk isn’t sick and he’s told by the senior monk to sweep, he should sweep. If he doesn’t, he commits an offense of wrong conduct.” }

On\marginnote{20.3.1} one occasion no seats were prepared in the observance-day hall. The monks sat on the ground. They became dirty, as did their robes. 

\scrule{“You should prepare seats in the observance-day hall.” }

The\marginnote{20.3.5} monks thought, “Who should prepare them?” 

\scrule{“The most senior monk should tell a junior monk.” }

Although\marginnote{20.3.9} told by the senior monk, the junior monks did not prepare them. 

\scrule{“If a monk isn’t sick and he’s told by the senior monk to prepare the seats, he should do so. If he doesn’t, he commits an offense of wrong conduct.” }

On\marginnote{20.4.1} one occasion there was no lamp in the observance-day hall. Because it was dark, the monks stepped on one another and on one another’s robes. 

\scrule{“You should light a lamp in the observance-day hall.” }

The\marginnote{20.4.5} monks thought, “Who should light it?” 

\scrule{“The most senior monk should tell a junior monk.” }

Although\marginnote{20.4.9} told by the senior monk, the junior monks did not light a lamp. 

\scrule{“If a monk isn’t sick and he’s told by the senior monk to light a lamp, he should do so. If he doesn’t, he commits an offense of wrong conduct.” }

On\marginnote{20.5.1} one occasion in a certain monastery, the resident monks didn’t set out water for drinking or water for washing. Newly-arrived monks complained and criticized them, “Why don’t the resident monks set out water for drinking and water for washing?” 

\scrule{“You should set out water for drinking and water for washing.” }

The\marginnote{20.6.1} monks thought, “Who should do it?” 

\scrule{“The most senior monk should tell a junior monk.” }

Although\marginnote{20.6.5} told by the senior monk, the junior monks did not do it. 

\scrule{“If a monk isn’t sick and he’s told by the senior monk to set them out, he should do so. If he doesn’t, he commits an offense of wrong conduct.” }

\section*{19. Those going to a different region, etc. }

On\marginnote{21.1.1} one occasion a number of ignorant and incompetent monks asked permission from their teachers and preceptors to go to a different region. They told the Buddha. 

\scrule{“A number of ignorant and incompetent monks might ask their teachers and preceptors for permission to go to a different region. The teachers and preceptors should then ask them where they’re going and who they’re going with. If they’re going with others who are ignorant and incompetent, the teachers and preceptors shouldn’t give them permission. If they do, they commit an offense of wrong conduct. }

\scrule{And if the students go without permission from their teachers and preceptors, they commit an offense of wrong conduct. }

A\marginnote{21.2.1} number of ignorant and incompetent monks may be staying in a certain monastery. They don’t know about the observance-day ceremony or the observance-day procedure, nor about the Monastic Code or its recitation. Then a monk arrives who is learned and a master of the tradition; who is an expert on the Teaching, the Monastic Law, and the Key Terms; who is knowledgeable and competent, has a sense of conscience, and is afraid of wrongdoing and fond of the training.\footnote{\textit{\textsanskrit{Mātikā}}, “Key Terms”, probably refers to the two \textsanskrit{Pātimokkhas}. Sp 5.325: \textit{\textsanskrit{Pātimokkhanti} dve \textsanskrit{mātikā} na \textsanskrit{jānāti}}, “\textit{\textsanskrit{Pātimokkha}}: he does not know the two (collections of) Key Terms.” } 

\scrule{Those monks should treat that learned monk with kindness. They should assist him and befriend him, and they should attend on him with bath powder, soap, tooth cleaners, and water for rinsing the mouth.\footnote{For an explanation of rendering \textit{\textsanskrit{cuṇṇa}} and \textit{mattika} as respectively “bath powder” and “soap”, see Appendix of Technical Terms. } If they don’t look after him in this way, they commit an offense of wrong conduct. }

On\marginnote{21.3.1} the observance day, a number of ignorant and incompetent monks may be staying in a certain monastery. They don’t know about the observance-day ceremony or the observance-day procedure, nor about the Monastic Code or its recitation. They should straightaway send a monk to a neighboring monastery to learn the Monastic Code, either in brief or in full. If he’s able to do this, it’s good. 

\scrule{If he’s not, then those monks should all go to a monastery where the monks know about the observance-day ceremony and the observance-day procedure, and about the Monastic Code and its recitation. If they don’t go, they commit an offense of wrong conduct. }

A\marginnote{21.4.1} number of ignorant and incompetent monks may be spending the rainy-season residence in a certain monastery. They don’t know about the observance-day ceremony or the observance-day procedure, nor about the Monastic Code or its recitation. They should straightaway send a monk to a neighboring monastery to learn the Monastic Code, either in brief or in full. If he’s able to do this, it’s good. If he’s not, then a monk should be sent under the seven-day allowance to learn the Monastic Code, either in brief or in full. If he’s able to do this, it’s good. 

\scrule{If he’s not, then those monks shouldn’t spend the rainy-season residence in that monastery. If they do, they commit an offense of wrong conduct.” }

\section*{20. Discussion of the passing on of purity }

Then\marginnote{22.1.1} the Buddha addressed the monks: “Gather, monks, for the Sangha to do the observance-day ceremony.” A monk said to the Buddha, “Sir, there’s a sick monk. He hasn’t come.” 

\scrule{“A sick monk should pass on his purity. }

And\marginnote{22.1.6} he should do it like this. The sick monk should approach a monk, arrange his upper robe over one shoulder, and squat on his heels. He should then raise his joined palms and say, ‘I pass on my purity; please convey my purity; please announce my purity.’ If he makes this understood by body, by speech, or by body and speech, then the purity has been passed on. If he doesn’t make this understood by body, by speech, or by body and speech, then the purity hasn’t been passed on. 

If\marginnote{22.2.1} he’s able to do this, it’s good. If he’s not, then the sick monk should be brought into the midst of the Sangha together with his bed or bench. They can then do the observance-day ceremony. But if the one who is nursing him says, ‘If we move him, his illness will get worse, or he’ll die,’ then the sick monk shouldn’t be moved. The Sangha should go to where the sick monk is and do the observance-day ceremony there. 

\scrule{You shouldn’t do the observance-day ceremony with an incomplete Sangha. If you do, you commit an offense of wrong conduct. }

If,\marginnote{22.3.1} after the purity has been passed on to him, the monk who is conveying the purity goes away right then and there, then the purity should be passed on to someone else.\footnote{Sp 3.164: \textit{Tattheva \textsanskrit{pakkamatīti} \textsanskrit{saṅghamajjhaṁ} \textsanskrit{anāgantvā} tatova katthaci gacchati}, “\textit{Tattheva pakkamati}: not having gone to the midst of the Sangha, he goes wherever.” } If, after the purity has been passed on to him, the monk who is conveying the purity disrobes right then and there, dies right then and there, admits right then and there that he’s a novice monk, admits right then and there that he’s renounced the training, admits right then and there that he’s committed the worst kind of offense, admits right then and there that he’s insane, admits right then and there that he’s deranged, admits right then and there that he’s overwhelmed by pain, admits right then and there that he’s been ejected for not recognizing an offense, admits right then and there that he’s been ejected for not making amends for an offense, admits right then and there that he’s been ejected for not giving up a bad view, admits right then and there that he’s a \textit{\textsanskrit{paṇḍaka}}, admits right then and there that he’s a fake monk, admits right then and there that he’s previously left to join the monastics of another religion, admits right then and there that he’s an animal, admits right then and there that he’s a matricide, admits right then and there that he’s a patricide, admits right then and there that he’s a murderer of a perfected one, admits right then and there that he’s raped a nun,\footnote{“Raped” renders \textit{\textsanskrit{dūsaka}}. See Appendix of Technical Terms. } admits right then and there that he’s caused a schism in the Sangha, admits right then and there that he’s caused the Buddha to bleed, or admits right then and there that he’s a hermaphrodite, then the purity should be passed on to someone else. 

If,\marginnote{22.4.1} after the purity has been passed on to him, the monk who is conveying the purity goes away while on his way to the observance-day ceremony, then the purity hasn’t been brought. If, after the purity has been passed on to him, the monk who is conveying the purity disrobes while on his way to the observance-day ceremony … or admits that he’s a hermaphrodite while on his way to the observance-day ceremony, then the purity hasn’t been brought. 

But\marginnote{22.4.4} if, after the purity has been passed on to him, the monk who is conveying the purity goes away after reaching the Sangha, then the purity has been brought. And if, after the purity has been passed on to him, the monk who is conveying the purity disrobes after reaching the Sangha … or admits that he’s a hermaphrodite after reaching the Sangha, then the purity has been brought. 

And\marginnote{22.4.7} if, after the purity has been passed on to him, the monk who is conveying the purity reaches the Sangha, but doesn’t announce the purity because he falls asleep or is heedless or gains a meditation attainment, then the purity has been brought. There’s no offense for the one who is conveying the purity. 

And\marginnote{22.4.9} if, after the purity has been passed on to him, the monk who is conveying the purity reaches the Sangha, but deliberately doesn’t announce the purity, then the purity has been brought. 

\scrule{But there’s an offense of wrong conduct for the one who is conveying the purity.” }

\section*{21. Discussion on giving consent }

The\marginnote{23.1.1} Buddha addressed the monks: “Gather, monks, for the Sangha to do a legal procedure.” A monk said to the Buddha, “Sir, there’s a sick monk. He hasn’t come.” 

\scrule{“A sick monk should give his consent. }

And\marginnote{23.1.6} he should give like this. The sick monk should approach a monk, arrange his upper robe over one shoulder, and squat on his heels. He should then raise his joined palms and say, ‘I give my consent; please convey my consent; please announce my consent.’ If he makes this understood by body, by speech, or by body and speech, then the consent has been given. If he doesn’t make this understood by body, by speech, or by body and speech, then the consent hasn’t been given. 

If\marginnote{23.2.1} he’s able to do this, it’s good. If he’s not, then the sick monk should be brought into the midst of the Sangha together with his bed or bench. They can then do the procedure. But if the one who is nursing him says, ‘If we move him, his illness will get worse, or he’ll die,’ then the sick monk shouldn’t be moved. The Sangha should go to where the sick monk is and do the procedure there. 

\scrule{You shouldn’t do a legal procedure with an incomplete sangha. If you do, you commit an offense of wrong conduct. }

If,\marginnote{23.3.1} after the consent has been given to him, the monk who is conveying the consent goes away right then and there, then the consent should be given to someone else. If, after the consent has been given to him, the monk who is conveying the consent disrobes right then and there, dies right then and there, admits right then and there that he’s a novice monk, admits right then and there that he’s renounced the training, admits right then and there that he’s committed the worst kind of offense, admits right then and there that he’s insane, admits right then and there that he’s deranged, admits right then and there that he’s overwhelmed by pain, admits right then and there that he’s been ejected for not recognizing an offense, admits right then and there that he’s been ejected for not making amends for an offense, admits right then and there that he’s been ejected for not giving up a bad view, admits right then and there that he’s a \textit{\textsanskrit{paṇḍaka}}, admits right then and there that he’s a fake monk, admits right then and there that he’s previously left to join the monastics of another religion, admits right then and there that he’s an animal, admits right then and there that he’s a matricide, admits right then and there that he’s a patricide, admits right then and there that he’s a murderer of a perfected one, admits right then and there that he’s raped a nun, admits right then and there that he’s caused a schism in the Sangha, admits right then and there that he’s caused the Buddha to bleed, or admits right then and there that he’s a hermaphrodite, then the consent should be given to someone else. 

If,\marginnote{23.3.24} after the consent has been given to him, the monk who is conveying the consent goes away while on his way to the legal procedure, then the consent hasn’t been brought. If, after the consent has been given to him, the monk who is conveying the consent disrobes while on his way to the legal procedure … or admits that he’s a hermaphrodite while on his way to the legal procedure, then the consent hasn’t been brought. 

But\marginnote{23.3.27} if, after the consent has been given to him, the monk who is conveying the consent goes away after reaching the Sangha, then the consent has been brought. And if, after the consent has been given to him, the monk who is conveying the consent disrobes after reaching the Sangha … or admits that he’s a hermaphrodite after reaching the Sangha, then the consent has been brought. 

And\marginnote{23.3.30} if, after the consent has been given to him, the monk who is conveying the consent reaches the Sangha, but doesn’t announce the consent because he falls asleep or is heedless or gains a meditation attainment, then the consent has been brought. There’s no offense for the one who is conveying the consent. 

And\marginnote{23.3.32} if, after the consent has been given to him, the monk who is conveying the consent reaches the Sangha, but deliberately doesn’t announce the consent, then the consent has been brought. 

\scrule{But there is an offense of wrong conduct for the one who is conveying the consent. }

\scrule{On the observance day, if the Sangha has business to be done, then anyone passing on their purity should also give their consent.” }

\section*{22. Discussion on being seized by relatives, etc. }

At\marginnote{24.1.1} one time on the observance day, a certain monk was seized by his relatives. They told the Buddha. 

“If\marginnote{24.1.3} a monk is seized by his relatives on the observance day, other monks should say to those relatives, ‘Listen, please release this monk for a short time so that he can take part in the observance-day ceremony.’ If they’re able to do this, it’s good. If not, they should say to those relatives, ‘Listen, please step aside for a moment while this monk passes on his purity.’ If they’re able to do this, it’s good. If not, they should say to those relatives, ‘Listen, please take this monk outside the monastery zone for a short time while the Sangha does the observance-day ceremony.’ If they’re able to do this, it’s good. 

\scrule{If not, you shouldn’t do the observance-day ceremony with an incomplete sangha. If you do, you commit an offense of wrong conduct. }

If\marginnote{24.3.1} on the observance day a monk is seized by kings, by bandits, by scoundrels, or by enemies of the monks, other monks should say to those enemies,\footnote{“Enemies of monks” is a translation of \textit{\textsanskrit{bhikkhupaccatthikā}}. At \href{https://suttacentral.net/pli-tv-bu-vb-pj1/en/brahmali\#9.3.1}{Bu Pj 1:9.3.1}, I have translated the same compound as “enemy monks”. In that rule this seems required because various people who are acting as enemies of monks are mentioned separately, such as kings, bandits, and scoundrels. Moreover, all of these are compounded with \textit{\textsanskrit{paccatthikā}}: \textit{\textsanskrit{bhikkhupaccatthikā}}, \textit{\textsanskrit{rājapaccatthikā}}, and so on. Since it seems reasonable to assume that all these compounds have the same structure, it follows that they should all be read as “enemies who are so-and-so” rather than “enemies of so-and-so”. This understanding is confirmed by Sp 1.58: \textit{\textsanskrit{bhikkhū} eva \textsanskrit{paccatthikā} \textsanskrit{bhikkhupaccatthikā}}, “\textit{\textsanskrit{Bhikkhupaccatthikā}} are just monks who are enemies.” In the present context, however, this interpretation does not seem to work. If \textit{\textsanskrit{bhikkhupaccatthikā}} refers to enemies who are monks, then they would have to be invited to take part in the ceremony, or some other arrangement would have to be made, but nothing is said about this in either the Pali or the commentaries. Moreover, kings, bandits, and scoundrels are in this case not compounded with \textit{\textsanskrit{paccatthikā}}, as they are in Bu Pj 1. I therefore conclude that the meaning here must be “enemies of monks”. } ‘Listen, please release this monk for a short time so that he can take part in the observance-day ceremony.’ If they’re able to do this, it’s good. If not, they should say to those enemies, ‘Listen, please step aside for a moment while this monk passes on his purity.’ If they’re able to do this, it’s good. If not, they should say to those enemies, ‘Listen, please take this monk outside the monastery zone for a short time while the Sangha does the observance-day ceremony.’ If they’re able to do this, it’s good. 

\scrule{If not, you shouldn’t do the observance-day ceremony with an incomplete sangha. If you do, you commit an offense of wrong conduct.” }

\section*{23. Agreement in regard to insanity }

Then\marginnote{25.1.1} the Buddha addressed the monks: “Gather, monks, there’s business for the Sangha.” A monk said to the Buddha, “Sir, there’s a monk called Gagga who is insane. He hasn’t come.” 

“Monks,\marginnote{25.1.5} there are two kinds of insane monks: there is the insane monk who sometimes remembers the observance day and sometimes doesn’t, who sometimes remembers the legal procedures of the Sangha and sometimes doesn’t, who sometimes goes to the observance-day ceremony and sometimes doesn’t, who sometimes goes to the legal procedures of the Sangha and sometimes doesn’t. Then there’s the insane monk who never remembers any of this. 

\scrule{For the first one of these, you should make an agreement in regard to insanity. }

And\marginnote{25.3.1} it should be made like this. A competent and capable monk should inform the Sangha: 

‘Please,\marginnote{25.3.3} venerables, I ask the Sangha to listen. The monk Gagga is insane. Sometimes he remembers the observance day and sometimes he doesn’t; sometimes he remembers the legal procedures of the Sangha and sometimes he doesn’t; sometimes he goes to the observance-day ceremony and sometimes he doesn’t; sometimes he goes to the legal procedures of the Sangha and sometimes he doesn’t. If the Sangha is ready, it should agree on the following in regard to the insanity of the monk Gagga: whether or not Gagga remembers either the observance day or the legal procedures of the Sangha, whether or not he comes to either, the Sangha should do the observance-day ceremony, it should do the legal procedures of the Sangha, with or without Gagga. This is the motion. 

Please,\marginnote{25.4.1} venerables, I ask the Sangha to listen. The monk Gagga is insane. Sometimes he remembers the observance day and sometimes he doesn’t; sometimes he remembers the legal procedures of the Sangha and sometimes he doesn’t; sometimes he goes to the observance-day ceremony and sometimes he doesn’t; sometimes he goes to the legal procedures of the Sangha and sometimes he doesn’t. The Sangha agrees on the following in regard to the insanity of the monk Gagga: whether or not Gagga remembers either the observance day or the legal procedures of the Sangha, whether or not he comes to either, the Sangha should do the observance-day ceremony, it should do the legal procedures of the Sangha, with or without Gagga. Any monk who approves of this agreement—whether or not Gagga remembers either the observance day or the legal procedures of the Sangha, whether or not he comes to either, the Sangha should do the observance-day ceremony, it should do the legal procedures of the Sangha, with or without Gagga—should remain silent. Any monk who doesn’t approve should speak up. 

The\marginnote{25.4.9} Sangha has agreed on the following in regard to the insanity of the monk Gagga: whether or not Gagga remembers either the observance-day ceremony or the legal procedures of the Sangha, whether or not he comes to either, the Sangha should do the observance-day ceremony, it should do the legal procedures of the Sangha, with or without Gagga. The Sangha approves and is therefore silent. I’ll remember it thus.’” 

\section*{24. Various kinds of observance days for the Sangha, etc. }

At\marginnote{26.1.1} one time on the observance day, there were four monks staying in a certain monastery. They thought, “The Buddha has laid down a rule that the observance-day ceremony should be done. Now there’s four of us. So how should we do the observance-day ceremony?” They told the Buddha. 

\scrule{“When there are four of you, you should recite the Monastic Code.” }

At\marginnote{26.2.1} one time on the observance day, there were three monks staying in a certain monastery. They thought, “The Buddha has instructed that the Monastic Code should be recited when there are four monks. But there’s only three of us. So how should we do the observance-day ceremony?” 

\scrule{“When there are three of you, you should do the observance-day ceremony by declaring your purity. }

And\marginnote{26.3.1} you should do it like this. A competent and capable monk should inform those monks: 

‘Please,\marginnote{26.3.3} venerables, I ask you to listen. Today is the observance day, the fifteenth. If the venerables are ready, we should do the observance-day ceremony by declaring purity to one another.’ 

The\marginnote{26.3.6} most senior monk should arrange his upper robe over one shoulder, squat on his heels, raise his joined palms, and say to the other monks:\footnote{\textit{Therena \textsanskrit{bhikkhunā}} could be rendered “a/the senior monk”. Yet the point is that only the most senior member of the Sangha should use the semi-informal address \textit{\textsanskrit{āvuso}}, whereas everyone else should use the formal equivalent \textit{bhante}. } ‘I’m pure. Please remember me as pure.’ And he should repeat this two more times. 

Each\marginnote{26.4.1} junior monk should arrange his upper robe over one shoulder, squat on his heels, raise his joined palms, and say to the other monks:\footnote{\textit{Navakena \textsanskrit{bhikkhunā}} could be rendered “a/the junior monk”. Yet the point here is that only the most senior member of the Sangha should use the semi-informal address \textit{\textsanskrit{āvuso}}, whereas everyone else should use the formal equivalent \textit{bhante}. In this context, then, \textit{navaka} does not have its normal meaning of “newly ordained” monk, but rather refers to any monk junior to the most senior one. } ‘I’m pure, venerable. Please remember me as pure.’ And he should repeat this two more times.” 

At\marginnote{26.5.1} one time on the observance day, there were two monks staying in a certain monastery. They thought, “The Buddha has instructed that the Monastic Code should be recited when there are four monks and that the observance-day ceremony should be done by declaring purity when there are three. But there’s only two of us. So how should we do the observance-day ceremony?” 

\scrule{“When there are two of you, you should do the observance-day ceremony by declaring your purity. }

And\marginnote{26.6.1} you should do it like this. 

The\marginnote{26.6.2} senior monk should arrange his upper robe over one shoulder, squat on his heels, raise his joined palms, and say to the junior monk: ‘I’m pure. Please remember me as pure.’ And he should repeat this two more times. 

The\marginnote{26.7.1} junior monk should arrange his upper robe over one shoulder, squat on his heels, raise his joined palms, and say to the senior monk: ‘I’m pure, venerable. Please remember me as pure.’ And he should repeat this two more times.” 

At\marginnote{26.8.1} one time on the observance day, a monk was staying in a certain monastery by himself. He thought, “The Buddha has instructed that the Monastic Code should be recited when there are four monks and that the observance-day ceremony should be done by declaring purity when there are two or three. But I’m here by myself. So how should I do the observance-day ceremony?” 

“On\marginnote{26.9.1} the observance day, a monk may be staying by himself in a certain monastery. He should sweep the place where the monks normally go: whether the assembly hall, under a roof cover, or at the foot of a tree. He should set out water for drinking and water for washing. He should prepare a seat, light a lamp, and sit down. 

If\marginnote{26.9.3} other monks arrive, he should do the observance-day ceremony with them. If not, he should determine: ‘Today is my observance day.’ 

\scrule{If he doesn’t make a determination, he commits an offense of wrong conduct. }

\scrule{Wherever four monks are staying together, three shouldn’t recite the Monastic Code, while the purity of the fourth is brought. If you do recite the Monastic Code, you commit an offense of wrong conduct. }

\scrule{Wherever three monks are staying together, two shouldn’t do the observance-day ceremony by declaring purity, while the purity of the third is brought. If you do declare purity, you commit an offense of wrong conduct. }

\scrule{Wherever two monks are staying together, one shouldn’t make a determination, while the purity of the other is brought. If you do make a determination, you commit an offense of wrong conduct.” }

\section*{25. The process for making amends for an offense }

At\marginnote{27.1.1} one time on the observance day, a certain monk committed an offense. He thought, “The Buddha has laid down a rule that one shouldn’t do the observance-day ceremony if one has an unconfessed offense.\footnote{It is not clear what this refers to. It could be that it is a reference to the \textsanskrit{Pātimokkha} preamble which states that “anyone who has committed an offense should reveal it” (\href{https://suttacentral.net/pli-tv-kd2/en/brahmali\#3.3.3}{Kd 2:3.3.3}). } And I’ve committed an offense. So what should I do?” They told the Buddha. 

“On\marginnote{27.1.8} the observance day, a monk may have committed an offense. He should approach a single monk, arrange his upper robe over one shoulder, squat on his heels, raise his joined palms, and say: 

‘I’ve\marginnote{27.1.10} committed such-and-such an offense. I confess it.’ The other should say, ‘Do you recognize the offense?’ —‘Yes, I recognize it.’ —‘You should restrain yourself in the future.’ 

On\marginnote{27.2.1} the observance day, a monk may be unsure if he’s committed an offense. He should approach a single monk, arrange his upper robe over one shoulder, squat on his heels, raise his joined palms, and say: 

‘I’m\marginnote{27.2.3} unsure if I’ve committed such-and-such an offense. I’ll make amends for it when I’m sure.’ He can then take part in the observance-day ceremony and listen to the recitation of the Monastic Code. This is not an obstacle to doing the observance-day ceremony.” 

On\marginnote{27.3.1} one occasion the monks from the group of six confessed shared offenses with one another. 

\scrule{“You shouldn’t confess shared offenses with one another. If you do, you commit an offense of wrong conduct.” }

On\marginnote{27.3.5} one occasion the monks from the group of six received the confession of shared offenses from one another. 

\scrule{“You shouldn’t receive the confession of shared offenses from one another. If you do, you commit an offense of wrong conduct.” }

\section*{26. The process for revealing an offense }

At\marginnote{27.4.1} one time a certain monk remembered an offense while the Monastic Code was being recited. He thought, “The Buddha has laid down a rule that one shouldn’t do the observance-day ceremony if one has an unconfessed offense. And I’ve committed an offense. So what should I do?” They told the Buddha. 

\scrule{“A monk may remember an offense while the Monastic Code is being recited. He should say to a monk sitting next to him, ‘I’ve committed such-and-such an offense. Once this ceremony is finished, I’ll make amends for it.’ They can then continue the observance-day ceremony and listen to the recitation of the Monastic Code. This is not an obstacle to doing the observance-day ceremony. }

\scrule{A monk may become unsure if he’s committed an offense while the Monastic Code is being recited. He should say to a monk sitting next to him, ‘I’m unsure if I’ve committed such-and-such an offense. I’ll make amends for it when I’m sure.’ They can then continue the observance-day ceremony and listen to the recitation of the Monastic Code. This is not an obstacle to doing the observance-day ceremony.” }

\section*{27. The process for making amends for a shared offense }

At\marginnote{27.6.1} one time on the observance day, the whole Sangha in a certain monastery had committed the same offense. The monks thought, “The Buddha has laid down a rule that one shouldn’t confess or receive the confession of shared offenses. Yet here the whole Sangha has committed the same offense. So what should we do?” 

\scrule{“On the observance day, the whole Sangha in a certain monastery may have committed the same offense. Those monks should straightaway send a monk to a neighboring monastery: ‘Go and make amends for this offense. When you return, we’ll make amends for it with you.’ }

If\marginnote{27.7.1} he’s able to do this, it’s good. If he’s not, then a competent and capable monk should inform the Sangha: 

‘Please,\marginnote{27.7.3} venerables, I ask the Sangha to listen. This whole Sangha has committed the same offense. When the Sangha sees another monk who is pure and free of offenses, it should make amends for this offense with him.’ 

Once\marginnote{27.7.6} this has been said, they can do the observance-day ceremony and listen to the recitation of the Monastic Code.  This is not an obstacle to doing the observance-day ceremony. 

On\marginnote{27.8.1} the observance day, the whole Sangha in a certain monastery may be unsure if it has committed the same offense. A competent and capable monk should then inform the Sangha: 

‘Please,\marginnote{27.8.3} venerables, I ask the Sangha to listen. This whole Sangha is unsure if it has committed the same offense. When the Sangha is sure, it should make amends for this offense.’ 

Once\marginnote{27.8.6} this has been said, they can do the observance-day ceremony and listen to the recitation of the Monastic Code. This is not an obstacle to doing the observance-day ceremony. 

\scrule{When a sangha has entered the rainy-season residence in a certain monastery, it may happen that the whole Sangha has committed the same offense. Those monks should straightaway send a monk to a neighboring monastery: ‘Go and make amends for this offense. When you return, we’ll make amends for it with you.’ }

If\marginnote{27.9.5} he’s able to do this, it’s good. If he’s not, they should send a monk under the seven-day allowance: ‘Go and make amends for this offense. When you return, we’ll make amends for it with you.’” 

Soon\marginnote{27.10.1} afterwards the whole Sangha in a certain monastery had committed the same offense. They did not know the name or the category of that offense. Then a monk arrived who was learned and a master of the tradition; who was an expert on the Teaching, the Monastic Law, and the Key Terms; who was knowledgeable and competent, had a sense of conscience, and was afraid of wrongdoing and fond of the training. A monk went up to him and asked, “When someone does such-and-such, what’s the name of the offense he’s committed?” The learned monk replied, “When someone does such-and-such, he’s committed an offense of this name. If you have committed this offense, you should make amends for it.” The other monk said, “It’s not just I alone who have committed this offense, but this whole Sangha.” The learned monk said, “What does it matter to you whether others have or haven’t committed an offense? Please clear yourself of your own offense.” 

Then,\marginnote{27.12.1} because of what the learned monk had said, the other monk made amends for that offense. He then went to the other monks and said, “When someone does such-and-such, he’s committed an offense of this name. This is the offense that you’ve committed. You should make amends for it.” But in spite of what he had said, those monks did not want to make amends for that offense. They told the Buddha. 

\scrule{“In a case such as this, if those monks do make amends for that offense because of what that monk has said, then this is good. If they don’t, then that monk doesn’t need to correct them if he doesn’t want to.” }

\scend{The second section for recitation on the grounds for accusations is finished. }

\section*{28. The group of fifteen on non-offenses }

At\marginnote{28.1.1} one time on the observance day, four or more resident monks had gathered together in a certain monastery. They did not know that there were other resident monks who had not arrived.\footnote{Here “resident monk” means a monk who is within the \textit{\textsanskrit{sīmā}}, the monastery zone. } Perceiving that they were acting according to the Teaching and the Monastic Law, perceiving that the assembly was complete although it was not, they did the observance-day ceremony and recited the Monastic Code. While they were doing it, a greater number of resident monks arrived. They told the Buddha. 

“On\marginnote{28.2.1} the observance day, four or more resident monks may have gathered together in a certain monastery. They don’t know there are other resident monks who haven’t arrived. Perceiving that they’re acting according to the Teaching and the Monastic Law, perceiving that the assembly is complete although it’s not, they do the observance-day ceremony and recite the Monastic Code. While they’re doing it, a greater number of resident monks arrive. 

\scrule{In such a case, those monks should recite the Monastic Code once more. There’s no offense for the reciters. }

On\marginnote{28.3.1} the observance day, four or more resident monks may have gathered together in a certain monastery. They don’t know there are other resident monks who haven’t arrived. Perceiving that they’re acting according to the Teaching and the Monastic Law, perceiving that the assembly is complete although it’s not, they do the observance-day ceremony and recite the Monastic Code. While they’re doing it, an equal number of resident monks arrive. 

\scrule{In such a case, what has been recited is valid, and the remainder should be listened to. There’s no offense for the reciters. }

On\marginnote{28.3.8} the observance day, four or more resident monks may have gathered together in a certain monastery. They don’t know there are other resident monks who haven’t arrived. Perceiving that they’re acting according to the Teaching and the Monastic Law, perceiving that the assembly is complete although it’s not, they do the observance-day ceremony and recite the Monastic Code. While they’re doing it, a smaller number of resident monks arrive. 

\scrule{In such a case, what has been recited is valid, and the remainder should be listened to. There’s no offense for the reciters. }

On\marginnote{28.4.1} the observance day, four or more resident monks may have gathered together in a certain monastery. They don’t know there are other resident monks who haven’t arrived. Perceiving that they’re acting according to the Teaching and the Monastic Law, perceiving that the assembly is complete although it’s not, they do the observance-day ceremony and recite the Monastic Code. When they’ve just finished, a greater number of resident monks arrive. 

\scrule{In such a case, those monks should recite the Monastic Code once more. There’s no offense for the reciters. }

On\marginnote{28.4.8} the observance day, four or more resident monks may have gathered together in a certain monastery. They don’t know there are other resident monks who haven’t arrived. Perceiving that they’re acting according to the Teaching and the Monastic Law, perceiving that the assembly is complete although it’s not, they do the observance-day ceremony and recite the Monastic Code. When they’ve just finished, an equal number of resident monks arrive. 

\scrule{In such a case, what has been recited is valid, and the late arrivals should announce their purity in the presence of the others. There’s no offense for the reciters. }

On\marginnote{28.4.15} the observance day, four or more resident monks may have gathered together in a certain monastery. They don’t know there are other resident monks who haven’t arrived. Perceiving that they’re acting according to the Teaching and the Monastic Law, perceiving that the assembly is complete although it’s not, they do the observance-day ceremony and recite the Monastic Code. When they’ve just finished, a smaller number of resident monks arrive. 

\scrule{In such a case, what has been recited is valid, and the late arrivals should announce their purity in the presence of the others. There’s no offense for the reciters. }

On\marginnote{28.5.1} the observance day, four or more resident monks may have gathered together in a certain monastery. They don’t know there are other resident monks who haven’t arrived. Perceiving that they’re acting according to the Teaching and the Monastic Law, perceiving that the assembly is complete although it’s not, they do the observance-day ceremony and recite the Monastic Code. When they’ve just finished, and none of the gathering has left, a greater number of resident monks arrive.\footnote{\textit{\textsanskrit{Avuṭṭhitāya} \textsanskrit{parisāya}} literally means that “the gathering has not got up”. The point, presumably, is that the meeting is not yet over and those present have not started to leave. } 

\scrule{In such a case, those monks should recite the Monastic Code once more. There’s no offense for the reciters. }

On\marginnote{28.5.8} the observance day, four or more resident monks may have gathered together in a certain monastery. They don’t know there are other resident monks who haven’t arrived. Perceiving that they’re acting according to the Teaching and the Monastic Law, perceiving that the assembly is complete although it’s not, they do the observance-day ceremony and recite the Monastic Code. When they’ve just finished, and none of the gathering has left, an equal number of resident monks arrive. 

\scrule{In such a case, what has been recited is valid, and the late arrivals should announce their purity in the presence of the others. There’s no offense for the reciters. }

On\marginnote{28.5.15} the observance day, four or more resident monks may have gathered together in a certain monastery. They don’t know there are other resident monks who haven’t arrived. Perceiving that they’re acting according to the Teaching and the Monastic Law, perceiving that the assembly is complete although it’s not, they do the observance-day ceremony and recite the Monastic Code. When they’ve just finished, and none of the gathering has left, a smaller number of resident monks arrive. 

\scrule{In such a case, what has been recited is valid, and the late arrivals should announce their purity in the presence of the others. There’s no offense for the reciters. }

On\marginnote{28.6.1} the observance day, four or more resident monks may have gathered together in a certain monastery. They don’t know there are other resident monks who haven’t arrived. Perceiving that they’re acting according to the Teaching and the Monastic Law, perceiving that the assembly is complete although it’s not, they do the observance-day ceremony and recite the Monastic Code. When they’ve just finished, and only some members of the gathering have left, a greater number of resident monks arrive. 

\scrule{In such a case, those monks should recite the Monastic Code once more. There’s no offense for the reciters. }

On\marginnote{28.6.8} the observance day, four or more resident monks may have gathered together in a certain monastery. They don’t know there are other resident monks who haven’t arrived. Perceiving that they’re acting according to the Teaching and the Monastic Law, perceiving that the assembly is complete although it’s not, they do the observance-day ceremony and recite the Monastic Code. When they’ve just finished, and only some members of the gathering have left, an equal number of resident monks arrive. 

\scrule{In such a case, what has been recited is valid, and the late arrivals should announce their purity in the presence of the others. There’s no offense for the reciters. }

On\marginnote{28.6.15} the observance day, four or more resident monks may have gathered together in a certain monastery. They don’t know there are other resident monks who haven’t arrived. Perceiving that they’re acting according to the Teaching and the Monastic Law, perceiving that the assembly is complete although it’s not, they do the observance-day ceremony and recite the Monastic Code. When they’ve just finished, and only some members of the gathering have left, a smaller number of resident monks arrive. 

\scrule{In such a case, what has been recited is valid, and the late arrivals should announce their purity in the presence of the others. There’s no offense for the reciters. }

On\marginnote{28.7.1} the observance day, four or more resident monks may have gathered together in a certain monastery. They don’t know there are other resident monks who haven’t arrived. Perceiving that they’re acting according to the Teaching and the Monastic Law, perceiving that the assembly is complete although it’s not, they do the observance-day ceremony and recite the Monastic Code. When they’ve just finished, and the entire gathering has left, a greater number of resident monks arrive. 

\scrule{In such a case, those monks should recite the Monastic Code once more. There’s no offense for the reciters. }

On\marginnote{28.7.8} the observance day, four or more resident monks may have gathered together in a certain monastery. They don’t know there are other resident monks who haven’t arrived. Perceiving that they’re acting according to the Teaching and the Monastic Law, perceiving that the assembly is complete although it’s not, they do the observance-day ceremony and recite the Monastic Code. When they’ve just finished, and the entire gathering has left, an equal number of resident monks arrive. 

\scrule{In such a case, what has been recited is valid, and the late arrivals should announce their purity in the presence of the others. There’s no offense for the reciters. }

On\marginnote{28.7.15} the observance day, four or more resident monks may have gathered together in a certain monastery. They don’t know there are other resident monks who haven’t arrived. Perceiving that they’re acting according to the Teaching and the Monastic Law, perceiving that the assembly is complete although it’s not, they do the observance-day ceremony and recite the Monastic Code. When they’ve just finished, and the entire gathering has left, a smaller number of resident monks arrive. 

\scrule{In such a case, what has been recited is valid, and the late arrivals should announce their purity in the presence of the others. There’s no offense for the reciters.” }

\scend{The group of fifteen on non-offenses is finished. }

\section*{29. The group of fifteen on perceiving an incomplete assembly as incomplete }

“On\marginnote{29.1.1} the observance day, four or more resident monks may have gathered together in a certain monastery. They know there are other resident monks who haven’t arrived. Perceiving that they’re acting according to the Teaching and the Monastic Law, yet correctly perceiving the assembly as incomplete, they do the observance-day ceremony and recite the Monastic Code. While they’re doing it, a greater number of resident monks arrive. 

\scrule{In such a case, those monks should recite the Monastic Code once more. There’s an offense of wrong conduct for the reciters. }

On\marginnote{29.2.1} the observance day, four or more resident monks may have gathered together in a certain monastery. They know there are other resident monks who haven’t arrived. Perceiving that they’re acting according to the Teaching and the Monastic Law, yet correctly perceiving the assembly as incomplete, they do the observance-day ceremony and recite the Monastic Code. While they’re doing it, an equal number of resident monks arrive. 

\scrule{In such a case, what has been recited is valid, and the remainder should be listened to. There’s an offense of wrong conduct for the reciters. }

On\marginnote{29.2.8} the observance day, four or more resident monks may have gathered together in a certain monastery. They know there are other resident monks who haven’t arrived. Perceiving that they’re acting according to the Teaching and the Monastic Law, yet correctly perceiving the assembly as incomplete, they do the observance-day ceremony and recite the Monastic Code. While they’re doing it, a smaller number of resident monks arrive. 

\scrule{In such a case, what has been recited is valid, and the remainder should be listened to. There’s an offense of wrong conduct for the reciters. }

On\marginnote{29.3.1} the observance day, four or more resident monks may have gathered together in a certain monastery. They know there are other resident monks who haven’t arrived. Perceiving that they’re acting according to the Teaching and the Monastic Law, yet correctly perceiving the assembly as incomplete, they do the observance-day ceremony and recite the Monastic Code. When they’ve just finished … When they’ve just finished, and none of the gathering has left … When they’ve just finished, and only some members of the gathering have left … When they’ve just finished, and the entire gathering has left, a greater number of resident monks arrive … an equal number of resident monks arrive … a smaller number of resident monks arrive. 

\scrule{In such a case, what has been recited is valid, and the late arrivals should announce their purity in the presence of the others. There’s an offense of wrong conduct for the reciters.” }

\scend{The group of fifteen on perceiving an incomplete assembly as incomplete is finished. }

\section*{30. The group of fifteen on being unsure }

“On\marginnote{30.1.1} the observance day, four or more resident monks may have gathered together in a certain monastery. They know there are other resident monks who haven’t arrived. They think, ‘Is it allowable for us to do the observance-day ceremony or not?’ Being unsure, they do the observance-day ceremony and recite the Monastic Code. While they’re doing it, a greater number of resident monks arrive. 

\scrule{In such a case, those monks should recite the Monastic Code once more. There’s an offense of wrong conduct for the reciters. }

On\marginnote{30.2.1} the observance day, four or more resident monks may have gathered together in a certain monastery. They know there are other resident monks who haven’t arrived. They think, ‘Is it allowable for us to do the observance-day ceremony or not?’ Being unsure, they do the observance-day ceremony and recite the Monastic Code. While they’re doing it, an equal number of resident monks arrive. 

\scrule{In such a case, what has been recited is valid, and the remainder should be listened to. There’s an offense of wrong conduct for the reciters. }

On\marginnote{30.2.9} the observance day, four or more resident monks may have gathered together in a certain monastery. They know there are other resident monks who haven’t arrived. They think, ‘Is it allowable for us to do the observance-day ceremony or not?’ Being unsure, they do the observance-day ceremony and recite the Monastic Code. While they’re doing it, a smaller number of resident monks arrive. 

\scrule{In such a case, what has been recited is valid, and the remainder should be listened to. There’s an offense of wrong conduct for the reciters. }

On\marginnote{30.2.17} the observance day, four or more resident monks may have gathered together in a certain monastery. They know there are other resident monks who haven’t arrived. They think, ‘Is it allowable for us to do the observance-day ceremony or not?’ Being unsure, they do the observance-day ceremony and recite the Monastic Code. When they’ve just finished … When they’ve just finished, and none of the gathering has left … When they’ve just finished, and only some members of the gathering have left … When they’ve just finished, and the entire gathering has left, a greater number of resident monks arrive … an equal number of resident monks arrive … a smaller number of resident monks arrive. 

\scrule{In such a case, what has been recited is valid, and the late arrivals should announce their purity in the presence of the others. There’s an offense of wrong conduct for the reciters.” }

\scend{The group of fifteen on being unsure is finished. }

\section*{31. The group of fifteen on being anxious }

“On\marginnote{31.1.1} the observance day, four or more resident monks may have gathered together in a certain monastery. They know there are other resident monks who haven’t arrived. They think, ‘It’s allowable for us to do the observance-day ceremony; it’s not unallowable.’ Being anxious, they do the observance-day ceremony and recite the Monastic Code. While they’re doing it, a greater number of resident monks arrive. 

\scrule{In such a case, those monks should recite the Monastic Code once more. There’s an offense of wrong conduct for the reciters. }

On\marginnote{31.2.1} the observance day, four or more resident monks may have gathered together in a certain monastery. They know there are other resident monks who haven’t arrived. They think, ‘It’s allowable for us to do the observance-day ceremony; it’s not unallowable.’ Being anxious, they do the observance-day ceremony and recite the Monastic Code. While they’re doing it, an equal number of resident monks arrive. 

\scrule{In such a case, what has been recited is valid, and the remainder should be listened to. There’s an offense of wrong conduct for the reciters. }

On\marginnote{31.2.9} the observance day, four or more resident monks may have gathered together in a certain monastery. They know there are other resident monks who haven’t arrived. They think, ‘It’s allowable for us to do the observance-day ceremony; it’s not unallowable.’ Being anxious, they do the observance-day ceremony and recite the Monastic Code. While they’re doing it, a smaller number of resident monks arrive. 

\scrule{In such a case, what has been recited is valid, and the remainder should be listened to. There’s an offense of wrong conduct for the reciters. }

On\marginnote{31.2.17} the observance day, four or more resident monks may have gathered together in a certain monastery. They know there are other resident monks who haven’t arrived. They think, ‘It’s allowable for us to do the observance-day ceremony; it’s not unallowable.’ Being anxious, they do the observance-day ceremony and recite the Monastic Code. When they’ve just finished … When they’ve just finished, and none of the gathering has left … When they’ve just finished, and only some members of the gathering have left … When they’ve just finished, and the entire gathering has left, a greater number of resident monks arrive … an equal number of resident monks arrive … a smaller number of resident monks arrive. 

\scrule{In such a case, what has been recited is valid, and the late arrivals should announce their purity in the presence of the others. There’s an offense of wrong conduct for the reciters.” }

\scend{The group of fifteen on being anxious is finished. }

\section*{32. The group of fifteen on aiming at schism }

“On\marginnote{32.1.1} the observance day, four or more resident monks may have gathered together in a certain monastery. They know there are other resident monks who haven’t arrived. They think, ‘May they get lost! May they disappear! We are better off without them.’ They then do the observance-day ceremony and recite the Monastic Code, aiming at schism. While they’re doing it, a greater number of resident monks arrive. 

\scrule{In such a case, those monks should recite the Monastic Code once more. And there’s a serious offense for the reciters. }

On\marginnote{32.2.1} the observance day, four or more resident monks may have gathered together in a certain monastery. They know there are other resident monks who haven’t arrived. They think, ‘May they get lost! May they disappear! We are better off without them.’ They then do the observance-day ceremony and recite the Monastic Code, aiming at schism. While they’re doing it, an equal number of resident monks arrive. 

\scrule{In such a case, what has been recited is valid, and the remainder should be listened to. And there’s a serious offense for the reciters. }

On\marginnote{32.2.9} the observance day, four or more resident monks may have gathered together in a certain monastery. They know there are other resident monks who haven’t arrived. They think, ‘May they get lost! May they disappear! We are better off without them.’ They then do the observance-day ceremony and recite the Monastic Code, aiming at schism. While they’re doing it, a smaller number of resident monks arrive. 

\scrule{In such a case, what has been recited is valid, and the remainder should be listened to. And there’s a serious offense for the reciters. }

On\marginnote{32.2.17} the observance day, four or more resident monks may have gathered together in a certain monastery. They know there are other resident monks who haven’t arrived. They think, ‘May they get lost! May they disappear! We are better off without them.’ They then do the observance-day ceremony and recite the Monastic Code, aiming at schism. When they’ve just finished, a greater number of resident monks arrive. 

\scrule{In such a case, those monks should recite the Monastic Code once more. And there’s a serious offense for the reciters. }

On\marginnote{32.2.25} the observance day, four or more resident monks may have gathered together in a certain monastery. They know there are other resident monks who haven’t arrived. They think, ‘May they get lost! May they disappear! We are better off without them.’ They then do the observance-day ceremony and recite the Monastic Code, aiming at schism. When they’ve just finished, an equal number of resident monks arrive. 

\scrule{In such a case, what has been recited is valid, and the late arrivals should announce their purity in the presence of the others. And there’s a serious offense for the reciters. }

On\marginnote{32.2.31} the observance day, four or more resident monks may have gathered together in a certain monastery. They know there are other resident monks who haven’t arrived. They think, ‘May they get lost! May they disappear! We are better off without them.’ They then do the observance-day ceremony and recite the Monastic Code, aiming at schism. When they’ve just finished, a smaller number of resident monks arrive. 

\scrule{In such a case, what has been recited is valid, and the late arrivals should announce their purity in the presence of the others. And there’s a serious offense for the reciters. }

On\marginnote{32.2.39} the observance day, four or more resident monks may have gathered together in a certain monastery. They know there are other resident monks who haven’t arrived. They think, ‘May they get lost! May they disappear! We are better off without them.’ They then do the observance-day ceremony and recite the Monastic Code, aiming at schism. When they’ve just finished, and none of the gathering has left, a greater number of resident monks arrive. 

\scrule{In such a case, those monks should recite the Monastic Code once more. And there’s a serious offense for the reciters. }

On\marginnote{32.2.47} the observance day, four or more resident monks may have gathered together in a certain monastery. They know there are other resident monks who haven’t arrived. They think, ‘May they get lost! May they disappear! We are better off without them.’ They then do the observance-day ceremony and recite the Monastic Code, aiming at schism. When they’ve just finished, and none of the gathering has left, an equal number of resident monks arrive. 

\scrule{In such a case, what has been recited is valid, and the late arrivals should announce their purity in the presence of the others. And there’s a serious offense for the reciters. }

On\marginnote{32.2.55} the observance day, four or more resident monks may have gathered together in a certain monastery. They know there are other resident monks who haven’t arrived. They think, ‘May they get lost! May they disappear! We are better off without them.’ They then do the observance-day ceremony and recite the Monastic Code, aiming at schism. When they’ve just finished, and none of the gathering has left, a smaller number of resident monks arrive. 

\scrule{In such a case, what has been recited is valid, and the late arrivals should announce their purity in the presence of the others. And there’s a serious offense for the reciters. }

On\marginnote{32.2.63} the observance day, four or more resident monks may have gathered together in a certain monastery. They know there are other resident monks who haven’t arrived. They think, ‘May they get lost! May they disappear! We are better off without them.’ They then do the observance-day ceremony and recite the Monastic Code, aiming at schism. When they’ve just finished, and only some members of the gathering have left, a greater number of resident monks arrive. 

\scrule{In such a case, those monks should recite the Monastic Code once more. And there’s a serious offense for the reciters. }

On\marginnote{32.2.71} the observance day, four or more resident monks may have gathered together in a certain monastery. They know there are other resident monks who haven’t arrived. They think, ‘May they get lost! May they disappear! We are better off without them.’ They then do the observance-day ceremony and recite the Monastic Code, aiming at schism. When they’ve just finished, and only some members of the gathering have left, an equal number of resident monks arrive. 

\scrule{In such a case, what has been recited is valid, and the late arrivals should announce their purity in the presence of the others. And there’s a serious offense for the reciters. }

On\marginnote{32.2.79} the observance day, four or more resident monks may have gathered together in a certain monastery. They know there are other resident monks who haven’t arrived. They think, ‘May they get lost! May they disappear! We are better off without them.’ They then do the observance-day ceremony and recite the Monastic Code, aiming at schism. When they’ve just finished, and only some members of the gathering have left, a smaller number of resident monks arrive. 

\scrule{In such a case, what has been recited is valid, and the late arrivals should announce their purity in the presence of the others. And there’s a serious offense for the reciters. }

On\marginnote{32.2.87} the observance day, four or more resident monks may have gathered together in a certain monastery. They know there are other resident monks who haven’t arrived. They think, ‘May they get lost! May they disappear! We are better off without them.’ They then do the observance-day ceremony and recite the Monastic Code, aiming at schism. When they’ve just finished, and the entire gathering has left, a greater number of resident monks arrive. 

\scrule{In such a case, those monks should recite the Monastic Code once more. And there’s a serious offense for the reciters. }

On\marginnote{32.2.95} the observance day, four or more resident monks may have gathered together in a certain monastery. They know there are other resident monks who haven’t arrived. They think, ‘May they get lost! May they disappear! We are better off without them.’ They then do the observance-day ceremony and recite the Monastic Code, aiming at schism. When they’ve just finished, and the entire gathering has left, an equal number of resident monks arrive. 

\scrule{In such a case, what has been recited is valid, and the late arrivals should announce their purity in the presence of the others. And there’s a serious offense for the reciters. }

On\marginnote{32.2.103} the observance day, four or more resident monks may have gathered together in a certain monastery. They know there are other resident monks who haven’t arrived. They think, ‘May they get lost! May they disappear! We are better off without them.’ They then do the observance-day ceremony and recite the Monastic Code, aiming at schism. When they’ve just finished, and the entire gathering has left, a smaller number of resident monks arrive. 

\scrule{In such a case, what has been recited is valid, and the late arrivals should announce their purity in the presence of the others. And there’s a serious offense for the reciters.” }

\scend{The group of fifteen on aiming at schism is finished. The group of seventy-five is finished. }

\section*{33. The successive series on entering a monastery zone }

“On\marginnote{33.1.1} the observance day, four or more resident monks may have gathered together in a certain monastery. They don’t know that other resident monks are entering the monastery zone. … They don’t know that other resident monks have entered the monastery zone. … They don’t see that other resident monks are entering the monastery zone. … They don’t see that other resident monks have entered the monastery zone. … They don’t hear that other resident monks are entering the monastery zone. … They don’t hear that other resident monks have entered the monastery zone. …” 

As\marginnote{33.1.12} there are one hundred and seventy-five sets of three for resident monks with resident monks, so there are for newly-arrived monks with resident monks, resident monks with newly-arrived monks, newly-arrived monks with newly-arrived monks. Thus by way of succession, there are seven hundred sets of three. 

“It\marginnote{34.1.1} may be, monks, that for the resident monks it’s the fourteenth day of the lunar half-month, but for the newly-arrived monks it’s the fifteenth. Then—

\scrule{If the number of resident monks is greater, the newly-arrived monks should fall in line with the resident monks. }

\scrule{If the number is the same, the newly-arrived monks should fall in line with the resident monks. }

\scrule{If the number of newly-arrived monks is greater, the resident monks should fall in line with the newly-arrived monks. }

It\marginnote{34.2.1} may be that for the resident monks it’s the fifteenth day of the lunar half-month, but for the newly-arrived monks it’s the fourteenth. Then—

\scrule{If the number of resident monks is greater, the newly-arrived monks should fall in line with the resident monks. }

\scrule{If the number is the same, the newly-arrived monks should fall in line with the resident monks. }

\scrule{If the number of newly-arrived monks is greater, the resident monks should fall in line with the newly-arrived monks. }

It\marginnote{34.3.1} may be that for the resident monks it’s the day after the observance day, but for the newly-arrived monks it’s the fifteenth day of the lunar half-month. Then—

\scrule{If the number of resident monks is greater, the resident monks may, if they’re willing, do the observance-day ceremony with the newly-arrived monks. Otherwise the newly-arrived monks should go outside the monastery zone and do the observance-day ceremony there. }

\scrule{If the number is the same, the resident monks may, if they’re willing, do the observance-day ceremony with the newly-arrived monks. Otherwise the newly-arrived monks should go outside the monastery zone and do the observance-day ceremony there. }

\scrule{If the number of newly-arrived monks is greater, the resident monks should do the observance-day ceremony with the newly-arrived monks, or they should go outside the monastery zone while the newly-arrived monks do the observance-day ceremony. }

It\marginnote{34.4.1} may be that for the resident monks it’s the fifteenth day of the lunar half-month, but for the newly-arrived monks it’s the day after the observance day. Then—

\scrule{If the number of resident monks is greater, the newly-arrived monks should do the observance-day ceremony with the resident monks, or they should go outside the monastery zone while the resident monks do the observance-day ceremony. }

\scrule{If the number is the same, the newly-arrived monks should do the observance-day ceremony with the resident monks, or they should go outside the monastery zone while the resident monks do the observance-day ceremony. }

\scrule{If the number of newly-arrived monks is greater, they may, if they’re willing, do the observance-day ceremony with the resident monks. Otherwise the resident monks should go outside the monastery zone and do the observance-day ceremony there.” }

\section*{34. The seeing of characteristics, etc. }

“It\marginnote{34.5.1} may happen that newly-arrived monks see signs and indications of resident monks: beds and benches that are made up, water for drinking and water for washing that are ready for use, yards that are well swept. As a consequence, they’re unsure whether or not there are resident monks there. Then—

\scrule{If they do the observance-day ceremony without investigating, there’s an offense of wrong conduct. }

\scrule{If they investigate, but don’t see anyone, and then do the observance-day ceremony, there’s no offense. }

\scrule{If they investigate, and they see someone, and then do the observance-day ceremony together, there’s no offense. }

\scrule{If they investigate, and they see someone, but then do the observance-day ceremony separately, there’s an offense of wrong conduct. }

\scrule{If they investigate, and they see someone, but think, ‘May they get lost! May they disappear! We are better off without them,’ and then do the observance-day ceremony aiming at schism, there’s a serious offense. }

It\marginnote{34.7.1} may happen that newly-arrived monks hear signs and indications of resident monks: the sound of the feet of someone doing walking meditation, the sound of recitation, the sound of coughing, the sound of sneezing. As a consequence, they’re unsure whether or not there are resident monks there. Then—

\scrule{If they do the observance-day ceremony without investigating, there’s an offense of wrong conduct. }

\scrule{If they investigate, but don’t see anyone, and then do the observance-day ceremony, there’s no offense. }

\scrule{If they investigate, and they see someone, and then do the observance-day ceremony together, there’s no offense. }

\scrule{If they investigate, and they see someone, but then do the observance-day ceremony separately, there’s an offense of wrong conduct. }

\scrule{If they investigate, and they see someone, but think, ‘May they get lost! May they disappear! We are better off without them,’ and then do the observance-day ceremony aiming at schism, there’s a serious offense. }

It\marginnote{34.8.1} may happen that resident monks see signs and indications of newly-arrived monks: an unknown almsbowl, an unknown robe, an unknown sitting mat, water poured on the ground from the washing of feet. As a consequence, they’re unsure whether or not monks have arrived. Then—

\scrule{If they do the observance-day ceremony without investigating, there’s an offense of wrong conduct. }

\scrule{If they investigate, but don’t see anyone, and then do the observance-day ceremony, there’s no offense. }

\scrule{If they investigate, and they see someone, and then do the observance-day ceremony together, there’s no offense. }

\scrule{If they investigate, and they see someone, but then do the observance-day ceremony separately, there’s an offense of wrong conduct. }

\scrule{If they investigate, and they see someone, but think, ‘May they get lost! May they disappear! We are better off without them,’ and then do the observance-day ceremony aiming at schism, there’s a serious offense. }

It\marginnote{34.9.1} may happen that resident monks hear signs and indications of newly-arrived monks: the sound of the feet of someone arriving, the sound of sandals being knocked together, the sound of coughing, the sound of sneezing. As a consequence, they’re unsure whether or not monks have arrived. Then—

\scrule{If they do the observance-day ceremony without investigating, there’s an offense of wrong conduct. }

\scrule{If they investigate, but don’t see anyone, and then do the observance-day ceremony, there’s no offense. }

\scrule{If they investigate, and they see someone, and then do the observance-day ceremony together, there’s no offense. }

\scrule{If they investigate, and they see someone, but then do the observance-day ceremony separately, there’s an offense of wrong conduct. }

\scrule{If they investigate, and they see someone, but think, ‘May they get lost! May they disappear! We are better off without them,’ and then do the observance-day ceremony aiming at schism, there’s a serious offense.” }

\section*{35. The doing of the observance-day ceremony with those belonging to a different Buddhist sect, etc. }

“It\marginnote{34.10.1} may happen that newly-arrived monks see resident monks who belong to a different Buddhist sect,\footnote{\textit{\textsanskrit{Nānāsaṁvāsaka}} (and \textit{\textsanskrit{samānasaṁvāsaka}}) need to be carefully distinguished from \textit{\textsanskrit{nānāsaṁvāsa}} (and \textit{\textsanskrit{samānasaṁvāsa}}). Only the former means “one belonging to a different Buddhist sect”. The latter means “belonging to a different community”, as decided by \textit{\textsanskrit{sīmās}}. } but they have the view that they belong to the same one. Then—

\scrule{If they don’t ask the resident monks about it, and then do the observance-day ceremony together, there’s no offense. }

\scrule{If they do ask the resident monks about it, but don’t reach a clear conclusion, and then do the observance-day ceremony together, there’s an offense of wrong conduct. }

\scrule{If they do ask the resident monks about it, but don’t reach a clear conclusion, and then do the observance-day ceremony separately, there’s no offense. }

It\marginnote{34.11.1} may happen that newly-arrived monks see resident monks who belong to the same Buddhist sect, but they have the view that they belong to a different one. Then—

\scrule{If they don’t ask the resident monks about it, and then do the observance-day ceremony together, there’s an offense of wrong conduct. }

\scrule{If they do ask the resident monks about it, and they change their view, but then do the observance-day ceremony separately, there’s an offense of wrong conduct. }

\scrule{If they do ask the resident monks about it, and they change their view, and then do the observance-day ceremony together, there’s no offense. }

It\marginnote{34.12.1} may happen that resident monks see newly-arrived monks who belong to a different Buddhist sect, but they have the view that they belong to the same one. Then—

\scrule{If they don’t ask the newly-arrived monks about it, and then do the observance-day ceremony together, there’s no offense. }

\scrule{If they do ask the newly-arrived monks about it, but don’t reach a clear conclusion, and then do the observance-day ceremony together, there’s an offense of wrong conduct. }

\scrule{If they do ask the newly-arrived monks about it, but don’t reach a clear conclusion, and then do the observance-day ceremony separately, there’s no offense. }

It\marginnote{34.13.1} may happen that resident monks see newly-arrived monks who belong to the same Buddhist sect, but they have the view that they belong to a different one. Then—

\scrule{If they don’t ask the newly-arrived monks about it, and then do the observance-day ceremony together, there’s an offense of wrong conduct. }

\scrule{If they do ask the newly-arrived monks about it, and they change their view, but then do the observance-day ceremony separately, there’s an offense of wrong conduct. }

\scrule{If they do ask the newly-arrived monks about it, and they change their view, and then do the observance-day ceremony together, there’s no offense.” }

\section*{36. The section on “you shouldn’t go” }

“On\marginnote{35.1.1} the observance day you shouldn’t go from a monastery with monks to a monastery without monks, except if you go with a sangha or there are dangers. On the observance day you shouldn’t go from a monastery with monks to a non-monastery without monks, except if you go with a sangha or there are dangers.\footnote{Here and below I understand a monastery, an \textit{\textsanskrit{āvāsa}}, to refer to a monastery with a properly defined zone, a \textit{\textsanskrit{sīmā}}. A non-monastery, an \textit{\textsanskrit{anāvāsa}}, is then a monastic residence without such a zone. } On the observance day you shouldn’t go from a monastery with monks to a monastery or a non-monastery without monks, except if you go with a sangha or there are dangers. 

On\marginnote{35.2.1} the observance day you shouldn’t go from a non-monastery with monks to a monastery without monks, except if you go with a sangha or there are dangers. On the observance day you shouldn’t go from a non-monastery with monks to a non-monastery without monks, except if you go with a sangha or there are dangers. On the observance day you shouldn’t go from a non-monastery with monks to a monastery or a non-monastery without monks, except if you go with a sangha or there are dangers. 

On\marginnote{35.3.1} the observance day you shouldn’t go from a monastery or a non-monastery with monks to a monastery without monks, except if you go with a sangha or there are dangers. On the observance day you shouldn’t go from a monastery or a non-monastery with monks to a non-monastery without monks, except if you go with a sangha or there are dangers. On the observance day you shouldn’t go from a monastery or a non-monastery with monks to a monastery or a non-monastery without monks, except if you go with a sangha or there are dangers. 

On\marginnote{35.4.1} the observance day you shouldn’t go from a monastery with monks to a monastery with monks who belong to a different Buddhist sect, except if you go with a sangha or there are dangers. On the observance day you shouldn’t go from a monastery with monks to a non-monastery with monks who belong to a different Buddhist sect, except if you go with a sangha or there are dangers. On the observance day you shouldn’t go from a monastery with monks to a monastery or a non-monastery with monks who belong to a different Buddhist sect, except if you go with a sangha or there are dangers. 

On\marginnote{35.4.4} the observance day you shouldn’t go from a non-monastery with monks to a monastery with monks who belong to a different Buddhist sect, except if you go with a sangha or there are dangers. On the observance day you shouldn’t go from a non-monastery with monks to a non-monastery with monks who belong to a different Buddhist sect, except if you go with a sangha or there are dangers. On the observance day you shouldn’t go from a non-monastery with monks to a monastery or a non-monastery with monks who belong to a different Buddhist sect, except if you go with a sangha or there are dangers. 

On\marginnote{35.4.7} the observance day you shouldn’t go from a monastery or a non-monastery with monks to a monastery with monks who belong to a different Buddhist sect, except if you go with a sangha or there are dangers. On the observance day you shouldn’t go from a monastery or a non-monastery with monks to a non-monastery with monks who belong to a different Buddhist sect, except if you go with a sangha or there are dangers. On the observance day you shouldn’t go from a monastery or a non-monastery with monks to a monastery or a non-monastery with monks who belong to a different Buddhist sect, except if you go with a sangha or there are dangers.” 

\section*{37. The section on “you may go” }

“On\marginnote{35.5.1} the observance day you may go from a monastery with monks to a monastery with monks who belong to the same Buddhist sect if you know you’ll get there on the same day. On the observance day you may go from a monastery with monks to a non-monastery with monks … to a monastery or a non-monastery with monks who belong to the same Buddhist sect if you know you’ll get there on the same day. 

On\marginnote{35.5.6} the observance day you may go from a non-monastery with monks to a monastery with monks … to a non-monastery with monks … to a monastery or a non-monastery with monks who belong to the same Buddhist sect if you know you’ll get there on the same day. 

On\marginnote{35.5.10} the observance day you may go from a monastery or a non-monastery with monks to a monastery with monks … to a non-monastery with monks … to a monastery or a non-monastery with monks who belong to the same Buddhist sect if you know you’ll get there on the same day.” 

\section*{38. The identification of persons to be avoided }

\scrule{“You shouldn’t recite the Monastic Code with a nun seated in the gathering. If you do, you commit an offense of wrong conduct. You shouldn’t recite the Monastic Code with a trainee nun, a novice monk, a novice nun, one who has renounced the training, or one who has committed the worst kind of offense seated in the gathering. If you do, you commit an offense of wrong conduct. }

\scrule{You shouldn’t recite the Monastic Code with one who has been ejected for not recognizing an offense seated in the gathering. If you do, you should be dealt with according to the rule. You shouldn’t recite the Monastic Code with one who has been ejected for not making amends for an offense seated in the gathering or with one who has been ejected for not giving up a bad view seated in the gathering. If you do, you should be dealt with according to the rule. }

\scrule{You shouldn’t recite the Monastic Code with a \textit{\textsanskrit{paṇḍaka}} seated in the gathering. If you do, you commit an offense of wrong conduct. You shouldn’t recite the Monastic Code with a fake monk, with one who has previously left to join the monastics of another religion, with an animal, with a matricide, with a patricide, with a murderer of a perfected one, with one who has raped a nun, with one has caused a schism in the Sangha, with one who has caused the Buddha to bleed, or with a hermaphrodite seated in the gathering. If you do, you commit an offense of wrong conduct. }

\scrule{You shouldn’t do the observance-day ceremony with a passed-on purity that has expired, except if the gathering is still seated together.\footnote{“A passed-on purity that has expired”, \textit{\textsanskrit{pārivāsikapārisuddhidānena}}, seems to refer to purity that was conveyed for a different occasion. So long as the assembly is still seated, the occasion is regarded as the same. See the discussion to Bi Pc 81 in Appendix on Individual \textsanskrit{Bhikkhunī} Rules in volume 3. } }

\scrule{You shouldn’t do the observance-day ceremony on a non-observance day, except to unify the Sangha.” }

\scend{The third section for recitation is finished. }

\scendsutta{The second chapter on the observance day is finished. }

\scuddanaintro{This is the summary: }

\begin{scuddana}%
“Ascetics\marginnote{36.4.6} of other religions, and \textsanskrit{Bimbisāra}, \\
To assemble, silent; \\
Teaching, seclusion, the Monastic Code, \\
Daily, then once. 

Separately,\marginnote{36.4.10} complete assembly, \\
Complete assembly, and Maddakucchi; \\
Monastery zone, large, with river, \\
One after another, two, and small. 

Juniors,\marginnote{36.4.14} and just in \textsanskrit{Rājagaha}, \\
May-stay-apart zone; \\
Should establish the monastery zone first, \\
Should abolish the monastery zone afterwards. 

Non-established\marginnote{36.4.18} zones of inhabited areas, \\
In a river, in the ocean, in a lake; \\
A splash of water, they made overlap, \\
And just so they enclosed. 

How\marginnote{36.4.22} many procedures, recitation, \\
Primitive tribes, and even when there were none; \\
A teaching, Monastic Law, they made threats, \\
Again Monastic Law, and threatening. 

Accusing,\marginnote{36.4.26} when permission is given, \\
Objecting to what is illegitimate; \\
Four or five, and others state, \\
Also if deliberately, one should make an effort. 

Included\marginnote{36.4.30} lay people, without being asked, \\
He did not know at \textsanskrit{Codanā}; \\
A number did not know, \\
And straightaway, would not go. 

Which,\marginnote{36.4.34} how many, faraway, \\
And to announce, he forgot; \\
Dirty, seat, lamp, \\
Regions, another who is learned. 

Straightaway,\marginnote{36.4.38} observance day in the rainy season, \\
Purity, and procedure, relatives; \\
Gagga, four, three, two, one, \\
Offense, shared, he remembered. 

The\marginnote{36.4.42} whole Sangha, unsure, \\
They did not know, one who is learned; \\
Greater, equal, smaller, \\
And none of the gathering has left. 

Some\marginnote{36.4.46} have left, entire, \\
And they know, unsure; \\
Anxious thinking, ‘It’s allowable’, \\
Knowing, seeing, and they hear. 

With\marginnote{36.4.50} resident, newly arrived, \\
The fourteenth and the fifteenth, again; \\
The day after, the fifteenth, \\
Characteristics, belonging to a Buddhist sect, hermaphrodite. 

That\marginnote{36.4.54} has expired, non-observance day \\
Except to unify the Sangha; \\
These summaries are detailed, \\
Making the topics clear.” 

%
\end{scuddana}

\scend{In this chapter there are eighty-six topics. }

\scendsutta{The chapter on the observance day is finished. }

%
\chapter*{{\suttatitleacronym Kd 3}{\suttatitletranslation The chapter on entering the rainy-season residence }{\suttatitleroot Vassūpanāyikakkhandhaka}}
\addcontentsline{toc}{chapter}{\tocacronym{Kd 3} \toctranslation{The chapter on entering the rainy-season residence } \tocroot{Vassūpanāyikakkhandhaka}}
\markboth{The chapter on entering the rainy-season residence }{Vassūpanāyikakkhandhaka}
\extramarks{Kd 3}{Kd 3}

\section*{1. The instruction to enter the rainy-season residence }

At\marginnote{1.1.1} one time the Buddha was staying at \textsanskrit{Rājagaha} in the Bamboo Grove, the squirrel sanctuary. At that time the Buddha had not yet laid down the rainy-season residence for the monks. And so the monks were wandering about in the winter, in the summer, and also during the rainy season. People complained and criticized them, “How can the Sakyan monastics go wandering in the winter, in the summer, and even during the rainy season? They’re trampling down the green grass, oppressing one-sensed life, and destroying many small creatures. Even the monastics of other religions, with their flawed teachings, settle down for the rainy season. Even birds make a nest in the top of a tree and settle down for the rainy-season. But not so the Sakyan monastics.” 

The\marginnote{1.3.1} monks heard the complaints of those people and told the Buddha. Soon afterwards he gave a teaching and addressed the monks: 

\scrule{“You should enter the rainy-season residence.”\footnote{\textit{\textsanskrit{Vassaṁ} \textsanskrit{upagantuṁ}}, literally, “to enter the rainy season”, but the idea of staying put in one place is implied. In these cases \textit{vassa}, “rainy season”, is used synonymously with \textit{\textsanskrit{vassāvāsa}}, “rainy-season residence” or “rains residence”. } }

The\marginnote{2.1.1} monks thought, “When should we enter the rains residence?” They told the Buddha. 

\scrule{“You should enter the rainy-season residence during the rainy season.” }

The\marginnote{2.2.1} monks thought, “How many entries to the rains residence are there?” 

\scrule{“There are two entries to the rainy-season residence: the first and the second. The first should be entered on the day after the full moon of July and the second one month after the same full moon.”\footnote{Although the match is not perfect, I have here translated \textsanskrit{Āsāḷha} as July. } }

\section*{2. The prohibition against wandering during the rainy season, etc. }

Soon\marginnote{3.1.1} afterwards the monks from the group of six entered the rains residence and then went wandering during the rainy season. People complained and criticized them just as they had before. 

The\marginnote{3.2.1} monks heard the complaints of those people and the monks of few desires complained and criticized them, “How could the monks from the group of six enter the rains residence and then go wandering during the rainy season?” And they told the Buddha. Soon afterwards he gave a teaching and addressed the monks: 

\scrule{“After entering the rainy-season residence, you should stay put for the first or the second three-month period before you go wandering. If you go wandering during the rainy-season residence period, you commit an offense of wrong conduct.” }

The\marginnote{4.1.1} monks from the group of six did not want to enter the rains residence. 

\scrule{“You should enter the rainy-season residence. If you don’t, you commit an offense of wrong conduct.” }

On\marginnote{4.2.1} the day of the entry to the rains residence, the monks from the group of six deliberately bypassed a monastery because they did not want to enter the rains residence. 

\scrule{“On the day of the entry to the rainy-season residence, you shouldn’t deliberately bypass a monastery because you don’t want to enter the rainy-season residence. If you do, you commit an offense of wrong conduct.” }

At\marginnote{4.3.1} one time King Seniya \textsanskrit{Bimbisāra} of Magadha wanted to postpone the rains residence. He sent a message to the monks: “Would the venerables please enter the rains residence during the next waxing phase of the moon?” They told the Buddha. 

\scrule{“You should comply with the wishes of kings.”\footnote{\textit{\textsanskrit{Rājūnaṁ} \textsanskrit{anuvattituṁ}}, literally, “(You should) behave according to the kings.” This is often understood to mean that monastics are obliged to follow the laws of the land in which they live. } }

\section*{3. The allowance for seven-day business }

When\marginnote{5.1.1} the Buddha had stayed at \textsanskrit{Rājagaha} for as long as he liked, he set out wandering toward \textsanskrit{Sāvatthī}. When he eventually arrived, he stayed in the Jeta Grove, \textsanskrit{Anāthapiṇḍika}’s Monastery. 

At\marginnote{5.1.4} that time the lay follower Udena had had a dwelling built for the Sangha in the Kosalan country. He sent a message to the monks: “Please come, venerables, I wish to make an offering, hear the Teaching, and see the monks.” 

The\marginnote{5.2.1} monks replied, “The Buddha has laid down a rule that a monk who’s entered the rains residence shouldn’t go wandering until after the rains. Please wait, Udena. Once we’ve completed the rains residence, we’ll come. But if the matter is urgent, then give the dwelling in the presence of the local monks.”\footnote{The Pali word translated here as “give” is \textit{\textsanskrit{patiṭṭhāpetu}}, which normally means “establish”. In the present context I understand it as “establishing a gift”, in the sense that the gift is meant for the Sangha but given in the presence of the local monks. The expression \textit{\textsanskrit{dakkhiṇaṁ} \textsanskrit{patiṭṭhāpeti}}, “to establish a gift”, is quite common in the Suttas, see \href{https://suttacentral.net/sn3.19/en/brahmali\#3.2}{SN 3.19:3.2}, \href{https://suttacentral.net/an4.61/en/brahmali\#16.1}{AN 4.61:16.1}, \href{https://suttacentral.net/an5.41/en/brahmali\#5.1}{AN 5.41:5.1}, \href{https://suttacentral.net/an5.227/en/brahmali\#2.3}{AN 5.227:2.3}, and \href{https://suttacentral.net/an6.37/en/brahmali\#1.2}{AN 6.37:1.2}. } 

Udena\marginnote{5.3.1} complained and criticized them, “How can the venerables not come when I’ve sent them a message? I’m a donor and I provide services. I’m a supporter of the Sangha!” 

The\marginnote{5.3.4} monks heard his complaints and they told the Buddha. Soon afterwards he gave a teaching and addressed the monks: 

\scrule{“If any of seven kinds of persons—a monk, a nun, a trainee nun, a novice monk, a novice nun, a male lay follower, or a female lay follower—asks you to come, I allow you to go for seven days, but only if you’re asked. And you should return within seven days.” }

\subsection*{Male lay followers}

“It\marginnote{5.5.1} may happen, monks, that a male lay follower has had a dwelling built for the Sangha and sends a message to the monks: ‘Please come, venerables, I wish to make an offering, hear the Teaching, and see the monks.’ You should go for seven days, but only if you’re asked. And you should return within seven days. 

It\marginnote{5.6.1} may happen that a male lay follower has had a stilt house built for the Sangha,\footnote{As elsewhere, I have rendered \textit{\textsanskrit{aḍḍhayoga}}, \textit{\textsanskrit{pāsāda}}, and \textit{hammiya} together as “stilt house”. According to the commentaries, the \textit{\textsanskrit{aḍḍhayoga}}, the \textit{\textsanskrit{pāsāda}}, and the \textit{hammiya}, are all different kinds of \textit{\textsanskrit{pāsāda}}, “stilt houses”. Rather than try to differentiate between these buildings, which is unlikely to be useful from a practical perspective, I have instead grouped them together as “stilt house”. Here is what the commentaries have to say. Sp 4.294: \textit{\textsanskrit{Aḍḍhayogoti} \textsanskrit{supaṇṇavaṅkagehaṁ}}, “An \textit{\textsanskrit{aḍḍhayoga}} is a house bent like a \textit{\textsanskrit{supaṇṇa}}.” Sp-\textsanskrit{ṭ} 4.294 clarifies: \textit{\textsanskrit{Supaṇṇavaṅkagehanti} \textsanskrit{garuḷapakkhasaṇṭhānena} \textsanskrit{katagehaṁ}}, “\textit{\textsanskrit{Supaṇṇavaṅkageha}}: a house made in the shape of the wings of a \textit{\textsanskrit{garuḷa}}.” A \textit{\textsanskrit{garuḷa}}, better known in its Sanskrit form \textit{\textsanskrit{garuḍa}}, is a mythological bird. Sp 4.294 continues: \textit{\textsanskrit{Pāsādoti} \textsanskrit{dīghapāsādo}. Hammiyanti \textsanskrit{upariākāsatale} \textsanskrit{patiṭṭhitakūṭāgāro} \textsanskrit{pāsādoyeva}}, “A \textit{\textsanskrit{pāsāda}} is a long stilt house. A \textit{hammiya} is just a \textit{\textsanskrit{pāsāda}} that has an upper room on top of its flat roof.” At Sp-\textsanskrit{ṭ} 3.74, however, we find slightly different explanations. It seems clear, however, that all three are stilt houses and that they are distinguished according to their shape and the kind of roof they possess. See also \textit{\textsanskrit{pāsāda}}, “stilt house”, in Appendix of Technical Terms. } has had a cave built,\footnote{For an explanation of rendering \textit{\textsanskrit{guhā}} as “cave”, see Appendix of Technical Terms. } a yard,\footnote{“Yard” renders \textit{\textsanskrit{pariveṇa}}. See Appendix of Technical Terms. } a gatehouse,\footnote{“Gatehouse” renders \textit{\textsanskrit{koṭṭhaka}}. See Appendix of Technical Terms. } an assembly hall, a water-boiling shed,\footnote{For an explanation of the rendering \textit{\textsanskrit{aggisālā}} as “water-boiling shed”, see Appendix of Technical Terms. } a food-storage hut, a restroom, a walking-meditation path, an indoor walking-meditation path, a well, a well house, a sauna,\footnote{For an explanation of rendering \textit{\textsanskrit{jantāghara}} as “sauna”, see Appendix of Technical Terms. } a sauna shed, a pond, a roof cover, a monastery, or has had a site for a monastery prepared for the Sangha, and sends a message to the monks: ‘Please come, venerables, I wish to make an offering, hear the Teaching, and see the monks.’ You should go for seven days, but only if you’re asked. And you should return within seven days. 

It\marginnote{5.7.1} may happen that a male lay follower has had a dwelling built for a number of monks … has had a dwelling built for a single monk, has had a stilt house built, a cave, a yard, a gatehouse, an assembly hall, a water-boiling shed, a food-storage hut, a restroom, a walking-meditation path, an indoor walking-meditation path, a well, a well house, a sauna, a sauna shed, a pond, a roof cover, a monastery, or has had a site for a monastery prepared, and sends a message to the monks: ‘Please come, venerables, I wish to make an offering, hear the Teaching, and see the monks.’ You should go for seven days, but only if you’re asked. And you should return within seven days. 

It\marginnote{5.8.1} may happen that a male lay follower has had a dwelling built for the Sangha of nuns, for a number of nuns, for a single nun, for a number of trainee nuns, for a single trainee nun, for a number of novice monks, for a single novice monk, for a number of novice nuns, or has had a dwelling built for a single novice nun … or has had a stilt house built, a cave, a yard, a gatehouse, an assembly hall, a water-boiling shed, a food-storage hut, a walking-meditation path, an indoor walking-meditation path, a well, a well house, a pond, a roof-cover, a monastery, or has had a site for a monastery prepared, and sends a message to the monks: ‘Please come, venerables, I wish to make an offering, hear the Teaching, and see the monks.’ You should go for seven days, but only if you’re asked. And you should return within seven days. 

It\marginnote{5.9.1} may happen that a male lay follower has had a house built for himself, has had a bedroom, a storehouse, a watchtower, a stall, a shop, a stilt house,\footnote{According to the commentaries the \textit{\textsanskrit{māḷa}}, the \textit{\textsanskrit{pāsāda}}, and the \textit{hammiya} are all different kinds of stilt houses. It is hard to make a meaningful distinction between these buildings from a modern perspective, and I have therefore grouped them together into the single category of “stilt house”. } a cave, a yard, a gatehouse, an assembly hall, a water-boiling shed, a kitchen, a walking-meditation path, an indoor walking-meditation path, a well, a well house, a pond, a roof cover, a park, or has had a site for a park prepared for himself; or his son is getting married, or his daughter is getting married, or he is sick, or he knows a discourse. If he then sends a message to the monks: ‘Please come, venerables, and learn this discourse before it disappears,’ or he has some duty or business and sends a message to the monks: ‘Please come, venerables, I wish to make an offering, hear the Teaching, and see the monks,’ you should go for seven days, but only if you’re asked. And you should return within seven days.” 

\subsection*{Improper cancellation of the invitation}

“It\marginnote{5.10.1} may happen that a female lay follower has had a dwelling built for the Sangha and sends a message to the monks: ‘Please come, venerables, I wish to make an offering, hear the Teaching, and see the monks.’ You should go for seven days, but only if you’re asked. And you should return within seven days. 

It\marginnote{5.11.1} may happen that a female lay follower has had a stilt house built for the Sangha, has had a cave built, a yard, a gatehouse, an assembly hall, a water-boiling shed, a food-storage hut, a restroom, a walking-meditation path, an indoor walking-meditation path, a well, a well house, a sauna, a sauna shed, a pond, a roof cover, a monastery, or has had a site for a monastery prepared for the Sangha and sends a message to the monks: ‘Please come, venerables, I wish to make an offering, hear the Teaching, and see the monks.’ You should go for seven days, but only if you’re asked. And you should return within seven days. 

It\marginnote{5.12.1} may happen that a female lay follower has had a dwelling built for a number of monks, for a single monk, for the Sangha of nuns, for a number of nuns, for a single nun, for a number of trainee nuns, for a single trainee nun, for a number of novice monks, for a single novice monk, for a number of novice nuns, or for a single novice nun … 

It\marginnote{5.12.12} may happen that a female lay follower has had a house built for herself, has had a bedroom, a storehouse, a watchtower, a stall, a shop, a stilt house, a cave, a yard, a gatehouse, an assembly hall, a water-boiling shed, a kitchen, a walking-meditation path, an indoor walking-meditation path, a well, a well house, a pond, a roof cover, a park, or has had a site for a park prepared for herself; or her son is getting married, or her daughter is getting married, or she is sick, or she knows a discourse. If she then sends a message to the monks: ‘Please come, venerables, and learn this discourse before it disappears,’ or she has some duty or business and sends a message to the monks: ‘Please come, venerables, I wish to make an offering, hear the Teaching, and see the monks,’ you should go for seven days, but only if you’re asked. And you should return within seven days. 

It\marginnote{5.13.1} may happen that a monk, a nun, a trainee nun, a novice monk, or a novice nun has had a dwelling built for the Sangha … for a number of monks, for a single monk, for the Sangha of nuns, for a number of nuns, for a single nun, for a number of trainee nuns, for a single trainee nun, for a number of novice monks, for a single novice monk, for a number of novice nuns, or for a single novice nun … or she’s had a dwelling built for herself, has had a stilt house built, a cave, a yard, a gatehouse, an assembly hall, a water-boiling shed, a food-storage hut, a walking-meditation path, an indoor walking-meditation path, a well, a well house, a pond, a roof-cover, a monastery, or has had a site for a monastery prepared for herself. If she then sends a message to the monks: ‘Please come, venerables, I wish to make an offering, hear the Teaching, and see the monks,’ you should go for seven days, but only if you’re asked. And you should return within seven days.” 

\section*{4. The allowance to go to any of five kinds of persons even if not asked }

On\marginnote{6.1.1} one occasion a certain monk was sick. He sent a message to the monks: “Please come, venerables, I’m sick.” They told the Buddha. 

\scrule{“Even if you’re not asked, let alone if you are, I allow you to go for seven days to any of five kinds of persons—a monk, a nun, a trainee nun, a novice monk, or a novice nun. But you should return within seven days.” }

\paragraph*{A monk sending a message }

“It\marginnote{6.2.1} may be that a sick monk sends a message to the monks: ‘Please come, venerables, I’m sick.’ Then, even if you’re not asked, let alone if you are, you should go for seven days, thinking, ‘I’ll look for food for the sick,’ ‘I’ll look for food for the nurses,’ ‘I’ll look for medicine,’ ‘I’ll enquire about his sickness,’ or ‘I’ll nurse him.’\footnote{Although the Pali expresses these alternatives as if they were a single thought, I take them to be individual reasons for taking the seven-day allowance. This is a common way throughout the Vinaya \textsanskrit{Piṭaka} of expressing such alternatives. } But you should return within seven days. 

It\marginnote{6.3.1} may be that a monk who is discontent with the spiritual life sends a message to the monks: ‘Please come, venerables, I’m discontent with the spiritual life.’ Then, even if you’re not asked, let alone if you are, you should go for seven days, thinking, ‘I’ll allay his discontent,’ ‘I’ll find someone to allay his discontent’, or ‘I’ll give him a teaching.’ But you should return within seven days. 

It\marginnote{6.4.1} may be that an anxious monk sends a message to the monks: ‘Please come, venerables, I’m anxious.’ Then, even if you’re not asked, let alone if you are, you should go for seven days, thinking, ‘I’ll dispel his anxiety,’ ‘I’ll find someone to dispel his anxiety,’ or ‘I’ll give him a teaching.’ But you should return within seven days. 

It\marginnote{6.5.1} may be that a monk who has wrong view sends a message to the monks: ‘Please come, venerables, I have wrong view.’ Then, even if you’re not asked, let alone if you are, you should go for seven days, thinking, ‘I’ll make him give up that wrong view,’ ‘I’ll get someone to make him give up that wrong view,’ or ‘I’ll give him a teaching.’ But you should return within seven days. 

It\marginnote{6.6.1} may be that a monk who has committed a heavy offense for which he deserves to be given probation sends a message to the monks: ‘Please come, venerables, I’ve committed a heavy offense for which I deserve to be given probation.’ Then, even if you’re not asked, let alone if you are, you should go for seven days, thinking, ‘I’ll make an effort to get him given probation,’ ‘I’ll do the proclamation,’ or ‘I’ll complete the quorum.’ But you should return within seven days. 

It\marginnote{6.7.1} may be that a monk who deserves to be sent back to the beginning sends a message to the monks: ‘Please come, venerables, I deserve to be sent back to the beginning.’ Then, even if you’re not asked, let alone if you are, you should go for seven days, thinking, ‘I’ll make an effort to get him sent back to the beginning,’ ‘I’ll do the proclamation,’ or ‘I’ll complete the quorum.’ But you should return within seven days. 

It\marginnote{6.8.1} may be that a monk who deserves the trial period sends a message to the monks: ‘Please come, venerables, I deserve to be given the trial period.’ Then, even if you’re not asked, let alone if you are, you should go for seven days, thinking, ‘I’ll make an effort to get him given the trial period,’ ‘I’ll do the proclamation,’ or ‘I’ll complete the quorum.’ But you should return within seven days. 

It\marginnote{6.9.1} may be that a monk who deserves rehabilitation sends a message to the monks: ‘Please come, venerables, I deserve rehabilitation.’ Then, even if you’re not asked, let alone if you are, you should go for seven days, thinking, ‘I’ll make an effort to get him rehabilitated,’ ‘I’ll do the proclamation,’ or ‘I’ll complete the quorum.’ But you should return within seven days. 

It\marginnote{6.10.1} may be that the Sangha wants to do a legal procedure against a monk—whether a procedure of condemnation, demotion, banishment, reconciliation, or ejection.\footnote{“Demotion” renders \textit{niyassa}. See Appendix of Technical Terms. } He sends a message to the monks: ‘Please come, venerables, the Sangha wants to do a legal procedure against me.’ Then, even if you’re not asked, let alone if you are, you should go for seven days, thinking, ‘How may the Sangha not do the procedure?’ or ‘How may the Sangha make it lighter?’ But you should return within seven days. 

Or\marginnote{6.11.1} it may be that the Sangha has done a legal procedure against him—whether a procedure of condemnation, demotion, banishment, reconciliation, or ejection. He sends a message to the monks: ‘Please come, venerables, the Sangha has done a legal procedure against me.’ Then, even if you’re not asked, let alone if you are, you should go for seven days, thinking, ‘How can I help him behave properly and suitably so as to deserve to be released?’ or ‘What can I do so that the Sangha lifts that procedure?’\footnote{The meaning of the first of these phrases, \textit{\textsanskrit{sammā} vattati}, is straightforward, but the last two, \textit{\textsanskrit{lomaṁ} \textsanskrit{pāteti}} and \textit{\textsanskrit{netthāraṁ} vattati}, are more difficult. Commenting on Bu Ss 13, Sp 1.435 explains: \textit{Na \textsanskrit{lomaṁ} \textsanskrit{pātentīti} \textsanskrit{anulomapaṭipadaṁ} \textsanskrit{appaṭipajjanatāya} na \textsanskrit{pannalomā} honti. Na \textsanskrit{netthāraṁ} \textsanskrit{vattantīti} attano \textsanskrit{nittharaṇamaggaṁ} na \textsanskrit{paṭipajjanti}}, “\textit{Na \textsanskrit{lomaṁ} \textsanskrit{pātenti}}: because of their non-practicing in conformity with the path, their bodily hairs are not flat. \textit{Na \textsanskrit{netthāraṁ} vattanti}: they are not practicing the path for their own getting out (of the offense).” My rendering attempts to capture the meaning in a non-literal way. } But you should return within seven days.” 

\paragraph*{A nun sending a message }

“It\marginnote{6.12.1} may be, monks, that a sick nun sends a message to the monks: ‘Please come, venerables, I’m sick.’ Then, even if you’re not asked, let alone if you are, you should go for seven days, thinking, ‘I’ll look for food for the sick,’ ‘I’ll look for food for the nurses,’ ‘I’ll look for medicine,’ ‘I’ll enquire about her sickness,’ or ‘I’ll nurse her.’ But you should return within seven days. 

It\marginnote{6.13.1} may be that a nun who is discontent with the spiritual life sends a message to the monks: ‘Please come, venerables, I’m discontent with the spiritual life.’ Then, even if you’re not asked, let alone if you are, you should go for seven days, thinking, ‘I’ll allay her discontent,’ ‘I’ll find someone to allay her discontent’, or ‘I’ll give her a teaching.’ But you should return within seven days. 

It\marginnote{6.14.1} may be that an anxious nun sends a message to the monks: ‘Please come, venerables, I’m anxious.’ Then, even if you’re not asked, let alone if you are, you should go for seven days, thinking, ‘I’ll dispel her anxiety,’ ‘I’ll find someone to dispel her anxiety,’ or ‘I’ll give her a teaching.’ But you should return within seven days. 

It\marginnote{6.15.1} may be that a nun who has wrong view sends a message to the monks: ‘Please come, venerables, I have wrong view.’ Then, even if you’re not asked, let alone if you are, you should go for seven days, thinking, ‘I’ll make her give up that wrong view,’ ‘I’ll get someone to make her give up that wrong view,’ or ‘I’ll give her a teaching.’ But you should return within seven days. 

It\marginnote{6.16.1} may be that a nun who has committed a heavy offense for which she deserves the trial period sends a message to the monks: ‘Please come, venerables, I deserve to be given the trial period.’ Then, even if you’re not asked, let alone if you are, you should go for seven days, thinking, ‘I’ll make an effort to get her given the trial period.’ But you should return within seven days. 

It\marginnote{6.17.1} may be that a nun who deserves to be sent back to the beginning sends a message to the monks: ‘Please come, venerables, I deserve to be sent back to the beginning.’ Then, even if you’re not asked, let alone if you are, you should go for seven days, thinking, ‘I’ll make an effort to get her sent back to the beginning.’ But you should return within seven days. 

It\marginnote{6.18.1} may be that a nun who deserves rehabilitation sends a message to the monks: ‘Please come, venerables, I deserve rehabilitation.’ Then, even if you’re not asked, let alone if you are, you should go for seven days, thinking, ‘I’ll make an effort to get her rehabilitated.’ But you should return within seven days. 

It\marginnote{6.19.1} may be that the Sangha wants to do a legal procedure against a nun—whether a procedure of condemnation, demotion, banishment, reconciliation, or ejection. She sends a message to the monks: ‘Please come, venerables, the Sangha wants to do a legal procedure against me.’ Then, even if you’re not asked, let alone if you are, you should go for seven days, thinking, ‘How may the Sangha not do the procedure?’ or ‘How may the Sangha make it lighter?’ But you should return within seven days. 

Or\marginnote{6.20.1} it may be that the Sangha has done a legal procedure against her—whether a procedure of condemnation, demotion, banishment, reconciliation, or ejection. She sends a message to the monks: ‘Please come, venerables, the Sangha has done a legal procedure against me.’ Then, even if you’re not asked, let alone if you are, you should go for seven days, thinking, ‘How can I help her behave properly and suitably so as to deserve to be released?’ or ‘What can I do so that the Sangha lifts that procedure?’ But you should return within seven days.” 

\paragraph*{Other monastics sending a message }

“It\marginnote{6.21.1} may be, monks, that a sick trainee nun sends a message to the monks: ‘Please come, venerables, I’m sick.’ Then, even if you’re not asked, let alone if you are, you should go for seven days, thinking, ‘I’ll look for food for the sick,’ ‘I’ll look for food for the nurses,’ ‘I’ll look for medicine,’ ‘I’ll enquire about her sickness,’ or ‘I’ll nurse her.’ But you should return within seven days. 

It\marginnote{6.22.1} may be that a trainee nun who is discontent with the spiritual life, who is anxious, who has wrong view, or who has failed in the training sends a message to the monks: ‘Please come, venerables, I’ve failed in the training.’ Then, even if you’re not asked, let alone if you are, you should go for seven days, thinking, ‘I’ll make an effort to get her to undertake the training.’ But you should return within seven days. 

It\marginnote{6.23.1} may be that a trainee nun who desires the full ordination sends a message to the monks: ‘Please come, venerables, I desire the full ordination.’ Then, even if you’re not asked, let alone if you are, you should go for seven days, thinking, ‘I’ll make an effort to get her the full ordination,’ ‘I’ll do the proclamation,’ or ‘I’ll complete the quorum.’ But you should return within seven days. 

It\marginnote{6.24.1} may be that a sick novice monk sends a message to the monks: ‘Please come, venerables, I’m sick.’ Then, even if you’re not asked, let alone if you are, you should go for seven days, thinking, ‘I’ll look for food for the sick,’ ‘I’ll look for food for the nurses,’ ‘I’ll look for medicine,’ ‘I’ll enquire about his sickness,’ or ‘I’ll nurse him.’ But you should return within seven days. 

It\marginnote{6.25.1} may be that a novice monk who is discontent with the spiritual life, who is anxious, who has wrong view, or who wants to ask about his age\footnote{\textit{Vassa} refers to the rainy season and by implication to a person’s age, that is, the number of rainy seasons. Perhaps the purpose of this question was to find out whether one is eligible for ordination. The commentaries are silent. } sends a message to the monks: ‘Please come, venerables, I want to ask about my age.’ Then, even if you’re not asked, let alone if you are, you should go for seven days, thinking, ‘I’ll ask him,’ or ‘I’ll inform him.’ But you should return within seven days. 

It\marginnote{6.26.1} may be that a novice monk who desires the full ordination sends a message to the monks: ‘Please come, venerables, I desire the full ordination.’ Then, even if you’re not asked, let alone if you are, you should go for seven days, thinking, ‘I’ll make an effort to get him the full ordination,’ ‘I’ll do the proclamation,’ or ‘I’ll complete the quorum.’ But you should return within seven days. 

It\marginnote{6.27.1} may be that a sick novice nun sends a message to the monks: ‘Please come, venerables, I’m sick.’ Then, even if you’re not asked, let alone if you are, you should go for seven days, thinking, ‘I’ll look for food for the sick,’ ‘I’ll look for food for the nurses,’ ‘I’ll look for medicine,’ ‘I’ll enquire about her sickness,’ or ‘I’ll nurse her.’ But you should return within seven days. 

It\marginnote{6.28.1} may be that a novice nun who is discontent with the spiritual life, who is anxious, who has wrong view, or who wants to ask about her age sends a message to the monks: ‘Please come, venerables, I want to ask about my age.’ Then, even if you’re not asked, let alone if you are, you should go for seven days, thinking, ‘I’ll ask her,’ or ‘I’ll inform her.’ But you should return within seven days. 

It\marginnote{6.29.1} may be that a novice nun who desires to undertake the training of a trainee nun sends a message to the monks: ‘Please come, venerables, I desire to undertake the training.’ Then, even if you’re not asked, let alone if you are, you should go for seven days, thinking, ‘I’ll make an effort for her to undertake the training of a trainee nun.’ But you should return within seven days.” 

\section*{5. The allowance to go to any of seven kinds of persons even if not asked }

On\marginnote{7.1.1} one occasion the mother of a certain monk was sick. She sent a message to her son: “Please come, I’m sick.” That monk thought, “The Buddha has laid down a rule that one should go for seven days to any of seven kinds of persons, but only when asked, and that one should go for seven days to any of five kinds of persons even if not asked, let alone if one is. My mother is sick, but she’s not a lay follower. So what should I do?” They told the Buddha. 

\scrule{“Even if you’re not asked, let alone if you are, I allow you to go for seven days to any of seven kinds of persons—a monk, a nun, a trainee nun, a novice monk, a novice nun, your mother, your father. But you should return within seven days. }

It\marginnote{7.3.1} may be that a monk’s mother is sick and sends a message to her son: ‘Please come, I’m sick.’ Then, even if you’re not asked, let alone if you are, you should go for seven days, thinking, ‘I’ll look for food for the sick,’ ‘I’ll look for food for the nurses,’ ‘I’ll look for medicine,’ ‘I’ll enquire about her sickness,’ or ‘I’ll nurse her.’ But you should return within seven days. 

It\marginnote{7.4.1} may be that a monk’s father is sick and sends a message to his son: ‘Please come, I’m sick.’ Then, even if you’re not asked, let alone if you are, you should go for seven days, thinking, ‘I’ll look for food for the sick,’ ‘I’ll look for food for the nurses,’ ‘I’ll look for medicine,’ ‘I’ll enquire about his sickness,’ or ‘I’ll nurse him.’ But you should return within seven days.” 

\section*{6. The allowance to go only when asked }

“It\marginnote{7.5.1} may be that a monk’s brother is sick and sends a message to his brother: ‘Please come, I’m sick.’ You should go for seven days, but only if you’re asked. And you should return within seven days. 

It\marginnote{7.6.1} may be that a monk’s sister is sick and sends a message to her brother: ‘Please come, I’m sick.’ You should go for seven days, but only if you’re asked. And you should return within seven days. 

It\marginnote{7.7.1} may be that a monk’s relative is sick and sends him a message: ‘Please come, venerable, I’m sick.’ You should go for seven days, but only if you’re asked. And you should return within seven days. 

It\marginnote{7.8.1} may be that one who is staying with the monks is sick\footnote{Sp 3.275: \textit{Bhikkhugatikoti \textsanskrit{ekasmiṁ} \textsanskrit{vihāre} \textsanskrit{bhikkhūhi} \textsanskrit{saddhiṁ} vasanakapuriso}, “\textit{Bhikkhugatika} means a man living with the monks in a particular monastery.” } and sends them a message: ‘Please come, venerables, I’m sick.’ You should go for seven days, but only if you’re asked. And you should return within seven days.” 

At\marginnote{8.1.1} one time one of the Sangha’s dwellings was falling apart. At that time the timber belonging to a certain lay follower had been cut up in the wilderness. He sent a message to the monks: “Venerables, if you retrieve that timber, I’ll give it to you.” They told the Buddha. 

\scrule{“I allow you to go on business for the Sangha. But you should return within seven days.” }

\scend{The section for recitation on the rainy-season residence is finished. }

\section*{7. The section on no offense for breaking the rains residence when there are dangers }

At\marginnote{9.1.1} one time in a certain monastery in the Kosalan country, monks who had entered the rains residence were harassed by predatory animals that attacked and grabbed hold of them. They told the Buddha. 

“It\marginnote{9.1.4} may happen that monks who have entered the rains residence are harassed by predatory animals that attack and grab hold of them. When there’s such a danger, you should leave. There’s no offense for breaking the rains residence. 

It\marginnote{9.1.8} may happen that monks who have entered the rains residence are harassed by creeping animals that attack and bite them. When there’s such a danger, you should leave. There’s no offense for breaking the rains residence. 

It\marginnote{9.2.1} may happen that monks who have entered the rains residence are harassed by criminals who steal from them and beat them up. When there’s such a danger, you should leave. There’s no offense for breaking the rains residence. 

It\marginnote{9.2.5} may happen that monks who have entered the rains residence are harassed by demons who take possession of them and kill them. When there’s such a danger, you should leave. There’s no offense for breaking the rains residence. 

It\marginnote{9.3.1} may happen that the village where monks have entered the rains residence burns down. As a consequence, they have trouble getting almsfood. When there’s such an obstacle, you should leave. There’s no offense for breaking the rains residence. 

It\marginnote{9.3.5} may happen that the dwellings where monks have entered the rains residence burn down. As a consequence, they have trouble getting dwellings. When there’s such an obstacle, you should leave. There’s no offense for breaking the rains residence. 

It\marginnote{9.4.1} may happen that the village where the monks have entered the rains residence is swept away by flooding. As a consequence, they have trouble getting almsfood. When there’s such an obstacle, you should leave. There’s no offense for breaking the rains residence. 

It\marginnote{9.4.5} may happen that the dwellings where the monks have entered the rains residence are swept away by flooding. As a consequence, they have trouble getting dwellings. When there’s such an obstacle, you should leave. There’s no offense for breaking the rains residence.” 

At\marginnote{10.1.1} one time in a certain monastery, the village where the monks had entered the rains residence relocated because of criminals. 

\scrule{“I allow you to move to where the village is.” }

The\marginnote{10.1.4} village was divided in two. 

\scrule{“I allow you to move to where the majority is.” }

The\marginnote{10.1.7} majority had no faith and confidence. 

\scrule{“I allow you to move to where those who have faith and confidence are.” }

At\marginnote{11.1.1} one time in a certain monastery in the Kosalan country, the monks who had entered the rains residence did not get enough food, whether coarse or fine. 

“It\marginnote{11.1.3} may happen that monks who have entered the rains residence don’t get enough food, whether coarse or fine. When there’s such an obstacle, you should leave. There’s no offense for breaking the rains residence. 

It\marginnote{11.1.6} may happen that monks who have entered the rains residence get enough food, whether coarse or fine, but the food isn’t suitable for them. When there’s such an obstacle, you should leave. There’s no offense for breaking the rains residence. 

It\marginnote{11.2.1} may happen that monks who have entered the rains residence get enough suitable food, whether coarse or fine, but they don’t get suitable medicines. When there’s such an obstacle, you should leave. There’s no offense for breaking the rains residence. 

It\marginnote{11.2.4} may happen that monks who have entered the rains residence get enough suitable food, whether coarse or fine, as well as suitable medicines, but they don’t get a suitable attendant. When there’s such an obstacle, you should leave. There’s no offense for breaking the rains residence. 

It\marginnote{11.3.1} may happen that a monk who has entered the rains residence is invited by a woman: ‘Come, venerable, I’ll give you money’, ‘I’ll give you gold’, ‘I’ll give you a field’, ‘I’ll give you land’, ‘I’ll give you an ox’, ‘I’ll give you a cow’, ‘I’ll give you a slave’, ‘I’ll give you my daughter as wife’, ‘I’ll be your wife’, ‘I’ll bring you another wife.’\footnote{“Money” renders \textit{\textsanskrit{hirañña}}, whereas \textit{\textsanskrit{suvaṇṇa}} is for “gold”. See Appendix of Technical Terms. } If that monk thinks, ‘The Buddha has said that the mind is volatile. This could be an obstacle to my monastic life,’ he should leave. There’s no offense for breaking the rains residence. 

It\marginnote{11.4.1} may happen that a monk who has entered the rains residence is invited by a sex worker, by a single woman, by a \textit{\textsanskrit{paṇḍaka}}, by relatives, by kings, by criminals, or by scoundrels: ‘Come, venerable, we’ll give you money’, ‘We’ll give you gold’, ‘We’ll give you a field’, ‘We’ll give you land’, ‘We’ll give you an ox’, ‘We’ll give you a cow’, ‘We’ll give you a slave’, ‘We’ll give you our daughter as wife’, ‘We’ll bring you another wife.’ If that monk thinks, ‘The Buddha has said that the mind is volatile. This could be an obstacle to my monastic life,’ he should leave. There’s no offense for breaking the rains residence. 

It\marginnote{11.4.12} may happen that a monk who has entered the rains residence sees an ownerless treasure. If he thinks, ‘The Buddha has said that the mind is volatile. This could be an obstacle to my monastic life,’ he should leave. There’s no offense for breaking the rains residence.” 

\section*{8. The section on no offense for breaking the rains residence when there is schism in the Sangha }

\paragraph*{Monks pursuing schism }

“It\marginnote{11.5.1} may happen that a monk who has entered the rains residence sees a number of monks who are pursuing schism in the Sangha. If he thinks, ‘The Buddha has said that schism in the Sangha is a serious matter. I don’t want the Sangha to be divided in my presence,’ he should leave. There’s no offense for breaking the rains residence. 

It\marginnote{11.5.6} may happen that a monk who has entered the rains residence hears that a number of monks in such-and-such a monastery are pursuing schism in the Sangha. If he thinks, ‘The Buddha has said that schism in the Sangha is a serious matter. I don’t want the Sangha to be divided in my presence,’ he should leave. There’s no offense for breaking the rains residence. 

It\marginnote{11.6.1} may happen that a monk who has entered the rains residence hears that a number of monks in such-and-such a monastery are pursuing schism in the Sangha. If he thinks, ‘Those monks are my friends. I must tell them that the Buddha has said that schism in the Sangha is a serious matter, and I must ask them not to consent to it. They will act on what I say. They will listen and pay careful attention,’ then he should leave. There’s no offense for breaking the rains residence. 

It\marginnote{11.7.1} may happen that a monk who has entered the rains residence hears that a number of monks in such-and-such a monastery are pursuing schism in the Sangha. If he thinks, ‘Those monks are not my friends, but we have friends in common. If I speak to my friends, they will tell those monks that the Buddha has said that schism in the Sangha is a serious matter, and they will ask them not to consent to it. Those monks will act on what my friends say. They will listen and pay careful attention,’ then he should leave. There’s no offense for breaking the rains residence. 

It\marginnote{11.8.1} may happen that a monk who has entered the rains residence hears that a number of monks in such-and-such a monastery have caused a schism in the Sangha. If he thinks, ‘Those monks are my friends. I must tell them that the Buddha has said that schism in the Sangha is a serious matter, and I must ask them not to consent to it. They will act on what I say. They will listen and pay careful attention,’ then he should leave. There’s no offense for breaking the rains residence. 

It\marginnote{11.9.1} may happen that a monk who has entered the rains residence hears that a number of monks in such-and-such a monastery have caused a schism in the Sangha. If he thinks, ‘Those monks are not my friends, but we have friends in common. If I speak to my friends, they will tell those monks that the Buddha has said that schism in the Sangha is a serious matter, and they will ask them not to consent to it. Those monks will act on what my friends say. They will listen and pay careful attention,’ then he should leave. There’s no offense for breaking the rains residence.” 

\paragraph*{Nuns pursuing schism }

“It\marginnote{11.10.1} may happen that a monk who has entered the rains residence hears that a number of nuns in such-and-such a monastery are pursuing schism in the Sangha. If he thinks, ‘Those nuns are my friends. I must tell them that the Buddha has said that schism in the Sangha is a serious matter, and I must ask them not to consent to it. They will act on what I say. They will listen and pay careful attention,’ then he should leave. There’s no offense for breaking the rains residence. 

It\marginnote{11.11.1} may happen that a monk who has entered the rains residence hears that a number of nuns in such-and-such a monastery are pursuing schism in the Sangha. If he thinks, ‘Those nuns are not my friends, but we have friends in common. If I speak to my friends, they will tell those nuns what the Buddha has said about schism in the Sangha being a serious matter, and they will ask them not to consent to it. Those nuns will act on what my friends say. They will listen and pay careful attention,’ then he should leave. There’s no offense for breaking the rains residence. 

It\marginnote{11.12.1} may happen that a monk who has entered the rains residence hears that a number of nuns in such-and-such a monastery have caused a schism in the Sangha. If he thinks, ‘Those nuns are my friends. I must tell them that the Buddha has said that schism in the Sangha is a serious matter, and I must ask them not to consent to it. They will act on what I say. They will listen and pay careful attention,’ then he should leave. There’s no offense for breaking the rains residence. 

It\marginnote{11.13.1} may happen that a monk who has entered the rains residence hears that a number of nuns in such-and-such a monastery have caused a schism in the Sangha. If he thinks, ‘Those nuns are not my friends, but we have friends in common. If I speak to my friends, they will tell those nuns what the Buddha has said about schism in the Sangha being a serious matter, and they will ask them not to consent to it. Those nuns will act on what my friends say. They will listen and pay careful attention,’ then he should leave. There’s no offense for breaking the rains residence.” 

\section*{9. Entering the rains residence in a cowherd’s dwelling, etc. }

On\marginnote{12.1.1} one occasion a certain monk wanted to enter the rains residence in a cowherd’s dwelling. They told the Buddha. 

\scrule{“I allow you to enter the rains residence in a cowherd’s dwelling.”\footnote{“Cowherd’s dwelling” renders \textit{vaja}. Sp 3.203: \textit{Vajoti \textsanskrit{gopālakānaṃ} \textsanskrit{nivāsaṭṭhānaṃ}}, “Vajo means the dwelling place of cowherds.” This is apparently a reasonably substantial dwelling with a door, as required by the commentary at Sp 3.204, and not just a sunshade as at \href{https://suttacentral.net/pli-tv-kd3/en/brahmali\#12.8.1}{Kd 3:12.8.1} below. } }

The\marginnote{12.1.4} cowherd’s dwelling was moved. 

\scrule{“I allow you to go where the cowherd’s dwelling is.” }

On\marginnote{12.2.1} one occasion, as the entry to the rains residence was getting close, a certain monk wanted to travel by caravan. 

\scrule{“I allow you to enter the rains residence in a caravan.” }

On\marginnote{12.2.4} one occasion, as the entry to the rains residence was getting close, a certain monk wanted to travel by boat. 

\scrule{“I allow you to enter the rains residence on a boat.” }

\section*{10. Places where the rains residence should not be entered }

At\marginnote{12.3.1} one time monks entered the rains residence in the hollow of a tree. People complained and criticized them, “They’re just like goblins.” 

\scrule{“You shouldn’t enter the rains residence in the hollow of a tree. If you do, you commit an offense of wrong conduct.” }

At\marginnote{12.4.1} one time monks entered the rains residence in the fork of a tree. People complained and criticized them, “They’re just like deer hunters.” 

\scrule{“You shouldn’t enter the rains residence in the fork of a tree. If you do, you commit an offense of wrong conduct.” }

At\marginnote{12.5.1} one time monks entered the rains residence out in the open. When it was raining, they ran for cover under trees and eaves. 

\scrule{“You shouldn’t enter the rains residence out in the open. If you do, you commit an offense of wrong conduct.” }

At\marginnote{12.6.1} one time monks entered the rains residence without a dwelling. They suffered in the cold and the heat. 

\scrule{“You shouldn’t enter the rains residence without a dwelling. If you do, you commit an offense of wrong conduct.” }

At\marginnote{12.7.1} one time monks entered the rains residence in a charnel house. People complained and criticized them, “They’re just like undertakers.” 

\scrule{“You shouldn’t enter the rains residence in a charnel house. If you do, you commit an offense of wrong conduct.” }

At\marginnote{12.8.1} one time monks entered the rains residence under a sunshade. People complained and criticized them, “They’re just like cowherds.” 

\scrule{“You shouldn’t enter the rains residence under a sunshade. If you do, you commit an offense of wrong conduct.” }

At\marginnote{12.9.1} one time monks entered the rains residence in a large earthenware pot.\footnote{Sp 3.204: \textit{\textsanskrit{Cāṭiyāti} \textsanskrit{etthāpi} mahantena kapallena}, “Here \textit{\textsanskrit{cāṭi}} is a large piece of earthenware.” } People complained and criticized them, “They’re just like the monastics of other religions.” 

\scrule{“You shouldn’t enter the rains residence in a large earthenware pot. If you do, you commit an offense of wrong conduct.” }

\section*{11. Illegitimate agreements }

At\marginnote{13.1.1} one time the Sangha at \textsanskrit{Sāvatthī} had made an agreement that they would not give the going forth during the rains residence. Then, one of \textsanskrit{Visākhā}’s grandsons went to the monks and asked for the going forth. The monks told him about their agreement, adding, “Please wait while the monks observe the rains residence. Once we’ve completed the rains residence, we’ll give you the going forth.” 

When\marginnote{13.1.8} they had completed the rains residence, the monks told \textsanskrit{Visākhā}’s grandson that they would give him the going forth. He replied, “If I had been given the going forth, venerables, I would have enjoyed it. But now I won’t do it.” \textsanskrit{Visākhā} complained and criticized those monks, “How could the venerables make an agreement that they wouldn’t give the going forth during the rains residence? Is there a time when the Teaching shouldn’t be practiced?” 

The\marginnote{13.2.5} monks heard \textsanskrit{Visākhā}’s complaints and told the Buddha. 

\scrule{“You shouldn’t make an agreement that you won’t give the going forth during the rains residence. If you do, you commit an offense of wrong conduct.” }

\section*{12. An offense of wrong conduct for agreeing }

On\marginnote{14.1.1} one occasion Venerable Upananda the Sakyan had agreed to spend the first rains residence at the invitation of King Pasenadi of Kosala.\footnote{This refers to the first of the two entries to the rainy-season residence, as set out above at \href{https://suttacentral.net/pli-tv-kd3/en/brahmali\#2.2.4}{Kd 3:2.2.4}. } As he was going to the monastery provided by the king, he saw two monasteries with much robe-cloth.\footnote{“Robe-cloth” renders \textit{\textsanskrit{cīvara}}. See Appendix of Technical Terms. } He thought, “Why don’t I spend the rains residence in these two monasteries? That way I’ll get much robe-cloth.” And he spent the rains residence in those two monasteries. 

King\marginnote{14.1.7} Pasenadi complained and criticized him, “How could Upananda agree to spend the rains residence in my monastery, but then break his word? Hasn’t the Buddha in many ways criticized lying and praised truthfulness?” 

The\marginnote{14.2.1} monks heard the king’s complaints, and the monks of few desires complained and criticized Upananda, “How could Upananda act like this?” And they told the Buddha. Soon afterwards the Buddha had the Sangha gathered and questioned Upananda: “Is it true that you acted like this?” 

“It’s\marginnote{14.3.4} true, sir.” 

The\marginnote{14.3.5} Buddha rebuked him … “Foolish man, how could you agree to spend the rains residence at the invitation of King Pasenadi, but then break your word? Haven’t I criticized lying in many ways and praised truthfulness? This will affect people’s confidence …” After rebuking him … he gave a teaching and addressed the monks: 

“It\marginnote{14.4.1} may happen that a monk agrees to spend the first rains residence in a particular monastery. While on his way to that monastery, he sees two monasteries with much robe-cloth. He thinks, ‘Why don’t I spend the rains residence in these two monasteries? That way I’ll get much robe-cloth.’ And he does spend the rains residence in those two monasteries. 

\scrule{The first rains residence doesn’t count for that monk. And there’s an offense of wrong conduct for agreeing.” }

\paragraph*{The first rains residence: observance-day outside monastery }

“It\marginnote{14.5.1} may happen that a monk agrees to spend the first rains residence in a particular monastery. While on his way to that monastery, he does the observance-day ceremony outside. On the following day, he enters and prepares the dwelling, sets out water for drinking and water for washing, and sweeps the yard. He then leaves on that very day, despite not having any business. 

\scrule{The first rains residence doesn’t count for that monk. And there’s an offense of wrong conduct for agreeing. }

It\marginnote{14.5.5} may happen that a monk agrees to spend the first rains residence in a particular monastery. While on his way to that monastery, he does the observance-day ceremony outside. On the following day, he enters and prepares the dwelling, sets out water for drinking and water for washing, and sweeps the yard. He then leaves on that very day because of business. 

\scrule{The first rains residence doesn’t count for that monk. And there’s an offense of wrong conduct for agreeing. }

It\marginnote{14.6.1} may happen that a monk agrees to spend the first rains residence in a particular monastery. While on his way to that monastery, he does the observance-day ceremony outside. On the following day, he enters and prepares the dwelling, sets out water for drinking and water for washing, and sweeps the yard. After staying there for two or three days, he leaves, despite not having any business. 

\scrule{The first rains residence doesn’t count for that monk. And there’s an offense of wrong conduct for agreeing. }

It\marginnote{14.6.5} may happen that a monk agrees to spend the first rains residence in a particular monastery. While on his way to that monastery, he does the observance-day ceremony outside. On the following day, he enters and prepares the dwelling, sets out water for drinking and water for washing, and sweeps the yard. After staying there for two or three days, he leaves because of business. 

\scrule{The first rains residence doesn’t count for that monk. And there’s an offense of wrong conduct for agreeing. }

It\marginnote{14.6.9} may happen that a monk agrees to spend the first rains residence in a particular monastery. While on his way to that monastery, he does the observance-day ceremony outside. On the following day, he enters and prepares the dwelling, sets out water for drinking and water for washing, and sweeps the yard. After staying there for two or three days, he leaves on seven-day business. But he stays away for more than seven days. 

\scrule{The first rains residence doesn’t count for that monk. And there’s an offense of wrong conduct for agreeing. }

It\marginnote{14.6.14} may happen that a monk agrees to spend the first rains residence in a particular monastery. While on his way to that monastery, he does the observance-day ceremony outside. On the following day, he enters and prepares the dwelling, sets out water for drinking and water for washing, and sweeps the yard. After staying there for two or three days, he leaves on seven-day business. And he returns within seven days. 

\scrule{The first rains residence does count for that monk. And there’s no offense for agreeing. }

It\marginnote{14.7.1} may happen that a monk agrees to spend the first rains residence in a particular monastery. While on his way to that monastery, he does the observance-day ceremony outside. On the following day, he enters and prepares the dwelling, sets out water for drinking and water for washing, and sweeps the yard. Seven days before the invitation ceremony, he leaves because of business. 

\scrule{Whether he returns to that monastery or not, the first rains residence does count for that monk. And there’s no offense for agreeing.” }

\paragraph*{The first rains residence: observance-day within monastery }

“It\marginnote{14.8.1} may happen that a monk agrees to spend the first rains residence in a particular monastery. When he’s arrived at that monastery, he does the observance-day ceremony. On the following day, he enters and prepares the dwelling, sets out water for drinking and water for washing, and sweeps the yard. He then leaves on that very day, despite not having any business. 

\scrule{The first rains residence doesn’t count for that monk. And there’s an offense of wrong conduct for agreeing. }

It\marginnote{14.9.1} may happen that a monk agrees to spend the first rains residence in a particular monastery. When he’s arrived at that monastery, he does the observance-day ceremony. On the following day, he enters and prepares the dwelling, sets out water for drinking and water for washing, and sweeps the yard. He then leaves on that very day because of business. … After staying there for two or three days, he leaves, despite not having any business. … After staying there for two or three days, he leaves because of business. … After staying there for two or three days, he leaves on seven-day business. But he stays away for more than seven days. 

\scrule{The first rains residence doesn’t count for that monk. And there’s an offense of wrong conduct for agreeing. }

…\marginnote{14.10.2} After staying there for two or three days, he leaves on seven-day business. And he returns within seven days. 

\scrule{The first rains residence does count for that monk. And there’s no offense for agreeing. }

…\marginnote{14.10.5} Seven days before the invitation ceremony, he leaves on seven-day business. 

\scrule{Whether he returns to that monastery or not, the first rains residence does count for that monk. And there’s no offense for agreeing.” }

\paragraph*{The second rains residence: observance-day outside monastery }

“It\marginnote{14.11.1} may happen that a monk agrees to spend the second rains residence in a particular monastery. While on his way to that monastery, he does the observance-day ceremony outside. On the following day, he enters and prepares the dwelling, sets out water for drinking and water for washing, and sweeps the yard. He then leaves on that very day, despite not having any business. 

\scrule{The second rains residence doesn’t count for that monk. And there’s an offense of wrong conduct for agreeing. }

It\marginnote{14.11.5} may happen that a monk agrees to spend the second rains residence in a particular monastery. While on his way to that monastery, he does the observance-day ceremony outside. On the following day, he enters and prepares the dwelling, sets out water for drinking and water for washing, and sweeps the yard. He then leaves on that very day because of business. 

\scrule{The second rains residence doesn’t count for that monk. And there’s an offense of wrong conduct for agreeing. }

It\marginnote{14.11.9} may happen that a monk agrees to spend the second rains residence in a particular monastery. While on his way to that monastery, he does the observance-day ceremony outside. On the following day, he enters and prepares the dwelling, sets out water for drinking and water for washing, and sweeps the yard. After staying there for two or three days, he leaves, despite not having any business. 

\scrule{The second rains residence doesn’t count for that monk. And there’s an offense of wrong conduct for agreeing. }

It\marginnote{14.11.13} may happen that a monk agrees to spend the second rains residence in a particular monastery. While on his way to that monastery, he does the observance-day ceremony outside. On the following day, he enters and prepares the dwelling, sets out water for drinking and water for washing, and sweeps the yard. After staying there for two or three days, he leaves because of business. 

\scrule{The second rains residence doesn’t count for that monk. And there’s an offense of wrong conduct for agreeing. }

It\marginnote{14.11.17} may happen that a monk agrees to spend the second rains residence in a particular monastery. While on his way to that monastery, he does the observance-day ceremony outside. On the following day, he enters and prepares the dwelling, sets out water for drinking and water for washing, and sweeps the yard. After staying there for two or three days, he leaves on seven-day business. But he stays away for more than seven days. 

\scrule{The second rains residence doesn’t count for that monk. And there’s an offense of wrong conduct for agreeing. }

It\marginnote{14.11.22} may happen that a monk agrees to spend the second rains residence in a particular monastery. While on his way to that monastery, he does the observance-day ceremony outside. On the following day, he enters and prepares the dwelling, sets out water for drinking and water for washing, and sweeps the yard. After staying there for two or three days, he leaves on seven-day business. And he returns within seven days. 

\scrule{The second rains residence does count for that monk. And there’s no offense for agreeing. }

It\marginnote{14.11.27} may happen that a monk agrees to spend the second rains residence in a particular monastery. While on his way to that monastery, he does the observance-day ceremony outside. On the following day, he enters and prepares the dwelling, sets out water for drinking and water for washing, and sweeps the yard. Seven days before \textsanskrit{Komudī}, the fourth full-moon day of the rainy season, he leaves because of business. 

\scrule{Whether he returns to that monastery or not, the second rains residence does count for that monk. And there’s no offense for agreeing.” }

\paragraph*{The second rains residence: observance-day within monastery }

“It\marginnote{14.11.31.1} may happen that a monk agrees to spend the second rains residence in a particular monastery. When he’s arrived at that monastery, he does the observance-day ceremony. On the following day, he enters and prepares the dwelling, sets out water for drinking and water for washing, and sweeps the yard. He then leaves on that very day, despite not having any business. 

\scrule{The second rains residence doesn’t count for that monk. And there’s an offense of wrong conduct for agreeing. }

It\marginnote{14.11.35} may happen that a monk agrees to spend the second rains residence in a particular monastery. When he’s arrived at that monastery, he does the observance-day ceremony. On the following day, he enters and prepares the dwelling, sets out water for drinking and water for washing, and sweeps the yard. He then leaves on that very day because of business. … After staying there for two or three days, he leaves, despite not having any business. … After staying there for two or three days, he leaves because of business. … After staying there for two or three days, he leaves on seven-day business. But he stays away for more than seven days. 

\scrule{The second rains residence doesn’t count for that monk. And there’s an offense of wrong conduct for agreeing. }

…\marginnote{14.11.43} After staying there for two or three days, he leaves on seven-day business. And he returns within seven days. 

\scrule{The second rains residence does count for that monk. And there’s no offense for agreeing. }

It\marginnote{14.11.46} may happen that a monk agrees to spend the second rains residence in a particular monastery. When he’s arrived at that monastery, he does the observance-day ceremony. On the following day, he enters and prepares the dwelling, sets out water for drinking and water for washing, and sweeps the yard. Seven days before \textsanskrit{Komudī}, the fourth full-moon day of the rainy season, he leaves because of business. 

\scrule{Whether he returns to that monastery or not, the second rains residence does count for that monk. And there’s no offense for agreeing.” }

\scendsutta{The third chapter on entering the rainy-season residence is finished. }

\scuddanaintro{This is the summary: }

\begin{scuddana}%
“To\marginnote{14.11.52} enter and when, \\
How many, and during the rains; \\
And they did not want, deliberately, \\
To postpone, lay follower. 

Sick,\marginnote{14.11.56} and mother, father, \\
And brother, then a relative; \\
One staying with the monks, dwelling, \\
And also predatory, creeping animals. 

And\marginnote{14.11.60} criminals, and demons, \\
And then burned down twice; \\
Swept away by flooding, it relocated, \\
And majority, donors. 

Coarse\marginnote{14.11.64} or fine, suitable, \\
And medicine, with attendant; \\
Woman, sex worker, and single woman, \\
A \textit{\textsanskrit{paṇḍaka}}, and by a relative. 

King,\marginnote{14.11.68} criminals, scoundrels, treasure, \\
And with eightfold on schism; \\
A cowherd’s dwelling, and a caravan, and a boat, \\
In a hollow, and in a fork. 

Rains\marginnote{14.11.72} residence out in the open, \\
And without a dwelling; \\
Charnel house, and under a sunshade, \\
And they entered in a large earthenware pot. 

Agreement,\marginnote{14.11.76} having agreed, \\
And observance days outside; \\
First, second, \\
Should be understood according to the same method. 

He\marginnote{14.11.80} departs without business, \\
And the same with business; \\
And two or three days, and again, \\
And on seven-day business. 

And\marginnote{14.11.84} returned within seven days, \\
Whether he returns or not; \\
Because of the gaps in the summary of topics, \\
One should attend carefully to the way of the passages of the Canonical text.” 

%
\end{scuddana}

\scend{In this chapter there are fifty-two topics. }

\scendsutta{The chapter on entering the rainy-season residence is finished. }

%
\chapter*{{\suttatitleacronym Kd 4}{\suttatitletranslation The chapter on the invitation ceremony }{\suttatitleroot Pavāraṇākkhandhaka}}
\addcontentsline{toc}{chapter}{\tocacronym{Kd 4} \toctranslation{The chapter on the invitation ceremony } \tocroot{Pavāraṇākkhandhaka}}
\markboth{The chapter on the invitation ceremony }{Pavāraṇākkhandhaka}
\extramarks{Kd 4}{Kd 4}

\section*{1. Being uncomfortable }

At\marginnote{1.1.1} one time the Buddha was staying at \textsanskrit{Sāvatthī} in the Jeta Grove, \textsanskrit{Anāthapiṇḍika}’s Monastery. At that time in a certain monastery in the Kosalan country a number of monks who were friends had entered rainy-season residence together. They thought, “How can we have a comfortable rains, live in peace and harmony, and get almsfood without trouble?” Then it occurred to them, “Let’s not talk to one another. Whoever returns first from almsround in the village should prepare the seats, and set out a foot stool, a foot scraper, and water for washing the feet. He should wash the bowl for leftovers and put it back out, and set out water for drinking and water for washing. Whoever returns last from almsround may eat whatever is left over, or he should discard it where there are no cultivated plants or in water without life.\footnote{\textit{Harita} could in principle refer to all plants, but it is elsewhere defined as what is cultivated, see \href{https://suttacentral.net/pli-tv-bu-vb-pc19/en/brahmali\#2.1.14}{Bu Pc 19:2.1.14} and \href{https://suttacentral.net/pli-tv-bi-vb-pc9/en/brahmali\#2.1.14}{Bi Pc 9:2.1.14}. } He should put away the seats and also the foot stool, the foot scraper, and the water for washing the feet. He should wash the bowl for leftovers and put it away, put away the water for drinking and the water for washing, and sweep the dining hall. Whoever sees that the pot for drinking water, the pot for washing water, or the waterpot in the restroom is empty should fill it. If he can’t do it by himself, he should call someone over by hand signal, and they should fill it together. He shouldn’t speak because of that. In this way we’ll have a comfortable rains, live in peace and harmony, and get almsfood without trouble.” 

And\marginnote{1.5.1} they did just that. 

Now\marginnote{1.8.1} it was the custom for monks who had completed the rainy-season residence to go and see the Buddha. And so, when the three months were over and they had completed the rains residence, they put their dwellings in order, took their bowls and robes, and set out for \textsanskrit{Sāvatthī}. When they eventually arrived, they went to the Jeta Grove, \textsanskrit{Anāthapiṇḍika}’s Monastery. There they approached the Buddha, bowed, and sat down. 

Since\marginnote{1.8.4} it is the custom for Buddhas to greet newly-arrived monks, the Buddha said to them, “I hope you’re keeping well, monks, I hope you’re getting by?  I hope you had a comfortable and harmonious rains, and got almsfood without trouble?” 

“We’re\marginnote{1.9.4} keeping well, sir, we’re getting by. We had a comfortable and harmonious rains, and got almsfood without trouble.” 

When\marginnote{1.10.1} Buddhas know what is going on, sometimes they ask and sometimes not. They know the right time to ask and when not to ask. Buddhas ask when it is beneficial, otherwise not, for Buddhas are incapable of doing what is unbeneficial.\footnote{“Incapable of doing” renders \textit{\textsanskrit{setughāta}}, literally, “destroyed the bridge”. Sp 1.16: \textit{Setu vuccati maggo, maggeneva \textsanskrit{tādisassa} vacanassa \textsanskrit{ghāto}, samucchedoti \textsanskrit{vuttaṁ} hoti}, “The path is called the bridge. What is said is that there is the destruction and cutting off of such speech by the path.” The commentary seems to take \textit{setu}, “bridge”, as a reference to the eightfold path. I prefer to understand “bridge” as a metaphor for access, that is, the Buddhas no longer have the possibility of doing what is unbeneficial. } Buddhas question the monks for two reasons: to give a teaching or to lay down a training rule. 

So\marginnote{1.10.6} the Buddha said to those monks, “In what way, monks, did you have a harmonious and comfortable rains? And how did you get almsfood without trouble?” 

When\marginnote{1.12.1} they had told him, the Buddha addressed the monks: 

“While\marginnote{1.12.2} being uncomfortable, these foolish men claim they were living in comfort. While living together like animals, they claim they were living in comfort. While living together like sheep, they claim they were living in comfort. While living together like enemies, they claim they were living in comfort. How could these foolish men take a vow of silence, like the monastics of other religions? This will affect people’s confidence …” After rebuking them and giving a teaching, he addressed the monks: 

\scrule{“You shouldn’t take a vow of silence, like the monastics of other religions. If you do, you commit an offense of wrong conduct. }

\scrule{When you have completed the rainy-season residence, you should invite the monks to correct you in regard to three things: what has been seen, heard, or suspected. }

This\marginnote{1.13.7} will help you live with one another in the proper way, help you clear yourself of offenses, and help you prioritize the training.\footnote{“Training” renders \textit{vinaya}. See Appendix of Technical Terms. } And you should do the invitation ceremony like this.\footnote{For the rendering “Should do the invitation ceremony”, see \textit{\textsanskrit{pavāraṇā}} in Appendix of Technical Terms. } A competent and capable monk should inform the Sangha: 

‘Please,\marginnote{1.14.3} venerables, I ask the Sangha to listen. Today is the invitation ceremony. If the Sangha is ready, it should do the invitation ceremony.’ 

The\marginnote{1.14.6} most senior monk should arrange his upper robe over one shoulder, squat on his heels, raise his joined palms, and say:\footnote{\textit{Therena \textsanskrit{bhikkhunā}} could be rendered “a/the senior monk”. Yet the point is that only the most senior member of the Sangha should use the semi-informal address \textit{\textsanskrit{āvuso}}, whereas everyone else should use the formal equivalent \textit{bhante}. } 

‘I\marginnote{1.14.7} invite the Sangha to correct me concerning what you have seen, heard, or suspect. Please correct me, venerables, out of compassion.\footnote{“Correct” renders \textit{vadantu}. See \textit{vadati} in Appendix of Technical Terms. } If I see a fault, I will make amends. For the second time, I invite the Sangha to correct me concerning what you have seen, heard, or suspect. Please correct me, venerables, out of compassion. If I see a fault, I will make amends. For the third time, I invite the Sangha to correct me concerning what you have seen, heard, or suspect. Please correct me, venerables, out of compassion. If I see a fault, I will make amends.’ 

Each\marginnote{1.14.16} junior monk should arrange his upper robe over one shoulder, squat on his heels, raise his joined palms, and say:\footnote{\textit{Navakena \textsanskrit{bhikkhunā}} could be rendered “a/the junior monk”. Yet the point here is that only the most senior member of the Sangha should use the semi-informal address \textit{\textsanskrit{āvuso}}, whereas everyone else should use the formal equivalent \textit{bhante}. In this context, then, \textit{navaka} does not have its normal meaning of “newly ordained” monk, but rather refers to any monk junior to the most senior one. } 

‘I\marginnote{1.14.17} invite the Sangha to correct me concerning what you have seen, heard, or suspect. Please correct me, venerables, out of compassion. If I see a fault, I will make amends. For the second time, I invite the Sangha to correct me concerning what you have seen, heard, or suspect. Please correct me, venerables, out of compassion. If I see a fault, I will make amends. For the third time, I invite the Sangha to correct me concerning what you have seen, heard, or suspect. Please correct me, venerables, out of compassion. If I see a fault, I will make amends.’” 

Soon\marginnote{2.1.1} afterwards the monks from the group of six remained seated while the senior monks were inviting correction, squatting on their heels. The monks of few desires complained and criticized them, “How can the monks from the group of six act like this?” They told the Buddha. … “Is it true, monks, that the monks from the group of six are acting like this?” 

“It’s\marginnote{2.1.6} true, sir.” 

The\marginnote{2.1.7} Buddha rebuked them … “How can those foolish men act like this?” This will affect people’s confidence …” After rebuking them … he gave a teaching and addressed the monks: 

\scrule{“You shouldn’t remain seated while the senior monks are inviting correction, squatting on their heels. If you do, you commit an offense of wrong conduct. }

\scrule{Everyone should squat on their heels during the invitation ceremony.” }

On\marginnote{2.2.1} one occasion, while squatting on his heels and waiting for everyone to finish, a senior monk who was weak from old age fainted and fell over. They told the Buddha. 

\scrule{“You should squat on your heels until you have invited correction. Once you have invited, you may sit down.” }

\section*{2. Breach of the invitation ceremony }

The\marginnote{3.1.1} monks thought, “How many invitation days are there?” 

\scrule{“There are two invitation days: the fourteenth and the fifteenth day of the lunar half-month.” }

The\marginnote{3.2.1} monks thought, “How many kinds of invitation procedures are there?” 

“There\marginnote{3.2.4} are four kinds: 

\begin{enumerate}%
\item The invitation procedure that is illegitimate and has an incomplete assembly. %
\item The invitation procedure that is illegitimate but has a complete assembly. %
\item The invitation procedure that is legitimate but has an incomplete assembly. %
\item The invitation procedure that is legitimate and has a complete assembly. %
\end{enumerate}

\scrule{The first, second, and third of these shouldn’t be done; I haven’t allowed such procedures. The fourth should be done; I have allowed such procedures. Therefore, monks, you should train like this: ‘We will do invitation procedures that are legitimate and have a complete assembly.’” }

\section*{3. The allowance to pass on the invitation }

The\marginnote{3.3.1} Buddha addressed the monks: “Gather, monks, for the Sangha to do the invitation ceremony.” A monk said to the Buddha, “Sir, there’s a sick monk. He hasn’t come.” 

\scrule{“A sick monk should pass on his invitation. }

And\marginnote{3.3.7} he should do it like this. The sick monk should approach a monk, arrange his upper robe over one shoulder, and squat on his heels. He should then raise his joined palms and say, ‘I pass on my invitation; please convey my invitation; please announce my invitation; please invite correction on my behalf.’ If he makes this understood by body, by speech, or by body and speech, then the invitation has been passed on. If he doesn’t make this understood by body, by speech, or by body and speech, then the invitation hasn’t been passed on. 

If\marginnote{3.4.1} he’s able to do this, it’s good. If he’s not, then the sick monk should be brought into the midst of the Sangha together with his bed or bench. They can then do the invitation ceremony. But if the one who is nursing him says, ‘If we move him, his illness will get worse, or he’ll die,’ then the sick monk shouldn’t be moved. The Sangha should go to where the sick monk is and do the invitation ceremony there. 

\scrule{You shouldn’t do the invitation ceremony with an incomplete sangha. If you do, you commit an offense of wrong conduct. }

If,\marginnote{3.5.1} after the invitation has been passed on to him, the monk who is conveying the invitation goes away right then and there, then the invitation should be passed on to someone else.\footnote{Sp 3.164: \textit{Tattheva \textsanskrit{pakkamatīti} \textsanskrit{saṅghamajjhaṁ} \textsanskrit{anāgantvā} tatova katthaci gacchati}, “\textit{Tattheva pakkamati}: not having gone to the midst of the Sangha, he goes wherever.” } If, after the invitation has been passed on to him, the monk who is conveying the invitation disrobes right then and there, dies right then and there, admits right then and there that he’s a novice monk, admits right then and there that he’s renounced the training, admits right then and there that he’s committed the worst kind of offense, admits right then and there that he’s insane, admits right then and there that he’s deranged, admits right then and there that he’s overwhelmed by pain, admits right then and there that he’s been ejected for not recognizing an offense, admits right then and there that he’s been ejected for not making amends for an offense, admits right then and there that he’s been ejected for not giving up a bad view, admits right then and there that he’s a \textit{\textsanskrit{paṇḍaka}}, admits right then and there that he’s a fake monk, admits right then and there that he’s previously left to join the monastics of another religion, admits right then and there that he’s an animal, admits right then and there that he’s a matricide, admits right then and there that he’s a patricide, admits right then and there that he’s a murderer of a perfected one, admits right then and there that he’s raped a nun, admits right then and there that he’s caused a schism in the Sangha, admits right then and there that he’s caused the Buddha to bleed, or admits right then and there that he’s a hermaphrodite, then the invitation should be passed on to someone else. 

If,\marginnote{3.5.24} after the invitation has been passed on to him, the monk who is conveying the invitation goes away while on his way to the invitation ceremony, then the invitation hasn’t been brought. If, after the invitation has been passed on to him, the monk who is conveying the invitation disrobes, dies, admits that he’s a novice monk, admits that he’s renounced the training, admits that he’s committed the worst kind of offense, admits that he’s insane, admits that he’s deranged, admits that he’s overwhelmed by pain, admits that he’s been ejected for not recognizing an offense, admits that he’s been ejected for not making amends for an offense, admits that he’s been ejected for not giving up a bad view, admits that he’s a \textit{\textsanskrit{paṇḍaka}}, admits that he’s a fake monk, admits that he’s previously left to join the monastics of another religion, admits that he’s an animal, admits that he’s a matricide, admits that he’s a patricide, admits that he’s a murderer of a perfected one, admits that he’s raped a nun, admits that he’s caused a schism in the Sangha, admits that he’s caused the Buddha to bleed, or admits that he’s a hermaphrodite while on his way to the invitation ceremony, then the invitation hasn’t been brought. 

But\marginnote{3.5.47} if, after the invitation has been passed on to him, the monk who is conveying the invitation goes away after reaching the Sangha, then the invitation has been brought. And if, after the invitation has been passed on to him, the monk who is conveying the invitation disrobes, dies, admits that he’s a novice monk, admits that he’s renounced the training, admits that he’s committed the worst kind of offense, admits that he’s insane, admits that he’s deranged, admits that he’s overwhelmed by pain, admits that he’s been ejected for not recognizing an offense, admits that he’s been ejected for not making amends for an offense, admits that he’s been ejected for not giving up a bad view, admits that he’s a \textit{\textsanskrit{paṇḍaka}}, admits that he’s a fake monk, admits that he’s previously left to join the monastics of another religion, admits that he’s an animal, admits that he’s a matricide, admits that he’s a patricide, admits that he’s a murderer of a perfected one, admits that he’s raped a nun, admits that he’s caused a schism in the Sangha, admits that he’s caused the Buddha to bleed, or admits that he’s a hermaphrodite after reaching the Sangha, then the invitation has been brought. 

And\marginnote{3.5.70} if, after the invitation has been passed on to him, the monk who is conveying the invitation reaches the Sangha, but doesn’t announce the invitation because he falls asleep, then the invitation has been brought. There’s no offense for the one who is conveying the invitation. And if, after the invitation has been passed on to him, the monk who is conveying the invitation reaches the Sangha, but doesn’t announce the invitation because he is heedless or because he gains a meditation attainment, then the invitation has been brought. There’s no offense for the one who is conveying the invitation. 

And\marginnote{3.5.75} if, after the invitation has been passed on to him, the monk who is conveying the invitation reaches the Sangha, but deliberately doesn’t announce the invitation, then the invitation has been brought. 

\scrule{But there’s an offense of wrong conduct for the one who is conveying the invitation. }

\scrule{On the invitation day, if the Sangha has business to be done, then anyone passing on their invitation should also give their consent.” }

\section*{4. Discussion on being seized by relatives, etc. }

At\marginnote{4.1.1} one time on the invitation day, a certain monk was seized by his relatives. They told the Buddha. 

“If\marginnote{4.1.3} a monk is seized by his relatives on the invitation day, other monks should say to those relatives, ‘Listen, please release this monk for a short time so that he can take part in the invitation ceremony.’ If they’re able to do this, it’s good. If not, they should say to those relatives, ‘Listen, please step aside for a moment while this monk passes on his invitation.’ If they’re able to do this, it’s good. If not, they should say to those relatives, ‘Listen, please take this monk outside the monastery zone for a short time while the Sangha does the invitation ceremony.’ If they’re able to do this, it’s good. 

\scrule{If not, you shouldn’t do the invitation ceremony with an incomplete sangha. If you do, you commit an offense of wrong conduct. }

If\marginnote{4.3.1} on the invitation day a monk is seized by kings, by bandits, by scoundrels, or by enemies of the monks,\footnote{“Enemies of monks” is a translation of \textit{\textsanskrit{bhikkhupaccatthikā}}. At \href{https://suttacentral.net/pli-tv-bu-vb-pj1/en/brahmali\#9.3.1}{Bu Pj 1:9.3.1}, I have translated the same compound as “enemy monks”. In that rule this seems required because various people who are acting as enemies of monks are mentioned separately, such as kings, bandits, and scoundrels. Moreover, all of these are compounded with \textit{\textsanskrit{paccatthikā}}: \textit{\textsanskrit{bhikkhupaccatthikā}}, \textit{\textsanskrit{rājapaccatthikā}}, and so on. Since it seems reasonable to assume that all these compounds have the same structure, it follows that they should all be read as “enemies who are so-and-so” rather than “enemies of so-and-so”. This understanding is confirmed by Sp 1.58: \textit{\textsanskrit{bhikkhū} eva \textsanskrit{paccatthikā} \textsanskrit{bhikkhupaccatthikā}}, “\textit{\textsanskrit{Bhikkhupaccatthikā}} are just monks who are enemies.” In the present context, however, this interpretation does not seem to work. If \textit{\textsanskrit{bhikkhupaccatthikā}} refers to enemies who are monks, then they would have to be invited to take part in the ceremony, or some other arrangement would have to be made, but nothing is said about this in either the Pali or the commentaries. Moreover, kings, bandits, and scoundrels are in this case not compounded with \textit{\textsanskrit{paccatthikā}}, as they are in Bu Pj 1. I therefore conclude that the meaning here must be “enemies of monks”. } other monks should say to those enemies, ‘Listen, please release this monk for a short time, so that he can take part in the invitation ceremony.’ If they’re able to do this, it’s good. If not, they should say to those enemies, ‘Listen, please step aside for a moment while this monk passes on his invitation.’ If they’re able to do this, it’s good. If not, they should say to those enemies, ‘Listen, please take this monk outside the monastery zone for a short time while the Sangha does the invitation ceremony.’ If they’re able to do this, it’s good. 

\scrule{If not, you shouldn’t do the invitation ceremony with an incomplete sangha. If you do, you commit an offense of wrong conduct.” }

\section*{5. Various kinds of invitation ceremonies for the Sangha, etc. }

At\marginnote{5.1.1} one time on the invitation day, there were five monks staying in a certain monastery. They thought, “The Buddha has laid down a rule that the invitation ceremony should be done with a sangha. Now there’s five of us. So how should we do the invitation ceremony?” They told the Buddha. 

\scrule{“When there are five of you, you should do the invitation ceremony in the Sangha.” }

At\marginnote{5.2.1} one time on the invitation day, there were four monks staying in a certain monastery. They thought, “The Buddha has instructed that the invitation ceremony should be done in the Sangha when there are five monks. But there’s only four of us. So how should we do the invitation ceremony?” 

\scrule{“When there are four of you, you should do the invitation ceremony with one another. }

And\marginnote{5.3.1} you should do it like this. A competent and capable monk should inform those monks: 

‘Please,\marginnote{5.3.3} venerables, I ask you to listen. Today is the invitation ceremony. If the venerables are ready, we should do the invitation ceremony with one another.’ 

The\marginnote{5.3.6} most senior monk should arrange his upper robe over one shoulder, squat on his heels, raise his joined palms, and say to the other monks: 

‘I\marginnote{5.3.7} invite you to correct me concerning what you have seen, heard, or suspect. Please correct me, venerables, out of compassion. If I see a fault, I will make amends. For the second time … For the third time, I invite you to correct me concerning what you have seen, heard, or suspect. Please correct me, venerables, out of compassion. If I see a fault, I will make amends.’ 

Each\marginnote{5.3.14} junior monk should arrange his upper robe over one shoulder, squat on his heels, raise his joined palms, and say to the other monks: 

‘I\marginnote{5.3.15} invite you to correct me concerning what you have seen, heard, or suspect. Please correct me, venerables, out of compassion. If I see a fault, I will make amends. For the second time … For the third time, I invite you to correct me concerning what you have seen, heard, or suspect. Please correct me, venerables, out of compassion. If I see a fault, I will make amends.’” 

At\marginnote{5.4.1} one time on the invitation day, there were three monks staying in a certain monastery. They thought, “The Buddha has instructed that the invitation ceremony should be done in the Sangha when there are five monks and with one another when there are four. But there’s only three of us. So how should we do the invitation ceremony?” 

\scrule{“When there are three of you, you should do the invitation ceremony with one another. }

And\marginnote{5.4.8} you should do it like this. A competent and capable monk should inform those monks: 

‘Please,\marginnote{5.4.10} venerables, I ask you to listen. Today is the invitation ceremony. If the venerables are ready, we should do the invitation ceremony with one another.’ 

The\marginnote{5.4.13} most senior monk should arrange his upper robe over one shoulder, squat on his heels, raise his joined palms, and say to the other monks: 

‘I\marginnote{5.4.14} invite you to correct me concerning what you have seen, heard, or suspect. Please correct me, venerables, out of compassion. If I see a fault, I will make amends. For the second time … For the third time, I invite you to correct me concerning what you have seen, heard, or suspect. Please correct me, venerables, out of compassion. If I see a fault, I will make amends.’ 

Each\marginnote{5.4.21} junior monk should arrange his upper robe over one shoulder, squat on his heels, raise his joined palms, and say to the other monks: 

‘I\marginnote{5.4.22} invite you to correct me concerning what you have seen, heard, or suspect. Please correct me, venerables, out of compassion. If I see a fault, I will make amends. For the second time … For the third time, I invite you to correct me concerning what you have seen, heard, or suspect. Please correct me, venerables, out of compassion. If I see a fault, I will make amends.’” 

At\marginnote{5.5.1} one time on the invitation day, there were two monks staying in a certain monastery. They thought, “The Buddha has instructed that the invitation ceremony should be done in the Sangha when there are five monks and with one another when there are three or four. But there’s only two of us. So how should we do the invitation ceremony?” 

\scrule{“When there are two of you, you should do the invitation ceremony with each other. }

And\marginnote{5.6.1} you should do it like this. The senior monk should arrange his upper robe over one shoulder, squat on his heels, raise his joined palms, and say to the junior monk: 

‘I\marginnote{5.6.3} invite you to correct me concerning what you have seen, heard, or suspect. Please correct me, venerable, out of compassion. If I see a fault, I will make amends. For the second time … For the third time, I invite you to correct me concerning what you have seen, heard, or suspect. Please correct me, venerable, out of compassion. If I see a fault, I will make amends.’ 

The\marginnote{5.6.10} junior monk should arrange his upper robe over one shoulder, squat on his heels, raise his joined palms, and say to the senior monk: 

‘I\marginnote{5.6.11} invite you to correct me concerning what you have seen, heard, or suspect. Please correct me, venerable, out of compassion. If I see a fault, I will make amends. For the second time … For the third time, I invite you to correct me concerning what you have seen, heard, or suspect. Please correct me, venerable, out of compassion. If I see a fault, I will make amends.’” 

At\marginnote{5.7.1} one time on the invitation day, a monk was staying in a certain monastery by himself. He thought, “The Buddha has instructed that the invitation ceremony should be done in the Sangha when there are five monks and with one another when there are two, three, or four. But I’m here by myself. So how should I do the invitation ceremony?” 

“On\marginnote{5.8.1} the invitation day, a monk may be staying by himself in a certain monastery. He should sweep the place where the monks normally go: whether the assembly hall, under a roof cover, or at the foot of a tree. He should set out water for drinking and water for washing. He should prepare a seat, light a lamp, and sit down. 

If\marginnote{5.8.3} other monks arrive, he should do the invitation ceremony with them. If not, he should determine: ‘Today is my invitation ceremony.’ 

\scrule{If he doesn’t make a determination, he commits an offense of wrong conduct. }

\scrule{Wherever five monks are staying together, four shouldn’t do the invitation ceremony in the Sangha, while the invitation of the fifth is brought. If you do the invitation in the Sangha, you commit an offense of wrong conduct. }

\scrule{Wherever four monks are staying together, three shouldn’t do the invitation ceremony with one another, while the invitation of the fourth is brought. If you do the invitation in this way, you commit an offense of wrong conduct. }

\scrule{Wherever three monks are staying together, two shouldn’t do the invitation ceremony with each other, while the invitation of the third is brought. If you do the invitation in this way, you commit an offense of wrong conduct. }

\scrule{Wherever two monks are staying together, one shouldn’t make a determination, while the invitation of the other is brought. If you do make a determination, you commit an offense of wrong conduct.” }

\section*{6. The process for making amends for an offense }

At\marginnote{6.1.1} one time on the invitation day, a certain monk had committed an offense. He thought, “The Buddha has laid down a rule that one shouldn’t invite correction if one has an unconfessed offense. And I’ve committed an offense. So what should I do?” They told the Buddha. 

“On\marginnote{6.1.8} the invitation day, a monk may have committed an offense. He should approach a single monk, arrange his upper robe over one shoulder, squat on his heels, raise his joined palms, and say: 

‘I’ve\marginnote{6.1.10} committed such-and-such an offense. I confess it.’ The other should say, ‘Do you recognize the offense?’ —‘Yes, I recognize it.’ —‘You should restrain yourself in the future.’ 

On\marginnote{6.1.15} the invitation day, a monk may be unsure if he’s committed an offense. He should approach a single monk, arrange his upper robe over one shoulder, squat on his heels, raise his joined palms, and say: 

‘I’m\marginnote{6.1.17} unsure if I’ve committed such-and-such an offense. I’ll make amends for it when I’m sure.’ They can then do the invitation ceremony. This is not an obstacle to doing the invitation ceremony.” 

\section*{7. The process for revealing an offense }

At\marginnote{6.2.1} one time a certain monk remembered an offense during the invitation ceremony. He thought, “The Buddha has laid down a rule that one shouldn’t invite correction if one has an unconfessed offense. And I’ve committed an offense. So what should I do?” They told the Buddha. 

\scrule{“A monk may remember an offense during the invitation ceremony. He should say to a monk sitting next to him, ‘I’ve committed such-and-such an offense. Once this ceremony is finished, I’ll make amends for it.’ They can then continue the invitation ceremony. This is not an obstacle to doing the invitation ceremony. }

\scrule{A monk may become unsure if he has committed an offense during the invitation ceremony. He should say to a monk sitting next to him, ‘I’m unsure if I’ve committed such-and-such an offense. I’ll make amends for it when I’m sure.’ They can then continue the invitation ceremony. This is not an obstacle to doing the invitation ceremony.” }

\section*{8. The process for making amends for a shared offense }

At\marginnote{6.3.6.1} one time on the invitation day, the whole Sangha in a certain monastery had committed the same offense. The monks thought, “The Buddha has laid down a rule that one shouldn’t confess or receive the confession of shared offenses. Yet here the whole Sangha has committed the same offense. So what should we do?” 

\scrule{“On the invitation day, the whole Sangha in a certain monastery may have committed the same offense. Those monks should straightaway send a monk to a neighboring monastery: ‘Go and make amends for this offense. When you return, we’ll make amends for it with you.’ }

If\marginnote{6.3.16} he’s able to do this, it’s good. If he’s not, then a competent and capable monk should inform the Sangha: 

‘Please,\marginnote{6.3.18} venerables, I ask the Sangha to listen. This whole Sangha has committed the same offense. When the Sangha sees another monk who is pure and free of offenses, it should make amends for this offense with him.’ 

Once\marginnote{6.3.21} this has been said, they can do the invitation ceremony. This is not an obstacle to doing the invitation ceremony. 

On\marginnote{6.3.23} the invitation day, the whole Sangha in a certain monastery may be unsure if it has committed the same offense. A competent and capable monk should then inform the Sangha: 

‘Please,\marginnote{6.3.25} venerables, I ask the Sangha to listen. This whole Sangha is unsure if it has committed the same offense. When the Sangha is sure, it should make amends for this offense.’ 

Once\marginnote{6.3.28} this has been said, they can do the invitation ceremony. This is not an obstacle to doing the invitation ceremony.” 

\scend{The first section for recitation is finished. }

\section*{9. The group of fifteen on non-offenses }

At\marginnote{7.1.1} one time on the invitation day, five or more resident monks had gathered together in a certain monastery. They did not know that there were other resident monks who had not arrived.\footnote{Here “resident monk” means a monk who is within the \textit{\textsanskrit{sīmā}}, the monastery zone. } Perceiving that they were acting according to the Teaching and the Monastic Law, perceiving that the assembly was complete although it was not, they did the invitation ceremony. While they were doing it, a greater number of resident monks arrived. They told the Buddha. 

“On\marginnote{7.2.1} the invitation day, five or more resident monks may have gathered together in a certain monastery. They don’t know there are other resident monks who haven’t arrived. Perceiving that they’re acting according to the Teaching and the Monastic Law, perceiving that the assembly is complete although it’s not, they do the invitation ceremony. While they’re doing it, a greater number of resident monks arrive. 

\scrule{In such a case, those monks should do the invitation ceremony once more. There’s no offense for those who already have invited. }

On\marginnote{7.3.1} the invitation day, five or more resident monks may have gathered together in a certain monastery. They don’t know there are other resident monks who haven’t arrived. Perceiving that they’re acting according to the Teaching and the Monastic Law, perceiving that the assembly is complete although it’s not, they do the invitation ceremony. While they’re doing it, an equal number of resident monks arrive. 

\scrule{In such a case, the invitations of those who already have invited are valid, but the others should invite. There’s no offense for those who already have invited. }

On\marginnote{7.3.8} the invitation day, five or more resident monks may have gathered together in a certain monastery. They don’t know there are other resident monks who haven’t arrived. Perceiving that they’re acting according to the Teaching and the Monastic Law, perceiving that the assembly is complete although it’s not, they do the invitation ceremony. While they’re doing it, a smaller number of resident monks arrive. 

\scrule{In such a case, the invitations of those who already have invited are valid, but the others should invite. There’s no offense for those who already have invited. }

On\marginnote{7.4.1} the invitation day, five or more resident monks may have gathered together in a certain monastery. They don’t know there are other resident monks who haven’t arrived. Perceiving that they’re acting according to the Teaching and the Monastic Law, perceiving that the assembly is complete although it’s not, they do the invitation ceremony. When they’ve just finished, a greater number of resident monks arrive. 

\scrule{In such a case, those monks should do the invitation ceremony once more. There’s no offense for those who already have invited. }

On\marginnote{7.4.8} the invitation day, five or more resident monks may have gathered together in a certain monastery. They don’t know there are other resident monks who haven’t arrived. Perceiving that they’re acting according to the Teaching and the Monastic Law, perceiving that the assembly is complete although it’s not, they do the invitation ceremony. When they’ve just finished, an equal number of resident monks arrive. 

\scrule{In such a case, the invitations of those who already have invited are valid, and the late arrivals should invite in the presence of the others. There’s no offense for those who already have invited. }

On\marginnote{7.4.15} the invitation day, five or more resident monks may have gathered together in a certain monastery. They don’t know there are other resident monks who haven’t arrived. Perceiving that they’re acting according to the Teaching and the Monastic Law, perceiving that the assembly is complete although it’s not, they do the invitation ceremony. When they’ve just finished, a smaller number of resident monks arrive. 

\scrule{In such a case, the invitations of those who already have invited are valid, and the late arrivals should invite in the presence of the others. There’s no offense for those who already have invited. }

On\marginnote{7.5.1} the invitation day, five or more resident monks may have gathered together in a certain monastery. They don’t know there are other resident monks who haven’t arrived. Perceiving that they’re acting according to the Teaching and the Monastic Law, perceiving that the assembly is complete although it’s not, they do the invitation ceremony. When they’ve just finished, and none of the gathering has left, a greater number of resident monks arrive.\footnote{\textit{\textsanskrit{Avuṭṭhitāya} \textsanskrit{parisāya}} literally means that “the gathering has not got up”. The point, presumably, is that the meeting is not yet over and those present have not started to leave. } 

\scrule{In such a case, those monks should do the invitation ceremony once more. There’s no offense for those who already have invited. }

On\marginnote{7.5.8} the invitation day, five or more resident monks may have gathered together in a certain monastery. They don’t know there are other resident monks who haven’t arrived. Perceiving that they’re acting according to the Teaching and the Monastic Law, perceiving that the assembly is complete although it’s not, they do the invitation ceremony. When they’ve just finished, and none of the gathering has left, an equal number of resident monks arrive. 

\scrule{In such a case, the invitations of those who already have invited are valid, and the late arrivals should invite in the presence of the others. There’s no offense for those who already have invited. }

On\marginnote{7.5.15} the invitation day, five or more resident monks may have gathered together in a certain monastery. They don’t know there are other resident monks who haven’t arrived. Perceiving that they’re acting according to the Teaching and the Monastic Law, perceiving that the assembly is complete although it’s not, they do the invitation ceremony. When they’ve just finished, and none of the gathering has left, a smaller number of resident monks arrive. 

\scrule{In such a case, the invitations of those who already have invited are valid, and the late arrivals should invite in the presence of the others. There’s no offense for those who already have invited. }

On\marginnote{7.5.22} the invitation day, five or more resident monks may have gathered together in a certain monastery. They don’t know there are other resident monks who haven’t arrived. Perceiving that they’re acting according to the Teaching and the Monastic Law, perceiving that the assembly is complete although it’s not, they do the invitation ceremony. When they’ve just finished, and only some members of the gathering have left, a greater number of resident monks arrive. 

\scrule{In such a case, those monks should do the invitation ceremony once more. There’s no offense for those who already have invited. }

On\marginnote{7.5.29} the invitation day, five or more resident monks may have gathered together in a certain monastery. They don’t know there are other resident monks who haven’t arrived. Perceiving that they’re acting according to the Teaching and the Monastic Law, perceiving that the assembly is complete although it’s not, they do the invitation ceremony. When they’ve just finished, and only some members of the gathering have left, an equal number of resident monks arrive. 

\scrule{In such a case, the invitations of those who already have invited are valid, and the late arrivals should invite in the presence of the others. There’s no offense for those who already have invited. }

On\marginnote{7.5.36} the invitation day, five or more resident monks may have gathered together in a certain monastery. They don’t know there are other resident monks who haven’t arrived. Perceiving that they’re acting according to the Teaching and the Monastic Law, perceiving that the assembly is complete although it’s not, they do the invitation ceremony. When they’ve just finished, and only some members of the gathering have left, a smaller number of resident monks arrive. 

\scrule{In such a case, the invitations of those who already have invited are valid, and the late arrivals should invite in the presence of the others. There’s no offense for those who already have invited. }

On\marginnote{7.5.43} the invitation day, five or more resident monks may have gathered together in a certain monastery. They don’t know there are other resident monks who haven’t arrived. Perceiving that they’re acting according to the Teaching and the Monastic Law, perceiving that the assembly is complete although it’s not, they do the invitation ceremony. When they’ve just finished, and the entire gathering has left, a greater number of resident monks arrive. 

\scrule{In such a case, those monks should do the invitation ceremony once more. There’s no offense for those who already have invited. }

On\marginnote{7.5.50} the invitation day, five or more resident monks may have gathered together in a certain monastery. They don’t know there are other resident monks who haven’t arrived. Perceiving that they’re acting according to the Teaching and the Monastic Law, perceiving that the assembly is complete although it’s not, they do the invitation ceremony. When they’ve just finished, and the entire gathering has left, an equal number of resident monks arrive. 

\scrule{In such a case, the invitations of those who already have invited are valid, and the late arrivals should invite in the presence of the others. There’s no offense for those who already have invited. }

On\marginnote{7.5.57} the invitation day, five or more resident monks may have gathered together in a certain monastery. They don’t know there are other resident monks who haven’t arrived. Perceiving that they’re acting according to the Teaching and the Monastic Law, perceiving that the assembly is complete although it’s not, they do the invitation ceremony. When they’ve just finished, and the entire gathering has left, a smaller number of resident monks arrive. 

\scrule{In such a case, the invitations of those who already have invited are valid, and the late arrivals should invite in the presence of the others. There’s no offense for those who already have invited.” }

\scend{The group of fifteen on non-offenses is finished. }

\section*{10. The group of fifteen on perceiving an incomplete assembly as incomplete }

“On\marginnote{8.1.1} the invitation day, five or more resident monks may have gathered together in a certain monastery. They know there are other resident monks who haven’t arrived. Perceiving that they’re acting according to the Teaching and the Monastic Law, yet correctly perceiving the assembly as incomplete, they do the invitation ceremony. While they’re doing it, a greater number of resident monks arrive. 

\scrule{In such a case, those monks should do the invitation ceremony once more. There’s an offense of wrong conduct for those who already have invited. }

On\marginnote{8.2.1} the invitation day, five or more resident monks may have gathered together in a certain monastery. They know there are other resident monks who haven’t arrived. Perceiving that they’re acting according to the Teaching and the Monastic Law, yet correctly perceiving the assembly as incomplete, they do the invitation ceremony. While they’re doing it, an equal number of resident monks arrive. 

\scrule{In such a case, the invitations of those who already have invited are valid, but the others should invite. There’s an offense of wrong conduct for those who already have invited. }

On\marginnote{8.2.8} the invitation day, five or more resident monks may have gathered together in a certain monastery. They know there are other resident monks who haven’t arrived. Perceiving that they’re acting according to the Teaching and the Monastic Law, yet correctly perceiving the assembly as incomplete, they do the invitation ceremony. While they’re doing it, a smaller number of resident monks arrive. 

\scrule{In such a case, the invitations of those who already have invited are valid, but the others should invite. There’s an offense of wrong conduct for those who already have invited. }

On\marginnote{8.3.1} the invitation day, five or more resident monks may have gathered together in a certain monastery. They know there are other resident monks who haven’t arrived. Perceiving that they’re acting according to the Teaching and the Monastic Law, yet correctly perceiving the assembly as incomplete, they do the invitation ceremony. When they’ve just finished … When they’ve just finished, and none of the gathering has left … When they’ve just finished, and only some members of the gathering have left … When they’ve just finished, and the entire gathering has left, a greater number of resident monks arrive … an equal number of resident monks arrive … a smaller number of resident monks arrive. 

\scrule{In such a case, the invitations of those who already have invited are valid, and the late arrivals should invite in the presence of the others. There’s an offense of wrong conduct for those who already have invited.” }

\scend{The group of fifteen on perceiving an incomplete assembly as incomplete is finished. }

\section*{11. The group of fifteen on being unsure }

“On\marginnote{9.1.1} the invitation day, five or more resident monks may have gathered together in a certain monastery. They know there are other resident monks who haven’t arrived. They think, ‘Is it allowable for us to do the invitation ceremony or not?’ Being unsure, they do the invitation ceremony. While they’re doing it, a greater number of resident monks arrive. 

\scrule{In such a case, those monks should do the invitation ceremony once more. There’s an offense of wrong conduct for those who already have invited. }

On\marginnote{9.2.1} the invitation day, five or more resident monks may have gathered together in a certain monastery. They know there are other resident monks who haven’t arrived. They think, ‘Is it allowable for us to do the invitation ceremony or not?’ Being unsure, they do the invitation ceremony. While they’re doing it, an equal number of resident monks arrive. 

\scrule{In such a case, the invitations of those who already have invited are valid, but the others should invite. There’s an offense of wrong conduct for those who already have invited. }

On\marginnote{9.2.8} the invitation day, five or more resident monks may have gathered together in a certain monastery. They know there are other resident monks who haven’t arrived. They think, ‘Is it allowable for us to do the invitation ceremony or not?’ Being unsure, they do the invitation ceremony. While they’re doing it, a smaller number of resident monks arrive. 

\scrule{In such a case, the invitations of those who already have invited are valid, but the others should invite. There’s an offense of wrong conduct for those who already have invited. }

On\marginnote{9.2.15} the invitation day, five or more resident monks may have gathered together in a certain monastery. They know there are other resident monks who haven’t arrived. They think, ‘Is it allowable for us to do the invitation ceremony or not?’ Being unsure, they do the invitation ceremony. When they’ve just finished … When they’ve just finished, and none of the gathering has left … When they’ve just finished, and only some members of the gathering have left … When they’ve just finished, and the entire gathering has left, a greater number of resident monks arrive … an equal number of resident monks arrive … a smaller number of resident monks arrive. 

\scrule{In such a case, the invitations of those who already have invited are valid, and the late arrivals should invite in the presence of the others. There’s an offense of wrong conduct for those who already have invited.” }

\scend{The group of fifteen on being unsure is finished. }

\section*{12. The group of fifteen on being anxious }

“On\marginnote{10.1.1} the invitation day, five or more resident monks may have gathered together in a certain monastery. They know there are other resident monks who haven’t arrived. They think, ‘It’s allowable for us to do the invitation ceremony; it’s not unallowable.’ Being anxious, they do the invitation ceremony. While they’re doing it, a greater number of resident monks arrive. 

\scrule{In such a case, those monks should do the invitation ceremony once more. There’s an offense of wrong conduct for those who already have invited. }

On\marginnote{10.2.1} the invitation day, five or more resident monks may have gathered together in a certain monastery. They know there are other resident monks who haven’t arrived. They think, ‘It’s allowable for us to do the invitation ceremony; it’s not unallowable.’ Being anxious, they do the invitation ceremony. While they’re doing it, an equal number of resident monks arrive. 

\scrule{In such a case, the invitations of those who already have invited are valid, but the others should invite. There’s an offense of wrong conduct for those who already have invited. }

On\marginnote{10.2.9} the invitation day, five or more resident monks may have gathered together in a certain monastery. They know there are other resident monks who haven’t arrived. They think, ‘It’s allowable for us to do the invitation ceremony; it’s not unallowable.’ Being anxious, they do the invitation ceremony. While they’re doing it, a smaller number of resident monks arrive. 

\scrule{In such a case, the invitations of those who already have invited are valid, but the others should invite. There’s an offense of wrong conduct for those who already have invited. }

On\marginnote{10.2.17} the invitation day, five or more resident monks may have gathered together in a certain monastery. They know there are other resident monks who haven’t arrived. They think, ‘It’s allowable for us to do the invitation ceremony; it’s not unallowable.’ Being anxious, they do the invitation ceremony. When they’ve just finished … When they’ve just finished, and none of the gathering has left … When they’ve just finished, and only some members of the gathering have left … When they’ve just finished, and the entire gathering has left, a greater number of resident monks arrive … an equal number of resident monks arrive … a smaller number of resident monks arrive. 

\scrule{In such a case, the invitations of those who already have invited are valid, and the late arrivals should invite in the presence of the others. There’s an offense of wrong conduct for those who already have invited.” }

\scend{The group of fifteen on being anxious is finished. }

\section*{13. The group of fifteen on aiming at schism }

“On\marginnote{11.1.1} the invitation day, five or more resident monks may have gathered together in a certain monastery. They know there are other resident monks who haven’t arrived. They think, ‘May they get lost! May they disappear! We are better off without them.’ They then do the invitation ceremony, aiming at schism. While they’re doing it, a greater number of resident monks arrive. 

\scrule{In such a case, those monks should do the invitation ceremony once more. And there’s a serious offense for those who already have invited. }

On\marginnote{11.2.1} the invitation day, five or more resident monks may have gathered together in a certain monastery. They know there are other resident monks who haven’t arrived. They think, ‘May they get lost! May they disappear! We are better off without them.’ They then do the invitation ceremony, aiming at schism. While they’re doing it, an equal number of resident monks arrive. 

\scrule{In such a case, the invitations of those who already have invited are valid, but the others should invite. And there’s a serious offense for those who already have invited. }

On\marginnote{11.2.9} the invitation day, five or more resident monks may have gathered together in a certain monastery. They know there are other resident monks who haven’t arrived. They think, ‘May they get lost! May they disappear! We are better off without them.’ They then do the invitation ceremony, aiming at schism. While they’re doing it, a smaller number of resident monks arrive. 

\scrule{In such a case, the invitations of those who already have invited are valid, but the others should invite. And there’s a serious offense for those who already have invited. }

On\marginnote{11.2.17} the invitation day, five or more resident monks may have gathered together in a certain monastery. They know there are other resident monks who haven’t arrived. They think, ‘May they get lost! May they disappear! We are better off without them.’ They then do the invitation ceremony, aiming at schism. When they’ve just finished, a greater number of resident monks arrive. 

\scrule{In such a case, those monks should do the invitation ceremony once more. And there’s a serious offense for those who already have invited. }

On\marginnote{11.2.25} the invitation day, five or more resident monks may have gathered together in a certain monastery. They know there are other resident monks who haven’t arrived. They think, ‘May they get lost! May they disappear! We are better off without them.’ They then do the invitation ceremony, aiming at schism. When they’ve just finished, an equal number of resident monks arrive. 

\scrule{In such a case, the invitations of those who already have invited are valid, and the late arrivals should invite in the presence of the others. And there’s a serious offense for those who already have invited. }

On\marginnote{11.2.33} the invitation day, five or more resident monks may have gathered together in a certain monastery. They know there are other resident monks who haven’t arrived. They think, ‘May they get lost! May they disappear! We are better off without them.’ They then do the invitation ceremony, aiming at schism. When they’ve just finished, a smaller number of resident monks arrive. 

\scrule{In such a case, the invitations of those who already have invited are valid, and the late arrivals should invite in the presence of the others. And there’s a serious offense for those who already have invited. }

On\marginnote{11.2.41} the invitation day, five or more resident monks may have gathered together in a certain monastery. They know there are other resident monks who haven’t arrived. They think, ‘May they get lost! May they disappear! We are better off without them.’ They then do the invitation ceremony, aiming at schism. When they’ve just finished, and none of the gathering has left, a greater number of resident monks arrive. 

\scrule{In such a case, those monks should do the invitation ceremony once more. And there’s a serious offense for those who already have invited. }

On\marginnote{11.2.49} the invitation day, five or more resident monks may have gathered together in a certain monastery. They know there are other resident monks who haven’t arrived. They think, ‘May they get lost! May they disappear! We are better off without them.’ They then do the invitation ceremony, aiming at schism. When they’ve just finished, and none of the gathering has left, an equal number of resident monks arrive. 

\scrule{In such a case, the invitations of those who already have invited are valid, and the late arrivals should invite in the presence of the others. And there’s a serious offense for those who already have invited. }

On\marginnote{11.2.57} the invitation day, five or more resident monks may have gathered together in a certain monastery. They know there are other resident monks who haven’t arrived. They think, ‘May they get lost! May they disappear! We are better off without them.’ They then do the invitation ceremony, aiming at schism. When they’ve just finished, and none of the gathering has left, a smaller number of resident monks arrive. 

\scrule{In such a case, the invitations of those who already have invited are valid, and the late arrivals should invite in the presence of the others. And there’s a serious offense for those who already have invited. }

On\marginnote{11.2.65} the invitation day, five or more resident monks may have gathered together in a certain monastery. They know there are other resident monks who haven’t arrived. They think, ‘May they get lost! May they disappear! We are better off without them.’ They then do the invitation ceremony, aiming at schism. When they’ve just finished, and only some members of the gathering have left, a greater number of resident monks arrive. 

\scrule{In such a case, those monks should do the invitation ceremony once more. And there’s a serious offense for those who already have invited. }

On\marginnote{11.2.73} the invitation day, five or more resident monks may have gathered together in a certain monastery. They know there are other resident monks who haven’t arrived. They think, ‘May they get lost! May they disappear! We are better off without them.’ They then do the invitation ceremony, aiming at schism. When they’ve just finished, and only some members of the gathering have left, an equal number of resident monks arrive. 

\scrule{In such a case, the invitations of those who already have invited are valid, and the late arrivals should invite in the presence of the others. And there’s a serious offense for those who already have invited. }

On\marginnote{11.2.81} the invitation day, five or more resident monks may have gathered together in a certain monastery. They know there are other resident monks who haven’t arrived. They think, ‘May they get lost! May they disappear! We are better off without them.’ They then do the invitation ceremony, aiming at schism. When they’ve just finished, and only some members of the gathering have left, a smaller number of resident monks arrive. 

\scrule{In such a case, the invitations of those who already have invited are valid, and the late arrivals should invite in the presence of the others. And there’s a serious offense for those who already have invited. }

On\marginnote{11.2.89} the invitation day, five or more resident monks may have gathered together in a certain monastery. They know there are other resident monks who haven’t arrived. They think, ‘May they get lost! May they disappear! We are better off without them.’ They then do the invitation ceremony, aiming at schism. When they’ve just finished, and the entire gathering has left, a greater number of resident monks arrive. 

\scrule{In such a case, those monks should do the invitation ceremony once more. And there’s a serious offense for those who already have invited. }

On\marginnote{11.2.97} the invitation day, five or more resident monks may have gathered together in a certain monastery. They know there are other resident monks who haven’t arrived. They think, ‘May they get lost! May they disappear! We are better off without them.’ They then do the invitation ceremony, aiming at schism. When they’ve just finished, and the entire gathering has left, an equal number of resident monks arrive. 

\scrule{In such a case, the invitations of those who already have invited are valid, and the late arrivals should invite in the presence of the others. And there’s a serious offense for those who already have invited. }

On\marginnote{11.2.105} the invitation day, five or more resident monks may have gathered together in a certain monastery. They know there are other resident monks who haven’t arrived. They think, ‘May they get lost! May they disappear! We are better off without them.’ They then do the invitation ceremony, aiming at schism. When they’ve just finished, and the entire gathering has left, a smaller number of resident monks arrive. 

\scrule{In such a case, the invitations of those who already have invited are valid, and the late arrivals should invite in the presence of the others. And there’s a serious offense for those who already have invited.” }

\scend{The group of fifteen on aiming at schism is finished. }

\scend{The group of seventy-five is finished. }

\section*{14. The successive series on entering a monastery zone }

“On\marginnote{12.1.1} the invitation day, five or more resident monks may have gathered together in a certain monastery. They don’t know that other resident monks are entering the monastery zone. … They don’t know that other resident monks have entered the monastery zone. … They don’t see that other resident monks are entering the monastery zone. … They don’t see that other resident monks have entered the monastery zone. … They don’t hear that other resident monks are entering the monastery zone. … They don’t hear that other resident monks have entered the monastery zone. …” 

As\marginnote{12.1.12} there are one hundred and seventy-five sets of three for resident monks with resident monks, so there is for newly-arrived monks with resident monks, resident monks with newly-arrived monks, newly-arrived monks with newly-arrived monks. Thus by way of succession, there are seven hundred sets of three. 

\section*{15. Different days }

“It\marginnote{13.1.1} may be, monks, that for the resident monks it’s the fourteenth day of the lunar half-month, but for the newly-arrived monks it’s the fifteenth. Then—

\scrule{If the number of resident monks is greater, the newly-arrived monks should fall in line with the resident monks. }

\scrule{If the number is the same, the newly-arrived monks should fall in line with the resident monks. }

\scrule{If the number of newly-arrived monks is greater, the resident monks should fall in line with the newly-arrived monks. }

It\marginnote{13.1.5} may be that for the resident monks it’s the fifteenth day of the lunar half-month, but for the newly-arrived monks it’s the fourteenth. Then—

\scrule{If the number of resident monks is greater, the newly-arrived monks should fall in line with the resident monks. }

\scrule{If the number is the same, the newly-arrived monks should fall in line with the resident monks. }

\scrule{If the number of newly-arrived monks is greater, the resident monks should fall in line with the newly-arrived monks. }

It\marginnote{13.1.9} may be that for the resident monks it’s the day after the invitation day, but for the newly-arrived monks it’s the fifteenth day of the lunar half-month. Then—

\scrule{If the number of resident monks is greater, the resident monks may, if they’re willing, do the invitation ceremony with the newly-arrived monks. Otherwise the newly-arrived monks should go outside the monastery zone and do the invitation ceremony there. }

\scrule{If the number is the same, the resident monks may, if they’re willing, do the invitation ceremony with the newly-arrived monks. Otherwise the newly-arrived monks should go outside the monastery zone and do the invitation ceremony there. }

\scrule{If the number of newly-arrived monks is greater, the resident monks should do the invitation ceremony with the newly-arrived monks, or they should go outside the monastery zone while the newly-arrived monks do the invitation ceremony. }

It\marginnote{13.1.15} may be that for the resident monks it’s the fifteenth day of the lunar half-month, but for the newly-arrived monks it’s the day after the invitation day. Then—

\scrule{If the number of resident monks is greater, the newly-arrived monks should do the invitation ceremony with the resident monks, or they should go outside the monastery zone while the resident monks do the invitation ceremony. }

\scrule{If the number is the same, the newly-arrived monks should do the invitation ceremony with the resident monks, or they should go outside the monastery zone while the resident monks do the invitation ceremony. }

\scrule{If the number of newly-arrived monks is greater, they may, if they’re willing, do the invitation ceremony with the resident monks. Otherwise the resident monks should go outside the monastery zone and do the invitation ceremony there.” }

\section*{16. The seeing of characteristics, etc. }

“It\marginnote{13.1.20.1} may happen that newly-arrived monks see signs and indications of resident monks: beds and benches that are made up, water for drinking and water for washing that are ready for use, yards that are well swept. As a consequence, they’re unsure whether or not there are resident monks there. Then—

\scrule{If they do the invitation ceremony without investigating, there’s an offense of wrong conduct.\footnote{The Pali text has ellipsis points at the end of this sentence, but this seems to be a mistake, cf. \href{https://suttacentral.net/pli-tv-kd2/en/brahmali\#34.6.3}{Kd 2:34.6.3}. } }

\scrule{If they investigate, but don’t see anyone, and then do the invitation ceremony, there’s no offense. }

\scrule{If they investigate, and they see someone, and then do the invitation ceremony together, there’s no offense. }

\scrule{If they investigate, and they see someone, but then do the invitation ceremony separately, there’s an offense of wrong conduct. }

\scrule{If they investigate, and they see someone, but think, ‘May they get lost! May they disappear! We are better off without them,’ and then do the invitation ceremony aiming at schism, there’s a serious offense. }

It\marginnote{13.1.27} may happen that newly-arrived monks hear signs and indications of resident monks: the sound of the feet of someone doing walking meditation, the sound of recitation, the sound of coughing, the sound of sneezing. As a consequence, they’re unsure whether or not there are resident monks there. Then—

\scrule{If they do the invitation ceremony without investigating, there’s an offense of wrong conduct. }

\scrule{If they investigate, but don’t see anyone, and then do the invitation ceremony, there’s no offense. }

\scrule{If they investigate, and they see someone, and then do the invitation ceremony together, there’s no offense. }

\scrule{If they investigate, and they see someone, but then do the invitation ceremony separately, there’s an offense of wrong conduct. }

\scrule{If they investigate, and they see someone, but think, ‘May they get lost! May they disappear! We are better off without them,’ and then do the invitation ceremony aiming at schism, there’s a serious offense. }

It\marginnote{13.1.34} may happen that resident monks see signs and indications of newly-arrived monks: an unknown almsbowl, an unknown robe, an unknown sitting mat, water poured on the ground from the washing of feet. As a consequence, they’re unsure whether or not monks have arrived. Then—

\scrule{If they do the invitation ceremony without investigating, there’s an offense of wrong conduct. }

\scrule{If they investigate, but don’t see anyone, and then do the invitation ceremony, there’s no offense. }

\scrule{If they investigate, and they see someone, and then do the invitation ceremony together, there’s no offense. }

\scrule{If they investigate, and they see someone, but then do the invitation ceremony separately, there’s an offense of wrong conduct. }

\scrule{If they investigate, and they see someone, but think, ‘May they get lost! May they disappear! We are better off without them,’ and then do the invitation ceremony aiming at schism, there’s a serious offense. }

It\marginnote{13.1.41} may happen that resident monks hear signs and indications of newly-arrived monks: the sound of the feet of someone arriving, the sound of sandals being knocked together, the sound of coughing, the sound of sneezing. As a consequence, they’re unsure whether or not monks have arrived. Then—

\scrule{If they do the invitation ceremony without investigating, there’s an offense of wrong conduct. }

\scrule{If they investigate, but don’t see anyone, and then do the invitation ceremony, there’s no offense. }

\scrule{If they investigate, and they see someone, and then do the invitation ceremony together, there’s no offense. }

\scrule{If they investigate, and they see someone, but then do the invitation ceremony separately, there’s an offense of wrong conduct. }

\scrule{If they investigate, and they see someone, but think, ‘May they get lost! May they disappear! We are better off without them,’ and then do the invitation ceremony aiming at schism, there’s a serious offense.” }

\section*{17. The doing of the invitation ceremony with those belonging to a different Buddhist sect, etc. }

“It\marginnote{13.1.48.1} may happen that newly-arrived monks see resident monks who belong to a different Buddhist sect,\footnote{\textit{\textsanskrit{Nānāsaṁvāsaka}} (and \textit{\textsanskrit{samānasaṁvāsaka}}) need to be carefully distinguished from \textit{\textsanskrit{nānāsaṁvāsa}} (and \textit{\textsanskrit{samānasaṁvāsa}}). Only the former means “one belonging to a different Buddhist sect”. The latter means “belonging to a different community”, as decided by \textit{\textsanskrit{sīmās}}. } but they have the view that they belong to the same one. Then—

\scrule{If they don’t ask the resident monks about it, and then do the invitation ceremony together, there’s no offense. }

\scrule{If they do ask the resident monks about it, but don’t reach a clear conclusion, and then do the invitation ceremony together, there’s an offense of wrong conduct. }

\scrule{If they do ask the resident monks about it, but don’t reach a clear conclusion, and then do the invitation ceremony separately, there’s no offense. }

It\marginnote{13.1.53} may happen that newly-arrived monks see resident monks who belong to the same Buddhist sect, but they have the view that they belong to a different one. Then—

\scrule{If they don’t ask the resident monks about it, and then do the invitation ceremony together, there’s an offense of wrong conduct. }

\scrule{If they do ask the resident monks about it, and they change their view, but then do the invitation ceremony separately, there’s an offense of wrong conduct. }

\scrule{If they do ask the resident monks about it, and they change their view, and then do the invitation ceremony together, there’s no offense. }

It\marginnote{13.1.57} may happen that resident monks see newly-arrived monks who belong to a different Buddhist sect, but they have the view that they belong to the same one. Then—

\scrule{If they don’t ask the newly-arrived monks about it, and then do the invitation ceremony together, there’s no offense. }

\scrule{If they do ask the newly-arrived monks about it, but don’t reach a clear conclusion, and then do the invitation ceremony together, there’s an offense of wrong conduct. }

\scrule{If they do ask the newly-arrived monks about it, but don’t reach a clear conclusion, and then do the invitation ceremony separately, there’s no offense. }

It\marginnote{13.1.62} may happen that resident monks see newly-arrived monks who belong to the same Buddhist sect, but they have the view that they belong to a different one. Then—

\scrule{If they don’t ask the newly-arrived monks about it, and then do the invitation ceremony together, there’s an offense of wrong conduct. }

\scrule{If they do ask the newly-arrived monks about it, and they change their view, but then do the invitation ceremony separately, there’s an offense of wrong conduct. }

\scrule{If they do ask the newly-arrived monks about it, and they change their view, and then do the invitation ceremony together, there’s no offense.” }

\section*{18. The section on “you shouldn’t go” }

“On\marginnote{13.1.67.1} the invitation day you shouldn’t go from a monastery with monks to a monastery without monks, except if you go with a sangha or there are dangers. On the invitation day you shouldn’t go from a monastery with monks to a non-monastery without monks, except if you go with a sangha or there are dangers.\footnote{Here and below I understand a monastery, an \textit{\textsanskrit{āvāsa}}, to refer to a monastery with a properly defined zone, a \textit{\textsanskrit{sīmā}}. A non-monastery, an \textit{\textsanskrit{anāvāsa}}, is then a monastic residence without such a zone. } On the invitation day you shouldn’t go from a monastery with monks to a monastery or a non-monastery without monks, except if you go with a sangha or there are dangers. 

On\marginnote{13.1.70} the invitation day you shouldn’t go from a non-monastery with monks to a monastery without monks, except if you go with a sangha or there are dangers. On the invitation day you shouldn’t go from a non-monastery with monks to a non-monastery without monks, except if you go with a sangha or there are dangers. On the invitation day you shouldn’t go from a non-monastery with monks to a monastery or a non-monastery without monks, except if you go with a sangha or there are dangers. 

On\marginnote{13.1.73} the invitation day you shouldn’t go from a monastery or a non-monastery with monks to a monastery without monks, except if you go with a sangha or there are dangers. On the invitation day you shouldn’t go from a monastery or a non-monastery with monks to a non-monastery without monks, except if you go with a sangha or there are dangers. On the invitation day you shouldn’t go from a monastery or a non-monastery with monks to a monastery or a non-monastery without monks, except if you go with a sangha or there are dangers. 

On\marginnote{13.1.76} the invitation day you shouldn’t go from a monastery with monks to a monastery with monks who belong to a different Buddhist sect, except if you go with a sangha or there are dangers. On the invitation day you shouldn’t go from a monastery with monks to a non-monastery with monks who belong to a different Buddhist sect, except if you go with a sangha or there are dangers. On the invitation day you shouldn’t go from a monastery with monks to a monastery or a non-monastery with monks who belong to a different Buddhist sect, except if you go with a sangha or there are dangers. 

On\marginnote{13.1.79} the invitation day you shouldn’t go from a non-monastery with monks to a monastery with monks who belong to a different Buddhist sect, except if you go with a sangha or there are dangers. On the invitation day you shouldn’t go from a non-monastery with monks to a non-monastery with monks who belong to a different Buddhist sect, except if you go with a sangha or there are dangers. On the invitation day you shouldn’t go from a non-monastery with monks to a monastery or a non-monastery with monks who belong to a different Buddhist sect, except if you go with a sangha or there are dangers. 

On\marginnote{13.1.82} the invitation day you shouldn’t go from a monastery or a non-monastery with monks to a monastery with monks who belong to a different Buddhist sect, except if you go with a sangha or there are dangers. On the invitation day you shouldn’t go from a monastery or a non-monastery with monks to a non-monastery with monks who belong to a different Buddhist sect, except if you go with a sangha or there are dangers. On the invitation day you shouldn’t go from a monastery or a non-monastery with monks to a monastery or a non-monastery with monks who belong to a different Buddhist sect, except if you go with a sangha or there are dangers.” 

\section*{19. The section on “you may go” }

“On\marginnote{13.1.85.1} the invitation day you may go from a monastery with monks to a monastery with monks who belong to the same Buddhist sect if you know you’ll get there on the same day. On the invitation day you may go from a monastery with monks to a non-monastery with monks … to a monastery or a non-monastery with monks who belong to the same Buddhist sect if you know you’ll get there on the same day. 

On\marginnote{13.1.88} the invitation day you may go from a non-monastery with monks to a monastery with monks … to a non-monastery with monks … to a monastery or a non-monastery with monks who belong to the same Buddhist sect if you know you’ll get there on the same day. 

On\marginnote{13.1.91} the invitation day you may go from a monastery or a non-monastery with monks to a monastery with monks … to a non-monastery with monks … to a monastery or a non-monastery with monks who belong to the same Buddhist sect if you know you’ll get there on the same day.” 

\section*{20. The identification of persons to be avoided }

\scrule{“You shouldn’t do the invitation ceremony with a nun seated in the gathering. If you do, you commit an offense of wrong conduct. You shouldn’t do the invitation ceremony with a trainee nun, a novice monk, a novice nun, one who has renounced the training, or one who has committed the worst kind of offense seated in the gathering. If you do, you commit an offense of wrong conduct. }

\scrule{You shouldn’t do the invitation ceremony with one who has been ejected for not recognizing an offense seated in the gathering. If you do, you should be dealt with according to the rule. You shouldn’t do the invitation ceremony with one who has been ejected for not making amends for an offense seated in the gathering or with one who has been ejected for not giving up a bad view seated in the gathering. If you do, you should be dealt with according to the rule. }

\scrule{You shouldn’t do the invitation ceremony with a \textit{\textsanskrit{paṇḍaka}} seated in the gathering. If you do, you commit an offense of wrong conduct. You shouldn’t do the invitation ceremony with a fake monk, with one who has previously left to join the monastics of another religion, with an animal, with a matricide, with a patricide, with a murderer of a perfected one, with one who has raped a nun, with one has caused a schism in the Sangha, with one who has caused the Buddha to bleed, or with a hermaphrodite seated in the gathering. If you do, you commit an offense of wrong conduct. }

\scrule{You shouldn’t do the invitation ceremony with a passed-on invitation that has expired, except if the gathering is still seated together.\footnote{“A passed-on invitation that has expired”, \textit{\textsanskrit{pārivāsikapavāraṇādānena}}, seems to refer to an invitation that was conveyed for a different occasion. So long as the assembly is still seated, the occasion is regarded as the same. See the discussion to Bi Pc 81 in Appendix on Individual \textsanskrit{Bhikkhunī} Rules in volume 3. } }

\scrule{You shouldn’t do the invitation ceremony on a non-invitation day, except to unify the Sangha.” }

\scend{The second section for recitation is finished. }

\section*{21. Invitation ceremonies by means of two statements }

At\marginnote{15.1.1} one time on the invitation day in a certain monastery in the Kosalan country, there was a threat from primitive tribes.\footnote{Sp 3.150: \textit{Savarabhayanti \textsanskrit{aṭavimanussabhayaṁ}}, “\textit{\textsanskrit{Savarabhayaṁ}}: threat from forest people.” } The monks were unable to do the invitation ceremony by means of three statements. 

\scrule{“I allow you to do the invitation ceremony by means of two statements.” }

The\marginnote{15.1.5} threat from primitive tribes increased. The monks were unable to do the invitation ceremony by means of two statements. 

\scrule{“I allow you to do the invitation ceremony by means of one statement.” }

The\marginnote{15.1.9} threat from primitive tribes increased further. The monks were unable to do the invitation ceremony by means of one statement. 

\scrule{“I allow you to do the invitation ceremony in groups according to the year of seniority.” }

On\marginnote{15.2.1} one occasion on the invitation day in a certain monastery, most of the night had been spent with people making offerings. The monks considered this and thought, “If the Sangha does the invitation ceremony by means of three statements, we won’t finish before dawn. What should we do?” 

“In\marginnote{15.3.1} such a case, a competent and capable monk should inform the Sangha: 

‘Please,\marginnote{15.3.5} venerables, I ask the Sangha to listen. Most of the night has been spent with people making offerings. If the Sangha does the invitation ceremony by means of three statements, we won’t finish before dawn. If the Sangha is ready, it should do the invitation ceremony by means of two statements.’ Or, ‘If the Sangha is ready, it should do the invitation ceremony by means of one statement.’ Or, ‘If the Sangha is ready, it should do the invitation ceremony in groups according to the year of seniority.’ 

It\marginnote{15.4.1} may happen on the invitation day that most of the night in a monastery is spent with monks giving teachings, with experts on the discourses reciting discourses, with experts on the Monastic Law discussing the Monastic Law, with expounders of the Teaching discussing the Teaching, or with the monks arguing. If the monks consider this and think, ‘If the Sangha does the invitation ceremony by means of three statements, we won’t finish before dawn,’ then a competent and capable monk should inform the Sangha: 

‘Please,\marginnote{15.4.9} venerables, I ask the Sangha to listen. Most of the night has been spent with the monks arguing. If the Sangha does the invitation ceremony by means of three statements, we won’t finish before dawn. If the Sangha is ready, it should do the invitation ceremony by means of two statements.’ Or, ‘If the Sangha is ready, it should do the invitation ceremony by means of one statement.’ Or, ‘If the Sangha is ready, it should do the invitation ceremony in groups according to the year of seniority.’” 

At\marginnote{15.5.1} one time on the invitation day in a certain monastery in the Kosalan country, a large sangha of monks had gathered. Just then a storm was approaching, but they only had a small sheltered area. The monks considered this and thought, “If the Sangha does the invitation ceremony by means of three statements, we won’t finish before it starts raining. What should we do?” They told the Buddha. 

“In\marginnote{15.6.1} such a case, a competent and capable monk should inform the Sangha: 

‘Please,\marginnote{15.6.6} venerables, I ask the Sangha to listen. This large Sangha of monks has gathered. A storm is approaching, but we only have a small sheltered area. If the Sangha does the invitation ceremony by means of three statements, we won’t finish before it starts raining. If the Sangha is ready, it should do the invitation ceremony by means of two statements.’ Or, ‘If the Sangha is ready, it should do the invitation ceremony by means of one statement.’ Or, ‘If the Sangha is ready, it should do the invitation ceremony in groups according to the year of seniority.’ 

It\marginnote{15.7.1} may happen on the invitation day in a certain monastery that there is a threat from kings, bandits, fire, floods, people, spirits, predatory animals, or creeping animals, or a threat to life, or a threat to the monastic life. If the monks consider this and think, ‘This is a threat to the monastic life. If the Sangha does the invitation ceremony by means of three statements, we won’t finish before the threat manifests,’ then a competent and capable monk should inform the Sangha: 

‘Please,\marginnote{15.7.14} venerables, I ask the Sangha to listen. This is a threat to the monastic life. If the Sangha does the invitation ceremony by means of three statements, we won’t finish before the threat manifests. If the Sangha is ready, it should do the invitation ceremony by means of two statements.’ Or, ‘If the Sangha is ready, it should do the invitation ceremony by means of one statement.’ Or, ‘If the Sangha is ready, it should do the invitation ceremony in groups according to the year of seniority.’” 

\section*{22. The cancellation of the invitation }

At\marginnote{16.1.1} that time the monks from the group of six invited correction while having unconfessed offenses. 

\scrule{“You shouldn’t invite correction if you have unconfessed offenses. If you do, you commit an offense of wrong conduct. If anyone invites correction with an unconfessed offense, you should get their permission and then accuse them of an offense.” }

Soon\marginnote{16.2.1} afterwards, when asked for permission, the monks from the group of six refused to give it. 

\scrule{“If anyone doesn’t give their permission, you should cancel their invitation. And it should be done like this. On the invitation day, whether the fourteenth or the fifteenth, in the midst of the Sangha and in the presence of that person, you should announce: }

‘Please,\marginnote{16.2.6} venerables, I ask the Sangha to listen. Such-and-such a person has an unconfessed offense. I cancel their invitation.\footnote{I use a gender neutral expression since monks are also allowed to cancel the invitation of nuns. } The invitation ceremony shouldn’t be done in their presence.’ 

Their\marginnote{16.2.10} invitation has then been canceled.” 

\subsection*{Improper cancellation of the invitation}

On\marginnote{16.3.1} one occasion the monks from the group of six—thinking to act before the good monks canceled their invitation, but having no reason for doing so—canceled the invitation of pure monks who had not committed any offenses. They also canceled the invitation of those who already had invited. 

\scrule{“When there is no reason for doing so, you shouldn’t cancel the invitation of pure monks who haven’t committed any offenses. If you do, you commit an offense of wrong conduct. And you shouldn’t cancel the invitation of those who already have invited. If you do, you commit an offense of wrong conduct. }

And\marginnote{16.4.1} this is how the invitation is canceled and how it isn’t canceled. 

If\marginnote{16.4.3} the invitation is canceled after a three-statement invitation has been spoken and concluded, then it isn’t canceled. If the invitation is canceled after a two-statement invitation … after a one-statement invitation … after an invitation done in groups according to the year of seniority has been spoken and concluded, then it isn’t canceled. 

If\marginnote{16.5.1} the invitation is canceled when a three-statement invitation hasn’t yet been concluded, then it’s canceled. If the invitation is canceled when a two-statement invitation … when a one-statement invitation … when an invitation done in groups according to the year of seniority hasn’t yet been concluded, then it’s canceled. 

It\marginnote{16.6.1} may happen on the invitation day that a monk cancels a second monk’s invitation. If other monks know about the first monk: ‘This venerable is impure in bodily conduct, verbal conduct, and livelihood; he’s ignorant and incompetent, incapable of answering properly when questioned,’ then they should press him by saying, ‘Enough. No more arguing and disputing,’ and the Sangha should then do the invitation ceremony. 

It\marginnote{16.7.1} may happen on the invitation day that a monk cancels a second monk’s invitation. If other monks know about the first monk: ‘This venerable is pure in bodily conduct, but impure in verbal conduct and livelihood; he’s ignorant and incompetent, incapable of answering properly when questioned,’ then they should press him by saying, ‘Enough. No more arguing and disputing,’ and the Sangha should then do the invitation ceremony. 

It\marginnote{16.8.1} may happen on the invitation day that a monk cancels a second monk’s invitation. If other monks know about the first monk: ‘This venerable is pure in bodily conduct and verbal conduct, but impure in livelihood; he’s ignorant and incompetent, incapable of answering properly when questioned,’ then they should press him by saying, ‘Enough. No more arguing and disputing,’ and the Sangha should then do the invitation ceremony. 

It\marginnote{16.9.1} may happen on the invitation day that a monk cancels a second monk’s invitation. If other monks know about the first monk: ‘This venerable is pure in bodily conduct, verbal conduct, and livelihood; but he’s ignorant and incompetent, incapable of answering properly when questioned,’ then they should press him by saying, ‘Enough. No more arguing and disputing,’ and the Sangha should then do the invitation ceremony.” 

\subsection*{Questioning of the accusing monk}

“It\marginnote{16.10.1} may happen on the invitation day that a monk cancels a second monk’s invitation. If other monks know about the first monk: ‘This venerable is pure in bodily conduct, verbal conduct, and livelihood; he’s knowledgeable and competent, capable of answering properly when questioned,’ then they should say to him, ‘Are you canceling this monk’s invitation because he has failed in morality, in conduct, or in view?’ 

If\marginnote{16.11.1} he says, ‘I’m canceling it because he has failed in morality,’ ‘I’m canceling it because he has failed in conduct,’ or ‘I’m canceling it because he has failed in view,’ he should be asked, ‘Do you know what failure in morality is?’ ‘Do you know what failure in conduct is?’ or ‘Do you know what failure in view is?’ 

If\marginnote{16.11.4} he says, ‘I do,’ he should be asked what they are. 

If\marginnote{16.12.1} he says, ‘The four offenses entailing expulsion and the thirteen entailing suspension are failure in morality,’ ‘The serious offenses, the offenses entailing confession, the offenses entailing acknowledgment, the offenses of wrong conduct, and the offenses of wrong speech are failure in conduct,’ ‘Wrong views and extreme views are failure in view,’ he should be asked, ‘Are you canceling this monk’s invitation because of what you’ve seen, what you’ve heard, or what you suspect?’ 

If\marginnote{16.13.1} he says, ‘I’m canceling it because of what I’ve seen,’ ‘I’m canceling it because of what I’ve heard,’ or ‘I’m canceling it because of what I suspect,’ he should be asked, ‘Since you’re canceling this monk’s invitation because of what you’ve seen, what have you seen? How did you see it? When did you see it? Where did you see it? Did you see him commit an offense entailing expulsion, an offense entailing suspension, a serious offense,  an offense entailing confession, an offense entailing acknowledgment, an offense of wrong conduct, or an offense of wrong speech? Where were you? Where was this monk? What were you doing? What was this monk doing?’ 

If\marginnote{16.14.1} he says, ‘I didn’t cancel this monk’s invitation because of what I’ve seen, but because of what I’ve heard,’ he should be asked, ‘Since you’re canceling this monk’s invitation because of what you’ve heard, what have you heard? How did you hear it? When did you hear it? Where did you hear it? Did you hear that he has committed an offense entailing expulsion, an offense entailing suspension, a serious offense, an offense entailing confession, an offense entailing acknowledgment, an offense of wrong conduct, or an offense of wrong speech? Did you hear it from a monk, a nun, a trainee nun, a novice monk, a novice nun, a male lay follower, or a female lay follower? Or did you hear it from kings, a king’s officials, the monastics of another religion, or the lay followers of another religion?’ 

If\marginnote{16.15.1} he says, ‘I didn’t cancel this monk’s invitation because of what I’ve heard, but because of what I suspect,’ he should be asked, ‘Since you’re canceling this monk’s invitation because of suspicion, what do you suspect? How do you suspect it? When did you suspect it? Where did you suspect it? Do you suspect that he has committed an offense entailing expulsion, an offense entailing suspension, a serious offense, an offense entailing confession, an offense entailing acknowledgment, an offense of wrong conduct, or an offense of wrong speech? Do you suspect it after hearing about it from a monk, a nun, a trainee nun, a novice monk, a novice nun, a male lay follower, or a female lay follower? Or do you suspect it after hearing about it from kings, a king’s officials, the monastics of another religion, or the lay followers of another religion?’ 

He\marginnote{16.16.1} might say,\footnote{\textit{Ce}, “if”, does not fit in the current context, since there is no main clause corresponding to the conditional clause. Perhaps this is an ancient mistake, whereby the \textit{ce} has been added on the pattern of the similar phrases above. I translate as if the \textit{ce} is not there. } ‘I didn’t cancel this monk’s invitation because of what I suspect. I don’t know why I canceled his invitation.’ 

If\marginnote{16.16.3} the accusing monk, when questioned, isn’t able to satisfy his discerning fellow monastics, they should conclude, ‘The accused monk is improperly accused.’ But if the accusing monk, when questioned, is able to satisfy his discerning fellow monastics, they should conclude, ‘The accused monk is properly accused.’\footnote{Vin-vn-\textsanskrit{ṭ} 2777: \textit{\textsanskrit{Sānuvādoti} ettha \textsanskrit{anuvādo} \textsanskrit{nāma} \textsanskrit{codanā}, saha \textsanskrit{anuvādena} \textsanskrit{vattatīti} \textsanskrit{sānuvādo}}, “\textit{\textsanskrit{Sānuvādo}}: here accusing is called \textit{\textsanskrit{anuvādo}}. \textit{\textsanskrit{Sānuvādo}} means he proceeds with an accusation.” } 

If\marginnote{16.17.1} the accusing monk admits to a groundless charge of an offense entailing expulsion, he should be charged with an offense entailing suspension. The Sangha should then do the invitation ceremony. If the accusing monk admits to a groundless charge of an offense entailing suspension, he should be dealt with according to the rule. The Sangha should then do the invitation ceremony. If the accusing monk admits to a groundless charge of a serious offense, an offense entailing confession, an offense entailing acknowledgment, an offense of wrong conduct, or an offense of wrong speech, he should be dealt with according to the rule. The Sangha should then do the invitation ceremony. 

If\marginnote{16.18.1} the accused monk admits to having committed an offense entailing expulsion, he should be expelled. The Sangha should then do the invitation ceremony. If the accused monk admits to having committed an offense entailing suspension, he should be charged with that offense. The Sangha should then do the invitation ceremony. If the accused monk admits to having committed a serious offense, an offense entailing confession, an offense entailing acknowledgment, an offense of wrong conduct, or an offense of wrong speech, he should be dealt with according to the rule. The Sangha should then do the invitation ceremony.” 

\section*{23. Grounds for a serious offense, etc. }

“On\marginnote{16.19.1} the invitation day, a monk may have committed a serious offense. Some monks regard it as a serious offense, but others as an offense entailing suspension. The monks who regard it as a serious offense should take that monk aside and deal with him according to the rule. They should then approach the Sangha and say: 

‘This\marginnote{16.19.4} monk has made amends for the offense he has committed. If the Sangha is ready, it should do the invitation ceremony.’ 

On\marginnote{16.20.1} the invitation day, a monk may have committed a serious offense. Some monks regard it as a serious offense, but others as an offense entailing confession. … Some monks regard it as a serious offense, but others as an offense entailing acknowledgment. … Some monks regard it as a serious offense, but others as an offense of wrong conduct. … Some monks regard it as a serious offense, but others as an offense of wrong speech. The monks who regard it as a serious offense should take that monk aside and deal with him according to the rule. They should then approach the Sangha and say: 

‘This\marginnote{16.20.7} monk has made amends for the offense he has committed. If the Sangha is ready, it should do the invitation ceremony.’ 

On\marginnote{16.21.1} the invitation day, a monk may have committed an offense entailing confession. … an offense entailing acknowledgment. … an offense of wrong conduct. … an offense of wrong speech. Some monks regard it as an offense of wrong speech, but others as an offense entailing suspension. The monks who regard it as an offense of wrong speech should take that monk aside and deal with him according to the rule. They should then approach the Sangha and say: 

‘This\marginnote{16.21.7} monk has made amends for the offense he has committed. If the Sangha is ready, it should do the invitation ceremony.’ 

On\marginnote{16.22.1} the invitation day, a monk may have committed an offense of wrong speech. Some monks regard it as an offense of wrong speech, but others as a serious offense. … Some monks regard it as an offense of wrong speech, but others as an offense entailing confession. … Some monks regard it as an offense of wrong speech, but others as an offense entailing acknowledgment. … Some monks regard it as an offense of wrong speech, but others as an offense of wrong conduct. The monks who regard it as an offense of wrong speech should take that monk aside and deal with him according to the rule. They should then approach the Sangha and say: 

‘This\marginnote{16.22.7} monk has made amends for the offense he has committed. If the Sangha is ready, it should do the invitation ceremony.’” 

\section*{24. Setting aside an offense, etc. }

“It\marginnote{16.23.1} may happen on the invitation day that a monk announces in the midst of the Sangha: 

‘Please,\marginnote{16.23.2} venerables, I ask the Sangha to listen. I know about an offense, but not who the offender is.\footnote{Sp 3.239: \textit{\textsanskrit{Idaṁ} vatthu \textsanskrit{paññāyati} na puggaloti ettha \textsanskrit{corā} kira \textsanskrit{araññavihāre} \textsanskrit{pokkharaṇito} macche \textsanskrit{gahetvā} \textsanskrit{pacitvā} \textsanskrit{khāditvā} \textsanskrit{agamaṁsu}. So \textsanskrit{taṁ} \textsanskrit{vippakāraṁ} \textsanskrit{disvā} \textsanskrit{ārāme} \textsanskrit{vā} \textsanskrit{kiñci} dhuttena \textsanskrit{kataṁ} \textsanskrit{vippakāraṁ} \textsanskrit{disvā} “bhikkhussa \textsanskrit{iminā} kammena bhavitabban”ti \textsanskrit{sallakkhetvā} \textsanskrit{evamāha}}, “I know about an offense, but not who the offender is: here, criminals catch fish from a pond near a forest monastery. They then cook it, eat it, and leave. Having seen the disturbance or having seen whatever disturbance was done by the scoundrels in the monastery, having reflected, ‘This action may have been done by a monk,’ he says thus.” } If the Sangha is ready, it should set aside the offense and then do the invitation ceremony.’\footnote{Sp 3.239: \textit{\textsanskrit{Vatthuṁ} \textsanskrit{ṭhapetvā} \textsanskrit{saṅgho} \textsanskrit{pavāreyyāti} “\textsanskrit{yadā} \textsanskrit{taṁ} \textsanskrit{puggalaṁ} \textsanskrit{jānissāma}, \textsanskrit{tadā} \textsanskrit{naṁ} \textsanskrit{codessāma}. \textsanskrit{Idāni} pana \textsanskrit{saṅgho} \textsanskrit{pavāretū}”ti ayamettha attho,} “\textit{\textsanskrit{Vatthuṁ} \textsanskrit{ṭhapetvā} \textsanskrit{saṅgho} \textsanskrit{pavāreyya}}: this is the meaning here: when we discover that person, we will accuse him. But now the Sangha should do the invitation ceremony.” } 

They\marginnote{16.23.5} should say to him, ‘The Buddha has laid down a rule that the invitation ceremony is for monks who are pure. If you know the offense, but not the offender, then say now who it is that you suspect.’\footnote{Sp 3.239: \textit{\textsanskrit{Idāneva} \textsanskrit{naṁ} \textsanskrit{vadehīti} sace \textsanskrit{iminā} \textsanskrit{vatthunā} \textsanskrit{kañci} \textsanskrit{puggalaṁ} \textsanskrit{parisaṅkasi}, \textsanskrit{idāneva} \textsanskrit{naṁ} \textsanskrit{apadisāhīti} attho}, “\textit{\textsanskrit{Idāneva} \textsanskrit{naṁ} vadehi}: the meaning is: if you suspect someone to have committed this offense, then indicate who it is now.” } 

It\marginnote{16.24.1} may happen on the invitation day that a monk announces in the midst of the Sangha: 

‘Please,\marginnote{16.24.2} venerables, I ask the Sangha to listen. I know of an offender, but not what the offense is. If the Sangha is ready, it should do the invitation ceremony without the offender.’ 

They\marginnote{16.24.5} should say to him, ‘The Buddha has laid down a rule that the invitation ceremony should be done in a complete assembly. If you know the offender, but not the offense, then say now what it is that you suspect.’\footnote{Sp 3.239: \textit{\textsanskrit{Idāneva} \textsanskrit{naṁ} \textsanskrit{vadehīti} \textsanskrit{yaṁ} \textsanskrit{tvaṁ} \textsanskrit{puggalaṁ} \textsanskrit{ṭhapesi}, tassa puggalassa \textsanskrit{idāneva} \textsanskrit{dosaṁ} vada}, “\textit{\textsanskrit{Idāneva} \textsanskrit{naṁ} vadehi}: say now the fault of the person of whom you are canceling the invitation.” } 

It\marginnote{16.25.1} may happen on the invitation day that a monk announces in the midst of the Sangha: 

‘Please,\marginnote{16.25.2} venerables, I ask the Sangha to listen. I know of an offender and his offense. If the Sangha is ready, it should set aside the offense and then do the invitation ceremony without the offender.’ 

They\marginnote{16.25.5} should say to him, ‘The Buddha has laid down a rule that the invitation ceremony should be done in a complete assembly by monks who are pure. If you know an offender and his offense, then say now what they are.’ 

If\marginnote{16.26.1} the offense is known about before the invitation ceremony, but the offender only afterwards, the offender should be corrected. If the offender is known about before the invitation ceremony, but the offense only afterwards, the offender should be corrected. If both the offense and the offender are known about before the invitation ceremony, and someone reopens the case after the invitation ceremony has been done, he commits an offense entailing confession for the reopening.” 

\section*{25. Creators of quarrels, etc. }

At\marginnote{17.1.1} one time in a certain monastery in the Kosalan country, a number of monks who were friends had entered the rainy-season residence together. Other monks who were quarrelsome, argumentative, and creators of legal issues in the Sangha had entered the rains residence nearby. They said to one another, “At the invitation ceremony, when those monks have completed the rains residence, we’ll cancel their invitation.” The monks who were friends heard about this and wondered what to do. They told the Buddha. 

\scrule{“In such a case, I allow those monks to do two or three observance-day ceremonies on the fourteenth day, with the aim of having their invitation ceremony before the other monks. }

If\marginnote{17.2.6} those quarrelsome and argumentative monks are on their way to the other monastery, the resident monks should gather quickly and do the invitation ceremony. When it has been done, they should say to the other monks, ‘Venerables, we have completed the invitation ceremony. Please do as you see fit.’ 

If\marginnote{17.3.1} those quarrelsome and argumentative monks arrive without prior notice, the resident monks should prepare seats and set out a foot stool, a foot scraper, and water for washing the feet. They should then go out to meet those monks, receive their bowls and robes, and ask if they want water to drink. Then, having distracted them, they should go outside the monastery zone and do the invitation ceremony there. When it has been done, they should say to the other monks, ‘Venerables, we have completed the invitation ceremony. Please do as you see fit.’ 

If\marginnote{17.4.1} they’re able to do this, it’s good. If not, then a resident monk who is competent and capable should inform the resident monks: 

‘Please,\marginnote{17.4.3} venerables, I ask the resident monks to listen. If the venerables are ready, we’ll now do the observance-day ceremony and recite the Monastic Code, and we’ll do the invitation ceremony during the next waning phase of the moon.’ 

If\marginnote{17.4.5} the quarrelsome and argumentative monks say, ‘Please do the invitation ceremony with us now,’ they should be told, ‘You have no authority over our invitation ceremony. We won’t do the procedure for the time being.’ 

If\marginnote{17.5.1} the quarrelsome and argumentative monks stay on until the new moon, then a resident monk who is competent and capable should inform the resident monks: 

‘Please,\marginnote{17.5.2} venerables, I ask the resident monks to listen. If the venerables are ready, we’ll now do the observance-day ceremony and recite the Monastic Code, and we’ll do the invitation ceremony during the next waxing phase of the moon.’ 

If\marginnote{17.5.4} the quarrelsome and argumentative monks say, ‘Please do the invitation ceremony with us now,’ they should be told, ‘You have no authority over our invitation ceremony. We won’t do the procedure for the time being.’ 

If\marginnote{17.6.1} the quarrelsome and argumentative monks stay on until the next full moon, then all the monks have no choice but to do the invitation ceremony on the day of \textsanskrit{Komudī}, the fourth full moon of the rainy season.” 

\subsection*{Invitation with the sick}

“If,\marginnote{17.7.1} while you’re doing the invitation ceremony, a sick monk cancels the invitation of a healthy monk, you should tell him, ‘You’re sick. The Buddha has said that a sick monk can’t endure being questioned. Please wait until you’re healthy. If you then wish, you may accuse him.’ If, in spite of this, he still accuses the other, he commits an offense entailing confession for disrespect. 

If,\marginnote{17.8.1} while you’re doing the invitation ceremony, a healthy monk cancels the invitation of a sick monk, you should tell him, ‘This monk is sick. The Buddha has said that a sick monk can’t endure being questioned. Please wait until he’s healthy. If you then wish, you may accuse him.’ If, in spite of this, he still accuses the other, he commits an offense entailing confession for disrespect. 

If,\marginnote{17.9.1} while you’re doing the invitation ceremony, a sick monk cancels the invitation of a sick monk, you should tell him, ‘You’re both sick. The Buddha has said that a sick monk can’t endure being questioned. Please wait until you’re both healthy. If you then wish, you may accuse him.’ If, in spite of this, he still accuses the other, he commits an offense entailing confession for disrespect. 

If,\marginnote{17.10.1} while you’re doing the invitation ceremony, a healthy monk cancels the invitation of a healthy monk, you should question and examine both and deal with them according to the rule. The Sangha should then continue the invitation ceremony.” 

\section*{26. Agreements about the invitation ceremony }

On\marginnote{18.1.1} one occasion in a certain monastery in the Kosalan country, a number of monks who were friends had entered the rainy-season residence together. While living together in peace and harmony, they were experiencing deep meditation.\footnote{Sp 3.241: \textit{\textsanskrit{Aññataro} \textsanskrit{phāsuvihāroti} \textsanskrit{taruṇasamatho} \textsanskrit{vā} \textsanskrit{taruṇavipassanā} \textsanskrit{vā}}, “\textit{\textsanskrit{Aññataro} \textsanskrit{phāsuvihāro}} means the early stages of stillness or clear seeing.” } They considered this and thought, “If we do the invitation ceremony now, the monks might set out wandering. We’ll then lose this deep meditation. So what should we do?” They told the Buddha. 

\scrule{“In such a case, I allow those monks to make an agreement about the invitation ceremony. }

And\marginnote{18.3.1} it should be made like this. Everyone should gather in one place. A competent and capable monk should then inform the Sangha: 

‘Please,\marginnote{18.3.4} venerables, I ask the Sangha to listen. While living together in peace and harmony, we’re experiencing deep meditation. If we do the invitation ceremony now, the monks might set out wandering. We’ll then lose this deep meditation. If the Sangha is ready, it should make an agreement about the invitation ceremony: we’ll now do the observance-day ceremony and recite the Monastic Code, and we’ll do the invitation ceremony on the day of \textsanskrit{Komudī}, the fourth full moon of the rainy season. This is the motion. 

Please,\marginnote{18.4.1} venerables, I ask the Sangha to listen. While living together in peace and harmony, we’re experiencing deep meditation. If we do the invitation ceremony now, the monks might set out wandering. We’ll then lose this deep meditation. The Sangha makes an agreement about the invitation ceremony: we’ll now do the observance-day ceremony and recite the Monastic Code, and we’ll do the invitation ceremony on the day of \textsanskrit{Komudī}, the fourth full moon of the rainy season. Any monk who approves of making this agreement about the invitation ceremony—that we’ll now do the observance-day ceremony and recite the Monastic Code and that we’ll do the invitation ceremony on the day of \textsanskrit{Komudī}, the fourth full moon of the rainy season—should remain silent. Any monk who doesn’t approve should speak up. 

The\marginnote{18.4.8} Sangha has made an agreement about the invitation ceremony: we’ll now do the observance-day ceremony and recite the Monastic Code, and we’ll do the invitation ceremony on the day of \textsanskrit{Komudī}, the fourth full moon of the rainy season. The Sangha approves and is therefore silent. I’ll remember it thus.’ 

If,\marginnote{18.5.1} when those monks have made an agreement about the invitation ceremony, a monk says, ‘I wish to go wandering in the country; I have business there,’ they should tell him, ‘That’s fine, but you have to do the invitation first.’ 

If,\marginnote{18.5.5} while that monk is doing the invitation, he cancels the invitation of another monk, the other monk should tell him, ‘You have no authority over my invitation until I invite.’ 

If,\marginnote{18.5.7} while that monk is doing the invitation, another monk cancels his invitation, the monks should question and examine both and deal with them according to the rule. 

If\marginnote{18.6.1} that monk finishes his business in the country and returns to that monastery before the full-moon day of \textsanskrit{Komudī}, and if, while the monks are doing the invitation ceremony, a monk cancels the invitation of the monk who has returned, the monk who has returned should tell him, ‘You have no authority over my invitation; I’ve already done it.’ 

If,\marginnote{18.6.4} while the monks are doing the invitation ceremony, the monk who has returned cancels the invitation of another monk, the monks should question and examine both and deal with them according to the rule. The Sangha should then continue the invitation ceremony.” 

\scendsutta{The fourth chapter on the invitation ceremony is finished. }

\scuddanaintro{This is the summary: }

\begin{scuddana}%
“Completed\marginnote{18.6.7} rains residence in Kosala, \\
They went to see the Teacher; \\
Living uncomfortably like animals, \\
One another in the proper way. 

Inviting,\marginnote{18.6.11} and in the seat,\footnote{\textit{\textsanskrit{Paṇāma}} does not refer directly to anything in the text, and so it may be a corruption. I follow the reading of SRT: \textit{\textsanskrit{āsane}}, “in the seat”. } \\
Legal procedure, sick, relatives; \\
King, and bandits, and scoundrels, \\
So enemies of monks. 

Five,\marginnote{18.6.15} four, three, two, one, \\
Committed, unsure, he remembered; \\
The whole Sangha, unsure, \\
Greater, and equal, smaller. 

Residents,\marginnote{18.6.19} fourteenth, \\
Characteristics, belonging to a Buddhist sect, both; \\
May go, not with seated, \\
About giving consent, invitation. 

With\marginnote{18.6.23} primitive tribes, spent, storm, \\
And threat, invitation; \\
They refused, before their, \\
And isn’t canceled, a monk’s. 

‘Or\marginnote{18.6.27} why’, and what, \\
Because of the seen, the heard, the suspected; \\
The accuser, and the accused, \\
Serious offense, offense, quarrel; \\
And agreement about the invitation, \\
One without authority, should invite.” 

%
\end{scuddana}

\scend{In this chapter there are forty-six topics. }

\scendsutta{The chapter on the invitation ceremony is finished. }

%
\chapter*{{\suttatitleacronym Kd 5}{\suttatitletranslation The chapter on skins }{\suttatitleroot Cammakkhandhaka}}
\addcontentsline{toc}{chapter}{\tocacronym{Kd 5} \toctranslation{The chapter on skins } \tocroot{Cammakkhandhaka}}
\markboth{The chapter on skins }{Cammakkhandhaka}
\extramarks{Kd 5}{Kd 5}

\section*{1. The account of \textsanskrit{Soṇa} \textsanskrit{Koḷivisa} }

At\marginnote{1.1.1} one time the Buddha was staying on the Vulture Peak at \textsanskrit{Rājagaha}. At that time King Seniya \textsanskrit{Bimbisāra} of Magadha ruled over eighty thousand villages, and at \textsanskrit{Campā} there was a wealthy merchant who had a son called \textsanskrit{Soṇa} \textsanskrit{Koḷivisa}. He had been raised in great comfort, so much so that he had hairs growing on the soles of his feet. 

On\marginnote{1.1.5} one occasion, King \textsanskrit{Bimbisāra} had the chiefs of those eighty thousand villages gathered because of some business. He then sent a message to \textsanskrit{Soṇa}, asking him to come. \textsanskrit{Soṇa}’s parents said to him, “\textsanskrit{Soṇa}, the king wishes to see your feet, but don’t point them at him. If you just sit down cross-legged in front him, he’ll be able to see them.” They then sent him away on a palanquin, and \textsanskrit{Soṇa} went to King \textsanskrit{Bimbisāra}. Upon arrival, he bowed to the king and sat down cross-legged in front of him. The king saw the hairs growing on the soles of his feet. 

Then,\marginnote{1.3.1} after instructing those eighty thousand chiefs in worldly matters, the king dismissed them, saying, “I’ve instructed you in worldly matters. Now go and visit the Buddha. He will instruct us about the afterlife.” 

Those\marginnote{1.3.5} village chiefs then went to the Vulture Peak. There they approached Venerable \textsanskrit{Sāgata}, who at that time was the Buddha’s attendant. They said to him, “Venerable, these eighty thousand chiefs have come to visit the Buddha. May we please see him?” 

“Well\marginnote{1.4.5} then, please wait here for a moment, while I announce you to the Buddha.” 

Then,\marginnote{1.5.1} while those village chiefs were watching, he sunk into the stone slab he was standing on and emerged in front of the Buddha. He said to the Buddha, “Sir, eighty thousand village chiefs have come to visit you. What would you like to do?” 

“Well\marginnote{1.5.4} then, \textsanskrit{Sāgata}, prepare a seat in the shade of the dwelling.” 

“Yes,\marginnote{1.6.1} sir.” 

He\marginnote{1.6.2} took a bench, sunk down in front of the Buddha, and as those village chiefs were watching, he once more emerged from that stone slab. He then prepared a seat in the shade of the dwelling, after which the Buddha came out and sat down. Those eighty thousand chiefs approached the Buddha, bowed, and sat down. But they were preoccupied with \textsanskrit{Sāgata}, not with the Buddha. 

After\marginnote{1.7.3} reading their minds, the Buddha said to \textsanskrit{Sāgata}, “Well then, \textsanskrit{Sāgata}, show us more superhuman abilities, more wonders of supernormal power.” 

Saying,\marginnote{1.7.5} “Yes, sir,” he rose up in the air, walked back and forth in space, and he stood, sat down, and lay down there. He emitted smoke and fire, and then disappeared. After this display of supernormal powers, he bowed down at the feet of the Buddha, and said, “Sir, you’re my teacher, and I’m your disciple.” Those eighty thousand chiefs thought, “It’s astonishing and amazing that even a disciple should be so powerful and mighty. Imagine what the teacher must be like!” Now they paid attention to the Buddha, not to \textsanskrit{Sāgata}. 

Having\marginnote{1.9.1} read their minds, the Buddha gave those eighty thousand chiefs a progressive talk—on generosity, morality, and heaven; on the downside, degradation, and defilement of worldly pleasures; and he revealed the benefits of renunciation. When the Buddha knew that their minds were ready, supple, without hindrances, joyful, and confident, he revealed the teaching unique to the Buddhas: suffering, its origin, its end, and the path. Just as a clean and stainless cloth absorbs dye properly, so too, while they were sitting right there, those eighty thousand village chiefs experienced the stainless vision of the Truth: “Anything that has a beginning has an end.” 

They\marginnote{1.10.1} had seen the Truth, had reached, understood, and penetrated it. They had gone beyond doubt and uncertainty, had attained to confidence, and had become independent of others in the Teacher’s instruction. They then said to the Buddha, “Wonderful, sir, wonderful! Just as one might set upright what’s overturned, or reveal what’s hidden, or show the way to one who’s lost, or bring a lamp into the darkness so that those with eyes might see what’s there—just so has the Buddha made the Teaching clear in many ways. We go for refuge to the Buddha, the Teaching, and the Sangha of monks. Please accept us as lay followers who have gone for refuge for life.” 

\subsection*{The going forth of \textsanskrit{Soṇa} \textsanskrit{Koḷivisa} }

But\marginnote{1.11.1} \textsanskrit{Soṇa} thought, “The way I understand the Buddha’s Teaching, it’s not easy for one who lives at home to lead the spiritual life perfectly complete and pure as a polished conch shell. Why don’t I cut off my hair and beard, put on the ocher robes, and go forth into homelessness?” 

When\marginnote{1.11.4} those eighty-four thousand chiefs had rejoiced and expressed their appreciation for the Buddha’s teaching, they got up from their seats, bowed down, circumambulated the Buddha with their right sides toward him, and left. 

Soon\marginnote{1.12.1} after they had left, \textsanskrit{Soṇa} approached the Buddha, bowed, sat down, and said, “Sir, the way I understand the Buddha’s Teaching, it’s not easy for one who lives at home to lead the spiritual life perfectly complete and pure as a polished conch shell. I want to cut off my hair and beard, put on the ocher robes, and go forth into homelessness. Please give me the going forth.” \textsanskrit{Soṇa} received the going forth and the full ordination in the Buddha’s presence. 

Soon\marginnote{1.12.7} after his ordination, while staying in Cool Grove, Venerable \textsanskrit{Soṇa} practiced walking meditation with so much energy that his feet cracked. His walking path became covered in blood, like a slaughterhouse. Then, while reflecting in private, he thought, “I’m one of the Buddha’s energetic disciples, yet my mind isn’t freed from the corruptions through letting go. But my family is wealthy. Why don’t I return to the lower life, enjoy wealth, and make merit?” 

Just\marginnote{1.14.1} then the Buddha read \textsanskrit{Soṇa}’s mind. And, as a strong man might bend or stretch his arm, the Buddha disappeared from the Vulture Peak and appeared in Cool Grove. 

Soon\marginnote{1.14.3} afterwards as the Buddha and a number of monks were walking about the dwellings, they came to \textsanskrit{Soṇa}’s walking path. The Buddha looked at it and asked the monks, “Whose walking path is this? It’s covered in blood, like a slaughterhouse.” They told him what had happened. 

The\marginnote{1.15.1} Buddha then went up to \textsanskrit{Soṇa}’s dwelling and sat down on the prepared seat. \textsanskrit{Soṇa} bowed and sat down, and the Buddha said to him, “\textsanskrit{Soṇa}, while reflecting in private, didn’t you think, ‘I’m one of the Buddha’s energetic disciples, yet my mind isn’t freed from the corruptions through letting go. But my family is wealthy. Why don’t I return to the lower life, enjoy wealth, and make merit’?” 

“Yes,\marginnote{1.15.10} sir.” 

“Well,\marginnote{1.15.11} let me ask you, \textsanskrit{Soṇa}: when you were previously a householder, weren’t you a skilled lute player?” 

“Yes.”\marginnote{1.15.12} 

“When\marginnote{1.15.13} the strings were too tight, was the lute in tune and easy to play?” 

“No.”\marginnote{1.15.14} 

“When\marginnote{1.16.1} the strings were too loose, was the lute in tune and easy to play?” 

“No.”\marginnote{1.16.2} 

“But\marginnote{1.16.3} when the strings were neither too tight nor too loose, but set to a balanced tension, was the lute then in tune and easy to play?” 

“Yes.”\marginnote{1.16.4} 

“Just\marginnote{1.16.5} so, \textsanskrit{Soṇa}, too much energy leads to restlessness and too little to laziness. So apply a balanced energy and bring about an evenness in the spiritual faculties. And that is where you should take up the meditation object.” 

“Yes,\marginnote{1.17.2} sir.” 

Then,\marginnote{1.17.3} as a strong man might bend or stretch his arm, the Buddha disappeared from the presence of \textsanskrit{Soṇa} in Cool Grove and appeared on the Vulture Peak. 

Soon\marginnote{1.18.1} \textsanskrit{Soṇa} applied a balanced energy and brought about an evenness in his spiritual faculties, which is where he took up his meditation object. He then stayed by himself, secluded, heedful, energetic, and diligent. In no long time in this very life, he realized with his own insight the supreme goal of the spiritual life for which gentlemen rightly go forth into homelessness. He understood that birth had come to an end, that the spiritual life had been fulfilled, that the job had been done, that there was no further state of existence. And Venerable \textsanskrit{Soṇa} became one of the perfected ones. 

He\marginnote{1.19.1} then thought, “Why don’t I declare perfect insight to the Buddha?” He then went to the Buddha, bowed, sat down, and said: 

“Sir,\marginnote{1.20.1} a monk who is a perfected one—who has ended the corruptions, fulfilled the spiritual life, done the job, put down the burden, realized the true goal, cut the bond to existence, gained release by right insight—he is committed to six things: to renunciation, seclusion, harmlessness, the end of grasping, the end of craving, and non-confusion. 

A\marginnote{1.21.1} venerable here might think, ‘No doubt this venerable is committed to renunciation simply because of faith.’ But this would be the wrong way to look at it. The monk who has ended the corruptions, who has fulfilled the spiritual life and done the job, doesn’t see anything to be done in himself, nor anything that needs improving. He is committed to renunciation because of the ending of sensual desire, because he is without sensual desire. He is committed to renunciation because of the ending of ill will, because he is without ill will. He is committed to renunciation because of the ending of confusion, because he is without confusion. 

A\marginnote{1.22.1} venerable here might think, ‘No doubt this venerable is committed to seclusion because he desires material support, honor, and praise.’ But this would be the wrong way to look at it. The monk who has ended the corruptions, who has fulfilled the spiritual life and done the job, doesn’t see anything to be done in himself, nor anything that needs improving. He is committed to seclusion because of the ending of sensual desire, because he is without sensual desire. He is committed to seclusion because of the ending of ill will, because he is without ill will. He is committed to seclusion because of the ending of confusion, because he is without confusion. 

A\marginnote{1.23.1} venerable here might think, ‘No doubt this venerable is committed to non-harming because he falls back on adhering to virtue and vows as the essence.’ But this would be the wrong way to look at it. The monk who has ended the corruptions, who has fulfilled the spiritual life and done the job, doesn’t see anything to be done in himself, nor anything that needs improving. He is committed to harmlessness because of the ending of sensual desire, because he is without sensual desire. He is committed to harmlessness because of the ending of ill will, because he is without ill will. He is committed to harmlessness because of the ending of confusion, because he is without confusion. 

He\marginnote{1.24.1} is committed to the end of grasping, to the end of craving, and to non-confusion because of the ending of sensual desire, because he is without sensual desire. 

He\marginnote{1.24.2} is committed to the end of grasping, to the end of craving, and to non-confusion because of the ending of ill will, because he is without ill will. 

He\marginnote{1.24.3} is committed to the end of grasping, to the end of craving, and to non-confusion because of the ending of confusion, because he is without confusion. 

Sir,\marginnote{1.25.1} for a monk who is fully freed in this way, even if he sees compelling sights, his mind is not overpowered by them. It remains unaffected, steady, and unshakeable, and he observes its disappearance. Even if he hears compelling sounds, smells compelling odors, tastes compelling flavors, touches compelling objects, or experiences compelling mental phenomena, his mind is not overpowered by them. It remains unaffected, steady, and unshakeable, and he observes its disappearance. 

It’s\marginnote{1.26.1} just like a granite mountain, a single, solid mass without cracks. It doesn’t shake or tremble when a powerful rainstorm arrives from any direction. The mind of the monk who is fully freed in this way is just like that. 

\begin{verse}%
For\marginnote{1.27.1} one committed to renunciation \\
And to seclusion of the mind, \\
For one committed to harmlessness \\
And to the end of grasping, 

For\marginnote{1.27.5} one committed to the end of craving \\
And to clarity of mind, \\
Having seen the arising of the senses, \\
Their mind is fully freed. 

For\marginnote{1.27.9} one who is fully freed, \\
The monastic with a peaceful mind, \\
There is nothing to improve \\
And nothing to be done. 

Just\marginnote{1.27.13} as a single, solid rock, \\
Is unshaken by the wind, \\
So too, all sights, and sounds, \\
Smells, tastes, and touches, 

And\marginnote{1.27.17} mental objects, good or bad, \\
Cannot move that kind of person. \\
Their mind is free and steady, \\
And they observe it as it disappears.” 

%
\end{verse}

\section*{2. The prohibition against sandals with double-layered soles, etc. }

Then\marginnote{1.28.1} the Buddha addressed the monks: “It’s in this way that a gentleman declares perfect insight. The matter is spoken of, but the person isn’t mentioned. Still some foolish men here seem to declare perfect insight just for fun. Soon enough they experience distress.” 

The\marginnote{1.29.1} Buddha then said to \textsanskrit{Soṇa}, “\textsanskrit{Soṇa}, you were brought up in great comfort. I allow you to use sandals with single-layered soles.” 

“When\marginnote{1.29.4} I went forth into homelessness, sir, I left behind eighty cartloads of gold coins and a troop of seven elephants.\footnote{“Gold coins” renders \textit{\textsanskrit{hirañña}}. See Appendix of Technical Terms. } If I were to walk around in sandals with single-layered soles, some people would say, ‘\textsanskrit{Soṇa} left all this behind when he went forth, and now he’s attached to sandals with single-layered soles.’ If you allow them to the Sangha of monks, I too will use them. If not, I won’t use them either.” The Buddha then gave a teaching and addressed the monks: 

\scrule{“I allow sandals with single-layered soles. But you shouldn’t wear sandals with double-layered soles, with triple-layered soles, or with multi-layered soles. If you do, you commit an offense of wrong conduct.” }

\section*{3. The prohibition against what is entirely blue, etc. }

Soon\marginnote{2.1.1} afterwards the monks from the group of six wore entirely blue sandals, entirely yellow sandals, entirely red sandals, entirely magenta sandals, entirely black sandals, entirely orange sandals, and entirely beige sandals. People complained and criticized them, “They’re just like householders who indulge in worldly pleasures!” They told the Buddha. 

\scrule{“You shouldn’t wear sandals that are entirely blue, entirely yellow, entirely red, entirely magenta, entirely black, entirely orange, or entirely beige.\footnote{According to SED, the \textit{\textsanskrit{mahāraṅga}} (sv. \textit{\textsanskrit{mahārajana}}) is the safflower, which is normally deep yellow or orange. Sp 3.246: \textit{\textsanskrit{Mahānāmarattā} \textsanskrit{sambhinnavaṇṇā} hoti \textsanskrit{paṇḍupalāsavaṇṇā}}, “\textit{\textsanskrit{Mahānāmaratta}} is a mixed color, the color of withered leaves.” } If you do, you commit an offense of wrong conduct.” }

At\marginnote{2.2.1} that time the monks from the group of six wore sandals with blue straps, yellow straps, red straps, magenta straps, black straps, orange straps, and beige straps. People complained and criticized them, “They’re just like householders who indulge in worldly pleasures!” 

\scrule{“You shouldn’t wear sandals with blue straps, yellow straps, red straps, magenta straps, black straps, orange straps, or beige straps. If you do, you commit an offense of wrong conduct.” }

At\marginnote{2.3.1} that time the monks from the group of six wore sandals containing leather, enclosing the shin and the foot, covering the foot, stuffed with cotton, looking like partridge feathers, having straps like ram horns, having straps like goat horns, having straps like scorpion claws, decorated with a peacock’s tail feather, and decorated in various ways. People complained and criticized them, “They’re just like householders who indulge in worldly pleasures!” 

\scrule{“You shouldn’t wear sandals containing leather,\footnote{Sp 3.246: \textit{\textsanskrit{Khallakabaddhāti} \textsanskrit{paṇhipidhānatthaṁ} tale \textsanskrit{khallakaṁ} \textsanskrit{bandhitvā} \textsanskrit{katā}}, “\textit{Khallakabaddha}: they are made by fastening leather at the sole for the purpose of covering the heel.” Vmv 3.246 adds: \textit{Khallakanti \textsanskrit{sabbapaṇhipidhānacammaṁ}}, “\textit{Khallaka}: a skin to cover the entire heel.” } enclosing the shin and the foot, covering the foot, stuffed with cotton, looking like partridge feathers, having straps like ram horns, having straps like goat horns, having straps like scorpion claws, decorated with a peacock’s tail feather, or decorated in various ways.\footnote{The various kinds of footwear listed here are explained as follows in the commentary. Sp 3.246: \textit{\textsanskrit{Puṭabaddhāti} \textsanskrit{yonakaupāhanā} vuccati, \textsanskrit{yā} \textsanskrit{yāvajaṅghato} \textsanskrit{sabbapādaṁ} \textsanskrit{paṭicchādeti}}, “Greek sandals are called \textit{\textsanskrit{puṭabaddha}}: whatever covers the entire foot as far as the shin.” Sp 3.246: \textit{\textsanskrit{Pāliguṇṭhimāti} \textsanskrit{paliguṇṭhitvā} \textsanskrit{katā}; \textsanskrit{yā} upari \textsanskrit{pādamattameva} \textsanskrit{paṭicchādeti}, na \textsanskrit{jaṅghaṁ}}, “\textit{\textsanskrit{Pāliguṇṭhima}}: they are made by covering: whatever covers merely the top of the foot, but not the shin.” Sp 3.246: \textit{\textsanskrit{Tūlapuṇṇikāti} \textsanskrit{tūlapicunā} \textsanskrit{pūretvā} \textsanskrit{katā}}, “\textit{\textsanskrit{Tūlapuṇṇika}}: they are made by filling with cotton down.” Sp 3.246: \textit{\textsanskrit{Tittirapattikāti} \textsanskrit{tittirapattasadisā} \textsanskrit{vicittabaddhā}}, “\textit{Tittirapattika}: they are colored, like the feathers of a partridge.” Sp 3.246: \textit{\textsanskrit{Meṇḍavisāṇavaddhikāti} \textsanskrit{kaṇṇikaṭṭhāne} \textsanskrit{meṇḍakasiṅgasaṇṭhāne} vaddhe \textsanskrit{yojetvā} \textsanskrit{katā}}, “\textit{\textsanskrit{Meṇḍavisāṇavaddhika}}: they are made by connecting a strap with the appearance of a ram horn at one corner.” Sp 3.246: \textit{\textsanskrit{Vicchikāḷikāpi} tattheva \textsanskrit{vicchikanaṅguṭṭhasaṇṭhāne} vaddhe \textsanskrit{yojetvā} \textsanskrit{katā}}, “\textit{\textsanskrit{Vicchikāḷika}}: they are made by connecting a strap with the appearance of scorpion claw.” Sp 3.246: \textit{\textsanskrit{Morapiñchaparisibbitāti} talesu \textsanskrit{vā} vaddhesu \textsanskrit{vā} \textsanskrit{morapiñchehi} suttakasadisehi \textsanskrit{parisibbitā}}, “\textit{\textsanskrit{Morapiñchaparisibbita}}: the tail feather of a peacock is sewn on the sole or on the strap, like a string of beads.” } If you do, you commit an offense of wrong conduct.” }

At\marginnote{2.4.1} that time the monks from the group of six wore sandals decorated with lionskin, tiger skin, leopard skin, deerskin, otter skin, cat skin, squirrel skin, and bat skin. People complained and criticized them, “They’re just like householders who indulge in worldly pleasures!” 

\scrule{“You shouldn’t wear sandals decorated with lionskin, tiger skin, leopard skin, deerskin, otter skin, cat skin, squirrel skin, or bat skin.\footnote{Sp 3.246: \textit{\textsanskrit{Lūvakacammaparikkhaṭāti} \textsanskrit{pakkhibiḷālacammaparikkhaṭā}}, “\textit{\textsanskrit{Luvakacammaparikkhaṭa}}: decorated with the skin of a winged cat.” Sp-yoj 3.246: \textit{\textsanskrit{Pakkhibiḷāloti} tuliyo}, “A winged cat is a flying fox.” } If you do, you commit an offense of wrong conduct.” }

\section*{4. The allowance for second-hand sandals with multi-layered soles }

One\marginnote{3.1.1} morning the Buddha robed up, took his bowl and robe, and entered \textsanskrit{Rājagaha} for almsfood together with an attendant monk. As the attendant followed behind the Buddha, he was limping. A certain lay follower wearing sandals with multi-layered soles saw the Buddha coming. He removed his sandals, approached the Buddha, and bowed.\footnote{I read \textit{\textsanskrit{orohitvā}} with the PTS version, against \textit{\textsanskrit{ārohitvā}} in MS. The MS text is saying that he is already wearing sandals, but then puts them on before going to meet the Buddha, which does not make good sense. } He then bowed to the attendant monk and asked him, “Venerable, why are you limping?” 

“Because\marginnote{3.2.2} my feet are cracked.” 

“Well\marginnote{3.2.3} then, take these sandals.” 

“Thanks,\marginnote{3.2.4} but the Buddha has prohibited sandals with multi-layered soles.” 

But\marginnote{3.2.5} the Buddha said, “Please take the sandals.” Soon afterwards the Buddha gave a teaching and addressed the monks: 

\scrule{“I allow second-hand sandals with multi-layered soles. But you shouldn’t wear new sandals with multi-layered soles. If you do, you commit an offense of wrong conduct.” }

\section*{5. The prohibition against sandals inside a monastery }

On\marginnote{4.1.1} one occasion the Buddha was doing walking meditation outside without sandals. The senior monks followed his example, but not the monks from the group of six. The monks of few desires complained and criticized them, “How can the monks from the group of six do walking meditation with their sandals on when the Teacher and the senior monks do it without?” They told the Buddha. … “Is it true, monks, that the monks from the group of six do this?” 

“It’s\marginnote{4.2.3} true, sir.” 

The\marginnote{4.2.4} Buddha rebuked them … “How can those foolish men do walking meditation with their sandals on when the Teacher and the senior monks do it without? Even the householders who wear white are respectful and deferential toward their teachers for teaching them the profession by which they make a living. And you who have gone forth on such a well-proclaimed spiritual path will shine if you’re respectful and deferential toward your teachers, your preceptors, or those of an equivalent standing.\footnote{Reading \textit{\textsanskrit{sagāravā} \textsanskrit{sappatissā} \textsanskrit{sabhāgavuttikā}} with SRT. } This will affect people’s confidence …” After rebuking them … the Buddha gave a teaching and addressed the monks: 

\scrule{“You shouldn’t do walking meditation with your sandals on when your teachers, your preceptors, or those of equivalent standing do it without. If you do, you commit an offense of wrong conduct. }

\scrule{And you shouldn’t wear sandals within a monastery. If you do, you commit an offense of wrong conduct.” }

Soon\marginnote{5.1.1} afterwards a certain monk was afflicted with a corn on his foot. The monks had to hold him while he urinated and defecated. Just then, the Buddha was walking about the dwellings and saw this. He went up to those monks and said to them, “What illness does this monk have?” 

“He\marginnote{5.2.2} has a corn on his foot, sir. That’s why we do this for him.” Soon afterwards the Buddha gave a teaching and addressed the monks: 

\scrule{“I allow you to wear sandals if your feet are painful or cracked, or you have a corn on your foot.” }

Then\marginnote{6.1.1} the monks made use of the beds and benches with dirty feet. Their robes and the furniture got dirty.\footnote{“Furniture” renders \textit{\textsanskrit{senāsana}}. See Appendix of Technical Terms. } 

\scrule{“When you know that you are about to make use of a bed or a bench, I allow you to wear sandals.” }

Then,\marginnote{6.2.1} when the monks were walking to the observance hall or to a meeting in the dark of night, they stepped on stumps and thorns, hurting their feet. 

\scrule{“I allow you to wear sandals within a monastery, and also to use a torch, a lamp, and a walking stick.” }

\section*{6. The prohibition against wooden shoes, etc. }

At\marginnote{6.3.1} one time the monks from the group of six got up early in the morning, put on wooden shoes, and walked back and forth outside, making a loud clacking noise. And they talked about all sorts of worldly things: about kings, gangsters, and officials; about armies, dangers, and battles; about food, drink, clothes, and beds; about garlands and perfumes; about relatives, vehicles, villages, towns, and countries; about women and heroes; gossip; about the departed; about various trivialities; about the world and the sea; about being this or that. They stepped on and killed insects, and they disturbed the monks in the stillness of meditation. 

The\marginnote{6.4.1} monks of few desires complained and criticized them, “How can the monks from the group of six act like this?” They told the Buddha. … “Is it true, monks, that the monks from the group of six are acting like this?” “It’s true, sir.” … After rebuking them … the Buddha gave a teaching and addressed the monks: 

\scrule{“You shouldn’t wear wooden shoes.\footnote{The shoe, \textit{\textsanskrit{pādukā}}, is introduced in this section, as distinct from the \textit{\textsanskrit{upāhanā}}, “sandal”, of the previous sections. They are both footwear and the distinction between them is not obvious. The best indication as to the difference is found at Sp-yoj 2.638: \textit{\textsanskrit{Pādukanti} \textsanskrit{upāhanaviseso}. So hi pajjate \textsanskrit{imāyāti} \textsanskrit{pādukāti} vuccati, \textsanskrit{sā} \textsanskrit{bahupaṭalā} \textsanskrit{cammamayā} \textsanskrit{vā} hoti \textsanskrit{kaṭṭhamayā} \textsanskrit{vā}}, “A \textit{\textsanskrit{pāduka}}: it is distinct from an \textit{\textsanskrit{upāhana}}. It is called a \textit{\textsanskrit{pāduka}}, because one should walk with it. It has much covering made of skin or wood.” Here the distinction between the two appears to hinge on the amount of covering, and thus translating the two terms as “shoe” and “sandal” seems justified. Moreover, the distinction made in Bhikkhu \textsanskrit{Ṭhānissaro}'s “The Buddhist Monastic Code I”, p. 444, and “The Buddhist Monastic Code II”, chapter III, that \textit{\textsanskrit{upāhāna}} refers to leather footwear whereas \textit{\textsanskrit{pāduka}} refers to non-leather footwear is here contradicted: it is specifically stated that a \textit{\textsanskrit{pāduka}} can be made of leather. } If you do, you commit an offense of wrong conduct.” }

When\marginnote{7.1.1} the Buddha had stayed at \textsanskrit{Rājagaha} for as long as he liked, he set out wandering toward Benares. When he eventually arrived, he stayed in the deer park at Isipatana. 

When\marginnote{7.1.4} the monks from the group of six heard that the Buddha had prohibited wooden shoes, they took cuttings from young palm trees and wore shoes made of palm leaves. The trees withered. People complained and criticized them, “How can the Sakyan monastics act like this? They are harming one-sensed life.” 

The\marginnote{7.2.1} monks heard the complaints of those people and they told the Buddha. … “Is it true, monks, that the monks from the group of six are acting like this?” 

“It’s\marginnote{7.2.5} true, sir.” 

The\marginnote{7.2.6} Buddha rebuked them … “How can those foolish men have cuttings made from young palm trees and wear shoes made of palm leaves, with the trees withering as a consequence? People regard trees as conscious. This will affect people’s confidence …” After rebuking them … the Buddha gave a teaching and addressed the monks: 

\scrule{“You shouldn’t wear shoes made of palm leaves. If you do, you commit an offense of wrong conduct.” }

When\marginnote{7.3.1} they heard that the Buddha had prohibited shoes made of palm leaves, the monks from the group of six had cuttings made from young bamboo and wore shoes made of bamboo leaves. The bamboo withered. People complained and criticized them, “How can the Sakyan monastics act like this? They are harming one-sensed life.” The monks heard the complaints of those people and they told the Buddha. … “… People regard trees as conscious … 

\scrule{You shouldn’t wear shoes made of bamboo leaves. If you do, you commit an offense of wrong conduct.” }

When\marginnote{8.1.1} the Buddha had stayed at Benares for as long as he liked, he set out wandering toward Bhaddiya. When he eventually arrived, he stayed in the \textsanskrit{Jātiyā} Grove. 

At\marginnote{8.1.4} that time the monks in Bhaddiya were fond of various kinds of nice shoes. They made shoes of grass, reed, fishtail-palm leaves, and wool, and they had them made. As a consequence, they neglected recitation, questioning, the higher morality, the higher mind, and the higher wisdom.\footnote{“Grass” covers two separate Pali terms, \textit{\textsanskrit{tiṇa}}, and \textit{kamala}. “Reed” covers two separate Pali terms, \textit{\textsanskrit{muñja}} and \textit{pabbaja}. } The monks of few desires complained and criticized them, “How can the monks in Bhaddiya do this?” 

They\marginnote{8.2.3} told the Buddha. … “Is it true, monks, that the monks in Bhaddiya do this?” 

“It’s\marginnote{8.2.6} true, sir.” 

The\marginnote{8.2.7} Buddha rebuked them … “How can those foolish men be fond of various kinds of nice shoes … and neglect recitation, questioning, the higher morality, the higher mind, and the higher wisdom? This will affect people’s confidence …” After rebuking them … the Buddha gave a teaching and addressed the monks: 

\scrule{“You shouldn’t wear shoes made of grass, reed, fishtail-palm leaves, or wool; or shoes made with gold, silver, gems, beryl, crystal, bronze, glass, tin, lead, or copper.\footnote{\textit{\textsanskrit{Hintāla}} is identified as the fishtail palm in SAF, p. 190. “Beryl” renders \textit{\textsanskrit{veḷuriya}}. Sp-\textsanskrit{ṭ} 1.281: \textit{\textsanskrit{Veḷuriyoti} \textsanskrit{vaṁsavaṇṇamaṇi}}, “The bamboo-colored gem is called \textit{\textsanskrit{veḷuriya}}.” According to PED, \textit{\textsanskrit{veḷuriya}} is lapis lazuli, which cannot be correct because lapis lazuli is blue. For the first four kinds of shoes I use the expression “made of”, but for the remainder, “made with”. It seems unlikely that the entire shoe would be made of these precious substances. } If you do, you commit an offense of wrong conduct. }

\scrule{And you shouldn’t use shoes.\footnote{Going by the commentarial definition (see the next note) the contextual meaning of \textit{\textsanskrit{saṅkamaniya}} is essentially “mobility”, which seems redundant on translation. } If you do, you commit an offense of wrong conduct. I allow three kinds of foot stands that are fixed in place and immobile:\footnote{Sp 3.251: \textit{\textsanskrit{Asaṅkamanīyāti} \textsanskrit{bhūmiyaṁ} \textsanskrit{suppatiṭṭhitā} \textsanskrit{niccalā} \textsanskrit{asaṁhāriyā}}, “\textit{\textsanskrit{Asaṅkamanīya}}: well-established on the ground, stable, not moving.” } foot stands for defecating, foot stands for urinating, and foot stands for restroom ablutions.”\footnote{\textit{\textsanskrit{Pāduka}}, translated as “shoe” above, I have here translated as “foot-stand”. This seems to be required from the current context. Sp 4.290: \textit{\textsanskrit{Passāvapādukanti} ettha \textsanskrit{pādukā} \textsanskrit{iṭṭhakāhipi} \textsanskrit{silāhipi} \textsanskrit{dārūhipi} \textsanskrit{kātuṁ} \textsanskrit{vaṭṭati}. \textsanskrit{Vaccapādukāyapi} eseva nayo}, “A \textit{\textsanskrit{pāduka}} for urinating: here it is allowable to make a \textit{\textsanskrit{pāduka}} of bricks, stone, or wood. The same method also applies for \textit{\textsanskrit{pāduka}} for defecating.” These fixtures seem more likely to be platforms or stands than shoes in any ordinary sense. } }

When\marginnote{9.1.1} the Buddha had stayed at Bhaddiya for as long as he liked, he set out wandering toward \textsanskrit{Sāvatthī}. When he eventually arrived, he stayed in the Jeta Grove, \textsanskrit{Anāthapiṇḍika}’s Monastery. 

At\marginnote{9.1.4} this time, the monks from the group of six would grab cattle as they were crossing the \textsanskrit{Aciravatī} river—by the horns, the ears, the neck, and the tail—and they would mount their backs and, motivated by lust, would touch their genitals. They even killed a calf by submerging it. People complained and criticized them, “How can the Sakyan monastics act like this? They’re just like householders who indulge in worldly pleasures!” 

The\marginnote{9.2.4} monks heard the complaints of those people and they told the Buddha. … “Is it true, monks …” “It’s true, sir.” … After rebuking them … the Buddha gave a teaching and addressed the monks: 

\scrule{“You shouldn’t grab cattle by the horns, the ears, the neck, or the tail, and you shouldn’t mount their backs. If you do mount their backs, you commit an offense of wrong conduct. }

\scrule{And you shouldn’t touch their genitals motivated by lust. If you do, you commit a serious offense. }

\scrule{And you shouldn’t kill a calf. If you do, you should be dealt with according to the rule.” }

\section*{7. The prohibition against vehicles, etc. }

At\marginnote{9.4.1} that time the monks from the group of six traveled in vehicles, sometimes pulled by a female animal with a man driving, at other times pulled by a male animal with a woman driving.\footnote{Sp 3.253: \textit{\textsanskrit{Itthiyuttenāti} dhenuyuttena. \textsanskrit{Purisantarenāti} \textsanskrit{purisasārathinā}. \textsanskrit{Purisayuttenāti} \textsanskrit{goṇayuttena}. \textsanskrit{Itthantarenāti} \textsanskrit{itthisārathinā}}, “\textit{Itthiyuttena}: with a yoked cow. \textit{Purisantarena}: with a man charioteer. \textit{Purisayuttena}: with a yoked bull. \textit{\textsanskrit{Itthantarenā}}: with a woman charioteer.” } People complained and criticized them, “You’d think they were at the Ganges festival!” They told the Buddha. 

\scrule{“You shouldn’t travel in a vehicle. If you do, you commit an offense of wrong conduct.” }

Soon\marginnote{10.1.1} afterwards a monk who was traveling through the Kosalan country on his way to visit the Buddha at \textsanskrit{Sāvatthī} became sick. He stepped off the path and sat down at the foot of a tree. People saw him and said to him, “Venerable, where are you going?” 

“I’m\marginnote{10.1.5} going to \textsanskrit{Sāvatthī} to visit the Buddha.” 

“Please\marginnote{10.2.1} come with us.” 

“I\marginnote{10.2.2} can’t. I’m sick.” 

“Then\marginnote{10.2.3} please come inside the vehicle.” 

“Thank\marginnote{10.2.4} you, but the Buddha has prohibited us from traveling in vehicles.” 

He\marginnote{10.2.5} did not accept because he was afraid of wrongdoing. Then, when he arrived at \textsanskrit{Sāvatthī}, he told the monks what had happened. They in turn told the Buddha. 

\scrule{“I allow a vehicle when you’re sick.” }

The\marginnote{10.3.1} monks thought, “Pulled by a female or by a male?” 

\scrule{“I allow a rickshaw pulled by men.”\footnote{Vin-vn-\textsanskrit{ṭ} 3084: \textit{\textsanskrit{Hatthavaṭṭakanti} hattheneva \textsanskrit{pavaṭṭetabbasakaṭaṁ}}, “\textit{\textsanskrit{Hatthavaṭṭakan}}: a cart to be moved only by hand.” } }

Soon\marginnote{10.3.5} afterwards a certain monk was even more uncomfortable when jolted around in a vehicle. 

\scrule{“I allow a palanquin and a litter.” }

\section*{8. The prohibition against high and luxurious beds }

At\marginnote{10.4.1} that time the monks from the group of six used high and luxurious beds, such as: high couches, luxurious couches, long-fleeced woolen rugs, multi-colored woolen rugs, white woolen rugs, red woolen rugs, cotton-down quilts, woolen rugs decorated with the images of predatory animals, woolen rugs with long fleece on one side, woolen rugs with long fleece on both sides, sheets of silk embroidered with gems, silken sheets, woolen rugs like a dancer’s rug, elephant-back rugs, horse-back rugs, carriage-seat rugs, rugs made of black antelope hide, exquisite sheets made of \textit{\textsanskrit{kadalī}}-deer hide, seats with canopies, seats with red cushions at each end. When people walking about the dwellings saw this, they complained and criticized them, “They’re just like householders who indulge in worldly pleasures!” They told the Buddha. 

\scrule{“You shouldn’t use high and luxurious beds, such as: high couches, luxurious couches, long-fleeced woolen rugs, multi-colored woolen rugs, white woolen rugs, red woolen rugs, cotton-down quilts, woolen rugs decorated with the images of predatory animals, woolen rugs with long fleece on one side, woolen rugs with long fleece on both sides, sheets of silk embroidered with gems, silken sheets, woolen rugs like a dancer’s rug, elephant-back rugs, horse-back rugs, carriage-seat rugs, rugs made of black antelope hide, exquisite sheets made of \textit{\textsanskrit{kadalī}}-deer hide, seats with canopies, seats with red cushions at each end.\footnote{For a further discussion of these, see Appendix of Furniture. } If you do, you commit an offense of wrong conduct.” }

\section*{9. The prohibition against all skins }

Soon\marginnote{10.6.1} afterwards when the monks from the group of six heard that the Buddha had prohibited high and luxurious beds, they used luxurious skins: lionskins, tiger skins, and leopard skins. They cut them to fit their beds and benches, and used them both there and elsewhere. When people walking about the dwellings saw this, they complained and criticized them, “They’re just like householders who indulge in worldly pleasures!” They told the Buddha. 

\scrule{“You shouldn’t use luxurious skins: lionskins, tiger skins, or leopard skins. If you do, you commit an offense of wrong conduct.” }

Soon\marginnote{10.7.1} afterwards when the monks from the group of six heard that the Buddha had prohibited luxurious skins, they used cattle hides. They cut them to fit their beds and benches, and used them both there and elsewhere. 

At\marginnote{10.7.5} this time a certain bad monk was associating with the family of a bad lay follower. One morning that monk robed up, took his bowl and robe, and went to that lay follower’s house, where he sat down on the prepared seat. The lay follower approached the monk, bowed, and sat down. 

At\marginnote{10.8.1} that time that lay follower had a beautiful young calf with variegated hide, just like a young leopard. When the bad monk stared at that calf, the lay follower asked him why. He replied, “I need the skin of that calf.” 

The\marginnote{10.8.6} bad lay follower then slaughtered the calf, skinned it, and gave the skin to the bad monk. The monk hid the skin under his outer robe and left. The mother-cow, longing for her calf, followed behind him. When the monks asked him why, he said he did not know. But his outer robe was smeared with blood, and so they said, “What happened to your outer robe?” 

When\marginnote{10.9.8} he told them what had happened, they asked, “So did you encourage someone to kill?” 

“Yes.”\marginnote{10.9.10} 

The\marginnote{10.9.11} monks of few desires complained and criticized him, “How can a monk encourage someone to kill? Hasn’t the Buddha in many ways criticized killing and praised abstention from killing?” They then told the Buddha. 

Soon\marginnote{10.10.1} afterwards the Buddha had the Sangha gathered and questioned that bad monk: “Is it true, monk, that you encouraged someone to kill?” 

“It’s\marginnote{10.10.3} true, sir.” … 

“Foolish\marginnote{10.10.4} man, how can you encourage someone to kill? Haven’t I in many ways criticized killing and praised abstention from killing? This will affect people’s confidence …” After rebuking him, the Buddha gave a teaching and addressed the monks: 

\scrule{“You shouldn’t make others kill. If you do, you should be dealt with according to the rule. }

\scrule{And you shouldn’t use cattle hide. If you do, you commit an offense of wrong conduct. }

\scrule{And you shouldn’t use any kind of skin. If you do, you commit an offense of wrong conduct.” }

\section*{10. The allowance regarding the belongings of a householder, etc. }

At\marginnote{11.1.1} that time people’s beds and benches were upholstered and covered with skin. Being afraid of wrongdoing, the monks did not sit on them. 

\scrule{“I allow you to sit down on what belongs to a householder, but not to lie down on it.” }

The\marginnote{11.1.5} dwellings were held together by straps of leather.\footnote{This refers to monastic dwellings, not regular houses. } Being afraid of wrongdoing, the monks did not sit down. 

\scrule{“I allow you to sit down against a mere binding made of skin.” }

At\marginnote{12.1.1} that time the monks from the group of six entered the village wearing sandals. People complained and criticized them, “They’re just like householders who indulge in worldly pleasures!” They told the Buddha. 

\scrule{“You shouldn’t enter the village wearing sandals. If you do, you commit an offense of wrong conduct.” }

Soon\marginnote{12.1.7} afterwards a certain sick monk was unable to go to the village without sandals. 

\scrule{“I allow sick monks to enter the village wearing sandals.” }

\section*{11. The account of \textsanskrit{Soṇa} \textsanskrit{Kuṭikaṇṇa} }

At\marginnote{13.1.1} one time Venerable \textsanskrit{Mahākaccāna} was staying in \textsanskrit{Avantī} on Papataka Hill at Kuraraghara. At that time the lay follower \textsanskrit{Soṇa} \textsanskrit{Kuṭikaṇṇa} was his supporter. 

On\marginnote{13.1.3} one occasion \textsanskrit{Soṇa} went to \textsanskrit{Mahākaccāna}, bowed, sat down, and said, “Venerable, the way I understand your teaching, it’s not easy for one who lives at home to lead the spiritual life perfectly complete and pure as a polished conch shell. I wish to cut off my hair and beard, put on the ocher robes, and go forth into homelessness. Please give me the going forth.” 

“It’s\marginnote{13.2.1} difficult, \textsanskrit{Soṇa}, to live the spiritual life all one’s life, eating one meal a day and sleeping by oneself. So follow the Buddhas’ instruction while remaining as a householder. At suitable times you can eat one meal a day, sleep by yourself, and abstain from sexuality.” As a result, \textsanskrit{Soṇa}’s intention to go forth died down. 

A\marginnote{13.2.4} second time \textsanskrit{Soṇa} asked \textsanskrit{Mahākaccāna} for the going forth, but got the same response. A third time he asked for the going forth and \textsanskrit{Mahākaccāna} finally relented. 

At\marginnote{13.2.11} that time in the southern region of \textsanskrit{Avantī}, there were few monks. Only after three years, with much trouble and difficulty, was \textsanskrit{Mahākaccāna} able to gather a sangha of ten monks from here and there to give the full ordination to Venerable \textsanskrit{Soṇa}. 

\section*{12. The making known of the five favors for \textsanskrit{Mahākaccāna} }

After\marginnote{13.3.1} completing the rainy-season residence, \textsanskrit{Soṇa} was reflecting in private: “I’ve heard that the Buddha is like this and like that, but I haven’t seen this for myself. If my preceptor allows me, I shall go and visit the Buddha, the Perfected One, the fully Awakened One.” 

Coming\marginnote{13.3.3} out from seclusion, \textsanskrit{Soṇa} went to \textsanskrit{Mahākaccāna}, bowed, sat down, and told him what he had thought. \textsanskrit{Mahākaccāna} said, “Well thought, \textsanskrit{Soṇa}! Please go and visit the Buddha, the Perfected and fully Awakened One. You will see someone who is pleasing to the eye and inspiring confidence; who is peaceful in mind and faculties; who is attained to the supreme subduing and calm; who is tamed, guarded, and restrained in his senses—a great being. Then, \textsanskrit{Soṇa}, in my name, pay respect with your head at the Buddha’s feet and say, ‘Sir, my preceptor, Venerable \textsanskrit{Mahākaccāna}, pays respect with his head at the Buddha’s feet.’ And then say this: 

‘In\marginnote{13.5.5} the southern region of \textsanskrit{Avantī}, sir, there are few monks. Only after three years, with much trouble and difficulty, was it possible to gather a sangha of ten monks from here and there to give me the full ordination. Would the Buddha allow a smaller group of monks to give the full ordination in \textsanskrit{Avantī}? 

In\marginnote{13.6.1} \textsanskrit{Avantī} the ground is dark and hard, made rough by the hooves of cattle. Would the Buddha allow sandals with multi-layered soles in \textsanskrit{Avantī}? 

In\marginnote{13.6.3} \textsanskrit{Avantī} people value bathing and cleanliness. Would the Buddha allow unrestricted bathing in \textsanskrit{Avantī}? 

In\marginnote{13.6.5} \textsanskrit{Avantī} sheepskins, goatskins, and deerskins are used as rugs, just as \textit{eragu} grass, chaff-flower grass, \textit{\textsanskrit{majjāru}} grass, and \textit{jantu} grass are used in the central Ganges plain.\footnote{For the term \textit{\textsanskrit{moragū}}, see Appendix of Plants. } Would the Buddha allow sheepskins, goatskins, and deerskins as rugs in \textsanskrit{Avantī}? 

At\marginnote{13.7.1} present people give robe-cloth to monks who are outside the monastery zone, saying,\footnote{“Robe-cloth” renders \textit{\textsanskrit{cīvara}}. See Appendix of Technical Terms. } “We give this robe-cloth to so-and-so.”\footnote{To clarify, the issue at stake is people giving cloth at a monastery for a monk who is away. The monk does not know he has been given cloth until he returns to the monastery. } When those monks return to the monastery, they are told, “Such-and-such people have given you robe-cloth.” But being afraid of wrongdoing, they don’t accept, thinking, “We might commit an offense entailing relinquishment.”\footnote{This refers to \href{https://suttacentral.net/pli-tv-bu-vb-np1/en/brahmali\#2.17.1}{Bu Np 1:2.17.1}/Bi Np 13, which prohibits a monk from keeping extra robe cloth for more than ten days. The point made here is that these monks would count the days from the moment the cloth was given. If they arrived at the monastery more than ten days after the cloth had been given, they would not be able to receive it without falling into an offense. } Would the Buddha point out a way to deal with robe-cloth?’” 

\textsanskrit{Soṇa}\marginnote{13.7.7} replied, “Yes, sir.” 

He\marginnote{13.7.8} got up from his seat, bowed down, and circumambulated \textsanskrit{Mahākāccāna} with his right side toward him. He then put his dwelling in order, took his bowl and robe, and set out for \textsanskrit{Sāvatthī}. When he eventually arrived, he went to the Jeta Grove, \textsanskrit{Anāthapiṇḍika}’s Monastery where he approached the Buddha, bowed, and sat down. 

The\marginnote{13.8.2} Buddha said to Venerable Ānanda, “Ānanda, please prepare a resting place for this newly-arrived monk.” Ānanda thought, “When the Buddha says this, it means he wishes to stay in the same dwelling as Venerable \textsanskrit{Soṇa}.” And he prepared a resting place for \textsanskrit{Soṇa} in the Buddha’s dwelling. 

Then,\marginnote{13.9.1} after spending much of the night outside, the Buddha entered the dwelling, as did \textsanskrit{Soṇa}. Rising early in the morning, the Buddha said to \textsanskrit{Soṇa}, “Recite a teaching, monk.” 

Saying,\marginnote{13.9.5} “Yes, sir,” he chanted the entire Chapter of Eights.\footnote{The fourth chapter of the Sutta \textsanskrit{Nipāta}. } 

When\marginnote{13.9.6} he was finished, the Buddha said, “Well done, \textsanskrit{Soṇa}, well done. You have learned the Chapter of Eights well. You have remembered it well. And you have a good voice—it’s clear, articulate, and gets the meaning across. How long have you been a monk?” 

“One\marginnote{13.9.11} year, sir.” 

“But\marginnote{13.10.1} why did it take you so long to go forth?” 

“Well,\marginnote{13.10.2} I have long seen the downside of worldly pleasures. Still, because household life is crowded and busy, I was not able to leave.”\footnote{\textit{Api ca \textsanskrit{sambādhā} \textsanskrit{gharāvāsā} \textsanskrit{bahukiccā} \textsanskrit{bahukaraṇīyāti}}, “Still, household life is crowded, with much business and many duties.” I have added a bit from the commentary to make the sentence clearer. Ud-a 46: \textit{\textsanskrit{Kāmesu} \textsanskrit{ādīnave} kenaci \textsanskrit{pakārena} \textsanskrit{diṭṭhepi} na \textsanskrit{tāvāhaṁ} \textsanskrit{gharāvāsato} \textsanskrit{nikkhamituṁ} \textsanskrit{asakkhiṁ}}, “Although I had seen the danger in sensual pleasures of whatever kind, I was not able to leave the household life for so long.” } 

Seeing\marginnote{13.10.3} the significance of this, the Buddha uttered a heartfelt exclamation: 

\begin{verse}%
“Having\marginnote{13.10.4} seen the downside of the world, \\
Knowing the Truth beyond ownership, \\
The noble one doesn’t delight in the bad; \\
In the bad, the pure one doesn’t delight.” 

%
\end{verse}

\textsanskrit{Soṇa}\marginnote{13.11.1} thought, “The Buddha approves of me! This is the time to bring up what my preceptor said.” He got up from his seat, arranged his upper robe over one shoulder, bowed down at the Buddha’s feet, and said, “Sir, my preceptor, Venerable \textsanskrit{Mahākaccāna}, pays respect with his head at the Buddha’s feet.” He then repeated everything \textsanskrit{Mahākaccāna} had asked him to say. 

Soon\marginnote{13.11.20} afterwards the Buddha gave a teaching and addressed the monks: 

\scrule{“In the southern region of \textsanskrit{Avantī} there are few monks. Outside the central Ganges plain, I allow the full ordination to be given by a group of five, including one expert on the Monastic Law. }

In\marginnote{13.12.1} this regard, the following is outside the central Ganges plain: 

\begin{itemize}%
\item In the eastern direction there is a town called \textit{\textsanskrit{Gajaṅgala}}, with another town called \textit{\textsanskrit{Mahāsālā}} just after it. Beyond it is outside the central Ganges plain. On the near side of it is the central Ganges plain. %
\item In the south-eastern direction there is a river called \textit{\textsanskrit{Sallavatī}}. Beyond it is outside the central Ganges plain. On the near side of it is the central Ganges plain. %
\item In the southern direction there is a town called \textit{\textsanskrit{Setakaṇṇika}}. Beyond it is outside the central Ganges plain. On the near side of it is the central Ganges plain. %
\item In the western direction there is a brahmin village called \textit{\textsanskrit{Thūṇa}}. Beyond it is outside the central Ganges plain. On the near side of it is the central Ganges plain. %
\item In the northern direction there is a mountain called \textit{\textsanskrit{Usīraddhaja}}. Beyond it is outside the central Ganges plain. On the near side of it is the central Ganges plain. %
\end{itemize}

In\marginnote{13.13.1} \textsanskrit{Avantī} the ground is dark and hard, made rough by the hooves of cattle. 

\scrule{Outside the central Ganges plain, I allow sandals with multi-layered soles. }

In\marginnote{13.13.3} \textsanskrit{Avantī} people value bathing and cleanliness. 

\scrule{Outside the central Ganges plain, I allow unrestricted bathing. }

In\marginnote{13.13.5} \textsanskrit{Avantī} sheepskins, goatskins, and deerskins are used as rugs, just as \textit{eragu} grass, chaff-flower grass, \textit{\textsanskrit{majjāru}} grass, and \textit{jantu} grass are used in the central Ganges plain. 

\scrule{Outside the central Ganges plain, I allow rugs of sheepskin, goatskin, and deerskin. }

And\marginnote{13.13.9} it may be that people give robe-cloth to monks who are outside the monastery zone, saying, ‘We give this robe-cloth to so-and-so.’ 

\scrule{I allow you to accept it and not start counting the days until you receive it in your hands.”\footnote{This relates to \href{https://suttacentral.net/pli-tv-bu-vb-np1/en/brahmali\#2.17.1}{Bu Np 1:2.17.1} and \href{https://suttacentral.net/pli-tv-bu-vb-np3/en/brahmali\#1.3.13.1}{Bu Np 3:1.3.13.1}. } }

\scendsutta{The fifth chapter on skins is finished. }

\scuddanaintro{This is the summary: }

\begin{scuddana}%
“The\marginnote{13.13.14} king of Magadha and \textsanskrit{Soṇa}, \\
Eighty thousand chiefs; \\
\textsanskrit{Sāgata} on the Vulture Peak, \\
Showed much that was super-human. 

Going\marginnote{13.13.18} forth, energetic, they cracked, \\
Lute, single-layered soles; \\
Blue, yellow, red, \\
Magenta, and just black. 

Orange,\marginnote{13.13.22} beige, \\
And he prohibited straps; \\
Leather, and enclosing, covering, \\
Cotton, partridge, ram, goat. 

Scorpion,\marginnote{13.13.26} peacock, and various, \\
Lion, and tiger, leopard; \\
Deer, otter, and cat, \\
Squirrel, bat, decorated. 

Cracked,\marginnote{13.13.30} sandals, corn, \\
Washed, stumps, clacking; \\
Palm, bamboo, and just grass, \\
Reed, fish-tail palm. 

Grass,\marginnote{13.13.34} wool, gold, \\
Silver, gems, beryl; \\
Crystal, bronze, and glass, \\
Tin, and lead, copper. 

Cow,\marginnote{13.13.38} vehicle, and sick, \\
Pulled by men, palanquin; \\
Beds, luxurious skins, \\
And the bad one with a cattle hide. 

Of\marginnote{13.13.42} householders, straps of leather, \\
They enter, being sick; \\
\textsanskrit{Mahākaccāyana}, \textsanskrit{Soṇa}, \\
Chanted the Chapter of Eights. 

Full\marginnote{13.13.46} ordination through five, \\
Multi-layered, unrestricted bathing; \\
He allowed rugs made of skin, \\
Not start the counting until; \\
The leader did these five favors,\footnote{I read \textit{\textsanskrit{adās}’ime} with SRT. } \\
For \textsanskrit{Soṇa}, the senior monk.” 

%
\end{scuddana}

\scend{In this chapter there are sixty-three topics. }

\scendsutta{The chapter on skins is finished. }

%
\chapter*{{\suttatitleacronym Kd 6}{\suttatitletranslation The chapter on medicines }{\suttatitleroot Bhesajjakkhandhaka}}
\addcontentsline{toc}{chapter}{\tocacronym{Kd 6} \toctranslation{The chapter on medicines } \tocroot{Bhesajjakkhandhaka}}
\markboth{The chapter on medicines }{Bhesajjakkhandhaka}
\extramarks{Kd 6}{Kd 6}

\section*{1. Discussion of the five tonics }

At\marginnote{1.1.1} one time the Buddha was staying at \textsanskrit{Sāvatthī} in the Jeta Grove, \textsanskrit{Anāthapiṇḍika}’s Monastery. At that time the monks were afflicted with autumn illness, and they could keep down neither congee nor other food. As a result, they became thin, haggard, and pale, with veins protruding all over their body. The Buddha noticed this and asked Venerable Ānanda why they were looking so sickly. Ānanda told him. 

Then,\marginnote{1.2.1} while reflecting in private, the Buddha thought, “What tonics might I allow the monks that are generally regarded as tonics, would serve as nourishment, but aren’t considered substantial food?” It then occurred to him, “There are these five tonics—\footnote{For an explanation of rendering \textit{bhesajja} as “tonics”, see Appendix of Technical Terms. } ghee, butter, oil, honey, and syrup—\footnote{I. B. Horner translates \textit{\textsanskrit{phāṇita}} as “molasses”, which doesn’t quite hit the mark. SED defines \textit{\textsanskrit{phāṇita}} as “the inspissated juice of the sugar cane or other plants”, in other words, “cane syrup”. According to the commentary at Sp 1.623 it can be either cooked or uncooked, the difference presumably being whether the juice is raw or concentrated. “Syrup” seems closer to the mark than “molasses”. } that are generally regarded as tonics, serve as nourishment, but aren’t considered substantial food. Why don’t I allow them these five tonics, to be received and consumed before midday?” 

In\marginnote{1.3.1} the evening, when the Buddha had come out from seclusion, he gave a teaching and then told the monks what he had thought, adding: 

\scrule{“I allow these five tonics, to be received and consumed before midday.” }

The\marginnote{1.4.1} monks then received and consumed the five tonics before midday. But even ordinary food did not agree with them, let alone greasy food. As result of both the autumn illness and the food not agreeing with them, they became even more thin, haggard, and pale. Once again the Buddha noticed this and asked Venerable Ānanda why they were looking even worse. Ānanda told him. The Buddha then gave a teaching, and addressed the monks: 

\scrule{“I allow the five tonics to be received and consumed both before and after midday.” }

At\marginnote{2.1.1} that time the sick monks needed fat as a tonic. They told the Buddha. 

\scrule{“I allow these fats as tonics: bear fat, fish fat, alligator fat, pig fat, and donkey fat. They should be received, melted, and mixed with oil before midday, and then used. If you receive, melt, and mix them with oil after midday, and then use them, you commit three offenses of wrong conduct. If you receive them before midday, but melt and mix them with oil after midday, and then use them, you commit two offenses of wrong conduct. If you receive and melt them before midday, but mix them with oil after midday, and then use them, you commit one offense of wrong conduct. If you receive, melt, and mix them with oil before midday, and then use them, there is no offense.” }

\section*{2. Discussion of root medicines, etc. }

At\marginnote{3.1.1} that time the sick monks needed medicinal roots. 

\scrule{“I allow these medicinal roots: turmeric, ginger, sweet flag, white sweet flag, atis root, black hellebore, vetiver root, nut grass, and whatever other medicinal roots there are that don’t serve as fresh or cooked food.\footnote{For a discussion of these names and those below, see Appendix of Plants. } After receiving them, you may keep them for life and use them when there’s a reason. If you use them when there’s no reason, you commit an offense of wrong conduct.” }

Soon\marginnote{3.2.1} afterwards the sick monks needed medicinal root flour. 

\scrule{“I allow a grinding stone.” }

The\marginnote{4.1.1} sick monks needed bitter medicines. 

\scrule{“I allow bitter medicines from these plants: neem tree, arctic snow, pointed gourd, white fig, Indian beech, and whatever other bitter medicines there are that don’t serve as fresh or cooked food. After receiving them, you may keep them for life and use them when there’s a reason. If you use them when there’s no reason, you commit an offense of wrong conduct.” }

The\marginnote{5.1.1} sick monks needed medicinal leaves. 

\scrule{“I allow medicinal leaves from these plants: neem tree, arctic snow, pointed gourd, holy basil, cotton plant, and whatever other leaf medicines there are that don’t serve as fresh or cooked food. After receiving them, you may keep them for life and use them when there’s a reason. If you use them when there’s no reason, you commit an offense of wrong conduct.” }

The\marginnote{6.1.1} sick monks needed medicinal fruits. 

\scrule{“I allow medicinal fruits from these plants: false black pepper, long pepper, black pepper, chebulic myrobalan, belleric myrobalan, emblic myrobalan, crepe ginger, and whatever other medicinal fruits there are that don’t serve as fresh or cooked food. After receiving them, you may keep them for life and use them when there’s a reason. If you use them when there’s no reason, you commit an offense of wrong conduct.” }

The\marginnote{7.1.1} sick monks needed medicinal gum. 

\scrule{“I allow the following medicinal gums: gum exuded from the asafoetida shrub, gum from the twigs and leaves of the asafoetida shrub, gum from the leaves of the asafoetida shrub, \textit{taka} gum, \textit{taka}-leaf gum, gum from heated \textit{taka} foliage, resin, and whatever other medicinal gums there are that don’t serve as fresh or cooked food. After receiving them, you may keep them for life and use them when there’s a reason. If you use them when there’s no reason, you commit an offense of wrong conduct.” }

The\marginnote{8.1.1} sick monks needed medicinal salts. 

\scrule{“I allow the following medicinal salts: sea salt, black salt, hill salt, soil salt, red salt, and whatever other medicinal salts there are that don’t serve as fresh or cooked food.\footnote{Sp 3.263: \textit{Sindhavanti \textsanskrit{setavaṇṇaṁ} pabbate \textsanskrit{uṭṭhahati}}, “\textit{Sindhava}: it appears as a white color in the hills.” Sp 3.263: \textit{Ubbhidanti \textsanskrit{bhūmito} \textsanskrit{aṅkuraṁ} \textsanskrit{uṭṭhahati}}, “\textit{Ubbhida}: it appears like a sprout from the earth.” But Sp-\textsanskrit{ṭ} 3.263 says: \textit{\textsanskrit{Ubbhidaṁ} \textsanskrit{nāma} \textsanskrit{ūsarapaṁsumayaṁ}}, “What is made from saline soil is called \textit{\textsanskrit{ubbhidaṁ}}.” Sp 3.263: \textit{Bilanti \textsanskrit{dabbasambhārehi} \textsanskrit{saddhiṁ} \textsanskrit{pacitaṁ}, \textsanskrit{taṁ} \textsanskrit{rattavaṇṇaṁ}}, “\textit{Bila}: it is boiled together with a material that has the color red.” } After receiving them, you may keep them for life and use them when there’s a reason. If you use them when there’s no reason, you commit an offense of wrong conduct.” }

\subsection*{Allowable medical equipment and more}

At\marginnote{9.1.1} this time Venerable Ānanda’s preceptor, Venerable \textsanskrit{Belaṭṭhasīsa}, had carbuncles, the pus making his robes adhere to his body. The monks kept on wetting his robes to remove the pus. As the Buddha was walking about the dwellings, he noticed this. He went up to them and said, “What sickness does this monk have?” 

“He\marginnote{9.1.5} has carbuncles, sir. That’s why we’re doing this.” Soon afterwards the Buddha gave a teaching and addressed the monks: 

\scrule{“For anyone who has an itch, a boil, a running sore, a carbuncle, or whose body smells, I allow medicinal powders.\footnote{Sp 2.539: \textit{\textsanskrit{Piḷakāti} \textsanskrit{lohitatuṇḍikā} \textsanskrit{sukhumapiḷakā}}, “\textit{\textsanskrit{Piḷaka}} is a minor \textit{\textsanskrit{piḷaka}} with blood on the tip.” Sp 2.539: \textit{Thullakacchu \textsanskrit{vā} \textsanskrit{ābādhoti} \textsanskrit{mahāpiḷakābādho} vuccati}, “\textit{Thullakacchu \textsanskrit{vā} \textsanskrit{ābādha}} is a sickness with large boils.” } If you’re not sick, I allow detergent, soap, and cleaning agents.\footnote{For an explanation of rendering \textit{\textsanskrit{chakaṇa}} and \textit{mattika} as respectively “detergent” and “soap”, see Appendix of Technical Terms. For an explanation of rendering \textit{rajananippakka} as “cleaning agents”, see Appendix of Medical Terminology. } And I allow a mortar and pestle.” }

Soon\marginnote{10.1.1} afterwards the sick monks needed sifted medicinal powders. 

\scrule{“I allow a powder sieve.” }

They\marginnote{10.1.4} needed finely sifted powder. 

\scrule{“I allow a cloth sieve.” }

On\marginnote{10.2.1} one occasion a monk was possessed by a spirit. His teacher and preceptor who were nursing him were not able to cure him. He then went to a pigs’ slaughterhouse to eat raw meat and drink blood. As a result, he became well. They told the Buddha. 

\scrule{“For one who is possessed, I allow raw meat and raw blood.”\footnote{See Appendix of Medical Terminology. } }

At\marginnote{11.1.1} that time a monk was afflicted with an eye-disease. The monks had to hold him while he urinated and defecated. Just then, as the Buddha was walking about the dwellings, he noticed this. He then went up to them and said, “What sickness does this monk have?” 

“He\marginnote{11.2.1} has an eye-disease, sir. That’s why we do this for him.” Soon afterwards the Buddha gave a teaching and addressed the monks: 

\scrule{“I allow these ointments: black ointment, mixed ointment, river ointment, red ocher, and soot.”\footnote{Sp 3.365: \textit{\textsanskrit{Rasañjanaṁ} \textsanskrit{nānāsambhārehi} \textsanskrit{kataṁ}}, “\textit{\textsanskrit{Rasañjana}} is made with many ingredients.” Sp 3.365: \textit{\textsanskrit{Sotañjananti} \textsanskrit{nadīsotādīsu} \textsanskrit{uppajjanakaṁ} \textsanskrit{añjanaṁ}}, “\textit{\textsanskrit{Sotañjana}}: an ointment being produced in the stream of rivers.” } }

They\marginnote{11.2.6} needed scented ointments. 

\scrule{“I allow sandal, crape jasmine, Indian valerian, coffee plum, and nut grass.”\footnote{For a discussion of these, see Appendix of Plants. } }

At\marginnote{12.1.1} that time the monks put their ointments in pots and scoops. The ointment was contaminated with grass, dust, and dirt. 

\scrule{“I allow an ointment box.” }

Soon\marginnote{12.1.5} afterwards the monks from the group of six used luxurious ointment boxes\footnote{I here render \textit{\textsanskrit{uccāvaca}} as luxurious. This rendering seems required by the context. See also use of this word at \href{https://suttacentral.net/sn2.29/en/brahmali\#7.1}{SN 2.29:7.1}. } made with gold or silver. People complained and criticized them, “They’re just like householders who indulge in worldly pleasures!” They told the Buddha. 

\scrule{“You shouldn’t use luxurious ointment boxes. If you do, you commit an offense of wrong conduct. }

\scrule{I allow ointment boxes made of bone, ivory, horn, reed, bamboo, wood, resin, fruit, metal, and shell.”\footnote{\textsanskrit{Khuddasikkhā}-\textsanskrit{purāṇaṭīkā} 185: \textit{\textsanskrit{Āmalakakakkādīhi} \textsanskrit{katā} \textsanskrit{phalamayā}}, “Made of fruit means made from ground emblic myrobalan, etc.” } }

At\marginnote{12.2.1} that time the ointment boxes were not covered. The ointment was contaminated with grass, dust, and dirt. 

\scrule{“I allow a lid.” }

The\marginnote{12.2.4} lids fell off. 

\scrule{“I allow you to tie it onto the ointment box with a string.” }

The\marginnote{12.2.7} ointment boxes split. 

\scrule{“I allow you to sew it together with thread.” }

At\marginnote{12.3.1} that time the monks put the ointment on with their fingers. As a result their eyes hurt. 

\scrule{“I allow an ointment stick.”\footnote{Appendix of Medical Terminology for a list of allowable medical equipment. } }

Soon\marginnote{12.3.4} afterwards the monks from the group of six used luxurious ointment sticks made with gold or silver. People complained and criticized them, “They’re just like householders who indulge in worldly pleasures!” They told the Buddha. 

\scrule{“You shouldn’t use luxurious ointment sticks. If you do, you commit an offense of wrong conduct. }

\scrule{I allow ointment sticks made of bone, ivory, horn, reed, bamboo, wood, resin, fruit, metal, and shell.” }

The\marginnote{12.4.1} monks dropped the ointment sticks on the ground. The sticks became rough. 

\scrule{“I allow a case for the ointment stick.”\footnote{Reading \textit{\textsanskrit{salākodhāniya}} with the PTS edition. } }

The\marginnote{12.4.4} monks carried the ointment boxes and sticks in their hands. 

\scrule{“I allow a bag for the ointment box.” }

They\marginnote{12.4.7} did not have a shoulder strap. 

\scrule{“I allow a shoulder strap and a string for tying it.”\footnote{Vin-\textsanskrit{ālaṅ}-\textsanskrit{ṭ} 34.67: \textit{\textsanskrit{Añjanitthavikāya} \textsanskrit{aṁse} \textsanskrit{lagganatthāya} \textsanskrit{aṁsabaddhakampi} bandhanasuttakampi \textsanskrit{vaṭṭati}}, “A shoulder strap and also a \textit{bandhanasuttaka} is allowed for the purpose of the hanging of the ointment-box bag from the shoulder.” } }

At\marginnote{13.1.1} one time Venerable Pilindavaccha had a headache. 

\scrule{“I allow oil for the head.” }

He\marginnote{13.1.4} did not get better. 

\scrule{“I allow treatment through the nose.” }

The\marginnote{13.1.7} oil dripped from the nose. 

\scrule{“I allow a nose dropper.” }

Soon\marginnote{13.1.10} afterwards the monks from the group of six used luxurious nose droppers made with gold or silver. People complained and criticized them, “They’re just like householders who indulge in worldly pleasures!” They told the Buddha. 

\scrule{“You shouldn’t use luxurious nose droppers. If you do, you commit an offense of wrong conduct. }

\scrule{I allow nose droppers made of bone, ivory, horn, reed, bamboo, wood, resin, fruit, metal, and shell.” }

The\marginnote{13.2.1} nose dropper dripped unevenly. 

\scrule{“I allow a double nose dropper.” }

He\marginnote{13.2.4} did not get better. 

\scrule{“I allow you to inhale smoke.” }

They\marginnote{13.2.7} just lit the wick and inhaled the smoke. They burned their throat. 

\scrule{“I allow a tube.” }

Soon\marginnote{13.2.10} the monks from the group of six used luxurious tubes made with gold or silver. People complained and criticized them, “They’re just like householders who indulge in worldly pleasures!” They told the Buddha. 

\scrule{“You shouldn’t use luxurious tubes. If you do, you commit an offense of wrong conduct. }

\scrule{I allow tubes made of bone, ivory, horn, reed, bamboo, wood, resin, fruit, metal, and shell.” }

At\marginnote{13.2.18} that time the tubes were not covered. Insects crawled inside of them. 

\scrule{“I allow a lid.” }

At\marginnote{13.2.21} that time the monks carried the tubes in their hands. 

\scrule{“I allow a bag for the tubes.” }

The\marginnote{13.2.24} tubes scratched each other. 

\scrule{“I allow a bag with two compartments.” }

They\marginnote{13.2.27} did not have a shoulder strap. 

\scrule{“I allow a shoulder strap and a string for fastening it.” }

\subsection*{Allowable medical treatments and more}

At\marginnote{14.1.1} one time Venerable Pilindavaccha had a certain disease.\footnote{\textit{\textsanskrit{Vātābādha}}, literally, “a wind disease”. According to the Indian system of classification, this included a number of illnesses, such as arthritis. Since intestinal gas is elsewhere called \textit{\textsanskrit{udaravātābādha}}, “stomach wind disease”, it is unclear what is meant in this context. The commentaries are silent. } The doctors said he needed a heated concoction of oil. 

\scrule{“I allow a heated concoction of oil.”\footnote{Sp 3.267: \textit{\textsanskrit{Anujānāmi} bhikkhave \textsanskrit{telapākanti} \textsanskrit{yaṅkiñci} \textsanskrit{bhesajjapakkhittaṁ} \textsanskrit{sabbaṁ} \textsanskrit{anuññātameva} hoti}, “I allow \textit{\textsanskrit{telapāka}}: whatever medicines are included are all allowed.” } }

They\marginnote{14.1.6} wanted to add alcohol to that concoction. 

\scrule{“I allow alcohol in a heated concoction of oil.” }

Soon\marginnote{14.1.9} afterwards the monks from the group of six heated oil with too much alcohol. They drank it and became drunk. 

\scrule{“You shouldn’t drink heated oil with too much alcohol. If you do, you should be dealt with according to the rule. }

\scrule{I allow you to drink heated oil if there is no discernible color, smell, or taste of alcohol.” }

The\marginnote{14.2.1} monks had heated much oil with too much alcohol. They did not know what to do with it. 

\scrule{“I allow you to determine it for external use.” }

Pilindavaccha\marginnote{14.2.5} had more heated oil, but there was no vessel for storing it. 

\scrule{“I allow three kinds of vessels: made of metal, made of wood, made of fruit.”\footnote{Vmv 3.305: \textit{Phalatumbo \textsanskrit{nāma} \textsanskrit{lābuādi}}, “A vessel made of fruit is a gourd, etc.” } }

At\marginnote{14.3.1} that time Pilindavaccha had arthritis of the hands and feet.\footnote{“Arthiritis of the hands and feet” renders \textit{\textsanskrit{aṅgavāta}}, literally “wind of the limbs”. I follow the commentarial explanation at Sp 3.267: \textit{\textsanskrit{Aṅgavātoti} \textsanskrit{hatthapāde} \textsanskrit{vāto}}, “\textit{\textsanskrit{Aṅgavāta}} means wind in the hands and the feet.” } 

\scrule{“I allow treatment through sweating.” }

He\marginnote{14.3.4} did not get better. 

\scrule{“I allow sweating with herbs.”\footnote{Sp 3.267: \textit{\textsanskrit{Sambhārasedanti} \textsanskrit{nānāvidhapaṇṇabhaṅgasedaṁ}}, “\textit{\textsanskrit{Sambhārasedanti}}: sweating with various shredded leaves.” } }

He\marginnote{14.3.7} still did not get better. 

\scrule{“I allow heavy sweating.” }

He\marginnote{14.3.10} still did not get better. 

\scrule{“I allow hemp water.”\footnote{Sp 3.267: \textit{\textsanskrit{Bhaṅgodakanti} \textsanskrit{nānāpaṇṇabhaṅgakuthitaṁ} \textsanskrit{udakaṁ}; tehi \textsanskrit{paṇṇehi} ca udakena ca \textsanskrit{siñcitvā} \textsanskrit{siñcitvā} sedetabbo}, “\textit{\textsanskrit{Bhaṅgodaka}}: water with various putrid, shredded leaves. One is made to sweat by repeated pouring the water and the leaves.” The commentary brings in the idea of sweating, saying that the hemp water was for external use, yet neither is mentioned in the Canonical text. In fact, although the use of \textsanskrit{bhaṅgodaka} in the Canonical text is immediately preceded by the three separate treatments that involved sweating (\href{https://suttacentral.net/pli-tv-kd6/en/brahmali\#14.3.3}{Kd 6:14.3.3}–14.3.9), it is not mentioned in connection with \textsanskrit{bhaṅgodaka}. Moreover, the commentary interprets \textsanskrit{bhaṅga} to mean shredded (leaves), with the idea of leaves merely implied. The more straightforward interpretation is that \textsanskrit{bhaṅga} refers to hemp, which is how I. B. Horner understands it. It seems possible, then, that this refers to hemp water, or cannabis water, that was taken as an internal medicine. Given that cannabis is known to alleviate arthritis symptoms, this is perhaps not as surprising as it may seem. } }

He\marginnote{14.3.13} still did not get better. 

\scrule{“I allow a bathtub.”\footnote{Sp 3.267: \textit{\textsanskrit{Udakakoṭṭhakanti} \textsanskrit{udakakoṭṭhe} \textsanskrit{cāṭiṁ} \textsanskrit{vā} \textsanskrit{doṇiṁ} \textsanskrit{vā} \textsanskrit{uṇhodakassa} \textsanskrit{pūretvā} tattha \textsanskrit{pavisitvā} \textsanskrit{sedakammakaraṇaṁ} \textsanskrit{anujānāmīti} attho}, “\textit{\textsanskrit{Udakakoṭṭhaka}}: the meaning is ‘I allow the causing of sweating by entering a tank or trough filled with hot water in a bathroom.’” } }

Pilindavaccha\marginnote{14.4.1} had arthritis.\footnote{“Arthiritis” renders \textit{\textsanskrit{pabbavāta}} Sp 3.267: \textit{\textsanskrit{Pabbavāto} \textsanskrit{hotīti} pabbe pabbe \textsanskrit{vāto} vijjhati}, “\textit{\textsanskrit{Pabbavāta}} means wind piercing in the various joints.” Although the exact meaning of \textit{\textsanskrit{vāta}} is not specified, it typically refers to arthritis. See SED. Here it presumably refers to joints other than the hands and feet, which are mentioned above. } 

\scrule{“I allow bloodletting.” }

He\marginnote{14.4.4} did not get better. 

\scrule{“I allow bloodletting and receiving it in a horn.”\footnote{I have not been able to trace any explanation of this seemingly strange practice, either in the commentaries or elsewhere. } }

Pilindavaccha\marginnote{14.4.7} had cracked feet. 

\scrule{“I allow salve for the feet.” }

He\marginnote{14.4.10} did not get better. 

\scrule{“I allow you to make foot salve.” }

At\marginnote{14.4.13} that time a monk was afflicted with abscesses.\footnote{“Abscess” renders \textit{\textsanskrit{gaṇḍa}}. For a discussion of this word, see Appendix of Technical Terms. } 

\scrule{“I allow surgery.” }

They\marginnote{14.4.16} needed bitter water. 

\scrule{“I allow bitter water.” }

They\marginnote{14.4.19} needed sesame paste. 

\scrule{“I allow sesame paste.” }

They\marginnote{14.5.1} needed flour paste. 

\scrule{“I allow flour paste.”\footnote{Sp 3.267: \textit{\textsanskrit{Kabaḷikanti} \textsanskrit{vaṇamukhe} \textsanskrit{sattupiṇḍaṁ} \textsanskrit{pakkhipituṁ}}, “\textit{\textsanskrit{Kabaḷika}} means to place a lump of flour on the sore.” Vmv 3.267: \textit{\textsanskrit{Kabaḷikāti} \textsanskrit{upanāhabhesajjaṁ}}, “\textit{\textsanskrit{Kabaḷika}}: a lasting medicine.” The definition in DOP is not quite right. } }

They\marginnote{14.5.4} needed a dressing. 

\scrule{“I allow a dressing.” }

The\marginnote{14.5.7} sore was itching. 

\scrule{“I allow you to sprinkle it with mustard powder.” }

The\marginnote{14.5.10} sore festered. 

\scrule{“I allow you to fumigate it.” }

The\marginnote{14.5.13} flesh protruded.\footnote{Sp 3.267: \textit{\textsanskrit{Vaḍḍhamaṁsanti} \textsanskrit{adhikamaṁsaṁ} \textsanskrit{āṇi} viya \textsanskrit{uṭṭhahati}}, “\textit{\textsanskrit{Vaḍḍhamaṁsa}} means an excess of flesh was sticking out like a peg.” } 

\scrule{“I allow you to cut it with a razor.”\footnote{Sp 3.267: \textit{\textsanskrit{Loṇasakkharikāya} chinditunti khurena \textsanskrit{chindituṁ}}, “\textit{\textsanskrit{Loṇasakkharikāya} \textsanskrit{chindituṁ}} means to cut with a razor.” } }

The\marginnote{14.5.16} sore did not heal. 

\scrule{“I allow oil for the sore.” }

The\marginnote{14.5.19} oil dripped off. 

\scrule{“I allow a bandage and all treatments for sores.” }

On\marginnote{14.6.1} one occasion a certain monk was bitten by a snake. 

\scrule{“I allow you to give him the four foul edibles: feces, urine, ash, and clay.” }

The\marginnote{14.6.5} monks thought, “Do they need to be received or not?”\footnote{The question seems to be whether these can be given to the bitten person in an unreceived state, \textit{\textsanskrit{appaṭiggahitāni}}, or whether they have to be received first, \textit{\textsanskrit{paṭiggahetabbānī}}. } 

\scrule{“They should be received if there is an attendant. If there isn’t, I allow you to take them yourself and then eat them.”\footnote{“Should” renders \textit{\textsanskrit{anujānāmi}}. For a discussion of this word, see Appendix of Technical Terms. } }

On\marginnote{14.6.9} one occasion a monk had drunk poison. 

\scrule{“I allow you to give him feces to drink.” }

The\marginnote{14.6.12} monks thought, “Does it need to be received or not?” 

\scrule{“I allow the one who is excreting it to receive it. When he’s received it, it doesn’t need to be received again.” }

On\marginnote{14.7.1} one occasion a monk was sick from a drug.\footnote{Sp 3.269: \textit{\textsanskrit{Gharadinnakābādhoti} \textsanskrit{vasīkaraṇapānakasamuṭṭhitarogo}}, “\textit{\textsanskrit{Gharadinnakābādha}} is a sickness coming from drinking an overpowering drink.” Sp-\textsanskrit{ṭ} 3.269: \textit{\textsanskrit{Gharadinnakābādho} \textsanskrit{nāma} \textsanskrit{vasīkaraṇatthāya} \textsanskrit{gharaṇiyā} \textsanskrit{dinnabhesajjasamuṭṭhito} \textsanskrit{ābādho}}, “\textit{\textsanskrit{Gharadinnakābādha}} is the name of a sickness coming from medicine given by a housewife for the purpose of overpowering.” The point seems to be that one is given a substance so that one can subsequently be overpowered. } 

\scrule{“I allow him to drink mud from a plow.”\footnote{\textit{\textsanskrit{Sītāloḷī}} literally means “what is mixed in a furrow”. Sp 3.269: \textit{\textsanskrit{Sītāloḷinti} \textsanskrit{naṅgalena} kasantassa \textsanskrit{phāle} \textsanskrit{laggamattikaṁ} udakena \textsanskrit{āloḷetvā} \textsanskrit{pāyetuṁ} \textsanskrit{anujānāmīti} attho}, “\textit{\textsanskrit{Sītāloḷī}}: the meaning is ‘I allow you to drink a mixture of water and the clay sticking to a plowshare of one plowing with a plow.’” } }

On\marginnote{14.7.4} one occasion a certain monk had indigestion.\footnote{Sp 3.269: \textit{\textsanskrit{Duṭṭhagahaṇikoti} \textsanskrit{vipannagahaṇiko}; kicchena \textsanskrit{uccāro} \textsanskrit{nikkhamatīti} attho}, “\textit{\textsanskrit{Duṭṭhagahaṇiko}}: one whose stomach has failed; the meaning is he has difficulty excreting feces.” } 

\scrule{“I allow him to drink lye.” }

On\marginnote{14.7.7} one occasion a certain monk suffered from jaundice. 

\scrule{“I allow him to drink chebulic myrobalan soaked in cattle urine.” }

On\marginnote{14.7.10} one occasion a certain monk suffered from a skin disease. 

\scrule{“I allow you to make a scented ointment.” }

On\marginnote{14.7.13} one occasion a monk’s body was full of impurities.\footnote{Sp 3.269: \textit{\textsanskrit{Abhisannakāyoti} \textsanskrit{ussannadosakāyo}}, “\textit{\textsanskrit{Abhisannakāya}} means the body is full of impurities.” } 

\scrule{“I allow him to drink a purgative.” }

He\marginnote{14.7.16} needed clear congee. 

\scrule{“I allow clear congee.”\footnote{Sp 3.269: \textsanskrit{Acchakañjiyanti} \textsanskrit{taṇḍulodakamaṇḍo}, “\textit{\textsanskrit{Acchakañjiya}}: the cream of rice water.” } }

He\marginnote{14.7.19} needed mung-bean broth. 

\scrule{“I allow mung-bean broth.”\footnote{Sp 3.269: \textit{\textsanskrit{Akaṭayusanti} asiniddho \textsanskrit{muggapacitapānīyo}}, “\textit{\textsanskrit{Akaṭayūsa}} is drinkable mung beans that have been boiled without oil.” Sp-\textsanskrit{ṭ} 3.269, however, says: \textit{\textsanskrit{Akaṭayūsenāti} \textsanskrit{anabhisaṅkhatena} \textsanskrit{muggayūsena}}, “\textit{\textsanskrit{Akaṭayūsena}} means the juice of unprepared mung beans.” This would seem to mean the raw juice of mung beans. I follow the more ancient authority. } }

He\marginnote{14.7.22} needed oily mung-bean broth. 

\scrule{“I allow oily mung-bean broth.”\footnote{Sp 3.269: \textit{\textsanskrit{Kaṭākaṭanti} sova dhotasiniddho}, “\textit{\textsanskrit{Kaṭākaṭa}} the same (as the previous) but washed in oil.” Sp-\textsanskrit{ṭ} 3.269, however, says: \textit{\textsanskrit{Kaṭākaṭenāti} mugge \textsanskrit{pacitvā} \textsanskrit{acāletvāva} \textsanskrit{parissāvitena} \textsanskrit{muggasūpenāti}}, “\textit{\textsanskrit{Kaṭākaṭa}} means mung-bean soup made by boiling mung beans and then filtering it without stirring.” But this seems indistinguishable from the previous medicine, the mung-bean broth. } }

He\marginnote{14.7.25} needed meat broth. 

\scrule{“I allow meat broth.”\footnote{Sp 3.269: \textit{\textsanskrit{Paṭicchādanīyenāti} \textsanskrit{maṁsarasena}}, “\textit{\textsanskrit{Paṭicchādanīyena}} means having the juice of meat.” } }

\section*{3. The account of Pilindavaccha }

At\marginnote{15.1.1} one time Venerable Pilindavaccha was having a hillside cleared near \textsanskrit{Rājagaha}, intending to build a shelter. Just then King Seniya \textsanskrit{Bimbisāra} of Magadha went to Pilindavaccha, bowed, sat down, and said, “Venerable, what are you having made?” 

“I’m\marginnote{15.1.5} clearing the hillside, great king. I want to build a shelter.” 

“Do\marginnote{15.1.6} you need a monastery worker?” 

“The\marginnote{15.1.7} Buddha hasn’t allowed monastery workers.” 

“Well\marginnote{15.1.8} then, sir, please ask the Buddha and tell me the outcome.” 

“Yes.”\marginnote{15.1.9} 

Pilindavaccha\marginnote{15.2.1} instructed, inspired, and gladdened King \textsanskrit{Bimbisāra} with a teaching, after which the king got up from his seat, bowed down, circumambulated Pilindavaccha with his right side toward him, and left. 

Soon\marginnote{15.2.3} afterwards Pilindavaccha sent a message to the Buddha: “Sir, King Seniya \textsanskrit{Bimbisāra} of Magadha wishes to provide a monastery worker. What should I tell him?” The Buddha then gave a teaching and addressed the monks: 

\scrule{“Monks, I allow monastery workers.” }

Once\marginnote{15.3.1} again King \textsanskrit{Bimbisāra} went to Pilindavaccha, bowed, sat down, and said, “Sir, has the Buddha allowed monastery workers?” 

“Yes,\marginnote{15.3.4} great king.” 

“Well\marginnote{15.3.5} then, I’ll provide you with a monastery worker.” 

Yet\marginnote{15.3.6} after making this promise, he forgot, and only remembered after a long time. He then addressed the official in charge of practical affairs: “Listen, has the monastery worker I promised been provided?” 

“No,\marginnote{15.3.8} sir, he hasn’t.” 

“How\marginnote{15.3.9} long has it been since we made that promise?” 

The\marginnote{15.4.1} official counted the days and said, “It’s been five hundred days.” 

“Well\marginnote{15.4.3} then, provide him with five hundred monastery workers.” 

“Yes.”\marginnote{15.4.4} 

The\marginnote{15.4.5} official provided Pilindavaccha with those monastery workers, and a separate village was established. They called it “The Monastery Workers’ Village” and “Pilinda Village”. And Pilindavaccha began associating with the families in that village. 

After\marginnote{15.4.8} robing up one morning, he took his bowl and robe and went to Pilinda Village for alms. At that time they were holding a celebration in that village, and the children were dressed up in ornaments and garlands. As Pilindavaccha was walking on continuous almsround, he came to the house of a certain monastery worker where he sat down on the prepared seat. Just then the daughter of the house had seen the other children dressed up in ornaments and garlands. She cried, saying, “I want a garland! I want ornaments!” Pilindavaccha asked her mother why the girl was crying. She told him, adding, “Poor people like us can’t afford garlands and ornaments.” Pilindavaccha then took a pad of grass and said to the mother, “Here, place this on the girl’s head.” She did, and it turned into a beautiful golden garland. Even the royal compound had nothing like it. 

People\marginnote{15.6.6} told King \textsanskrit{Bimbisāra}, “Sir, in the house of a such-and-such a monastery worker there’s a beautiful golden garland. Even in your court, sir, there’s nothing like it. So how did those poor people get it? They must have stolen it.” King \textsanskrit{Bimbisāra} then had that family imprisoned. 

Once\marginnote{15.7.1} again Pilindavaccha robed up in the morning, took his bowl and robe, and went to Pilinda Village for alms. As he was walking on continuous almsround, he came to the house of that same monastery worker. He then asked the neighbors what had happened to that family. 

“The\marginnote{15.7.4} king had jailed them, venerable, because of that golden garland.” 

Pilindavaccha\marginnote{15.7.5} went to King \textsanskrit{Bimbisāra}’s house, where he sat down on the prepared seat. King \textsanskrit{Bimbisāra} approached Pilindavaccha, bowed, and sat down. Pilindavaccha said, “Great king, why have you jailed the family of that monastery worker?” 

“Sir,\marginnote{15.8.2} in the house of that monastery worker there was a beautiful golden garland. Even the royal compound has nothing like it. So how did those poor people get it? They must have stolen it.” 

Pilindavaccha\marginnote{15.8.6} then focused his mind on turning King \textsanskrit{Bimbisāra}’s stilt house into gold. As a result, the whole house became gold. He said, “Great king, how did you get so much gold?” 

“Understood,\marginnote{15.8.9} sir! It’s your supernormal power.” And he released that family. 

People\marginnote{15.9.1} said, “They say Venerable Pilindavaccha has performed a superhuman feat, a wonder of supernormal power, for the king and his court!” Delighted, and gaining confidence in Pilindavaccha, they brought him the five tonics: ghee, butter, oil, honey, and syrup. Ordinarily, too, Pilindavaccha was getting the five tonics. Since he was getting so much, he gave it away to his followers, who ended up with an abundance of tonics. After filling up basins and waterpots and setting these aside, they filled their water filters and bags and hung these in the windows. But as the tonics dripped, the dwellings became infested with rats. When people walking about the dwellings noticed this, they complained and criticized them, “These Sakyan monastics are hoarding goods indoors, just like King Seniya \textsanskrit{Bimbisāra} of Magadha!” 

The\marginnote{15.10.1} monks heard the complaints of those people and the monks of few desires complained and criticized those monks, “How can these monks choose to live with such abundance?” 

After\marginnote{15.10.4} rebuking those monks in many ways, they told the Buddha. Soon afterwards he had the Sangha gathered and questioned the monks: “Is it true, monks, that there are monks who live like this?” “It’s true, sir.” … After rebuking them, the Buddha gave a teaching and addressed the monks: 

\scrule{“After being received, the tonics allowable for sick monks—that is, ghee, butter, oil, honey, and syrup—should be used from storage for at most seven days. If you use them longer than that, you should be dealt with according to the rule.” }

\scend{The first section for recitation on allowable medicines is finished. }

\section*{4. The allowance for sugar, etc. }

When\marginnote{16.1.1} the Buddha had stayed at \textsanskrit{Sāvatthī} for as long as he liked, he set out wandering toward \textsanskrit{Rājagaha}. While they were traveling, Venerable Revata the Doubter saw a sugar factory. As he approached, he noticed they were mixing the sugar with flour and ash. He thought,\footnote{Sp-\textsanskrit{ṭ} 3.272: \textit{\textsanskrit{Guḷakaraṇanti} \textsanskrit{guḷakaraṇaṭṭhānaṁ}, \textsanskrit{ucchusālanti} \textsanskrit{vuttaṁ} hoti}, “\textit{\textsanskrit{Guḷakaraṇan}}: a place for making sugar. It is called a sugar-cane building.” } “Sugar mixed with food is unallowable, and so it’s unallowable to eat sugar at the wrong time,” and being afraid of wrongdoing, he and his followers did not take sugar. They told the Buddha. “Why are they adding flour and ash to the sugar?” 

“To\marginnote{16.1.8} harden it, sir.” 

\scrule{“If they add flour or ash to sugar to harden it, it’s still considered sugar. I allow you to eat as much sugar as you like.” }

While\marginnote{16.2.1} still traveling, Revata noticed mung beans sprouting from feces. He thought, “Mung beans are unallowable. They sprout even after being digested,” and being afraid of wrongdoing, he and his followers did not eat mung beans. They told the Buddha. 

\scrule{“Although mung beans may sprout after being digested, I allow you to eat as much of it as you like.” }

On\marginnote{16.3.1} one occasion a certain monk who had a stomachache drank a salty purgative and was cured. 

\scrule{“I allow salty purgatives when you’re sick. If you’re not sick, I allow you to drink it mixed with water.” }

\section*{5. Discussion of the prohibition against storing indoors, etc. }

Wandering\marginnote{17.1.1} on, the Buddha eventually arrived at \textsanskrit{Rājagaha} where he stayed in the Bamboo Grove, the squirrel sanctuary. 

Soon\marginnote{17.1.2} afterwards the Buddha had a stomachache. Venerable Ānanda thought, “Previously, when the Buddha had a stomachache, he was comfortable after drinking the threefold pungent congee.”\footnote{\textit{\textsanskrit{Tekaṭulayāgu}} is commonly rendered as “rice porridge having three pungent ingredients”. The three are sesame seeds, rice, and mung beans, yet rice and mung beans can hardly be called pungent. I would suggest it is the taste of the combination of the three that is pungent. } He then asked for sesame seeds, rice, and mung beans, stored them indoors, cooked them himself indoors, and brought them to the Buddha, saying, “Sir, please drink the threefold pungent congee.” 

When\marginnote{17.2.1} Buddhas know what is going on, sometimes they ask and sometimes not. They know the right time to ask and when not to ask. Buddhas ask when it is beneficial, otherwise not, for Buddhas are incapable of doing what is unbeneficial.\footnote{“Incapable of doing” renders \textit{\textsanskrit{setughāta}}, literally, “destroyed the bridge”. Sp 1.16: \textit{Setu vuccati maggo, maggeneva \textsanskrit{tādisassa} vacanassa \textsanskrit{ghāto}, samucchedoti \textsanskrit{vuttaṁ} hoti}, “The path is called the bridge. What is said is that there is the destruction and cutting off of such speech by the path.” The commentary seems to take \textit{setu}, “bridge”, as a reference to the eightfold path. I prefer to understand “bridge” as a metaphor for access, that is, the Buddhas no longer have the possibility of doing what is unbeneficial. } Buddhas question the monks for two reasons: to give a teaching or to lay down a training rule. 

So\marginnote{17.2.6} he said to Ānanda, “Ānanda, where does this congee come from?” Ānanda told him. 

The\marginnote{17.3.1} Buddha rebuked him, “It’s not suitable, Ānanda, it’s not proper, it’s not worthy of a monastic, it’s not allowable, it’s not to be done. How could you be so indulgent? What’s been stored indoors in a monastery is unallowable;\footnote{Sp 3.274: \textit{Antovutthanti \textsanskrit{akappiyakuṭiyaṁ} \textsanskrit{vutthaṁ}}, “\textit{Antovuttha}: stored, apart from the food-storage hut.” } what’s been cooked indoors in a monastery is unallowable;\footnote{\textsanskrit{Khuddasikkhā}-\textsanskrit{abhinavaṭīkā} 112: \textit{Antopakketi \textsanskrit{akappiyakuṭiyā} anto pakke}, “\textit{Antopakka}: cooked indoors, apart from the food-storage hut.” } what’s been cooked by oneself is unallowable. This will affect people’s confidence …” After rebuking him, he gave a teaching and addressed the monks: 

\scrule{“You shouldn’t eat what’s been stored indoors in a monastery, what’s been cooked indoors in a monastery, or what you have cooked yourself. If you do, you commit an offense of wrong conduct. }

\scrule{If you eat what’s been stored indoors in a monastery, cooked indoors in a monastery, and cooked by yourselves, you commit three offenses of wrong conduct. }

\scrule{If you eat what’s been stored indoors in a monastery, cooked indoors in a monastery, but cooked by others, you commit two offenses of wrong conduct. }

\scrule{If you eat what’s been stored indoors in a monastery, but cooked outside, yet cooked by yourselves, you commit two offenses of wrong conduct. }

\scrule{If you eat what’s been stored outside, but cooked indoors in a monastery, and cooked by yourselves, you commit two offenses of wrong conduct. }

\scrule{If you eat what’s been stored indoors in a monastery, but cooked outside, and cooked by others, you commit one offense of wrong conduct. }

\scrule{If you eat what’s been stored outside, but cooked indoors in a monastery, yet cooked by others, you commit one offense of wrong conduct. }

\scrule{If you eat what’s been stored outside, and cooked outside, but cooked by yourselves, you commit one offense of wrong conduct. }

\scrule{If you eat what’s been stored outside, and cooked outside, and cooked by others, there is no offense.” }

When\marginnote{17.6.1} the monks heard that the Buddha had prohibited cooking, being afraid of wrongdoing, they did not reheat. 

\scrule{“I allow you to reheat what’s already been cooked.” }

At\marginnote{17.7.1} that time \textsanskrit{Rājagaha} was short of food. People brought salt, oil, rice, and fresh food to the monastery.\footnote{“Fresh food” renders \textit{\textsanskrit{khādanīya}}. See Appendix of Technical Terms. } The monks stored it outdoors, but it was eaten by vermin and stolen by thieves. 

\scrule{“I allow you to store food indoors.” }

The\marginnote{17.7.7} monks stored it indoors, but it was cooked outside. They were surrounded by scrap-eaters, and the monks ate in fear. 

\scrule{“I allow cooking indoors.” }

Because\marginnote{17.7.12} of the famine, the attendants took more for themselves and gave less to the monks. 

\scrule{“I allow you to cook. I allow you to store food indoors in a monastery, to cook indoors in a monastery, and to cook yourselves.” }

\section*{6. Receiving what has been picked up }

On\marginnote{17.8.1} one occasion a number of monks who had completed the rainy-season residence in \textsanskrit{Kāsi} were traveling to \textsanskrit{Rājagaha} to visit the Buddha. While on their way, they did not receive sufficient food, whether fine or coarse. Yet there was much fruit, but no attendant to offer it. 

When\marginnote{17.8.4} the monks arrived at \textsanskrit{Rājagaha}, they were exhausted. They went to the Bamboo Grove, approached the Buddha, bowed, and sat down. Since it is the custom for Buddhas to greet newly-arrived monks, he said to them, “I hope you’re keeping well, monks, I hope you’re getting by?  I hope you’re not tired from traveling? And where have you come from?” 

“We’re\marginnote{17.9.1} keeping well, sir, we’re getting by,” and they told him what had happened. Soon afterwards the Buddha gave a teaching and addressed the monks: 

\scrule{“If there is no attendant, but you see fruit, I allow you to pick it up yourself. You should then carry it until you see an attendant, put it on the ground, and have it received. You may then eat it. I allow you to receive what you have picked up.” }

On\marginnote{18.1.1} one occasion a certain brahmin had obtained fresh sesame seed and fresh honey. He thought, “Why don’t I give this to the Sangha of monks headed by the Buddha?” He then went to the Buddha, exchanged pleasantries with him, and said, “Please accept tomorrow’s meal from me together with the Sangha of monks.” The Buddha consented by remaining silent. Knowing that the Buddha had consented, the brahmin left. 

The\marginnote{18.2.1} following morning that brahmin had various kinds of fine foods prepared. He then had the Buddha informed that the meal was ready. 

The\marginnote{18.2.3} Buddha robed up, took his bowl and robe, and went to that brahmin’s house where he sat down on the prepared seat. That brahmin personally served various kinds of fine foods to the Sangha of monks headed by the Buddha. When the Buddha had finished his meal, the brahmin sat down to one side. The Buddha then instructed, inspired, and gladdened him with a teaching, after which he got up and left. 

Soon\marginnote{18.3.1} after the Buddha had left, that brahmin thought, “I invited the Sangha of monks headed by the Buddha to give them the fresh sesame seed and honey, but I forgot. Why don’t I take the sesame seed and honey to the monastery in basins and waterpots?” And he did just that. He then went up to the Buddha and said, “When I invited you for a meal, I forgot to give you these fresh sesame seeds and this honey. Please accept it.” 

“Well\marginnote{18.4.3} then, brahmin, give it to the monks.” 

At\marginnote{18.4.4} this time, food was scarce, and the monks were refusing invitations to eat more even after taking just a little. After reflection, they were even declining altogether.\footnote{“The monks refused an invitation to eat more even after taking just a little” renders \textit{appamattakepi \textsanskrit{pavārenti}}. The phrase is explained in the sub-commentary at Sp-\textsanskrit{ṭ} 3.276: \textit{Appamattakepi \textsanskrit{pavārentīti} appamattakepi gahite \textsanskrit{pavārenti}, “bahumhi gahite \textsanskrit{aññesaṁ} \textsanskrit{nappahotī}”ti \textsanskrit{maññamānā}}, “\textit{Appamattakepi \textsanskrit{pavārenti}}: even when they have taken just a little, they refuse an invitation to eat more, thinking, ‘If much is taken, there will not be enough for others.’” } And so now that the whole Sangha was being invited, being afraid of wrongdoing, they did not accept.\footnote{Apart from their restraint due to the scarcity of food, they did not want to fall into an offense under \href{https://suttacentral.net/pli-tv-bu-vb-pc35/en/brahmali\#2.15.1}{Bu Pc 35:2.15.1}. } 

\scrule{“Accept, monks, and eat. I allow one who has finished his meal and refused an invitation to eat more to eat non-leftovers that have been brought out.”\footnote{Sp 3.276: \textit{Tato \textsanskrit{nīhaṭanti} yattha \textsanskrit{nimantitā} \textsanskrit{bhuñjanti}, tato \textsanskrit{nīhaṭaṁ}}, “\textit{Tato \textsanskrit{nīhaṭan}}: brought out from where they ate their invitational meal.” } }

\section*{7. The allowance for what has been received, etc. }

On\marginnote{19.1.1} one occasion a family who was supporting Venerable Upananda the Sakyan sent fresh food to the Sangha, saying, “After showing it to Venerable Upananda, it’s to be given to the Sangha.” Just then Upananda had gone to the village for alms. When those people arrived at the monastery, they asked for Upananda and were told where he was. They said, “Venerables, after showing it to Venerable Upananda, this fresh food is to be given to the Sangha.” The monks told the Buddha. He said, “Well then, receive it and put it aside until Upananda returns.” But because Upananda visited families before eating, he returned late to the monastery. 

At\marginnote{19.2.2} this time, food was scarce, and the monks were refusing invitations to eat more even after taking just a little. After reflection, they were even declining altogether. And so now that the whole Sangha was being invited, being afraid of wrongdoing, they did not accept. 

\scrule{“Accept, monks, and eat. I allow one who has finished his meal and refused an invitation to eat more to eat non-leftovers that were received before the meal.” }

When\marginnote{20.1.1} the Buddha had stayed at \textsanskrit{Rājagaha} for as long as he liked, he set out wandering toward \textsanskrit{Sāvatthī}. When he eventually arrived, he stayed in the Jeta Grove, \textsanskrit{Anāthapiṇḍika}’s Monastery. 

At\marginnote{20.1.4} that time Venerable \textsanskrit{Sāriputta} had a fever. Venerable \textsanskrit{Mahāmoggallāna} went to him and asked, “When you previously had a fever, \textsanskrit{Sāriputta}, how did you get better?” 

“I\marginnote{20.1.7} had lotus roots and tubers.” 

Then,\marginnote{20.1.8} just as a strong man might bend or stretch his arm, \textsanskrit{Mahāmoggallāna} disappeared from the Jeta Grove and reappeared on the banks of the \textsanskrit{Mandākinī} lotus pond. An elephant saw \textsanskrit{Mahāmoggallāna} coming and said to him, “Welcome, Venerable \textsanskrit{Mahāmoggallāna}, please come. What do you need, venerable? What may I give?” 

“I\marginnote{20.2.5} need lotus roots and tubers.” 

The\marginnote{20.2.6} elephant told another elephant, “Listen, give as many roots and tubers as the venerable needs.” It plunged into the \textsanskrit{Mandākinī} lotus pond and pulled up lotus roots and tubers with his trunk. It gave them a good rinse, bound them in a bundle, and went up to \textsanskrit{Mahāmoggallāna}. Then, just as a strong man might bend or stretch his arm, \textsanskrit{Mahāmoggallāna} disappeared from the banks of the \textsanskrit{Mandākinī} lotus pond and reappeared in the Jeta Grove. And that elephant did the same. It had the roots and tubers offered to \textsanskrit{Mahāmoggallāna}, before returning to the \textsanskrit{Mandākinī} lotus pond in the same manner. \textsanskrit{Mahāmoggallāna} then brought those lotus roots and tubers to \textsanskrit{Sāriputta}. When he had eaten them, his fever subsided. But there was much left over. 

At\marginnote{20.4.1} this time, food was scarce, and the monks were refusing invitations to eat more even after taking just a little. After reflection, they were even declining altogether. And so now that the whole Sangha was being invited, being afraid of wrongdoing, they did not accept. 

\scrule{“Accept, monks, and eat. I allow one who has finished his meal and refused an invitation to eat more to eat non-leftovers coming from the forest or a lotus pond.”\footnote{“Non-leftovers” is here used in its technical sense of \href{https://suttacentral.net/pli-tv-bu-vb-pc35/en/brahmali\#3.1.9}{Bu Pc 35:3.1.9}. In other words, the lotus roots and tubers were leftovers that were turned into non-leftovers through the appropriate procedure. } }

On\marginnote{21.1.1} one occasion in \textsanskrit{Sāvatthī}, much fruit had been given, but there was no attendant. Being afraid of wrongdoing, the monks did not eat it. 

\scrule{“I allow you to eat fruit that hasn’t been made allowable if it’s seedless or the seeds have been removed.”\footnote{“The seeds have been removed” renders \textit{\textsanskrit{nibbattabīja}}. \textit{Nibbatta} is a past participle that normally means “come into being” or “developed”. According to SED (sv. \textit{nir-\textsanskrit{vṛit}}), however, it can also mean “removed”. Sp 3.278: \textit{\textsanskrit{Nibbaṭṭabījanti} \textsanskrit{bījaṁ} \textsanskrit{nibbaṭṭetvā} \textsanskrit{apanetvā} \textsanskrit{paribhuñjitabbakaṁ} \textsanskrit{ambapanasādi}}, “\textit{\textsanskrit{Nibbattabīja}} means having \textit{nibbatta}-ed, having removed the seed, mangoes, jackfruit, etc., may be eaten.” } }

\section*{8. Discussion of the prohibition against surgery }

When\marginnote{22.1.1} the Buddha had stayed at \textsanskrit{Sāvatthī} for as long as he liked, he set out wandering toward \textsanskrit{Rājagaha}. When he eventually arrived, he stayed in the Bamboo Grove, the squirrel sanctuary. 

At\marginnote{22.1.4} that time the doctor \textsanskrit{Ākāsagotta} performed surgery on a certain monk who had hemorrhoids. Just then, while walking about the dwellings, the Buddha came to this monk’s dwelling. \textsanskrit{Ākāsagotta} saw the Buddha coming and said to him, “Good Gotama, please come and see this monk’s anus. It’s just like the mouth of a lizard.” 

The\marginnote{22.2.3} Buddha thought, “This foolish man is mocking me,” and he turned around right there. Soon afterwards he had the Sangha gathered and questioned the monks: “Is there a sick monk in such-and-such a dwelling?” 

“There\marginnote{22.2.7} is, sir.” 

“What’s\marginnote{22.2.8} his illness?” 

“He\marginnote{22.2.9} has hemorrhoids, and the doctor \textsanskrit{Ākāsagotta} is performing surgery.” 

The\marginnote{22.3.1} Buddha rebuked him, “It’s not suitable, monks, for that foolish man, it’s not proper, it’s not worthy of a monastic, it’s not allowable, it’s not to be done. How can he have surgery on the private parts? The skin is delicate in that area, sores heel with difficulty, and a scalpel is hard to wield there. This will affect people’s confidence …” After rebuking him … he gave a teaching and addressed the monks: 

\scrule{“You shouldn’t have surgery on the private parts.\footnote{Vjb 3.279: \textit{\textsanskrit{Sambādheti} vaccamagge bhikkhussa \textsanskrit{bhikkhuniyā} ca \textsanskrit{passāvamaggepi} anulomato}, “\textit{\textsanskrit{Sambādha}} means the anus of a monk or a nun, and also the genital area accords with this.” } If you do, you commit a serious offense.” }

When\marginnote{22.4.1} they heard that the Buddha had prohibited surgery, the monks from the group of six had enemas.\footnote{Sp 3.279: \textit{Yena kenaci pana cammena \textsanskrit{vā} vatthena \textsanskrit{vā} \textsanskrit{vatthipīḷanampi} na \textsanskrit{kātabbaṁ}}, “One should not do bladder-action, \textit{\textsanskrit{vatthipīḷana}}, with whatever skin or cloth.” Vmv 3.279: \textit{\textsanskrit{Vatthipīḷananti} \textsanskrit{yathā} \textsanskrit{vatthigatatelādi} \textsanskrit{antosarīre} \textsanskrit{ārohanti}, \textsanskrit{evaṁ} hatthena \textsanskrit{vatthimaddanaṁ}}, “\textit{\textsanskrit{Vatthipīḷana}}: in order for oils, etc., in a bladder to go up inside the body, thus one squeezes the bladder with the hand.” The meaning is not entirely clear. My rendering is no more than a suggestion. } The monks of few desires complained and criticized them, “How can the monks from the group of six have enemas?” They told the Buddha what had happened. “Is it true, monks, that the monks from the group of six are having enemas?” “It’s true, sir.” … After rebuking them, the Buddha gave a teaching and addressed the monks: 

\scrule{“You shouldn’t get surgery within 3.5 centimeters of the private parts or have enemas.\footnote{That is, two fingerbreadths. For a discussion of the \textit{\textsanskrit{aṅgula}}, see \textit{sugata} in Appendix of Technical Terms. } If you do, you commit a serious offense.” }

\section*{9. Discussion of the prohibition against human flesh }

When\marginnote{23.1.1} the Buddha had stayed at \textsanskrit{Rājagaha} for as long as he liked, he set out wandering toward Benares. When he eventually arrived, he stayed in the deer park at Isipatana. 

At\marginnote{23.1.4} that time in Benares there were two lay-followers, Suppiya and \textsanskrit{Suppiyā}, husband and wife, both with confidence in Buddhism. They were donors and benefactors, and they attended on the Sangha. 

On\marginnote{23.1.5} one occasion \textsanskrit{Suppiyā} went to the monastery. She walked from dwelling to dwelling, from yard to yard, asking the monks, “Is anyone sick? What may I bring?” Just then a certain monk had drunk a purgative. He told \textsanskrit{Suppiyā} about this, adding, “I need meat broth.” “No problem, I’ll organize it.” 

She\marginnote{23.2.6} then returned to her house and told a servant, “Go and get some meat.”\footnote{\textit{\textsanskrit{Pavattamaṁsa}} refers to meat ready for sale, that is, not specially slaughtered. Sp 3.280: \textit{\textsanskrit{Pavattamaṁsanti} matassa \textsanskrit{maṁsaṁ}}, “\textit{\textsanskrit{Pavattamaṁsa}}: meat from a dead (animal).” } Saying, “Yes, ma’am,” he walked around the whole of Benares, but could not find any. So he returned to \textsanskrit{Suppiyā} and said, “There’s no meat, ma’am. There’s no slaughter today.” 

\textsanskrit{Suppiyā}\marginnote{23.3.1} thought, “If that monk doesn’t get meat broth, his illness will get worse or he’ll die. Because I’ve already agreed to provide it, it would not be right if I didn’t.” She then took a knife, cut flesh from her own thigh, and gave it to a slave, saying, “Prepare this meat and give it to the sick monk in such-and-such a dwelling. If anyone asks for me, tell them I’m sick.” She then wrapped her thigh in her upper robe, entered her bedroom, and lay down on the bed. 

When\marginnote{23.4.1} Suppiya returned home, he asked the slave where his wife was. The slave told him. 

He\marginnote{23.4.4} then went to see her, and she told him what had happened. He thought, “It’s astonishing and amazing how much faith and confidence \textsanskrit{Suppiyā} has, as she gives up even her own flesh. Is there anything she would not give?” 

Delighted\marginnote{23.4.13} and joyful he went to the Buddha. He bowed, sat down, and said, “Sir, please accept tomorrow’s meal from me together with the Sangha of monks.” The Buddha consented by remaining silent. Knowing that the Buddha had consented, Suppiya got up from his seat, bowed down, circumambulated the Buddha with his right side toward him, and left. 

The\marginnote{23.5.5} following morning Suppiya had various kinds of fine foods prepared. He then had the Buddha informed that the meal was ready. 

The\marginnote{23.5.6} Buddha robed up, took his bowl and robe, and went to Suppiya’s house where he sat down on the prepared seat together with the Sangha of monks. Suppiya approached the Buddha and bowed down to him. When the Buddha asked him where \textsanskrit{Suppiyā} was, he replied that she was sick. 

“Well\marginnote{23.6.4} then, please tell her to come.” 

“She’s\marginnote{23.6.5} not able, sir.” 

“Well\marginnote{23.6.6} then, carry her in here.” And he did. The moment \textsanskrit{Suppiyā} saw the Buddha that great wound healed and was perfectly covered with skin and hairs. Suppiya and \textsanskrit{Suppiyā} exclaimed, “The great power and might of the Buddha is truly astonishing and amazing!” Delighted and joyful, they personally served various kinds of fine foods to the Sangha of monks headed by the Buddha. When the Buddha had finished his meal, they sat down to one side. The Buddha instructed, inspired, and gladdened them with a teaching, after which he got up from his seat and left. 

Soon\marginnote{23.8.1} afterwards the Buddha had the Sangha gathered and questioned the monks: “Who asked \textsanskrit{Suppiyā} for meat?” The responsible monk told the Buddha. 

“Did\marginnote{23.8.4} you get the meat?” 

“I\marginnote{23.8.5} did, sir.” 

“Did\marginnote{23.8.6} you eat it?” 

“Yes.”\marginnote{23.8.7} 

“Were\marginnote{23.8.8} you circumspect about it?” 

“No,\marginnote{23.8.9} sir.” 

The\marginnote{23.9.1} Buddha rebuked him … “Foolish man, how can you eat meat without circumspection? You have eaten human flesh. This will affect people’s confidence …” After rebuking him, he gave a teaching and addressed the monks: 

\scrule{“There are people who have faith and confidence, even to the point of giving up their own flesh. You shouldn’t eat human flesh. If you do, you commit a serious offense. }

\scrule{You shouldn’t eat flesh without being circumspect. If you do, you commit an offense of wrong conduct.” }

\section*{10. Discussion of the prohibition against elephant meat, etc. }

At\marginnote{23.10.1} one time the king’s elephants had died. Because there was a shortage of food, people ate the elephant meat. They also gave elephant meat to monks who were walking for alms. When the monks ate it, people complained and criticized them, “How can the Sakyan monastics eat elephant meat? Elephants are an attribute of kingship. If the king knew, he would not be pleased with those monks.” They told the Buddha. 

\scrule{“You shouldn’t eat elephant meat. If you do, you commit an offense of wrong conduct.” }

At\marginnote{23.11.1} one time the king’s horses had died. Because there was a shortage of food, people ate the horse meat. They also gave horse meat to monks who were walking for alms. When the monks ate it, people complained and criticized them, “How can the Sakyan monastics eat horse meat? Horses are an attribute of kingship. If the king knew, he would not be pleased with those monks.” They told the Buddha. 

\scrule{“You shouldn’t eat horse meat. If you do, you commit an offense of wrong conduct.” }

At\marginnote{23.12.1} one time when there was a shortage of food, people ate dog meat. They also gave dog meat to monks who were walking for alms. When the monks ate it, people complained and criticized them, “How can the Sakyan monastics eat dog meat? Dogs are disgusting and repulsive.” They told the Buddha. 

\scrule{“You shouldn’t eat dog meat. If you do, you commit an offense of wrong conduct.” }

At\marginnote{23.13.1} one time when there was a shortage of food, people ate snake meat. They also gave snake meat to monks who were walking for alms. When the monks ate it, people complained and criticized them, “How can the Sakyan monastics eat snake meat? Snakes are disgusting and repulsive.” Even Supassa the king of dragons went to see the Buddha. He bowed down to the Buddha\footnote{The \textit{\textsanskrit{nāgas}}, here rendered as “dragons”, were supernormal serpents who protected the snakes. } and said, “Sir, there are dragons without faith and confidence. They might harm the monks even over small matters. Please ask the venerables not to eat snake meat.” The Buddha instructed, inspired, and gladdened him with a teaching, after which Supassa bowed down, circumambulated the Buddha with his right side toward him, and left. Soon afterwards the Buddha gave a teaching and addressed the monks: 

\scrule{“You shouldn’t eat snake meat. If you do, you commit an offense of wrong conduct.” }

On\marginnote{23.14.1} one occasion hunters killed a lion and ate the lion meat. They also gave lion meat to monks who were walking for alms. After eating it, those monks returned to the wilderness. And because of the smell of lion meat, lions attacked them. 

\scrule{“You shouldn’t eat lion meat. If you do, you commit an offense of wrong conduct.” }

On\marginnote{23.15.1} one occasion hunters killed a tiger … a leopard … a bear … a hyena and ate the hyena meat. They also gave hyena meat to monks who were walking for alms. After eating it, those monks returned to the wilderness. And because of the smell of hyena meat, hyenas attacked them. 

\scrule{“You shouldn’t eat tiger meat, leopard meat, bear meat, or hyena meat. If you do, you commit an offense of wrong conduct.” }

\scend{The second section for recitation on \textsanskrit{Suppiyā} is finished. }

\section*{11. The allowance for congee and honey balls }

When\marginnote{24.1.1} the Buddha had stayed at Benares for as long as he liked, he set out wandering toward Andhakavinda together with a large sangha of twelve-hundred and fifty monks. On this occasion the country people had loaded large quantities of salt, oil, rice, and fresh food onto carts, and were following behind the Sangha of monks headed by the Buddha, thinking, “When our turn comes, we’ll prepare a meal.” Five hundred people living on scraps were also following along. 

Eventually\marginnote{24.1.4} the Buddha arrived at Andhakavinda and stayed there. Soon afterwards a certain brahmin whose turn to offer a meal had not yet come, thought, “I’ve been following the Sangha of monks headed by the Buddha for two months waiting to offer them a meal, and I’m still waiting. Moreover, I am all alone, and all my household business is being neglected. Why don’t I inspect the dining hall\footnote{\textit{Bhattagga} is literally “a meal house”. The name suggests that the \textit{bhattagga} was a separate building for eating. They were found both in private houses and in monasteries (\href{https://suttacentral.net/pli-tv-kd10/en/brahmali\#4.5.7}{Kd 10:4.5.7}). Since they were part of houses or a compound of private buildings, “refectory” is not a satisfactory rendering. The fact that kitchens are not mentioned separately may mean that they were part of the \textit{bhattagga}, except in monasteries. This is supported by a passage at (\href{https://suttacentral.net/pli-tv-bu-vb-pj3/en/brahmali\#5.3.1}{Bu Pj 3:5.3.1}) that mentions a cooking implement, a pestle, being stored in a village \textit{bhattagga}. } and prepare whatever is lacking?” When he did, he saw that two things were missing: congee and honey balls. He then went to Venerable Ānanda and told what he had been thinking, adding, “Good Ānanda, if I were to prepare congee and honey balls, would Good Gotama accept it?” 

“Well,\marginnote{24.3.10} brahmin, let me ask the Buddha.” Venerable Ānanda told the Buddha, who said, “Allow it to be prepared, Ānanda.” Ānanda passed the message on to the brahmin. 

The\marginnote{24.4.4} following morning that brahmin prepared much congee and many honey balls and brought it to the Buddha, saying, “Good Gotama, please accept the congee and the honey balls.” 

“Well\marginnote{24.4.6} then, brahmin, give it to the monks.” 

But\marginnote{24.4.7} being afraid of wrongdoing, the monks did not accept. The Buddha said, “Accept, monks, and eat.” That brahmin then personally served much congee and many honey balls to the Sangha of monks headed by the Buddha. When the Buddha had finished his meal, the brahmin sat down to one side. And the Buddha said this to him: 

“Brahmin,\marginnote{24.6.1} there are these ten benefits of congee.\footnote{This is a partial parallel to \href{https://suttacentral.net/an5.207/en/brahmali}{AN 5.207}. } One who gives congee gives life, beauty, happiness, strength, and eloquence; drinking congee stills hunger, allays thirst, gets rid of wind, cleans out the bladder, and helps the digestion of food remnants. 

\begin{verse}%
One\marginnote{24.6.5} who gives congee respectfully at the right time \\
To the restrained ones who live on the gifts of others, \\
Such a one supplies them with ten things: \\
Long life, beauty, happiness, and strength, 

And\marginnote{24.6.9} eloquence, too, one gets from that; \\
Hunger, thirst, and wind are removed, \\
The bladder is cleaned and the food digested. \\
This tonic is praised by the Accomplished One. 

Therefore,\marginnote{24.6.13} for a person looking for happiness—\\
One wishing for heavenly bliss \\
Or desiring human prosperity—\\
It’s appropriate to give congee regularly.” 

%
\end{verse}

The\marginnote{24.7.1} Buddha then got up from his seat and left. Soon afterwards he gave a teaching and addressed the monks: 

\scrule{“I allow congee and honey balls.” }

\section*{12. The government official with recently acquired faith }

When\marginnote{25.1.1} people heard that the Buddha had allowed congee and honey balls, they prepared rice porridge and honey balls early in the morning.\footnote{“Rice porridge” renders \textit{\textsanskrit{bhojjayāgu}}. Sp 3.283: \textit{\textsanskrit{Bhojjayāgunti} \textsanskrit{yā} \textsanskrit{pavāraṇaṁ} janeti}, “\textit{\textsanskrit{Bhojjayāgu}}: what gives rise to satisfaction.” Vin-vn-\textsanskrit{ṭ} 309: \textit{Ettha ca \textsanskrit{bhojjayāgu} \textsanskrit{nāma} \textsanskrit{bahalayāgu}}, “In this case it is rice porridge that is called \textit{\textsanskrit{bhojjayāgu}}.” } After eating rice porridge and honey balls to their satisfaction in the morning, the monks did not eat as much as they had intended in the dining hall. 

At\marginnote{25.1.4} this time a certain government official who had recently acquired faith in Buddhism had invited the Sangha of monks headed by the Buddha for the meal on the following day. He thought, “Why don’t I prepare twelve hundred and fifty bowls of meat for the twelve hundred and fifty monks? I can then give one bowl to each and every monk.” 

The\marginnote{25.2.1} following morning that official had various kinds of fine foods prepared, as well as twelve hundred and fifty bowls of meat. He then had the Buddha informed that the meal was ready. The Buddha robed up, took his bowl and robe, and went to that brahmin’s house where he sat down on the prepared seat together with the Sangha of monks. That official then served the monks in the dining hall. As he did so, the monks said, “Only a little, thanks.” 

“Please\marginnote{25.3.4} don’t say that because I’ve only recently acquired faith in Buddhism. I’ve prepared much food of various kinds, as well as twelve-hundred and fifty bowls of meat. I’ll bring one bowl of meat to each and every one of you. Venerables, please accept as much as you like.” 

“We’re\marginnote{25.3.8} not taking so little because of that, but because we ate rice porridge and honey balls to our satisfaction early in the morning.” 

The\marginnote{25.4.1} official complained and criticized them, “When the venerables have been invited by me, how can they eat someone else’s rice porridge? Am I incapable of giving them as much as they like?” Angry and aiming to criticize, he walked around filling the monks’ almsbowls, saying, “Eat it or take it away.” 

When\marginnote{25.4.4} he had personally served the various kinds of fine foods to the Sangha of monks headed by the Buddha, and the Buddha had finished his meal, the official sat down to one side. The Buddha instructed, inspired, and gladdened him with a teaching, after which he got up from his seat and left. 

Soon\marginnote{25.5.1} after the Buddha had left, that official felt anxiety and remorse, thinking, “It’s bad for me, truly bad, that I acted like this. I wonder, did I make much merit or demerit?” He then went to the Buddha, bowed, sat down, and told him what he had been thinking, adding, “I wonder, sir, did I make much merit or demerit?” 

“When\marginnote{25.6.1} you invited the Sangha of monks headed by the Buddha for a meal on the following day, you made much merit. When each and every monk received rice from you, you made much merit. You are heading for heaven.” 

When\marginnote{25.6.3} the official heard this, he was joyful and elated. He got up from his seat, bowed down, circumambulated the Buddha with his right side toward him, and left. Soon afterwards the Buddha had the Sangha gathered and questioned the monks: “Is it true, monks, that monks who had been invited for a meal ate someone else’s rice porridge beforehand?” 

“It’s\marginnote{25.7.3} true, sir.” 

The\marginnote{25.7.4} Buddha rebuked them … “How can those foolish men eat someone else’s congee beforehand when they have been invited for a meal? This will affect people’s confidence …” After rebuking them, he gave a teaching and addressed the monks: 

\scrule{“When you have been invited to a meal, you shouldn’t eat someone else’s rice porridge beforehand. If you do, you should be dealt with according to the rule.”\footnote{This refers to \href{https://suttacentral.net/pli-tv-bu-vb-pc33/en/brahmali\#3.15.1}{Bu Pc 33:3.15.1}. } }

\section*{13. The account of \textsanskrit{Belaṭṭha} \textsanskrit{Kaccāna} }

When\marginnote{26.1.1} the Buddha had stayed at Andhakavinda for as long as he liked, he set out wandering toward \textsanskrit{Rājagaha} together with a large sangha of twelve-hundred and fifty monks. Just then \textsanskrit{Belaṭṭha} \textsanskrit{Kaccāna} was traveling from \textsanskrit{Rājagaha} to Andhakavinda with five hundred carts, all of them filled with jars of sugar. When the Buddha saw \textsanskrit{Belaṭṭha} \textsanskrit{Kaccāna} coming, he stepped off the road and sat down at the foot of a tree. 

\textsanskrit{Belaṭṭha}\marginnote{26.2.1} \textsanskrit{Kaccāna} went up to the Buddha, bowed, and said, “Sir, I would like to give one jar of sugar to each and every monk.” 

“Well\marginnote{26.2.4} then, \textsanskrit{Kaccāna}, just bring one jar of sugar.” 

Saying,\marginnote{26.2.5} “Yes, sir,” he got a jar of sugar, returned to the Buddha, and said, “Here is the jar. What should I do next?” 

“Now\marginnote{26.2.8} give sugar to the monks.” 

Saying,\marginnote{26.3.1} “Yes, sir,” he did just that. He then said to the Buddha, “I’ve given sugar to the monks, but there’s much left over. What should I do with that?” 

“Give\marginnote{26.3.4} the monks as much sugar as they need.” 

Saying,\marginnote{26.3.5} “Yes, sir,” he did as requested. He then said to the Buddha, “I’ve given the monks as much sugar as they need, but there’s much left over. What should I do with that?” 

“Give\marginnote{26.3.8} the monks as much sugar as they want.” 

Saying,\marginnote{26.3.9} “Yes, sir,” he again did as requested. Some monks filled their almsbowls and even their water filters and bags. When he was finished, he said to the Buddha, “I’ve given the monks as much sugar as they want, but there’s much left over. What should I do with that?” 

“Give\marginnote{26.4.4} to those who live on scraps.” 

Saying,\marginnote{26.4.5} “Yes, sir,” he again did as requested. He then said to the Buddha, “I’ve given them sugar, but there’s much left over. What should I do with that?” 

“Give\marginnote{26.4.8} them as much sugar as they need.” 

Saying,\marginnote{26.5.1} “Yes, sir,” he again did as requested. He then said to the Buddha, “I’ve given them as much sugar as they need, but there’s much left over. What should I do with that?” 

“Give\marginnote{26.5.4} them as much sugar as they want.” 

Saying,\marginnote{26.5.5} “Yes, sir,” he once again did as requested. Some of those who lived on scraps filled basins, waterpots, and baskets, and some even their laps. When he was finished, he said to the Buddha, “I’ve given them as much sugar as they want, but there’s much left over. What should I do with that?” 

“\textsanskrit{Kaccāna},\marginnote{26.6.4} I don’t see anyone in this world with its gods, lords of death, and supreme beings, in this society with its monastics and brahmins, its gods and humans, who would be able to properly digest that sugar except a Buddha or his disciple. So discard that sugar where there are no cultivated plants or in water without life.”\footnote{\textit{Appaharita}, literally, “few green plants”. \href{https://suttacentral.net/pli-tv-bu-vb-pc19/en/brahmali\#2.1.14}{Bu Pc 19:2.1.14}: \textit{\textsanskrit{Haritaṁ} \textsanskrit{nāma} \textsanskrit{pubbaṇṇaṁ} \textsanskrit{aparaṇṇaṁ}}, “\textit{Harita} means: vegetables and grains.” } 

Saying,\marginnote{26.6.6} “Yes, sir,” he dumped that sugar in water without life. As he did so, that sugar hissed, sputtered, fumed, and smoked—just like a plowshare heated the whole day hisses, sputters, fumes, and smokes when dropped in water. 

\textsanskrit{Belaṭṭha}\marginnote{26.7.4} \textsanskrit{Kaccāna} was awestruck, with goose bumps all over. He approached the Buddha, bowed, and sat down. The Buddha then gave him a progressive talk—on generosity, morality, and heaven; on the downside, degradation, and defilement of worldly pleasures; and he revealed the benefits of renunciation. When the Buddha knew that his mind was ready, supple, without hindrances, joyful, and confident, he revealed the teaching unique to the Buddhas: suffering, its origin, its end, and the path. And just as a clean and stainless cloth absorbs dye properly, so too, while he was sitting right there, \textsanskrit{Belaṭṭha} \textsanskrit{Kaccāna} experienced the stainless vision of the Truth: “Anything that has a beginning has an end.” 

He\marginnote{26.9.1} had seen the Truth, had reached, understood, and penetrated it. He had gone beyond doubt and uncertainty, had attained to confidence, and had become independent of others in the Teacher’s instruction. He then said to the Buddha, “Wonderful, sir, wonderful! Just as one might set upright what’s overturned, or reveal what’s hidden, or show the way to one who’s lost, or bring a lamp into the darkness so that one with eyes might see what’s there—just so has the Buddha made the Teaching clear in many ways. I go for refuge to the Buddha, the Teaching, and the Sangha of monks. Please accept me as a lay follower who’s gone for refuge for life.” 

The\marginnote{27.1.1} Buddha then continued wandering toward \textsanskrit{Rājagaha}. When he eventually arrived, he stayed in the Bamboo Grove, the squirrel sanctuary. At that time there was an abundance of sugar in \textsanskrit{Rājagaha}. The monks thought, “The Buddha has only allowed sugar for the sick,” and being afraid of wrongdoing, they did not eat it. 

\scrule{“I allow you to take sugar when you’re sick and sugar mixed in water when you’re not.” }

\section*{14. \textsanskrit{Pāṭaligāma} }

When\marginnote{28.1.1} the Buddha had stayed at \textsanskrit{Rājagaha} for as long as he liked, he set out wandering toward \textsanskrit{Pāṭaligāma} with a large sangha of twelve-hundred and fifty monks.\footnote{Sections 14 to 18 are essentially the same as sections 5–8 and 11 in the \textsanskrit{Mahāparinibbāna} Sutta at \href{https://suttacentral.net/dn16/en/brahmali\#1.19.1}{DN 16:1.19.1}–2.3.10 and \href{https://suttacentral.net/dn16/en/brahmali\#2.14.1}{DN 16:2.14.1}–2.19.9. } When he eventually arrived, he stayed there. 

When\marginnote{28.1.3} the lay followers of \textsanskrit{Pāṭaligāma} heard that he had arrived, they went to see him, bowed, and sat down on one side. The Buddha instructed, inspired, and gladdened them with a teaching. They then said to the Buddha, “Sir, please visit our guesthouse together with the Sangha of monks.” The Buddha consented by remaining silent. Knowing that he had consented, they got up from their seats, bowed down, and circumambulated him with their right sides toward him. They then went to the guesthouse, spread mats on the floor, prepared seats, put out a large waterpot, and hung up an oil lamp, after which they returned to the Buddha, bowed, and told him that everything was prepared, adding, “Sir, please come when you’re ready.” 

The\marginnote{28.3.7} Buddha robed up, took his bowl and robe, and went to the guesthouse together with the Sangha of monks. He washed his feet, entered the guesthouse, and sat down facing east, leaning on the central pillar. The monks washed their feet too, entered the guesthouse, and sat down facing east with the Buddha in front of them, leaning against the western wall. The lay followers of \textsanskrit{Pāṭaligāma} followed suit and sat down facing west with the Buddha in front of them, leaning against the eastern wall. The Buddha then addressed those lay followers: 

“There\marginnote{28.4.1} are these five dangers for one who is immoral because of failure in morality. Because of heedlessness, they lose much wealth. They get a bad reputation. Whenever they come to a gathering of people—whether a gathering of aristocrats, brahmins, householders, or monastics—they are shy and timid. They die confused. After death, they are reborn in a lower realm. 

There\marginnote{28.5.1} are these five benefits for one who is moral because of success in morality. Because of heedfulness, they gain much wealth. They get a good reputation. Whenever they come to a gathering of people—whether a gathering of aristocrats, brahmins, householders, or monastics—they are confident and self-assured. They die with a clear mind. After death, they are reborn in heaven.” 

The\marginnote{28.6.1} Buddha instructed, inspired, and gladdened them by teaching for much of the night. He then dismissed them, saying, “It’s late. Please go when you’re ready.” 

Saying,\marginnote{28.6.4} “Yes, sir,” they got up from their seats, bowed down, circumambulated him with their right sides toward him, and left. Soon after the lay followers of \textsanskrit{Pāṭaligāma} had left, the Buddha entered an empty cubicle.\footnote{\textit{\textsanskrit{Suññāgāra}} normally means solitude or an empty dwelling, but here the context suggests another meaning is intended. DN-a 1.151, commenting on the parallel at \href{https://suttacentral.net/dn16/en/brahmali\#1.25.4}{DN 16:1.25.4}, says: \textit{\textsanskrit{Suññāgāranti} \textsanskrit{pāṭiyekkaṁ} \textsanskrit{suññāgāraṁ} \textsanskrit{nāma} natthi, tattheva pana ekapasse \textsanskrit{sāṇipākārena} \textsanskrit{parikkhipitvā} – “idha \textsanskrit{satthā} \textsanskrit{vissamissatī}”ti \textsanskrit{mañcakaṁ} \textsanskrit{paññapesuṁ}}, “\textit{\textsanskrit{Suññāgāra}}: it is not a distinct (dwelling) that is called a \textit{\textsanskrit{suññāgāra}}. But they prepared a bed to one side right there, having surrounded it with a curtain, thinking, ‘The Teacher will rest here.’” } 

\section*{15. Sunidha and \textsanskrit{Vassakāra} }

At\marginnote{28.7.2.1} that time Sunidha and \textsanskrit{Vassakāra}, the government officials of Magadha, were building a fortress at \textsanskrit{Pāṭaligāma} to defend against the Vajjians. The Buddha got up early in the morning and, with his superhuman and purified clairvoyance, he saw a number of gods taking possession of sites around \textsanskrit{Pāṭaligāma}. And wherever powerful gods took possession of a site was where powerful kings and government officials tended to build their houses. Wherever gods of middle standing took possession of a site was where the kings and government officials of middle standing tended to build their houses. Wherever the lower ranked gods took possession of a site was where the lower ranked kings and government officials tended to build their houses. 

The\marginnote{28.7.7} Buddha said to Venerable Ānanda, “Who’s building a fortress in \textsanskrit{Pāṭaligāma}?” 

“Sunidha\marginnote{28.8.1} and \textsanskrit{Vassakāra}, sir.” 

“They\marginnote{28.8.2} are building the fortress, Ānanda, as if they had consulted with the \textsanskrit{Tāvatiṁsa} gods.” The Buddha told Ānanda what he had seen, adding, “As far, Ānanda, as the extent of the Indian realm, as far as the routes of commerce, \textsanskrit{Pāṭaliputta} will be the chief city, the destination for merchandise.\footnote{For \textit{\textsanskrit{puṭabhedana}}, see Oskar von Hinüber, “Hoary Past and Hazy Memory”, p. 203. } And there will be three dangers for \textsanskrit{Pāṭaliputta}: fire, water, and internal dissent.” 

Sunidha\marginnote{28.9.1} and \textsanskrit{Vassakāra} then went to the Buddha and exchanged pleasantries with him, adding, “Please accept tomorrow’s meal from us together with the Sangha of monks.” The Buddha consented by remaining silent. Knowing that he had consented, they left. 

Having\marginnote{28.10.5} had various kinds of fine foods prepared, they had the Buddha informed that the meal was ready. The Buddha robed up, took his bowl and robe, and went to Sunidha and \textsanskrit{Vassakāra}’s meal offering where he sat down on the prepared seat together with the Sangha of monks. Sunidha and \textsanskrit{Vassakāra} then personally served various kinds of fine foods to the Sangha of monks headed by the Buddha. When the Buddha had finished his meal, they sat down to one side. And the Buddha expressed his appreciation with these verses: 

\begin{verse}%
“In\marginnote{28.11.1} whatever place \\
The wise decide to live, \\
There they feed the virtuous, \\
The restrained monastics. 

One\marginnote{28.11.5} should dedicate the offering \\
To whatever gods are there. \\
Being revered and honored, \\
They return the favor to you. 

And\marginnote{28.11.9} they have compassion for you, \\
As a mother for her own child. \\
The person the gods have compassion for \\
Always has good fortune.” 

%
\end{verse}

The\marginnote{28.11.13} Buddha then got up from his seat and left. 

But\marginnote{28.12.1} Sunidha and \textsanskrit{Vassakāra} followed behind him, thinking, “Whatever gate the ascetic Gotama leaves from, we’ll name the Gotama Gate. Whatever ford he uses to cross the river Ganges, we’ll name the Gotama Ford.” 

And\marginnote{28.12.4} so the gate through which he left was named the Gotama Gate. The Buddha then went to the river Ganges. At that time the river was full to the brim. Among the people who wanted to cross, some were looking for a boat, some for a barge, and some were putting together a raft. 

The\marginnote{28.13.1} Buddha saw this. Then, just as a strong man might bend or stretch his arm, the Buddha disappeared from the near shore of the river and reappeared on the far shore together with the Sangha of monks. 

Seeing\marginnote{28.13.2} the significance of this, the Buddha uttered a heartfelt exclamation: 

\begin{verse}%
“Whoever\marginnote{28.13.3} crosses the flowing mass of water, \\
They build a bridge, leaving the water behind.\footnote{I understand \textit{\textsanskrit{pallalāni}} as a poetic term for any body of water. } \\
While ordinary people put together a raft, \\
The wise have crossed already.” 

%
\end{verse}

\section*{16. Discussion of the truths at \textsanskrit{Koṭigāma} }

The\marginnote{29.1.1} Buddha then went to \textsanskrit{Koṭigāma} and stayed there. And he addressed the monks: 

“It’s\marginnote{29.1.4} because of not awakening to or penetrating these four noble truths that you and I have wandered on and transmigrated for such a long time: the noble truth of suffering, the noble truth of the origin of suffering, the noble truth of the end of suffering, the noble truth of the path leading to the end of suffering. But now, monks, the noble truth of suffering has been awakened to and penetrated, likewise the noble truth of the origin of suffering, the noble truth of the end of suffering, and the noble truth of the path leading to the end of suffering. Craving for existence has been cut off; the passage to existence has been destroyed; now there is no further existence. 

\begin{verse}%
Because\marginnote{29.2.2} of not properly seeing \\
The four noble truths, \\
You have transmigrated for a long time \\
Among the various kinds of rebirth. 

But\marginnote{29.2.6} now they have been seen, \\
The passage to existence has been destroyed, \\
The root of suffering has been cut off, \\
And there is no further existence.” 

%
\end{verse}

\section*{17–18. The account of \textsanskrit{Ambapālī} and the \textsanskrit{Licchavīs} }

The\marginnote{30.1.1} courtesan \textsanskrit{Ambapālī} heard that the Buddha had arrived at \textsanskrit{Koṭigāma}. She had her best carriages harnessed, mounted one of them, and left \textsanskrit{Vesālī} to visit the Buddha. She went by carriage as far as the ground would allow, dismounted, and then approached the Buddha on foot. After bowing down to the Buddha, she sat down, and the Buddha instructed, inspired, and gladdened her with a teaching. She then said to the Buddha, “Sir, please accept tomorrow’s meal from me together with the Sangha of monks.” The Buddha consented by remaining silent. Knowing that he had consented, she got up from her seat, bowed down, circumambulated him with her right side toward him, and left. 

The\marginnote{30.3.1} \textsanskrit{Licchavīs} of \textsanskrit{Vesālī}, too, heard that the Buddha had arrived at \textsanskrit{Koṭigāma}. They had their best carriages harnessed, mounted one of them, and left \textsanskrit{Vesālī} to visit the Buddha. Some of them wore blue, with blue makeup, blue clothes, and blue ornaments, and likewise, some of them wore yellow, some red, and some white. 

When\marginnote{30.3.5} \textsanskrit{Ambapālī} met the young \textsanskrit{Licchavīs}, she turned her carriage around and drove up next to them, pole to pole, yoke to yoke, wheel to wheel, axle to axle. The \textsanskrit{Licchavīs} said, “What on earth are you doing?” and she replied, “I’m doing this, sirs, because I’ve invited the Buddha and the Sangha of monks for tomorrow’s meal!” 

“We’ll\marginnote{30.4.4} give you a hundred thousand for this meal, \textsanskrit{Ambapālī}.” 

“Even\marginnote{30.4.5} if you gave me the whole of \textsanskrit{Vesālī} and the adjoining countryside, I would not give you this meal.”\footnote{\textit{\textsanskrit{Vesāliṁ} \textsanskrit{sāhāraṁ}}, literally, “\textsanskrit{Vesālī} with its support”. Sp 3.289: \textit{\textsanskrit{Sāhāraṁ} \textsanskrit{dajjeyyāthāti} \textsanskrit{sajanapadaṁ} \textsanskrit{dadeyyātha}}, “\textit{\textsanskrit{Sāhāraṁ} \textsanskrit{dajjeyyātha}} means you should give me (\textsanskrit{Vesālī}) together with the country.” } 

The\marginnote{30.4.6} \textsanskrit{Licchavīs} snapped their fingers in dismay, saying, “Dammit, we’ve been beaten by the mango woman!” And they continued on their way to the Buddha. 

When\marginnote{30.5.2} the Buddha saw them coming, he said to the monks, “Those of you who haven’t seen the \textsanskrit{Tāvatiṁsa} gods, look at the \textsanskrit{Licchavīs}. The \textsanskrit{Licchavīs} are similar to the \textsanskrit{Tāvatiṁsa} gods.” 

The\marginnote{30.5.6} \textsanskrit{Licchavīs} went by carriage as far as the ground would allow, dismounted, and then approached the Buddha on foot. After bowing down to the Buddha, they sat down, and the Buddha instructed, inspired, and gladdened them with a teaching. They then said to the Buddha, “Sir, please accept tomorrow’s meal from us together with the Sangha of monks.” 

“I\marginnote{30.5.11} have already accepted tomorrow’s meal from \textsanskrit{Ambapālī}.” 

The\marginnote{30.5.12} \textsanskrit{Licchavīs} snapped their fingers in dismay, saying, “Dammit, we’ve been beaten by the mango woman.” After rejoicing in the Buddha’s words, they got up from their seats, bowed down, circumambulated him with their right sides toward him, and left. 

When\marginnote{30.6.1} the Buddha had stayed at \textsanskrit{Koṭigāma} for as long as he liked, he went to \textsanskrit{Nātikā}, where he stayed in the brick guesthouse. 

The\marginnote{30.6.3} following morning \textsanskrit{Ambapālī} had various kinds of fine foods prepared in her own park. She then had the Buddha informed that the meal was ready. The Buddha robed up, took his bowl and robe, and went to \textsanskrit{Ambapālī}’s meal offering where he sat down on the prepared seat together with the Sangha of monks. \textsanskrit{Ambapālī} personally served various kinds of fine foods to the Sangha of monks headed by the Buddha. When the Buddha had finished his meal, she sat down to one side and said, “Sir, I give this mango grove to the Sangha of monks headed by the Buddha.” The Buddha accepted the park. After instructing, inspiring, and gladdening her with a teaching, he got up from his seat and went to the Great Wood near \textsanskrit{Vesālī}, where he stayed in the hall with the peaked roof. 

\scend{The third section for recitation on the \textsanskrit{Licchavīs} is finished. }

\section*{19. The account of General \textsanskrit{Sīha} }

On\marginnote{31.1.1} one occasion a number of well-known \textsanskrit{Licchavīs} were seated together in the public hall, praising the Buddha, the Teaching, and the Sangha in many ways.\footnote{The \textit{\textsanskrit{santhāgāra}} seems to have been a multi-purpose building. In the present context it is used as a meeting place, the exact nature of the meeting not being spelled out. Other contexts show that the \textit{\textsanskrit{santhāgāra}} was used for a number of purposes. At \href{https://suttacentral.net/mn51/en/brahmali\#10.3}{MN 51:10.3} it is used as a place to perform a ritual; from \href{https://suttacentral.net/mn53/en/brahmali\#2.1}{MN 53:2.1} and similar contexts we can deduce from the word \textit{\textsanskrit{anajjhāvuṭṭha}}, “not (previously) lived in”, that it was used as a place of lodging; at \href{https://suttacentral.net/dn3/en/brahmali\#1.13.4}{DN 3:1.13.4} the Sakyans were enjoying themselves in their \textit{\textsanskrit{santhāgāra}}; according to \href{https://suttacentral.net/dn16/en/brahmali\#5.20.1}{DN 16:5.20.1} and other \textit{suttas}, the official meetings were held there; and at \href{https://suttacentral.net/sn56.45/en/brahmali\#1.3}{SN 56.45:1.3} the \textsanskrit{Licchavīs} were practicing archery in the \textit{\textsanskrit{santhāgāra}}. By contrast the \textit{\textsanskrit{sabhā}} seems to have been used exclusively for official meetings. As a consequence I translate \textit{\textsanskrit{santhāgāra}} as “public hall” and \textit{\textsanskrit{sabhā}} as “public meeting hall”. Most of this section is parallel to \href{https://suttacentral.net/an8.12/en/brahmali}{AN 8.12}. } \textsanskrit{Sīha} the general, a disciple of the Jains, was seated in that gathering. He thought, “No doubt that Buddha is perfected, a fully Awakened One, since these well-known \textsanskrit{Licchavīs} praise the Buddha, the Teaching, and the Sangha in this way. Why don’t I go and visit that Buddha?” He then went to the Jain ascetic from \textsanskrit{Ñātika} and said,\footnote{Bhikkhu \textsanskrit{Sujāto} has this to say about the name \textsanskrit{Nigaṇṭha} \textsanskrit{Nāṭaputta}: “(The \textsanskrit{Ñātika} clan) were perhaps the second-most important of the clans that made up the Vajjian League (after the \textsanskrit{Licchavīs}), yet there is little information about them, and they seem almost absent from the Pali texts. One of the rather noteworthy aspects of the clan is how variable the spelling of their name is. We find \textsanskrit{Jṇātṛika} or \textsanskrit{Jṇātaka} in Sanskrit; \textsanskrit{Ñātaka} in Pali, \textsanskrit{Nāyika} in Jain Prakrit, and well as \textsanskrit{Nāṭaka}, and so on. The variety of forms and dialectical variations is forbidding, but it appears that the sense of the word is simply “the clan”, i.e. it is \textit{\textsanskrit{ñāti}} as in “family”. By far the most famous member of the clan was \textsanskrit{Mahāvīra}, the leader of the Jains. In Pali, he is known as \textsanskrit{Nigaṇṭha} \textsanskrit{Nāṭaputta}. The latter name is explained by the commentary as “son of a dancer”; it is also sometimes spelled \textsanskrit{Nāthaputta} (son of a lord). However given the universal Jain tradition that he was a \textsanskrit{Jṇātṛika}, it seems certain that this is a misunderstanding, and that \textsanskrit{Nāṭaputta} in fact means “a son of the \textsanskrit{Jṇātṛi} clan”, i.e. a \textsanskrit{Jṇātṛika}. It is the same pattern as \textsanskrit{Sākyaputta}, which means “Sakyan”. Given this, perhaps we should reconsider how we present his name. \textsanskrit{Nigaṇṭha} means “knotless”, but it is just a term for a Jain ascetic (as \textit{bhikkhu} is for Buddhists). Perhaps we should translate his name as “the Jain monk of the \textsanskrit{Ñātika} clan.” See full discussion at https://discourse.suttacentral.net/t/the-lost-vajjian-clan-of-the-natikas. } “Sir, I wish to visit the ascetic Gotama.” 

“But\marginnote{31.2.3} \textsanskrit{Sīha}, why visit the ascetic Gotama who believes that actions don’t have results when you believe that they do? For the ascetic Gotama believes in inaction, teaches that, and trains his disciples in that.” \textsanskrit{Sīha}’s intention to go died down. 

The\marginnote{31.3.1} same sequence of events happened a second time. 

A\marginnote{31.3.10} third time a number of well-known \textsanskrit{Licchavīs} were seated together in the public hall, praising the Buddha, the Teaching, and the Sangha in many ways. \textsanskrit{Sīha} heard this, and he had the same thoughts as before. And it occurred to him, “What can the Jain ascetics do to me, whether I get their permission or not? Let me go and visit the Buddha, the Perfected and fully Awakened One, without getting permission from the Jains.” 

Soon\marginnote{31.4.1} afterwards, in the middle of the day, General \textsanskrit{Sīha} set out from \textsanskrit{Vesālī} with five hundred carriages to visit the Buddha. He went by carriage as far as the ground would allow, dismounted, and then approached the Buddha on foot. He bowed, sat down, and said, “Sir, I have heard that the ascetic Gotama believes in inaction, that he teaches inaction, and that he trains his disciples in that. Those who say this, do they say what you have said without falsely misrepresenting you? Do they explain according to the Teaching so that they can’t be legitimately criticized? I don’t wish to misrepresent you.” 

“There’s\marginnote{31.5.1} a way, \textsanskrit{Sīha}, one could rightly say of me that I believe in inaction, that I teach inaction, and that I train my disciples in that. What’s that way? 

I\marginnote{31.6.1} teach the non-doing of misconduct by body, speech, and mind. I teach the non-doing of the various kinds of bad and unwholesome actions. 

There’s\marginnote{31.6.7} also a way one could rightly say of me that I believe in action, that I teach action, and that I train my disciples in that. What’s that way? I teach the doing of good conduct by body, speech, and mind. I teach the doing of the various kinds of good and wholesome actions. 

There’s\marginnote{31.7.1} a way one could rightly say of me that I’m an annihilationist, that I teach for the sake of annihilation, and that I train my disciples in that. What’s that way? I teach the annihilation of sensual desire, ill will, and confusion. I teach the annihilation of the various kinds of bad and unwholesome qualities. 

There’s\marginnote{31.7.7} a way one could rightly say of me that I’m disgusting, that I teach for the sake of disgust, and that I train my disciples in that. What’s that way?\footnote{The literal meaning is “Good Gotama is disgusted,” but I am taking literary license to make it more meaningful and punchy. Sp 1.7: \textit{Puna \textsanskrit{brāhmaṇo} “jigucchati \textsanskrit{maññe} \textsanskrit{samaṇo} gotamo \textsanskrit{idaṁ} \textsanskrit{vayovuḍḍhānaṁ} \textsanskrit{abhivādanādikulasamudācārakammaṁ}, tena \textsanskrit{taṁ} na \textsanskrit{karotī}”ti \textsanskrit{maññamāno} \textsanskrit{bhagavantaṁ} \textsanskrit{jegucchīti} \textsanskrit{āha}}, “Again, the brahmin says ‘The Buddha is disgusted’ because he thinks, ‘It seems the ascetic Gotama is disgusted with doing the wholesome actions of bowing down, etc., to elders.’” The brahmin clearly didn’t approve of such conduct, perhaps even finding it disgusting. } I am disgusted by misconduct by body, speech, and mind. I am disgusted by the various kinds of bad and unwholesome qualities. 

There’s\marginnote{31.8.1} a way one could rightly say of me that I’m an exterminator, that I teach for the sake of extermination, and that I train my disciples in that. What’s that way? I teach the extermination of sensual desire, ill will, and confusion, the extermination of the various kinds of bad and unwholesome qualities. 

There’s\marginnote{31.8.7} a way one could rightly say of me that I’m austere, that I teach for the sake of austerity, and that I train my disciples in that. What’s that way? I say that bad, unwholesome qualities—misconduct by body, speech, and mind—are to be disciplined. One who has abandoned them, cut them off at the root, made them like a palm stump, eradicated them, and made them incapable of reappearing in the future—such a one I call austere. Indeed the Buddha has abandoned the bad, unwholesome qualities that are to be disciplined, has cut them off at the root, made them like a palm stump, eradicated them, and made them incapable of reappearing in the future. 

There’s\marginnote{31.9.1} a way one could rightly say of me that I’m retiring, that I teach for the sake of retiring, and that I train my disciples in that. What’s that way?\footnote{“Retiring” renders \textit{apagabbha}, explained in the commentaries, at Sp 1.10, as: \textit{Gabbhato apagatoti apagabbho}, “\textit{Apagabbha} means departed from the womb.” However, there is an alternative, and perhaps more convincing, derivation of this word. According to SED, in Vedic Sanskrit we find the word \textit{apagalbha} in the meaning “wanting in boldness” or “timid”. It seems possible, then, that we here have a play on words, where the brahmin refers to “timid” whereas the Buddha responds according to the meaning “departed from the womb” or “retired from rebirth”. I have used the word “retiring” in an attempt at catching this pun. } One who has retired from any future conception in a womb, any rebirth in a future life, who has cut it off at the root, made it like a palm stump, eradicated it, and made it incapable of reappearing in the future—such a one I call retiring. Indeed the Buddha’s future conception in a womb, his rebirth in a future life, is abandoned and cut off at the root, made like a palm stump, eradicated, and incapable of reappearing in the future. 

There’s\marginnote{31.9.7} a way one could rightly say of me that I’m at ease, that I teach for the sake of ease, and that I train my disciples in that. What’s that way? I’m at ease in the highest sense, I proclaim my Teaching for the sake of ease, and I train my disciples in that.” 

When\marginnote{31.10.1} the Buddha had finished, \textsanskrit{Sīha} exclaimed, “Wonderful, sir, wonderful! Just as one might set upright what’s overturned, or reveal what’s hidden, or show the way to one who’s lost, or bring a lamp into the darkness so that one with eyes might see what’s there—just so has the Buddha made the Teaching clear in many ways. I go for refuge to the Buddha, the Teaching, and the Sangha of monks. Please accept me as a lay follower who’s gone for refuge for life.” 

“Consider\marginnote{31.10.4} it carefully, \textsanskrit{Sīha}. It’s good for well-known people such as yourself to reflect carefully.” 

“Now\marginnote{31.10.5} I’m even more pleased with you, sir. Had I become a lay follower of another religion, they would’ve carried a banner all over \textsanskrit{Vesālī} to proclaim it. But you tell me to consider it carefully. For the second time, I go for refuge to the Buddha, the Teaching, and the Sangha of monks. Please accept me as a lay follower who’s gone for refuge for life.” 

“For\marginnote{31.11.1} a long time, \textsanskrit{Sīha}, your family has been a wellspring of support for the Jain ascetics. When they come to you, you should still consider giving them almsfood.” 

“Now\marginnote{31.11.2} I’m even more pleased with you, sir. I had heard that you say that offerings should only be given to you and your disciples, not to anyone else, and only offerings given to you and your disciples are fruitful, not what’s given to others. But in reality you encourage me to give to the Jain ascetics. Indeed, I shall know the right time for that. For the third time, I go for refuge to the Buddha, the Teaching, and the Sangha of monks. Please accept me as a lay follower who’s gone for refuge for life.” 

The\marginnote{31.12.1} Buddha then gave \textsanskrit{Sīha} a progressive talk—on generosity, morality, and heaven; on the downside, degradation, and defilement of worldly pleasures; and he revealed the benefits of renunciation. When the Buddha knew that his mind was ready, supple, without hindrances, joyful, and confident, he revealed the teaching unique to the Buddhas: suffering, its origin, its end, and the path. And just as a clean and stainless cloth absorbs dye properly, so too, while he was sitting right there, \textsanskrit{Sīha} experienced the stainless vision of the Truth: “Anything that has a beginning has an end.” He had seen the Truth, had reached, understood, and penetrated it. He had gone beyond doubt and uncertainty, had attained to confidence, and had become independent of others in the Teacher’s instruction. 

He\marginnote{31.12.3} then said to the Buddha, “Sir, Please accept tomorrow’s meal from me together with the Sangha of monks.” The Buddha consented by remaining silent. Knowing that the Buddha had consented, \textsanskrit{Sīha} got up from his seat, bowed down, circumambulated the Buddha with his right side toward him, and left. 

\textsanskrit{Sīha}\marginnote{31.12.7} then told a man, “Go and get some meat.” The following morning \textsanskrit{Sīha} had various kinds of fine foods prepared. He then had the Buddha informed that the meal was ready. 

The\marginnote{31.12.10} Buddha robed up, took his bowl and robe, and went to General \textsanskrit{Sīha}’s house where he sat down on the prepared seat together with the Sangha of monks. 

Just\marginnote{31.13.1} then a number of Jain ascetics were walking around \textsanskrit{Vesālī}, from street to street, from intersection to intersection, waiving their arms and calling out, “General \textsanskrit{Sīha} has killed a large animal and made a meal for the ascetic Gotama. The ascetic Gotama is eating that meat, knowing that the animal was killed for his sake!” 

A\marginnote{31.13.3} certain man went up to \textsanskrit{Sīha} and whispered to him what the Jains were doing. \textsanskrit{Sīha} said, “Forget about it. For a long time those venerables have wanted to disparage the Buddha, the Teaching, and the Sangha of monks. They’ll grow old and still keep on misrepresenting the Buddha with lies. Besides, I wouldn’t kill a living being even for the sake of my life.” 

\textsanskrit{Sīha}\marginnote{31.14.1} then personally served various kinds of fine foods to the Sangha of monks headed by the Buddha. When the Buddha had finished his meal, \textsanskrit{Sīha} sat down to one side. The Buddha instructed, inspired, and gladdened him with a teaching. He then got up from his seat and left. 

Soon\marginnote{31.14.3} afterwards the Buddha gave a teaching and addressed the monks: 

\scrule{“You shouldn’t eat meat when you know the animal was killed for your sake. If you do, you commit an offense of wrong conduct. I allow you to eat meat and fish that’s pure in three respects: you haven’t seen, heard, or suspected that the animal was killed for your sake.” }

\section*{20. The allowance for a food-storage area }

Some\marginnote{32.1.1} time later in \textsanskrit{Vesālī}, there was plenty of food, the crops were abundant, and there was no problem getting by on almsfood. Then, while the Buddha was reflecting in private, he thought, “Those things I allowed the monks when there was a shortage of food, the crops were meager, and it was hard to get by on alms—that is, what’s been stored indoors in a monastery, what’s been cooked indoors in a monastery, what’s been cooked by the monks themselves, what’s been received after picking it up, what’s been brought out, what’s been received before the meal, what’s come from the forest or a lotus pond—do the monks still make use of these?” 

When\marginnote{32.1.4} the Buddha had come out from seclusion, he asked Venerable Ānanda about this. He replied, “They do, sir.” 

Soon\marginnote{32.2.1} afterwards the Buddha gave a teaching and addressed the monks: 

“Those\marginnote{32.2.2} things I allowed you when there was a shortage of food, the crops were meager, and it was hard to get by on alms, I prohibit from today onward. 

\scrule{You shouldn’t eat what’s been stored indoors in a monastery, what’s been cooked indoors in a monastery, what’s been cooked by yourselves, or what’s been received after picking it up. If you do, you commit an offense of wrong conduct. }

\scrule{If you have finished your meal and refused an invitation to eat more, you shouldn’t eat non-leftovers that have been brought out, that have been received before the meal, or that have come from the forest or a lotus pond. If you do, you should be dealt with according to the rule.”\footnote{That is, \href{https://suttacentral.net/pli-tv-bu-vb-pc35/en/brahmali\#2.15.1}{Bu Pc 35:2.15.1}. } }

At\marginnote{33.1.1} that time people from the country loaded much salt, oil, rice, and fresh food onto carts, brought them to outside the monastery gatehouse, and waited for their turn to cook a meal. Just then a storm was approaching. Those people went to Venerable Ānanda and told him what was happening, adding, “What should we do now?” Ānanda told the Buddha, who said, “Well then, Ānanda, the Sangha should designate a building at the edge of the monastery as a food-storage area and then store the food there—whether a dwelling, a stilt house, or a cave.\footnote{“Stilt house” combines \textit{\textsanskrit{aḍḍhayoga}}, \textit{\textsanskrit{pāsāda}}, and \textit{hammiya } in one word. All of these, according to the commentaries, are different kinds of \textit{\textsanskrit{pāsāda}}, “stilt houses”. Rather than try to differentiate between these buildings, which is unlikely to be useful from a practical perspective, I have instead grouped them together as “stilt house”. Here is what the commentaries have to say. Sp 4.294: \textit{\textsanskrit{Aḍḍhayogoti} \textsanskrit{supaṇṇavaṅkagehaṁ}}, “An \textit{\textsanskrit{aḍḍhayoga}} is a house bent like a \textit{\textsanskrit{supaṇṇa}}.” Sp-\textsanskrit{ṭ} 4.294 clarifies: \textit{\textsanskrit{Supaṇṇavaṅkagehanti} \textsanskrit{garuḷapakkhasaṇṭhānena} \textsanskrit{katagehaṁ}}, “\textit{\textsanskrit{Supaṇṇavaṅkageha}}: a house made in the shape of the wings of a \textit{\textsanskrit{garuḷa}}.” A \textit{\textsanskrit{garuḷa}}, better known in its Sanskrit form \textit{\textsanskrit{garuḍa}}, is a mythological bird. Sp 4.294 continues: \textit{\textsanskrit{Pāsādoti} \textsanskrit{dīghapāsādo}. Hammiyanti \textsanskrit{upariākāsatale} \textsanskrit{patiṭṭhitakūṭāgāro} \textsanskrit{pāsādoyeva}}, “A \textit{\textsanskrit{pāsāda}} is a long stilt house. A \textit{hammiya} is just a \textit{\textsanskrit{pāsāda}} that has an upper room on top of its flat roof.” At Sp-\textsanskrit{ṭ} 3.74, however, we find slightly different explanations. Still, it seems clear that all three are stilt houses and that they are distinguished according to their shape and the kind of roof they possess. For an explanation of the rendering “stilt house” for \textit{\textsanskrit{pāsāda}}, see Appendix of Technical Terms. } And it should be done like this. A competent and capable monk should inform the Sangha: 

‘Please,\marginnote{33.2.4} venerables, I ask the Sangha to listen. If the Sangha is ready, it should designate such-and-such a dwelling as a food-storage area.  This is the motion. 

Please,\marginnote{33.2.7} venerables, I ask the Sangha to listen. The Sangha designates such-and-such a dwelling as a food-storage area. Any monk who approves of designating such-and-such a dwelling as a food-storage area should remain silent. Any monk who doesn’t approve should speak up. 

The\marginnote{33.2.11} Sangha has designated such-and-such a dwelling as a food-storage area. The Sangha approves and is therefore silent. I’ll remember it thus.’” 

Soon\marginnote{33.3.1} afterwards people used the designated food-storage area for various purposes: to cook congee and rice, to prepare curries, to chop meat, and to split firewood. Getting up early in the morning, the Buddha heard loud noises, like the cawing of crows. He asked Venerable Ānanda what was going on, and Ānanda told him. Soon afterwards the Buddha gave a teaching and addressed the monks: 

\scrule{“You shouldn’t use a designated food-storage area.\footnote{The sequence of events suggests the following: (1) The Buddha initially allows a dwelling (\textit{\textsanskrit{vihāra}}) to be used as a food storage area. (2) In the present rule he then disallows this. (3) Finally, in the next rule, he allows certain other buildings to be used in this way. } If you do, you commit an offense of wrong conduct. I allow three places as food-storage areas: a building made according to a proclamation, a cow stall, and a building given for the purpose by a householder.” }

Soon\marginnote{33.5.1} afterwards Venerable Yasoja was sick. People brought him tonics and the monks stored them outside. Vermin ate them and thieves stole them. 

\scrule{“I allow you to use a designated food-storage area. I allow four places as food-storage areas: a building made according to a proclamation, a cow stall, a building given for the purpose by a householder, and a building designated by the Sangha.”\footnote{The commentary explains these terms as follows. First the “building made according to a proclamation”. Sp 3.295: \textit{\textsanskrit{Paṭhamathambhaṁ} pana \textsanskrit{paṭhamabhittipādaṁ} \textsanskrit{vā} \textsanskrit{patiṭṭhāpentehi} \textsanskrit{bahūhi} \textsanskrit{samparivāretvā} “\textsanskrit{kappiyakuṭiṁ} karoma, \textsanskrit{kappiyakuṭiṁ} \textsanskrit{karomā}”ti \textsanskrit{vācaṁ} \textsanskrit{nicchārentehi} manussesu \textsanskrit{ukkhipitvā} \textsanskrit{patiṭṭhāpentesu} \textsanskrit{āmasitvā} \textsanskrit{vā} \textsanskrit{sayaṁ} \textsanskrit{ukkhipitvā} \textsanskrit{vā} thambhe \textsanskrit{vā} \textsanskrit{bhittipādo} \textsanskrit{vā} \textsanskrit{patiṭṭhāpetabbo}}, “After the many who are installing have surrounded the first pillar or the first base for a wall, after extolling among people by saying, ‘We are making a food-store’, having touched those who are installing or oneself having extolled, the base of the wall is to be established or at a pillar.” The details are not entirely clear, but the main point seems to be that one announces in the presence of others that one is building a food-store. Next the “cow stall”. Sp 3.295: \textit{Ettha \textsanskrit{kappiyakuṭiṁ} \textsanskrit{laddhuṁ} \textsanskrit{vaṭṭati}}, “To obtain a food-store here is allowable.” Sp 3.295: \textit{\textsanskrit{Gahapatīti} \textsanskrit{manussā} \textsanskrit{āvāsaṁ} \textsanskrit{katvā} “\textsanskrit{kappiyakuṭiṁ} dema, \textsanskrit{paribhuñjathā}”ti vadanti}, “‘A building given for the purpose by a householder’: having made a building, the people say, ‘We give a food-store, please use it.’” Sp 3.295: \textit{\textsanskrit{Sammutikā} \textsanskrit{nāma} \textsanskrit{kammavācaṁ} \textsanskrit{sāvetvā} \textsanskrit{katāti}}, “What is constructed/designated after making an official proclamation in the Sangha is called ‘a building designated by the Sangha’.” } }

\scend{The fourth section for recitation on \textsanskrit{Sīha} is finished. }

\section*{21. The account of the householder \textsanskrit{Meṇḍaka} }

At\marginnote{34.1.1} that time in the town of Bhaddiya there was a householder called \textsanskrit{Meṇḍaka} who had supernormal powers. He would wash his hair, sweep out his granary, and sit down outside the door. A shower of grain would then fall out of the sky and fill his granary. His wife, too, had supernormal powers. She would sit down next to a pot of rice and a pot of curry and serve a meal to the slaves, servants, and workers. The food would not be exhausted until she got up. His son, too, had supernormal powers. He would get a bag containing a thousand coins and give the slaves, servants, and workers their wages for six months. That purse would not go empty as long as he held it. His daughter-in-law, too, had supernormal powers. She would sit down next to a four-liter basket and give out rice for six months to the slaves, servants, and workers. The rice would not be exhausted until she got up.\footnote{\textit{\textsanskrit{Catudoṇika} \textsanskrit{piṭaka}}, “A basket with a capacity of four \textit{\textsanskrit{doṇas}}.” According to T. W. Rhys Davids in “On the Ancient Coins and Measures of Ceylon: with a discussion of the Ceylon date of the Buddha's death”, p. 18, one \textit{\textsanskrit{doṇa}} is equivalent to 64 handfuls. It may well be that this amounts to more than one liter, but given the uncertainty one liter seems like a suitably round number. } Even his slave had supernormal powers. While plowing with a single plow, he made seven furrows. 

King\marginnote{34.3.1} Seniya \textsanskrit{Bimbisāra} of Magadha heard that within his kingdom, in the town of Bhaddiya, there was a householder called \textsanskrit{Meṇḍaka} with all these abilities. The king told the official in charge of practical affairs about this, adding, “Go and investigate it. If you see it, it will be as if I see it myself.” 

Saying,\marginnote{34.5.11} “Yes, sir,” he set out for Bhaddiya with the fourfold army. When he eventually arrived, he went up to \textsanskrit{Meṇḍaka} and said, “I’ve been told by the king to investigate your supernormal powers. Please show them to me.” \textsanskrit{Meṇḍaka} then washed his hair, swept out his granary, and sat down outside the door. A shower of grain fell out of the sky and filled his granary. 

“Good.\marginnote{34.6.11} Now show me your wife’s supernormal powers.” \textsanskrit{Meṇḍaka} told his wife, “Please serve a meal to the fourfold army.” She sat down next to a pot of rice and a pot of curry and served a meal to the fourfold army. The food was not exhausted until she got up. 

“Good.\marginnote{34.7.4} Now show me your son’s supernormal powers.” \textsanskrit{Meṇḍaka} told his son, “Please give wages for six months to the fourfold army.” He got a bag containing a thousand coins and gave the fourfold army its wages for six months. That purse did not go empty as long as he held it. 

“Good.\marginnote{34.8.4} Now show me your daughter-in-law’s supernormal powers.” \textsanskrit{Meṇḍaka} told his daughter-in-law, “Please give rice for six months to the fourfold army.” She sat down next to a four-liter basket and gave rice for six months to the fourfold army. The rice was not exhausted until she got up. 

“Good.\marginnote{34.9.4} Now show me your slave’s supernormal powers.” 

“Sir,\marginnote{34.9.6} we have to go to the field to see that.” 

“Forget\marginnote{34.9.7} about it, then. I consider it as seen.” 

That\marginnote{34.9.8} official then returned to \textsanskrit{Rājagaha} with the fourfold army and told the king what had happened. 

When\marginnote{34.10.1} the Buddha had stayed at \textsanskrit{Vesālī} for as long as he liked, he set out wandering toward Bhaddiya with a large sangha of twelve-hundred and fifty monks. When he eventually arrived, he stayed in the \textsanskrit{Jātiyā} Grove. 

\textsanskrit{Meṇḍaka}\marginnote{34.11.1} heard: “Sir, the ascetic Gotama, the Sakyan, who has gone forth from the Sakyan clan, has arrived at Bhaddiya and is staying in the \textsanskrit{Jātiyā} Grove with a large sangha of twelve-hundred and fifty monks. That good Gotama has a fine reputation: 

‘He\marginnote{34.11.4} is a Buddha, perfected and fully awakened, complete in insight and conduct, happy, knower of the world, supreme leader of trainable people, teacher of gods and humans, awakened, a Buddha. With his own insight he has seen this world with its gods, its lords of death, and its supreme beings, this society with its monastics and brahmins, its gods and humans, and he makes it known to others. He has a Teaching that’s good in the beginning, good in the middle, and good in the end. It has a true goal and is well articulated. He sets out a perfectly complete and pure spiritual life.’ It’s good to see such perfected ones.” 

\textsanskrit{Meṇḍaka}\marginnote{34.12.1} then had his best carriages harnessed, mounted one of them, and set out from Bhaddiya to visit the Buddha. A number of monastics from other religions saw \textsanskrit{Meṇḍaka} coming, and they said to him, “Where are you going, householder?” 

“I’m\marginnote{34.12.4} going to visit the Buddha, sir, the ascetic Gotama.” 

“But\marginnote{34.12.5} why visit the ascetic Gotama who believes that actions don’t have results when you believe that they do? For the ascetic Gotama believes in inaction, teaches that, and trains his disciples in that.” 

\textsanskrit{Meṇḍaka}\marginnote{34.13.1} thought, “No doubt he must be a Buddha, a Perfected and fully Awakened One, since these monastics of other religions are jealous.” He then went by carriage as far as the ground would allow, dismounted, and then approached the Buddha on foot. After bowing down to the Buddha, he sat down, and the Buddha gave him a progressive talk—on generosity, morality, and heaven; on the downside, degradation, and defilement of worldly pleasures; and he revealed the benefits of renunciation. When the Buddha knew that his mind was ready, supple, without hindrances, joyful, and confident, he revealed the teaching unique to the Buddhas: suffering, its origin, its end, and the path. And just as a clean and stainless cloth absorbs dye properly, so too, while he was sitting right there, \textsanskrit{Meṇḍaka} experienced the stainless vision of the Truth: “Anything that has a beginning has an end.” He had seen the Truth, had reached, understood, and penetrated it. He had gone beyond doubt and uncertainty, had attained to confidence, and had become independent of others in the Teacher’s instruction. 

He\marginnote{34.13.6} then said to the Buddha, “Wonderful, sir, wonderful! Just as one might set upright what’s overturned, or reveal what’s hidden, or show the way to one who’s lost, or bring a lamp into the darkness so that one with eyes might see what’s there—just so has the Buddha made the Teaching clear in many ways. I go for refuge to the Buddha, the Teaching, and the Sangha of monks. Please accept me as a lay follower who’s gone for refuge for life. And please accept tomorrow’s meal from me together with the Sangha of monks.” The Buddha consented by remaining silent. Knowing that the Buddha had consented, \textsanskrit{Meṇḍaka} got up from his seat, bowed down, circumambulated the Buddha with his right side toward him, and left. 

The\marginnote{34.14.2} following morning \textsanskrit{Meṇḍaka} had various kinds of fine foods prepared and then had the Buddha informed that the meal was ready. 

The\marginnote{34.14.3} Buddha robed up, took his bowl and robe, and went to \textsanskrit{Meṇḍaka}’s house where he sat down on the prepared seat together with the Sangha of monks. Then \textsanskrit{Meṇḍaka}’s wife, son, daughter-in-law, and slave approached the Buddha, bowed, and sat down. The Buddha gave them a progressive talk, just as he had done to \textsanskrit{Meṇḍaka}. They, too, experienced the stainless vision of the Truth, and they expressed their appreciation in the same way and became lay followers. \textsanskrit{Meṇḍaka} then personally served various kinds of fine foods to the Sangha of monks headed by the Buddha. When the Buddha had finished his meal, \textsanskrit{Meṇḍaka} sat down to one side and said, “Sir, as long as you’re staying in Bhaddiya, I would like to offer a regular meal to the Sangha of monks headed by the Buddha.” The Buddha then instructed, inspired, and gladdened him with a teaching, after which he got up from his seat and left. 

\section*{22. The allowance for the five products of a cow, etc. }

When\marginnote{34.17.1} the Buddha had stayed at Bhaddiya for as long as he liked, he set out wandering toward \textsanskrit{Aṅguttarāpa} with a large sangha of twelve-hundred and fifty monks. He had not informed \textsanskrit{Meṇḍaka}. When \textsanskrit{Meṇḍaka} heard about it, he told his slaves and workers, “Load lots of salt, oil, rice, and fresh food onto the carts, and bring along twelve-hundred and fifty cowherds and twelve-hundred and fifty dairy cows. We’ll give the Buddha fresh milk wherever we see him.” 

\textsanskrit{Meṇḍaka}\marginnote{34.18.1} caught up with the Buddha while he was crossing a wilderness area. \textsanskrit{Meṇḍaka} approached the Buddha, bowed down, and said, “Sir, please accept tomorrow’s meal from me together with the Sangha of monks.” The Buddha consented by remaining silent. Knowing that the Buddha had consented, \textsanskrit{Meṇḍaka} bowed down, circumambulated the Buddha with his right side toward him, and left. 

The\marginnote{34.18.7} following morning \textsanskrit{Meṇḍaka} had various kinds of fine foods prepared and then had the Buddha informed that the meal was ready. 

The\marginnote{34.19.1} Buddha robed up, took his bowl and robe, and went to \textsanskrit{Meṇḍaka}’s meal offering where he sat down on the prepared seat together with the Sangha of monks. \textsanskrit{Meṇḍaka} told the twelve-hundred and fifty cowherds, “Listen, bring one cow for each and every monk and give them fresh milk.” \textsanskrit{Meṇḍaka} then personally served various kinds of fine foods to the Sangha of monks headed by the Buddha, and he gave them fresh milk. Being afraid of wrongdoing, the monks did not accept.\footnote{It’s not clear why they refused to accept the milk. } The Buddha said, “Accept, monks, and drink.” When the Buddha had finished his meal, \textsanskrit{Meṇḍaka} sat down to one side, and said, “Sir, there are wilderness roads where there is little water and little food, where it’s not easy to travel without provisions. Please allow provisions.” The Buddha then instructed, inspired, and gladdened \textsanskrit{Meṇḍaka} with a teaching, after which he got up from his seat and left. 

Soon\marginnote{34.21.1} afterwards the Buddha gave a teaching and addressed the monks: 

\scrule{“I allow five products from cows: milk, curd, buttermilk, butter, and ghee. }

\scrule{There are wilderness roads where there’s little water and little food, where it’s not easy to travel without provisions. I allow you to look for provisions: whatever you need of rice, mung beans, black gram, salt, sugar, oil, and ghee. }

\scrule{There are people who have faith and confidence. They may deposit money with an attendant, saying,\footnote{“Money” renders \textit{\textsanskrit{hirañña}}. See Appendix of Technical Terms. } “With this, please get something allowable for the venerable.” I allow you to consent to anything allowable from that fund. But I say that under no circumstances should you look for or consent to gold, silver, or money.”\footnote{“Gold, silver, and money” renders \textit{\textsanskrit{jātarūparajata}}. For a discussion of this compound, see Appendix of Technical Terms. } }

\section*{23. The account of \textsanskrit{Keṇiya} the dreadlocked ascetic }

Wandering\marginnote{35.1.1} on, the Buddha eventually arrived at \textsanskrit{Āpaṇa}. \textsanskrit{Keṇiya} the dreadlocked ascetic heard, “The ascetic Gotama, the Sakyan, who’s gone forth from the Sakyan clan, has arrived at \textsanskrit{Āpaṇa}.” And he heard about the Buddha’s qualities just as \textsanskrit{Meṇḍaka} had. He thought, “What should I take to the ascetic Gotama?” And it occurred to him, “There are those ancient sages of the brahmins, the creators and teachers of the Vedas, that is, \textsanskrit{Aṭṭhaka}, \textsanskrit{Vāmaka}, \textsanskrit{Vāmadeva}, \textsanskrit{Vessāmitta}, Yamataggi, \textsanskrit{Aṅgīrasa}, \textsanskrit{Bhāradvāja}, \textsanskrit{Vāseṭṭha}, Kassapa, and Bhagu. The brahmins at present still sing and proclaim the ancient verses that they sang, proclaimed, and collected. Now those ancient sages abstained from eating at night and at the wrong time, yet they consented to certain drinks. The ascetic Gotama also abstains from eating at night and at the wrong time. It would be appropriate for him to consent to the same drinks.” 

He\marginnote{35.3.2} then had a large quantity of drinks prepared. Lifting them with carrying poles, he went to the Buddha. He exchanged pleasantries with the Buddha and said, “Good Gotama, please accept these drinks.” 

“Please\marginnote{35.3.6} give them to the monks, \textsanskrit{Keṇiya}.” 

He\marginnote{35.3.7} did, but being afraid of wrongdoing, the monks did not accept them. The Buddha said, “Accept, monks, and drink.” \textsanskrit{Keṇiya} then personally served that large quantity of drinks to the Sangha of monks headed by the Buddha. When the Buddha had finished his meal, \textsanskrit{Keṇiya} sat down to one side. The Buddha instructed, inspired, and gladdened him with a teaching, and \textsanskrit{Keṇiya} said, “Good Gotama, please accept tomorrow’s meal from me together with the Sangha of monks.” 

“The\marginnote{35.5.1} Sangha is large, \textsanskrit{Keṇiya}. There are twelve hundred and fifty monks. And you have faith in the brahmins.” 

\textsanskrit{Keṇiya}\marginnote{35.5.2} acknowledged what the Buddha had said, but repeated his invitation a second time. The Buddha replied as before, and \textsanskrit{Keṇiya} repeated his invitation a third time. The Buddha then consented by remaining silent. Knowing that the Buddha had consented, \textsanskrit{Keṇiya} got up from his seat and left. 

Soon\marginnote{35.6.1} afterwards the Buddha gave a teaching and addressed the monks: 

\scrule{“I allow eight kinds of drinks: mango drinks, rose-apple drinks, drinks from bananas with seeds, drinks from seedless bananas, licorice drinks, grape drinks, drinks made from lotus tubers, and falsa fruit drinks.\footnote{Sp 3.300: \textit{\textsanskrit{Cocapānanti} \textsanskrit{aṭṭhikehi} kadaliphalehi \textsanskrit{katapānaṁ}}, “\textit{\textsanskrit{Cocapāna}}: a drink made with plantain fruits that have seeds.” Sp 3.300: \textit{\textsanskrit{Mocapānanti} \textsanskrit{anaṭṭhikehi} kadaliphalehi \textsanskrit{katapānaṁ}}, “\textit{\textsanskrit{Mocapāna}}: a drink made with seedless plantain fruit.” \textit{\textsanskrit{Madhūkapāna}} is literally “a drink from the honey tree (fruit)”, \textit{Bassia latifolia} or \textit{Madhuca longifolia}. Sp 3.300: \textit{\textsanskrit{Madhukapānanti} \textsanskrit{madhukānaṁ} \textsanskrit{jātirasena} \textsanskrit{katapānaṁ}}, “\textit{\textsanskrit{Madhukapāna}}: a drink made with the natural juice from honey tree fruits.” This might also refer to sap from the honey tree. Sp 3.300: \textit{\textsanskrit{Muddikapānanti} \textsanskrit{muddikā} udake \textsanskrit{madditvā} \textsanskrit{ambapānaṁ} viya \textsanskrit{katapānaṁ}}, “\textit{\textsanskrit{Muddikapāna}}: having crushed grapes in water, it is a drink made like a mango drink.” Sp 3.300: \textit{\textsanskrit{Sālūkapānanti} \textsanskrit{rattuppalanīluppalādīnaṁ} \textsanskrit{sālūke} \textsanskrit{madditvā} \textsanskrit{katapānaṁ}}, “\textit{\textsanskrit{Sālūkapāna}}: a drink made by having crushed the tubers of red and blue lotuses.” SED identifies the \textit{\textsanskrit{phārusakapāna}} as the \textit{Grewia Asiatica}, sv. \textit{\textsanskrit{parūsha}}. Sp 3.300: \textit{\textsanskrit{Phārusakapānanti} \textsanskrit{phārusakaphalehi} \textsanskrit{ambapānaṁ} viya \textsanskrit{katapānaṁ}}, “\textit{\textsanskrit{Phārusakapāna}}: a drink made like a mango drink but with \textit{\textsanskrit{phārusaka}} fruits.” } }

\scrule{I allow juice from all fruits, except grain. I allow juice from all leaves, except the leaves of potherbs. I allow juice from all flowers, except licorice flowers. I allow sugarcane juice.” }

The\marginnote{35.7.1} following morning \textsanskrit{Keṇiya} had various kinds of fine foods prepared in his own hermitage and then had the Buddha informed that the meal was ready. 

The\marginnote{35.7.2} Buddha robed up, took his bowl and robe, and went to \textsanskrit{Keṇiya}’s hermitage where he sat down on the prepared seat together with the Sangha of monks. \textsanskrit{Keṇiya} then personally served various kinds of fine foods to the Sangha of monks headed by the Buddha. When the Buddha had finished his meal, \textsanskrit{Keṇiya} sat down to one side, and the Buddha expressed his appreciation with these verses: 

\begin{verse}%
“Sacrifice\marginnote{35.8.2} is the best fire worship, \\
\textsanskrit{Sāvittī} the best meter;\footnote{“Meter” refers to the rhythmical pattern of verse. Pali, and presumably other Indian languages, divide syllables into two categories, long and short. The meter then specifies the pattern of long and short syllables in a line of verse. The \textsanskrit{Sāvittī} meter mentioned here will have a particular patten of such syllables. } \\
A king is the best of humans, \\
The ocean the chief of rivers. 

The\marginnote{35.8.6} moon is the best in the night sky, \\
The sun the best of all that shines. \\
But for those making offerings, desiring merit, \\
The Sangha is indeed the best.” 

%
\end{verse}

And\marginnote{35.8.10} the Buddha got up from his seat and left. 

\section*{24. The account of Roja the Mallian }

When\marginnote{36.1.1} the Buddha had stayed at \textsanskrit{Āpaṇa} for as long as he liked, he set out wandering toward \textsanskrit{Kusinārā} with a large sangha of twelve-hundred and fifty monks. When the Mallians of \textsanskrit{Kusinārā} heard that the Buddha was coming, they made an agreement that whoever did not go out to meet the Buddha would be fined five hundred coins. 

At\marginnote{36.1.4} that time Venerable Ānanda had a friend called Roja the Mallian. As the Buddha was approaching \textsanskrit{Kusinārā}, the Mallians, including Roja, went out to meet him. Roja then went to Ānanda and bowed, and Ānanda said to him, “It’s noble of you to come out to meet the Buddha.” 

“I’m\marginnote{36.2.5} not doing this out of respect for the Buddha, the Teaching, or the Sangha. I’m doing it because I would get fined by the Mallians if I didn’t.” 

Ānanda\marginnote{36.2.8} was disappointed with his friend. He went to the Buddha, bowed, sat down, and said, “Sir, Roja the Mallian is a well-known person. It’s of great benefit when such well-known people gain confidence in this spiritual path. Sir, please inspire confidence in Roja.” 

“That’s\marginnote{36.3.6} not difficult for the Buddha, Ānanda.” 

The\marginnote{36.4.1} Buddha then suffused Roja with a mind of loving kindness, before getting up from his seat and entering his dwelling. When Roja was suffused with loving kindness, he acted just like a young calf looking for its mother: he went from dwelling to dwelling, from yard to yard, asking, “Venerables, where’s the Buddha staying, the Perfected and fully Awakened One? I wish to see him.” 

“In\marginnote{36.4.5} that dwelling, Roja, with the closed door. Go there quietly and slowly, enter the porch, clear your throat, and knock on the door. The Buddha will then open the door for you.”\footnote{“Door” renders \textit{\textsanskrit{aggaḷa}}. For a discussion of this word, see Appendix of Technical Terms. } 

Roja\marginnote{36.5.1} did just that, and the Buddha opened the door for him. He entered the dwelling, bowed, and sat down. The Buddha then gave him a progressive talk—on generosity, morality, and heaven; on the downside, degradation, and defilement of worldly pleasures; and he revealed the benefits of renunciation. When the Buddha knew that his mind was ready, supple, without hindrances, joyful, and confident, he revealed the teaching unique to the Buddhas: suffering, its origin, its end, and the path. And just as a clean and stainless cloth absorbs dye properly, so too, while he was sitting right there, Roja experienced the stainless vision of the Truth: “Anything that has a beginning has an end.” He had seen the Truth, had reached, understood, and penetrated it. He had gone beyond doubt and uncertainty, had attained to confidence, and had become independent of others in the Teacher’s instruction. 

He\marginnote{36.5.5} then said to the Buddha, “Sir, please have the venerables accept robe-cloth, almsfood, dwellings, and medicinal supplies from me, and not from others.” 

“Roja,\marginnote{36.5.7} those who have seen the Truth with a trainee’s knowledge and vision, as you have, think like this. But listen, Roja, the monks will have to receive both from you and others.” 

At\marginnote{36.6.1} this time in \textsanskrit{Kusinārā} there was a succession of fine meals. Not being able to get a turn, Roja thought, “Why don’t I inspect the dining hall and then prepare whatever is lacking?” When he did, he saw that two things were missing: potherbs and fresh food made of flour.\footnote{Sp 3.302: \textit{\textsanskrit{Piṭṭhakhādanīyanti} \textsanskrit{piṭṭhamayaṁ} \textsanskrit{khādanīyaṁ}}: “\textit{\textsanskrit{Piṭṭhakhādanīya}}: fresh food made from flour.” } He then went to Venerable Ānanda and told him what he had been thinking, adding, “Venerable Ānanda, if I were to prepare potherbs and fresh food made of flour, would the Buddha accept it?” 

“Well,\marginnote{36.6.12} Roja, let me ask the Buddha.” Venerable Ānanda told the Buddha, who said, “Allow it to be prepared, Ānanda.” Ānanda passed the message on to Roja. 

The\marginnote{36.7.4} following morning Roja prepared many potherbs and much fresh food made with flour and brought it to the Buddha, saying, “Sir, please accept the potherbs and the fresh food made with flour.” 

“Well\marginnote{36.7.6} then, Roja, give it to the monks.” He did, but being afraid of wrongdoing, they did not accept. The Buddha said, “Accept, monks, and eat.” Roja then personally served many potherbs and much fresh food made with flour to the Sangha of monks headed by the Buddha. When the Buddha had finished his meal, Roja sat down to one side. The Buddha instructed, inspired, and gladdened him with a teaching, after which he got up from his seat and left. Soon afterwards the Buddha gave a teaching and addressed the monks: 

\scrule{“I allow all potherbs and all fresh food made of flour.” }

\section*{25. The account of the one who had gone forth when old }

When\marginnote{37.1.1} the Buddha had stayed at \textsanskrit{Kusinārā} for as long as he liked, he set out wandering toward \textsanskrit{Ātumā} with a large sangha of twelve-hundred and fifty monks. At that time at \textsanskrit{Ātumā} there was a monk who was previously a barber and who had gone forth when old. He had two boys, sweet-voiced and articulate, who were skilled barbers.\footnote{Sp 3.303: \textit{Dve \textsanskrit{dārakāti} \textsanskrit{sāmaṇerabhūmiyaṁ} \textsanskrit{ṭhitā} dve \textsanskrit{puttā}}, “\textit{Dve \textsanskrit{dārakā}}: two sons who were novice monks.” } 

The\marginnote{37.2.1} monk who had gone forth when old heard that the Buddha was coming to \textsanskrit{Ātumā}, and he said to those boys, “The Buddha is coming to \textsanskrit{Ātumā} with a large sangha of twelve-hundred and fifty monks. Now go and get the barber equipment, and then go from house to house with a box and collect salt, oil, rice, and fresh food. When the Buddha has arrived, we’ll make a congee drink.” 

Saying,\marginnote{37.3.1} “Yes,” they did just that. When people saw those sweet-voiced and articulate boys, they used their services even if they did not really want to. And they gave much in return. Soon the boys had collected a large amount of salt, oil, rice, and fresh food. 

When\marginnote{37.4.1} the Buddha eventually arrived at \textsanskrit{Ātumā}, he stayed in a dwelling made of husk.\footnote{Sp-\textsanskrit{ṭ} 3.303: \textit{\textsanskrit{Bhusāgāreti} bhusamaye \textsanskrit{agārake}}, “\textit{\textsanskrit{Bhusāgāre}}: a house made of husk.” } The following morning that monk who had gone forth when old had much congee prepared and brought it to the Buddha, saying, “Sir, please accept the congee.” 

When\marginnote{37.4.5} Buddhas know what is going on, sometimes they ask and sometimes not. They know the right time to ask and when not to ask. Buddhas ask when it is beneficial, otherwise they do not, for Buddhas are incapable of doing what is unbeneficial. Buddhas question the monks for two reasons: to give a teaching or to lay down a training rule. 

The\marginnote{37.4.7} Buddha then said to him, “Where does this congee come from?” He told him, and the Buddha rebuked him, “It’s not suitable, foolish man, it’s not proper, it’s not worthy of a monastic, it’s not allowable, it’s not to be done. How can you who have gone forth encourage others in what’s unallowable? This will affect people’s confidence …” After rebuking him, he gave a teaching and addressed the monks: 

\scrule{“You shouldn’t encourage others to do what’s unallowable. If you do, you commit an offense of wrong conduct. And if you were previously a barber, you shouldn’t carry barber equipment around. If you do, you commit an offense of wrong conduct.” }

When\marginnote{38.1.1} the Buddha had stayed at \textsanskrit{Ātumā} for as long as he liked, he set out wandering toward \textsanskrit{Sāvatthī}. When he eventually arrived, he stayed in the Jeta Grove, \textsanskrit{Anāthapiṇḍika}’s Monastery. At that time in \textsanskrit{Sāvatthī} there was much fruit.\footnote{It is not immediately clear whether there was much fruit in general or whether the Sangha had received much fruit. Normally the word \textit{uppanna} is construed with the genitive of the recipient. Since in this case there is no genitive, we can only assume that the fruit had not (yet) been given to the Sangha. } The monks thought, “Which fruits has the Buddha allowed and which not?” They told the Buddha. 

\scrule{“I allow all fruits.” }

On\marginnote{39.1.1} one occasion seeds belonging to the Sangha had been planted on land belonging to an individual and seeds belonging to an individual on land belonging to the Sangha. 

\scrule{“If seeds belonging to the Sangha have been planted on land belonging to an individual, that person should be given a share, and the produce may then be eaten.\footnote{Sp 3.304: \textit{\textsanskrit{Bhāgaṁ} \textsanskrit{datvāti} \textsanskrit{dasamabhāgaṁ} \textsanskrit{datvā}; \textsanskrit{idaṁ} kira \textsanskrit{jambudīpe} \textsanskrit{porāṇakacārittaṁ}, \textsanskrit{tasmā} \textsanskrit{dasakoṭṭhāse} \textsanskrit{katvā} eko \textsanskrit{koṭṭhāso} \textsanskrit{bhūmisāmikānaṁ} \textsanskrit{dātabbo}}, “\textit{\textsanskrit{Bhāgaṁ} \textsanskrit{datvā}}: having given a tenth part. They say this is the ancient custom in India. Therefore, having divided it into ten parts, one part is to be given to the owner of the land.” } If seeds belonging to an individual have been planted on land belonging to the Sangha, the Sangha should be given a share, and the produce may then be eaten.” }

\section*{26. Discussion of the four great standards }

At\marginnote{40.1.1} that time the monks were anxious about all sorts of matters, thinking, “What has the Buddha allowed and what hasn’t he allowed?” They told the Buddha. 

\scrule{“If I haven’t specifically prohibited something, then it’s unallowable to you if it’s similar to what’s unallowable and opposed to what’s allowable. If I haven’t specifically prohibited something, then it’s allowable to you if it’s similar to what’s allowable and opposed to what’s unallowable. If I haven’t specifically allowed something, then it’s unallowable to you if it’s similar to what’s unallowable and opposed to what’s allowable. If I haven’t specifically allowed something, then it’s allowable to you if it’s similar to what’s allowable and opposed to what’s unallowable.” }

Then\marginnote{40.2.1} the monks thought, “Are post-midday tonics mixed with ordinary food allowable or unallowable? Are seven-day tonics mixed with ordinary food allowable or unallowable? Are lifetime tonics mixed with ordinary food allowable or unallowable? Are seven-day tonics mixed with post-midday tonics allowable or unallowable? Are lifetime tonics mixed with post-midday tonics allowable or unallowable? Are lifetime tonics mixed with seven-day tonics allowable or unallowable?” They told the Buddha. 

\scrule{“When mixed with ordinary food, post-midday tonics are allowable before midday on the day they are received, but not after midday. When mixed with ordinary food, seven-day tonics are allowable before midday on the day they are received, but not after midday. When mixed with ordinary food, lifetime tonics are allowable before midday on the day they are received, but not after midday. When mixed with post-midday tonics, seven-day tonics are allowable after midday on the day they are received, but not beyond dawn.\footnote{The point here is that the day ends at dawn. The mixture has the same allowable period as post-midday tonics do on their own. } When mixed with post-midday tonics, lifetime tonics are allowable after midday on the day they are received, but not beyond dawn. When mixed with seven-day tonics, lifetime tonics are allowable for seven days, but not beyond.” }

\scendsutta{The sixth chapter on medicines is finished. }

\scuddanaintro{This is the summary: }

\begin{scuddana}%
“In\marginnote{40.3.9} autumn, also after midday, \\
Fat, about root, and with flours; \\
With bitter, leaf, fruit, \\
Gum, salt, and detergent. 

Powder,\marginnote{40.3.13} sieve, and meat, \\
Ointment, scented; \\
Ointment box, luxurious, uncovered, \\
Ointment stick, ointment stick case. 

Bag,\marginnote{40.3.17} shoulder strap, string, \\
Head oil, and nose; \\
Nose dropper, and smoke, \\
And tube, lid, bag. 

In\marginnote{40.3.21} a concoction of oil, and alcohol, \\
Too much, external use; \\
Vessel, sweat, and herbs, \\
Heavy, and so hemp water. 

Bathtub,\marginnote{40.3.25} and blood, \\
Horn, salve for the feet; \\
Foot salve, knife, and bitter, \\
Sesame paste, flour paste. 

Cloth,\marginnote{40.3.29} and mustard powder, \\
Smoke, and with a razor; \\
Sore oil, bandage, \\
And foul, receiving. 

Feces,\marginnote{40.3.33} excreting, and mixture, \\
Lye, chebulic myrobalan in urine; \\
Scented, and purgative, \\
Clear congee, mung-bean broth, oily mung-bean broth. 

Meat\marginnote{40.3.37} broth, hillside, \\
Monastery, and with seven days; \\
Sugar, mung beans, and purgative, \\
Cooking oneself, reheating. 

He\marginnote{40.3.41} allowed again, when short of food, \\
And fruit, sesame, fresh food; \\
Before eating, fever, \\
And removed, hemorrhoids. 

And\marginnote{40.3.45} enema, and Suppi, \\
And human flesh; \\
Elephant, horse, and dog, \\
Snake, lion, leopard. 

Bear,\marginnote{40.3.49} and hyena flesh, \\
And turn, and congee; \\
Recent, apart from, sugar, \\
Sunidha, guesthouse. 

Ganges,\marginnote{40.3.53} \textsanskrit{Koṭi}, speaking the truths, \\
And \textsanskrit{Ambapālī}, \textsanskrit{Licchavī}; \\
Killed for, plenty of food, \\
He prohibited again. 

Storm,\marginnote{40.3.57} Yasa, and \textsanskrit{Meṇḍaka}, \\
Product of a cow, and with provisions; \\
\textsanskrit{Keṇi}, mango, rose apple, bananas with seeds, \\
Seedless bananas, licorice, grapes, lotus tubers. 

Falsa\marginnote{40.3.61} fruit, potherbs, flour, \\
At Ātuma, barber; \\
At \textsanskrit{Sāvatthī}, fruit, seed, \\
And about all sorts of matters, in the time period.” 

%
\end{scuddana}

\scend{In this chapter there are one hundred and six topics. }

\scendsutta{The chapter on medicines is finished. }

%
\chapter*{{\suttatitleacronym Kd 7}{\suttatitletranslation The chapter on the robe-making ceremony }{\suttatitleroot Kathinakkhandhaka}}
\addcontentsline{toc}{chapter}{\tocacronym{Kd 7} \toctranslation{The chapter on the robe-making ceremony } \tocroot{Kathinakkhandhaka}}
\markboth{The chapter on the robe-making ceremony }{Kathinakkhandhaka}
\extramarks{Kd 7}{Kd 7}

\section*{1. The allowance for a robe-making ceremony }

At\marginnote{1.1.1} one time the Buddha was staying at \textsanskrit{Sāvatthī} in the Jeta Grove, \textsanskrit{Anāthapiṇḍika}’s Monastery. At that time thirty monks from \textsanskrit{Pāvā}—all wilderness dwellers, almsfood-only eaters, rag-robe wearers, and three-robe owners—were traveling to \textsanskrit{Sāvatthī} to visit the Buddha. Because the entry to the rainy-season residence was approaching, they were unable to reach \textsanskrit{Sāvatthī}, and they entered the rains residence at \textsanskrit{Sāketa} while still on their way. They spent the rains residence discontented, thinking, “The Buddha is only 80 kilometers away, yet we don’t get to see him.” 

When\marginnote{1.1.5} they had completed the rainy-season residence and done the invitation ceremony at the end of the three months, it was raining, with water and mud everywhere. As they traveled to \textsanskrit{Sāvatthī}, they were exhausted, their robes soaked. 

When\marginnote{1.1.6} they arrived at \textsanskrit{Sāvatthī}, they went to \textsanskrit{Anāthapiṇḍika}’s Monastery, bowed to the Buddha, and sat down.  Since it is the custom for Buddhas to greet newly-arrived monks, 

the\marginnote{1.2.2} Buddha said to them, “I hope you’re keeping well, monks, I hope you’re getting by?  I hope you had a comfortable rains, that you lived together in peace and harmony, and got almsfood without trouble?” 

“We’re\marginnote{1.2.5} keeping well, sir, we’re getting by. We had a comfortable rains, lived together in peace and harmony, and had no trouble getting almsfood.” They told the Buddha what had happened during the rains and while traveling to \textsanskrit{Sāvatthī}. 

Soon\marginnote{1.3.1} afterwards the Buddha gave a teaching and addressed the monks: 

\scrule{“I allow monks who have completed the rainy-season residence to participate in a robe-making ceremony.\footnote{For an explanation of rendering \textit{kathina} as “robe-making ceremony”, see Appendix of Technical Terms. } Once you have participated in the robe-making ceremony, five things are allowable for you: going without informing, going without taking, eating in a group, as much robe-cloth as you need, and whatever robe-cloth is given there is for you.\footnote{For the first four of these five see \href{https://suttacentral.net/pli-tv-bu-vb-pc46/en/brahmali\#5.6.1}{Bu Pc 46:5.6.1}, \href{https://suttacentral.net/pli-tv-bu-vb-np2/en/brahmali\#2.39.1}{Bu Np 2:2.39.1}, \href{https://suttacentral.net/pli-tv-bu-vb-pc32/en/brahmali\#8.15.1}{Bu Pc 32:8.15.1}, and \href{https://suttacentral.net/pli-tv-bu-vb-np1/en/brahmali\#2.17.1}{Bu Np 1:2.17.1} respectively. “Robe-cloth” renders \textit{\textsanskrit{cīvara}}, for which see Appendix of Technical Terms. } }

And\marginnote{1.3.6} the robe-making ceremony should be performed like this. A competent and capable monk should inform the Sangha: 

‘Please,\marginnote{1.4.2} venerables, I ask the Sangha to listen. This cloth has been given to the Sangha for the robe-making ceremony. If the Sangha is ready, it should give this cloth to monk so-and-so to perform the robe-making ceremony. This is the motion. 

Please,\marginnote{1.4.6} venerables, I ask the Sangha to listen. This cloth has been given to the Sangha for the robe-making ceremony. The Sangha gives this cloth to monk so-and-so to perform the robe-making ceremony. Any monk who approves of giving this cloth to monk so-and-so to perform the robe-making ceremony should remain silent. Any monk who doesn’t approve should speak up. 

The\marginnote{1.4.11} Sangha has given this cloth to monk so-and-so to perform the robe-making ceremony. The Sangha approves and is therefore silent. I’ll remember it thus.’ 

And,\marginnote{1.5.2} monks, how has the robe-making ceremony not been performed? The robe-making ceremony hasn’t been performed merely by marking the cloth,\footnote{Sp 3.308: \textit{\textsanskrit{Ullikhitamattenāti} \textsanskrit{dīghato} ca puthulato ca \textsanskrit{pamāṇaggahaṇamattena}}, “\textit{Ullikhitamattena}: merely by taking the measure lengthwise or crosswise.” } merely by washing the cloth,  merely by planning the robe,\footnote{Sp 3.308: \textit{\textsanskrit{Cīvaravicāraṇamattenāti} “\textsanskrit{pañcakaṁ} \textsanskrit{vā} \textsanskrit{sattakaṁ} \textsanskrit{vā} \textsanskrit{navakaṁ} \textsanskrit{vā} \textsanskrit{ekādasakaṁ} \textsanskrit{vā} \textsanskrit{hotū}”ti \textsanskrit{evaṁ} \textsanskrit{vicāritamattena}}, “\textit{\textsanskrit{Cīvaravicāraṇamattena}}: merely by planning the robe in this way: let it consist of five, seven, nine, or eleven.” Vmv 3.308 specifies: \textit{\textsanskrit{Pañcakanti} \textsanskrit{pañcakhaṇḍaṁ}}, “\textit{\textsanskrit{Pañcaka}} means: five sections.” }  merely by cutting the cloth, merely by tacking the cloth,\footnote{Sp 3.308: \textit{\textsanskrit{Bandhanamattenāti} \textsanskrit{moghasuttakāropanamattena}}, “\textit{Bandhanamattena}: merely by inserting a false thread.” } merely by sewing a hem,\footnote{Sp 3.308: \textit{\textsanskrit{Ovaṭṭiyakaraṇamattenāti} \textsanskrit{moghasuttakānusārena} \textsanskrit{dīghasibbitamattena}}, “\textit{\textsanskrit{Ovaṭṭiyakaraṇamattena}}: merely by sewing a long seam in conformity with the false thread.” } merely by marking with a strip of cloth,\footnote{Sp 3.308: \textit{\textsanskrit{Kaṇḍusakaraṇamattenāti} muddhiyapattabandhanamattena}, “\textit{\textsanskrit{Kaṇḍusakaraṇamattena}} means merely by fixing a panel for calculating.” Vjb 3.308: \textit{\textsanskrit{Kaṇḍusaṁ} \textsanskrit{nāma} pubbabandhana}, “\textit{\textsanskrit{Kaṇḍusa}} is a prior fixing.” } merely by strengthening, merely by adding a border lengthwise,\footnote{Sp 3.308: \textit{\textsanskrit{Anuvātakaraṇamattenāti} \textsanskrit{piṭṭhianuvātāropanamattena}}, “\textit{\textsanskrit{Anuvātakaraṇamattena}} means merely by mounting a border at the back.” This is further explained at Sp-\textsanskrit{ṭ} 3.308: \textit{\textsanskrit{Piṭṭhianuvātāropanamattenāti} \textsanskrit{dīghato} \textsanskrit{anuvātassa} \textsanskrit{āropanamattena}}, “\textit{\textsanskrit{Piṭṭhianuvātāropanamattena}} means merely by mounting a border lengthwise.” } merely by adding a border crosswise,\footnote{Sp 3.308: \textit{\textsanskrit{Paribhaṇḍakaraṇamattenāti} \textsanskrit{kucchianauvātāropanamattena}}, “\textit{\textsanskrit{Paribhaṇḍakaraṇamattena}} means merely by mounting a border at the belly.” This is further explained at Sp-\textsanskrit{ṭ} 3.308: \textit{\textsanskrit{Kucchianuvātāropanamattenāti} puthulato \textsanskrit{anuvātassa} \textsanskrit{āropanamattena}}, “\textit{\textsanskrit{Kucchianuvātāropanamattena}} means merely by adding a border crosswise.” } merely by patching,\footnote{Sp 3.308: \textit{\textsanskrit{Ovaddheyyakaraṇamattenāti} \textsanskrit{āgantukapattāropanamattena}; \textsanskrit{kathinacīvarato} \textsanskrit{vā} \textsanskrit{pattaṁ} \textsanskrit{gahetvā} \textsanskrit{aññasmiṁ} \textsanskrit{akathinacīvare} \textsanskrit{pattāropanamattena}}, “\textit{\textsanskrit{Ovaddheyyakaraṇamattena}}: merely by adding a panel to an external (robe); having taken a panel from the cloth for the robe-making ceremony, then adding it to another robe, which is not the cloth for the robe-making ceremony.” } merely by partial dyeing;\footnote{Sp 3.308: \textit{\textsanskrit{Kambalamaddanamattenāti} \textsanskrit{ekavāraṁyeva} rajane pakkhittena \textsanskrit{dantavaṇṇena} \textsanskrit{paṇḍupalāsavaṇṇena} \textsanskrit{vā}}, “\textit{Kambalamaddanamattena}: dyeing it just once by putting it into the color of ivory or beige.” The implication seems to be that the \textit{kathina} ceremony can be done by a process of proper dyeing. Sp 3.308: \textit{Sace pana \textsanskrit{sakiṁ} \textsanskrit{vā} \textsanskrit{dvikkhattuṁ} \textsanskrit{vā} rattampi \textsanskrit{sāruppaṁ} hoti, \textsanskrit{vaṭṭati}}, “But if it is suitably dyed, once or twice, it is allowable.” } nor has it been performed if a monk has made an indication,\footnote{Sp 3.308: \textit{\textsanskrit{Nimittakatenāti} “‘\textsanskrit{iminā} dussena \textsanskrit{kathinaṁ} \textsanskrit{attharissāmī}’ti \textsanskrit{evaṁ} nimittakatena. Ettakameva hi \textsanskrit{parivāre} \textsanskrit{vuttaṁ}. \textsanskrit{Aṭṭhakathāsu} pana ‘\textsanskrit{ayaṁ} \textsanskrit{sāṭako} sundaro, \textsanskrit{sakkā} \textsanskrit{iminā} \textsanskrit{kathinaṁ} attharitu’nti \textsanskrit{evaṁ} \textsanskrit{nimittakammaṁ} \textsanskrit{katvā} \textsanskrit{laddhenā}”ti \textsanskrit{vuttaṁ}}, “\textit{Nimittakatena}: it is said in the Compendium that it means making an indication in this way: ‘I will do the robe-making ceremony with this cloth.’ But it is said in the commentaries that it is by obtaining (a robe) after making an indication in this way: ‘This cloth is beautiful; it is possible to do the robe-making ceremony with it.’” } if a monk has given a hint,\footnote{Sp 3.308: \textit{\textsanskrit{Parikathākatenāti} “\textsanskrit{kathinaṁ} \textsanskrit{nāma} \textsanskrit{dātuṁ} \textsanskrit{vaṭṭati}, \textsanskrit{kathinadāyako} \textsanskrit{bahuṁ} \textsanskrit{puññaṁ} \textsanskrit{pasavatī}”ti \textsanskrit{evaṁ} \textsanskrit{parikathāya} \textsanskrit{uppāditena}}, “\textit{\textsanskrit{Parikathākatena}}: by one who causes it to be given by hinting in this way: ‘It is allowable to give a cloth for the robe-making ceremony; one who gives this makes much merit.’” } if the robe-cloth has been borrowed,\footnote{Sp 3.308: \textit{\textsanskrit{Kukkukatenāti} \textsanskrit{tāvakālikena}}, “\textit{Kukkukatena}: with one that is borrowed.” } if it has been stored, if it is to be relinquished, if it hasn’t been marked,\footnote{For the meaning of \textit{akappakatena} see \href{https://suttacentral.net/pli-tv-bu-vb-pc58/en/brahmali\#2.1.2}{Bu Pc 58:2.1.2}. } if it’s not an outer robe or an upper robe or a sarong; nor has it been performed if the robe hasn’t been made on that very day with five or more cut sections with panels,\footnote{Sp 3.308 explains \textit{\textsanskrit{pañcakena} \textsanskrit{vā} \textsanskrit{atirekapañcakena} \textsanskrit{vā}}, “five or more”, as \textit{\textsanskrit{pañca} \textsanskrit{vā} \textsanskrit{atirekāni} \textsanskrit{vā} \textsanskrit{khaṇḍāni}}, “five or more sections”. Each section is made up of a large panel (\textit{\textsanskrit{maṇḍala}}) and a medium-sized panel (\textit{\textsanskrit{aḍḍhamaṇḍala}}) with a strip (\textit{\textsanskrit{aḍḍhakusi}}) in between. In this case \textit{\textsanskrit{maṇḍala}} seems to be used as an umbrella term for both \textit{\textsanskrit{maṇḍala}} and \textit{\textsanskrit{aḍḍhamaṇḍala}}. Sp 3.308: \textit{\textsanskrit{Mahāmaṇḍalaaḍḍhamaṇḍalāni} \textsanskrit{dassetvā}}, “Showing large panels and medium-sized panels.” See also \href{https://suttacentral.net/pli-tv-kd8/en/brahmali\#12.2.3}{Kd 8:12.2.3}. } if the robe-making ceremony wasn’t performed by an individual,\footnote{Sp 3.308: \textit{\textsanskrit{Aññatra} puggalassa \textsanskrit{atthārāti} puggalassa \textsanskrit{atthāraṁ} \textsanskrit{ṭhapetvā} na \textsanskrit{aññena} \textsanskrit{saṅghassa} \textsanskrit{vā} \textsanskrit{gaṇassa} \textsanskrit{vā} \textsanskrit{atthārena} \textsanskrit{atthataṁ} hoti}, “\textit{\textsanskrit{Aññatra} puggalassa \textsanskrit{atthārā}}: apart from an individual performing it, there is no other performing it by a sangha or by a group.” The performing, literally, “spreading”, does not refer to the making of the robe, but to the declaration made when the robe is complete. Sp 3.306: \textit{\textsanskrit{Katapariyositaṁ} pana \textsanskrit{kathinaṁ} \textsanskrit{gahetvā} \textsanskrit{atthārakena} \textsanskrit{bhikkhunā} “sace \textsanskrit{saṅghāṭiyā} \textsanskrit{kathinaṁ} \textsanskrit{attharitukāmo} hoti, \textsanskrit{porāṇikā} \textsanskrit{saṅghāṭi} \textsanskrit{paccuddharitabbā}, \textsanskrit{navā} \textsanskrit{saṅghāṭi} \textsanskrit{adhiṭṭhātabbā}, ‘\textsanskrit{Imāya} \textsanskrit{saṅghāṭiyā} \textsanskrit{kathinaṁ} \textsanskrit{attharāmī}’ti \textsanskrit{vācā} \textsanskrit{bhinditabbā}”\textsanskrit{tiādinā} \textsanskrit{parivāre} \textsanskrit{vuttavidhānena} \textsanskrit{kathinaṁ} \textsanskrit{attharitabbaṁ}}, “By the monk who is performing the ceremony, having taken the completed \textit{kathina} (robe), the \textit{kathina} ceremony is performed by the ceremony spoken of in the Compendium: ‘If he wants to perform the \textit{kathina} ceremony with an outer robe, he should first relinquish his old outer robe and determine the new one, and then say, “I perform the \textit{kathina} ceremony with this outer robe”’, etc.” } or if the robe-making ceremony has been performed correctly but the appreciation for the ceremony was expressed outside the monastery zone.\footnote{Sp 3.306 explains the appreciation as follows: \textit{Tehi anumodakehi \textsanskrit{bhikkhūhi} \textsanskrit{ekaṁsaṁ} \textsanskrit{uttarāsaṅgaṁ} \textsanskrit{karitvā} \textsanskrit{añjaliṁ} \textsanskrit{paggahetvā} evamassa \textsanskrit{vacanīyo} – “\textsanskrit{atthataṁ} \textsanskrit{āvuso} \textsanskrit{saṅghassa} \textsanskrit{kathinaṁ}, dhammiko \textsanskrit{kathinatthāro}, \textsanskrit{anumodāmā}”ti \textsanskrit{evamādinā} \textsanskrit{parivāre} \textsanskrit{vuttavidhāneneva} \textsanskrit{anumodāpetabbaṁ}}, “The expression of appreciation is to be done by the ceremony spoken of in the Compendium, thus: the monks who express their appreciation should put their upper robe over one shoulder, put the palms of their hands together, and say this: ‘The \textit{kathina} ceremony has been done by the Sangha, it is legitimate, we express our appreciation.’” “Monastery zone” renders \textit{\textsanskrit{sīmā}}. See Appendix of Technical Terms. } In this way the robe-making ceremony hasn’t been performed. 

And\marginnote{1.6.1} how has the robe-making ceremony been performed? The robe-making ceremony has been performed if the cloth is brand new, if it’s nearly new, if it’s old, if it’s a rag, if it’s from a shop; it has been performed if a monk hasn’t made an indication, if a monk hasn’t given a hint, if the robe-cloth hasn’t been borrowed, if it hasn’t been stored, if it’s not to be relinquished, if it has been marked, if it’s an outer robe or an upper robe or a sarong; it has been performed if the robe has been made on that very day with five or more cut sections with panels, if the robe-making ceremony was performed by an individual, if the robe-making ceremony has been performed correctly and if the appreciation for the ceremony was expressed inside the monastery zone.\footnote{Sp 3.309: \textit{\textsanskrit{Ahatenāti} aparibhuttena}, “\textit{Ahatena}: not used.” Sp 3.309: \textit{\textsanskrit{Ahatakappenāti} ahatasadisena \textsanskrit{ekavāraṁ} \textsanskrit{vā} \textsanskrit{dvikkhattuṁ} \textsanskrit{vā} dhotena}, “\textit{Ahatakappena}: similar to one that is brand new; washed once or twice.” Sp 3.309: \textit{\textsanskrit{Pilotikāyāti} \textsanskrit{hatavatthakasāṭakena}}, “\textit{\textsanskrit{Pilotikāya}}: a used robe-cloth.” } In this way the robe-making ceremony has been performed. 

And\marginnote{1.7.1} how does the robe season come to an end? There are these eight key phrases for when the robe season ends: when he departs from the monastery, when the robe is finished, when he makes a decision, when the robe-cloth is lost, when he hears about the end of the robe season, when an expectation of more robe-cloth is disappointed, when he is outside the monastery zone, ending together.” 

\section*{2. The group of seven on “takes” }

A\marginnote{2.1.1} monk who has participated in the robe-making ceremony takes a finished robe and leaves the monastery, thinking, “I won’t return.” For that monk the robe season ends when he departs from the monastery. 

A\marginnote{2.1.3} monk who has participated in the robe-making ceremony takes robe-cloth and leaves the monastery. When he is outside the monastery zone, he thinks, “I’ll make the robe right here. I won’t return.” He then has the robe made. For that monk the robe season ends when the robe is finished. 

A\marginnote{2.1.7} monk who has participated in the robe-making ceremony takes robe-cloth and leaves the monastery. When he is outside the monastery zone, he thinks, “I won’t make a robe, and I won’t return.” For that monk the robe season ends when he makes that decision. 

A\marginnote{2.1.10} monk who has participated in the robe-making ceremony takes robe-cloth and leaves the monastery. When he is outside the monastery zone, he thinks, “I’ll make the robe right here. I won’t return.” He has the robe made, but it is lost while being made. For that monk the robe season ends when the robe-cloth is lost. 

A\marginnote{2.2.1} monk who has participated in the robe-making ceremony takes robe-cloth and leaves the monastery, thinking, “I’ll return.” When he is outside the monastery zone, he has the robe made. When the robe has been made, he hears that they have made an end to the robe season in that monastery. For that monk the robe season ends when he hears about the end of the robe season. 

A\marginnote{2.2.5} monk who has participated in the robe-making ceremony takes robe-cloth and leaves the monastery, thinking, “I’ll return.” When he is outside the monastery zone, he has the robe made. When the robe has been made, he still thinks, “I’ll return,” but he remains outside the monastery zone until the end of the robe season. For that monk the robe season ends while he is outside the monastery zone. 

A\marginnote{2.2.9} monk who has participated in the robe-making ceremony takes robe-cloth and leaves the monastery, thinking, “I’ll return.” When he is outside the monastery zone, he has the robe made. When the robe has been made, he still thinks, “I’ll return,” and they reach the end of the robe season together. For that monk the robe season ends together with the other monks.\footnote{The point seems to be that he makes it back to the monastery before the end of the robe season. } 

\scend{The group of seven on “takes” is finished. }

\section*{3. The group of seven on “with” }

A\marginnote{3.1.1} monk who has participated in the robe-making ceremony leaves the monastery with a finished robe, thinking, “I won’t return.” For that monk the robe season ends when he departs from the monastery. 

A\marginnote{3.1.3} monk who has participated in the robe-making ceremony leaves the monastery with robe-cloth. When he is outside the monastery zone, he thinks, “I’ll make the robe right here. I won’t return.” He then has the robe made. For that monk the robe season ends when the robe is finished. 

A\marginnote{3.1.7} monk who has participated in the robe-making ceremony leaves the monastery with robe-cloth. When he is outside the monastery zone, he thinks, “I won’t make a robe, and I won’t return.” For that monk the robe season ends when he makes that decision. 

A\marginnote{3.1.10} monk who has participated in the robe-making ceremony leaves the monastery with robe-cloth. When he is outside the monastery zone, he thinks, “I’ll make the robe right here. I won’t return.” He has the robe made, but it is lost while being made. For that monk the robe season ends when the robe-cloth is lost. 

A\marginnote{3.2.1} monk who has participated in the robe-making ceremony leaves the monastery with robe-cloth, thinking, “I’ll return.” When he is outside the monastery zone, he has the robe made. When the robe has been made, he hears that they have made an end to the robe season in that monastery. For that monk the robe season ends when he hears about the end of the robe season. 

A\marginnote{3.2.5} monk who has participated in the robe-making ceremony leaves the monastery with robe-cloth, thinking, “I’ll return.” When he is outside the monastery zone, he has the robe made. When the robe has been made, he still thinks, “I’ll return,” but he remains outside the monastery zone until the end of the robe season. For that monk the robe season ends while he is outside the monastery zone. 

A\marginnote{3.2.9} monk who has participated in the robe-making ceremony leaves the monastery with robe-cloth, thinking, “I’ll return.” When he is outside the monastery zone, he has the robe made. When the robe has been made, he still thinks, “I’ll return,” and they reach the end of the robe season together. For that monk the robe season ends together with the other monks. 

\scend{The group of seven on “with” is finished. }

\section*{4. The group of six on “takes” }

A\marginnote{4.1.1} monk who has participated in the robe-making ceremony takes an unfinished robe and leaves the monastery. When he is outside the monastery zone, he thinks, “I’ll make the robe right here. I won’t return.” He then has the robe made. For that monk the robe season ends when the robe is finished. 

A\marginnote{4.1.5} monk who has participated in the robe-making ceremony takes an unfinished robe and leaves the monastery. When he is outside the monastery zone, he thinks, “I won’t make a robe, and I won’t return.” For that monk the robe season ends when he makes that decision. 

A\marginnote{4.1.8} monk who has participated in the robe-making ceremony takes an unfinished robe and leaves the monastery. When he is outside the monastery zone, he thinks, “I’ll make the robe right here. I won’t return.” He has the robe made, but it is lost while being made. For that monk the robe season ends when the robe-cloth is lost. 

A\marginnote{4.1.13} monk who has participated in the robe-making ceremony takes an unfinished robe and leaves the monastery, thinking, “I’ll return.” When he is outside the monastery zone, he has the robe made. When the robe has been made, he hears that they have made an end to the robe season in that monastery. For that monk the robe season ends when he hears about the end of the robe season. 

A\marginnote{4.1.18} monk who has participated in the robe-making ceremony takes an unfinished robe and leaves the monastery, thinking, “I’ll return.” When he is outside the monastery zone, he has the robe made. When the robe has been made, he still thinks, “I’ll return,” but he remains outside the monastery zone until the end of the robe season. For that monk the robe season ends while he is outside the monastery zone. 

A\marginnote{4.1.22} monk who has participated in the robe-making ceremony takes an unfinished robe and leaves the monastery, thinking, “I’ll return.” When he is outside the monastery zone, he has the robe made. When the robe has been made, he still thinks, “I’ll return,” and they reach the end of the robe season together. For that monk the robe season ends together with the other monks. 

\scend{The group of six on “takes” is finished. }

\section*{5. The group of six on “with” }

A\marginnote{5.1.1} monk who has participated in the robe-making ceremony leaves the monastery with an unfinished robe. When he is outside the monastery zone, he thinks, “I’ll make the robe right here. I won’t return.” He then has the robe made. For that monk the robe season ends when the robe is finished. 

A\marginnote{5.1.5} monk who has participated in the robe-making ceremony leaves the monastery with an unfinished robe. When he is outside the monastery zone, he thinks, “I won’t make a robe, and I won’t return.” For that monk the robe season ends when he makes that decision. 

A\marginnote{5.1.8} monk who has participated in the robe-making ceremony leaves the monastery with an unfinished robe. When he is outside the monastery zone, he thinks, “I’ll make the robe right here. I won’t return.” He has the robe made, but it is lost while being made. For that monk the robe season ends when the robe-cloth is lost. 

A\marginnote{5.1.12} monk who has participated in the robe-making ceremony leaves the monastery with an unfinished robe, thinking, “I’ll return.” When he is outside the monastery zone, he has the robe made. When the robe has been made, he hears that they have made an end to the robe season in that monastery. For that monk the robe season ends when he hears about the end of the robe season. 

A\marginnote{5.1.16} monk who has participated in the robe-making ceremony leaves the monastery with an unfinished robe, thinking, “I’ll return.” When he is outside the monastery zone, he has the robe made. When the robe has been made, he still thinks, “I’ll return,” but he remains outside the monastery zone until the end of the robe season. For that monk the robe season ends while he is outside the monastery zone. 

A\marginnote{5.1.20} monk who has participated in the robe-making ceremony leaves the monastery with an unfinished robe, thinking, “I’ll return.” When he is outside the monastery zone, he has the robe made. When the robe has been made, he still thinks, “I’ll return,” and they reach the end of the robe season together. For that monk the robe season ends together with the other monks. 

\scend{The group of six on “with” is finished. }

\section*{6. The group of fifteen on “takes” }

A\marginnote{6.1.1} monk who has participated in the robe-making ceremony takes robe-cloth and leaves the monastery. When he is outside the monastery zone, he thinks, “I’ll make the robe right here. I won’t return.” He then has the robe made. For that monk the robe season ends when the robe is finished. 

A\marginnote{6.1.5} monk who has participated in the robe-making ceremony takes robe-cloth and leaves the monastery. When he is outside the monastery zone, he thinks, “I won’t make a robe, and I won’t return.” For that monk the robe season ends when he makes that decision. 

A\marginnote{6.1.8} monk who has participated in the robe-making ceremony takes robe-cloth and leaves the monastery. When he is outside the monastery zone, he thinks, “I’ll make the robe right here. I won’t return.” He has the robe made, but it is lost while being made. For that monk the robe season ends when the robe-cloth is lost. 

\scend{The group of three is finished. }

A\marginnote{6.2.1} monk who has participated in the robe-making ceremony takes robe-cloth and leaves the monastery, thinking, “I won’t return.” When he is outside the monastery zone, he thinks, “I’ll make the robe right here.” He then has the robe made. For that monk the robe season ends when the robe is finished. 

A\marginnote{6.2.5} monk who has participated in the robe-making ceremony takes robe-cloth and leaves the monastery, thinking, “I won’t return.” When he is outside the monastery zone, he thinks, “I won’t make a robe.” For that monk the robe season ends when he makes that decision. 

A\marginnote{6.2.8} monk who has participated in the robe-making ceremony takes robe-cloth and leaves the monastery, thinking, “I won’t return.” When he is outside the monastery zone, he thinks, “I’ll make the robe right here.” He has the robe made, but it is lost while being made. For that monk the robe season ends when the robe-cloth is lost. 

\scend{The group of three is finished. }

A\marginnote{6.3.1} monk who has participated in the robe-making ceremony takes robe-cloth and leaves the monastery. He has not decided whether he will return or not. When he is outside the monastery zone, he thinks, “I’ll make the robe right here. I won’t return.” He then has the robe made. For that monk the robe season ends when the robe is finished. 

A\marginnote{6.3.5} monk who has participated in the robe-making ceremony takes robe-cloth and leaves the monastery. He has not decided whether he will return or not. When he is outside the monastery zone, he thinks, “I won’t make a robe, and I won’t return.” For that monk the robe season ends when he makes that decision. 

A\marginnote{6.3.8} monk who has participated in the robe-making ceremony takes robe-cloth and leaves the monastery. He has not decided whether he will return or not. When he is outside the monastery zone, he thinks, “I’ll make the robe right here. I won’t return.” He has the robe made, but it is lost while being made. For that monk the robe season ends when the robe-cloth is lost. 

\scend{The group of three is finished. }

A\marginnote{6.4.1} monk who has participated in the robe-making ceremony takes robe-cloth and leaves the monastery, thinking, “I’ll return.” When he is outside the monastery zone, he thinks, “I’ll make the robe right here. I won’t return.” He then has the robe made. For that monk the robe season ends when the robe is finished. 

A\marginnote{6.4.5} monk who has participated in the robe-making ceremony takes robe-cloth and leaves the monastery, thinking, “I’ll return.” When he is outside the monastery zone, he thinks, “I won’t make a robe, and I won’t return.” For that monk the robe season ends when he makes that decision. 

A\marginnote{6.4.8} monk who has participated in the robe-making ceremony takes robe-cloth and leaves the monastery, thinking, “I’ll return.” When he is outside the monastery zone, he thinks, “I’ll make the robe right here. I won’t return.” He has the robe made, but it is lost while being made. For that monk the robe season ends when the robe-cloth is lost. 

A\marginnote{6.4.12} monk who has participated in the robe-making ceremony takes robe-cloth and leaves the monastery, thinking, “I’ll return.” When he is outside the monastery zone, he has the robe made. When the robe has been made, he hears that they have made an end to the robe season in that monastery. For that monk the robe season ends when he hears about the end of the robe season. 

A\marginnote{6.4.16} monk who has participated in the robe-making ceremony takes robe-cloth and leaves the monastery, thinking, “I’ll return.” When he is outside the monastery zone, he has the robe made. When the robe has been made, he still thinks, “I’ll return,” but he remains outside the monastery zone until the end of the robe season. For that monk the robe season ends while he is outside the monastery zone. 

A\marginnote{6.4.20} monk who has participated in the robe-making ceremony takes robe-cloth and leaves the monastery, thinking, “I’ll return.” When he is outside the monastery zone, he has the robe made. When the robe has been made, he still thinks, “I’ll return,” and they reach the end of the robe season together. For that monk the robe season ends together with the other monks. 

\scend{The group of six is finished. The group of fifteen on “takes” is finished. }

\section*{7. The group of fifteen on “with”, etc. }

A\marginnote{7.1.1} monk who has participated in the robe-making ceremony leaves the monastery with robe-cloth. … 

(To\marginnote{7.1.2} be expanded in detail as in the section on “takes”, \href{https://suttacentral.net/pli-tv-kd7\#6.1.1}{Kd 7:6.1.1}–6.4.23.) 

\subsection*{The group of fifteen on “takes an unfinished” }

A\marginnote{7.1.3.1} monk who has participated in the robe-making ceremony takes an unfinished robe and leaves the monastery. When he is outside the monastery zone, he thinks, “I’ll make the robe right here. I won’t return.” He then has the robe made. For that monk the robe season ends when the robe is finished. 

(To\marginnote{7.1.7} be expanded in detail as in the section on “with”, \href{https://suttacentral.net/pli-tv-kd7\#7.1.1}{Kd 7:7.1.1}–7.1.2 = \href{https://suttacentral.net/pli-tv-kd7\#6.1.1}{Kd 7:6.1.1}–6.4.23.) 

\section*{8. The group of fifteen on “with an unfinished” }

A\marginnote{7.1.8.1} monk who has participated in the robe-making ceremony leaves the monastery with an unfinished robe. When he is outside the monastery zone, he thinks, “I’ll make the robe right here. I won’t return.” He then has the robe made. For that monk the robe season ends when the robe is finished. 

A\marginnote{7.1.12} monk who has participated in the robe-making ceremony leaves the monastery with an unfinished robe. When he is outside the monastery zone, he thinks, “I won’t make a robe, and I won’t return.” For that monk the robe season ends when he makes that decision. 

A\marginnote{7.1.15} monk who has participated in the robe-making ceremony leaves the monastery with an unfinished robe. When he is outside the monastery zone, he thinks, “I’ll make the robe right here. I won’t return.” He has the robe made, but it is lost while being made. For that monk the robe season ends when the robe-cloth is lost. 

\scend{The group of three is finished. }

A\marginnote{7.1.20} monk who has participated in the robe-making ceremony leaves the monastery with an unfinished robe, thinking, “I won’t return.” When he is outside the monastery zone, he thinks, “I’ll make the robe right here.” He then has the robe made. For that monk the robe season ends when the robe is finished. 

A\marginnote{7.1.24} monk who has participated in the robe-making ceremony leaves the monastery with an unfinished robe, thinking, “I won’t return.” When he is outside the monastery zone, he thinks, “I won’t make a robe.” For that monk the robe season ends when he makes that decision. 

A\marginnote{7.1.27} monk who has participated in the robe-making ceremony leaves the monastery with an unfinished robe, thinking, “I won’t return.” When he is outside the monastery zone, he thinks, “I’ll make the robe right here.” He has the robe made, but it is lost while being made. For that monk the robe season ends when the robe-cloth is lost. 

\scend{The group of three is finished. }

A\marginnote{7.1.32} monk who has participated in the robe-making ceremony leaves the monastery with an unfinished robe. He has not decided whether he will return or not. When he is outside the monastery zone, he thinks, “I’ll make the robe right here. I won’t return.” He then has the robe made. For that monk the robe season ends when the robe is finished. 

A\marginnote{7.1.36} monk who has participated in the robe-making ceremony leaves the monastery with an unfinished robe. He has not decided whether he will return or not. When he is outside the monastery zone, he thinks, “I won’t make a robe, and I won’t return.” For that monk the robe season ends when he makes that decision. 

A\marginnote{7.1.39} monk who has participated in the robe-making ceremony leaves the monastery with an unfinished robe. He has not decided whether he will return or not. When he is outside the monastery zone, he thinks, “I’ll make the robe right here. I won’t return.” He has the robe made, but it is lost while being made. For that monk the robe season ends when the robe-cloth is lost. 

\scend{The group of three is finished. }

A\marginnote{7.1.44} monk who has participated in the robe-making ceremony leaves the monastery with an unfinished robe, thinking, “I’ll return.” When he is outside the monastery zone, he thinks, “I’ll make the robe right here. I won’t return.” He then has the robe made. For that monk the robe season ends when the robe is finished. 

A\marginnote{7.1.48} monk who has participated in the robe-making ceremony leaves the monastery with an unfinished robe, thinking, “I’ll return.” When he is outside the monastery zone, he thinks, “I won’t make a robe, and I won’t return.” For that monk the robe season ends when he makes that decision. 

A\marginnote{7.1.51} monk who has participated in the robe-making ceremony leaves the monastery with an unfinished robe, thinking, “I’ll return.” When he is outside the monastery zone, he thinks, “I’ll make the robe right here. I won’t return.” He has the robe made, but it is lost while being made. For that monk the robe season ends when the robe-cloth is lost. 

A\marginnote{7.1.55} monk who has participated in the robe-making ceremony leaves the monastery with an unfinished robe, thinking, “I’ll return.” When he is outside the monastery zone, he has the robe made. When the robe has been made, he hears that they have made an end to the robe season in that monastery. For that monk the robe season ends when he hears about the end of the robe season. 

A\marginnote{7.1.59} monk who has participated in the robe-making ceremony leaves the monastery with an unfinished robe, thinking, “I’ll return.” When he is outside the monastery zone, he has the robe made. When the robe has been made, he still thinks, “I’ll return,” but he remains outside the monastery zone until the end of the robe season. For that monk the robe season ends while he is outside the monastery zone. 

A\marginnote{7.1.63} monk who has participated in the robe-making ceremony leaves the monastery with an unfinished robe, thinking, “I’ll return.” When he is outside the monastery zone, he has the robe made. When the robe has been made, he still thinks, “I’ll return,” and they reach the end of the robe season together. For that monk the robe season ends together with the other monks. 

\scend{The group of six is finished. The group of fifteen on “with” is finished. }

\scend{The section for recitation on “takes” is finished. }

\section*{9. The group of twelve on “not as expected” }

A\marginnote{8.1.1} monk who has participated in the robe-making ceremony leaves the monastery while expecting more robe-cloth. When he is outside the monastery zone, he deals with that expectation. He gets robe-cloth, but not what he had expected. He thinks, “I’ll make the robe right here. I won’t return.” He then has the robe made. For that monk the robe season ends when the robe is finished. 

A\marginnote{8.1.7} monk who has participated in the robe-making ceremony leaves the monastery while expecting more robe-cloth. When he is outside the monastery zone, he deals with that expectation. He gets robe-cloth, but not what he had expected. He thinks, “I won’t make a robe, and I won’t return.” For that monk the robe season ends when he makes that decision. 

A\marginnote{8.1.12} monk who has participated in the robe-making ceremony leaves the monastery while expecting more robe-cloth. When he is outside the monastery zone, he deals with that expectation. He gets robe-cloth, but not what he had expected. He thinks, “I’ll make the robe right here. I won’t return.” He has the robe made, but it is lost while being made. For that monk the robe season ends when the robe-cloth is lost. 

A\marginnote{8.1.18} monk who has participated in the robe-making ceremony leaves the monastery while expecting more robe-cloth. When he is outside the monastery zone, he thinks, “I’ll deal with that expectation right here. I won’t return.” He then deals with that expectation, but it is disappointed. For that monk the robe season ends when the expectation is disappointed. 

A\marginnote{8.2.1} monk who has participated in the robe-making ceremony leaves the monastery while expecting more robe-cloth, thinking, “I won’t return.” When he is outside the monastery zone, he deals with that expectation. He gets robe-cloth, but not what he had expected. He thinks, “I’ll make the robe right here.” He then has the robe made. For that monk the robe season ends when the robe is finished. 

A\marginnote{8.2.7} monk who has participated in the robe-making ceremony leaves the monastery while expecting more robe-cloth, thinking, “I won’t return.” When he is outside the monastery zone, he deals with that expectation. He gets robe-cloth, but not what he had expected. He thinks, “I won’t make a robe.” For that monk the robe season ends when he makes that decision. 

A\marginnote{8.2.12} monk who has participated in the robe-making ceremony leaves the monastery while expecting more robe-cloth, thinking, “I won’t return.” When he is outside the monastery zone, he deals with that expectation. He gets robe-cloth, but not what he had expected. He thinks, “I’ll make the robe right here.” He has the robe made, but it is lost while being made. For that monk the robe season ends when the robe-cloth is lost. 

A\marginnote{8.2.18} monk who has participated in the robe-making ceremony leaves the monastery while expecting more robe-cloth, thinking, “I won’t return.” When he is outside the monastery zone, he thinks, “I’ll deal with that expectation right here.” He then deals with that expectation, but it is disappointed. For that monk the robe season ends when the expectation is disappointed. 

A\marginnote{8.3.1} monk who has participated in the robe-making ceremony leaves the monastery while expecting more robe-cloth. He has not decided whether he will return or not. When he is outside the monastery zone, he deals with that expectation. He gets robe-cloth, but not what he had expected. He thinks, “I’ll make the robe right here. I won’t return.” He then has the robe made. For that monk the robe season ends when the robe is finished. 

A\marginnote{8.3.7} monk who has participated in the robe-making ceremony leaves the monastery while expecting more robe-cloth. He has not decided whether he will return or not. When he is outside the monastery zone, he deals with that expectation. He gets robe-cloth, but not what he had expected. He thinks, “I won’t make a robe, and I won’t return.” For that monk the robe season ends when he makes that decision. 

A\marginnote{8.3.12} monk who has participated in the robe-making ceremony leaves the monastery while expecting more robe-cloth. He has not decided whether he will return or not. When he is outside the monastery zone, he deals with that expectation. He gets robe-cloth, but not what he had expected. He thinks, “I’ll make the robe right here. I won’t return.” He has the robe made, but it is lost while being made. For that monk the robe season ends when the robe-cloth is lost. 

A\marginnote{8.3.18} monk who has participated in the robe-making ceremony leaves the monastery while expecting more robe-cloth. He has not decided whether he will return or not. When he is outside the monastery zone, he thinks, “I’ll deal with that expectation right here. I won’t return.” He then deals with that expectation, but it is disappointed. For that monk the robe season ends when the expectation is disappointed. 

\scend{The group of twelve on “not as expected” is finished. }

\section*{10. The group of twelve on “as expected” }

A\marginnote{9.1.1} monk who has participated in the robe-making ceremony leaves the monastery while expecting more robe-cloth, thinking, “I’ll return.” When he is outside the monastery zone, he deals with that expectation, getting what he had expected. He thinks, “I’ll make the robe right here. I won’t return.” He then has the robe made. For that monk the robe season ends when the robe is finished. 

A\marginnote{9.1.7} monk who has participated in the robe-making ceremony leaves the monastery while expecting more robe-cloth, thinking, “I’ll return.” When he is outside the monastery zone, he deals with that expectation, getting what he had expected. He thinks, “I won’t make a robe, and I won’t return.” For that monk the robe season ends when he makes that decision. 

A\marginnote{9.1.12} monk who has participated in the robe-making ceremony leaves the monastery while expecting more robe-cloth, thinking, “I’ll return.” When he is outside the monastery zone, he deals with that expectation, getting what he had expected. He thinks, “I’ll make the robe right here. I won’t return.” He has the robe made, but it is lost while being made. For that monk the robe season ends when the robe-cloth is lost. 

A\marginnote{9.1.18} monk who has participated in the robe-making ceremony leaves the monastery while expecting more robe-cloth, thinking, “I’ll return.” When he is outside the monastery zone, he thinks, “I’ll deal with that expectation right here. I won’t return.” He then deals with that expectation, but it is disappointed. For that monk the robe season ends when the expectation is disappointed. 

A\marginnote{9.2.1} monk who has participated in the robe-making ceremony leaves the monastery while expecting more robe-cloth, thinking, “I’ll return.” When he is outside the monastery zone, he hears that they have made an end to the robe season in that monastery. He thinks, “Since they have made an end to the robe season in that monastery, I’ll deal with that expectation right here.” He then deals with that expectation, getting what he had expected. He thinks, “I’ll make the robe right here. I won’t return.” He then has the robe made. For that monk the robe season ends when the robe is finished. 

A\marginnote{9.2.8} monk who has participated in the robe-making ceremony leaves the monastery while expecting more robe-cloth, thinking, “I’ll return.” When he is outside the monastery zone, he hears that they have made an end to the robe season in that monastery. He thinks, “Since they have made an end to the robe season in that monastery, I’ll deal with that expectation right here.” He then deals with that expectation, getting what he had expected. He thinks, “I won’t make a robe, and I won’t return.” For that monk the robe season ends when he makes that decision. 

A\marginnote{9.2.14} monk who has participated in the robe-making ceremony leaves the monastery while expecting more robe-cloth, thinking, “I’ll return.” When he is outside the monastery zone, he hears that they have made an end to the robe season in that monastery. He thinks, “Since they have made an end to the robe season in that monastery, I’ll deal with that expectation right here.” He then deals with that expectation, getting what he had expected. He thinks, “I’ll make the robe right here. I won’t return.” He has the robe made, but it is lost while being made. For that monk the robe season ends when the robe-cloth is lost. 

A\marginnote{9.2.21} monk who has participated in the robe-making ceremony leaves the monastery while expecting more robe-cloth, thinking, “I’ll return.” When he is outside the monastery zone, he hears that they have made an end to the robe season in that monastery. He thinks, “Since they have made an end to the robe season in that monastery, I’ll deal with that expectation right here. I won’t return.” He then deals with that expectation, but it is disappointed. For that monk the robe season ends when the expectation is disappointed. 

A\marginnote{9.3.1} monk who has participated in the robe-making ceremony leaves the monastery while expecting more robe-cloth, thinking, “I’ll return.” When he is outside the monastery zone, he deals with that expectation, getting what he had expected. He then has the robe made. When the robe has been made, he hears that they have made an end to the robe season in that monastery. For that monk the robe season ends when he hears about the end of the robe season. 

A\marginnote{9.3.5} monk who has participated in the robe-making ceremony leaves the monastery while expecting more robe-cloth, thinking, “I’ll return.” When he is outside the monastery zone, he thinks, “I’ll deal with that expectation right here. I won’t return.” He then deals with that expectation, but it is disappointed. For that monk the robe season ends when the expectation is disappointed. 

A\marginnote{9.3.9} monk who has participated in the robe-making ceremony leaves the monastery while expecting more robe-cloth, thinking, “I’ll return.” When he is outside the monastery zone, he deals with that expectation, getting what he had expected. He then has the robe made. When the robe has been made, he still thinks, “I’ll return,” but he remains outside the monastery zone until the end of the robe season. For that monk the robe season ends while he is outside the monastery zone. 

A\marginnote{9.3.14} monk who has participated in the robe-making ceremony leaves the monastery while expecting more robe-cloth, thinking, “I’ll return.” When he is outside the monastery zone, he deals with that expectation, getting what he had expected. He then has the robe made. When the robe has been made, he still thinks, “I’ll return,” and they reach the end of the robe season together. For that monk the robe season ends together with the other monks. 

\scend{The group of twelve on “as expected” is finished. }

\section*{11. The group of twelve on business }

A\marginnote{10.1.1} monk who has participated in the robe-making ceremony leaves the monastery on some business. When he is outside the monastery zone, he comes to expect more robe-cloth. He deals with that expectation. He gets robe-cloth, but not what he had expected. He thinks, “I’ll make the robe right here. I won’t return.” He then has the robe made. For that monk the robe season ends when the robe is finished. 

A\marginnote{10.1.7} monk who has participated in the robe-making ceremony leaves the monastery on some business. When he is outside the monastery zone, he comes to expect more robe-cloth. He deals with that expectation. He gets robe-cloth, but not what he had expected. He thinks, “I won’t make a robe, and I won’t return.” For that monk the robe season ends when he makes that decision. 

A\marginnote{10.1.12} monk who has participated in the robe-making ceremony leaves the monastery on some business. When he is outside the monastery zone, he comes to expect more robe-cloth. He deals with that expectation. He gets robe-cloth, but not what he had expected. He thinks, “I’ll make the robe right here. I won’t return.” He has the robe made, but it is lost while being made. For that monk the robe season ends when the robe-cloth is lost. 

A\marginnote{10.1.18} monk who has participated in the robe-making ceremony leaves the monastery on some business. When he is outside the monastery zone, he comes to expect more robe-cloth. He thinks, “I’ll deal with that expectation right here. I won’t return.” He then deals with that expectation, but it is disappointed. For that monk the robe season ends when the expectation is disappointed. 

A\marginnote{10.2.1} monk who has participated in the robe-making ceremony leaves the monastery on some business, thinking, “I won’t return.” When he is outside the monastery zone, he comes to expect more robe-cloth. He deals with that expectation. He gets robe-cloth, but not what he had expected. He thinks, “I’ll make the robe right here.” He then has the robe made. For that monk the robe season ends when the robe is finished. 

A\marginnote{10.2.7} monk who has participated in the robe-making ceremony leaves the monastery on some business, thinking, “I won’t return.” When he is outside the monastery zone, he comes to expect more robe-cloth. He deals with that expectation. He gets robe-cloth, but not what he had expected. He thinks, “I won’t make a robe.” For that monk the robe season ends when he makes that decision. 

A\marginnote{10.2.12} monk who has participated in the robe-making ceremony leaves the monastery on some business, thinking, “I won’t return.” When he is outside the monastery zone, he comes to expect more robe-cloth. He deals with that expectation. He gets robe-cloth, but not what he had expected. He thinks, “I’ll make the robe right here.” He has the robe made, but it is lost while being made. For that monk the robe season ends when the robe-cloth is lost. 

A\marginnote{10.2.18} monk who has participated in the robe-making ceremony leaves the monastery on some business, thinking, “I won’t return.” When he is outside the monastery zone, he comes to expect more robe-cloth. He thinks, “I’ll deal with that expectation right here.” He then deals with that expectation, but it is disappointed. For that monk the robe season ends when the expectation is disappointed. 

A\marginnote{10.3.1} monk who has participated in the robe-making ceremony leaves the monastery on some business. He has not decided whether he will return or not. When he is outside the monastery zone, he comes to expect more robe-cloth. He deals with that expectation. He gets robe-cloth, but not what he had expected. He thinks, “I’ll make the robe right here. I won’t return.” He then has the robe made. For that monk the robe season ends when the robe is finished. 

A\marginnote{10.3.7} monk who has participated in the robe-making ceremony leaves the monastery on some business. He has not decided whether he will return or not. When he is outside the monastery zone, he comes to expect more robe-cloth. He deals with that expectation. He gets robe-cloth, but not what he had expected. He thinks, “I won’t make a robe, and I won’t return.” For that monk the robe season ends when he makes that decision. 

A\marginnote{10.3.12} monk who has participated in the robe-making ceremony leaves the monastery on some business. He has not decided whether he will return or not. When he is outside the monastery zone, he comes to expect more robe-cloth. He deals with that expectation. He gets robe-cloth, but not what he had expected. He thinks, “I’ll make the robe right here. I won’t return.” He has the robe made, but it is lost while being made. For that monk the robe season ends when the robe-cloth is lost. 

A\marginnote{10.3.18} monk who has participated in the robe-making ceremony leaves the monastery on some business. He has not decided whether he will return or not. When he is outside the monastery zone, he comes to expect more robe-cloth. He thinks, “I’ll deal with that expectation right here. I won’t return.” He then deals with that expectation, but it is disappointed. For that monk the robe season ends when the expectation is disappointed. 

\scend{The group of twelve on business is finished. }

\section*{12. The group of nine on “without taking” }

A\marginnote{11.1.1} monk who has participated in the robe-making ceremony leaves the monastery for a different region without taking his share of robe-cloth. When he has gone to that region, the monks there ask him, “Where did you complete the rains residence? Where’s your share of robe-cloth?” He replies, “I completed the rains residence in such-and-such a monastery. That’s where my share of robe-cloth is.” They say, “Go and get that robe-cloth, and we’ll make a robe for you.” He then goes to that monastery and asks the monks, “Where’s my share of the robe-cloth?” They reply, “This is your share. Where are you going?” He says, “I’m going to such-and-such a monastery. The monks there will make me a robe.” They say, “There’s no need to go. We’ll make a robe for you here.” He thinks, “I’ll make the robe right here. I won’t return.”\footnote{Presumably this means he will leave the monastery once the robe is finished and not return before the end of the robe season. } He then has the robe made. For that monk the robe season ends when the robe is finished. 

A\marginnote{11.1.12} monk who has participated in the robe-making ceremony leaves the monastery for a different region … “I won’t make a robe, and I won’t return.” For that monk the robe season ends when he makes that decision. 

A\marginnote{11.1.15} monk who has participated in the robe-making ceremony leaves the monastery for a different region … “I’ll make the robe right here. I won’t return.” He then has the robe made, but it is lost while being made. For that monk the robe season ends when the robe-cloth is lost. 

A\marginnote{11.2.1} monk who has participated in the robe-making ceremony leaves the monastery for a different region without taking his share of robe-cloth. When he has gone to that region, the monks there ask him, “Where did you complete the rains residence? Where’s your share of robe-cloth?” He replies, “I completed the rains residence in such-and-such a monastery. That’s where my share of robe-cloth is.” They say, “Go and get that robe-cloth, and we’ll make a robe for you.” He then goes to that monastery and asks the monks, “Where’s my share of the robe-cloth?” They reply, “This is your share.” He takes that robe-cloth and sets out for the other monastery. While he is on his way, monks ask him, “Where are you going?” He says, “I’m going to such-and-such a monastery. The monks there will make me a robe.” They say, “There’s no need to go. We’ll make a robe for you here.” He thinks, “I’ll make the robe right here. I won’t return.” He then has the robe made. For that monk the robe season ends when the robe is finished. 

A\marginnote{11.2.14} monk who has participated in the robe-making ceremony leaves the monastery for a different region without taking his share of robe-cloth. When he has gone to that region, the monks there ask him, “Where did you complete the rains residence? Where’s your share of robe-cloth?” He replies, “I completed the rains residence in such-and-such a monastery. That’s where my share of robe-cloth is.” They say, “Go and get that robe-cloth, and we’ll make a robe for you.” He then goes to that monastery and asks the monks, “Where’s my share of the robe-cloth?” They reply, “This is your share.” He takes that robe-cloth and sets out for the other monastery. While he is on his way, monks ask him, “Where are you going?” He says, “I’m going to such-and-such a monastery. The monks there will make me a robe.” They say, “There’s no need to go. We’ll make a robe for you here.” He thinks, “I won’t make a robe, and I won’t return.” For that monk the robe season ends when he makes that decision. 

A\marginnote{11.2.26} monk who has participated in the robe-making ceremony leaves the monastery for a different region … “I’ll make the robe right here. I won’t return.” He then has the robe made, but it is lost while being made. For that monk the robe season ends when the robe-cloth is lost. 

A\marginnote{11.3.1} monk who has participated in the robe-making ceremony leaves the monastery for a different region without taking his share of robe-cloth. When he has gone to that region, the monks there ask him, “Where did you complete the rains residence? Where’s your share of robe-cloth?” He replies, “I completed the rains residence in such-and-such a monastery. That’s where my share of robe-cloth is.” They say, “Go and get that robe-cloth, and we’ll make a robe for you.” He then goes to that monastery and asks the monks, “Where’s my share of the robe-cloth?” They reply, “This is your share.” He takes that robe-cloth and returns to the other monastery. When he has arrived, he thinks, “I’ll make the robe right here. I won’t return.” He then has the robe made. For that monk the robe season ends when the robe is finished. 

A\marginnote{11.3.11} monk who has participated in the robe-making ceremony leaves the monastery for a different region … “I won’t make a robe, and I won’t return.” For that monk the robe season ends when he makes that decision. 

A\marginnote{11.3.14} monk who has participated in the robe-making ceremony leaves the monastery for a different region … “I’ll make the robe right here. I won’t return.” He then has the robe made, but it is lost while being made. For that monk the robe season ends when the robe-cloth is lost. 

\scend{The group of nine on “without taking” is finished. }

\section*{13. The group of five on “meditation going well” }

A\marginnote{12.1.1} monk who has participated in the robe-making ceremony and whose meditation is going well takes his robe-cloth and leaves the monastery, thinking,\footnote{“Whose meditation is going well” renders \textit{\textsanskrit{phāsuvihārika}}. \textit{\textsanskrit{Vihāra}} is a common Sutta term for a state of meditation, as in \textit{\textsanskrit{diṭṭhadhammasukhavihāra}}, “a happy (meditation) abiding in this very life”, which is a reference to the four absorptions. \textit{\textsanskrit{Phāsu}} means “comfortable” or “at ease”. Although this may be understood quite broadly, it seems likely that meditation would be the main connotation. } “I’ll go to such-and-such a monastery. If my meditation goes well there, I’ll stay. If not, I’ll go to such-and-such a monastery. If my meditation goes well there, I’ll stay. If not, I’ll go to such-and-such a monastery. If my meditation goes well there, I’ll stay. If not, I’ll return.” When he is outside the monastery zone, he thinks, “I’ll make the robe right here. I won’t return.” He then has the robe made. For that monk the robe season ends when the robe is finished. 

A\marginnote{12.1.9} monk who has participated in the robe-making ceremony and whose meditation is going well takes his robe-cloth and leaves the monastery, thinking, “I’ll go to such-and-such a monastery. If my meditation goes well there, I’ll stay. If not, I’ll go to such-and-such a monastery. If my meditation goes well there, I’ll stay. If not, I’ll go to such-and-such a monastery. If my meditation goes well there, I’ll stay. If not, I’ll return.” When he is outside the monastery zone, he thinks, “I won’t make a robe, and I won’t return.” For that monk the robe season ends when he makes that decision. 

A\marginnote{12.1.16} monk who has participated in the robe-making ceremony and whose meditation is going well takes his robe-cloth and leaves the monastery, thinking, “I’ll go to such-and-such a monastery. If my meditation goes well there, I’ll stay. If not, I’ll go to such-and-such a monastery. If my meditation goes well there, I’ll stay. If not, I’ll go to such-and-such a monastery. If my meditation goes well there, I’ll stay. If not, I’ll return.” When he is outside the monastery zone, he thinks, “I’ll make the robe right here. I won’t return.” He then has the robe made, but it is lost while being made. For that monk the robe season ends when the robe-cloth is lost. 

A\marginnote{12.1.23} monk who has participated in the robe-making ceremony and whose meditation is going well takes his robe-cloth and leaves the monastery, thinking, “I’ll go to such-and-such a monastery. If my meditation goes well there, I’ll stay. If not, I’ll go to such-and-such a monastery. If my meditation goes well there, I’ll stay. If not, I’ll go to such-and-such a monastery. If my meditation goes well there, I’ll stay. If not, I’ll return.” When he is outside the monastery zone, he has a robe made. When the robe has been made, he still thinks, “I’ll return,” but he remains outside the monastery zone until the end of the robe season. For that monk the robe season ends while he is outside the monastery zone. 

A\marginnote{12.1.30} monk who has participated in the robe-making ceremony and whose meditation is going well takes his robe-cloth and leaves the monastery, thinking, “I’ll go to such-and-such a monastery. If my meditation goes well there, I’ll stay. If not, I’ll go to such-and-such a monastery. If my meditation goes well there, I’ll stay. If not, I’ll go to such-and-such a monastery. If my meditation goes well there, I’ll stay. If not, I’ll return.” When he is outside the monastery zone, he has a robe made. When the robe has been made, he still thinks, “I’ll return,” and they reach the end of the robe season together. For that monk the robe season ends together with the other monks. 

\scend{The group of five on “meditation going well” is finished. }

\section*{14. Discussion on obstacles and removal of obstacles }

“Monks,\marginnote{13.1.1} there are two obstacles for the ending of the robe season: the monastery obstacle and the robe obstacle. What’s the monastery obstacle? A monk stays in that monastery or he leaves intending to return. What’s the robe obstacle? A monk hasn’t made a robe, or he hasn’t finished it, or he’s expecting more robe-cloth. 

There\marginnote{13.2.1} are two removals of obstacles for the ending of the robe season: the removal of the monastery obstacle and the removal of the robe obstacle. What’s the removal of the monastery obstacle? A monk leaves that monastery without intending to return. What’s the removal of the robe obstacle? A monk has made a robe; or the robe-cloth is lost, destroyed, or burned; or his expectation of more robe-cloth is disappointed.” 

\scendsutta{The seventh chapter on the robe-making ceremony is finished. }

\scuddanaintro{This is the summary: }

\begin{scuddana}%
“Thirty\marginnote{13.2.12} monks from \textsanskrit{Pāva}, \\
Stayed discontented in \textsanskrit{Sāketa}; \\
Completed the rains, with soaked, \\
Went to see the Victor. 

This\marginnote{13.2.16} is the basis for the robe-making ceremony, \\
And five things are allowable; \\
Without informing, going without taking, \\
Just so eating in a group. 

And\marginnote{13.2.20} as much as you need, the given, \\
Is for those who have participated in the robe-making ceremony; \\
Motion, just thus performed, \\
Just thus not performed. 

Marking,\marginnote{13.2.24} and just washing, \\
And planning, cutting; \\
Tacking, hem, strip of cloth, \\
Strengthening, border lengthwise. 

Border\marginnote{13.2.28} crosswise, patch, \\
Dyeing, indication, hint; \\
Borrowed, stored, to be relinquished, \\
Not marked, apart from those three. 

Apart\marginnote{13.2.32} from five or more, \\
With cut sections with panels; \\
Not apart from an individual, correctly, \\
He appreciates outside the monastery zone. 

The\marginnote{13.2.36} robe-making ceremony is not performed, \\
Thus it was taught by the Buddha; \\
Brand new, nearly new, old, \\
Rag, and from a shop. 

Without\marginnote{13.2.40} indication, without hint, \\
And not borrowed, not stored; \\
Not to be relinquished, marked, \\
And so with the three robes. 

Five\marginnote{13.2.44} or more, \\
Cut sections made with panels; \\
Performed by an individual, correctly, \\
He appreciates inside the monastery zone. 

In\marginnote{13.2.48} this way is the robe-making ceremony performed, \\
Eight key phrases for ending; \\
Departing, finished, \\
And decision, lost. 

Hearing,\marginnote{13.2.52} disappointed expectation, \\
Monastery zone, ending together as the eighth; \\
Takes a finished robe, \\
He goes, thinking, “I won’t return.” 

So,\marginnote{13.2.56} for him the robe season ends, \\
When he departs; \\
He goes taking robe-cloth, \\
Outside the monastery zone he thinks: 

“I’ll\marginnote{13.2.60} make it. I won’t return.” \\
For him the robe season ends when it’s finished; \\
Takes outside the monastery zone, thinking, “Just not, \\
And I won’t return.” 

So,\marginnote{13.2.64} for him the robe season ends, \\
When he decides; \\
He goes taking robe-cloth, \\
Outside the monastery zone he thinks: 

“I’ll\marginnote{13.2.68} make it. I won’t return.” \\
While making it, it is lost; \\
So, for him the robe season ends, \\
When it is lost. 

Taking\marginnote{13.2.72} it, he goes, thinking, “I’ll return”, \\
He has the robe made outside; \\
When his robe is finished, he hears, \\
There the robe season has ended. 

So,\marginnote{13.2.76} for him the robe season ends, \\
When he hears about it; \\
Taking it, he goes, thinking, “I’ll return”, \\
He has the robe made outside. 

When\marginnote{13.2.80} the robe is finished, outside, \\
He remains until the robe season ends; \\
So, for him the robe season ends, \\
When he is outside the monastery zone. 

Taking\marginnote{13.2.84} it, he goes, thinking, “I’ll return”, \\
He has the robe made outside; \\
When the robe is finished, thinking, “I’ll return”, \\
The robe season ends together with. 

So,\marginnote{13.2.88} for him the robe season ends, \\
Together with the monks; \\
And takes, with, \\
Seven with sevenfold outcome. 

There\marginnote{13.2.92} is no ending by departing, \\
The outcome in the unfinished set of six; \\
Takes, outside the monastery zone, \\
“I’ll make”, he produces. 

Finished,\marginnote{13.2.96} and decision, \\
Lost, these three; \\
Taking it, he goes, thinking, “I won’t return”, \\
“I’ll make outside the monastery zone”. 

Finished,\marginnote{13.2.100} also decision, \\
Also lost, these three; \\
Not decided, he does not think, \\
Below is the threefold method. 

Taking\marginnote{13.2.104} it, he goes, thinking, “I’ll return”, \\
When outside the monastery zone, thinking, “I’ll make”; \\
“I’ll not return”, he has it made, \\
The robe season ends when the robe is finished. 

Decision,\marginnote{13.2.108} and lost, \\
Hearing, outside the monastery zone; \\
Together with the monks, \\
Thus is the outcome of the fifteen. 

With,\marginnote{13.2.112} unfinished, \\
Thus again with; \\
These four sections, \\
Are all fifteenfold. 

And\marginnote{13.2.116} not as expected, as expected, \\
And business those three; \\
By this method one should here understand, \\
Three, twelve, twelve. 

Here\marginnote{13.2.120} the nine on without taking, \\
Fivefold on good meditation there; \\
Obstacles, removal of obstacles, \\
The summary is made from this method.” 

%
\end{scuddana}

\scend{In this chapter there are one hundred and eighteen topics by means of groups of twelve with repetition. }

\scendsutta{The chapter on the robe-making ceremony is finished. }

%
\chapter*{{\suttatitleacronym Kd 8}{\suttatitletranslation The chapter on robes }{\suttatitleroot Cīvarakkhandhaka}}
\addcontentsline{toc}{chapter}{\tocacronym{Kd 8} \toctranslation{The chapter on robes } \tocroot{Cīvarakkhandhaka}}
\markboth{The chapter on robes }{Cīvarakkhandhaka}
\extramarks{Kd 8}{Kd 8}

\section*{1. The account of \textsanskrit{Jīvaka} }

At\marginnote{1.1.1} one time the Buddha was staying at \textsanskrit{Rājagaha} in the Bamboo Grove, the squirrel sanctuary. At that time \textsanskrit{Vesālī} was prosperous and crowded with people, and there was plenty of food. There were seven thousand seven hundred and seven stilt houses, and the same number of halls with peaked roofs, parks, and lotus ponds. And there was the courtesan \textsanskrit{Ambapālī} who was attractive and gracious and had the most beautiful complexion. She was skilled in dancing, singing, and instrumental music. She was highly desired, charging fifty coins for a night. Because of her, \textsanskrit{Vesālī} was even more splendid. 

On\marginnote{1.2.1} one occasion the householder association of \textsanskrit{Rājagaha} traveled to \textsanskrit{Vesālī} on business,\footnote{Sp 3.326: \textit{Negamoti \textsanskrit{kuṭumbiyagaṇo}}, “\textit{Negama}: a group of householders.” This definition is missing in DOP. } and they saw all these marvelous qualities of the city, including \textsanskrit{Ambapālī}. When they had concluded their business, they returned to \textsanskrit{Rājagaha}. They then went to King Seniya \textsanskrit{Bimbisāra} of Magadha and told him about everything they had seen, adding, “Sir, please appoint a courtesan.” 

“Well\marginnote{1.2.14} then, find a suitable girl.” 

At\marginnote{1.3.1} that time in \textsanskrit{Rājagaha} there was a girl called \textsanskrit{Sālavatī} who was attractive and gracious and had the most beautiful complexion, and the householder association appointed her as courtesan. Soon afterwards she became skilled in dancing, singing, and instrumental music. She was highly desired, charging a hundred coins for a night. 

Soon\marginnote{1.3.4} enough \textsanskrit{Sālavatī} became pregnant. She thought, “Men don’t like pregnant women. If anyone finds out about this, it will ruin my career. Let me announce that I’m sick.” She told her doorman, “Don’t allow any man to enter. If anyone asks for me, tell them I’m sick.” 

“Yes,\marginnote{1.3.13} ma’am.” 

Her\marginnote{1.4.1} pregnancy progressed, and eventually she gave birth to a son. She told her slave, “Listen, take this boy away in a winnowing basket and throw him on the trash heap.” 

Saying,\marginnote{1.4.4} “Yes, ma’am,” she did just that. 

On\marginnote{1.4.5} the same morning, as Prince Abhaya was walking to an audience with the king, he saw that boy surrounded by crows. He asked his companions, “What’s that surrounded by crows?” 

“It’s\marginnote{1.4.7} a boy, sir.” 

“Is\marginnote{1.4.8} he alive?” 

“Yes,\marginnote{1.4.9} he’s alive.” 

“Well\marginnote{1.4.10} then, take him to our compound and give him to the wet-nurses to feed.”\footnote{“Compound” renders \textit{antepura}. See Appendix of Technical Terms. } 

Saying,\marginnote{1.4.11} “Yes,” they did as requested. 

When\marginnote{1.4.13} they knew that he would live, they gave him the name \textsanskrit{Jīvaka}, “Survivor”. And because a prince brought him up, they also gave him the name \textsanskrit{Komārabhacca}, “Prince-reared”. 

When\marginnote{1.5.1} \textsanskrit{Jīvaka} reached the age of discernment, he went to Prince Abhaya and asked him, “Who, sir, are my mother and father?” 

“I\marginnote{1.5.4} don’t know who your mother is, but I’m your father, because I brought you up.” 

On\marginnote{1.5.7} a later occasion \textsanskrit{Jīvaka} thought, “It’s not easy to make a living in a royal family without a profession. Why don’t I learn a profession?” 

At\marginnote{1.5.10} that time the pre-eminent physician in the world was living at \textsanskrit{Takkasilā}. Then, without asking permission from Prince Abhaya, \textsanskrit{Jīvaka} left for \textsanskrit{Takkasilā}. When he eventually arrived, he went to that physician and said, “Teacher, I wish to learn the profession.” 

“Well\marginnote{1.6.4} then, \textsanskrit{Jīvaka}, please do so.” 

\textsanskrit{Jīvaka}\marginnote{1.6.5} learned much, and he learned quickly; he remembered well and did not forget. After seven years, \textsanskrit{Jīvaka} thought, “I’m a good learner, and I’ve been studying for seven years. And yet there’s no end in sight to learning this profession.” 

He\marginnote{1.7.1} went to that physician and told him what he had thought, adding, “When will I complete the training for this profession?” 

“Listen,\marginnote{1.7.4} \textsanskrit{Jīvaka}. Take a spade and walk as far as 13 kilometers all around \textsanskrit{Takkasilā} and bring back whatever plant you see that’s not medicinal.” 

Saying,\marginnote{1.7.5} “Yes, teacher,” he did just that. But he did not see any plant that was not medicinal. He then went back to the physician and told him what had happened. The physician said, “You’re well-trained, \textsanskrit{Jīvaka}. It’s enough for you to live on.” And he gave \textsanskrit{Jīvaka} a small amount of provisions for the journey. 

\textsanskrit{Jīvaka}\marginnote{1.8.1} left for \textsanskrit{Rājagaha}, but the provisions were exhausted by the time he got to \textsanskrit{Sāketa}. \textsanskrit{Jīvaka} thought, “These roads go through the wilderness where there’s little water and little food. It’s not easy to travel there without provisions. Let me search for provisions.” 

\section*{2. The account of the wealthy merchant’s wife }

At\marginnote{1.8.6.1} that time in \textsanskrit{Sāketa} there was a wealthy merchant whose wife had had a headache for seven years. Many of the most famous physicians in the world had come to see her, but none was able to cure her. And they were very expensive. When \textsanskrit{Jīvaka} arrived at \textsanskrit{Sāketa}, he asked people, “Is there anyone who’s sick who I might treat?” 

“There’s\marginnote{1.8.11} a wealthy merchant whose wife has had a headache for seven years. Go, doctor, and treat her.” 

\textsanskrit{Jīvaka}\marginnote{1.9.1} went to that merchant’s house and told the doorman, “Go and say this to the merchant’s wife, Ma’am, a doctor has arrived. He wishes to see you.’” 

Saying,\marginnote{1.9.4} “Yes, doctor,” he did as asked. 

She\marginnote{1.9.7} replied, “What sort of doctor is it?” 

“A\marginnote{1.9.8} young one.” 

“Forget\marginnote{1.9.9} it. I don’t need a young doctor. Many of the most famous physicians in the world have been here, but none was able to cure me. And they were very expensive too.” 

The\marginnote{1.10.1} doorman then returned to \textsanskrit{Jīvaka} and told him what the merchant’s wife had said. 

\textsanskrit{Jīvaka}\marginnote{1.10.6} replied, “Go and tell her that she doesn’t have to pay anything in advance. When she is cured, she can pay whatever she likes.” 

Saying,\marginnote{1.10.10} “Yes, doctor,” he told the merchant’s wife. 

She\marginnote{1.10.14} said, “Well then, let him in.” 

Saying,\marginnote{1.10.15} “Yes, ma’am,” he went to \textsanskrit{Jīvaka} and told him. 

\textsanskrit{Jīvaka}\marginnote{1.11.1} then approached the merchant’s wife. After examining her, he said to her, “Ma’am, I need a handful of ghee.” She got him a handful of ghee. \textsanskrit{Jīvaka} cooked that ghee with a number of medicines. He then had her lie down on her back on a bed, and he gave her the medicine through the nose. The medicine emerged in her mouth. She then spat it out into a container and told a slave, “Listen, save this ghee in a cotton wad.” 

\textsanskrit{Jīvaka}\marginnote{1.12.1} thought, “It’s astonishing how wretched this housewife is in saving this ghee in a cotton wad, when it should be discarded. Many of my valuable medicines went into it, but she might not give me anything for my services.” 

Seeing\marginnote{1.12.5} his body language, the merchant’s wife asked him what he was concerned about. He told her, and she said, “We householders know the benefit of such frugality. This ghee is good for the slaves and workers, for ointment for the feet, or for using in lamps. Don’t be concerned, doctor, your fee will be abundant.” 

\textsanskrit{Jīvaka}\marginnote{1.13.1} cured the headache of the merchant’s wife with a single treatment through the nose. When she was well, she gave him four thousand coins. When her son and daughter-in-law found out that she was well, they too gave him four thousand coins each, as did her husband. The merchant also gave him a male and a female slave, and a carriage with horses. 

\textsanskrit{Jīvaka}\marginnote{1.13.9} took those sixteen thousand coins, as well as the male and female slaves and the carriage with horses, and he left for \textsanskrit{Rājagaha}. When he eventually arrived, he went to Prince Abhaya and said, “For my first job, sir, I earned sixteen thousand coins, a male and a female slave, and a carriage with horses. Please accept it for bringing me up.” 

“There’s\marginnote{1.13.13} no need, \textsanskrit{Jīvaka}. You should keep it all. But please build a house in our compound.” 

Saying,\marginnote{1.13.16} “Yes,” he did just that. 

\section*{3. The account of King \textsanskrit{Bimbisāra} }

At\marginnote{1.14.1} that time King Seniya \textsanskrit{Bimbisāra} of Magadha had hemorrhoids. His wrap garments were soiled with blood, and the queens made fun of him, “Sir, you’re menstruating; your fertile period has arrived. Soon you’ll give birth.” The king felt humiliated. 

Soon\marginnote{1.14.6} afterwards he told Prince Abhaya what had happened, adding, “Abhaya, please find a doctor to treat me.” 

“Sir,\marginnote{1.14.10} our young doctor \textsanskrit{Jīvaka} is excellent. He’ll treat you.” 

“Well\marginnote{1.14.12} then, Abhaya, send for \textsanskrit{Jīvaka}.” 

Prince\marginnote{1.15.1} Abhaya then sent for \textsanskrit{Jīvaka}. Saying, “Yes, sir,” \textsanskrit{Jīvaka} took some medicine on his nail and went to King \textsanskrit{Bimbisāra}. He said, “Sir, let me see your affliction.” \textsanskrit{Jīvaka} then cured King \textsanskrit{Bimbisāra}’s hemorrhoids with one application of ointment. When the king was well, he had five hundred women adorned with every kind of ornament. He then had the ornaments removed and made into a pile. And he said to \textsanskrit{Jīvaka}, “\textsanskrit{Jīvaka}, these ornaments from five hundred women are all yours.” 

“There’s\marginnote{1.15.8} no need. Please just remember my act of service.” 

“Well\marginnote{1.15.9} then, \textsanskrit{Jīvaka}, please attend on me, the harem, and the Sangha of monks headed by the Buddha.” 

“Yes,\marginnote{1.15.10} sir.” 

\section*{4. The account of the wealthy merchant of \textsanskrit{Rājagaha} }

At\marginnote{1.16.1} that time a wealthy merchant of \textsanskrit{Rājagaha} had had a headache for seven years. Many of the most famous physicians in the world had come to see him, but none was able to cure him. They were very expensive, yet they gave up on him. Some of them said, “The merchant will die in five days.” Others said, “The merchant will die in seven days.” 

The\marginnote{1.16.9} householder association of \textsanskrit{Rājagaha} considered, “This merchant has done much for the king and for this association, and now the doctors have given up on him. But there’s \textsanskrit{Jīvaka}, the king’s excellent young doctor. Let’s ask the king for \textsanskrit{Jīvaka} to treat the merchant.” 

They\marginnote{1.17.1} then went to the king and told him about the merchant, adding, “Sir, please ask doctor \textsanskrit{Jīvaka} to treat the merchant.” 

And\marginnote{1.17.9} that’s what the king did. Saying, “Yes, sir,” \textsanskrit{Jīvaka} went to that merchant, examined him, and said, “If I were to cure you, what fee would you pay me?” 

“I\marginnote{1.17.13} would give you all my wealth, doctor, and I would become your slave.” 

“Are\marginnote{1.18.1} you able to lie on one side for seven months?” 

“I\marginnote{1.18.2} am.” 

“Are\marginnote{1.18.3} you able to lie on the other side for seven months?” 

“I\marginnote{1.18.4} am.” 

“Are\marginnote{1.18.5} you able to lie on your back for seven months?” 

“I\marginnote{1.18.6} am.” 

\textsanskrit{Jīvaka}\marginnote{1.18.7} then had the merchant lie down on a bed. He bound him to the bed, removed some skin from his head, opened a suture in the skull, and removed two insects. He showed them to the crowd, saying, “Sirs, look at these two insects, one small and one large. The doctors who said he would live for five days had seen the large insect. In five days it would have destroyed the merchant’s brain. Because of that he would have died. Those doctors were right. And those doctors who said he would live for seven days had seen the small insect. In seven days it would have destroyed the merchant’s brain. Because of that he would have died. Those doctors were right, too.” He then closed the suture in the skull, sewed the skin back together, and applied an ointment. 

After\marginnote{1.19.1} seven days the merchant said to \textsanskrit{Jīvaka}, “Doctor, I’m unable to lie on one side for seven months.” 

“But\marginnote{1.19.3} didn’t you say you were?” 

“I\marginnote{1.19.5} did, but I’ll die. I’m unable to do it.” 

“Well\marginnote{1.19.6} then, lie on the other side for seven months.” 

After\marginnote{1.19.7} seven days the merchant said to \textsanskrit{Jīvaka}, “Doctor, I’m unable to lie on the other side for seven months.” 

“But\marginnote{1.19.9} didn’t you say you were?” 

“I\marginnote{1.19.11} did, but I’ll die. I’m unable to do it.” 

“Well\marginnote{1.19.12} then, lie on your back for seven months.” 

After\marginnote{1.19.13} seven days the merchant said to \textsanskrit{Jīvaka}, “Doctor, I’m unable to lie on my back for seven months.” 

“But\marginnote{1.19.15} didn’t you say you were?” 

“I\marginnote{1.19.17} did, but I’ll die. I’m unable to do it.” 

“If\marginnote{1.20.1} I hadn’t said this to you, you wouldn’t have been able to lie down for so long. I already knew that you would be well in three times seven days. Get up, you’re cured. But do you remember my fee?” 

“All\marginnote{1.20.5} my wealth is yours, doctor, and I’m your slave.” 

“There’s\marginnote{1.20.6} no need for that. Just give one hundred thousand coins to the king and another one hundred thousand to me.” And being well, he did just that. 

\section*{5. The account of the wealthy merchant’s son }

On\marginnote{1.21.1} one occasion the son of a wealthy merchant in Benares twisted his gut while turning somersaults. Because of that, he was not able to digest congee or food, and he couldn’t urinate or defecate. He became thin, haggard, and pale, with veins protruding all over his body. The merchant considered this and thought, “Why don’t I go to \textsanskrit{Rājagaha} and ask the king for doctor \textsanskrit{Jīvaka} to treat my son?” 

He\marginnote{1.21.7} then traveled to \textsanskrit{Rājagaha}, went to King \textsanskrit{Bimbisāra}, and told him about his son, adding, “Sir, please ask doctor \textsanskrit{Jīvaka} to treat my son.” 

And\marginnote{1.22.1} that’s what the king did. \textsanskrit{Jīvaka} consented and then traveled to Benares where he went to that merchant. He examined his son, dismissed the people there, put up a curtain all around, and tied him to a pillar. He then had his wife stand in front of him, cut open his belly, and pulled out his twisted gut. He showed it to his wife, saying, “See, this is your husband’s affliction. It’s because of this that he’s in such a bad state.” He then untwisted the gut, put it back, sewed his belly back together, and applied ointment. Soon the merchant’s son was healthy again. His father gave sixteen thousand coins to \textsanskrit{Jīvaka}, and \textsanskrit{Jīvaka} returned to \textsanskrit{Rājagaha}. 

\section*{6. The account of King Pajjota }

At\marginnote{1.23.1} that time King Pajjota had jaundice. Many of the most famous physicians in the world had come to see him, but none was able to cure him. And they were very expensive. King Pajjota then sent a message to King \textsanskrit{Bimbisāra}: “Sir, I have such-and-such a disease. Please ask doctor \textsanskrit{Jīvaka} to treat me.” 

King\marginnote{1.23.6} \textsanskrit{Bimbisāra} told \textsanskrit{Jīvaka}, “Go to \textsanskrit{Ujjenī}, \textsanskrit{Jīvaka}, and treat King Pajjota.” \textsanskrit{Jīvaka} consented and traveled to \textsanskrit{Ujjenī}. He then went to King Pajjota, examined him, and said, “Please give me some ghee, sir. I’ll make a medicine from it for you to drink.” 

“Forget\marginnote{1.24.3} it, \textsanskrit{Jīvaka}. Make whatever will cure me that doesn’t contain ghee. I hate ghee; it’s disgusting.” 

\textsanskrit{Jīvaka}\marginnote{1.24.5} thought, “I won’t be able to cure this sickness without ghee. Why don’t I prepare medicine from ghee, but with a bitter color, smell, and taste?” \textsanskrit{Jīvaka} then cooked ghee with a number of medicines, but he made sure it had the color, smell, and taste of a bitter substance. But it occurred to him, “When the king is digesting the ghee after drinking it, it will make him vomit. And because he’s temperamental, he might have me executed. Let me take leave in advance.” He then went to the king and said, “Sir, we doctors need to pull up roots and collect medicines at short notice. Please instruct the stables and the gates: ‘\textsanskrit{Jīvaka} may ride on whatever animal he wishes, depart by whatever gate he desires, and he may come and go as he pleases.’” The king did as \textsanskrit{Jīvaka} had asked. 

At\marginnote{1.25.6} that time King Pajjota had a she-elephant called \textsanskrit{Bhaddavatikā}, which could traverse 650 kilometers in a day. After bringing the ghee to the king and having him drink it, \textsanskrit{Jīvaka} went to the elephant stables and fled the city on \textsanskrit{Bhaddavatikā}. 

Soon\marginnote{1.26.1} afterwards, while King Pajjota was digesting that ghee, he vomited. He said to his men, “That scoundrel \textsanskrit{Jīvaka} made me drink ghee. Find out where he is.” 

“Sir,\marginnote{1.26.5} he’s fled the city on \textsanskrit{Bhaddavatikā}.” 

At\marginnote{1.26.6} that time King Pajjota had a slave called \textsanskrit{Kāka}, whose mother was a spirit and who could traverse 780 kilometers in a day. The king told \textsanskrit{Kāka}, “Go, \textsanskrit{Kāka}, and make \textsanskrit{Jīvaka} turn back. Tell him that the king is asking him to return. But \textsanskrit{Kāka}, these doctors are full of tricks. Don’t receive anything from him.” 

\textsanskrit{Kāka}\marginnote{1.27.1} caught up with \textsanskrit{Jīvaka} at \textsanskrit{Kosambī}, while he was still on his way and having breakfast. \textsanskrit{Kāka} said, “Doctor, the king is asking you to return.” 

“Wait,\marginnote{1.27.4} \textsanskrit{Kāka}, until I’ve finished eating. Why don’t you have something too?” 

“There’s\marginnote{1.27.6} no need. The king told me that doctors are full of tricks and that I shouldn’t receive anything from you.” 

\textsanskrit{Jīvaka}\marginnote{1.27.8} then removed the medicinal part of an emblic myrobalan fruit with his nail, before eating it and drinking water. And he said to \textsanskrit{Kāka}, “Here, \textsanskrit{Kāka}, have some emblic myrobalan and water.” \textsanskrit{Kāka} thought, “The doctor is eating and drinking it. It can’t be anything bad.” So he ate half a fruit and drank the water. When he had eaten it, he vomited right there. He asked \textsanskrit{Jīvaka}, “Will I survive?” 

“Don’t\marginnote{1.28.7} be afraid, \textsanskrit{Kāka}. Both you and the king will be well. But the king is temperamental and might have me executed. Because of that I won’t return.” 

After\marginnote{1.28.9} handing back the she-elephant \textsanskrit{Bhaddavatikā} to \textsanskrit{Kāka}, he continued on to \textsanskrit{Rājagaha}. When he eventually arrived, he went to King \textsanskrit{Bimbisāra} and told him what had happened. The king said, “It’s good, \textsanskrit{Jīvaka}, that you didn’t return. That king is temperamental and might even have had you executed.” 

When\marginnote{1.29.1} King Pajjota was cured, he sent a message to \textsanskrit{Jīvaka}: “Come, \textsanskrit{Jīvaka}, I wish to give you a gift.” 

He\marginnote{1.29.3} replied, “There’s no need, sir. Please just remember my act of service.” 

\section*{7. The account of the two valuable cloths }

Soon\marginnote{1.29.4.1} afterwards King Pajjota obtained two valuable cloths. They were extremely exquisite and fine, one in a million. The king sent them to \textsanskrit{Jīvaka}. \textsanskrit{Jīvaka} thought, “No-one is worthy of these except the Buddha, the Perfected and fully Awakened One, or King \textsanskrit{Bimbisāra}.” 

\section*{8. The account of the thirty purgings }

On\marginnote{1.30.1} one occasion the Buddha’s body was full of impurities. He said to Venerable Ānanda, “Ānanda, my body is full of impurities. I would like to take a purgative.” Ānanda went to \textsanskrit{Jīvaka} and told him what the Buddha had said. And \textsanskrit{Jīvaka} replied, “Well then, Venerable Ānanda, oil the Buddha’s body for a few days.” 

After\marginnote{1.30.9} doing as instructed, Ānanda returned to \textsanskrit{Jīvaka} to let him know, adding, “Please continue the treatment.” 

\textsanskrit{Jīvaka}\marginnote{1.31.1} thought, “It would not be appropriate for me to give the Buddha a powerful purgative.” He then prepared three handfuls of lotus flowers with a variety of medicines, went to the Buddha, and gave him one handful, saying, “Sir, please smell the first handful. This will purge you ten times.” He then brought him the second and the third handful, repeating the instruction, adding, “In this way you’ll have thirty purgings.” After giving the Buddha thirty purgings, he bowed down, circumambulated him with his right side toward him, and left. 

When\marginnote{1.32.1} he was outside the gatehouse, \textsanskrit{Jīvaka} thought, “I’ve given thirty purgings to the Buddha, for his body is full of impurities. But he’ll only be purged twenty-nine times. After being purged, however, he’ll take a bath, which will count as one purging. In this way the Buddha will have had exactly thirty purgings.” 

The\marginnote{1.32.8} Buddha read \textsanskrit{Jīvaka}’s mind. He told Ānanda about it, adding, “Well then, Ānanda, prepare hot water,” and Ānanda did as requested. 

\textsanskrit{Jīvaka}\marginnote{1.33.1} then went back to the Buddha, bowed, sat down, and asked, “Sir, are you purged?” 

“I\marginnote{1.33.4} am, \textsanskrit{Jīvaka}.” 

\textsanskrit{Jīvaka}\marginnote{1.33.5} told him what he had thought outside the gatehouse, adding, “Sir, please bathe.” The Buddha had a hot bath. When he had bathed, the Buddha was purged once, adding up to a total of thirty purgings. \textsanskrit{Jīvaka} said to the Buddha, “Sir, until your body is back to normal, please don’t take any mung-bean broth.” 

\section*{9. The account of asking for a favor }

Soon\marginnote{1.33.18.1} the Buddha’s body was back to normal. \textsanskrit{Jīvaka} then took those two valuable cloths and went to the Buddha. He bowed, sat down, and said, “Sir, I wish to ask for a favor.” 

“Buddhas\marginnote{1.34.4} don’t grant favors, \textsanskrit{Jīvaka}.” 

“It’s\marginnote{1.34.5} allowable and blameless.” 

“Well\marginnote{1.34.6} then, say what it is.” 

“The\marginnote{1.34.7} Buddha and the Sangha of monks are rag-robe wearers. But I’ve received these two valuable cloths from King Pajjota that are extremely exquisite and fine—they are one in a million. Please accept them, and please allow the Sangha of monks to use robe-cloth given by householders.” 

The\marginnote{1.34.12} Buddha received the two valuable cloths. He then instructed, inspired, and gladdened \textsanskrit{Jīvaka} with a teaching, after which \textsanskrit{Jīvaka} got up from his seat, bowed down, circumambulated the Buddha with his right side toward him, and left. Soon afterwards the Buddha gave a teaching and addressed the monks: 

\scrule{“Monks, I allow you to use robe-cloth given by householders. Anyone who wishes may wear rag-robes and anyone who wishes may accept robe-cloth from householders. But I praise contentment with one or the other.” }

The\marginnote{1.35.6} people of \textsanskrit{Rājagaha} heard that the Buddha had allowed the monks to use robe-cloth given by householders. They were excited and joyful, thinking, “Now we’ll give gifts and make merit.” In just a single day many thousands of robes were given at \textsanskrit{Rājagaha}. And the same thing happened in the country. 

At\marginnote{1.36.1} that time, a fleecy robe was offered to the Sangha.\footnote{“Was offered” renders \textit{uppanna}. This word, which literally means “arisen”, varies slightly in meaning dependent on the context. Often it refers to a requisite that has just been given to the Sangha or an individual monastic. Occasionally however, such as here, this does not fit the context. Here we need to assume that the monks had not yet received it, seeing as they ask the Buddha whether or not the robe is allowable. In other words, here \textit{uppanna} happens first, and only then is the robe given. The meaning, then, must be that the monks had been given an offer or a promise of this robe, but had not yet received it. In a sense, the robe had “become available” to them. The most common way for a requisite to become available to a monastic is that an offer is made. I translate accordingly. See also DOP for this meaning of \textit{uppanna}. } They told the Buddha. 

\scrule{“I allow fleecy robes.”\footnote{Sp 3.337: \textit{\textsanskrit{Pāvāroti} salomako \textsanskrit{kappāsādibhedo}}, “\textit{\textsanskrit{Pāvāra}} means cotton, etc., with hair.” Sp-yoj 3.337: \textit{\textsanskrit{Pāvāroti} \textsanskrit{uttarāsaṅgo}}, “\textit{\textsanskrit{Pāvāro}} means upper robe.” } }

And\marginnote{1.36.4} a silken, fleecy robe was offered. 

\scrule{“I allow silken, fleecy robes.” }

And\marginnote{1.36.7} a woolen, fleecy robe was offered. 

\scrule{“I allow woolen, fleecy robes.”\footnote{Vin-\textsanskrit{ālaṅ}-\textsanskrit{ṭ} 34.57: \textit{Kojavanti \textsanskrit{uṇṇāmayo} \textsanskrit{pāvārasadiso}}, “\textit{Kojava} is like a \textit{\textsanskrit{pāvāra}} made of wool.” } }

\scend{The first section for recitation is finished. }

\section*{10. Discussion on the allowance of wool }

On\marginnote{2.1.1} one occasion the king of \textsanskrit{Kāsi} sent a valuable, woolen \textsanskrit{Kāsi} cloth to \textsanskrit{Jīvaka}. \textsanskrit{Jīvaka} took the cloth and went to the Buddha. He bowed, sat down, and told him what had happened, adding, “Sir, please accept this woolen cloth for my long-lasting benefit and happiness.” The Buddha received the woolen cloth. He then instructed, inspired, and gladdened \textsanskrit{Jīvaka} with a teaching, after which \textsanskrit{Jīvaka} got up from his seat, bowed down, circumambulated the Buddha with his right side toward him, and left. Soon afterwards the Buddha gave a teaching and addressed the monks: 

\scrule{“I allow wool.”\footnote{In connection with \href{https://suttacentral.net/pli-tv-bu-vb-np26/en/brahmali\#1.23.1}{Bu Np 26:1.23.1}, which concerns thread used for weaving robes, Sp 1.636 says: \textit{Kambalanti \textsanskrit{eḷakalomasuttaṁ}}, “\textit{Kambala} means a thread of wool.” } }

At\marginnote{3.1.1} that time various kinds of robe-cloth were offered to the Sangha.\footnote{“Robe-cloth” renders \textit{\textsanskrit{cīvara}}. See Appendix of Technical Terms. } The monks thought, “What kind of robe-cloth has and hasn’t the Buddha allowed?” They told the Buddha. 

\scrule{“I allow six kinds of robe-cloth: linen, cotton, silk, wool, sunn hemp, and hemp.”\footnote{“Cotton” renders \textit{\textsanskrit{kappāsika}}, “sunn hemp” \textit{\textsanskrit{sāṇa}}, and “hemp” \textit{\textsanskrit{bhaṅga}}. See Appendix of Plants. } }

Soon\marginnote{3.2.1} afterwards the monks thought, “The Buddha has only allowed one kind of robe, not two,” and being afraid of wrongdoing, they did not accept rags after receiving robe-cloth from householders. 

\scrule{“I allow you to accept rags after receiving robe-cloth from a householder. But I praise contentment with both.” }

\section*{11. Discussion on searching for rags }

On\marginnote{4.1.1} one occasion when a number of monks were traveling through the Kosalan country, some of them entered a charnel ground to look for rags, while the others walked on. The former monks got hold of rags, and the others said, “Please give us a share.” 

“But\marginnote{4.1.7} why didn’t you wait, then? We won’t give you a share.” They told the Buddha. 

\scrule{“If you’re unwilling, you don’t have to give a share to those who don’t wait.” }

On\marginnote{4.2.1} another occasion when a number of monks were traveling through the Kosalan country, some of them entered a charnel ground to look for rags, while the others waited. The former monks got hold of rags, and the others said, “Please give us a share.” 

“But\marginnote{4.2.7} why didn’t you come with us, then? We won’t give you a share.” 

\scrule{“Even if you’re unwilling, you should give a share to those who wait.” }

On\marginnote{4.3.1} yet another occasion when a number of monks were traveling through the Kosalan country, some of them entered a charnel ground to look for rags first, while the other monks entered afterwards. Those who entered first got hold of rags, but not those who entered afterwards. The latter monks said, “Please give us a share.” 

“But\marginnote{4.3.8} why did you come in afterwards, then? We won’t give you a share.” 

\scrule{“If you’re unwilling, you don’t have to give a share to those who enter afterwards.” }

On\marginnote{4.4.1} yet another occasion when a number of monks were traveling through the Kosalan country, they all entered a charnel ground together to look for rags.\footnote{Sp 3.340: \textit{\textsanskrit{Sadisā} \textsanskrit{susānaṁ} \textsanskrit{okkamiṁsūti} sabbe \textsanskrit{samaṁ} \textsanskrit{okkamiṁsu}}, “\textit{\textsanskrit{Sadisā} \textsanskrit{susānaṁ} \textsanskrit{okkamiṁsu}}: they all entered together.” } Some of them got hold of rags, while others did not. The latter monks said, “Please give us a share.” 

“But\marginnote{4.4.7} why didn’t you get any? We won’t give you a share.” 

\scrule{“Even if you’re unwilling, you should give a share to those who enter together with you.” }

On\marginnote{4.5.1} yet another occasion when a number of monks were traveling through the Kosalan country, they entered a charnel ground together to look for rags after making an agreement to share.\footnote{Sp 3.340: \textit{Te \textsanskrit{katikaṁ} \textsanskrit{katvāti} \textsanskrit{laddhaṁ} \textsanskrit{paṁsukūlaṁ} sabbe \textsanskrit{bhājetvā} \textsanskrit{gaṇhissāmāti} bahimeva \textsanskrit{katikaṁ} \textsanskrit{katvā}}, “After making an agreement about it means: having made an agreement outside, as follows: ‘We will get (rags) by distributing the obtained rags to everyone.’” } Some of them got hold of rags, while others did not. The latter monks said, “Please give us a share.” 

“But\marginnote{4.5.7} why didn’t you get any? We won’t give you a share.” 

\scrule{“Even if you’re unwilling, if you have made an agreement about it, you should give a share to those who enter.” }

\section*{12. Discussion on the appointment of a receiver of robe-cloth }

At\marginnote{5.1.1} that time people brought robe-cloth to the monastery, but not finding anyone to receive it, they took it back. As a result, only a little robe-cloth was given at that monastery. They told the Buddha. 

\scrule{“You should appoint a monk who has five qualities as the receiver of robe-cloth: he’s not biased by favoritism, ill will, confusion, or fear, and he knows what has and what hasn’t been received. }

And\marginnote{5.2.1} this is how he should be appointed. First a monk should be asked, and then a competent and capable monk should inform the Sangha: 

‘Please,\marginnote{5.2.4} venerables, I ask the Sangha to listen. If the Sangha is ready, it should appoint monk so-and-so as the receiver of robe-cloth. This is the motion. 

Please,\marginnote{5.2.7} venerables, I ask the Sangha to listen. The Sangha appoints monk so-and-so as the receiver of robe-cloth. Any monk who approves of appointing monk so-and-so as the receiver of robe-cloth should remain silent. Any monk who doesn’t approve should speak up. 

The\marginnote{5.2.11} Sangha has appointed monk so-and-so as the receiver of robe-cloth. The Sangha approves and is therefore silent. I’ll remember it thus.’” 

Soon,\marginnote{6.1.1} after receiving cloth, the receivers of robe-cloth left it right there and went away. The robe-cloth was lost. 

\scrule{“You should appoint a monk who has five qualities as the keeper of robe-cloth: he’s not biased by favoritism, ill will, confusion, or fear, and he knows what is and what isn’t stored. }

And\marginnote{6.2.1} this is how he should be appointed. First a monk should be asked, and then a competent and capable monk should inform the Sangha: 

‘Please,\marginnote{6.2.4} venerables, I ask the Sangha to listen. If the Sangha is ready, it should appoint monk so-and-so as the keeper of robe-cloth. This is the motion. 

Please,\marginnote{6.2.7} venerables, I ask the Sangha to listen. The Sangha appoints monk so-and-so as the keeper of robe-cloth. Any monk who approves of appointing monk so-and-so as the keeper of robe-cloth should remain silent. Any monk who doesn’t approve should speak up. 

The\marginnote{6.2.11} Sangha has appointed monk so-and-so as the keeper of robe-cloth. The Sangha approves and is therefore silent. I’ll remember it thus.’” 

\section*{13. Discussion on the designation of a storeroom, etc. }

Soon\marginnote{7.1.1} afterwards the monk who was the keeper of robe-cloth stored it under a roof cover, at the foot of a tree, and under the eaves of a building. Rats and termites ate it. 

\scrule{“I allow you to designate a dwelling, a stilt house, or a cave as a storeroom.\footnote{Apart from the \textit{\textsanskrit{vihāra}}, “a dwelling”, and the \textit{\textsanskrit{guhā}}, “a cave”, the Pali mentions three kinds of buildings, the \textit{\textsanskrit{aḍḍhayoga}}, the \textit{\textsanskrit{pāsāda}}, and the \textit{hammiya}, all of which, according to the commentaries, are different kinds of \textit{\textsanskrit{pāsāda}}, “stilt houses”. Rather than try to differentiate between these buildings, which is unlikely to be useful from a practical perspective, I have instead grouped them together as “stilt house”. Here is what the commentaries have to say. Sp 4.294: \textit{\textsanskrit{Aḍḍhayogoti} \textsanskrit{supaṇṇavaṅkagehaṁ}}, “An \textit{\textsanskrit{aḍḍhayoga}} is a house bent like a \textit{\textsanskrit{supaṇṇa}}.” Sp-\textsanskrit{ṭ} 4.294 clarifies: \textit{\textsanskrit{Supaṇṇavaṅkagehanti} \textsanskrit{garuḷapakkhasaṇṭhānena} \textsanskrit{katagehaṁ}}, “\textit{\textsanskrit{Supaṇṇavaṅkageha}}: a house made in the shape of the wings of a \textit{\textsanskrit{garuḷa}}.” A \textit{\textsanskrit{garuḷa}}, better known in its Sanskrit form \textit{\textsanskrit{garuḍa}}, is a mythological bird. Sp 4.294 continues: \textit{\textsanskrit{Pāsādoti} \textsanskrit{dīghapāsādo}. Hammiyanti \textsanskrit{upariākāsatale} \textsanskrit{patiṭṭhitakūṭāgāro} \textsanskrit{pāsādoyeva}}, “A \textit{\textsanskrit{pāsāda}} is a long stilt house. A \textit{hammiya} is just a \textit{\textsanskrit{pāsāda}} that has an upper room on top of its flat roof.” At Sp-\textsanskrit{ṭ} 3.74, however, we find slightly different explanations. It seems clear, however, that all three are stilt houses and that they are distinguished according to their shape and the kind of roof they possess. } }

And\marginnote{7.2.1} this is how it should be designated. A competent and capable monk should inform the Sangha: 

‘Please,\marginnote{7.2.3} venerables, I ask the Sangha to listen. If the Sangha is ready, it should designate such-and-such a dwelling as a storeroom. This is the motion. 

Please,\marginnote{7.2.6} venerables, I ask the Sangha to listen. The Sangha designates such-and-such a dwelling as a storeroom. Any monk who approves of designating such-and-such a dwelling as a storeroom should remain silent. Any monk who doesn’t approve should speak up. 

The\marginnote{7.2.10} Sangha has designated such-and-such a dwelling as a storeroom. The Sangha approves and is therefore silent. I’ll remember it thus.’” 

Soon,\marginnote{8.1.1} the robe-cloth in the Sangha’s storeroom was not looked after. 

\scrule{“You should appoint a monk who has five qualities as the storeman: he’s not biased by favoritism, ill will, confusion, or fear, and he knows what is and what isn’t protected.\footnote{Sp 3.343: \textit{\textsanskrit{Guttāguttañca} \textsanskrit{jāneyyāti} ettha yassa \textsanskrit{tāva} \textsanskrit{chadanādīsu} koci doso natthi, \textsanskrit{taṁ} \textsanskrit{guttaṁ}}, “\textit{\textsanskrit{Guttāguttañca} \textsanskrit{jāneyya}}: here, in so far as there is no fault in the roofing, etc., it is protected.” The commentary then goes on to say he should make repairs if the stored goods are unprotected. } }

And\marginnote{8.1.4} this is how he should be appointed. First a monk should be asked, and then a competent and capable monk should inform the Sangha: 

‘Please,\marginnote{8.1.7} venerables, I ask the Sangha to listen. If the Sangha is ready, it should appoint monk so-and-so as the storeman. This is the motion. 

Please,\marginnote{8.1.10} venerables, I ask the Sangha to listen. The Sangha appoints monk so-and-so as the storeman. Any monk who approves of appointing monk so-and-so as the storeman should remain silent. Any monk who doesn’t approve should speak up. 

The\marginnote{8.1.14} Sangha has appointed monk so-and-so as the storeman. The Sangha approves and is therefore silent. I’ll remember it thus.’” 

Soon\marginnote{8.2.1} afterwards the monks from the group of six dismissed the storeman. 

\scrule{“You shouldn’t dismiss the storeman. If you do, you commit an offense of wrong conduct.” }

\subsection*{The distribution of robe-cloth}

At\marginnote{9.1.1} one time there was much robe-cloth in the Sangha’s storeroom. 

\scrule{“The present Sangha should distribute it.” }

Soon\marginnote{9.1.4} afterwards there was a racket as the Sangha was distributing that robe-cloth. 

\scrule{“You should appoint a monk who has five qualities as the distributor of robe-cloth: he’s not biased by favoritism, ill will, confusion, or fear, and he knows what has and what has not been distributed. }

And\marginnote{9.1.8} this is how he should be appointed. First a monk should be asked, and then a competent and capable monk should inform the Sangha: 

‘Please,\marginnote{9.1.11} venerables, I ask the Sangha to listen. If the Sangha is ready, it should appoint monk so-and-so as the distributor of robe-cloth. This is the motion. 

Please,\marginnote{9.1.14} venerables, I ask the Sangha to listen. The Sangha appoints monk so-and-so as the distributor of robe-cloth. Any monk who approves of appointing monk so-and-so as the distributor of robe-cloth should remain silent. Any monk who doesn’t approve should speak up. 

The\marginnote{9.1.18} Sangha has appointed monk so-and-so as the distributor of robe-cloth. The Sangha approves and is therefore silent. I’ll remember it thus.’” 

The\marginnote{9.2.1} monks who were the distributors of robe-cloth thought, “How should we distribute the robe-cloth?” 

\scrule{“You should first sort the cloth, then estimate its value, followed by grouping it according to value, counting the monks, gathering the monks into groups, and finally fixing the shares of robe-cloth.”\footnote{Sp 3.343: \textit{\textsanskrit{Uccinitvāti} “‘\textsanskrit{idaṁ} \textsanskrit{thūlaṁ}, \textsanskrit{idaṁ} \textsanskrit{saṇhaṁ}, \textsanskrit{idaṁ} \textsanskrit{ghanaṁ}, \textsanskrit{idaṁ} \textsanskrit{tanukaṁ}, \textsanskrit{idaṁ} \textsanskrit{paribhuttaṁ}, \textsanskrit{idaṁ} \textsanskrit{aparibhuttaṁ}, \textsanskrit{idaṁ} \textsanskrit{dīghato} \textsanskrit{ettakaṁ} puthulato ettaka’nti \textsanskrit{evaṁ} \textsanskrit{vatthāni} \textsanskrit{vicinitvā}. \textsanskrit{Tulayitvāti} ‘\textsanskrit{idaṁ} \textsanskrit{ettakaṁ} agghati, \textsanskrit{idaṁ} ettakan’ti \textsanskrit{evaṁ} \textsanskrit{agghaparicchedaṁ} \textsanskrit{katvā}. ‘\textsanskrit{Vaṇṇāvaṇṇaṁ} \textsanskrit{katvā}’ti ‘sace \textsanskrit{sabbesaṁ} ekekameva \textsanskrit{dasagghanakaṁ} \textsanskrit{pāpuṇāti}, \textsanskrit{iccetaṁ} \textsanskrit{kusalaṁ}; no ce \textsanskrit{pāpuṇāti}, \textsanskrit{yaṁ} nava \textsanskrit{vā} \textsanskrit{aṭṭha} \textsanskrit{vā} agghati, \textsanskrit{taṁ} \textsanskrit{aññena} ekaagghanakena ca dviagghanakena ca \textsanskrit{saddhiṁ} \textsanskrit{bandhitvā} etena \textsanskrit{upāyena} same \textsanskrit{paṭivīse} \textsanskrit{ṭhapetvā}’ti attho. ‘\textsanskrit{Bhikkhū} \textsanskrit{gaṇetvā} \textsanskrit{vaggaṁ} \textsanskrit{bandhitvā}’ti ‘sace ekekassa \textsanskrit{diyamāne} divaso nappahoti, dasa dasa \textsanskrit{bhikkhū} \textsanskrit{gaṇetvā} dasa dasa \textsanskrit{cīvarapaṭivīse} \textsanskrit{ekavaggaṁ} \textsanskrit{bandhitvā} \textsanskrit{ekaṁ} \textsanskrit{bhaṇḍikaṁ} \textsanskrit{katvā} \textsanskrit{evaṁ} \textsanskrit{cīvarapaṭivīsaṁ} \textsanskrit{ṭhapetuṁ} \textsanskrit{anujānāmī}’”ti attho}, “\textit{\textsanskrit{Uccinitvā}} means having distinguished the cloth in this way: ‘This is coarse, this is soft, this is thick, this is thin, this is used, this is unused; this is its length and this its width.’ \textit{\textsanskrit{Tulayitvā}} means dividing it up according to value in this way: ‘This is worth so much, this so much.’ \textit{\textsanskrit{Vaṇṇāvaṇṇaṁ} \textsanskrit{katvā}}: the meaning is: ‘If each one of them obtains what has a value of ten, it is good. If not, then what has the value of nine or eight should be bound with another (piece) that has the value of one or two, in this way fixing equal shares’. \textit{\textsanskrit{Bhikkhū} \textsanskrit{gaṇetvā} \textsanskrit{vaggaṁ} \textsanskrit{bandhitvā}}: the meaning is: ‘I allow, if one day is not sufficient to give it out to each one separately, to count the monks in groups of ten, then to bind ten shares of robe-cloth for each group, then to make one bundle, and in this way to fix the shares of robe-cloth.’” } }

The\marginnote{9.2.5} monks who were distributors of robe-cloth thought, “What share of the robe-cloth should we give to the novices?” 

\scrule{“I allow you to give half a share to the novices.” }

On\marginnote{9.3.1} one occasion a monk wanted to take his own share and leave.\footnote{\textit{\textsanskrit{Uttaritukāma}}, literally, “desiring to cross over”. Sp 3.343: \textit{\textsanskrit{Uttaritukāmoti} \textsanskrit{nadiṁ} \textsanskrit{vā} \textsanskrit{kantāraṁ} \textsanskrit{vā} \textsanskrit{uttaritukāmo}; \textsanskrit{satthaṁ} \textsanskrit{labhitvā} \textsanskrit{disā} \textsanskrit{pakkamitukāmoti} attho}, “\textit{\textsanskrit{Uttaritukāmo}}: means desiring to cross over a river or a wilderness. The meaning is: ‘Having found a caravan, he desires to leave for the districts.’” } 

\scrule{“You should give a share to one who’s leaving.” }

On\marginnote{9.3.4} another occasion a monk wanted to take an extra share and leave. 

\scrule{“I allow you to give an extra share to anyone who gives something in return.” }

The\marginnote{9.4.1} distributors of robe-cloth thought, “How should we give out the shares of robe-cloth? According to the order in which the monks have arrived or according to seniority?” 

\scrule{“You should satisfy those in need and then give out the remainder by drawing lots.”\footnote{The meaning of this is not clear. \textit{Toseti}, which is the causative formation of \textit{tussati}, means “to please” or “to satisfy” someone. Judging from the usage of this verb elsewhere, it seems to be exclusively used with reference to living beings, that is, the patient of the verb will invariably be a living being. Reading \textit{vikalake} as an accusative plural, the meaning of \textit{vikalake \textsanskrit{tosetvā}} might be construed as follows: “to satisfy those who are short (of robes or cloth)”. The meaning of \textit{\textsanskrit{kusapātaṁ} \textsanskrit{kātuṁ}}, literally, “the \textit{kusa}-grass should be dropped”, is equally unclear. Yet according to how \textit{kusa}-grass is used elsewhere as an aid to sharing out requisites (e.g. at \href{https://suttacentral.net/pli-tv-kd8/en/brahmali\#24.4.4}{Kd 8:24.4.4}), I take it to refer to a randomized method for giving out shares of robe-cloth. I translate accordingly. I have added the phrase “give out the remainder” to clarify the overall meaning. The commentary, however, sees most of this quite differently. Sp 3.343: \textit{Vikalake \textsanskrit{tosetvāti} … \textsanskrit{Cīvaravikalakaṁ} \textsanskrit{nāma} \textsanskrit{sabbesaṁ} \textsanskrit{pañca} \textsanskrit{pañca} \textsanskrit{vatthāni} \textsanskrit{pattāni}, \textsanskrit{sesānipi} atthi, \textsanskrit{ekekaṁ} pana na \textsanskrit{pāpuṇāti}, \textsanskrit{chinditvā} \textsanskrit{dātabbāni}. … \textsanskrit{Chinditvā} dinne pana \textsanskrit{taṁ} \textsanskrit{tositaṁ} hoti, atha \textsanskrit{kusapāto} \textsanskrit{kātabbo}}, “\textit{Vikalake \textsanskrit{tosetvā}}: … \textit{\textsanskrit{Cīvaravikalaka}} means when five cloths are obtained by everyone and there are leftovers, but not sufficient for each one, the cloth should be cut up and then given out. … When it is given out after cutting it up, that means the remainder is shared out. Then the \textit{kusa}-grass should be dropped.” } }

\section*{14. Discussion on the dyeing of robes }

At\marginnote{10.1.1} that time the monks dyed the robes even with dung and beige clay. The robes were discolored. 

\scrule{“I allow you to use six kinds of dye: dye from roots, dye from wood, dye from bark, dye from leaves, dye from flowers, and dye from fruit.” }

The\marginnote{10.2.1} monks dyed the robes in cold water. The robes were smelly. 

\scrule{“I allow a dye-pot to boil the dye.” }

The\marginnote{10.2.5} dye boiled over. 

\scrule{“I allow you to attach a collar.”\footnote{Sp 3.344: \textit{\textsanskrit{Uttarāḷumpanti} \textsanskrit{vaṭṭādhārakaṁ}, \textsanskrit{rajanakumbhiyā} majjhe \textsanskrit{ṭhapetvā} \textsanskrit{taṁ} \textsanskrit{ādhārakaṁ} \textsanskrit{parikkhipitvā} \textsanskrit{rajanaṁ} \textsanskrit{pakkhipituṁ} \textsanskrit{anujānāmīti} attho. \textsanskrit{Evañhi} kate \textsanskrit{rajanaṁ} na uttarati}, “\textit{\textsanskrit{Uttarāḷumpa}}: the meaning is a circular collar; having fixed it on the middle of the dyeing-pot, having made a circle of it, you should add the dye. For when it is done in this way, the dye does not overflow.” Sp-\textsanskrit{ṭ} 3.344 specifies that the collar goes inside the pot, \textit{\textsanskrit{antorajanakumbhiyā}}. Vmv 3.344 adds: \textit{\textsanskrit{Evañhi} kateti \textsanskrit{vaṭṭādhārassa} anto \textsanskrit{rajanodakaṁ}, bahi \textsanskrit{challikañca} \textsanskrit{katvā} viyojane kate. Na \textsanskrit{uttaratīti} \textsanskrit{kevalaṁ} udakato \textsanskrit{pheṇuṭṭhānābhāvā} na uttarati}, “‘For when it is done in this way’ means: after placing the dyeing water inside the circular collar and the (dyeing-)bark on the outside, they are kept separate. ‘It does not overflow’ means: the foam rising completely from the water does not overflow.” } }

The\marginnote{10.2.8} monks did not know whether the dye was ready or not. 

\scrule{“You should put a drop in water or on the back of your nail.” }

To\marginnote{10.3.1} empty the pot, the monks tilted it.\footnote{Reading \textit{\textsanskrit{āvajjanti}} with the PTS edition. } The pot broke. 

\scrule{“I allow a dye-ladle, with or without a handle.” }

The\marginnote{10.3.5} monks did not have a vessel for the dye. 

\scrule{“I allow a basin for dye, a waterpot for dye.”\footnote{Vin-\textsanskrit{ālaṅ}-\textsanskrit{ṭ} 34.57: \textit{Tattha rajanakolambanti \textsanskrit{rajanakuṇḍaṁ}. Tattha \textsanskrit{rajanakuṇḍanti} \textsanskrit{pakkarajanaṭṭhapanakaṁ} \textsanskrit{mahāghaṭaṁ}}, “There the \textit{rajanakolamba} is a \textit{\textsanskrit{rajanakuṇḍa}}. There the \textit{\textsanskrit{rajanakuṇḍa}} is a large waterpot (\textit{\textsanskrit{ghaṭa}}) for the placing of finished dye.” } }

The\marginnote{10.3.8} monks were kneading the robes in basins and bowls. The robes tore. 

\scrule{“I allow a dyeing trough.” }

The\marginnote{11.1.1} monks spread the robes on the ground. The robes became dirty. 

\scrule{“I allow a spread of grass.” }

The\marginnote{11.1.5} grass was eaten by termites. 

\scrule{“I allow a bamboo robe rack and a clothesline.” }

They\marginnote{11.1.8} hung up the robes by the middle. The dye dripped from both sides. 

\scrule{“You should fasten them at the edge.” }

The\marginnote{11.1.12} edge became worn. 

\scrule{“I allow a string for the edge.”\footnote{Sp 1.85: \textit{\textsanskrit{Yaṁ} pana “\textsanskrit{anujānāmi}, bhikkhave, \textsanskrit{kaṇṇasuttaka}”nti \textsanskrit{evaṁ} \textsanskrit{anuññātaṁ}, \textsanskrit{taṁ} \textsanskrit{anuvāte} \textsanskrit{pāsakaṁ} \textsanskrit{katvā} \textsanskrit{bandhitabbaṁ} \textsanskrit{rajanakāle} \textsanskrit{lagganatthāya}}, “But that which is allowed in this way, \textit{\textsanskrit{anujānāmi}, bhikkhave, \textsanskrit{kaṇṇasuttaka}}, having made a loop at the long edge, it is to be bound for the purpose of hanging up at the time of dyeing.” } }

The\marginnote{11.1.15} dye dripped from one edge. 

\scrule{“You should dye them by repeatedly turning them over, and you shouldn’t leave while they’re still dripping.” }

The\marginnote{11.2.1} robes were starchy.\footnote{Sp 3.344: \textit{Patthinnanti \textsanskrit{atirajitattā} \textsanskrit{thaddhaṁ}}, “\textit{Patthinna}: stiff because of too much dye.” } 

\scrule{“You should rinse them in water.” }

The\marginnote{11.2.4} robes were stiff. 

\scrule{“You should beat them with your hands.” }

At\marginnote{11.2.7} that time the monks wore robes consisting of a single piece of cloth, the color of ivory. People complained and criticized them, “They’re just like householders who indulge in worldly pleasures!” They told the Buddha. 

\scrule{“You shouldn’t wear robes consisting of a single piece of cloth. If you do, you commit an offense of wrong conduct.” }

\section*{15. The instruction on robes made of pieces }

After\marginnote{12.1.1} staying at \textsanskrit{Rājagaha} for as long as he liked, the Buddha set out wandering toward the southern hills. He saw the fields of Magadha laid out in rectangles defined by long and short boundaries and their intersections. He said to Venerable Ānanda,\footnote{Sp 3.345: \textit{Acchibaddhanti \textsanskrit{caturassakedārakabaddhaṁ}}, “\textit{Acchibaddha}: a rectangular field with borders.” } “Ānanda, have a look at these fields.” 

“Yes,\marginnote{12.1.4} sir.” 

“Are\marginnote{12.1.5} you able to make this kind of robe for the monks?”\footnote{Sp 3.345: \textit{\textsanskrit{Saṁvidahitunti} \textsanskrit{kātuṁ}}, “\textit{\textsanskrit{Saṁvidahituṁ}} means to make.” } 

“I\marginnote{12.1.6} am.” 

After\marginnote{12.1.7} staying in the southern hills for as long as he liked, the Buddha returned to \textsanskrit{Rājagaha}. Ānanda then made robes for a number of monks. He went to the Buddha and said, “Sir, please have a look at the robes I’ve made.” 

Soon\marginnote{12.2.1} afterwards the Buddha gave a teaching and addressed the monks: 

“Ānanda\marginnote{12.2.2} is clever. He understands the detailed meaning of what I’ve spoken in brief. He can make long strips, short strips, large panels, medium-sized panels, middle sections, intermediate sections, a neckpiece, a calf-piece, and outer sections. In this way the robe will be made of pieces, making it worthless, appropriate for monastics, and undesirable for one’s enemies.\footnote{Vin-vn-\textsanskrit{ṭ} 563: \textit{Kusinti \textsanskrit{āyāmato} ca \textsanskrit{vitthārato} ca \textsanskrit{anuvātaṁ} \textsanskrit{cīvaramajjhe} \textsanskrit{tādisameva} \textsanskrit{dīghapattañca}}, “A \textit{kusi} is a lengthwise or crosswise border in the middle of the robe, just like a long panel.” Vin-vn-\textsanskrit{ṭ} 563: \textit{\textsanskrit{Aḍḍhakusinti} \textsanskrit{anuvātasadisaṁ} \textsanskrit{cīvaramajjhe} tattha tattha \textsanskrit{rassapattaṁ}}, “An \textit{\textsanskrit{aḍḍhakusi}} is a short panel like a border, here and there in the middle of the robe.” Sp 3.245: \textit{\textsanskrit{Maṇḍalanti} \textsanskrit{pañcakhaṇḍikacīvarassa} \textsanskrit{ekekasmiṁ} \textsanskrit{khaṇḍe} \textsanskrit{mahāmaṇḍalaṁ}}, “A \textit{\textsanskrit{maṇḍala}} is the large panel in each section of a robe with five sections.” Commenting on the \textit{\textsanskrit{vivaṭṭa}}, “the middle section”, Vin-vn-\textsanskrit{ṭ} 563 says: \textit{\textsanskrit{Vivaṭṭanti} \textsanskrit{maṇḍalaṁ}, \textsanskrit{aḍḍhamaṇḍalañcāti} dve ekato \textsanskrit{katvā} \textsanskrit{sibbitaṁ} vemajjhe \textsanskrit{khaṇḍaṁ}}, “The \textit{\textsanskrit{vivaṭṭa}} is the section in the middle, which is made by sewing together a large panel (\textit{\textsanskrit{maṇḍala}}) and a medium-sized panel (\textit{\textsanskrit{aḍḍhamaṇḍala}}).” The \textit{\textsanskrit{vivaṭṭa}}, “section in the middle”, is one of usually five main sections of the robe, see below. Sp 3.245: \textit{\textsanskrit{Vivaṭṭanti} \textsanskrit{maṇḍalañca} \textsanskrit{aḍḍhamaṇḍalañca} ekato \textsanskrit{katvā} \textsanskrit{sibbitaṁ} \textsanskrit{majjhimakhaṇḍaṁ}}, “The \textit{\textsanskrit{vivaṭṭa}} is the sewn-together section in the middle, made by making the large panel and the medium-sized panel into one.” Sp 3.245: \textit{\textsanskrit{Anuvivaṭṭanti} tassa ubhosu passesu dve \textsanskrit{khaṇḍāni}}, “The \textit{\textsanskrit{anuvivaṭṭa}}s are the two sections on either side of it.” Vin-vn-\textsanskrit{ṭ} 563 clarifies that “it” refers to the \textit{\textsanskrit{majjhimakhaṇḍa}}, “the middle section”. Sp 3.245: \textit{\textsanskrit{Gīveyyakanti} \textsanskrit{gīvāveṭhanaṭṭhāne} \textsanskrit{daḷhīkaraṇatthaṁ} \textsanskrit{aññaṁ} \textsanskrit{suttasaṁsibbitaṁ} \textsanskrit{āgantukapattaṁ}}, “The \textit{\textsanskrit{gīveyyaka}} is another added panel, sewn on with thread and for the purpose of strengthening, wrapping the neck area.” Sp 3.245: \textit{\textsanskrit{Jaṅgheyyakanti} \textsanskrit{jaṅghapāpuṇanaṭṭhāne} tatheva \textsanskrit{saṁsibbitaṁ} \textsanskrit{pattaṁ}}, “The \textit{\textsanskrit{jaṅgheyyaka}} is a panel sewn on in the same way (as the neckpiece) at the place reaching the calves.” Sp 3.245: \textit{\textsanskrit{Bāhantanti} \textsanskrit{anuvivaṭṭānaṁ} bahi \textsanskrit{ekekaṁ} \textsanskrit{khaṇḍaṁ}}, “The \textit{\textsanskrit{bāhanta}} is the single section on the outside of (each of) the intermediate sections.” Sp 3.245 then adds: \textit{Iti \textsanskrit{pañcakhaṇḍikacīvarenetaṁ} \textsanskrit{vicāritanti}}, “In this way is the layout of a robe with five sections.” } 

\scrule{Your outer robe should be made of pieces and so should your upper robe and sarong.” }

\section*{16. The instruction on the three robes }

After\marginnote{13.1.1} staying at \textsanskrit{Rājagaha} for as long as he liked, the Buddha set out wandering toward \textsanskrit{Vesālī}. On the road between \textsanskrit{Rājagaha} and \textsanskrit{Vesālī} the Buddha saw a number of monks walking along, loaded up with robes on their heads, shoulders, and hips. He thought, “These foolish men have turned to an abundance in robes too readily. Let me set a limit on robes for the monks.” 

Wandering\marginnote{13.2.1} on, the Buddha eventually arrived at \textsanskrit{Vesālī} where he stayed at the Gotamaka Shrine. At that time it was midwinter, when the days are cold and snowy. The Buddha sat outside at night without being cold, wearing only one robe. Becoming cold at the end of the first part of the night, he put on a second robe. Becoming cold once again at the end of the middle part of the night, he put on a third robe. At the end of the last part of the night, when the sky was flaring up at dawn, he became cold once more. Putting on a fourth robe, he was fine. He thought, “Even those on this spiritual path who come from good families, who are sensitive to the cold and fear the cold, are able to get by with three robes. Let me set a limit on robes for the monks. Let me allow them three robes.” 

Soon\marginnote{13.3.3} afterwards the Buddha gave a teaching and addressed the monks. He told them what had happened and what he had thought, adding: 

\scrule{“I allow you three robes: a double-layered outer robe, a single-layered upper robe, and a single-layered sarong.” }

\section*{17. Discussion on extra robes }

When\marginnote{13.6.1} they heard that the Buddha had allowed three robes, the monks from the group of six went to the village in one set of three, stayed in the monastery in another set, and went bathing in yet another set. The monks of few desires complained and criticized them, “How can the monks from the group of six keep extra robes?” They told the Buddha. Soon afterwards the Buddha gave a teaching and addressed the monks: 

\scrule{“You shouldn’t keep extra robes. If you do, you should be dealt with according to the rule.”\footnote{That is, \href{https://suttacentral.net/pli-tv-bu-vb-np1/en/brahmali\#2.17.1}{Bu Np1:2.17.1}. } }

Soon,\marginnote{13.7.1} Venerable Ānanda was offered an extra robe. He wanted to give it to Venerable \textsanskrit{Sāriputta} who was staying at \textsanskrit{Sāketa}. Knowing that the Buddha had laid down a rule against having an extra robe, he thought, “What should I do now?” He told the Buddha, who said, “How long is it, Ānanda, before \textsanskrit{Sāriputta} returns?” 

“Nine\marginnote{13.7.13} or ten days.” 

Soon\marginnote{13.7.14} afterwards the Buddha gave a teaching and addressed the monks: 

\scrule{“You should keep an extra robe for ten days at the most.” }

Soon\marginnote{13.8.1} the monks were given extra robes. Not knowing what to do with them, they told the Buddha. 

\scrule{“I allow you to assign an extra robe to another.”\footnote{For an explanation of the idea of \textit{\textsanskrit{vikappanā}}, see Appendix of Technical Terms. } }

After\marginnote{14.1.1} staying at \textsanskrit{Vesālī} for as long as he liked, the Buddha set out wandering toward Benares. When he eventually arrived, he stayed in the deer park at Isipattana. 

On\marginnote{14.1.4} that occasion the sarong of a certain monk had a hole. He thought, “The Buddha has allowed three robes: a double-layered outer robe, a single-layered upper robe, and a single-layered sarong. Since my sarong has a hole, let me add a patch. It will have a double layer of cloth on all sides, but only a single layer in the middle.”\footnote{The point, presumably, is that there would be a double layer of cloth wherever the patch overlapped with the original robe, but only a single layer over the hole. } And that’s what he did. 

Just\marginnote{14.2.2} then the Buddha was walking about the dwellings, and he saw that monk patching his robe. He went up to that monk and said, “What are you doing, monk?” 

“I’m\marginnote{14.2.4} patching my robe, sir.” 

“Well\marginnote{14.2.5} done. It’s good that you are patching your robe.” 

Soon\marginnote{14.2.7} afterwards the Buddha gave a teaching and addressed the monks: 

\scrule{“When the cloth is new or nearly new, I allow a double-layered outer robe, a single-layered upper robe, and a single-layered sarong. When the cloth is worn, I allow an outer robe of four layers, a double-layered upper robe, and a double-layered sarong. With rags, you may have as much as you like. With scraps of cloth from a shop, you should search for them. And I allow patches, mending, hems, strips of cloth for marking, and strengthening.”\footnote{For a discussion of rendering \textit{\textsanskrit{aggaḷa}} as “patch”, see Appendix of Technical Terms. Sp 3.348: \textit{Suttena \textsanskrit{saṁsibbitaṁ} \textsanskrit{tunnaṁ}}, “\textit{Tunna} is the sewing on (of the patch) with a thread.” Sp 3.348: \textit{\textsanskrit{Vaṭṭetvā} \textsanskrit{karaṇaṁ} \textsanskrit{ovaṭṭikaṁ}}, “Having folded, there is the making of the \textit{\textsanskrit{ovaṭṭika}}.” Sp 3.348: \textit{\textsanskrit{Kaṇḍusakaṁ} vuccati \textsanskrit{muddikā}}, “Calculating is called \textit{\textsanskrit{kaṇḍusaka}}.” Vjb 3.308 adds: \textit{\textsanskrit{Kaṇḍusaṁ} \textsanskrit{nāma} pubbabandhana}, “\textit{\textsanskrit{Kaṇḍusa}} is a prior fixing.” The meaning of this is not clear to me. I follow the definition given in CPD. See also \href{https://suttacentral.net/pli-tv-kd7/en/brahmali\#1.5.9}{Kd 7:1.5.9}. } }

\section*{18. The account of \textsanskrit{Visākhā} }

After\marginnote{15.1.1} staying at Benares for as long as he liked, the Buddha set out wandering toward \textsanskrit{Sāvatthī}. When he eventually arrived, he stayed in the Jeta Grove, \textsanskrit{Anāthapiṇḍika}’s Monastery. 

Soon\marginnote{15.1.4} afterwards \textsanskrit{Visākhā} \textsanskrit{Migāramātā} went to the Buddha, bowed, and sat down. When the Buddha had instructed, inspired, and gladdened her with a teaching, \textsanskrit{Visākhā} said, “Sir, please accept tomorrow’s meal from me together with the Sangha of monks.” The Buddha consented by remaining silent. Knowing that the Buddha had consented, \textsanskrit{Visākhā} got up from her seat, bowed down, circumambulated him with her right side toward him, and left. 

The\marginnote{15.2.1} following morning it was pouring down from a great storm extending over the four continents. The Buddha said to the monks, “It rains on the four continents just as it rains here in the Jeta Grove. Bathe in the rain, monks. This is the last great storm of this kind.” 

Saying,\marginnote{15.2.6} “Yes, sir,” they took off their robes and bathed in the rain. 

When\marginnote{15.3.1} \textsanskrit{Visākhā} had had various kinds of fine foods prepared, she told a slave, “Go to the monastery and tell the Buddha that the meal is ready.” 

Saying,\marginnote{15.3.3} “Yes, ma’am,” she went to the monastery and saw the monks bathing naked in the rain. She thought, “There are no monks in the monastery, just \textsanskrit{Ājīvaka} ascetics bathing in the rain.” She returned and told \textsanskrit{Visākhā} what had happened. Being wise and discerning, \textsanskrit{Visākhā} thought, “No doubt the venerables are bathing naked in the rain. It’s only because of her ignorance that she thinks as she does.” So she sent the slave back to the monastery with the same instructions. 

When\marginnote{15.4.1} the monks had cooled their bodies and felt invigorated, they took their robes and entered their dwellings. When the slave arrived, she didn’t see any monks. She thought, “There are no monks in the monastery. It’s empty.” She returned and told \textsanskrit{Visākhā} what had happened. Once again \textsanskrit{Visākhā} realized what was going on, and she told the slave to go to the monastery one more time. 

When\marginnote{15.5.1} the message had been delivered, the Buddha said to the monks, “Get your bowls and robes. It’s time for the meal.” 

“Yes,\marginnote{15.5.3} sir.” 

The\marginnote{15.5.4} Buddha robed up and took his bowl and robe. Then, just as a strong man might bend or stretch his arm, the Buddha disappeared from the Jeta Grove and appeared at \textsanskrit{Visākhā}’s gatehouse. He sat down on the prepared seat together with the Sangha of monks. 

\textsanskrit{Visākhā}\marginnote{15.6.1} thought, “The power and might of the Buddha are truly amazing. The water is flowing knee-deep, even waist-deep, yet not a single monk has wet feet or wet robes.” Delighted and joyful, she personally served various kinds of fine foods to the Sangha of monks headed by the Buddha. When the Buddha had finished his meal, she sat down to one side and said, “Sir, I wish to ask for eight favors.” 

“Buddhas\marginnote{15.6.6} don’t grant favors, \textsanskrit{Visākhā}.” 

“It’s\marginnote{15.6.7} allowable and blameless.” 

“Well\marginnote{15.6.8} then, say what it is.” 

“For\marginnote{15.7.1} as long as I live I wish to give rainy-season robes to the Sangha, and I wish to give meals to the newly-arrived and departing monastics, as well as to those who are sick and those nursing the sick. I also wish to give medicines to the sick, a regular supply of congee, and bathing robes to the nuns.” 

“But,\marginnote{15.7.2} \textsanskrit{Visākhā}, what reason do you have for asking me for these eight favors?” 

\textsanskrit{Visākhā}\marginnote{15.7.3} then told the Buddha what had happened to her slave, adding, “Nakedness is gross, disgusting, and repulsive. This is why I wish to give rainy-season robes to the Sangha for as long as I live. 

Also,\marginnote{15.8.1} not knowing the streets or where to go, newly-arrived monks will get exhausted while walking for alms. But if they eat a meal from me, they will get to know the streets and where to go for alms, and they will avoid getting exhausted. This is why I wish to give meals to the newly-arrived monks for as long as I live. 

Also,\marginnote{15.8.4} in trying to get a meal, departing monks may bother the Teacher, or they may arrive late at their destination. Or, if they fail to get a meal, they’ll be weak while traveling. But if they eat a meal from me, they won’t bother the Teacher, they’ll arrive at their destination at an appropriate time, and they won’t be weak while traveling. This is why I wish to give meals to the departing monks for as long as I live. 

Also,\marginnote{15.9.1} if sick monks don’t get suitable food, their illness might get worse, or they might die. But if they eat a meal from me, their illness won’t get worse, and they won’t die. This is why I wish to give meals to the sick monks for as long as I live. 

Also,\marginnote{15.9.4} if the monks who nurse the sick have to get their own meals, they won’t be able to bring back meals for the sick until after midday, and then the sick won’t be able to eat. But if they eat a meal from me, they’ll be able to bring back meals for the sick in good time, and the sick will be able to eat. This is why I wish to give meals to those monks who nurse the sick for as long as I live. 

Also,\marginnote{15.10.1} if the sick monks don’t get suitable medicines, their illness might get worse or they might die. But if they get medicine from me, their illness won’t get worse, and they won’t die. This is why I wish to give medicine to the Sangha for as long as I live. 

Also,\marginnote{15.10.4} while staying at Andhakavinda, the Buddha allowed congee, seeing ten benefits in it. This is why I wish to give a regular supply of congee to the Sangha for as long as I live. 

Also,\marginnote{15.11.1} sir, on one occasion the nuns were bathing naked at a ford in the river \textsanskrit{Aciravatī} together with sex workers. The sex workers teased them, ‘Venerables, why practice the spiritual life when you’re still young? Why not enjoy worldly pleasures? You can practice the spiritual life when you’re old. In this way you’ll get the benefits of both.’ The nuns felt humiliated. Nakedness in women is gross, disgusting, and repulsive. This is why I wish to give bathing robes to the Sangha of nuns for as long as I live.”\footnote{See also \href{https://suttacentral.net/Bi Pc 21/en/brahmali\#1.16.1}{Bi Pc 21}, which prohibits a nun from bathing naked. } 

“But,\marginnote{15.12.1} \textsanskrit{Visākhā}, what benefit do you see that you ask me for these eight favors?” 

“Well,\marginnote{15.12.2} it will happen that monks who have completed the rainy-season residence in the various regions will come to \textsanskrit{Sāvatthī} to visit the Buddha. If a monk has died, they’ll ask you about his destination, and you’ll tell them whether he’s reached the fruit of stream-entry, the fruit of once-returning, the fruit of non-returning, or perfection. I’ll then ask those monks whether that dead monk had previously visited \textsanskrit{Sāvatthī}. If they say he had, I may conclude, ‘No doubt that venerable will have enjoyed a rainy-season robe supplied by me. Or he will have enjoyed a meal for newly-arrived monks, a meal for departing monks, a meal for sick monks, a meal for those nursing the sick, medicines, or a regular supply of congee—all given by me.’ When I recall that, I’ll be glad. The gladness will give rise to joy, and the mental joy will make me tranquil. When I’m tranquil, I’ll feel bliss. And when I’m blissful, my mind will be stilled. In this way I’ll develop the spiritual faculties, the spiritual powers, and the factors of awakening. It’s because of this benefit that I ask for these eight favors.” 

“Well\marginnote{15.14.1} said, \textsanskrit{Visākhā}. It’s good that you ask me for these eight favors for the sake of this benefit. I grant you these eight favors.” The Buddha then expressed his appreciation to \textsanskrit{Visākhā} with these verses: 

\begin{verse}%
“Rejoicing\marginnote{15.14.5} in giving food and drink, \\
A virtuous disciple of the Accomplished One, \\
Overcoming stinginess, gives a gift. \\
It leads to heaven, eliminates sadness, and brings bliss. 

By\marginnote{15.14.9} means of the stainless path, \\
She obtains heaven and long life. \\
Desiring merit, happy and healthy, \\
She rejoices in heaven for a long time.” 

%
\end{verse}

When\marginnote{15.14.13} the Buddha had expressed his appreciation, he got up from his seat and left. Soon afterwards the Buddha gave a teaching and addressed the monks: 

\scrule{“I allow a rainy-season robe, meals for newly-arrived monastics, meals for departing monastics, meals for the sick, meals for those nursing the sick, medicine for the sick, a regular supply of congee, and bathing robes for the Sangha of nuns.” }

\scend{The section for recitation on \textsanskrit{Visākhā} is finished. }

\section*{19. The allowance of a sitting mat }

At\marginnote{16.1.1} one time the monks ate fine food, and then went to sleep absentminded and heedless. They emitted semen while dreaming, soiling the furniture.\footnote{“Furniture” renders \textit{\textsanskrit{senāsana}}. See Appendix of Technical Terms. } 

Soon\marginnote{16.1.3} afterwards the Buddha was walking about the dwellings with Venerable Ānanda as his attendant. Noticing that soiled furniture, he asked Ānanda what had happened. Ānanda told him, and the Buddha said, “That’s how it is, Ānanda. For those who go to sleep absentminded and heedless, semen is emitted while dreaming. But for those who fall asleep mindful and heedful, this doesn’t happen, nor does it for ordinary people who are free from sensual desire. And, Ānanda, it’s impossible for a perfected one to emit semen.” 

Soon\marginnote{16.2.6} afterwards the Buddha gave a teaching and addressed the monks, telling them what had happened. He then said: 

“There\marginnote{16.3.1} are these five drawbacks to going to sleep absentminded and heedless:\footnote{This is parallel to \href{https://suttacentral.net/an5.210/en/brahmali}{AN 5.210}. } you don’t sleep well; you wake up feeling miserable; you have nightmares; the gods don’t guard you; and you emit semen. 

And\marginnote{16.3.4} there are these five benefits in going to sleep mindful and heedful: you sleep well; you wake up feeling good; you don’t have nightmares; the gods guard you; and you don’t emit semen. 

\scrule{And, monks, I allow a sitting mat to protect the body, the robes, and the furniture.”\footnote{“Sitting mat” renders \textit{\textsanskrit{nisīdana}}. See Appendix of Technical Terms. } }

But\marginnote{16.4.1} the sitting mat was too small. It did not protect the entire piece of furniture. 

\scrule{“I allow you to make a sheet as large as you like.” }

On\marginnote{17.1.1} one occasion Venerable Ānanda’s preceptor, Venerable \textsanskrit{Belaṭṭhasīsa}, had carbuncles, with pus causing his robes to stick to his body. To detach them, the monks kept on moistening his robes with water. As the Buddha was walking about the dwellings, he noticed the monks doing this. He went up to them and said, “What illness does this monk have?” 

“He\marginnote{17.1.6} has carbuncles, sir. That’s why we’re doing this.” Soon afterwards the Buddha gave a teaching and addressed the monks: 

\scrule{“For anyone who has an itch, a boil, a running sore, or a carbuncle, I allow an itch-covering cloth.” }

On\marginnote{18.1.1} one occasion \textsanskrit{Visākhā} took a washcloth and went to the Buddha. She bowed, sat down, and said, “Sir, please accept this washcloth for my long-lasting benefit and happiness.” The Buddha accepted it and then instructed, inspired, and gladdened her with a teaching. She got up from her seat, bowed down, circumambulated him with her right side toward him, and left. Soon afterwards the Buddha gave a teaching and addressed the monks: 

\scrule{“I allow washcloths.” }

At\marginnote{19.1.1} that time Venerable Ānanda had a friend called Roja the Mallian. Roja had left an old linen cloth with Ānanda, and it so happened that Ānanda needed such a cloth. They told the Buddha. 

\scrule{“I allow you to take things on trust from someone who has five qualities: they’re a friend, they’re a close companion, they’ve spoken about it, they’re alive, and you know they’ll be pleased if you take it.” }

At\marginnote{20.1.1} that time the monks had enough robes, but they needed water filters and bags. 

\scrule{“I allow cloth for requisites.” }

\section*{20. Discussion of the smallest robe-cloth that can be assigned to another, etc. }

Soon\marginnote{20.2.1} afterwards the monks thought, “These things that have been allowed by the Buddha—the three robes, the rainy-season robe, the sitting mat, the sheet, the itch-covering cloth, the washcloth, and the cloth for requisites—are they all to be determined or to be assigned to another?” They told the Buddha. 

\scrule{“You should determine the three robes, not assign them to another; you should determine the rainy-season robe for the four months of the rainy season, and apart from that assign it to another; you should determine the sitting mat, not assign it to another; you should determine a sheet, not assign it to another; you should determine an itch-covering cloth for as long as you’re sick, and apart from that assign it to another; you should determine a washcloth, not assign it to another; you should determine a cloth for requisites, not assign it to another.” }

The\marginnote{21.1.1} monks thought, “What’s the size of the smallest robe-cloth that can be assigned to another?” 

\scrule{“The smallest robe-cloth you should assign to another is one that’s eight standard fingerbreadths long and four wide.” }

At\marginnote{21.1.5} that time Venerable \textsanskrit{Mahākassapa}’s rag robes were heavy. 

\scrule{“I allow you to mend roughly with thread.”\footnote{Sp 3.359: \textit{\textsanskrit{Suttalūkhaṁ} \textsanskrit{kātunti} sutteneva \textsanskrit{aggaḷaṁ} \textsanskrit{kātunti} attho}, “\textit{\textsanskrit{Suttalūkhaṁ} \textsanskrit{kātuṁ}} means to patch just using thread.” } }

The\marginnote{21.1.8} corners became deformed.\footnote{Sp 3.359: \textit{\textsanskrit{Vikaṇṇo} \textsanskrit{hotīti} \textsanskrit{suttaṁ} \textsanskrit{acchetvā} \textsanskrit{acchetvā} \textsanskrit{sibbantānaṁ} eko \textsanskrit{saṅghāṭikoṇo} \textsanskrit{dīgho} hoti}, “\textit{\textsanskrit{Vikaṇṇo} hoti}: when those who are sewing repeatedly pull (\textit{\textsanskrit{añchitvā} \textsanskrit{añchitvā}}) the thread, one corner of the outer robe becomes long.” Here, instead of reading \textit{\textsanskrit{acchetvā} \textsanskrit{acchetvā}} (“having repeatedly left uncut”), which does not fit the context well, I read \textit{\textsanskrit{añchitvā} \textsanskrit{añchitvā}} with Sp-yoj 3.359. } 

\scrule{“I allow you to remove the deformity.”\footnote{Sp 3.359: \textit{\textsanskrit{Vikaṇṇaṁ} uddharitunti \textsanskrit{dīghakoṇaṁ} \textsanskrit{chindituṁ}}, “\textit{\textsanskrit{Vikaṇṇaṁ} \textsanskrit{uddharituṁ}} means to cut off the long corner.” } }

The\marginnote{21.1.11} cloth frayed. 

\scrule{“I allow you to add a lengthwise border and a crosswise border.”\footnote{\textit{\textsanskrit{Anuvāta}} and \textit{\textsanskrit{paribhaṇḍa}}, refer to long and short borders respectively. This is what the commentaries have to say. Sp 3.308: \textit{\textsanskrit{Anuvātakaraṇamattenāti} \textsanskrit{piṭṭhianuvātāropanamattena}}, “\textit{\textsanskrit{Anuvātakaraṇamattena}} means merely by mounting a border at the back.” Which is further explained at Sp-\textsanskrit{ṭ} 3.308: \textit{\textsanskrit{Piṭṭhianuvātāropanamattenāti} \textsanskrit{dīghato} \textsanskrit{anuvātassa} \textsanskrit{āropanamattena}}, “\textit{\textsanskrit{Piṭṭhianuvātāropanamattena}} means merely by mounting a border lengthwise.” Sp 3.308: \textit{\textsanskrit{Paribhaṇḍakaraṇamattenāti} \textsanskrit{kucchianauvātāropanamattena}}, “\textit{\textsanskrit{Paribhaṇḍakaraṇamattena}} means merely by mounting a border at the belly.” Which is further explained at Sp-\textsanskrit{ṭ} 3.308: \textit{\textsanskrit{Kucchianuvātāropanamattenāti} puthulato \textsanskrit{anuvātassa} \textsanskrit{āropanamattena}}, “\textit{\textsanskrit{Kucchianuvātāropanamattena}} means merely by adding a border crosswise.” } }

On\marginnote{21.1.14} one occasion the panels of an outer robe were breaking up. 

\scrule{“I allow you to darn.”\footnote{Sp 3.359: \textit{\textsanskrit{Aṭṭhapadakaṁ} \textsanskrit{kātunti} \textsanskrit{aṭṭhapadakacchannena} \textsanskrit{pattamukhaṁ} \textsanskrit{sibbituṁ}}, “\textit{\textsanskrit{Aṭṭhapadakaṁ} \textsanskrit{kātuṁ}} means to sew the opening in the panel with a network-covering.” CPD suggests “network” for \textit{\textsanskrit{aṭṭhapadaka}}. } }

At\marginnote{21.2.1} one time, while making a set of three robes for a monk, there was not enough cloth to make all three out of pieces. 

\scrule{“I allow two robes made of pieces and one that isn’t.” }

There\marginnote{21.2.4} was not enough cloth to make two out of pieces. 

\scrule{“I allow one robe made of pieces and two that are not.” }

There\marginnote{21.2.7} was not enough cloth to make one out of pieces. 

\scrule{“I allow you to add a further supply.\footnote{Sp 3.360: \textit{\textsanskrit{Anvādhikampi} \textsanskrit{āropetunti} \textsanskrit{āgantukapattampi} \textsanskrit{dātuṁ}}, “\textit{\textsanskrit{Anvādhikampi} \textsanskrit{āropetuṁ}}: to give an extra panel.” The exact significance of this is unclear. } But you shouldn’t wear robes none of which are cut into pieces. If you do, you commit an offense of wrong conduct.” }

On\marginnote{22.1.1} one occasion a monk who had been given much robe-cloth wanted to give it to his mother and father. 

\scrule{“If you’re giving to your mother and father, what can I say? I allow you to give to your mother and father. But, monks, a gift of faith shouldn’t be ruined.\footnote{Sp 3.361: \textit{Ettha \textsanskrit{sesañātīnaṁ} dento \textsanskrit{vinipātetiyeva}}, “In this context it goes to ruin when given to other relatives.” Presumably this means one should not give to anyone who is not a monastic, apart from one’s parents. } If you do, you commit an offense of wrong conduct.” }

On\marginnote{23.1.1} one occasion a monk left one of his robes in the Blind Men’s Grove and then entered the village for alms in just his sarong and upper robe. Thieves stole that robe, and as a result he became poorly dressed. When other monks asked him why, he told them what had happened. 

\scrule{“You shouldn’t enter a village in just your sarong and upper robe. If you do, you commit an offense of wrong conduct.” }

Soon\marginnote{23.2.1} afterwards Venerable Ānanda, being absentminded, entered a village in just his sarong and upper robe. The monks said to him, “Hasn’t the Buddha laid down a rule against entering the village in just a sarong and an upper robe?” Ānanda told them what had happened. They told the Buddha. 

\scrule{“There are five reasons for leaving behind the outer robe, the upper robe, or the sarong:\footnote{This passage suggests that each of the three robes could be used as a substitute for the remaining two, which in turn suggests they were similar to each other. } you’re sick; it’s the rainy season; you’re going to cross a river; the dwelling is protected by a door; you have participated in the robe-making ceremony.\footnote{Sp 3.362: \textit{\textsanskrit{Vassikasaṅketanti} vassike \textsanskrit{cattāro} \textsanskrit{māse}}, “\textit{\textsanskrit{Vassikasaṅketa}} means the four months of the rainy season.” “Door” renders \textit{\textsanskrit{aggaḷa}}. For a discussion of this word, see Appendix of Technical Terms. } }

\scrule{There are five reasons for leaving behind the rainy-season robe: you’re sick; you’re going outside the monastery zone; you’re going to cross a river; the dwelling is protected by a door; the rainy-season robe hasn’t been sewn or is unfinished.” }

\section*{21. Discussion of robe-cloth given to the Sangha }

At\marginnote{24.1.1} that time a certain monk had spent the rainy season by himself. People gave him robe-cloth, intending it for the Sangha. He thought, “The Buddha has laid down that a sangha consists of a group of at least four, but I’m here by myself. Now these people have given robe-cloth, intending it for the Sangha. Let me take it to \textsanskrit{Sāvatthī}.” He then took that robe-cloth to \textsanskrit{Sāvatthī} and told the Buddha what had happened. The Buddha said, “This robe-cloth is yours until the end of the robe season. 

\scrule{It may be that a monk spends the rainy season by himself, yet people give him robe-cloth, intending it for the Sangha. That robe-cloth is his until the end of the robe season.” }

At\marginnote{24.3.1} one time a certain monk was living by himself outside the rainy season. People gave him robe-cloth, intending it for the Sangha. He thought, “The Buddha has laid down that a sangha consists of a group of at least four, but I’m here by myself. Now these people have given robe-cloth, intending it for the Sangha. Let me take it to \textsanskrit{Sāvatthī}.” He then took that robe-cloth to \textsanskrit{Sāvatthī} and told the monks, who in turn told the Buddha. He said, “The present Sangha should distribute it. 

\scrule{It may be that a monk is living by himself outside the rainy season, yet people give him robe-cloth, intending it for the Sangha. I allow that monk to determine that robe-cloth as his. But if another monk arrives before he has determined that robe-cloth, then he should be given an equal share. If yet another monk arrives before they’ve distributed that robe-cloth by drawing lots, he too should be given an equal share. If still another monk arrives, but after they’ve distributed that robe-cloth by drawing lots, they don’t need to give him a share if they’re unwilling.” }

On\marginnote{24.5.1} one occasion, after completing the rainy-season residence at \textsanskrit{Sāvatthī}, two senior monks who were brothers, Venerable \textsanskrit{Isidāsa} and Venerable \textsanskrit{Isibhaṭa}, went to a certain village monastery. Because it was a long time since they had been there, people gave meals together with robe-cloth. The resident monks asked them, “Venerables, this robe-cloth belonging to the Sangha was given because of you. Will you accept a share?” They replied, “As we understand the teaching of the Buddha, this robe-cloth is just for you until the end of the robe season.” 

At\marginnote{24.6.1} that time there were three monks who were spending the rains residence at \textsanskrit{Rājagaha}. People gave them robe-cloth, intending it for the Sangha. The monks considered, “The Buddha has laid down that a sangha consists of a group of at least four, but we’re just three. Now these people have given robe-cloth, intending it for the Sangha. What should we do?” 

On\marginnote{24.6.6} that occasion there were a number of senior monks—Venerable \textsanskrit{Nilavāsī}, Venerable \textsanskrit{Sāṇavāsī}, Venerable Gotaka, Venerable Bhagu, and Venerable \textsanskrit{Phaḷikasantāna}—staying at \textsanskrit{Pāṭaliputta} in the \textsanskrit{Kukkuṭa} Monastery. The monks from \textsanskrit{Rājagaha} went to \textsanskrit{Pāṭaliputta} to ask them. They replied, “As we understand the teaching of the Buddha, that robe-cloth is just for you until the end of the robe season.” 

\section*{22. Upananda the Sakyan }

At\marginnote{25.1.1} one time, after completing the rainy-season residence at \textsanskrit{Sāvatthī}, Venerable Upananda the Sakyan went to a certain village monastery. Just then the monks there had gathered to distribute the robe-cloth. They said to Upananda, “We’re distributing the Sangha’s robe-cloth. Would you like a share?” 

“Yes,\marginnote{25.1.4} I would.” 

He\marginnote{25.1.5} then took that share of robe-cloth and went to another monastery. There, too, the monks had gathered to distribute the robe-cloth. They said to Upananda, “We’re distributing the Sangha’s robe-cloth. Would you like a share?” 

“Yes,\marginnote{25.1.8} I would.” 

He\marginnote{25.1.9} then took that share, too, and went to yet another monastery. There, too, the monks had gathered to distribute the robe-cloth. They said to Upananda, “We’re distributing the Sangha’s robe-cloth. Would you like a share?” 

“Yes,\marginnote{25.1.12} I would.” 

He\marginnote{25.1.13} then took that share too, made a large bundle of robe-cloth, and returned to \textsanskrit{Sāvatthī}. 

The\marginnote{25.2.1} monks said to him, “You have much merit, Upananda, as you’ve been given so much robe-cloth.” 

“This\marginnote{25.2.2} has nothing to do with merit.” And he told them how he had obtained so much robe-cloth. 

“So\marginnote{25.3.1} you spent the rains residence in one place and accepted a share of the robe-cloth from somewhere else?” 

“Yes.”\marginnote{25.3.2} 

The\marginnote{25.3.3} monks of few desires complained and criticized Upananda, “How could Venerable Upananda spend the rains residence in one place and then accept a share of the robe-cloth from somewhere else?” They told the Buddha. … “Is it true, Upananda, that you did this?” 

“It’s\marginnote{25.3.7} true, sir.” 

The\marginnote{25.3.8} Buddha rebuked him … “Foolish man, how could you spend the rains residence in one place and then accept a share of the robe-cloth from somewhere else? This will affect people’s confidence …” After rebuking him … he gave a teaching and addressed the monks: 

\scrule{“You shouldn’t spend the rainy-season residence in one place and then accept a share of the robe-cloth from somewhere else. If you do, you commit an offense of wrong conduct.” }

At\marginnote{25.4.1} one time Venerable Upananda spent the rains residence in two separate monasteries, thinking, “In this way I’ll get much robe-cloth.” The monks thought, “What share of the robe-cloth should be given to Venerable Upananda?” They told the Buddha. “Give the foolish man one person’s share. 

\scrule{It may be that a monk spends the rains residence in two monasteries with the intention of getting much robe-cloth. If he spends half the time in each monastery, he should be given half a share of the robe-cloth in each. Otherwise he should be given his share of the robe-cloth in the monastery where he spends the most time.” }

\section*{23. The account of the one who was sick }

On\marginnote{26.1.1} one occasion there was a monk who had dysentery and was lying in his own feces and urine. Just then, as the Buddha was walking about the dwellings with Venerable Ānanda as his attendant, he came to the dwelling of this monk. When he saw his condition, he went up to him and said, “What’s your illness, monk?” 

“I\marginnote{26.1.6} have dysentery, sir.” 

“But\marginnote{26.1.7} don’t you have a nurse?” 

“No.”\marginnote{26.1.8} 

“Why\marginnote{26.1.9} don’t the monks nurse you?” 

“Because\marginnote{26.1.10} I don’t do anything for them.” 

The\marginnote{26.2.1} Buddha said to Ānanda, “Go and get some water, Ānanda. Let’s give him a wash.” 

Saying,\marginnote{26.2.3} “Yes, sir,” he did just that. And so the Buddha poured the water, while Ānanda cleaned him up. Then, the Buddha lifting him by the head and Ānanda by the feet, they lay him on a bed. 

Soon\marginnote{26.3.1} afterwards the Buddha had the Sangha gathered and questioned the monks: “Is there a sick monk in that dwelling?” 

“Yes,\marginnote{26.3.3} sir.” 

“What’s\marginnote{26.3.4} his illness?” 

“He\marginnote{26.3.5} has dysentery.” 

“Does\marginnote{26.3.6} he have a nurse?” 

“No.”\marginnote{26.3.7} 

“But\marginnote{26.3.8} why don’t you nurse him?” 

“Because\marginnote{26.3.9} he doesn’t do anything for us.” 

“Monks,\marginnote{26.3.11} you have no mother or father to nurse you. If you don’t nurse one another, who will? Whoever would nurse me should nurse one who is sick. 

\scrule{If you have a preceptor, he should nurse you for life; he shouldn’t go anywhere until you’ve recovered. If you have a teacher, he should nurse you for life; he shouldn’t go anywhere until you’ve recovered. If you have a student, he should nurse you for life; he shouldn’t go anywhere until you’ve recovered. If you have a pupil, he should nurse you for life; he shouldn’t go anywhere until you’ve recovered. If you have a co-student, he should nurse you for life; he shouldn’t go anywhere until you’ve recovered. If you have a co-pupil, he should nurse you for life; he shouldn’t go anywhere until you’ve recovered. If you have none of these, the Sangha should nurse you. If you don’t nurse one who is sick, you commit an offense of wrong conduct. }

“A\marginnote{26.5.1} sick person who has five qualities is hard to nurse: they do what’s detrimental for curing the sickness; they don’t know the right amount in what’s beneficial; they don’t take their medicine; they don’t accurately describe the state of their illness—whether it’s getting worse, better, or remaining the same—to the one who’s nursing them and wishing them well; they’re unable to bear up with bodily feelings that are painful, severe, sharp, and life-threatening. 

A\marginnote{26.6.1} sick person who has five qualities is easy to nurse: they do what’s beneficial for curing the sickness; they know the right amount in what’s beneficial; they take their medicine; they accurately describe the state of their illness—whether it’s getting worse, better, or remaining the same—to the one who’s nursing them and wishing them well; they’re able to bear up with bodily feelings that are painful, severe, sharp, and life-threatening. 

An\marginnote{26.7.1} attendant who has five qualities is unsuited to nurse the sick: they’re incapable of preparing medicine; not knowing what’s beneficial and what’s harmful, they bring what’s harmful and remove what’s beneficial; they nurse the sick for the sake of worldly gain, not with a mind of good will; they’re disgusted at having to clean up feces, urine, spit, or vomit; they’re incapable of instructing, inspiring, and gladdening the sick person with a teaching from time to time.\footnote{Sp 3.366: \textit{\textsanskrit{Bhesajjaṁ} \textsanskrit{saṁvidhātunti} \textsanskrit{bhesajjaṁ} \textsanskrit{yojetuṁ} asamattho hoti}, “\textit{\textsanskrit{Bhesajjaṁ} \textsanskrit{saṁvidhātuṁ}}: he is incapable of preparing medicine.” } 

An\marginnote{26.8.1} attendant who has five qualities is suited to nurse the sick: they’re capable of preparing medicine; knowing what’s beneficial and what’s harmful, they remove what’s harmful and bring what’s beneficial; they nurse the sick with a mind of good will, not for the sake of worldly gain; they’re not disgusted at having to clean up feces, urine, spit, or vomit; they’re capable of instructing, inspiring, and gladdening the sick person with a teaching from time to time.” 

\section*{24. Discussion of inheritance }

On\marginnote{27.1.1} one occasion two monks were traveling through the Kosalan country, when they arrived at a monastery with a sick monk. They thought, “The Buddha has praised nursing the sick, so let’s nurse this monk.” But while they were nursing him, he died. They then took his bowl and robes and went to \textsanskrit{Sāvatthī}, where they told the Buddha what had happened. 

\scrule{“When a monk dies, the Sangha becomes the owner of his bowl and robes. Still, the nurses have been very helpful. I allow the Sangha to give the three robes and the bowl to the nurses. }

And\marginnote{27.2.4} they should be given like this. The monk nurse should approach the Sangha and say, ‘Venerables, monk so-and-so has died. Here are his three robes and bowl.’ A competent and capable monk should then inform the Sangha: 

‘Please,\marginnote{27.2.9} venerables, I ask the Sangha to listen. Monk so-and-so has died. Here are his three robes and bowl. If the Sangha is ready, it should give the three robes and the bowl to the nurses. This is the motion. 

Please,\marginnote{27.2.14} venerables, I ask the Sangha to listen. Monk so-and-so has died. Here are his three robes and bowl. The Sangha gives the three robes and the bowl to the nurses. Any monk who approves of giving the three robes and the bowl to the nurses should remain silent. Any monk who doesn’t approve should speak up. 

The\marginnote{27.2.20} Sangha has given the three robes and the bowl to the nurses. The Sangha approves and is therefore silent. I’ll remember it thus.’” 

On\marginnote{27.3.1} one occasion a novice monk had died. 

\scrule{“When a novice monk dies, the Sangha becomes the owner of his bowl and robes. Still, the nurses have been very helpful. I allow the Sangha to give the robes and the bowl to the nurses. }

And\marginnote{27.3.6} they should be given like this. The monk nurse should approach the Sangha and say, ‘Venerables, the novice monk so-and-so has died. Here are his robes and bowl.’ A competent and capable monk should then inform the Sangha: 

‘Please,\marginnote{27.3.10} venerables, I ask the Sangha to listen. The novice monk so-and-so has died. Here are his robes and bowl. If the Sangha is ready, it should give the robes and the bowl to the nurses. This is the motion. 

Please,\marginnote{27.3.15} venerables, I ask the Sangha to listen. The novice monk so-and-so has died. Here are his robes and bowl. The Sangha gives the robes and the bowl to the nurses. Any monk who approves of giving the robes and the bowl to the nurses should remain silent. Any monk who doesn’t approve should speak up. 

The\marginnote{27.3.21} Sangha has given the robes and the bowl to the nurses. The Sangha approves and is therefore silent. I’ll remember it thus.’” 

On\marginnote{27.4.1} one occasion a monk and a novice monk were nursing someone together. While they were doing so, the patient died. The monk nurse thought, “What share of the robes should be given to the novice-monk nurse?” 

\scrule{“I allow you to give an equal share to the novice-monk nurse.” }

On\marginnote{27.5.1} one occasion a monk with many possessions had died. 

\scrule{“When a monk dies, the Sangha becomes the owner of his bowl and robes.  Still, the nurses have been very helpful. I allow the Sangha to give the three robes and the bowl to the nurses. The present Sangha should distribute his ordinary possessions.\footnote{“Possessions” renders \textit{\textsanskrit{parikkhāra}}. See Appendix of Technical Terms. } His valuable possessions are for the Sangha as a whole, both present and future. They’re not to be given out, not to be distributed.” }

\section*{25. Discussion of the prohibition against nakedness }

On\marginnote{28.1.1} one occasion a naked monk went to the Buddha and said, “In many ways, sir, you praise fewness of wishes, contentment, self-effacement, ascetic practices, being inspiring, reduction in things, and being energetic. Being naked leads to all those things. Please allow nakedness for the monks.” 

The\marginnote{28.1.5} Buddha rebuked him, “Foolish man, it’s not suitable, it’s not proper, it’s not worthy of a monastic, it’s not allowable, it’s not to be done. How can you undertake the practice of nakedness, like the monastics of other religions? This will affect people’s confidence …” After rebuking him … he gave a teaching and addressed the monks: 

\scrule{“You shouldn’t undertake the practice of nakedness, like the monastics of other religions. If you do, you commit a serious offense.” }

\section*{26. Discussion of the prohibition against grass robes, etc. }

On\marginnote{28.2.1} one occasion a monk put on a sarong made of grass … a sarong made of bark … a sarong made of bits of wood …\footnote{Sp 1.67: \textit{\textsanskrit{Phalakacīraṁ} \textsanskrit{nāma} \textsanskrit{phalakasaṇṭhānāni} \textsanskrit{phalakāni} \textsanskrit{sibbitvā} \textsanskrit{katacīraṁ}}, “\textit{\textsanskrit{Phalakacīra}}: a robe made by sewing together bits of wood or what has the appearance of wood.” } a sarong made of human hair … a sarong made of horse-hair … a sarong made of owls’ wings … a sarong made of antelope hide, went to the Buddha, and said, “In many ways, sir, you praise fewness of wishes, contentment, self-effacement, ascetic practices, being inspiring, reduction in things, and being energetic. A robe made of antelope hide leads to all those things. Please allow robes of antelope hide for the monks.” 

The\marginnote{28.2.12} Buddha rebuked him, “Foolish man, it’s not suitable, it’s not proper, it’s not worthy of a monastic, it’s not allowable, it’s not to be done. How can you wear a robe made of antelope hide, that sign of monastics of other religions? This will affect people’s confidence …” After rebuking him … he gave a teaching and addressed the monks: 

\scrule{“You shouldn’t wear a robe made of antelope hide, that sign of monastics of other religions. If you do, you commit a serious offense.” }

On\marginnote{28.3.1} one occasion a monk dressed in a sarong made of stalks of crown flower … in a sarong made of jute, went to the Buddha, and said, “In many ways, sir, you praise fewness of wishes, contentment, self-effacement, ascetic practices, being inspiring, reduction in things, and being energetic. A jute robe leads to all those things. Please allow jute robes for the monks.” 

The\marginnote{28.3.6} Buddha rebuked him, “Foolish man, it’s not suitable, it’s not proper, it’s not worthy of a monastic, it’s not allowable, it’s not to be done. How can you dress in a sarong made of jute? This will affect people’s confidence …” After rebuking him … he gave a teaching and addressed the monks: 

\scrule{“You shouldn’t dress in a sarong made of jute.\footnote{Sp 3.371: \textit{Potthakoti makacimayo vuccati}, “What is made of \textit{makaci} is called \textit{potthaka}.” N\&E, p. 90, identifies \textit{potthaka} as jute. } If you do, you commit an offense of wrong conduct.” }

\section*{27. Discussion of the prohibition against what is entirely blue, etc. }

At\marginnote{29.1.1} that time the monks from the group of six wore entirely blue robes, entirely yellow robes, entirely red robes, entirely magenta robes, entirely black robes, entirely orange robes,\footnote{According to the SED, the \textit{\textsanskrit{mahāraṅga}} (sv. \textit{\textsanskrit{mahārajana}}) is the safflower, which is normally deep yellow or orange. } and entirely beige robes;\footnote{Sp 3.246: \textit{\textsanskrit{Mahānāmarattā} \textsanskrit{sambhinnavaṇṇā} hoti \textsanskrit{paṇḍupalāsavaṇṇā}}, “\textit{\textsanskrit{Mahānāmaratta}} is a mixed color, the color of withered leaves.” } and robes with borders made from a single piece of cloth, robes with long borders, robes with floral borders, robes with borders decorated with snakes’ hoods, close-fitting jackets,\footnote{CPD: “\textit{\textsanskrit{Kañcuk}(\textsanskrit{ā}/a)}, m. and f. (ts.), a dress that fits close to the upper part of the body, hence: 1. jacket; 2.a. suit of armour; …”. } Lodh-tree robes,\footnote{Sp 3.372: \textit{\textsanskrit{Tirīṭakaṁ} pana \textsanskrit{rukkhachallimayaṁ}}, “But \textit{\textsanskrit{tirīṭaka}} means made of tree bark.” \textsanskrit{Khuddasikkhā}-\textsanskrit{abhinavaṭīkā} 57 adds: \textit{\textsanskrit{Tirīṭaketi} \textsanskrit{taṁ} \textsanskrit{nāmaka} rukkhatace}, “\textit{\textsanskrit{Tirīṭake}} is the bark of a tree with that name.” The \textit{\textsanskrit{tirīṭa}} is identified as the Lodh tree in SAF, p. 73. } and turbans. People complained and criticized them, “How can the Sakyan monastics wear turbans? They’re just like householders who indulge in worldly pleasures!” 

\scrule{“You shouldn’t wear entirely blue robes, entirely yellow robes, entirely red robes, entirely magenta robes, entirely black robes, entirely orange robes, entirely beige robes, robes with borders made from a single piece of cloth, robes with long borders, robes with floral borders, robes with borders decorated with snakes’ hoods, close-fitting jackets, Lodh-tree robes, or turbans. If you do, you commit an offense of wrong conduct.” }

\section*{28. Discussion of robe-cloth that has not yet been offered to those who have completed the rainy-season residence }

At\marginnote{30.1.1} that time, the monks who had completed the rainy-season residence left before the robe-cloth was offered. They disrobed, died, admitted to being novice monks, admitted to having renounced the training, admitted to having committed the worst kind of offense, admitted to being insane, admitted to being deranged, admitted to being overwhelmed by pain, admitted to having been ejected for not recognizing an offense, admitted to having been ejected for not making amends for an offense, admitted to having been ejected for not giving up a bad view, admitted to being \textit{\textsanskrit{paṇḍakas}}, admitted to being fake monks, admitted to having previously left to join the monastics of another religion, admitted to being animals, admitted to being matricides, admitted to being patricides, admitted to being murderers of a perfected one, admitted to having raped a nun, admitted to having caused a schism in the Sangha, admitted to having caused the Buddha to bleed, or admitted to being a hermaphrodite before the robe-cloth was offered. They told the Buddha. 

\scrule{“It may be that a monk who’s completed the rains residence leaves before the robe-cloth is offered. If there’s anyone suitable to receive it, it should be given.\footnote{Sp 3.374 says: \textit{\textsanskrit{Patirūpe} \textsanskrit{gāhaketi} sace koci bhikkhu “\textsanskrit{ahaṁ} tassa \textsanskrit{gaṇhāmī}”ti \textsanskrit{gaṇhāti}, \textsanskrit{dātabbanti} attho}, “\textit{\textsanskrit{Patirūpe} \textsanskrit{gāhake}} means: if there is any monk who thinks, ‘I’ll take it for him’, he should take it.” } }

\scrule{It may be that a monk who’s completed the rains residence disrobes, dies, admits to being a novice monk, admits to having renounced the training, or admits to having committed the worst kind of offense before the robe-cloth is offered. Then the Sangha becomes the owner of that robe-cloth. }

\scrule{It may be that a monk who’s completed the rains residence admits to being insane, to being deranged, to being overwhelmed by pain, to having been ejected for not recognizing an offense, to having been ejected for not making amends for an offense, or to having been ejected for not giving up a bad view before the robe-cloth is offered. If there’s anyone suitable to receive it, it should be given. }

\scrule{It may be that a monk who’s completed the rains residence admits to being a \textit{\textsanskrit{paṇḍaka}}, to being a fake monk, to having previously left to join the monastics of another religion, to being an animal, to being a matricide, to being a patricide, to being a murderer of a perfected one, to having raped a nun, to having caused a schism in the Sangha, to having caused the Buddha to bleed, or to being a hermaphrodite before the robe-cloth is offered. Then the Sangha becomes the owner of that robe-cloth. }

\scrule{It may be that, after robe-cloth has been offered but before it’s distributed, a monk who’s completed the rains residence leaves. If there’s anyone suitable to receive the robe-cloth, it should be given.\footnote{“Has been offered” renders \textit{uppanna}. This word, which literally means “arisen”, varies slightly in meaning dependent on the context. Often it refers to a requisite that has just been given to the Sangha or an individual monastic. Occasionally however, such as here, this does not fit the context, because the giving of the cloth is specifically said to happen afterwards. In other words, here \textit{uppanna} happens first, and only then is the robe given. The meaning, then, must be that the monks had been given an offer or a promise of robe-cloth, but had not yet received it. In a sense, the robe-cloth had “become available” to them. The most common way for a requisite to become available to a monastic is that an offer is made. I translate accordingly. See also DOP for this meaning of \textit{uppanna}. } }

\scrule{It may be that, after robe-cloth has been offered but before it’s distributed, a monk who’s completed the rains residence disrobes, dies, admits to being a novice monk, admits to having renounced the training, or admits to having committed the worst kind of offense. Then the Sangha becomes the owner of that robe-cloth. }

\scrule{It may be that, after robe-cloth has been offered but before it’s distributed, a monk who’s completed the rains residence admits to being insane, to being deranged, to being overwhelmed by pain, to having been ejected for not recognizing an offense, to having been ejected for not making amends for an offense, or to having been ejected for not giving up a bad view. If there’s anyone suitable to receive the robe-cloth, it should be given. }

\scrule{It may be that, after robe-cloth has been offered but before it’s distributed, a monk who’s completed the rains residence admits to being a \textit{\textsanskrit{paṇḍaka}}, to being a fake monk, to having previously left to join the monastics of another religion, to being an animal, to being a matricide, to being a patricide, to being a murderer of a perfected one, to having raped a nun, to having caused a schism in the Sangha, to having caused the Buddha to bleed, or to being a hermaphrodite. Then the Sangha becomes the owner of that robe-cloth.” }

\section*{29. Discussion of robe-cloth that is offered when the Sangha is divided }

\scrule{“It may be that the Sangha splits before robe-cloth is offered to those monks who have completed the rains residence. If people then give water to one side and robe-cloth to the other, saying, ‘We give to the Sangha,’ it’s all for the Sangha. }

\scrule{It may be that the Sangha splits before robe-cloth is offered to those monks who have completed the rains residence. If people then give water and robe-cloth to the same side, saying, ‘We give to the Sangha,’ it’s all for the Sangha. }

\scrule{It may be that the Sangha splits before robe-cloth is offered to those monks who have completed the rains residence. If people then give water to one side and robe-cloth to the other, saying, ‘We give to this side,’ it’s all for that side.\footnote{Sp 3.376: \textit{Pakkhassevetanti \textsanskrit{evaṁ} dinne yassa \textsanskrit{koṭṭhāsassa} \textsanskrit{udakaṁ} \textsanskrit{dinnaṁ}, tassa udakameva hoti; yassa \textsanskrit{cīvaraṁ} \textsanskrit{dinnaṁ}, tasseva \textsanskrit{cīvaraṁ}}, “\textit{Pakkhassevetan}: when given in this way, the water is for the side to which water was given, and the robe-cloth is for the side to which robe-cloth was given.” And so, when the Canonical text says, “It’s all for that side,” this means that all the water is for the side where they give water, and all the cloth is for the side where they give cloth. } }

\scrule{It may be that the Sangha splits before robe-cloth is offered to those monks who have completed the rains residence. If people then give water and robe-cloth to the same side, saying, ‘We give to this side,’ it’s all for that side. }

\scrule{It may be that the Sangha splits before the robe-cloth is distributed but after it was offered to those monks who have completed the rains residence. The robe-cloth is then to be distributed equally to everyone.” }

\section*{30. Discussion of what is properly and improperly taken }

On\marginnote{31.1.1} one occasion, Venerable Revata gave a robe to a monk to take to Venerable \textsanskrit{Sāriputta}, saying, “Please give this robe to the elder.” While on his way, that monk took that robe on trust from Revata. 

Later,\marginnote{31.1.4} when Revata met \textsanskrit{Sāriputta}, Revata asked him whether he had received that robe. He replied that he had not. 

Revata\marginnote{31.1.8} then asked the other monk, “I gave you a robe to take to the elder. Where’s that robe?” 

“I\marginnote{31.1.11} took it on trust from you.” They told the Buddha. 

\scrule{“It may be that a monk gives a robe to a monk to take to yet another monk, saying, ‘Give this robe to so-and-so.’ If, while on his way, he takes it on trust from the sender, it’s properly taken. But if he takes it on trust from the intended recipient, it’s improperly taken. }

\scrule{It may be that a monk gives a robe to a monk to take to yet another monk, saying, ‘Give this robe to so-and-so.’ If, while on his way, he takes it on trust from the intended recipient, it’s improperly taken. But if he takes it on trust from the sender, it’s properly taken. }

\scrule{It may be that a monk gives a robe to a monk to take to yet another monk, saying, ‘Give this robe to so-and-so.’ If, while on his way, he hears that the sender has died and he determines it as a robe inherited from the sender, it’s properly determined. But if he takes it on trust from the intended recipient, it’s improperly taken. }

\scrule{It may be that a monk gives a robe to a monk to take to yet another monk, saying, ‘Give this robe to so-and-so.’ If, while on his way, he hears that the intended recipient has died and he determines it as a robe inherited from the intended recipient, it’s improperly determined. But if he takes it on trust from the sender, it’s properly taken. }

\scrule{It may be that a monk gives a robe to a monk to take to yet another monk, saying, ‘Give this robe to so-and-so.’ If, while on his way, he hears that both have died and he determines it as a robe inherited from the sender, it’s properly determined. But if he determines it as a robe inherited from the intended recipient, it’s improperly determined. }

\scrule{It may be that a monk gives a robe to a monk to take to yet another monk, saying, ‘I give this robe to so-and-so.’ If, while on his way, he takes it on trust from the sender, it’s improperly taken. But if he takes it on trust from the intended recipient, it’s properly taken. }

\scrule{It may be that a monk gives a robe to a monk to take to yet another monk, saying, ‘I give this robe to so-and-so.’ If, while on his way, he takes it on trust from the intended recipient, it’s properly taken. But if he takes it on trust from the sender, it’s improperly taken. }

\scrule{It may be that a monk gives a robe to a monk to take to yet another monk, saying, ‘I give this robe to so-and-so.’ If, while on his way, he hears that the sender has died and he determines it as a robe inherited from the sender, it’s improperly determined. But if he takes it on trust from the intended recipient, it’s properly taken. }

\scrule{It may be that a monk gives a robe to a monk to take to yet another monk, saying, ‘I give this robe to so-and-so.’ If, while on his way, he hears that the intended recipient has died and he determines it as a robe inherited from the intended recipient, it’s properly determined. But if he takes it on trust from the sender, it’s improperly taken. }

\scrule{It may be that a monk gives a robe to a monk to take to yet another monk, saying, ‘I give this robe to so-and-so.’ If, while on his way, he hears that both have died and he determines it as a robe inherited from the sender, it’s improperly determined. But if he determines it as a robe inherited from the intended recipient, it’s properly determined.” }

\section*{31. Eight key phrases on robe-cloth }

“Monks,\marginnote{32.1.1} there are these eight key phrases for the giving of robe-cloth: someone gives within a monastery zone; someone gives to a recipient who has made an agreement; someone gives where alms are prepared; someone gives to the Sangha; someone gives to both Sanghas; someone gives to a sangha that has completed the rainy-season residence; someone gives according to a specification; someone gives to an individual.\footnote{Sp 3.379: \textit{… \textsanskrit{sīmaṁ} \textsanskrit{parāmasitvā} dento \textsanskrit{sīmāya} deti \textsanskrit{nāma}}, “… giving while touching the zone is called ‘giving within a zone’.” } 

\begin{enumerate}%
\item Someone gives within a monastery zone: it should be distributed by the monks within that zone. %
\item Someone gives to a recipient who has made an agreement: when a number of monasteries have the same material support, then when it’s given in one monastery, it’s given to all.\footnote{Sp 3.379: \textit{\textsanskrit{Katikāyāti} \textsanskrit{samānalābhakatikāya}. \textsanskrit{Tenevāha} – “\textsanskrit{sambahulā} \textsanskrit{āvāsā} \textsanskrit{samānalābhā} \textsanskrit{hontī}”ti. \textsanskrit{Tatrevaṁ} \textsanskrit{katikā} \textsanskrit{kātabbā}, \textsanskrit{ekasmiṁ} \textsanskrit{vihāre} sannipatitehi \textsanskrit{bhikkhūhi} \textsanskrit{yaṁ} \textsanskrit{vihāraṁ} \textsanskrit{saṅgaṇhitukāmā} \textsanskrit{samānalābhaṁ} \textsanskrit{kātuṁ} icchanti, tassa \textsanskrit{nāmaṁ} \textsanskrit{gahetvā} asuko \textsanskrit{nāma} \textsanskrit{vihāro} \textsanskrit{porāṇakoti} \textsanskrit{vā} \textsanskrit{buddhādhivutthoti} \textsanskrit{vā} \textsanskrit{appalābhoti} \textsanskrit{vā} \textsanskrit{yaṁkiñci} \textsanskrit{kāraṇaṁ} \textsanskrit{vatvā} \textsanskrit{taṁ} \textsanskrit{vihāraṁ} \textsanskrit{iminā} \textsanskrit{vihārena} \textsanskrit{saddhiṁ} \textsanskrit{ekalābhaṁ} \textsanskrit{kātuṁ} \textsanskrit{saṅghassa} \textsanskrit{ruccatīti} \textsanskrit{tikkhattuṁ} \textsanskrit{sāvetabbaṁ}. \textsanskrit{Ettāvatā} \textsanskrit{tasmiṁ} \textsanskrit{vihāre} nisinnopi idha nisinnova hoti, \textsanskrit{tasmiṁ} \textsanskrit{vihārepi} \textsanskrit{saṅghena} evameva \textsanskrit{kātabbaṁ}. \textsanskrit{Ettāvatā} idha nisinnopi \textsanskrit{tasmiṁ} nisinnova hoti. \textsanskrit{Ekasmiṁ} \textsanskrit{lābhe} \textsanskrit{bhājiyamāne} \textsanskrit{itarasmiṁ} \textsanskrit{ṭhitassa} \textsanskrit{bhāgaṁ} \textsanskrit{gahetuṁ} \textsanskrit{vaṭṭati}. \textsanskrit{Evaṁ} ekena \textsanskrit{vihārena} \textsanskrit{saddhiṁ} \textsanskrit{bahūpi} \textsanskrit{āvāsā} \textsanskrit{ekalābhā} \textsanskrit{kātabbā}}, “\textit{\textsanskrit{Katikāya}}: an agreement on equality in material support. Because of this, it was said: \textit{\textsanskrit{Sambahulā} \textsanskrit{āvāsā} \textsanskrit{samānalābhā} honti}. In regard to this, the agreement is to be made in this way: by the monks gathered in one monastery, in the monastery where they desire to collect (material support) to make an equality in material support, having taken its name, a monastery called such-and-such, whether it is old or was lived in by the Buddha or gets little support, for whatever reason, having said this, he should proclaim three times, “That monastery together with this monastery approves of the Sangha to make a unity in material support.” With this much, even if seated in that monastery, it is as if seated here. Also, if this were to be done by the Sangha in that monastery, then, with this much, even if seated here, it is as if seated there. When distributing the material support in one place, one is allowed to take a share for one in the other. In this way, even many monasteries are to be made a unity in material support with one monastery.” } %
\item Someone gives where alms are prepared: someone gives where the Sangha is regularly working.\footnote{\textit{\textsanskrit{Saṅghassa} \textsanskrit{dhuvakārā} kariyyanti} can be construed either as the Sangha regularly doing work or as work regularly being done for the Sangha. The commentary merely offer various scenarios for this sort of situation. Sp 3.379: \textit{\textsanskrit{Bhikkhāpaññattiyāti} attano \textsanskrit{pariccāgapaññāpanaṭṭhāne}. \textsanskrit{Tenevāha} – “yattha \textsanskrit{saṅghassa} \textsanskrit{dhuvakārā} \textsanskrit{kariyantī}”ti. Tassattho – \textsanskrit{yasmiṁ} \textsanskrit{vihāre} imassa \textsanskrit{cīvaradāyakassa} \textsanskrit{santakaṁ} \textsanskrit{saṅghassa} \textsanskrit{pākavaṭṭaṁ} \textsanskrit{vā} vattati, \textsanskrit{yasmiṁ} \textsanskrit{vā} \textsanskrit{vihāre} \textsanskrit{bhikkhū} attano \textsanskrit{bhāraṁ} \textsanskrit{katvā} \textsanskrit{sadā} gehe bhojeti, yattha \textsanskrit{vā} anena \textsanskrit{āvāso} \textsanskrit{kārito}, \textsanskrit{salākabhattādīni} \textsanskrit{vā} \textsanskrit{nibaddhāni}, yena pana sakalopi \textsanskrit{vihāro} \textsanskrit{patiṭṭhāpito}, tattha vattabbameva natthi, ime \textsanskrit{dhuvakārā} \textsanskrit{nāma}. \textsanskrit{Tasmā} sace so “yattha \textsanskrit{mayhaṁ} \textsanskrit{dhuvakārā} \textsanskrit{karīyanti}, tattha \textsanskrit{dammī}”ti \textsanskrit{vā} “tattha \textsanskrit{dethā}”ti \textsanskrit{vā} \textsanskrit{bhaṇati}, \textsanskrit{bahūsu} cepi \textsanskrit{ṭhānesu} \textsanskrit{dhuvakārā} honti, sabbattha dinnameva hoti}, “\textit{\textsanskrit{Bhikkhāpaññattiyā}}: in the place of preparing one’s own offering. Because of that, this is said: \textit{Yattha \textsanskrit{saṅghassa} \textsanskrit{dhuvakārā} kariyanti}. This is its meaning: ‘In the monastery where the belongings of this robe-giver are; or where there is a regular supply of cooked food for the Sangha; or in the monastery where, having created their own burden, the monks are always fed in a house; or where a monastery is built by him (Sp-yoj 3.379: \textit{\textsanskrit{cīvaradāyakena}}, “by the robe-giver”), or when meals decided by lots, etc., are regular, by whom even an entire monastery is established, (even if) nothing is to be done there—these are called regular work. Therefore, if he thinks, “I will give where constant work is being done by me,” or he says, “Give there,” then even if there is constant work in many places, it is given everywhere.’” } %
\item Someone gives to the Sangha: the present Sangha should distribute it. %
\item Someone gives to both Sanghas: even when there are many monks and just a single nun, she should be given half; even when there are many nuns and just a single monk, he should be given half. %
\item Someone gives to a sangha that has completed the rainy-season residence: it’s to be distributed by the monks who have completed the rains residence in that monastery. %
\item Someone gives according to a specification: relating to congee, a meal, fresh food, robe-cloth, a dwelling, or medicine.\footnote{Sp 3.379: \textit{Ādissa \textsanskrit{detīti} … \textsanskrit{Tatrāyaṁ} \textsanskrit{yojanā} – \textsanskrit{bhikkhū} \textsanskrit{ajjatanāya} \textsanskrit{vā} \textsanskrit{svātanāya} \textsanskrit{vā} \textsanskrit{yāguyā} \textsanskrit{nimantetvā} \textsanskrit{tesaṁ} \textsanskrit{gharaṁ} \textsanskrit{paviṭṭhānaṁ} \textsanskrit{yāguṁ} deti, \textsanskrit{yāguṁ} \textsanskrit{datvā} \textsanskrit{pītāya} \textsanskrit{yāguyā} “\textsanskrit{imāni} \textsanskrit{cīvarāni}, yehi \textsanskrit{mayhaṁ} \textsanskrit{yāgu} \textsanskrit{pītā}, \textsanskrit{tesaṁ} \textsanskrit{dammī}”ti deti, yehi nimantitehi \textsanskrit{yāgu} \textsanskrit{pītā}, \textsanskrit{tesaṁyeva} \textsanskrit{pāpuṇāti}}. “\textit{Ādissa deti} … This is the meaning: having invited monks to congee on the same or the following day, one then gives congee to those who have entered the house. When the congee has been given and it has been drunk, one then gives, saying, ‘I give these robes to those who drank my congee.’ The drinkers of congee among those who were invited, only they obtain (robes).” } %
\item Someone gives to an individual: ‘I give this robe-cloth to so-and-so.’” %
\end{enumerate}

\scendsutta{The eighth chapter on robes is finished. }

\scuddanaintro{This is the summary: }

\begin{scuddana}%
“The\marginnote{32.1.16} householder association of \textsanskrit{Rājagaha}, \\
Having seen the courtesan in \textsanskrit{Vesālī}; \\
Returned to \textsanskrit{Rājagaha}, \\
Announced it to the king. 

The\marginnote{32.1.20} son of \textsanskrit{Sālavatī}, \\
But the child of Abhaya; \\
Because the boy lived, \\
He was called \textsanskrit{Jīvaka}. 

He\marginnote{32.1.24} went to \textsanskrit{Takkasilā}, \\
Having learned, a great physician; \\
A seven-year illness, \\
He cured by nose treatment. 

The\marginnote{32.1.28} king’s hemorrhoids, \\
Applied ointment; \\
Attended on me and the harem, \\
And the Buddha and the Sangha. 

And\marginnote{32.1.32} the merchant of \textsanskrit{Rājagaha}, \\
Treated the twisted gut; \\
The great illness of Pajjota, \\
He cured with a drink of ghee. 

And\marginnote{32.1.36} service, valuable cloth, \\
Full of, he oiled; \\
With three handfuls of lotus flowers, \\
Thirty purgings exactly. 

He\marginnote{32.1.40} asked for a blameless favor, \\
And he received the valuable cloths; \\
And robes given by householders, \\
Was allowed by the Buddha. 

In\marginnote{32.1.44} \textsanskrit{Rājagaha}, in the country, \\
Many robes were given; \\
A fleecy robe, and silken, \\
Woolen fleecy robe, valuable \textsanskrit{Kāsi} cloth. 

And\marginnote{32.1.48} various kinds, contented, \\
Didn’t wait, and did wait; \\
First, after, together, \\
And agreement, took it back. 

Storeroom,\marginnote{32.1.52} and not looked after, \\
And just so they dismissed; \\
Much, and racket, \\
How should one distribute, what should one give. 

His\marginnote{32.1.56} own, with an extra share, \\
How should a share be given; \\
With dung, cold water, \\
Boiled over, they did not know. 

Tilting,\marginnote{32.1.60} and vessel, \\
And in a basin, and on the ground; \\
Termites, in the middle, they became worn, \\
From one edge, and with starch. 

Stiff,\marginnote{32.1.64} uncut, rectangles, \\
He saw them loaded up; \\
Having tested, the Sakyan Sage, \\
Allowed three robes. 

With\marginnote{32.1.68} another extra, \\
Was given, and just a hole; \\
Four-continent, she asked for a favor, \\
To give a rainy-season robe. 

And\marginnote{32.1.72} newly-arrived, departing, and sick, \\
And nurse, medicine; \\
Regular, and bathing robe, \\
Fine, too small. 

Carbuncles,\marginnote{32.1.76} washcloth, linen, \\
Enough, determining; \\
Smallest, made heavy, \\
Deformed corner, frayed. 

They\marginnote{32.1.80} broke up, not enough, \\
And a further supply, and much; \\
In the Blind Men’s Grove, through absentmindedness, \\
The rains by himself, and outside the rainy season. 

Two\marginnote{32.1.84} brothers, in \textsanskrit{Rājagaha}, \\
Upananda, again in two; \\
Dysentery, illness, \\
And just both, belonging to the sick. 

Naked,\marginnote{32.1.88} grass, bark, \\
Bits of wood, human hair; \\
Horse-hair, and owls’ wings, \\
Antelope, stalks of crown flower. 

Jute,\marginnote{32.1.92} and blue, yellow, \\
Red, and with magenta; \\
Black, orange, beige, \\
So uncut borders. 

Long,\marginnote{32.1.96} floral, snake’s hood borders, \\
Jacket, Lodh tree, turban; \\
Not yet offered, he left, \\
The Sangha is divided just then. 

They\marginnote{32.1.100} give to one side, to the Sangha, \\
Venerable Revata sent; \\
Taking on trust, determined, \\
Eight key phrases on robes.” 

%
\end{scuddana}

\scend{In this chapter there are ninety-six topics. }

\scendsutta{The chapter on robes is finished. }

%
\chapter*{{\suttatitleacronym Kd 9}{\suttatitletranslation The chapter connected with Campā }{\suttatitleroot Campeyyakkhandhaka}}
\addcontentsline{toc}{chapter}{\tocacronym{Kd 9} \toctranslation{The chapter connected with Campā } \tocroot{Campeyyakkhandhaka}}
\markboth{The chapter connected with Campā }{Campeyyakkhandhaka}
\extramarks{Kd 9}{Kd 9}

\section*{1. The account of the monk Kassapagotta }

At\marginnote{1.1.1} one time the Buddha was staying at \textsanskrit{Campā} on the banks of the \textsanskrit{Gaggarā} lotus pond. At that time in the country of \textsanskrit{Kāsi} there was a village called \textsanskrit{Vāsabha} with a resident monk called Kassapagotta. He was dedicated to the local monastery,\footnote{Sp 3.380: \textit{Tantibaddhoti \textsanskrit{tasmiṁ} \textsanskrit{āvāse} \textsanskrit{kattabbatātantipaṭibaddho}}, “\textit{Tantibaddho}: bound to what is to be done in regard to that monastery”. } trying to get good monks to come, to help those who had come be comfortable, and to make the local monastery grow and reach maturity. 

At\marginnote{1.1.5} this time a number of monks who were wandering in \textsanskrit{Kāsi} arrived at \textsanskrit{Vāsabha}. When Kassapagotta saw those monks coming, he prepared seats, and he set out a foot stool, a foot scraper, and water for washing the feet. He then went out to meet them, received their bowls and robes, and asked if they wanted water to drink. He made sure they had a bath, and he helped them get congee, fresh foods, and meals. Those newly-arrived monks thought, “He’s great, this resident monk, since he helps us with all these things. Let’s settle down right here in \textsanskrit{Vāsabha}.” And they did just that. 

Soon\marginnote{1.2.1} afterwards Kassapagotta thought, “These monks are now rid of their tiredness from traveling. And by now they know where to get alms. Also, in the long run it’s hard work to seek support from unrelated folk, and people don’t like to be asked. Why don’t I stop helping them get congee, fresh foods, and meals?” And he did. 

Those\marginnote{1.2.7} newly-arrived monks considered, “Previously this resident monk made sure we got a bath, and he helped us get congee, fresh foods, and meals. But now he’s stopped. He’s become hostile, this resident monk. Well then, let’s eject him.” 

Soon\marginnote{1.3.1} afterwards those newly-arrived monks gathered and confronted Kassapagotta with what had happened, adding, “You’ve committed an offense. Do you recognize it?” 

“No.\marginnote{1.3.6} I haven’t committed any offense that I should recognize.” 

Those\marginnote{1.3.7} newly-arrived monks then ejected Kassapagotta for not recognizing an offense. 

Kassapagotta\marginnote{1.3.8} thought, “I don’t actually know whether this was an offense or not, whether I’ve committed one or not, whether I’ve been ejected or not, whether it was legitimate or not, whether it’s reversible or not, whether it’s fit to stand or not. Let me go to \textsanskrit{Campā} and ask the Buddha.” 

He\marginnote{1.4.1} then put his dwelling in order, took his bowl and robe, and set out for \textsanskrit{Campā}. When he eventually arrived, he went to the Buddha, bowed, and sat down. Since it is the custom for Buddhas to greet newly-arrived monks, the Buddha said to Kassapagotta, “I hope you’re keeping well, monk, I hope you’re getting by?  I hope you’re not tired from traveling?  And where have you come from?” 

“I’m\marginnote{1.4.8} keeping well, sir, I’m getting by. I’m not tired from traveling.” And he told the Buddha all that had happened, adding, “That’s where I’ve come from.” 

“Well,\marginnote{1.6.1} that’s not an offense, monk, and you haven’t been ejected. You’ve been ejected by an illegitimate legal procedure that’s reversible and unfit to stand. Go back and stay right there in the village of \textsanskrit{Vāsabha}.” 

“Yes,\marginnote{1.6.6} sir.” He got up from his seat, bowed down, circumambulated the Buddha with his right side toward him, and set out for \textsanskrit{Vāsabha}. 

Soon\marginnote{1.7.1} those newly-arrived monks became anxious and remorseful: “It’s truly bad for us that we have ejected, without reason, a pure monk who hadn’t committed any offense. Well then, let’s go to \textsanskrit{Campā} and confess our mistake to the Buddha.” 

They\marginnote{1.7.5} then put their dwellings in order, took their bowls and robes, and set out for \textsanskrit{Campā}. When they eventually arrived, they went to the Buddha, bowed, and sat down. Since it is the custom for Buddhas to greet newly-arrived monks, the Buddha said to them, “I hope you’re keeping well, monks, I hope you’re getting by?  I hope you’re not tired from traveling?  And where have you come from?” 

“We’re\marginnote{1.7.12} keeping well, sir, we’re getting by. We’re not tired from traveling. There’s a village in the country of \textsanskrit{Kāsi} called \textsanskrit{Vāsabha}. That’s where we’ve come from.” 

“Are\marginnote{1.8.1} you the ones who ejected the resident monk?” 

“Yes,\marginnote{1.8.2} sir.” 

“For\marginnote{1.8.3} what reason?” 

“Without\marginnote{1.8.4} any reason.” 

The\marginnote{1.8.5} Buddha rebuked them, “Foolish men, it’s not suitable, it’s not proper, it’s not worthy of a monastic, it’s not allowable, it’s not to be done. How could you, without reason, eject a pure monk who hadn’t committed any offense? This will affect people’s confidence …” After rebuking them … he gave a teaching and addressed the monks: 

\scrule{“You shouldn’t, without reason, eject a pure monk who hasn’t committed any offense. If you do, you commit an offense of wrong conduct.” }

Those\marginnote{1.9.1} monks then got up from their seats, arranged their upper robes over one shoulder, bowed down with their heads at the Buddha’s feet, and said, “Sir, we have made a mistake. We’ve been foolish, confused, and unskillful in ejecting, without reason, a pure monk who hadn’t committed any offense. Please accept our confession so that we may restrain ourselves in the future.” 

“You\marginnote{1.9.4} have certainly made a mistake. You’ve been foolish, confused, and unskillful. But since you acknowledge your mistake and make proper amends, I forgive you. For this is called growth in the training of the noble ones: acknowledging a mistake, making proper amends, and undertaking restraint for the future.” 

\section*{2. Discussion of illegitimate legal procedures done by an incomplete assembly, etc. }

At\marginnote{2.1.1} that time the monks at \textsanskrit{Campā} did legal procedures such as these: illegitimate legal procedures done by an incomplete assembly, illegitimate legal procedures done by a unanimous assembly, legitimate legal procedures done by an incomplete assembly, legitimate-like legal procedures done by an incomplete assembly, legitimate-like legal procedures done by a unanimous assembly, one person ejecting another, one ejecting two, one ejecting three, one ejecting a sangha,\footnote{“Three” renders \textit{sambahula}. See Appendix of Technical Terms. } two ejecting one, two ejecting two, two ejecting three, two ejecting a sangha, three ejecting one, three ejecting two, three ejecting three, three ejecting a sangha, a sangha ejecting a sangha. 

The\marginnote{2.2.1} monks of few desires complained and criticized them, “How can the monks at \textsanskrit{Campā} do such legal procedures?” 

They\marginnote{2.2.4} told the Buddha. … “Is it true, monks, that the monks at \textsanskrit{Campā} do this?” 

“It’s\marginnote{2.2.8} true, sir.” 

The\marginnote{2.2.9} Buddha rebuked them, “It’s not suitable for those foolish men, it’s not proper, it’s not worthy of a monastic, it’s not allowable, it’s not to be done. How can they do such legal procedures? This will affect people’s confidence …” After rebuking them … he gave a teaching and addressed the monks: 

\begin{itemize}%
\item “Illegitimate legal procedures done by an incomplete assembly are invalid and not to be done. %
\item Illegitimate legal procedures done by a unanimous assembly are invalid and not to be done. %
\item Legitimate legal procedures done by an incomplete assembly are invalid and not to be done. %
\item Legitimate-like legal procedures done by an incomplete assembly are invalid and not to be done. %
\item Legitimate-like legal procedures done by a unanimous assembly are invalid and not to be done. %
\item One person ejecting another is invalid and not to be done. %
\item One ejecting two is invalid and not to be done. %
\item One ejecting three is invalid and not to be done. %
\item One ejecting a sangha is invalid and not to be done. %
\item Two ejecting one is invalid and not to be done. %
\item Two ejecting two is invalid and not to be done. %
\item Two ejecting three is invalid and not to be done. %
\item Two ejecting a sangha is invalid and not to be done. %
\item Three ejecting one is invalid and not to be done. %
\item Three ejecting two is invalid and not to be done. %
\item Three ejecting three is invalid and not to be done. %
\item Three ejecting a sangha is invalid and not to be done. %
\item A sangha ejecting a sangha is invalid and not to be done. %
\end{itemize}

There\marginnote{2.4.1} are four kinds of legal procedures: an illegitimate legal procedure done by an incomplete assembly, an illegitimate legal procedure done by a unanimous assembly, a legitimate legal procedure done by an incomplete assembly, and a legitimate legal procedure done by a unanimous assembly. 

\begin{itemize}%
\item The illegitimate legal procedure done by an incomplete assembly is reversible and unfit to stand, because it’s illegitimate and the assembly is incomplete. You shouldn’t do such procedures. I haven’t allowed such procedures. %
\item The illegitimate legal procedure done by a unanimous assembly is reversible and unfit to stand, because it’s illegitimate. You shouldn’t do such procedures. I haven’t allowed such procedures. %
\item The legitimate legal procedure done by an incomplete assembly is reversible and unfit to stand, because the assembly is incomplete. You shouldn’t do such procedures. I haven’t allowed such procedures. %
\item The legitimate legal procedure done by a unanimous assembly is irreversible and fit to stand, because it’s legitimate and the assembly is unanimous. You should do such procedures. I have allowed such procedures. %
\end{itemize}

And\marginnote{2.4.11} so, monks, you should train yourselves like this: ‘We will perform legitimate legal procedures done by a unanimous assembly.’” 

\section*{3. Discussion of legal procedures deficient in motion, etc. }

At\marginnote{3.1.1} that time the monks from the group of six did legal procedures such as these: illegitimate procedures done by an incomplete assembly; illegitimate procedures done by a unanimous assembly; legitimate procedures done by an incomplete assembly; legitimate-like procedures done by an incomplete assembly; legitimate-like procedures done by a unanimous assembly; procedures deficient in motion but complete in announcement; procedures deficient in announcement but complete in motion; procedures deficient in both motion and announcement;\footnote{“Announcement” renders \textit{\textsanskrit{anussāvana}}. When used to describe elements of a \textit{\textsanskrit{saṅghakamma}}, “a legal procedure”, \textit{\textsanskrit{anussāvana}} and \textit{\textsanskrit{kammavācā}} are used synonymously. In these cases they refer to the one or three “announcements” that follow the motion, and so I render them both as “announcement”. Occasionally, however, \textit{\textsanskrit{anussāvana}}, but not \textit{\textsanskrit{kammavācā}}, is used to describe the full legal procedure of both motion and announcements. In such instances I render it as “proclamation”. } procedures not done according to the Teaching; procedures not done according to the Monastic Law; procedures not done according to the Teacher’s instructions; procedures that had been objected to, that were illegitimate, reversible, and unfit to stand. 

The\marginnote{3.1.7} monks of few desires complained and criticized them, “How can the monks from the group of six do such legal procedures?” 

They\marginnote{3.1.14} told the Buddha. … “Is it true, monks, that the monks from the group of six do this?” 

“It’s\marginnote{3.1.18} true, sir.” 

The\marginnote{3.1.19} Buddha rebuked them … He then gave a teaching and addressed the monks: 

\begin{itemize}%
\item “Illegitimate legal procedures done by an incomplete assembly are invalid and not to be done. %
\item Illegitimate legal procedures done by a unanimous assembly are invalid and not to be done. %
\item Legitimate legal procedures done by an incomplete assembly are invalid and not to be done. %
\item Legitimate-like legal procedures done by an incomplete assembly are invalid and not to be done. %
\item Legitimate-like legal procedures done by a unanimous assembly are invalid and not to be done. %
\item Legal procedures deficient in motion but complete in announcement are invalid and not to be done. %
\item Legal procedures deficient in announcement but complete in motion are invalid and not to be done. %
\item Legal procedures deficient in both motion and announcement are invalid and not to be done. %
\item Legal procedures not done according to the Teaching are invalid and not to be done. %
\item Legal procedures not done according to the Monastic Law are invalid and not to be done. %
\item Legal procedures not done according to the Teacher’s instructions are invalid and not to be done. %
\item Legal procedures that have been objected to, that are illegitimate, reversible, and unfit to stand are invalid and not to be done. %
\end{itemize}

And,\marginnote{3.3.1} monks, there are six kinds of legal procedures: illegitimate legal procedures, legal procedures done by an incomplete assembly, legal procedures done by a unanimous assembly, legitimate-like legal procedures done by an incomplete assembly, legitimate-like legal procedures done by a unanimous assembly, legitimate legal procedures done by a unanimous assembly. 

What’s\marginnote{3.3.3} an illegitimate legal procedure? 

If\marginnote{3.3.4} a procedure requires one motion and one announcement, but they do it with one motion and no announcement, it’s an illegitimate legal procedure. If a procedure requires one motion and one announcement, but they do it with two motions and no announcement, it’s an illegitimate legal procedure. If a procedure requires one motion and one announcement, but they do it with one announcement and no motion, it’s an illegitimate legal procedure. If a procedure requires one motion and one announcement, but they do it with two announcements and no motion, it’s an illegitimate legal procedure. 

If\marginnote{3.4.1} a procedure requires one motion and three announcements, but they do it with one motion and no announcement, it is an illegitimate legal procedure. If a procedure requires one motion and three announcements, but they do it with two motions and no announcement, it’s an illegitimate legal procedure. If a procedure requires one motion and three announcements, but they do it with three motions and no announcement, it’s an illegitimate legal procedure. If a procedure requires one motion and three announcements, but they do it with four motions and no announcement, it’s an illegitimate legal procedure. If a procedure requires one motion and three announcements, but they do it with one announcement and no motion, it’s an illegitimate legal procedure. If a procedure requires one motion and three announcements, but they do it with two announcements and no motion, it’s an illegitimate legal procedure. If a procedure requires one motion and three announcements, but they do it with three announcements and no motion, it’s an illegitimate legal procedure. If a procedure requires one motion and three announcements, but they do it with four announcements and no motion, it’s an illegitimate legal procedure. 

And\marginnote{3.5.1} what’s a legal procedure done by an incomplete assembly? 

When\marginnote{3.5.2} a procedure requires one motion and one announcement, but the monks who should take part haven’t all arrived, and the consent hasn’t been brought for those who are eligible to give their consent, and someone present objects to the decision, then it’s a legal procedure done by an incomplete assembly.\footnote{Sp 3.388: \textit{Kammappattoti \textsanskrit{kammaṁ} patto, kammayutto \textsanskrit{kammāraho}; na \textsanskrit{kiñci} \textsanskrit{kammaṁ} \textsanskrit{kātuṁ} \textsanskrit{nārahatīti} attho}, “‘Who should take part’: who are able in regard to the legal procedure, suitable for the legal procedure, fit for the legal procedure. The meaning is that one should not not do any kind of legal procedure.” The last line means one should or must take part in the legal procedure. } When a procedure requires one motion and one announcement, and the monks who should take part have arrived, but the consent hasn’t been brought for those who are eligible to give their consent, and someone present objects to the decision, then it’s a legal procedure done by an incomplete assembly. When a procedure requires one motion and one announcement, and the monks who should take part have arrived, and consent has been brought for those who are eligible to give their consent, but someone present objects to the decision, then it’s a legal procedure done by an incomplete assembly. 

When\marginnote{3.5.5} a procedure requires one motion and three announcements, but the monks who should take part haven’t all arrived, and consent hasn’t been brought for those who are eligible to give their consent, and someone present objects to the decision, then it’s a legal procedure done by an incomplete assembly. When a procedure requires one motion and three announcements, and the monks who should take part have arrived, but consent hasn’t been brought for those who are eligible to give their consent, and someone present objects to the decision, then it’s a legal procedure done by an incomplete assembly. When a procedure requires one motion and three announcements, and the monks who should take part have arrived, and consent has been brought for those who are eligible to give their consent, but someone present objects to the decision, then it’s a legal procedure done by an incomplete assembly. 

And\marginnote{3.6.1} what’s a legal procedure done by a unanimous assembly? 

When\marginnote{3.6.2} a procedure requires one motion and one announcement, and the monks who should take part have arrived, and consent has been brought for those who are eligible to give their consent, and no-one present objects to the decision, then it’s a legal procedure done by a unanimous assembly. When a procedure requires one motion and three announcements, and the monks who should take part have arrived, and consent has been brought for those who are eligible to give their consent, and no-one present objects to the decision, then it’s a legal procedure done by a unanimous assembly. 

And\marginnote{3.7.1} what’s a legitimate-like legal procedure done by an incomplete assembly? 

When\marginnote{3.7.2} a procedure requires one motion and one announcement, but they make the announcement first and put forward the motion afterwards, and the monks who should take part haven’t all arrived, and consent hasn’t been brought for those who are eligible to give their consent, and someone present objects to the decision, then it’s a legitimate-like legal procedure done by an incomplete assembly. When a procedure requires one motion and one announcement, but they make the announcement first and put forward the motion afterwards, yet the monks who should take part have arrived, but consent hasn’t been brought for those who are eligible to give their consent, and someone present objects to the decision, then it’s a legitimate-like legal procedure done by an incomplete assembly. When a procedure requires one motion and one announcement, but they make the announcement first and put forward the motion afterwards, yet the monks who should take part have arrived, and consent has been brought for those who are eligible to give their consent, but someone present objects to the decision, then it’s a legitimate-like legal procedure done by an incomplete assembly. 

When\marginnote{3.7.5} a procedure requires one motion and three announcements, but they make the announcements first and put forward the motion afterwards, and if the monks who should take part haven’t all arrived, and consent hasn’t been brought for those who are eligible to give their consent, and someone present objects to the decision, then it’s a legitimate-like legal procedure done by an incomplete assembly. When a procedure requires one motion and three announcements, but they make the announcements first and put forward the motion afterwards, yet the monks who should take part have arrived, but consent hasn’t been brought for those who are eligible to give their consent, and someone present objects to the decision, then it’s a legitimate-like legal procedure done by an incomplete assembly. When a procedure requires one motion and three announcements, but they make the announcements first and put forward the motion afterwards, yet the monks who should take part have arrived, and consent has been brought for those who are eligible to give their consent, but someone present objects to the decision, then it’s a legitimate-like legal procedure done by an incomplete assembly. 

And\marginnote{3.8.1} what’s a legitimate-like legal procedure done by a unanimous assembly? 

When\marginnote{3.8.2} a procedure requires one motion and one announcement, but they make the announcement first and put forward the motion afterwards, yet the monks who should take part have arrived, and consent has been brought for those who are eligible to give their consent, and no-one present objects to the decision, then it’s a legitimate-like legal procedure done by a unanimous assembly. When a procedure requires one motion and three announcements, but they make the announcements first and put forward the motion afterwards, yet the monks who should take part have arrived, and consent has been brought for those who are eligible to give their consent, and no-one present objects to the decision, then it’s a legitimate-like legal procedure done by a unanimous assembly. 

And\marginnote{3.9.1} what is a legitimate legal procedure done by a unanimous assembly? 

When\marginnote{3.9.2} a procedure requires one motion and one announcement, and they put forward the motion first and make the announcement afterwards, and the monks who should take part have arrived, and consent has been brought for those who are eligible to give their consent, and no-one present objects to the decision, then it’s a legitimate legal procedure done by a unanimous assembly. When a procedure requires one motion and three announcements, and they put forward the motion first and make the announcements afterwards, and the monks who should take part have arrived, and consent has been brought for those who are eligible to give their consent, and no-one present objects to the decision, then it’s a legitimate legal procedure done by a unanimous assembly.” 

\section*{4. Discussion of what can be done by a group of four, etc. }

“There\marginnote{4.1.1} are five kinds of sangha: a sangha of monks consisting of a group of four, a sangha of monks consisting of a group of five, a sangha of monks consisting of a group of ten, a sangha of monks consisting of a group of twenty, a sangha of monks consisting of a group of more than twenty. 

\begin{enumerate}%
\item A Sangha of monks consisting of a group of four—unanimous, acting legitimately—is able to do all legal procedures except three: ordination, invitation, and rehabilitation. %
\item A Sangha of monks consisting of a group of five—unanimous, acting legitimately—is able to do all legal procedures except two: ordination within the central Ganges plain and rehabilitation. %
\item A Sangha of monks consisting of a group of ten—unanimous, acting legitimately—is able to do all legal procedures except one: rehabilitation. %
\item A Sangha of monks consisting of a group of twenty—unanimous, acting legitimately—is able to do all legal procedures. %
\item A Sangha of monks consisting of a group of more than twenty—unanimous, acting legitimately—is able to do all legal procedures. %
\end{enumerate}

If\marginnote{4.2.1} a legal procedure that requires a group of four is done with a nun as the fourth member, it’s invalid and not to be done. If a legal procedure that requires a group of four is done with a trainee nun as the fourth member, with a novice monk as the fourth member, with a novice nun as the fourth member, with one who’s renounced the training as the fourth member, with one who’s committed the worst kind of offense as the fourth member,\footnote{Sp-yoj 5.483: \textit{Antimavatthunti \textsanskrit{pārājikavatthuṁ}}; “\textit{Antimavatthu}: an action that is the basis for an offense entailing expulsion.” } with one who’s been ejected for not recognizing an offense as the fourth member, with one who’s been ejected for not making amends for an offense as the fourth member, with one who’s been ejected for not giving up a bad view as the fourth member, with a \textit{\textsanskrit{paṇḍaka}} as the fourth member, with a fake monk as the fourth member, with one who’s previously left to join the monastics of another religion as the fourth member, with an animal as the fourth member, with a matricide as the fourth member, with a patricide as the fourth member, with a murderer of a perfected one as the fourth member, with one who’s raped a nun as the fourth member, with one who’s caused a schism in the Sangha as the fourth member, with one who’s caused the Buddha to bleed as the fourth member, with a hermaphrodite as the fourth member, with one belonging to a different Buddhist sect as the fourth member,\footnote{\textit{\textsanskrit{Nānāsaṁvāsaka}} (and \textit{\textsanskrit{samānasaṁvāsaka}}) need to be carefully distinguished from \textit{\textsanskrit{nānāsaṁvāsa}} (and \textit{\textsanskrit{samānasaṁvāsa}}). Only the former means “one belonging to a different Buddhist sect”. The latter means “belonging to a different community”, as decided by \textit{\textsanskrit{sīmās}}. } with one who’s outside the monastery zone as the fourth member,\footnote{\textit{\textsanskrit{Nānāsīmāya}} literally means “within a different monastery zone”. Sp 3.389: \textit{\textsanskrit{Nānāsīmāya} \textsanskrit{ṭhitacatutthoti} \textsanskrit{sīmantarikāya} \textsanskrit{vā} \textsanskrit{bahisīmāya} \textsanskrit{vā} \textsanskrit{hatthapāse} \textsanskrit{ṭhitenāpi} \textsanskrit{saddhiṁ} catuvaggo \textsanskrit{hutvāti} attho}, “\textit{\textsanskrit{Nānāsīmāya} \textsanskrit{ṭhitacatuttho}}, the meaning is: having been a group of four, including one who, even if within arm’s reach, is in the space between monastery zones or outside the monastery zone.” } with one floating in the air by supernormal power as the fourth member, or with one who’s subject to the legal procedure as the fourth member, it’s invalid and not to be done.” 

\scend{Procedures requiring a group of four is finished. }

“If\marginnote{4.3.2} a legal procedure that requires a group of five is done with a nun as the fifth member, it’s invalid and not to be done. If a legal procedure that requires a group of five is done with a trainee nun as the fifth member, with a novice monk as the fifth member, with a novice nun as the fifth member, with one who’s renounced the training as the fifth member, with one who’s committed the worst kind of offense as the fifth member, with one who’s been ejected for not recognizing an offense as the fifth member, with one who’s been ejected for not making amends for an offense as the fifth member, with one who’s been ejected for not giving up a bad view as the fifth member, with a \textit{\textsanskrit{paṇḍaka}} as the fifth member, with a fake monk as the fifth member, with one who’s previously left to join the monastics of another religion as the fifth member, with an animal as the fifth member, with a matricide as the fifth member, with a patricide as the fifth member, with a murderer of a perfected one as the fifth member, with one who’s raped a nun as the fifth member, with one who’s caused a schism in the Sangha as the fifth member, with one who’s caused the Buddha to bleed as the fifth member, with a hermaphrodite as the fifth member, with one belonging to a different Buddhist sect as the fifth member, with one who’s outside the monastery zone as the fifth member, with one floating in the air by supernormal power as the fifth member, or with one who’s subject to the legal procedure as the fifth member, it’s invalid and not to be done.” 

\scend{Procedures requiring a group of five is finished. }

“If\marginnote{4.4.2} a legal procedure that requires a group of ten is done with a nun as the tenth member, it’s invalid and not to be done. If a legal procedure that requires a group of ten is done with a trainee nun as the tenth member, with a novice monk as the tenth member, with a novice nun as the tenth member, with one who’s renounced the training as the tenth member, with one who’s committed the worst kind of offense as the tenth member, with one who’s been ejected for not recognizing an offense as the tenth member, with one who’s been ejected for not making amends for an offense as the tenth member, with one who’s been ejected for not giving up a bad view as the tenth member, with a \textit{\textsanskrit{paṇḍaka}} as the tenth member, with one living in the community by theft as the tenth member, with one who’s previously left to join the monastics of another religion as the tenth member, with an animal as the tenth member, with a matricide as the tenth member, with a patricide as the tenth member, with a murderer of a perfected one as the tenth member, with one who’s raped a nun as the tenth member, with one who’s caused a schism in the Sangha as the tenth member, with one who’s caused the Buddha to bleed as the tenth member, with a hermaphrodite as the tenth member, with one belonging to a different Buddhist sect as the tenth member, with one who’s outside the monastery zone as the tenth member, with one floating in the air by supernormal power as the tenth member, or with one who’s subject to the legal procedure as the tenth member, it’s invalid and not to be done.” 

\scend{Procedures requiring a group of ten is finished. }

“If\marginnote{4.5.2} a legal procedure that requires a group of twenty is done with a nun as the twentieth member, it’s invalid and not to be done. If a legal procedure that requires a group of twenty is done with a trainee nun as the twentieth member, with a novice monk as the twentieth member, with a novice nun as the twentieth member, with one who’s renounced the training as the twentieth member, with one who’s committed the worst kind of offense as the twentieth member, with one who’s been ejected for not recognizing an offense as the twentieth member, with one who’s been ejected for not making amends for an offense as the twentieth member, with one who’s been ejected for not giving up a bad view as the twentieth member, with a \textit{\textsanskrit{paṇḍaka}} as the twentieth member, with a fake monk as the twentieth member, with one who’s previously left to join the monastics of another religion as the twentieth member, with an animal as the twentieth member, with a matricide as the twentieth member, with a patricide as the twentieth member, with a murderer of a perfected one as the twentieth member, with one who’s raped a nun as the twentieth member, with one who’s caused a schism in the Sangha as the twentieth member, with one who’s caused the Buddha to bleed as the twentieth member, with a hermaphrodite as the twentieth member, with one belonging to a different Buddhist sect as the twentieth member, with one who’s outside the monastery zone as the twentieth member, with one floating in the air by supernormal power as the twentieth member, or with one who’s subject to the legal procedure as the twentieth member, it’s invalid and not to be done.” 

\scend{Procedures requiring a group of twenty is finished. }

\section*{5. Discussion of the one on probation, etc. }

“If\marginnote{4.6.2.1} a group with one on probation as the fourth member gives probation, sends back to the beginning, or gives the trial period, or a group with one on probation as the twentieth member rehabilitates, it’s invalid and not to be done. If a group with one deserving to be sent back to the beginning as the fourth member gives probation, sends back to the beginning, or gives the trial period, or a group with one deserving to be sent back to the beginning as the twentieth member rehabilitates, it’s invalid and not to be done. If a group with one deserving the trial period as the fourth member gives probation, sends back to the beginning, or gives the trial period, or a group with one deserving a trial period as the twentieth member rehabilitates, it’s invalid and not to be done. If a group with one undertaking the trial period as the fourth member gives probation, sends back to the beginning, or gives the trial period, or a group with one undertaking a trial period as the twentieth member rehabilitates, it’s invalid and not to be done. If a group with one deserving rehabilitation as the fourth member gives probation, sends back to the beginning, or gives the trial period, or a group with one deserving rehabilitation as the twentieth member rehabilitates, it’s invalid and not to be done. 

In\marginnote{4.7.1} the midst of the Sangha, the objections of some are valid, not the objections of others. Whose objections are invalid in the midst of the Sangha? 

In\marginnote{4.7.3} the midst of the Sangha, the objection of a nun is invalid. In the midst of the Sangha, the objection of a trainee nun, of a novice monk, of a novice nun, of one who’s renounced the training, of one who’s committed the worst kind of offense, of one who’s insane, of one who’s deranged, of one who’s overwhelmed by pain, of one who’s been ejected for not recognizing an offense, of one who’s been ejected for not making amends for an offense, of one who’s been ejected for not giving up a bad view, of a \textit{\textsanskrit{paṇḍaka}}, of a fake monk, of one who’s previously left to join the monastics of another religion, of an animal, of a matricide, of a patricide, of a murderer of a perfected one, of one who’s raped a nun, of one who’s caused a schism in the Sangha, of one who’s caused the Buddha to bleed, of a hermaphrodite, of one who belongs to a different Buddhist sect, of one who’s outside the monastery zone, of one floating in the air by supernormal power, or of one who’s subject to the legal procedure is invalid. 

And\marginnote{4.8.1} whose objections are valid in the midst of the Sangha? 

In\marginnote{4.8.2} the midst of the Sangha, the objection of a regular monk, who belongs to the same Buddhist sect and is staying within the same monastery zone, even if just declared to a monk sitting next to him, is valid. 

\section*{6. Discussion of the two kinds of sending away, etc. }

“There\marginnote{4.9.1} are two kinds of sending away. If the Sangha sends away someone who doesn’t have the attributes needed to be sent away, the sending away may succeed or fail. 

When\marginnote{4.9.4} does it fail? It fails if the monk is pure, without offenses. 

When\marginnote{4.9.7} does it succeed? It succeeds if the monk is ignorant, incompetent, often committing offenses, lacking in boundaries, constantly and improperly socializing with householders.\footnote{According to CPD, apparently quoting the commentary (“Bu”), \textit{\textsanskrit{anapadāna}} means “‘who is unable to discern (what is an offense)’, or ‘not setting a good example’.” It is not clear, however, why \textit{\textsanskrit{apadāna}} should be rendered as “discern”. Sp 3.407: \textit{\textsanskrit{Apadānaṁ} vuccati paricchedo; \textsanskrit{āpattiparicchedavirahitoti} attho}, “Limit is called \textit{\textsanskrit{apadāna}}; the meaning is ‘without limit to offenses’.” Sp-\textsanskrit{ṭ} 3.395: \textit{Natthi etassa \textsanskrit{apadānaṁ} \textsanskrit{avakhaṇḍanaṁ} \textsanskrit{āpattipariyantoti} \textsanskrit{anapadāno}}, “\textit{\textsanskrit{Anapadāno}}: he has no \textit{\textsanskrit{apadāna}}, no cutting off, no limit with offenses.” } 

There\marginnote{4.10.1} are two kinds of admittance. If the Sangha admits someone who doesn’t have the attributes needed to be admitted, the admittance may succeed or fail. 

When\marginnote{4.10.3} does it fail? A \textit{\textsanskrit{paṇḍaka}} doesn’t have the attributes needed to be admitted, and if the Sangha admits him, his admittance fails. A fake monk, one who’s previously left to join the monastics of another religion, an animal, a matricide, a patricide, a murderer of a perfected one, one who’s raped a nun, one who’s caused a schism in the Sangha, one who’s caused the Buddha to bleed, or a hermaphrodite doesn’t have the attributes needed to be admitted, and if the Sangha admits him, his admittance fails. 

When\marginnote{4.11.1} does it succeed? One without a hand doesn’t have the attributes needed to be admitted, but if the Sangha admits him, his admittance succeeds. One without a foot, one without a hand and a foot, one without an ear, one without nose, one without an ear and nose, one without a finger or toe,\footnote{This single phrase combines two Pali terms, \textit{\textsanskrit{aṅgulicchinna}} and \textit{\textsanskrit{aḷacchinna}}. The latter refers to a thumb or a big toe, whereas the former refers to any of the remaining four fingers or toes. } one with a cut tendon, one with joined fingers,\footnote{\textit{\textsanskrit{Phaṇahatthaka}}, literally, “one who has a hand like a snake’s hood”. Sp 3.119: \textit{\textsanskrit{Phaṇahatthakoti} yassa \textsanskrit{vaggulipakkhakā} viya \textsanskrit{aṅguliyo} \textsanskrit{sambaddhā} honti}, “\textit{\textsanskrit{Phaṇahatthako}}: one whose fingers are connected like the wings of a bat.” } a hunchback, a dwarf, one with goiter, one who’s branded, one who’s been whipped, a sentenced criminal,\footnote{\textit{Likhitaka}, literally, “one who has been written about”. Sp 3.119: \textit{Atha kho yo koci \textsanskrit{corikaṁ} \textsanskrit{vā} \textsanskrit{aññaṁ} \textsanskrit{vā} \textsanskrit{garuṁ} \textsanskrit{rājāparādhaṁ} \textsanskrit{katvā} \textsanskrit{palāto}, \textsanskrit{rājā} ca \textsanskrit{naṁ} \textsanskrit{paṇṇe} \textsanskrit{vā} potthake \textsanskrit{vā} “\textsanskrit{itthannāmo} yattha dissati, tattha \textsanskrit{gahetvā} \textsanskrit{māretabbo}”ti \textsanskrit{vā} “\textsanskrit{hatthapādānissa} \textsanskrit{chinditabbānī}”ti \textsanskrit{vā} “\textsanskrit{ettakaṁ} \textsanskrit{nāma} \textsanskrit{daṇḍaṁ} \textsanskrit{āharāpetabbo}”ti \textsanskrit{vā} \textsanskrit{likhāpeti}, \textsanskrit{ayaṁ} likhitako \textsanskrit{nāma}}, “When someone has run away after stealing or doing another serious offense against the king, and the king causes the writing about him on a leaf or in a book that ‘wherever so-and-so is seen, he should be seized and executed’ or ‘his hands and feet are to be cut off’ or ‘this penalty is to be imposed’, this a called a sentenced criminal.” } one with elephantiasis, one with a serious sickness, one with abnormal appearance,\footnote{\textit{\textsanskrit{Parisadūsaka}}, literally, “one who defiles an assembly”. Sp 3.93: \textit{\textsanskrit{Parisadūsakoti} yo attano \textsanskrit{virūpatāya} \textsanskrit{parisaṁ} \textsanskrit{dūseti}; \textsanskrit{atidīgho} \textsanskrit{vā} hoti \textsanskrit{aññesaṁ} \textsanskrit{sīsappamāṇanābhippadeso}, atirasso \textsanskrit{vā} …}, “\textit{\textsanskrit{Parisadūsaka}}: whoever defiles an assembly through his own bad appearance. He is too tall, a head taller than others, or he is too short …” } one blind in one eye, one with a crooked limb, one who’s lame, one paralyzed on one side,\footnote{Sp 3.119: \textit{Pakkhahatoti yassa eko hattho \textsanskrit{vā} \textsanskrit{pādo} \textsanskrit{vā} \textsanskrit{aḍḍhasarīraṁ} \textsanskrit{vā} \textsanskrit{sukhaṁ} na vahati}, “\textit{Pakkhahata}: for whom one hand or one foot or half the body does not work properly.” } one crippled,\footnote{\textit{\textsanskrit{Chinniriyāpatha}}, literally, “the ways of movement have been cut off”. Sp 3.119: \textit{\textsanskrit{Chinniriyāpathoti} \textsanskrit{pīṭhasappi} vuccati}, “One who crawls is called \textit{\textsanskrit{chinniriyāpatha}}.” The exact meaning is not clear. } one weak from old age, one who’s blind, one who’s mute, one who’s deaf, one who’s blind and mute, one who’s blind and deaf, one who’s mute and deaf, or one who’s blind and mute and deaf doesn’t have the attributes needed to be admitted, but if the Sangha admits him, his admittance succeeds.” 

\scend{The first section for recitation on the village of \textsanskrit{Vāsabha} is finished. }

\section*{7. Discussion of illegitimate legal procedures, etc. }

\subsection*{A pure monk}

“It\marginnote{5.1.1.1} may be that a monk doesn’t have any offense he needs to recognize, yet a Sangha, several monks, or an individual monk accuses him, saying:\footnote{“Several” renders \textit{sambahula}. See Appendix of Technical Terms. } ‘You’ve committed an offense. Do you recognize it?’ If he says, ‘I haven’t committed any offense that I should recognize,’ yet the Sangha ejects him for not recognizing an offense, then the legal procedure is illegitimate. 

It\marginnote{5.1.8} may be that a monk doesn’t have any offense he needs to make amends for, yet a Sangha, several monks, or an individual monk accuses him, saying: ‘You’ve committed an offense. Make amends for it.’ If he says, ‘I haven’t committed any offense that I should make amends for,’ yet the Sangha ejects him for not making amends for an offense, then the legal procedure is illegitimate. 

It\marginnote{5.1.15} may be that a monk doesn’t have any bad view he needs to give up, yet a Sangha, several monks, or an individual monk accuses him, saying: ‘You have a bad view that you need to give up.’ If he says, ‘I don’t have any bad view that I should give up,’ yet the Sangha ejects him for not giving up a bad view, then the legal procedure is illegitimate. 

It\marginnote{5.2.1} may be that a monk doesn’t have any offense he needs to recognize, nor any he needs to make amends for, yet a Sangha, several monks, or an individual monk accuses him, saying: ‘You’ve committed an offense. Do you recognize it? Make amends for it.’ If he says, ‘I haven’t committed any offense that I should recognize, nor any I should make amends for,’ yet the Sangha ejects him for not recognizing an offense or for not making amends for it, then the legal procedure is illegitimate. 

It\marginnote{5.3.1} may be that a monk doesn’t have any offense he needs to recognize, nor any bad view he needs to give up, yet a Sangha, several monks, or an individual monk accuses him, saying: ‘You’ve committed an offense. Do you recognize it? And you have a bad view that you need to give up.’ If he says, ‘I haven’t committed any offense that I should recognize, nor do I have any bad view that I should give up,’ yet the Sangha ejects him for not recognizing an offense or for not giving up a bad view, then the legal procedure is illegitimate. 

It\marginnote{5.4.1} may be that a monk doesn’t have any offense he needs to make amends for, nor any bad view he needs to give up, yet a Sangha, several monks, or an individual monk accuses him, saying: ‘You’ve committed an offense. Make amends for it. And you have a bad view that you need to give up.’ If he says, ‘I haven’t committed any offense that I should make amends for, nor do I have any bad view that I should give up,’ yet the Sangha ejects him for not making amends for an offense or for not giving up a bad view, then the legal procedure is illegitimate. 

It\marginnote{5.5.1} may be that a monk doesn’t have any offense he needs to recognize, nor any offense he needs to make amends for, nor any bad view he needs to give up, yet a Sangha, several monks, or an individual monk accuses him, saying: ‘You’ve committed an offense. Do you recognize it? Make amends for it. And you have a bad view that you need to give up.’ If he says, ‘I haven’t committed any offense that I should recognize, nor any I should make amends for, nor do I have any bad view that I should give up,’ yet the Sangha ejects him for not recognizing an offense, for not making amends for an offense, or for not giving up a bad view, then the legal procedure is illegitimate.” 

\subsection*{A monk who recognises his offense, etc.}

“It\marginnote{5.6.1} may be that a monk has an offense he needs to recognize, and a Sangha, several monks, or an individual monk accuses him, saying: ‘You’ve committed an offense. Do you recognize it?’ If he says, ‘I do,’ yet the Sangha ejects him for not recognizing an offense, then the legal procedure is illegitimate. 

It\marginnote{5.6.8} may be that a monk has an offense he needs to make amends for, and a Sangha, several monks, or an individual monk accuses him, saying: ‘You’ve committed an offense. Make amends for it.’ If he says, ‘I will,’ yet the Sangha ejects him for not making amends for an offense, then the legal procedure is illegitimate. 

It\marginnote{5.6.15} may be that a monk has a bad view he needs to give up, and a Sangha, several monks, or an individual monk accuses him, saying: ‘You have a bad view that you need to give up.’ If he says, ‘I’ll give it up,’ yet the Sangha ejects him for not giving up a bad view, then the legal procedure is illegitimate. 

It\marginnote{5.7.1} may be that a monk has an offense he needs to recognize and an offense he needs to make amends for … an offense he needs to recognize and a bad view he needs to give up … an offense he needs to make amends for and a bad view he needs to give up … an offense he needs to recognize, an offense he needs to make amends for, and a bad view he needs to give up, and a Sangha, several monks, or an individual monk accuses him, saying: ‘You’ve committed an offense. Do you recognize it? Make amends for it. And you have a bad view that you need to give up.’ If he says, ‘I recognize it, I’ll make amends for it, and I’ll give up that view,’ yet the Sangha ejects him for not recognizing an offense, for not making amends for an offense, or for not giving up a bad view, then the legal procedure is illegitimate.” 

\subsection*{A monk who does not recognise his offense, etc.}

“It\marginnote{5.8.1} may be that a monk has an offense he needs to recognize, and a Sangha, several monks, or an individual monk accuses him, saying: ‘You’ve committed an offense. Do you recognize it?’ If he says, ‘I haven’t committed any offense that I should recognize,’ and the Sangha ejects him for not recognizing an offense, then the legal procedure is legitimate. 

It\marginnote{5.8.8} may be that a monk has an offense he needs to make amends for, and a Sangha, several monks, or an individual monk accuses him, saying: ‘You’ve committed an offense. Make amends for it.’ If he says, ‘I haven’t committed any offense that I should make amends for,’ and the Sangha ejects him for not making amends for an offense, then the legal procedure is legitimate. 

It\marginnote{5.8.15} may be that a monk has a bad view he needs to give up, and a Sangha, several monks, or an individual monk accuses him, saying: ‘You have a bad view that you need to give up.’ If he says, ‘I don’t have any bad view that I should give up,’ and the Sangha ejects him for not giving up a bad view, then the legal procedure is legitimate. 

It\marginnote{5.9.1} may be that a monk has an offense he needs to recognize and an offense he needs to make amends for … an offense he needs to recognize and a bad view he needs to give up … an offense he needs to make amends for and a bad view he needs to give up … an offense he needs to recognize, an offense he needs to make amends for, and a bad view he needs to give up, and a Sangha, several monks, or an individual monk accuses him, saying: ‘You’ve committed an offense. Do you recognize it? Make amends for it. And you have a bad view that you need to give up.’ If he says, ‘I haven’t committed any offense that I should recognize, nor any I should make amends for, nor do I have any bad view that I should give up,’ and the Sangha ejects him for not recognizing an offense, for not making amends for an offense, or for not giving up a bad view, then the legal procedure is legitimate.” 

\section*{8. The discussion of \textsanskrit{Upāli}’s questions }

On\marginnote{6.1.1} one occasion Venerable \textsanskrit{Upāli} went to the Buddha, bowed, sat down, and said, “If, sir, a unanimous Sangha doesn’t do a legal procedure face-to-face that should be done face-to-face, is that a legitimate procedure, in accordance with the Monastic Law?” 

“That\marginnote{6.1.4} legal procedure, \textsanskrit{Upāli}, is illegitimate, contrary to the Monastic Law.” 

“If\marginnote{6.2.1} a unanimous Sangha does a procedure without questioning that should be done with questioning, does a procedure without admission that should be done with admission, applies resolution because of past insanity to one deserving resolution through recollection, does a procedure of further penalty against one deserving resolution because of past insanity, does a procedure of condemnation against one deserving a procedure of further penalty, does a procedure of demotion against one deserving a procedure of condemnation,\footnote{“Demotion” renders \textit{niyassa}. See Appendix of Technical Terms. } does a procedure of banishment against one deserving a procedure of demotion, does a procedure of reconciliation against one deserving a procedure of banishment, does a procedure of ejection against one deserving a procedure of reconciliation, gives probation to one deserving a procedure of ejection, sends back to the beginning one deserving probation, gives the trial period to one deserving to be sent back to the beginning, rehabilitates one deserving the trial period, or gives full ordination to one deserving rehabilitation, is that a legitimate procedure, in accordance with the Monastic Law?” 

“That\marginnote{6.3.1} legal procedure, \textsanskrit{Upāli}, is illegitimate, contrary to the Monastic Law. If a unanimous Sangha doesn’t do a legal procedure face-to-face that should be done face-to-face, that procedure is illegitimate, contrary to the Monastic Law, and the Sangha is at fault. If a unanimous Sangha does a legal procedure without questioning that should be done with questioning, does a legal procedure without admission that should be done with admission, applies resolution because of past insanity to one deserving resolution through recollection, does a legal procedure of further penalty against one deserving resolution because of past insanity, does a legal procedure of condemnation against one deserving a procedure of further penalty, does a legal procedure of demotion against one deserving a procedure of condemnation, does a legal procedure of banishment against one deserving a procedure of demotion, does a legal procedure of reconciliation against one deserving a procedure of banishment, does a legal procedure of ejection against one deserving a procedure of reconciliation, gives probation to one deserving a procedure of ejection, sends back to the beginning one deserving probation, gives the trial period to one deserving to be sent back to the beginning, rehabilitates one deserving the trial period, or gives full ordination to one deserving rehabilitation, that procedure is illegitimate, contrary to the Monastic Law, and the Sangha is at fault.” 

“But\marginnote{6.4.1} if, sir, a unanimous Sangha does a legal procedure face-to-face that should be done face-to-face, is that a legitimate procedure, in accordance with the Monastic Law?” 

“That\marginnote{6.4.2} legal procedure, \textsanskrit{Upāli}, is legitimate, in accordance with the Monastic Law.” 

“If\marginnote{6.4.3} a unanimous Sangha does a procedure with questioning that should be done with questioning, does a procedure with admission that should be done with admission, applies resolution through recollection to one deserving resolution through recollection, applies resolution because of past insanity to one deserving resolution because of past insanity, does a procedure of further penalty against one deserving a procedure of further penalty, does a procedure of condemnation against one deserving a procedure of condemnation, does a procedure of demotion against one deserving a procedure of demotion, does a procedure of banishment against one deserving a procedure of banishment, does a procedure of reconciliation against one deserving a procedure of reconciliation, does a procedure of ejection against one deserving a procedure of ejection, gives probation to one deserving probation,  sends back to the beginning one deserving to be sent back to the beginning, gives the trial period to one deserving the trial period, rehabilitates one deserving rehabilitation, or gives full ordination to one deserving full ordination, is that a legitimate procedure, in accordance with the Monastic Law?” 

“That\marginnote{6.4.18} legal procedure, \textsanskrit{Upāli}, is legitimate, in accordance with the Monastic Law. If a unanimous Sangha does a legal procedure face-to-face that should be done face-to-face, that procedure is legitimate, in accordance with the Monastic Law, and the Sangha isn’t at fault. If a unanimous Sangha does a legal procedure with questioning that should be done with questioning, does a legal procedure with admission that should be done with admission, applies resolution through recollection to one deserving resolution through recollection, applies resolution because of past insanity to one deserving resolution because of past insanity, does a legal procedure of further penalty against one deserving a procedure of further penalty, does a legal procedure of condemnation against one deserving a procedure of condemnation, does a legal procedure of demotion against one deserving a procedure of demotion, does a legal procedure of banishment against one deserving a procedure of banishment, does a legal procedure of reconciliation against one deserving a procedure of reconciliation, does a legal procedure of ejection against one deserving a procedure of ejection, gives probation to one deserving probation, sends back to the beginning one deserving to be sent back to the beginning, gives the trial period to one deserving the trial period, rehabilitates one deserving rehabilitation, or gives full ordination to one deserving full ordination, that procedure is legitimate, in accordance with the Monastic Law, and the Sangha isn’t at fault.” 

“If,\marginnote{6.5.1} sir, a unanimous Sangha applies resolution because of past insanity to one deserving resolution through recollection and applies resolution through recollection to one deserving resolution because of past insanity, is that a legitimate procedure, in accordance with the Monastic Law?”\footnote{The text is not clear about the relationship between the two applications of resolution, but usually an “and” is the default conjunction if nothing else is specified in the text. This seems to be confirmed by the commentary. Sp 3.400: \textit{\textsanskrit{Dvimūlake} \textsanskrit{yathā} sativinayo \textsanskrit{amūḷhavinayena} \textsanskrit{saddhiṁ} \textsanskrit{ekā} \textsanskrit{pucchā} \textsanskrit{katā}}, “When there are two items, as with resolution through recollection together with resolution through past insanity, a single question is posed.” In other words, the \textit{\textsanskrit{saṅghakamma}} seems to be against two individuals together. } 

“That\marginnote{6.5.2} legal procedure, \textsanskrit{Upāli}, is illegitimate, contrary to the Monastic Law.” 

“If\marginnote{6.5.3} a unanimous Sangha does a procedure of further penalty against one deserving resolution because of past insanity and applies resolution because of past insanity to one deserving a procedure of further penalty, does a procedure of condemnation against one deserving a procedure of further penalty and does a procedure of further penalty against one deserving a procedure of condemnation, does a procedure of demotion against one deserving a procedure of condemnation and does a procedure of condemnation against one deserving a procedure of demotion, does a procedure of banishment against one deserving a procedure of demotion and does a procedure of demotion against one deserving a procedure of banishment, does a procedure of reconciliation against one deserving a procedure of banishment and does a procedure of banishment against one deserving a procedure of reconciliation, does a procedure of ejection against one deserving a procedure of reconciliation and does a procedure of reconciliation against one deserving a procedure of ejection, gives probation to one deserving a procedure of ejection and does a procedure of ejection against one deserving probation, sends back to the beginning one deserving probation and gives probation to one deserving to be sent back to the beginning, gives the trial period to one deserving to be sent back to the beginning and sends back to the beginning one deserving the trial period, rehabilitates one deserving the trial period and gives the trial period to one deserving rehabilitation, or gives full ordination to one deserving rehabilitation and rehabilitates one deserving to be given full ordination, is that a legitimate procedure, in accordance with the Monastic Law?” 

“That\marginnote{6.6.1} legal procedure, \textsanskrit{Upāli}, is illegitimate, contrary to the Monastic Law. If a unanimous Sangha applies resolution because of past insanity to one deserving resolution through recollection and applies resolution through recollection to one deserving resolution because of past insanity, that procedure is illegitimate, contrary to the Monastic Law, and the Sangha is at fault. If a unanimous Sangha does a legal procedure of further penalty against one deserving resolution because of past insanity and applies resolution because of past insanity to one deserving a procedure of further penalty, does a legal procedure of condemnation against one deserving a procedure of further penalty and does a procedure of further penalty against one deserving a procedure of condemnation, does a legal procedure of demotion against one deserving a procedure of condemnation and does a procedure of condemnation against one deserving a procedure of demotion, does a legal procedure of banishment against one deserving a procedure of demotion and does a procedure of demotion against one deserving a procedure of banishment, does a legal procedure of reconciliation against one deserving a procedure of banishment and does a procedure of banishment against one deserving a procedure of reconciliation, does a legal procedure of ejection against one deserving a procedure of reconciliation and does a procedure of reconciliation against one deserving a procedure of ejection, gives probation to one deserving a procedure of ejection and does a procedure of ejection against one deserving probation, sends back to the beginning one deserving probation and gives probation to one deserving to be sent back to the beginning, gives the trial period to one deserving to be sent back to the beginning and sends back to the beginning one deserving the trial period, rehabilitates one deserving the trial period and gives the trial period to one deserving rehabilitation, gives full ordination to one deserving rehabilitation and rehabilitates one deserving full ordination, that procedure is illegitimate, contrary to the Monastic Law, and the Sangha is at fault.” 

“But\marginnote{6.7.1} if, sir, a unanimous Sangha applies resolution through recollection to one deserving resolution through recollection and applies resolution because of past insanity to one deserving resolution because of past insanity, is that a legitimate procedure, in accordance with the Monastic Law?” 

“That\marginnote{6.7.2} legal procedure, \textsanskrit{Upāli}, is legitimate, in accordance with the Monastic Law.” 

“If\marginnote{6.7.3} a unanimous Sangha applies resolution because of past insanity to one deserving resolution because of past insanity, does a procedure of further penalty against one deserving a procedure of further penalty, does a procedure of condemnation against one deserving a procedure of condemnation, does a procedure of demotion against one deserving a procedure of demotion, does a procedure of banishment against one deserving a procedure of banishment, does a procedure of reconciliation against one deserving a procedure of reconciliation, does a procedure of ejection against one deserving a procedure of ejection, gives probation to one deserving probation, sends back to the beginning one deserving to be sent back to the beginning, gives the trial period to one deserving the trial period, or rehabilitates one deserving rehabilitation and gives full ordination to one deserving full ordination, is that a legitimate procedure, in accordance with the Monastic Law?” 

“That\marginnote{6.8.1} legal procedure, \textsanskrit{Upāli}, is legitimate, in accordance with the Monastic Law. If a unanimous Sangha applies resolution through recollection to one deserving resolution through recollection and applies resolution because of past insanity to one deserving resolution because of past insanity, that procedure is legitimate, in accordance with the Monastic Law, and the Sangha isn’t at fault. If a unanimous Sangha applies resolution because of past insanity to one deserving resolution because of past insanity, does a procedure of further penalty against one deserving a procedure of further penalty, does a procedure of condemnation against one deserving a procedure of condemnation, does a procedure of demotion against one deserving a procedure of demotion, does a procedure of banishment against one deserving a procedure of banishment, does a procedure of reconciliation against one deserving a procedure of reconciliation, does a procedure of ejection against one deserving a procedure of ejection, gives probation to one deserving probation, sends back to the beginning one deserving to be sent back to the beginning, gives the trial period to one deserving the trial period, or rehabilitates one deserving rehabilitation and gives full ordination to one deserving full ordination, that procedure is legitimate, in accordance with the Monastic Law, and the Sangha isn’t at fault.” 

Soon\marginnote{6.9.1} afterwards the Buddha addressed the monks: “If a unanimous Sangha applies resolution because of past insanity to one deserving resolution through recollection, that procedure is illegitimate, contrary to the Monastic Law, and the Sangha is at fault. If a unanimous Sangha does a legal procedure of further penalty against one deserving resolution through recollection, does a legal procedure of condemnation against one deserving resolution through recollection, does a legal procedure of demotion against one deserving resolution through recollection, does a legal procedure of banishment against one deserving resolution through recollection, does a legal procedure of reconciliation against one deserving resolution through recollection, does a legal procedure of ejection against one deserving resolution through recollection, gives probation to one deserving resolution through recollection, sends back to the beginning one deserving resolution through recollection, gives the trial period to one deserving resolution through recollection, rehabilitates one deserving resolution through recollection, gives full ordination to one deserving resolution through recollection, that procedure is illegitimate, contrary to the Monastic Law, and the Sangha is at fault. 

If\marginnote{6.9.16} a unanimous Sangha does a legal procedure of further penalty against one deserving resolution because of past insanity, that procedure is illegitimate, contrary to the Monastic Law, and the Sangha is at fault. If a unanimous Sangha does a legal procedure of condemnation against one deserving resolution because of past insanity, does a legal procedure of demotion against one deserving resolution because of past insanity, does a legal procedure of banishment against one deserving resolution because of past insanity, does a legal procedure of reconciliation against one deserving resolution because of past insanity, does a legal procedure of ejection against one deserving resolution because of past insanity, gives probation to one deserving resolution because of past insanity, sends back to the beginning one deserving resolution because of past insanity, gives the trial period to one deserving resolution because of past insanity, rehabilitates one deserving resolution because of past insanity, gives full ordination to one deserving resolution because of past insanity, or applies resolution through recollection to one deserving resolution because of past insanity, that procedure is illegitimate, contrary to the Monastic Law, and the Sangha is at fault. 

If\marginnote{6.9.30} a unanimous Sangha does a legal procedure of condemnation against one deserving a procedure of further penalty … against one deserving a procedure of condemnation … against one deserving a procedure of demotion … against one deserving a procedure of banishment … against one deserving a procedure of reconciliation … against one deserving a procedure of ejection … against one deserving probation … against one deserving to be sent back to the beginning … against one deserving the trial period … against one deserving rehabilitation … or applies resolution through recollection to one deserving full ordination, that procedure is illegitimate, contrary to the Monastic Law, and the Sangha is at fault. 

If\marginnote{6.9.42} a unanimous Sangha applies resolution because of past insanity to one deserving full ordination, does a legal procedure of further penalty against one deserving full ordination, does a legal procedure of condemnation against one deserving full ordination, does a legal procedure of demotion against one deserving full ordination, does a legal procedure of banishment against one deserving full ordination, does a legal procedure of reconciliation against one deserving full ordination, does a legal procedure of ejection against one deserving full ordination, gives probation to one deserving full ordination, sends back to the beginning one deserving full ordination, gives the trial period to one deserving full ordination, or rehabilitates one deserving full ordination, that procedure is illegitimate, contrary to the Monastic Law, and the Sangha is at fault.” 

\scend{The second section for recitation on \textsanskrit{Upāli}’s questions is finished. }

\section*{9. Discussion of the legal procedure of condemnation }

“It\marginnote{7.1.1} may be, monks, that a monk is quarrelsome and argumentative, one who creates legal issues in the Sangha. The monks consider, ‘This monk is quarrelsome and argumentative, one who creates legal issues in the Sangha. Well then, let’s do a legal procedure of condemnation against him.’ They do a procedure of condemnation against him—illegitimately and with an incomplete assembly. 

He\marginnote{7.1.7} then goes to another monastery. There too the monks consider, ‘The Sangha did a legal procedure of condemnation against this monk—illegitimately and with an incomplete assembly. Well then, let’s do a procedure of condemnation against him.’ They do a procedure of condemnation against him—illegitimately but with a unanimous assembly. 

He\marginnote{7.1.13} then goes to yet another monastery. There too the monks consider, ‘The Sangha did a legal procedure of condemnation against this monk—illegitimately but with a unanimous assembly. Well then, let’s do a procedure of condemnation against him.’ They do a procedure of condemnation against him—legitimately but with an incomplete assembly. 

He\marginnote{7.1.19} then goes to yet another monastery. There too the monks consider, ‘The Sangha did a legal procedure of condemnation against this monk—legitimately but with an incomplete assembly. Well then, let’s do a procedure of condemnation against him.’ They do a procedure of condemnation against him—in a legitimate-like way and with an incomplete assembly. 

He\marginnote{7.1.25} then goes to yet another monastery. There too the monks consider, ‘The Sangha did a legal procedure of condemnation against this monk—in a legitimate-like way and with an incomplete assembly. Well then, let’s do a procedure of condemnation against him.’ They do a procedure of condemnation against him—in a legitimate-like way but with a unanimous assembly. 

“It\marginnote{7.2.1} may be that a monk is quarrelsome and argumentative, one who creates legal issues in the Sangha. The monks consider, ‘This monk is quarrelsome and argumentative, one who creates legal issues in the Sangha. Well then, let’s do a legal procedure of condemnation against him.’ They do a procedure of condemnation against him—illegitimately but with a unanimous assembly. 

He\marginnote{7.2.7} then goes to another monastery. There too the monks consider, ‘The Sangha did a legal procedure of condemnation against this monk—illegitimately but with a unanimous assembly. Well then, let’s do a procedure of condemnation against him.’ They do a procedure of condemnation against him—legitimately but with an incomplete assembly. 

He\marginnote{7.2.13} then goes to yet another monastery. There too the monks consider, ‘The Sangha did a legal procedure of condemnation against this monk—legitimately but with an incomplete assembly. Well then, let’s do a procedure of condemnation against him.’ They do a procedure of condemnation against him—in a legitimate-like way and with an incomplete assembly. 

He\marginnote{7.2.19} then goes to yet another monastery. There too the monks consider, ‘The Sangha did a legal procedure of condemnation against this monk—in a legitimate-like way and with an incomplete assembly. Well then, let’s do a procedure of condemnation against him.’ They do a procedure of condemnation against him—in a legitimate-like way but with a unanimous assembly. 

He\marginnote{7.2.25} then goes to yet another monastery. There too the monks consider, ‘The Sangha did a legal procedure of condemnation against this monk—in a legitimate-like way but with a unanimous assembly. Well then, let’s do a procedure of condemnation against him.’ They do a procedure of condemnation against him—illegitimately and with an incomplete assembly. 

“It\marginnote{7.3.1} may be that a monk is quarrelsome and argumentative, one who creates legal issues in the Sangha. The monks consider, ‘This monk is quarrelsome and argumentative, one who creates legal issues in the Sangha. Well then, let’s do a legal procedure of condemnation against him.’ They do a procedure of condemnation against him—legitimately but with an incomplete assembly. 

He\marginnote{7.3.7} then goes to another monastery. There too the monks consider, ‘The Sangha did a legal procedure of condemnation against this monk—legitimately but with an incomplete assembly. Well then, let’s do a procedure of condemnation against him.’ They do a procedure of condemnation against him—in a legitimate-like way and with an incomplete assembly. 

He\marginnote{7.3.13} then goes to yet another monastery. There too the monks consider, ‘The Sangha did a legal procedure of condemnation against this monk—in a legitimate-like way and with an incomplete assembly. Well then, let’s do a procedure of condemnation against him.’ They do a procedure of condemnation against him—in a legitimate-like way but with a unanimous assembly. 

He\marginnote{7.3.19} then goes to yet another monastery. There too the monks consider, ‘The Sangha did a legal procedure of condemnation against this monk—in a legitimate-like way but with a unanimous assembly. Well then, let’s do a procedure of condemnation against him.’ They do a procedure of condemnation against him—illegitimately and with an incomplete assembly. 

He\marginnote{7.3.25} then goes to yet another monastery. There too the monks consider, ‘The Sangha did a legal procedure of condemnation against this monk—illegitimately and with an incomplete assembly. Well then, let’s do a procedure of condemnation against him.’ They do a procedure of condemnation against him—illegitimately but with a unanimous assembly. 

“It\marginnote{7.4.1} may be that a monk is quarrelsome and argumentative, one who creates legal issues in the Sangha. The monks consider, ‘This monk is quarrelsome and argumentative, one who creates legal issues in the Sangha. Well then, let’s do a legal procedure of condemnation against him.’ They do a procedure of condemnation against him—in a legitimate-like way and with an incomplete assembly. 

He\marginnote{7.4.7} then goes to another monastery. There too the monks consider, ‘The Sangha did a legal procedure of condemnation against this monk—in a legitimate-like way and with an incomplete assembly. Well then, let’s do a procedure of condemnation against him.’ They do a procedure of condemnation against him—in a legitimate-like way but with a unanimous assembly. 

He\marginnote{7.4.13} then goes to yet another monastery. There too the monks consider, ‘The Sangha did a legal procedure of condemnation against this monk—in a legitimate-like way with a unanimous assembly. Well then, let’s do a procedure of condemnation against him.’ They do a procedure of condemnation against him—illegitimately and with an incomplete assembly. 

He\marginnote{7.4.19} then goes to yet another monastery. There too the monks consider, ‘The Sangha did a legal procedure of condemnation against this monk—illegitimately and with an incomplete assembly. Well then, let’s do a procedure of condemnation against him.’ They do a procedure of condemnation against him—illegitimately but with a unanimous assembly. 

He\marginnote{7.4.25} then goes to yet another monastery. There too the monks consider, ‘The Sangha did a legal procedure of condemnation against this monk—illegitimately but with a unanimous assembly. Well then, let’s do a procedure of condemnation against him.’ They do a procedure of condemnation against him—legitimately but with an incomplete assembly. 

“It\marginnote{7.5.1} may be that a monk is quarrelsome and argumentative, one who creates legal issues in the Sangha. The monks consider, ‘This monk is quarrelsome and argumentative, one who creates legal issues in the Sangha. Well then, let’s do a legal procedure of condemnation against him.’ They do a procedure of condemnation against him—in a legitimate-like way but with a unanimous assembly. 

He\marginnote{7.5.7} then goes to another monastery. There too the monks consider, ‘The Sangha did a legal procedure of condemnation against this monk—in a legitimate-like way but with a unanimous assembly. Well then, let’s do a procedure of condemnation against him.’ They do a procedure of condemnation against him—illegitimately and with an incomplete assembly. 

He\marginnote{7.5.13} then goes to yet another monastery. There too the monks consider, ‘The Sangha did a legal procedure of condemnation against this monk—illegitimately and with an incomplete assembly. Well then, let’s do a procedure of condemnation against him.’ They do a procedure of condemnation against him—illegitimately but with a unanimous assembly. 

He\marginnote{7.5.19} then goes to yet another monastery. There too the monks consider, ‘The Sangha did a legal procedure of condemnation against this monk—illegitimately but with a unanimous assembly. Well then, let’s do a procedure of condemnation against him.’ They do a procedure of condemnation against him—legitimately but with an incomplete assembly. 

He\marginnote{7.5.25} then goes to yet another monastery. There too the monks consider, ‘The Sangha did a legal procedure of condemnation against this monk—legitimately but with an incomplete assembly. Well then, let’s do a procedure of condemnation against him.’ They do a procedure of condemnation against him—in a legitimate-like way and with an incomplete assembly.” 

\section*{10. Discussion of the legal procedure of demotion }

“It\marginnote{7.6.1} may be, monks, that a monk is ignorant, incompetent, often committing offenses, lacking in boundaries, constantly and improperly socializing with householders. The monks consider, ‘This monk is ignorant, incompetent, often committing offenses, lacking in boundaries, constantly and improperly socializing with householders. Well then, let’s do a legal procedure of demotion against him.’ They do a procedure of demotion against him—illegitimately and with an incomplete assembly. 

He\marginnote{7.6.7} then goes to another monastery. There too the monks consider, ‘The Sangha did a legal procedure of demotion against this monk—illegitimately and with an incomplete assembly. Well then, let’s do a procedure of demotion against him.’ They do a procedure of demotion against him—illegitimately but with a unanimous assembly. … legitimately but with an incomplete assembly. … in a legitimate-like way and with an incomplete assembly. … in a legitimate-like way but with a unanimous assembly. …” 

The\marginnote{7.6.16} permutation series is to be expanded as above. 

\section*{11. Discussion of the legal procedure of banishment }

“It\marginnote{7.7.1} may be that a monk is a corrupter of families and badly behaved. The monks consider, ‘This monk is a corrupter of families and badly behaved. Well then, let’s do a procedure of banishment against him.’ They do a procedure of banishment against him—illegitimately and with an incomplete assembly. 

He\marginnote{7.7.7} then goes to another monastery. There too the monks consider, ‘The Sangha did a legal procedure of banishment against this monk—illegitimately and with an incomplete assembly. Well then, let’s do a procedure of banishment against him.’ They do a procedure of banishment against him—illegitimately but with a unanimous assembly. … legitimately but with an incomplete assembly. … in a legitimate-like way and with an incomplete assembly. … in a legitimate-like way but with a unanimous assembly. …” 

The\marginnote{7.7.16} permutation series is to be expanded. 

\section*{12. Discussion of the legal procedure of reconciliation }

“It\marginnote{7.8.1} may be that a monk abuses and reviles householders. The monks consider, ‘This monk abuses and reviles householders. Well then, let’s do a procedure of reconciliation against him.’ They do a procedure of reconciliation against him—illegitimately and with an incomplete assembly. 

He\marginnote{7.8.7} then goes to another monastery. There too the monks consider, ‘The Sangha did a legal procedure of reconciliation against this monk—illegitimately and with an incomplete assembly. Well then, let’s do a procedure of reconciliation against him.’ They do a procedure of reconciliation against him—illegitimately but with a unanimous assembly. … legitimately but with an incomplete assembly. … in a legitimate-like way and with an incomplete assembly. … in a legitimate-like way but with a unanimous assembly. …” 

The\marginnote{7.8.16} permutation series is to be expanded. 

\section*{13. Discussion of the legal procedure of ejection for not recognizing }

“It\marginnote{7.9.1} may be that a monk commits an offense but refuses to recognize it. The monks consider, ‘This monk has committed an offense but refuses to recognize it. Well then, let’s do a procedure of ejection against him for not recognizing an offense.’ They do a procedure of ejection against him—illegitimately and with an incomplete assembly. 

He\marginnote{7.9.7} then goes to another monastery. There too the monks consider, ‘The Sangha did a legal procedure of ejection against this monk for not recognizing an offense—illegitimately and with an incomplete assembly. Well then, let’s do a procedure of ejection against him.’ They do a procedure of ejection against him—illegitimately but with a unanimous assembly. … legitimately but with an incomplete assembly. … in a legitimate-like way and with an incomplete assembly. … in a legitimate-like way but with a unanimous assembly. …” 

The\marginnote{7.9.16} permutation series is to be expanded. 

\section*{14. Discussion of the legal procedure of ejection for not making amends }

“It\marginnote{7.10.1} may be that a monk commits an offense but refuses to make amends for it. The monks consider, ‘This monk has committed an offense but refuses to make amends for it. Well then, let’s do a procedure of ejection against him for not making amends for an offense.’ They do a procedure of ejection against him—illegitimately and with an incomplete assembly. 

He\marginnote{7.10.7} then goes to another monastery. There too the monks consider, ‘The Sangha did a legal procedure of ejection against this monk for not making amends for an offense—illegitimately and with an incomplete assembly. Well then, let’s do a procedure of ejection against him.’ They do a procedure of ejection against him—illegitimately but with a unanimous assembly. … legitimately but with an incomplete assembly. … in a legitimate-like way and with an incomplete assembly. … in a legitimate-like way but with a unanimous assembly. …” 

The\marginnote{7.10.16} permutation series is to be expanded. 

\section*{15. Discussion of the legal procedure of ejection for not giving up a bad view }

“It\marginnote{7.11.1} may be that a monk refuses to give up a bad view. The monks consider, ‘This monk refuses to give up a bad view. Well then, let’s do a procedure of ejection against him for not giving up a bad view.’ They do a procedure of ejection against him—illegitimately and with an incomplete assembly. 

He\marginnote{7.11.7} then goes to another monastery. There too the monks consider, ‘The Sangha did a legal procedure of ejection against this monk for not giving up a bad view—illegitimately and with an incomplete assembly. Well then, let’s do a procedure of ejection against him.’ They do a procedure of ejection against him—illegitimately but with a unanimous assembly. … legitimately but with an incomplete assembly. … in a legitimate-like way and with an incomplete assembly. … in a legitimate-like way but with a unanimous assembly. …” 

The\marginnote{7.11.16} permutation series is to be expanded. 

\section*{16. Discussion of the lifting of the legal procedure of condemnation }

“It\marginnote{7.12.1} may be, monks, that the Sangha has done a legal procedure of condemnation against a monk, that he then conducts himself properly and suitably so as to deserve to be released, and that he then asks for the lifting of that procedure.\footnote{The meaning of the first of these phrases, \textit{\textsanskrit{sammā} vattati}, is straightforward, but the last two, \textit{\textsanskrit{lomaṁ} \textsanskrit{pāteti}} and \textit{\textsanskrit{netthāraṁ} vattati}, are more difficult. Commenting on Bu Ss 13, Sp 1.435 says: \textit{Na \textsanskrit{lomaṁ} \textsanskrit{pātentīti} \textsanskrit{anulomapaṭipadaṁ} \textsanskrit{appaṭipajjanatāya} na \textsanskrit{pannalomā} honti. Na \textsanskrit{netthāraṁ} \textsanskrit{vattantīti} attano \textsanskrit{nittharaṇamaggaṁ} na \textsanskrit{paṭipajjanti}}, “\textit{Na \textsanskrit{lomaṁ} \textsanskrit{pātenti}}: because of their non-practicing in conformity with the path, their bodily hairs are not flat. \textit{Na \textsanskrit{netthāraṁ} vattanti}: they are not practicing the path for their own getting out (of the offense).” My rendering attempts to capture the meaning in a non-literal way. } The monks consider, ‘The Sangha has done a legal procedure of condemnation against this monk. He has conducted himself properly and suitably so as to deserve to be released, and now asks for the lifting of that procedure. Well then, let’s lift that procedure.’ They lift that procedure—illegitimately and with an incomplete assembly. 

He\marginnote{7.12.7} then goes to another monastery. There too the monks consider, ‘The Sangha has lifted a legal procedure of condemnation against this monk—illegitimately and with an incomplete assembly. Well then, let’s lift that procedure.’ They lift that procedure—illegitimately but with a unanimous assembly. 

He\marginnote{7.12.13} then goes to yet another monastery. There too the monks consider, ‘The Sangha has lifted a legal procedure of condemnation against this monk—illegitimately but with a unanimous assembly. Well then, let’s lift that procedure.’ They lift that procedure—legitimately but with an incomplete assembly. 

He\marginnote{7.12.19} then goes to yet another monastery. There too the monks consider, ‘The Sangha has lifted a legal procedure of condemnation against this monk—legitimately but with an incomplete assembly. Well then, let’s lift that procedure.’ They lift that procedure—in a legitimate-like way and with an incomplete assembly. 

He\marginnote{7.12.25} then goes to yet another monastery. There too the monks consider, ‘The Sangha has lifted a legal procedure of condemnation against this monk—in a legitimate-like way and with an incomplete assembly. Well then, let’s lift that procedure.’ They lift that procedure—in a legitimate-like way but with a unanimous assembly. 

“It\marginnote{7.13.1} may be that the Sangha has done a legal procedure of condemnation against a monk, that he then conducts himself properly and suitably so as to deserve to be released, and that he then asks for the lifting of that procedure. The monks consider, ‘The Sangha has done a legal procedure of condemnation against this monk. He has conducted himself properly and suitably so as to deserve to be released, and now asks for the lifting of that procedure. Well then, let’s lift that procedure.’ They lift that procedure—illegitimately but with a unanimous assembly. 

He\marginnote{7.13.7} then goes to another monastery. There too the monks consider, ‘The Sangha has lifted a legal procedure of condemnation against this monk—illegitimately but with a unanimous assembly. Well then, let’s lift that procedure.’ They lift that procedure—legitimately but with an incomplete assembly. 

He\marginnote{7.13.13} then goes to yet another monastery. There too the monks consider, ‘The Sangha has lifted a legal procedure of condemnation against this monk—legitimately but with an incomplete assembly. Well then, let’s lift that procedure.’ They lift that procedure—in a legitimate-like way and with an incomplete assembly. 

He\marginnote{7.13.19} then goes to yet another monastery. There too the monks consider, ‘The Sangha has lifted a legal procedure of condemnation against this monk—in a legitimate-like way and with an incomplete assembly. Well then, let’s lift that procedure.’ They lift that procedure—in a legitimate-like way but with a unanimous assembly. 

He\marginnote{7.13.25} then goes to yet another monastery. There too the monks consider, ‘The Sangha has lifted a legal procedure of condemnation against this monk—in a legitimate-like way but with a unanimous assembly. Well then, let’s lift that procedure.’ They lift that procedure—illegitimately and with an incomplete assembly. 

“It\marginnote{7.13.31} may be that the Sangha has done a legal procedure of condemnation against a monk, that he then conducts himself properly and suitably so as to deserve to be released, and that he then asks for the lifting of that procedure. The monks consider, ‘The Sangha has done a legal procedure of condemnation against this monk. He has conducted himself properly and suitably so as to deserve to be released, and now asks for the lifting of that procedure. Well then, let’s lift that procedure.’ They lift that procedure—legitimately but with an incomplete assembly. 

He\marginnote{7.13.37} then goes to another monastery. There too the monks consider, ‘The Sangha has lifted a legal procedure of condemnation against this monk—legitimately but with an incomplete assembly. Well then, let’s lift that procedure.’ They lift that procedure—in a legitimate-like way and with an incomplete assembly. 

He\marginnote{7.13.43} then goes to yet another monastery. There too the monks consider, ‘The Sangha has lifted a legal procedure of condemnation against this monk—in a legitimate-like way and with an incomplete assembly. Well then, let’s lift that procedure.’ They lift that procedure—in a legitimate-like way but with a unanimous assembly. 

He\marginnote{7.13.49} then goes to yet another monastery. There too the monks consider, ‘The Sangha has lifted a legal procedure of condemnation against this monk—in a legitimate-like way but with a unanimous assembly. Well then, let’s lift that procedure.’ They lift that procedure—illegitimately and with an incomplete assembly. 

He\marginnote{7.13.55} then goes to yet another monastery. There too the monks consider, ‘The Sangha has lifted a legal procedure of condemnation against this monk—illegitimately and with an incomplete assembly. Well then, let’s lift that procedure.’ They lift that procedure—illegitimately but with a unanimous assembly. 

“It\marginnote{7.13.61} may be that the Sangha has done a legal procedure of condemnation against a monk, that he then conducts himself properly and suitably so as to deserve to be released, and that he then asks for the lifting of that procedure. The monks consider, ‘The Sangha has done a legal procedure of condemnation against this monk. He has conducted himself properly and suitably so as to deserve to be released, and now asks for the lifting of that procedure. Well then, let’s lift that procedure.’ They lift that procedure—in a legitimate-like way and with an incomplete assembly. 

He\marginnote{7.13.67} then goes to another monastery. There too the monks consider, ‘The Sangha has lifted a legal procedure of condemnation against this monk—in a legitimate-like way and with an incomplete assembly. Well then, let’s lift that procedure.’ They lift that procedure—in a legitimate-like way but with a unanimous assembly. 

He\marginnote{7.13.73} then goes to yet another monastery. There too the monks consider, ‘The Sangha has lifted a legal procedure of condemnation against this monk—in a legitimate-like way but with a unanimous assembly. Well then, let’s lift that procedure.’ They lift that procedure—illegitimately and with an incomplete assembly. 

He\marginnote{7.13.79} then goes to yet another monastery. There too the monks consider, ‘The Sangha has lifted a legal procedure of condemnation against this monk—illegitimately and with an incomplete assembly. Well then, let’s lift that procedure.’ They lift that procedure—illegitimately but with a unanimous assembly. 

He\marginnote{7.13.85} then goes to yet another monastery. There too the monks consider, ‘The Sangha has lifted a legal procedure of condemnation against this monk—illegitimately but with a unanimous assembly. Well then, let’s lift that procedure.’ They lift that procedure—legitimately but with an incomplete assembly. 

“It\marginnote{7.13.91} may be that the Sangha has done a legal procedure of condemnation against a monk, that he then conducts himself properly and suitably so as to deserve to be released, and that he then asks for the lifting of that procedure. The monks consider, ‘The Sangha has done a legal procedure of condemnation against this monk. He has conducted himself properly and suitably so as to deserve to be released, and now asks for the lifting of that procedure. Well then, let’s lift that procedure.’ They lift that procedure—in a legitimate-like way but with a unanimous assembly. 

He\marginnote{7.13.97} then goes to another monastery. There too the monks consider, ‘The Sangha has lifted a legal procedure of condemnation against this monk—in a legitimate-like way but with a unanimous assembly. Well then, let’s lift that procedure.’ They lift that procedure—illegitimately and with an incomplete assembly. 

He\marginnote{7.13.103} then goes to yet another monastery. There too the monks consider, ‘The Sangha has lifted a legal procedure of condemnation against this monk—illegitimately and with an incomplete assembly. Well then, let’s lift that procedure.’ They lift that procedure—illegitimately but with a unanimous assembly. 

He\marginnote{7.13.109} then goes to yet another monastery. There too the monks consider, ‘The Sangha has lifted a legal procedure of condemnation against this monk—illegitimately but with a unanimous assembly. Well then, let’s lift that procedure.’ They lift that procedure—legitimately but with an incomplete assembly. 

He\marginnote{7.13.115} then goes to yet another monastery. There too the monks consider, ‘The Sangha has lifted a legal procedure of condemnation against this monk—legitimately but with an incomplete assembly. Well then, let’s lift that procedure.’ They lift that procedure—in a legitimate-like way and with an incomplete assembly.” 

\section*{17. Discussion of the lifting of the legal procedure of demotion }

“It\marginnote{7.14.1} may be, monks, that the Sangha has done a legal procedure of demotion against a monk, that he then conducts himself properly and suitably so as to deserve to be released, and that he then asks for the lifting of that procedure. The monks consider, ‘The Sangha has done a legal procedure of demotion against this monk. He has conducted himself properly and suitably so as to deserve to be released, and now asks for the lifting of that procedure. Well then, let’s lift that procedure.’ They lift that procedure—illegitimately and with an incomplete assembly. 

He\marginnote{7.14.7} then goes to another monastery. There too the monks consider, ‘The Sangha has lifted a legal procedure of demotion against this monk—illegitimately and with an incomplete assembly. Well then, let’s lift that procedure.’ They lift that procedure—illegitimately but with a unanimous assembly. … legitimately but with an incomplete assembly. … in a legitimate-like way and with an incomplete assembly. … in a legitimate-like way but with a unanimous assembly. …” 

The\marginnote{7.14.16} permutation series is to be expanded. 

\section*{18. Discussion of the lifting of the legal procedure of banishment }

“It\marginnote{7.14.17.1} may be that the Sangha has done a legal procedure of banishment against a monk, that he then conducts himself properly and suitably so as to deserve to be released, and that he then asks for the lifting of that procedure. The monks consider, ‘The Sangha has done a legal procedure of banishment against this monk. He has conducted himself properly and suitably so as to deserve to be released, and now asks for the lifting of that procedure. Well then, let’s lift that procedure.’ They lift that procedure—illegitimately and with an incomplete assembly. 

He\marginnote{7.14.23} then goes to another monastery. There too the monks consider, ‘The Sangha has lifted a legal procedure of banishment against this monk—illegitimately and with an incomplete assembly. Well then, let’s lift that procedure.’ They lift that procedure—illegitimately but with a unanimous assembly. … legitimately but with an incomplete assembly. … in a legitimate-like way and with an incomplete assembly. … in a legitimate-like way but with a unanimous assembly. …” 

The\marginnote{7.14.32} permutation series is to be expanded. 

\section*{19. Discussion of the lifting of the legal procedure of reconciliation }

“It\marginnote{7.14.33.1} may be that the Sangha has done a legal procedure of reconciliation against a monk, that he then conducts himself properly and suitably so as to deserve to be released, and that he then asks for the lifting of that procedure. The monks consider, ‘The Sangha has done a legal procedure of reconciliation against this monk. He has conducted himself properly and suitably so as to deserve to be released, and now asks for the lifting of that procedure. Well then, let’s lift that procedure.’ They lift that procedure—illegitimately and with an incomplete assembly. 

He\marginnote{7.14.39} then goes to another monastery. There too the monks consider, ‘The Sangha has lifted a legal procedure of reconciliation against this monk—illegitimately and with an incomplete assembly. Well then, let’s lift that procedure.’ They lift that procedure—illegitimately but with a unanimous assembly. … legitimately but with an incomplete assembly. … in a legitimate-like way and with an incomplete assembly. … in a legitimate-like way but with a unanimous assembly. …” 

The\marginnote{7.14.48} permutation series is to be expanded. 

\section*{20. Discussion of the lifting of the legal procedure of ejection for not recognizing }

“It\marginnote{7.14.49.1} may be that the Sangha has done a legal procedure of ejection against a monk for not recognizing an offense, that he then conducts himself properly and suitably so as to deserve to be released, and that he then asks for the lifting of that procedure. The monks consider, ‘The Sangha has done a legal procedure of ejection against this monk for not recognizing an offense. He has conducted himself properly and suitably so as to deserve to be released, and now asks for the lifting of that procedure. Well then, let’s lift that procedure.’ They lift that procedure—illegitimately and with an incomplete assembly. 

He\marginnote{7.14.55} then goes to another monastery. There too the monks consider, ‘The Sangha has lifted a legal procedure of ejection against this monk for not recognizing an offense—illegitimately and with an incomplete assembly. Well then, let’s lift that procedure.’ They lift that procedure—illegitimately but with a unanimous assembly. … legitimately but with an incomplete assembly. … in a legitimate-like way and with an incomplete assembly. … in a legitimate-like way but with a unanimous assembly. …” 

The\marginnote{7.14.64} permutation series is to be expanded. 

\section*{21. Discussion of the lifting of the legal procedure of ejection for not making amends }

“It\marginnote{7.14.65.1} may be that the Sangha has done a legal procedure of ejection against a monk for not making amends for an offense, that he then conducts himself properly and suitably so as to deserve to be released, and that he then asks for the lifting of that procedure. The monks consider, ‘The Sangha has done a legal procedure of ejection against this monk for not making amends for an offense. He has conducted himself properly and suitably so as to deserve to be released, and now asks for the lifting of that procedure. Well then, let’s lift that procedure.’ They lift that procedure—illegitimately and with an incomplete assembly. 

He\marginnote{7.14.71} then goes to another monastery. There too the monks consider, ‘The Sangha has lifted a legal procedure of ejection against this monk for not making amends for an offense—illegitimately and with an incomplete assembly. Well then, let’s lift that procedure.’ They lift that procedure—illegitimately but with a unanimous assembly. … legitimately but with an incomplete assembly. … in a legitimate-like way and with an incomplete assembly. … in a legitimate-like way but with a unanimous assembly. …” 

The\marginnote{7.14.81} permutation series is to be expanded. 

\section*{22. Discussion of the lifting of the legal procedure of ejection for not giving up a bad view }

“It\marginnote{7.14.82.1} may be that the Sangha has done a legal procedure of ejection against a monk for not giving up a bad view, that he then conducts himself properly and suitably so as to deserve to be released, and that he then asks for the lifting of that procedure. The monks consider, ‘The Sangha has done a legal procedure of ejection against this monk for not giving up a bad view. He has conducted himself properly and suitably so as to deserve to be released, and now asks for the lifting of that procedure. Well then, let’s lift that procedure.’ They lift that procedure—illegitimately and with an incomplete assembly. 

He\marginnote{7.14.88} then goes to another monastery. There too the monks consider, ‘The Sangha has lifted a legal procedure of ejection against this monk for not giving up a bad view—illegitimately and with an incomplete assembly. Well then, let’s lift that procedure.’ They lift that procedure—illegitimately but with a unanimous assembly. … legitimately but with an incomplete assembly. … in a legitimate-like way and with an incomplete assembly. … in a legitimate-like way but with a unanimous assembly. …” 

The\marginnote{7.14.98} permutation series is to be expanded. 

\section*{23. Discussion of disputes on the legal procedure of condemnation }

“It\marginnote{7.15.1} may be, monks, that a monk is quarrelsome and argumentative, one who creates legal issues in the Sangha. The monks consider, ‘This monk is quarrelsome and argumentative, one who creates legal issues in the Sangha. Well then, let’s do a legal procedure of condemnation against him.’ They do the procedure—illegitimately and with an incomplete assembly. 

The\marginnote{7.15.8} Sangha there starts disputing: ‘It was an illegitimate procedure done with an incomplete assembly,’ ‘It was an illegitimate procedure done with a unanimous assembly,’ ‘It was a legitimate procedure done with an incomplete assembly,’ ‘It was a legitimate-like procedure done with an incomplete assembly,’ ‘It was a legitimate-like procedure done with a unanimous assembly,’ ‘The procedure is invalid, it was badly done, and it needs to be done again.’ Those monks who say, ‘It was an illegitimate legal procedure done with an incomplete assembly,’ and those who say, ‘The legal procedure is invalid, it was badly done, and it needs to be done again,’ they are the ones there who speak in accordance with the Teaching. 

“It\marginnote{7.16.1} may be that a monk is quarrelsome and argumentative, one who creates legal issues in the Sangha. The monks consider, ‘This monk is quarrelsome and argumentative, one who creates legal issues in the Sangha. Well then, let’s do a legal procedure of condemnation against him.’ They do the procedure—illegitimately but with a unanimous assembly. 

The\marginnote{7.16.9} Sangha there starts disputing: ‘It was an illegitimate procedure done with an incomplete assembly,’ ‘It was an illegitimate procedure done with a unanimous assembly,’ ‘It was a legitimate procedure done with an incomplete assembly,’ ‘It was a legitimate-like procedure done with an incomplete assembly,’ ‘It was a legitimate-like procedure done with a unanimous assembly,’ ‘The procedure is invalid, it was badly done, and it needs to be done again.’ Those monks who say, ‘It was an illegitimate legal procedure done with a unanimous assembly,’ and those who say, ‘The legal procedure is invalid, it was badly done, and it needs to be done again,’ they are the ones there who speak in accordance with the Teaching. 

“It\marginnote{7.16.14} may be that a monk is quarrelsome and argumentative, one who creates legal issues in the Sangha. The monks consider, ‘This monk is quarrelsome and argumentative, one who creates legal issues in the Sangha. Well then, let’s do a legal procedure of condemnation against him.’ They do the procedure—legitimately but with an incomplete assembly. 

The\marginnote{7.16.20} Sangha there starts disputing: ‘It was an illegitimate procedure done with an incomplete assembly,’ ‘It was an illegitimate procedure done with a unanimous assembly,’ ‘It was a legitimate procedure done with an incomplete assembly,’ ‘It was a legitimate-like procedure done with an incomplete assembly,’ ‘It was a legitimate-like procedure done with a unanimous assembly,’ ‘The procedure is invalid, it was badly done, and it needs to be done again.’ Those monks who say, ‘It was a legitimate legal procedure done with an incomplete assembly,’ and those who say, ‘The legal procedure is invalid, it was badly done, and it needs to be done again,’ they are the ones there who speak in accordance with the Teaching. 

“It\marginnote{7.16.25} may be that a monk is quarrelsome and argumentative, one who creates legal issues in the Sangha. The monks consider, ‘This monk is quarrelsome and argumentative, one who creates legal issues in the Sangha. Well then, let’s do a legal procedure of condemnation against him.’ They do the procedure—in a legitimate-like way and with an incomplete assembly. 

The\marginnote{7.16.33} Sangha there starts disputing: ‘It was an illegitimate procedure done with an incomplete assembly,’ ‘It was an illegitimate procedure done with a unanimous assembly,’ ‘It was a legitimate procedure done with an incomplete assembly,’ ‘It was a legitimate-like procedure done with an incomplete assembly,’ ‘It was a legitimate-like procedure done with a unanimous assembly,’ ‘The procedure is invalid, it was badly done, and it needs to be done again.’ Those monks who say, ‘It was a legitimate-like legal procedure done with an incomplete assembly,’ and those who say,\footnote{The Pali mistakenly reads \textit{samagga}, “a complete assembly”, instead of \textit{vagga}, “an incomplete assembly”. } ‘The legal procedure is invalid, it was badly done, and it needs to be done again,’ they are the ones there who speak in accordance with the Teaching. 

“It\marginnote{7.16.38} may be that a monk is quarrelsome and argumentative, one who creates legal issues in the Sangha. The monks consider, ‘This monk is quarrelsome and argumentative, one who creates legal issues in the Sangha. Well then, let’s do a legal procedure of condemnation against him.’ They do the procedure—in a legitimate-like way but with a unanimous assembly. 

The\marginnote{7.16.46} Sangha there starts disputing: ‘It was an illegitimate procedure done with an incomplete assembly,’ ‘It was an illegitimate procedure done with a unanimous assembly,’ ‘It was a legitimate procedure done with an incomplete assembly,’ ‘It was a legitimate-like procedure done with an incomplete assembly,’ ‘It was a legitimate-like procedure done with a unanimous assembly,’ ‘The procedure is invalid, it was badly done, and it needs to be done again.’ Those monks who say, ‘It was a legitimate-like legal procedure done with a unanimous assembly,’ and those who say, ‘The legal procedure is invalid, it was badly done, and it needs to be done again,’ they are the ones there who speak in accordance with the Teaching.” 

\section*{24. Discussion of disputes on the legal procedure of demotion }

“It\marginnote{7.17.1} may be, monks, that a monk is ignorant, incompetent, often committing offenses, lacking in boundaries, constantly and improperly socializing with householders. The monks consider, ‘This monk is ignorant, incompetent, often committing offenses, lacking in boundaries, constantly and improperly socializing with householders. Well then, let’s do a legal procedure of demotion against him.’ They do the procedure—illegitimately and with an incomplete assembly. … illegitimately but with a unanimous assembly. … legitimately but with an incomplete assembly. … in a legitimate-like way and with an incomplete assembly. … in a legitimate-like way but with a unanimous assembly. 

The\marginnote{7.17.11} Sangha there starts disputing: ‘It was an illegitimate procedure done with an incomplete assembly,’ ‘It was an illegitimate procedure done with a unanimous assembly,’ ‘It was a legitimate procedure done with an incomplete assembly,’ ‘It was a legitimate-like procedure done with an incomplete assembly,’ ‘It was a legitimate-like procedure done with a unanimous assembly,’ ‘The procedure is invalid, it was badly done, and it needs to be done again.’ Those monks who say, ‘It was a legitimate-like legal procedure done with a unanimous assembly,’ and those who say, ‘The legal procedure is invalid, it was badly done, and it needs to be done again,’ they are the ones there who speak in accordance with the Teaching.” 

\scend{These five contracted sections are finished. }

\section*{25. Discussion of disputes on the legal procedure of banishment }

“It\marginnote{7.18.1} may be that a monk is a corrupter of families and badly behaved. The monks consider, ‘This monk is a corrupter of families and badly behaved. Well then, let’s do a legal procedure of banishment against him.’ They do the procedure—illegitimately and with an incomplete assembly. … illegitimately but with a unanimous assembly. … legitimately but with an incomplete assembly. … in a legitimate-like way and with an incomplete assembly. … in a legitimate-like way but with a unanimous assembly. 

The\marginnote{7.18.11} Sangha there starts disputing: ‘It was an illegitimate procedure done with an incomplete assembly,’ ‘It was an illegitimate procedure done with a unanimous assembly,’ ‘It was a legitimate procedure done with an incomplete assembly,’ ‘It was a legitimate-like procedure done with an incomplete assembly,’ ‘It was a legitimate-like procedure done with a unanimous assembly,’ ‘The procedure is invalid, it was badly done, and it needs to be done again.’ Those monks who say, ‘It was a legitimate-like legal procedure done with a unanimous assembly,’ and those who say, ‘The legal procedure is invalid, it was badly done, and it needs to be done again,’ they are the ones there who speak in accordance with the Teaching.” 

\scend{These five contracted sections are finished. }

\section*{26. Discussion of disputes on the legal procedure of reconciliation }

“It\marginnote{7.18.17.1} may be that a monk abuses and reviles householders. The monks consider, ‘This monk abuses and reviles householders. Well then, let’s do a legal procedure of reconciliation against him.’ They do the procedure—illegitimately and with an incomplete assembly. … illegitimately but with a unanimous assembly. … legitimately but with an incomplete assembly. … in a legitimate-like way and with an incomplete assembly. … in a legitimate-like way but with a unanimous assembly. 

The\marginnote{7.18.27} Sangha there starts disputing: ‘It was an illegitimate procedure done with an incomplete assembly,’ ‘It was an illegitimate procedure done with a unanimous assembly,’ ‘It was a legitimate procedure done with an incomplete assembly,’ ‘It was a legitimate-like procedure done with an incomplete assembly,’ ‘It was a legitimate-like procedure done with a unanimous assembly,’ ‘The procedure is invalid, it was badly done, and it needs to be done again.’ Those monks who say, ‘It was a legitimate-like legal procedure done with a unanimous assembly,’ and those who say, ‘The legal procedure is invalid, it was badly done, and it needs to be done again,’ they are the ones there who speak in accordance with the Teaching.” 

\scend{These five contracted sections are finished. }

\section*{27. Discussion of disputes on the legal procedure of ejection for not recognizing }

“It\marginnote{7.18.33.1} may be that a monk commits an offense but refuses to recognize it. The monks consider, ‘This monk has committed an offense but refuses to recognize it. Well then, let’s do a legal procedure of ejection against him for not recognizing an offense.’ They do the procedure—illegitimately and with an incomplete assembly. … illegitimately but with a unanimous assembly. … legitimately but with an incomplete assembly. … in a legitimate-like way and with an incomplete assembly. … in a legitimate-like way but with a unanimous assembly. 

The\marginnote{7.18.43} Sangha there starts disputing: ‘It was an illegitimate procedure done with an incomplete assembly,’ ‘It was an illegitimate procedure done with a unanimous assembly,’ ‘It was a legitimate procedure done with an incomplete assembly,’ ‘It was a legitimate-like procedure done with an incomplete assembly,’ ‘It was a legitimate-like procedure done with a unanimous assembly,’ ‘The procedure is invalid, it was badly done, and it needs to be done again.’ Those monks who say, ‘It was a legitimate-like legal procedure done with a unanimous assembly,’ and those who say, ‘The legal procedure is invalid, it was badly done, and it needs to be done again,’ they are the ones there who speak in accordance with the Teaching.” 

\scend{These five contracted sections are finished. }

\section*{28. Discussion of disputes on the legal procedure of ejection for not making amends }

“It\marginnote{7.18.49.1} may be that a monk commits an offense but refuses to make amends for it. The monks consider, ‘This monk has committed an offense but refuses to make amends for it. Well then, let’s do a legal procedure of ejection against him for not making amends for an offense.’ They do the procedure—illegitimately and with an incomplete assembly. … illegitimately but with a unanimous assembly. … legitimately but with an incomplete assembly. … in a legitimate-like way and with an incomplete assembly. … in a legitimate-like way but with a unanimous assembly. 

The\marginnote{7.18.59} Sangha there starts disputing: ‘It was an illegitimate procedure done with an incomplete assembly,’ ‘It was an illegitimate procedure done with a unanimous assembly,’ ‘It was a legitimate procedure done with an incomplete assembly,’ ‘It was a legitimate-like procedure done with an incomplete assembly,’ ‘It was a legitimate-like procedure done with a unanimous assembly,’ ‘The procedure is invalid, it was badly done, and it needs to be done again.’ Those monks who say, ‘It was a legitimate-like legal procedure done with a unanimous assembly,’ and those who say, ‘The legal procedure is invalid, it was badly done, and it needs to be done again,’ they are the ones there who speak in accordance with the Teaching.” 

\scend{These five contracted sections are finished. }

\section*{29. Discussion of disputes on the legal procedure of ejection for not giving up }

“It\marginnote{7.18.66.1} may be that a monk refuses to give up a bad view. The monks consider, ‘This monk refuses to give up a bad view. Well then, let’s do a legal procedure of ejection against him for not giving up a bad view.’ They do the procedure—illegitimately and with an incomplete assembly. … illegitimately but with a unanimous assembly. … legitimately but with an incomplete assembly. … in a legitimate-like way and with an incomplete assembly. … in a legitimate-like way but with a unanimous assembly. 

The\marginnote{7.18.76} Sangha there starts disputing: ‘It was an illegitimate procedure done with an incomplete assembly,’ ‘It was an illegitimate procedure done with a unanimous assembly,’ ‘It was a legitimate procedure done with an incomplete assembly,’ ‘It was a legitimate-like procedure done with an incomplete assembly,’ ‘It was a legitimate-like procedure done with a unanimous assembly,’ ‘The procedure is invalid, it was badly done, and it needs to be done again.’ Those monks who say, ‘It was a legitimate-like legal procedure done with a unanimous assembly,’ and those who say, ‘The legal procedure is invalid, it was badly done, and it needs to be done again,’ they are the ones there who speak in accordance with the Teaching.” 

\scend{These five contracted sections are finished. }

\section*{30. Discussion of the lifting of the legal procedure of condemnation }

“It\marginnote{7.19.1} may be, monks, that the Sangha has done a legal procedure of condemnation against a monk, that he then conducts himself properly and suitably so as to deserve to be released, and that he then asks for the lifting of that procedure. The monks consider, ‘The Sangha has done a legal procedure of condemnation against this monk. He has conducted himself properly and suitably so as to deserve to be released, and now asks for the lifting of that procedure. Well then, let’s lift that procedure.’ They lift that procedure—illegitimately and with an incomplete assembly. 

The\marginnote{7.19.7} Sangha there starts disputing: ‘It was an illegitimate procedure done with an incomplete assembly,’ ‘It was an illegitimate procedure done with a unanimous assembly,’ ‘It was a legitimate procedure done with an incomplete assembly,’ ‘It was a legitimate-like procedure done with an incomplete assembly,’ ‘It was a legitimate-like procedure done with a unanimous assembly,’ ‘The procedure is invalid, it was badly done, and it needs to be done again.’ Those monks who say, ‘It was an illegitimate legal procedure done with an incomplete assembly,’ and those who say, ‘The legal procedure is invalid, it was badly done, and it needs to be done again,’ they are the ones there who speak in accordance with the Teaching. 

“It\marginnote{7.19.12} may be that the Sangha has done a legal procedure of condemnation against a monk, that he then conducts himself properly and suitably so as to deserve to be released, and that he then asks for the lifting of that procedure. The monks consider, ‘The Sangha has done a legal procedure of condemnation against this monk. He has conducted himself properly and suitably so as to deserve to be released, and now asks for the lifting of that procedure. Well then, let’s lift that procedure.’ They lift that procedure—illegitimately but with a unanimous assembly. 

The\marginnote{7.19.18} Sangha there starts disputing: ‘It was an illegitimate procedure done with an incomplete assembly,’ ‘It was an illegitimate procedure done with a unanimous assembly,’ ‘It was a legitimate procedure done with an incomplete assembly,’ ‘It was a legitimate-like procedure done with an incomplete assembly,’ ‘It was a legitimate-like procedure done with a unanimous assembly,’ ‘The procedure is invalid, it was badly done, and it needs to be done again.’ Those monks who say, ‘It was an illegitimate legal procedure done with a unanimous assembly,’ and those who say, ‘The legal procedure is invalid, it was badly done, and it needs to be done again,’ they are the ones there who speak in accordance with the Teaching. 

“It\marginnote{7.19.23} may be that the Sangha has done a legal procedure of condemnation against a monk, that he then conducts himself properly and suitably so as to deserve to be released, and that he then asks for the lifting of that procedure. The monks consider, ‘The Sangha has done a legal procedure of condemnation against this monk. He has conducted himself properly and suitably so as to deserve to be released, and now asks for the lifting of that procedure. Well then, let’s lift that procedure.’ They lift that procedure—legitimately but with an incomplete assembly. 

The\marginnote{7.19.29} Sangha there starts disputing: ‘It was an illegitimate procedure done with an incomplete assembly,’ ‘It was an illegitimate procedure done with a unanimous assembly,’ ‘It was a legitimate procedure done with an incomplete assembly,’ ‘It was a legitimate-like procedure done with an incomplete assembly,’ ‘It was a legitimate-like procedure done with a unanimous assembly,’ ‘The procedure is invalid, it was badly done, and it needs to be done again.’ Those monks who say, ‘It was a legitimate legal procedure done with an incomplete assembly,’ and those who say, ‘The legal procedure is invalid, it was badly done, and it needs to be done again,’ they are the ones there who speak in accordance with the Teaching. 

“It\marginnote{7.19.34} may be that the Sangha has done a legal procedure of condemnation against a monk, that he then conducts himself properly and suitably so as to deserve to be released, and that he then asks for the lifting of that procedure. The monks consider, ‘The Sangha has done a legal procedure of condemnation against this monk. He has conducted himself properly and suitably so as to deserve to be released, and now asks for the lifting of that procedure. Well then, let’s lift that procedure.’ They lift that procedure—in a legitimate-like way and with an incomplete assembly. 

The\marginnote{7.19.40} Sangha there starts disputing: ‘It was an illegitimate procedure done with an incomplete assembly,’ ‘It was an illegitimate procedure done with a unanimous assembly,’ ‘It was a legitimate procedure done with an incomplete assembly,’ ‘It was a legitimate-like procedure done with an incomplete assembly,’ ‘It was a legitimate-like procedure done with a unanimous assembly,’ ‘The procedure is invalid, it was badly done, and it needs to be done again.’ Those monks who say, ‘It was a legitimate-like legal procedure done with an incomplete assembly,’ and those who say, ‘The legal procedure is invalid, it was badly done, and it needs to be done again,’ they are the ones there who speak in accordance with the Teaching. 

“It\marginnote{7.19.45} may be that the Sangha has done a legal procedure of condemnation against a monk, that he then conducts himself properly and suitably so as to deserve to be released, and that he then asks for the lifting of that procedure. The monks consider, ‘The Sangha has done a legal procedure of condemnation against this monk. He has conducted himself properly and suitably so as to deserve to be released, and now asks for the lifting of that procedure. Well then, let’s lift that procedure.’ They lift that procedure—in a legitimate-like way but with a unanimous assembly. 

The\marginnote{7.19.51} Sangha there starts disputing: ‘It was an illegitimate procedure done with an incomplete assembly,’ ‘It was an illegitimate procedure done with a unanimous assembly,’ ‘It was a legitimate procedure done with an incomplete assembly,’ ‘It was a legitimate-like procedure done with an incomplete assembly,’ ‘It was a legitimate-like procedure done with a unanimous assembly,’ ‘The procedure is invalid, it was badly done, and it needs to be done again.’ Those monks who say, ‘It was a legitimate-like legal procedure done with a unanimous assembly,’ and those who say, ‘The legal procedure is invalid, it was badly done, and it needs to be done again,’ they are the ones there who speak in accordance with the Teaching.” 

\section*{31. Discussion of the lifting of the legal procedure of demotion }

“It\marginnote{7.20.1} may be, monks, that the Sangha has done a legal procedure of demotion against a monk, that he then conducts himself properly and suitably so as to deserve to be released, and that he then asks for the lifting of that procedure. The monks consider, ‘The Sangha has done a legal procedure of demotion against this monk. He has conducted himself properly and suitably so as to deserve to be released, and now asks for the lifting of that procedure. Well then, let’s lift that procedure.’ They lift that procedure—illegitimately and with an incomplete assembly. … illegitimately but with a unanimous assembly. … legitimately but with an incomplete assembly. … in a legitimate-like way and with an incomplete assembly. … in a legitimate-like way but with a unanimous assembly. 

The\marginnote{7.20.11} Sangha there starts disputing: ‘It was an illegitimate procedure done with an incomplete assembly,’ ‘It was an illegitimate procedure done with a unanimous assembly,’ ‘It was a legitimate procedure done with an incomplete assembly,’ ‘It was a legitimate-like procedure done with an incomplete assembly,’ ‘It was a legitimate-like procedure done with a unanimous assembly,’ ‘The procedure is invalid, it was badly done, and it needs to be done again.’ Those monks who say, ‘It was a legitimate-like legal procedure done with a unanimous assembly,’ and those who say, ‘The legal procedure is invalid, it was badly done, and it needs to be done again,’ they are the ones there who speak in accordance with the Teaching.” 

\scend{These five contracted sections, too, are finished. }

\section*{32. Discussion of the lifting of the legal procedure of banishment }

“It\marginnote{7.20.17.1} may be that the Sangha has done a legal procedure of banishment against a monk, that he then conducts himself properly and suitably so as to deserve to be released, and that he then asks for the lifting of that procedure. The monks consider, ‘The Sangha has done a legal procedure of banishment against this monk. He has conducted himself properly and suitably so as to deserve to be released, and now asks for the lifting of that procedure. Well then, let’s lift that procedure.’ They lift that procedure—illegitimately and with an incomplete assembly. … illegitimately but with a unanimous assembly. … legitimately but with an incomplete assembly. … in a legitimate-like way and with an incomplete assembly. … in a legitimate-like way but with a unanimous assembly. 

The\marginnote{7.20.27} Sangha there starts disputing: ‘It was an illegitimate procedure done with an incomplete assembly,’ ‘It was an illegitimate procedure done with a unanimous assembly,’ ‘It was a legitimate procedure done with an incomplete assembly,’ ‘It was a legitimate-like procedure done with an incomplete assembly,’ ‘It was a legitimate-like procedure done with a unanimous assembly,’ ‘The procedure is invalid, it was badly done, and it needs to be done again.’ Those monks who say, ‘It was a legitimate-like legal procedure done with a unanimous assembly,’ and those who say, ‘The legal procedure is invalid, it was badly done, and it needs to be done again,’ they are the ones there who speak in accordance with the Teaching.” 

\scend{These five contracted sections, too, are finished. }

\section*{33. Discussion of the lifting of the legal procedure of reconciliation }

“It\marginnote{7.20.33.1} may be that the Sangha has done a legal procedure of reconciliation against a monk, that he then conducts himself properly and suitably so as to deserve to be released, and that he then asks for the lifting of that procedure. The monks consider, ‘The Sangha has done a legal procedure of reconciliation against this monk. He has conducted himself properly and suitably so as to deserve to be released, and now asks for the lifting of that procedure. Well then, let’s lift that procedure.’ They lift that procedure—illegitimately and with an incomplete assembly. … illegitimately but with a unanimous assembly. … legitimately but with an incomplete assembly. … in a legitimate-like way and with an incomplete assembly. … in a legitimate-like way but with a unanimous assembly. 

The\marginnote{7.20.43} Sangha there starts disputing: ‘It was an illegitimate procedure done with an incomplete assembly,’ ‘It was an illegitimate procedure done with a unanimous assembly,’ ‘It was a legitimate procedure done with an incomplete assembly,’ ‘It was a legitimate-like procedure done with an incomplete assembly,’ ‘It was a legitimate-like procedure done with a unanimous assembly,’ ‘The procedure is invalid, it was badly done, and it needs to be done again.’ Those monks who say, ‘It was a legitimate-like legal procedure done with a unanimous assembly,’ and those who say, ‘The legal procedure is invalid, it was badly done, and it needs to be done again,’ they are the ones there who speak in accordance with the Teaching.” 

\scend{These five contracted sections, too, are finished. }

\section*{34. Discussion of the lifting of the legal procedure of ejection for not recognizing }

“It\marginnote{7.20.50.1} may be that the Sangha has done a legal procedure of ejection against a monk for not recognizing an offense, that he then conducts himself properly and suitably so as to deserve to be released, and that he then asks for the lifting of that procedure. The monks consider, ‘The Sangha has done a legal procedure of ejection against this monk for not recognizing an offense. He has conducted himself properly and suitably so as to deserve to be released, and now asks for the lifting of that procedure. Well then, let’s lift that procedure.’ They lift that procedure—illegitimately and with an incomplete assembly. … illegitimately but with a unanimous assembly. … legitimately but with an incomplete assembly. … in a legitimate-like way and with an incomplete assembly. … in a legitimate-like way but with a unanimous assembly. 

The\marginnote{7.20.60} Sangha there starts disputing: ‘It was an illegitimate procedure done with an incomplete assembly,’ ‘It was an illegitimate procedure done with a unanimous assembly,’ ‘It was a legitimate procedure done with an incomplete assembly,’ ‘It was a legitimate-like procedure done with an incomplete assembly,’ ‘It was a legitimate-like procedure done with a unanimous assembly,’ ‘The procedure is invalid, it was badly done, and it needs to be done again.’ Those monks who say, ‘It was a legitimate-like legal procedure done with a unanimous assembly,’ and those who say, ‘The legal procedure is invalid, it was badly done, and it needs to be done again,’ they are the ones there who speak in accordance with the Teaching.” 

\scend{These five contracted sections, too, are finished. }

\section*{35. Discussion of the lifting of the legal procedure of ejection for not making amends }

“It\marginnote{7.20.66.1} may be that the Sangha has done a legal procedure of ejection against a monk for not making amends for an offense, that he then conducts himself properly and suitably so as to deserve to be released, and that he then asks for the lifting of that procedure. The monks consider, ‘The Sangha has done a legal procedure of ejection against this monk for not making amends for an offense. He has conducted himself properly and suitably so as to deserve to be released, and now asks for the lifting of that procedure. Well then, let’s lift that procedure.’ They lift that procedure—illegitimately and with an incomplete assembly. … illegitimately but with a unanimous assembly. … legitimately but with an incomplete assembly. … in a legitimate-like way and with an incomplete assembly. … in a legitimate-like way but with a unanimous assembly. 

The\marginnote{7.20.76} Sangha there starts disputing: ‘It was an illegitimate procedure done with an incomplete assembly,’ ‘It was an illegitimate procedure done with a unanimous assembly,’ ‘It was a legitimate procedure done with an incomplete assembly,’ ‘It was a legitimate-like procedure done with an incomplete assembly,’ ‘It was a legitimate-like procedure done with a unanimous assembly,’ ‘The procedure is invalid, it was badly done, and it needs to be done again.’ Those monks who say, ‘It was a legitimate-like legal procedure done with a unanimous assembly,’ and those who say, ‘The legal procedure is invalid, it was badly done, and it needs to be done again,’ they are the ones there who speak in accordance with the Teaching.” 

\scend{These five contracted sections, too, are finished. }

\section*{36. Discussion of the lifting of the legal procedure of ejection for not giving up a bad view }

“It\marginnote{7.20.82.1} may be that the Sangha has done a legal procedure of ejection against a monk for not giving up a bad view, that he then conducts himself properly and suitably so as to deserve to be released, and that he then asks for the lifting of that procedure. The monks consider, ‘The Sangha has done a legal procedure of ejection against this monk for not giving up a bad view. He has conducted himself properly and suitably so as to deserve to be released, and now asks for the lifting of that procedure. Well then, let’s lift that procedure.’ They lift that procedure—illegitimately and with an incomplete assembly. … illegitimately but with a unanimous assembly. … legitimately but with an incomplete assembly. … in a legitimate-like way and with an incomplete assembly. … in a legitimate-like way but with a unanimous assembly. 

The\marginnote{7.20.92} Sangha there starts disputing: ‘It was an illegitimate procedure done with an incomplete assembly,’ ‘It was an illegitimate procedure done with a unanimous assembly,’ ‘It was a legitimate procedure done with an incomplete assembly,’ ‘It was a legitimate-like procedure done with an incomplete assembly,’ ‘It was a legitimate-like procedure done with a unanimous assembly,’ ‘The procedure is invalid, it was badly done, and it needs to be done again.’ Those monks who say, ‘It was a legitimate-like legal procedure done with a unanimous assembly,’ and those who say, ‘The legal procedure is invalid, it was badly done, and it needs to be done again,’ they are the ones there who speak in accordance with the Teaching.” 

\scend{These five contracted sections, too, are finished. }

\scendsutta{The ninth chapter connected with \textsanskrit{Campā} is finished. }

\scuddanaintro{This is the summary: }

\begin{scuddana}%
“The\marginnote{7.20.100} Buddha was at \textsanskrit{Campā}, \\
The account of the village of \textsanskrit{Vāsabha}; \\
Helping the newly arrived, \\
He worked for what they wanted. 

Knowing,\marginnote{7.20.104} ‘They are knowledgeable’, \\
He made no effort then; \\
Ejected, ‘He did not’, \\
He went to the Victor. 

Illegitimate\marginnote{7.20.108} legal procedures with incomplete assembly, \\
And illegitimate legal procedures with unanimous assembly; \\
And legitimate legal procedures with incomplete assembly, \\
Legitimate-like with incomplete assembly. 

Legitimate-like\marginnote{7.20.112} with unanimous assembly, \\
One person ejects another; \\
And one ejects two or three, \\
One ejects a sangha. 

The\marginnote{7.20.116} same for two and three, \\
And a sangha ejects a sangha; \\
The Excellent Omniscient One having heard, \\
Prohibited the illegitimate. 

A\marginnote{7.20.120} procedure deficient in motion, \\
But complete in announcement; \\
One deficient in announcement, \\
But complete in motion. 

And\marginnote{7.20.124} one deficient in both, \\
And not according to the Teaching; \\
The Monastic Law, the Teacher, objected to, \\
Reversible, unfit to stand. 

Illegitimate\marginnote{7.20.128} with incomplete assembly, with unanimous assembly, \\
Legitimate, two legitimate-like; \\
Just legitimate with a unanimous assembly, \\
Was allowed by the Buddha. 

A\marginnote{7.20.132} group of four, a group of five, \\
And a group of ten, twenty; \\
And a group of more than twenty, \\
Thus a five-fold sangha. 

Apart\marginnote{7.20.136} from ordination, \\
And the procedure of invitation; \\
Together with the procedure of rehabilitation, \\
Is done by a group of four. 

Apart\marginnote{7.20.140} from two procedures, \\
Ordination in the Middle Country; \\
Rehabilitation, a group of five, \\
Does all procedures. 

Apart\marginnote{7.20.144} from rehabilitation, \\
Is a group of ten monks; \\
A sangha that does all procedures, \\
Is twenty, a doer of all. 

A\marginnote{7.20.148} nun, and a trainee nun, \\
A novice monk, a novice nun; \\
Who has renounced, the worst kind of offense, \\
Ejected for not seeing an offense. 

For\marginnote{7.20.152} not making amends, for a bad view, \\
A \textsanskrit{paṇḍaka}, a fake monk; \\
Monastics of another religion, animal, \\
Killer of mother, and father. 

A\marginnote{7.20.156} perfected one, a rapist of a nun, \\
A schismatic, a shedder of blood; \\
A hermaphrodite, a different Buddhist sect, \\
Outside the monastery zone, by supernormal power. 

The\marginnote{7.20.160} one who is subject to the legal procedure, \\
These twenty-four are; \\
Prohibited by the Fully Awakened One, \\
For these do not complete the quorum. 

If,\marginnote{7.20.164} with one on probation as the fourth, \\
It should give probation;\footnote{In these cases, the third person singular agent, the “it”, is presumably the Sangha. } \\
Or send to the beginning, give trial, rehabilitate, \\
It’s invalid, not to be done. 

One\marginnote{7.20.168} deserving sending back, deserving trial, on trial, \\
And even deserving rehabilitation; \\
These five cannot do a procedure, \\
Explained the Fully Awakened One. 

A\marginnote{7.20.172} nun, and a trainee nun, \\
A novice monk, a novice nun; \\
Who has renounced, the worst kind, insane, \\
Deranged, pain, for not seeing. 

For\marginnote{7.20.176} not making amends, for a bad view, \\
And also a \textit{\textsanskrit{paṇḍaka}}, hermaphrodite; \\
One from a different Buddhist sect, monastery zone, \\
Air, and the subject of the procedure. 

Of\marginnote{7.20.180} these eighteen, \\
An objection is invalid; \\
Of a regular monk, \\
An objection is valid. 

For\marginnote{7.20.184} one who is pure, the sending away fails, \\
For the fool it succeeds; \\
The \textit{\textsanskrit{paṇḍaka}}, living together by theft, \\
Joined, animal. 

Of\marginnote{7.20.188} mother, of father, a perfected One, \\
A rapist, a schismatic; \\
And a shedder of blood, \\
And one who is a hermaphrodite. 

Of\marginnote{7.20.192} these eleven, \\
The admittance fails; \\
Hand, foot, both of them, \\
Ear, nose, both of them. 

Finger,\marginnote{7.20.196} thumb, tendon, \\
Joined, and hunchback, dwarf; \\
Goiter, branded, and whipped, \\
And sentenced, elephantiasis. 

Serious,\marginnote{7.20.200} abnormal, and blind in one eye, \\
Crooked limb, lame, and also the paralyzed; \\
Crippled, weak, \\
Blind, and mute, deaf. 

Blind\marginnote{7.20.204} and mute, blind and deaf, \\
Mute and deaf; \\
And blind and mute and deaf, \\
Thirty-two exactly. 

For\marginnote{7.20.208} them there is admittance, \\
Explained the Fully Awakened one; \\
They are to be seen, to be remedied, \\
There is no sending away. 

A\marginnote{7.20.212} procedure of ejection against one, \\
Seven are illegitimate; \\
If committed but acting properly, \\
Those seven too are illegitimate. 

If\marginnote{7.20.216} committed and not acting properly, \\
Seven procedures are legitimate; \\
Face-to-face, and questioning, \\
And done with admission. 

Recollection,\marginnote{7.20.220} insanity, penalty, \\
Condemnation, and with demotion; \\
Banishment, reconciliation, \\
Ejection, and probation. 

Beginning,\marginnote{7.20.224} trial, rehabilitation, \\
Just so ordination; \\
If it does one in place of another,\footnote{Again, in these cases, the third person singular agent, the “it”, is presumably the Sangha. } \\
These sixteen are illegitimate. 

If\marginnote{7.20.228} it does the right one,\footnote{Ditto. } \\
These sixteen are legitimate; \\
It would counter accuse reciprocally,\footnote{Ditto. } \\
These sixteen are illegitimate. 

Two\marginnote{7.20.232} and two having that basis, \\
Also these sixteen are legitimate; \\
The permutation with a one-by-one basis, \\
‘Illegitimate’, said the Victor. 

It\marginnote{7.20.236} did a legal procedure of condemnation, \\
The Sangha, the one who is quarrelsome; \\
An illegitimate procedure with incomplete assembly, \\
He went to another monastery. 

There\marginnote{7.20.240} unanimous assembly with illegitimate, \\
Did condemnation against him; \\
Another incomplete assembly with legitimate, \\
Did condemnation against him. 

Also\marginnote{7.20.244} incomplete assembly with legitimate-like, \\
So did a unanimous assembly; \\
And a unanimous assembly with illegitimate, \\
And an incomplete assembly with legitimate. 

And\marginnote{7.20.248} incomplete assembly with legitimate-like, \\
And unanimous assembly, in these cases; \\
Having done the basis one by one, \\
A discerning one would link the permutation series. 

Demotion\marginnote{7.20.252} for the incompetent fool,\footnote{Reading \textit{\textsanskrit{bālābyattassa}}. } \\
The corrupter of families should be banished; \\
And a procedure of reconciliation, \\
Should be done to the abuser. 

In\marginnote{7.20.256} not recognizing, in not making amends, \\
And one who would not give up a view; \\
For them there is the procedure of ejection, \\
Said the Caravan Leader. 

With\marginnote{7.20.260} regard to the procedures that have a method,\footnote{See CPD for this use of \textit{upari}. } \\
A wise one should determine condemnation; \\
For those who act suitably, \\
One who conducts himself properly, he should ask. 

The\marginnote{7.20.264} lifting of those procedures, \\
And in accordance with the method for the procedure as above; \\
In regard to whichever procedure, \\
And there they dispute. 

Invalid,\marginnote{7.20.268} and just badly done, \\
And to be done again; \\
And also for the lifting of procedures, \\
Those monks speak in accordance with the Teaching. 

Having\marginnote{7.20.272} seen those afflicted by the disease of failure, \\
To those who are ready for the legal procedure; \\
The Great Sage declared the lifting, \\
Like a surgeon applies the medicine.” 

%
\end{scuddana}

\scend{In this chapter there are thirty-six topics. }

\scendsutta{The chapter connected with \textsanskrit{Campā} is finished. }

%
\chapter*{{\suttatitleacronym Kd 10}{\suttatitletranslation The chapter connected with Kosambī }{\suttatitleroot Kosambakakkhandhaka}}
\addcontentsline{toc}{chapter}{\tocacronym{Kd 10} \toctranslation{The chapter connected with Kosambī } \tocroot{Kosambakakkhandhaka}}
\markboth{The chapter connected with Kosambī }{Kosambakakkhandhaka}
\extramarks{Kd 10}{Kd 10}

\section*{1. The account of the dispute at \textsanskrit{Kosambī} }

At\marginnote{1.1.1} one time when the Buddha was staying at \textsanskrit{Kosambī} in Ghosita’s Monastery, a certain monk had committed an offense. He regarded it as an offense, but there were other monks who did not. Some time later he no longer regarded it as an offense, but there were other monks who did. They said to him, “You’ve committed an offense. Do you recognize it?” 

“No,\marginnote{1.1.9} I haven’t committed any offense that I should recognize.” 

Soon\marginnote{1.1.10} afterwards the monks achieved unanimity, and they ejected that monk for not recognizing the offense. But that monk was learned, a master of the tradition; he was an expert on the Teaching, the Monastic Law, and the Key Terms; he was knowledgeable and competent, had a sense of conscience, and was afraid of wrongdoing and fond of the training. He went to his friends and said, “This isn’t an offense, and so I haven’t committed any. And I haven’t been ejected, for the legal procedure was illegitimate, reversible, and unfit to stand. Please side with me, venerables, in accordance with the Teaching and the Monastic Law.” He was able to form a faction. He then sent the same message to his friends in the country, and again he was able to form a faction. 

The\marginnote{1.3.1} monks who sided with him went to the monks who had ejected him and said, “This isn’t an offense, and so this monk hasn’t committed any. He hasn’t been ejected, for the legal procedure was illegitimate, reversible, and unfit to stand.” 

They\marginnote{1.3.6} replied, “This is an offense, and he’s committed it. And he’s been ejected. The legal procedure was legitimate, irreversible, and fit to stand. Venerables, don’t side with this monk.” But they still sided with him. 

Soon\marginnote{1.4.1} afterwards a certain monk went to the Buddha, bowed, sat down, and told him all that had happened. 

Realizing\marginnote{1.5.1} that the Sangha of monks was divided, the Buddha got up from his seat, went to those monks who had done the ejecting, and sat down on the prepared seat. He then said to those monks: 

“Don’t\marginnote{1.5.3} just eject a monk for any kind of offense merely because it seems clear to you that he’s committed it. 

It\marginnote{1.6.1} may be that a monk has committed an offense. He doesn’t regard it as an offense, but there are other monks who do. If they know, ‘This monk is learned and a master of the tradition; he’s an expert on the Teaching, the Monastic Law, and the Key Terms; he’s knowledgeable and competent, has a sense of conscience, and is afraid of wrongdoing and fond of the training. If we eject him for not recognizing an offense, we won’t be to able do the observance-day ceremony with him. Because of this, there’ll be arguments and disputes in the Sangha; there’ll be schism, fracture, and separation in the Sangha,’ and if they understand the gravity of schism, they shouldn’t eject that monk. 

It\marginnote{1.7.1} may be that a monk has committed an offense. He doesn’t regard it as an offense, but there are other monks who do. If they know, ‘This monk is learned and a master of the tradition; he’s an expert on the Teaching, the Monastic Law, and the Key Terms; he’s knowledgeable and competent, has a sense of conscience, and is afraid of wrongdoing and fond of the training. If we eject him for not recognizing an offense, we won’t be able to do the invitation ceremony with him; we won’t be able to do legal procedures with him; we won’t share a seat with him; we won’t drink congee with him; we won’t sit in the dining hall with him; we won’t stay in the same room with him; we won’t bow down, stand up, raise our joined palms, or do acts of respect toward one another according to seniority. Because of this, there’ll be arguments and disputes in the Sangha; there’ll be schism, fracture, and separation in the Sangha,’ and if they understand the gravity of schism, they shouldn’t eject that monk.” 

The\marginnote{1.8.1} Buddha got up from his seat, went to those monks who were siding with the ejected monk, and sat down on the prepared seat. He then said to those monks: 

“If\marginnote{1.8.2} you’ve committed an offense, don’t refuse to make amends for it just because you think that you haven’t committed it. 

It\marginnote{1.8.3} may be that a monk has committed an offense. He doesn’t regard it as an offense, but there are other monks who do. If he knows, ‘These monks are learned and masters of the tradition; they’re experts on the Teaching, the Monastic Law, and the Key Terms; they’re knowledgeable and competent, have a sense of conscience, and are afraid of wrongdoing and fond of the training. They’re unlikely, because of me or anyone else, to act wrongly out of favoritism, ill will, confusion, or fear. And if these monks eject me for not recognizing an offense, they won’t be able do the observance-day ceremony with me. Because of this, there’ll be arguments and disputes in the Sangha; there’ll be schism, fracture, and separation in the Sangha,’ and if he understands the gravity of schism, he should confess the offense even out of confidence in the others. 

It\marginnote{1.8.7} may be that a monk has committed an offense. He doesn’t regard it as an offense, but there are other monks who do. If he knows, ‘These monks are learned and masters of the tradition; they’re experts on the Teaching, the Monastic Law, and the Key Terms; they’re knowledgeable and competent, have a sense of conscience, and are afraid of wrongdoing and fond of the training. They’re unlikely, because of me or anyone else, to act wrongly out of favoritism, ill will, confusion, or fear. And if these monks eject me for not recognizing an offense, they won’t be able to do the invitation ceremony with me; they won’t be able to do legal procedures with me; they won’t share a seat with me; they won’t drink congee with me; they won’t sit in the dining hall with me; they won’t stay in the same room with me; we won’t bow down, stand up, raise our joined palms, or do acts of respect toward one another according to seniority. Because of this, there’ll be arguments and disputes in the Sangha; there’ll be schism, fracture, and separation in the Sangha,’ and if he understands the gravity of schism, he should confess the offense even out of confidence in the others.” The Buddha then got up from his seat and left. 

\subsection*{Monks belonging to different Buddhist sects}

Soon\marginnote{1.9.1} those monks who sided with the ejected monk did the observance-day ceremony and legal procedures right there within the monastery zone. But the monks who had ejected him went outside the monastery zone and did the observance-day ceremony and legal procedures there. One of the monks who had done the ejecting went to the Buddha, bowed, sat down, and told him what was happening. 

The\marginnote{1.9.5} Buddha replied: “If those monks who side with the ejected monk do the observance-day ceremony and legal procedures right there within the monastery zone, and it’s in accordance with the motion and announcements as I’ve laid them down, then those procedures are legitimate, irreversible, and fit to stand. And if you, the monks who did the ejecting, do the observance-day ceremony and legal procedures right there within the monastery zone, and it’s in accordance with the motion and announcements as I’ve laid them down, then those procedures too are legitimate, irreversible, and fit to stand. This is so because you now belong to a different Buddhist sect. 

There\marginnote{1.10.2} are these two grounds for belonging to a different Buddhist sect. Either one makes oneself belong to a different Buddhist sect, or a unanimous assembly ejects one for not recognizing an offense, for not making amends for an offense, or for not giving up a bad view. And there are these two grounds for belonging to the same Buddhist sect. Either one makes oneself belong to the same Buddhist sect, or a unanimous assembly readmits one who had been ejected for not recognizing an offense, for not making amends for an offense, or for not giving up a bad view.”\footnote{\textit{\textsanskrit{Nānāsaṁvāsa}} (and \textit{\textsanskrit{samānasaṁvāsa}}) need to be carefully distinguished from \textit{\textsanskrit{nānāsaṁvāsaka}} (and \textit{\textsanskrit{samānasaṁvāsaka}}) . The former means “belonging to a different community”, as decided by \textit{\textsanskrit{sīmās}}. The latter means “one belonging to a different Buddhist sect”. } 

\subsection*{Proper conduct when the Sangha is divided}

At\marginnote{2.1.1} this time the monks were arguing and disputing in the dining halls in inhabited areas, behaving improperly by body and speech, such as grabbing one another. People complained and criticized them, “How can the Sakyan monastics behave like this?” 

The\marginnote{2.1.4} monks heard the complaints of those people, and the monks of few desires complained and criticized them, “How can monks behave like this?” They told the Buddha. … “Is it true, monks, that monks are behaving like this?” 

“It’s\marginnote{2.1.9} true, sir.” 

The\marginnote{2.1.10} Buddha rebuked them … He then gave a teaching and addressed the monks: 

\scrule{“When the Sangha is divided and the monks are behaving contrary to the Teaching and are not on friendly terms, they should sit down and reflect, ‘We won’t behave improperly by body or speech, such as grabbing one another.’ When the Sangha is divided, but the monks are behaving in accordance with the Teaching and are on friendly terms, they should sit down one seat apart.”\footnote{Sp 3.456: \textit{\textsanskrit{Ekekaṁ} \textsanskrit{āsanaṁ} \textsanskrit{antaraṁ} \textsanskrit{katvā} \textsanskrit{nisīditabbaṁ}}, “They should sit down, having made a gap between each seat.” Presumably this refers to any situation where monks from different sides are sitting next to each other. } }

The\marginnote{2.2.1} monks were also arguing and disputing in the midst of the Sangha, attacking one another verbally, and were unable to resolve that legal issue. A certain monk went to the Buddha, bowed, and told him what was happening, adding, “Sir, please go to those monks out of compassion.” The Buddha consented by remaining silent. 

He\marginnote{2.2.8} then went to those monks, sat down on the prepared seat, and said, “Enough, monks, don’t quarrel and dispute.” 

A\marginnote{2.2.10} certain monk who spoke contrary to the Teaching replied, “Wait, sir, you’re the Lord of the Teaching. Be at ease and enjoy the happiness of meditation. We’ll face the consequences of this quarrelling and disputing.” The Buddha repeated his appeal to those monks, but got the same reply. 

\section*{2. The account of \textsanskrit{Dīghāvu} }

The\marginnote{2.3.1} Buddha then said: 

“At\marginnote{2.3.2} one time in Benares, monks, there was a king of \textsanskrit{Kāsi} called Brahmadatta. He was rich and powerful, had many vehicles and transport animals, and possessed a large kingdom and much wealth. Then there was \textsanskrit{Dīghīti}, the king of Kosala, who was poor and had little power, who had few vehicles and transport animals, and who possessed only a small kingdom and little wealth. 

At\marginnote{2.3.4} one time King Brahmadatta, armed with his fourfold army, marched out to attack King \textsanskrit{Dīghīti}. When King \textsanskrit{Dīghīti} heard about this, he reflected on King Brahmadatta’s superior wealth and power, and he concluded, ‘I’m incapable of repelling even a single strike from Brahmadatta. Let me flee the town before he arrives.’ 

And\marginnote{2.3.8} he fled the town together with his queen. King Brahmadatta then conquered and seized King \textsanskrit{Dīghīti}’s army, vehicles, and transport animals, as well as his country and wealth. 

King\marginnote{2.3.10} \textsanskrit{Dīghīti} and his wife set out for Benares. When they eventually arrived, they stayed in the house of a potter on the edge of the town, disguised as wanderers. 

Soon\marginnote{2.4.1} the queen became pregnant. She craved to see the fully equipped fourfold army arrayed on even ground at sunrise and to drink water from the washing of swords. She told the king. He said, ‘How can we possibly achieve this when things are so difficult for us?’ 

She\marginnote{2.4.5} replied, ‘Well, if I don’t get it, I’ll die.’ 

At\marginnote{2.5.1} that time King Brahmadatta had a brahmin counselor who was a friend of King \textsanskrit{Dīghīti}. King \textsanskrit{Dīghīti} went to his friend and told him about his wife’s pregnancy and craving. The brahmin replied, ‘Well then, let me see the queen.’ 

The\marginnote{2.5.5} queen then went to that brahmin. When he saw her coming, he got up from his seat, arrange his upper robe over one shoulder, raise his joined palms, and uttered a heartfelt exclamation three times: 

‘You\marginnote{2.5.7} have the king of Kosala in your womb!’ And he added, ‘Be pleased, lady. You’ll get to see the fully equipped fourfold army arrayed on even ground at sunrise and to drink water from the washing of swords.’ 

The\marginnote{2.6.1} brahmin counselor then went to King Brahmadatta and said, ‘The omens are such, sir, that tomorrow you should have the fully equipped fourfold army arrayed on even ground at sunrise and have the swords washed.’ The king told his people to act accordingly. As a consequence, the queen was able to satisfy her craving. 

When\marginnote{2.6.5} she reached her term, the queen gave birth to a son. They called him \textsanskrit{Dīghāvu}. Soon enough Prince \textsanskrit{Dīghāvu} became self-reliant.\footnote{\textit{\textsanskrit{Viññutaṁ} \textsanskrit{pāpuṇi}}, literally, “reached discernment”. Commenting on a similar context at \href{https://suttacentral.net/pli-tv-kd20/en/brahmali\#25.1.6}{Kd 20:25.1.6}, Sp 4.432 explains: \textit{\textsanskrit{Yāva} so \textsanskrit{dārako} \textsanskrit{viññutaṁ} \textsanskrit{pāpuṇātīti} \textsanskrit{yāva} \textsanskrit{khādituṁ} \textsanskrit{bhuñjituṁ} \textsanskrit{nahāyituñca} \textsanskrit{maṇḍituñca} attano \textsanskrit{dhammatāya} \textsanskrit{sakkotīti} attho}, “\textit{\textsanskrit{Yāva} so \textsanskrit{dārako} \textsanskrit{viññutaṁ} \textsanskrit{pāpuṇāti}} means until he is able to eat, bathe, and groom himself.” } King \textsanskrit{Dīghīti} thought, ‘This King Brahmadatta has caused us much misfortune; he’s taken our army, our vehicles and transport animals, and our country and wealth. If he finds out about us, he’ll kill all three of us. Let me take Prince \textsanskrit{Dīghāvu} to live out of town.’ And he did just that. As he was living outside of town, Prince \textsanskrit{Dīghāvu} was soon training in all branches of knowledge. 

At\marginnote{2.8.1} this time King \textsanskrit{Dīghīti}’s old barber was living at King Brahmadatta’s court. On one occasion he saw King \textsanskrit{Dīghīti} and his wife staying in that potter’s house, disguised as wanderers. He then went to King Brahmadatta and told him. The king ordered his people to get King \textsanskrit{Dīghīti} and his wife. When they had done so, he said, ‘Bind their arms behind their backs with a strong rope and shave their heads. Parade them from street to street and square to square to the beat of a harsh drum. Then take them out of town through the southern gate, cut them in four, and place the pieces at the four directions.’ Saying, ‘Yes, sir,’ they bound and shaved King \textsanskrit{Dīghīti} and his wife, and paraded them as instructed. 

Just\marginnote{2.10.1} then Prince \textsanskrit{Dīghāvu} thought, ‘I haven’t seen my parents for a long time. Why don’t I pay them a visit?’ When he entered Benares, he saw what was happening to his parents. As he approached them, King \textsanskrit{Dīghīti} said to him, ‘My dear \textsanskrit{Dīghāvu}, see neither long nor short. For hatred never ends through hatred; hatred only ends through love.’ 

The\marginnote{2.11.1} people there said to King \textsanskrit{Dīghīti}, ‘You’re insane, King \textsanskrit{Dīghīti}, you’re babbling. Who’s \textsanskrit{Dīghāvu}? Who are you saying this to?’ 

‘I’m\marginnote{2.11.6} not insane, I’m not babbling. The wise will understand.’ 

King\marginnote{2.11.7} \textsanskrit{Dīghīti} repeated what he had said to the prince a second and a third time, and the people there reacted as before. 

Then,\marginnote{2.11.20} when the parading was finished, they took King \textsanskrit{Dīghīti} and his wife through the southern gate and cut them in four. They placed the pieces at the four directions, set up guard, and departed. 

Prince\marginnote{2.12.1} \textsanskrit{Dīghāvu} entered Benares, brought back some alcohol, and gave it to the guards. When they were lying drunken on the ground, he collected sticks, built a funeral pyre, and lifted his parents’ bodies on top. He then lit the pyre, and raising his joined palms, he circumambulated it with his right side toward it. 

Just\marginnote{2.12.3} then King Brahmadatta was up in his finest stilt house, and he saw Prince \textsanskrit{Dīghāvu} doing those funeral rites. He thought, ‘No doubt this is a relative of King \textsanskrit{Dīghīti}. This is surely a sign of trouble for me, in that nobody has told me.’ 

The\marginnote{2.13.1} prince then went into the wilderness and cried his heart out. Wiping away his tears, he entered Benares and went to the elephant stables next to the royal compound. He said to the elephant trainer, ‘Teacher, I wish to learn your profession.’ 

‘Well\marginnote{2.13.3} then, young brahmin, I’ll teach you.’ 

Soon\marginnote{2.13.4} the prince was getting up early in the morning, singing sweetly and playing his lute in the elephant stables. King Brahmadatta, too, was getting up early, and he heard that music. He asked his people who it was. They replied that it was a young brahmin who was an apprentice of such-and-such an elephant trainer. 

‘Well\marginnote{2.14.3} then, bring him here.’ 

They\marginnote{2.14.4} brought the prince, and the king asked him whether he was the one who had been singing and playing the lute. When the prince confirmed that it was he, the king said, ‘Well then, sing and play right here.’ \textsanskrit{Dīghāvu} consented and did his best to please the king. The king said, ‘Now then, young man, please attend on me.’ The prince agreed. 

The\marginnote{2.14.11} prince then got up before the king and went to bed after him. He willingly performed any services and was pleasant in his conduct and speech. Soon the king put the prince in an intimate position of trust. 

On\marginnote{2.15.1} one occasion the king said to the prince, ‘Listen, young man. Harness a chariot, and let’s go hunting.’ He did as asked and told the king, ‘Sir, the chariot is ready. You may leave when you’re ready.’ The king mounted the chariot, with the prince driving it. He then drove the chariot away from the army. 

When\marginnote{2.15.8} they had gone a long way, the king said to the prince, ‘Listen, unharness the chariot. I’m tired. I wish to lie down.’ He did as asked and then sat down cross-legged on the ground. The king lay down, resting his head on the prince’s lap. And because he was tired, he quickly fell asleep. The prince thought, ‘This king has caused us much misfortune. He took our army, our vehicles and transport animals, and our country and wealth. He killed my mother and father. This is my chance to take revenge.’ And he drew his sword from its scabbard. 

He\marginnote{2.16.6} then thought, ‘At the time of his death, my father said to me, “My dear \textsanskrit{Dīghāvu}, see neither long nor short. For hatred never ends through hatred; hatred only ends through love.” It wouldn’t be right for me to ignore my father’s advice.’ And he returned the sword to its scabbard. 

A\marginnote{2.16.11} second and a third time he had the same thoughts, and each time he ended up returning the sword to its scabbard. 

Just\marginnote{2.16.28} then King Brahmadatta suddenly got up, frightened and alarmed. The prince asked him what was the matter, and the king said, ‘I just dreamed that Prince \textsanskrit{Dīghāvu}, the son of \textsanskrit{Dīghīti} the king of Kosala, attacked me with a sword.’ Seizing the king’s head with his left hand and drawing his sword with his right hand, the prince said to the king, ‘Sir, I’m that Prince \textsanskrit{Dīghāvu}, the son of \textsanskrit{Dīghīti} the king of Kosala. You’ve caused us much misfortune. You took our army, our vehicles and transport animals, and our country and wealth. You killed my mother and father. This is my chance to take revenge.’ 

The\marginnote{2.17.7} king bowed down with his head at the prince’s feet and said, ‘Dear \textsanskrit{Dīghāvu}, please spare my life.’ 

‘Who\marginnote{2.17.9} am I to spare your life? Sir, it’s you who should spare mine.’ 

‘Well\marginnote{2.17.11} then, \textsanskrit{Dīghāvu}, if you spare my life, I’ll spare yours.’ 

The\marginnote{2.17.12} king and \textsanskrit{Dīghāvu} spared each other’s lives. They shook hands and made a vow not to harm one another. 

The\marginnote{2.17.13} king said to the prince, ‘Well then, \textsanskrit{Dīghāvu}, harness the chariot and let’s go.’ He did as asked and told to the king, ‘Sir, the chariot is ready. You may leave when you’re ready.’ The king mounted the chariot, with the prince driving it. And he drove it so that it soon rejoined the army. 

When\marginnote{2.18.1} he was back in Benares, the king gathered his court and said, ‘Now, let me ask you: if you saw Prince \textsanskrit{Dīghāvu}, the son of \textsanskrit{Dīghīti} the king of Kosala, what would you do to him?’ 

They\marginnote{2.18.3} variously replied, ‘Sir, we’d cut off his hands;’ ‘We’d cut off his feet;’ ‘We’d cut off both his hands and feet;’ ‘We’d cut off his ears;’ ‘We’d cut off his nose;’ ‘We’d cut off both his ears and nose;’ ‘We’d cut off his head.’ 

‘Well,\marginnote{2.18.11} this is Prince \textsanskrit{Dīghāvu}, the son of \textsanskrit{Dīghīti} the king of Kosala. You shouldn’t do anything to harm him. I’ve spared his life and he’s spared mine.’ 

Soon\marginnote{2.19.1} afterwards the king said to \textsanskrit{Dīghāvu}, ‘\textsanskrit{Dīghāvu}, what’s the meaning of that which your father told you at the time of his death?’ 

‘When\marginnote{2.19.5} he said, “Not long,” he meant, “Don’t harbor hate for a long time.” When he said, “Not short,” he meant, “Don’t hastily break with your friends.” And when he said, “For hatred never ends through hatred; hatred only ends through love,” he was referring to your killing of my mother and father. For if I had killed you, those who wish you well would’ve killed me, and those who wish me well would in turn have killed them. In this way the hatred would never end through hatred. But now you’ve spared my life and I’ve spared yours. In this way hatred ends through love.’ 

The\marginnote{2.20.1} king thought, ‘It’s amazing how wise \textsanskrit{Dīghāvu} is, as he’s able to fully understand the meaning of his father’s brief statement.’ He gave him back his father’s army, his vehicles and transport animals, and his country and wealth. And he also gave him his own daughter. 

“In\marginnote{2.20.4} this way, monks, those kings who had the authority to punish were actually patient and gentle. But right here, you who’ve gone forth on this well-proclaimed spiritual path, do you shine with your patience and gentleness?” 

A\marginnote{2.20.6} third time the Buddha said to those monks, “Enough, monks, don’t quarrel and dispute.” And a third time that monk who spoke contrary to the Teaching replied, “Wait, sir, you’re the Lord of the Teaching. Be at ease and enjoy the happiness of meditation. We’ll face the consequences of this quarreling and disputing.” 

The\marginnote{2.20.12} Buddha thought, “These foolish men are consumed by emotions. It’s not easy to persuade them,” and he got up from his seat and left. 

\scend{The first section for recitation on \textsanskrit{Dīghāvu} is finished. }

Then,\marginnote{3.1.1} after robing up in the morning, the Buddha took his bowl and robe and entered \textsanskrit{Kosambī} for alms. When he had completed his almsround, eaten his meal, and returned, he put his dwelling in order. He then took his bowl and robe, and while standing in the midst of the Sangha, he spoke these verses: 

\begin{verse}%
“When\marginnote{3.1.3} many voices shout at once, \\
No-one thinks they are a fool. \\
Even as the Sangha splits, \\
They do not think it through.\footnote{Literally, “They do not think there is something more.” Sp 3.464: \textit{\textsanskrit{Nāññaṁ} bhiyyo \textsanskrit{amaññarunti} koci ekopi “\textsanskrit{ahaṁ} \textsanskrit{bālo}”ti ca na \textsanskrit{maññittha}; bhiyyo ca \textsanskrit{saṅghasmiṁ} \textsanskrit{bhijjamāne} \textsanskrit{aññampi} \textsanskrit{ekaṁ} “\textsanskrit{mayhaṁ} \textsanskrit{kāraṇā} \textsanskrit{saṅgho} \textsanskrit{bhijjatī}”ti \textsanskrit{idaṁ} \textsanskrit{kāraṇaṁ} na \textsanskrit{maññitthāti} attho}, “The meaning of \textit{\textsanskrit{nāññaṁ} bhiyyo \textsanskrit{amaññarun}} is: not even one thinks, ‘I am a fool.’ And more, when the Sangha is being split, not even one other thinks of this action that, ‘The Sangha is being split because of me.’” } 

Forgetting\marginnote{3.1.7} to speak wisely, \\
They are obsessed by speech; \\
Saying whatever they like, \\
They don’t know what leads them on. 

‘They\marginnote{3.1.11} abused me, they hit me, \\
They defeated me, they robbed me.’ \\
For those who carry on like this, \\
Hatred cannot end. 

‘They\marginnote{3.1.15} abused me, they hit me, \\
They defeated me, they robbed me.’ \\
For those who do not carry on like this, \\
Hatred has an end. 

For\marginnote{3.1.19} never does hatred \\
End through hatred; \\
Only through love does it end—\\
This is an ancient law. 

Others\marginnote{3.1.23} do not know \\
That here we need restraint; \\
But there are those there who know,\footnote{Sp 3.464: \textit{Ye ca tattha \textsanskrit{vijānantīti} ye tattha \textsanskrit{paṇḍitā} “\textsanskrit{mayaṁ} \textsanskrit{maccusamīpaṁ} \textsanskrit{gacchāmā}”ti \textsanskrit{vijānanti}}, “\textit{Ye ca tattha \textsanskrit{vijānanti}}: those there who are wise, they understand: ‘We are going close to death.’” } \\
That quarrels end like this.\footnote{Sp 3.464: \textit{Tato sammanti \textsanskrit{medhagāti} \textsanskrit{evañhi} te \textsanskrit{jānantā} \textsanskrit{yonisomanasikāraṁ} \textsanskrit{uppādetvā} \textsanskrit{medhagānaṁ} \textsanskrit{kalahānaṁ} \textsanskrit{vūpasamāya} \textsanskrit{paṭipajjanti}}, “\textit{Tato sammanti \textsanskrit{medhagā}}: for they know this by giving rise to wise attention. They practice for the ending of quarrel and strife.” } 

Those\marginnote{3.1.27} breaking bones and killing, \\
Those taking cows, horses, and wealth, \\
Those plundering the country, \\
Even they can stay together—\\
Why then cannot you? 

If\marginnote{3.1.32} you find a discerning friend, \\
A steadfast companion, good to live with, \\
Then overcome all problems, \\
And go with them, glad and mindful. 

If\marginnote{3.1.36} you do not find a discerning friend, \\
A steadfast companion, good to live with, \\
Then like a king giving up his kingdom, \\
Wander alone like a mighty elephant in the forest. 

It’s\marginnote{3.1.40} better to wander alone, \\
For there is no friendship with fools. \\
Wander alone and do no bad, \\
Unconcerned, like a mighty elephant in the forest.” 

%
\end{verse}

\section*{3. The account of going to \textsanskrit{Bālakaloṇaka} }

After\marginnote{4.1.1} speaking these verses, the Buddha went to the village of \textsanskrit{Bālakaloṇaka}. At that time Venerable Bhagu was staying near that village. When Bhagu saw the Buddha coming, he prepared a seat and set out a foot stool, a foot scraper, and water for washing the feet. He then went out to meet the Buddha, receiving his bowl and robe. The Buddha sat down on the prepared seat and washed his feet. When Bhagu had bowed and sat down, the Buddha said to him, “I hope you’re keeping well, monk, I hope you’re getting by? I hope you’re not having any trouble getting almsfood?” 

“I’m\marginnote{4.1.9} keeping well, sir, I’m getting by. I’m having no trouble getting almsfood.” 

The\marginnote{4.1.11} Buddha instructed, inspired, and gladdened Bhagu with a teaching. He then got up from his seat and went to the Eastern Bamboo Park. 

\section*{4. The account of going to the Eastern Bamboo Park }

At\marginnote{4.2.1} this time Venerable Anuruddha, Venerable Nandiya, and Venerable Kimila were staying at the Eastern Bamboo Park. The park keeper saw the Buddha coming and said to him, “Ascetic, don’t enter this park. There are three gentlemen here, practicing for their own good. Please, don’t disturb them.” When Anuruddha heard the park keeper advising the Buddha, he said, “Please don’t block the Buddha. It’s our teacher who’s arrived.” Anuruddha then went to Nandiya and Kimila and said, “Come out, venerables, our teacher has arrived.” 

The\marginnote{4.3.1} three of them went out to meet the Buddha. One received his bowl and robe, one prepared a seat, and one put out a foot stool, a foot scraper, and water for washing the feet. The Buddha sat down on the prepared seat and washed his feet. When they had bowed and sat down, the Buddha said to them, “I hope you’re all keeping well, Anuruddha, I hope you’re getting by? I hope you’re not having any trouble getting almsfood?” 

“We’re\marginnote{4.3.7} keeping well, sir, we’re getting by. We’re not having any trouble getting almsfood.” 

“I\marginnote{4.3.9} hope you’re living together in peace and harmony, blending like milk and water, and regarding one another with affection?” 

“Yes,\marginnote{4.3.10} we are.” 

“And\marginnote{4.3.11} how do you do this?” 

“I\marginnote{4.4.1} think like this, ‘How fortunate I am to be living with such fellow monastics!’ And I do acts of good will toward them by body, speech, and mind, both in public and in private. I think, ‘Why don’t I set aside what I wish to do and instead do what these venerables wish?’ And that’s what I do. We’re separate in body, but it might seem as if we’re one in mind.” 

Nandiya\marginnote{4.4.10} and Kimila then repeated what Anuruddha had said. 

“I\marginnote{4.5.1} hope, Anuruddha, that you’re heedful and energetic?” 

“Yes,\marginnote{4.5.2} sir, we are.” 

“And\marginnote{4.5.3} how is it that you’re heedful and energetic?” 

“Whoever\marginnote{4.5.4} returns first from almsround in the village, he prepares the seats and sets out a foot stool, a foot scraper, and water for washing the feet. He washes the bowl for leftovers and puts it back out, and sets out water for drinking and water for washing. Whoever returns last from almsround may eat the leftovers, or he discards them where there are no cultivated plants\footnote{\textit{Harita} could in principle refer to all plants, but it is elsewhere defined as what is cultivated, see \href{https://suttacentral.net/pli-tv-bu-vb-pc19/en/brahmali\#2.1.14}{Bu Pc 19:2.1.14} and \href{https://suttacentral.net/pli-tv-bi-vb-pc9/en/brahmali\#2.1.14}{Bi Pc 9:2.1.14}. } or in water without life. He puts away the seats and also the foot stool, the foot scraper, and the water for washing the feet. He washes the bowl for leftovers and puts it away, puts away the water for drinking and the water for washing, and sweeps the dining hall.\footnote{“Dining hall” renders \textit{bhattagga}, literally, “an eating house”. The name suggests that the \textit{bhattagga} was a separate building for eating. They were found both in private houses and in monasteries, as can be seen from the present passage. Since they were part of houses, “refectory” is not a satisfactory rendering. The fact that kitchens are not mentioned separately may mean that they were part of the \textit{bhattagga}, except in monasteries. This is supported by a passage at \href{https://suttacentral.net/pli-tv-bu-vb-pj3/en/brahmali\#5.3.1}{Bu Pj 3:5.3.1} that mentions a cooking implement, a pestle, being stored in a \textit{bhattagga}. } Whoever sees that the pot for drinking water, the pot for washing water, or the waterpot in the restroom is empty fills it. If he can’t do it by himself, he calls someone over by hand signal, and they move it together. We don’t speak because of that. And every five days we sit together the whole night to discuss the Teaching.” 

\section*{5. The account of going to \textsanskrit{Pālileyyaka} }

The\marginnote{4.6.1} Buddha then instructed, inspired, and gladdened Venerable Anuruddha, Venerable Nandiya, and Venerable Kimila with a teaching. He then got up from his seat and set out wandering toward \textsanskrit{Pālileyyaka}. When he eventually arrived, he stayed in a protected forest grove, at the foot of an auspicious sal tree. 

Then,\marginnote{4.6.4} while he was reflecting in private, the Buddha thought, “Previously, when I was surrounded by those quarreling monks at \textsanskrit{Kosambī}, I wasn’t at ease. But now that I’m alone, away from those monks, I’m happy and at ease.” 

At\marginnote{4.6.7} that time there was a large bull elephant who lived surrounded by a herd—by males and females, by juveniles and babies. He ate grass with the tips broken off and drank muddy water. Other elephants ate the branches that he had pulled down. And when he was immersed in a pool, the female elephants came rubbing their bodies against his. He considered this and thought, “Why don’t I leave the herd and stay by myself?” 

He\marginnote{4.7.1} then left the herd and went to \textsanskrit{Pālileyyaka}, to where the Buddha was at the foot of the auspicious sal tree. And he attended on the Buddha, using his trunk to set out water for drinking and water for washing, and to clear the vegetation. 

He\marginnote{4.7.2} thought, “Previously, when I was surrounded by the other elephants, I wasn’t at ease. But now that I’m alone, away from those elephants, I’m happy and at ease.” 

After\marginnote{4.7.5} considering his own seclusion and reading the mind of the elephant, the Buddha uttered a heartfelt exclamation: 

\begin{verse}%
“The\marginnote{4.7.6} mind of this mighty elephant, \\
With tusks like chariot poles, \\
Agrees with the mind of the Sage, \\
Since they each delight in the forest solitude.” 

%
\end{verse}

When\marginnote{5.1.1} the Buddha had stayed at \textsanskrit{Pālileyyaka} for as long as he liked, he set out wandering toward \textsanskrit{Sāvatthī}. When he eventually arrived, he stayed in the Jeta Grove, \textsanskrit{Anāthapiṇḍika}’s Monastery. 

Soon\marginnote{5.1.4} the lay followers in \textsanskrit{Kosambī} considered, “These venerable monks at \textsanskrit{Kosambī} have caused us much misfortune. The Buddha himself left because he was troubled by them. Well then, let’s not bow down, rise up, raise our joined palms, or do acts of respect toward them. And let’s not honor, respect, esteem, or associate with them, nor give them almsfood. Then, they’ll either leave, disrobe, or reconcile with the Buddha.” And they did just that. 

Soon\marginnote{5.2.2} the monks at \textsanskrit{Kosambī} said, “Well then, let’s go to \textsanskrit{Sāvatthī} and resolve this legal issue in the presence of the Buddha.” 

\section*{6. The account of the eighteen grounds }

The\marginnote{5.2.4.1} monks at \textsanskrit{Kosambī} put their dwellings in order, took their bowls and robes, and went to \textsanskrit{Sāvatthī}. When Venerable \textsanskrit{Sāriputta} heard that they were coming, he went to the Buddha, bowed, sat down, and told him, adding, “Sir, how should I act toward these monks?” 

“Take\marginnote{5.3.7} your stand in accordance with the Teaching.” 

“And\marginnote{5.3.8} how do I know what accords with the Teaching and what doesn’t?” 

“There\marginnote{5.4.1} are eighteen grounds for knowing that someone is speaking contrary to the Teaching: 

\begin{enumerate}%
\item A monk proclaims what’s contrary to the Teaching as being in accordance with it,\footnote{Sp 4.351: \textit{\textsanskrit{Adhammaṁ} dhammoti \textsanskrit{dīpentītiādīsu} \textsanskrit{aṭṭhārasasu} \textsanskrit{bhedakaravatthūsu} \textsanskrit{suttantapariyāyena} \textsanskrit{tāva} dasa \textsanskrit{kusalakammapathā} dhammo, dasa \textsanskrit{akusalakammapathā} adhammo. \textsanskrit{Tathā} \textsanskrit{cattāro} \textsanskrit{satipaṭṭhānā}, \textsanskrit{cattāro} \textsanskrit{sammappadhānā}, \textsanskrit{cattāro} \textsanskrit{iddhipādā}, \textsanskrit{pañcindriyāni}, \textsanskrit{pañca} \textsanskrit{balāni}, satta \textsanskrit{bojjhaṅgā}, ariyo \textsanskrit{aṭṭhaṅgiko} maggoti \textsanskrit{sattatiṁsa} \textsanskrit{bodhipakkhiyadhammā} dhammo \textsanskrit{nāma}; tayo \textsanskrit{satipaṭṭhānā}, tayo \textsanskrit{sammappadhānā}, tayo \textsanskrit{iddhipādā}, cha \textsanskrit{indriyāni}, cha \textsanskrit{balāni}, \textsanskrit{aṭṭha} \textsanskrit{bojjhaṅgā}, \textsanskrit{navaṅgiko} maggoti ca \textsanskrit{cattāro} \textsanskrit{upādānā}, \textsanskrit{pañca} \textsanskrit{nīvaraṇā}, satta \textsanskrit{anusayā}, \textsanskrit{aṭṭha} \textsanskrit{micchattāti} ca \textsanskrit{ayaṁ} adhammo}, “In regard to the meaning of ‘they proclaim what’s contrary to the Teaching as being in accordance with it,’ etc., according to the exposition in the discourses of the eighteen grounds for schism, the ten wholesome ways of action are in accordance with the Teaching, while the ten unwholesome ways of action are contrary to the Teaching. In the same way, the thirty-seven aids to awakening—the four focuses of mindfulness, the four right efforts, the four bases for spiritual power, the five faculties, the five powers, the seven factors of awakening, the noble eightfold path—are in accordance with the Teaching; while the three focuses of mindfulness, the three right efforts, the three bases for spiritual power, the six faculties, the six powers, the eight factors of awakening, the noble ninefold path, as well as the four graspings, the five hindrances, the seven underlying tendencies, and the eight kinds of wrongness are all contrary to the Teaching.” } %
\item and what’s in accordance with the Teaching as contrary to it. %
\item He proclaims what’s contrary to the Monastic Law as being in accordance with it, %
\item and what’s in accordance with the Monastic Law as contrary to it. %
\item He proclaims what hasn’t been spoken by the Buddha as spoken by him, %
\item and what’s been spoken by the Buddha as not spoken by him. %
\item He proclaims what wasn’t practiced by the Buddha as practiced by him, %
\item and what was practiced by the Buddha as not practiced by him. %
\item He proclaims what wasn’t laid down by the Buddha as laid down by him, %
\item and what was laid down by the Buddha as not laid down by him. %
\item He proclaims a non-offense as an offense, %
\item and an offense as a non-offense. %
\item He proclaims a light offense as heavy, %
\item and a heavy offense as light. %
\item He proclaims a curable offense as incurable, %
\item and an incurable offense as curable. %
\item He proclaims a grave offense as minor, %
\item and a minor offense as grave. %
\end{enumerate}

And\marginnote{5.5.1} there are eighteen grounds for knowing that someone is speaking in accordance with the Teaching: 

\begin{enumerate}%
\item A monk proclaims what’s contrary to the Teaching as such, %
\item and what’s in accordance with the Teaching as such. %
\item He proclaims what’s contrary to the Monastic Law as such, %
\item and what’s in accordance with the Monastic Law as such. %
\item He proclaims what hasn’t been spoken by the Buddha as such, %
\item and what’s been spoken by the Buddha as such. %
\item He proclaims what wasn’t practiced by the Buddha as such, %
\item and what was practiced by the Buddha as such. %
\item He proclaims what wasn’t laid down by the Buddha as such, %
\item and what was laid down by the Buddha as such. %
\item He proclaims a non-offense as such, %
\item and an offense as such. %
\item He proclaims a light offense as light, %
\item and a heavy offense as heavy. %
\item He proclaims a curable offense as curable, %
\item and an incurable offense as incurable. %
\item He proclaims a grave offense as grave, %
\item and a minor offense as minor.” %
\end{enumerate}

When\marginnote{5.6.1} Venerable \textsanskrit{Mahāmoggallāna} heard … When Venerable \textsanskrit{Mahākassapa} heard … When Venerable \textsanskrit{Mahākaccāna} heard … When Venerable \textsanskrit{Mahākoṭṭhika} heard … When Venerable \textsanskrit{Mahākappina} heard … When Venerable \textsanskrit{Mahācunda} heard … When Venerable Anuruddha heard … When Venerable Revata heard … When Venerable \textsanskrit{Upāli} heard … When Venerable Ānanda heard … When Venerable \textsanskrit{Rāhula} heard that they were coming, he too went to the Buddha, bowed, sat down, and told him, adding, “Sir, how should I act toward these monks?” 

“Take\marginnote{5.6.17} your stand in accordance with the Teaching.” 

“And\marginnote{5.6.18} how do I know what accords with the Teaching and what doesn’t?” The Buddha told him, too, about the eighteen grounds for knowing that someone is speaking contrary to the Teaching 

and\marginnote{5.6.30} the eighteen grounds for knowing that someone is speaking in accordance with the Teaching. 

When\marginnote{5.7.1} \textsanskrit{Mahāpajāpati} \textsanskrit{Gotamī} heard that they were coming, she too went to the Buddha, bowed, and told him, adding, “Sir, how should I act toward these monks?” 

“Well,\marginnote{5.7.7} \textsanskrit{Gotamī}, listen to the teaching from both sides. Then approve of the views, beliefs, and persuasion of those who speak in accordance with the Teaching. And whatever support the Sangha of nuns seeks from the Sangha of monks, they should get it all from those who speak in accordance with the Teaching.” 

When\marginnote{5.8.1} \textsanskrit{Anāthapiṇḍika} heard that they were coming, he too went to the Buddha, bowed, sat down, and told him, adding, “Sir, how should I act toward these monks?” 

“Well,\marginnote{5.8.7} householder, make offerings to both sides and listen to their teachings. Then approve of the views, beliefs, and persuasion of those who speak in accordance with the Teaching.” 

When\marginnote{5.9.1} \textsanskrit{Visākhā} \textsanskrit{Migāramātā} heard that they were coming, she too went to the Buddha, bowed, sat down, and told him, adding, “Sir, how should I act toward these monks?” 

“Well,\marginnote{5.9.7} \textsanskrit{Visākhā}, make offerings to both sides and listen to their teachings. Then approve of the views, beliefs, and persuasion of those who speak in accordance with the Teaching.” 

Eventually\marginnote{5.10.1} those monks from \textsanskrit{Kosambī} arrived at \textsanskrit{Sāvatthī}. Venerable \textsanskrit{Sāriputta} went to the Buddha, bowed, sat down, and told him, adding, “How should we prepare dwellings for these monks?” 

“Give\marginnote{5.10.6} them dwellings in a separate place.” 

“But\marginnote{5.10.7} what should we do if there are no dwellings in a separate place?” 

“In\marginnote{5.10.8} that case, create separate resting places and then give them out. 

\scrule{Under no circumstances, \textsanskrit{Sāriputta}, should a dwelling be reserved for a more senior monk.\footnote{The point seems to be that if the incoming monks, at least one of whom now belong to a different Buddhist sect, are to stay in the same place as the other monks, then they must be given dwellings according to seniority. If, however, they are staying in a separate location, then seniority only counts within that location. Vmv 3.473: \textit{\textsanskrit{Vivittaṁ} \textsanskrit{katvāpi} \textsanskrit{dātabbanti} \textsanskrit{vuttattā} pana \textsanskrit{yathāvuḍḍhaṁ} \textsanskrit{varasenāsanaṁ} \textsanskrit{adatvā} \textsanskrit{vuḍḍhānampi} \textsanskrit{asaññatānaṁ} \textsanskrit{saññatehi} \textsanskrit{vivittaṁ} \textsanskrit{katvā} \textsanskrit{dātabbanti} \textsanskrit{daṭṭhabbaṁ}}, “Because of what has been said, ‘Create separate resting places and then give them out’ is to be understood like this: not having given the best dwellings according to seniority, (the dwellings) are to be given out after separating the unrestrained senior monastics from the restrained ones.” } If you do, you commit an offense of wrong conduct.” }

“And\marginnote{5.10.11} what should we do regarding food and requisites?” 

“Food\marginnote{5.10.12} and requisites should be distributed equally to everyone.” 

\section*{7. The instruction to readmit }

Then\marginnote{5.11.1} that ejected monk reflected on the Teaching and the Monastic Law, and he concluded, “This is an offense and I’ve committed it. I’ve been ejected, for the legal procedure was legitimate, irreversible, and fit to stand.” He went to those who were siding with him and told them what he had been thinking, adding, “Come, venerables, please readmit me.” 

They\marginnote{5.12.1} then took that monk to the Buddha, bowed, sat down, and told him what had happened, adding, “Sir, what should we do now?” 

“This\marginnote{5.12.11} is an offense, monks, and this monk has committed it. He’s been ejected, for the legal procedure was legitimate, irreversible, and fit to stand. But since he recognizes this, he should be readmitted.” 

\section*{8. Discussion of unity in the Sangha }

Soon\marginnote{5.13.1} afterwards the monks who had been siding with the ejected monk readmitted him. They then went to the monks who had ejected him and said, “This monk has recognized that he had committed an offense and was ejected. He’s now been readmitted. Because of this, the basis for the arguments and disputes in the Sangha, for the schism, fracture, and separation in the Sangha, has been removed.\footnote{A literal translation of the Pali might read as follows: “In regard to which reason there was quarrel, argument, conflict, dispute, schism, fracture, division, and separation in the Sangha, this monk has committed, and has been ejected, and he has seen, and he has been reinstated.” I understand this to mean that the basis for the conflict has been dealt with. } To resolve this matter, let’s unify the Sangha.” 

The\marginnote{5.13.4} monks who had done the ejecting went to the Buddha, bowed, sat down, and told him what had happened, adding, “How should we proceed with this?” 

“This\marginnote{5.14.1} being the case, you should resolve this matter by unifying the Sangha. And it should be done like this. Everyone should gather in one place, including those who are sick. No-one should give their consent. A competent and capable monk should then inform the Sangha: 

‘Please,\marginnote{5.14.6} venerables, I ask the Sangha to listen. This monk has recognized that he had committed an offense and was ejected. He’s now been readmitted. Because of this, the basis for the arguments and disputes in the Sangha, for the schism, fracture, and separation in the Sangha, has been removed. If the Sangha is ready, let’s resolve this matter by unifying the Sangha. This is the motion. 

Please,\marginnote{5.14.10} venerables, I ask the Sangha to listen. This monk has recognized that he had committed an offense and was ejected. He’s now been readmitted. Because of this, the basis for the arguments and disputes in the Sangha, for the schism, fracture, and separation in the Sangha, has been removed. The Sangha resolves this matter by unifying the Sangha. Any monk who approves of resolving this matter by unifying Sangha should remain silent. Any monk who doesn’t approve should speak up. 

The\marginnote{5.14.14} Sangha has resolved this matter by unifying the Sangha. The schism in the Sangha has come to an end. The fracture in the Sangha has come to an end. The separation in the Sangha has come to an end. The Sangha approves and is therefore silent. I’ll remember it thus.’ 

The\marginnote{5.14.17} observance-day ceremony, the recitation of the Monastic Code, should be done straightaway.” 

\section*{9. \textsanskrit{Upāli}’s questions about unity in the Sangha }

Soon\marginnote{6.1.1} afterwards Venerable \textsanskrit{Upāli} went to the Buddha, bowed, sat down, and said, “Sir, if the basis for the arguments and disputes in the Sangha, for the schism, fracture, and separation in the Sangha, hasn’t been decided by the Sangha, hasn’t been resolved by the Sangha, yet the Sangha unifies the Sangha—is that unity in the Sangha legitimate?” 

“That\marginnote{6.1.4} unity in the Sangha is illegitimate.” 

“If\marginnote{6.1.5} the basis for the arguments and disputes in the Sangha, for the schism, fracture, and separation in the Sangha, has been decided by the Sangha, has been resolved by the Sangha, and the Sangha then unifies the Sangha—is that unity in the Sangha legitimate?” 

“That\marginnote{6.1.6} unity in the Sangha is legitimate.” 

“And\marginnote{6.2.1} sir, how many kinds of unity in the Sangha are there?” 

“There\marginnote{6.2.2} are two kinds of unity in the Sangha. There’s the unity in the Sangha where the wording is fulfilled, but not the purpose. And there’s the unity in the Sangha where both the wording and the purpose are fulfilled. If the basis for the arguments and disputes in the Sangha, for the schism, fracture, and separation in the Sangha, hasn’t been decided by the Sangha, hasn’t been resolved by the Sangha, yet the Sangha unifies the Sangha, this is called unity in the Sangha where the wording is fulfilled, but not the purpose. If the basis for the arguments and disputes in the Sangha, for the schism, fracture, and separation in the Sangha, has been decided by the Sangha, has been resolved by the Sangha, and the Sangha then unifies the Sangha, this is called unity in the Sangha where both the wording and the purpose are fulfilled.” 

\textsanskrit{Upāli}\marginnote{6.3.1} then got up from his seat, arrange his upper robe over one shoulder, raise his joined palms, and spoke to the Buddha in verse: 

\begin{verse}%
“In\marginnote{6.3.2} regard to the duties and discussions of the Sangha, \\
In regard to the business that arises and the investigations—\\
A person of great value, how does he handle these? \\
How is a monk fit to deal with these?” 

“Blameless\marginnote{6.3.6} in the basic morality, \\
Watching his own behavior, with senses well-restrained—\\
His enemies cannot legitimately criticize him; \\
There’s nothing for them to correct in him. 

Having\marginnote{6.3.10} such purity of conduct, \\
Enabled, he speaks confidently; \\
Without fear, he doesn’t tremble in a gathering; \\
He doesn’t neglect the meaning and speaks naturally. 

If\marginnote{6.3.14} then asked a question in a gathering, \\
He’s neither shy nor timid. \\
His words are timely and pertinent; \\
He watchfully satisfies a discerning gathering. 

Respectful\marginnote{6.3.18} of more senior monks, \\
Having confidence in his teacher, \\
Able to investigate, clever in discussion, \\
Skilled in defeating his opponents.\footnote{Sp 3.477: \textit{Viraddhikovidoti \textsanskrit{viraddhaṭṭhānakusalo}}, “\textit{Viraddhikovido}: skilled in the cases of failure.” That is, the failure of his opponents. } 

Wherever\marginnote{6.3.22} his opponents turn, he refutes them, \\
And the crowd is convinced. \\
He doesn’t abandon his position, \\
Yet answers questions without hurting anyone. 

He’s\marginnote{6.3.26} capable of acting as messenger, \\
And about the business of the Sangha, they speak to him. \\
When speaking, or sent out by the community of monks,\footnote{Sp 3.477: \textit{\textsanskrit{Karaṁ} vacoti \textsanskrit{vacanaṁ} karonto}, “\textit{\textsanskrit{Karaṁ} vaco}: when doing speech.” } \\
He doesn’t think, ‘I’m doing it.’ 

As\marginnote{6.3.30} far as the actions by which one commits offenses, \\
And how they’re cleared, \\
Both these analyses he has learned well. \\
He’s skilled in the ways of clearing offenses. 

If\marginnote{6.3.34} one is sent away for one’s conduct, \\
But once sent away one acts rightly, \\
There’s readmittance for one who lives thus. \\
This too he knows, the one skilled in analysis. 

Respectful\marginnote{6.3.38} of more senior monks, \\
Yet whether junior, senior, or of middle standing, \\
The wise practice for the benefit of the many—\\
Such a monk is fit to deal with these.” 

%
\end{verse}

\scendsutta{The tenth chapter on those from \textsanskrit{Kosambī} is finished. }

\scuddanaintro{This is the summary: }

\begin{scuddana}%
“The\marginnote{6.3.44} splendid Victor was in \textsanskrit{Kosambī}, \\
When disputing for not seeing an offense; \\
One should not eject for just any offense, \\
One should confess an offense out of faith. 

Just\marginnote{6.3.48} there inside the monastery zone, \\
And just \textsanskrit{Bālaka}, \textsanskrit{Vaṁsadā}; \\
And \textsanskrit{Pālileyyā}, \textsanskrit{Sāvatthī}, \\
And \textsanskrit{Sāriputta}, Kolita. 

\textsanskrit{Mahākassapa},\marginnote{6.3.52} and \textsanskrit{Kaccāna}, \\
\textsanskrit{Koṭṭhika}, and with Kappina; \\
\textsanskrit{Mahācunda}, Anuruddha, \\
And both Revata and \textsanskrit{Upāli}. 

Ānanda,\marginnote{6.3.56} and also \textsanskrit{Rāhula}, \\
\textsanskrit{Gotamī}, \textsanskrit{Anāthapiṇḍika}; \\
And separate dwellings, \\
And food and requisites equally. 

No-one\marginnote{6.3.60} is to give their consent, \\
Questioned by \textsanskrit{Upāli}; \\
Blameless in morality, \\
Harmonious in the Teaching of the Victor.” 

%
\end{scuddana}

\scendsutta{The chapter connected with \textsanskrit{Kosambī} is finished. }

\scendbook{The Great Division is finished. }

\scendbook{The canonical text of the Great Division is finished. }

%
\backmatter%
%
\chapter*{Appendices}
\addcontentsline{toc}{chapter}{Appendices}
\markboth{Appendices}{Appendices}

\emph{Appendices for all volumes may be found at the end of the first volume, The Great Analysis, part I.}

%
\chapter*{Colophon}
\addcontentsline{toc}{chapter}{Colophon}
\markboth{Colophon}{Colophon}

\section*{The Translator}

Bhikkhu Brahmali was born Norway in 1964. He first became interested in Buddhism and meditation in his early 20s after a visit to Japan. Having completed degrees in engineering and finance, he began his monastic training as an anagarika (keeping the eight precepts) in England at Amaravati and Chithurst Buddhist Monastery.

After hearing teachings from Ajahn Brahm he decided to travel to Australia to train at Bodhinyana Monastery. Bhikkhu Brahmali has lived at Bodhinyana Monastery since 1994, and was ordained as a Bhikkhu, with Ajahn Brahm as his preceptor, in 1996. In 2015 he entered his 20th Rains Retreat as a fully ordained monastic and received the title Maha Thera (Great Elder).

Bhikkhu Brahmali’s knowledge of the Pali language and of the Suttas is excellent. Bhikkhu Bodhi, who translated most of the Pali Canon into English for Wisdom Publications, called him one of his major helpers for the 2012 translation of \emph{The Numerical Discourses of the Buddha}. He has also published two essays on Dependent Origination and a book called \emph{The Authenticity of the Early Buddhist Texts} with the Buddhist Publication Society in collaboration with Bhante Sujato.

The monastics of the Buddhist Society of WA (BSWA) often turn to him to clarify Vinaya (monastic discipline) or Sutta questions. They also greatly appreciate his Sutta and Pali classes. Furthermore he has been instrumental in most of the building and maintenance projects at Bodhinyana Monastery and at the emerging Hermit Hill property in Serpentine.

\section*{Creation Process}

Translated from the Pali. The primary source was the \textsanskrit{Mahāsaṅgīti} edition, with occasional reference to other Pali editions, especially the \textsanskrit{Chaṭṭha} \textsanskrit{Saṅgāyana} edition and the Pali Text Society edition. I cross-checked with I.B. Horner’s English translation, “The Book of the Discipline”, as well as Bhikkhu \textsanskrit{Ñāṇatusita}’s “A Translation and Analysis of the \textsanskrit{Pātimokkha}” and Ajahn \textsanskrit{Ṭhānissaro}’s “Buddhist Monastic Code”.

\section*{The Translation}

This is the first complete translation of the Vinaya \textsanskrit{Piṭaka} in English. The aim has been to produce a translation that is easy to read, clear, and accurate, and also modern in vocabulary and style.

\section*{About SuttaCentral}

SuttaCentral publishes early Buddhist texts. Since 2005 we have provided root texts in Pali, Chinese, Sanskrit, Tibetan, and other languages, parallels between these texts, and translations in many modern languages. Building on the work of generations of scholars, we offer our contribution freely.

SuttaCentral is driven by volunteer contributions, and in addition we employ professional developers. We offer a sponsorship program for high quality translations from the original languages. Financial support for SuttaCentral is handled by the SuttaCentral Development Trust, a charitable trust registered in Australia.

\section*{About Bilara}

“Bilara” means “cat” in Pali, and it is the name of our Computer Assisted Translation (CAT) software. Bilara is a web app that enables translators to translate early Buddhist texts into their own language. These translations are published on SuttaCentral with the root text and translation side by side.

\section*{About SuttaCentral Editions}

The SuttaCentral Editions project makes high quality books from selected Bilara translations. These are published in formats including HTML, EPUB, PDF, and print.

You are welcome to print any of our Editions.

%
\end{document}