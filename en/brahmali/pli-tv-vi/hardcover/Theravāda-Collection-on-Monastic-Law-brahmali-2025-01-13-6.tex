\documentclass[12pt,openany]{book}%
\usepackage{lastpage}%
%
\usepackage{ragged2e}
\usepackage{verse}
\usepackage[a-3u]{pdfx}
\usepackage[inner=1in, outer=1in, top=.7in, bottom=1in, papersize={6in,9in}, headheight=13pt]{geometry}
\usepackage{polyglossia}
\usepackage[12pt]{moresize}
\usepackage{soul}%
\usepackage{microtype}
\usepackage{tocbasic}
\usepackage{realscripts}
\usepackage{epigraph}%
\usepackage{setspace}%
\usepackage{sectsty}
\usepackage{fontspec}
\usepackage{marginnote}
\usepackage[bottom]{footmisc}
\usepackage{enumitem}
\usepackage{fancyhdr}
\usepackage{emptypage}
\usepackage{extramarks}
\usepackage{graphicx}
\usepackage{relsize}
\usepackage{etoolbox}

% improve ragged right headings by suppressing hyphenation and orphans. spaceskip plus and minus adjust interword spacing; increase rightskip stretch to make it want to push a word on the first line(s) to the next line; reduce parfillskip stretch to make line length more equal . spacefillskip and xspacefillskip can be deleted to use defaults.
\protected\def\BalancedRagged{
\leftskip     0pt
\rightskip    0pt plus 10em
\spaceskip=1\fontdimen2\font plus .5\fontdimen3\font minus 1.5\fontdimen4\font
\xspaceskip=1\fontdimen2\font plus 1\fontdimen3\font minus 1\fontdimen4\font
\parfillskip  0pt plus 15em
\relax
}

\hypersetup{
colorlinks=true,
urlcolor=black,
linkcolor=black,
citecolor=black,
allcolors=black
}

% use a small amount of tracking on small caps
\SetTracking[ spacing = {25*,166, } ]{ encoding = *, shape = sc }{ 25 }

% add a blank page
\newcommand{\blankpage}{
\newpage
\thispagestyle{empty}
\mbox{}
\newpage
}

% define languages
\setdefaultlanguage[]{english}
\setotherlanguage[script=Latin]{sanskrit}

%\usepackage{pagegrid}
%\pagegridsetup{top-left, step=.25in}

% define fonts
% use if arno sanskrit is unavailable
%\setmainfont{Gentium Plus}
%\newfontfamily\Marginalfont[]{Gentium Plus}
%\newfontfamily\Allsmallcapsfont[RawFeature=+c2sc]{Gentium Plus}
%\newfontfamily\Noligaturefont[Renderer=Basic]{Gentium Plus}
%\newfontfamily\Noligaturecaptionfont[Renderer=Basic]{Gentium Plus}
%\newfontfamily\Fleuronfont[Ornament=1]{Gentium Plus}

% use if arno sanskrit is available. display is applied to \chapter and \part, subhead to \section and \subsection.
\setmainfont[
  FontFace={sb}{n}{Font = {Arno Pro Semibold}},
  FontFace={sb}{it}{Font = {Arno  Pro Semibold Italic}}
]{Arno Pro}

% create commands for using semibold
\DeclareRobustCommand{\sbseries}{\fontseries{sb}\selectfont}
\DeclareTextFontCommand{\textsb}{\sbseries}

\newfontfamily\Marginalfont[RawFeature=+subs]{Arno Pro Regular}
\newfontfamily\Allsmallcapsfont[RawFeature=+c2sc]{Arno Pro}
\newfontfamily\Noligaturefont[Renderer=Basic]{Arno Pro}
\newfontfamily\Noligaturecaptionfont[Renderer=Basic]{Arno Pro Caption}

% chinese fonts
\newfontfamily\cjk{Noto Serif TC}
\newcommand*{\langlzh}[1]{\cjk{#1}\normalfont}%

% logo
\newfontfamily\Logofont{sclogo.ttf}
\newcommand*{\sclogo}[1]{\large\Logofont{#1}}

% use subscript numerals for margin notes
\renewcommand*{\marginfont}{\Marginalfont}

% ensure margin notes have consistent vertical alignment
\renewcommand*{\marginnotevadjust}{-.17em}

% use compact lists
\setitemize{noitemsep,leftmargin=1em}
\setenumerate{noitemsep,leftmargin=1em}
\setdescription{noitemsep, style=unboxed, leftmargin=1em}

% style ToC
\DeclareTOCStyleEntries[
  raggedentrytext,
  linefill=\hfill,
  pagenumberwidth=.5in,
  pagenumberformat=\normalfont,
  entryformat=\normalfont
]{tocline}{chapter,section}


  \setlength\topsep{0pt}%
  \setlength\parskip{0pt}%

% define new \centerpars command for use in ToC. This ensures centering, proper wrapping, and no page break after
\def\startcenter{%
  \par
  \begingroup
  \leftskip=0pt plus 1fil
  \rightskip=\leftskip
  \parindent=0pt
  \parfillskip=0pt
}
\def\stopcenter{%
  \par
  \endgroup
}
\long\def\centerpars#1{\startcenter#1\stopcenter}

% redefine part, so that it adds a toc entry without page number
\let\oldcontentsline\contentsline
\newcommand{\nopagecontentsline}[3]{\oldcontentsline{#1}{#2}{}}

    \makeatletter
\renewcommand*\l@part[2]{%
  \ifnum \c@tocdepth >-2\relax
    \addpenalty{-\@highpenalty}%
    \addvspace{0em \@plus\p@}%
    \setlength\@tempdima{3em}%
    \begingroup
      \parindent \z@ \rightskip \@pnumwidth
      \parfillskip -\@pnumwidth
      {\leavevmode
       \setstretch{.85}\large\scshape\centerpars{#1}\vspace*{-1em}\llap{#2}}\par
       \nobreak
         \global\@nobreaktrue
         \everypar{\global\@nobreakfalse\everypar{}}%
    \endgroup
  \fi}
\makeatother

\makeatletter
\def\@pnumwidth{2em}
\makeatother

% define new sectioning command, which is only used in volumes where the pannasa is found in some parts but not others, especially in an and sn

\newcommand*{\pannasa}[1]{\clearpage\thispagestyle{empty}\begin{center}\vspace*{14em}\setstretch{.85}\huge\itshape\scshape\MakeLowercase{#1}\end{center}}

    \makeatletter
\newcommand*\l@pannasa[2]{%
  \ifnum \c@tocdepth >-2\relax
    \addpenalty{-\@highpenalty}%
    \addvspace{.5em \@plus\p@}%
    \setlength\@tempdima{3em}%
    \begingroup
      \parindent \z@ \rightskip \@pnumwidth
      \parfillskip -\@pnumwidth
      {\leavevmode
       \setstretch{.85}\large\itshape\scshape\lowercase{\centerpars{#1}}\vspace*{-1em}\llap{#2}}\par
       \nobreak
         \global\@nobreaktrue
         \everypar{\global\@nobreakfalse\everypar{}}%
    \endgroup
  \fi}
\makeatother

% don't put page number on first page of toc (relies on etoolbox)
\patchcmd{\chapter}{plain}{empty}{}{}

% global line height
\setstretch{1.05}

% allow linebreak after em-dash
\catcode`\—=13
\protected\def—{\unskip\textemdash\allowbreak}

% style headings with secsty. chapter and section are defined per-edition
\partfont{\setstretch{.85}\normalfont\centering\textsc}
\subsectionfont{\setstretch{.95}\normalfont\BalancedRagged}%
\subsubsectionfont{\setstretch{1}\normalfont\itshape\BalancedRagged}

% style elements of suttatitle
\newcommand*{\suttatitleacronym}[1]{\smaller[2]{#1}\vspace*{.3em}}
\newcommand*{\suttatitletranslation}[1]{\linebreak{#1}}
\newcommand*{\suttatitleroot}[1]{\linebreak\smaller[2]\itshape{#1}}

\DeclareTOCStyleEntries[
  indent=3.3em,
  dynindent,
  beforeskip=.2em plus -2pt minus -1pt,
]{tocline}{section}

\DeclareTOCStyleEntries[
  indent=0em,
  dynindent,
  beforeskip=.4em plus -2pt minus -1pt,
]{tocline}{chapter}

\newcommand*{\tocacronym}[1]{\hspace*{-3.3em}{#1}\quad}
\newcommand*{\toctranslation}[1]{#1}
\newcommand*{\tocroot}[1]{(\textit{#1})}
\newcommand*{\tocchapterline}[1]{\bfseries\itshape{#1}}


% redefine paragraph and subparagraph headings to not be inline
\makeatletter
% Change the style of paragraph headings %
\renewcommand\paragraph{\@startsection{paragraph}{4}{\z@}%
            {-2.5ex\@plus -1ex \@minus -.25ex}%
            {1.25ex \@plus .25ex}%
            {\noindent\normalfont\itshape\small}}

% Change the style of subparagraph headings %
\renewcommand\subparagraph{\@startsection{subparagraph}{5}{\z@}%
            {-2.5ex\@plus -1ex \@minus -.25ex}%
            {1.25ex \@plus .25ex}%
            {\noindent\normalfont\itshape\footnotesize}}
\makeatother

% use etoolbox to suppress page numbers on \part
\patchcmd{\part}{\thispagestyle{plain}}{\thispagestyle{empty}}
  {}{\errmessage{Cannot patch \string\part}}

% and to reduce margins on quotation
\patchcmd{\quotation}{\rightmargin}{\leftmargin 1.2em \rightmargin}{}{}
\AtBeginEnvironment{quotation}{\small}

% titlepage
\newcommand*{\titlepageTranslationTitle}[1]{{\begin{center}\begin{large}{#1}\end{large}\end{center}}}
\newcommand*{\titlepageCreatorName}[1]{{\begin{center}\begin{normalsize}{#1}\end{normalsize}\end{center}}}

% halftitlepage
\newcommand*{\halftitlepageTranslationTitle}[1]{\setstretch{2.5}{\begin{Huge}\uppercase{\so{#1}}\end{Huge}}}
\newcommand*{\halftitlepageTranslationSubtitle}[1]{\setstretch{1.2}{\begin{large}{#1}\end{large}}}
\newcommand*{\halftitlepageFleuron}[1]{{\begin{large}\Fleuronfont{{#1}}\end{large}}}
\newcommand*{\halftitlepageByline}[1]{{\begin{normalsize}\textit{{#1}}\end{normalsize}}}
\newcommand*{\halftitlepageCreatorName}[1]{{\begin{LARGE}{\textsc{#1}}\end{LARGE}}}
\newcommand*{\halftitlepageVolumeNumber}[1]{{\begin{normalsize}{\Allsmallcapsfont{\textsc{#1}}}\end{normalsize}}}
\newcommand*{\halftitlepageVolumeAcronym}[1]{{\begin{normalsize}{#1}\end{normalsize}}}
\newcommand*{\halftitlepageVolumeTranslationTitle}[1]{{\begin{Large}{\textsc{#1}}\end{Large}}}
\newcommand*{\halftitlepageVolumeRootTitle}[1]{{\begin{normalsize}{\Allsmallcapsfont{\textsc{\itshape #1}}}\end{normalsize}}}
\newcommand*{\halftitlepagePublisher}[1]{{\begin{large}{\Noligaturecaptionfont\textsc{#1}}\end{large}}}

% epigraph
\renewcommand{\epigraphflush}{center}
\renewcommand*{\epigraphwidth}{.85\textwidth}
\newcommand*{\epigraphTranslatedTitle}[1]{\vspace*{.5em}\footnotesize\textsc{#1}\\}%
\newcommand*{\epigraphRootTitle}[1]{\footnotesize\textit{#1}\\}%
\newcommand*{\epigraphReference}[1]{\footnotesize{#1}}%

% map
\newsavebox\IBox

% custom commands for html styling classes
\newcommand*{\scnamo}[1]{\begin{Center}\textit{#1}\end{Center}\bigskip}
\newcommand*{\scendsection}[1]{\begin{Center}\begin{small}\textit{#1}\end{small}\end{Center}\addvspace{1em}}
\newcommand*{\scendsutta}[1]{\begin{Center}\textit{#1}\end{Center}\addvspace{1em}}
\newcommand*{\scendbook}[1]{\bigskip\begin{Center}\uppercase{#1}\end{Center}\addvspace{1em}}
\newcommand*{\scendkanda}[1]{\begin{Center}\textbf{#1}\end{Center}\addvspace{1em}} % use for ending vinaya rule sections and also samyuttas %
\newcommand*{\scend}[1]{\begin{Center}\begin{small}\textit{#1}\end{small}\end{Center}\addvspace{1em}}
\newcommand*{\scendvagga}[1]{\begin{Center}\textbf{#1}\end{Center}\addvspace{1em}}
\newcommand*{\scrule}[1]{\textsb{#1}}
\newcommand*{\scadd}[1]{\textit{#1}}
\newcommand*{\scevam}[1]{\textsc{#1}}
\newcommand*{\scspeaker}[1]{\hspace{2em}\textit{#1}}
\newcommand*{\scbyline}[1]{\begin{flushright}\textit{#1}\end{flushright}\bigskip}
\newcommand*{\scexpansioninstructions}[1]{\begin{small}\textit{#1}\end{small}}
\newcommand*{\scuddanaintro}[1]{\medskip\noindent\begin{footnotesize}\textit{#1}\end{footnotesize}\smallskip}

\newenvironment{scuddana}{%
\setlength{\stanzaskip}{.5\baselineskip}%
  \vspace{-1em}\begin{verse}\begin{footnotesize}%
}{%
\end{footnotesize}\end{verse}
}%

% custom command for thematic break = hr
\newcommand*{\thematicbreak}{\begin{center}\rule[.5ex]{6em}{.4pt}\begin{normalsize}\quad\Fleuronfont{•}\quad\end{normalsize}\rule[.5ex]{6em}{.4pt}\end{center}}

% manage and style page header and footer. "fancy" has header and footer, "plain" has footer only

\pagestyle{fancy}
\fancyhf{}
\fancyfoot[RE,LO]{\thepage}
\fancyfoot[LE,RO]{\footnotesize\lastleftxmark}
\fancyhead[CE]{\setstretch{.85}\Noligaturefont\MakeLowercase{\textsc{\firstrightmark}}}
\fancyhead[CO]{\setstretch{.85}\Noligaturefont\MakeLowercase{\textsc{\firstleftmark}}}
\renewcommand{\headrulewidth}{0pt}
\fancypagestyle{plain}{ %
\fancyhf{} % remove everything
\fancyfoot[RE,LO]{\thepage}
\fancyfoot[LE,RO]{\footnotesize\lastleftxmark}
\renewcommand{\headrulewidth}{0pt}
\renewcommand{\footrulewidth}{0pt}}
\fancypagestyle{plainer}{ %
\fancyhf{} % remove everything
\fancyfoot[RE,LO]{\thepage}
\renewcommand{\headrulewidth}{0pt}
\renewcommand{\footrulewidth}{0pt}}

% style footnotes
\setlength{\skip\footins}{1em}

\makeatletter
\newcommand{\@makefntextcustom}[1]{%
    \parindent 0em%
    \thefootnote.\enskip #1%
}
\renewcommand{\@makefntext}[1]{\@makefntextcustom{#1}}
\makeatother

% hang quotes (requires microtype)
\microtypesetup{
  protrusion = true,
  expansion  = true,
  tracking   = true,
  factor     = 1000,
  patch      = all,
  final
}

% Custom protrusion rules to allow hanging punctuation
\SetProtrusion
{ encoding = *}
{
% char   right left
  {-} = {    , 500 },
  % Double Quotes
  \textquotedblleft
      = {1000,     },
  \textquotedblright
      = {    , 1000},
  \quotedblbase
      = {1000,     },
  % Single Quotes
  \textquoteleft
      = {1000,     },
  \textquoteright
      = {    , 1000},
  \quotesinglbase
      = {1000,     }
}

% make latex use actual font em for parindent, not Computer Modern Roman
\AtBeginDocument{\setlength{\parindent}{1em}}%
%

% Default values; a bit sloppier than normal
\tolerance 1414
\hbadness 1414
\emergencystretch 1.5em
\hfuzz 0.3pt
\clubpenalty = 10000
\widowpenalty = 10000
\displaywidowpenalty = 10000
\hfuzz \vfuzz
 \raggedbottom%

\title{Theravāda Collection on Monastic Law}
\author{Bhikkhu Brahmali}
\date{}%
% define a different fleuron for each edition
\newfontfamily\Fleuronfont[Ornament=9]{Arno Pro}

% Define heading styles per edition for chapter and section. Suttatitle can be either of these, depending on the volume. 

\let\oldfrontmatter\frontmatter
\renewcommand{\frontmatter}{%
\chapterfont{\setstretch{.85}\normalfont\centering}%
\sectionfont{\setstretch{.85}\normalfont\BalancedRagged}%
\oldfrontmatter}

\let\oldmainmatter\mainmatter
\renewcommand{\mainmatter}{%
\chapterfont{\setstretch{.85}\normalfont\centering}%
\sectionfont{\setstretch{.85}\normalfont\BalancedRagged}%
\oldmainmatter}

\let\oldbackmatter\backmatter
\renewcommand{\backmatter}{%
\chapterfont{\setstretch{.85}\normalfont\centering}%
\sectionfont{\setstretch{.85}\normalfont\BalancedRagged}%
\pagestyle{plainer}%
\oldbackmatter}

% for reasons, flat texts align too far in the margin in ToC, this fixes it. 
\renewcommand*{\tocacronym}[1]{\hspace*{0em}{#1}\quad}%
%
\begin{document}%
\normalsize%
\frontmatter%
\setlength{\parindent}{0cm}

\pagestyle{empty}

\maketitle

\blankpage%
\begin{center}

\vspace*{2.2em}

\halftitlepageTranslationTitle{Theravāda Collection on Monastic Law}

\vspace*{1em}

\halftitlepageTranslationSubtitle{A translation of the Pali Vinaya Piṭaka into English}

\vspace*{2em}

\halftitlepageFleuron{•}

\vspace*{2em}

\halftitlepageByline{translated and introduced by}

\vspace*{.5em}

\halftitlepageCreatorName{Bhikkhu Brahmali}

\vspace*{4em}

\halftitlepageVolumeNumber{Volume 6}

\smallskip

\halftitlepageVolumeAcronym{Pvr}

\smallskip

\halftitlepageVolumeTranslationTitle{The Compendium}

\smallskip

\halftitlepageVolumeRootTitle{Parivāra}

\vspace*{\fill}

\sclogo{0}
 \halftitlepagePublisher{SuttaCentral}

\end{center}

\newpage
%
\setstretch{1.05}

\begin{footnotesize}

\textit{Theravāda Collection on Monastic Law} is a translation of the Theravāda Vinayapiṭaka by Bhikkhu Brahmali.

\medskip

Creative Commons Zero (CC0)

To the extent possible under law, Bhikkhu Brahmali has waived all copyright and related or neighboring rights to \textit{Theravāda Collection on Monastic Law}.

\medskip

This work is published from Australia.

\begin{center}
\textit{This translation is an expression of an ancient spiritual text that has been passed down by the Buddhist tradition for the benefit of all sentient beings. It is dedicated to the public domain via Creative Commons Zero (CC0). You are encouraged to copy, reproduce, adapt, alter, or otherwise make use of this translation. The translator respectfully requests that any use be in accordance with the values and principles of the Buddhist community.}
\end{center}

\medskip

\begin{description}
    \item[Web publication date] 2021
    \item[This edition] 2025-01-13 01:01:43
    \item[Publication type] hardcover
    \item[Edition] ed3
    \item[Number of volumes] 6
    \item[Publication ISBN] 978-1-76132-006-4
    \item[Volume ISBN] 978-1-76132-012-5
    \item[Publication URL] \href{https://suttacentral.net/editions/pli-tv-vi/en/brahmali}{https://suttacentral.net/editions/pli-tv-vi/en/brahmali}
    \item[Source URL] \href{https://github.com/suttacentral/bilara-data/tree/published/translation/en/brahmali/vinaya}{https://github.com/suttacentral/bilara-data/tree/published/translation/en/brahmali/vinaya}
    \item[Publication number] scpub8
\end{description}

\medskip

Map of Jambudīpa is by Jonas David Mitja Lang, and is released by him under Creative Commons Zero (CC0).

\medskip

Published by SuttaCentral

\medskip

\textit{SuttaCentral,\\
c/o Alwis \& Alwis Pty Ltd\\
Kaurna Country,\\
Suite 12,\\
198 Greenhill Road,\\
Eastwood,\\
SA 5063,\\
Australia}

\end{footnotesize}

\newpage

\setlength{\parindent}{1em}%%
\tableofcontents
\newpage
\pagestyle{fancy}
%
\chapter*{Introduction to the \textsanskrit{Parivāra}, “The Compendium”}
\addcontentsline{toc}{chapter}{Introduction to the \textsanskrit{Parivāra}, “The Compendium”}
\markboth{Introduction to the \textsanskrit{Parivāra}, “The Compendium”}{Introduction to the \textsanskrit{Parivāra}, “The Compendium”}

\scbyline{Bhikkhu Brahmali, 2024}

The present volume is the last of six, the total of which constitutes a complete translation of the Vinaya \textsanskrit{Piṭaka}, the Monastic Law. This volume consists of the \textsanskrit{Parivāra}, “the Compendium”. In the present introduction, I will do a brief survey of the contents of volume 6 and make observations of points of particular interest. For a general introduction to the Monastic Law, see volume 1.

The word \textit{\textsanskrit{parivāra}} normally means “entourage”, as in a king’s retinue, or “accompaniment”, as in side dishes to the main part of a meal. In the present context, it would then seem to mean something that “accompanies” the Vinaya proper, that is, the Sutta-\textsanskrit{vibhaṅga} and the Khandhakas. As to its content, the \textsanskrit{Parivāra} is a detailed and condensed systematization of the most important rules and regulations of the first five volumes of the Vinaya. It is to reflect this context that I have chosen to render \textsanskrit{Parivāra} as Compendium.\footnote{Here is the definition of compendium from Oxford Dictionaries online: “A concise but detailed information about a particular subject, especially in a book or other publication”. }

The \textsanskrit{Parivāra} is significantly later than the rest of the Canonical Vinaya. As we shall see, the work speaks of the arrival of the \textsanskrit{Tipiṭaka} in Sri Lanka and gives a long list of Vinaya Masters at the \textsanskrit{Mahāvihāra},\footnote{At \href{https://suttacentral.net/pli-tv-pvr1.1/en/brahmali\#3.56}{Pvr~1.1:3.56}–19.4. } the most important monastery in Sri Lanka in the early centuries after its arrival there. Oskar von Hinüber estimates that it was composed at the earliest in the first century CE.\footnote{Hinüber, 2000, § 42. } I. B. Horner (BD, p. ix) quotes Rhys Davids and Oldenberg as follows:

\begin{quotation}%
“… [T]his work [the \textsanskrit{Parivāra}], an abstract of the other parts of the Vinaya, is in fact a very much later compilation, and probably the work of a Ceylonese Thera.”\footnote{Rhys Davids, 1881, part I, p. xxiv. }

%
\end{quotation}

The work as a whole has no discernible structure. Each chapter has its own inner logic, but there is no obvious connection between the individual chapters. The overall impression is of a work that has been haphazardly assembled over time, with new material simply added to the end on an ongoing basis.

Still, the \textsanskrit{Parivāra} has its own peculiar style. Most of the material is presented in a question-and-answer format,\footnote{Only a few chapters do not have this format, namely, Pvr 3, 9, 14–15, 17, and in part 13 and 21. The total page count of these is less than 20\% of the \textsanskrit{Parivāra}. } and occasionally as questions without answers, as in \href{https://suttacentral.net/pli-tv-pvr20/en/brahmali}{Pvr~20}. This, combined with its systematic organization of the Vinaya material, suggests that the \textsanskrit{Parivāra} was used as a manual for students. This is what I. B. Horner has to say of the matter:

\begin{quotation}%
“Indeed, to provide a manual for instructors and students may well have been a reason for its compilation.”\footnote{BD, vol. VI, p. x. }

%
\end{quotation}

It may even be that it was used to test advanced students, such as those who were preparing to be Vinaya teachers. This is suggested especially by Pvr 20, delightfully named “The sudorific verses”,\footnote{In plain English, “the sweat-inducing verses”. } which consists of a series of Vinaya conundrums that only especially knowledgeable monastics would be able to solve. The questions look a bit like an entrance exam to an elite university! The solutions are only found in the commentaries.

Overall, I refer more often to the commentaries in this volume than I have in the previous five. The reason for this is the significant number of cryptic passages throughout the \textsanskrit{Parivāra}.

Because the \textsanskrit{Parivāra} is mostly a systematic presentation of material from the rest of the Vinaya \textsanskrit{Piṭaka}, it does not contain much that is new. Occasionally, however, the \textsanskrit{Parivāra} does add details not found elsewhere. This is especially noticeable in Pvr 16, “The subdivision on the Robe-making ceremony”, where important details fill in the otherwise incomplete description at Kd 7. Similarly, Pvr 14 and 15 add details not found elsewhere in the Vinaya. These are examples of the \textsanskrit{Parivāra} acting as a kind of early commentary. This, then, is a secondary function of the \textsanskrit{Parivāra}.

To restate what we have seen so far, I would propose that the \textsanskrit{Parivāra} can be regarded as having at least three distinct purposes:

\begin{itemize}%
\item It is a concise restatement of the content of the earlier parts of the Vinaya \textsanskrit{Piṭaka}%
\item It contains questions to be used for testing students%
\item It is an early commentary that explains passages in the first five volume of the Canonical Vinaya texts.%
\end{itemize}

Before I move on to the content of the individual chapters, I need to briefly comment on the mnemonic verses at the end of each chapter, the so-called \textit{\textsanskrit{uddānas}}. As we found for the Khandhakas, the \textit{\textsanskrit{uddānas}} do not always fully reflect the content of the chapter they belong to. For instance, the \textit{\textsanskrit{uddāna}} that immediately follows Pvr 9, summarizing the content of the \textsanskrit{Parivāra} up to this point, is missing references to chapters 3 and 6, while chapter 8 is called “invitation ceremony” instead of the expected “observance-day ceremony”. Careful study of these anomalies may help us reconstruct the historical development of the \textsanskrit{Parivāra}, and as such this area deserves detailed investigation.

Another noteworthy aspect of the \textit{\textsanskrit{uddānas}} is that they do not always conform to what we find in the earlier volumes of the Vinaya. For instance, in the first two chapters of the \textsanskrit{Parivāra}, which are concerned with the \textsanskrit{Pātimokkha} rules and their analysis, we would expect the \textit{\textsanskrit{uddānas}} to be the same as what we have in the Sutta-\textsanskrit{vibhaṅga}. Yet this is not always the case. Here is a comparison of two parallel \textit{\textsanskrit{uddāna}} verses, each containing a list of the first ten \textit{bhikkhu nissaggiya \textsanskrit{pācittiya}} rules.

\begin{verse}%
\textbf{Sutta-\textsanskrit{vibhaṅga} (Bu NP 1–10):}

\textit{\textsanskrit{Ubbhataṁ} \textsanskrit{kathinaṁ} \textsanskrit{tīṇi}},\\
\textit{\textsanskrit{dhovanañca} \textsanskrit{paṭiggaho}};\\
\textit{\textsanskrit{Aññātakāni} \textsanskrit{tīṇeva}},\\
\textit{\textsanskrit{ubhinnaṁ} \textsanskrit{dūtakena} \textsanskrit{cāti}}.

“Three on the ended robe season,\\

And washing, receiving;\\

Three on those who are unrelated,\\

Of both, and with messenger.”\footnote{At \href{https://suttacentral.net/pli-tv-bu-vb-np10/en/brahmali\#2.4.11}{Bu~NP~10:2.4.11}. }

%
\end{verse}

\begin{verse}%
\textbf{\textsanskrit{Parivāra} (Bu NP 1–10):}

\textit{\textsanskrit{Dasekarattimāso} ca},\\
\textit{\textsanskrit{dhovanañca} \textsanskrit{paṭiggaho}};\\
\textit{\textsanskrit{Aññātaṁ} \textsanskrit{tañca} uddissa},\\
\textit{\textsanskrit{ubhinnaṁ} \textsanskrit{dūtakena} ca}.

“Ten, one day, and a month;\\

And washing, receiving;\\

Unrelated, and that one, for the sake of;\\

Of both, and with messenger.”\footnote{At \href{https://suttacentral.net/pli-tv-pvr1.1/en/brahmali\#116.1}{Pvr~1.1:116.1}. }

%
\end{verse}

As can be seen, the differences are quite substantial, including different names for the same rule. Similar differences exist for other rules, including Bu Pc 11–30.

Why are there such differences in texts that ostensibly hail from the same tradition? Could it be that the tradition is not the same after all? Or that it is mixed? Or might it be that the \textit{\textsanskrit{uddānas}} in the \textsanskrit{Parivāra} are a recent addition to a work that was never recited orally? Unfortunately, I will have to leave the exploration of these interesting questions for another occasion.

Now let us turn to a brief overview of the content of the 21 chapters of the \textsanskrit{Parivāra}.

\section*{Pvr 1.1–2.16}

Pvr 1 and 2, which are each divided into sixteen sections, \textit{\textsanskrit{vāras}}, are the two longest chapters of the \textsanskrit{Parivāra}. They would have been much longer except for the heavy abbreviations, yet even in their current form, they account for almost 40\% of the \textsanskrit{Parivāra}’s total page count in the PTS edition. They are a summary and classification of the content of the Sutta-\textsanskrit{vibhaṅgas}, the first three volumes of the present translation series. Pvr 1 concerns the \textsanskrit{Mahā}-\textsanskrit{vibhaṅga}, whereas Pvr 2 is about the \textsanskrit{Bhikkhunī}-\textsanskrit{vibhaṅga}. Both have a question-and-answer format.

\href{https://suttacentral.net/pli-tv-pvr1.1/en/brahmali}{Pvr~1.1}, “Questions and answers on the monks’ \textsanskrit{Pātimokkha} rules and their analysis”, classifies the rules according to a standard set of Vinaya criteria. These include whether the rules are main or subsidiary, whether they are universally applicable or not, whether they apply to both Sanghas or only one, whether they are moral transgressions or otherwise, etc. Some of these classifications are more perfunctory than meaningful. For instance, the question is raised of what the Monastic Code is, with the reply being that it is the training rules.\footnote{At \href{https://suttacentral.net/pli-tv-pvr1.1/en/brahmali\#3.34}{Pvr~1.1:3.34}. }

The last question for each rule is “Who handed it down?”, to which the reply is a long list of teachers, starting with \textsanskrit{Upāli} who recited the Vinaya at the first Council.\footnote{At \href{https://suttacentral.net/pli-tv-pvr1.1/en/brahmali\#3.56}{Pvr~1.1:3.56}–19.4. } It then lists another four teachers, all based in India, ending with Moggaliputta. According to the Pali tradition, Moggaliputta presided over the so-called third Council, which effectively authorized the school of Buddhism that initially flourished in Sri Lanka and later in much of South-east Asia, and which is now known as Theravada Buddhism.\footnote{The third Council is described in the Vinaya commentary, the \textsanskrit{Samantapāsādikā}. }

The same list continues with another five names, headed by Ashoka’s son Mahinda, who were the monks that brought the \textsanskrit{Tipiṭaka} from India to Sri Lanka.\footnote{At \href{https://suttacentral.net/pli-tv-pvr1.1/en/brahmali\#5.1}{Pvr~1.1:5.1}–6.4. } Then follows another 29 monks who are described as the Vinaya masters of Sri Lanka with the following words: “These mighty beings of great wisdom, knowers of the Monastic Law and skilled in the path, proclaimed the Collection of Monastic Law on the island of Sri Lanka”.\footnote{\textit{Ete \textsanskrit{nāgā} \textsanskrit{mahāpaññā}, \textsanskrit{vinayaññū} \textsanskrit{maggakovidā}, \textsanskrit{vinayaṁ} \textsanskrit{dīpe} \textsanskrit{pakāsesuṁ}, \textsanskrit{piṭakaṁ} \textsanskrit{tambapaṇṇiyāti}}. (\href{https://suttacentral.net/pli-tv-pvr1.1/en/brahmali\#19.1}{Pvr~1.1:19.1}) } This succession of Vinaya teachers, and especially Moggaliputta and Mahinda, deserves careful study. The overall pattern, including the full list of Vinaya masters, is repeated for every offense in the \textsanskrit{Pātimokkha}. The amount of repetition is huge.

Pvr 1.1 also includes an interesting passage on the little understood term \textit{abhivinaya}, and by extension the related word \textit{abhidhamma}. Here is that passage:

\begin{quotation}%
“What is the Monastic Law (\textit{vinaya}) there? What is concerned with the Monastic Law (\textit{abhivinaya}) there?” The rules (\textit{\textsanskrit{paññatti}}) are the Monastic Law. Their analysis (\textit{vibhatti}) is concerned with the Monastic Law.\footnote{\textit{Ko tattha vinayo, ko tattha abhivinayoti? \textsanskrit{Paññatti} vinayo, vibhatti abhivinayo.} (\href{https://suttacentral.net/pli-tv-pvr1.1/en/brahmali\#3.32}{Pvr~1.1:3.32}–3.35) }

%
\end{quotation}

\textit{Vibhatti} is just an alternative to the term \textit{\textsanskrit{vibhaṅga}}, both words being derived from the same underlying root and prefix, \textit{vi} + \textit{bhaj}. Here is the commentarial explanation:

\begin{quotation}%
The definitions (\textit{\textsanskrit{padabhājana}}) are called \textit{vibhatti}; for \textit{vibhatti} is a name for the analysis (\textit{\textsanskrit{vibhaṅga}}).\footnote{Sp 5.2: \textit{\textsanskrit{Vibhattīti} \textsanskrit{padabhājanaṁ} vuccati; \textsanskrit{vibhattīti} hi \textsanskrit{vibhaṅgassevetaṁ} \textsanskrit{nāmaṁ}.} }

%
\end{quotation}

We have seen in the introduction to volume 1 that the \textsanskrit{Vibhaṅga} is an early commentary on the \textsanskrit{Pātimokkha} rules, and as such it is “about” those rules. A reasonable rendering of the prefix \textit{abhi}-, then, is “about”, with \textit{abhivinaya} becoming “about the Vinaya”.

If we extend this understanding to \textit{abhidhamma}, we get the meaning “about the Dhamma”, which would then be a reference to a commentarial kind of literature. This suggests that in the earliest period the new \textit{abhidhamma} material was considered a commentary on the Dhamma. Rather than being the ultimate expression of the Dhamma, it was understood to be a secondary kind of literature.

I will just briefly summarize the content of the remainder of Pvr 1 and 2. The next section, \href{https://suttacentral.net/pli-tv-pvr1.2/en/brahmali}{Pvr~1.2}, “The number of offenses within each offense”, lists every class of offense that may be committed under each rule, including any derived offenses mentioned in the \textsanskrit{Vibhaṅga} material. The next five sections follow in a similar vein, listing a variety of matters in relation to each offense. Section 1.8 then repeats the content of the previous six sections. The text is so repetitive as to be almost entirely abbreviated away. This is even more true of next eight sections, Pvr 1.9–1.16, which are an almost verbatim repetition of the previous eight, with only a marginal change in wording.\footnote{The main difference in wording can be exemplified by Bu Ss 2: “offense entailing suspension \textbf{for} making physical contact with a woman” at \href{https://suttacentral.net/pli-tv-pvr1.1/en/brahmali\#44.1}{Pvr~1.1:44.1} vs. “offense entailing suspension \textbf{that is a result of} making physical contact with a woman” at \href{https://suttacentral.net/pli-tv-pvr1.9/en/brahmali\#15.1}{Pvr~1.9:15.1}. }

In Pvr 2.1–2.16 the entire process is repeated for the \textsanskrit{Bhikkhunī}-\textsanskrit{vibhaṅga}.

\section*{Pvr 3–6}

\href{https://suttacentral.net/pli-tv-pvr3/en/brahmali}{Pvr~3}, “The origination of offenses”, sets up an elaborate scheme of how the three doors of action—body, speech, and mind—in various combinations serve as the origination of offenses. In the opening verses, which are largely a homage to the Buddha, we encounter the word \textsanskrit{Parivāra}, the only time it occurs in the text itself. It is then said that those who are virtuous and love the Dhamma should train in this Compendium. Following the opening verses, the main content of this chapter is little more than a list of the various originations and which \textsanskrit{Pātimokkha} rules they are associated with. \href{https://suttacentral.net/pli-tv-pvr4/en/brahmali}{Pvr~4}, “More on the origination of offenses”, continues the discussion of Pvr 3, adding material on the relationship between the originations and other aspects of the training, such as the classes of offenses, the kinds of failures (\textit{vipatti}), and the kinds of legal issues.

This focus on the origination (\textit{\textsanskrit{samuṭṭhāna}}) of offenses is a hallmark of the \textsanskrit{Parivāra}. Whereas the rest of the Vinaya \textsanskrit{Piṭaka} hardly makes use of this term,\footnote{It is found twice in the Samatha-kkhandhaka at \href{https://suttacentral.net/pli-tv-kd14/en/brahmali\#14.6.1}{Kd~14:14.6.1}. } it is found over a thousand times in this text. Although intention, and therefore the nature of the origination of an offense, is a factor in a significant number of \textsanskrit{Pātimokkha} rules, it is here presented from a largely theoretical perspective. We can perhaps sense a shift in the priorities of the Sangha in the centuries after the Buddha passed away. It seems that parts of the Sangha were becoming more involved in academic pursuits and less in the practical implementation of the path to awakening.

\href{https://suttacentral.net/pli-tv-pvr5/en/brahmali}{Pvr~5}, “The legal issues and their settling”, concerns the four kinds of legal issues, \textit{\textsanskrit{adhikaraṇas}}, especially the causes for their arising and the means for settling them. This chapter is particularly dry. Much of the content is taken up with abstract relationships between the seven principles for settling legal issues.

Pvr 5 is closely connected to Kd 14, for instance in its use of the triad wholesome, unwholesome, and indeterminate, which is otherwise only found in the Abhidhamma and the later parts of the Sutta \textsanskrit{Piṭaka}. Further on, Pvr 5 uses the expression \textit{\textsanskrit{yasmiṁ} samaye}, “on whatever occasion”, in a way that is reminiscent of the Abhidhamma. The parallel expression in the Suttas is instead \textit{\textsanskrit{ekaṁ} \textsanskrit{samayaṁ}}, “on one occasion”, and in the Vinaya \textit{tena samayena}, “on that occasion”. The influence of the Abhidhamma on the \textsanskrit{Parivāra} is unmistakable.

\href{https://suttacentral.net/pli-tv-pvr6/en/brahmali}{Pvr~6}, “Offenses in the Chapters”, is a short chapter in verse, stating the number of classes of offenses found in each \textit{khandhaka}. Interestingly, it says that no offenses are laid down in Kd 21. This must mean that the offenses of wrong conduct ascribed to Ānanda at the First Council were not considered offenses proper.\footnote{See \href{https://suttacentral.net/pli-tv-kd21/en/brahmali\#1.10.1}{Kd~21:1.10.1}–1.10.23. } It was presumably understood that the Council elders did not have the authority to lay down new regulations.

\section*{Pvr 7}

\href{https://suttacentral.net/pli-tv-pvr7/en/brahmali}{Pvr~7}, “The numerical method”, is one of the longest and in some ways the most interesting chapter of the \textsanskrit{Parivāra}. Its structure is almost certainly modeled on the \textsanskrit{Aṅguttara} \textsanskrit{Nikāya}, with each section containing a collection of lists or statements (as opposed to \textit{suttas}) arranged according to the number of items they contain, ranging from one to eleven. And so, there are altogether eleven sections.

Let us consider the nature of these numbered lists. To frame the discussion, I will assume, as discussed above, that much of the purpose of the \textsanskrit{Parivāra} is to educate students. We can then group the lists into three distinct categories, which may be described as follows:

\begin{enumerate}%
\item These lists consist of riddle-like items, presumably to be solved by students of the Vinaya.%
\item Here there is no proper list, but instead a statement declaring that there are x items of a certain kind. Possibly a student was then supposed to elaborate on those items.%
\item These lists simply state what the items are.%
\end{enumerate}

To make it a bit clearer how this works in practice, here is an example of each category:

\begin{enumerate}%
\item “There are offenses one commits at the right time, not at the wrong time. There are offenses one commits at the wrong time, not at the right time. There are offenses one commits both at the right time and also at the wrong time.” (\href{https://suttacentral.net/pli-tv-pvr7/en/brahmali\#25.4}{Pvr~7:25.4})%
\item “There are three kinds of illegitimate cancellations of the Monastic Code.” (\href{https://suttacentral.net/pli-tv-pvr7/en/brahmali\#27.1}{Pvr~7:27.1})%
\item “When a monk has three qualities a legal procedure may be done against him: he is shameless, ignorant, and not a regular monk.” (\href{https://suttacentral.net/pli-tv-pvr7/en/brahmali\#29.3}{Pvr~7:29.3})%
\end{enumerate}

What is interesting about these three categories is that members of the same category tend to appear together in groups. Moreover, these groups do not occur randomly within each section (ones to elevens), but instead in a fairly regular pattern. We can make the following observations:

\begin{itemize}%
\item Category (1)—lists with riddle-like items—always occurs at the beginning, or it does not occur at all. If we assume that this chapter developed over a period of time and that new elements were added at the end of sections, we can say that the earliest version of Pvr 7 consisted of category (1) lists. The riddle-like questions would have ensured a significant involvement from the students.%
\item Category (2) was added next. Here the students are still involved, in the sense of having to recall the individual items. But the involvement is somewhat reduced from the first category, where students need to apply their imagination.%
\item Category (3) is consistently last, with the partial exception of section 5. In this category, the students are merely given a list to be memorized, without any further involvement or testing.%
\item Sections 1 and 5–11 were not part of the earliest version of Pvr 7. It would seem that new sections were added until there was a total of eleven, matching the structure of the \textsanskrit{Aṅguttara} \textsanskrit{Nikāya}.%
\end{itemize}

Given these considerations, it seems possible that the impetus for creating the \textsanskrit{Parivāra}, or at least Pvr 7, may have been the new idea of testing students’ Vinaya knowledge with riddle-like questions. As time went on and the initial impetus faded, new material was added that did not have the same innovative value. These were the category (2) statements. Lastly, we have the category (3) lists, where the students’ involvement is reduced to memorizing the items. As we carry on our investigation of the \textsanskrit{Parivāra}, we shall see that these three categories, especially (1) and (3), are reflected also in other chapters.

Now let us consider the content of these numerical lists. Although I have not attempted any systematic study of this, it seems likely that Pvr 7 is an inventory of all things related to the Vinaya, some of it quite tangential, that can be classified according to a numerical system. This is no doubt one reason why some of the sections are long and repetitive, sometimes even pedantic in their insistence on covering everything at the cost of readability.

Most of the content of Pvr 7 is taken from the first five volumes of the Vinaya \textsanskrit{Piṭaka}. Occasionally, however, it does include content that is not found anywhere else in the Pali scriptures. For instance, we find a rather quirky paragraph on the five benefits of sweeping (\href{https://suttacentral.net/pli-tv-pvr7/en/brahmali\#63.58}{Pvr~7:63.58}), and the unexpected list of five gifts without merit (\href{https://suttacentral.net/pli-tv-pvr7/en/brahmali\#63.54}{Pvr~7:63.54}). Then there is the occasional coinage of new terminology, such as \textit{anusandhivacanapatha}, “the sequence of statements”, which is related to the proper performance of \textit{\textsanskrit{saṅghakamma}} (\href{https://suttacentral.net/pli-tv-pvr7/en/brahmali\#64.16}{Pvr~7:64.16}). Then there are a few lists that are otherwise just found in the Suttas, such as the \textit{\textsanskrit{āghātavatthūni}}, “the ten grounds for resentment” (\href{https://suttacentral.net/pli-tv-pvr7/en/brahmali\#126.1}{Pvr~7:126.1}). The vast majority of the material, however, is closely tied to the rest of the Vinaya.

\section*{Pvr 8–15}

\href{https://suttacentral.net/pli-tv-pvr8/en/brahmali}{Pvr~8}, “Aspects of the legal procedures”, describes a number of \textit{\textsanskrit{saṅghakammas}} in terms of their beginning, middle, and end. There is nothing new in this short chapter.

\href{https://suttacentral.net/pli-tv-pvr9/en/brahmali}{Pvr~9}, “The ten reasons for the training rules”, is another short chapter, in this case setting out the ten reasons the Buddha laid down the training rules, the \textit{\textsanskrit{sikkhāpada}}. This chapter shows how these reasons are so closely related that they are essentially one and the same.

\href{https://suttacentral.net/pli-tv-pvr10/en/brahmali}{Pvr~10}, “Verses on the training rules”, is a collection of 79 verses, beginning with a list of the places where the \textsanskrit{Pātimokkha} rules were laid down. It then looks at the four kinds of failures and matches them to the different classes of rules. This is followed by a detailed classification of the \textsanskrit{Pātimokkha} rules, in accordance with their class and whether they are held in common or not between the monks and the nuns. Finally, each class of rule is explained. The chapter ends with a few inspiring verses, including some that are not found anywhere else in the Canonical literature (\href{https://suttacentral.net/pli-tv-pvr10/en/brahmali\#80.1}{Pvr~10:80.1}).

\href{https://suttacentral.net/pli-tv-pvr11/en/brahmali}{Pvr~11}, “The four legal issues and their resolution”, is a technical chapter on these topics. It begins with the legal issues, covering their reopening, their causes, offenses committed because of them, and connections between them. The chapter then turns to consider the seven principles for the resolving of legal issues, starting with some connections between them, before moving on to their causes. This is followed by further analyses of the legal issues. Overall, the content is dry and largely rehashes material found in Kd 14.

\href{https://suttacentral.net/pli-tv-pvr12/en/brahmali}{Pvr~12}, “The verses on how to accuse properly”, is a short chapter on the proper way of accusing another monastic of an offense. This is expanded on in \href{https://suttacentral.net/pli-tv-pvr13/en/brahmali}{Pvr~13}, “The process of investigation”, in which the process to be followed by an investigator is set out, focusing on the examination of the accuser and the accused. We then have a short section on the appropriate behavior of the parties to an accusation, before the chapter ends with a description of the dangers in false accusations.

\href{https://suttacentral.net/pli-tv-pvr14/en/brahmali}{Pvr~14}, “The short chapter on conflict”, continues in much the same vein by setting out the right attitude of an investigator of a conflict. Much of the content, especially the concept of an investigator, adds to what we find in the Khandhakas. By way of illustration, here is an aspect of the duties of an investigator:

\begin{quotation}%
“He should gladden those who are confused, comfort those who are frightened, restrain those who are fierce, and expose those who are impure.” (\href{https://suttacentral.net/pli-tv-pvr14/en/brahmali\#4.1}{Pvr~14:4.1})

%
\end{quotation}

The theme of conflict is also the topic of \href{https://suttacentral.net/pli-tv-pvr15/en/brahmali}{Pvr~15}, “The great chapter on conflict”. As in Pvr 14, the main concern of this chapter is to set out the qualities of one who is involved in settling a conflict. This chapter repeats much of the material of Pvr 13 on the process of investigation.

\section*{Pvr 16}

\href{https://suttacentral.net/pli-tv-pvr16/en/brahmali}{Pvr~16}, “The robe-making ceremony”, starts with a few useful definitions that clarify what is meant at Kd 7, which contains the main exposition on the robe-making ceremony. Next comes a section that uses the Abhidhamma method of conditionality to show the correct sequence in which the different parts of the robe-making ceremony should be performed. This use of conditionality is almost unique to a Canonical text outside the Abhidhamma.\footnote{There are few references to such conditionality in \textsanskrit{Paṭisambhidāmagga}, a late Abhidhamma-style text of the Khuddaka \textsanskrit{Nikāya}, and in the \textsanskrit{Peṭakopadesa}, a para-Canonical text. The conditionality in question are the 24 conditions spoken of in the summary of conditions at the beginning of the \textsanskrit{Paṭṭhāna}, the last book of the Abhidhamma. } Moreover, it is somewhat discordant that such a practical ceremony should be framed using such abstract terminology. It is hard to discern any real reason for bringing in the Abhidhamma, except perhaps as an attempt to magnify the importance of the \textsanskrit{Parivāra} by associating it with a framework that was gradually gaining in influence and prestige.

Pvr 16 continues with a technical discussion on how the various aspects of the ceremony condition each other, followed by further definitions and explanations of how the ceremony is to be performed. The chapter ends with a discussion on the ending of the robe season.

Before moving on to the next chapter, some versions of the text have the phrase \textit{\textsanskrit{parivāraṁ} \textsanskrit{niṭṭhitaṁ}}, “the \textsanskrit{Parivāra} is finished”.\footnote{This is found in the PTS version and in SRT. } It seems likely that the text at some point ended here.

\section*{Pvr 17}

\href{https://suttacentral.net/pli-tv-pvr17/en/brahmali}{Pvr~17}, “Ven. \textsanskrit{Upāli} questions the Buddha”, is a longish chapter that collects all the questions of \textsanskrit{Upāli} in one place. As the Sangha’s foremost Vinaya expert, \textsanskrit{Upāli} is known for his regular questioning of the Buddha, as can be seen throughout the Khandhakas.\footnote{Specifically at Kd 9–10, 12, 17, and 19. } This chapter, however, does not simply assemble his existing questions, but adds a large number of its own, mostly presenting a variety of Vinaya material as if it originated through \textsanskrit{Upāli}’s questioning. This chapter, then, is another extensive collection of assorted Vinaya material.

Pvr 17 has fourteen subchapters with the following content:

\begin{itemize}%
\item 1–3 are mostly concerned with monastics with various bad qualities and the consequences this has from a Vinaya perspective, (\href{https://suttacentral.net/pli-tv-pvr17/en/brahmali\#1.1}{Pvr~17:1.1}–51.1)%
\item 4 sets out various Vinaya-related activities and how these can be done either legitimately or illegitimately, (\href{https://suttacentral.net/pli-tv-pvr17/en/brahmali\#54.1}{Pvr~17:54.1})%
\item 5 is broadly concerned with accusations and how to deal with them in an appropriate manner, (\href{https://suttacentral.net/pli-tv-pvr17/en/brahmali\#87.1}{Pvr~17:87.1})%
\item 6 concerns ascetic practices, (\href{https://suttacentral.net/pli-tv-pvr17/en/brahmali\#126.1}{Pvr~17:126.1})%
\item 7 is a mix of topics, (\href{https://suttacentral.net/pli-tv-pvr17/en/brahmali\#131.1}{Pvr~17:131.1})%
\item 8 concerns the relationship between monks and nuns, (\href{https://suttacentral.net/pli-tv-pvr17/en/brahmali\#147.1}{Pvr~17:147.1})%
\item 9 describes what sort of monk should not be appointed to a committee and what sort of monk is considered ignorant, (\href{https://suttacentral.net/pli-tv-pvr17/en/brahmali\#169.1}{Pvr~17:169.1})\footnote{The committees referred to are those set up to resolve disagreements in the Sangha under the seven principles for settling legal issues, specifically the fifth principle of majority decision, \textit{yebhuyyasika}, see \href{https://suttacentral.net/pli-tv-kd14/en/brahmali\#14.19.1}{Kd~14:14.19.1}. }%
\item 10 describes the sort of monk who is unqualified to resolve legal issues and then discusses schism in the Sangha, (\href{https://suttacentral.net/pli-tv-pvr17/en/brahmali\#186.1}{Pvr~17:186.1})%
\item 11–13 discuss the karmic consequences for a schismatic and of doing one’s duties for the Sangha in a biased fashion, (\href{https://suttacentral.net/pli-tv-pvr17/en/brahmali\#201.1}{Pvr~17:201.1}–236.1)%
\item 14 has a variety of topics, but with an emphasis on whom one should pay respect to, (\href{https://suttacentral.net/pli-tv-pvr17/en/brahmali\#240.1}{Pvr~17:240.1}).%
\end{itemize}

\section*{Pvr 18–21}

\href{https://suttacentral.net/pli-tv-pvr18/en/brahmali}{Pvr~18}, “The origination of offenses”, is another chapter on the origination of offenses and is closely related to Pvr 3. As I’ve noted above, the origination of offenses has an outsized presence in the \textsanskrit{Parivāra} compared to the rest of the Vinaya \textsanskrit{Piṭaka}.

\href{https://suttacentral.net/pli-tv-pvr19/en/brahmali}{Pvr~19}, “Verses on offenses, training rules, and legal procedures”, is a collection of 109 verses focused for the most part on offenses and \textsanskrit{Pātimokkha} rules. The verses have a question-and-answer format, some of them presented as Vinaya conundrums similar to what we have seen in Pvr 7. In the last one-and-a-half subchapters, legal procedures are a major topic.

\href{https://suttacentral.net/pli-tv-pvr20/en/brahmali}{Pvr~20}, the chapter marvelously named “The sudorific verses”—that is, the sweat-inducing verses—consists of a series of 43 cryptic Vinaya questions. Of all the questions in the \textsanskrit{Parivāra}, these are by far the most difficult. Perhaps they functioned as a test for those aiming for the highest distinction in Vinaya scholarship. Here is a sample:

\begin{verse}%
“A monk, by means of begging, builds a hut,\\

Whose site has been approved, which is the right size,\\

where no harm will be done, and which has\\

a space on all sides.\\

How, then, does he commit an offense?” (\href{https://suttacentral.net/pli-tv-pvr20/en/brahmali\#8.1}{Pvr~20:8.1})

%
\end{verse}

Although this question is framed in terms of Bu Ss 6, the answer, says the commentary, is found in the origin story to Bu Pj 2, in which there is an offense of wrong conduct for building a hut made entirely of clay \href{https://suttacentral.net/pli-tv-bu-vb-pj2/en/brahmali\#1.2.11}{Bu~Pj~2}.

\href{https://suttacentral.net/pli-tv-pvr21/en/brahmali}{Pvr~21}, “Legal procedures, why the Monastic Law, resolution of legal issues”, begins with a summary of the five ways in which legal procedures fail. This is followed by three subchapters on the reasons why the Buddha laid down the training rules and other Vinaya procedures. The final subchapter is about the resolution of legal issues.

\thematicbreak
The above is no more than a brief overview of the content of the \textsanskrit{Parivāra}. There are number of issues that deserve further attention, but that will have to wait for another time. Apart from those mentioned in the text above, here is a short list of such issues:

\begin{itemize}%
\item It would be helpful to study the non-Vinaya content of the \textsanskrit{Parivāra}, such as the \textit{cha \textsanskrit{sāraṇīyā} \textsanskrit{dhammā}}, “the six aspects of friendliness”.\footnote{At \href{https://suttacentral.net/pli-tv-pvr7/en/brahmali\#81.19}{Pvr~7:81.19}. } This would aid our understanding of how this collection evolved.%
\item Occasionally the \textsanskrit{Parivāra} contradicts the content of the other parts of the Vinaya, such as stating that a nun with ten years’ seniority may give the full ordination.\footnote{At \href{https://suttacentral.net/pli-tv-pvr7/en/brahmali\#135.26}{Pvr~7:135.26}. } Such contradictions seem too blatant to have happened by accident. So, why are they there?%
\item The \textsanskrit{Parivāra} sometimes includes rules not found anywhere else in the Vinaya \textsanskrit{Piṭaka}. For instance, there is the following rule for nuns: “When, being unsure, she conceals [an offense], she commits a serious offense.”\footnote{At \href{https://suttacentral.net/pli-tv-pvr2.2/en/brahmali\#2.4}{Pvr~2.2:2.4}. } Where do such rules come from?%
\item We find occasional differences in vocabulary and spelling between the \textsanskrit{Parivāra} and the earlier books of the Vinaya \textsanskrit{Piṭaka}. One such instance is \textit{\textsanskrit{samādisati}} in the Sutta-\textsanskrit{vibhaṅga}, whereas the equivalent in the \textsanskrit{Parivāra} is \textit{\textsanskrit{saṁvidahitvā}} instead.\footnote{Respectively at \href{https://suttacentral.net/pli-tv-bu-vb-ss6/en/brahmali\#3.5.1}{Bu~Ss~6:3.5.1} and \href{https://suttacentral.net/pli-tv-pvr4/en/brahmali\#42.3}{Pvr~4:42.3}. } A systematic survey of such differences, with a discussion of their origins, would be helpful.%
\item All seven books of the Abhidhamma are mentioned in the \textsanskrit{Parivāra}.\footnote{At \href{https://suttacentral.net/pli-tv-pvr1.1/en/brahmali\#7.2}{Pvr~1.1:7.2}. } Such details are helpful to help us establish the relative chronology of the Canonical texts.%
\end{itemize}

%
\chapter*{Abbreviations}
\addcontentsline{toc}{chapter}{Abbreviations}
\markboth{Abbreviations}{Abbreviations}

\begin{description}%
\item[AN] \textsanskrit{Aṅguttara} Nikāya (references are to Nipāta and \textit{sutta} numbers)%
\item[AN-a] \textsanskrit{Aṅguttara} Nikāya \textsanskrit{aṭṭhakathā}, the commentary on the \textsanskrit{Aṅguttara} Nikāya%
\item[As] \textit{\textsanskrit{adhikaraṇasamathadhamma}}%
\item[Ay] \textit{aniyata}%
\item[Bi] \textit{\textsanskrit{bhikkhunī}}%
\item[Bu] \textit{bhikkhu}%
\item[CPD] Critical Pali Dictionary%
\item[DN] \textsanskrit{Dīgha} \textsanskrit{Nikāya} (references are to \textit{sutta} numbers)%
\item[DN-a] \textsanskrit{Dīgha} \textsanskrit{Nikāya} \textsanskrit{aṭṭhakathā}, the commentary on the \textsanskrit{Dīgha} \textsanskrit{Nikāya}%
\item[DOP] Dictionary of Pali%
\item[f, ff] and the following page, pages%
\item[Iti] Itivuttaka (references are to verse numbers)%
\item[Ja] \textsanskrit{Jātaka} and \textsanskrit{Jātaka} \textsanskrit{aṭṭhakathā}%
\item[Kd] Khandhaka%
\item[Khuddas-\textsanskrit{pṭ}] \textsanskrit{Khuddasikkhā}-\textsanskrit{purāṇaṭīkā} (references are to paragraph numbers)%
\item[Khuddas-\textsanskrit{nṭ}] \textsanskrit{Khuddasikkhā}-\textsanskrit{abhinavaṭīkā} (references are to paragraph numbers)%
\item[Kkh] \textsanskrit{Kaṅkha}̄\textsanskrit{vitaraṇi}̄%
\item[Kkh-\textsanskrit{pṭ}] \textsanskrit{Kaṅkhāvitaraṇīpurāṇa}-\textsanskrit{ṭīkā}%
\item[MN] Majjhima \textsanskrit{Nikāya} (references are to \textit{sutta} numbers)%
\item[MN-a] Majjhima \textsanskrit{Nikāya} \textsanskrit{aṭṭhakathā}, the commentary on the Majjhima \textsanskrit{Nikāya}%
\item[MS] \textsanskrit{Mahāsaṅgīti} \textsanskrit{Tipiṭaka} (the version of the \textsanskrit{Tipiṭaka} found on SuttaCentral)%
\item[N\&E] “Nature and the Environment in Early Buddhism”, Bhante Dhammika%
\item[Nidd-a] \textsanskrit{Mahāniddesa} \textsanskrit{aṭṭhakathā} (references are to VRI edition paragraph numbers)%
\item[NP] \textit{nissaggiya \textsanskrit{pācittiya}}%
\item[p., pp.] page, pages%
\item[Pc] \textit{\textsanskrit{pācittiya}}%
\item[Pd] \textit{\textsanskrit{pāṭidesanīya}}%
\item[PED] Pali English Dictionary%
\item[Pj] \textit{\textsanskrit{pārājika}}%
\item[PTS] Pali Text Society%
\item[Pvr] \textsanskrit{Parivāra}%
\item[SAF] “South Asian Flora as reflected in the twelfth-century Pali lexicon \textsanskrit{Abhidhānapadīpikā}”, J. Liyanaratne%
\item[SED] Sanskrit English Dictionary%
\item[Sk] \textit{sekhiya}%
\item[SN] \textsanskrit{Saṁyutta} \textsanskrit{Nikāya} (references are to \textsanskrit{Saṁyutta} and \textit{sutta} numbers)%
\item[SN-a] \textsanskrit{Saṁyutta} \textsanskrit{Nikāya} \textsanskrit{aṭṭhakathā}, the commentary on the \textsanskrit{Saṁyutta} \textsanskrit{Nikāya} (references are to volume number and paragraph numbers of the VRI version)%
\item[Sp] Samantapāsādikā, the commentary on the Vinaya \textsanskrit{Piṭaka} (references are to volume and paragraph numbers of the VRI version)%
\item[Sp‑ṭ] Sāratthadīpanī-ṭīkā (references follow the division into five volumes of the Canonical text and then add the paragraph number of the VRI version of the sub-commentary)%
\item[Sp‑yoj] \textsanskrit{Pācityādiyojanā} (volume numbers match those of Sp of the online VRI version, which, given that Sp‑yoj starts with the \textit{bhikkhu \textsanskrit{pācittiyas}}, means that Sp‑yoj is divided into four volumes, starting at volume 2; paragraph numbers are those of the VRI version)%
\item[SRT] Siamrath \textsanskrit{Tipiṭaka}, official edition of the \textsanskrit{Tipiṭaka} published in Thailand%
\item[Ss] \textit{\textsanskrit{saṅghādisesa}}%
\item[sv.] \textit{sub voce}, see under%
\item[\textsanskrit{Thīg}] \textsanskrit{Therīgāthā}%
\item[Ud-a] \textsanskrit{Udāna} \textsanskrit{aṭṭhakathā}, the commentary on the \textsanskrit{Udāna} (references are to \textit{sutta} number)%
\item[Vb] \textsanskrit{Vibhaṅga}, the second book of the Abhidhamma \textsanskrit{Piṭaka}%
\item[Vin-\textsanskrit{ālaṅ}-\textsanskrit{ṭ}] \textsanskrit{Vinayālaṅkāra}-\textsanskrit{ṭīkā} (references are to chapter number and paragraph numbers of the VRI version)%
\item[Vin-vn-\textsanskrit{ṭ}] \textsanskrit{Vinayavinicchayaṭīkā} (references are to paragraph numbers of the VRI version)%
\item[Vjb] \textsanskrit{Vajirabuddhiṭīkā} (references are to volume and paragraph numbers of the VRI version)%
\item[Vmv] \textsanskrit{Vimativinodanī}-\textsanskrit{ṭīkā} (references are to volume and paragraph numbers of the VRI version)%
\item[VRI] Vipassana Research Institute, the publisher of the online version of the Sixth Council edition of the Pali Canon at https://www.tipitaka.org%
\item[Vv-a] \textsanskrit{Vimānavatthu} \textsanskrit{aṭṭhakathā}, the commentary on the \textsanskrit{Vimānavatthu} (references are to paragraph numbers of the VRI edition).%
\end{description}

%
\mainmatter%
\pagestyle{fancy}%
\addtocontents{toc}{\let\protect\contentsline\protect\nopagecontentsline}
\part*{The Compendium }
\addcontentsline{toc}{part}{The Compendium }
\markboth{}{}
\addtocontents{toc}{\let\protect\contentsline\protect\oldcontentsline}

%
%
\chapter*{{\suttatitleacronym Pvr 1.1}{\suttatitletranslation Questions and answers on the monks’ Pātimokkha rules and their analysis }{\suttatitleroot Katthapaññattivāra}}
\addcontentsline{toc}{chapter}{\tocacronym{Pvr 1.1} \toctranslation{Questions and answers on the monks’ Pātimokkha rules and their analysis } \tocroot{Katthapaññattivāra}}
\markboth{Questions and answers on the monks’ Pātimokkha rules and their analysis }{Katthapaññattivāra}
\extramarks{Pvr 1.1}{Pvr 1.1}

\section*{The chapter on offenses entailing expulsion }

\scnamo{Homage to the Buddha, the Perfected One, the fully Awakened One }

“The\marginnote{2.1} first offense entailing expulsion was laid down by the Buddha who knows and sees, the Perfected One, the fully Awakened One. Where was it laid down? Whom is it about? What is it about? Is there a rule, an addition to the rule, an unprompted rule?\footnote{Sp 5.2: \textit{\textsanskrit{Ayañhi} \textsanskrit{anuppannapaññatti} \textsanskrit{nāma} anuppanne dose \textsanskrit{paññattā}; \textsanskrit{sā} \textsanskrit{aṭṭhagarudhammavasena} \textsanskrit{bhikkhunīnaṁyeva} \textsanskrit{āgatā}, \textsanskrit{aññatra} natthi}, “Laid down when no fault has occurred, this is called \textit{\textsanskrit{anuppannapaññatti}}. It has come down for the nuns on account of the eight important principles. There are no other cases.” } Is it a rule that applies everywhere or in a particular place? Is it a rule that the monks and nuns have in common or not in common? Is it a rule for one Sangha or for both? In which of the five ways of reciting the Monastic Code is it contained and included? In which recitation is it included? To which of the four kinds of failure does it belong? To which of the seven classes of offenses does it belong? Through how many of the six kinds of originations of offenses does it originate? To which of the four kinds of legal issues does it belong? Through how many of the seven principles for settling legal issues is it settled? What is the Monastic Law there? What is concerned with the Monastic Law there? What is the Monastic Code there? What is concerned with the Monastic Code there? What is failure? What is success? What is the practice? For how many reasons did the Buddha lay down the first offense entailing expulsion? Who are those who train? Who have finished the training? Established in what? Who master it? Whose pronouncement was it? Who handed it down?” 

“The\marginnote{3.1} first offense entailing expulsion was laid down by the Buddha who knows and sees, the Perfected One, the fully Awakened One. Where was it laid down?” At \textsanskrit{Vesālī}. “Whom is it about?” Sudinna the Kalandian. “What is it about?” Sudinna having sexual intercourse with his ex-wife. “Is there a rule, an addition to the rule, an unprompted rule?” There is one rule. There are two additions to the rule.\footnote{Sp 5.2: \textit{Dve \textsanskrit{anupaññattiyoti} “antamaso \textsanskrit{tiracchānagatāyapī}”ti ca, “\textsanskrit{sikkhaṁ} \textsanskrit{apaccakkhāyā}”ti ca \textsanskrit{makkaṭivajjiputtakavatthūnaṁ} vasena \textsanskrit{vuttā} – \textsanskrit{imā} dve \textsanskrit{anupaññattiyo}}, “Two \textit{\textsanskrit{anupaññattiyo}} is said because of the accounts of the female monkey and the Vajjians. These are the two additions to the rule: ‘even with a female animal’ and ‘without first renouncing the training’.” } There is no unprompted rule. “Is it a rule that applies everywhere or in a particular place?” Everywhere.\footnote{Sp 5.2: \textit{\textsanskrit{Sabbatthapaññattīti} majjhimadese ceva paccantimajanapadesu ca \textsanskrit{sabbatthapaññatti}}, “\textit{\textsanskrit{Sabbatthapaññatti}}: both in the central Ganges plain and in the outlying countries, this is \textit{\textsanskrit{sabbatthapaññatti}}.” } “Is it a rule that the monks and nuns have in common or not in common?” In common. “Is it a rule for one Sangha or for both?” For both.\footnote{Sp 5.2: \textit{\textsanskrit{Byañjanamattameva} hi ettha \textsanskrit{nānaṁ}, \textsanskrit{bhikkhūnaṁ} \textsanskrit{bhikkhunīnampi} \textsanskrit{sādhāraṇattā} \textsanskrit{sādhāraṇapaññatti}, ubhinnampi \textsanskrit{paññattattā} \textsanskrit{ubhatopaññattīti}}, “For here merely the wording is different, since a commonality between the monks and the nuns is a common rule, whereas a ruling for both is a rule for both sides.” The point seems to be that a rule “in common” is synonymous with a rule “for both”. } “In which of the five ways of reciting the Monastic Code is it contained and included?” In the introduction.\footnote{Sp 5.2: \textit{\textsanskrit{Nidānogadhanti} “yassa \textsanskrit{siyā} \textsanskrit{āpatti} so \textsanskrit{āvikareyyā}”ti ettha \textsanskrit{sabbāpattīnaṁ} \textsanskrit{anupaviṭṭhattā} \textsanskrit{nidānogadhaṁ}; \textsanskrit{nidāne} \textsanskrit{anupaviṭṭhanti} attho}, “\textit{\textsanskrit{Nidānogadha}}: it is contained in the introduction because of the entry here of all offenses: ‘Anyone who has committed an offense should reveal it.’ The meaning is they are entered in the introduction.” } “In which recitation is it included?” In the second recitation.\footnote{Sp 5.2: \textit{Dutiyena \textsanskrit{uddesenāti} \textsanskrit{nidānogadhaṁ} \textsanskrit{nidānapariyāpannampi} \textsanskrit{samānaṁ} “tatrime \textsanskrit{cattāro} \textsanskrit{pārājikā} \textsanskrit{dhammā}”\textsanskrit{tiādinā} dutiyeneva uddesena \textsanskrit{uddesaṁ} \textsanskrit{āgacchati}}, “In the second recitation: being contained and included in the introduction, it comes to be recited with ‘Now these four rules on expulsion’ etc.” } “To which of the four kinds of failure does it belong?” Failure in morality. “To which of the seven classes of offenses does it belong?” The class of offenses entailing expulsion. “Through how many of the six kinds of originations of offenses does it originate?” It originates in one way: from body and mind, not from speech. “To which of the four kinds of legal issues does it belong?” Legal issues arising from an offense. “Through how many of the seven principles for settling legal issues is it settled?” Through two of them: by resolution face-to-face and by acting according to what has been admitted. “What is the Monastic Law there? What is concerned with the Monastic Law there?” The rule is the Monastic Law. Its analysis is concerned with the Monastic Law.\footnote{Sp 5.2: \textit{\textsanskrit{Vibhattīti} \textsanskrit{padabhājanaṁ} vuccati; \textsanskrit{vibhattīti} hi \textsanskrit{vibhaṅgassevetaṁ} \textsanskrit{nāmaṁ}}, “The word analysis is called \textit{vibhatti}; for \textit{vibhatti} is a name for the analysis.” The first of these refers to the \textsanskrit{Pātimokkha}, which is mentioned in the next entry, whereas the latter refers to the \textit{\textsanskrit{vibhaṅga}}. This shows us that the prefix \textit{abhi-} means “about” in certain contexts, such as this one, and refers to a commentary style text. It may well be, then, that the Abhidhamma, which is sometimes paired with \textit{abhivinaya} in the Canonical texts, originally was regarded as a commentary on the content of the Sutta \textsanskrit{Piṭaka}. } “What is the Monastic Code there? What is concerned with the Monastic Code there?” The rule is the Monastic Code. Its analysis is concerned with the Monastic Code. “What is failure?” Lack of restraint. “What is success?” Restraint. “What is the practice?” Thinking, “I won’t do such a thing,” one undertakes to train in the training rules for life. “For how many reasons did the Buddha lay down the first offense entailing expulsion?” He laid it down for the following ten reasons: for the well-being of the Sangha, for the comfort of the Sangha, for the restraint of bad people, for the ease of good monks, for the restraint of the corruptions relating to the present life, for the restraint of the corruptions relating to future lives, to give rise to confidence in those without it, to increase the confidence of those who have it, for the longevity of the true Teaching, and for supporting the training. “Who are those who train?” They are the trainees and the good ordinary people. “Who have finished the training?” The perfected ones. “Established in what?” In fondness for the training. “Who master it?” Those who learn it.\footnote{Sp 5.2: \textit{\textsanskrit{Yesaṁ} \textsanskrit{vattatīti} \textsanskrit{yesaṁ} \textsanskrit{vinayapiṭakañca} \textsanskrit{aṭṭhakathā} ca \textsanskrit{sabbā} \textsanskrit{paguṇāti} attho}, “\textit{\textsanskrit{Yesaṁ} vattati} means for those who learn the whole collection of Monastic Law and the commentaries.” } “Whose pronouncement was it?” It was the pronouncement of the Buddha, the Perfected One, the fully Awakened One. “Who handed it down?” The lineage: 

\begin{verse}%
“\textsanskrit{Upāli}\marginnote{4.1} and \textsanskrit{Dāsaka}, \\
\textsanskrit{Soṇaka} and so Siggava; \\
With Moggaliputta as the fifth—\\
These were in India, the land named after the glorious rose apple.\footnote{Vmv 1.1: \textit{Jambusirivhayeti jambusadiso sirimanto avhayo \textsanskrit{nāmaṁ} yassa \textsanskrit{dīpassa}, \textsanskrit{tasmiṁ} \textsanskrit{jambudīpeti} \textsanskrit{vuttaṁ} hoti}, “\textit{Jambusirivhaye}: this is said: in that rose-apple land that is like a glorious rose apple and is so-called, so-named.” } 

Then\marginnote{5.1} Mahinda, \textsanskrit{Iṭṭiya}, \\
Uttiya and so Sambala; \\
And the wise one named Bhadda. 

These\marginnote{6.1} mighty beings of great wisdom, \\
Came here from India; \\
They taught the Collection on Monastic Law, \\
In Sri Lanka. 

And\marginnote{7.1} the five Collections of Discourses, \\
And the seven works of philosophy;\footnote{Sp-\textsanskrit{ṭ} 1.0: \textit{Satta ceva \textsanskrit{pakaraṇeti} \textsanskrit{dhammasaṅgaṇīvibhaṅgādike} satta \textsanskrit{abhidhammappakaraṇe} ca \textsanskrit{vācesunti} attho} “\textit{Satta ceva \textsanskrit{pakaraṇe}}: it means they also taught the seven treatises of philosophy, starting with the \textsanskrit{Dhammasaṅgaṇī} and the \textsanskrit{Vibhaṅga}.” } \\
Then \textsanskrit{Ariṭṭha} the discerning, \\
And the wise Tissadatta. 

The\marginnote{8.1} confident \textsanskrit{Kālasumana}, \\
And the senior monk named \textsanskrit{Dīgha}; \\
And the wise \textsanskrit{Dīghasumana}. 

Another\marginnote{9.1} \textsanskrit{Kālasumana}, \\
And the senior monk \textsanskrit{Nāga}, Buddharakkhita; \\
And the discerning senior monk Tissa, \\
And the wise senior monk Deva. 

Another\marginnote{10.1} discerning Sumana, \\
Confident in the Monastic Law; \\
The learned \textsanskrit{Cūlanāga}, \\
Invincible, like an elephant. 

And\marginnote{11.1} the one named \textsanskrit{Dhammapālita}, \\
\textsanskrit{Rohaṇa}, venerated as a saint; \\
His student Khema of great wisdom, \\
A master of the three Collections. 

Like\marginnote{12.1} the king of the stars on the island, \\
He outshone others in his wisdom; \\
And the discerning Upatissa, \\
Phussadeva the great speaker. 

Another\marginnote{13.1} discerning Sumana, \\
The learned one named Puppha; \\
\textsanskrit{Mahāsīva} the great speaker, \\
Skilled in the entire Collection. 

Another\marginnote{14.1} discerning \textsanskrit{Upāli}, \\
Confident in the Monastic Law; \\
\textsanskrit{Mahānāga} of great wisdom, \\
Skilled in the tradition of the true Teaching. 

Another\marginnote{15.1} discerning Abhaya, \\
Skilled in the entire Collection; \\
And the discerning senior monk Tissa, \\
Confident in the Monastic Law. 

His\marginnote{16.1} student of great wisdom, \\
The learned one named Puppha; \\
Guarding Buddhism, \\
He established himself in India. 

And\marginnote{17.1} the discerning \textsanskrit{Cūlābhaya}, \\
Confident in the Monastic Law; \\
And the discerning senior monk Tissa, \\
Skilled in the tradition of the true Teaching. 

And\marginnote{18.1} the discerning \textsanskrit{Cūladeva}, \\
Confident in the Monastic Law; \\
And the discerning senior monk Siva, \\
Skilled in the entire Monastic Law—

These\marginnote{19.1} mighty beings of great wisdom, \\
Knowers of the Monastic Law and skilled in the path; \\
Proclaimed the Collection of Monastic Law, \\
On the island of Sri Lanka.” 

%
\end{verse}

“The\marginnote{20.1} second offense entailing expulsion was laid down by the Buddha who knows and sees, the Perfected One, the fully Awakened One. Where was it laid down?” At \textsanskrit{Rājagaha}. “Whom is it about?” Dhaniya the potter. “What is it about?” Dhaniya stealing timber from the king. There is one rule. There is one addition to the rule. “Through how many of the six kinds of originations of offenses does it originate?” It originates in three ways: from body and mind, not from speech; or from speech and mind, not from body; or from body, speech, and mind. … 

“There\marginnote{21.1} is the third offense entailing expulsion. Where was it laid down?” At \textsanskrit{Vesālī}. “Whom is it about?” A number of monks. “What is it about?” Those monks killing one another. There is one rule. There is one addition to the rule. “Through how many of the six kinds of originations of offenses does it originate?” It originates in three ways: from body and mind, not from speech; or from speech and mind, not from body; or from body, speech, and mind. … 

“There\marginnote{22.1} is the fourth offense entailing expulsion. Where was it laid down?” At \textsanskrit{Vesālī}. “Whom is it about?” The monks from the banks of the \textsanskrit{Vaggumudā}. “What is it about?” Those monks praising one another’s superhuman qualities to householders. There is one rule. There is one addition to the rule. “Through how many of the six kinds of originations of offenses does it originate?” It originates in three ways: from body and mind, not from speech; or from speech and mind, not from body; or from body, speech, and mind. … 

\scend{The four offenses entailing expulsion are finished. }

\scuddanaintro{This is the summary: }

\begin{scuddana}%
“Sexual\marginnote{25.1} intercourse, and stealing, \\
Person, super—\\
The four offenses entailing expulsion, \\
Definitive grounds for cutting off.” 

%
\end{scuddana}

\section*{2. The chapter on offenses entailing suspension }

“The\marginnote{26.1} offense entailing suspension for emitting semen by means of effort was laid down by the Buddha who knows and sees, the Perfected One, the fully Awakened One. Where was it laid down? Whom is it about? What is it about? Is there a rule, an addition to the rule, an unprompted rule? Is it a rule that applies everywhere or in a particular place? Is it a rule that the monks and nuns have in common or not in common? Is it a rule for one Sangha or for both? In which of the five ways of reciting the Monastic Code is it contained and included? In which recitation is it included? To which of the four kinds of failure does it belong? To which of the seven classes of offenses does it belong? Through how many of the six kinds of originations of offenses does it originate? To which of the four kinds of legal issues does it belong? Through how many of the seven principles for settling legal issues is it settled? What is the Monastic Law there? What is concerned with the Monastic Law there? What is the Monastic Code there? What is concerned with the Monastic Code there? What is failure? What is success? What is the practice? For how many reasons did the Buddha lay down the offense entailing suspension for emitting semen by means of effort? Who are those who train? Who have finished the training? Established in what? Who master it? Whose pronouncement was it? Who handed it down?” 

“The\marginnote{27.1} offense entailing suspension for emitting semen by means of effort was laid down by the Buddha who knows and sees, the Perfected One, the fully Awakened One. Where was it laid down?” At \textsanskrit{Sāvatthī}. “Whom is it about?” Venerable Seyyasaka. “What is it about?” Seyyasaka masturbating. “Is there a rule, an addition to the rule, an unprompted rule?” There is one rule. There is one addition to the rule. There is no unprompted rule. “Is it a rule that applies everywhere or in a particular place?” Everywhere. “Is it a rule that the monks and nuns have in common or not in common?” Not in common. “Is it a rule for one Sangha or for both?” For one. “In which of the five ways of reciting the Monastic Code is it contained and included?” In the introduction. “In which recitation is it included?” In the third recitation. “To which of the four kinds of failure does it belong?” Failure in morality. “To which of the seven classes of offenses does it belong?” The class of offenses entailing suspension. “Through how many of the six kinds of originations of offenses does it originate?” It originates in one way: from body and mind, not from speech. “To which of the four kinds of legal issues does it belong?” Legal issues arising from an offense. “Through how many of the seven principles for settling legal issues is it settled?” Through two of them: by resolution face-to-face and by acting according to what has been admitted. “What is the Monastic Law there? What is concerned with the Monastic Law there?” The rule is the Monastic Law. Its analysis is concerned with the Monastic Law. “What is the Monastic Code there? What is concerned with the Monastic Code there?” The rule is the Monastic Code. Its analysis is concerned with the Monastic Code. “What is failure?” Lack of restraint. “What is success?” Restraint. “What is the practice?” Thinking, “I won’t do such a thing,” one undertakes to train in the training rules for life. “For how many reasons did the Buddha lay down the offense entailing suspension for emitting semen by means of effort?” He laid it down for the following ten reasons: for the well-being of the Sangha, for the comfort of the Sangha, for the restraint of bad people, for the ease of good monks, for the restraint of the corruptions relating to the present life, for the restraint of the corruptions relating to future lives, to give rise to confidence in those without it, to increase the confidence of those who have it, for the longevity of the true Teaching, and for supporting the training. “Who are those who train?” They are the trainees and the good ordinary people. “Who have finished the training?” The perfected ones. “Established in what?” In fondness for the training. “Who master it?” Those who learn it. “Whose pronouncement was it?” It was the pronouncement of the Buddha, the Perfected One, the fully Awakened One. “Who handed it down?” The lineage: 

\begin{verse}%
“\textsanskrit{Upāli}\marginnote{28.1} and \textsanskrit{Dāsaka}, \\
\textsanskrit{Soṇaka} and so Siggava; \\
With Moggaliputta as the fifth—\\
These were in India, the land named after the glorious rose apple. 

Then\marginnote{29.1} Mahinda, \textsanskrit{Iṭṭiya}, \\
Uttiya and so Sambala; \\
And the wise one named Bhadda. 

These\marginnote{30.1} mighty beings of great wisdom, \\
Came here from India; \\
They taught the Collection on Monastic Law, \\
In Sri Lanka. 

And\marginnote{31.1} the five Collections of Discourses, \\
And the seven works of philosophy; \\
Then \textsanskrit{Ariṭṭha} the discerning, \\
And the wise Tissadatta. 

The\marginnote{32.1} confident \textsanskrit{Kālasumana}, \\
And the senior monk named \textsanskrit{Dīgha}; \\
And the wise \textsanskrit{Dīghasumana}. 

Another\marginnote{33.1} \textsanskrit{Kālasumana}, \\
And the senior monk \textsanskrit{Nāga}, Buddharakkhita; \\
And the discerning senior monk Tissa, \\
And the wise senior monk Deva. 

Another\marginnote{34.1} discerning Sumana, \\
Confident in the Monastic Law; \\
The learned \textsanskrit{Cūlanāga}, \\
Invincible, like an elephant. 

And\marginnote{35.1} the one named \textsanskrit{Dhammapālita}, \\
\textsanskrit{Rohaṇa}, venerated as a saint; \\
His student Khema of great wisdom, \\
A master of the three Collections. 

Like\marginnote{36.1} the king of the stars on the island, \\
He outshone others in his wisdom; \\
And the discerning Upatissa, \\
Phussadeva the great speaker. 

Another\marginnote{37.1} discerning Sumana, \\
The learned one named Puppha; \\
\textsanskrit{Mahāsīva} the great speaker, \\
Skilled in the entire Collection. 

Another\marginnote{38.1} discerning \textsanskrit{Upāli}, \\
Confident in the Monastic Law; \\
\textsanskrit{Mahānāga} of great wisdom, \\
Skilled in the tradition of the true Teaching. 

Another\marginnote{39.1} discerning Abhaya, \\
Skilled in the entire Collection; \\
And the discerning senior monk Tissa, \\
Confident in the Monastic Law, 

His\marginnote{40.1} student of great wisdom, \\
The learned one named Puppha; \\
Guarding Buddhism, \\
He established himself in India. 

And\marginnote{41.1} the discerning \textsanskrit{Cūlābhaya}, \\
Confident in the Monastic Law; \\
And the discerning senior monk Tissa, \\
Skilled in the tradition of the true Teaching. 

And\marginnote{42.1} the discerning \textsanskrit{Cūladeva}, \\
Confident in the Monastic Law; \\
And the discerning senior monk Siva, \\
Skilled in the entire Monastic Law—

These\marginnote{43.1} mighty beings of great wisdom, \\
Knowers of the Monastic Law and skilled in the path; \\
Proclaimed the Collection of Monastic Law, \\
On the island of Sri Lanka.” 

%
\end{verse}

“The\marginnote{44.1} offense entailing suspension for making physical contact with a woman was laid down by the Buddha who knows and sees, the Perfected One, the fully Awakened One. Where was it laid down?” At \textsanskrit{Sāvatthī}. “Whom is it about?” Venerable \textsanskrit{Udāyī}. “What is it about?” \textsanskrit{Udāyī} making physical contact with a woman. There is one rule. Of the six kinds of originations of offenses, it originates in one way: from body and mind, not from speech. … 

“There\marginnote{45.1} is an offense entailing suspension for speaking indecently to a woman. Where was it laid down?” At \textsanskrit{Sāvatthī}. “Whom is it about?” Venerable \textsanskrit{Udāyī}. “What is it about?” \textsanskrit{Udāyī} speaking indecently to a woman. There is one rule. Of the six kinds of originations of offenses, it originates in three ways: from body and mind, not from speech; or from speech and mind, not from body; or from body, speech, and mind. … 

“There\marginnote{46.1} is an offense entailing suspension for encouraging a woman to satisfy one’s own desires. Where was it laid down?” At \textsanskrit{Sāvatthī}. “Whom is it about?” Venerable \textsanskrit{Udāyī}. “What is it about?” \textsanskrit{Udāyī} encouraging a woman to satisfy his own desires. There is one rule. Of the six kinds of originations of offenses, it originates in three ways: from body and mind, not from speech; or from speech and mind, not from body; or from body, speech, and mind. … 

“There\marginnote{47.1} is an offense entailing suspension for acting as a matchmaker. Where was it laid down?” At \textsanskrit{Sāvatthī}. “Whom is it about?” Venerable \textsanskrit{Udāyī}. “What is it about?” \textsanskrit{Udāyī} acting as a matchmaker. There is one rule. There is one addition to the rule. Of the six kinds of originations of offenses, it originates in six ways: from body, not from speech or mind; or from speech, not from body or mind; or from body and speech, not from mind; or from body and mind, not from speech; or from speech and mind, not from body; or from body, speech, and mind. … 

“There\marginnote{48.1} is an offense entailing suspension for having a hut built by means of begging. Where was it laid down?” At \textsanskrit{Āḷavī}. “Whom is it about?” The monks of \textsanskrit{Āḷavī}. “What is it about?” Those monks having huts made by means of begging. There is one rule. Of the six kinds of originations of offenses, it originates in six ways: … 

“There\marginnote{49.1} is an offense entailing suspension for having a large dwelling built. Where was it laid down?” At \textsanskrit{Kosambī}. “Whom is it about?” Venerable Channa. “What is it about?” Channa having a tree that served as a shrine felled to clear a site for a dwelling. There is one rule. Of the six kinds of originations of offenses, it originates in six ways: … 

“There\marginnote{50.1} is an offense entailing suspension for groundlessly charging a monk with an offense entailing expulsion. Where was it laid down?” At \textsanskrit{Rājagaha}. “Whom is it about?” The monks Mettiya and \textsanskrit{Bhūmajaka}. “What is it about?” Those monks groundlessly charging Venerable Dabba the Mallian with an offense entailing expulsion. There is one rule. Of the six kinds of originations of offenses, it originates in three ways: … 

“There\marginnote{51.1} is an offense entailing suspension for charging a monk with an offense entailing expulsion, using an unrelated legal issue as a pretext. Where was it laid down?” At \textsanskrit{Rājagaha}. “Whom is it about?” The monks Mettiya and \textsanskrit{Bhūmajaka}. “What is it about?” Those monks charging Venerable Dabba the Mallian with an offense entailing expulsion, using an unrelated legal issue as a pretext. There is one rule. Of the six kinds of originations of offenses, it originates in three ways: … 

“There\marginnote{52.1} is an offense entailing suspension for a monk who does not stop pursuing schism in the Sangha when pressed for the third time. Where was it laid down?” At \textsanskrit{Rājagaha}. “Whom is it about?” Devadatta. “What is it about?” Devadatta pursuing schism in a united Sangha. There is one rule. Of the six kinds of originations of offenses, it originates in one way: from body, speech, and mind. … 

“There\marginnote{53.1} is an offense entailing suspension for monks who do not stop siding with one who is pursuing schism in the Sangha when pressed for the third time. Where was it laid down?” At \textsanskrit{Rājagaha}. “Whom is it about?” Several monks. “What is it about?” Those monks siding with and supporting Devadatta’s pursuit of schism in the Sangha. There is one rule. Of the six kinds of originations of offenses, it originates in one way: from body, speech, and mind. … 

“There\marginnote{54.1} is an offense entailing suspension for a monk who does not stop being difficult to correct when pressed for the third time. Where was it laid down?” At \textsanskrit{Kosambī}. “Whom is it about?” Venerable Channa. “What is it about?” Channa making himself incorrigible when legitimately spoken to by the monks. There is one rule. Of the six kinds of originations of offenses, it originates in one way: from body, speech, and mind. … 

“There\marginnote{55.1} is an offense entailing suspension for a monk who does not stop being a corrupter of families when pressed for the third time. Where was it laid down?” At \textsanskrit{Sāvatthī}. “Whom is it about?” The monks Assaji and Punabbasuka. “What is it about?” Those monks, when the Sangha did a legal procedure of banishment against them, slandering the monks as acting out of favoritism, ill will, confusion, and fear. There is one rule. Of the six kinds of originations of offenses, it originates in one way: from body, speech, and mind. … 

\scend{The thirteen rules entailing suspension are finished. }

\scuddanaintro{This is the summary: }

\begin{scuddana}%
“Emission,\marginnote{58.1} physical contact, \\
Indecent, and his own needs; \\
Matchmaking, and a hut, \\
And a dwelling, groundless. 

A\marginnote{59.1} pretext, and schism, \\
Those who side with him; \\
Difficult to correct, and corrupter of families—\\
The thirteen offenses entailing suspension.” 

%
\end{scuddana}

\section*{3. The chapter on undetermined offenses }

“The\marginnote{60.1} first undetermined offense was laid down by the Buddha who knows and sees, the Perfected One, the fully Awakened One. Where was it laid down? Whom is it about? What is it about? Is there a rule, an addition to the rule, an unprompted rule? Is it a rule that applies everywhere or in a particular place? Is it a rule that the monks and nuns have in common or not in common? Is it a rule for one Sangha or for both? In which of the five ways of reciting the Monastic Code is it contained and included? In which recitation is it included? To which of the four kinds of failure does it belong? To which of the seven classes of offenses does it belong? Through how many of the six kinds of originations of offenses does it originate? To which of the four kinds of legal issues does it belong? Through how many of the seven principles for settling legal issues is it settled? What is the Monastic Law there? What is concerned with the Monastic Law there? What is the Monastic Code there? What is concerned with the Monastic Code there? What is failure? What is success? What is the practice? For how many reasons did the Buddha lay down the first undetermined offense? Who are those who train? Who have finished the training? Established in what? Who master it? Whose pronouncement was it? Who handed it down?” 

“The\marginnote{61.1} first undetermined offense was laid down by the Buddha who knows and sees, the Perfected One, the fully Awakened One. Where was it laid down?” At \textsanskrit{Sāvatthī}. “Whom is it about?” Venerable \textsanskrit{Udāyī}. “What is it about?” \textsanskrit{Udāyī} sitting down in private alone with a woman on a concealed seat suitable for the deed. “Is there a rule, an addition to the rule, an unprompted rule?” There is one rule. There is no addition to the rule. There is no unprompted rule. “Is it a rule that applies everywhere or in a particular place?” Everywhere. “Is it a rule that the monks and nuns have in common or not in common?” Not in common. “Is it a rule for one Sangha or for both?” For one. “In which of the five ways of reciting the Monastic Code is it contained and included?” In the introduction. “In which recitation is it included?” In the fourth recitation. “To which of the four kinds of failure does it belong?” It may be failure in morality or failure in conduct. “To which of the seven classes of offenses does it belong?” It may be in the class of offenses entailing expulsion, in the class of offenses entailing suspension, or in the class of offenses entailing confession. “Through how many of the six kinds of originations of offenses does it originate?” It originates in one way: from body and mind, not from speech. “To which of the four kinds of legal issues does it belong?” Legal issues arising from an offense. “Through how many of the seven principles for settling legal issues is it settled?” Through three of them: it may be settled by resolution face-to-face and by acting according to what has been admitted; or it may be settled by resolution face-to-face and by covering over as if with grass.\footnote{That is, the three principles are resolution face-to-face, acting according to what has been admitted, and covering over as if with grass. } “What is the Monastic Law there? What is concerned with the Monastic Law there?” The rule is the Monastic Law. Its analysis is concerned with the Monastic Law. “What is the Monastic Code there? What is concerned with the Monastic Code there?” The rule is the Monastic Code. Its analysis is concerned with the Monastic Code. “What is failure?” Lack of restraint. “What is success?” Restraint. “What is the practice?” Thinking, “I won’t do such a thing,” one undertakes to train in the training rules for life. “For how many reasons did the Buddha lay down the first undetermined offense?” He laid it down for the following ten reasons: for the well-being of the Sangha, for the comfort of the Sangha, for the restraint of bad people, for the ease of good monks, for the restraint of the corruptions relating to the present life, for the restraint of the corruptions relating to future lives, to give rise to confidence in those without it, to increase the confidence of those who have it, for the longevity of the true Teaching, and for supporting the training. “Who are those who train?” They are the trainees and the good ordinary people. “Who have finished the training?” The perfected ones. “Established in what?” In fondness for the training. “Who master it?” Those who learn it. “Whose pronouncement was it?” It was the pronouncement of the Buddha, the Perfected One, the fully Awakened One. “Who handed it down?” The lineage: 

\begin{verse}%
“\textsanskrit{Upāli}\marginnote{62.1} and \textsanskrit{Dāsaka}, \\
\textsanskrit{Soṇaka} and so Siggava; \\
With Moggaliputta as the fifth—\\
These were in India, the land named after the glorious rose apple. 

Then\marginnote{63.1} Mahinda, \textsanskrit{Iṭṭiya}, \\
Uttiya and so Sambala; \\
And the wise one named Bhadda. 

These\marginnote{64.1} mighty beings of great wisdom, \\
Came here from India; \\
They taught the Collection on Monastic Law, \\
In Sri Lanka. 

And\marginnote{65.1} the five Collections of Discourses, \\
And the seven works of philosophy; \\
Then \textsanskrit{Ariṭṭha} the discerning, \\
And the wise Tissadatta. 

The\marginnote{66.1} confident \textsanskrit{Kālasumana}, \\
And the senior monk named \textsanskrit{Dīgha}; \\
And the wise \textsanskrit{Dīghasumana}. 

Another\marginnote{67.1} \textsanskrit{Kālasumana}, \\
And the senior monk \textsanskrit{Nāga}, Buddharakkhita; \\
And the discerning senior monk Tissa, \\
And the wise senior monk Deva. 

Another\marginnote{68.1} discerning Sumana, \\
Confident in the Monastic Law; \\
The learned \textsanskrit{Cūlanāga}, \\
Invincible, like an elephant. 

And\marginnote{69.1} the one named \textsanskrit{Dhammapālita}, \\
\textsanskrit{Rohaṇa}, venerated as a saint; \\
His student Khema of great wisdom, \\
A master of the three Collections. 

Like\marginnote{70.1} the king of the stars on the island, \\
He outshone others in his wisdom; \\
And the discerning Upatissa, \\
Phussadeva the great speaker. 

Another\marginnote{71.1} discerning Sumana, \\
The learned one named Puppha; \\
\textsanskrit{Mahāsīva} the great speaker, \\
Skilled in the entire Collection. 

Another\marginnote{72.1} discerning \textsanskrit{Upāli}, \\
Confident in the Monastic Law; \\
\textsanskrit{Mahānāga} of great wisdom, \\
Skilled in the tradition of the true Teaching. 

Another\marginnote{73.1} discerning Abhaya, \\
Skilled in the entire Collection; \\
And the discerning senior monk Tissa, \\
Confident in the Monastic Law. 

His\marginnote{74.1} student of great wisdom, \\
The learned one named Puppha; \\
Guarding Buddhism, \\
He established himself in India. 

And\marginnote{75.1} the discerning \textsanskrit{Cūlābhaya}, \\
Confident in the Monastic Law; \\
And the discerning senior monk Tissa, \\
Skilled in the tradition of the true Teaching. 

And\marginnote{76.1} the discerning \textsanskrit{Cūladeva}, \\
Confident in the Monastic Law; \\
And the discerning senior monk Siva, \\
Skilled in the entire Monastic Law—

These\marginnote{77.1} mighty beings of great wisdom, \\
Knowers of the Monastic Law and skilled in the path; \\
Proclaimed the Collection of Monastic Law, \\
On the island of Sri Lanka.” 

%
\end{verse}

“The\marginnote{78.1} second undetermined offense was laid down by the Buddha who knows and sees, the Perfected One, the fully Awakened One. Where was it laid down?” At \textsanskrit{Sāvatthī}. “Whom is it about?” Venerable \textsanskrit{Udāyī}. “What is it about?” \textsanskrit{Udāyī} sitting down in private alone with a woman. “Is there a rule, an addition to the rule, an unprompted rule?” There is one rule. There is no addition to the rule. There is no unprompted rule. “Is it a rule that applies everywhere or in a particular place?” Everywhere. “Is it a rule that the monks and nuns have in common or not in common?” Not in common. “Is it a rule for one Sangha or for both?” For one. “In which of the five ways of reciting the Monastic Code is it contained and included?” In the introduction. “In which recitation is it included?” In the fourth recitation. “To which of the four kinds of failure does it belong?” It may be failure in morality or failure in conduct. “To which of the seven classes of offenses does it belong?” It may be in the class of offenses entailing suspension, or in the class of offenses entailing confession. “Through how many of the six kinds of originations of offenses does it originate?” It originates in three ways: from body and mind, not from speech; or from speech and mind, not from body; or from body, speech, and mind. “To which of the four kinds of legal issues does it belong?” Legal issues arising from an offense. “Through how many of the seven principles for settling legal issues is it settled?” Through three of them: it may be settled by resolution face-to-face and by acting according to what has been admitted; or it may be settled by resolution face-to-face and by covering over as if with grass. … 

\scend{The two undetermined offense are finished. }

\scuddanaintro{This is the summary: }

\begin{scuddana}%
“Suitable\marginnote{81.1} for the deed, \\
And then not so—\\
The undetermined offenses have been well laid down, \\
By the Stable One, the Buddha who is the best.” 

%
\end{scuddana}

\section*{4. The chapter on relinquishment }

\subsection*{The subchapter on the robe season }

“The\marginnote{82.1} offense entailing relinquishment and confession for keeping an extra robe more than ten days that was laid down by the Buddha who knows and sees, the Perfected One, the fully Awakened One. Where was it laid down?” At \textsanskrit{Vesālī}. “Whom is it about?” The monks from the group of six. “What is it about?” Those monks keeping an extra robe. There is one rule. There is one addition to the rule. Of the six kinds of originations of offenses, it originates in two ways: from body and speech, not from mind; or from body, speech, and mind. … 

“There\marginnote{83.1} is an offense entailing relinquishment and confession for staying apart from one’s three robes for one day. Where was it laid down?” At \textsanskrit{Sāvatthī}. “Whom is it about?” A number of monks. “What is it about?” Those monks storing one of their robes with other monks and then leaving to wander the country in a sarong and an upper robe. There is one rule. There is one addition to the rule. Of the six kinds of originations of offenses, it originates in two ways: from body and speech, not from mind; or from body, speech, and mind. … 

“There\marginnote{84.1} is an offense entailing relinquishment and confession for receiving out-of-season robe-cloth and then keeping it for more than a month. Where was it laid down?” At \textsanskrit{Sāvatthī}. “Whom is it about?” A number of monks. “What is it about?” Those monks receiving out-of-season robe-cloth and then keeping it for more than a month. There is one rule.\footnote{MS says there is one addition to the rule, \textit{\textsanskrit{ekā} \textsanskrit{anupaññatti}}, which is incorrect. I follow SRT, which has the correct reading. } Of the six kinds of originations of offenses, it originates in two ways: from body and speech, not from mind; or from body, speech, and mind. … 

“There\marginnote{85.1} is an offense entailing relinquishment and confession for having an unrelated nun wash a used robe. Where was it laid down?” At \textsanskrit{Sāvatthī}. “Whom is it about?” Venerable \textsanskrit{Udāyī}. “What is it about?” \textsanskrit{Udāyī} having an unrelated nun wash a used robe. There is one rule. Of the six kinds of originations of offenses, it originates in six ways: … 

“There\marginnote{86.1} is an offense entailing relinquishment and confession for receiving a robe directly from an unrelated nun. Where was it laid down?” At \textsanskrit{Rājagaha}. “Whom is it about?” Venerable \textsanskrit{Udāyī}. “What is it about?” \textsanskrit{Udāyī} receiving a robe directly from an unrelated nun. There is one rule. There is one addition to the rule. Of the six kinds of originations of offenses, it originates in six ways: … 

“There\marginnote{87.1} is an offense entailing relinquishment and confession for asking an unrelated male or female householder for a robe. Where was it laid down?” At \textsanskrit{Sāvatthī}. “Whom is it about?” Venerable Upananda the Sakyan. “What is it about?” Upananda asking the son of an unrelated merchant for a robe. There is one rule. There is one addition to the rule. Of the six kinds of originations of offenses, it originates in six ways: … 

“There\marginnote{88.1} is an offense entailing relinquishment and confession for asking an unrelated male or female householder for too many robes. Where was it laid down?” At \textsanskrit{Sāvatthī}. “Whom is it about?” The monks from the group of six. “What is it about?” Those monks not knowing moderation and asking for many robes. There is one rule. Of the six kinds of originations of offenses, it originates in six ways: … 

“There\marginnote{89.1} is an offense entailing relinquishment and confession for, without first being invited, going to an unrelated householder and specifying the kind of robe-cloth one wants. Where was it laid down?” At \textsanskrit{Sāvatthī}. “Whom is it about?” Venerable Upananda the Sakyan. “What is it about?” Upananda, without first being invited, going to an unrelated householder and specifying the kind of robe-cloth he wanted. There is one rule. Of the six kinds of originations of offenses, it originates in six ways: … 

“There\marginnote{90.1} is an offense entailing relinquishment and confession for, without first being invited, going to unrelated householders and specifying the kind of robe-cloth one wants. Where was it laid down?” At \textsanskrit{Sāvatthī}. “Whom is it about?” Venerable Upananda the Sakyan. “What is it about?” Upananda, without first being invited, going to unrelated householders and specifying the kind of robe-cloth he wanted. There is one rule. Of the six kinds of originations of offenses, it originates in six ways: … 

“There\marginnote{91.1} is an offense entailing relinquishment and confession for getting robe-cloth after prompting more than three times and standing more than six times. Where was it laid down?” At \textsanskrit{Sāvatthī}. “Whom is it about?” Venerable Upananda the Sakyan. “What is it about?” Upananda not agreeing when asked by a lay follower to wait for one day. There is one rule. Of the six kinds of originations of offenses, it originates in six ways: … 

\scendvagga{The first subchapter on the robe season is finished. }

\subsection*{The subchapter on silk }

“There\marginnote{93.1} is an offense entailing relinquishment and confession for having a blanket made that contains silk. Where was it laid down?” At \textsanskrit{Āḷavī}. “Whom is it about?” The monks from the group of six. “What is it about?” Those monks going to the silk-makers and saying, “Please boil a heap of silkworms and give us silk. We want to make a blanket containing silk.” There is one rule. Of the six kinds of originations of offenses, it originates in six ways: … 

“There\marginnote{94.1} is an offense entailing relinquishment and confession for having a blanket made entirely of black wool. Where was it laid down?” At \textsanskrit{Vesālī}. “Whom is it about?” The monks from the group of six. “What is it about?” Those monks having a blanket made entirely of black wool. There is one rule. Of the six kinds of originations of offenses, it originates in six ways: … 

“There\marginnote{95.1} is an offense entailing relinquishment and confession for having a new blanket made without using one measure of white wool and one measure of brown. Where was it laid down?” At \textsanskrit{Sāvatthī}. “Whom is it about?” The monks from the group of six. “What is it about?” Those monks adding just a little bit of white on the edge, effectively having a blanket made entirely of black wool. There is one rule. Of the six kinds of originations of offenses, it originates in six ways: … 

“There\marginnote{96.1} is an offense entailing relinquishment and confession for having a blanket made every year. Where was it laid down?” At \textsanskrit{Sāvatthī}. “Whom is it about?” A number of monks. “What is it about?” Those monks having a blanket made every year. There is one rule. There is one addition to the rule. Of the six kinds of originations of offenses, it originates in six ways: … 

“There\marginnote{97.1} is an offense entailing relinquishment and confession for having a new sitting blanket made without incorporating a piece of one standard handspan from the border of an old blanket. Where was it laid down?” At \textsanskrit{Sāvatthī}. “Whom is it about?” A number of monks. “What is it about?” Those monks discarding their blankets and undertaking the practice of staying in the wilderness, of eating only almsfood, and of wearing rag-robes. There is one rule. Of the six kinds of originations of offenses, it originates in six ways: … 

“There\marginnote{98.1} is an offense entailing relinquishment and confession for receiving wool and then taking it more than 40 kilometers. Where was it laid down?” At \textsanskrit{Sāvatthī}. “Whom is it about?” A certain monk. “What is it about?” That monk receiving wool and then taking it more than 40 kilometers. There is one rule. Of the six kinds of originations of offenses, it originates in two ways: from body, not from speech or mind; or from body and mind, not from speech. … 

“There\marginnote{99.1} is an offense entailing relinquishment and confession for having an unrelated nun wash wool. Where was it laid down?” In the Sakyan country. “Whom is it about?” The monks from the group of six. “What is it about?” Those monks having wool washed by unrelated nuns. There is one rule. Of the six kinds of originations of offenses, it originates in six ways: … 

“There\marginnote{100.1} is an offense entailing relinquishment and confession for receiving money. Where was it laid down?” At \textsanskrit{Rājagaha}. “Whom is it about?” Venerable Upananda the Sakyan. “What is it about?” Upananda receiving money. There is one rule. Of the six kinds of originations of offenses, it originates in six ways: … 

“There\marginnote{101.1} is an offense entailing relinquishment and confession for engaging in various kinds of trades involving money. Where was it laid down?” At \textsanskrit{Sāvatthī}. “Whom is it about?” The monks from the group of six. “What is it about?” Those monks engaging in various kinds of trades involving money. There is one rule. Of the six kinds of originations of offenses, it originates in six ways: … 

“There\marginnote{102.1} is an offense entailing relinquishment and confession for engaging in in various kinds of barter. Where was it laid down?” At \textsanskrit{Sāvatthī}. “Whom is it about?” Venerable Upananda the Sakyan. “What is it about?” Upananda bartering with a wanderer. There is one rule. Of the six kinds of originations of offenses, it originates in six ways: … 

\scendvagga{The second subchapter on silk is finished. }

\subsection*{The subchapter on almsbowls }

“There\marginnote{104.1} is an offense entailing relinquishment and confession for keeping an extra almsbowl for more than ten days. Where was it laid down?” At \textsanskrit{Sāvatthī}. “Whom is it about?” The monks from the group of six. “What is it about?” Those monks keeping an extra bowl. There is one rule. There is one addition to the rule. Of the six kinds of originations of offenses, it originates in two ways: from body and speech, not from mind; or from body, speech, and mind. … 

“There\marginnote{105.1} is an offense entailing relinquishment and confession for exchanging an almsbowl with fewer than five mends for a new almsbowl. Where was it laid down?” In the Sakyan country. “Whom is it about?” The monks from the group of six. “What is it about?” Those monks asking for many bowls even when their existing almsbowls only had a minor chip or scratch. There is one rule. Of the six kinds of originations of offenses, it originates in six ways: … 

“There\marginnote{106.1} is an offense entailing relinquishment and confession for receiving tonics and then keeping them for more than seven days. Where was it laid down?” At \textsanskrit{Sāvatthī}. “Whom is it about?” A number of monks. “What is it about?” Those monks receiving tonics and then keeping them for more than seven days. There is one rule. Of the six kinds of originations of offenses, it originates in two ways: … (as in the rule on the robe season) … 

“There\marginnote{107.1} is an offense entailing relinquishment and confession for looking for a rainy-season robe when there is more than a month left of summer. Where was it laid down?” At \textsanskrit{Sāvatthī}. “Whom is it about?” The monks from the group of six. “What is it about?” Those monks looking for a rainy-season robe when there was more than a month left of summer. There is one rule. Of the six kinds of originations of offenses, it originates in six ways: … 

“There\marginnote{108.1} is an offense entailing relinquishment and confession for giving a robe to a monk and then taking it back in anger. Where was it laid down?” At \textsanskrit{Sāvatthī}. “Whom is it about?” Venerable Upananda the Sakyan. “What is it about?” Upananda giving a robe to a monk and then taking it back in anger. There is one rule. Of the six kinds of originations of offenses, it originates in three ways: … 

“There\marginnote{109.1} is an offense entailing relinquishment and confession for asking for thread and then having weavers weave robe-cloth. Where was it laid down?” At \textsanskrit{Rājagaha}. “Whom is it about?” The monks from the group of six. “What is it about?” Those monks asking for thread and then having weavers weave robe-cloth. There is one rule. Of the six kinds of originations of offenses, it originates in six ways: … 

“There\marginnote{110.1} is an offense entailing relinquishment and confession for going, without first being invited, to an unrelated householder’s weavers and then specifying the kind of robe-cloth one wants. Where was it laid down?” At \textsanskrit{Sāvatthī}. “Whom is it about?” Venerable Upananda the Sakyan. “What is it about?” Upananda going, without first being invited, to an unrelated householder’s weavers and then specifying the kind of robe-cloth he wanted. There is one rule. Of the six kinds of originations of offenses, it originates in six ways: … 

“There\marginnote{111.1} is an offense entailing relinquishment and confession for receiving a haste-cloth and then keeping it beyond the robe season. Where was it laid down?” At \textsanskrit{Sāvatthī}. “Whom is it about?” A number of monks. “What is it about?” Those monks receiving a haste-cloth and then keeping it beyond the robe season. There is one rule. Of the six kinds of originations of offenses, it originates in two ways: … (as in the rule on the robe season) … 

“There\marginnote{112.1} is an offense entailing relinquishment and confession for storing one of one’s three robes in an inhabited area and then staying apart from it for more than six days. Where was it laid down?” At \textsanskrit{Sāvatthī}. “Whom is it about?” A number of monks. “What is it about?” Those monks storing one of their three robes in an inhabited area and then staying apart from it for more than six days. There is one rule. Of the six kinds of originations of offenses, it originates in two ways: … (as in the rule on the robe season) … 

“There\marginnote{113.1} is an offense entailing relinquishment and confession for diverting to oneself material support that one knows was intended for the Sangha. Where was it laid down?” At \textsanskrit{Sāvatthī}. “Whom is it about?” The monks from the group of six. “What is it about?” Those monks diverting to themselves material support that they knew was intended for the Sangha. There is one rule. Of the six kinds of originations of offenses, it originates in three ways: … 

\scendvagga{The third subchapter on almsbowls is finished. }

\scend{The thirty rules on relinquishment and confession are finished. }

\scuddanaintro{This is the summary: }

\begin{scuddana}%
“Ten,\marginnote{116.1} one day, and a month; \\
And washing, receiving; \\
Unrelated, and that one, for the sake of; \\
Of both, and with messenger. 

Silk,\marginnote{117.1} entirely, two parts, \\
Six years, sitting blanket; \\
And two on wool, should take, \\
Two on various kinds. 

Two\marginnote{118.1} on bowls, and tonics, \\
Rainy season, the fifth on a gift; \\
Oneself, having woven, haste, \\
Risky, and with the Sangha.” 

%
\end{scuddana}

\section*{5. The chapter on offenses entailing confession }

\subsection*{The subchapter on lying }

“The\marginnote{119.1} offense entailing confession for lying in full awareness was laid down by the Buddha who knows and sees, the Perfected One, the fully Awakened One. Where was it laid down?” At \textsanskrit{Sāvatthī}. “Whom is it about?” Hatthaka the Sakyan. “What is it about?” Hatthaka, when talking with the monastics of other religions, asserting things after denying them and denying things after asserting them. There is one rule. Of the six kinds of originations of offenses, it originates in three ways: from body and mind, not from speech; or from speech and mind, not from body; or from body, speech, and mind. … 

“There\marginnote{120.1} is an offense entailing confession for speaking abusively. Where was it laid down?” At \textsanskrit{Sāvatthī}. “Whom is it about?” The monks from the group of six. “What is it about?” Those monks arguing with and abusing good monks. There is one rule. Of the six kinds of originations of offenses, it originates in three ways: … 

“There\marginnote{121.1} is an offense entailing confession for malicious talebearing between monks. Where was it laid down?” At \textsanskrit{Sāvatthī}. “Whom is it about?” The monks from the group of six. “What is it about?” Those monks engaging in malicious talebearing between monks who were arguing. There is one rule. Of the six kinds of originations of offenses, it originates in three ways: … 

“There\marginnote{122.1} is an offense entailing confession for instructing a person who is not fully ordained to memorize the Teaching. Where was it laid down?” At \textsanskrit{Sāvatthī}. “Whom is it about?” The monks from the group of six. “What is it about?” Those monks instructing lay followers to memorize the Teaching. There is one rule. Of the six kinds of originations of offenses, it originates in two ways: from speech, not from body or mind; or from speech and mind, not from body. … 

“There\marginnote{123.1} is an offense entailing confession for lying down more than two or three nights in the same sleeping place as a person who is not fully ordained. Where was it laid down?” At \textsanskrit{Āḷavī}. “Whom is it about?” A number of monks. “What is it about?” Those monks lying down in the same sleeping place as a person who was not fully ordained. There is one rule. There is one addition to the rule. Of the six kinds of originations of offenses, it originates in two ways: from body, not from speech or mind; or from body and mind, not from speech. … 

“There\marginnote{124.1} is an offense entailing confession for lying down in the same sleeping place as a woman. Where was it laid down?” At \textsanskrit{Sāvatthī}. “Whom is it about?” Venerable Anuruddha. “What is it about?” Anuruddha lying down in the same sleeping place as a woman. There is one rule. Of the six kinds of originations of offenses, it originates in two ways: … (as in the rule on wool) … 

“There\marginnote{125.1} is an offense entailing confession for giving a teaching of more than five or six sentences to a woman. Where was it laid down?” At \textsanskrit{Sāvatthī}. “Whom is it about?” Venerable \textsanskrit{Udāyī}. “What is it about?” \textsanskrit{Udāyī} giving a teaching of more than five or six sentences to a woman. There is one rule. There are two additions to the rule. Of the six kinds of originations of offenses, it originates in two ways: … (as in the rule on memorizing the Teaching) … 

“There\marginnote{126.1} is an offense entailing confession for truthfully telling a person who is not fully ordained of a superhuman quality. Where was it laid down?” At \textsanskrit{Vesālī}. “Whom is it about?” The monks from the banks of the \textsanskrit{Vaggumudā}. “What is it about?” Those monks praising one another’s superhuman qualities to householders. There is one rule. Of the six kinds of originations of offenses, it originates in three ways: from body, not from speech or mind; or from speech, not from body or mind; or from body and speech, not from mind. … 

“There\marginnote{127.1} is an offense entailing confession for telling a person who is not fully ordained about a monk’s grave offense. Where was it laid down?” At \textsanskrit{Sāvatthī}. “Whom is it about?” The monks from the group of six. “What is it about?” Those monks telling a person who is not fully ordained about a monk’s grave offense. There is one rule. Of the six kinds of originations of offenses, it originates in three ways: … 

“There\marginnote{128.1} is an offense entailing confession for digging the earth. Where was it laid down?” At \textsanskrit{Āḷavī}. “Whom is it about?” The monks of \textsanskrit{Āḷavī}. “What is it about?” Those monks digging the earth. There is one rule. Of the six kinds of originations of offenses, it originates in three ways: … 

\scendvagga{The first subchapter on lying is finished. }

\subsection*{The subchapter on plants }

“There\marginnote{130.1} is an offense entailing confession for destroying a plant. Where was it laid down?” At \textsanskrit{Āḷavī}. “Whom is it about?” The monks of \textsanskrit{Āḷavī}. “What is it about?” Those monks cutting down a tree. There is one rule. Of the six kinds of originations of offenses, it originates in three ways: … 

“There\marginnote{131.1} is an offense entailing confession for speaking evasively or harassing. Where was it laid down?” At \textsanskrit{Kosambī}. “Whom is it about?” Venerable Channa. “What is it about?” Channa speaking evasively when examined about an offense in the midst of the Sangha. There is one rule. There is one addition to the rule. Of the six kinds of originations of offenses, it originates in three ways: … 

“There\marginnote{132.1} is an offense entailing confession for complaining or criticizing. Where was it laid down?” At \textsanskrit{Rājagaha}. “Whom is it about?” The monks Mettiya and \textsanskrit{Bhūmajaka}. “What is it about?” Those monks complaining to monks about Venerable Dabba the Mallian. There is one rule. There is one addition to the rule. Of the six kinds of originations of offenses, it originates in three ways: … 

“There\marginnote{133.1} is an offense entailing confession for taking a bed, a bench, a mattress, or a stool belonging to the Sangha and putting it outside, and then departing without putting it away or informing anyone. Where was it laid down?” At \textsanskrit{Sāvatthī}. “Whom is it about?” A number of monks. “What is it about?” Those monks taking furniture belonging to the Sangha outside and then departing without putting it away or informing anyone. There is one rule. There is one addition to the rule. Of the six kinds of originations of offenses, it originates in two ways: … (as in the rule on the robe season) … 

“There\marginnote{134.1} is an offense entailing confession for putting out bedding in a dwelling belonging to the Sangha, and then departing without putting it away or informing anyone. Where was it laid down?” At \textsanskrit{Sāvatthī}. “Whom is it about?” The monks from the group of seventeen. “What is it about?” Those monks putting out bedding in a dwelling belonging to the Sangha, and then departing without putting it away or informing anyone. There is one rule. Of the six kinds of originations of offenses, it originates in two ways: … (as in the rule on the robe season) … 

“There\marginnote{135.1} is an offense entailing confession for arranging one’s sleeping place, in a dwelling belonging to the Sangha, in a way that encroaches on a monk that one knows arrived there before oneself. Where was it laid down?” At \textsanskrit{Sāvatthī}. “Whom is it about?” The monks from the group of six. “What is it about?” Those monks arranging their sleeping places in a way that encroached on the senior monks. There is one rule. Of the six kinds of originations of offenses, it originates in one way: from body and mind, not from speech. … 

“There\marginnote{136.1} is an offense entailing confession for angrily throwing a monk out of a dwelling belonging to the Sangha. Where was it laid down?” At \textsanskrit{Sāvatthī}. “Whom is it about?” The monks from the group of six. “What is it about?” Those monks angrily throwing monks out of a dwelling belonging to the Sangha. There is one rule. Of the six kinds of originations of offenses, it originates in three ways: … 

“There\marginnote{137.1} is an offense entailing confession for sitting down on a bed or a bench with detachable legs on an upper story in a dwelling belonging to the Sangha. Where was it laid down?” At \textsanskrit{Sāvatthī}. “Whom is it about?” A certain monk. “What is it about?” That monk sitting down hastily on a bed with detachable legs on an upper story in a dwelling belonging to the Sangha. There is one rule. Of the six kinds of originations of offenses, it originates in two ways: from body, not from speech or mind; or from body and mind, not from speech. … 

“There\marginnote{138.1} is an offense entailing confession for applying more than two or three courses. Where was it laid down?” At \textsanskrit{Kosambī}. “Whom is it about?” Venerable Channa. “What is it about?” Channa having a finished dwelling roofed and plastered over and over, so that it collapsed from overloading. There is one rule. Of the six kinds of originations of offenses, it originates in six ways: … 

“There\marginnote{139.1} is an offense entailing confession for pouring water that one knows contains living beings onto grass or clay. Where was it laid down?” At \textsanskrit{Āḷavī}. “Whom is it about?” The monks of \textsanskrit{Āḷavī}. “What is it about?” Those monks pouring water that they knew contained living beings onto grass and clay. There is one rule. Of the six kinds of originations of offenses, it originates in three ways: … 

\scendvagga{The second subchapter on plants is finished. }

\subsection*{The subchapter on the instruction }

“There\marginnote{141.1} is an offense entailing confession for instructing the nuns without being appointed. Where was it laid down?” At \textsanskrit{Sāvatthī}. “Whom is it about?” The monks from the group of six. “What is it about?” Those monks instructing the nuns without being appointed. “Is there a rule, an addition to the rule, an unprompted rule?” There is one rule. There is one addition to the rule. There is no unprompted rule. Of the six kinds of originations of offenses, it originates in two ways: from speech, not from body or mind; or from speech and mind, not from body. … 

“There\marginnote{142.1} is an offense entailing confession for instructing the nuns after sunset. Where was it laid down?” At \textsanskrit{Sāvatthī}. “Whom is it about?” Venerable \textsanskrit{Cūlapanthaka}. “What is it about?” \textsanskrit{Cūlapanthaka} instructing the nuns after sunset. There is one rule. Of the six kinds of originations of offenses, it originates in two ways: … (as in the rule on memorizing the Teaching) … 

“There\marginnote{143.1} is an offense entailing confession for going to the nuns’ dwelling place and instructing the nuns. Where was it laid down?” In the Sakyan country. “Whom is it about?” The monks from the group of six. “What is it about?” Those monks going to the nuns’ dwelling place and instructing the nuns. There is one rule. There is one addition to the rule. Of the six kinds of originations of offenses, it originates in two ways: … (as in the rule on the robe season) … 

“There\marginnote{144.1} is an offense entailing confession for saying that the monks are instructing the nuns for the sake of worldly gain. Where was it laid down?” At \textsanskrit{Sāvatthī}. “Whom is it about?” The monks from the group of six. “What is it about?” Those monks saying that the monks were instructing the nuns for the sake of worldly gain. There is one rule. Of the six kinds of originations of offenses, it originates in three ways: … 

“There\marginnote{145.1} is an offense entailing confession for giving robe-cloth to an unrelated nun. Where was it laid down?” At \textsanskrit{Sāvatthī}. “Whom is it about?” A certain monk. “What is it about?” That monk giving robe-cloth to an unrelated nun. There is one rule. There is one addition to the rule. Of the six kinds of originations of offenses, it originates in six ways: … 

“There\marginnote{146.1} is an offense entailing confession for sewing a robe for an unrelated nun. Where was it laid down?” At \textsanskrit{Sāvatthī}. “Whom is it about?” Venerable \textsanskrit{Udāyī}. “What is it about?” \textsanskrit{Udāyī} sewing a robe for an unrelated nun. There is one rule. Of the six kinds of originations of offenses, it originates in six ways: … 

“There\marginnote{147.1} is an offense entailing confession for traveling by arrangement with a nun. Where was it laid down?” At \textsanskrit{Sāvatthī}. “Whom is it about?” The monks from the group of six. “What is it about?” Those monks traveling by arrangement with nuns. There is one rule. There is one addition to the rule. Of the six kinds of originations of offenses, it originates in four ways: from body, not from speech or mind; or from body and speech, not from mind; or from body and mind, not from speech; or from body, speech, and mind. … 

“There\marginnote{148.1} is an offense entailing confession for boarding a boat by arrangement with a nun. Where was it laid down?” At \textsanskrit{Sāvatthī}. “Whom is it about?” The monks from the group of six. “What is it about?” Those monks boarding a boat by arrangement with nuns. There is one rule. There is one addition to the rule. Of the six kinds of originations of offenses, it originates in four ways: … 

“There\marginnote{149.1} is an offense entailing confession for eating almsfood knowing that a nun had it prepared. Where was it laid down?” At \textsanskrit{Rājagaha}. “Whom is it about?” Devadatta. “What is it about?” Devadatta eating almsfood knowing that a nun had it prepared. There is one rule. There is one addition to the rule. Of the six kinds of originations of offenses, it originates in one way: from body and mind, not from speech. … 

“There\marginnote{150.1} is an offense entailing confession for sitting down in private alone with a nun. Where was it laid down?” At \textsanskrit{Sāvatthī}. “Whom is it about?” Venerable \textsanskrit{Udāyī}. “What is it about?” \textsanskrit{Udāyī} sitting down in private alone with a nun. There is one rule. Of the six kinds of originations of offenses, it originates in one way: from body and mind, not from speech. … 

\scendvagga{The third subchapter on the instruction is finished. }

\subsection*{The subchapter on eating }

“There\marginnote{152.1} is an offense entailing confession for eating alms too often at a public guesthouse. Where was it laid down?” At \textsanskrit{Sāvatthī}. “Whom is it about?” The monks from the group of six. “What is it about?” Those monks staying on and on, eating alms at a public guesthouse. There is one rule. There is one addition to the rule. Of the six kinds of originations of offenses, it originates in two ways: … (as in the rule on wool) … 

“There\marginnote{153.1} is an offense entailing confession for eating in a group. Where was it laid down?” At \textsanskrit{Rājagaha}. “Whom is it about?” Devadatta. “What is it about?” Devadatta and his followers eating at invitations after repeatedly asking. There is one rule. There are seven additions to the rule. Of the six kinds of originations of offenses, it originates in two ways: … (as in the rule on wool) … 

“There\marginnote{154.1} is an offense entailing confession for eating one meal before another. Where was it laid down?” At \textsanskrit{Vesālī}. “Whom is it about?” A number of monks. “What is it about?” Those monks eating elsewhere when invited for a meal. There is one rule. There are four additions to the rule.\footnote{There are three additions that impact the wording of the rule, and one addition that establishes a method for assigning a meal to another. } Of the six kinds of originations of offenses, it originates in two ways: … (as in the rule on the robe season) … 

“There\marginnote{155.1} is an offense entailing confession for accepting more than two or three bowlfuls of cookies. Where was it laid down?” At \textsanskrit{Sāvatthī}. “Whom is it about?” A number of monks. “What is it about?” Those monks receiving without moderation. There is one rule. Of the six kinds of originations of offenses, it originates in six ways: … 

“There\marginnote{156.1} is an offense entailing confession for having finished one’s meal and refused an invitation to eat more, and then eating fresh or cooked food that is not left over. Where was it laid down?” At \textsanskrit{Sāvatthī}. “Whom is it about?” A number of monks. “What is it about?” Those monks having finished their meal and refused an invitation to eat more, and then eating elsewhere. There is one rule. There is one addition to the rule. Of the six kinds of originations of offenses, it originates in two ways: … (as in the rule on the robe season) … 

“There\marginnote{157.1} is an offense entailing confession for inviting a monk who has finished his meal and refused an invitation to eat more to eat fresh or cooked food that is not left over. Where was it laid down?” At \textsanskrit{Sāvatthī}. “Whom is it about?” A certain monk. “What is it about?” That monk inviting a monk who had finished his meal and refused an invitation to eat more to eat food that was not left over. There is one rule. Of the six kinds of originations of offenses, it originates in three ways: … 

“There\marginnote{158.1} is an offense entailing confession for eating fresh or cooked food at the wrong time. Where was it laid down?” At \textsanskrit{Rājagaha}. “Whom is it about?” The monks from the group of seventeen. “What is it about?” Those monks eating at the wrong time. There is one rule. Of the six kinds of originations of offenses, it originates in two ways: … (as in the rule on wool) … 

“There\marginnote{159.1} is an offense entailing confession for storing and then eating fresh or cooked food. Where was it laid down?” At \textsanskrit{Sāvatthī}. “Whom is it about?” Venerable \textsanskrit{Belaṭṭhasīsa}. “What is it about?” \textsanskrit{Belaṭṭhasīsa} storing food and then eating it. There is one rule. Of the six kinds of originations of offenses, it originates in two ways: … (as in the rule on wool) … 

“There\marginnote{160.1} is an offense entailing confession for eating fine foods that one has requested for oneself. Where was it laid down?” At \textsanskrit{Sāvatthī}. “Whom is it about?” The monks from the group of six. “What is it about?” Those monks eating fine foods that they had requested for themselves. There is one rule. There is one addition to the rule. Of the six kinds of originations of offenses, it originates in four ways: … 

“There\marginnote{161.1} is an offense entailing confession for eating food that has not been given. Where was it laid down?” At \textsanskrit{Vesālī}. “Whom is it about?” A certain monk. “What is it about?” That monk eating food that had not been given. There is one rule. There is one addition to the rule. Of the six kinds of originations of offenses, it originates in two ways: … (as in the rule on wool) … 

\scendvagga{The fourth subchapter on eating is finished. }

\subsection*{The subchapter on naked ascetics }

“There\marginnote{163.1} is an offense entailing confession for personally giving fresh or cooked food to a naked ascetic, to a male wanderer, or to a female wanderer. Where was it laid down?” At \textsanskrit{Vesālī}. “Whom is it about?” Venerable Ānanda. “What is it about?” Ānanda giving two cookies, thinking they were one, to a certain a female wanderer. There is one rule. Of the six kinds of originations of offenses, it originates in two ways: … (as in the rule on wool) … 

“There\marginnote{164.1} is an offense entailing confession for saying to a monk, ‘Come, let’s go to the village or town for alms,’ and then, whether he has had food given to him or not, sending him away. Where was it laid down?” At \textsanskrit{Sāvatthī}. “Whom is it about?” Venerable Upananda the Sakyan. “What is it about?” Upananda saying to a monk, “Come, let’s go to the village or town for alms,” and then, without having had food given to him, sending him away. There is one rule. Of the six kinds of originations of offenses, it originates in three ways: … 

“There\marginnote{165.1} is an offense entailing confession for sitting down intruding on a lustful couple. Where was it laid down?” At \textsanskrit{Sāvatthī}. “Whom is it about?” Venerable Upananda the Sakyan. “What is it about?” Upananda sitting down intruding on a lustful couple. There is one rule. Of the six kinds of originations of offenses, it originates in one way: from body and mind, not from speech. … 

“There\marginnote{166.1} is an offense entailing confession for sitting down in private on a concealed seat with a woman. Where was it laid down?” At \textsanskrit{Sāvatthī}. “Whom is it about?” Venerable Upananda the Sakyan. “What is it about?” Upananda sitting down in private on a concealed seat with a woman. There is one rule. Of the six kinds of originations of offenses, it originates in one way: from body and mind, not from speech. … 

“There\marginnote{167.1} is an offense entailing confession for sitting down in private alone with a woman. Where was it laid down?” At \textsanskrit{Sāvatthī}. “Whom is it about?” Venerable Upananda the Sakyan. “What is it about?” Upananda sitting down in private alone with a woman. There is one rule. Of the six kinds of originations of offenses, it originates in one way: from body and mind, not from speech. … 

“There\marginnote{168.1} is an offense entailing confession for being invited to a meal and then visiting families beforehand or afterwards without informing an available monk. Where was it laid down?” At \textsanskrit{Rājagaha}. “Whom is it about?” Venerable Upananda the Sakyan. “What is it about?” Upananda having been invited to a meal and then visiting families beforehand and afterwards. There is one rule. There are four additions to the rule. Of the six kinds of originations of offenses, it originates in two ways: … (as in the rule on the robe season) … 

“There\marginnote{169.1} is an offense entailing confession for asking for too many tonics. Where was it laid down?” In the Sakyan country. “Whom is it about?” The monks from the group of six. “What is it about?” Those monks not waiting for one day when asked by \textsanskrit{Mahānāma} the Sakyan. There is one rule. Of the six kinds of originations of offenses, it originates in six ways: … 

“There\marginnote{170.1} is an offense entailing confession for going to see an army. Where was it laid down?” At \textsanskrit{Sāvatthī}. “Whom is it about?” The monks from the group of six. “What is it about?” Those monks going to see an army. There is one rule. There is one addition to the rule. Of the six kinds of originations of offenses, it originates in two ways: … (as in the rule on wool) … 

“There\marginnote{171.1} is an offense entailing confession for staying with the army for more than three nights. Where was it laid down?” At \textsanskrit{Sāvatthī}. “Whom is it about?” The monks from the group of six. “What is it about?” Those monks staying with the army for more than three nights. There is one rule. Of the six kinds of originations of offenses, it originates in two ways: … (as in the rule on wool) … 

“There\marginnote{172.1} is an offense entailing confession for going to a battle. Where was it laid down?” At \textsanskrit{Sāvatthī}. “Whom is it about?” The monks from the group of six. “What is it about?” Those monks going to a battle. There is one rule. Of the six kinds of originations of offenses, it originates in two ways: … (as in the rule on wool) … 

\scendvagga{The fifth subchapter on naked ascetics is finished. }

\subsection*{The subchapter on drinking alcohol }

“There\marginnote{174.1} is an offense entailing confession for drinking an alcoholic drink. Where was it laid down?” At \textsanskrit{Kosambī}. “Whom is it about?” Venerable \textsanskrit{Sāgata}. “What is it about?” \textsanskrit{Sāgata} drinking alcohol. There is one rule. Of the six kinds of originations of offenses, it originates in two ways: from body, not from speech or mind; or from body and mind, not from speech. … 

“There\marginnote{175.1} is an offense entailing confession for tickling. Where was it laid down?” At \textsanskrit{Sāvatthī}. “Whom is it about?” The monks from the group of six. “What is it about?” Those monks tickling a monk to make him laugh. There is one rule. Of the six kinds of originations of offenses, it originates in one way: from body and mind, not from speech. … 

“There\marginnote{176.1} is an offense entailing confession for playing in water. Where was it laid down?” At \textsanskrit{Sāvatthī}. “Whom is it about?” The monks from the group of seventeen. “What is it about?” Those monks playing in the water of the river \textsanskrit{Aciravatī}. There is one rule. Of the six kinds of originations of offenses, it originates in one way: from body and mind, not from speech. … 

“There\marginnote{177.1} is an offense entailing confession for disrespect. Where was it laid down?” At \textsanskrit{Kosambī}. “Whom is it about?” Venerable Channa. “What is it about?” Channa acting disrespectfully. There is one rule. Of the six kinds of originations of offenses, it originates in three ways: … 

“There\marginnote{178.1} is an offense entailing confession for scaring a monk. Where was it laid down?” At \textsanskrit{Sāvatthī}. “Whom is it about?” The monks from the group of six. “What is it about?” Those monks scaring a monk. There is one rule. Of the six kinds of originations of offenses, it originates in three ways: … 

“There\marginnote{179.1} is an offense entailing confession for lighting a fire and warming oneself. Where was it laid down?” In the \textsanskrit{Bhaggā} country. “Whom is it about?” A number of monks. “What is it about?” Those monks warming themselves by lighting a fire. There is one rule. There are two additions to the rule. Of the six kinds of originations of offenses, it originates in six ways: … 

“There\marginnote{180.1} is an offense entailing confession for bathing at intervals of less than a half-month. Where was it laid down?” At \textsanskrit{Rājagaha}. “Whom is it about?” A number of monks. “What is it about?” Those monks bathing without moderation, even after seeing the king. There is one rule. There are six additions to the rule. “Is it a rule that applies everywhere or in a particular place?” In a particular place. Of the six kinds of originations of offenses, it originates in two ways: … (as in the rule on wool) … 

“There\marginnote{181.1} is an offense entailing confession for using a new robe without first applying one of the three kinds of stains. Where was it laid down?” At \textsanskrit{Sāvatthī}. “Whom is it about?” A number of monks. “What is it about?” Those monks not recognizing their own robes. There is one rule. Of the six kinds of originations of offenses, it originates in two ways: … (as in the rule on wool) … 

“There\marginnote{182.1} is an offense entailing confession for assigning the ownership of a robe to a monk, a nun, a trainee nun, a novice monk, or a novice nun, and then using it without the other first relinquishing it. Where was it laid down?”\footnote{For an explanation of the idea of \textit{\textsanskrit{vikappanā}}, see Appendix of Technical Terms. } At \textsanskrit{Sāvatthī}. “Whom is it about?” Venerable Upananda the Sakyan. “What is it about?” Upananda assigning the ownership of a robe to a monk and then using it without that monk first relinquishing it. There is one rule. Of the six kinds of originations of offenses, it originates in two ways: … (as in the rule on the robe season) … 

“There\marginnote{183.1} is an offense entailing confession for hiding a monk’s bowl, robe, sitting mat, needle case, or belt. Where was it laid down?” At \textsanskrit{Sāvatthī}. “Whom is it about?” The monks from the group of six. “What is it about?” Those monks hiding other monks’ bowls and robes. There is one rule. Of the six kinds of originations of offenses, it originates in three ways: … 

\scendvagga{The sixth subchapter of alcoholic drinks is finished. }

\subsection*{The subchapter on containing living beings }

“There\marginnote{185.1} is an offense entailing confession for intentionally killing a living being. Where was it laid down?” At \textsanskrit{Sāvatthī}. “Whom is it about?” Venerable \textsanskrit{Udāyī}. “What is it about?” \textsanskrit{Udāyī} intentionally killing a living being. There is one rule. Of the six kinds of originations of offenses, it originates in three ways: … 

“There\marginnote{186.1} is an offense entailing confession for using water that one knows contains living beings. Where was it laid down?” At \textsanskrit{Sāvatthī}. “Whom is it about?” The monks from the group of six. “What is it about?” Those monks using water that they knew contained living beings. There is one rule. Of the six kinds of originations of offenses, it originates in three ways: … 

“There\marginnote{187.1} is an offense entailing confession for reopening a legal issue that one knows has been legitimately settled. Where was it laid down?” At \textsanskrit{Sāvatthī}. “Whom is it about?” The monks from the group of six. “What is it about?” Those monks reopening a legal issue that they knew had been legitimately settled. There is one rule. Of the six kinds of originations of offenses, it originates in three ways: … 

“There\marginnote{188.1} is an offense entailing confession for knowingly concealing a monk’s grave offense. Where was it laid down?” At \textsanskrit{Sāvatthī}. “Whom is it about?” A certain monk. “What is it about?” That monk knowingly concealing a monk’s grave offense. There is one rule. Of the six kinds of originations of offenses, it originates in one way: from body, speech, and mind. … 

“There\marginnote{189.1} is an offense entailing confession for giving the full ordination to a person one knows is less than twenty years old. Where was it laid down?” At \textsanskrit{Rājagaha}. “Whom is it about?” A number of monks. “What is it about?” Those monks giving the full ordination to a person they knew was less than twenty years old. There is one rule. Of the six kinds of originations of offenses, it originates in three ways: … 

“There\marginnote{190.1} is an offense entailing confession for knowingly traveling by arrangement with a group of thieves. Where was it laid down?” At \textsanskrit{Sāvatthī}. “Whom is it about?” A certain monk. “What is it about?” That monk knowingly traveling by arrangement with a group of thieves. There is one rule. Of the six kinds of originations of offenses, it originates in two ways: from body and mind, not from speech; or from body, speech, and mind. … 

“There\marginnote{191.1} is an offense entailing confession for traveling by arrangement with a woman. Where was it laid down?” At \textsanskrit{Sāvatthī}. “Whom is it about?” A certain monk. “What is it about?” That monk traveling by arrangement with a woman. There is one rule. Of the six kinds of originations of offenses, it originates in four ways: … 

“There\marginnote{192.1} is an offense entailing confession for not giving up a bad view when pressed for the third time. Where was it laid down?” At \textsanskrit{Sāvatthī}. “Whom is it about?” The monk \textsanskrit{Ariṭṭha}, an ex-vulture-killer. “What is it about?” \textsanskrit{Ariṭṭha} not giving up a bad view when pressed for the third time. There is one rule. Of the six kinds of originations of offenses, it originates in one way: from body, speech, and mind. … 

“There\marginnote{193.1} is an offense entailing confession for living with a monk who one knows is saying such things, who has not made amends according to the rule, and who has not given up that view. Where was it laid down?”\footnote{“Such things” and “that view” refer to the idea that sexual intercourse is not an obstacle to spiritual progress, see \href{https://suttacentral.net/pli-tv-bu-vb-pc68/en/brahmali\#1.49.1}{Bu Pc 68:1.49.1}. } At \textsanskrit{Sāvatthī}. “Whom is it about?” The monks from the group of six. “What is it about?” Those monks living with the monk \textsanskrit{Ariṭṭha} who they knew was saying such things, who had not made amends according to the rule, and who had not given up that view. There is one rule. Of the six kinds of originations of offenses, it originates in three ways: … 

“There\marginnote{194.1} is an offense entailing confession for befriending a novice monastic who one knows has been expelled in this way. Where was it laid down?”\footnote{For the meaning of “in this way”, see \href{https://suttacentral.net/pli-tv-bu-vb-pc70/en/brahmali\#1.46.1}{Bu Pc 70:1.46.1}. } At \textsanskrit{Sāvatthī}. “Whom is it about?” The monks from the group of six. “What is it about?” Those monks befriended the novice monastic \textsanskrit{Kaṇṭaka} who they knew had been expelled in this way. There is one rule. Of the six kinds of originations of offenses, it originates in three ways: … 

\scendvagga{The seventh subchapter on containing living beings is finished. }

\subsection*{The subchapter on legitimately }

“When\marginnote{196.1} legitimately corrected by the monks, there is an offense entailing confession for saying, ‘I won’t practice this training rule until I’ve questioned a monk who is an expert on the Monastic Law.’ Where was it laid down?” At \textsanskrit{Kosambī}. “Whom is it about?” Venerable Channa. “What is it about?” Channa, when legitimately corrected by the monks, saying, “I won’t practice this training rule until I’ve questioned a monk who is an expert on the Monastic Law”. There is one rule. Of the six kinds of originations of offenses, it originates in three ways: … 

“There\marginnote{197.1} is an offense entailing confession for disparaging the Monastic Law. Where was it laid down?” At \textsanskrit{Sāvatthī}. “Whom is it about?” The monks from the group of six. “What is it about?” Those monks disparaging the Monastic Law. There is one rule. Of the six kinds of originations of offenses, it originates in three ways: … 

“There\marginnote{198.1} is an offense entailing confession for the act of deception. Where was it laid down?” At \textsanskrit{Sāvatthī}. “Whom is it about?” The monks from the group of six. “What is it about?” Those monks acting to deceive. There is one rule. Of the six kinds of originations of offenses, it originates in three ways: … 

“There\marginnote{199.1} is an offense entailing confession for hitting a monk in anger. Where was it laid down?” At \textsanskrit{Sāvatthī}. “Whom is it about?” The monks from the group of six. “What is it about?” Those monks hitting other monks in anger. There is one rule. Of the six kinds of originations of offenses, it originates in one way: from body and mind, not from speech. … 

“There\marginnote{200.1} is an offense entailing confession for raising a hand in anger against a monk. Where was it laid down?” At \textsanskrit{Sāvatthī}. “Whom is it about?” The monks from the group of six. “What is it about?” Those monks raising a hand in anger against other monks. There is one rule. Of the six kinds of originations of offenses, it originates in one way: from body and mind, not from speech. … 

“There\marginnote{201.1} is an offense entailing confession for groundlessly charging a monk with an offense entailing suspension. Where was it laid down?” At \textsanskrit{Sāvatthī}. “Whom is it about?” The monks from the group of six. “What is it about?” Those monks groundlessly charging a monk with an offense entailing suspension. There is one rule. Of the six kinds of originations of offenses, it originates in three ways: … 

“There\marginnote{202.1} is an offense entailing confession for intentionally making a monk anxious. Where was it laid down?” At \textsanskrit{Sāvatthī}. “Whom is it about?” The monks from the group of six. “What is it about?” Those monks intentionally making monks anxious. There is one rule. Of the six kinds of originations of offenses, it originates in three ways: … 

“There\marginnote{203.1} is an offense entailing confession for eavesdropping on monks who are arguing and disputing. Where was it laid down?” At \textsanskrit{Sāvatthī}. “Whom is it about?” The monks from the group of six. “What is it about?” Those monks eavesdropping on monks who were arguing and disputing. There is one rule. Of the six kinds of originations of offenses, it originates in two ways: from body and mind, not from speech; or from body, speech, and mind. … 

“There\marginnote{204.1} is an offense entailing confession for giving one’s consent to legitimate legal procedures and then criticizing them afterwards. Where was it laid down?” At \textsanskrit{Sāvatthī}. “Whom is it about?” The monks from the group of six. “What is it about?” Those monks giving their consent to legitimate legal procedures and then criticizing them afterwards. There is one rule. Of the six kinds of originations of offenses, it originates in three ways: … 

“There\marginnote{205.1} is an offense entailing confession for, without first giving one’s consent, getting up from one’s seat and leaving while the Sangha is in the middle of a discussion. Where was it laid down?” At \textsanskrit{Sāvatthī}. “Whom is it about?” A certain monk. “What is it about?” That monk getting up from his seat and leaving while the Sangha was in the middle of a discussion, without first giving his consent. There is one rule. Of the six kinds of originations of offenses, it originates in one way: from body, speech, and mind. … 

“There\marginnote{206.1} is an offense entailing confession for giving out a robe as part of a unanimous Sangha and then criticizing it afterwards. Where was it laid down?” At \textsanskrit{Rājagaha}. “Whom is it about?” The monks from the group of six. “What is it about?” Those monks giving out a robe as part of a unanimous Sangha and then criticizing it afterwards. There is one rule. Of the six kinds of originations of offenses, it originates in three ways: … 

“There\marginnote{207.1} is an offense entailing confession for diverting to an individual material support that one knows was intended for the Sangha. Where was it laid down?” At \textsanskrit{Sāvatthī}. “Whom is it about?” The monks from the group of six. “What is it about?” Those monks diverting to an individual material support that they knew was intended for the Sangha. There is one rule. Of the six kinds of originations of offenses, it originates in three ways: … 

\scendvagga{The eighth subchapter on legitimately is finished. }

\subsection*{The subchapter on kings }

“There\marginnote{209.1} is an offense entailing confession for entering the royal compound without first being announced. Where was it laid down?” At \textsanskrit{Sāvatthī}. “Whom is it about?” Venerable Ānanda. “What is it about?” Ānanda entering the royal compound without first being announced. There is one rule. Of the six kinds of originations of offenses, it originates in two ways: … (as in the rule on the robe season) … 

“There\marginnote{210.1} is an offense entailing confession for picking up something precious. Where was it laid down?” At \textsanskrit{Sāvatthī}. “Whom is it about?” A certain monk. “What is it about?” That monk picking up something precious. There is one rule. There are two additions to the rule. Of the six kinds of originations of offenses, it originates in six ways: … 

“There\marginnote{211.1} is an offense entailing confession for entering an inhabited area at the wrong time without informing an available monk. Where was it laid down?” At \textsanskrit{Sāvatthī}. “Whom is it about?” The monks from the group of six. “What is it about?” Those monks entering an inhabited area at the wrong time. There is one rule. There are three additions to the rule. Of the six kinds of originations of offenses, it originates in two ways: … (as in the rule on the robe season) … 

“There\marginnote{212.1} is an offense entailing confession for having a needle case made from bone, ivory, or horn. Where was it laid down?” In the Sakyan country. “Whom is it about?” A number of monks. “What is it about?” Those monks having no sense of moderation and asking for many needle cases. There is one rule. Of the six kinds of originations of offenses, it originates in six ways: … 

“There\marginnote{213.1} is an offense entailing confession for having a bed or a bench made that exceeds the right height. Where was it laid down?” At \textsanskrit{Sāvatthī}. “Whom is it about?” Venerable Upananda the Sakyan. “What is it about?” Upananda sleeping on a high bed. There is one rule. Of the six kinds of originations of offenses, it originates in six ways: … 

“There\marginnote{214.1} is an offense entailing confession for having a bed or a bench made upholstered with cotton down. Where was it laid down?” At \textsanskrit{Sāvatthī}. “Whom is it about?” The monks from the group of six. “What is it about?” Those monks having a bed or a bench made upholstered with cotton down. There is one rule. Of the six kinds of originations of offenses, it originates in six ways: … 

“There\marginnote{215.1} is an offense entailing confession for having a sitting mat made that exceeds the right size. Where was it laid down?” At \textsanskrit{Sāvatthī}. “Whom is it about?” The monks from the group of six. “What is it about?” Those monks using inappropriately-sized sitting mats. There is one rule. There is one addition to the rule. Of the six kinds of originations of offenses, it originates in six ways: … 

“There\marginnote{216.1} is an offense entailing confession for having an itch-covering cloth made that exceeds the right size. Where was it laid down?” At \textsanskrit{Sāvatthī}. “Whom is it about?” The monks from the group of six. “What is it about?” Those monks wearing inappropriately-sized itch-covering cloths. There is one rule. Of the six kinds of originations of offenses, it originates in six ways: … 

“There\marginnote{217.1} is an offense entailing confession for having a rainy-season robe made that exceeds the right size. Where was it laid down?” At \textsanskrit{Sāvatthī}. “Whom is it about?” The monks from the group of six. “What is it about?” Those monks wearing inappropriately-sized rainy-season robes. There is one rule. Of the six kinds of originations of offenses, it originates in six ways: … 

“There\marginnote{218.1} is an offense entailing confession for having a robe made that is the standard robe size. Where was it laid down?” At \textsanskrit{Sāvatthī}. “Whom is it about?” Venerable Nanda. “What is it about?” Nanda wearing a robe that was the standard robe size. There is one rule. Of the six kinds of originations of offenses, it originates in six ways: … 

\scendvagga{The ninth subchapter on kings is finished.\footnote{In the \textsanskrit{Bhikkhuvibhaṅga}, this is called “The subchapter on precious things”, named after the second rule of the subchapter. } }

\scend{The ninety-two offenses entailing confession are finished. }

\scend{The section on minor rules has been completed. }

\scuddanaintro{This is the summary: }

\begin{scuddana}%
“Falsely,\marginnote{221.1} abusive, and malicious talebearing, \\
Memorizing, bed, and with a woman; \\
Except with one who understands, true, \\
Grave offense, digging. 

Plant,\marginnote{222.1} with evasion, complaining, \\
Bed, and it is called bedding; \\
Before, throwing out, detachable, \\
Door, and containing living beings. 

Not\marginnote{223.1} appointed, set, \\
Dwelling place, and worldly gain; \\
Should he give, should he sew, by arrangement, \\
Boat, should eat, together. 

Alms,\marginnote{224.1} group, another, cookie, \\
Himself invited, another invited; \\
At the wrong time, store, milk, \\
With tooth cleaner—those are the ten. 

Naked\marginnote{225.1} ascetic, sending away, intruding on, \\
Concealed, and private; \\
Invited, with requisites, \\
Army, staying, battle. 

Alcohol,\marginnote{226.1} finger, and laughter, \\
And disrespect, scaring; \\
Fire, bathing, stain, \\
Himself, and with hiding. 

Intentionally,\marginnote{227.1} water, and legal procedure, \\
Grave, less than twenty; \\
Thieves, woman, not taught, \\
In the community, and with one who has been expelled. 

Legitimately,\marginnote{228.1} oppression, \\
Deception, on hitting, should he raise; \\
And groundless, intentionally, \\
‘I’ll hear,’ criticism, should he leave. 

After\marginnote{229.1} giving a robe with the Sangha, \\
Should he divert to an individual; \\
And a king’s, precious things, available, \\
Needle, and bed, cotton down; \\
Sitting mat, itch-covering cloth, \\
Rainy-season, and by the standard.” 

%
\end{scuddana}

\scuddanaintro{This is the summary of the subchapters: }

\begin{scuddana}%
“Falsely,\marginnote{231.1} and plants, instruction, \\
Eating, and with a naked ascetic; \\
Alcohol, containing living beings, legitimately, \\
With the subchapter on kings—these nine.” 

%
\end{scuddana}

\section*{6. The chapter on offenses entailing acknowledgment }

“The\marginnote{232.1} offense entailing acknowledgment for eating fresh or cooked food that was received directly from an unrelated nun who had entered an inhabited area was laid down by the Buddha who knows and sees, the Perfected One, the fully Awakened One. Where was it laid down?” At \textsanskrit{Sāvatthī}. “Whom is it about?” A certain monk. “What is it about?” That monk receiving food directly from an unrelated nun who had entered an inhabited area. There is one rule. Of the six kinds of originations of offenses, it originates in two ways: from body, not from speech or mind; or from body and mind, not from speech. … 

“There\marginnote{233.1} is an offense entailing acknowledgment for eating without having restrained a nun who is giving directions. Where was it laid down?” At \textsanskrit{Rājagaha}. “Whom is it about?” The monks from the group of six. “What is it about?” Those monks not restraining a nun who was giving directions. There is one rule. Of the six kinds of originations of offenses, it originates in two ways: from body and speech, not from mind; or from body, speech, and mind. … 

“There\marginnote{234.1} is an offense entailing acknowledgment for eating fresh or cooked food after personally receiving it from families designated as ‘in training’. Where was it laid down?” At \textsanskrit{Sāvatthī}. “Whom is it about?” A number of monks. “What is it about?” Those monks receiving with no sense of moderation. There is one rule. There are two additions to the rule. Of the six kinds of originations of offenses, it originates in two ways: from body, not from speech or mind; or from body and mind, not from speech. … 

“There\marginnote{235.1} is an offense entailing acknowledgment for eating fresh or cooked food after personally receiving it inside a wilderness monastery without first making an announcement. Where was it laid down?” In the Sakyan country. “Whom is it about?” A number of monks. “What is it about?” Those monks not informing that there were bandits staying in the monastery. There is one rule. There is one addition to the rule. Of the six kinds of originations of offenses, it originates in two ways: from body and speech, not from mind; or from body, speech, and mind. … 

\scend{The four offenses entailing acknowledgment are finished. }

\scuddanaintro{This is the summary: }

\begin{scuddana}%
“From\marginnote{238.1} one who is unrelated, giving directions, \\
In training, and with wilderness—\\
The four offenses entailing acknowledgment, \\
Proclaimed by the Awakened One.” 

%
\end{scuddana}

\section*{7. The chapter on training }

\subsection*{The subchapter on evenly all around }

“The\marginnote{239.1} offense of wrong conduct for, out of disrespect, wearing one’s sarong hanging down in front or behind was laid down by the Buddha who knows and sees, the Perfected One, the fully Awakened One. Where was it laid down?” At \textsanskrit{Sāvatthī}. “Whom is it about?” The monks from the group of six. “What is it about?” Those monks wearing their sarongs hanging down in front and behind. There is one rule. Of the six kinds of originations of offenses, it originates in one way: from body and mind, not from speech. … 

“There\marginnote{240.1} is an offense of wrong conduct for, out of disrespect, wearing one’s upper robe hanging down in front or behind. Where was it laid down?” At \textsanskrit{Sāvatthī}. “Whom is it about?” The monks from the group of six. “What is it about?” Those monks wearing their upper robes hanging down in front and behind. There is one rule. Of the six kinds of originations of offenses, it originates in one way: from body and mind, not from speech. … 

“There\marginnote{241.1} is an offense of wrong conduct for, out of disrespect, walking in an inhabited area with one’s body uncovered …” … There is one rule. It originates in one way: from body and mind, not from speech. … 

“There\marginnote{242.1} is an offense of wrong conduct for, out of disrespect, sitting in an inhabited area with one’s body uncovered …” … There is one rule. It originates in one way: from body and mind, not from speech. … 

“There\marginnote{243.1} is an offense of wrong conduct for, out of disrespect, walking in an inhabited area, playing with one’s hands and feet …” … There is one rule. It originates in one way: from body and mind, not from speech. … 

“There\marginnote{244.1} is an offense of wrong conduct for, out of disrespect, sitting in an inhabited area, playing with one’s hands and feet …” … There is one rule. It originates in one way: from body and mind, not from speech. … 

“There\marginnote{245.1} is an offense of wrong conduct for, out of disrespect, walking in an inhabited area, looking here and there …” … There is one rule. It originates in one way: from body and mind, not from speech. … 

“There\marginnote{246.1} is an offense of wrong conduct for, out of disrespect, sitting in an inhabited area, looking here and there …” … There is one rule. It originates in one way: from body and mind, not from speech. … 

“There\marginnote{247.1} is an offense of wrong conduct for, out of disrespect, walking in an inhabited area with a lifted robe …” … There is one rule. It originates in one way: from body and mind, not from speech. … 

“There\marginnote{248.1} is an offense of wrong conduct for, out of disrespect, sitting in an inhabited area with a lifted robe …” … There is one rule. It originates in one way: from body and mind, not from speech. … 

\scendvagga{The first subchapter on evenly all around is finished. }

\subsection*{The subchapter on laughing loudly }

“There\marginnote{250.1} is an offense of wrong conduct for, out of disrespect, laughing loudly while walking in an inhabited area. Where was it laid down?” At \textsanskrit{Sāvatthī}. “Whom is it about?” The monks from the group of six. “What is it about?” Those monks laughing loudly while walking in an inhabited area. There is one rule. Of the six kinds of originations of offenses, it originates in one way: from body, speech, and mind. … 

“There\marginnote{251.1} is an offense of wrong conduct for, out of disrespect, laughing loudly while sitting in an inhabited area. Where was it laid down?” At \textsanskrit{Sāvatthī}. “Whom is it about?” The monks from the group of six. “What is it about?” Those monks laughing loudly while sitting in an inhabited area. There is one rule. Of the six kinds of originations of offenses, it originates in one way: from body, speech, and mind. … 

“There\marginnote{252.1} is an offense of wrong conduct for, out of disrespect, being noisy while walking in an inhabited area. Where was it laid down?” At \textsanskrit{Sāvatthī}. “Whom is it about?” The monks from the group of six. “What is it about?” Those monks being noisy while walking in an inhabited area. There is one rule. Of the six kinds of originations of offenses, it originates in one way: from body, speech, and mind. … 

“There\marginnote{253.1} is an offense of wrong conduct for, out of disrespect, being noisy while sitting in an inhabited area. Where was it laid down?” At \textsanskrit{Sāvatthī}. “Whom is it about?” The monks from the group of six. “What is it about?” Those monks being noisy while sitting in an inhabited area. There is one rule. Of the six kinds of originations of offenses, it originates in one way: from body, speech, and mind. … 

“There\marginnote{254.1} is an offense of wrong conduct for, out of disrespect, swaying one’s body while walking in an inhabited area …” … There is one rule. It originates in one way: from body and mind, not from speech. … 

“There\marginnote{255.1} is an offense of wrong conduct for, out of disrespect, swaying one’s body while sitting in an inhabited area …” … There is one rule. It originates in one way: from body and mind, not from speech. … 

“There\marginnote{256.1} is an offense of wrong conduct for, out of disrespect, swinging one’s arms while walking in an inhabited area …” … There is one rule. It originates in one way: from body and mind, not from speech. … 

“There\marginnote{257.1} is an offense of wrong conduct for, out of disrespect, swinging one’s arms while sitting in an inhabited area …” … There is one rule. It originates in one way: from body and mind, not from speech. … 

“There\marginnote{258.1} is an offense of wrong conduct for, out of disrespect, swaying one’s head while walking in an inhabited area …” … There is one rule. It originates in one way: from body and mind, not from speech. … 

“There\marginnote{259.1} is an offense of wrong conduct for, out of disrespect, swaying one’s head while sitting in an inhabited area …” … There is one rule. It originates in one way: from body and mind, not from speech. … 

\scendvagga{The second subchapter on laughing loudly is finished. }

\subsection*{The subchapter on hands on hips }

“There\marginnote{261.1} is an offense of wrong conduct for, out of disrespect, walking in an inhabited area with one’s hands on one’s hips …” … There is one rule. It originates in one way: from body and mind, not from speech. … 

“There\marginnote{262.1} is an offense of wrong conduct for, out of disrespect, sitting in an inhabited area with one’s hands on one’s hips …” … There is one rule. It originates in one way: from body and mind, not from speech. … 

“There\marginnote{263.1} is an offense of wrong conduct for, out of disrespect, walking in an inhabited area with a covered head. Where was it laid down?” At \textsanskrit{Sāvatthī}. “Whom is it about?” The monks from the group of six. “What is it about?” Those monks walking in an inhabited area with their upper robes covering their heads. There is one rule. Of the six kinds of originations of offenses, it originates in one way: from body and mind, not from speech. … 

“There\marginnote{264.1} is an offense of wrong conduct for, out of disrespect, sitting in an inhabited area with a covered head. Where was it laid down?” At \textsanskrit{Sāvatthī}. “Whom is it about?” The monks from the group of six. “What is it about?” Those monks sitting in an inhabited area with their upper robes covering their heads. There is one rule. Of the six kinds of originations of offenses, it originates in one way: from body and mind, not from speech. … 

“There\marginnote{265.1} is an offense of wrong conduct for, out of disrespect, moving about while squatting on one’s heels in an inhabited area …” … There is one rule. It originates in one way: from body and mind, not from speech. … 

“There\marginnote{266.1} is an offense of wrong conduct for, out of disrespect, clasping one’s knees while sitting in an inhabited area …” … There is one rule. It originates in one way: from body and mind, not from speech. … 

“There\marginnote{267.1} is an offense of wrong conduct for, out of disrespect, receiving almsfood contemptuously …” … There is one rule. It originates in one way: from body and mind, not from speech. … 

“There\marginnote{268.1} is an offense of wrong conduct for, out of disrespect, receiving almsfood while looking here and there …” … There is one rule. It originates in one way: from body and mind, not from speech. … 

“There\marginnote{269.1} is an offense of wrong conduct for, out of disrespect, receiving large amounts of bean curry …” … There is one rule. It originates in one way: from body and mind, not from speech. … 

“There\marginnote{270.1} is an offense of wrong conduct for, out of disrespect, receiving almsfood in a heap …” … There is one rule. It originates in one way: from body and mind, not from speech. … 

\scendvagga{The third subchapter on hands on hips is finished. }

\subsection*{The subchapter on almsfood }

“There\marginnote{272.1} is an offense of wrong conduct for, out of disrespect, eating almsfood contemptuously …” … There is one rule. It originates in one way: from body and mind, not from speech. … 

“There\marginnote{273.1} is an offense of wrong conduct for, out of disrespect, eating almsfood while looking here and there …” … There is one rule. It originates in one way: from body and mind, not from speech. … 

“There\marginnote{274.1} is an offense of wrong conduct for, out of disrespect, eating almsfood picking here and there …” … There is one rule. It originates in one way: from body and mind, not from speech. … 

“There\marginnote{275.1} is an offense of wrong conduct for, out of disrespect, eating large amounts of bean curry …” … There is one rule. It originates in one way: from body and mind, not from speech. … 

“There\marginnote{276.1} is an offense of wrong conduct for, out of disrespect, eating almsfood after making a heap …” … There is one rule. It originates in one way: from body and mind, not from speech. … 

“There\marginnote{277.1} is an offense of wrong conduct for, out of disrespect, covering one’s curries with rice …” … There is one rule. It originates in one way: from body and mind, not from speech. … 

“There\marginnote{278.1} is an offense of wrong conduct for, out of disrespect, eating bean curry or rice that, when one is not sick, one has requested for oneself. Where was it laid down?” At \textsanskrit{Sāvatthī}. “Whom is it about?” The monks from the group of six. “What is it about?” Those monks eating bean curry and rice that they had requested for themselves. There is one rule. There is one addition to the rule. Of the six kinds of originations of offenses, it originates in two ways: from body and mind, not from speech; or from body, speech, and mind. … 

“There\marginnote{279.1} is an offense of wrong conduct for, out of disrespect, looking at the almsbowl of another finding fault …” … There is one rule. It originates in one way: from body and mind, not from speech. … 

“There\marginnote{280.1} is an offense of wrong conduct for, out of disrespect, making a large mouthful …” … There is one rule. It originates in one way: from body and mind, not from speech. … 

“There\marginnote{281.1} is an offense of wrong conduct for, out of disrespect, making an elongated mouthful …” … There is one rule. It originates in one way: from body and mind, not from speech. … 

\scendvagga{The fourth subchapter on almsfood is finished. }

\subsection*{The subchapter on mouthfuls }

“There\marginnote{283.1} is an offense of wrong conduct for, out of disrespect, opening one’s mouth without bringing a mouthful to it …” … There is one rule. It originates in one way: from body and mind, not from speech. … 

“There\marginnote{284.1} is an offense of wrong conduct for, out of disrespect, putting one’s whole hand in one’s mouth while eating …” … There is one rule. It originates in one way: from body and mind, not from speech. … 

“There\marginnote{285.1} is an offense of wrong conduct for, out of disrespect, speaking with food in one’s mouth. Where was it laid down?” At \textsanskrit{Sāvatthī}. “Whom is it about?” The monks from the group of six. “What is it about?” Those monks speaking with food in their mouths. There is one rule. Of the six kinds of originations of offenses, it originates in one way: from body, speech, and mind. … 

“There\marginnote{286.1} is an offense of wrong conduct for, out of disrespect, eating from a lifted ball of food …” … There is one rule. It originates in one way: from body and mind, not from speech. … 

“There\marginnote{287.1} is an offense of wrong conduct for, out of disrespect, eating breaking up mouthfuls …” … There is one rule. It originates in one way: from body and mind, not from speech. … 

“There\marginnote{288.1} is an offense of wrong conduct for, out of disrespect, eating stuffing one’s cheeks …” … There is one rule. It originates in one way: from body and mind, not from speech. … 

“There\marginnote{289.1} is an offense of wrong conduct for, out of disrespect, eating shaking one’s hand …” … There is one rule. It originates in one way: from body and mind, not from speech. … 

“There\marginnote{290.1} is an offense of wrong conduct for, out of disrespect, eating scattering rice …” … There is one rule. It originates in one way: from body and mind, not from speech. … 

“There\marginnote{291.1} is an offense of wrong conduct for, out of disrespect, eating sticking out one’s tongue …” … There is one rule. It originates in one way: from body and mind, not from speech. … 

“There\marginnote{292.1} is an offense of wrong conduct for, out of disrespect, eating making a chomping sound …” … There is one rule. It originates in one way: from body and mind, not from speech. … 

\scendvagga{The fifth subchapter on mouthfuls is finished. }

\subsection*{The subchapter on slurping }

“There\marginnote{294.1} is an offense of wrong conduct for, out of disrespect, eating making a slurping sound. Where was it laid down?” At \textsanskrit{Kosambī}. “Whom is it about?” A number of monks. “What is it about?” Those monks slurping while drinking milk. There is one rule. Of the six kinds of originations of offenses, it originates in one way: from body and mind, not from speech. … 

“There\marginnote{295.1} is an offense of wrong conduct for, out of disrespect, eating licking one’s hands …” … There is one rule. It originates in one way: from body and mind, not from speech. … 

“There\marginnote{296.1} is an offense of wrong conduct for, out of disrespect, eating licking one’s almsbowl …” … There is one rule. It originates in one way: from body and mind, not from speech. … 

“There\marginnote{297.1} is an offense of wrong conduct for, out of disrespect, eating licking one’s lips …” … There is one rule. It originates in one way: from body and mind, not from speech. … 

“There\marginnote{298.1} is an offense of wrong conduct for, out of disrespect, receiving the drinking-water vessel with a hand soiled with food. Where was it laid down?” In the \textsanskrit{Bhaggā} country. “Whom is it about?” A number of monks. “What is it about?” Those monks receiving the drinking-water vessel with a hand soiled with food. There is one rule. Of the six kinds of originations of offenses, it originates in one way: from body and mind, not from speech. … 

“There\marginnote{299.1} is an offense of wrong conduct for, out of disrespect, discarding bowl-washing water containing rice in an inhabited area. Where was it laid down?” In the \textsanskrit{Bhaggā} country. “Whom is it about?” A number of monks. “What is it about?” Those monks discarding bowl-washing water containing rice in an inhabited area. There is one rule. Of the six kinds of originations of offenses, it originates in one way: from body and mind, not from speech. … 

“There\marginnote{300.1} is an offense of wrong conduct for, out of disrespect, giving a teaching to someone holding a sunshade. Where was it laid down?” At \textsanskrit{Sāvatthī}. “Whom is it about?” The monks from the group of six. “What is it about?” Those monks giving a teaching to someone holding a sunshade. There is one rule. There is one addition to the rule. Of the six kinds of originations of offenses, it originates in one way: from speech and mind, not from body. … 

“There\marginnote{301.1} is an offense of wrong conduct for, out of disrespect, giving a teaching to someone holding a staff …” … There is one rule. There is one addition to the rule. Of the six kinds of originations of offenses, it originates in one way: from speech and mind, not from body. … 

“There\marginnote{302.1} is an offense of wrong conduct for, out of disrespect, giving a teaching to someone holding a knife …” … There is one rule. There is one addition to the rule. Of the six kinds of originations of offenses, it originates in one way: from speech and mind, not from body. … 

“There\marginnote{303.1} is an offense of wrong conduct for, out of disrespect, giving a teaching to someone holding a weapon …” … There is one rule. There is one addition to the rule. Of the six kinds of originations of offenses, it originates in one way: from speech and mind, not from body. … 

\scendvagga{The sixth subchapter on slurping is finished. }

\subsection*{The subchapter on shoes }

“There\marginnote{305.1} is an offense of wrong conduct for, out of disrespect, giving a teaching to someone wearing shoes …” … There is one rule. There is one addition to the rule. Of the six kinds of originations of offenses, it originates in one way: from speech and mind, not from body. … 

“There\marginnote{306.1} is an offense of wrong conduct for, out of disrespect, giving a teaching to someone wearing sandals …” … There is one rule. There is one addition to the rule. Of the six kinds of originations of offenses, it originates in one way: from speech and mind, not from body. … 

“There\marginnote{307.1} is an offense of wrong conduct for, out of disrespect, giving a teaching to someone in a vehicle …” … There is one rule. There is one addition to the rule. Of the six kinds of originations of offenses, it originates in one way: from speech and mind, not from body. … 

“There\marginnote{308.1} is an offense of wrong conduct for, out of disrespect, giving a teaching to someone lying down …” … There is one rule. There is one addition to the rule. Of the six kinds of originations of offenses, it originates in one way: from speech and mind, not from body. … 

“There\marginnote{309.1} is an offense of wrong conduct for, out of disrespect, giving a teaching to someone seated clasping their knees …” … There is one rule. There is one addition to the rule. Of the six kinds of originations of offenses, it originates in one way: from speech and mind, not from body. … 

“There\marginnote{310.1} is an offense of wrong conduct for, out of disrespect, giving a teaching to someone with a headdress …” … There is one rule. There is one addition to the rule. Of the six kinds of originations of offenses, it originates in one way: from speech and mind, not from body. … 

“There\marginnote{311.1} is an offense of wrong conduct for, out of disrespect, giving a teaching to someone with a covered head …” … There is one rule. There is one addition to the rule. Of the six kinds of originations of offenses, it originates in one way: from speech and mind, not from body. … 

“There\marginnote{312.1} is an offense of wrong conduct for, out of disrespect, giving a teaching while sitting on the ground to someone sitting on a seat …” … There is one rule. There is one addition to the rule. Of the six kinds of originations of offenses, it originates in one way: from body, speech, and mind. … 

“There\marginnote{313.1} is an offense of wrong conduct for, out of disrespect, giving a teaching while sitting on a low seat to someone sitting on a high seat …” … There is one rule. There is one addition to the rule. Of the six kinds of originations of offenses, it originates in one way: from body, speech, and mind. … 

“There\marginnote{314.1} is an offense of wrong conduct for, out of disrespect, giving a teaching while standing to someone sitting …” … There is one rule. There is one addition to the rule. Of the six kinds of originations of offenses, it originates in one way: from body, speech, and mind. … 

“There\marginnote{315.1} is an offense of wrong conduct for, out of disrespect, giving a teaching to someone walking in front of oneself …” … There is one rule. There is one addition to the rule. Of the six kinds of originations of offenses, it originates in one way: from body, speech, and mind. … 

“There\marginnote{316.1} is an offense of wrong conduct for, out of disrespect, giving a teaching while walking next to the path to someone walking on the path …” … There is one rule. There is one addition to the rule. Of the six kinds of originations of offenses, it originates in one way: from body, speech, and mind. … 

“There\marginnote{317.1} is an offense of wrong conduct for, out of disrespect, defecating or urinating while standing …” … There is one rule. There is one addition to the rule. Of the six kinds of originations of offenses, it originates in one way: from body and mind, not from speech. … 

“There\marginnote{318.1} is an offense of wrong conduct for, out of disrespect, defecating, urinating, or spitting on cultivated plants …” … There is one rule. There is one addition to the rule. Of the six kinds of originations of offenses, it originates in one way: from body and mind, not from speech. … 

“There\marginnote{319.1} is an offense of wrong conduct for, out of disrespect, defecating, urinating, or spitting in water. Where was it laid down?”\footnote{The ellipses points in the Pali are an editorial mistake. } At \textsanskrit{Sāvatthī}. “Whom is it about?” The monks from the group of six. “What is it about?” Those monks defecating, urinating, and spitting in water. There is one rule. There is one addition to the rule. Of the six kinds of originations of offenses, it originates in one way: from body and mind, not from speech. … 

\scendvagga{The seventh subchapter on shoes is finished. }

\scend{The seventy-five rules to be trained in are finished. }

\scuddanaintro{This is the summary: }

\begin{scuddana}%
“Evenly\marginnote{322.1} all around, covered, \\
Well-restrained, lowered eyes; \\
Lifted robe, laughing loudly, noise, \\
And three on swaying. 

Hands\marginnote{323.1} on hips, and covered head, \\
Squatting on the heels, and clasping the knees; \\
Respectfully, and attention on the bowl, \\
The right proportion of bean curry, an even level. 

Respectfully,\marginnote{324.1} and attention on the bowl, \\
In order, the right proportion of bean curry; \\
Making a heap, covering, \\
Requesting, finding fault. 

Not\marginnote{325.1} large, round, mouth, \\
Whole hand, should not speak; \\
Lifted, breaking up, cheek, \\
Shaking, scattering rice. 

And\marginnote{326.1} sticking out the tongue, \\
Chomping, slurping; \\
Hand, and bowl, and lips, \\
With food, and containing rice. 

To\marginnote{327.1} one holding a sunshade, \\
The Buddhas do not give the true Teaching; \\
Nor to one holding a staff, \\
A knife, or a weapon. 

Shoes,\marginnote{328.1} and sandals, \\
And to one in a vehicle, and to one lying down; \\
To one seated clasping their knees, \\
To one with a headdress, and to one with a covered head. 

The\marginnote{329.1} ground, on a low seat, standing, \\
Behind, and next to the path; \\
Not to be done while standing, \\
On cultivated plants, and in water.” 

%
\end{scuddana}

\scuddanaintro{This is the summary of the subchapters: }

\begin{scuddana}%
“Evenly\marginnote{331.1} all around, laughing loudly, \\
Hands on hips, and also almsfood; \\
Mouthfuls, and slurping, \\
And with shoe as the seventh.” 

%
\end{scuddana}

\scendsutta{The questions and answers on the monks’ \textsanskrit{Pātimokkha} rules and their analysis in the Great Analysis are finished. }

%
\chapter*{{\suttatitleacronym Pvr 1.2}{\suttatitletranslation The number of offenses within each offense }{\suttatitleroot Katāpattivāra}}
\addcontentsline{toc}{chapter}{\tocacronym{Pvr 1.2} \toctranslation{The number of offenses within each offense } \tocroot{Katāpattivāra}}
\markboth{The number of offenses within each offense }{Katāpattivāra}
\extramarks{Pvr 1.2}{Pvr 1.2}

\section*{The chapter on offenses entailing expulsion }

When\marginnote{1.1} having sexual intercourse, how many kinds of offenses does one commit? One commits three kinds of offenses: when one has sexual intercourse with an undecomposed corpse, one commits an offense entailing expulsion; when one has sexual intercourse with a mostly decomposed corpse, one commits a serious offense; when one inserts one’s penis into a wide open mouth without touching it, one commits an offense of wrong conduct. 

When\marginnote{2.1} stealing, how many kinds of offenses does one commit? One commits three kinds of offenses: when, intending to steal, one steals something worth five \textit{\textsanskrit{māsaka}} coins or more, one commits an offense entailing expulsion; when, intending to steal, one steals something worth more than one \textit{\textsanskrit{māsaka}} coin but less than five, one commits a serious offense; when, intending to steal, one steals something worth one \textit{\textsanskrit{māsaka}} coin or less, one commits an offense of wrong conduct. 

When\marginnote{3.1} intentionally killing a human being, how many kinds of offenses does one commit? One commits three kinds of offenses: when one digs a pit for a human being, thinking, “Falling into it, they will die,” one commits an offense of wrong conduct; when they experience pain after falling in, one commits a serious offense; when they die, one commits an offense entailing expulsion. 

When\marginnote{4.1} claiming a non-existent superhuman quality, how many kinds of offenses does one commit? One commits three kinds of offenses: when, having bad desires, overcome by desire, one claims a non-existent superhuman quality, one commits an offense entailing expulsion; when one says, “The monk who stays in your dwelling is a perfected one,” and the listener understands, one commits a serious offense; when the listener does not understand, one commits an offense of wrong conduct. 

\scend{The four offenses entailing expulsion are finished. }

\section*{2. The chapter on offenses entailing suspension }

(…)\marginnote{6.1} When emitting semen by means of effort, one commits three kinds of offenses: when one intends and makes an effort, and semen is emitted, one commits an offense entailing suspension; when one intends and makes an effort, but semen is not emitted, one commits a serious offense; for the effort there is an offense of wrong conduct. 

When\marginnote{7.1} making physical contact with a woman, one commits three kinds of offenses: when one makes physical contact, body with body, one commits an offense entailing suspension; when one makes physical contact with something connected to her body, one commits a serious offense; when one, with something connected to one’s own body, makes physical contact with something connected to her body, one commits an offense of wrong conduct. 

When\marginnote{8.1} speaking indecently to a woman, one commits three kinds of offenses: when, referring to the anus or the vagina, one praises or disparages, one commits an offense entailing suspension; when, referring to any part below the collar bone but above the knees, apart from the anus or the vagina, one praises or disparages, one commits a serious offense; when, referring to anything connected to the body, one praises or disparages, one commits an offense of wrong conduct. 

When\marginnote{9.1} encouraging the satisfaction of one’s own desires, one commits three kinds of offenses: when one encourages a woman to satisfy one’s own desires, one commits an offense entailing suspension; when one encourages a \textit{\textsanskrit{paṇḍaka}} to satisfy one’s own desires, one commits a serious offense; when one encourages an animal to satisfy one’s own desires, one commits an offense of wrong conduct. 

When\marginnote{10.1} acting as a matchmaker, one commits three kinds of offenses: when one accepts the mission, finds out the response, and reports back, one commits an offense entailing suspension; when one accepts the mission, and finds out the response, but does not report back, one commits a serious offense; when one accepts the mission, but neither finds out the response, nor reports back, one commits an offense of wrong conduct. 

When\marginnote{11.1} having a hut built by means of begging, one commits three kinds of offenses: when one is having it built, then for the effort there is an offense of wrong conduct; when there is one piece left to complete the hut, one commits a serious offense; when the last piece is finished, one commits an offense entailing suspension. 

When\marginnote{12.1} having a large dwelling built, one commits three kinds of offenses: when one is having it built, then for the effort there is an offense of wrong conduct; when there is one piece left to complete the dwelling, one commits a serious offense; when the last piece is finished, one commits an offense entailing suspension. 

When\marginnote{13.1} groundlessly charging a monk with an offense entailing expulsion, one commits three kinds of offenses: when, without having gotten his permission, one speaks with the aim of making him leave the monastic life, one commits one offense entailing suspension and one offense of wrong conduct; when, having gotten his permission, one speaks with the aim of abusing him, one commits an offense for abusive speech. 

When\marginnote{14.1} charging a monk with an offense entailing expulsion, using an unrelated legal issue as a pretext, one commits three kinds of offenses: when, without having gotten his permission, one speaks with the aim of making him leave the monastic life, one commits one offense entailing suspension and one offense of wrong conduct; when, having gotten his permission, one speaks with the aim of abusing him, one commits an offense for abusive speech. 

When\marginnote{15.1} not stopping when pressed for the third time, a monk who is pursuing schism in the Sangha commits three kinds of offenses: after the motion, he commits an offense of wrong conduct; after each of the first two announcements, he commits a serious offense; when the last announcement is finished, he commits an offense entailing suspension. 

When\marginnote{16.1} not stopping when pressed for the third time, monks who side with a monk who is pursuing schism commit three kinds of offenses: after the motion, they commit an offense of wrong conduct; after each of the first two announcements, they commit a serious offense; when the last announcement is finished, they commit an offense entailing suspension. 

When\marginnote{17.1} not stopping when pressed for the third time, a monk who is difficult to correct commits three kinds of offenses: after the motion, he commits an offense of wrong conduct; after each of the first two announcements, he commits a serious offense; when the last announcement is finished, he commits an offense entailing suspension. 

When\marginnote{18.1} not stopping when pressed for the third time, a monk who is a corrupter of families commits three kinds of offenses: after the motion, he commits an offense of wrong conduct; after each of the first two announcements, he commits a serious offense; when the last announcement is finished, he commits an offense entailing suspension. 

\scend{The thirteen rules entailing suspension are finished. }

\section*{4. The chapter on relinquishment }

\subsection*{The subchapter on the robe season }

When\marginnote{20.1} keeping an extra robe more than ten days, one commits one kind of offense: an offense entailing relinquishment and confession. 

When\marginnote{21.1} staying apart from one’s three robes for one day, one commits one kind of offense: an offense entailing relinquishment and confession. 

When\marginnote{22.1} receiving out-of-season robe-cloth and then keeping it for more than a month, one commits one kind of offense: an offense entailing relinquishment and confession. 

When\marginnote{23.1} having an unrelated nun wash a used robe, one commits two kinds of offenses: when one is having it washed, then for the effort there is an offense of wrong conduct; when one has had it washed, one commits an offense entailing relinquishment and confession. 

When\marginnote{24.1} receiving a robe directly from an unrelated nun, one commits two kinds of offenses: when one is in the process of taking it, then for the effort there is an offense of wrong conduct; when one has taken it, one commits an offense entailing relinquishment and confession. 

When\marginnote{25.1} asking an unrelated male or female householder for a robe, one commits two kinds of offenses: when one is in the process of asking, then for the effort there is an offense of wrong conduct; when one has asked, one commits an offense entailing relinquishment and confession. 

When\marginnote{26.1} asking an unrelated male or female householder for too many robes, one commits two kinds of offenses: when one is in the process of asking, then for the effort there is an offense of wrong conduct; when one has asked, one commits an offense entailing relinquishment and confession. 

When,\marginnote{27.1} without first being invited, one goes to an unrelated householder and specifies the kind of robe-cloth one wants, one commits two kinds of offenses: when one is in the process of specifying it, then for the effort there is an offense of wrong conduct; when one has specified it, one commits an offense entailing relinquishment and confession. 

When,\marginnote{28.1} without first being invited, one goes to unrelated householders and specifies the kind of robe-cloth one wants, one commits two kinds of offenses: when one is in the process of specifying it, then for the effort there is an offense of wrong conduct; when one has specified it, one commits an offense entailing relinquishment and confession. 

When\marginnote{29.1} getting robe-cloth after prompting more than three times and standing more than six times, one commits two kinds of offenses: when one is in the process of getting it, then for the effort there is an offense of wrong conduct; when one has gotten it, one commits an offense entailing relinquishment and confession. 

\scendvagga{The first subchapter on the robe season is finished. }

\subsection*{The subchapter on silk }

When\marginnote{31.1} having a blanket made that contains silk, one commits two kinds of offenses: when one is having it made, then for the effort there is an offense of wrong conduct; when one has had it made, one commits an offense entailing relinquishment and confession. 

When\marginnote{32.1} having a blanket made entirely of black wool, one commits two kinds of offenses: when one is having it made, then for the effort there is an offense of wrong conduct; when one has had it made, one commits an offense entailing relinquishment and confession. 

When\marginnote{33.1} having a new blanket made without using one measure of white wool and one measure of brown, one commits two kinds of offenses: when one is having it made, then for the effort there is an offense of wrong conduct; when one has had it made, one commits an offense entailing relinquishment and confession. 

When\marginnote{34.1} having a blanket made every year, one commits two kinds of offenses: when one is having it made, then for the effort there is an offense of wrong conduct; when one has had it made, one commits an offense entailing relinquishment and confession. 

When\marginnote{35.1} having a new sitting blanket made without incorporating a piece of one standard handspan from the border of an old blanket, one commits two kinds of offenses: when one is having it made, then for the effort there is an offense of wrong conduct; when one has had it made, one commits an offense entailing relinquishment and confession. 

When\marginnote{36.1} receiving wool and then taking it more than 40 kilometers, one commits two kinds of offenses: when one goes beyond 40 kilometers with the first foot, one commits an offense of wrong conduct; when one goes beyond with the second foot, one commits an offense entailing relinquishment and confession. 

When\marginnote{37.1} having an unrelated nun wash wool, one commits two kinds of offenses: when one is having it washed, then for the effort there is an offense of wrong conduct; when one has had it washed, one commits an offense entailing relinquishment and confession. 

When\marginnote{38.1} receiving money, one commits two kinds of offenses: when one is in the process of taking it, then for the effort there is an offense of wrong conduct; when one has taken it, one commits an offense entailing relinquishment and confession. 

When\marginnote{39.1} engaging in various kinds of trade involving money, one commits two kinds of offenses: when one is in the process of trading, then for the effort there is an offense of wrong conduct; when one has traded, one commits an offense entailing relinquishment and confession. 

When\marginnote{40.1} engaging in various kinds of barter, one commits two kinds of offenses: when one is in the process of bartering, then for the effort there is an offense of wrong conduct; when one has bartered, one commits an offense entailing relinquishment and confession. 

\scendvagga{The second subchapter on silk is finished. }

\subsection*{The subchapter on almsbowls }

When\marginnote{42.1} keeping an extra almsbowl for more than ten days, one commits one kind of offense: an offense entailing relinquishment and confession. 

When\marginnote{43.1} exchanging an almsbowl with fewer than five mends for a new almsbowl, one commits two kinds of offenses: when one is in the process of exchanging it, then for the effort there is an offense of wrong conduct; when one has exchanged it, one commits an offense entailing relinquishment and confession. 

When\marginnote{44.1} receiving tonics and then keeping them for more than seven days, one commits one kind of offense: an offense entailing relinquishment and confession. 

When\marginnote{45.1} looking for a rainy-season robe when there is more than a month left of summer, one commits two kinds of offenses: when one is in the process of looking for it, then for the effort there is an offense of wrong conduct; when one has looked for it, one commits an offense entailing relinquishment and confession.\footnote{That is, one has obtained it. This interpretation is required by the fact that it is an offense entailing relinquishment. } 

When\marginnote{46.1} giving a robe to a monk and then taking it back in anger, one commits two kinds of offenses: when one is in the process of taking it back, then for the effort there is an offense of wrong conduct; when one has taken it back, one commits an offense entailing relinquishment and confession. 

When\marginnote{47.1} asking for thread and then having weavers weave robe-cloth, one commits two kinds of offenses: when one is having it woven, then for the effort there is an offense of wrong conduct; when one has had it woven, one commits an offense entailing relinquishment and confession. 

When,\marginnote{48.1} without first being invited, one goes to an unrelated householder’s weavers and specifies the kind of robe-cloth one wants, one commits two kinds of offenses: when one is in the process of specifying it, then for the effort there is an offense of wrong conduct; when one has specified it, one commits an offense entailing relinquishment and confession. 

When\marginnote{49.1} receiving a haste-cloth and then keeping it beyond the robe season, one commits one kind of offense: an offense entailing relinquishment and confession. 

When\marginnote{50.1} storing one of one’s three robes in an inhabited area and then staying apart from it for more than six days, one commits one kind of offense: an offense entailing relinquishment and confession. 

When\marginnote{51.1} diverting to oneself material support that one knows was intended for the Sangha, one commits two kinds of offenses: when one is in the process of diverting it, then for the effort there is an offense of wrong conduct; when one has diverted it, one commits an offense entailing relinquishment and confession. 

\scendvagga{The third subchapter on almsbowls is finished. }

\scend{The thirty rules on relinquishment and confession are finished. }

\section*{5. The chapter on offenses entailing confession }

\subsection*{The subchapter on lying }

When\marginnote{53.1} lying in full awareness, how many kinds of offenses does one commit? When lying in full awareness, one commits five kinds of offenses: when, having bad desires, overcome by desire, one claims a non-existent superhuman quality, one commits an offense entailing expulsion; when one groundlessly charges a monk with an offense entailing expulsion, one commits an offense entailing suspension; when one says, “The monk who stays in your dwelling is a perfected one,” and the listener understands, one commits a serious offense; when the listener does not understand, one commits an offense of wrong conduct; when one lies in full awareness, one commits an offense entailing confession. 

When\marginnote{54.1} speaking abusively, one commits two kinds of offenses: when one speaks abusively to one who is fully ordained, one commits an offense entailing confession; when one speaks abusively to one who is not fully ordained, one commits an offense of wrong conduct. 

When\marginnote{55.1} engaging in malicious talebearing, one commits two kinds of offenses: when one engages in malicious talebearing to one who is fully ordained, one commits an offense entailing confession; when one engages in malicious talebearing to one who is not fully ordained, one commits an offense of wrong conduct. 

When\marginnote{56.1} instructing a person who is not fully ordained to memorize the Teaching, one commits two kinds of offenses: when one is in the process of instructing, then for the effort there is an offense of wrong conduct; for every line, one commits an offense entailing confession. 

When\marginnote{57.1} lying down more than two or three nights in the same sleeping place as a person who is not fully ordained, one commits two kinds of offenses: when one is in the process of lying down, then for the effort there is an offense of wrong conduct; when one is lying down, one commits an offense entailing confession. 

When\marginnote{58.1} lying down in the same sleeping place as a woman, one commits two kinds of offenses: when one is in the process of lying down, then for the effort there is an offense of wrong conduct; when one is lying down, one commits an offense entailing confession. 

When\marginnote{59.1} giving a teaching of more than five or six sentences to a woman, one commits two kinds of offenses: when one is in the process of teaching, then for the effort there is an offense of wrong conduct; for every line, one commits an offense entailing confession. 

When\marginnote{60.1} truthfully telling a person who is not fully ordained of a superhuman quality, one commits two kinds of offenses: when one is in the process of telling, then for the effort there is an offense of wrong conduct; when one has finished telling, one commits an offense entailing confession. 

When\marginnote{61.1} telling a person who is not fully ordained about a monk’s grave offense, one commits two kinds of offenses: when one is in the process of telling, then for the effort there is an offense of wrong conduct; when one has finished telling, one commits an offense entailing confession. 

When\marginnote{62.1} digging the earth, one commits two kinds of offenses: when one is in the process of digging, then for the effort there is an offense of wrong conduct; for every strike, one commits an offense entailing confession. 

\scendvagga{The first subchapter on lying is finished. }

\subsection*{The subchapter on plants }

When\marginnote{64.1} destroying a plant, one commits two kinds of offenses: when one is having it cut down, then for the effort there is an offense of wrong conduct; for every strike, one commits an offense entailing confession. 

When\marginnote{65.1} speaking evasively, one commits two kinds of offenses: when one speaks evasively without having been charged with evasive speech, one commits an offense of wrong conduct; when one speaks evasively after having been charged with evasive speech, one commits an offense entailing confession. 

When\marginnote{66.1} complaining about a monk, one commits two kinds of offenses: when one is in the process of complaining, then for the effort there is an offense of wrong conduct; when one has complained, one commits an offense entailing confession. 

When\marginnote{67.1} taking a bed, a bench, a mattress, or a stool belonging to the Sangha and putting it outside, and then departing without putting it away or informing anyone, one commits two kinds of offenses: when one goes beyond the distance of a stone’s throw with the first foot, one commits an offense of wrong conduct; when one goes beyond with the second foot, one commits an offense entailing confession. 

When\marginnote{68.1} putting out bedding in a dwelling belonging to the Sangha, and then departing without putting it away or informing anyone, one commits two kinds of offenses: when one crosses the boundary with the first foot, one commits an offense of wrong conduct; when one crosses with the second foot, one commits an offense entailing confession. 

When\marginnote{69.1} arranging one’s sleeping place, in a dwelling belonging to the Sangha, in a way that encroaches on a monk that one knows arrived there before oneself, one commits two kinds of offenses: when one is in the process of lying down, then for the effort there is an offense of wrong conduct; when one is lying down, one commits an offense entailing confession. 

When\marginnote{70.1} angrily throwing a monk out of a dwelling belonging to the Sangha, one commits two kinds of offenses: when one is in the process of throwing him out, then for the effort there is an offense of wrong conduct; when one has thrown him out, one commits an offense entailing confession. 

When\marginnote{71.1} sitting down on a bed or a bench with detachable legs on an upper story in a dwelling belonging to the Sangha, one commits two kinds of offenses: when one is in the process of sitting down, then for the effort there is an offense of wrong conduct; when one is seated, one commits an offense entailing confession. 

When\marginnote{72.1} applying more than two or three courses, one commits two kinds of offenses: when one is in the process of applying them, then for the effort there is an offense of wrong conduct; when one has applied them, one commits an offense entailing confession. 

When\marginnote{73.1} pouring water that one knows contains living beings onto grass or clay, one commits two kinds of offenses: when one is in the process pouring, then for the effort there is an offense of wrong conduct; when one has finished pouring, one commits an offense entailing confession. 

\scendvagga{The second subchapter on plants is finished. }

\subsection*{The subchapter on the instruction }

When\marginnote{75.1} instructing the nuns without being appointed, one commits two kinds of offenses: when one is in the process of instructing, then for the effort there is an offense of wrong conduct; when one has finished instructing, one commits an offense entailing confession. 

When\marginnote{76.1} instructing the nuns after sunset, one commits two kinds of offenses: when one is in the process of instructing, then for the effort there is an offense of wrong conduct; when one has finished instructing, one commits an offense entailing confession. 

When\marginnote{77.1} one goes to the nuns’ dwelling place and instructs the nuns, one commits two kinds of offenses: when one is in the process of instructing, then for the effort there is an offense of wrong conduct; when one has finished instructing, one commits an offense entailing confession. 

When\marginnote{78.1} saying that the monks are instructing the nuns for the sake of worldly gain, one commits two kinds of offenses: when one is in the process of saying it, then for the effort there is an offense of wrong conduct; when one has said it, one commits an offense entailing confession. 

When\marginnote{79.1} giving robe-cloth to an unrelated nun, one commits two kinds of offenses: when one is in the process of giving it, then for the effort there is an offense of wrong conduct; when one has given it, one commits an offense entailing confession. 

When\marginnote{80.1} sewing a robe for an unrelated nun, one commits two kinds of offenses: when one is in the process of sewing it, then for the effort there is an offense of wrong conduct; for every stitch, one commits an offense entailing confession. 

When\marginnote{81.1} traveling by arrangement with a nun, one commits two kinds of offenses: when one is in the process of traveling, then for the effort there is an offense of wrong conduct; when one has traveled, one commits an offense entailing confession. 

When\marginnote{82.1} boarding a boat by arrangement with a nun, one commits two kinds of offenses: when one is in the process of boarding, then for the effort there is an offense of wrong conduct; when one has boarded, one commits an offense entailing confession. 

When\marginnote{83.1} eating almsfood knowing that a nun had it prepared, one commits two kinds of offenses: when one receives with the intention to eat, one commits an offense of wrong conduct; for every mouthful swallowed, one commits an offense entailing confession. 

When\marginnote{84.1} sitting down in private alone with a nun, one commits two kinds of offenses: when one is in the process of sitting down, then for the effort there is an offense of wrong conduct; when one is seated, one commits an offense entailing confession. 

\scendvagga{The third subchapter on the instruction is finished. }

\subsection*{The subchapter on eating }

When\marginnote{86.1} eating alms too often at a public guesthouse, one commits two kinds of offenses: when one receives with the intention to eat, one commits an offense of wrong conduct; for every mouthful swallowed, one commits an offense entailing confession. 

When\marginnote{87.1} eating in a group, one commits two kinds of offenses: when one receives with the intention to eat, one commits an offense of wrong conduct; for every mouthful swallowed, one commits an offense entailing confession. 

When\marginnote{88.1} eating one meal before another, one commits two kinds of offenses: when one receives with the intention to eat, one commits an offense of wrong conduct; for every mouthful swallowed, one commits an offense entailing confession. 

When\marginnote{89.1} accepting more than two or three bowlfuls of cookies, one commits two kinds of offenses: when one is in the process of taking, then for the effort there is an offense of wrong conduct; when one has finished taking, one commits an offense entailing confession. 

When\marginnote{90.1} one has finished one’s meal and refused an invitation to eat more, and then eats fresh or cooked food that is not left over, one commits two kinds of offenses: when one receives with the intention to eat, one commits an offense of wrong conduct; for every mouthful swallowed, one commits an offense entailing confession. 

When\marginnote{91.1} inviting a monk who has finished his meal and refused an invitation to eat more to eat fresh or cooked food that is not left over, one commits two kinds of offenses: when, because of what one says, the other receives with the intention to eat, one commits an offense of wrong conduct; when the meal is finished, one commits an offense entailing confession. 

When\marginnote{92.1} eating fresh or cooked food at the wrong time, one commits two kinds of offenses: when one receives with the intention to eat, one commits an offense of wrong conduct; for every mouthful swallowed, one commits an offense entailing confession. 

When\marginnote{93.1} storing and then eating fresh or cooked food, one commits two kinds of offenses: when one receives with the intention to eat, one commits an offense of wrong conduct; for every mouthful swallowed, one commits an offense entailing confession. 

When\marginnote{94.1} eating fine foods that one has requested for oneself, one commits two kinds of offenses: when one receives with the intention to eat, one commits an offense of wrong conduct; for every mouthful swallowed, one commits an offense entailing confession. 

When\marginnote{95.1} eating food that has not been given, one commits two kinds of offenses: when one receives with the intention to eat, one commits an offense of wrong conduct; for every mouthful swallowed, one commits an offense entailing confession. 

\scendvagga{The fourth subchapter on eating is finished. }

\subsection*{The subchapter on naked ascetics }

When\marginnote{97.1} personally giving fresh or cooked food to a naked ascetic, to a male wanderer, or to a female wanderer, one commits two kinds of offenses: when one is in the process of giving, then for the effort there is an offense of wrong conduct; when one has given, one commits an offense entailing confession. 

When\marginnote{98.1} one says to a monk, “Come, let’s go to the village or town for alms,” and then, whether one has had food given to him or not, sends him away, one commits two kinds of offenses: when one is in the process of sending him away, then for the effort there is an offense of wrong conduct; when one has sent him away, one commits an offense entailing confession. 

When\marginnote{99.1} sitting down intruding on a lustful couple, one commits two kinds of offenses: when one is in the process of sitting down, then for the effort there is an offense of wrong conduct; when one is seated, one commits an offense entailing confession. 

When\marginnote{100.1} sitting down in private on a concealed seat with a woman, one commits two kinds of offenses: when one is in the process of sitting down, then for the effort there is an offense of wrong conduct; when one is seated, one commits an offense entailing confession. 

When\marginnote{101.1} sitting down in private alone with a woman, one commits two kinds of offenses: when one is in the process of sitting down, then for the effort there is an offense of wrong conduct; when one is seated, one commits an offense entailing confession. 

When\marginnote{102.1} one is invited to a meal, and then visits families beforehand or afterwards, one commits two kinds of offenses: when one crosses the threshold with the first foot, one commits an offense of wrong conduct; when one crosses with the second foot, one commits an offense entailing confession. 

When\marginnote{103.1} asking for too many tonics, one commits two kinds of offenses: when one is in the process of asking, then for the effort there is an offense of wrong conduct; when one has asked, one commits an offense entailing confession. 

When\marginnote{104.1} going to see an army, one commits two kinds of offenses: when one is in the process of going, one commits an offense of wrong conduct; wherever one stands to see it, one commits an offense entailing confession. 

When\marginnote{105.1} staying with the army for more than three nights, one commits two kinds of offenses: when one is staying, then for the effort there is an offense of wrong conduct; when one has stayed, one commits an offense entailing confession. 

When\marginnote{106.1} going to a battle, one commits two kinds of offenses: when one is in the process of going, one commits an offense of wrong conduct; wherever one stands to see it, one commits an offense entailing confession. 

\scendvagga{The fifth subchapter on naked ascetics is finished. }

\subsection*{The subchapter on drinking alcohol }

When\marginnote{108.1} drinking an alcoholic drink, one commits two kinds of offenses: when one receives with the intention to drink, one commits an offense of wrong conduct; for every mouthful swallowed, one commits an offense entailing confession. 

When\marginnote{109.1} one makes a monk laugh by tickling, one commits two kinds of offenses: when one is in the process of making him laugh, then for the effort there is an offense of wrong conduct; when one has made him laugh, one commits an offense entailing confession. 

When\marginnote{110.1} playing in water, one commits two kinds of offenses: when one is playing in water less than ankle deep, one commits an offense of wrong conduct. when one is playing in water more than ankle deep, one commits an offense entailing confession. 

When\marginnote{111.1} being disrespectful, one commits two kinds of offenses: when one is in the process of doing it, then for the effort there is an offense of wrong conduct; when one has done it, one commits an offense entailing confession. 

When\marginnote{112.1} scaring a monk, one commits two kinds of offenses: when one is in the process of scaring him, then for the effort there is an offense of wrong conduct; when one has scared him, one commits an offense entailing confession. 

When\marginnote{113.1} one lights a fire and warms oneself, one commits two kinds of offenses: when one is in the process of lighting it, then for the effort there is an offense of wrong conduct; when one has lit it, one commits an offense entailing confession. 

When\marginnote{114.1} bathing at intervals of less than a half-month, one commits two kinds of offenses: when one is in the process of bathing, then for the effort there is an offense of wrong conduct; when the bath is finished, one commits an offense entailing confession. 

When\marginnote{115.1} using a new robe without first applying one of the three kinds of stains, one commits two kinds of offenses: when one is using it, then for the effort there is an offense of wrong conduct; when one has used it, one commits an offense entailing confession. 

When\marginnote{116.1} assigning the ownership of a robe to a monk, a nun, a trainee nun, a novice monk, or a novice nun, and then using it without the other first relinquishing it, one commits two kinds of offenses:\footnote{For an explanation of the idea of \textit{\textsanskrit{vikappanā}}, see Appendix of Technical Terms. } when one is using it, then for the effort there is an offense of wrong conduct; when one has used it, one commits an offense entailing confession. 

When\marginnote{117.1} hiding a monk’s bowl, robe, sitting mat, needle case, or belt, one commits two kinds of offenses: when one is in the process of hiding it, then for the effort there is an offense of wrong conduct; when one has hid it, one commits an offense entailing confession. 

\scendvagga{The sixth subchapter of alcoholic drinks is finished. }

\subsection*{The subchapter on containing living beings }

When\marginnote{119.1} intentionally killing a living being, how many kinds of offenses does one commit? When intentionally killing a living being, one commits four kinds of offenses: when one digs a non-specific pit, thinking, “Whatever falls into it will die,” one commits an offense of wrong conduct; when a person falls into it and dies, one commits an offense entailing expulsion; when a spirit, ghost, or animal in human form falls into it and dies, one commits a serious offense; when an animal falls into it and dies, one commits an offense entailing confession. 

When\marginnote{120.1} using water that one knows contains living beings, one commits two kinds of offenses: when one is using it, then for the effort there is an offense of wrong conduct; when one has used it, one commits an offense entailing confession. 

When\marginnote{121.1} reopening a legal issue that one knows has been legitimately settled, one commits two kinds of offenses: when one is in the process of reopening it, then for the effort there is an offense of wrong conduct; when one has reopened it, one commits an offense entailing confession. 

When\marginnote{122.1} knowingly concealing a monk’s grave offense, one commits one kind of offense: an offense entailing confession. 

When\marginnote{123.1} giving the full ordination to a person one knows is less than twenty years old, one commits two kinds of offenses: when one is in the process of giving the full ordination, then for the effort there is an offense of wrong conduct; when one has given the full ordination, one commits an offense entailing confession. 

When\marginnote{124.1} knowingly traveling by arrangement with a group of thieves, one commits two kinds of offenses: when one is in the process of traveling, then for the effort there is an offense of wrong conduct; when one has traveled, one commits an offense entailing confession. 

When\marginnote{125.1} traveling by arrangement with a woman, one commits two kinds of offenses: when one is in the process of traveling, then for the effort there is an offense of wrong conduct; when one has traveled, one commits an offense entailing confession. 

When\marginnote{126.1} not giving up a bad view when pressed for the third time, one commits two kinds of offenses: after the motion, one commits an offense of wrong conduct; when the last announcement is finished, one commits an offense entailing confession. 

When\marginnote{127.1} living with a monk who one knows is saying such things, who has not made amends according to the rule, and who has not given up that view, one commits two kinds of offenses:\footnote{“Such things” and “that view” refer to the idea that sexual intercourse is not an obstacle to spiritual progress, see \href{https://suttacentral.net/pli-tv-bu-vb-pc68/en/brahmali\#1.49.1}{Bu Pc 68:1.49.1}. } when one is living with him, then for the effort there is an offense of wrong conduct; when one has lived with him, one commits an offense entailing confession. 

When\marginnote{128.1} befriending a novice monastic who one knows has been expelled in this way, one commits two kinds of offenses:\footnote{For the meaning of “in this way”, see \href{https://suttacentral.net/pli-tv-bu-vb-pc70/en/brahmali\#1.46.1}{Bu Pc 70:1.46.1}. } when one befriends him, then for the effort there is an offense of wrong conduct; when one has befriended him, one commits an offense entailing confession. 

\scendvagga{The seventh subchapter on containing living beings is finished. }

\subsection*{The subchapter on legitimately }

When\marginnote{130.1} legitimately corrected by the monks, saying, “I won’t practice this training rule until I’ve questioned a monk who’s an expert on the Monastic Law”, one commits two kinds of offenses: when one is in the process of saying it, then for the effort there is an offense of wrong conduct; when one has said it, one commits an offense entailing confession. 

When\marginnote{131.1} disparaging the Monastic Law, one commits two kinds of offenses: when one is in the process of disparaging it, then for the effort there is an offense of wrong conduct; when one has disparaged it, one commits an offense entailing confession. 

When\marginnote{132.1} deceiving, one commits two kinds of offenses: when one deceives without having been charged with deception, one commits an offense of wrong conduct; when one deceives after having been charged with deception, one commits an offense entailing confession. 

When\marginnote{133.1} hitting a monk in anger, one commits two kinds of offenses: when one is hitting, then for the effort there is an offense of wrong conduct; when one has hit, one commits an offense entailing confession. 

When\marginnote{134.1} raising a hand in anger against a monk, one commits two kinds of offenses: when one raises it, then for the effort there is an offense of wrong conduct; when one has raised it, one commits an offense entailing confession. 

When\marginnote{135.1} groundlessly charging a monk with an offense entailing suspension, one commits two kinds of offenses: when one is in the process of making the charge, then for the effort there is an offense of wrong conduct; when one has made the charge, one commits an offense entailing confession. 

When\marginnote{136.1} intentionally making a monk anxious, one commits two kinds of offenses: when one is in the process of doing it, then for the effort there is an offense of wrong conduct; when one has done it, one commits an offense entailing confession. 

When\marginnote{137.1} eavesdropping on monks who are arguing and disputing, one commits two kinds of offenses: when going with the intention to listen, one commits an offense of wrong conduct; wherever one stands to listen, one commits an offense entailing confession. 

When\marginnote{138.1} one gives one’s consent to legitimate legal procedures and then criticizes them afterwards, one commits two kinds of offenses: when one is criticizing, then for the effort there is an offense of wrong conduct; when one has criticized, one commits an offense entailing confession. 

When,\marginnote{139.1} without first giving one’s consent, one gets up from one’s seat and leaves while the Sangha is in the middle of a discussion, one commits two kinds of offenses: when one is in the process of going beyond arm’s reach of the gathering, one commits an offense of wrong conduct; when one has gone beyond, one commits an offense entailing confession. 

When\marginnote{140.1} one gives out a robe as part of a unanimous Sangha and then criticizes it afterwards, one commits two kinds of offenses: when one is criticizing it, then for the effort there is an offense of wrong conduct; when one has criticized it, one commits an offense entailing confession. 

When\marginnote{141.1} diverting to an individual material support that one knows was intended for the Sangha, one commits two kinds of offenses: when one is in the process of diverting it, then for the effort there is an offense of wrong conduct; when one has diverted it, one commits an offense entailing confession. 

\scendvagga{The eighth subchapter on legitimately is finished. }

\subsection*{The subchapter on kings }

When\marginnote{143.1} entering the royal compound without first being announced, one commits two kinds of offenses: when one crosses the threshold with the first foot, one commits an offense of wrong conduct; when one crosses with the second foot, one commits an offense entailing confession. 

When\marginnote{144.1} picking up something precious, one commits two kinds of offenses: when one is in the process of taking hold of it, then for the effort there is an offense of wrong conduct; when one has taken hold of it, one commits an offense entailing confession. 

When\marginnote{145.1} entering an inhabited area at the wrong time without informing an available monk, one commits two kinds of offenses: when one crosses the boundary with the first foot, one commits an offense of wrong conduct; when one crosses with the second foot, one commits an offense entailing confession. 

When\marginnote{146.1} having a needle case made from bone, ivory, or horn, one commits two kinds of offenses: when one is having it made, then for the effort there is an offense of wrong conduct; when one has had it made, one commits an offense entailing confession. 

When\marginnote{147.1} having a bed or a bench made that exceeds the right height, one commits two kinds of offenses: when one is having it made, then for the effort there is an offense of wrong conduct; when one has had it made, one commits an offense entailing confession. 

When\marginnote{148.1} having a bed or a bench made upholstered with cotton down, one commits two kinds of offenses: when one is having it made, then for the effort there is an offense of wrong conduct; when one has had it made, one commits an offense entailing confession. 

When\marginnote{149.1} having a sitting mat made that exceeds the right size, one commits two kinds of offenses: when one is having it made, then for the effort there is an offense of wrong conduct; when one has had it made, one commits an offense entailing confession. 

When\marginnote{150.1} having an itch-covering cloth made that exceeds the right size, one commits two kinds of offenses: when one is having it made, then for the effort there is an offense of wrong conduct; when one has had it made, one commits an offense entailing confession. 

When\marginnote{151.1} having a rainy-season robe made that exceeds the right size, one commits two kinds of offenses: when one is having it made, then for the effort there is an offense of wrong conduct; when one has had it made, one commits an offense entailing confession. 

When\marginnote{152.1} having a robe made that is the standard robe size, how many kinds of offenses does one commit? When having a robe made that is the standard robe size, one commits two kinds of offenses: when one is having it made, then for the effort there is an offense of wrong conduct; when one has had it made, one commits an offense entailing confession. 

\scendvagga{The ninth subchapter on kings is finished. }

\scend{The section on minor rules is finished. }

\section*{6. The chapter on offenses entailing acknowledgment }

When\marginnote{154.1} eating fresh or cooked food that was received directly from an unrelated nun who had entered an inhabited area, how many kinds of offenses does one commit? When eating fresh or cooked food that was received directly from an unrelated nun who had entered an inhabited area, one commits two kinds of offenses: when one receives with the intention to eat, one commits an offense of wrong conduct; for every mouthful swallowed, one commits an offense entailing acknowledgment. 

When\marginnote{155.1} eating without having restrained a nun who is giving directions, one commits two kinds of offenses: when one receives with the intention to eat, one commits an offense of wrong conduct; for every mouthful swallowed, one commits an offense entailing acknowledgment. 

When\marginnote{156.1} eating fresh or cooked food after personally receiving it from families designated as “in training”, one commits two kinds of offenses: when one receives with the intention to eat, one commits an offense of wrong conduct; for every mouthful swallowed, one commits an offense entailing acknowledgment. 

When\marginnote{157.1} eating fresh or cooked food after personally receiving it inside a wilderness monastery without first making an announcement, how many kinds of offenses does one commit? When eating fresh or cooked food after personally receiving it inside a wilderness monastery without first making an announcement, one commits two kinds of offenses: when one receives with the intention to eat, one commits an offense of wrong conduct; for every mouthful swallowed, one commits an offense entailing acknowledgment. 

\scend{The four offenses entailing acknowledgment are finished. }

\section*{7. The chapter on training }

\subsection*{The subchapter on evenly all around }

When,\marginnote{159.1} out of disrespect, one wears one’s sarong hanging down in front or behind, how many kinds of offenses does one commit? When, out of disrespect, one wears one’s sarong hanging down in front or behind, one commits one kind of offense: an offense of wrong conduct. 

When,\marginnote{160.1} out of disrespect, one wears one’s upper robe hanging down in front or behind, one commits one kind of offense: an offense of wrong conduct. 

When,\marginnote{161.1} out of disrespect, one walks in an inhabited area with one’s body uncovered, one commits one kind of offense: an offense of wrong conduct. 

When,\marginnote{162.1} out of disrespect, one sits in an inhabited area with one’s body uncovered, one commits one kind of offense: an offense of wrong conduct. 

When,\marginnote{163.1} out of disrespect, one walks in an inhabited area, playing with one’s hands and feet, one commits one kind of offense: an offense of wrong conduct. 

When,\marginnote{164.1} out of disrespect, one sits in an inhabited area, playing with one’s hands and feet, one commits one kind of offense: an offense of wrong conduct. 

When,\marginnote{165.1} out of disrespect, one walks in an inhabited area, looking here and there, one commits one kind of offense: an offense of wrong conduct. 

When,\marginnote{166.1} out of disrespect, one sits in an inhabited area, looking here and there, one commits one kind of offense: an offense of wrong conduct. 

When,\marginnote{167.1} out of disrespect, one walks in an inhabited area with a lifted robe, one commits one kind of offense: an offense of wrong conduct. 

When,\marginnote{168.1} out of disrespect, one sits in an inhabited area with a lifted robe, one commits one kind of offense: an offense of wrong conduct. 

\scendvagga{The first subchapter on evenly all around is finished. }

\subsection*{The subchapter on laughing loudly }

When,\marginnote{170.1} out of disrespect, one laughs loudly while walking in an inhabited area, one commits one kind of offense: an offense of wrong conduct. 

When,\marginnote{171.1} out of disrespect, one laughs loudly while sitting in an inhabited area, one commits one kind of offense: an offense of wrong conduct. 

When,\marginnote{172.1} out of disrespect, one is noisy while walking in an inhabited area, one commits one kind of offense: an offense of wrong conduct. 

When,\marginnote{173.1} out of disrespect, one is noisy while sitting in an inhabited area, one commits one kind of offense: an offense of wrong conduct. 

When,\marginnote{174.1} out of disrespect, one sways one’s body while walking in an inhabited area, one commits one kind of offense: an offense of wrong conduct. 

When,\marginnote{175.1} out of disrespect, one sways one’s body while sitting in an inhabited area, one commits one kind of offense: an offense of wrong conduct. 

When,\marginnote{176.1} out of disrespect, one swings one’s arms while walking in an inhabited area, one commits one kind of offense: an offense of wrong conduct. 

When,\marginnote{177.1} out of disrespect, one swings one’s arms while sitting in an inhabited area, one commits one kind of offense: an offense of wrong conduct. 

When,\marginnote{178.1} out of disrespect, one sways one’s head while walking in an inhabited area, one commits one kind of offense: an offense of wrong conduct. 

When,\marginnote{179.1} out of disrespect, one sways one’s head while sitting in an inhabited area, one commits one kind of offense: an offense of wrong conduct. 

\scendvagga{The second subchapter on laughing loudly is finished. }

\subsection*{The subchapter on hands on hips }

When,\marginnote{181.1} out of disrespect, one walks in an inhabited area with one’s hands on one’s hips, one commits one kind of offense: an offense of wrong conduct. 

When,\marginnote{182.1} out of disrespect, one sits in an inhabited area with one’s hands on one’s hips, one commits one kind of offense: an offense of wrong conduct. 

When,\marginnote{183.1} out of disrespect, one walks in an inhabited area with a covered head, one commits one kind of offense: an offense of wrong conduct. 

When,\marginnote{184.1} out of disrespect, one sits in an inhabited area with a covered head, one commits one kind of offense: an offense of wrong conduct. 

When,\marginnote{185.1} out of disrespect, one moves about while squatting on one’s heels in an inhabited area, one commits one kind of offense: an offense of wrong conduct. 

When,\marginnote{186.1} out of disrespect, one clasps one’s knees while sitting in an inhabited area, one commits one kind of offense: an offense of wrong conduct. 

When,\marginnote{187.1} out of disrespect, one receives almsfood contemptuously, one commits one kind of offense: an offense of wrong conduct. 

When,\marginnote{188.1} out of disrespect, one receives almsfood while looking here and there, one commits one kind of offense: an offense of wrong conduct. 

When,\marginnote{189.1} out of disrespect, one receives large amounts of bean curry, one commits one kind of offense: an offense of wrong conduct. 

When,\marginnote{190.1} out of disrespect, one receives almsfood in a heap, one commits one kind of offense: an offense of wrong conduct. 

\scendvagga{The third subchapter on hands on hips is finished. }

\subsection*{The subchapter on almsfood }

When,\marginnote{192.1} out of disrespect, one eats almsfood contemptuously, one commits one kind of offense: an offense of wrong conduct. 

When,\marginnote{193.1} out of disrespect, one eats almsfood while looking here and there, one commits one kind of offense: an offense of wrong conduct. 

When,\marginnote{194.1} out of disrespect, one eats almsfood picking here and there, one commits one kind of offense: an offense of wrong conduct. 

When,\marginnote{195.1} out of disrespect, one eats large amounts of bean curry, one commits one kind of offense: an offense of wrong conduct. 

When,\marginnote{196.1} out of disrespect, one eats almsfood after making a heap, one commits one kind of offense: an offense of wrong conduct. 

When,\marginnote{197.1} out of disrespect, one covers one’s curries with rice, one commits one kind of offense: an offense of wrong conduct. 

When,\marginnote{198.1} out of disrespect, one eats bean curry or rice that, when one is not sick, one has requested for oneself, one commits one kind of offense: an offense of wrong conduct. 

When,\marginnote{199.1} out of disrespect, one looks at the almsbowl of another finding fault, one commits one kind of offense: an offense of wrong conduct. 

When,\marginnote{200.1} out of disrespect, one makes a large mouthful, one commits one kind of offense: an offense of wrong conduct. 

When,\marginnote{201.1} out of disrespect, one makes an elongated mouthful, one commits one kind of offense: an offense of wrong conduct. 

\scendvagga{The fourth subchapter on almsfood is finished. }

\subsection*{The subchapter on mouthfuls }

When,\marginnote{203.1} out of disrespect, one opens one’s mouth without bringing a mouthful to it, one commits one kind of offense: an offense of wrong conduct. 

When,\marginnote{204.1} out of disrespect, one puts one’s whole hand in one’s mouth while eating, one commits one kind of offense: an offense of wrong conduct. 

When,\marginnote{205.1} out of disrespect, one speaks with food in one’s mouth, one commits one kind of offense: an offense of wrong conduct. 

When,\marginnote{206.1} out of disrespect, one eats from a lifted ball of food, one commits one kind of offense: an offense of wrong conduct. 

When,\marginnote{207.1} out of disrespect, one eats breaking up mouthfuls, one commits one kind of offense: an offense of wrong conduct. 

When,\marginnote{208.1} out of disrespect, one eats stuffing one’s cheeks, one commits one kind of offense: an offense of wrong conduct. 

When,\marginnote{209.1} out of disrespect, one eats shaking one’s hand, one commits one kind of offense: an offense of wrong conduct. 

When,\marginnote{210.1} out of disrespect, one eats scattering rice, one commits one kind of offense: an offense of wrong conduct. 

When,\marginnote{211.1} out of disrespect, one eats sticking out one’s tongue, one commits one kind of offense: an offense of wrong conduct. 

When,\marginnote{212.1} out of disrespect, one eats making a chomping sound, one commits one kind of offense: an offense of wrong conduct. 

\scendvagga{The fifth subchapter on mouthfuls is finished. }

\subsection*{The subchapter on slurping }

When,\marginnote{214.1} out of disrespect, one eats making a slurping sound, one commits one kind of offense: an offense of wrong conduct. 

When,\marginnote{215.1} out of disrespect, one eats licking one’s hands, one commits one kind of offense: an offense of wrong conduct. 

When,\marginnote{216.1} out of disrespect, one eats licking one’s almsbowl, one commits one kind of offense: an offense of wrong conduct. 

When,\marginnote{217.1} out of disrespect, one eats licking one’s lips, one commits one kind of offense: an offense of wrong conduct. 

When,\marginnote{218.1} out of disrespect, one receives the drinking-water vessel with a hand soiled with food, one commits one kind of offense: an offense of wrong conduct. 

When,\marginnote{219.1} out of disrespect, one discards bowl-washing water containing rice in an inhabited area, one commits one kind of offense: an offense of wrong conduct. 

When,\marginnote{220.1} out of disrespect, one gives a teaching to someone holding a sunshade, one commits one kind of offense: an offense of wrong conduct. 

When,\marginnote{221.1} out of disrespect, one gives a teaching to someone holding a staff, one commits one kind of offense: an offense of wrong conduct. 

When,\marginnote{222.1} out of disrespect, one gives a teaching to someone holding a knife, one commits one kind of offense: an offense of wrong conduct. 

When,\marginnote{223.1} out of disrespect, one gives a teaching to someone holding a weapon, one commits one kind of offense: an offense of wrong conduct. 

\scendvagga{The sixth subchapter on slurping is finished. }

\subsection*{The subchapter on shoes }

When,\marginnote{225.1} out of disrespect, one gives a teaching to someone wearing shoes, one commits one kind of offense: an offense of wrong conduct. 

When,\marginnote{226.1} out of disrespect, one gives a teaching to someone wearing sandals, one commits one kind of offense: an offense of wrong conduct. 

When,\marginnote{227.1} out of disrespect, one gives a teaching to someone in a vehicle, one commits one kind of offense: an offense of wrong conduct. 

When,\marginnote{228.1} out of disrespect, one gives a teaching to someone lying down, one commits one kind of offense: an offense of wrong conduct. 

When,\marginnote{229.1} out of disrespect, one gives a teaching to someone seated clasping their knees, one commits one kind of offense: an offense of wrong conduct. 

When,\marginnote{230.1} out of disrespect, one gives a teaching to someone with a headdress, one commits one kind of offense: an offense of wrong conduct. 

When,\marginnote{231.1} out of disrespect, one gives a teaching to someone with a covered head, one commits one kind of offense: an offense of wrong conduct. 

When,\marginnote{232.1} out of disrespect, one gives a teaching while sitting on the ground to someone sitting on a seat, one commits one kind of offense: an offense of wrong conduct. 

When,\marginnote{233.1} out of disrespect, one gives a teaching while sitting on a low seat to someone sitting on a high seat, one commits one kind of offense: an offense of wrong conduct. 

When,\marginnote{234.1} out of disrespect, one gives a teaching while standing to someone sitting, one commits one kind of offense: an offense of wrong conduct. 

When,\marginnote{235.1} out of disrespect, one gives a teaching to someone walking in front of oneself, one commits one kind of offense: an offense of wrong conduct. 

When,\marginnote{236.1} out of disrespect, one gives a teaching while walking next to the path to someone walking on the path, one commits one kind of offense: an offense of wrong conduct. 

When,\marginnote{237.1} out of disrespect, one defecates or urinates while standing, one commits one kind of offense: an offense of wrong conduct. 

When,\marginnote{238.1} out of disrespect, one defecates, urinates, or spits on cultivated plants, one commits one kind of offense: an offense of wrong conduct. 

When,\marginnote{239.1} out of disrespect, one defecates, urinates, or spits in water, how many kinds of offenses does one commit? When, out of disrespect, one defecates, urinates, or spits in water, one commits one kind of offense: an offense of wrong conduct. 

\scendvagga{The seventh subchapter on shoes is finished. }

\scend{The rules to be trained in are finished. }

\scendsutta{The number of offenses within each offense, the second, is finished. }

%
\chapter*{{\suttatitleacronym Pvr 1.3}{\suttatitletranslation The classes of failure for each offense }{\suttatitleroot Vipattivāra}}
\addcontentsline{toc}{chapter}{\tocacronym{Pvr 1.3} \toctranslation{The classes of failure for each offense } \tocroot{Vipattivāra}}
\markboth{The classes of failure for each offense }{Vipattivāra}
\extramarks{Pvr 1.3}{Pvr 1.3}

When\marginnote{1.1} it comes to the offenses for having sexual intercourse, to how many of the four kinds of failure do they belong? They belong to two kinds of failure: they may be failure in morality; they may be failure in conduct. … 

When\marginnote{2.1} it comes to the offense for, out of disrespect, defecating, urinating, or spitting in water, to how many of the four kinds of failure does it belong? It belongs to one kind of failure: failure in conduct. 

\scendsutta{The classes of failure for each offense, the third, are finished. }

%
\chapter*{{\suttatitleacronym Pvr 1.4}{\suttatitletranslation The classes of offenses in each offense }{\suttatitleroot Saṅgahitavāra}}
\addcontentsline{toc}{chapter}{\tocacronym{Pvr 1.4} \toctranslation{The classes of offenses in each offense } \tocroot{Saṅgahitavāra}}
\markboth{The classes of offenses in each offense }{Saṅgahitavāra}
\extramarks{Pvr 1.4}{Pvr 1.4}

When\marginnote{1.1} it comes to the offenses for having sexual intercourse, in how many of the seven classes of offenses are they found? They are found in three: they may be in the class of offenses entailing expulsion; they may be in the class of serious offenses; they may be in the class of offenses of wrong conduct. … 

When\marginnote{2.1} it comes to the offense for, out of disrespect, defecating, urinating, or spitting in water, in how many of the seven classes of offenses is it found? It is found in one: in the class of offenses of wrong conduct. 

\scendsutta{The classes of offenses in each offense, the fourth, are finished. }

%
\chapter*{{\suttatitleacronym Pvr 1.5}{\suttatitletranslation The originations of each offense }{\suttatitleroot Samuṭṭhānavāra}}
\addcontentsline{toc}{chapter}{\tocacronym{Pvr 1.5} \toctranslation{The originations of each offense } \tocroot{Samuṭṭhānavāra}}
\markboth{The originations of each offense }{Samuṭṭhānavāra}
\extramarks{Pvr 1.5}{Pvr 1.5}

When\marginnote{1.1} it comes to the offenses for having sexual intercourse, through how many of the six kinds of originations of offenses do they originate? They originate in one way: from body and mind, not from speech. … 

When\marginnote{2.1} it comes to the offense for, out of disrespect, defecating, urinating, or spitting in water, through how many of the six kinds of originations of offenses does it originate? It originates in one way: from body and mind, not from speech. 

\scendsutta{The originations of each offense, the fifth, are finished. }

%
\chapter*{{\suttatitleacronym Pvr 1.6}{\suttatitletranslation The legal issues to which each offense belongs }{\suttatitleroot Adhikaraṇavāra}}
\addcontentsline{toc}{chapter}{\tocacronym{Pvr 1.6} \toctranslation{The legal issues to which each offense belongs } \tocroot{Adhikaraṇavāra}}
\markboth{The legal issues to which each offense belongs }{Adhikaraṇavāra}
\extramarks{Pvr 1.6}{Pvr 1.6}

When\marginnote{1.1} it comes to the offenses for having sexual intercourse, to which of the four kinds of legal issues do they belong? They belong to legal issues arising from an offense. … 

When\marginnote{2.1} it comes to the offense for, out of disrespect, defecating, urinating, or spitting in water, to which of the four kinds of legal issues does it belong? It belongs to legal issues arising from an offense. 

\scendsutta{The legal issues to which each offense belongs, the sixth, are finished. }

%
\chapter*{{\suttatitleacronym Pvr 1.7}{\suttatitletranslation How each offense is settled }{\suttatitleroot Samathavāra}}
\addcontentsline{toc}{chapter}{\tocacronym{Pvr 1.7} \toctranslation{How each offense is settled } \tocroot{Samathavāra}}
\markboth{How each offense is settled }{Samathavāra}
\extramarks{Pvr 1.7}{Pvr 1.7}

When\marginnote{1.1} it comes to the offenses for having sexual intercourse, through how many of the seven principles for settling legal issues are they settled? Through three of them: they may be settled by resolution face-to-face and by acting according to what has been admitted; or they may be settled by resolution face-to-face and by covering over as if with grass. … 

When\marginnote{2.1} it comes to the offense for, out of disrespect, defecating, urinating, or spitting in water, through how many of the seven principles for settling legal issues is it settled? Through three of them: it may be settled by resolution face-to-face and by acting according to what has been admitted; or it may be settled by resolution face-to-face and by covering over as if with grass. 

\scendsutta{How each offense is settled, the seventh, is finished. }

%
\chapter*{{\suttatitleacronym Pvr 1.8}{\suttatitletranslation Summary of the previous six sections }{\suttatitleroot Samuccayavāra}}
\addcontentsline{toc}{chapter}{\tocacronym{Pvr 1.8} \toctranslation{Summary of the previous six sections } \tocroot{Samuccayavāra}}
\markboth{Summary of the previous six sections }{Samuccayavāra}
\extramarks{Pvr 1.8}{Pvr 1.8}

When\marginnote{1.1} having sexual intercourse, how many kinds of offenses does one commit? One commits three kinds of offenses: when one has sexual intercourse with an undecomposed corpse, one commits an offense entailing expulsion; when one has sexual intercourse with a mostly decomposed corpse, one commits a serious offense; when one inserts one’s penis into a wide open mouth without touching it, one commits an offense of wrong conduct. 

When\marginnote{2.1} it comes to these offenses, to how many of the four kinds of failure do they belong? In how many of the seven classes of offenses are they found? Through how many of the six kinds of originations of offenses do they originate? To which of the four kinds of legal issues do they belong? Through how many of the seven principles for settling legal issues are they settled? 

They\marginnote{2.2} belong to two kinds of failure: they may be failure in morality; they may be failure in conduct. They are found in three classes of offenses: they may be in the class of offenses entailing expulsion; they may be in the class of serious offenses; they may be in the class of offenses of wrong conduct. They originate in one way: from body and mind, not from speech. They belong to legal issues arising from an offense. They are settled through three principles: they may be settled by resolution face-to-face and by acting according to what has been admitted; or they may be settled by resolution face-to-face and by covering over as if with grass. … 

When,\marginnote{3.1} out of disrespect, one defecates, urinates, or spits in water, how many kinds of offenses does one commit? One commits one kind of offense: an offense of wrong conduct. 

When\marginnote{4.1} it comes to this offense, to how many of the four kinds of failure does it belong? In how many of the seven classes of offenses is it found? Through how many of the six kinds of originations of offenses does it originate? To which of the four kinds of legal issues does it belong? Through how many of the seven principles for settling legal issues is it settled? It belongs to one kind of failure: 

failure\marginnote{4.3} in conduct. It is found in one class of offenses: in the class of offenses of wrong conduct. It originates in one way: from body and mind, not from speech. It belongs to legal issues arising from an offense. It is settled through three principles: it may be settled by resolution face-to-face and by acting according to what has been admitted; or it may be settled by resolution face-to-face and by covering over as if with grass. 

\scendsutta{The summary of the previous six sections, the eighth, is finished. }

\scend{These eight sections were written down through the method of recitation. }

\scuddanaintro{This is the summary: }

\begin{scuddana}%
“Where\marginnote{7.1} was it laid down, and how many, \\
Failure, and being found in; \\
Originations, legal issues, \\
Settling, and with gathering up.” 

%
\end{scuddana}

%
\chapter*{{\suttatitleacronym Pvr 1.9}{\suttatitletranslation Questions and answers on the monks’ Pātimokkha rules and their analysis }{\suttatitleroot Katthapaññattivāra}}
\addcontentsline{toc}{chapter}{\tocacronym{Pvr 1.9} \toctranslation{Questions and answers on the monks’ Pātimokkha rules and their analysis } \tocroot{Katthapaññattivāra}}
\markboth{Questions and answers on the monks’ Pātimokkha rules and their analysis }{Katthapaññattivāra}
\extramarks{Pvr 1.9}{Pvr 1.9}

\section*{The chapter on offenses entailing expulsion }

“The\marginnote{1.1} offense entailing expulsion that is a result of having sexual intercourse was laid down by the Buddha who knows and sees, the Perfected One, the fully Awakened One. Where was it laid down? Whom is it about? What is it about? … Who handed it down?” 

“The\marginnote{2.1} offense entailing expulsion that is a result of having sexual intercourse was laid down by the Buddha who knows and sees, the Perfected One, the fully Awakened One. Where was it laid down?” At \textsanskrit{Vesālī}. “Whom is it about?” Sudinna the Kalandian. “What is it about?” Sudinna having sexual intercourse with his ex-wife. “Is there a rule, an addition to the rule, an unprompted rule?” There is one rule. There are two additions to the rule. There is no unprompted rule. “Is it a rule that applies everywhere or in a particular place?” Everywhere. “Is it a rule that the monks and nuns have in common or not in common?” In common. “Is it a rule for one Sangha or for both?” For both. “In which of the five ways of reciting the Monastic Code is it contained and included?” In the introduction. “In which recitation is it included?” In the second recitation. “To which of the four kinds of failure does it belong?” Failure in morality. “To which of the seven classes of offenses does it belong?” The class of offenses entailing expulsion. “Through how many of the six kinds of originations of offenses does it originate?” It originates in one way: from body and mind, not from speech. … “Who handed it down?” The lineage: 

\begin{verse}%
“\textsanskrit{Upāli}\marginnote{3.1} and \textsanskrit{Dāsaka}, \\
\textsanskrit{Soṇaka} and so Siggava; \\
With Moggaliputta as the fifth—\\
These were in India, the land named after the glorious rose apple. 

…\marginnote{4.1} 

These\marginnote{5.1} mighty beings of great wisdom, \\
Knowers of the Monastic Law and skilled in the path; \\
Proclaimed the Collection of Monastic Law, \\
On the island of Sri Lanka.” 

%
\end{verse}

“The\marginnote{6.1} offense entailing expulsion that is a result of stealing was laid down by the Buddha who knows and sees, the Perfected One, the fully Awakened One. Where was it laid down?” At \textsanskrit{Rājagaha}. “Whom is it about?” Dhaniya the potter. “What is it about?” Dhaniya stealing timber from the king. There is one rule. There is one addition to the rule. Of the six kinds of originations of offenses, it originates in three ways: from body and mind, not from speech; or from speech and mind, not from body; or from body, speech, and mind. … 

“There\marginnote{7.1} is an offense entailing expulsion that is a result of intentionally killing a human being. Where was it laid down?” At \textsanskrit{Vesālī}. “Whom is it about?” A number of monks. “What is it about?” Those monks killing one another. There is one rule. There is one addition to the rule. Of the six kinds of originations of offenses, it originates in three ways: from body and mind, not from speech; or from speech and mind, not from body; or from body, speech, and mind. … 

“There\marginnote{8.1} is an offense entailing expulsion that is a result of claiming a non-existent superhuman quality. Where was it laid down?” At \textsanskrit{Vesālī}. “Whom is it about?” The monks from the banks of the \textsanskrit{Vaggumudā}. “What is it about?” Those monks praising one another’s superhuman qualities to householders. There is one rule. There is one addition to the rule. Of the six kinds of originations of offenses, it originates in three ways: from body and mind, not from speech; or from speech and mind, not from body; or from body, speech, and mind. … 

\scend{The four offenses entailing expulsion are finished. }

\section*{2. The chapter on offenses entailing suspension, etc. }

“The\marginnote{10.1} offense entailing suspension that is a result of emitting semen by means of effort was laid down by the Buddha who knows and sees, the Perfected One, the fully Awakened One. Where was it laid down? Whom is it about? What is it about? … Who handed it down?” 

“The\marginnote{11.1} offense entailing suspension that is a result of emitting semen by means of effort was laid down by the Buddha who knows and sees, the Perfected One, the fully Awakened One. Where was it laid down?” At \textsanskrit{Sāvatthī}. “Whom is it about?” Venerable Seyyasaka. “What is it about?” Seyyasaka masturbating. “Is there a rule, an addition to the rule, an unprompted rule?” There is one rule. There is one addition to the rule. There is no unprompted rule. “Is it a rule that applies everywhere or in a particular place?” Everywhere. “Is it a rule that the monks and nuns have in common or not in common?” Not in common. “Is it a rule for one Sangha or for both?” For one. “In which of the five ways of reciting the Monastic Code is it contained and included?” In the introduction. “In which recitation is it included?” In the third recitation. “To which of the four kinds of failure does it belong?” Failure in morality. “To which of the seven classes of offenses does it belong?” The class of offenses entailing suspension. “Through how many of the six kinds of originations of offenses does it originate?” It originates in one way: from body and mind, not from speech. … “Who handed it down?” The lineage: 

\begin{verse}%
“\textsanskrit{Upāli}\marginnote{12.1} and \textsanskrit{Dāsaka}, \\
\textsanskrit{Soṇaka} and so Siggava; \\
With Moggaliputta as the fifth—\\
These were in India, the land named after the glorious rose apple. 

…\marginnote{13.1} 

These\marginnote{14.1} mighty beings of great wisdom, \\
Knowers of the Monastic Law and skilled in the path; \\
Proclaimed the Collection of Monastic Law, \\
On the island of Sri Lanka.” 

%
\end{verse}

“There\marginnote{15.1} is an offense entailing suspension that is a result of making physical contact with a woman. Where was it laid down?” At \textsanskrit{Sāvatthī}. “Whom is it about?” Venerable \textsanskrit{Udāyī}. “What is it about?” \textsanskrit{Udāyī} making physical contact with a woman. There is one rule. Of the six kinds of originations of offenses, it originates in one way: from body and mind, not from speech. … 

“There\marginnote{16.1} is an offense entailing suspension that is a result of speaking indecently to a woman. Where was it laid down?” At \textsanskrit{Sāvatthī}. “Whom is it about?” Venerable \textsanskrit{Udāyī}. “What is it about?” \textsanskrit{Udāyī} speaking indecently to a woman. There is one rule. Of the six kinds of originations of offenses, it originates in three ways: from body and mind, not from speech; or from speech and mind, not from body; or from body, speech, and mind. … 

“There\marginnote{17.1} is an offense entailing suspension that is a result of encouraging a woman to satisfy one’s own desires. Where was it laid down?” At \textsanskrit{Sāvatthī}. “Whom is it about?” Venerable \textsanskrit{Udāyī}. “What is it about?” \textsanskrit{Udāyī} encouraging a woman to satisfy his own desires. There is one rule. Of the six kinds of originations of offenses, it originates in three ways: … 

“There\marginnote{18.1} is an offense entailing suspension that is a result of acting as a matchmaker. Where was it laid down?” At \textsanskrit{Sāvatthī}. “Whom is it about?” Venerable \textsanskrit{Udāyī}. “What is it about?” \textsanskrit{Udāyī} acting as a matchmaker. There is one rule. There is one addition to the rule. Of the six kinds of originations of offenses, it originates in six ways: from body, not from speech or mind; or from speech, not from body or mind; or from body and speech, not from mind; or from body and mind, not from speech; or from speech and mind, not from body; or from body, speech, and mind. … 

“There\marginnote{19.1} is an offense entailing suspension that is a result of having a hut built by means of begging. Where was it laid down?” At \textsanskrit{Āḷavī}. “Whom is it about?” The monks of \textsanskrit{Āḷavī}. “What is it about?” Those monks having huts made by means of begging. There is one rule. Of the six kinds of originations of offenses, it originates in six ways: … 

“There\marginnote{20.1} is an offense entailing suspension that is a result of having a large dwelling built. Where was it laid down?” At \textsanskrit{Kosambī}. “Whom is it about?” Venerable Channa. “What is it about?” Channa having a tree that served as a shrine felled to clear a site for a dwelling. There is one rule. Of the six kinds of originations of offenses, it originates in six ways: … 

“There\marginnote{21.1} is an offense entailing suspension that is a result of groundlessly charging a monk with an offense entailing expulsion. Where was it laid down?” At \textsanskrit{Rājagaha}. “Whom is it about?” The monks Mettiya and \textsanskrit{Bhūmajaka}. “What is it about?” Those monks groundlessly charging Venerable Dabba the Mallian with an offense entailing expulsion. There is one rule. Of the six kinds of originations of offenses, it originates in three ways: … 

“There\marginnote{22.1} is an offense entailing suspension that is a result of charging a monk with an offense entailing expulsion, using an unrelated legal issue as a pretext. Where was it laid down?” At \textsanskrit{Rājagaha}. “Whom is it about?” The monks Mettiya and \textsanskrit{Bhūmajaka}. “What is it about?” Those monks charging Venerable Dabba the Mallian with an offense entailing expulsion, using an unrelated legal issue as a pretext. There is one rule. Of the six kinds of originations of offenses, it originates in three ways: … 

“There\marginnote{23.1} is an offense entailing suspension that is a result of a monk not stopping with pursuing schism in the Sangha when pressed for the third time. Where was it laid down?” At \textsanskrit{Rājagaha}. “Whom is it about?” Devadatta. “What is it about?” Devadatta pursuing schism in a united Sangha. There is one rule. Of the six kinds of originations of offenses, it originates in one way: from body, speech, and mind. … 

“There\marginnote{24.1} is an offense entailing suspension that is a result of monks not stopping siding with one who is pursuing schism in the Sangha when pressed for the third time. Where was it laid down?” At \textsanskrit{Rājagaha}. “Whom is it about?” Several monks. “What is it about?” Those monks siding with and supporting Devadatta’s pursuit of schism in the Sangha. There is one rule. Of the six kinds of originations of offenses, it originates in one way: from body, speech, and mind. … 

“There\marginnote{25.1} is an offense entailing suspension that is a result of a monk not stopping with being difficult to correct when pressed for the third time. Where was it laid down?” At \textsanskrit{Kosambī}. “Whom is it about?” Venerable Channa. “What is it about?” Channa making himself incorrigible when legitimately spoken to by the monks. There is one rule. Of the six kinds of originations of offenses, it originates in one way: from body, speech, and mind. … 

“There\marginnote{26.1} is an offense entailing suspension that is a result of a monk not stopping with being a corrupter of families when pressed for the third time. Where was it laid down?” At \textsanskrit{Sāvatthī}. “Whom is it about?” The monks Assaji and Punabbasuka. “What is it about?” Those monks, when the Sangha did a legal procedure of banishment against them, slandering the monks as acting out of favoritism, ill will, confusion, and fear. There is one rule. Of the six kinds of originations of offenses, it originates in one way: from body, speech, and mind. … 

“There\marginnote{27.1} is an offense of wrong conduct that is a result of, out of disrespect, defecating, urinating, or spitting in water. Where was it laid down?” At \textsanskrit{Sāvatthī}. “Whom is it about?” The monks from the group of six. “What is it about?” Those monks defecating, urinating, and spitting in water. There is one rule. There is one addition to the rule. Of the six kinds of originations of offenses, it originates in one way: from body and mind, not from speech. … 

\scendsutta{The questions and answers on the monks’ \textsanskrit{Pātimokkha} rules and their analysis, the first, are finished. }

%
\chapter*{{\suttatitleacronym Pvr 1.10}{\suttatitletranslation The number of offenses within each offense }{\suttatitleroot Katāpattivāra}}
\addcontentsline{toc}{chapter}{\tocacronym{Pvr 1.10} \toctranslation{The number of offenses within each offense } \tocroot{Katāpattivāra}}
\markboth{The number of offenses within each offense }{Katāpattivāra}
\extramarks{Pvr 1.10}{Pvr 1.10}

\section*{The chapter on offenses entailing expulsion }

How\marginnote{1.1} many kinds of offenses does one commit as a result of having sexual intercourse? One commits four kinds of offenses: when one has sexual intercourse with an undecomposed corpse, one commits an offense entailing expulsion; when one has sexual intercourse with a mostly decomposed corpse, one commits a serious offense; when one inserts one’s penis into a wide open mouth without touching it, one commits an offense of wrong conduct; when one uses a dildo, one commits an offense entailing confession. 

How\marginnote{2.1} many kinds of offenses does one commit as a result of stealing? One commits three kinds of offenses: when, intending to steal, one steals something worth five \textit{\textsanskrit{māsaka}} coins or more, one commits an offense entailing expulsion; when, intending to steal, one steals something worth more than one \textit{\textsanskrit{māsaka}} coin but less than five, one commits a serious offense; when, intending to steal, one steals something worth one \textit{\textsanskrit{māsaka}} coin or less, one commits an offense of wrong conduct. 

How\marginnote{3.1} many kinds of offenses does one commit as a result of intentionally killing a human being? One commits three kinds of offenses: when one digs a pit for a human being, thinking, “Falling into it, they will die,” one commits an offense of wrong conduct; when they experience pain after falling in, one commits a serious offense; when they die, one commits an offense entailing expulsion. 

How\marginnote{4.1} many kinds of offenses does one commit as a result of claiming a non-existent superhuman quality? One commits three kinds of offenses: when, having bad desires, overcome by desire, one claims a non-existent superhuman quality, one commits an offense entailing expulsion; when one says, “The monk who stays in your dwelling is a perfected one,” and the listener understands, one commits a serious offense; when the listener does not understand, one commits an offense of wrong conduct. 

\scend{The four offenses entailing expulsion are finished. }

\section*{2. The chapter on offenses entailing suspension, etc. }

How\marginnote{6.1} many kinds of offenses does one commit as a result of emitting semen by means of effort? One commits three kinds of offenses: when one intends and makes an effort, and semen is emitted, one commits an offense entailing suspension; when one intends and makes an effort, but semen is not emitted, one commits a serious offense; for the effort there is an offense of wrong conduct. 

How\marginnote{7.1} many kinds of offenses does one commit as a result of making physical contact with a woman? One commits five kinds of offenses: when a lustful nun consents to a lustful man taking hold of her anywhere below the collar bone but above the knees, she commits an offense entailing expulsion; when a monk makes physical contact, body with body, he commits an offense entailing suspension; when, with one’s own body, one makes physical contact with something connected to their body, one commits a serious offense; when, with something connected to one’s own body, one makes physical contact with something connected to their body, one commits an offense of wrong conduct; for tickling, one commits an offense entailing confession. 

As\marginnote{8.1} a result of speaking indecently to a woman, one commits three kinds of offenses: when, referring to the anus or the vagina, one praises or disparages, one commits an offense entailing suspension; when, referring to any part below the collar bone but above the knees, apart from the anus or the vagina, one praises or disparages, one commits a serious offense; when, referring to anything connected to the body, one praises or disparages, one commits an offense of wrong conduct. 

As\marginnote{9.1} a result of encouraging the satisfaction of one’s own desires, one commits three kinds of offenses: when one encourages a woman to satisfy one’s own desires, one commits an offense entailing suspension; when one encourages a \textit{\textsanskrit{paṇḍaka}} to satisfy one’s own desires, one commits a serious offense; when one encourages an animal to satisfy one’s own desires, one commits an offense of wrong conduct. 

As\marginnote{10.1} a result of acting as a matchmaker, one commits three kinds of offenses: when one accepts the mission, finds out the response, and reports back, one commits an offense entailing suspension; when one accepts the mission, and finds out the response, but does not report back, one commits a serious offense; when one accepts the mission, but neither finds out the response, nor reports back, one commits an offense of wrong conduct. 

As\marginnote{11.1} a result of having a hut built by means of begging, one commits three kinds of offenses: when one is having it built, then for the effort there is an offense of wrong conduct; when there is one piece left to complete the hut, one commits a serious offense; when the last piece is finished, one commits an offense entailing suspension. 

As\marginnote{12.1} a result of having a large dwelling built, one commits three kinds of offenses: when one is having it built, then for the effort there is an offense of wrong conduct; when there is one piece left to complete the hut, one commits a serious offense; when the last piece is finished, one commits an offense entailing suspension. 

As\marginnote{13.1} a result of groundlessly charging a monk with an offense entailing expulsion, one commits three kinds of offenses: when, without having gotten his permission, one speaks with the aim of making him leave the monastic life, one commits one offense entailing suspension and one offense of wrong conduct; when, having gotten his permission, one speaks with the aim of abusing him, one commits an offense for abusive speech. 

As\marginnote{14.1} a result of charging a monk with an offense entailing expulsion, using an unrelated legal issue as a pretext, one commits three kinds of offenses: when, without having gotten his permission, one speaks with the aim of making him leave the monastic life, one commits one offense entailing suspension and one offense of wrong conduct; when, having gotten his permission, one speaks with the aim of abusing him, one commits an offense for abusive speech. 

As\marginnote{15.1} a result of not stopping when pressed for the third time, a monk who is pursuing schism in the Sangha commits three kinds of offenses: after the motion, he commits an offense of wrong conduct; after each of the first two announcements, he commits a serious offense; when the last announcement is finished, he commits an offense entailing suspension. 

As\marginnote{16.1} a result of not stopping when pressed for the third time, monks who side with a monk who is pursuing schism commit three kinds of offenses: after the motion, they commit an offense of wrong conduct; after each of the first two announcements, they commit a serious offense; when the last announcement is finished, they commit an offense entailing suspension. 

As\marginnote{17.1} a result of not stopping when pressed for the third time, a monk who is difficult to correct commits three kinds of offenses: after the motion, he commits an offense of wrong conduct; after each of the first two announcements, he commits a serious offense; when the last announcement is finished, he commits an offense entailing suspension. 

As\marginnote{18.1} a result of not stopping when pressed for the third time, a monk who is a corrupter of families commits three kinds of offenses: after the motion, he commits an offense of wrong conduct; after each of the first two announcements, he commits a serious offense; when the last announcement is finished, he commits an offense entailing suspension. … 

As\marginnote{19.1} a result of, out of disrespect, defecating, urinating, or spitting in water, how many kinds of offenses does one commit? One commits one kind of offense: an offense of wrong conduct. 

\scendsutta{The number of offenses within each offense, the second, is finished. }

%
\chapter*{{\suttatitleacronym Pvr 1.11}{\suttatitletranslation The classes of failure for each offense }{\suttatitleroot Vipattivāra}}
\addcontentsline{toc}{chapter}{\tocacronym{Pvr 1.11} \toctranslation{The classes of failure for each offense } \tocroot{Vipattivāra}}
\markboth{The classes of failure for each offense }{Vipattivāra}
\extramarks{Pvr 1.11}{Pvr 1.11}

When\marginnote{1.1} it comes to the offenses that are a result of having sexual intercourse, to how many of the four kinds of failure do they belong? They belong to two kinds of failure: they may be failure in morality; they may be failure in conduct. … 

When\marginnote{2.1} it comes to the offense that is a result of, out of disrespect, defecating, urinating, or spitting in water, to how many of the four kinds of failure does it belong? It belongs to one kind of failure: failure in conduct. 

\scendsutta{The classes of failure for each offense, the third, are finished. }

%
\chapter*{{\suttatitleacronym Pvr 1.12}{\suttatitletranslation The classes of offenses in each offense }{\suttatitleroot Saṅgahitavāra}}
\addcontentsline{toc}{chapter}{\tocacronym{Pvr 1.12} \toctranslation{The classes of offenses in each offense } \tocroot{Saṅgahitavāra}}
\markboth{The classes of offenses in each offense }{Saṅgahitavāra}
\extramarks{Pvr 1.12}{Pvr 1.12}

When\marginnote{1.1} it comes to the offenses that are a result of having sexual intercourse, in how many of the seven classes of offenses are they found? They are found in four: they may be in the class of offenses entailing expulsion; they may be in the class of serious offenses; they may be in the class of offenses entailing confession; they may be in the class of offenses of wrong conduct. … 

When\marginnote{2.1} it comes to the offense that is a result of, out of disrespect, defecating, urinating, or spitting in water, in how many of the seven classes of offenses is it found? It is found in one: in the class of offenses of wrong conduct. 

\scendsutta{The classes of offenses in each offense, the fourth, are finished. }

%
\chapter*{{\suttatitleacronym Pvr 1.13}{\suttatitletranslation The originations of each offense }{\suttatitleroot Samuṭṭhānavāra}}
\addcontentsline{toc}{chapter}{\tocacronym{Pvr 1.13} \toctranslation{The originations of each offense } \tocroot{Samuṭṭhānavāra}}
\markboth{The originations of each offense }{Samuṭṭhānavāra}
\extramarks{Pvr 1.13}{Pvr 1.13}

When\marginnote{1.1} it comes to the offenses that are a result of having sexual intercourse, through how many of the six kinds of originations of offenses do they originate? They originate in one way: from body and mind, not from speech. … 

When\marginnote{2.1} it comes to the offense that is a result of, out of disrespect, defecating, urinating, or spitting in water, through how many of the six kinds of originations of offenses does it originate? It originates in one way: from body and mind, not from speech. 

\scendsutta{The originations of each offense, the fifth, are finished. }

%
\chapter*{{\suttatitleacronym Pvr 1.14}{\suttatitletranslation The legal issues to which each offense belongs }{\suttatitleroot Adhikaraṇavāra}}
\addcontentsline{toc}{chapter}{\tocacronym{Pvr 1.14} \toctranslation{The legal issues to which each offense belongs } \tocroot{Adhikaraṇavāra}}
\markboth{The legal issues to which each offense belongs }{Adhikaraṇavāra}
\extramarks{Pvr 1.14}{Pvr 1.14}

When\marginnote{1.1} it comes to the offenses that are a result of having sexual intercourse, to which of the four kinds of legal issues do they belong? They belong to legal issues arising from an offense. … 

When\marginnote{2.1} it comes to the offense that is a result of, out of disrespect, defecating, urinating, or spitting in water, to which of the four kinds of legal issues does it belong? It belongs to legal issues arising from an offense. 

\scendsutta{The legal issues to which each offense belongs, the sixth, are finished. }

%
\chapter*{{\suttatitleacronym Pvr 1.15}{\suttatitletranslation How each offense is settled }{\suttatitleroot Samathavāra}}
\addcontentsline{toc}{chapter}{\tocacronym{Pvr 1.15} \toctranslation{How each offense is settled } \tocroot{Samathavāra}}
\markboth{How each offense is settled }{Samathavāra}
\extramarks{Pvr 1.15}{Pvr 1.15}

When\marginnote{1.1} it comes to the offenses that are a result of having sexual intercourse, through how many of the seven principles for settling legal issues are they settled? Through three of them: they may be settled by resolution face-to-face and by acting according to what has been admitted; or they may be settled by resolution face-to-face and by covering over as if with grass. … 

When\marginnote{2.1} it comes to the offense that is a result of, out of disrespect, defecating, urinating, or spitting in water, through how many of the seven principles for settling legal issues is it settled? Through three of them: it may be settled by resolution face-to-face and by acting according to what has been admitted; or it may be settled by resolution face-to-face and by covering over as if with grass. 

\scendsutta{How each offense is settled, the seventh, is finished. }

%
\chapter*{{\suttatitleacronym Pvr 1.16}{\suttatitletranslation Summary of the previous six sections }{\suttatitleroot Samuccayavāra}}
\addcontentsline{toc}{chapter}{\tocacronym{Pvr 1.16} \toctranslation{Summary of the previous six sections } \tocroot{Samuccayavāra}}
\markboth{Summary of the previous six sections }{Samuccayavāra}
\extramarks{Pvr 1.16}{Pvr 1.16}

As\marginnote{1.1} a result of having sexual intercourse, how many kinds of offenses does one commit? One commits four kinds of offenses: when one has sexual intercourse with an undecomposed corpse, one commits an offense entailing expulsion; when one has sexual intercourse with a mostly decomposed corpse, one commits a serious offense; when one inserts one’s penis into a wide open mouth without touching it, one commits an offense of wrong conduct; when one uses a dildo, one commits an offense entailing confession. 

When\marginnote{1.7} it comes to these offenses, to how many of the four kinds of failure do they belong? In how many of the seven classes of offenses are they found? Through how many of the six kinds of originations of offenses do they originate? To which of the four kinds of legal issues do they belong? Through how many of the seven principles for settling legal issues are they settled? 

They\marginnote{1.13} belong to two kinds of failure: they may be failure in morality; they may be failure in conduct. They are found in four classes of offenses: they may be in the class of offenses entailing expulsion; they may be in the class of serious offenses; they may be in the class of offenses entailing confession; they may be in the class of offenses of wrong conduct. They originate in one way: from body and mind, not from speech. They belong to legal issues arising from an offense. They are settled through three principles: they may be settled by resolution face-to-face and by acting according to what has been admitted; or they may be settled by resolution face-to-face and by covering over as if with grass. … 

As\marginnote{2.1} a result of, out of disrespect, defecating, urinating, or spitting in water, how many kinds of offenses does one commit? One commits one kind of offense: an offense of wrong conduct. 

When\marginnote{2.4} it comes to this offense, to how many of the four kinds of failure does it belong? In how many of the seven classes of offenses is it found? Through how many of the six kinds of originations of offenses does it originate? To which of the four kinds of legal issues does it belong? Through how many of the seven principles for settling legal issues is it settled? 

It\marginnote{2.10} belongs to one kind of failure: failure in conduct. It is found in one class of offenses: in the class of offenses of wrong conduct. It originates in one way: from body and mind, not from speech. It belongs to legal issues arising from an offense. It is settled through three principles: it may be settled by resolution face-to-face and by acting according to what has been admitted; or it may be settled by resolution face-to-face and by covering over as if with grass. 

\scendsutta{The summary of the previous six sections, the eighth, is finished. }

\scend{The eight sections on “as a result of” are finished. }

\scend{The sixteen great sections in The Great Analysis are finished. }

\scend{The great section of the Monks’ Analysis is finished. }

%
\chapter*{{\suttatitleacronym Pvr 2.1}{\suttatitletranslation Questions and answers on the nuns’ Pātimokkha rules and their analysis }{\suttatitleroot Katthapaññattivāra}}
\addcontentsline{toc}{chapter}{\tocacronym{Pvr 2.1} \toctranslation{Questions and answers on the nuns’ Pātimokkha rules and their analysis } \tocroot{Katthapaññattivāra}}
\markboth{Questions and answers on the nuns’ Pātimokkha rules and their analysis }{Katthapaññattivāra}
\extramarks{Pvr 2.1}{Pvr 2.1}

\section*{The chapter on offenses entailing expulsion }

“The\marginnote{1.1} nuns’ fifth offense entailing expulsion was laid down by the Buddha who knows and sees, the Perfected One, the fully Awakened One. Where was it laid down? Whom is it about? What is it about? Is there a rule, an addition to the rule, an unprompted rule? Is it a rule that applies everywhere or in a particular place? Is it a rule that the monks and nuns have in common or not in common? Is it a rule for one Sangha or for both? In which of the four ways of reciting the Monastic Code is it contained and included? In which recitation is it included? To which of the four kinds of failure does it belong? To which of the seven classes of offenses does it belong? Through how many of the six kinds of originations of offenses does it originate? To which of the four kinds of legal issues does it belong? Through how many of the seven principles for settling legal issues is it settled? What is the Monastic Law there? What is concerned with the Monastic Law there? What is the Monastic Code there? What is concerned with the Monastic Code there? What is failure? What is success? What is the practice? For how many reasons did the Buddha lay down the nun’s fifth offense entailing expulsion? Who are those who train? Who have finished the training? Established in what? Who master it? Whose pronouncement was it? Who handed it down?” 

“The\marginnote{2.1} nuns’ fifth offense entailing expulsion was laid down by the Buddha who knows and sees, the Perfected One, the fully Awakened One. Where was it laid down?” At \textsanskrit{Sāvatthī}. “Whom is it about?” The nun \textsanskrit{Sundarīnandā}. “What is it about?” The lustful nun \textsanskrit{Sundarīnandā} consenting to a lustful man making physical contact with her. “Is there a rule, an addition to the rule, an unprompted rule?” There is one rule. There is no addition to the rule. There is no unprompted rule. “Is it a rule that applies everywhere or in a particular place?” Everywhere. “Is it a rule that the monks and nuns have in common or not in common?” Not in common. “Is it a rule for one Sangha or for both?” For one. “In which of the four ways of reciting the Monastic Code is it contained and included?” In the introduction.\footnote{Sp 5.2: \textit{\textsanskrit{Nidānogadhanti} “yassa \textsanskrit{siyā} \textsanskrit{āpatti} so \textsanskrit{āvikareyyā}”ti ettha \textsanskrit{sabbāpattīnaṁ} \textsanskrit{anupaviṭṭhattā} \textsanskrit{nidānogadhaṁ}; \textsanskrit{nidāne} \textsanskrit{anupaviṭṭhanti} attho}, “\textit{\textsanskrit{Nidānogadha}}: it is contained in the introduction because of the entry here of all offenses: ‘Anyone who has committed an offense should reveal it.’ The meaning is they are entered in the introduction.” } “In which recitation is it included?” In the second recitation. “To which of the four kinds of failure does it belong?” Failure in morality. “To which of the seven classes of offenses does it belong?” The class of offenses entailing expulsion. “Through how many of the six kinds of originations of offenses does it originate?” It originates in one way: from body and mind, not from speech. “To which of the four kinds of legal issues does it belong?” Legal issues arising from an offense. “Through how many of the seven principles for settling legal issues is it settled?” Through two of them: by resolution face-to-face and by acting according to what has been admitted. “What is the Monastic Law there? What is concerned with the Monastic Law there?” The rule is the Monastic Law. Its analysis is concerned with the Monastic Law. “What is the Monastic Code there? What is concerned with the Monastic Code there?” The rule is the Monastic Code. Its analysis is concerned with the Monastic Code. “What is failure?” Lack of restraint. “What is success?” Restraint. “What is the practice?” Thinking, “I won’t do such a thing,” one undertakes to train in the training rules for life. “For how many reasons did the Buddha lay down the nun’s fifth offense entailing expulsion?” He laid it down for the following ten reasons: for the well-being of the Sangha, for the comfort of the Sangha, for the restraint of bad nuns, for the ease of good nuns, for the restraint of the corruptions relating to the present life, for the restraint of the corruptions relating to future lives, to give rise to confidence in those without it, to increase the confidence of those who have it, for the longevity of the true Teaching, and for supporting the training. “Who are those who train?” They are the trainees and the good ordinary people. “Who have finished the training?” The perfected ones. “Established in what?” In fondness for the training. “Who master it?” Those who learn it. “Whose pronouncement was it?” It was the pronouncement of the Buddha, the Perfected One, the fully Awakened One. “Who handed it down?” The lineage: 

\begin{verse}%
“\textsanskrit{Upāli}\marginnote{3.1} and \textsanskrit{Dāsaka}, \\
\textsanskrit{Soṇaka} and so Siggava; \\
With Moggaliputta as the fifth—\\
These were in India, the land named after the glorious rose apple. 

Then\marginnote{4.1} Mahinda, \textsanskrit{Iṭṭiya}, \\
Uttiya and so Sambala; \\
And the wise one named Bhadda: 

These\marginnote{5.1} mighty beings of great wisdom, \\
Came here from India; \\
They taught the Collection on Monastic Law, \\
In Sri Lanka. 

And\marginnote{6.1} the five Collections of Discourses, \\
And the seven works of philosophy; \\
Then \textsanskrit{Ariṭṭha} the discerning, \\
And the wise Tissadatta. 

The\marginnote{7.1} confident \textsanskrit{Kālasumana}, \\
And the senior monk named \textsanskrit{Dīgha}; \\
And the wise \textsanskrit{Dīghasumana}. 

Another\marginnote{8.1} \textsanskrit{Kālasumana}, \\
And the senior monk \textsanskrit{Nāga}, Buddharakkhita; \\
And the discerning senior monk Tissa, \\
And the wise senior monk Deva. 

Another\marginnote{9.1} discerning Sumana, \\
Confident in the Monastic Law; \\
The learned \textsanskrit{Cūlanāga}, \\
Invincible, like an elephant. 

And\marginnote{10.1} the one named \textsanskrit{Dhammapālita}, \\
\textsanskrit{Rohaṇa}, venerated as a saint; \\
His student Khema of great wisdom, \\
A master of the three Collections. 

Like\marginnote{11.1} the king of the stars on the island, \\
He outshone others in his wisdom; \\
And the discerning Upatissa, \\
Phussadeva the great speaker. 

Another\marginnote{12.1} discerning Sumana, \\
The learned one named Puppha; \\
\textsanskrit{Mahāsīva} the great speaker, \\
Skilled in the entire Collection. 

Another\marginnote{13.1} discerning \textsanskrit{Upāli}, \\
Confident in the Monastic Law; \\
\textsanskrit{Mahānāga} of great wisdom, \\
Skilled in the tradition of the true Teaching. 

Another\marginnote{14.1} discerning Abhaya, \\
Skilled in the entire Collection; \\
And the discerning senior monk Tissa, \\
Confident in the Monastic Law. 

His\marginnote{15.1} student of great wisdom, \\
The learned one named Puppha; \\
Guarding Buddhism, \\
He established himself in India. 

And\marginnote{16.1} the discerning \textsanskrit{Cūlābhaya}, \\
Confident in the Monastic Law; \\
And the discerning senior monk Tissa, \\
Skilled in the tradition of the true Teaching. 

And\marginnote{17.1} the discerning \textsanskrit{Cūladeva}, \\
Confident in the Monastic Law; \\
And the discerning senior monk Siva, \\
Skilled in the entire Monastic Law—

These\marginnote{18.1} mighty beings of great wisdom, \\
Knowers of the Monastic Law and skilled in the path; \\
Proclaimed the Collection of Monastic Law, \\
On the island of Sri Lanka.” 

%
\end{verse}

“The\marginnote{19.1} nuns’ sixth offense entailing expulsion was laid down by the Buddha who knows and sees, the Perfected One, the fully Awakened One. Where was it laid down?” At \textsanskrit{Sāvatthī}. “Whom is it about?” The nun \textsanskrit{Thullanandā}. “What is it about?” The nun \textsanskrit{Thullanandā}, knowing that a nun had committed an offense entailing expulsion, neither confronting her herself nor telling the community. There is one rule. Of the six kinds of originations of offenses, it originates in one way: from body, speech, and mind. … 

“There\marginnote{20.1} is the nun’s seventh offense entailing expulsion. Where was it laid down?” At \textsanskrit{Sāvatthī}. “Whom is it about?” The nun \textsanskrit{Thullanandā}. “What is it about?” The nun \textsanskrit{Thullanandā} taking sides with the monk \textsanskrit{Ariṭṭha}, an ex-vulture-killer, who had been ejected by a unanimous Sangha. There is one rule. Of the six kinds of originations of offenses, it originates in one way: through abandoning one’s duty. … 

“There\marginnote{21.1} is the nun’s eighth offense entailing expulsion. Where was it laid down?” At \textsanskrit{Sāvatthī}. “Whom is it about?” The nuns from the group of six. “What is it about?” The nuns from the group of six fulfilling the eight parts. There is one rule. Of the six kinds of originations of offenses, it originates in one way: through abandoning one’s duty. … 

\scend{The eight offenses entailing expulsion are finished. }

\scuddanaintro{This is the summary: }

\begin{scuddana}%
“Sexual\marginnote{24.1} intercourse, and stealing, \\
Person, super; \\
Physical contact, conceals, \\
Ejected, eight parts—\\
The Great Hero laid down, \\
The definitive grounds for cutting off.” 

%
\end{scuddana}

\section*{2. The chapter on offenses entailing suspension }

“The\marginnote{25.1} offense entailing suspension for a litigious nun initiating a lawsuit was laid down by the Buddha who knows and sees, the Perfected One, the fully Awakened One. Where was it laid down? Whom is it about? What is it about? … Who handed it down?” 

“The\marginnote{26.1} offense entailing suspension for a litigious nun initiating a lawsuit was laid down by the Buddha who knows and sees, the Perfected One, the fully Awakened One. Where was it laid down?” At \textsanskrit{Sāvatthī}. “Whom is it about?” The nun \textsanskrit{Thullanandā}. “What is it about?” The nun \textsanskrit{Thullanandā} taking legal action. “Is there a rule, an addition to the rule, an unprompted rule?” There is one rule. There is no addition to the rule. There is no unprompted rule. “Is it a rule that applies everywhere or in a particular place?” Everywhere. “Is it a rule that the monks and nuns have in common or not in common?” Not in common. “Is it a rule for one Sangha or for both?” For one. “In which of the four ways of reciting the Monastic Code is it contained and included?” In the introduction. “In which recitation is it included?” In the third recitation. “To which of the four kinds of failure does it belong?” Failure in morality. “To which of the seven classes of offenses does it belong?” The class of offenses entailing suspension. “Through how many of the six kinds of originations of offenses does it originate?” It originates in two ways: from body and speech, not from mind; or from body, speech, and mind. … “Who handed it down?” The lineage: 

\begin{verse}%
“\textsanskrit{Upāli}\marginnote{27.1} and \textsanskrit{Dāsaka}, \\
\textsanskrit{Soṇaka} and so Siggava; \\
With Moggaliputta as the fifth—\\
These were in India, the land named after the glorious rose apple. 

…\marginnote{28.1} 

These\marginnote{29.1} mighty beings of great wisdom, \\
Knowers of the Monastic Law and skilled in the path; \\
Proclaimed the Collection of Monastic Law, \\
On the island of Sri Lanka.” 

%
\end{verse}

“There\marginnote{30.1} is an offense entailing suspension for giving the full admission to a female criminal. Where was it laid down?” At \textsanskrit{Sāvatthī}. “Whom is it about?” The nun \textsanskrit{Thullanandā}. “What is it about?” The nun \textsanskrit{Thullanandā} giving the full admission to a female criminal. There is one rule. Of the six kinds of originations of offenses, it originates in two ways: from speech and mind, not from body; or from body, speech, and mind. … 

“There\marginnote{31.1} is an offense entailing suspension for walking to the next inhabited area by oneself. Where was it laid down?” At \textsanskrit{Sāvatthī}. “Whom is it about?” A certain nun. “What is it about?” A certain nun walking to the next inhabited area by herself. There is one rule. There are three additions to the rule. Of the six kinds of originations of offenses, it originates in one way: … (as in the first offense entailing expulsion) … 

“There\marginnote{32.1} is an offense entailing suspension for readmitting a nun who had been ejected by a unanimous Sangha in accordance with the Teaching, the Monastic Law, and the Teacher’s instruction, without first getting permission from the Sangha that did the legal procedure and without the consent of the community. Where was it laid down?” At \textsanskrit{Sāvatthī}. “Whom is it about?” The nun \textsanskrit{Thullanandā}. “What is it about?” The nun \textsanskrit{Thullanandā} readmitting a nun who had been ejected by a unanimous Sangha in accordance with the Teaching, the Monastic Law, and the Teacher’s instruction, without first getting permission from the Sangha that did the legal procedure and without the consent of the community. There is one rule. Of the six kinds of originations of offenses, it originates in one way: through abandoning one’s duty. … 

“There\marginnote{33.1} is an offense entailing suspension for a lustful nun eating fresh or cooked food after receiving it directly from a lustful man. Where was it laid down?” At \textsanskrit{Sāvatthī}. “Whom is it about?” The nun \textsanskrit{Sundarīnandā}. “What is it about?” The nun \textsanskrit{Sundarīnandā}, being lustful, receiving food directly from a lustful man. There is one rule. Of the six kinds of originations of offenses, it originates in one way: … (as in the first offense entailing expulsion) … 

“There\marginnote{34.1} is an offense entailing suspension for urging a nun on, saying, ‘Venerable, what can this man do to you, whether he has lust or not, if you’re without? Go on, venerable, receive it with your own hands and then eat whatever fresh or cooked food he gives you.’ Where was it laid down?” At \textsanskrit{Sāvatthī}. “Whom is it about?” A certain nun. “What is it about?” A certain nun urging a nun on, saying, “Venerable, what can this man do to you, whether he has lust or not, if you’re without? Go on, venerable, receive it with your own hands and then eat whatever fresh or cooked food he gives you.” There is one rule. Of the six kinds of originations of offenses, it originates in three ways: … 

“There\marginnote{35.1} is an offense entailing suspension for an angry nun not to stop when pressed for the third time. Where was it laid down?” At \textsanskrit{Sāvatthī}. “Whom is it about?” The nun \textsanskrit{Caṇḍakālī}. “What is it about?” The nun \textsanskrit{Caṇḍakālī} saying in anger, “I renounce the Buddha, I renounce the Teaching, I renounce the Sangha, I renounce the training!” There is one rule. Of the six kinds of originations of offenses, it originates in one way: through abandoning one’s duty. … 

“There\marginnote{36.1} is an offense entailing suspension for a nun who has lost a legal case not to stop when pressed for the third time. Where was it laid down?” At \textsanskrit{Sāvatthī}. “Whom is it about?” The nun \textsanskrit{Caṇḍakālī}. “What is it about?” The nun \textsanskrit{Caṇḍakālī}, who had lost a legal case, saying in anger, “The nuns are acting out of favoritism, ill will, confusion, and fear.” There is one rule. Of the six kinds of originations of offenses, it originates in one way: through abandoning one’s duty. … 

“There\marginnote{37.1} is an offense entailing suspension for socializing nuns not to stop when pressed for the third time. Where was it laid down?” At \textsanskrit{Sāvatthī}. “Whom is it about?” A number of nuns. “What is it about?” A number of nuns socializing. There is one rule. Of the six kinds of originations of offenses, it originates in one way: through abandoning one’s duty. … 

“There\marginnote{38.1} is an offense entailing suspension for urging nuns in this way: ‘Venerables, you should socialize. Don’t live separately,’ and then not to stop when pressed for the third time. Where was it laid down?” At \textsanskrit{Sāvatthī}. “Whom is it about?” The nun \textsanskrit{Thullanandā}. “What is it about?” The nun \textsanskrit{Thullanandā} urging the nuns on, saying, “Venerables, you should socialize. Don’t live separately.” There is one rule. Of the six kinds of originations of offenses, it originates in one way: through abandoning one’s duty. … 

\scend{The ten rules entailing suspension are finished. }

\scuddanaintro{This is the summary: }

\begin{scuddana}%
“Litigious,\marginnote{41.1} a criminal, the next inhabited area, \\
Ejected, and with fresh food; \\
What to you, angry, a legal issue, \\
Socializing, the same method: those are the ten.” 

%
\end{scuddana}

\section*{3. The chapter on relinquishment }

“The\marginnote{42.1} offense entailing relinquishment and confession for collecting almsbowls was laid down by the Buddha who knows and sees, the Perfected One, the fully Awakened One. Where was it laid down?” At \textsanskrit{Sāvatthī}. “Whom is it about?” The nuns from the group of six. “What is it about?” The nuns from the group of six collecting almsbowls. There is one rule. Of the six kinds of originations of offenses, it originates in two ways: … (as in the rule on the robe season) … 

“There\marginnote{43.1} is an offense entailing relinquishment and confession for determining out-of-season robe-cloth as ‘in-season’, and then distributing it. Where was it laid down?” At \textsanskrit{Sāvatthī}. “Whom is it about?” The nun \textsanskrit{Thullanandā}. “What is it about?” The nun \textsanskrit{Thullanandā} determining out-of-season robe-cloth as “in-season”, and then distributing it. There is one rule. Of the six kinds of originations of offenses, it originates in three ways: … 

“There\marginnote{44.1} is an offense entailing relinquishment and confession for trading robes with a nun and then taking it back. Where was it laid down?” At \textsanskrit{Sāvatthī}. “Whom is it about?” The nun \textsanskrit{Thullanandā}. “What is it about?” The nun \textsanskrit{Thullanandā} trading a robe with a nun and then taking it back. There is one rule. Of the six kinds of originations of offenses, it originates in three ways: … 

“There\marginnote{45.1} is an offense entailing relinquishment and confession for asking for one thing and then for something else. Where was it laid down?” At \textsanskrit{Sāvatthī}. “Whom is it about?” The nun \textsanskrit{Thullanandā}. “What is it about?” The nun \textsanskrit{Thullanandā} asking for one thing and then for something else. There is one rule. Of the six kinds of originations of offenses, it originates in six ways: … 

“There\marginnote{46.1} is an offense entailing relinquishment and confession for getting one thing in exchange and then something else. Where was it laid down?” At \textsanskrit{Sāvatthī}. “Whom is it about?” The nun \textsanskrit{Thullanandā}. “What is it about?” The nun \textsanskrit{Thullanandā} getting one thing in exchange and then something else. There is one rule. Of the six kinds of originations of offenses, it originates in six ways: … 

“There\marginnote{47.1} is an offense entailing relinquishment and confession for exchanging for something else a requisite belonging to the Sangha that is designated for a specific purpose. Where was it laid down?” At \textsanskrit{Sāvatthī}. “Whom is it about?” A number of nuns. “What is it about?” A number of nuns exchanging for something else a requisite belonging to the Sangha that was designated for a specific purpose. There is one rule. Of the six kinds of originations of offenses, it originates in six ways: … 

“There\marginnote{48.1} is an offense entailing relinquishment and confession for exchanging for something else a requisite belonging to the Sangha that is designated for a specific purpose and was asked for. Where was it laid down?” At \textsanskrit{Sāvatthī}. “Whom is it about?” A number of nuns. “What is it about?” A number of nuns exchanging for something else a requisite belonging to the Sangha that was designated for a specific purpose and was asked for. There is one rule. Of the six kinds of originations of offenses, it originates in six ways: … 

“There\marginnote{49.1} is an offense entailing relinquishment and confession for exchanging for something else a collective requisite that is designated for a specific purpose. Where was it laid down?” At \textsanskrit{Sāvatthī}. “Whom is it about?” A number of nuns. “What is it about?” A number of nuns exchanging for something else a collective requisite that was designated for a specific purpose. There is one rule. Of the six kinds of originations of offenses, it originates in six ways: … 

“There\marginnote{50.1} is an offense entailing relinquishment and confession for exchanging for something else a collective requisite that is designated for a specific purpose and was asked for. Where was it laid down?” At \textsanskrit{Sāvatthī}. “Whom is it about?” A number of nuns. “What is it about?” A number of nuns exchanging for something else a collective requisite that was designated for a specific purpose and was asked for. There is one rule. Of the six kinds of originations of offenses, it originates in six ways: … 

“There\marginnote{51.1} is an offense entailing relinquishment and confession for exchanging for something else a personal requisite that is designated for a specific purpose and was asked for. Where was it laid down?” At \textsanskrit{Sāvatthī}. “Whom is it about?” The nun \textsanskrit{Thullanandā}. “What is it about?” The nun \textsanskrit{Thullanandā} exchanging for something else a personal requisite that was designated for a specific purpose and was asked for. There is one rule. Of the six kinds of originations of offenses, it originates in six ways: … 

“There\marginnote{52.1} is an offense entailing relinquishment and confession for getting in exchange a heavy cloak worth more than four \textit{\textsanskrit{kaṁsa}} coins. Where was it laid down?” At \textsanskrit{Sāvatthī}. “Whom is it about?” The nun \textsanskrit{Thullanandā}. “What is it about?” The nun \textsanskrit{Thullanandā} asking the king for a woolen cloak. There is one rule. Of the six kinds of originations of offenses, it originates in six ways: … 

“There\marginnote{53.1} is an offense entailing relinquishment and confession for getting in exchange a light cloak worth more than two-and-a-half \textit{\textsanskrit{kaṁsa}} coins. Where was it laid down?” At \textsanskrit{Sāvatthī}. “Whom is it about?” The nun \textsanskrit{Thullanandā}. “What is it about?” The nun \textsanskrit{Thullanandā} asking the king for a linen cloak. There is one rule. Of the six kinds of originations of offenses, it originates in six ways: … 

\scend{The twelve rules on relinquishment and confession are finished. }

\scuddanaintro{This is the summary: }

\begin{scuddana}%
“Bowl,\marginnote{56.1} out-of-season as in-season, \\
And should trade, should ask; \\
Getting in exchange, a specific purpose, \\
And belonging to the Sangha, collective; \\
Asked for, personal, \\
Four \textit{\textsanskrit{kaṁsa}} coins, two-and-a-half.” 

%
\end{scuddana}

\section*{4. The chapter on offenses entailing confession }

\subsection*{The subchapter on garlic }

“The\marginnote{57.1} offense entailing confession for eating garlic was laid down by the Buddha who knows and sees, the Perfected One, the fully Awakened One. Where was it laid down?” At \textsanskrit{Sāvatthī}. “Whom is it about?” The nun \textsanskrit{Thullanandā}. “What is it about?” The nun \textsanskrit{Thullanandā} taking garlic without any sense of moderation. There is one rule. Of the six kinds of originations of offenses, it originates in two ways: … (as in the rule on wool) … 

“There\marginnote{58.1} is an offense entailing confession for removing hair from the private parts. Where was it laid down?” At \textsanskrit{Sāvatthī}. “Whom is it about?” The nuns from the group of six. “What is it about?” The nuns from the group of six removing hair from their private parts. There is one rule. Of the six kinds of originations of offenses, it originates in four ways: … 

“There\marginnote{59.1} is an offense entailing confession for slapping the genitals with the palm of the hand. Where was it laid down?” At \textsanskrit{Sāvatthī}. “Whom is it about?” Two nuns. “What is it about?” Two nuns slapping their genitals with the palms of their hands. There is one rule. Of the six kinds of originations of offenses, it originates in one way: … (as in the first offense entailing expulsion) … 

“There\marginnote{60.1} is an offense entailing confession for using a dildo. Where was it laid down?” At \textsanskrit{Sāvatthī}. “Whom is it about?” A certain nun. “What is it about?” A certain nun using a dildo. There is one rule. Of the six kinds of originations of offenses, it originates in one way: … (as in the first offense entailing expulsion) … 

“There\marginnote{61.1} is an offense entailing confession for cleaning oneself with water by inserting more than two finger joints. Where was it laid down?” In the Sakyan country. “Whom is it about?” A certain nun. “What is it about?” A certain nun cleaning herself too deeply with water. There is one rule. Of the six kinds of originations of offenses, it originates in one way: … (as in the first offense entailing expulsion) … 

“There\marginnote{62.1} is an offense entailing confession for attending on a monk who is eating with drinking water or a fan. Where was it laid down?” At \textsanskrit{Sāvatthī}. “Whom is it about?” A certain nun. “What is it about?” A certain nun attending on a monk who is eating with drinking water or a fan. There is one rule. Of the six kinds of originations of offenses, it originates in two ways: … (as in the rule on wool) … 

“There\marginnote{63.1} is an offense entailing confession for asking for raw grain and then eating it. Where was it laid down?” At \textsanskrit{Sāvatthī}. “Whom is it about?” A number of nuns. “What is it about?” A number of nuns asking for raw grain and then eating it. There is one rule. Of the six kinds of originations of offenses, it originates in four ways: … 

“There\marginnote{64.1} is an offense entailing confession for disposing of feces, urine, trash, or food scraps over a wall. Where was it laid down?” At \textsanskrit{Sāvatthī}. “Whom is it about?” A certain nun. “What is it about?” A certain nun disposing of feces over a wall. There is one rule. Of the six kinds of originations of offenses, it originates in six ways: … 

“There\marginnote{65.1} is an offense entailing confession for disposing of feces, urine, trash, or food scraps on cultivated plants. Where was it laid down?” At \textsanskrit{Sāvatthī}. “Whom is it about?” A number of nuns. “What is it about?” A number of nuns disposing of feces, urine, trash, and food scraps on cultivated plants. There is one rule. Of the six kinds of originations of offenses, it originates in six ways: … 

“There\marginnote{66.1} is an offense entailing confession for going to see dancing, singing, or music. Where was it laid down?” At \textsanskrit{Rājagaha}. “Whom is it about?” The nuns from the group of six. “What is it about?” The nuns from the group of six going to see dancing, singing, and music. There is one rule. Of the six kinds of originations of offenses, it originates in two ways: … (as in the rule on wool) … 

\scendvagga{The first subchapter on garlic is finished. }

\subsection*{The subchapter on the dark of the night }

“There\marginnote{68.1} is an offense entailing confession for standing alone with a man in the dark of the night without a lamp. Where was it laid down?” At \textsanskrit{Sāvatthī}. “Whom is it about?” A certain nun. “What is it about?” A certain nun standing alone with a man in the dark of the night without a lamp. There is one rule. Of the six kinds of originations of offenses, it originates in two ways: … (as in the rule on a group of traveling thieves) … 

“There\marginnote{69.1} is an offense entailing confession for standing alone with a man in a concealed place. Where was it laid down?” At \textsanskrit{Sāvatthī}. “Whom is it about?” A certain nun. “What is it about?” A certain nun standing alone with a man in a concealed place. There is one rule. Of the six kinds of originations of offenses, it originates in two ways: … (as in the rule on a group of traveling thieves) … 

“There\marginnote{70.1} is an offense entailing confession for standing alone with a man out in the open. Where was it laid down?” At \textsanskrit{Sāvatthī}. “Whom is it about?” A certain nun. “What is it about?” A certain nun standing alone with a man out in the open. There is one rule. Of the six kinds of originations of offenses, it originates in two ways: … (as in the rule on a group of traveling thieves) … 

“There\marginnote{71.1} is an offense entailing confession for standing alone with a man on a street, in a cul-de-sac, or at an intersection. Where was it laid down?” At \textsanskrit{Sāvatthī}. “Whom is it about?” The nun \textsanskrit{Thullanandā}. “What is it about?” The nun \textsanskrit{Thullanandā} standing alone with a man on a street, in a cul-de-sac, and at an intersection. There is one rule. Of the six kinds of originations of offenses, it originates in two ways: … (as in the rule on a group of traveling thieves) … 

“There\marginnote{72.1} is an offense entailing confession for visiting families before the meal, sitting down on a seat, and then departing without informing the owners. Where was it laid down?” At \textsanskrit{Sāvatthī}. “Whom is it about?” A certain nun. “What is it about?” A certain nun visiting families before the meal, sitting down on a seat, and then departing without informing the owners. There is one rule. Of the six kinds of originations of offenses, it originates in two ways: … (as in the rule on the robe season) … 

“There\marginnote{73.1} is an offense entailing confession for visiting families after the meal and then sitting down on a seat without asking permission of the owners. Where was it laid down?” At \textsanskrit{Sāvatthī}. “Whom is it about?” The nun \textsanskrit{Thullanandā}. “What is it about?” The nun \textsanskrit{Thullanandā} visiting families after the meal and then sitting down on a seat without asking permission of the owners. There is one rule. Of the six kinds of originations of offenses, it originates in two ways: … (as in the rule on the robe season) … 

“There\marginnote{74.1} is an offense entailing confession for visiting families at the wrong time, putting out bedding without asking permission of the owners, or having it put out, and then sitting down. Where was it laid down?” At \textsanskrit{Sāvatthī}. “Whom is it about?” A number of nuns. “What is it about?” A number of nuns visiting families at the wrong time, putting out bedding without asking permission of the owners, and then sitting down. There is one rule. Of the six kinds of originations of offenses, it originates in two ways: … (as in the rule on the robe season) … 

“There\marginnote{75.1} is an offense entailing confession for complaining about someone because of a misunderstanding and a lack of proper reflection. Where was it laid down?” At \textsanskrit{Sāvatthī}. “Whom is it about?” A certain nun. “What is it about?” A certain nun complaining about someone because of a misunderstanding and a lack of proper reflection. There is one rule. Of the six kinds of originations of offenses, it originates in three ways: … 

“There\marginnote{76.1} is an offense entailing confession for referring to hell or the spiritual life to curse oneself or someone else. Where was it laid down?” At \textsanskrit{Sāvatthī}. “Whom is it about?” The nun \textsanskrit{Caṇḍakālī}. “What is it about?” The nun \textsanskrit{Caṇḍakālī} referring to hell and the spiritual life to curse herself and someone else. There is one rule. Of the six kinds of originations of offenses, it originates in three ways: … 

“There\marginnote{77.1} is an offense entailing confession for crying after repeatedly beating oneself. Where was it laid down?” At \textsanskrit{Sāvatthī}. “Whom is it about?” The nun \textsanskrit{Caṇḍakālī}. “What is it about?” The nun \textsanskrit{Caṇḍakālī} crying after repeatedly beating herself. There is one rule. Of the six kinds of originations of offenses, it originates in one way: through abandoning one’s duty. … 

\scendvagga{The second subchapter on the dark of the night is finished. }

\subsection*{The subchapter on bathing }

“There\marginnote{79.1} is an offense entailing confession for bathing naked. Where was it laid down?” At \textsanskrit{Sāvatthī}. “Whom is it about?” A number of nuns. “What is it about?” A number of nuns bathing naked. There is one rule. Of the six kinds of originations of offenses, it originates in two ways: … (as in the rule on wool) … 

“There\marginnote{80.1} is an offense entailing confession for making a bathing robe that exceeds the right size. Where was it laid down?” At \textsanskrit{Sāvatthī}. “Whom is it about?” The nuns from the group of six. “What is it about?” The nuns from the group of six wearing bathing robes that exceeded the right size. There is one rule. Of the six kinds of originations of offenses, it originates in six ways: … 

“There\marginnote{81.1} is an offense entailing confession for unstitching a nun’s robe, or having it unstitched, and then neither sewing it oneself nor making any effort to have someone else sew it. Where was it laid down?” At \textsanskrit{Sāvatthī}. “Whom is it about?” The nun \textsanskrit{Thullanandā}. “What is it about?” The nun \textsanskrit{Thullanandā} unstitching a nun’s robe and then neither sewing it herself nor making any effort to have someone else sew it. There is one rule. Of the six kinds of originations of offenses, it originates in one way: through abandoning one’s duty. … 

“There\marginnote{82.1} is an offense entailing confession for not moving one’s robes for more than five days. Where was it laid down?” At \textsanskrit{Sāvatthī}. “Whom is it about?” A number of nuns. “What is it about?” A number of nuns storing a robe with other nuns and then leaving to wander the country in a sarong and an upper robe. There is one rule. Of the six kinds of originations of offenses, it originates in two ways: … (as in the rule on the robe season) … 

“There\marginnote{83.1} is an offense entailing confession for wearing a robe taken on loan. Where was it laid down?” At \textsanskrit{Sāvatthī}. “Whom is it about?” A certain nun. “What is it about?” A certain nun putting on another nun’s robe without asking permission. There is one rule. Of the six kinds of originations of offenses, it originates in two ways: … (as in the rule on the robe season) … 

“There\marginnote{84.1} is an offense entailing confession for creating an obstacle for the community to get robe-cloth. Where was it laid down?” At \textsanskrit{Sāvatthī}. “Whom is it about?” The nun \textsanskrit{Thullanandā}. “What is it about?” The nun \textsanskrit{Thullanandā} creating an obstacle for a community to get robe-cloth. There is one rule. Of the six kinds of originations of offenses, it originates in three ways: … 

“There\marginnote{85.1} is an offense entailing confession for blocking a legitimate distribution of robe-cloth. Where was it laid down?” At \textsanskrit{Sāvatthī}. “Whom is it about?” The nun \textsanskrit{Thullanandā}. “What is it about?” The nun \textsanskrit{Thullanandā} blocking a legitimate distribution of robe-cloth. There is one rule. Of the six kinds of originations of offenses, it originates in three ways: … 

“There\marginnote{86.1} is an offense entailing confession for giving a monastic robe to a householder, a male wanderer, or a female wanderer. Where was it laid down?” At \textsanskrit{Sāvatthī}. “Whom is it about?” The nun \textsanskrit{Thullanandā}. “What is it about?” The nun \textsanskrit{Thullanandā} giving a monastic robe to a householder. There is one rule. Of the six kinds of originations of offenses, it originates in six ways: … 

“There\marginnote{87.1} is an offense entailing confession for letting the robe season expire because of an uncertain expectation of robe-cloth. Where was it laid down?” At \textsanskrit{Sāvatthī}. “Whom is it about?” The nun \textsanskrit{Thullanandā}. “What is it about?” The nun \textsanskrit{Thullanandā} letting the robe season expire because of an uncertain expectation of robe-cloth. There is one rule. Of the six kinds of originations of offenses, it originates in three ways: … 

“There\marginnote{88.1} is an offense entailing confession for blocking a legitimate ending of the robe season. Where was it laid down?” At \textsanskrit{Sāvatthī}. “Whom is it about?” The nun \textsanskrit{Thullanandā}. “What is it about?” The nun \textsanskrit{Thullanandā} blocking a legitimate ending of the robe season. There is one rule. Of the six kinds of originations of offenses, it originates in three ways: … 

\scendvagga{The third subchapter on bathing is finished. }

\subsection*{The subchapter on lying down }

“There\marginnote{90.1} is an offense entailing confession for two nuns to lie down on the same bed. Where was it laid down?” At \textsanskrit{Sāvatthī}. “Whom is it about?” A number of nuns. “What is it about?” A number of nuns lying down in pairs on the same bed. There is one rule. Of the six kinds of originations of offenses, it originates in two ways: … (as in the rule on wool) … 

“There\marginnote{91.1} is an offense entailing confession for two nuns to lie down on the same sheet and under the same cover. Where was it laid down?” At \textsanskrit{Sāvatthī}. “Whom is it about?” A number of nuns. “What is it about?” A number of nuns lying down in pairs on the same sheet and under the same cover. There is one rule. Of the six kinds of originations of offenses, it originates in two ways: … (as in the rule on wool) … 

“There\marginnote{92.1} is an offense entailing confession for intentionally making a nun ill at ease. Where was it laid down?” At \textsanskrit{Sāvatthī}. “Whom is it about?” The nun \textsanskrit{Thullanandā}. “What is it about?” The nun \textsanskrit{Thullanandā} intentionally making a nun ill at ease. There is one rule. Of the six kinds of originations of offenses, it originates in three ways: … 

“There\marginnote{93.1} is an offense entailing confession for not nursing a suffering disciple, nor making any effort to have someone else nurse her. Where was it laid down?” At \textsanskrit{Sāvatthī}. “Whom is it about?” The nun \textsanskrit{Thullanandā}. “What is it about?” The nun \textsanskrit{Thullanandā} not nursing a suffering disciple, nor making any effort to have someone else nurse her. There is one rule. Of the six kinds of originations of offenses, it originates in one way: through abandoning one’s duty. … 

“There\marginnote{94.1} is an offense entailing confession for giving a dwelling place to a nun, and then, in anger, throwing her out. Where was it laid down?” At \textsanskrit{Sāvatthī}. “Whom is it about?” The nun \textsanskrit{Thullanandā}. “What is it about?” The nun \textsanskrit{Thullanandā} giving a dwelling place to a nun, and then, in anger, throwing her out. There is one rule. Of the six kinds of originations of offenses, it originates in three ways: … 

“There\marginnote{95.1} is an offense entailing confession for a socializing nun not to stop when pressed for the third time. Where was it laid down?” At \textsanskrit{Sāvatthī}. “Whom is it about?” The nun \textsanskrit{Caṇḍakālī}. “What is it about?” The nun \textsanskrit{Caṇḍakālī} socializing. There is one rule. Of the six kinds of originations of offenses, it originates in one way: through abandoning one’s duty. … 

“There\marginnote{96.1} is an offense entailing confession for wandering without a group of travelers where it is considered risky and dangerous within one’s own country. Where was it laid down?” At \textsanskrit{Sāvatthī}. “Whom is it about?” A number of nuns. “What is it about?” A number of nuns wandering without a group of travelers where it was considered risky and dangerous within their own country. There is one rule. Of the six kinds of originations of offenses, it originates in two ways: … (as in the rule on wool) … 

“There\marginnote{97.1} is an offense entailing confession for wandering without a group of travelers where it is considered risky and dangerous outside one’s own country. Where was it laid down?” At \textsanskrit{Sāvatthī}. “Whom is it about?” A number of nuns. “What is it about?” A number of nuns wandering without a group of travelers where it was considered risky and dangerous outside their own country. There is one rule. Of the six kinds of originations of offenses, it originates in two ways: … (as in the rule on wool) … 

“There\marginnote{98.1} is an offense entailing confession for wandering during the rainy season. Where was it laid down?” At \textsanskrit{Rājagaha}. “Whom is it about?” A number of nuns. “What is it about?” A number of nuns wandering during the rainy season. There is one rule. Of the six kinds of originations of offenses, it originates in two ways: … (as in the rule on wool) … 

“There\marginnote{99.1} is an offense entailing confession for a nun who has completed the rainy-season residence not to go wandering. Where was it laid down?” At \textsanskrit{Rājagaha}. “Whom is it about?” A number of nuns. “What is it about?” A number of nuns who did not go wandering after completing the rainy-season residence. There is one rule. Of the six kinds of originations of offenses, it originates in one way: … (as in the first offense entailing expulsion) … 

\scendvagga{The fourth subchapter on lying down is finished. }

\subsection*{The subchapter on galleries }

“There\marginnote{101.1} is an offense entailing confession for visiting a royal house, a gallery, a park, a garden, or a lotus pond. Where was it laid down?” At \textsanskrit{Sāvatthī}. “Whom is it about?” The nuns from the group of six. “What is it about?” The nuns from the group of six visiting a royal house and a gallery. There is one rule. Of the six kinds of originations of offenses, it originates in two ways: … (as in the rule on wool) … 

“There\marginnote{102.1} is an offense entailing confession for using a high or luxurious couch. Where was it laid down?” At \textsanskrit{Sāvatthī}. “Whom is it about?” A number of nuns. “What is it about?” A number of nuns using high and luxurious couches. There is one rule. Of the six kinds of originations of offenses, it originates in two ways: … (as in the rule on wool) … 

“There\marginnote{103.1} is an offense entailing confession for spinning yarn. Where was it laid down?” At \textsanskrit{Sāvatthī}. “Whom is it about?” The nuns from the group of six. “What is it about?” The nuns from the group of six spinning yarn. There is one rule. Of the six kinds of originations of offenses, it originates in two ways: … (as in the rule on wool) … 

“There\marginnote{104.1} is an offense entailing confession for providing services for a householder. Where was it laid down?” At \textsanskrit{Sāvatthī}. “Whom is it about?” A number of nuns. “What is it about?” A number of nuns providing services for householders. There is one rule. Of the six kinds of originations of offenses, it originates in two ways: … (as in the rule on wool) … 

“There\marginnote{105.1} is an offense entailing confession for agreeing, when requested by a nun, to resolve a legal issue, but then neither resolving it nor making any effort to resolve it. Where was it laid down?” At \textsanskrit{Sāvatthī}. “Whom is it about?” The nun \textsanskrit{Thullanandā}. “What is it about?” The nun \textsanskrit{Thullanandā} agreeing, when asked by a nun, to resolve a legal issue, but then neither resolving it nor making any effort to resolve it. There is one rule. Of the six kinds of originations of offenses, it originates in one way: through abandoning one’s duty. … 

“There\marginnote{106.1} is an offense entailing confession for personally giving fresh or cooked food to a householder, a male wanderer, or a female wanderer. Where was it laid down?” At \textsanskrit{Sāvatthī}. “Whom is it about?” The nun \textsanskrit{Thullanandā}. “What is it about?” The nun \textsanskrit{Thullanandā} personally giving fresh and cooked food to a householder. There is one rule. Of the six kinds of originations of offenses, it originates in two ways: … (as in the rule on wool) … 

“There\marginnote{107.1} is an offense entailing confession for not relinquishing but continuing to use a communal robe. Where was it laid down?” At \textsanskrit{Sāvatthī}. “Whom is it about?” The nun \textsanskrit{Thullanandā}. “What is it about?” The nun \textsanskrit{Thullanandā} not relinquishing but continuing to use a communal robe. There is one rule. Of the six kinds of originations of offenses, it originates in two ways: … (as in the rule on the robe season) … 

“There\marginnote{108.1} is an offense entailing confession for going wandering without relinquishing one’s lodging. Where was it laid down?” At \textsanskrit{Sāvatthī}. “Whom is it about?” The nun \textsanskrit{Thullanandā}. “What is it about?” The nun \textsanskrit{Thullanandā} going wandering without relinquishing her lodging. There is one rule. Of the six kinds of originations of offenses, it originates in two ways: … (as in the rule on the robe season) … 

“There\marginnote{109.1} is an offense entailing confession for studying worldly subjects. Where was it laid down?” At \textsanskrit{Sāvatthī}. “Whom is it about?” The nuns from the group of six. “What is it about?” The nuns from the group of six studying worldly subjects. There is one rule. Of the six kinds of originations of offenses, it originates in two ways: … (as in the rule on memorizing the Teaching) … 

“There\marginnote{110.1} is an offense entailing confession for teaching worldly subjects. Where was it laid down?” At \textsanskrit{Sāvatthī}. “Whom is it about?” The nuns from the group of six. “What is it about?” The nuns from the group of six teaching worldly subjects. There is one rule. Of the six kinds of originations of offenses, it originates in two ways: … (as in the rule on memorizing the Teaching) … 

\scendvagga{The fifth subchapter on galleries is finished. }

\subsection*{The subchapter on monasteries }

“There\marginnote{112.1} is an offense entailing confession for entering a monastery without asking permission, yet knowing that there are monks there. Where was it laid down?” At \textsanskrit{Sāvatthī}. “Whom is it about?” A number of nuns. “What is it about?” A number of nuns entering a monastery without asking permission. There is one rule. There are two additions to the rule. Of the six kinds of originations of offenses, it originates in one way: through abandoning one’s duty. … 

“There\marginnote{113.1} is an offense entailing confession for abusing or reviling a monk. Where was it laid down?” At \textsanskrit{Vesālī}. “Whom is it about?” The nuns from the group of six. “What is it about?” The nuns from the group of six abusing Venerable \textsanskrit{Upāli}. There is one rule. Of the six kinds of originations of offenses, it originates in three ways: … 

“There\marginnote{114.1} is an offense entailing confession for furiously reviling the community. Where was it laid down?” At \textsanskrit{Sāvatthī}. “Whom is it about?” The nun \textsanskrit{Thullanandā}. “What is it about?” The nun \textsanskrit{Thullanandā} furiously reviling the community. There is one rule. Of the six kinds of originations of offenses, it originates in three ways: … 

“There\marginnote{115.1} is an offense entailing confession, when invited to a meal, for refusing an offer to eat more, and then eating fresh or cooked food elsewhere. Where was it laid down?” At \textsanskrit{Sāvatthī}. “Whom is it about?” A number of nuns. “What is it about?” A number of nuns eating elsewhere after finishing their meal and refusing an offer to eat more. There is one rule. Of the six kinds of originations of offenses, it originates in four ways: … 

“There\marginnote{116.1} is an offense entailing confession for keeping a family to oneself. Where was it laid down?” At \textsanskrit{Sāvatthī}. “Whom is it about?” A certain nun. “What is it about?” A certain nun keeping a family to herself. There is one rule. Of the six kinds of originations of offenses, it originates in three ways: … 

“There\marginnote{117.1} is an offense entailing confession for spending the rainy-season residence in a monastery without monks. Where was it laid down?” At \textsanskrit{Sāvatthī}. “Whom is it about?” A number of nuns. “What is it about?” A number of nuns spending the rainy-season residence in a monastery without monks. There is one rule. Of the six kinds of originations of offenses, it originates in two ways: … (as in the rule on wool) … 

“There\marginnote{118.1} is an offense entailing confession for a nun who has completed the rainy-season residence not to invite correction from both Sanghas in regard to three things. Where was it laid down?” At \textsanskrit{Sāvatthī}. “Whom is it about?” A number of nuns. “What is it about?” A number of nuns who had completed the rainy-season residence not inviting the Sangha of monks for correction. There is one rule. Of the six kinds of originations of offenses, it originates in one way: through abandoning one’s duty. … 

“There\marginnote{119.1} is an offense entailing confession for not going to the instruction or to a formal meeting of the community. Where was it laid down?” In the Sakyan country. “Whom is it about?” The nuns from the group of six. “What is it about?” The nuns from the group of six not going to the instruction. There is one rule. Of the six kinds of originations of offenses, it originates in one way: … (as in the first offense entailing expulsion) … 

“There\marginnote{120.1} is an offense entailing confession for not enquiring about the observance day nor asking for the instruction. Where was it laid down?” At \textsanskrit{Sāvatthī}. “Whom is it about?” A number of nuns. “What is it about?” A number of nuns not enquiring about the observance day nor asking for the instruction. There is one rule. Of the six kinds of originations of offenses, it originates in one way: through abandoning one’s duty. … 

“There\marginnote{121.1} is an offense entailing confession for being alone with a man and having him rupture an abscess or a wound situated on the lower part of one’s body, without getting permission from the Sangha or a group. Where was it laid down?” At \textsanskrit{Sāvatthī}. “Whom is it about?” A certain nun. “What is it about?” A certain nun being alone with a man and having him rupture an abscess situated on the lower part of her body. There is one rule. Of the six kinds of originations of offenses, it originates in two ways: … (as in the rule on the robe season) … 

\scendvagga{The sixth subchapter on monasteries is finished. }

\subsection*{The subchapter on pregnant women }

“There\marginnote{123.1} is an offense entailing confession for giving the full admission to a pregnant woman. Where was it laid down?” At \textsanskrit{Sāvatthī}. “Whom is it about?” A number of nuns. “What is it about?” A number of nuns giving the full admission to a pregnant woman. There is one rule. Of the six kinds of originations of offenses, it originates in three ways: … 

“There\marginnote{124.1} is an offense entailing confession for giving the full admission to a woman who is breastfeeding. Where was it laid down?” At \textsanskrit{Sāvatthī}. “Whom is it about?” A number of nuns. “What is it about?” A number of nuns giving the full admission to a woman who was breastfeeding. There is one rule. Of the six kinds of originations of offenses, it originates in three ways: … 

“There\marginnote{125.1} is an offense entailing confession for giving the full admission to a trainee nun who has not trained for two years in the six rules. Where was it laid down?” At \textsanskrit{Sāvatthī}. “Whom is it about?” A number of nuns. “What is it about?” A number of nuns giving the full admission to a trainee nun who had not trained for two years in the six rules. There is one rule. Of the six kinds of originations of offenses, it originates in three ways: … 

“There\marginnote{126.1} is an offense entailing confession for giving the full admission to a trainee nun who has trained for two years in the six rules, but who has not been approved by the Sangha. Where was it laid down?” At \textsanskrit{Sāvatthī}. “Whom is it about?” A number of nuns. “What is it about?” A number of nuns giving the full admission to a trainee nun who had trained for two years in the six rules, but who had not been approved by the Sangha. There is one rule. Of the six kinds of originations of offenses, it originates in three ways: … 

“There\marginnote{127.1} is an offense entailing confession for giving the full admission to a married girl who is less than twelve years old. Where was it laid down?” At \textsanskrit{Sāvatthī}. “Whom is it about?” A number of nuns. “What is it about?” A number of nuns giving the full admission to a married girl who was less than twelve years old. There is one rule. Of the six kinds of originations of offenses, it originates in three ways: … 

“There\marginnote{128.1} is an offense entailing confession for giving the full admission to a married girl who is more than twelve years old, but who has not trained for two years in the six rules. Where was it laid down?” At \textsanskrit{Sāvatthī}. “Whom is it about?” A number of nuns. “What is it about?” A number of nuns giving the full admission to a married girl who was more than twelve years old, but who had not trained for two years in the six rules. There is one rule. Of the six kinds of originations of offenses, it originates in three ways: … 

“There\marginnote{129.1} is an offense entailing confession for giving the full admission to a married girl who is more than twelve years old and who has trained for two years in the six rules, but who has not been approved by the Sangha. Where was it laid down?” At \textsanskrit{Sāvatthī}. “Whom is it about?” A number of nuns. “What is it about?” A number of nuns giving the full admission to a married girl who was more than twelve years old and who had trained for two years in the six rules, but who had not been approved by the Sangha. There is one rule. Of the six kinds of originations of offenses, it originates in three ways: … 

“There\marginnote{130.1} is an offense entailing confession for giving the full admission to a disciple, and then, for the next two years, neither guiding her nor having her guided. Where was it laid down?” At \textsanskrit{Sāvatthī}. “Whom is it about?” The nun \textsanskrit{Thullanandā}. “What is it about?” The nun \textsanskrit{Thullanandā} giving the full admission to a disciple, and then, for the next two years, neither guiding her nor having her guided. There is one rule. Of the six kinds of originations of offenses, it originates in one way: through abandoning one’s duty. … 

“There\marginnote{131.1} is an offense entailing confession for not following the mentor who gave one the full admission for two years. Where was it laid down?” At \textsanskrit{Sāvatthī}. “Whom is it about?” A number of nuns. “What is it about?” A number of nuns not following the mentor who gave them the full admission for two years. There is one rule. Of the six kinds of originations of offenses, it originates in one way: … (as in the first offense entailing expulsion) … 

“There\marginnote{132.1} is an offense entailing confession for giving the full admission to a disciple, and then neither sending her away nor having her sent away. Where was it laid down?” At \textsanskrit{Sāvatthī}. “Whom is it about?” The nun \textsanskrit{Thullanandā}. “What is it about?” The nun \textsanskrit{Thullanandā} giving the full admission to a disciple, and then neither sending her away nor having her sent away. There is one rule. Of the six kinds of originations of offenses, it originates in one way: through abandoning one’s duty. … 

\scendvagga{The seventh subchapter on pregnant women is finished. }

\subsection*{The subchapter on unmarried women }

“There\marginnote{134.1} is an offense entailing confession for giving the full admission to an unmarried woman who is less than twenty years old. Where was it laid down?” At \textsanskrit{Sāvatthī}. “Whom is it about?” A number of nuns. “What is it about?” A number of nuns giving the full admission to an unmarried woman who was less than twenty years old. There is one rule. Of the six kinds of originations of offenses, it originates in three ways: … 

“There\marginnote{135.1} is an offense entailing confession for giving the full admission to an unmarried woman who is more than twenty years old, but who has not trained for two years in the six rules. Where was it laid down?” At \textsanskrit{Sāvatthī}. “Whom is it about?” A number of nuns. “What is it about?” A number of nuns giving the full admission to an unmarried woman who was more than twenty years old, but who had not trained for two years in the six rules. There is one rule. Of the six kinds of originations of offenses, it originates in three ways: … 

“There\marginnote{136.1} is an offense entailing confession for giving the full admission to an unmarried woman who is more than twenty years old and who has trained for two years in the six rules, but who has not been approved by the Sangha. Where was it laid down?” At \textsanskrit{Sāvatthī}. “Whom is it about?” A number of nuns. “What is it about?” A number of nuns giving the full admission to an unmarried woman who was more than twenty years old and who had trained for two years in the six rules, but who had not been approved by the Sangha. There is one rule. Of the six kinds of originations of offenses, it originates in three ways: … 

“There\marginnote{137.1} is an offense entailing confession for one who has less than twelve years of seniority giving the full admission. Where was it laid down?” At \textsanskrit{Sāvatthī}. “Whom is it about?” A number of nuns. “What is it about?” A number of nuns who had less than twelve years of seniority giving the full admission. There is one rule. Of the six kinds of originations of offenses, it originates in three ways: … 

“There\marginnote{138.1} is an offense entailing confession for one who has twelve years of seniority giving the full admission without approval from the Sangha. Where was it laid down?” At \textsanskrit{Sāvatthī}. “Whom is it about?” A number of nuns. “What is it about?” A number of nuns who had twelve years of seniority giving the full admission without approval from the Sangha. There is one rule. Of the six kinds of originations of offenses, it originates in three ways: … (as in the second offense entailing expulsion) … 

“There\marginnote{139.1} is an offense entailing confession for verbally consenting when being told, ‘Venerable, you’ve given enough full admissions for now,’ but then criticizing it afterwards. Where was it laid down?” At \textsanskrit{Sāvatthī}. “Whom is it about?” The nun \textsanskrit{Caṇḍakālī}. “What is it about?” The nun \textsanskrit{Caṇḍakālī} verbally consenting when being told, “Venerable, you’ve given enough full admissions for now,” but then criticizing it afterwards. There is one rule. Of the six kinds of originations of offenses, it originates in three ways: … 

“There\marginnote{140.1} is an offense entailing confession for telling a trainee nun, ‘If you give me a robe, venerable, I’ll give you the full admission,’ but then neither giving her the full admission nor making any effort to have her fully admitted. Where was it laid down?” At \textsanskrit{Sāvatthī}. “Whom is it about?” The nun \textsanskrit{Thullanandā}. “What is it about?” The nun \textsanskrit{Thullanandā} telling a trainee nun, “If you give me a robe, venerable, I’ll give you the full admission,” but then neither giving her the full admission nor making any effort to have her fully admitted. There is one rule. Of the six kinds of originations of offenses, it originates in one way: through abandoning one’s duty. … 

“There\marginnote{141.1} is an offense entailing confession for telling a trainee nun, ‘If you follow me for two years, Venerable, I’ll give you the full admission,’ but then neither giving her the full admission nor making any effort to have her fully admitted. Where was it laid down?” At \textsanskrit{Sāvatthī}. “Whom is it about?” The nun \textsanskrit{Thullanandā}. “What is it about?” The nun \textsanskrit{Thullanandā} telling a trainee nun, “If you follow me for two years, venerable, I’ll give you the full admission,” but then neither giving her the full admission nor making any effort to have her fully admitted. There is one rule. Of the six kinds of originations of offenses, it originates in one way: through abandoning one’s duty. … 

“There\marginnote{142.1} is an offense entailing confession for giving the full admission to a trainee nun who is socializing with men and boys and who is temperamental and difficult to live with. Where was it laid down?” At \textsanskrit{Sāvatthī}. “Whom is it about?” The nun \textsanskrit{Thullanandā}. “What is it about?” The nun \textsanskrit{Thullanandā} giving the full admission to a trainee nun who was socializing with men and boys and who was temperamental and difficult to live with. There is one rule. Of the six kinds of originations of offenses, it originates in three ways: … 

“There\marginnote{143.1} is an offense entailing confession for giving the full admission to a trainee nun who has not been given permission by her parents or her husband. Where was it laid down?” At \textsanskrit{Sāvatthī}. “Whom is it about?” The nun \textsanskrit{Thullanandā}. “What is it about?” The nun \textsanskrit{Thullanandā} giving the full admission to a trainee nun who had not been given permission by her parents and her husband. There is one rule. Of the six kinds of originations of offenses, it originates in four ways: from speech, not from body or mind; or from body and speech, not from mind; or from speech and mind, not from body; or from body, speech, and mind. … 

“There\marginnote{144.1} is an offense entailing confession for giving the full admission to a trainee nun after a given consent has expired. Where was it laid down?” At \textsanskrit{Rājagaha}. “Whom is it about?” The nun \textsanskrit{Thullanandā}. “What is it about?” The nun \textsanskrit{Thullanandā} giving the full admission to a trainee nun after the given consent had expired. There is one rule. Of the six kinds of originations of offenses, it originates in three ways: … 

“There\marginnote{145.1} is an offense entailing confession for giving full admission every year. Where was it laid down?” At \textsanskrit{Sāvatthī}. “Whom is it about?” A number of nuns. “What is it about?” A number of nuns giving full admission every year. There is one rule. Of the six kinds of originations of offenses, it originates in three ways: … 

“There\marginnote{146.1} is an offense entailing confession for giving the full admission to two women in the same year. Where was it laid down?” At \textsanskrit{Sāvatthī}. “Whom is it about?” A number of nuns. “What is it about?” A number of nuns giving the full admission to two women in the same year. There is one rule. Of the six kinds of originations of offenses, it originates in three ways: … 

\scendvagga{The eighth subchapter on unmarried women is finished. }

\subsection*{The subchapter on sunshades and sandals }

“There\marginnote{148.1} is an offense entailing confession for using a sunshade and sandals. Where was it laid down?” At \textsanskrit{Sāvatthī}. “Whom is it about?” The nuns from the group of six. “What is it about?” The nuns from the group of six using sunshades and sandals. There is one rule. There is one addition to the rule. Of the six kinds of originations of offenses, it originates in two ways: … (as in the rule on wool) … 

“There\marginnote{149.1} is an offense entailing confession for traveling in a vehicle. Where was it laid down?” At \textsanskrit{Sāvatthī}. “Whom is it about?” The nuns from the group of six. “What is it about?” The nuns from the group of six traveling in a vehicle. There is one rule. There is one addition to the rule. Of the six kinds of originations of offenses, it originates in two ways: … (as in the rule on wool) … 

“There\marginnote{150.1} is an offense entailing confession for wearing a hip ornament. Where was it laid down?” At \textsanskrit{Sāvatthī}. “Whom is it about?” A certain nun. “What is it about?” A certain nun wearing a hip ornament. There is one rule. Of the six kinds of originations of offenses, it originates in two ways: … (as in the rule on wool) … 

“There\marginnote{151.1} is an offense entailing confession for wearing jewelry. Where was it laid down?” At \textsanskrit{Sāvatthī}. “Whom is it about?” The nuns from the group of six. “What is it about?” The nuns from the group of six wearing jewelry. There is one rule. Of the six kinds of originations of offenses, it originates in two ways: … (as in the rule on wool) … 

“There\marginnote{152.1} is an offense entailing confession for bathing with scents and colors. Where was it laid down?” At \textsanskrit{Sāvatthī}. “Whom is it about?” The nuns from the group of six. “What is it about?” The nuns from the group of six bathing with scents and colors. There is one rule. Of the six kinds of originations of offenses, it originates in two ways: … (as in the rule on wool) … 

“There\marginnote{153.1} is an offense entailing confession for bathing with scented sesame paste. Where was it laid down?” At \textsanskrit{Sāvatthī}. “Whom is it about?” The nuns from the group of six. “What is it about?” The nuns from the group of six bathing with scented sesame paste. There is one rule. Of the six kinds of originations of offenses, it originates in two ways: … (as in the rule on wool) … 

“There\marginnote{154.1} is an offense entailing confession for having a nun massage or rub oneself. Where was it laid down?” At \textsanskrit{Sāvatthī}. “Whom is it about?” A number of nuns. “What is it about?” A number of nuns having a nun massage and rub them. There is one rule. Of the six kinds of originations of offenses, it originates in two ways: … (as in the rule on wool) … 

“There\marginnote{155.1} is an offense entailing confession for having a trainee nun massage or rub oneself. Where was it laid down?” At \textsanskrit{Sāvatthī}. “Whom is it about?” A number of nuns. “What is it about?” A number of nuns having a trainee nun massage and rub them. There is one rule. Of the six kinds of originations of offenses, it originates in two ways: … (as in the rule on wool) … 

“There\marginnote{156.1} is an offense entailing confession for having a novice nun massage or rub oneself. Where was it laid down?” At \textsanskrit{Sāvatthī}. “Whom is it about?” A number of nuns. “What is it about?” A number of nuns having a novice nun massage and rub them. There is one rule. Of the six kinds of originations of offenses, it originates in two ways: … (as in the rule on wool) … 

“There\marginnote{157.1} is an offense entailing confession for having a female householder massage or rub oneself. Where was it laid down?” At \textsanskrit{Sāvatthī}. “Whom is it about?” A number of nuns. “What is it about?” A number of nuns having a female householder massage and rub them. There is one rule. Of the six kinds of originations of offenses, it originates in two ways: … (as in the rule on wool) … 

“There\marginnote{158.1} is an offense entailing confession for sitting down on a seat in front of a monk without asking permission. Where was it laid down?” At \textsanskrit{Sāvatthī}. “Whom is it about?” A number of nuns. “What is it about?” A number of nuns sitting down on seats in front of a monk without asking permission. There is one rule. Of the six kinds of originations of offenses, it originates in two ways: … (as in the rule on the robe season) … 

“There\marginnote{159.1} is an offense entailing confession for asking a question of a monk who has not given permission. Where was it laid down?” At \textsanskrit{Sāvatthī}. “Whom is it about?” A number of nuns. “What is it about?” A number of nuns asking a question of a monk who had not given them permission. There is one rule. Of the six kinds of originations of offenses, it originates in two ways: … (as in the rule on memorizing the Teaching) … 

“There\marginnote{160.1} is an offense entailing confession for entering an inhabited area without wearing one’s chest wrap. Where was it laid down?” At \textsanskrit{Sāvatthī}. “Whom is it about?” A certain nun. “What is it about?” A certain nun entering an inhabited area without wearing her chest wrap. There is one rule. Of the six kinds of originations of offenses, it originates in two ways: from body, not from speech or mind; or from body and mind, not from speech. … 

\scendvagga{The ninth subchapter on sunshades and sandals is finished. }

\scend{The section on minor rules in nine subchapters is finished. }

\scuddanaintro{This is the summary: }

\begin{scuddana}%
“Garlic,\marginnote{163.1} hair on the private parts, \\
And palm, dildo, cleaning; \\
Eating, of raw grains, \\
Two with food scraps, seeing. 

In\marginnote{164.1} the dark, concealed, \\
Out in the open, and on a street; \\
Before, after, and at the wrong time, \\
Misunderstanding, hell, she beat. 

Naked,\marginnote{165.1} water, having unstitched, \\
Five days, taken on loan; \\
The community, distribution, monastic, \\
Uncertain, and with the robe season. 

With\marginnote{166.1} the same bed, and with the same sheet, \\
Intentionally, disciple; \\
Gives, and socializing, within, \\
Outside, rainy season, should she not go. 

Royal,\marginnote{167.1} high couch, and yarn, \\
Householder, and with resolving; \\
Should she give, robe, lodging, \\
And learning, should she teach. 

Monastery,\marginnote{168.1} abusing, and furious, \\
Should she eat, keeps a family to herself; \\
Should she spend, inviting correction, instruction, \\
Two things, and with the lower part of the body. 

A\marginnote{169.1} pregnant woman, a breastfeeding woman, six rules, \\
One who has not been approved, less than twelve; \\
And more than twelve, by the Sangha, \\
Disciple, admission, and five to six. 

An\marginnote{170.1} unmarried girl, and two, by the Sangha, \\
Twelve, and with one who has not been approved;\footnote{Reading \textit{\textsanskrit{dvādasāsammatena} } with SRT. } \\
Enough, and if, for two years, \\
Socializing, and by the husband. 

Expired,\marginnote{171.1} every year, \\
And with the admission of two; \\
Sunshade, in a vehicle, hip ornament, \\
Jewelry, with colors. 

Sesame\marginnote{172.1} paste, and a nun, \\
And a trainee nun, a novice nun; \\
A female householder, in front of a monk, \\
Not permission, a chest wrap.” 

%
\end{scuddana}

\scuddanaintro{This is the summary of the subchapters: }

\begin{scuddana}%
“Garlic,\marginnote{174.1} the dark, bathing, \\
Lying down, gallery; \\
Monastery, and pregnant women, \\
Unmarried girls, sunshades and sandals.” 

%
\end{scuddana}

\section*{5. The chapter on offenses entailing acknowledgment }

“There\marginnote{175.1} is an offense entailing acknowledgment for asking for ghee and then eating it. Where was it laid down?” At \textsanskrit{Sāvatthī}. “Whom is it about?” The nuns from the group of six. “What is it about?” The nuns from the group of six asking for ghee and then eating it. There is one rule. There is one addition to the rule. Of the six kinds of originations of offenses, it originates in four ways: … 

“There\marginnote{176.1} is an offense entailing acknowledgment for asking for oil and then eating it. Where was it laid down?” At \textsanskrit{Sāvatthī}. “Whom is it about?” The nuns from the group of six. “What is it about?” The nuns from the group of six asking for oil and then eating it. There is one rule. There is one addition to the rule. Of the six kinds of originations of offenses, it originates in four ways: … 

“There\marginnote{177.1} is an offense entailing acknowledgment for asking for honey and then eating it. Where was it laid down?” At \textsanskrit{Sāvatthī}. “Whom is it about?” The nuns from the group of six. “What is it about?” The nuns from the group of six asking for honey and then eating it. There is one rule. There is one addition to the rule. Of the six kinds of originations of offenses, it originates in four ways: … 

“There\marginnote{178.1} is an offense entailing acknowledgment for asking for syrup and then eating it. Where was it laid down?” At \textsanskrit{Sāvatthī}. “Whom is it about?” The nuns from the group of six. “What is it about?” The nuns from the group of six asking for syrup and then eating it. There is one rule. There is one addition to the rule. Of the six kinds of originations of offenses, it originates in four ways: … 

“There\marginnote{179.1} is an offense entailing acknowledgment for asking for fish and then eating it. Where was it laid down?” At \textsanskrit{Sāvatthī}. “Whom is it about?” The nuns from the group of six. “What is it about?” The nuns from the group of six asking for fish and then eating it. There is one rule. There is one addition to the rule. Of the six kinds of originations of offenses, it originates in four ways: … 

“There\marginnote{180.1} is an offense entailing acknowledgment for asking for meat and then eating it. Where was it laid down?” At \textsanskrit{Sāvatthī}. “Whom is it about?” The nuns from the group of six. “What is it about?” The nuns from the group of six asking for meat and then eating it. There is one rule. There is one addition to the rule. Of the six kinds of originations of offenses, it originates in four ways: … 

“There\marginnote{181.1} is an offense entailing acknowledgment for asking for milk and then drinking it. Where was it laid down?” At \textsanskrit{Sāvatthī}. “Whom is it about?” The nuns from the group of six. “What is it about?” The nuns from the group of six asking for milk and then drinking it. There is one rule. There is one addition to the rule. Of the six kinds of originations of offenses, it originates in four ways: … 

“There\marginnote{182.1} is an offense entailing acknowledgment for asking for curd and then eating it. Where was it laid down?” At \textsanskrit{Sāvatthī}. “Whom is it about?” The nuns from the group of six. “What is it about?” The nuns from the group of six asking for curd and then eating it. There is one rule. There is one addition to the rule. Of the six kinds of originations of offenses, it originates in four ways: from body, not from speech or mind; or from body and speech, not from mind; or from body and mind, not from speech; or from body, speech, and mind. … 

\scend{The eight offenses entailing acknowledgment are finished. }

\scuddanaintro{This is the summary: }

\begin{scuddana}%
“Ghee,\marginnote{185.1} oil, and honey, \\
Syrup, and fish; \\
Meat, milk, and curd: \\
A nun asked for—\\
The eight offenses entailing acknowledgment, \\
Taught by the Buddha himself.” 

%
\end{scuddana}

The\marginnote{186.1} training rules given in full in the Monks’ Analysis are contracted in the Nuns’ Analysis. 

\scendsutta{The questions and answers on the nuns’ \textsanskrit{Pātimokkha} rules and their analysis, the first, are finished. }

%
\chapter*{{\suttatitleacronym Pvr 2.2}{\suttatitletranslation The number of offenses within each offense }{\suttatitleroot Katāpattivāra}}
\addcontentsline{toc}{chapter}{\tocacronym{Pvr 2.2} \toctranslation{The number of offenses within each offense } \tocroot{Katāpattivāra}}
\markboth{The number of offenses within each offense }{Katāpattivāra}
\extramarks{Pvr 2.2}{Pvr 2.2}

\section*{The chapter on offenses entailing expulsion }

When\marginnote{1.1} a lustful nun consents to a lustful man making physical contact with her, how many kinds of offenses does she commit? She commits three kinds of offenses: when she consents to him taking hold of her anywhere below the collar bone but above the knees, she commits an offense entailing expulsion; when she consents to him taking hold of her above the collar bone or below the knees, she commits a serious offense; when she consents to him taking hold of something connected to her body, she commits an offense of wrong conduct. 

When\marginnote{2.1} a nun conceals an offense, how many kinds of offenses does she commit? She commits three kinds of offenses: when she knowingly conceals an offense entailing expulsion, she commits an offense entailing expulsion; when, being unsure, she conceals it, she commits a serious offense;\footnote{This ruling does not seem to be found in the Canonical text, either for monks or nuns. } when she conceals a failure in conduct, she commits an offense of wrong conduct. 

When\marginnote{3.1} a nun takes sides with one who has been ejected and does not stop when pressed for the third time, how many kinds of offenses does she commit? She commits three kinds of offenses: after the motion, she commits an offense of wrong conduct; after each of the first two announcements, she commits a serious offense; when the last announcement is finished, she commits an offense entailing expulsion. 

When\marginnote{4.1} fulfilling the eight parts, how many kinds of offenses does she commit? She commits three kinds of offenses: when she goes to such-and-such a place when told by a man to do so, she commits an offense of wrong conduct; when she enters within arm’s reach of the man, she commits a serious offense; when she fulfills the eight parts, she commits an offense entailing expulsion. 

\scend{The offenses entailing expulsion are finished. }

\section*{2. The chapter on offenses entailing suspension }

When\marginnote{6.1} a litigious nun initiates a lawsuit, she commits three kinds of offenses: when she tells one other person, she commits an offense of wrong conduct; when she tells a second person, she commits a serious offense; when the lawsuit is finished, she commits an offense entailing suspension. 

When\marginnote{7.1} giving the full admission to a female criminal, she commits three kinds of offenses: after the motion, she commits an offense of wrong conduct; after each of the first two announcements, she commits a serious offense; when the last announcement is finished, she commits an offense entailing suspension. 

When\marginnote{8.1} walking to the next inhabited area by herself, she commits three kinds of offenses: when she is in the process of going, she commits an offense of wrong conduct; when she crosses the boundary with her first foot, she commits a serious offense; when she crosses with her second foot, she commits an offense entailing suspension. 

When\marginnote{9.1} readmitting a nun who had been ejected by a unanimous Sangha in accordance with the Teaching, the Monastic Law, and the Teacher’s instruction, without first getting permission from the Sangha that did the legal procedure and without the consent of the community, she commits three kinds of offenses: after the motion, she commits an offense of wrong conduct; after each of the first two announcements, she commits a serious offense; when the last announcement is finished, she commits an offense entailing suspension. 

When\marginnote{10.1} a lustful nun eats fresh or cooked food after receiving it directly from a lustful man, she commits three kinds of offenses: when she receives fresh or cooked food with the intention of eating it, she commits a serious offense; for every mouthful swallowed, she commits an offense entailing suspension; if she receives water or a tooth cleaner, she commits an offense of wrong conduct. 

When\marginnote{11.1} urging a nun on, saying, “Venerable, what can this man do to you, whether he has lust or not, if you’re without? Go on, venerable, receive it with your own hands and then eat whatever fresh or cooked food he gives you,” she commits three kinds of offenses: when, because of her statement, the other nun receives it with the intention of eating it, she commits an offense of wrong conduct; for every mouthful swallowed, she commits a serious offense; when the meal is finished, she commits an offense entailing suspension. 

When\marginnote{12.1} an angry nun does not stop when pressed for the third time, she commits three kinds of offenses: after the motion, she commits an offense of wrong conduct; after each of the first two announcements, she commits a serious offense; when the last announcement is finished, she commits an offense entailing suspension. 

When\marginnote{13.1} a nun who has lost a legal case does not stop when pressed for the third time, she commits three kinds of offenses: after the motion, she commits an offense of wrong conduct; after each of the first two announcements, she commits a serious offense; when the last announcement is finished, she commits an offense entailing suspension. 

When\marginnote{14.1} socializing nuns do not stop when pressed for the third time, they commit three kinds of offenses: after the motion, they commit an offense of wrong conduct; after each of the first two announcements, they commit a serious offense; when the last announcement is finished, they commit an offense entailing suspension. 

When\marginnote{15.1} urging the nuns on, saying, “Venerables, you should socialize. Don’t live separately,” and not stopping when pressed for the third time, she commits three kinds of offenses: after the motion, she commits an offense of wrong conduct; after each of the first two announcements, she commits a serious offense; when the last announcement is finished, she commits an offense entailing suspension. 

\scend{The offenses entailing suspension are finished. }

\section*{3. The chapter on relinquishment }

When\marginnote{17.1} collecting almsbowls, she commits one kind of offense: an offense entailing relinquishment and confession. 

When\marginnote{18.1} determining out-of-season robe-cloth as “in-season”, and then distributing it, she commits two kinds of offenses: when she is in the process of distributing it, then for the effort there is an offense of wrong conduct; when she has distributed it, she commits an offense entailing relinquishment and confession. 

When\marginnote{19.1} trading robes with a nun and then taking it back, she commits two kinds of offenses: when she is in the process of taking it back, then for the effort there is an offense of wrong conduct; when she has taken it back, she commits an offense entailing relinquishment and confession. 

When\marginnote{20.1} asking for one thing and then for something else, she commits two kinds of offenses: when she is in the process of asking, then for the effort there is an offense of wrong conduct; when she has asked, she commits an offense entailing relinquishment and confession. 

When\marginnote{21.1} getting one thing in exchange and then something else, she commits two kinds of offenses: when she is in the process of getting it in exchange, then for the effort there is an offense of wrong conduct; when she has received it in exchange, she commits an offense entailing relinquishment and confession. 

When\marginnote{22.1} exchanging for something else a requisite belonging to the Sangha that is designated for a specific purpose, she commits two kinds of offenses: when she is in the process of getting it in exchange, then for the effort there is an offense of wrong conduct; when she has received it in exchange, she commits an offense entailing relinquishment and confession. 

When\marginnote{23.1} exchanging for something else a requisite belonging to the Sangha that is designated for a specific purpose and was asked for, she commits two kinds of offenses: when she is in the process of getting it in exchange, then for the effort there is an offense of wrong conduct; when she has received it in exchange, she commits an offense entailing relinquishment and confession. 

When\marginnote{24.1} exchanging for something else a collective requisite that is designated for a specific purpose, she commits two kinds of offenses: when she is in the process of getting it in exchange, then for the effort there is an offense of wrong conduct; when she has received it in exchange, she commits an offense entailing relinquishment and confession. 

When\marginnote{25.1} exchanging for something else a collective requisite that is designated for a specific purpose and was asked for, she commits two kinds of offenses: when she is in the process of getting it in exchange, then for the effort there is an offense of wrong conduct; when she has received it in exchange, she commits an offense entailing relinquishment and confession. 

When\marginnote{26.1} exchanging for something else a personal requisite that is designated for a specific purpose and was asked for, she commits two kinds of offenses: when she is in the process of getting it in exchange, then for the effort there is an offense of wrong conduct; when she has received it in exchange, she commits an offense entailing relinquishment and confession. 

When\marginnote{27.1} getting in exchange a heavy cloak worth more than four \textit{\textsanskrit{kaṁsa}} coins, she commits two kinds of offenses: when she is in the process of getting it in exchange, then for the effort there is an offense of wrong conduct; when she has received it in exchange, she commits an offense entailing relinquishment and confession. 

When\marginnote{28.1} getting in exchange a light cloak worth more than two-and-a-half \textit{\textsanskrit{kaṁsa}} coins, she commits two kinds of offenses: when she is in the process of getting it in exchange, then for the effort there is an offense of wrong conduct; when she has received it in exchange, she commits an offense entailing relinquishment and confession. 

\scend{The rules on relinquishment and confession are finished. }

\section*{4. The chapter on offenses entailing confession }

\subsection*{The subchapter on garlic }

When\marginnote{30.1} eating garlic, she commits two kinds of offenses: when she receives it with the intention of eating it, she commits an offense of wrong conduct; for every mouthful swallowed, she commits an offense entailing confession. 

When\marginnote{31.1} removing hair from the private parts, she commits two kinds of offenses: when she is in the process of removing it, then for the effort there is an offense of wrong conduct; when she has removed it, she commits an offense entailing confession. 

When\marginnote{32.1} slapping her genitals with the palm of her hand, she commits two kinds of offenses: when she is in the process of slapping, then for the effort there is an offense of wrong conduct; when she is done, she commits an offense entailing confession. 

When\marginnote{33.1} using a dildo, she commits two kinds of offenses: when she is using it, then for the effort there is an offense of wrong conduct; when she is done, she commits an offense entailing confession. 

When\marginnote{34.1} cleaning herself with water by inserting more than two finger joints, she commits two kinds of offenses: when she is in the process of cleaning, then for the effort there is an offense of wrong conduct; when she is done, she commits an offense entailing confession. 

When\marginnote{35.1} attending on a monk who is eating with drinking water or a fan, she commits two kinds of offenses: when standing within arm’s reach, she commits an offense entailing confession; when standing beyond arm’s reach, she commits an offense of wrong conduct. 

When\marginnote{36.1} asking for raw grain and then eating it, she commits two kinds of offenses: when she receives with the intention to eat, she commits an offense of wrong conduct; for every mouthful swallowed, she commits an offense entailing confession. 

When\marginnote{37.1} disposing of feces, urine, trash, or food scraps over a wall or over an encircling wall, she commits two kinds of offenses: when she is in the process of discarding it, then for the effort there is an offense of wrong conduct; when she has discarded it, she commits an offense entailing confession. 

When\marginnote{38.1} disposing of feces, urine, trash, or food scraps on cultivated plants, she commits two kinds of offenses: when she is in the process of discarding it, then for the effort there is an offense of wrong conduct; when she has discarded it, she commits an offense entailing confession. 

When\marginnote{39.1} going to see dancing, singing, or music, she commits two kinds of offenses: when she is in the process of going, she commits an offense of wrong conduct; wherever she stands to see it or hear it, she commits an offense entailing confession. 

\scendvagga{The first subchapter on garlic is finished. }

\subsection*{The subchapter on the dark of the night }

When\marginnote{41.1} standing alone with a man in the dark of the night without a lamp, she commits two kinds of offenses: when she stands within arm’s reach, she commits an offense entailing confession; when she stands beyond arm’s reach, she commits an offense of wrong conduct. 

When\marginnote{42.1} standing alone with a man in a concealed place, she commits two kinds of offenses: when she stands within arm’s reach, she commits an offense entailing confession; when she stands beyond arm’s reach, she commits an offense of wrong conduct. 

When\marginnote{43.1} standing alone with a man out in the open, she commits two kinds of offenses: when she stands within arm’s reach, she commits an offense entailing confession; when she stands beyond arm’s reach, she commits an offense of wrong conduct. 

When\marginnote{44.1} standing alone with a man on a street, in a cul-de-sac, or at an intersection, she commits two kinds of offenses: when she stands within arm’s reach, she commits an offense entailing confession; when she stands beyond arm’s reach, she commits an offense of wrong conduct. 

When\marginnote{45.1} visiting families before the meal, sitting down on a seat, and then departing without informing the owners, she commits two kinds of offenses: when she goes beyond the roof cover of the house with the first foot, she commits an offense of wrong conduct; when she goes beyond with the second foot, she commits an offense entailing confession. 

When\marginnote{46.1} visiting families after the meal and then sitting down on a seat without asking permission of the owners, she commits two kinds of offenses: when she is in the process of sitting down, then for the effort there is an offense of wrong conduct; when she is seated, she commits an offense entailing confession. 

When\marginnote{47.1} visiting families at the wrong time, putting out bedding without asking permission of the owners, or having it put out, and then sitting down, she commits two kinds of offenses: when she is in the process of sitting down, then for the effort there is an offense of wrong conduct; when she is seated, she commits an offense entailing confession. 

When\marginnote{48.1} complaining about someone because of a misunderstanding and a lack of proper reflection, she commits two kinds of offenses: when she is in the process of complaining, then for the effort there is an offense of wrong conduct; when she has complained, she commits an offense entailing confession. 

When\marginnote{49.1} referring to hell or the spiritual life to curse oneself or someone else, she commits two kinds of offenses: when she is in the process of cursing, then for the effort there is an offense of wrong conduct; when she has finished cursing, she commits an offense entailing confession. 

When\marginnote{50.1} crying after repeatedly beating herself, she commits two kinds of offenses: when beating herself and crying, she commits an offense entailing confession; when beating herself but not crying, she commits an offense of wrong conduct. 

\scendvagga{The second subchapter on the dark of the night is finished. }

\subsection*{The subchapter on bathing }

When\marginnote{52.1} bathing naked, she commits two kinds of offenses: when she is bathing, then for the effort there is an offense of wrong conduct; when she is finished bathing, she commits an offense entailing confession. 

When\marginnote{53.1} making a bathing robe that exceeds the right size, she commits two kinds of offenses: when she is in the process of making it, then for the effort there is an offense of wrong conduct; when she has made it, she commits an offense entailing confession. 

When\marginnote{54.1} unstitching a nun’s robe, or having it unstitched, and then neither sewing it herself nor making any effort to have someone else sew it, she commits one kind of offense: an offense entailing confession. 

When\marginnote{55.1} not moving her robes for more than five days, she commits one kind of offense: an offense entailing confession. 

When\marginnote{56.1} wearing a  robe taken on loan, she commits two kinds of offenses: when she is wearing it, then for the effort there is an offense of wrong conduct; when she has worn it, she commits an offense entailing confession. 

When\marginnote{57.1} creating an obstacle for the community to get robe-cloth, she commits two kinds of offenses: when she is in the process of creating it, then for the effort there is an offense of wrong conduct; when she has created it, she commits an offense entailing confession. 

When\marginnote{58.1} blocking a legitimate distribution of robe-cloth, she commits two kinds of offenses: when she is in the process of blocking it, then for the effort there is an offense of wrong conduct; when she has blocked it, she commits an offense entailing confession. 

When\marginnote{59.1} giving a monastic robe to a householder, a male wanderer, or a female wanderer, she commits two kinds of offenses: when she is in the process of giving it, then for the effort there is an offense of wrong conduct; when she has given it, she commits an offense entailing confession. 

When\marginnote{60.1} letting the robe season expire because of an uncertain expectation of robe-cloth, she commits two kinds of offenses: when she is in the process of letting it expire, then for the effort there is an offense of wrong conduct; when she has let it expire, she commits an offense entailing confession. 

When\marginnote{61.1} blocking a legitimate ending of the robe season, she commits two kinds of offenses: when she is in the process of blocking it, then for the effort there is an offense of wrong conduct; when she has blocked it, she commits an offense entailing confession. 

\scendvagga{The third subchapter on bathing is finished. }

\subsection*{The subchapter on lying down }

When\marginnote{63.1} two nuns lie down on the same bed, they commit two kinds of offenses: when they are in the process of lying down, then for the effort there is an offense of wrong conduct; when they are lying down, they commit an offense entailing confession. 

When\marginnote{64.1} two nuns lie down on the same sheet and under the same cover, they commit two kinds of offenses: when they are in the process of lying down, then for the effort there is an offense of wrong conduct; when they are lying down, they commit an offense entailing confession. 

When\marginnote{65.1} intentionally making a nun ill at ease, she commits two kinds of offenses: when she is doing it, then for the effort there is an offense of wrong conduct; when she has done it, she commits an offense entailing confession. 

When\marginnote{66.1} not nursing a suffering disciple, nor making any effort to have someone nurse her, she commits one kind of offense: an offense entailing confession. 

When\marginnote{67.1} giving a dwelling place to a nun, and then, in anger, throwing her out, she commits two kinds of offenses: when she is in the process of throwing her out, then for the effort there is an offense of wrong conduct; when she has thrown her out, she commits an offense entailing confession. 

When\marginnote{68.1} a socializing nun does not stop when pressed for the third time, she commits two kinds of offenses: after the motion, she commits an offense of wrong conduct; when the last announcement is finished, she commits an offense entailing confession. 

When\marginnote{69.1} wandering without a group of travelers where it is considered risky and dangerous within her own country, she commits two kinds of offenses: when she is traveling, then for the effort there is an offense of wrong conduct; when she has traveled, she commits an offense entailing confession. 

When\marginnote{70.1} wandering without a group of travelers where it is considered risky and dangerous outside her own country, she commits two kinds of offenses: when she is traveling, then for the effort there is an offense of wrong conduct; when she has traveled, she commits an offense entailing confession. 

When\marginnote{71.1} wandering during the rainy season, she commits two kinds of offenses: when she is traveling, then for the effort there is an offense of wrong conduct; when she has traveled, she commits an offense entailing confession. 

When\marginnote{72.1} a nun who has completed the rainy-season residence does not go wandering, she commits one kind of offense: an offense entailing confession. 

\scendvagga{The fourth subchapter on lying down is finished. }

\subsection*{The subchapter on galleries }

When\marginnote{74.1} visiting a royal house, a gallery, a park, a garden, or a lotus pond, she commits two kinds of offenses: when she is in the process of going there, she commits an offense of wrong conduct; wherever she stands to see them, she commits an offense entailing confession. 

When\marginnote{75.1} using a high or luxurious couch, she commits two kinds of offenses: when she is using it, then for the effort there is an offense of wrong conduct; when she has used it, she commits an offense entailing confession. 

When\marginnote{76.1} spinning yarn, she commits two kinds of offenses: when she is spinning, then for the effort there is an offense of wrong conduct; for every pull, she commits an offense entailing confession. 

When\marginnote{77.1} providing services for a householder, she commits two kinds of offenses: when she is in the process of providing them, then for the effort there is an offense of wrong conduct; when she has provided them, she commits an offense entailing confession. 

When\marginnote{78.1} agreeing, when requested by a nun, to resolve a legal issue, but then neither resolving it nor making any effort to resolve it, she commits one kind of offense: an offense entailing confession. 

When\marginnote{79.1} personally giving fresh or cooked food to a householder, a male wanderer, or a female wanderer, she commits two kinds of offenses: when she is in the process of giving, then for the effort there is an offense of wrong conduct; when she has given, she commits an offense entailing confession. 

When\marginnote{80.1} not relinquishing but continuing to use a communal robe, she commits two kinds of offenses: when she is using it, then for the effort there is an offense of wrong conduct; when she has used it, she commits an offense entailing confession. 

When\marginnote{81.1} going wandering without relinquishing her lodging, she commits two kinds of offenses: when she crosses the boundary with her first foot, she commits an offense of wrong conduct; when she crosses with her second foot, she commits an offense entailing confession. 

When\marginnote{82.1} studying worldly subjects, she commits two kinds of offenses: when she is studying, then for the effort there is an offense of wrong conduct; for every line, she commits an offense entailing confession. 

When\marginnote{83.1} teaching worldly subjects, she commits two kinds of offenses: when she is teaching, then for the effort there is an offense of wrong conduct; for every line, she commits an offense entailing confession. 

\scendvagga{The fifth subchapter on galleries is finished. }

\subsection*{The subchapter on monasteries }

When\marginnote{85.1} entering a monastery without asking permission, yet knowing that there are monks there, she commits two kinds of offenses: when she crosses the boundary with her first foot, she commits an offense of wrong conduct; when she crosses with her second foot, she commits an offense entailing confession. 

When\marginnote{86.1} abusing or reviling a monk, she commits two kinds of offenses: when she is in the process of abusing, then for the effort there is an offense of wrong conduct; when she is finished abusing, she commits an offense entailing confession. 

When\marginnote{87.1} furiously reviling the community, she commits two kinds of offenses: when she is in the process of reviling, then for the effort there is an offense of wrong conduct; when she is finished reviling, she commits an offense entailing confession. 

When\marginnote{88.1} invited to a meal, refusing an offer to eat more, and then eating fresh or cooked food, she commits two kinds of offenses: when receiving with the intention to eat, she commits an offense of wrong conduct; for every mouthful swallowed, she commits an offense entailing confession. 

When\marginnote{89.1} keeping a family to herself, she commits two kinds of offenses: when she is in the process of keeping it for herself, then for the effort there is an offense of wrong conduct; when she has kept it for herself, she commits an offense entailing confession. 

When\marginnote{90.1} spending the rainy-season residence in a monastery without monks, she commits two kinds of offenses: when she thinks, “I’ll stay here for the rainy-season residence,” and she prepares a dwelling, sets out water for drinking and water for washing, and sweeps the yard, she commits an offense of wrong conduct; at dawn, she commits an offense entailing confession. 

When\marginnote{91.1} a nun who has completed the rainy-season residence does not invite correction from both Sanghas in regard to three things, she commits one kind of offense: an offense entailing confession. 

When\marginnote{92.1} not going to the instruction or to a formal meeting of the community, she commits one kind of offense: an offense entailing confession. 

When\marginnote{93.1} not enquiring about the observance day nor asking for the instruction, she commits one kind of offense: an offense entailing confession. 

When\marginnote{94.1} being alone with a man and having him rupture an abscess or a wound situated on the lower part of her body, without getting permission from the Sangha or a group, she commits two kinds of offenses: when she is in the process of having it ruptured, then for the effort there is an offense of wrong conduct; when she has had it ruptured, she commits an offense entailing confession. 

\scendvagga{The sixth subchapter on monasteries is finished. }

\subsection*{The subchapter on pregnant women }

When\marginnote{96.1} giving the full admission to a pregnant woman, she commits two kinds of offenses: when she is in the process of giving the full admission, then for the effort there is an offense of wrong conduct; when she has given the full admission, she commits an offense entailing confession. 

When\marginnote{97.1} giving the full admission to a woman who is breastfeeding, she commits two kinds of offenses: when she is in the process of giving the full admission, then for the effort there is an offense of wrong conduct; when she has given the full admission, she commits an offense entailing confession. 

When\marginnote{98.1} giving the full admission to a trainee nun who has not trained for two years in the six rules, she commits two kinds of offenses: when she is in the process of giving the full admission, then for the effort there is an offense of wrong conduct; when she has given the full admission, she commits an offense entailing confession. 

When\marginnote{99.1} giving the full admission to a trainee nun who has trained for two years in the six rules, but who has not been approved by the Sangha, she commits two kinds of offenses: when she is in the process of giving the full admission, then for the effort there is an offense of wrong conduct; when she has given the full admission, she commits an offense entailing confession. 

When\marginnote{100.1} giving the full admission to a married girl who is less than twelve years old, she commits two kinds of offenses: when she is in the process of giving the full admission, then for the effort there is an offense of wrong conduct; when she has given the full admission, she commits an offense entailing confession. 

When\marginnote{101.1} giving the full admission to a married girl who is more than twelve years old, but who has not trained for two years in the six rules, she commits two kinds of offenses: when she is in the process of giving the full admission, then for the effort there is an offense of wrong conduct; when she has given the full admission, she commits an offense entailing confession. 

When\marginnote{102.1} giving the full admission to a married girl who is more than twelve years old and who has trained for two years in the six rules, but who has not been approved by the Sangha, she commits two kinds of offenses: when she is in the process of giving the full admission, then for the effort there is an offense of wrong conduct; when she has given the full admission, she commits an offense entailing confession. 

When\marginnote{103.1} giving the full admission to a disciple, and then, for the next two years, neither guiding her nor having her guided, she commits one kind of offense: an offense entailing confession. 

When\marginnote{104.1} not following the mentor who gave her the full admission for two years, she commits one kind of offense: an offense entailing confession. 

When\marginnote{105.1} giving the full admission to a disciple, and then neither sending her away nor having her sent away, she commits one kind of offense: an offense entailing confession. 

\scendvagga{The seventh subchapter on pregnant women is finished. }

\subsection*{The subchapter on unmarried women }

When\marginnote{107.1} giving the full admission to an unmarried woman who is less than twenty years old, she commits two kinds of offenses: when she is in the process of giving the full admission, then for the effort there is an offense of wrong conduct; when she has given the full admission, she commits an offense entailing confession. 

When\marginnote{108.1} giving the full admission to an unmarried woman who is more than twenty years old, but who has not trained for two years in the six rules, she commits two kinds of offenses: when she is in the process of giving the full admission, then for the effort there is an offense of wrong conduct; when she has given the full admission, she commits an offense entailing confession. 

When\marginnote{109.1} giving the full admission to an unmarried woman who is more than twenty years old and who has trained for two years in the six rules, but who has not been approved by the Sangha, she commits two kinds of offenses: when she is in the process of giving the full admission, then for the effort there is an offense of wrong conduct; when she has given the full admission, she commits an offense entailing confession. 

When\marginnote{110.1} one who has less than twelve years of seniority gives the full admission, she commits two kinds of offenses: when she is in the process of giving the full admission, then for the effort there is an offense of wrong conduct; when she has given the full admission, she commits an offense entailing confession. 

When\marginnote{111.1} one who has twelve years of seniority gives the full admission without approval from the Sangha, she commits two kinds of offenses: when she is in the process of giving the full admission, then for the effort there is an offense of wrong conduct; when she has given the full admission, she commits an offense entailing confession. 

When\marginnote{112.1} verbally consenting when being told, “Venerable, you’ve given enough full admissions for now,” but then criticizing it afterwards, she commits two kinds of offenses: when she is criticizing, then for the effort there is an offense of wrong conduct; when she has criticized, she commits an offense entailing confession. 

When\marginnote{113.1} telling a trainee nun, “If you give me a robe, venerable, I’ll give you the full admission,” but then neither giving her the full admission nor making any effort to have her fully admitted, she commits one kind of offense: an offense entailing confession. 

When\marginnote{114.1} telling a trainee nun, “If you follow me for two years, venerable, I’ll give you the full admission,” but then neither giving her the full admission nor making any effort to have her fully admitted, she commits one kind of offense: an offense entailing confession. 

When\marginnote{115.1} giving the full admission to a trainee nun who is socializing with men and boys and who is temperamental and difficult to live with, she commits two kinds of offenses: when she is in the process of giving the full admission, then for the effort there is an offense of wrong conduct; when she has given the full admission, she commits an offense entailing confession. 

When\marginnote{116.1} giving the full admission to a trainee nun who has not been given permission by her parents or her husband, she commits two kinds of offenses: when she is in the process of giving the full admission, then for the effort there is an offense of wrong conduct; when she has given the full admission, she commits an offense entailing confession. 

When\marginnote{117.1} giving the full admission to a trainee nun after a given consent has expired, she commits two kinds of offenses: when she is in the process of giving the full admission, then for the effort there is an offense of wrong conduct; when she has given the full admission, she commits an offense entailing confession. 

When\marginnote{118.1} giving full admission every year, she commits two kinds of offenses: when she is in the process of giving the full admission, then for the effort there is an offense of wrong conduct; when she has given the full admission, she commits an offense entailing confession. 

When\marginnote{119.1} giving the full admission to two women in the same year, she commits two kinds of offenses: when she is in the process of giving the full admission, then for the effort there is an offense of wrong conduct; when she has given the full admission, she commits an offense entailing confession. 

\scendvagga{The eighth subchapter on unmarried women is finished. }

\subsection*{The subchapter on sunshades and sandals }

When\marginnote{121.1} using a sunshade and sandals, she commits two kinds of offenses: when she is using them, then for the effort there is an offense of wrong conduct; when she has used them, she commits an offense entailing confession. 

When\marginnote{122.1} traveling in a vehicle, she commits two kinds of offenses: when she is traveling, then for the effort there is an offense of wrong conduct; when she has traveled, she commits an offense entailing confession. 

When\marginnote{123.1} wearing a hip ornament, she commits two kinds of offenses: when she is wearing it, then for the effort there is an offense of wrong conduct; when she has worn it, she commits an offense entailing confession. 

When\marginnote{124.1} wearing jewelry, she commits two kinds of offenses: when she is wearing it, then for the effort there is an offense of wrong conduct; when she has worn it, she commits an offense entailing confession. 

When\marginnote{125.1} bathing with scents and colors, she commits two kinds of offenses: when she is bathing, then for the effort there is an offense of wrong conduct; when she has bathed, she commits an offense entailing confession. 

When\marginnote{126.1} bathing with scented sesame paste, she commits two kinds of offenses: when she is bathing, then for the effort there is an offense of wrong conduct; when she has bathed, she commits an offense entailing confession. 

When\marginnote{127.1} having a nun massage or rub her, she commits two kinds of offenses: when she is getting massaged, then for the effort there is an offense of wrong conduct; when she has gotten massaged, she commits an offense entailing confession. 

When\marginnote{128.1} having a trainee nun massage or rub her, she commits two kinds of offenses: when she is getting massaged, then for the effort there is an offense of wrong conduct; when she has gotten massaged, she commits an offense entailing confession. 

When\marginnote{129.1} having a novice nun massage or rub her, she commits two kinds of offenses: when she is getting massaged, then for the effort there is an offense of wrong conduct; when she has gotten massaged, she commits an offense entailing confession. 

When\marginnote{130.1} having a female householder massage or rub her, she commits two kinds of offenses: when she is getting massaged, then for the effort there is an offense of wrong conduct; when she has gotten massaged, she commits an offense entailing confession. 

When\marginnote{131.1} sitting down on a seat in front of a monk without asking permission, she commits two kinds of offenses: when she is in the process of sitting down, then for the effort there is an offense of wrong conduct; when she is seated, she commits an offense entailing confession. 

When\marginnote{132.1} asking a question of a monk who has not given her permission, she commits two kinds of offenses: when she is asking, then for the effort there is an offense of wrong conduct; when she has asked, she commits an offense entailing confession. 

When\marginnote{133.1} entering an inhabited area without wearing her chest wrap, she commits two kinds of offenses: when she crosses the boundary with her first foot, she commits an offense of wrong conduct; when she crosses with her second foot, she commits an offense entailing confession. 

\scendvagga{The ninth subchapter on sunshades and sandals is finished. }

\scend{The section on minor rules is finished. }

\section*{5. The chapter on offenses entailing acknowledgment }

When\marginnote{135.1} asking for ghee and then eating it, she commits two kinds of offenses: when she receives it with the intention of eating it, she commits an offense of wrong conduct; for every mouthful swallowed, she commits an offense entailing acknowledgment. 

When\marginnote{136.1} asking for oil and then eating it, she commits two kinds of offenses: when she receives it with the intention of eating it, she commits an offense of wrong conduct; for every mouthful swallowed, she commits an offense entailing acknowledgment. 

When\marginnote{137.1} asking for honey and then eating it, she commits two kinds of offenses: when she receives it with the intention of eating it, she commits an offense of wrong conduct; for every mouthful swallowed, she commits an offense entailing acknowledgment. 

When\marginnote{138.1} asking for syrup and then eating it, she commits two kinds of offenses: when she receives it with the intention of eating it, she commits an offense of wrong conduct; for every mouthful swallowed, she commits an offense entailing acknowledgment. 

When\marginnote{139.1} asking for fish and then eating it, she commits two kinds of offenses: when she receives it with the intention of eating it, she commits an offense of wrong conduct; for every mouthful swallowed, she commits an offense entailing acknowledgment. 

When\marginnote{140.1} asking for meat and then eating it, she commits two kinds of offenses: when she receives it with the intention of eating it, she commits an offense of wrong conduct; for every mouthful swallowed, she commits an offense entailing acknowledgment. 

When\marginnote{141.1} asking for milk and then drinking it, she commits two kinds of offenses: when she receives it with the intention of drinking it, she commits an offense of wrong conduct; for every mouthful swallowed, she commits an offense entailing acknowledgment. 

When\marginnote{142.1} asking for curd and then eating it, she commits two kinds of offenses: when she receives it with the intention of eating it, she commits an offense of wrong conduct; for every mouthful swallowed, she commits an offense entailing acknowledgment. 

\scend{The eight offenses entailing acknowledgment are finished. }

\scendsutta{The number of offenses within each offense, the second, is finished. }

%
\chapter*{{\suttatitleacronym Pvr 2.3}{\suttatitletranslation The classes of failure for each offense }{\suttatitleroot Vipattivāra}}
\addcontentsline{toc}{chapter}{\tocacronym{Pvr 2.3} \toctranslation{The classes of failure for each offense } \tocroot{Vipattivāra}}
\markboth{The classes of failure for each offense }{Vipattivāra}
\extramarks{Pvr 2.3}{Pvr 2.3}

When\marginnote{1.1} it comes to the offenses for a lustful nun consenting to a lustful man making physical contact with her, to how many of the four kinds of failure do they belong? They belong to two kinds of failure: they may be failure in morality; they may be failure in conduct. … 

When\marginnote{2.1} it comes to the offenses for asking for curd and then eating it, to how many of the four kinds of failure do they belong? They belong to one kind of failure: failure in conduct. 

\scendsutta{The classes of failure for each offense, the third, are finished. }

%
\chapter*{{\suttatitleacronym Pvr 2.4}{\suttatitletranslation The classes of offenses in each offense }{\suttatitleroot Saṅgahavāra}}
\addcontentsline{toc}{chapter}{\tocacronym{Pvr 2.4} \toctranslation{The classes of offenses in each offense } \tocroot{Saṅgahavāra}}
\markboth{The classes of offenses in each offense }{Saṅgahavāra}
\extramarks{Pvr 2.4}{Pvr 2.4}

When\marginnote{1.1} it comes to the offenses for a lustful nun consenting to a lustful man making physical contact with her, in how many of the seven classes of offenses are they found? They are found in three: they may be in the class of offenses entailing expulsion; they may be in the class of serious offenses; they may be in the class of offenses of wrong conduct. … 

When\marginnote{2.1} it comes to the offenses for asking for curd and then eating it, in how many of the seven classes of offenses are they found? They are found in two: they may be in the class of offenses entailing acknowledgment; they may be in the class of offenses of wrong conduct. … 

\scendsutta{The classes of offenses in each offense, the fourth, are finished. }

%
\chapter*{{\suttatitleacronym Pvr 2.5}{\suttatitletranslation The originations of each offense }{\suttatitleroot Samuṭṭhānavāra}}
\addcontentsline{toc}{chapter}{\tocacronym{Pvr 2.5} \toctranslation{The originations of each offense } \tocroot{Samuṭṭhānavāra}}
\markboth{The originations of each offense }{Samuṭṭhānavāra}
\extramarks{Pvr 2.5}{Pvr 2.5}

When\marginnote{1.1} it comes to the offenses for a lustful nun consenting to a lustful man making physical contact with her, through how many of the six kinds of originations of offenses do they originate? They originate in one way: from body and mind, not from speech. … 

When\marginnote{2.1} it comes to the offenses for asking for curd and then eating it, through how many of the six kinds of originations of offenses do they originate? They originate in four ways: from body, not from speech or mind; or from body and speech, not from mind; or from body and mind, not from speech; or from body, speech, and mind. 

\scendsutta{The originations of each offense, the fifth, are finished. }

%
\chapter*{{\suttatitleacronym Pvr 2.6}{\suttatitletranslation The legal issues to which each offense belongs }{\suttatitleroot Adhikaraṇavāra}}
\addcontentsline{toc}{chapter}{\tocacronym{Pvr 2.6} \toctranslation{The legal issues to which each offense belongs } \tocroot{Adhikaraṇavāra}}
\markboth{The legal issues to which each offense belongs }{Adhikaraṇavāra}
\extramarks{Pvr 2.6}{Pvr 2.6}

When\marginnote{1.1} it comes to the offenses for a lustful nun consenting to a lustful man making physical contact with her, to which of the four kinds of legal issues do they belong? They belong to legal issues arising from an offense. … 

When\marginnote{2.1} it comes to the offenses for asking for curd and then eating it, to which of the four kinds of legal issues do they belong? They belong to legal issues arising from an offense. 

\scendsutta{The legal issues to which each offense belongs, the sixth, are finished. }

%
\chapter*{{\suttatitleacronym Pvr 2.7}{\suttatitletranslation How each offense is settled }{\suttatitleroot Samathavāra}}
\addcontentsline{toc}{chapter}{\tocacronym{Pvr 2.7} \toctranslation{How each offense is settled } \tocroot{Samathavāra}}
\markboth{How each offense is settled }{Samathavāra}
\extramarks{Pvr 2.7}{Pvr 2.7}

When\marginnote{1.1} it comes to the offenses for a lustful nun consenting to a lustful man making physical contact with her, through how many of the seven principles for settling legal issues are they settled? Through three of them: they may be settled by resolution face-to-face and by acting according to what has been admitted; or they may be settled by resolution face-to-face and by covering over as if with grass. … 

When\marginnote{2.1} it comes to the offenses for asking for curd and then eating it, through how many of the seven principles for settling legal issues are they settled? Through three of them: they may be settled by resolution face-to-face and by acting according to what has been admitted; or they may be settled by resolution face-to-face and by covering over as if with grass. 

\scendsutta{How each offense is settled, the seventh, is finished. }

%
\chapter*{{\suttatitleacronym Pvr 2.8}{\suttatitletranslation Summary of the previous six sections }{\suttatitleroot Samuccayavāra}}
\addcontentsline{toc}{chapter}{\tocacronym{Pvr 2.8} \toctranslation{Summary of the previous six sections } \tocroot{Samuccayavāra}}
\markboth{Summary of the previous six sections }{Samuccayavāra}
\extramarks{Pvr 2.8}{Pvr 2.8}

When\marginnote{1.1} a lustful nun consents to a lustful man making physical contact with her, how many kinds of offenses does she commit? She commits three kinds of offenses: when she consents to him taking hold of her anywhere below the collar bone but above the knees, she commits an offense entailing expulsion; when she consents to him taking hold of her above the collar bone or below the knees, she commits a serious offense; when she consents to him taking hold of something connected to her body, she commits an offense of wrong conduct. 

When\marginnote{2.1} it comes to these offenses, to how many of the four kinds of failure do they belong? In how many of the seven classes of offenses are they found? Through how many of the six kinds of originations of offenses do they originate? To which of the four kinds of legal issues do they belong? Through how many of the seven principles for settling legal issues are they settled? 

They\marginnote{2.6} belong to two kinds of failure: they may be failure in morality; they may be failure in conduct. They are found in three classes of offenses: they may be in the class of offenses entailing expulsion; they may be in the class of serious offenses; they may be in the class of offenses of wrong conduct. They originate in one way: from body and mind, not from speech. They belong to legal issues arising from an offense. They are settled through three principles: they may be settled by resolution face-to-face and by acting according to what has been admitted; or they may be settled by resolution face-to-face and by covering over as if with grass. … 

When\marginnote{3.1} asking for curd and then eating it, how many kinds of offenses does she commit? She commits two kinds of offenses: when she receives it with the intention of eating it, she commits an offense of wrong conduct; for every mouthful swallowed, she commits an offense entailing acknowledgment. 

When\marginnote{4.1} it comes to these offenses, to how many of the four kinds of failure do they belong? In how many of the seven classes of offenses are they found? Through how many of the six kinds of originations of offenses do they originate? To which of the four kinds of legal issues do they belong? Through how many of the seven principles for settling legal issues are they settled? 

They\marginnote{4.6} belong to one kind of failure: failure in conduct. They are found in two classes of offenses: they may be in the class of offenses entailing acknowledgment; they may be in the class of offenses of wrong conduct. They originate in four ways: from body, not from speech or mind; or from body and speech, not from mind; or from body and mind, not from speech; or from body, speech, and mind. They belong to legal issues arising from an offense. They are settled through three principles: they may be settled by resolution face-to-face and by acting according to what has been admitted; or they may be settled by resolution face-to-face and by covering over as if with grass. 

\scendsutta{The summary of the previous six sections, the eighth, is finished. }

%
\chapter*{{\suttatitleacronym Pvr 2.9}{\suttatitletranslation Questions and answers on the nuns’ Pātimokkha rules and their analysis }{\suttatitleroot Katthapaññattivāra}}
\addcontentsline{toc}{chapter}{\tocacronym{Pvr 2.9} \toctranslation{Questions and answers on the nuns’ Pātimokkha rules and their analysis } \tocroot{Katthapaññattivāra}}
\markboth{Questions and answers on the nuns’ Pātimokkha rules and their analysis }{Katthapaññattivāra}
\extramarks{Pvr 2.9}{Pvr 2.9}

\section*{1. The chapter on offenses entailing expulsion }

“The\marginnote{1.1} offense entailing expulsion that is a result of consenting to physical contact was laid down by the Buddha who knows and sees, the Perfected One, the fully Awakened One. Where was it laid down? Whom is it about? What is it about? … Who handed it down?” 

“The\marginnote{2.1} offense entailing expulsion that is a result of consenting to physical contact was laid down by the Buddha who knows and sees, the Perfected One, the fully Awakened One. Where was it laid down?” At \textsanskrit{Sāvatthī}. “Whom is it about?” The nun \textsanskrit{Sundarīnandā}. “What is it about?” The lustful nun \textsanskrit{Sundarīnandā} consenting to a lustful man making physical contact with her. “Is there a rule, an addition to the rule, an unprompted rule?” There is one rule. There is no addition to the rule. There is no unprompted rule. “Is it a rule that applies everywhere or in a particular place?” Everywhere. “Is it a rule that the monks and nuns have in common or not in common?” Not in common. “Is it a rule for one Sangha or for both?” For one. “In which of the four ways of reciting the Monastic Code is it contained and included?” In the introduction. “In which recitation is it included?” In the second recitation. “To which of the four kinds of failure does it belong?” Failure in morality. “To which of the seven classes of offenses does it belong?” The class of offenses entailing expulsion. “Through how many of the six kinds of originations of offenses does it originate?” It originates in one way: from body and mind, not from speech. … “Who handed it down?” The lineage: 

\begin{verse}%
“\textsanskrit{Upāli}\marginnote{3.1} and \textsanskrit{Dāsaka}, \\
\textsanskrit{Soṇaka} and so Siggava; \\
With Moggaliputta as the fifth—\\
These were in India, the land named after the glorious rose apple. 

…\marginnote{4.1} 

These\marginnote{5.1} mighty beings of great wisdom, \\
Knowers of the Monastic Law and skilled in the path; \\
Proclaimed the Collection of Monastic Law, \\
On the island of Sri Lanka.” 

%
\end{verse}

“There\marginnote{6.1} is an offense entailing expulsion that is a result of concealing an offense. Where was it laid down?” At \textsanskrit{Sāvatthī}. “Whom is it about?” The nun \textsanskrit{Thullanandā}. “What is it about?” The nun \textsanskrit{Thullanandā}, knowing that a nun had committed an offense entailing expulsion, neither confronting her herself nor telling the community. There is one rule. Of the six kinds of originations of offenses, it originates in one way: through abandoning one’s duty. …\footnote{There is an inconsistency here. According to the parallel passage at \href{https://suttacentral.net/pli-tv-pvr2.1/en/brahmali\#19.9}{Pvr 2.1:19.9} this should read \textit{\textsanskrit{kāyato} ca \textsanskrit{vācato} ca cittato ca \textsanskrit{samuṭṭhāti}}, “from body, speech, and mind”. } 

“There\marginnote{7.1} is an offense entailing expulsion that is a result of not stopping when pressed for the third time. Where was it laid down?” At \textsanskrit{Sāvatthī}. “Whom is it about?” The nun \textsanskrit{Thullanandā}. “What is it about?” The nun \textsanskrit{Thullanandā} taking sides with the monk \textsanskrit{Ariṭṭha}, an ex-vulture-killer, who had been ejected by a unanimous Sangha. There is one rule. Of the six kinds of originations of offenses, it originates in one way: through abandoning one’s duty. … 

“There\marginnote{8.1} is an offense entailing expulsion that is a result of fulfilling the eight parts. Where was it laid down?” At \textsanskrit{Sāvatthī}. “Whom is it about?” The nuns from the group of six. “What is it about?” The nuns from the group of six fulfilling the eight parts. There is one rule. Of the six kinds of originations of offenses, it originates in one way: through abandoning one’s duty. … 

\scendsutta{The offenses entailing expulsion are finished. }

\section*{2. The chapter on offenses entailing suspension, etc. }

“The\marginnote{9.1} offense entailing suspension that is a result of a litigious nun initiating a lawsuit was laid down by the Buddha who knows and sees, the Perfected One, the fully Awakened One. Where was it laid down? Whom is it about? What is it about? … Who handed it down?” 

“The\marginnote{10.1} offense entailing suspension that is a result of a litigious nun initiating a lawsuit was laid down by the Buddha who knows and sees, the Perfected One, the fully Awakened One. Where was it laid down?” At \textsanskrit{Sāvatthī}. “Whom is it about?” The nun \textsanskrit{Thullanandā}. “What is it about?” The nun \textsanskrit{Thullanandā} taking legal action. “Is there a rule, an addition to the rule, an unprompted rule?” There is one rule. There is no addition to the rule. There is no unprompted rule. “Is it a rule that applies everywhere or in a particular place?” Everywhere. “Is it a rule that the monks and nuns have in common or not in common?” Not in common. “Is it a rule for one Sangha or for both?” For one. “In which of the four ways of reciting the Monastic Code is it contained and included?” In the introduction. “In which recitation is it included?” In the third recitation. “To which of the four kinds of failure does it belong?” Failure in morality. “To which of the seven classes of offenses does it belong?” The class of offenses entailing suspension. “Through how many of the six kinds of originations of offenses does it originate?” It originates in two ways: from body and speech, not from mind; or from body, speech, and mind. … “Who handed it down?” The lineage: 

\begin{verse}%
“\textsanskrit{Upāli}\marginnote{11.1} and \textsanskrit{Dāsaka}, \\
\textsanskrit{Soṇaka} and so Siggava; \\
With Moggaliputta as the fifth—\\
These were in India, the land named after the glorious rose apple. 

…\marginnote{12.1} 

These\marginnote{13.1} mighty beings of great wisdom, \\
Knowers of the Monastic Law and skilled in the path; \\
Proclaimed the Collection of Monastic Law, \\
On the island of Sri Lanka.” 

%
\end{verse}

“There\marginnote{14.1} is an offense entailing suspension that is a result of giving the full admission to a female criminal. Where was it laid down?” At \textsanskrit{Sāvatthī}. “Whom is it about?” The nun \textsanskrit{Thullanandā}. “What is it about?” The nun \textsanskrit{Thullanandā} giving the full admission to a female criminal. There is one rule. Of the six kinds of originations of offenses, it originates in two ways: from speech and mind, not from body; or from body, speech, and mind. … 

“There\marginnote{15.1} is an offense entailing suspension that is a result of walking to the next inhabited area by oneself. Where was it laid down?” At \textsanskrit{Sāvatthī}. “Whom is it about?” A certain nun. “What is it about?” A certain nun walking to the next village by herself. There is one rule. There are three additions to the rule. Of the six kinds of originations of offenses, it originates in one way: … (as in the first offense entailing expulsion) … 

“There\marginnote{16.1} is an offense entailing suspension that is a result of readmitting a nun who had been ejected by a unanimous Sangha in accordance with the Teaching, the Monastic Law, and the Teacher’s instruction, without first getting permission from the Sangha that did the legal procedure and without the consent of the community. Where was it laid down?” At \textsanskrit{Sāvatthī}. “Whom is it about?” The nun \textsanskrit{Thullanandā}. “What is it about?” The nun \textsanskrit{Thullanandā} readmitting a nun who had been ejected by a unanimous Sangha in accordance with the Teaching, the Monastic Law, and the Teacher’s instruction, without first getting permission from the Sangha that did the legal procedure and without the consent of the community. There is one rule. Of the six kinds of originations of offenses, it originates in one way: through abandoning one’s duty. … 

“There\marginnote{17.1} is an offense entailing suspension that is a result of a lustful nun eating fresh or cooked food after receiving it directly from a lustful man. Where was it laid down?” At \textsanskrit{Sāvatthī}. “Whom is it about?” The nun \textsanskrit{Sundarīnandā}. “What is it about?” The nun \textsanskrit{Sundarīnandā}, being lustful, receiving food directly from a lustful man. There is one rule. Of the six kinds of originations of offenses, it originates in one way: … (as in the first offense entailing expulsion) … 

“There\marginnote{18.1} is an offense entailing suspension that is a result of urging a nun on, saying, ‘Venerable, what can this man do to you, whether he has lust or not, if you’re without? Go on, venerable, receive it with your own hands and then eat whatever fresh or cooked food he gives you.’ Where was it laid down?” At \textsanskrit{Sāvatthī}. “Whom is it about?” A certain nun. “What is it about?” A certain nun urging a nun on, saying, “Venerable, what can this man do to you, whether he has lust or not, if you’re without? Go on, venerable, receive it with your own hands and then eat whatever fresh or cooked food he gives you.” There is one rule. Of the six kinds of originations of offenses, it originates in three ways: … 

“There\marginnote{19.1} is an offense entailing suspension that is a result of an angry nun not stopping when pressed for the third time. Where was it laid down?” At \textsanskrit{Sāvatthī}. “Whom is it about?” The nun \textsanskrit{Caṇḍakālī}. “What is it about?” The nun \textsanskrit{Caṇḍakālī} saying in anger, “I renounce the Buddha, I renounce the Teaching, I renounce the Sangha, I renounce the training!” There is one rule. Of the six kinds of originations of offenses, it originates in one way: through abandoning one’s duty. … 

“There\marginnote{20.1} is an offense entailing suspension that is a result of a nun who has lost a legal case not stopping when pressed for the third time. Where was it laid down?” At \textsanskrit{Sāvatthī}. “Whom is it about?” The nun \textsanskrit{Caṇḍakālī}. “What is it about?” The nun \textsanskrit{Caṇḍakālī}, who had lost a legal case, saying in anger, “The nuns are acting out of favoritism, ill will, confusion, and fear.” There is one rule. Of the six kinds of originations of offenses, it originates in one way: through abandoning one’s duty. … 

“There\marginnote{21.1} is an offense entailing suspension that is a result of socializing nuns not stopping when pressed for the third time. Where was it laid down?” At \textsanskrit{Sāvatthī}. “Whom is it about?” A number of nuns. “What is it about?” A number of nuns socializing. There is one rule. Of the six kinds of originations of offenses, it originates in one way: through abandoning one’s duty. … 

“There\marginnote{22.1} is an offense entailing suspension that is a result of urging nuns on in this way: ‘Venerables, you should socialize. Don’t live separately,’ and then not stopping when pressed for the third time. Where was it laid down?” At \textsanskrit{Sāvatthī}. “Whom is it about?” The nun \textsanskrit{Thullanandā}. “What is it about?” The nun \textsanskrit{Thullanandā} urging the nuns on, saying, “Venerables, you should socialize. Don’t live separately.” There is one rule. Of the six kinds of originations of offenses, it originates in one way: through abandoning one’s duty. … 

“There\marginnote{23.1} is an offense entailing acknowledgment that is a result of asking for curd and then eating it. Where was it laid down?” At \textsanskrit{Sāvatthī}. “Whom is it about?” The nuns from the group of six. “What is it about?” The nuns from the group of six asking for curd and then eating it. There is one rule. There is one addition to the rule. Of the six kinds of originations of offenses, it originates in four ways: … 

\scendsutta{The questions and answers on the nuns’ \textsanskrit{Pātimokkha} rules and their analysis, the first, are finished. }

%
\chapter*{{\suttatitleacronym Pvr 2.10}{\suttatitletranslation The number of offenses within each offense }{\suttatitleroot Katāpattivāra}}
\addcontentsline{toc}{chapter}{\tocacronym{Pvr 2.10} \toctranslation{The number of offenses within each offense } \tocroot{Katāpattivāra}}
\markboth{The number of offenses within each offense }{Katāpattivāra}
\extramarks{Pvr 2.10}{Pvr 2.10}

\section*{1. The chapter on offenses entailing expulsion }

How\marginnote{1.1} many kinds of offenses does one commit as a result of consenting to physical contact? One commits five kinds of offenses: when a lustful nun consents to a lustful man taking hold of her anywhere below the collar bone but above the knees, she commits an offense entailing expulsion; when a monk makes physical contact, body with body, he commits an offense entailing suspension; when, with one’s own body, one makes physical contact with something connected to their body, one commits a serious offense; when, with something connected to one’s own body, one makes physical contact with something connected to their body, one commits an offense of wrong conduct; for tickling, one commits an offense entailing confession. 

How\marginnote{2.1} many kinds of offenses does one commit as a result of concealing an offense? One commits four kinds of offenses: when a nun knowingly conceals an offense entailing expulsion, she commits an offense entailing expulsion; when, being unsure, she conceals it, she commits a serious offense;\footnote{This ruling does not seem to be found in the Canonical text, either for monks or nuns. } when a monk conceals an offense entailing suspension, he commits an offense entailing confession; when one conceals a failure in conduct, one commits an offense of wrong conduct. 

“How\marginnote{3.1} many kinds of offenses does one commit as a result of not stopping when pressed for the third time?” One commits five kinds of offenses: when a nun takes sides with one who has been ejected and does not stop when pressed for the third time, then after the motion, she commits an offense of wrong conduct; after each of the first two announcements, she commits a serious offense; when the last announcement is finished, she commits an offense entailing expulsion; when a nun sides with a monk who is pursuing schism and she does not stop when pressed for the third time, she commits an offense entailing suspension; when not giving up a bad view after being pressed for the third time, one commits an offense entailing confession. 

How\marginnote{4.1} many kinds of offenses does she commit as a result of fulfilling the eight parts? She commits three kinds of offenses: when she goes to such-and-such a place when told by a man to do so, she commits an offense of wrong conduct; when she enters within arm’s reach of the man, she commits a serious offense; when she fulfills the eight parts, she commits an offense entailing expulsion. 

\scend{The offenses entailing expulsion are finished. }

\section*{2. The chapter on offenses entailing suspension, etc. }

As\marginnote{5.1} a result of initiating a lawsuit, a litigious nun commits three kinds of offenses: when she tells one other person, she commits an offense of wrong conduct; when she tells a second person, she commits a serious offense; when the lawsuit is finished, she commits an offense entailing suspension. 

As\marginnote{6.1} a result of giving the full admission to a female criminal, she commits three kinds of offenses: after the motion, she commits an offense of wrong conduct; after each of the first two announcements, she commits a serious offense; when the last announcement is finished, she commits an offense entailing suspension. 

As\marginnote{7.1} a result of walking to the next inhabited area by herself, she commits three kinds of offenses: when she is in the process of going, she commits an offense of wrong conduct; when she crosses the boundary with her first foot, she commits a serious offense; when she crosses with her second foot, she commits an offense entailing suspension. 

As\marginnote{8.1} a result of readmitting—without first getting permission from the Sangha that did the legal procedure and without the consent of the community—a nun who had been ejected by a unanimous Sangha in accordance with the Teaching, the Monastic Law, and the Teacher’s instruction, she commits three kinds of offenses: after the motion, she commits an offense of wrong conduct; after each of the first two announcements, she commits a serious offense; when the last announcement is finished, she commits an offense entailing suspension. 

As\marginnote{9.1} a result of eating fresh or cooked food after receiving it directly from a lustful man, a lustful nun commits three kinds of offenses: when she receives fresh or cooked food with the intention of eating it, she commits a serious offense; for every mouthful swallowed, she commits an offense entailing suspension; if she receives water or a tooth cleaner, she commits an offense of wrong conduct. 

As\marginnote{10.1} a result of urging a nun on, saying, “Venerable, what can this man do to you, whether he has lust or not, if you’re without? Go on, venerable, receive it with your own hands and then eat whatever fresh or cooked food he gives you,” she commits three kinds of offenses: when, because of her statement, the other nun receives it with the intention of eating it, she commits an offense of wrong conduct; for every mouthful swallowed, she commits a serious offense; when the meal is finished, she commits an offense entailing suspension. 

As\marginnote{11.1} a result of not stopping when pressed for the third time, an angry nun commits three kinds of offenses: after the motion, she commits an offense of wrong conduct; after each of the first two announcements, she commits a serious offense; when the last announcement is finished, she commits an offense entailing suspension. 

As\marginnote{12.1} a result of not stopping when pressed for the third time, a nun who has lost a legal case commits three kinds of offenses: after the motion, she commits an offense of wrong conduct; after each of the first two announcements, she commits a serious offense; when the last announcement is finished, she commits an offense entailing suspension. 

As\marginnote{13.1} a result of not stopping when pressed for the third time, a socializing nun commits three kinds of offenses: after the motion, she commits an offense of wrong conduct; after each of the first two announcements, she commits a serious offense; when the last announcement is finished, she commits an offense entailing suspension. 

As\marginnote{14.1} a result of urging the nuns on, saying, “Venerables, you should socialize. Don’t live separately,” and not stopping when pressed for the third time, she commits three kinds of offenses: after the motion, she commits an offense of wrong conduct; after each of the first two announcements, she commits a serious offense; when the last announcement is finished, she commits an offense entailing suspension. 

\scend{The ten rules entailing suspension are finished. … }

(To\marginnote{16.1} be expanded as above, (pli-tv-pvr2.2:17.0) to (pli-tv-pvr2.2:141.3), with the only difference being the addition of “as a result of”.) 

How\marginnote{17.1} many kinds of offenses does she commit as a result of asking for curd and then eating it? She commits two kinds of offenses: when she receives it with the intention of eating it, she commits an offense of wrong conduct; for every mouthful swallowed, she commits an offense entailing acknowledgment. 

\scendsutta{The number of offenses within each offense, the second, is finished. }

%
\chapter*{{\suttatitleacronym Pvr 2.11}{\suttatitletranslation The classes of failure for each offense }{\suttatitleroot Vipattivāra}}
\addcontentsline{toc}{chapter}{\tocacronym{Pvr 2.11} \toctranslation{The classes of failure for each offense } \tocroot{Vipattivāra}}
\markboth{The classes of failure for each offense }{Vipattivāra}
\extramarks{Pvr 2.11}{Pvr 2.11}

When\marginnote{1.1} it comes to the offenses that are a result of consenting to physical contact, to how many of the four kinds of failure do they belong? They belong to two kinds of failure: they may be failure in morality; they may be failure in conduct. … 

When\marginnote{2.1} it comes to the offenses that are a result of asking for curd and then eating it, to how many of the four kinds of failure do they belong? They belong to one kind of failure: failure in conduct. 

\scendsutta{The classes of failure for each offense, the third, are finished. }

%
\chapter*{{\suttatitleacronym Pvr 2.12}{\suttatitletranslation The classes of offenses in each offense }{\suttatitleroot Saṅgahavāra}}
\addcontentsline{toc}{chapter}{\tocacronym{Pvr 2.12} \toctranslation{The classes of offenses in each offense } \tocroot{Saṅgahavāra}}
\markboth{The classes of offenses in each offense }{Saṅgahavāra}
\extramarks{Pvr 2.12}{Pvr 2.12}

When\marginnote{1.1} it comes to the offenses that are a result of consenting to physical contact, in how many of the seven classes of offenses are they found? They are found in five: they may be in the class of offenses entailing expulsion; they may be in the class of offenses entailing suspension; they may be in the class of serious offenses; they may be in the class of offenses entailing confession; they may be in the class of offenses of wrong conduct. … 

When\marginnote{2.1} it comes to the offenses that are a result of asking for curd and then eating it, in how many of the seven classes of offenses are they found? They are found in two: they may be in the class of offenses entailing acknowledgment; they may be in the class of offenses of wrong conduct. 

\scendsutta{The classes of offenses in each offense, the fourth, are finished. }

%
\chapter*{{\suttatitleacronym Pvr 2.13}{\suttatitletranslation The originations of each offense }{\suttatitleroot Samuṭṭhānavāra}}
\addcontentsline{toc}{chapter}{\tocacronym{Pvr 2.13} \toctranslation{The originations of each offense } \tocroot{Samuṭṭhānavāra}}
\markboth{The originations of each offense }{Samuṭṭhānavāra}
\extramarks{Pvr 2.13}{Pvr 2.13}

When\marginnote{1.1} it comes to the offenses that are a result of consenting to physical contact, through how many of the six kinds of originations of offenses do they originate? They originate in one way: from body and mind, not from speech. … 

When\marginnote{2.1} it comes to the offenses that are a result of asking for curd and then eating it, through how many of the six kinds of originations of offenses do they originate? They originate in four ways: from body, not from speech or mind; or from body and speech, not from mind; or from body and mind, not from speech; or from body, speech, and mind. 

\scendsutta{The originations of each offense, the fifth, are finished. }

%
\chapter*{{\suttatitleacronym Pvr 2.14}{\suttatitletranslation The legal issues to which each offense belongs }{\suttatitleroot Adhikaraṇavāra}}
\addcontentsline{toc}{chapter}{\tocacronym{Pvr 2.14} \toctranslation{The legal issues to which each offense belongs } \tocroot{Adhikaraṇavāra}}
\markboth{The legal issues to which each offense belongs }{Adhikaraṇavāra}
\extramarks{Pvr 2.14}{Pvr 2.14}

When\marginnote{1.1} it comes to the offenses that are a result of consenting to physical contact, to which of the four kinds of legal issues do they belong? They belong to legal issues arising from an offense. … 

When\marginnote{2.1} it comes to the offenses that are a result of asking for curd and then eating it, to which of the four kinds of legal issues do they belong? They belong to legal issues arising from an offense. 

\scendsutta{The legal issues to which each offense belongs, the sixth, are finished. }

%
\chapter*{{\suttatitleacronym Pvr 2.15}{\suttatitletranslation How each offense is settled }{\suttatitleroot Samathavāra}}
\addcontentsline{toc}{chapter}{\tocacronym{Pvr 2.15} \toctranslation{How each offense is settled } \tocroot{Samathavāra}}
\markboth{How each offense is settled }{Samathavāra}
\extramarks{Pvr 2.15}{Pvr 2.15}

When\marginnote{1.1} it comes to the offenses that are a result of consenting to physical contact, through how many of the seven principles for settling legal issues are they settled? Through three of them: they may be settled by resolution face-to-face and by acting according to what has been admitted; or they may be settled by resolution face-to-face and by covering over as if with grass. … 

When\marginnote{2.1} it comes to the offenses that are a result of asking for curd and then eating it, through how many of the seven principles for settling legal issues are they settled? Through three of them: they may be settled by resolution face-to-face and by acting according to what has been admitted; or they may be settled by resolution face-to-face and by covering over as if with grass. 

\scendsutta{How each offense is settled, the seventh, is finished. }

%
\chapter*{{\suttatitleacronym Pvr 2.16}{\suttatitletranslation Summary of the previous six sections }{\suttatitleroot Samuccayavāra}}
\addcontentsline{toc}{chapter}{\tocacronym{Pvr 2.16} \toctranslation{Summary of the previous six sections } \tocroot{Samuccayavāra}}
\markboth{Summary of the previous six sections }{Samuccayavāra}
\extramarks{Pvr 2.16}{Pvr 2.16}

As\marginnote{1.1} a result of consenting to physical contact, how many kinds of offenses does one commit? One commits five kinds of offenses: when a lustful nun consents to a lustful man taking hold of her anywhere below the collar bone but above the knees, she commits an offense entailing expulsion; when a monk makes physical contact, body with body, he commits an offense entailing suspension; when, with one’s own body, one makes physical contact with something connected to their body, one commits a serious offense; when, with something connected to one’s own body, one makes physical contact with something connected to their body, one commits an offense of wrong conduct; for tickling, one commits an offense entailing confession. 

When\marginnote{2.1} it comes to these offenses, to how many of the four kinds of failure do they belong? In how many of the seven classes of offenses are they found? Through how many of the six kinds of originations of offenses do they originate? To which of the four kinds of legal issues do they belong? Through how many of the seven principles for settling legal issues are they settled? 

They\marginnote{2.6} belong to two kinds of failure: they may be failure in morality; they may be failure in conduct. They are found in five classes of offenses: they may be in the class of offenses entailing expulsion; they may be in the class of offenses entailing suspension; they may be in the class of serious offenses; they may be in the class of offenses entailing confession; they may be in the class of offenses of wrong conduct. They originate in one way: from body and mind, not from speech. They belong to legal issues arising from an offense. They are settled through three principles: they may be settled by resolution face-to-face and by acting according to what has been admitted; or they may be settled by resolution face-to-face and by covering over as if with grass. … 

As\marginnote{3.1} a result of asking for curd and then eating it, how many kinds of offenses does she commit? She commits two kinds of offenses: when she receives it with the intention of eating it, she commits an offense of wrong conduct; for every mouthful swallowed, she commits an offense entailing acknowledgment. 

When\marginnote{4.1} it comes to these offenses, to how many of the four kinds of failure do they belong? In how many of the seven classes of offenses are they found? Through how many of the six kinds of originations of offenses do they originate? To which of the four kinds of legal issues do they belong? Through how many of the seven principles for settling legal issues are they settled? 

They\marginnote{4.6} belong to one kind of failure: failure in conduct. They are found in two classes of offenses: they may be in the class of offenses entailing acknowledgment; they may be in the class of offenses of wrong conduct. They originate in four ways: from body, not from speech or mind; or from body and speech, not from mind; or from body and mind, not from speech; or from body, speech, and mind. They belong to legal issues arising from an offense. They are settled through three principles: they may be settled by resolution face-to-face and by acting according to what has been admitted; or they may be settled by resolution face-to-face and by covering over as if with grass. 

\scendsutta{The summary of the previous six sections, the eighth, is finished. }

\scend{The eight sections on “as a result of” are finished. }

\scend{The sixteen great sections of the Nuns’ Analysis are finished. }

%
\chapter*{{\suttatitleacronym Pvr 3}{\suttatitletranslation The origination of offenses }{\suttatitleroot Samuṭṭhānasīsasaṅkhepa}}
\addcontentsline{toc}{chapter}{\tocacronym{Pvr 3} \toctranslation{The origination of offenses } \tocroot{Samuṭṭhānasīsasaṅkhepa}}
\markboth{The origination of offenses }{Samuṭṭhānasīsasaṅkhepa}
\extramarks{Pvr 3}{Pvr 3}

\section*{Summary of originations }

\begin{verse}%
“All\marginnote{1.1} phenomena are impermanent, \\
And suffering, nonself, made up; \\
Indeed the description extinguishment, \\
Is the conviction about nonself. 

When\marginnote{2.1} the Buddha moon has not appeared, \\
When the Buddha sun has not yet risen; \\
Then even the name is not known, \\
Of the things that are the same as those.\footnote{This renders \textit{\textsanskrit{tesaṁ} \textsanskrit{sabhāgadhammānaṁ}}. The point seems to be that the names \textit{anicca}, etc., mentioned above, are not even known if a Buddha does not arise. } 

Having\marginnote{3.1} done many difficult things, \\
Having fulfilled the perfections; \\
The Great Heroes emerge, \\
Endowed with vision, in this world with its supreme beings.\footnote{Reading \textit{sabrahmake} as a shorthand for \textit{sabrahmake loke}. } 

They\marginnote{4.1} instruct in the True Teaching, \\
That ends suffering and brings happiness; \\
\textsanskrit{Angīrasa}, the Sakyan Sage, \\
Who has compassion for all beings. 

The\marginnote{5.1} best of all creatures, the lion, \\
Taught the three Collections: \\
The Discourses, and the Philosophy, \\
And the Monastic Law, of great quality. 

Thus\marginnote{6.1} the true Teaching carries on, \\
So long as the Monastic Law remains; \\
As well as both Analyses, \\
The Chapters, and the Key Terms—

A\marginnote{7.1} garland bound, \\
By the quality of the string of the Compendium; \\
In this very Compendium, \\
Originations are determined. 

Combination\marginnote{8.1} and source are another, \\
Which are pointed out in the list below. \\
Therefore, they should train in the Compendium, \\
Those who love the Teaching and are highly virtuous.” 

%
\end{verse}

\section*{The thirteen originations }

\begin{verse}%
“Laid\marginnote{9.1} down in the two analyses, \\
They recite on the observance day; \\
I will declare the origination, \\
According to the method: listen to me. 

The\marginnote{10.1} first offense entailing expulsion,\footnote{Starting with the first offense entailing expulsion, here are the headings for the thirteen categories of origination, that is, offenses that are \textit{like} the first \textit{\textsanskrit{pārājika}}, etc. } \\
And then the second; \\
Matchmaking and pressing, \\
And an extra robe. 

Wool,\marginnote{11.1} memorizing the Teaching, \\
True, and by arrangement; \\
Thieves, teaching, and female criminal, \\
Lack of permission is the thirteenth. 

These\marginnote{12.1} are the thirteen originations, \\
The method thought out by the wise; \\
In regard to a single origination, \\
Those that are alike are shown here. 

%
\end{verse}

\subsection*{1. Originations like the first offense entailing expulsion }

\begin{verse}%
Sexual\marginnote{13.1} intercourse, semen, contact,\footnote{Bu Pj 1, and Bu Ss 1 and 2. This and the following rule identifications are all taken from the commentary. } \\
The first undetermined offense; \\
Arrived before, had prepared,\footnote{Bu Pc 16 and 29. } \\
With a nun in private.\footnote{Bu Pc 30. } 

Lustful,\marginnote{14.1} and two in private,\footnote{Bu Pc 43, 44, and 45. } \\
Finger, playing in the water;\footnote{Bu Pc 51 and 52. } \\
Should he hit, and should he raise,\footnote{Bu Pc 74 and 75. } \\
And fifty-three on training.\footnote{Sk 1-10, 15-36, 38-42, 44-56, and 73-75. } 

Below\marginnote{15.1} the collar bone, village, lustful,\footnote{Bi Pj 5, and Bi Ss 3 and 5. } \\
Palm, and dildo, cleaning;\footnote{Bi Pc 3, 4, and 5. } \\
And completed the rainy-season residence, instruction,\footnote{Bi Pc 40 and 58. } \\
If she should not follow her mentor.\footnote{Bi Pc 69. } 

These\marginnote{16.1} seventy-six training rules, \\
Done from body and mind; \\
All have one origination, \\
Like the first offense entailing expulsion. 

%
\end{verse}

\scendsutta{Originations like the first offense entailing expulsion are finished. }

\subsection*{2. Originations like the second offense entailing expulsion }

\begin{verse}%
Not\marginnote{18.1} given, form, super,\footnote{Bu Pj 2, 3, and 4. } \\
Indecent, his own needs;\footnote{Bu Ss 3 and 4. } \\
Groundless, unrelated,\footnote{Bu Ss 8 and 9. } \\
The second undetermined offense. 

Should\marginnote{19.1} he take back, in intending,\footnote{Bu Np 25 and 30. } \\
Falsely, abusive, malicious talebearing;\footnote{Bu Pc 1, 2, and 3. } \\
Grave offense, should he dig the earth,\footnote{Bu Pc 9 and 10. } \\
Plant, with evasion, should he complain.\footnote{Bu Pc 11, 12, and 13. } 

Throwing\marginnote{20.1} out, and sprinkling,\footnote{Bu Pc 17 and 20. } \\
For the sake of worldly gain, who has finished his meal;\footnote{Bu Pc 24 and 36. } \\
Come, disrespect, scaring,\footnote{Bu Pc 42, 54, and 55. } \\
And should he hide, life.\footnote{Bu Pc 60 and 61. } 

That\marginnote{21.1} one knows contains living beings, legal procedure,\footnote{Bu Pc 62 and 63. } \\
Less than, doing formal meetings with, expulsion;\footnote{Bu Pc 65, 69, and 70. } \\
Legitimately, annoyance,\footnote{Bu Pc 71 and 72. } \\
Deception, and with groundless.\footnote{Bu Pc 73 and 76. } 

Anxious,\marginnote{22.1} legitimate, gives out a robe,\footnote{Bu Pc 77, 79, and 81. } \\
Should he divert to an individual;\footnote{Bu Pc 82. } \\
What to you, out-of-season, should she take back,\footnote{Bi Ss 6, and Bi Np 2 and 3. } \\
Because of misunderstanding, and with hell.\footnote{Bi Pc 18 and 19. } 

Group,\marginnote{23.1} distribution, uncertain,\footnote{Bi Pc 26, 27, and 29. } \\
Robe season, ill at ease, dwelling place;\footnote{Bi Pc 30, 33, and 35. } \\
Abuse, furious, keep for herself,\footnote{Bi Pc 52, 53, and 55. } \\
And pregnant woman, breastfeeding woman.\footnote{Bi Pc 61 and 62. } 

Two\marginnote{24.1} years’ training, by the Sangha,\footnote{Bi Pc 63 and 64. } \\
And three on married girls;\footnote{Bi Pc 65, 66, and 67. } \\
And three on unmarried women,\footnote{Bi Pc 71, 72, and 73. } \\
Less than twelve, without approval.\footnote{Bi Pc 74 and 75. } 

Enough,\marginnote{25.1} difficult to live with,\footnote{Bi Pc 76 and 79. } \\
Consent, and every year, two;\footnote{Bi Pc 81, 82, and 83. } \\
These seventy training rules,\footnote{All versions of the Pali have seventy at this point, yet counting according to the commentary, I get seventy-one. Seventy-one does, in fact, seem to be required so as to account for all the rules in the two \textsanskrit{Pātimokkhas}. } \\
Have three originations: 

From\marginnote{26.1} body and mind, not from speech, \\
Or from speech and mind, not from body; \\
Or they are produced from the three doors, \\
As is the second offense entailing expulsion. 

%
\end{verse}

\scendsutta{Originations like the second offense entailing expulsion are finished. }

\subsection*{3. Originations like the offense for matchmaking }

\begin{verse}%
Matchmaking,\marginnote{28.1} hut, dwelling,\footnote{Bu Ss 5, 6, and 7. } \\
And washing, receiving;\footnote{Bu Np 4 and 5. } \\
Asking, invites to take more than,\footnote{Bu Np 6 and 7. } \\
Of both, and with messenger.\footnote{Bu Np 8, 9 and 10. } 

Silk,\marginnote{29.1} entirely, two parts,\footnote{Bu Np 11, 12 and 13. } \\
Six years, sitting mat;\footnote{Bu Np 14 and 15. } \\
They neglect, and money,\footnote{Bu Np 17 and 18. } \\
Two on various kinds.\footnote{Bu Np 19 and 20. } 

Fewer\marginnote{30.1} than five mends, rainy season,\footnote{Bu Np 22 and 24. } \\
Thread, and by assigning;\footnote{Bu Np 26 and 27. } \\
Door, and giving, and sewing,\footnote{Bu Pc 19, 25, and 26. } \\
Cookie, requisite, and fire.\footnote{Bu Pc 34, 47, and 56. } 

Precious\marginnote{31.1} things, needle, and bed,\footnote{Bu Pc 84, 86, and 87. } \\
Cotton down, sitting mat, and itch;\footnote{Bu Pc 88, 89, and 90. } \\
Rainy-season, and by the standard,\footnote{Bu Pc 91 and 92. } \\
Asking, exchanging for something else.\footnote{Bi Np 4 and 5. } 

Two\marginnote{32.1} on belonging to the Sangha, two on collective,\footnote{Bi Np 6, 7, 8, and 9. } \\
Individual, light, heavy;\footnote{Bi Np 10, 11, and 12. } \\
Two on food scraps, and bathing robe,\footnote{Bi Pc 8, 9, and 22. } \\
And with a monastic robe.\footnote{Bi Pc 28. } 

These\marginnote{33.1} exactly fifty rules, \\
Are produced for six reasons: \\
From body, not from speech or mind; \\
From speech, not from body or mind; 

From\marginnote{34.1} body and speech, not from mind; \\
From body and mind, not from speech; \\
From speech and mind, not from body; \\
Or they are produced from the three doors. \\
They have six originations, \\
As it is with matchmaking. 

%
\end{verse}

\scendsutta{Originations like the offense for matchmaking are finished. }

\subsection*{4. Originations like the offense on pressing }

\begin{verse}%
Schism,\marginnote{36.1} those who side with, difficult to correct,\footnote{Bu Ss 10, 11, and 12. } \\
Corrupter, grave, and view;\footnote{Bu Ss 13, and Bu Pc 64 and 69. } \\
Consent, and two on laughing loudly,\footnote{Bu Pc 80, and Sk 11 and 12. } \\
And two on noise, should not speak.\footnote{Sk 13, 14, and 43. } 

The\marginnote{37.1} ground, on a low seat, standing,\footnote{Sk 68, 69 and 70. } \\
Behind, and next to the path;\footnote{Sk 71 and 72. } \\
Offenses, taking sides with, holding,\footnote{Bi Pj 6, 7, and 8. } \\
Should readmit, renouncing.\footnote{Bi Ss 4 and 10. } 

A,\marginnote{38.1} two on socializing, beating,\footnote{Bi Ss 11, 12, and 13, and Bi Pc 20. } \\
Should she unstitch, and with suffering;\footnote{Bi Pc 23 and 34. } \\
Again socializing, should she not resolve,\footnote{Bi Pc 36 and 45. } \\
And monastery, inviting correction.\footnote{Bi Pc 51 and 57. } 

Every\marginnote{39.1} half, two on disciple,\footnote{Bi Pc 59, 68, and 70. } \\
Robe, following.\footnote{Bi Pc 77 and 78. } \\
These thirty-seven rules \\
Are from body, speech and mind; \\
All have one origination, \\
As the rule on pressing. 

%
\end{verse}

\scendsutta{Originations like the offense on pressing are finished. }

\subsection*{5. Originations like the offense on the robe-making ceremony }

\begin{verse}%
Three\marginnote{41.1} on the end of the robe season,\footnote{Bu Np 1, 2, and 3. } \\
The first on bowl, tonics;\footnote{Bu Np 21 and 23. } \\
And also haste, risky,\footnote{Bu Np 28 and 29. } \\
And two with departing.\footnote{Bu Pc 14 and 15. } 

Dwelling\marginnote{42.1} place, one before another,\footnote{Bu Pc 23 and 33. } \\
Not left over, invitation;\footnote{Bu Pc 35 and 46. } \\
Assignment, of a king, at the wrong time,\footnote{Bu Pc 59, 83, and 85. } \\
Giving directions, and with wilderness.\footnote{Bu Pd 2 and 4. } 

Litigious,\marginnote{43.1} and collection,\footnote{Bi Ss 1 and Bi Np 1. } \\
Before, after, and at the wrong time;\footnote{Bi Pc 15, 16, and 17. } \\
Five days, borrowed,\footnote{Bi Pc 24 and 25. } \\
And also two with lodging.\footnote{Bi Pc 47 and 48. } 

On\marginnote{44.1} the lower part of the body, and on a seat—\footnote{Bi Pc 60 and 94. } \\
These twenty-nine \\
Are from body and speech, not from mind, \\
Or they are produced from the three doors; \\
All have two originations, \\
The same as with the robe-making ceremony. 

%
\end{verse}

\scendsutta{Originations like the offense on the robe-making ceremony are finished. }

\subsection*{6. Originations like the offense on wool }

\begin{verse}%
Wool,\marginnote{46.1} two on sleeping place,\footnote{Bu Np 16, and Bu Pc 5 and 6. } \\
Detachable, eating an almsmeal;\footnote{Bu Pc 18 and 31. } \\
Group, at the wrong time, store,\footnote{Bu Pc 32, 37 and 38. } \\
With tooth cleaner, naked ascetic.\footnote{Bu Pc 40 and 41. } 

(Mobilized)\marginnote{47.1} army, army, battle,\footnote{Bu Pc 48, 49, and 50. } \\
Alcohol, bathing at less than;\footnote{Bu Pc 51 and 57. } \\
On stains, two on acknowlegment,\footnote{Bu Pc 57, and Bu Pd 1 and 3. } \\
Garlic, should she attend on, dancing.\footnote{Bi Pc 1, 6, and 10. } 

Bathing,\marginnote{48.1} sheet, sleeping place,\footnote{Bi Pc 21, 31, and 32. } \\
Within her own country, and so outside;\footnote{Bi Pc 37 and 38. } \\
During the rainy season, gallery,\footnote{Bi Pc 39 and 41. } \\
High couch, spinning yarn.\footnote{Bi Pc 42 and 43. } 

Service,\marginnote{49.1} and personally,\footnote{Bi Pc 44 and 46. } \\
And with a monastery without monks;\footnote{Bi Pc 56. } \\
Sunshade, and vehicle, hip ornament,\footnote{Bi Pc 84, 85, and 86. } \\
Jewelry, scents, scented.\footnote{Bi Pc 87, 88, and 89. } 

Nun,\marginnote{50.1} and trainee nun,\footnote{Bi Pc 90 and 91. } \\
Novice nun, and with female householder;\footnote{Bi Pc 92 and 93. } \\
The offense for not wearing a chest wrap:\footnote{Bi Pc 96. } \\
The forty-four rules 

Are\marginnote{51.1} from body, not from speech or mind, \\
Or from body and mind, not from speech; \\
All have two originations, \\
The same as ‘wool’. 

%
\end{verse}

\scendsutta{Originations like the offense on wool are finished. }

\subsection*{7. Originations like the offense for memorizing the Teaching }

\begin{verse}%
Memorizing,\marginnote{53.1} except, not appointed,\footnote{Bu Pc 4, 7, and 21. } \\
And so with sunset;\footnote{Bu Pc 22. } \\
Two were spoken on worldly subjects,\footnote{Bi Pc 46 and 50. } \\
And asking without permission.\footnote{Bi Pc 95. } 

These\marginnote{54.1} seven training rules \\
Are from speech, not from body or mind; \\
Or they are produced from speech and mind, \\
Not from body; \\
All have two originations, \\
As in memorizing the Teaching. 

%
\end{verse}

\scendsutta{Originations like the offense for memorizing the Teaching are finished. }

\subsection*{8. Originations like the offense for traveling }

\begin{verse}%
Traveling,\marginnote{56.1} boat, fine,\footnote{Bu Pc 27, 28, and 39. } \\
With a woman, should she remove;\footnote{Bu Pc 67 and Bi Pc 2. } \\
Grain, and invited,\footnote{Bi Pc 7 and 54. } \\
And the eight to be acknowledged.\footnote{Bi Pd 1-8. } 

These\marginnote{57.1} fifteen training rules \\
Are from body, not from speech or mind; \\
Or they are produced from body and speech, \\
Not from mind; 

Or\marginnote{58.1} they are produced from body and mind, \\
Not from speech; \\
Or from body, speech and mind—\\
The fourfold origination; \\
Laid down through the Buddha’s knowledge, \\
The same as with the rule on traveling. 

%
\end{verse}

\scendsutta{Originations like the offense for traveling are finished. }

\subsection*{9. Originations like the offense on a group of traveling thieves }

\begin{verse}%
A\marginnote{60.1} group of traveling thieves, eavesdropping,\footnote{Bu Pc 66 and 78. } \\
And with asking for bean curry;\footnote{Sk 37. } \\
Night, and concealed, out in the open,\footnote{Bi Pc 11, 12, and 13. } \\
With a cul-de-sac: these seven\footnote{Bi Pc 14. } 

Are\marginnote{61.1} produced from body and mind, \\
Not from speech; \\
Or they are produced from the three doors. \\
They have two originations; \\
As the origination of a group of traveling thieves, \\
Taught by the Kinsman of the Sun. 

%
\end{verse}

\scendsutta{Originations like the offense on a group of traveling thieves are finished. }

\subsection*{10. Originations like the offenses on giving a Teaching }

\begin{verse}%
The\marginnote{63.1} true Teaching to someone holding a sunshade,\footnote{Sk 57. } \\
The Buddhas do not instruct; \\
Nor to someone holding a staff,\footnote{Sk 58. } \\
Or to someone holding a knife, or a weapon.\footnote{Sk 59 and 60. } 

Shoes,\marginnote{64.1} sandals, vehicle,\footnote{Sk 61, 62, and 63. } \\
Lying down, and clasping their knees;\footnote{Sk 64 and 65. } \\
Headdress, and covered head:\footnote{Sk 66 and 67. } \\
Eleven rules, not less. 

Produced\marginnote{65.1} from speech and mind, \\
Not from body; \\
All have one origination, \\
Like the offenses on giving a teaching. 

%
\end{verse}

\scendsutta{Originations like the offenses on giving a Teaching are finished. }

\subsection*{11. Originations like the offense for telling truthfully }

\begin{verse}%
Truthfully:\marginnote{67.1} produced from the body,\footnote{Bu Pc 8. } \\
Not from speech or mind; \\
Or it originates from speech, \\
Not from body or mind; 

Or\marginnote{68.1} it is produced from body and speech, \\
Not from mind; \\
So informing of what is true \\
Is produced for three reasons. 

%
\end{verse}

\scendsutta{Originations like the offense for telling truthfully are finished. }

\subsection*{12. Originations like the offense for the admission of a female criminal }

\begin{verse}%
A\marginnote{70.1} female criminal: produced from speech or mind,\footnote{Bi Ss 2. } \\
Not from body; \\
Or produced from the three doors. \\
This offense for admitting a female criminal \\
Has two originations, \\
As spoken by the King of the Teaching. 

%
\end{verse}

\scendsutta{Originations like the offense for the admission of a female criminal are finished. }

\subsection*{13. Originations like the offense for lack of permission }

\begin{verse}%
Lack\marginnote{72.1} of permission: from speech,\footnote{Bi Pc 80. } \\
Not from body or mind; \\
Or produced from body and speech, \\
Not from mind; 

Or\marginnote{73.1} produced from speech and mind, \\
Not from body; \\
Or produced from the three doors. \\
One that has four bases is not done.\footnote{This seems to say that there is no fourth door for committing offenses. } 

%
\end{verse}

\scendsutta{Originations like the offense for lack of permission are finished. }

\begin{verse}%
For\marginnote{75.1} the contraction on originations \\
Has thirteen classes that have been well taught; \\
A cause for non-delusion, \\
In accordance with the Teaching that guides; \\
The wise person remembering this, \\
Is not confused about origination.” 

%
\end{verse}

\scend{The origination of offenses is finished. }

%
\chapter*{{\suttatitleacronym Pvr 4}{\suttatitletranslation More on the origination of offenses }{\suttatitleroot Antarapeyyāla}}
\addcontentsline{toc}{chapter}{\tocacronym{Pvr 4} \toctranslation{More on the origination of offenses } \tocroot{Antarapeyyāla}}
\markboth{More on the origination of offenses }{Antarapeyyāla}
\extramarks{Pvr 4}{Pvr 4}

\section*{The section on questioning “how many?” }

How\marginnote{1.1} many kinds of offenses? How many classes of offenses? How many grounds of training? How many kinds of disrespect? How many kinds of respect? How many grounds of training? How many kinds of failure? How many kinds of originations of offenses? How many roots of disputes? How many roots of accusations? How many aspects of friendliness? How many grounds for schism? How many kinds of legal issues? How many principles for settling them? 

There\marginnote{2.1} are five kinds of offenses. There are five classes of offenses. There are five grounds of training. There are seven kinds of offenses. There are seven classes of offenses. There are seven grounds of training. There are six kinds of disrespect. There are six kinds of respect. There are six grounds of training. There are four kinds of failure. There are six kinds of originations of offenses. There are six roots of disputes. There are six roots of accusations. There are six aspects of friendliness. There are eighteen grounds for schism. There are four kinds of legal issues. There are seven principles for settling them. 

What\marginnote{3.1} are the five kinds of offenses? Offenses entailing expulsion, offenses entailing suspension, offenses entailing confession, offenses entailing acknowledgment, offenses of wrong conduct. 

What\marginnote{4.1} are the five classes of offenses? The class of offenses entailing expulsion, the class of offenses entailing suspension, the class of offenses entailing confession, the class of offenses entailing acknowledgment, the class of offenses of wrong conduct. 

What\marginnote{5.1} are the five grounds of training? The refraining from, the keeping away from, the desisting from, the abstaining from, the non-doing of, the non-performing of, the non-committing of, the non-transgressing the boundary of, the incapability with respect to the five classes of offenses. 

What\marginnote{6.1} are the seven kinds of offenses? Offenses entailing expulsion, offenses entailing suspension, serious offenses, offenses entailing confession, offenses entailing acknowledgment, offenses of wrong conduct, offenses of wrong speech. 

What\marginnote{7.1} are the seven classes of offenses? The class of offenses entailing expulsion, the class of offenses entailing suspension, the class of serious offenses, the class of offenses entailing confession, the class of offenses entailing acknowledgment, the class of offenses of wrong conduct, the class of offenses of wrong speech. 

What\marginnote{8.1} are the seven grounds of training? The refraining from, the keeping away from, the desisting from, the abstaining from, the non-doing of, the non-performing of, the non-committing of, the non-transgressing the boundary of, the incapability with respect to the seven classes of offenses. 

What\marginnote{9.1} are the six kinds of disrespect? Disrespect for the Buddha, disrespect for the Teaching, disrespect for the Sangha, disrespect for the training, disrespect for heedfulness, disrespect for hospitality. 

What\marginnote{10.1} are the six kinds of respect? Respect for the Buddha, respect for the Teaching, respect for the Sangha, respect for the training, respect for heedfulness, respect for hospitality. 

What\marginnote{11.1} are the six grounds of training? The refraining from, the keeping away from, the desisting from, the abstaining from, the non-doing of, the non-performing of, the non-committing of, the non-transgressing the boundary of, the incapability with respect to the six kinds of disrespect. 

What\marginnote{12.1} are four kinds of failure? Failure in morality, failure in conduct, failure in view, failure in livelihood. 

What\marginnote{13.1} are the six kinds of originations of offenses? There are offenses that originate from body, but not from speech or mind; there are offenses that originate from speech, but not from body or mind; there are offenses that originate from body and speech, but not from mind; there are offenses that originate from body and mind, but not from speech; there are offenses that originate from speech and mind, but not from body; there are offenses that originate from body, speech, and mind. 

What\marginnote{14.1} are the six roots of disputes? (1) It may be that a monk is angry and resentful. One who is angry and resentful is disrespectful and undeferential toward the Teacher, the Teaching, and the Sangha, and he doesn’t fulfill the training. Such a person creates disputes in the Sangha. Disputes are unbeneficial and a cause of unhappiness for humanity; they are harmful, detrimental, and a cause of suffering for gods and humans. When you see such a root of disputes either in yourself or in others, you should make an effort to get rid of it. If you don’t see such a root either in yourself or in others, you should practice so that it has no future effect. In this way that bad root of disputes is abandoned. In this way that bad root of disputes has no future effect. 

(2)\marginnote{15.1} Or it may be that a monk is denigrating and domineering, (3) envious and stingy, (4) treacherous and deceitful, (5) one who has bad desires and wrong views, or (6) one who obstinately grasps his own views and only gives them up with difficulty. Any of these is disrespectful and undeferential toward the Teacher, the Teaching, and the Sangha, and he doesn’t fulfill the training. Such a person creates disputes in the Sangha. Disputes are unbeneficial and a cause of unhappiness for humanity; they are harmful, detrimental, and a cause of suffering for gods and humans. When you see such a root of disputes either in yourself or in others, you should make an effort to get rid of it. If you don’t see such a root either in yourself or in others, you should practice so that it has no future effect. In this way that bad root of disputes is abandoned. In this way that bad root of disputes has no future effect. 

What\marginnote{16.1} are the six roots of accusations? (1) It may be that a monk is angry and resentful. One who is angry and resentful is disrespectful and undeferential toward the Teacher, the Teaching, and the Sangha, and he doesn’t fulfill the training. Such a person creates accusations in the Sangha. Accusations are unbeneficial and a cause of unhappiness for humanity; they are harmful, detrimental, and a cause of suffering for gods and humans. When you see such a root of accusations either in yourself or in others, you should make an effort to get rid of it. If you don’t see such a root either in yourself or in others, you should practice so that it has no future effect. In this way that bad root of accusations is abandoned. In this way that bad root of accusations has no future effect. 

(2)\marginnote{17.1} Or it may be that a monk is denigrating and domineering, (3) envious and stingy, (4) treacherous and deceitful, (5) one who has bad desires and wrong views, or (6) one who obstinately grasps his own views and only gives them up with difficulty. Any of these is disrespectful and undeferential toward the Teacher, the Teaching, and the Sangha, and he doesn’t fulfill the training. Such a person creates accusations in the Sangha. Accusations are unbeneficial and a cause of unhappiness for humanity; they are harmful, detrimental, and a cause of suffering for gods and humans. When you see such a root of accusations either in yourself or in others, you should make an effort to get rid of it. If you don’t see such a root either in yourself or in others, you should practice so that it has no future effect. In this way that bad root of accusations is abandoned. In this way that bad root of accusations has no future effect. 

What\marginnote{18.1} are the six aspects of friendliness? (1) As to this, a monk acts with good will toward his fellow monastics, both in public and in private. (2) Furthermore, a monk speaks with good will to his fellow monastics, both in public and in private. (3) Furthermore, a monk thinks with good will about his fellow monastics, both in public and in private. (4) Furthermore, whatever a monk has gained in an appropriate manner, even the content of his almsbowl, he shares without reservation with his virtuous fellow monastics. (5) Furthermore, a monk lives with his fellow monastics, both in public and in private, with moral conduct that is unbroken, consistent, spotless, pure, liberating, praised by the wise, ungrasped, and leading to stillness. (6) Furthermore, a monk lives with his fellow monastics, both in public and in private, with that noble view that is liberating and leads one who acts in accordance with it to the complete end of suffering. These aspects of friendliness create love and respect, and lead to coming together, concord, harmony, and unity. 

What\marginnote{24.1} are the eighteen grounds for schism? In this case a monk proclaims what is contrary to the Teaching as being in accordance with it and what is in accordance with the Teaching as contrary to it. He proclaims what is contrary to the Monastic Law as being in accordance with it, and what is in accordance with the Monastic Law as contrary to it. He proclaims what hasn’t been spoken by the Buddha as spoken by him, and what has been spoken by the Buddha as not spoken by him. He proclaims what was not practiced by the Buddha as practiced by him, and what was practiced by the Buddha as not practiced by him. He proclaims what was not laid down by the Buddha as laid down by him, and what was laid down by the Buddha as not laid down by him. He proclaims a non-offense as an offense, and an offense as a non-offense. He proclaims a light offense as heavy, and a heavy offense as light. He proclaims a curable offense as incurable, and an incurable offense as curable. He proclaims a grave offense as minor, and a minor offense as grave. 

What\marginnote{25.1} are the four kinds of legal issues? Legal issues arising from disputes, legal issues arising from accusations, legal issues arising from offenses, legal issues arising from business. 

What\marginnote{26.1} are the seven principles for settling them? Resolution face-to-face, resolution through recollection, resolution because of past insanity, acting according to what has been admitted, majority decision, further penalty, covering over as if with grass. 

\scendsutta{The section on questioning “how many?” is finished. }

\scuddanaintro{This is the summary: }

\begin{scuddana}%
“Offense,\marginnote{29.1} classes of offenses, \\
Training, again sevenfold; \\
Training, and disrespect, \\
Respect, and root. 

Again\marginnote{30.1} training, failure, \\
Originations, disputes; \\
Accusations, friendliness, \\
Schism, and with legal issues; \\
Seven are said on settling, \\
These are the seventeen items.” 

%
\end{scuddana}

\section*{1. The section on the six kinds of originations of offenses }

“Is\marginnote{31.1} it possible to commit an offense entailing expulsion through the first kind of origination of offenses?”—“No.”—“An offense entailing suspension?”—“One might.”—“A serious offense?”—“One might.”—“An offense entailing confession?”—“One might.”—“An offense entailing acknowledgment?”—“One might.”—“An offense of wrong conduct?”—“One might.”—“An offense of wrong speech?”—“No.” 

“Is\marginnote{32.1} it possible to commit an offense entailing expulsion through the second kind of origination of offenses?”—“No.”—“An offense entailing suspension?”—“One might.”—“A serious offense?”—“One might.”—“An offense entailing confession?”—“One might.”—“An offense entailing acknowledgment?”—“No.”—“An offense of wrong conduct?”—“One might.”—“An offense of wrong speech?”—“No.” 

“Is\marginnote{33.1} it possible to commit an offense entailing expulsion through the third kind of origination of offenses?”—“No.”—“An offense entailing suspension?”—“One might.”—“A serious offense?”—“One might.”—“An offense entailing confession?”—“One might.”—“An offense entailing acknowledgment?”—“One might.”—“An offense of wrong conduct?”—“One might.”—“An offense of wrong speech?”—“No.” 

“Is\marginnote{34.1} it possible to commit an offense entailing expulsion through the fourth kind of origination of offenses?”—“One might.”—“An offense entailing suspension?”—“One might.”—“A serious offense?”—“One might.”—“An offense entailing confession?”—“One might.”—“An offense entailing acknowledgment?”—“One might.”—“An offense of wrong conduct?”—“One might.”—“An offense of wrong speech?”—“No.” 

“Is\marginnote{35.1} it possible to commit an offense entailing expulsion through the fifth kind of origination of offenses?”—“One might.”—“An offense entailing suspension?”—“One might.”—“A serious offense?”—“One might.”—“An offense entailing confession?”—“One might.”—“An offense entailing acknowledgment?”—“No.”—“An offense of wrong conduct?”—“One might.”—“An offense of wrong speech?”—“One might.” 

“Is\marginnote{36.1} it possible to commit an offense entailing expulsion through the sixth kind of origination of offenses?”—“One might.”—“An offense entailing suspension?”—“One might.”—“A serious offense?”—“One might.”—“An offense entailing confession?”—“One might.”—“An offense entailing acknowledgment?”—“One might.”—“An offense of wrong conduct?”—“One might.”—“An offense of wrong speech?”—“No.” 

\scendsutta{The first section on the six kinds of originations of offenses is finished. }

\section*{2. The section on “how many kinds of offenses?” }

“How\marginnote{38.1} many kinds of offenses does one commit through the first kind of origination of offenses? Five: (1) when a monk—perceiving it as allowable and by means of begging—builds a hut whose site has not been approved, which exceeds the right size, where harm will be done, and which lacks a space on all sides, then for the effort there is an offense of wrong conduct; (2) when there is one piece left to complete the hut, he commits a serious offense; (3) when the last piece is finished, he commits an offense entailing suspension; (4) when a monk, perceiving it as allowable, eats cooked food at the wrong time, he commits an offense entailing confession; (5) when a monk, perceiving it as allowable, receives fresh or cooked food directly from an unrelated nun who has entered an inhabited area, and then eats it, he commits an offense entailing acknowledgment. 

When\marginnote{39.1} it comes to these offenses, to how many of the four kinds of failure do they belong? In how many of the seven classes of offenses are they found? Through how many of the six kinds of originations of offenses do they originate? To which of the four kinds of legal issues do they belong? Through how many of the seven principles for settling legal issues are they settled? They belong to two kinds of failure: they may be failure in morality; they may be failure in conduct. They are found in five classes of offenses: they may be in the class of offenses entailing suspension; they may be in the class of serious offenses; they may be in the class of offenses entailing confession; they may be in the class of offenses entailing acknowledgment; they may be in the class of offenses of wrong conduct. They originate in one way: from body, not from speech or mind. They belong to legal issues arising from offenses. They are settled through three principles: they may be settled by resolution face-to-face and by acting according to what has been admitted; or they may be settled by resolution face-to-face and by covering over as if with grass. 

How\marginnote{40.1} many kinds of offenses does one commit through the second kind of origination of offenses? Four: (1) when a monk, perceiving it as allowable, appoints someone to build him a hut, and they build a hut whose site has not been approved, which exceeds the right size, where harm will be done, and which lacks a space on all sides, then for the effort there is an offense of wrong conduct; (2) when there is one piece left to complete the hut, he commits a serious offense; (3) when the last piece is finished, he commits an offense entailing suspension; (4) when a monk, perceiving it as allowable, instructs a person who is not fully ordained to memorize the Teaching, he commits an offense entailing confession. 

When\marginnote{41.1} it comes to these offenses, to how many of the four kinds of failure do they belong? … Through how many of the seven principles for settling legal issues are they settled? They belong to two kinds of failure: they may be failure in morality; they may be failure in conduct. They are found in four classes of offenses: they may be in the class of offenses entailing suspension; they may be in the class of serious offenses; they may be in the class of offenses entailing confession; they may be in the class of offenses of wrong conduct. They originate in one way: from speech, not from body or mind. They belong to legal issues arising from offenses. They are settled through three principles: they may be settled by resolution face-to-face and by acting according to what has been admitted; or they may be settled by resolution face-to-face and by covering over as if with grass. 

How\marginnote{42.1} many kinds of offenses does one commit through the third kind of origination of offenses? Five: (1) when a monk—perceiving it as allowable and having appointed someone—builds a hut whose site has not been approved, which exceeds the right size, where harm will be done, and which lacks a space on all sides, then for the effort there is an offense of wrong conduct;\footnote{The Pali reads \textit{\textsanskrit{saṁvidahitvā}}, whereas the text at \href{https://suttacentral.net/pli-tv-bu-vb-ss6/en/brahmali\#3.5.1}{Bu Ss 6:3.5.1} reads \textit{\textsanskrit{samādisati}}. Presumably this is only for grammatical reasons. } (2) when there is one piece left to complete the hut, he commits a serious offense; (3) when the last piece is finished, he commits an offense entailing suspension; (4) when a monk, perceiving it as allowable, eats fine foods that he has requested for himself, he commits an offense entailing confession; (5) when a monk, perceiving it as allowable, eats without having restrained a nun who is giving directions, he commits an offense entailing acknowledgment. 

When\marginnote{43.1} it comes to these offenses, to how many of the four kinds of failure do they belong? … Through how many of the seven principles for settling legal issues are they settled? They belong to two kinds of failure: they may be failure in morality; they may be failure in conduct. They are found in five classes of offenses: they may be in the class of offenses entailing suspension; they may be in the class of serious offenses; they may be in the class of offenses entailing confession; they may be in the class of offenses entailing acknowledgment; they may be in the class of offenses of wrong conduct. They originate in one way: from body and speech, not from mind. They belong to legal issues arising from offenses. They are settled through three principles: they may be settled by resolution face-to-face and by acting according to what has been admitted; or they may be settled by resolution face-to-face and by covering over as if with grass. 

How\marginnote{44.1} many kinds of offenses does one commit through the fourth kind of origination of offenses? Six: (1) when a monk has sexual intercourse, he commits an offense entailing expulsion; (2) when a monk—perceiving it as unallowable and by means of begging—builds a hut whose site has not been approved, which exceeds the right size, where harm will be done, and which lacks a space on all sides, then for the effort there is an offense of wrong conduct; (3) when there is one piece left to complete the hut, he commits a serious offense; (4) when the last piece is finished, he commits an offense entailing suspension; (5) when a monk, perceiving it as unallowable, eats cooked food at the wrong time, he commits an offense entailing confession; (6) when a monk, perceiving it as unallowable, receives fresh or cooked food directly from an unrelated nun who has entered an inhabited area, and then eats it, he commits an offense entailing acknowledgment. 

When\marginnote{45.1} it comes to these offenses, to how many of the four kinds of failure do they belong? … Through how many of the seven principles for settling legal issues are they settled? They belong to two kinds of failure: they may be failure in morality; they may be failure in conduct. They are found in six classes of offenses: they may be in the class of offenses entailing expulsion; they may be in the class of offenses entailing suspension; they may be in the class of serious offenses; they may be in the class of offenses entailing confession; they may be in the class of offenses entailing acknowledgment; they may be in the class of offenses of wrong conduct. They originate in one way: from body and mind, not from speech. They belong to legal issues arising from offenses. They are settled through three principles: they may be settled by resolution face-to-face and by acting according to what has been admitted; or they may be settled by resolution face-to-face and by covering over as if with grass. 

How\marginnote{46.1} many kinds of offenses does one commit through the fifth kind of origination of offenses? Six: (1) when a monk, having bad desires, overcome by desire, claims a non-existent superhuman quality, he commits an offense entailing expulsion; (2) when a monk, perceiving it as unallowable, appoints someone to build him a hut, and they build a hut whose site has not been approved, which exceeds the right size, where harm will be done, and which lacks a space on all sides, then for the effort there is an offense of wrong conduct; (3) when there is one piece left to complete the hut, he commits a serious offense; (4) when the last piece is finished, he commits an offense entailing suspension; (5) when a monk, perceiving it as unallowable, instructs a person who is not fully ordained to memorize the Teaching, he commits an offense entailing confession; (6) when—not wishing to revile, not wishing to insult, not wishing to humiliate, but wanting to have fun—one says what is low to one who is low, one commits an offense of wrong speech. 

When\marginnote{47.1} it comes to these offenses, to how many of the four kinds of failure do they belong? … Through how many of the seven principles for settling legal issues are they settled? They belong to two kinds of failure: they may be failure in morality; they may be failure in conduct. They are found in six classes of offenses: they may be in the class of offenses entailing expulsion; they may be in the class of offenses entailing suspension; they may be in the class of serious offenses; they may be in the class of offenses entailing confession; they may be in the class of offenses of wrong conduct; they may be in the class of offenses of wrong speech. They originate in one way: from speech and mind, not from body. They belong to legal issues arising from offenses. They are settled through three principles: they may be settled by resolution face-to-face and by acting according to what has been admitted; or they may be settled by resolution face-to-face and by covering over as if with grass. 

How\marginnote{48.1} many kinds of offenses does one commit through the sixth kind of origination of offenses? Six: (1) when a monk, having agreed with others, steals goods, he commits an offense entailing expulsion; (2) when a monk—perceiving it as unallowable and having appointed someone—builds a hut whose site has not been approved, which exceeds the right size, where harm will be done, and which lacks a space on all sides, then for the effort there is an offense of wrong conduct; (3) when there is one piece left to complete the hut, he commits a serious offense; (4) when the last piece is finished, he commits an offense entailing suspension; (5) when a monk, perceiving it as unallowable, eats fine foods that he has requested for himself, he commits an offense entailing confession; (6) when a monk, perceiving it as unallowable, eats without having restrained a nun who is giving directions, he commits an offense entailing acknowledgment. 

When\marginnote{49.1} it comes to these offenses, to how many of the four kinds of failure do they belong? In how many of the seven classes of offenses are they found? Through how many of the six kinds of originations of offenses do they originate? To which of the four kinds of legal issues do they belong? Through how many of the seven principles for settling legal issues are they settled? They belong to two kinds of failure: they may be failure in morality; they may be failure in conduct. They are found in six classes of offenses: they may be in the class of offenses entailing expulsion; they may be in the class of offenses entailing suspension; they may be in the class of serious offenses; they may be in the class of offenses entailing confession; they may be in the class of offenses entailing acknowledgment; they may be in the class of offenses of wrong conduct. They originate in one way: from body, speech, and mind. They belong to legal issues arising from offenses. They are settled through three principles: they may be settled by resolution face-to-face and by acting according to what has been admitted; or they may be settled by resolution face-to-face and by covering over as if with grass.” 

\scend{The second section on “how many kinds of offenses?” for the six kinds of originations of offenses is finished. }

\section*{3. The verses on the kinds of originations of offenses }

\begin{verse}%
“Origination\marginnote{51.1} from body has been declared by the one who benefits the world, \\
The one of boundless vision, seeing seclusion: \\
I ask how many kinds of offenses originate from that—\\
You who are skilled in analysis, please say. 

Origination\marginnote{52.1} from body has been declared by the one who benefits the world, \\
The one of boundless vision, seeing seclusion: \\
Five kinds of offenses originate from that—\\
I declare this to you, you who are skilled in analysis. 

Origination\marginnote{53.1} from speech has been declared by the one who benefits the world, \\
The one of boundless vision, seeing seclusion: \\
I ask how many kinds of offenses originate from that—\\
You who are skilled in analysis, please say. 

Origination\marginnote{54.1} from speech has been declared by the one who benefits the world, \\
The one of boundless vision, seeing seclusion: \\
Four kinds of offenses originate from that—\\
I declare this to you, you who are skilled in analysis. 

Origination\marginnote{55.1} from body and speech has been declared by the one who benefits the world, \\
The one of boundless vision, seeing seclusion: \\
I ask how many kinds of offenses originate from that—\\
You who are skilled in analysis, please say. 

Origination\marginnote{56.1} from body and speech has been declared by the one who benefits the world, \\
The one of boundless vision, seeing seclusion: \\
Five kinds of offenses originate from that—\\
I declare this to you, you who are skilled in analysis. 

Origination\marginnote{57.1} from body and mind has been declared by the one who benefits the world, \\
The one of boundless vision, seeing seclusion: \\
I ask how many kinds of offenses originate from that—\\
You who are skilled in analysis, please say. 

Origination\marginnote{58.1} from body and mind has been declared by the one who benefits the world, \\
The one of boundless vision, seeing seclusion: \\
Six kinds of offenses originate from that—\\
I declare this to you, you who are skilled in analysis. 

Origination\marginnote{59.1} from speech and mind has been declared by the one who benefits the world, \\
The one of boundless vision, seeing seclusion: \\
I ask how many kinds of offenses originate from that—\\
You who are skilled in analysis, please say. 

Origination\marginnote{60.1} from speech and mind has been declared by the one who benefits the world, \\
The one of boundless vision, seeing seclusion: \\
Six kinds of offenses originate from that—\\
I declare this to you, you who are skilled in analysis. 

Origination\marginnote{61.1} from body, speech, and mind has been declared by the one who benefits the world, \\
The one of boundless vision, seeing seclusion: \\
I ask how many kinds of offenses originate from that—\\
You who are skilled in analysis, please say. 

Origination\marginnote{62.1} from body, speech, and mind has been declared by the one who benefits the world, \\
The one of boundless vision, seeing seclusion: \\
Six kinds of offenses originate from that—\\
I declare this to you, you who are skilled in analysis.” 

%
\end{verse}

\scendsutta{The third section on the verses on the kinds of originations of offenses is finished. }

\section*{4. The section on “as a result of failure” }

How\marginnote{64.1} many kinds of offenses does one commit as a result of failure in morality? Four: when a nun knowingly conceals an offense entailing expulsion, she commits an offense entailing expulsion; when, being unsure, she conceals it, she commits a serious offense; when a monk conceals an offense entailing suspension, he commits an offense entailing confession; when he conceals a grave offense of his own, he commits an offense of wrong conduct. 

When\marginnote{65.1} it comes to these offenses, to how many of the four kinds of failure do they belong? … Through how many of the seven principles for settling legal issues are they settled? They belong to two kinds of failure: they may be failure in morality; they may be failure in conduct. They are found in four classes of offenses: they may be in the class of offenses entailing expulsion; they may be in the class of serious offenses; they may be in the class of offenses entailing confession; they may be in the class of offenses of wrong conduct. They originate in one way: from body, speech, and mind. They belong to legal issues arising from offenses. They are settled through three principles: they may be settled by resolution face-to-face and by acting according to what has been admitted; or they may be settled by resolution face-to-face and by covering over as if with grass. 

How\marginnote{66.1} many kinds of offenses does one commit as a result of failure in conduct? One: when one conceals a failure in conduct, one commits an offense of wrong conduct. 

When\marginnote{67.1} it comes to this offense, to how many of the four kinds of failure does it belong? … Through how many of the seven principles for settling legal issues is it settled? It belongs to one kind of failure: failure in conduct. It is found in one class of offenses: in the class of offenses of wrong conduct. They originate in one way: from body, speech, and mind. It belongs to legal issues arising from offenses. They are settled through three principles: they may be settled by resolution face-to-face and by acting according to what has been admitted; or they may be settled by resolution face-to-face and by covering over as if with grass. 

How\marginnote{68.1} many kinds of offenses does one commit as a result of failure in view? Two: when not giving up a bad view after being pressed for the third time, then after the motion, one commits an offense of wrong conduct; when the last announcement is finished, one commits an offense entailing confession. 

When\marginnote{69.1} it comes to these offenses, to how many of the four kinds of failure do they belong? … Through how many of the seven principles for settling legal issues are they settled? They belong to one kind of failure: failure in conduct. They are found in two classes of offenses: they may be in the class of offenses entailing confession; they may be in the class of offenses of wrong conduct. They originate in one way: from body, speech, and mind. They belong to legal issues arising from offenses. They are settled through three principles: they may be settled by resolution face-to-face and by acting according to what has been admitted; or they may be settled by resolution face-to-face and by covering over as if with grass. 

How\marginnote{70.1} many kinds of offenses does one commit as a result of failure in livelihood? Six: (1) when, to make a living, having bad desires, overcome by desire, one claims a non-existent superhuman quality, one commits an offense entailing expulsion; (2) when, to make a living, one acts as a matchmaker, one commits an offense entailing suspension; (3) when, to make a living, one says, “The monk who stays in your dwelling is a perfected one,” and the listener understands, one commits a serious offense; (4) when, to make a living, a monk eats fine foods that he has requested for himself, he commits an offense entailing confession; (5) when, to make a living, a nun eats fine foods that she has requested for herself, she commits an offense entailing acknowledgment; (6) when, to make a living, one eats bean curry or rice that one has requested for oneself, one commits an offense of wrong conduct. 

When\marginnote{71.1} it comes to these offenses, to how many of the four kinds of failure do they belong? … Through how many of the seven principles for settling legal issues are they settled? They belong to two kinds of failure: they may be failure in morality; they may be failure in conduct. They are found in six classes of offenses: they may be in the class of offenses entailing expulsion; they may be in the class of offenses entailing suspension; they may be in the class of serious offenses; they may be in the class of offenses entailing confession; they may be in the class of offenses entailing acknowledgment; they may be in the class of offenses of wrong conduct. They originate in six ways: from body, not from speech or mind; or from speech, not from body or mind; or from body and speech, not from mind; or from body and mind, not from speech; or from speech and mind, not from body; or from body, speech, and mind. They belong to legal issues arising from offenses. They are settled through three principles: they may be settled by resolution face-to-face and by acting according to what has been admitted; or they may be settled by resolution face-to-face and by covering over as if with grass. 

\scendsutta{The fourth section on “as a result of failure” is finished. }

\section*{5. The section on “as a result of legal issues” }

“How\marginnote{73.1} many kinds of offenses does one commit as a result of legal issues arising from disputes? Two: when one speaks abusively to one who is fully ordained, one commits an offense entailing confession; when one speaks abusively to one who is not fully ordained, one commits an offense of wrong conduct. 

When\marginnote{74.1} it comes to these offenses, to how many of the four kinds of failure do they belong? … Through how many of the seven principles for settling legal issues are they settled? They belong to one kind of failure: failure in conduct. They are found in two classes of offenses: they may be in the class of offenses entailing confession; they may be in the class of offenses of wrong conduct. They originate in three ways: from body and mind, not from speech; or from speech and mind, not from body; or from body, speech, and mind. They belong to legal issues arising from offenses. They are settled through three principles: they may be settled by resolution face-to-face and by acting according to what has been admitted; or they may be settled by resolution face-to-face and by covering over as if with grass. 

How\marginnote{75.1} many kinds of offenses does one commit as a result of legal issues arising from accusations? Three: when one groundlessly charges a monk with an offense entailing expulsion, one commits an offense entailing suspension; when one groundlessly charges him with an offense entailing suspension, one commits an offense entailing confession; when one groundlessly charges him with failure in conduct, one commits an offense of wrong conduct. 

When\marginnote{76.1} it comes to these offenses, to how many of the four kinds of failure do they belong? … Through how many of the seven principles for settling legal issues are they settled? They belong to two kinds of failure: they may be failure in morality; they may be failure in conduct. They are found in three classes of offenses: they may be in the class of offenses entailing suspension; they may be in the class of offenses entailing confession; they may be in the class of offenses of wrong conduct. They originate in three ways: from body and mind, not from speech; or from speech and mind, not from body; or from body, speech, and mind. They belong to legal issues arising from offenses. They are settled through three principles: they may be settled by resolution face-to-face and by acting according to what has been admitted; or they may be settled by resolution face-to-face and by covering over as if with grass. 

How\marginnote{77.1} many kinds of offenses does one commit as a result of legal issues arising from offenses? Four: when a nun knowingly conceals an offense entailing expulsion, she commits an offense entailing expulsion; when, being unsure, she conceals it, she commits a serious offense; when a monk conceals an offense entailing suspension, he commits an offense entailing confession; when one conceals a failure in conduct, one commits an offense of wrong conduct. 

When\marginnote{78.1} it comes to these offenses, to how many of the four kinds of failure do they belong? … Through how many of the seven principles for settling legal issues are they settled? They belong to two kinds of failure: they may be failure in morality; they may be failure in conduct. They are found in four classes of offenses: they may be in the class of offenses entailing expulsion; they may be in the class of serious offenses; they may be in the class of offenses entailing confession; they may be in the class of offenses of wrong conduct. They originate in one way: from body, speech, and mind. They belong to legal issues arising from offenses. They are settled through three principles: they may be settled by resolution face-to-face and by acting according to what has been admitted; or they may be settled by resolution face-to-face and by covering over as if with grass. 

How\marginnote{79.1} many kinds of offenses does one commit as a result of legal issues arising from business? Five: (1) when a nun takes sides with one who has been ejected and does not stop when pressed for the third time, then after the motion, she commits an offense of wrong conduct; (2) after each of the first two announcements, she commits a serious offense; (3) when the last announcement is finished, she commits an offense entailing expulsion; (4) when monks who side with a monk who is pursuing schism do not stop when pressed for the third time, they commit an offense entailing suspension; (5) when not giving up a bad view after being pressed for the third time, one commits an offense entailing confession. 

When\marginnote{80.1} it comes to these offenses, to how many of the four kinds of failure do they belong? … Through how many of the seven principles for settling legal issues are they settled? They belong to two kinds of failure: they may be failure in morality; they may be failure in conduct. They are found in five classes of offenses: they may be in the class of offenses entailing expulsion; they may be in the class of offenses entailing suspension; they may be in the class of serious offenses; they may be in the class of offenses entailing confession; they may be in the class of offenses of wrong conduct. They originate in one way: from body, speech, and mind. They belong to legal issues arising from offenses. They are settled through three principles: they may be settled by resolution face-to-face and by acting according to what has been admitted; or they may be settled by resolution face-to-face and by covering over as if with grass. 

Apart\marginnote{81.1} from the seven kinds of offenses and the seven classes of offenses, when it comes to the rest of the offenses, to how many of the four kinds of failure do they belong? In how many of the seven classes of offenses are they found? Through how many of the six kinds of originations of offenses do they originate? To which of the four kinds of legal issues do they belong? Through how many of the seven principles for settling legal issues are they settled? Apart from the seven kinds of offenses and the seven classes of offenses, the rest of the offenses do not belong to any of the four kinds of failure. They are not found in any of the seven classes of offenses. They do not originate through any of the six kinds of originations of offenses. They do not belong to any of the four kinds of legal issues. They are not settled through any of the seven principles for settling legal issues. Why is that? Apart from the seven kinds of offenses and the seven classes of offenses, there are no other offenses.” 

\scendsutta{The fifth section on “as a result of legal issues” is finished. }

\scend{More on the origination of offenses is finished. }

\scuddanaintro{This is the summary: }

\begin{scuddana}%
“Questioning\marginnote{85.1} ‘how many?’, originations, \\
And so ‘how many kinds of offenses?’ \\
Originations, and failure, \\
And so with legal issues.” 

%
\end{scuddana}

%
\chapter*{{\suttatitleacronym Pvr 5}{\suttatitletranslation The legal issues and their settling }{\suttatitleroot Samathabheda}}
\addcontentsline{toc}{chapter}{\tocacronym{Pvr 5} \toctranslation{The legal issues and their settling } \tocroot{Samathabheda}}
\markboth{The legal issues and their settling }{Samathabheda}
\extramarks{Pvr 5}{Pvr 5}

\section*{1. The section with a succession on legal issues }

In\marginnote{1.1} regard to legal issues arising from disputes: What is the forerunner? How many reasons are there? How many grounds? How many foundations? How many causes? How many roots? Through how many motives does one dispute? Through how many principles for settling is a legal issue arising from a dispute settled?\footnote{In this chapter \textit{samatha}, “settling”, seems to be used as a shorthand for \textit{samathadhamma}, “principle for settling”. } 

In\marginnote{2.1} regard to legal issues arising from accusations: What is the forerunner? How many reasons are there? How many grounds? How many foundations? How many causes? How many roots? Through how many motives does one accuse? Through how many principles for settling is a legal issue arising from an accusation settled? 

In\marginnote{3.1} regard to legal issues arising from offenses: What is the forerunner? How many reasons are there? How many grounds? How many foundations? How many causes? How many roots? Through how many motives does one commit an offense? Through how many principles for settling is a legal issue arising from an offense settled? 

In\marginnote{4.1} regard to legal issues arising from business: What is the forerunner? How many reasons are there? How many grounds? How many foundations? How many causes? How many roots? Through how many motives does one give rise to business? Through how many principles for settling is a legal issue arising from business settled? 

“What\marginnote{5.1} is the forerunner of legal issues arising from disputes?” Desire is a forerunner, ill will is a forerunner, confusion is a forerunner, non-desire is a forerunner, non-ill will is a forerunner, non-confusion is a forerunner. “How many reasons are there?” The eighteen grounds for schism.\footnote{A more literal but clunky rendering would be: “The reasons which are the eighteen grounds for schism.” } “How many grounds are there?” The eighteen grounds for schism. “How many foundations are there?” The eighteen grounds for schism. “How many causes are there?” Nine: there are three wholesome causes, three unwholesome causes, and three indeterminate causes. “How many roots are there?” Twelve. “Through how many motives does one dispute?” Through two: through a view that accords with the Teaching or through a view that is contrary to the Teaching. “Through how many principles for settling is a legal issue arising from a dispute settled?” Through two of them: through resolution face-to-face and through a majority decision. 

“What\marginnote{6.1} is the forerunner of legal issues arising from accusations?” Desire is a forerunner, ill will is a forerunner, confusion is a forerunner, non-desire is a forerunner, non-ill will is a forerunner, non-confusion is a forerunner. “How many reasons are there?” The four failures. “How many grounds are there?” The four failures. “How many foundations are there?” The four failures. “How many causes are there?” Nine: there are three wholesome causes, three unwholesome causes, and three indeterminate causes. “How many roots are there?” Fourteen. “Through how many motives does one accuse?” Through two: through action or through offense. “Through how many principles for settling is a legal issue arising from an accusation settled?” Through four of them: through resolution face-to-face, through resolution through recollection, through resolution because of past insanity, and through a further penalty. 

“What\marginnote{7.1} is the forerunner of legal issues arising from offenses?” Desire is a forerunner, ill will is a forerunner, confusion is a forerunner, non-desire is a forerunner, non-ill will is a forerunner, non-confusion is a forerunner. “How many reasons are there?” The seven classes of offenses. “How many grounds are there?” The seven classes of offenses. “How many foundations are there?” The seven classes of offenses. “How many causes are there?” Six: three unwholesome causes and three indeterminate causes. “How many roots are there?” The six originations of offenses. “Through how many motives does one commit an offense?” Through six: through shamelessness, through ignorance, through being overcome by anxiety, through perceiving what is unallowable as allowable, through perceiving what is allowable as unallowable, through absentmindedness.\footnote{“Through being overcome by anxiety” renders \textit{\textsanskrit{kukkuccapakatā}}. Sp 3.175: \textit{\textsanskrit{Evaṁ} \textsanskrit{pubbabhāge} \textsanskrit{sanniṭṭhānaṁ} \textsanskrit{katvāpi} \textsanskrit{karaṇakkhaṇe} akappiye \textsanskrit{akappiyasaññitāsaṅkhātena} kukkuccena \textsanskrit{abhibhūtā} “\textsanskrit{kukkuccapakatā}”ti}, “\textit{\textsanskrit{Kukkuccapakatā}}: thus, having decided first, then at the moment of the unallowable action, one is overwhelmed by anxiety due to the perception of unallowableness.” } “Through how many principles for settling is a legal issue arising from an offense settled?” Through three of them: through resolution face-to-face and through acting according to what has been admitted, and through resolution face-to-face and through covering over as if with grass. 

“What\marginnote{8.1} is the forerunner of legal issues arising from business?” Desire is a forerunner, ill will is a forerunner, confusion is a forerunner, non-desire is a forerunner, non-ill will is a forerunner, non-confusion is a forerunner. “How many reasons are there?” The four legal procedures. “How many grounds are there?” The four legal procedures. “How many foundations are there?” The four legal procedures. “How many causes are there?” Nine: there are three wholesome causes, three unwholesome causes, and three indeterminate causes. “How many roots are there?” One: The Sangha. “Through how many motives does one give rise to business?” Through two: through a motion or through getting permission. “Through how many principles for settling is a legal issue arising from business settled?” Through one: through resolution face-to-face. 

How\marginnote{9.1} many principles for settling are there? Seven: resolution face-to-face, resolution through recollection, resolution because of past insanity, acting according to what has been admitted, a majority decision, a further penalty, and covering over as if with grass. 

With\marginnote{10.1} a different presentation, might the seven principles for settling become ten, and the ten become seven? They might. 

How?\marginnote{11.1} A legal issue arising from a dispute is settled through two principles, a legal issue arising from an accusation is settled through four principles, a legal issue arising from an offense is settled through three principles, a legal issue arising from business is settled through one principle. In this way, the seven principles for settling become ten, and the ten become seven. 

\scendsutta{The sixth section on “a different presentation” is finished. }

\section*{2. The section on “in common” }

How\marginnote{13.1} many of the principles for settling legal issues arising from disputes do the monks and the nuns have in common? How many do they not have in common? How many of the principles for settling legal issues arising from accusations do the monks and the nuns have in common? How many do they not have in common? How many of the principles for settling legal issues arising from offenses do the monks and the nuns have in common? How many do they not have in common? How many of the principles for settling legal issues arising from business do the monks and the nuns have in common? How many do they not have in common? 

They\marginnote{14.1} have two principles for settling legal issues arising from disputes in common: resolution face-to-face and a majority decision. And they have five not in common: resolution through recollection, resolution because of past insanity, acting according to what has been admitted, a further penalty, and covering over as if with grass. 

They\marginnote{15.1} have four principles for settling legal issues arising from accusations in common: resolution face-to-face, resolution through recollection, resolution because of past insanity, and a further penalty. And they have three not in common: a majority decision, acting according to what has been admitted, and covering over as if with grass. 

They\marginnote{16.1} have three principles for settling legal issues arising from offenses in common: resolution face-to-face, acting according to what has been admitted, and covering over as if with grass. And they have four not in common: a majority decision, resolution through recollection, resolution because of past insanity, and a further penalty. 

They\marginnote{17.1} have one principle for settling legal issues arising from business in common: resolution face-to-face. And they have six not in common: a majority decision, resolution through recollection, resolution because of past insanity, acting according to what has been admitted, a further penalty, and covering over as if with grass. 

\scendsutta{The seventh section on “in common” is finished. }

\section*{3. The section on “belonging to that” }

How\marginnote{19.1} many principles for settling are there that belong to legal issues arising from disputes? How many belong to something else? How many principles for settling are there that belong to legal issues arising from accusations? How many belong to something else? How many principles for settling are there that belong to legal issues arising from offenses? How many belong to something else? How many principles for settling are there that belong to legal issues arising from business? How many belong to something else? 

There\marginnote{20.1} are two principles for settling that belong to legal issues arising from disputes: resolution face-to-face and a majority decision. And there are five that belong to something else: resolution through recollection, resolution because of past insanity, acting according to what has been admitted, a further penalty, and covering over as if with grass. 

There\marginnote{21.1} are four principles for settling that belong to legal issues arising from accusations: resolution face-to-face, resolution through recollection, resolution because of past insanity, and a further penalty. And there are three that belong to something else: a majority decision, acting according to what has been admitted, and covering over as if with grass. 

There\marginnote{22.1} are three principles for settling that belong to legal issues arising from offenses: resolution face-to-face, acting according to what has been admitted, and covering over as if with grass. And there are four that belong to something else: a majority decision, resolution through recollection, resolution because of past insanity, and a further penalty. 

There\marginnote{23.1} is one principle for settling that belongs to legal issues arising from business: resolution face-to-face. And there are six that belong to something else: a majority decision, resolution through recollection, resolution because of past insanity, acting according to what has been admitted, a further penalty, and covering over as if with grass. 

\scendsutta{The eighth section on “belonging to that” is finished }

\section*{4. The section on different principles for settling used in common }

One\marginnote{25.1} principle for settling is used in common with another principle for settling, and one principle for settling is not used in common with another principle for settling. It may be that one principle for settling is used in common with another principle for settling, and it may be that one principle for settling is not used in common with another principle for settling. 

How\marginnote{26.1} may it be that one principle for settling is used in common with another principle for settling? How may it be that one principle for settling is not used in common with another principle for settling? 

A\marginnote{26.3} majority decision is used in common with resolution face-to-face. It is not used in common with resolution through recollection, resolution because of past insanity, acting according to what has been admitted, a further penalty, or covering over as if with grass. 

Resolution\marginnote{27.1} through recollection is used in common with resolution face-to-face. It is not used in common with resolution because of past insanity, acting according to what has been admitted, a further penalty, covering over as if with grass, or a majority decision. 

Resolution\marginnote{28.1} because of past insanity is used in common with resolution face-to-face. It is not used in common with acting according to what has been admitted, a further penalty, covering over as if with grass, a majority decision, or resolution through recollection. 

Acting\marginnote{29.1} according to what has been admitted is used in common with resolution face-to-face. It is not used in common with a further penalty, covering over as if with grass, a majority decision, resolution through recollection, or resolution because of past insanity. 

A\marginnote{30.1} further penalty is used in common with resolution face-to-face. It is not used in common with covering over as if with grass, a majority decision, resolution through recollection, resolution because of past insanity, or acting according to what has been admitted. 

Covering\marginnote{31.1} over as if with grass is used in common with resolution face-to-face. It is not used in common with a majority decision, resolution through recollection, resolution because of past insanity, acting according to what has been admitted, or a further penalty. 

\scendsutta{The ninth section on different principles for settling used in common is finished. }

\section*{5. The section on different principles for settling belonging with each other }

One\marginnote{34.1} principle for settling belongs with another principle for settling, and one principle for settling does not belong with another principle for settling. It may be that one principle for settling belongs with another principle for settling, and it may be that one principle for settling does not belong with another principle for settling. 

How\marginnote{35.1} may it be that one principle for settling belongs with another principle for settling? How may it be that one principle for settling does not belong with another principle for settling? 

A\marginnote{35.3} majority decision belongs with resolution face-to-face. It does not belong with resolution through recollection, resolution because of past insanity, acting according to what has been admitted, a further penalty, or covering over as if with grass. 

Resolution\marginnote{36.1} through recollection belongs with resolution face-to-face. It does not belong with resolution because of past insanity, acting according to what has been admitted, a further penalty, covering over as if with grass, or a majority decision. 

Resolution\marginnote{37.1} because of past insanity belongs with resolution face-to-face. It does not belong with acting according to what has been admitted, a further penalty, covering over as if with grass, a majority decision, or resolution through recollection. 

Acting\marginnote{38.1} according to what has been admitted belongs with resolution face-to-face. It does not belong with a further penalty, covering over as if with grass, a majority decision, resolution through recollection, or resolution because of past insanity. 

A\marginnote{39.1} further penalty belongs with resolution face-to-face. It does not belong with covering over as if with grass, a majority decision, resolution through recollection, resolution because of past insanity, or acting according to what has been admitted. 

Covering\marginnote{40.1} over as if with grass belongs with resolution face-to-face. It does not belong with a majority decision, resolution through recollection, resolution because of past insanity, acting according to what has been admitted, or a further penalty. 

\scendsutta{The tenth section on different principles for settling belonging with each other is finished. }

\section*{6. The section on “is a principle for settling also resolution face-to-face?” }

Is\marginnote{43.1} a principle for settling also resolution face-to-face, and is resolution face-to-face also a principle for settling? Is a principle for settling also a majority decision, and is a majority decision also a principle for settling? Is a principle for settling also resolution through recollection, and is resolution through recollection also a principle for settling? Is a principle for settling also resolution because of past insanity, and is resolution because of past insanity also a principle for settling? Is a principle for settling also acting according to what has been admitted, and is acting according to what has been admitted also a principle for settling? Is a principle for settling also a further penalty, and is a further penalty also a principle for settling? Is a principle for settling also covering over as if with grass, and is covering over as if with grass also a principle for settling? 

A\marginnote{44.1} majority decision, resolution through recollection, resolution because of past insanity, acting according to what has been admitted, a further penalty, and covering over as if with grass: these principles for settling are principles for settling, but they are not resolution face-to-face. Resolution face-to-face is both a principle for settling and also resolution face-to-face. 

Resolution\marginnote{45.1} through recollection, resolution because of past insanity, acting according to what has been admitted, a further penalty, covering over as if with grass, and resolution face-to-face: these principles for settling are principles for settling, but they are not a majority decision. A majority decision is both a principle for settling and also a majority decision. 

Resolution\marginnote{46.1} because of past insanity, acting according to what has been admitted, a further penalty, covering over as if with grass, resolution face-to-face, and a majority decision: these principles for settling are principles for settling, but they are not resolution through recollection. Resolution through recollection is both a principle for settling and also resolution through recollection. 

Acting\marginnote{47.1} according to what has been admitted, a further penalty, covering over as if with grass, resolution face-to-face, a majority decision, and resolution through recollection: these principles for settling are principles for settling, but they are not resolution because of past insanity. Resolution because of past insanity is both a principle for settling and also resolution because of past insanity. 

A\marginnote{48.1} further penalty, covering over as if with grass, resolution face-to-face, a majority decision, resolution through recollection, and resolution because of past insanity: these principles for settling are principles for settling, but they are not acting according to what has been admitted. Acting according to what has been admitted is both a principle for settling and also acting according to what has been admitted. 

Covering\marginnote{49.1} over as if with grass, resolution face-to-face, a majority decision, resolution through recollection, resolution because of past insanity, and acting according to what has been admitted: these principles for settling are principles for settling, but they are not a further penalty. A further penalty is both a principle for settling and also a further penalty. 

Resolution\marginnote{50.1} face-to-face, a majority decision, resolution through recollection, resolution because of past insanity, acting according to what has been admitted, and a further penalty: these principles for settling are principles for settling, but they are not covering over as if with grass. Covering over as if with grass is both a principle for settling and also covering over as if with grass. 

\scendsutta{The eleventh section on “is a principle for settling also resolution face-to-face?” is finished. }

\section*{7. The section on resolution }

Is\marginnote{52.1} a resolution also resolution face-to-face, and is resolution face-to-face also a resolution? Is a resolution also a majority decision, and is a majority decision also a resolution? Is a resolution also resolution through recollection, and is resolution through recollection also a resolution? Is a resolution also resolution because of past insanity, and is resolution because of past insanity also a resolution? Is a resolution also acting according to what has been admitted, and is acting according to what has been admitted also a resolution? Is a resolution also a further penalty, and is a further penalty also a resolution? Is a resolution also covering over as if with grass, and is covering over as if with grass also a resolution? 

A\marginnote{53.1} resolution may be resolution face-to-face, or it may not be resolution face-to-face. Resolution face-to-face is both a resolution and also resolution face-to-face. 

A\marginnote{54.1} resolution may be a majority decision, or it may not be a majority decision. A majority decision is both a resolution and also a majority decision. 

A\marginnote{55.1} resolution may be resolution through recollection, or it may not be resolution through recollection. Resolution through recollection is both a resolution and also resolution through recollection. 

A\marginnote{56.1} resolution may be resolution because of past insanity, or it may not be resolution because of past insanity. Resolution because of past insanity is both a resolution and also resolution because of past insanity. 

A\marginnote{57.1} resolution may be acting according to what has been admitted, or it may not be acting according to what has been admitted. Acting according to what has been admitted is both a resolution and also  acting according to what has been admitted. 

A\marginnote{58.1} resolution may be a further penalty, or it may not be a further penalty. A further penalty is both a resolution and also a further penalty. 

A\marginnote{59.1} resolution may be covering over as if with grass, or it may not be covering over as if with grass. Covering over as if with grass is both a resolution and also covering over as if with grass. 

\scendsutta{The twelfth section on resolution is finished. }

\section*{8. The section on the wholesome }

Is\marginnote{61.1} resolution face-to-face wholesome, unwholesome, or indeterminate? Is a majority decision wholesome, unwholesome, or indeterminate? Is resolution through recollection wholesome, unwholesome, or indeterminate? Is resolution because of past insanity wholesome, unwholesome, or indeterminate? Is acting according to what has been admitted wholesome, unwholesome, or indeterminate? Is a further penalty wholesome, unwholesome, or indeterminate? Is covering over as if with grass wholesome, unwholesome, or indeterminate? 

Resolution\marginnote{62.1} face-to-face may be wholesome or indeterminate; resolution face-to-face is never unwholesome. A majority decision may be wholesome, unwholesome, or indeterminate. Resolution through recollection may be wholesome, unwholesome, or indeterminate. Resolution because of past insanity may be wholesome, unwholesome, or indeterminate. Acting according to what has been admitted may be wholesome, unwholesome, or indeterminate. A further penalty may be wholesome, unwholesome, or indeterminate. Covering over as if with grass, may be wholesome, unwholesome, or indeterminate. 

Is\marginnote{69.1} a legal issue arising from a dispute wholesome, unwholesome, or indeterminate? Is a legal issue arising from an accusation wholesome, unwholesome, or indeterminate? Is a legal issue arising from an offense wholesome, unwholesome, or indeterminate? Is a legal issues arising from business wholesome, unwholesome, or indeterminate? 

A\marginnote{70.1} legal issue arising from a dispute may be wholesome, unwholesome, or indeterminate. A legal issue arising from an accusation may be wholesome, unwholesome, or indeterminate. A legal issue arising from an offense may be unwholesome or indeterminate; a legal issue arising from an offense is never wholesome. A legal issue arising from business may be wholesome, unwholesome, or indeterminate. 

\scendsutta{The thirteenth section on the wholesome is finished. }

\section*{9. The section on “where”, the section on questioning }

Resolution\marginnote{75.1} face-to-face is appropriate where a majority decision is appropriate. A majority decision is appropriate where resolution face-to-face is appropriate. But there, resolution through recollection is not appropriate, nor is resolution because of past insanity, acting according to what has been admitted, a further penalty, or covering over as if with grass. 

Resolution\marginnote{76.1} face-to-face is appropriate where resolution through recollection is appropriate. Resolution through recollection is appropriate where resolution face-to-face is appropriate. But there, resolution because of past insanity is not appropriate, nor is acting according to what has been admitted, a further penalty, covering over as if with grass, or a majority decision. 

Resolution\marginnote{77.1} face-to-face is appropriate where resolution because of past insanity is appropriate. Resolution because of past insanity is appropriate where resolution face-to-face is appropriate. But there, acting according to what has been admitted is not appropriate, nor is a further penalty, covering over as if with grass, a majority decision, or resolution through recollection. 

Resolution\marginnote{78.1} face-to-face is appropriate where acting according to what has been admitted is appropriate. Acting according to what has been admitted is appropriate where resolution face-to-face is appropriate. But there, a further penalty is not appropriate, nor is covering over as if with grass, a majority decision, resolution through recollection, or resolution because of past insanity. 

Resolution\marginnote{79.1} face-to-face is appropriate where a further penalty is appropriate. A further penalty is appropriate where resolution face-to-face is appropriate. But there, covering over as if with grass is not appropriate, nor is a majority decision, resolution through recollection, resolution because of past insanity, or acting according to what has been admitted. 

Resolution\marginnote{80.1} face-to-face is appropriate where covering over as if with grass is appropriate. Covering over as if with grass is appropriate where resolution face-to-face is appropriate. But there, a majority decision is not appropriate, nor is resolution through recollection, resolution because of past insanity, acting according to what has been admitted, or a further penalty. 

Where\marginnote{81.1} there is a majority decision, there is resolution face-to-face. Where there is resolution face-to-face, there is a majority decision. But there is no resolution through recollection there, nor resolution because of past insanity, acting according to what has been admitted, a further penalty, or covering over as if with grass. 

Where\marginnote{82.1} there is resolution through recollection, there is resolution face-to-face. Where there is resolution face-to-face, there is resolution through recollection. But there is no resolution because of past insanity there, nor acting according to what has been admitted, a further penalty, covering over as if with grass, or a majority decision. Resolution face-to-face to be done as the basis … 

Where\marginnote{83.1} there is covering over as if with grass, there is  resolution face-to-face. Where there is resolution face-to-face, there is covering over as if with grass. But there is no majority decision there, nor resolution through recollection, resolution because of past insanity, acting according to what has been admitted, or a further penalty. 

\scend{The successive permutation series. }

\scendsutta{The fourteenth section on “where” is finished. }

\section*{10. The section on settling, the section on responding }

On\marginnote{85.1} an occasion when a legal issue is being resolved through resolution face-to-face and a majority decision, then, where a majority decision is appropriate, there resolution face-to-face is appropriate, and where resolution face-to-face is appropriate, there a majority decision is appropriate. But there resolution through recollection is not appropriate, nor is resolution because of past insanity, acting according to what has been admitted, a further penalty, or covering over as if with grass. 

On\marginnote{86.1} an occasion when a legal issue is being resolved through resolution face-to-face and resolution through recollection, then, where resolution through recollection is appropriate, there resolution face-to-face is appropriate, and where resolution face-to-face is appropriate, there resolution through recollection is appropriate. But there resolution because of past insanity is not appropriate, nor is acting according to what has been admitted, a further penalty, covering over as if with grass, or a majority decision. 

On\marginnote{87.1} an occasion when a legal issue is being resolved through resolution face-to-face and resolution because of past insanity, then, where resolution because of past insanity is appropriate, there resolution face-to-face is appropriate, and where resolution face-to-face is appropriate, there resolution because of past insanity is appropriate. But there acting according to what has been admitted is not appropriate, nor is a further penalty, covering over as if with grass, a majority decision, or resolution through recollection. 

On\marginnote{88.1} an occasion when a legal issue is being resolved through resolution face-to-face and acting according to what has been admitted, then, where acting according to what has been admitted is appropriate, there resolution face-to-face is appropriate, and where resolution face-to-face is appropriate, there acting according to what has been admitted is appropriate. But there a further penalty is not appropriate, nor is covering over as if with grass, a majority decision, resolution through recollection, or resolution because of past insanity. 

On\marginnote{89.1} an occasion when a legal issue is being resolved through resolution face-to-face and a further penalty, then, where a further penalty is appropriate, there resolution face-to-face is appropriate, and where resolution face-to-face is appropriate, there a further penalty is appropriate. But there covering over as if with grass is not appropriate, nor is a majority decision, resolution through recollection, resolution because of past insanity, or acting according to what has been admitted. 

On\marginnote{90.1} an occasion when a legal issue is being resolved through resolution face-to-face and covering over as if with grass, then, where covering over as if with grass is appropriate, there resolution face-to-face is appropriate, and where resolution face-to-face is appropriate, there covering over as if with grass is appropriate. But there a majority decision is not appropriate, nor is resolution through recollection, resolution because of past insanity, acting according to what has been admitted, or a further penalty. 

\scendsutta{The fifteenth section on settling is finished. }

\section*{11. The section on connection }

“Are\marginnote{92.1} ‘legal issue’ and ‘principle for settling’ connected or disconnected? Is it possible to completely separate them and point to their difference?” 

“‘Legal\marginnote{93.1} issue’ and ‘principle for settling’ are disconnected, not connected, and it’s possible to completely separate them and point to their difference”: they should be told not to say this. “‘Legal issue’ and ‘principle for settling’ are connected, not disconnected, and it’s impossible to completely separate them and point to their difference. Why’s that? Didn’t the Buddha say that there are four kinds of legal issues and seven principles for settling? Legal issues are settled through the principles for settling; the principles for settling are settled through legal issues.\footnote{The latter of these two is explained at Vmv 5.306: \textit{\textsanskrit{Samathāadhikaraṇehi} \textsanskrit{sammantīti} \textsanskrit{apalokanādīhi} \textsanskrit{catūhi} \textsanskrit{kiccādhikaraṇehi} sabbepi \textsanskrit{samathā} \textsanskrit{niṭṭhānaṁ} gacchanti, \textsanskrit{nāññehīti} \textsanskrit{imamatthaṁ} \textsanskrit{sandhāya} \textsanskrit{vuttaṁ}}, “\textit{\textsanskrit{Samathāadhikaraṇehi} sammanti}: with the four legal issues arising from business, starting with the legal procedure of getting permission, all principles for settling, too, come to an end, but not with any other legal issue. It was said with reference to this.” } In this way they are connected, not disconnected, and it’s impossible to completely separate them and point to their difference.” 

\scendsutta{The sixteenth section on connection is finished. }

\section*{12. The section on settling }

Through\marginnote{95.1} how many principles for settling is a legal issue arising from a dispute settled? Through how many principles for settling is a legal issue arising from an accusation settled? Through how many principles for settling is a legal issue arising from an offense settled? Through how many principles for settling is a legal issue arising from business settled? 

A\marginnote{96.1} legal issue arising from a dispute is settled through two principles: through resolution face-to-face and through a majority decision. 

A\marginnote{97.1} legal issue arising from an accusation is settled through four principles: through resolution face-to-face, through resolution through recollection, through resolution because of past insanity, and through a further penalty. 

A\marginnote{98.1} legal issue arising from an offense is settled through three principles: through resolution face-to-face, through acting according to what has been admitted, and through covering over as if with grass. 

A\marginnote{99.1} legal issue arising from business is settled through one principle: through resolution face-to-face. 

Through\marginnote{100.1} how many principles for settling is a legal issue arising from a dispute and a legal issue arising from an accusation settled? Through five: through resolution face-to-face, through a majority decision, through resolution through recollection, through resolution because of past insanity, and through a further penalty. 

Through\marginnote{101.1} how many principles for settling is a legal issue arising from a dispute and a legal issue arising from an offense settled? Through four: through resolution face-to-face, through a majority decision, through acting according to what has been admitted, and through covering over as if with grass. 

Through\marginnote{102.1} how many principles for settling is a legal issue arising from a dispute and a legal issue arising from business settled? Through two: through resolution face-to-face, and through a majority decision. 

Through\marginnote{103.1} how many principles for settling is a legal issue arising from an accusation and a legal issue arising from an offense settled? Through six: through resolution face-to-face, through resolution through recollection, through resolution because of past insanity, through acting according to what has been admitted, through a further penalty, and through covering over as if with grass. 

Through\marginnote{104.1} how many principles for settling is a legal issue arising from an accusation and a legal issue arising from business settled? Through four: through resolution face-to-face, through resolution through recollection, through resolution because of past insanity, and through a further penalty. 

Through\marginnote{105.1} how many principles for settling is a legal issue arising from an offense and a legal issue arising from business settled? Through three: through resolution face-to-face, through acting according to what has been admitted, and through covering over as if with grass. 

Through\marginnote{106.1} how many principles for settling is a legal issue arising from a dispute, a legal issue arising from an accusation, and a legal issue arising from an offense settled? Through seven: through resolution face-to-face, through a majority decision, through resolution through recollection, through resolution because of past insanity, through acting according to what has been admitted, through a further penalty, and through covering over as if with grass. 

Through\marginnote{107.1} how many principles for settling is a legal issue arising from a dispute, a legal issue arising from an accusation, and a legal issue arising from business settled? Through five: through resolution face-to-face, through a majority decision, through resolution through recollection, through resolution because of past insanity, and through a further penalty. 

Through\marginnote{108.1} how many principles for settling is a legal issue arising from an accusation, a legal issue arising from an offense, and a legal issue arising from business settled?\footnote{The combination legal issue arising from a dispute, legal issue arising from an offense, and legal issue arising from business is not mentioned. } Through six: through resolution face-to-face, through resolution through recollection, through resolution because of past insanity, through acting according to what has been admitted, through a further penalty, and through covering over as if with grass. 

Through\marginnote{109.1} how many principles for settling is a legal issue arising from a dispute, a legal issue arising from an accusation, a legal issue arising from an offense, and a legal issue arising from business settled? Through seven: through resolution face-to-face, through a majority decision, through resolution through recollection, through resolution because of past insanity, through acting according to what has been admitted, through a further penalty, and through covering over as if with grass. 

\scendsutta{The seventeenth section on settling is finished. }

\section*{13. The section on settling and not settling }

Through\marginnote{111.1} how many principles for settling is a legal issue arising from a dispute settled, and through how many principles is it not settled? Through how many principles for settling is a legal issue arising from an accusation settled, and through how many principles is it not settled? Through how many principles for settling is a legal issue arising from an offense settled, and through how many principles is it not settled? Through how many principles for settling is a legal issue arising from business settled, and through how many principles is it not settled? 

A\marginnote{112.1} legal issue arising from a dispute is settled through two principles: through resolution face-to-face, and through a majority decision. It is not settled through five principles: through resolution through recollection, through resolution because of past insanity, through acting according to what has been admitted, through a further penalty, and through covering over as if with grass. 

A\marginnote{113.1} legal issue arising from an accusation is settled through four principles: through resolution face-to-face, through resolution through recollection, through resolution because of past insanity, and through a further penalty. It is not settled through three principles: through a majority decision, through acting according to what has been admitted, and through covering over as if with grass. 

A\marginnote{114.1} legal issue arising from an offense is settled through three principles: through resolution face-to-face, through acting according to what has been admitted, and through covering over as if with grass. It is not settled through four principles: through a majority decision, through resolution through recollection, through resolution because of past insanity, and through a further penalty. 

A\marginnote{115.1} legal issue arising from business is settled through one principle: through resolution face-to-face. It is not settled through six principles: through a majority decision, through resolution through recollection, through resolution because of past insanity, through acting according to what has been admitted, through a further penalty, and through covering over as if with grass. 

Through\marginnote{116.1} how many principles for settling are a legal issue arising from a dispute and a legal issue arising from an accusation settled? Through how many principles are they not settled? They are settled through five principles: through resolution face-to-face, through a majority decision, through resolution through recollection, through resolution because of past insanity, and through a further penalty. They are not settled through two principles: through acting according to what has been admitted, and through covering over as if with grass. 

Through\marginnote{117.1} how many principles for settling are a legal issue arising from a dispute and a legal issue arising from an offense settled? Through how many principles are they not settled? They are settled through four principles: through resolution face-to-face, through a majority decision, through acting according to what has been admitted, and through covering over as if with grass. They are not settled through three principles: through resolution through recollection, through resolution because of past insanity, and through a further penalty. 

Through\marginnote{118.1} how many principles for settling are a legal issue arising from a dispute and a legal issue arising from business settled? Through how many principles are they not settled? They are settled through two principles: through resolution face-to-face, and through a majority decision. They are not settled through five principles: through resolution through recollection, through resolution because of past insanity, through acting according to what has been admitted, through a further penalty, and through covering over as if with grass. 

Through\marginnote{119.1} how many principles for settling are a legal issue arising from an accusation and a legal issue arising from an offense settled? Through how many principles are they not settled? They are settled through six principles: through resolution face-to-face, through resolution through recollection, through resolution because of past insanity, through acting according to what has been admitted, through a further penalty, and through covering over as if with grass. They are not settled through one principle: through a majority decision. 

Through\marginnote{120.1} how many principles for settling are a legal issue arising from an accusation and a legal issue arising from business settled? Through how many principles are they not settled? They are settled through four principles: through resolution face-to-face, through resolution through recollection, through resolution because of past insanity, and through a further penalty. They are not settled through three principles: through a majority decision, through acting according to what has been admitted, and through covering over as if with grass. 

Through\marginnote{121.1} how many principles for settling are a legal issue arising from an offense and a legal issue arising from business settled? Through how many principles are they not settled? They are settled through three principles: through resolution face-to-face, through acting according to what has been admitted, and through covering over as if with grass. They are not settled through four principles: through a majority decision, through  resolution through recollection, through resolution because of past insanity, and through a further penalty. 

Through\marginnote{122.1} how many principles for settling are a legal issue arising from a dispute, a legal issue arising from an accusation, and a legal issue arising from an offense settled? Through how many principles are they not settled? They are settled through seven principles: through resolution face-to-face, through a majority decision, through resolution through recollection, through resolution because of past insanity, through acting according to what has been admitted, through a further penalty, and through covering over as if with grass. 

Through\marginnote{123.1} how many principles for settling are a legal issue arising from a dispute, a legal issue arising from an accusation, and a legal issue arising from business settled? Through how many principles are they not settled? They are settled through five principles: through resolution face-to-face, through a majority decision, through resolution through recollection, through resolution because of past insanity, and through a further penalty. They are not settled through two principles: through acting according to what has been admitted, and through covering over as if with grass. 

Through\marginnote{124.1} how many principles for settling are a legal issue arising from an accusation, a legal issue arising from an offense, and a legal issue arising from business settled?\footnote{Again, the combination legal issue arising from a dispute, legal issue arising from an offense, and legal issue arising from business is not mentioned. } Through how many principles are they not settled? They are settled through six principles: through resolution face-to-face, through resolution through recollection, through resolution because of past insanity, through acting according to what has been admitted, through a further penalty, and through covering over as if with grass. They are not settled through one principle: through a majority decision. 

Through\marginnote{125.1} how many principles for settling are a legal issue arising from a dispute, a legal issue arising from an accusation, a legal issue arising from an offense, and a legal issue arising from business settled? Through how many principles are they not settled? They are settled through seven principles: through resolution face-to-face, through a majority decision, through resolution through recollection, through  resolution because of past insanity, through acting according to what has been admitted, through a further penalty, and through covering over as if with grass. 

\scendsutta{The eighteenth section on settling and not settling is finished. }

\section*{14. The section on principles for settling and legal issues }

Are\marginnote{127.1} principles for settling settled through principles for settling? Are principles for settling settled through legal issues? Are legal issues settled through principles for settling? Are legal issues settled through legal issues? 

It\marginnote{128.1} may be that principles for settling are settled through principles for settling; it may be that principles for settling are not settled through principles for settling. It may be that principles for settling are settled through legal issues; it may be that principles for settling are not settled through legal issues. It may be that legal issues are settled through principles for settling; it may be that legal issues are not settled through principles for settling. It may be that legal issues are settled through legal issues; it may be that legal issues are not settled through legal issues. 

How\marginnote{129.1} may it be that principles for settling are settled through principles for settling? And how may it be that principles for settling are not settled through principles for settling? 

A\marginnote{129.2} majority decision is settled through resolution face-to-face. It is not settled through resolution through recollection, through resolution because of past insanity, through acting according to what has been admitted, through a further penalty, or through covering over as if with grass. 

Resolution\marginnote{130.1} through recollection is settled through resolution face-to-face. It is not settled through resolution because of past insanity, through acting according to what has been admitted, through a further penalty, through covering over as if with grass, or through a majority decision. 

Resolution\marginnote{131.1} because of past insanity is settled through resolution face-to-face. It is not settled through acting according to what has been admitted, through a further penalty, through covering over as if with grass, through a majority decision, or through resolution through recollection. 

Acting\marginnote{132.1} according to what has been admitted is settled through resolution face-to-face. It is not settled through a further penalty, through covering over as if with grass, through a majority decision, through resolution through recollection, or through resolution because of past insanity. 

A\marginnote{133.1} further penalty is settled through resolution face-to-face. It is not settled through covering over as if with grass, through a majority decision, through resolution through recollection, through resolution because of past insanity, or through acting according to what has been admitted. 

Covering\marginnote{134.1} over as if with grass is settled through resolution face-to-face. It is not settled through a majority decision, through resolution through recollection, through resolution because of past insanity, through acting according to what has been admitted, or through a further penalty. 

How\marginnote{135.1} may it be that principles for settling are settled through legal issues? How may it be that principles for settling are not settled through legal issues? 

Resolution\marginnote{135.2} face-to-face is not settled through a legal issue arising from a dispute, a legal issue arising from an accusation, or a legal issue arising from an offense. It is settled through a legal issue arising from business. 

A\marginnote{136.1} majority decision is not settled through a legal issue arising from a dispute, a legal issue arising from an accusation, or a legal issue arising from an offense. It is settled through a legal issue arising from business. 

Resolution\marginnote{137.1} through recollection is not settled through a legal issue arising from a dispute, a legal issue arising from an accusation, or a legal issue arising from an offense. It is settled through a legal issue arising from business. 

Resolution\marginnote{138.1} because of past insanity is not settled through a legal issue arising from a dispute, a legal issue arising from an accusation, or a legal issue arising from an offense. It is settled through a legal issue arising from business. 

Acting\marginnote{139.1} according to what has been admitted is not settled through a legal issue arising from a dispute, a legal issue arising from an accusation, or a legal issue arising from an offense. It is settled through a legal issue arising from business. 

A\marginnote{140.1} further penalty is not settled through a legal issue arising from a dispute, a legal issue arising from an accusation, or a legal issue arising from an offense. It is settled through a legal issue arising from business. 

Covering\marginnote{141.1} over as if with grass is not settled through a legal issue arising from a dispute, a legal issue arising from an accusation, or a legal issue arising from an offense. It is settled through a legal issue arising from business. 

How\marginnote{142.1} may it be that legal issues are settled through principles for settling? How may it be that legal issues are not settled through principles for settling? 

A\marginnote{142.2} legal issue arising from a dispute is settled through resolution face-to-face and through a majority decision. It is not settled through resolution through recollection, through resolution because of past insanity, through acting according to what has been admitted, through a further penalty, or through covering over as if with grass. 

A\marginnote{143.1} legal issue arising from an accusation is settled through resolution face-to-face, through resolution through recollection, through resolution because of past insanity, and through a further penalty. It is not settled through a majority decision, through acting according to what has been admitted, or through covering over as if with grass. 

A\marginnote{144.1} legal issue arising from an offense is settled through resolution face-to-face, through acting according to what has been admitted, and through covering over as if with grass. It is not settled through a majority decision, through resolution through recollection, through resolution because of past insanity, or through a further penalty. 

A\marginnote{145.1} legal issue arising from business is settled through resolution face-to-face. It is not settled through a majority decision, through resolution through recollection, through resolution because of past insanity, through acting according to what has been admitted, through a further penalty, or through covering over as if with grass. 

How\marginnote{146.1} may it be that legal issues are settled through legal issues? How may it be that legal issues are not settled through legal issues? 

A\marginnote{146.2} legal issue arising from a dispute is not settled through a legal issue arising from a dispute, through a legal issue arising from an accusation, or through a legal issue arising from an offense. It is settled through a legal issue arising from business. 

A\marginnote{147.1} legal issue arising from an accusation is not settled through a legal issue arising from a dispute, through a legal issue arising from an accusation, or through a legal issue arising from an offense. It is settled through a legal issue arising from business. 

A\marginnote{148.1} legal issue arising from an offense is not settled through a legal issue arising from a dispute, through a legal issue arising from an accusation, or through a legal issue arising from an offense. It is settled through a legal issue arising from business. 

A\marginnote{149.1} legal issue arising from business is not settled through a legal issue arising from a dispute, through a legal issue arising from an accusation, or through a legal issue arising from an offense. It is settled through a legal issue arising from business. 

The\marginnote{150.1} six principles for settling and the four legal issues are settled through resolution face-to-face. Resolution face-to-face is not settled through anything. 

\scendsutta{The nineteenth section on principles for settling and legal issues is finished. }

\section*{15. The section on causing to originate }

Which\marginnote{152.1} of the four legal issues causes a legal issue arising from a dispute to originate? None of them. Nevertheless, the four legal issues are produced from a legal issue arising from a dispute. How is that? It may be that monks are disputing, saying, “This is the Teaching”, “This is contrary to the Teaching”, “This is the Monastic Law”, “This is contrary to the Monastic Law”, “This was spoken by the Buddha”, “This was not spoken by the Buddha”, “This was practiced by the Buddha”, “This was not practiced by the Buddha”, “This was laid down by the Buddha”, “This was not laid down by the Buddha”, “This is an offense”, “This is not an offense”, “This is a light offense”, “This is a heavy offense”, “This is a curable offense”, “This is an incurable offense”, “This is a grave offense”, or “This is a minor offense.” In regard to this, whatever there is of quarreling, arguing, conflict, disputing, variety in opinion, difference in opinion, heated speech, or strife—this is called a legal issue arising from a dispute. When, during a legal issue arising from a dispute, the Sangha disputes, there is a legal issue arising from a dispute.\footnote{The punctuation of the Pali in this and the next two segments does not seem right. The purpose of this and the next three sentences is to answer the question posed just above about how the four legal issues are produced from a legal issue arising from a dispute. For this to work, I propose the following punctuation: \textit{\textsanskrit{Vivādādhikaraṇe} \textsanskrit{saṅgho} vivadati \textsanskrit{vivādādhikaraṇaṁ}. \textsanskrit{Vivadamāno} anuvadati \textsanskrit{anuvādādhikaraṇaṁ}. \textsanskrit{Anuvadamāno} \textsanskrit{āpattiṁ} \textsanskrit{āpajjati} \textsanskrit{āpattādhikaraṇaṁ}. \textsanskrit{Tāya} \textsanskrit{āpattiyā} \textsanskrit{saṅgho} \textsanskrit{kammaṁ} karoti \textsanskrit{kiccādhikaraṇaṁ}.} This suggestion is supported by the punctuation found in the parallel passage at \href{https://suttacentral.net/pli-tv-pvr11/en/brahmali\#32.9}{Pvr 11:32.9}–32.11. I translate accordingly. And also below for the other three kinds of legal issues. } When one who is disputing makes an accusation, there is a legal issue arising from an accusation. When one who is accusing commits an offense, there is a legal issue arising from an offense. When the Sangha does a legal procedure because of that offense, there is a legal issue arising from business. 

Which\marginnote{153.1} of the four legal issues causes a legal issue arising from an accusation to originate? None of them. Nevertheless, the four legal issues are produced from a legal issue arising from an accusation. How is that? It may be that the monks accuse a monk of failure in morality, failure in conduct, failure in view, or failure in livelihood. In regard to this, whatever there is of accusations, accusing, allegations, blame, taking sides because of friendship, taking part in the accusation, or supporting the accusation—this is called a legal issue arising from an accusation. When, during a legal issue arising from an accusation, the Sangha disputes, there is a legal issue arising from a dispute. When one who is disputing makes an accusation, there is a legal issue arising from an accusation. When one who is accusing commits an offense, there is a legal issue arising from an offense. When the Sangha does a legal procedure because of that offense, there is a legal issue arising from business. 

Which\marginnote{154.1} of the four legal issues causes a legal issue arising from an offense to originate? None of them. Nevertheless, the four legal issues are produced from a legal issue arising from an offense. How is that? There are legal issues arising from offenses because of the five classes of offenses, and there are legal issues arising from offenses because of the seven classes of offenses—these are called legal issues arising from offenses. When, during a legal issue arising from an offense, the Sangha disputes, there is a legal issue arising from a dispute. When one who is disputing makes an accusation, there is a legal issue arising from an accusation. When one who is accusing commits an offense, there is a legal issue arising from an offense. When the Sangha does a legal procedure because of that offense, there is a legal issue arising from business. 

Which\marginnote{155.1} of the four legal issues causes a legal issue arising from business to originate? None of them. Nevertheless, the four legal issues are produced from a legal issue arising from business. How is that? Whatever is the duty or the business of the Sangha—a legal procedure consisting of getting permission, a legal procedure consisting of one motion, a legal procedure consisting of one motion and one announcement, a legal procedure consisting of one motion and three announcements—this is called a legal issue arising from business. When, during a legal issue arising from business, the Sangha disputes, there is a legal issue arising from a dispute. When one who is disputing makes an accusation, there is a legal issue arising from an accusation. When one who is accusing commits an offense, there is a legal issue arising from an offense. When the Sangha does a legal procedure because of that offense, there is a legal issue arising from business. 

\scendsutta{The twentieth section on causing to originate is finished. }

\section*{16. The section on “belonging to” }

“To\marginnote{157.1} which of the four legal issues does a legal issue arising from a dispute belong? Which legal issue does it depend on? Which legal issue is it included in? Which legal issue is it grouped with? 

To\marginnote{158.1} which of the four legal issues does a legal issue arising from an accusation belong? Which legal issue does it depend on? Which legal issue is it included in? Which legal issue is it grouped with? 

To\marginnote{159.1} which of the four legal issues does a legal issue arising from an offense belong? Which legal issue does it depend on? Which legal issue is it included in? Which legal issue is it grouped with? 

To\marginnote{160.1} which of the four legal issues does a legal issue arising from business belong? Which legal issue does it depend on? Which legal issue is it included in? Which legal issue is it grouped with? 

A\marginnote{161.1} legal issue arising from a dispute belongs to legal issues arising from disputes; it depends on legal issues arising from disputes; it is included in legal issues arising from disputes; it is grouped with legal issues arising from disputes. 

A\marginnote{162.1} legal issue arising from an accusation belongs to legal issues arising from accusations; it depends on legal issues arising from accusations; it is included in legal issues arising from accusations; it is grouped with legal issues arising from accusations. 

A\marginnote{163.1} legal issue arising from an offense belongs to legal issues arising from offenses; it depends on legal issues arising from offenses; it is included in legal issues arising from offenses; it is grouped with legal issues arising from offenses. 

A\marginnote{164.1} legal issue arising from business belongs to legal issues arising from business; it depends on legal issues arising from business; it is included in legal issues arising from business; it is grouped with legal issues arising from business. 

To\marginnote{165.1} how many of the seven principles for settling does a legal issue arising from a dispute belong? On how many principles for settling does it depend? In how many principles for settling is it included? With how many principles for settling is it grouped? Through how many principles for settling is it settled? 

To\marginnote{166.1} how many of the seven principles for settling does a legal issue arising from an accusation belong? On how many principles for settling does it depend? In how many principles for settling is it included? With how many principles for settling is it grouped? Through how many principles for settling is it settled? 

To\marginnote{167.1} how many of the seven principles for settling does a legal issue arising from an offense belong? On how many principles for settling does it depend? In how many principles for settling is it included? With how many principles for settling is it grouped? Through how many principles for settling is it settled? 

To\marginnote{168.1} how many of the seven principles for settling does a legal issue arising from business belong? On how many principles for settling does it depend? In how many principles for settling is it included? With how many principles for settling is it grouped? Through how many principles for settling is it settled? 

A\marginnote{169.1} legal issue arising from a dispute belongs to two principles for settling; it depends on two principles for settling; it is included in two principles for settling; it is grouped with two principles for settling; it is settled through two principles for settling: through resolution face-to-face and through a majority decision. 

A\marginnote{170.1} legal issue arising from an accusation belongs to four principles for settling; it depends on four principles for settling; it is included in four principles for settling; it is grouped with four principles for settling; it is settled through four principles for settling: through resolution face-to-face, through resolution through recollection, through resolution because of past insanity, and through a further penalty. 

A\marginnote{171.1} legal issue arising from an offense belongs to three principles for settling; it depends on three principles for settling; it is included in three principles for settling; it is grouped with three principles for settling; it is settled through three principles for settling: through resolution face-to-face, through acting according to what has been admitted, and through covering over as if with grass. 

A\marginnote{172.1} legal issue arising from business belongs to one principle for settling; it depends on one principle for settling; it is included in one principle for settling; it is grouped with one principle for settling; it is settled through one principle for settling: through resolution face-to-face.” 

\scendsutta{The twenty-first section on “belonging to” is finished. }

\scend{The legal issues and their settling are finished. }

\scuddanaintro{This is the summary: }

\begin{scuddana}%
“Legal\marginnote{176.1} issue, a different presentation, \\
And in common, belonging to; \\
Principles for settling used in common, \\
Principles for settling that belong with that. 

Principle\marginnote{177.1} for settling is also resolution face-to-face, \\
On resolution, and on the wholesome; \\
Where, on settling, on connection, \\
On settling, and on not settling; \\
And on principles for settling and legal issues, \\
Origination, and belonging to.” 

%
\end{scuddana}

%
\chapter*{{\suttatitleacronym Pvr 6}{\suttatitletranslation Offenses in the Khandhakas }{\suttatitleroot Khandhakapucchāvāra}}
\addcontentsline{toc}{chapter}{\tocacronym{Pvr 6} \toctranslation{Offenses in the Khandhakas } \tocroot{Khandhakapucchāvāra}}
\markboth{Offenses in the Khandhakas }{Khandhakapucchāvāra}
\extramarks{Pvr 6}{Pvr 6}

\begin{verse}%
“I\marginnote{1.1} will ask about the full ordination, together with its origin stories and its detailed explanations:\footnote{It is noteworthy that it is not called the Great Chapter. “Full ordination” is close to what this chapter is called in other schools, that is, “going forth”. } \\
how many offenses are laid down in its exalted clauses? \\
I will answer about the full ordination, together with its origin stories and its detailed explanations: \\
two offenses are laid down in its exalted clauses.\footnote{\href{https://suttacentral.net/Sp 4.320/en/brahmali}{SP 4.320}: \textit{Dve \textsanskrit{āpattiyoti} \textsanskrit{ūnavīsativassaṁ} \textsanskrit{upasampādentassa} \textsanskrit{pācittiyaṁ}, sesesu sabbapadesu \textsanskrit{dukkaṭaṁ}}, “Two offenses: there is an offense entailing confession for one ordaining a person less than twenty years old. There is an offense of wrong conduct in regard to all the other clauses.” In the following cases, I will just note the classes of offenses that are mentioned in the commentary. } 

I\marginnote{2.1} will ask about the observance day, together with its origin stories and its detailed explanations: \\
how many offenses are laid down in its exalted clauses? \\
I will answer about the observance day, together with its origin stories and its detailed explanations: \\
three offenses are laid down in its exalted clauses.\footnote{The serious offense, the offense entailing confession, and the offense of wrong conduct. } 

I\marginnote{3.1} will ask about entering the rainy-season residence, together with its origin stories and its detailed explanations: \\
how many offenses are laid down in its exalted clauses? \\
I will answer about entering the rainy-season residence, together with its origin stories and its detailed explanations: \\
one offense is laid down in its exalted clauses.\footnote{The offense of wrong conduct. } 

I\marginnote{4.1} will ask about the invitation ceremony, together with its origin stories and its detailed explanations: \\
how many offenses are laid down in its exalted clauses? \\
I will answer about the invitation ceremony, together with its origin stories and its detailed explanations: \\
three offenses are laid down in its exalted clauses.\footnote{The serious offense, the offense entailing confession, and the offense of wrong conduct. } 

I\marginnote{5.1} will ask about that which is connected with skins, together with its origin stories and its detailed explanations: \\
how many offenses are laid down in its exalted clauses? \\
I will answer about that which is connected with skins, together with its origin stories and its detailed explanations: \\
three offenses are laid down in its exalted clauses.\footnote{The offense entailing confession, the serious offense, and the offense of wrong conduct. } 

I\marginnote{6.1} will ask about medicines, together with its origin stories and its detailed explanations: \\
how many offenses are laid down in its exalted clauses? \\
I will answer about medicines, together with its origin stories and its detailed explanations: \\
three offenses are laid down in its exalted clauses.\footnote{The serious offense, the offense entailing confession, and the offense of wrong conduct. } 

I\marginnote{7.1} will ask about the robe-making ceremony, together with its origin stories and its detailed explanations:\footnote{For a discussion of the word \textit{kathina}, see Appendix of Technical Terms. } \\
how many offenses are laid down in its exalted clauses? \\
I will answer about the robe-making ceremony, together with its origin stories and its detailed explanations: \\
there is no offense laid down in its exalted clauses. 

I\marginnote{8.1} will ask about that which is connected with robes, together with its origin stories and its detailed explanations: \\
how many offenses are laid down in its exalted clauses? \\
I will answer about that which is connected with robes, together with its origin stories and its detailed explanations: \\
three offenses are laid down in its exalted clauses.\footnote{The serious offense, the offense entailing relinquishment and confession, and the offense of wrong conduct. } 

I\marginnote{9.1} will ask about that which is connected with \textsanskrit{Campā}, together with its origin stories and its detailed explanations: \\
how many offenses are laid down in its exalted clauses? \\
I will answer about that which is connected with \textsanskrit{Campā}, together with its origin stories and its detailed explanations: \\
one offense is laid down in its exalted clauses.\footnote{The offense of wrong conduct. } 

I\marginnote{10.1} will ask about that which is connected with \textsanskrit{Kosambī}, together with its origin stories and its detailed explanations: \\
how many offenses are laid down in its exalted clauses? \\
I will answer about that which is connected with \textsanskrit{Kosambī}, together with its origin stories and its detailed explanations: \\
one offense is laid down in its exalted clauses.\footnote{The offense of wrong conduct. } 

I\marginnote{11.1} will ask about the Chapter on Legal Procedures, together with its origin stories and its detailed explanations: \\
how many offenses are laid down in its exalted clauses? \\
I will answer about the Chapter on Legal Procedures, together with its origin stories and its detailed explanations: \\
one offense is laid down in its exalted clauses.\footnote{The offense of wrong conduct. } 

I\marginnote{12.1} will ask about those on probation, together with its origin stories and its detailed explanations: \\
how many offenses are laid down in its exalted clauses? \\
I will answer about those on probation, together with its origin stories and its detailed explanations: \\
one offense is laid down in its exalted clauses.\footnote{The offense of wrong conduct. } 

I\marginnote{13.1} will ask about gathering up, together with its origin stories and its detailed explanations: \\
how many offenses are laid down in its exalted clauses? \\
I will answer about the gathering up, together with its origin stories and its detailed explanations: \\
one offense is laid down in its exalted clauses.\footnote{The offense of wrong conduct. } 

I\marginnote{14.1} will ask about the settling of legal issues, together with its origin stories and its detailed explanations: \\
how many offenses are laid down in its exalted clauses? \\
I will answer about the settling of legal issues, together with its origin stories and its detailed explanations: \\
two offenses are laid down in its exalted clauses.\footnote{The offense entailing confession, and the offense of wrong conduct. } 

I\marginnote{15.1} will ask about minor topics, together with its origin stories and its detailed explanations: \\
how many offenses are laid down in its exalted clauses? \\
I will answer about the minor topics, together with its origin stories and its detailed explanations: \\
three offenses are laid down in its exalted clauses.\footnote{The serious offense, the offense entailing confession, and the offense of wrong conduct. } 

I\marginnote{16.1} will ask about resting places, together with its origin stories and its detailed explanations:\footnote{For an explanation of the rendering “resting places” for \textit{\textsanskrit{senāsana}}, see Appendix of Technical Terms. } \\
how many offenses are laid down in its exalted clauses? \\
I will answer about resting places, together with its origin stories and its detailed explanations: \\
three offenses are laid down in its exalted clauses.\footnote{The serious offense, the offense entailing confession, and the offense of wrong conduct. } 

I\marginnote{17.1} will ask about schism in the Sangha, together with its origin stories and its detailed explanations: \\
how many offenses are laid down in its exalted clauses? \\
I will answer about schism in the Sangha, together with its origin stories and its detailed explanations: \\
two offenses are laid down in its exalted clauses.\footnote{The serious offense, and the offense entailing confession. } 

I\marginnote{18.1} will ask about conduct, together with its origin stories and its detailed explanations: \\
how many offenses are laid down in its exalted clauses? \\
I will answer about conduct, together with its origin stories and its detailed explanations: \\
one offense is laid down in its exalted clauses.\footnote{The offense of wrong conduct. } 

I\marginnote{19.1} will ask about cancellation, together with its origin stories and its detailed explanations: \\
how many offenses are laid down in its exalted clauses? \\
I will answer about cancellation, together with its origin stories and its detailed explanations: \\
one offense is laid down in its exalted clauses.\footnote{The offense of wrong conduct. } 

I\marginnote{20.1} will ask about the Chapter on Nuns, together with its origin stories and its detailed explanations: \\
how many offenses are laid down in its exalted clauses? \\
I will answer about the Chapter on Nuns, together with its origin stories and its detailed explanations: \\
two offenses are laid down in its exalted clauses.\footnote{The offense entailing confession, and the offense of wrong conduct. } 

I\marginnote{21.1} will ask about the group of five hundred, together with its origin stories and its detailed explanations: \\
how many offenses are laid down in its exalted clauses? \\
I will answer about the group of five hundred, together with its origin stories and its detailed explanations: \\
there is no offense laid down in its exalted clauses. 

I\marginnote{22.1} will ask about the group of seven hundred, together with its origin stories and its detailed explanations: \\
how many offenses are laid down in its exalted clauses? \\
I will answer about the group of seven hundred, together with its origin stories and its detailed explanations: \\
there is no offense laid down in its exalted clauses.” 

%
\end{verse}

\scend{The section on offenses in the Khandhakas, the first, is finished. }

\scuddanaintro{This is the summary: }

\begin{scuddana}%
“Full\marginnote{25.1} ordination, observance day, \\
Entering the rainy-season residence, invitation ceremony; \\
Skins, medicines, robe-making ceremony, \\
Robes, and that which is connected with \textsanskrit{Campā}, 

The\marginnote{26.1} Chapter on \textsanskrit{Kosambī}, legal procedures, \\
Those on probation, the gathering up; \\
Settling of legal issues, minor topics, resting places, \\
Schism in the Sangha, conduct; \\
Cancellation, the Chapter on Nuns, \\
And with the five and seven hundred.” 

%
\end{scuddana}

%
\chapter*{{\suttatitleacronym Pvr 7}{\suttatitletranslation The numerical method }{\suttatitleroot Ekuttarikanaya}}
\addcontentsline{toc}{chapter}{\tocacronym{Pvr 7} \toctranslation{The numerical method } \tocroot{Ekuttarikanaya}}
\markboth{The numerical method }{Ekuttarikanaya}
\extramarks{Pvr 7}{Pvr 7}

\section*{1. The section on ones }

“The\marginnote{1.1} things that produce offenses should be known. The things that do not produce offenses should be known. Offenses should be known. Non-offenses should be known. Light offenses should be known. Serious offenses should be known. Curable offenses should be known. Incurable offenses should be known. Grave offenses should be known. Minor offenses should be known. Offenses that require making amends should be known. Offenses that do not require making amends should be known. Offenses that are confessable should be known. Offenses that are not confessable should be known. Obstructive offenses should be known.\footnote{Sp 5.321: \textit{\textsanskrit{Antarāyikāti} sattapi \textsanskrit{āpattiyo} \textsanskrit{sañcicca} \textsanskrit{vītikkantā} \textsanskrit{saggantarāyañceva} \textsanskrit{mokkhantarāyañca} \textsanskrit{karontīti}}, “\textit{\textsanskrit{Antarāyikā}}: the intentional transgression of the seven kinds of offenses creates an obstacle to heaven and an obstacle to liberation.” } Unobstructive offenses should be known.\footnote{Sp 5.321: \textit{\textsanskrit{Ajānantena} \textsanskrit{vītikkantā} pana \textsanskrit{paṇṇattivajjāpatti} neva \textsanskrit{saggantarāyaṁ} na \textsanskrit{mokkhantarāyaṁ} \textsanskrit{karotīti} \textsanskrit{anantarāyikā}}, “\textit{\textsanskrit{Anantarāyikā}}: but the unknowing transgression of an offense that is a fault by convention creates an obstacle neither to heaven nor to liberation.” } Offenses designated as blameworthy should be known.\footnote{Sp 5.321: \textit{\textsanskrit{Sāvajjapaññattīti} \textsanskrit{lokavajjā}}, “\textit{\textsanskrit{Sāvajjapaññatti}}: faults according to the world.” } Offenses designated as blameless should be known.\footnote{Sp 5.321: \textit{\textsanskrit{Anavajjapaññattīti} \textsanskrit{paṇṇattivajjā}}, “\textit{\textsanskrit{Anavajjapaññatti}}: faults by convention.” } Offenses originating from action should be known. Offenses originating from non-action should be known. Offenses originating from both action and non-action should be known. Initial offenses should be known. Subsequent offenses should be known. Offenses committed while making amends for an initial offense should be known.\footnote{Sp 5.321: \textit{\textsanskrit{Pubbāpattīnaṁ} \textsanskrit{antarāpatti} \textsanskrit{nāma} \textsanskrit{parivāse} \textsanskrit{āpannā}}, “What is committed during the probation is called \textit{\textsanskrit{pubbāpattīnaṁ} \textsanskrit{antarāpatti}}.” } Offenses committed while making amends for a subsequent offense should be known.\footnote{Sp 5.321: \textit{“\textsanskrit{Aparāpattīnaṁ} \textsanskrit{antarāpatti} \textsanskrit{nāma} \textsanskrit{mānattacāre} \textsanskrit{āpannā}”ti \textsanskrit{vuttaṁ}}, “It is said that what is committed during the trial period is called \textit{\textsanskrit{aparāpattīnaṁ} \textsanskrit{antarāpatti}}.” } Offenses that are fit to be counted as confessed should be known.\footnote{Sp 5.321: \textit{\textsanskrit{Desitā} \textsanskrit{gaṇanūpagā} \textsanskrit{nāma} \textsanskrit{yā} \textsanskrit{dhuranikkhepaṁ} \textsanskrit{katvā} puna na \textsanskrit{āpajjissāmīti} \textsanskrit{desitā} hoti. \textsanskrit{Agaṇanūpagā} \textsanskrit{nāma} \textsanskrit{yā} \textsanskrit{dhuranikkhepaṁ} \textsanskrit{akatvā} \textsanskrit{saussāheneva} cittena aparisuddhena \textsanskrit{desitā} hoti. \textsanskrit{Ayañhi} \textsanskrit{desitāpi} \textsanskrit{desitagaṇanaṁ} na upeti. \textsanskrit{Aṭṭhame} \textsanskrit{vatthusmiṁ} \textsanskrit{bhikkhuniyā} \textsanskrit{pārājikameva} hoti}, “\textit{\textsanskrit{Desitā} \textsanskrit{gaṇanūpagā}} means cases where, having put down the burden, it is confessed, saying, ‘I will not commit again.’ \textit{\textsanskrit{Agaṇanūpagā}} means cases where, not having put down the burden, one confesses with an impure mind that is still effective. This confession does not go towards counting as confessed. This concerns only the eighth factor of the offense entailing confession of the nuns.” Vin-vn-\textsanskrit{ṭ} 2005: \textit{\textsanskrit{Dhuranikkhepanaṁ} \textsanskrit{katvāti} “na \textsanskrit{punevaṁ} \textsanskrit{karissāmī}”ti \textsanskrit{dhuraṁ} \textsanskrit{nikkhipitvā}. \textsanskrit{Desitā} \textsanskrit{gaṇanūpikāti} \textsanskrit{desitā} \textsanskrit{desitagaṇanameva} upeti, \textsanskrit{pārājikassa} \textsanskrit{aṅgaṁ} na \textsanskrit{hotīti} attho. … \textsanskrit{Saussāhāya} \textsanskrit{desitāti} puna \textsanskrit{āpajjane} \textsanskrit{anikkhittadhurāya} \textsanskrit{bhikkhuniyā} \textsanskrit{desitāpi} \textsanskrit{āpatti} \textsanskrit{desanāgaṇanaṁ} na upeti}, “\textit{\textsanskrit{Dhuranikkhepanaṁ} \textsanskrit{katvā}} means having put down the burden, saying, ‘I will not do it again.’ \textit{\textsanskrit{Desitā} \textsanskrit{gaṇanūpika}} means what has been confessed comes to be counted as confessed; the meaning is that it is not a factor of the offense entailing expulsion. \textit{\textsanskrit{Saussāhāya} \textsanskrit{desitā}} means, if a nun has not put down the burden in regard to what has been committed, then even if the offense is confessed, it does not come to be counted as a confession.” Sp-\textsanskrit{ṭ} 5.321: \textit{\textsanskrit{Saussāheneva} \textsanskrit{cittenāti} “punapi \textsanskrit{āpajjissāmī}”ti}, “\textit{\textsanskrit{Saussāheneva} cittena} means thinking, ‘I will do it again.’” } Offenses that are unfit to be counted as confessed should be known. The rule should be known. An addition to the rule down should be known. An unprompted rule should be known. Rules that apply everywhere should be known. Rules that apply in a particular place should be known. Rules that the monks and nuns have in common should be known. Rules they do not have in common should be known. Rules for one Sangha should be known. Rules for both Sanghas should be known. Heavy offenses should be known.\footnote{Sp 5.321: \textit{\textsanskrit{Thullavajjāti} thulladose \textsanskrit{paññattā} \textsanskrit{garukāpatti}}, “\textit{Thullavajja}: a serious offense laid down in regard to a serious fault.” } Light offenses should be known.\footnote{Sp 5.321: \textit{\textsanskrit{Athullavajjāti} \textsanskrit{lahukāpatti}}, “\textit{Athullavajja}: a light offense.” } Offenses connected with householders should be known. Offenses not connected with householders should be known. Offenses with fixed rebirth should be known.\footnote{Sp 5.321: \textit{\textsanskrit{Pañcānantariyakammāpatti} \textsanskrit{niyatā}, \textsanskrit{sesā} \textsanskrit{aniyatā}}, “The five offenses of actions with immediate results have a fixed rebirth, the rest do not.” } Offenses with undetermined rebirth should be known. The person who is the first offender should be known. The subsequent offenders should be known. The occasional offender should be known.\footnote{Sp 5.321: \textit{\textsanskrit{Adhiccāpattiko} \textsanskrit{nāma} yo \textsanskrit{kadāci} karahaci \textsanskrit{āpattiṁ} \textsanskrit{āpajjati}}, “Whoever commits an offense only once in a while is called \textit{\textsanskrit{adhiccāpattiko}}.” } The frequent offender should be known. The accusing person should be known. The accused person should be known. The person who accuses illegitimately should be known. The person who is accused illegitimately should be known. The person who accuses legitimately should be known. The person who is accused legitimately should be known. The person with fixed future should be known. The person with undetermined future should be known. The person incapable of an offense should be known. The person capable of an offense should be known. The ejected person should be known. The unejected person should be known. The expelled person should be known. The unexpelled person should be known. The person who belongs to the same Buddhist sect should be known. The person who belongs to a different Buddhist sect should be known. Cancellation should be known.” 

\scend{The section on ones is finished. }

\scuddanaintro{This is the summary: }

\begin{scuddana}%
“That\marginnote{4.1} produce, offense, light, \\
And curable, grave; \\
Making amends, and confession, \\
Obstructive, blameworthy, action. 

Both\marginnote{5.1} action and non-action, initial, \\
While making amends, fit to be counted; \\
Rule, unprompted, \\
Everywhere, and in common, for one Sangha. 

Heavy,\marginnote{6.1} householder, and fixed, \\
First, occasional, accusing; \\
Illegitimately, legitimately, fixed, \\
Incapable, ejected, expelled; \\
The same, and cancellation: \\
This is the summary of the ones.” 

%
\end{scuddana}

\section*{2. The section on twos }

There\marginnote{7.1} are offenses for which perception is a factor, and offenses for which it is not. — There are offenses for which the attainment has been achieved, and offenses for which it has not. —\footnote{Sp 5.322: \textit{\textsanskrit{Laddhasamāpattikassa} \textsanskrit{āpatti} \textsanskrit{nāma} \textsanskrit{bhūtārocanāpatti}, \textsanskrit{aladdhasamāpattikassa} \textsanskrit{āpatti} \textsanskrit{nāma} \textsanskrit{abhūtārocanāpatti}}, “The offense for telling truthfully is called an offense for which an attainment has been achieved. The offense for telling untruthfully is called an offense for which an attainment has not been achieved.” Sp-yoj 5.322: \textit{\textsanskrit{Bhūtārocanāpattīti} \textsanskrit{pācittiyāpatti}. \textsanskrit{Abhūtārocanāpattīti} \textsanskrit{pārājikathullaccayāpatti}}, “‘The offense for telling truthfully’ is an offense entailing confession. ‘The offense for telling untruthfully’ is an offense entailing expulsion.” } There are offenses that are connected with the true Teaching, and offenses that are not. —\footnote{Sp 5.322: \textit{\textsanskrit{Saddhammapaṭisaññuttā} \textsanskrit{nāma} \textsanskrit{padasodhammādikā}, \textsanskrit{asaddhammapaṭisaññuttā} \textsanskrit{nāma} \textsanskrit{duṭṭhullavācāpatti}}, “Memorizing the Teaching, etc., are called connected with the true Teaching. The offense of indecent speech is called not connected with the true Teaching.” } There are offenses that are connected with one’s own requisites, and offenses that are connected with someone else’s requisites. —\footnote{Sp 5.322: \textit{\textsanskrit{Saparikkhārapaṭisaññuttā} \textsanskrit{nāma} nissaggiyavatthuno \textsanskrit{anissajjitvā} paribhoge, \textsanskrit{pattacīvarānaṁ} nidahane, \textsanskrit{kiliṭṭhacīvarānaṁ} adhovane, malaggahitapattassa apacaneti \textsanskrit{evaṁ} ayuttaparibhoge \textsanskrit{āpatti}. \textsanskrit{Paraparikkhārapaṭisaññuttā} \textsanskrit{nāma} \textsanskrit{saṅghikamañcapīṭhādīnaṁ} \textsanskrit{ajjhokāse} \textsanskrit{santharaṇaanāpucchāgamanādīsu} \textsanskrit{āpajjitabbā} \textsanskrit{āpatti}}, “Connected with one’s own requisites: using an item to be relinquished without first relinquishing it, storing a bowl or robe, not washing a soiled robe, not firing a stained bowl—this is called connected with one’s own requisites. Connected with someone else’s requisites: there is an offense to be committed in leaving without informing after putting a bed or bench belonging to the Sangha out in the open, etc.” } There are offenses that are connected with oneself, and offenses that are connected with others. — There are heavy offenses committed by one speaking the truth, and light offenses committed by one speaking falsely. There are heavy offenses committed by one speaking falsely, and light offenses committed by one speaking the truth. — There are offenses committed by one on the ground, not by one above ground.\footnote{Sp 5.322: \textit{\textsanskrit{Saṅghakammaṁ} \textsanskrit{vaggaṁ} \textsanskrit{karissāmīti} \textsanskrit{antosīmāya} ekamante \textsanskrit{nisīdanto} \textsanskrit{bhūmigato} \textsanskrit{āpajjati} \textsanskrit{nāma}}, “Sitting down to one side within the monastery zone, thinking, ‘I will ensure the legal procedure is done by an incomplete assembly’—it is called ‘committed by one on the ground’.” } There are offenses committed by one above ground, not by one on the ground. —\footnote{Sp 5.322 gives the example of \href{https://suttacentral.net/pli-tv-bu-vb-pc18/en/brahmali\#1.18.1}{Bu Pc 18:1.18.1}. } There are offenses committed by one who is leaving, not by one who is entering.\footnote{Sp 5.322 gives the example of the section on “The proper conduct for departing monks” at \href{https://suttacentral.net/pli-tv-kd18/en/brahmali\#3.1.0}{Kd 18:3.1.0} in The Chapter on Proper Conduct. } There are offenses committed by one who is entering, not by one who is leaving. —\footnote{Sp 5.322 gives the example of the section on “The proper conduct for newly arrived monks” at \href{https://suttacentral.net/pli-tv-kd18/en/brahmali\#0.4}{Kd 18:0.4} in The Chapter on Proper Conduct. } There are offenses committed by applying, and offenses committed by not applying. —\footnote{Sp 5.322 gives the examples of \href{https://suttacentral.net/pli-tv-bi-vb-pc5/en/brahmali\#1.2.12.1}{Bi Pc 5:1.2.12.1} for the former and \href{https://suttacentral.net/pli-tv-bu-vb-pc58/en/brahmali\#1.15.1}{Bu Pc 58:1.15.1} for the latter. } There are offenses committed by undertaking, and offenses committed by not undertaking. —\footnote{Sp 5.322: \textit{\textsanskrit{Mūgabbatādīni} \textsanskrit{titthiyavattāni} \textsanskrit{samādiyanto} \textsanskrit{samādiyanto} \textsanskrit{āpajjati} \textsanskrit{nāma}. \textsanskrit{Pārivāsikādayo} pana \textsanskrit{tajjanīyādikammakatā} \textsanskrit{vā} attano \textsanskrit{vattaṁ} \textsanskrit{asamādiyantā} \textsanskrit{āpajjanti}, te \textsanskrit{sandhāya} \textsanskrit{vuttaṁ} “\textsanskrit{atthāpatti} na \textsanskrit{samādiyanto} \textsanskrit{āpajjatī}”ti}, “Undertaking the conduct of ascetics of other religions, such as the vow of silence, etc., is called ‘committed by undertaking’. They commit by not undertaking the conduct of one undertaking probation, etc., who has had a procedure of condemnation, etc., done against them—it is with reference to those that it is said, ‘offenses committed by not undertaking’.” } There are offenses committed by doing, and offenses committed by not doing. —\footnote{Sp 5.322: \textit{\textsanskrit{Aññātikāya} \textsanskrit{bhikkhuniyā} \textsanskrit{cīvaraṁ} sibbanto \textsanskrit{vejjakammabhaṇḍāgārikakammacittakammādīni} \textsanskrit{vā} karonto karonto \textsanskrit{āpajjati} \textsanskrit{nāma}. \textsanskrit{Upajjhāyavattādīni} akaronto akaronto \textsanskrit{āpajjati} \textsanskrit{nāma}}, “Sewing a robe for an unrelated nun or doing the work of a doctor or the work of a storekeeper or the work of a decorator, etc., is called ‘committed by doing’. Not doing the duties of a preceptor, etc., is called ‘committed by not doing’.” } There are offenses committed by giving, and offenses committed by not giving. —\footnote{Sp 5.322: \textit{\textsanskrit{Aññātikāya} \textsanskrit{bhikkhuniyā} \textsanskrit{cīvaraṁ} \textsanskrit{dadamāno} dento \textsanskrit{āpajjati} \textsanskrit{nāma}. \textsanskrit{Saddhivihārikaantevāsikānaṁ} \textsanskrit{cīvarādīni} adento adento \textsanskrit{āpajjati} \textsanskrit{nāma}}, “Giving a robe to an unrelated nun is called ‘committed by giving’. Not giving robes, etc., to students and pupils is called ‘committed by not giving’.” } There are offenses committed by teaching, and offenses committed by not teaching. — There are offenses committed by receiving, and offenses committed by not receiving. —\footnote{Sp 5.322: \textit{\textsanskrit{Aññātikāya} \textsanskrit{bhikkhuniyā} \textsanskrit{cīvaraṁ} \textsanskrit{gaṇhanto} \textsanskrit{paṭiggaṇhanto} \textsanskrit{āpajjati} \textsanskrit{nāma}. “Na bhikkhave \textsanskrit{ovādo} na gahetabbo”ti vacanato \textsanskrit{ovādaṁ} \textsanskrit{agaṇhanto} na \textsanskrit{paṭiggaṇhanto} \textsanskrit{āpajjati} \textsanskrit{nāma}}, “Taking a robe from an unrelated nun is called ‘committed by receiving’. Because of the ruling, ‘Monks, you should agree to give the instruction’, then, not agreeing to give the instruction is called ‘committed by not receiving’.” This may seem a bit confusing in English. The idea is that the monk “accepts” the job of giving the instruction. The acceptance, \textit{\textsanskrit{gaṇhāti}}, is here expressed with the verb \textit{\textsanskrit{paṭigaṇhāti}}, which normally means “to receive”. } There are offenses committed by using, and offenses committed by not using. —\footnote{Sp 5.322: \textit{\textsanskrit{Nissaggiyavatthuṁ} \textsanskrit{anissajjitvā} \textsanskrit{paribhuñjanto} paribhogena \textsanskrit{āpajjati} \textsanskrit{nāma}. \textsanskrit{Pañcāhikaṁ} \textsanskrit{saṅghāṭicāraṁ} \textsanskrit{atikkāmayamānā} aparibhogena \textsanskrit{āpajjati} \textsanskrit{nāma}}, “Using an item to be relinquished without first relinquishing it is called ‘committed by using’. Not moving the robes for more than five days is called ‘committed by not using’.” } There are offenses committed at night, not by day. There are offenses committed by day, not at night. — There are offenses committed at dawn, and offenses committed not at dawn. — There are offenses committed by cutting, and offenses committed by not cutting. —\footnote{Sp 5.322: \textit{\textsanskrit{Bhūtagāmañceva} \textsanskrit{aṅgajātañca} chindanto chindanto \textsanskrit{āpajjati} \textsanskrit{nāma}, kese \textsanskrit{vā} nakhe \textsanskrit{vā} na chindanto na chindanto \textsanskrit{āpajjati} \textsanskrit{nāma}}, “Cutting plants or the penis is called ‘committed by cutting’. Not cutting the hair or the nails is called ‘committed by not cutting’.” } There are offenses committed by covering, and offenses committed by not covering. —\footnote{Sp 5.322: \textit{\textsanskrit{Āpattiṁ} \textsanskrit{chādento} \textsanskrit{chādento} \textsanskrit{āpajjati} \textsanskrit{nāma}, “\textsanskrit{tiṇena} \textsanskrit{vā} \textsanskrit{paṇṇena} \textsanskrit{vā} \textsanskrit{paṭicchādetvā} \textsanskrit{āgantabbaṁ}, natveva naggena \textsanskrit{āgantabbaṁ}, yo \textsanskrit{āgaccheyya} \textsanskrit{āpatti} \textsanskrit{dukkaṭassā}”ti \textsanskrit{imaṁ} pana \textsanskrit{āpattiṁ} na \textsanskrit{chādento} \textsanskrit{āpajjati} \textsanskrit{nāma}}, “Covering over an offense is called ‘committed by covering’. ‘He should cover up with grass and leaves before going on; he should not go on while naked’—this offense is called ‘committed by not covering’.” } There are offenses committed by wearing, and offenses committed by not wearing.\footnote{Sp 5.322: \textit{\textsanskrit{Kusacīrādīni} \textsanskrit{dhārento} \textsanskrit{dhārento} \textsanskrit{āpajjati} \textsanskrit{nāma}, “\textsanskrit{ayaṁ} te bhikkhu patto \textsanskrit{yāva} \textsanskrit{bhedanāya} \textsanskrit{dhāretabbo}”ti \textsanskrit{imaṁ} \textsanskrit{āpattiṁ} na \textsanskrit{dhārento} \textsanskrit{āpajjati} \textsanskrit{nāma}}, “Wearing grass robes, etc., is called ‘committed by wearing’. ‘Monk, this bowl is yours; keep it until it breaks’—this offense is called ‘committed by not wearing/keeping’.” } 

There\marginnote{8.1} are two observance days: the fourteenth and the fifteenth day of the lunar half-month. — There are two invitation days: the fourteenth and the fifteenth day of the lunar half-month. — There are two kinds of legal procedures: the procedure consisting of getting permission, and the procedure consisting of one motion. — There are two other kinds of legal procedures: the procedure consisting of one motion and one announcement, and the procedure consisting of one motion and three announcements. — There are two kinds of cases of legal procedures: the cases of procedures consisting of getting permission, and the cases of procedures consisting of one motion. — There are two other kinds of cases of legal procedures: the cases of procedures consisting of one motion and one announcement, and the cases of procedures consisting of one motion and three announcements. — There are two kinds of flaws in legal procedures: the flaw in a procedure consisting of getting permission, and the flaw in a procedure consisting of one motion. — There are two other kinds of flaws in legal procedures: the flaw in a procedure consisting of one motion and one announcement, and the flaw in a procedure consisting of one motion and three announcements. — There are two kinds of successes in legal procedures: the success of a procedure consisting of getting permission, and the success of a procedure consisting of one motion. — There are two other kinds of successes in legal procedures: the success of a procedure consisting of one motion and one announcement, and the success of a procedure consisting of one motion and three announcements. 

There\marginnote{8.21} are two grounds for belonging to a different Buddhist sect: either one makes oneself belong to a different Buddhist sect, or a unanimous assembly ejects one for not recognizing an offense, for not making amends for an offense, or for not giving up a bad view. — There are two grounds for belonging to the same Buddhist sect: either one makes oneself belong to the same Buddhist sect, or a unanimous assembly readmits one who has been ejected for recognizing an offense, for making amends for an offense, or for giving up a bad view. — There are two kinds of offenses entailing expulsion: for monks and for nuns. — There are two kinds of offenses entailing suspension, two kinds of serious offenses, two kinds of offenses entailing confession, two kinds of offenses entailing acknowledgment, two kinds of offenses of wrong conduct, two kinds of offenses of wrong speech: for monks and for nuns. — There are seven kinds of offenses and seven classes of offenses. — Schism in the Sangha occurs in two ways: through a legal procedure or through a vote. 

Two\marginnote{9.1} kinds of people should not be given the full ordination: one lacking in age, and one lacking in limbs. —\footnote{Sp 5.322: \textit{\textsanskrit{Addhānahīno} \textsanskrit{nāma} \textsanskrit{ūnavīsativasso}. \textsanskrit{Aṅgahīno} \textsanskrit{nāma} \textsanskrit{hatthacchinnādibhedo}}, “One less than twenty years old is called \textit{\textsanskrit{addhānahīno}}. The category of one without a hand etc. is called \textit{\textsanskrit{aṅgahīno}}.” } Another two kinds of people should not be given the full ordination: one who is deficient as object, and one who has acted wrongly. —\footnote{Sp 5.322: \textit{Vatthuvipanno \textsanskrit{nāma} \textsanskrit{paṇḍako} \textsanskrit{tiracchānagato} \textsanskrit{ubhatobyañjanako} ca. \textsanskrit{Avasesā} \textsanskrit{theyyasaṁvāsakādayo} \textsanskrit{aṭṭha} \textsanskrit{abhabbapuggalā} \textsanskrit{karaṇadukkaṭakā} \textsanskrit{nāma}}, “A \textit{\textsanskrit{paṇḍaka}}, an animal, and a hermaphrodite are called ‘deficient as object’. The remaining eight incapable people, starting with the fake monk, are called ‘one who has acted wrongly’.” } Another two kinds of people should not be given the full ordination: one who is incomplete, and one who is complete but who has not asked for it. —\footnote{Sp 5.322: \textit{\textsanskrit{Aparipūro} \textsanskrit{nāma} \textsanskrit{aparipuṇṇapattacīvaro}}, “\textit{\textsanskrit{Aparipūro}}: not complete in bowl and robes.” } One should not live with formal support from two kinds of persons:\footnote{For an explanation of the rendering “formal support” for \textit{\textsanskrit{nissāya}}, see Appendix of Technical Terms. } one who is shameless, and one who is ignorant. — One should not give formal support to two kinds of people: one who is shameless, and one who has a sense of conscience but who has not asked for it. — One should give formal support to two kinds of people: one who is ignorant, and one who has a sense of conscience and who has asked for it. — Two kinds of people are incapable of committing an offense: Buddhas and solitary Buddhas. — Two kinds of people are capable of committing an offense: monks and nuns. — Two kinds of people are incapable of intentionally committing an offense: monks and nuns who are noble persons. — Two kinds of people are capable of intentionally committing an offense: monks and nuns who are ordinary persons. — Two kinds of people are incapable of intentionally committing an action that goes too far: monks and nuns who are noble persons. — Two kinds of people are capable of intentionally committing an action that goes too far: monks and nuns who are ordinary persons. 

There\marginnote{10.1} are two kinds of objections: one objects by body, or one objects by speech. — There are two kinds of sending away: if the Sangha sends away those who don’t have the attributes needed for being sent away, the sending away of some succeeds, while the sending away of others fails. — There are two kinds of admittance: if the Sangha admits those who don’t have the attributes needed for being admitted, the admittance of some succeeds, while the admittance of others fails. — There are two kinds of admitting: one admits by body, or one admits by speech. — There are two kinds of receiving: one receives by body, or one receives by what is connected to the body. — There are two kinds of prohibitions: one prohibits by body, or one prohibits by speech. — There are two kinds of harming: harming of the training, and harming of possessions. — There are two kinds of accusing: one accuses by body, or one accuses by speech. 

There\marginnote{10.17} are two obstacles for the ending of the robe season: the monastery obstacle, and the robe obstacle. — There are two removals of obstacles for the ending of the robe season: the removal of the monastery obstacle, and the removal of the robe obstacle. — There are two kinds of robes: from householders, and from rags. — There are two kinds of almsbowls: iron bowls, and ceramic bowls. — There are two kinds of bowl rests: bowl-rests made of tin, and bowl-rests made of lead. — There are two ways of determining an almsbowl: one determines it by body, or one determines it by speech. — There are two ways of determining a robe: one determines it by body, or one determines it by speech. — There are two kinds of assignment: assignment in the presence of, and assignment in the absence of.\footnote{For an explanation of the idea of \textit{\textsanskrit{vikappanā}}, see Appendix of Technical Terms. } 

There\marginnote{10.33} are two Monastic Laws: for the monks, and for the nuns. — There are two things that belong to the Monastic Law: the rules, and what accords with the rules. — There are two kinds of self-effacement through the Monastic Law: ending access to what is unallowable, and moderation in what is allowable. — One commits an offense in two ways: one commits it by body, or one commits it by speech. — One is cleared of an offense in two ways: one is cleared by body, or one is cleared by speech. — There are two kinds of probation: probation for concealed offenses, and probation for unconcealed offenses. — There are two other kinds of probation: purifying probation, and simultaneous probation. — There are two kinds of trial periods: trial periods for concealed offenses, and trial periods for unconcealed offenses. — There are two other kinds of trial periods: trial periods for a half-month, and simultaneous trial periods. — Not counting a day is for two kinds of people: for one on probation, and for one undertaking the trial period. — There are two kinds of disrespect: disrespect for the person, and disrespect for the rule. 

There\marginnote{10.55} are two kinds of salt: natural, and artificial. — There are two other kinds of salt: sea salt, and black salt. — There are two other kinds of salt: hill salt, and soil salt. — There are two other kinds of salt: salt from the Roma country, and grain salt. —\footnote{Sp-\textsanskrit{ṭ} 5.322: \textit{Romajanapade \textsanskrit{jātaṁ} \textsanskrit{romakaṁ}. \textsanskrit{Pakkālakanti} \textsanskrit{yavakkhāraṁ}}, “\textit{Romaka} means grown in the Roma country. \textit{\textsanskrit{Pakkālaka}} means salt from grain.” } There are two kinds of using: using internally, and using externally. —\footnote{Sp 5.322: \textit{Abbhantaraparibhogo \textsanskrit{nāma} \textsanskrit{ajjhoharaṇaparibhogo}. \textsanskrit{Bāhiraparibhogo} \textsanskrit{nāma} \textsanskrit{sīsamakkhanādi}}, “\textit{Abbhantaraparibhogo}: using for consumption. \textit{\textsanskrit{Bāhiraparibhogo}}: what can be smeared on the head, etc.” } There are two kinds of name-calling: low name-calling, and high name-calling. — There is malicious talebearing in two ways: for one wanting to endear himself, or for one aiming at division. — Eating in a group comes about in two ways: through an invitation, or through asking. — There are two entries to the rainy-season residence: the first and the second. — There are two kinds of illegitimate cancellations of the Monastic Code. — There are two kinds of legitimate cancellations of the Monastic Code. 

There\marginnote{11.1} are two kinds of fools: one who takes on future responsibilities, and one who does not take on current responsibilities. — There are two kinds of wise persons: one who does not take on future responsibilities, and one who takes on current responsibilities. — There are two other kinds of fools: one who perceives what is unallowable as allowable, and one who perceives what is allowable as unallowable. — And there are two kinds of wise persons: one who perceives what is unallowable as unallowable, and one who perceives what is allowable as allowable. — There are two other kinds of fools: one who perceives a non-offense as an offense, and one who perceives an offense as a non-offense. — And there are two kinds of wise persons: one who perceives an offense as an offense, and one who perceives a non-offense as a non-offense. — There are two other kinds of fools: one who perceives what is contrary to the Teaching as the Teaching, and one who perceives what is the Teaching as contrary to the Teaching. — And there are two kinds of wise persons: one who perceives what is contrary to the Teaching as contrary to the Teaching, and one who perceives what is the Teaching as the Teaching. — There are two other kinds of fools: one who perceives what is contrary to the Monastic Law as the Monastic Law, and one who perceives what is the Monastic Law as contrary to the Monastic Law. — And there are two kinds of wise persons: one who perceives what is contrary to the Monastic Law as contrary to the Monastic Law, and one who perceives what is the Monastic Law as the Monastic Law. 

The\marginnote{12.1} corruptions increase for two kinds of persons: one who is afraid of wrongdoing when one should not be, and one who is not afraid of wrongdoing when one should be. — The corruptions do not increase for two kinds of persons: one who is not afraid of wrongdoing when one should not be, and one who is afraid of wrongdoing when one should be. — The corruptions increase for two other kinds of persons: one who perceives what is unallowable as allowable, and one who perceives what is allowable as unallowable. — The corruptions do not increase for two other kinds of persons: one who perceives what is unallowable as unallowable, and one who perceives what is allowable as allowable. — The corruptions increase for two other kinds of persons: one who perceives a non-offense as an offense, and one who perceives an offense as a non-offense. — The corruptions do not increase for two other kinds of persons: one who perceives a non-offense as a non-offense, and one who perceives an offense as an offense. — The corruptions increase for two other kinds of persons: one who perceives what is contrary to the Teaching as the Teaching, and one who perceives what is the Teaching as contrary to the Teaching. — The corruptions do not increase for two other kinds of persons: one who perceives what is contrary to the Teaching as contrary to the Teaching, and one who perceives what is the Teaching as the Teaching. — The corruptions increase for two other kinds of persons: one who perceives what is contrary to the Monastic Law as the Monastic Law, and one who perceives what is the Monastic Law as contrary to the Monastic Law. — The corruptions do not increase for two other kinds of persons: one who perceives what is contrary to the Monastic Law as contrary to the Monastic Law, and one who perceives what is the Monastic Law as the Monastic Law. 

\scend{The section on twos is finished. }

\scuddanaintro{This is the summary: }

\begin{scuddana}%
“Perception,\marginnote{15.1} and achieved, the true Teaching, \\
And requisites, others; \\
Truth, ground, leaving, \\
Applying, undertaking. 

Doing,\marginnote{16.1} giving, receiving, \\
By using, and night; \\
Dawn, cutting, covering, \\
And wearing, observance days. 

Invitation\marginnote{17.1} days, legal procedures, other, \\
Object, other, and flaws; \\
Other, and two successes, \\
Different, and the same. 

Expulsion,\marginnote{18.1} suspension, serious offense, \\
Confession, acknowledgment; \\
Wrong conduct, and wrong speech, \\
Seven, and classes of offenses. 

Schism,\marginnote{19.1} full ordination, \\
And two other; \\
Should not live, should not give, \\
Incapable, and capable. 

Intentionally,\marginnote{20.1} and goes too far, \\
Objections, sending away; \\
Admittance, and admitting, \\
Receiving, prohibitions. 

Harming,\marginnote{21.1} and accusing, \\
And so two on the robe season; \\
Robes, bowls, bowl rests, \\
And so two on determining. 

And\marginnote{22.1} assignment, Monastic Laws, \\
And belong to the Monastic Law, self-effacement; \\
And one commits, one is cleared, \\
Probation, two others. 

Two\marginnote{23.1} trial periods, others, \\
Not counting a day, disrespect; \\
Two salts, three others, \\
Using, and with name-calling. 

And\marginnote{24.1} malicious talebearing, group, rainy-season residence, \\
Cancellations, responsibilities, allowable; \\
Non-offense, contrary to the Teaching, the Teaching, \\
The Monastic Law, and so corruptions.” 

%
\end{scuddana}

\section*{3. The section on threes }

“(1)\marginnote{25.1} There are offenses one commits while the Buddha is alive, not after his extinguishment.\footnote{Sp 5.323: \textit{Tattha \textsanskrit{lohituppādāpattiṁ} \textsanskrit{tiṭṭhante} \textsanskrit{āpajjati}}, “Therein, the offense of causing to bleed is committed while he is alive.” } (2) There are offenses one commits after the Buddha’s extinguishment, not while he is alive.\footnote{Sp 5.323: \textit{\textsanskrit{Theraṁ} \textsanskrit{āvusovādena} \textsanskrit{samudācaraṇapaccayā} \textsanskrit{āpattiṁ} parinibbute bhagavati \textsanskrit{āpajjati}, no \textsanskrit{tiṭṭhante}}, “The offense due to addressing a senior monastic as ‘friend’ is committed after the extinguishment of the Buddha, not while he is alive.” } (3) There are offenses one commits both while the Buddha is alive and also after his extinguishment.\footnote{Sp 5.323: \textit{\textsanskrit{Imā} dve \textsanskrit{āpattiyo} \textsanskrit{ṭhapetvā} \textsanskrit{avasesā} dharantepi bhagavati \textsanskrit{āpajjati}, parinibbutepi}, “Apart from these two offenses, the rest are committed both while the Buddha is alive and after the extinguishment.” } 

(1)\marginnote{25.4} There are offenses one commits at the right time, not at the wrong time. (2) There are offenses one commits at the wrong time, not at the right time. (3) There are offenses one commits both at the right time and also at the wrong time. 

(1)\marginnote{25.7} There are offenses one commits at night, not by day. (2) There are offenses one commits by day, not at night. (3) There are offenses one commits both at night and also by day. 

(1)\marginnote{25.10} There are offenses one commits when one has ten years of seniority, not less.\footnote{Sp 5.323: \textit{“Dasavassomhi \textsanskrit{atirekadasavassomhī}”ti \textsanskrit{bālo} abyatto \textsanskrit{parisaṁ} \textsanskrit{upaṭṭhāpento} dasavasso \textsanskrit{āpajjati} no \textsanskrit{ūnadasavasso}}, “An ignorant and incompetent person who has ten years of seniority commits an offense when creating an entourage, thinking, ‘I have ten years or more of seniority,’ but not one who has less than ten years of seniority.” } (2) There are offenses one commits when one has less than ten years of seniority, not ten.\footnote{Sp 5.323: \textit{“\textsanskrit{Ahaṁ} \textsanskrit{paṇḍito} byatto”ti navo \textsanskrit{vā} majjhimo \textsanskrit{vā} \textsanskrit{parisaṁ} \textsanskrit{upaṭṭhāpento} \textsanskrit{ūnadasavasso} \textsanskrit{āpajjati} no dasavasso ca}, “A junior monk or one of middle standing who has less than ten years of seniority commits an offense when creating an entourage, thinking, ‘I am wise and competent,’ but not one who has ten years of seniority.” } (3) There are offenses one commits both when one has ten years of seniority and also when one has less.\footnote{Sp 5.323: \textit{\textsanskrit{Sesā} dasavasso ceva \textsanskrit{āpajjati} \textsanskrit{ūnadasavasso} ca}, “The rest are committed both by those who have ten years of seniority and by those who have less than ten years of seniority.” } 

(1)\marginnote{25.13} There are offenses one commits when one has five years of seniority, not less.\footnote{Sp 5.323: \textit{“\textsanskrit{Pañcavassomhī}”ti \textsanskrit{bālo} abyatto \textsanskrit{anissāya} vasanto \textsanskrit{pañcavasso} \textsanskrit{āpajjati}}, “An ignorant and incompetent person who has five years of seniority commits an offense when living without formal support, thinking, ‘I have five years of seniority.’” } (2) There are offenses one commits when one has less than five years of seniority, not five.\footnote{Sp 5.323: \textit{“\textsanskrit{Ahaṁ} \textsanskrit{paṇḍito} byatto”ti navako \textsanskrit{anissāya} vasanto \textsanskrit{ūnapañcavasso} \textsanskrit{āpajjati}}, “A junior monk who has less than five years of seniority commits an offense when living without formal support, thinking, ‘I am wise and competent.’” } (3) There are offenses one commits both when one has five years of seniority and also when one has less.\footnote{Sp 5.323: \textit{\textsanskrit{Sesaṁ} \textsanskrit{pañcavasso} ceva \textsanskrit{āpajjati} \textsanskrit{ūnapañcavasso} ca}, “The rest are committed both by those who have five years of seniority and by those who have less than five years of seniority.” } 

(1)\marginnote{25.16} There are offenses one commits with a wholesome mind. (2) There are offenses one commits with an unwholesome mind. (3) There are offenses one commits with an indeterminate mind. 

(1)\marginnote{25.19} There are offenses one commits while experiencing pleasant feelings. (2) There are offenses one commits while experiencing unpleasant feelings. (3) There are offenses one commits while experiencing neither pleasant nor unpleasant feelings. 

There\marginnote{25.22} are three grounds for an accusation: what is seen, what is heard, and what is suspected. — There are three ways of voting: a secret ballot, an open vote, and whispering in the ear. — Three things are opposed: great desires, discontent, and self-inflation. — Three things are allowed: fewness of wishes, contentment, and self-effacement. — Three other things are opposed: great desires, discontent, and lacking a sense of moderation. — And three things are allowed: fewness of wishes, contentment, and having a sense of moderation. — There are three kinds of rules: a rule, an addition to a rule, and an unprompted rule. — There are three other kinds of rules: a rule that applies everywhere, a rule that applies in a particular place, and a rule that the monks and nuns have in common. — There are three other kinds of rules: a rule the monks and nuns do not have in common, a rule for one Sangha, and a rule for both Sanghas. 

(1)\marginnote{26.1} There are offenses committed by fools, not by the wise. (2) There are offenses committed by the wise, not by fools. (3) There are offenses committed by both by fools and the wise. 

(1)\marginnote{26.4} There are offenses committed during the waning phase of the moon, not during the waxing phase.\footnote{Sp 5.323: \textit{\textsanskrit{Vassaṁ} anupagacchanto \textsanskrit{kāḷe} \textsanskrit{āpajjati} no \textsanskrit{juṇhe}}, “When not entering the rainy-season residence, one commits an offense during the waning phase of the moon, not during the waxing phase.” } (2) There are offenses committed during the waxing phase of the moon, not during the waning phase.\footnote{Sp 5.323: \textit{\textsanskrit{Mahāpavāraṇāya} \textsanskrit{appavārento} \textsanskrit{juṇhe} \textsanskrit{āpajjati} no \textsanskrit{kāḷe}}, “When not inviting during the great invitation ceremony, one commits an offense during the waxing phase of the moon, not during the waning phase.” } (3) There are offenses committed both during the waning and the waxing phases of the moon.\footnote{Sp 5.323: \textit{\textsanskrit{Avasesaṁ} \textsanskrit{kāḷe} ceva \textsanskrit{āpajjati} \textsanskrit{juṇhe} ca}, “The rest one commits both during the waning and the waxing phases of the moon.” } 

(1)\marginnote{26.7} There are things that are allowable during the waning phase of the moon, not during the waxing phase.\footnote{Sp 5.323: \textit{\textsanskrit{Vassūpagamanaṁ} \textsanskrit{kāḷe} kappati no \textsanskrit{juṇhe}}, “Entering the rainy-season residence is allowable during the waning phase of the moon, not during the waxing phase.” } (2) There are things that are allowable during the waxing phase of the moon, not during the waning phase.\footnote{Sp 5.323: \textit{\textsanskrit{Mahāpavāraṇāya} \textsanskrit{pavāraṇā} \textsanskrit{juṇhe} kappati no \textsanskrit{kāḷe}}, “Inviting during the great invitation ceremony is allowable during the waxing phase of the moon, not during the waning phase.” } (3) There are things that are allowable both during the waning and the waxing phases of the moon.\footnote{Sp 5.323: \textit{\textsanskrit{Sesaṁ} \textsanskrit{anuññātakaṁ} \textsanskrit{kāḷe} ceva kappati \textsanskrit{juṇhe} ca}, “The rest of what is allowable is allowable both during the waning and the waxing phases of the moon.” } 

(1)\marginnote{26.10} There are offenses committed during winter, not during summer or the rainy season.\footnote{Sp 5.323: \textit{\textsanskrit{Kattikapuṇṇamāsiyā} pacchime \textsanskrit{pāṭipadadivase} \textsanskrit{vikappetvā} \textsanskrit{ṭhapitaṁ} \textsanskrit{vassikasāṭikaṁ} \textsanskrit{nivāsento} hemante \textsanskrit{āpajjati}}, “Apart from assigning it to another on the day after the observance day of the last \textit{Kattika} month, if one dresses in the rainy-season bathing cloth, then one commits an offense during winter.” } (2) There are offenses committed during summer, not during winter or the rainy season.\footnote{Sp 5.323: \textit{\textsanskrit{Atirekamāse} sese \textsanskrit{gimhāne} pariyesanto \textsanskrit{atirekaḍḍhamāse} sese \textsanskrit{katvā} \textsanskrit{nivāsento} ca gimhe \textsanskrit{āpajjati} \textsanskrit{nāma}}, “If one searches when there is more than a month left of summer, or if, after making it, one dresses (in the rainy-season bathing cloth) when there is more than half a month left, then it is called committed during summer.” } (3) There are offenses committed during the rainy season, not during winter or summer.\footnote{Sp 5.323: \textit{\textsanskrit{Satiyā} \textsanskrit{vassikasāṭikāya} naggo \textsanskrit{kāyaṁ} \textsanskrit{ovassāpento} vasse \textsanskrit{āpajjati} \textsanskrit{nāma}}, “If one has a rainy-season bathing cloth, yet still bathes naked in the rain, then it is called committed during the rainy season.” } 

(1)\marginnote{26.13} There are offenses committed by a sangha, not by several monastics or an individual.\footnote{Sp 5.323: \textit{\textsanskrit{Pārisuddhiuposathaṁ} \textsanskrit{vā} \textsanskrit{adhiṭṭhānuposathaṁ} \textsanskrit{vā} karonto \textsanskrit{saṅgho} \textsanskrit{āpajjati}}, “In doing the observance-day ceremony by declaring one’s purity or by making a determination, a Sangha commits an offense.” } (2) There are offenses committed by several monastics, not by a sangha or an individual.\footnote{Sp 5.323: \textit{\textsanskrit{Suttuddesañca} \textsanskrit{adhiṭṭhānuposathañca} karonto \textsanskrit{gaṇo} \textsanskrit{āpajjati}}, “In reciting the Monastic Code or doing the observance-day ceremony by making a determination, several monastics commit an offense.” } (3) There are offenses committed by an individual, not by a sangha or several monastics.\footnote{Sp 5.323: \textit{Ekako \textsanskrit{suttuddesaṁ} \textsanskrit{pārisuddhiuposathañca} karonto puggalo \textsanskrit{āpajjati}}, “If, on one’s own, one either recites the Monastic Code or does the observance-day ceremony by declaring one’s purity, then the individual commits an offense.” } 

(1)\marginnote{26.16} There are things allowable for a sangha, not for several monastics or an individual.\footnote{Sp 5.323: \textit{\textsanskrit{Saṅghuposatho} ca \textsanskrit{saṅghapavāraṇā} ca \textsanskrit{saṅghasseva} kappati}, “The observance-day ceremony and the invitation ceremony for a sangha are allowable for a sangha.” } (2) There are things allowable for several monastics, not for a sangha or an individual.\footnote{Sp 5.323: \textit{\textsanskrit{Gaṇuposatho} ca \textsanskrit{gaṇapavāraṇā} ca \textsanskrit{gaṇasseva} kappati}, “The observance-day ceremony and the invitation ceremony for several monastics are allowable for several monastics.” } (3) There are things allowable for an individual, not for a sangha or several monastics.\footnote{Sp 5.323: \textit{\textsanskrit{Adhiṭṭhānuposatho} ca \textsanskrit{adhiṭṭhānapavāraṇā} ca puggalasseva kappati}, “The observance-day ceremony and the invitation ceremony done by making a determination are allowable for an individual.” } 

There\marginnote{26.19} are three kinds of concealing: (1) one conceals the action that was the basis for the offense, not the offense. (2) one conceals the offense, not the action that was the basis for the offense. (3) one conceals both the action that was the basis for the offense and also the offense. 

There\marginnote{26.23} are three coverings: (1) a sauna, (2) water, and (3) a cloth.\footnote{For an explanation of the rendering “sauna” for \textit{\textsanskrit{jantāghara}}, see Appendix of Technical Terms. } 

Three\marginnote{26.25} things happen concealed, not openly: (1) Women are married with a veil, not unveiled.\footnote{This triplet is a parallel to \href{https://suttacentral.net/an3.131/en/brahmali\#1.1}{AN 3.131:1.1}, with the difference that the verb there reads \textit{\textsanskrit{āvahati}} rather than \textit{vahati}, as here. I follow the reading in the Sutta, which makes better sense. } (2) The mantras of the brahmins are transmitted in secret, not openly. (3) Wrong view is transmitted in secret, not openly. 

Three\marginnote{26.29} things shine in the open, not when concealed: (1) The disc of the moon shines in the open, not when concealed. (2) The disc of the sun shines in the open, not when concealed. (3) The spiritual path proclaimed by the Buddha shines in the open, not when concealed.\footnote{For an explanation of the rendering “spiritual path” for \textit{dhammavinaya}, see Appendix of Technical Terms. } 

There\marginnote{26.33} are three times for the allocation of dwellings: (1) the first, (2) the second, and (3) when given up in between. 

(1)\marginnote{26.35} There are offenses that one commits when sick, not when not sick.\footnote{Sp 5.323: \textit{\textsanskrit{Aññena} bhesajjena \textsanskrit{karaṇīyena} \textsanskrit{aññaṁ} \textsanskrit{viññāpento} \textsanskrit{gilāno} \textsanskrit{āpajjati}}, “A sick person commits an offense when asking for a medicine different from the one they need.” } (2) There are offenses that one commits when not sick, not when sick.\footnote{Sp 5.323: \textit{Na bhesajjena \textsanskrit{karaṇīyena} \textsanskrit{bhesajjaṁ} \textsanskrit{viññāpento} \textsanskrit{agilāno} \textsanskrit{āpajjati}}, “A person who is not sick commits an offense when asking for a medicine they do not need.” } (3) There are offenses that one commits both when sick and when not sick.\footnote{Sp 5.323: \textit{\textsanskrit{Avasesaṁ} \textsanskrit{āpattiṁ} \textsanskrit{gilāno} ceva \textsanskrit{āpajjati} \textsanskrit{agilāno} ca}, “The rest are committed both by one who is sick and by one who is not.” } 

There\marginnote{27.1} are three kinds of illegitimate cancellations of the Monastic Code. — There are three kinds of legitimate cancellations of the Monastic Code. — There are three kinds of probation: probation for concealed offenses, probation for unconcealed offenses, and purifying probation. — There are three kinds of trial period: trial period for concealed offenses, trial period for unconcealed offenses, and trial period for a half-month. — There are three things that stop a monk on probation from counting a particular day toward his probationary period: he stays in the same room as a regular monk; he stays apart from other monks; he doesn’t inform other monks of his status. 

(1)\marginnote{27.9} There are offenses that one commits inside, not outside.\footnote{According to Sp 5.323 this refers to \href{https://suttacentral.net/pli-tv-bu-vb-pc16/en/brahmali\#1.16.1}{Bu Pc 16:1.16.1}. } (2) There are offenses that one commits outside, not inside.\footnote{According to Sp 5.323 this refers to \href{https://suttacentral.net/pli-tv-bu-vb-pc14/en/brahmali\#1.1.9.1}{Bu Pc 14:1.1.9.1}. } (3) There are offenses that one commits both inside and outside.\footnote{According to Sp 5.323 this refers to the rest of the rules. } 

(1)\marginnote{27.12} There are offenses that one commits inside the monastery zone, not outside.\footnote{Sp 5.323: \textit{\textsanskrit{Antosīmāyāti} \textsanskrit{āgantuko} \textsanskrit{āgantukavattaṁ} \textsanskrit{adassetvā} \textsanskrit{sachattupāhano} \textsanskrit{vihāraṁ} pavisanto \textsanskrit{upacārasīmaṁ} okkantamattova \textsanskrit{āpajjati}}, “\textit{\textsanskrit{Antosīmāya}}: if a newly arrived monk who does not display the duties of newly arrived monks enters a monastery, or even its vicinity, while holding a sunshade and wearing sandals, then he commits an offense.” For an explanation of the rendering “monastery zone” for \textit{\textsanskrit{sīmā}}, see Appendix of Technical Terms. } (2) There are offenses that one commits outside the monastery zone, not inside.\footnote{Sp 5.323: \textit{\textsanskrit{Bahisīmāyāti} gamiko \textsanskrit{dārubhaṇḍapaṭisāmanādigamikavattaṁ} \textsanskrit{apūretvā} pakkamanto \textsanskrit{upacārasīmaṁ} atikkantamattova \textsanskrit{āpajjati}}, “\textit{\textsanskrit{Bahisīmāya}}: if a departing monk leaves, or even just goes beyond the vicinity (of the monastery zone), without fulfilling the duties of a departing monk, that is, setting the wooden goods in order, etc., then he commits an offense.” } (3) There are offenses that one commits both inside and outside the monastery zone.\footnote{Sp 5.323: \textit{\textsanskrit{Avasesaṁ} \textsanskrit{antosīmāya} ceva \textsanskrit{āpajjati} \textsanskrit{bahisīmāya} ca}, “The rest are committed both inside and outside the monastery zone.” } 

One\marginnote{27.15} commits an offense in three ways: one commits an offense by body, by speech, or by body and speech. — One commits an offense in three other ways: in the midst of the Sangha, in the midst of a group, or in the presence of an individual. — One clears an offense in three ways: one clears an offense by body, by speech, or by body and speech. — One clears an offense in three other ways: in the midst of the Sangha, in the midst of a group, or in the presence of an individual. — There are three illegitimate grantings of resolution because of past insanity. — There are three legitimate grantings of resolution because of past insanity. 

When\marginnote{28.1} a monk has three qualities, the Sangha may, if it wishes, do a procedure of condemnation against him: (1) he is quarrelsome, argumentative, and a creator of legal issues in the Sangha; (2) he is ignorant and incompetent, often committing offenses, and lacking in boundaries; (3) he is constantly and improperly socializing with householders. 

When\marginnote{28.3} a monk has three qualities, the Sangha may, if it wishes, do a procedure of demotion against him:\footnote{For an explanation of the rendering “demotion” for \textit{niyassa}, see Appendix of Technical Terms. } (1) he is quarrelsome, argumentative, and a creator of legal issues in the Sangha; (2) he is ignorant and incompetent, often committing offenses, and lacking in boundaries; (3) he is constantly and improperly socializing with householders. 

When\marginnote{28.5} a monk has three qualities, the Sangha may, if it wishes, do a procedure of banishing him: (1) he is quarrelsome, argumentative, and a creator of legal issues in the Sangha; (2) he is ignorant and incompetent, often committing offenses, and lacking in boundaries; (3) he is a corrupter of families and badly behaved, and his bad behavior has been seen and heard about. 

When\marginnote{28.7} a monk has three qualities, the Sangha may, if it wishes, do a procedure of reconciliation against him: (1) he is quarrelsome, argumentative, and a creator of legal issues in the Sangha; (2) he is ignorant and incompetent, often committing offenses, and lacking in boundaries; (3) he abuses and reviles householders. 

When\marginnote{28.9} a monk has three qualities, the Sangha may, if it wishes, do a procedure of ejecting him for not recognizing an offense: (1) he is quarrelsome, argumentative, and a creator of legal issues in the Sangha; (2) he is ignorant and incompetent, often committing offenses, and lacking in boundaries; (3) after committing an offense, he refuses to recognize it. 

When\marginnote{28.11} a monk has three qualities, the Sangha may, if it wishes, do a procedure of ejecting him for not making amends for an offense: (1) he is quarrelsome, argumentative, and a creator of legal issues in the Sangha; (2) he is ignorant and incompetent, often committing offenses, and lacking in boundaries; (3) after committing an offense, he refuses to make amends for it. 

When\marginnote{28.13} a monk has three qualities, the Sangha may, if it wishes, do a procedure of ejecting him for not giving up a bad view: (1) he is quarrelsome, argumentative, and a creator of legal issues in the Sangha; (2) he is ignorant and incompetent, often committing offenses, and lacking in boundaries; (3) he refuses to give up a bad view. 

When\marginnote{29.1} a monk has three qualities, the Sangha may, if it wishes, plan a strong action against him:\footnote{Sp 5.323: \textit{\textsanskrit{Āgāḷhāya} \textsanskrit{ceteyyāti} \textsanskrit{āgāḷhāya} \textsanskrit{daḷhabhāvāya} ceteyya; \textsanskrit{tajjanīyakammādikatassa} \textsanskrit{vattaṁ} na \textsanskrit{pūrayato} \textsanskrit{icchamāno} \textsanskrit{saṅgho} \textsanskrit{ukkhepanīyakammaṁ} \textsanskrit{kareyyāti} attho}, “\textit{\textsanskrit{Āgāḷhāya} ceteyya}: it may plan what is strong and firm. The meaning is that the Sangha, if it wishes, may do a procedure of ejection against one who is not fulfilling the duties of one who has had a procedure of condemnation, etc., done against him.” } (1) he is quarrelsome, argumentative, and a creator of legal issues in the Sangha; (2) he is ignorant and incompetent, often committing offenses, and lacking in boundaries; (3) he is constantly and improperly socializing with householders. 

When\marginnote{29.3} a monk has three qualities a legal procedure may be done against him: he is shameless, ignorant, and not a regular monk. —\footnote{Sp 5.425: \textit{\textsanskrit{Apakatattassāti} ukkhittakassa \textsanskrit{vā}, yassa \textsanskrit{vā} \textsanskrit{uposathapavāraṇā} \textsanskrit{ṭhapitā} honti }, “Not regular: one who has been ejected, or one who has had the recitation of the Monastic Code or the invitation ceremony canceled.” } When a monk has three other qualities a legal procedure may be done against him: he has failed in the higher morality; he has failed in conduct; he has failed in view. — When a monk has three other qualities a legal procedure may be done against him: his bodily conduct is frivolous; his verbal conduct is frivolous; his bodily and verbal conduct are frivolous. —\footnote{Sp 5.323: \textit{\textsanskrit{Kāyiko} davo \textsanskrit{nāma} \textsanskrit{pāsakādīhi} \textsanskrit{jūtakīḷanādibhedo} \textsanskrit{anācāro}; \textsanskrit{vācasiko} davo \textsanskrit{nāma} \textsanskrit{mukhālambarakaraṇādibhedo} \textsanskrit{anācāro}; \textsanskrit{kāyikavācasiko} \textsanskrit{nāma} \textsanskrit{naccanagāyanādibhedo} \textsanskrit{dvīhipi} \textsanskrit{dvārehi} \textsanskrit{anācāro}}, “Frivolous bodily conduct: the category of misconduct like gaming and playing, etc., with dice, etc. Frivolous verbal conduct: the category of misconduct like making noises through the mouth, etc. Frivolous bodily and verbal conduct: the category of misconduct through two doors, like dancing and singing, etc.” } When a monk has three other qualities a legal procedure may be done against him: he is improperly behaved by body; he is improperly behaved by speech; he is improperly behaved by body and speech. — When a monk has three other qualities a legal procedure may be done against him: his bodily conduct is harmful; his verbal conduct is harmful; his bodily and verbal conduct are harmful. — When a monk has three other qualities a legal procedure may be done against him: he has wrong livelihood by body; he has wrong livelihood by speech; he has wrong livelihood by body and speech. — When a monk has three other qualities a legal procedure may be done against him: if, after committing an offense and having a legal procedure done against him, he gives the full ordination, gives formal support, has a novice monk attend on him. — When a monk has three other qualities a legal procedure may be done against him: he commits the same offense for which the Sangha did the legal procedure against him; he commits an offense similar to the one for which the Sangha did the legal procedure against him; he commits an offense worse than the one for which the Sangha did the legal procedure against him. — When a monk has three other qualities a legal procedure may be done against him: he disparages the Buddha; he disparages the Teaching; he disparages the Sangha. 

When\marginnote{30.1} a monk has three qualities and is having the observance-day ceremony canceled in the midst of the Sangha, then, after pressing him by saying, ‘Enough, no more arguing and disputing,’ the Sangha should do the observance-day ceremony. These are the three qualities:\footnote{For an explanation of the rendering “observance-day ceremony” for \textit{uposatha}, see Appendix of Technical Terms. } he is shameless, ignorant, and not a regular monk. — When a monk has three qualities and is having the invitation ceremony canceled in the midst of the Sangha, then, after pressing him by saying, ‘Enough, no more arguing and disputing,’ the Sangha should do the invitation ceremony. These are the three qualities:\footnote{For an explanation of the rendering “invitation ceremony” for \textit{\textsanskrit{pavāraṇā}}, see Appendix of Technical Terms. } he is shameless, ignorant, and not a regular monk. — The Sangha should not give any formal approval to a monk who has three qualities: he is shameless, ignorant, and not a regular monk. — When a monk has three qualities, he should not speak in the Sangha: he is shameless, ignorant, and not a regular monk. — When a monk has three qualities, he should not be put in any position of authority:\footnote{Sp 5.323: \textit{Na \textsanskrit{kismiñci} \textsanskrit{paccekaṭṭhāneti} \textsanskrit{kismiñci} \textsanskrit{bījanaggāhādike} ekasmimpi \textsanskrit{jeṭṭhakaṭṭhāne} na \textsanskrit{ṭhapetabboti} attho}, “\textit{Na \textsanskrit{kismiñci} \textsanskrit{paccekaṭṭhāne}} means he should not be put in any fan-holding position, etc., or in a position of seniority.” } he is shameless, ignorant, and not a regular monk. — When a monk has three qualities, one should not live with formal support from him: he is shameless, ignorant, and not a regular monk. — When a monk has three qualities, he should not give formal support: he is shameless, ignorant, and not a regular monk. — When a monk has three qualities, he is not qualified to get permission to correct another:\footnote{Sp 5.323: \textit{\textsanskrit{Okāsakammaṁ} \textsanskrit{kārentassāti} “karotu \textsanskrit{āyasmā} \textsanskrit{okāsaṁ}, \textsanskrit{ahaṁ} \textsanskrit{taṁ} \textsanskrit{vattukāmo}”ti \textsanskrit{evaṁ} \textsanskrit{okāsaṁ} \textsanskrit{kārentassa}} “\textit{\textsanskrit{Okāsakammaṁ} \textsanskrit{kārentassa}} means asking permission in this way: ‘Venerable, give me permission; I wish to correct you.’” } he is shameless, ignorant, and not a regular monk. — When a monk has three qualities, he should not be allowed to direct anyone:\footnote{Sp 5.323: \textit{\textsanskrit{Savacanīyaṁ} \textsanskrit{nādātabbanti} \textsanskrit{vacanaṁ} na \textsanskrit{ādātabbaṁ}, vacanampi na \textsanskrit{sotabbaṁ}}, “\textit{\textsanskrit{Savacanīyaṁ} \textsanskrit{nādātabba}} means he should not undertake to correct someone; and even if he does, he should not be listened to.” } he is shameless, ignorant, and not a regular monk. — When a monk has three qualities, he should not be asked about the Monastic Law: he is shameless, ignorant, and not a regular monk. — When a monk has three qualities, he should not ask about the Monastic Law: he is shameless, ignorant, and not a regular monk. — When a monk has three qualities, his questions about the Monastic Law should not be replied to: he is shameless, ignorant, and not a regular monk. — When a monk has three qualities, he should not reply to questions about the Monastic Law: he is shameless, ignorant, and not a regular monk. — When a monk has three qualities, he should not be allowed to ask questions:\footnote{Sp 5.323: \textit{Anuyogo na \textsanskrit{dātabboti} “\textsanskrit{idaṁ} \textsanskrit{kappatī}”ti pucchantassa \textsanskrit{pucchāya} \textsanskrit{okāso} na \textsanskrit{dātabbo}, “\textsanskrit{aññaṁ} \textsanskrit{pucchā}”ti vattabbo. Iti so neva pucchitabbo \textsanskrit{nāssa} \textsanskrit{pucchā} \textsanskrit{sotabbāti} attho}, “\textit{Anuyogo na \textsanskrit{dātabbo}}: when he asks, ‘Is this allowable?’ he should be not be given the opportunity to question. They should say, ‘Ask someone else.’ In this way, he should neither be asked nor should his questions be listened to. This is the meaning.” } he is shameless, ignorant, and not a regular monk. — When a monk has three qualities, one should not discuss the Monastic Law with him: he is shameless, ignorant, and not a regular monk. — When a monk has three qualities, he should not give the full ordination, give formal support, or have a novice monk attend on him: he is shameless, ignorant, and not a regular monk. 

There\marginnote{31.1} are three kinds of observance-day ceremonies: on the fourteenth, on the fifteenth, and the observance-day ceremony for the sake of unity. — There are three other kinds of observance-day ceremonies: the observance-day ceremony for a sangha, the observance-day ceremony for a group, and the observance-day ceremony for an individual. — There are three other kinds of observance-day ceremonies: the observance-day ceremony which consists of reciting the Monastic Code, the observance-day ceremony which consists of declaring purity, and the observance-day ceremony which consists of a determination. 

There\marginnote{32.1} are three kinds of invitation ceremonies: on the fourteenth, on the fifteenth, and the invitation ceremony for the sake of unity. — There are three other kinds of invitation ceremonies: the invitation ceremony for a sangha, the invitation ceremony for a group, and the invitation ceremony for an individual. — There are three other kinds of invitation ceremonies: the invitation ceremony done by means of three statements, the invitation ceremony done by means of two statements, the invitation ceremony done by means of groups according to the year of seniority. 

There\marginnote{33.1} are three kinds of persons bound for hell: (1) one who, not having abandoned it, does not abstain from sexuality, while claiming to do so; (2) one who groundlessly charges someone who lives a pure spiritual life with not abstaining from sexuality; (3) one who has and declares a view such as this: ‘There is no fault in worldly pleasures,’ and then indulges in them. 

There\marginnote{33.3} are three unwholesome roots: desire, ill will, and confusion. — There are three wholesome roots: non-desire, non-ill will, and non-confusion. — There are three kinds of misconduct: misconduct by body, misconduct by speech, and misconduct by mind. — There are three kinds of good conduct: good conduct by body, good conduct by speech, and good conduct by mind. — There are three reasons why the Buddha laid down the rule on eating in groups of no more than three:\footnote{According to Vmv 4.343 this is a reference to \href{https://suttacentral.net/pli-tv-bu-vb-pc32/en/brahmali\#8.15.1}{Bu Pc 32:8.15.1}. } for the restraint of bad people; for the ease of good monks, stopping those with bad desires from creating a faction and then splitting the Sangha; and out of compassion for families. — It’s because he was overcome and consumed by three bad qualities that Devadatta was irredeemably destined to an eon in hell: bad desires; bad friends; and after trifling successes, he stopped short of the goal. — There are three kinds of approval: approval to use a staff, approval to use a carrying net, and approval to use both a staff and a carrying net. — There are three kinds of foot stands that are fixed in place and immobile: foot stands for defecating, foot stands for urinating, foot stands for restroom ablutions. — There are three kinds of foot scrubbers: stones, pebbles, and pumice.” 

\scend{The section on threes is finished. }

\scuddanaintro{This is the summary: }

\begin{scuddana}%
“While\marginnote{36.1} he is alive, at the right time, and at night, \\
Ten, five, with wholesome; \\
Feelings, grounds for an accusation, \\
Voting, two on prohibited. 

Rule,\marginnote{37.1} and two others, \\
Fools, and during the waning phase of the moon, it is allowable; \\
During winter, a sangha, for a sangha, \\
And concealings, a covering. 

Concealed,\marginnote{38.1} and in the open, \\
Dwelling, sick; \\
Monastic Code, probation, \\
Trial period, those on probation. 

Inside,\marginnote{39.1} and inside the monastery zone, \\
One commits, again another; \\
One clears, and another, \\
Two on resolution because of past insanity. 

Condemnation,\marginnote{40.1} and demotion, \\
Banishing, reconciliation; \\
Not recognizing, making amends, \\
And not giving up a view. 

Strong,\marginnote{41.1} legal procedure, in the higher morality, \\
Frivolous, improperly behaved, harmful; \\
Livelihood, committing, similar, \\
Disparages, and with observance-day ceremony. 

Invitation\marginnote{42.1} ceremony, and formal approval, \\
Speak, and with authority; \\
Should not live, should not give, \\
So one should not ask for permission. 

One\marginnote{43.1} should not direct, \\
Two on those who should not be asked; \\
And two on one should not reply, \\
And one should not be allowed to ask. 

Discussion,\marginnote{44.1} full ordination, \\
Formal support, and novice monk; \\
Three on three observance-day ceremonies, \\
Three on three invitation ceremonies. 

Bound\marginnote{45.1} for the lower, unwholesome, \\
Wholesome, two on conduct; \\
Eating in groups of no more than three, in bad qualities, \\
Approval, and with foot stands; \\
And foot scrubbers—\\
This is the summary for the threes.” 

%
\end{scuddana}

\section*{4. The section on fours }

(1)\marginnote{46.1} There are offenses that one commits through one’s own speech, but clears through someone else’s speech.\footnote{Sp 5.324: \textit{\textsanskrit{Sakavācāya} \textsanskrit{āpajjati} \textsanskrit{paravācāya} \textsanskrit{vuṭṭhātīti} \textsanskrit{vacīdvārikaṁ} \textsanskrit{padasodhammādibhedaṁ} \textsanskrit{āpattiṁ} \textsanskrit{āpajjitvā} \textsanskrit{tiṇavatthārakasamathaṭṭhānaṁ} gato parassa \textsanskrit{kammavācāya} \textsanskrit{vuṭṭhāti}}, “One commits through one’s own speech, but clears through someone else’s speech: having committed an offense in the category of memorizing the teaching, etc., through the speech door, one goes to the place where it can be settled by covering over as if with grass, and one then clears it through someone else’s legal-procedure announcement.” } (2) There are offenses that one commits through someone else’s speech, but clears through one’s own speech.\footnote{Sp 5.324: \textit{\textsanskrit{Paravācāya} \textsanskrit{āpajjati} \textsanskrit{sakavācāya} \textsanskrit{vuṭṭhātīti} \textsanskrit{pāpikāya} \textsanskrit{diṭṭhiyā} \textsanskrit{appaṭinissagge} parassa \textsanskrit{kammavācāya} \textsanskrit{āpajjati}, puggalassa santike desento \textsanskrit{sakavācāya} \textsanskrit{vuṭṭhāti}}, “One commits through someone else’s speech, but clears through one’s own speech: having committed an offense through someone else’s legal-procedure announcement in regard to not giving up a bad view, one clears it through one’s own speech by confessing in the presence of an individual.” } (3) There are offenses that one commits through one’s own speech and clears through one’s own speech.\footnote{Sp 5.324: \textit{\textsanskrit{Sakavācāya} \textsanskrit{āpajjati} \textsanskrit{sakavācāya} \textsanskrit{vuṭṭhātīti} \textsanskrit{vacīdvārikaṁ} \textsanskrit{padasodhammādibhedaṁ} \textsanskrit{āpattiṁ} \textsanskrit{sakavācāya} \textsanskrit{āpajjati}, \textsanskrit{desetvā} \textsanskrit{vuṭṭhahantopi} \textsanskrit{sakavācāya} \textsanskrit{vuṭṭhāti}}, “One commits through one’s own speech and clears through one’s own speech: having committed, through one’s own speech, an offense in the category of memorizing the teaching, etc., through the speech door, one clears it through one’s own speech by confessing in the presence of an individual.” } (4) There are offenses that one commits through someone else’s speech and clears through someone else’s speech.\footnote{Sp 5.324: \textit{\textsanskrit{Paravācāya} \textsanskrit{āpajjati} \textsanskrit{paravācāya} \textsanskrit{vuṭṭhātīti} \textsanskrit{yāvatatiyakaṁ} \textsanskrit{saṅghādisesaṁ} parassa \textsanskrit{kammavācāya} \textsanskrit{āpajjati}, \textsanskrit{vuṭṭhahantopi} parassa \textsanskrit{parivāsakammavācādīhi} \textsanskrit{vuṭṭhāti}}, “One commits through someone else’s speech and clears through someone else’s speech: one commits a third announcement offense entailing suspension through someone else’s legal-procedure announcement, and then clears it through someone else’s legal-procedure announcement concerning probation, etc.” } 

(1)\marginnote{46.5} There are offenses that one commits by body, but clears by speech.\footnote{Sp 5.324: \textit{\textsanskrit{Kāyadvārikaṁ} \textsanskrit{kāyena} \textsanskrit{āpajjati}, desento \textsanskrit{vācāya} \textsanskrit{vuṭṭhāti}}, “One commits a body-door offense by body, and one clears it by speech through confession.” } (2) There are offenses that one commits by speech, but clears by body.\footnote{Sp 5.324: \textit{\textsanskrit{Vacīdvārikaṁ} \textsanskrit{vācāya} \textsanskrit{āpajjati}, \textsanskrit{tiṇavatthārake} \textsanskrit{kāyena} \textsanskrit{vuṭṭhāti}}, “One commits a speech-door offense by speech, and one clears it by body through covering over as if with grass.” } (3) There are offenses that one commits by body and clears by body.\footnote{Sp 5.324: \textit{\textsanskrit{Kāyadvārikaṁ} \textsanskrit{kāyenaāpajjati}, tameva \textsanskrit{tiṇavatthārake} \textsanskrit{kāyena} \textsanskrit{vuṭṭhāti}}, “One commits a body-door offense by body, and one clears it by body through covering over as if with grass.” } (4) There are offenses that one commits by speech and clears by speech.\footnote{Sp 5.324: \textit{\textsanskrit{Vacīdvārikaṁ} \textsanskrit{vācāya} \textsanskrit{āpajjati}, tameva desento \textsanskrit{vācāya} \textsanskrit{vuṭṭhāti}}, “One commits a speech-door offense by speech, and one clears it by speech through confession.” } 

(1)\marginnote{46.9} There are offenses that one commits while sleeping, but clears while awake.\footnote{Sp 5.324: \textit{\textsanskrit{Saṅghikamañcassa} attano \textsanskrit{paccattharaṇena} anattharato \textsanskrit{kāyasamphusane} \textsanskrit{lomagaṇanāya} \textsanskrit{āpajjitabbāpattiṁ} \textsanskrit{sahagāraseyyāpattiñca} pasutto \textsanskrit{āpajjati}, \textsanskrit{pabujjhitvā} pana \textsanskrit{āpannabhāvaṁ} \textsanskrit{ñatvā} desento \textsanskrit{paṭibuddho} \textsanskrit{vuṭṭhāti}}, “For one not covering a bed belonging to the Sangha with his own sheet, an offense is committed through the counting of hairs when the body touches. And the offense of sharing a bed in a house is also committed while sleeping. But having woken up and knowing that one has committed an offense, one clears it by confessing while awake.” } (2) There are offenses that one commits while awake, but clears while sleeping.\footnote{Sp 5.324: \textit{Jagganto \textsanskrit{āpajjitvā} pana \textsanskrit{tiṇavatthārakasamathaṭṭhāne} sayanto \textsanskrit{paṭibuddho} \textsanskrit{āpajjati} pasutto \textsanskrit{vuṭṭhāti} \textsanskrit{nāma}}, “Having committed an offense while awake and then sleeping in the place where it is settled by covering over as if with grass—this is called ‘one commits while awake, but clears while sleeping’.” } (3) There are offenses that one commits while sleeping and clears while sleeping.\footnote{Sp 5.324: \textit{Pacchimapadadvayampi \textsanskrit{vuttānusāreneva} \textsanskrit{veditabbaṁ}}, “The last two cases are to be known through conformity with what has been said.” This means that the last two cases follow the example of the first two. } (4) There are offenses that one commits while awake and clears while awake. 

(1)\marginnote{46.13} There are offenses that one commits unintentionally, but clears intentionally.\footnote{Sp 5.324: \textit{\textsanskrit{Acittakāpattiṁ} acittako \textsanskrit{āpajjati} \textsanskrit{nāma}. \textsanskrit{Pacchā} desento sacittako \textsanskrit{vuṭṭhāti}}, “An unintentional offense is called committed unintentionally. When confessing it later, one clears it intentionally.” } (2) There are offenses that one commits intentionally, but clears unintentionally.\footnote{Sp 5.324: \textit{\textsanskrit{Sacittakāpattiṁ} sacittako \textsanskrit{āpajjati} \textsanskrit{nāma}. \textsanskrit{Tiṇavatthārakaṭṭhāne} sayanto acittako \textsanskrit{vuṭṭhāti}}, “An intentional offense is called committed intentionally. When sleeping at the place (they do the legal procedure) of covering over as if with grass, one clears it unintentionally.” } (3) There are offenses that one commits unintentionally and clears unintentionally.\footnote{Sp 5.324: \textit{Sesapadadvayampi \textsanskrit{vuttānusāreneva} \textsanskrit{veditabbaṁ}}, “The remaining pair, too, is to be understood in accordance with what has been said.” That is, one is to combine each of the two parts of the previous two cases as appropriate to explain this case and the next one. } (4) There are offenses that one commits intentionally and clears intentionally. 

(1)\marginnote{46.17} There are offenses where one confesses an offense while committing an offense.\footnote{Sp 5.324: \textit{Yo \textsanskrit{sabhāgaṁ} \textsanskrit{āpattiṁ} deseti, \textsanskrit{ayaṁ} \textsanskrit{desanāpaccayā} \textsanskrit{dukkaṭaṁ} \textsanskrit{āpajjanto}}, “Whoever confesses a shared offense, commits an offense of wrong conduct on account of the confession.” } (2) There are offenses where one commits an offense while confessing an offense.\footnote{Sp 5.324: \textit{Yo \textsanskrit{sabhāgaṁ} \textsanskrit{āpattiṁ} deseti, \textsanskrit{ayaṁ} \textsanskrit{desanāpaccayā} \textsanskrit{dukkaṭaṁ} \textsanskrit{āpajjanto} \textsanskrit{pācittiyādīsu} \textsanskrit{aññataraṁ} deseti, \textsanskrit{tañca} desento \textsanskrit{dukkaṭaṁ} \textsanskrit{āpajjati}}, “Whoever confesses a shared offense, committing an offense of wrong conduct on account of the confession, if he confesses an offense among the offenses entailing confession etc., in confessing that he commits an offense of wrong conduct.” } (3) There are offenses where one clears an offense while committing an offense.\footnote{Sp 5.324: \textit{\textsanskrit{Taṁ} pana \textsanskrit{dukkaṭaṁ} \textsanskrit{āpajjanto} \textsanskrit{pācittiyādito} \textsanskrit{vuṭṭhāti}}, “But in committing that offense of wrong conduct, he clears the offense entailing confession, etc.” } (4) There are offenses where one commits an offense while clearing an offense.\footnote{Sp 5.324: \textit{\textsanskrit{Pācittiyādito} ca \textsanskrit{vuṭṭhahanto} \textsanskrit{taṁ} \textsanskrit{āpajjati}. Iti ekassa puggalassa ekameva \textsanskrit{payogaṁ} \textsanskrit{sandhāya} “\textsanskrit{āpattiṁ} \textsanskrit{āpajjanto} \textsanskrit{desetī}”ti \textsanskrit{idaṁ} \textsanskrit{catukkaṁ} vuttanti \textsanskrit{veditabbaṁ}}, “In clearing the offense entailing confession, etc., he commits that offense. Thus it is to be understood that this fourfold statement was said with reference to just a single effort of a single individual, that is, ‘One confesses an offense while committing an offense’.” } 

(1)\marginnote{46.21} There are offenses that one commits through action, but clears through non-action.\footnote{Sp 5.324: \textit{Kammacatukke \textsanskrit{pāpikāya} \textsanskrit{diṭṭhiyā} \textsanskrit{appaṭinissaggāpattiṁ} kammena \textsanskrit{āpajjati}, desento akammena \textsanskrit{vuṭṭhāti}}, “In the tetrad on legal procedures, one commits the offense for not giving up a bad view through a legal procedure, but clears it without a legal procedure when confessing it.” Sp-yoj 5.324: \textit{\textsanskrit{Kammenāti} \textsanskrit{samanubhāsanakammena}}: “Through a legal procedure means through the legal procedure of pressing.” See \href{https://suttacentral.net/pli-tv-bu-vb-pc68/en/brahmali\#1.49.1}{Bu Pc 68:1.49.1}. } (2) There are offenses that one commits through non-action, but clears through action.\footnote{Sp 5.324: \textit{\textsanskrit{Vissaṭṭhiādikaṁ} akammena \textsanskrit{āpajjati}, \textsanskrit{parivāsādinā} kammena \textsanskrit{vuṭṭhāti}}, “One commits the offense of emission without a legal procedure, but clears it through the legal procedure of probation, etc.” } (3) There are offenses that one commits through action and clears through action.\footnote{Sp 5.324: \textit{\textsanskrit{Samanubhāsanaṁ} kammeneva \textsanskrit{āpajjati}, kammena \textsanskrit{vuṭṭhāti}}, “One commits the offense through a legal procedure of pressing and clears it through a legal procedure.” } (4) There are offenses that one commits through non-action and clears through non-action.\footnote{Sp 5.324: \textit{\textsanskrit{Sesaṁ} akammena \textsanskrit{āpajjati}, akammena \textsanskrit{vuṭṭhāti}}, “One commits the rest without a legal procedure and clears them without a legal procedure.” } 

There\marginnote{47.1} are four kinds of ignoble speech: (1) saying that one has seen what one has not seen; (2) saying that one has heard what one has not heard; (3) saying that one has sensed what one has not sensed; (4) saying that one has known what one has not known. 

There\marginnote{47.3} are four kinds of noble speech: (1) saying that one has not seen what one has not seen; (2) saying that one has not heard what one has not heard; (3) saying that one has not sensed what one has not sensed; (4) saying that one has not known what one has not known. 

There\marginnote{47.5} are four other kinds of ignoble speech: (1) saying that one has not seen what one has seen; (2) saying that one has not heard what one has heard; (3) saying that one has not sensed what one has sensed; (4) saying that one has not known what one has known. 

There\marginnote{47.7} are four other kinds of noble speech: (1) saying that one has seen what one has seen; (2) saying that one has heard what one has heard; (3) saying that one has sensed what one has sensed; (4) saying that one has known what one has known. 

The\marginnote{48.1} monks have four offenses entailing expulsion in common with the nuns. — The nuns have four offenses entailing expulsion not in common with the monks. 

There\marginnote{48.3} are four kinds of requisites:\footnote{Sp 5.324: \textit{\textsanskrit{Parikkhāracatukke} \textsanskrit{paṭhamo} \textsanskrit{sakaparikkhāro}, dutiyo \textsanskrit{saṅghikova} tatiyo cetiyasantako, catuttho \textsanskrit{gihiparikkhāro}}, “In the tetrad on requisites, the first is one’s own requisites, the second the requisites of the Sangha, the third what is owned by a shrine, and the fourth the requisites of a householder.” For a dicsussion of the word \textit{vibbhamati}, see Appendix of Technical Terms. } (1) There are requisites that should be guarded, taken as a personal possession, and made use of. (2) There are requisites that should be guarded and made use of, but not taken as a personal possession.\footnote{I translate according to the reading found in SRT: \textit{Atthi \textsanskrit{parikkhāro} rakkhitabbo gopetabbo na \textsanskrit{mamāyitabbo} \textsanskrit{paribhuñjitabbo}}. The Pali reading found in the current text does not fit the commentarial explanation that this concerns a requisite belonging to the Sangha, for Sangha requisites cannot be taken as one’s own. So it seems the commentarial explanation must be based on the SRT reading. } (3) There are requisites that should be guarded, but not taken as a personal possession or made use of. (4) There are requisites that should neither be guarded, nor taken as a personal possession, nor made use of. 

(1)\marginnote{48.8} There are offenses one commits in the presence of someone, but clears in their absence.\footnote{Sp 5.324: \textit{\textsanskrit{Sammukhācatukke} \textsanskrit{pāpikāya} \textsanskrit{diṭṭhiyā} \textsanskrit{appaṭinissaggāpattiṁ} \textsanskrit{saṅghassa} \textsanskrit{sammukhā} \textsanskrit{āpajjati}, \textsanskrit{vuṭṭhānakāle} pana \textsanskrit{saṅghena} \textsanskrit{kiccaṁ} \textsanskrit{natthīti} \textsanskrit{parammukhā} \textsanskrit{vuṭṭhāti}}, “In the tetrad on presence, one commits an offense for not giving up a bad view in the presence of the Sangha, but at the time of clearing, one clears it in its absence, thinking, ‘There is no duty for the Sangha’.” } (2) There are offenses one commits in the absence of someone, but clears in their presence.\footnote{Sp 5.324: \textit{\textsanskrit{Vissaṭṭhiādikaṁ} \textsanskrit{parammukhā} \textsanskrit{āpajjati}, \textsanskrit{saṅghassa} \textsanskrit{sammukhā} \textsanskrit{vuṭṭhāti}}, “One commits an offense of emission, etc., in the absence of the Sangha, but clears it in its presence.” } (3) There are offenses one commits in the presence of someone and clears in their presence.\footnote{Sp 5.324: \textit{\textsanskrit{Samanubhāsanaṁ} \textsanskrit{saṅghassa} \textsanskrit{sammukhā} eva \textsanskrit{āpajjati}, \textsanskrit{sammukhā} \textsanskrit{vuṭṭhāti}}, “One commits an offense when pressed in the presence of the Sangha and also clears it in its presence.” } (4) There are offenses one commits in the absence of someone and clears in their absence.\footnote{Sp 5.324: \textit{\textsanskrit{Sesaṁ} \textsanskrit{sampajānamusāvādādibhedaṁ} \textsanskrit{parammukhāva} \textsanskrit{āpajjati}, \textsanskrit{parammukhāva} \textsanskrit{vuṭṭhāti}}, “For the remainder, that is, the category of lying in full awareness, etc., one commits an offense in the absence of the Sangha and also clears it in its absence.” } 

(1)\marginnote{48.12} There are offenses one commits unknowingly, but clears knowingly.\footnote{Sp 5.324: \textit{\textsanskrit{Ajānantacatukkaṁ} \textsanskrit{acittakacatukkasadisaṁ}}, “The tetrad on knowing is similar to the tetrad on unintentional.” } (2) There are offenses one commits knowingly, but clears unknowingly. (3) There are offenses one commits unknowingly and clears unknowingly. (4) There are offenses one commits knowingly and clears knowingly. 

One\marginnote{49.1} commits offenses in four ways: by body, by speech, by body and speech, through a legal procedure. — One commits offenses in four other ways: in the midst of the Sangha, in the midst of a group, in the presence of an individual, through the appearance of sexual characteristics. — One clears offenses in four ways: by body, by speech, by body and speech, through a legal procedure. — One clears offenses in four other ways: in the midst of the Sangha, in the midst of a group, in the presence of an individual, through the appearance of sexual characteristics. — When one gets it: one abandons the former, one is established in the latter, asking for things comes to a stop, rules come to an end. —\footnote{Sp 5.324: \textit{\textsanskrit{Sahapaṭilābhacatukke} yassa bhikkhuno \textsanskrit{liṅgaṁ} parivattati, so saha \textsanskrit{liṅgapaṭilābhena} \textsanskrit{paṭhamaṁ} uppannavasena \textsanskrit{seṭṭhabhāvena} ca \textsanskrit{purimaṁ} \textsanskrit{purisaliṅgaṁ} jahati, pacchime \textsanskrit{itthiliṅge} \textsanskrit{patiṭṭhāti}, \textsanskrit{purisakuttapurisākārādivasena} \textsanskrit{pavattā} \textsanskrit{kāyavacīviññattiyo} \textsanskrit{paṭippassambhanti}, \textsanskrit{bhikkhūti} \textsanskrit{vā} purisoti \textsanskrit{vā} \textsanskrit{evaṁ} \textsanskrit{pavattā} \textsanskrit{paṇṇattiyo} nirujjhanti, \textsanskrit{yāni} \textsanskrit{bhikkhunīhi} \textsanskrit{asādhāraṇāni} \textsanskrit{chacattālīsa} \textsanskrit{sikkhāpadāni} tehi \textsanskrit{anāpattiyeva} hoti}, “In the tetrad on ‘when one gets it’, the characteristics of a monk are changed. Together with the appearance of the characteristics, then, on account of their arising first and being the best, he abandons the former characteristics of a man. He is established in the latter characteristics of a woman. The asking for things by body and speech that happens on account of the ways of a man, etc., that comes to a stop. Expressions used such as ‘monk’ and ‘man’ come to an end. The forty-six training rules of the monks that are not in common with the nuns are now non-offenses.” } When one gets it: one abandons the latter, one is established in the former, asking for things comes to a stop, rules come to an end. —\footnote{Sp 5.324: \textit{Dutiyacatukke pana \textsanskrit{yassā} \textsanskrit{bhikkhuniyā} \textsanskrit{liṅgaṁ} parivattati, \textsanskrit{sā} \textsanskrit{pacchāsamuppattiyā} \textsanskrit{vā} \textsanskrit{hīnabhāvena} \textsanskrit{vā} pacchimanti \textsanskrit{saṅkhyaṁ} \textsanskrit{gataṁ} \textsanskrit{itthiliṅgaṁ} jahati, \textsanskrit{vuttappakārena} purimanti \textsanskrit{saṅkhyaṁ} gate \textsanskrit{purisaliṅge} \textsanskrit{patiṭṭhāti}. … \textsanskrit{bhikkhunīti} \textsanskrit{vā} \textsanskrit{itthīti} \textsanskrit{vā} \textsanskrit{evaṁ} \textsanskrit{pavattā} \textsanskrit{paṇṇattiyopi} nirujjhanti, \textsanskrit{yāni} \textsanskrit{bhikkhūhi} \textsanskrit{asādhāraṇāni} \textsanskrit{sataṁ} \textsanskrit{tiṁsañca} \textsanskrit{sikkhāpadāni}, tehi \textsanskrit{anāpattiyeva} hoti}, “In the second tetrad, the characteristics of a nun change. Because of their later arising or because of being inferior, she is reckoned as abandoning the latter female characteristic, and in the said way is reckoned as established in the former characteristics of a man. … Expressions used such as ‘nun’ and ‘woman’ come to an end. The one hundred and thirty training rules of the nuns that are not in common with the monks are now non-offenses.” } There are four kinds of accusing: one accuses someone for failure in morality, one accuses someone for failure in conduct, one accuses someone for failure in view, one accuses someone for failure in livelihood. — There are four kinds of probation: probation for concealed offenses, probation for unconcealed offenses, purifying probation, and simultaneous probation. — There are four kinds of trial periods: trial periods for concealed offenses, trial periods for unconcealed offenses, trial periods for a half-month, and simultaneous trial periods. — There are four things that stop a monk who is undertaking the trial period from counting a particular day toward his trial period: he stays in the same room as a regular monk; he stays apart from other monks; he doesn’t inform other monks of his status; he travels without a group. — There are four unique things. —\footnote{Sp 5.324: \textit{\textsanskrit{Cattāro} \textsanskrit{sāmukkaṁsāti} \textsanskrit{cattāro} \textsanskrit{mahāpadesā}}, “The four unique things are the four great standards.” For an explanation of the rendering “standard” for \textit{sugata}, see Appendix of Technical Terms. } There are four things that need to be received: ordinary food, post-midday tonics, seven-day tonics, and lifetime tonics. — There are four foul edibles: feces, urine, ash, and clay. — There are four kinds of legal procedures: procedures consisting of getting permission, procedures consisting of one motion, procedures consisting of one motion and one announcement, or procedures consisting of one motion and three announcements. — There are four other kinds of legal procedures: illegitimate legal procedures done by an incomplete assembly, illegitimate legal procedures done unanimously, legitimate legal procedures done by an incomplete assembly, and legitimate legal procedures done unanimously. — There are four kinds of failure: failure in morality, failure in conduct, failure in view, and failure in livelihood. — There are four kinds of legal issues: legal issues arising from disputes, legal issues arising from accusations, legal issues arising from offenses, and legal issues arising from business. — There are four kinds of people who corrupt a gathering: an immoral monk with bad qualities, an immoral nun with bad qualities, an immoral male lay follower with bad qualities, an immoral female lay follower with bad qualities. — There are four kinds of people who make a gathering shine: a moral monk with good qualities, a moral nun with good qualities, a moral male lay follower with good qualities, a moral female lay follower with good qualities. 

(1)\marginnote{50.1} There are offenses committed by new arrivals, not by residents.\footnote{For this tetrad, see \href{https://suttacentral.net/pli-tv-kd18/en/brahmali\#1.1.1}{Kd 18:1.1.1} and \href{https://suttacentral.net/pli-tv-kd18/en/brahmali\#2.1.1}{Kd 18:2.1.1}. } (2) There are offenses committed by residents, not by new arrivals. (3) There are offenses committed both by new arrivals and by residents. (4) There are offenses committed neither by new arrivals nor by residents. 

(1)\marginnote{50.5} There are offenses committed by those departing, not by residents.\footnote{For this tetrad, see \href{https://suttacentral.net/pli-tv-kd18/en/brahmali\#3.1.1}{Kd 18:3.1.1} and \href{https://suttacentral.net/pli-tv-kd18/en/brahmali\#2.1.1}{Kd 18:2.1.1}. } (2) There are offenses committed by residents, not by those departing. (3) There are offenses committed both by those departing and by residents. (4) There are offenses committed neither by those departing nor by residents. 

(1)\marginnote{50.9} There are rules that have variety in the action that is the basis for the offense, but not in the offense. (2) There are rules that have variety in the offense, but not in the action that is the basis for the offense. (3) There are rules that have variety both in the action that is the basis for the offense and in the offense. (4) There are rules that have variety neither in the action that is the basis for the offense nor in the offense.\footnote{Sp 5.324: \textit{\textsanskrit{Vatthunānattatādicatukke} \textsanskrit{catunnaṁ} \textsanskrit{pārājikānaṁ} \textsanskrit{aññamaññaṁ} \textsanskrit{vatthunānattatāva} hoti,\textsanskrit{naāpattinānattatā}. \textsanskrit{Sabbāpi} hi \textsanskrit{sā} \textsanskrit{pārājikāpattiyeva}. \textsanskrit{Saṅghādisesādīsupi} eseva nayo. Bhikkhussa ca \textsanskrit{bhikkhuniyā} ca \textsanskrit{aññamaññaṁ} \textsanskrit{kāyasaṁsagge} bhikkhussa \textsanskrit{saṅghādiseso} \textsanskrit{bhikkhuniyā} \textsanskrit{pārājikanti} \textsanskrit{evaṁ} \textsanskrit{āpattinānattatāva} hoti, na \textsanskrit{vatthunānattatā}, ubhinnampi hi \textsanskrit{kāyasaṁsaggova} vatthu. \textsanskrit{Tathā} “\textsanskrit{lasuṇakkhādane} \textsanskrit{bhikkhuniyā} \textsanskrit{pācittiyaṁ}, bhikkhussa \textsanskrit{dukkaṭa}”nti \textsanskrit{evamādināpettha} nayena \textsanskrit{yojanā} \textsanskrit{veditabbā}. \textsanskrit{Catunnaṁ} \textsanskrit{pārājikānaṁ} terasahi \textsanskrit{saṅghādisesehi} \textsanskrit{saddhiṁ} \textsanskrit{vatthunānattatā} ceva \textsanskrit{āpattinānattatā} ca. \textsanskrit{Evaṁ} \textsanskrit{saṅghādisesādīnaṁ} \textsanskrit{aniyatādīhi}. Ādito \textsanskrit{paṭṭhāya} \textsanskrit{cattāri} \textsanskrit{pārājikāni} ekato \textsanskrit{āpajjantānaṁ} \textsanskrit{bhikkhubhikkhunīnaṁ} neva \textsanskrit{vatthunānattatā} no \textsanskrit{āpattinānattatā}}, “In the tetrad on ‘variety in the action that is the basis for the offense’, for each of the four offenses entailing expulsion, there is variety in the action that is the basis for the offense but no variety in the offense. For all it is just an offense entailing expulsion. The same method applies to the offenses entailing suspension, etc. Regarding physical contact, each for a monk or a nun, there is an offense entailing suspension for a monk and an offense entailing expulsion for a nun, and thus there is variety in the offense, but not in the action that is the basis for the offense, for both have physical contact as the basis for the action. So, in eating garlic, there is an offense of confession for a nun but an offense of wrong conduct for a monk. Here the meaning is to be understood through this method, etc. For the four offenses entailing expulsion, together with the thirteen offenses entailing suspension, there is variety both in the action that is the basis for the offense and in the offense. Thus it is for the offenses entailing suspension, etc., together with the undetermined offenses, etc. Starting from the four offenses entailing expulsion on one side, for the committing monks and nuns, there is variety neither in the action that is the basis for the offense nor in the offense” } 

(1)\marginnote{50.10} There are rules where the action that is the basis for the offense is shared, but not the offense. (2) There are rules where the offense is shared, but not the action that is the basis for the offense. (3) There are rules where both the action that is the basis for the offense and the offense are shared. (4) There are rules where neither the action that is the basis for the offense nor the offense is shared.\footnote{Sp 5.324: \textit{\textsanskrit{Vatthusabhāgādicatukke} bhikkhussa ca \textsanskrit{bhikkhuniyā} ca \textsanskrit{kāyasaṁsagge} \textsanskrit{vatthusabhāgatā}, no \textsanskrit{āpattisabhāgatā}, \textsanskrit{catūsu} \textsanskrit{pārājikesu} \textsanskrit{āpattisabhāgatā}, no \textsanskrit{vatthusabhāgatā}. Esa nayo \textsanskrit{saṅghādisesādīsu}. Bhikkhussa ca \textsanskrit{bhikkhuniyā} ca \textsanskrit{catūsu} \textsanskrit{pārājikesu} \textsanskrit{vatthusabhāgatā} ceva \textsanskrit{āpattisabhāgatā} ca. Esa nayo \textsanskrit{sabbāsu} \textsanskrit{sādhāraṇāpattīsu}. \textsanskrit{Asādhāraṇāpattiyaṁ} neva \textsanskrit{vatthusabhāgatā} no \textsanskrit{āpattisabhāgatā}}, “In the tetrad on ‘where the action that is the basis for the offense is shared, etc.’, regarding physical contact for a monk or a nun, the action that is the basis for the offense is shared, but not the offense. For the four offenses entailing expulsion, the offenses are shared, but not the action that is the basis for the offense. This is the method for the offenses entailing suspension. For the four offenses entailing expulsion of both the monks and the nuns, both the action that is the basis for the offense and the offense are shared. This is the method for all the common offenses. For the offenses that are not in common, neither the action that is the basis for the offense nor the offense is shared.” } 

(1)\marginnote{50.11} There are offenses committed by the preceptor, but not the student. (2) There are offenses committed by the student, but not the preceptor. (3) There are offenses committed by both the preceptor and the student. (4) There are offenses committed by neither the preceptor nor the student.\footnote{Sp 5.324: \textit{\textsanskrit{Upajjhāyacatukke} \textsanskrit{saddhivihārikassa} \textsanskrit{upajjhāyena} kattabbavattassa \textsanskrit{akaraṇe} \textsanskrit{āpattiṁ} \textsanskrit{upajjhāyo} \textsanskrit{āpajjati}, no \textsanskrit{saddhivihāriko} \textsanskrit{upajjhāyassa} \textsanskrit{kattabbavattaṁ} akaronto \textsanskrit{saddhivihāriko} \textsanskrit{āpajjati}, no \textsanskrit{upajjhāyo}; \textsanskrit{sesaṁ} ubhopi \textsanskrit{āpajjanti}, \textsanskrit{asādhāraṇaṁ} ubhopi \textsanskrit{nāpajjanti}}, “In the tetrad on the preceptor, the preceptor commits an offense in not doing the duties to be done by a preceptor, not the student. The student commits an offense in not doing the duties to be done towards a preceptor, not the preceptor. The rest are committed by both. Offenses not in common between the monks and the nuns are committed by neither.” } 

(1)\marginnote{50.12} There are offenses committed by the teacher, but not the pupil. (2) There are offenses committed by the pupil, but not the teacher. (3) There are offenses committed by both the teacher and the pupil. (4) There are offenses committed by neither the teacher nor the pupil.\footnote{Sp 5.324: \textit{Ācariyacatukkepi eseva nayo}, “Also in the tetrad on the teacher, this is the method.” } 

There\marginnote{50.13} is no offense for breaking the rainy-season residence for these four reasons: there is a schism in the Sangha; there are some who want to cause a schism in the Sangha; there is a threat to life; there is a threat to the monastic life. — There are four kinds of bad conduct by speech: lying, divisive speech, harsh speech, and idle speech. — There are four kinds of good conduct by speech: truthful speech, non-divisive speech, gentle speech, and meaningful speech. 

(1)\marginnote{50.19} There are offenses that are serious when taking for oneself, but light when inciting someone else.\footnote{Sp 5.324: \textit{Ādiyantacatukke \textsanskrit{pādaṁ} \textsanskrit{vā} \textsanskrit{atirekapādaṁ} \textsanskrit{vā} \textsanskrit{sahatthā} \textsanskrit{ādiyanto} \textsanskrit{garukaṁ} \textsanskrit{āpajjati}, \textsanskrit{ūnakapādaṁ} \textsanskrit{gaṇhāhīti} \textsanskrit{āṇattiyā} \textsanskrit{aññaṁ} payojento \textsanskrit{lahukaṁ} \textsanskrit{āpajjati}. Etena nayena \textsanskrit{sesapadattayaṁ} \textsanskrit{veditabbaṁ}}, “In the tetrad on taking for oneself, one commits a serious offense when personally taking a \textit{\textsanskrit{pāda}} coin or more than a \textit{\textsanskrit{pāda}}, but one commits a light offense when inciting someone else by asking them to take less than a \textit{\textsanskrit{pāda}}. The remaining three cases are to be understood through this method.” } (2) There are offenses that are light when taking for oneself, but serious when inciting someone else. (3) There are offenses that are serious both when taking for oneself and when inciting someone else. (4) There are offenses that are light both when taking for oneself and when inciting someone else. 

(1)\marginnote{51.1} There are people who deserve being bowed down to, but not being stood up for.\footnote{Sp 5.324: \textit{\textsanskrit{Abhivādanārahacatukke} \textsanskrit{bhikkhunīnaṁ} \textsanskrit{tāva} bhattagge navamabhikkhunito \textsanskrit{paṭṭhāya} \textsanskrit{upajjhāyāpi} \textsanskrit{abhivādanārahā} no \textsanskrit{paccuṭṭhānārahā}. Avisesena ca vippakatabhojanassa bhikkhussa yo koci \textsanskrit{vuḍḍhataro}}, “In the tetrad on ‘those who deserve being bowed down to’, in the dining hall, as far as the nuns are concerned, starting from the ninth nun, the preceptor deserves being bowed down to, but not being stood up for; and through non-discrimination, whatever monk is more senior and has not finished his meal.” } (2) There are people who deserve being stood up for, but not to being bowed down to.\footnote{Sp 5.324: \textit{\textsanskrit{Saṭṭhivassassāpi} \textsanskrit{pārivāsikassa} \textsanskrit{samīpagato} tadahupasampannopi \textsanskrit{paccuṭṭhānāraho} no \textsanskrit{abhivādanāraho}}, “A monk of sixty years seniority who is on probation, when coming close to one ordained on that very day, deserves being stood up for, but not to being bowed down to.” } (3) There are people who deserve both being bowed down to and being stood up for.\footnote{Sp 5.324: \textit{\textsanskrit{Appaṭikkhittesu} \textsanskrit{ṭhānesu} \textsanskrit{vuḍḍho} navakassa \textsanskrit{abhivādanāraho} ceva \textsanskrit{paccuṭṭhānāraho} ca}, “When there are no prohibiting grounds, then a more senior monk deserves to have a more junior monk bow down to him and stand up for him.” } (4) There are people who deserve neither being bowed down to nor being stood up for.\footnote{Sp 5.324: \textit{Navako pana \textsanskrit{vuḍḍhassa} neva \textsanskrit{abhivādanāraho} na \textsanskrit{paccuṭṭhānāraho}}, “But a more junior monk does not deserve to have a more senior monk bow down to him and stand up for him.” } 

(1)\marginnote{51.5} There are people who deserve a seat, but not being bowed down to.\footnote{Sp 5.324: \textit{\textsanskrit{Āsanārahacatukkassa} \textsanskrit{paṭhamapadaṁ} purimacatukke dutiyapadena, \textsanskrit{dutiyapadañca} \textsanskrit{paṭhamapadena} atthato \textsanskrit{sadisaṁ}}, “The first case in the tetrad on ‘those who deserve a seat’ is parallel in meaning to the second case in the previous tetrad, and the second case is parallel to the first case.” } (2) There are people who deserve being bowed down to, but not a seat. (3) There are people who deserve both a seat and being bowed down to. (4) There are people who deserve neither a seat nor being bowed down to. 

(1)\marginnote{51.9} There are offenses that one commits at the right time, not at the wrong time.\footnote{Sp 5.324: \textit{\textsanskrit{Kālacatukke} \textsanskrit{pavāretvā} \textsanskrit{bhuñjanto} \textsanskrit{kāle} \textsanskrit{āpajjati} no \textsanskrit{vikāle}}, “In the tetrad on ‘at the right time’, if one eats after refusing an invitation to eat more, one commits an offense at the right time, not at the wrong time.” } (2) There are offenses that one commits at the wrong time, not at the right time.\footnote{Sp 5.324: \textit{\textsanskrit{Vikālabhojanāpattiṁ} \textsanskrit{vikāle} \textsanskrit{āpajjati} no \textsanskrit{kāle}}, “The offense of eating at the wrong time is committed at the wrong time not at the right time.” } (3) There are offenses that one commits both at the right time and at the wrong time.\footnote{Sp 5.324: \textit{\textsanskrit{Sesaṁ} \textsanskrit{kāle} ceva \textsanskrit{āpajjati} \textsanskrit{vikāle} ca}, “The rest one commits both at the right time and at the wrong time.” } (4) There are offenses that one commits neither at the right time nor at the wrong time.\footnote{Sp 5.324: \textit{\textsanskrit{Asādhāraṇaṁ} neva \textsanskrit{kāle} no \textsanskrit{vikāle}}, “An offense that is not in common between the monks and the nuns is committed neither at the right time, nor at the wrong time.” } 

(1)\marginnote{51.13} There are things that when received are allowable at the right time, but not at the wrong time.\footnote{Sp 5.324: \textit{\textsanskrit{Paṭiggahitacatukke} \textsanskrit{purebhattaṁ} \textsanskrit{paṭiggahitāmisaṁ} \textsanskrit{kāle} kappati no \textsanskrit{vikāle}}, “In the tetrad on receiving, food received before the meal is allowable at the right time, not at the wrong time.” } (2) There are things that when received are allowable at the wrong time, but not at the right time.\footnote{Sp 5.324: \textit{\textsanskrit{Pānakaṁ} \textsanskrit{vikāle} kappati, punadivasamhi no \textsanskrit{kāle}}, “A drink is allowable at the wrong time, but not at the right time on the next day.” “A drink” refers to the juice drinks that are allowable in the afternoon. } (3) There are things that when received are allowable both at the right time and at the wrong time.\footnote{Sp 5.324: \textit{\textsanskrit{Sattāhakālikaṁ} \textsanskrit{yāvajīvikaṁ} \textsanskrit{kāle} ceva kappati \textsanskrit{vikāle} ca}, “Seven-day tonics and lifetime tonics are allowable both at the right time and at the wrong time.” } (4) There are things that when received are allowable neither at the right time nor at the wrong time.\footnote{Sp 5.324: \textit{Attano attano \textsanskrit{kālātītaṁ} \textsanskrit{yāvakālikādittayaṁ} \textsanskrit{akappiyamaṁsaṁ} \textsanskrit{uggahitakamappaṭiggahitakañca} neva \textsanskrit{kāle} kappati no \textsanskrit{vikāle}}, “(1) For each and every person, the triad beginning with ordinary food, when the right time has lapsed; (2) unallowable meat; (3) what has been picked up; and (4) what has not been received—are all allowable neither at the right time nor at the wrong time.” “The triad beginning with ordinary food” refers to ordinary food, post-midday tonics, and seven-day tonics. } 

(1)\marginnote{51.17} There are offenses that one commits outside the central Ganges plain, but not within it.\footnote{Sp 5.324: \textit{Paccantimacatukke samudde \textsanskrit{sīmaṁ} bandhanto paccantimesu janapadesu \textsanskrit{āpajjati}, no majjhimesu}, “In the tetrad on ‘outside the central Ganges plain’, if one creates a monastery zone in the ocean, one commits an offense outside the central Ganges plain, not within it.” } (2) There are offenses that one commits within the central Ganges plain, but not outside it.\footnote{Sp 5.324: \textit{\textsanskrit{Pañcavaggena} \textsanskrit{gaṇena} \textsanskrit{upasampādento} \textsanskrit{guṇaṅguṇūpāhanaṁ} \textsanskrit{dhuvanahānaṁ} \textsanskrit{cammattharaṇāni} ca majjhimesu janapadesu \textsanskrit{āpajjati} no paccantimesu}, “Giving the full ordination in a group of five, wearing sandals with multilayered soles, in unrestricted bathing, and having rugs made of skins, one commits an offense within the central Ganges plain, but not outside it.” } (3) There are offenses that one commits both outside the central Ganges plain and within it.\footnote{Sp 5.324: \textit{\textsanskrit{Imāni} \textsanskrit{cattāri} “idha na \textsanskrit{kappantī}”ti vadantopi paccantimesu \textsanskrit{āpajjati}, “idha \textsanskrit{kappantī}”ti vadanto pana majjhimesu \textsanskrit{āpajjati}. \textsanskrit{Sesāpattiṁ} ubhayattha \textsanskrit{āpajjati}}, “In saying that these four are not allowable here, one commits an offense outside the central Ganges plain, but in saying that they are allowable here, one commits an offense within the central Ganges plain. The rest one commits in both places.” } (4) There are offenses that one commits neither outside the central Ganges plain nor within it.\footnote{Sp 5.324: \textit{\textsanskrit{Asādhāraṇaṁ} na katthaci \textsanskrit{āpajjati}}, “The offenses that are not in common are not committed anywhere.” The point, presumably, is that the offenses of the monks that are not offenses for the nuns are never committed anywhere by the nuns. And the same for the parallel case of offenses of the nuns that are not offenses for the monks. } 

(1)\marginnote{51.21} There are things that are allowable outside the central Ganges plain, but not within it.\footnote{Sp 5.324: \textit{Dutiyacatukke \textsanskrit{pañcavaggena} \textsanskrit{gaṇena} \textsanskrit{upasampadādi} catubbidhampi vatthu paccantimesu janapadesu kappati. “\textsanskrit{Idaṁ} \textsanskrit{kappatī}”ti \textsanskrit{dīpetumpi} tattheva kappati no majjhimesu}, “In the second tetrad, the fourfold action that is the basis for an offense, starting with the full ordination in a group of five, is allowable outside the central Ganges plain. Also to proclaim that, ‘This is allowable,’ is allowable just there, not within the central Ganges plain.” } (2) There are things that are allowable within the central Ganges plain, but not outside of it.\footnote{Sp 5.324: \textit{“\textsanskrit{Idaṁ} na \textsanskrit{kappatī}”ti \textsanskrit{dīpetuṁ} pana majjhimesu janapadesu kappati no paccantimesu}, “But to proclaim, ‘This isn’t allowable,’ is allowable within the central Ganges plain, but not outside of it.” } (3) There are things that are allowable both outside the central Ganges plain and within it.\footnote{Sp 5.324: \textit{\textsanskrit{Sesaṁ} “\textsanskrit{anujānāmi} bhikkhave \textsanskrit{pañca} \textsanskrit{loṇānī}”\textsanskrit{tiādi} \textsanskrit{anuññātakaṁ} ubhayattha kappati}, “The rest that is allowed, starting with, ‘Monks, I allow five kinds of salt,’ is allowable in both places.” } (4) There are things that are allowable neither outside the central Ganges plain nor within it.\footnote{Sp 5.324: \textit{\textsanskrit{Yaṁ} pana akappiyanti \textsanskrit{paṭikkhittaṁ}, \textsanskrit{taṁ} \textsanskrit{ubhayatthāpi} na kappati}, “But whatever is prohibited as unallowable is unallowable in both places.” } 

(1)\marginnote{51.25} There are offenses that one commits inside, but not outside.\footnote{Sp 5.324: \textit{\textsanskrit{Antoādicatukke} anupakhajja \textsanskrit{seyyādiṁ} anto \textsanskrit{āpajjati} no bahi}, “In the tetrad on inside etc., when encroaching with a sleeping place, etc., one commits an offense inside, but not outside.” “Encroaching with a sleeping place” refers to \href{https://suttacentral.net/pli-tv-bu-vb-pc16/en/brahmali\#1.16.1}{Bu Pc 16:1.16.1}. } (2) There are offenses that one commits outside, but not inside.\footnote{Sp 5.324: \textit{\textsanskrit{Ajjhokāse} \textsanskrit{saṅghikamañcādīni} \textsanskrit{nikkhipitvā} pakkamanto bahi \textsanskrit{āpajjati} no anto}, “Putting a bed, etc., belonging to the Sangha outside and then leaving, one commits an offense outside, but not inside.” } (3) There are offenses that one commits both inside and outside.\footnote{Sp 5.324: \textit{\textsanskrit{Sesaṁ} anto ceva bahi ca}, “The rest are committed both inside and outside.” } (4) There are offenses that one commits neither inside nor outside.\footnote{Sp 5.324: \textit{\textsanskrit{Asādhāraṇaṁ} neva anto na bahi}, “Offenses not in common are committed neither inside nor outside.” } 

(1)\marginnote{51.29} There are offenses that one commits inside the monastery zone, not outside.\footnote{Sp 5.324: \textit{\textsanskrit{Antosīmādicatukke} \textsanskrit{āgantuko} \textsanskrit{vattaṁ} \textsanskrit{apūrento} \textsanskrit{antosīmāya} \textsanskrit{āpajjati}}, “In the tetrad on inside the monastery zone etc., not fulfilling the duties of a newly arrived monk, one commits an offense inside the monastery zone.” } (2) There are offenses that one commits outside the monastery zone, not inside.\footnote{Sp 5.324: \textit{Gamiyo \textsanskrit{bahisīmāya} … \textsanskrit{āpajjati}}, “One departing commits it outside the monastery zone.” } (3) There are offenses that one commits both inside and outside the monastery zone.\footnote{Sp 5.324: \textit{\textsanskrit{Musāvādādiṁ} \textsanskrit{antosīmāya} ca \textsanskrit{bahisīmāya} ca \textsanskrit{āpajjati}}, “In lying, etc., one commits an offense both inside and outside the monastery zone.” } (4) There are offenses that one commits neither inside nor outside the monastery zone.\footnote{Sp 5.324: \textit{\textsanskrit{Asādhāraṇaṁ} na katthaci}, “Offenses not in common are not committed anywhere.” } 

(1)\marginnote{51.33} There are offenses that one commits in inhabited areas, not in the wilderness.\footnote{Sp 5.324: \textit{\textsanskrit{Gāmacatukke} \textsanskrit{antaragharapaṭisaṁyuttaṁ} \textsanskrit{sekhiyapaññattiṁ} \textsanskrit{gāme} \textsanskrit{āpajjati} no \textsanskrit{araññe}}, “In the tetrad on inhabited areas, regarding the rules of training connected with inhabited areas, one commits an offense in an inhabited area, not in the wilderness.” For an explanation of the rendering “inhabited area” for \textit{\textsanskrit{gāma}}, see Appendix of Technical Terms. } (2) There are offenses that one commits in the wilderness, not in inhabited areas.\footnote{Sp 5.324: \textit{\textsanskrit{Bhikkhunī} \textsanskrit{aruṇaṁ} \textsanskrit{uṭṭhāpayamānā} \textsanskrit{araññe} \textsanskrit{āpajjati} no \textsanskrit{gāme}}, “A nun who lets the dawn arise commits an offense in the wilderness, not in an inhabited area.” This is presumably a reference to \href{https://suttacentral.net/pli-tv-bi-vb-ss3/en/brahmali\#4.14.1}{Bi Ss 3:4.14.1}, according to which a \textit{\textsanskrit{bhikkhunī}} cannot spend the night by herself in the wilderness. The offense is committed at dawn. } (3) There are offenses that one commits both in inhabited areas and in the wilderness.\footnote{Sp 5.324: \textit{\textsanskrit{Musāvādādiṁ} \textsanskrit{gāme} ceva \textsanskrit{āpajjati} \textsanskrit{araññe} ca}, “In lying, etc., one commits an offense both in an inhabited area and in the wilderness.” } (4) There are offenses that one commits neither in inhabited areas nor in the wilderness.\footnote{Sp 5.324: \textit{\textsanskrit{Asādhāraṇaṁ} na katthaci}, “Offenses not in common are not committed anywhere.” } 

There\marginnote{52.1} are four kinds of accusing: pointing out the action that is the basis for an offense, pointing out the offense, refusing to live together, refusing to act respectfully. — There are four kinds of preliminary actions. —\footnote{Sp 5.324: \textit{“\textsanskrit{Chandapārisuddhiutukkhānaṁ} \textsanskrit{bhikkhugaṇanā} ca \textsanskrit{ovādo}”ti ime pana “\textsanskrit{cattāro} \textsanskrit{pubbakiccā}”ti \textsanskrit{veditabbā}}, “Consent and purity, announcing the season, counting the monks, and the instruction—these are to be understood as the four kinds of preliminary actions.” } There are four kinds of readiness. —\footnote{Sp 5.324: \textit{\textsanskrit{Cattāro} \textsanskrit{pattakallāti} uposatho \textsanskrit{yāvatikā} ca \textsanskrit{bhikkhū} \textsanskrit{kammappattā} te \textsanskrit{āgatā} honti, \textsanskrit{sabhāgāpattiyo} na vijjanti, \textsanskrit{vajjanīyā} ca \textsanskrit{puggalā} \textsanskrit{tasmiṁ} na honti, pattakallanti \textsanskrit{vuccatīti}}, “The four kinds of readiness: it is the observance day, all the monks who should be present have arrived, there are no shared offenses, and there is no-one there who should not be present—this is called ‘readiness’.” } There are four offenses entailing confession concerning “no other”. —\footnote{Sp 5.324: \textit{\textsanskrit{Cattāri} \textsanskrit{anaññapācittiyānīti} “etadeva \textsanskrit{paccayaṁ} \textsanskrit{karitvā} \textsanskrit{anaññaṁ} \textsanskrit{pācittiya}”nti \textsanskrit{evaṁ} \textsanskrit{vuttāni} \textsanskrit{anupakhajjaseyyākappanasikkhāpadaṁ} “\textsanskrit{ehāvuso} \textsanskrit{gāmaṁ} \textsanskrit{vā} \textsanskrit{nigamaṁ} \textsanskrit{vā}”ti \textsanskrit{sikkhāpadaṁ}, \textsanskrit{sañcicca} \textsanskrit{kukkuccaupadahanaṁ}, \textsanskrit{upassutitiṭṭhananti} \textsanskrit{imāni} \textsanskrit{cattāri}}, “The four offenses entailing confession concerning ‘no other’: those spoken like this: ‘Having done it for this reason and no other, there is an offense entailing confession’, that is, the training rule on encroaching with a sleeping place, the training rule on ‘Come to the town or village’, intentionally giving rise to anxiety, and eavesdropping—these four.” } There are four kinds of approval from the monks. —\footnote{Sp 5.324: \textit{Catasso bhikkhusammutiyoti “ekarattampi ce bhikkhu \textsanskrit{ticīvarena} vippavaseyya \textsanskrit{aññatra} \textsanskrit{bhikkhusammutiyā}, \textsanskrit{aññaṁ} \textsanskrit{navaṁ} \textsanskrit{santhataṁ} \textsanskrit{kārāpeyya} \textsanskrit{aññatra} \textsanskrit{bhikkhusammutiyā}, tato ce uttari vippavaseyya \textsanskrit{aññatra} \textsanskrit{bhikkhusammutiyā}, \textsanskrit{duṭṭhullaṁ} \textsanskrit{āpattiṁ} anupasampannassa \textsanskrit{āroceyya} \textsanskrit{aññatra} \textsanskrit{bhikkhusammutiyā}”ti}, “The four kinds of approval from the monks: if a monk stays apart from his three robes even for a single day, except with the approval of the monks; if he makes another new blanket, except with the approval of the monks; if he stays apart longer than that, except with the approval of the monks; if he informs a person who is not fully ordained of a grave offense, except with the approval of the monks.” } There are four ways of acting that are wrong: one is biased by desire, ill will, confusion, or fear. — There are four ways of acting that are not wrong: one is not biased by desire, ill will, confusion, or fear. — When a shameless monk has four qualities, he causes a schism in the Sangha: he is biased by desire, ill will, confusion, or fear. — When a good monk has four qualities, he unites a divided Sangha: he is not biased by desire, ill will, confusion, or fear. — When a monk has four qualities, he should not be asked about the Monastic Law: he is biased by desire, ill will, confusion, or fear. — When a monk has four qualities, he should not ask about the Monastic Law: he is biased by desire, ill will, confusion, or fear. — When a monk has four qualities, his questions about the Monastic Law should not be replied to: he is biased by desire, ill will, confusion, or fear. — When a monk has four qualities, he should not reply to questions about the Monastic Law: he is biased by desire, ill will, confusion, or fear. — When a monk has four qualities, he should not be allowed to ask questions:\footnote{Sp 5.323: \textit{Anuyogo na \textsanskrit{dātabboti} “\textsanskrit{idaṁ} \textsanskrit{kappatī}”ti pucchantassa \textsanskrit{pucchāya} \textsanskrit{okāso} na \textsanskrit{dātabbo}, “\textsanskrit{aññaṁ} \textsanskrit{pucchā}”ti vattabbo. Iti so neva pucchitabbo \textsanskrit{nāssa} \textsanskrit{pucchā} \textsanskrit{sotabbāti} attho}, “\textit{Anuyogo na \textsanskrit{dātabbo}}: when he asks, ‘Is this allowable?’ he should be not be given the opportunity to question. They should say, ‘Ask someone else.’ In this way, he should neither be asked nor should his questions be listened to. This is the meaning.” } he is biased by desire, ill will, confusion, or fear. — When a monk has four qualities, you should not discuss the Monastic Law with him: he is biased by desire, ill will, confusion, or fear. — There are offenses that one commits when sick, not when not sick;\footnote{Sp 5.324: \textit{\textsanskrit{Gilānacatukke} \textsanskrit{aññabhesajjena} \textsanskrit{karaṇīyena} \textsanskrit{lolatāya} \textsanskrit{aññaṁ} \textsanskrit{viññāpento} \textsanskrit{gilāno} \textsanskrit{āpajjati}}, “In the tetrad on sickness, when one needs one kind of medicine, but because of greed asks for another, one commits an offense when sick.” } there are offenses that one commits when not sick, not when sick;\footnote{Sp 5.324: \textit{\textsanskrit{Abhesajjakaraṇīyena} \textsanskrit{bhesajjaṁ} \textsanskrit{viññāpento} \textsanskrit{agilāno} \textsanskrit{āpajjati}}, “When one needs something other than medicine, yet asks for medicine, one commits an offense when not sick.” } there are offenses that one commits both when sick and when not sick;\footnote{Sp 5.324: \textit{\textsanskrit{Musāvādādiṁ} ubhopi \textsanskrit{āpajjanti}}, “When lying, etc., one commits an offense at both times.” } there are offenses that one commits neither when sick nor when not sick. —\footnote{Sp 5.324: \textit{\textsanskrit{Asādhāraṇaṁ} ubhopi \textsanskrit{nāpajjanti}}, “Offenses not in common are not committed at either time.” } There are four kinds of illegitimate cancellations of the Monastic Code. — There are four kinds of legitimate cancellations of the Monastic Code. 

\scend{The section on fours is finished. }

\scuddanaintro{This is the summary: }

\begin{scuddana}%
“Through\marginnote{55.1} one’s own speech, by body, \\
While sleeping, unintentionally; \\
And while committing, through action, \\
And so four on speech. 

The\marginnote{56.1} monks have, and the nuns have, \\
And requisite, in the presence of; \\
Unknowingly, by body, and in the midst of, \\
And so twofold on clears. 

When\marginnote{57.1} one gets it, accusing, \\
And it is called probation; \\
Trial period, and also undertaking, \\
Unique things, received. 

Foul\marginnote{58.1} edibles, legal procedures, \\
Again legal procedures, failures; \\
Legal issues, and immoral ones, \\
Shining, and by a new arrival. 

One\marginnote{59.1} departing, variety in the action, \\
Shared, and with preceptor; \\
Teacher, or reasons, \\
Bad conduct, good conduct. 

Taking\marginnote{60.1} for oneself, and people, \\
Who deserves a seat; \\
And at the right time, and it is allowable, \\
Outside the central Ganges plain, allowable. 

Inside,\marginnote{61.1} and inside the monastery zone, \\
And in an inhabited area, and with accusing; \\
Preliminary action, readiness, \\
‘No other’, and approvals. 

Wrong\marginnote{62.1} acting, and not wrong acting, \\
Shameless, and with good; \\
And two on should be asked, \\
And another two on should reply; \\
And question, discussion, \\
Sick, and with cancellation.” 

%
\end{scuddana}

\section*{5. The section on fives }

“There\marginnote{63.1} are five kinds of offenses. —\footnote{At \href{https://suttacentral.net/pli-tv-pvr4/en/brahmali\#3.2}{Pvr 4:3.2} these are said to be the offenses entailing expulsion, the offenses entailing suspension, the offenses entailing confession, the offenses entailing acknowledgment, and the offenses of wrong conduct. } There are five classes of offenses. —\footnote{As for the previous item. } There are five grounds of training. —\footnote{At \href{https://suttacentral.net/pli-tv-pvr4/en/brahmali\#5.2}{Pvr 4:5.2} these are said to be the refraining from, the keeping away from, the desisting from, the abstaining from, the non-doing of, the non-performing of, the non-committing of, the non-transgressing the boundary of, the incapability with respect to the five classes of offenses. } There are five kinds of actions with results in the next life. —\footnote{Vibh 941: \textit{\textsanskrit{Mātā} \textsanskrit{jīvitā} \textsanskrit{voropitā} hoti, \textsanskrit{pitā} \textsanskrit{jīvitā} voropito hoti, arahanto \textsanskrit{jīvitā} voropito hoti, \textsanskrit{duṭṭhena} cittena \textsanskrit{tathāgatassa} \textsanskrit{lohitaṁ} \textsanskrit{uppāditaṁ} hoti, \textsanskrit{saṅgho} bhinno hoti – \textsanskrit{imāni} \textsanskrit{pañca} \textsanskrit{kammāni} \textsanskrit{ānantarikāni}}, “Killing one’s mother; killing one’s father; killing a perfected one; with a malicious mind, causing the Buddha to bleed; causing a schism in the Sangha—these five actions have results in the next life.” } There are five kinds of people with fixed rebirth. —\footnote{Sp 5.325: \textit{\textsanskrit{Pañcakesu} \textsanskrit{pañca} \textsanskrit{puggalā} \textsanskrit{niyatāti} \textsanskrit{ānantariyānamevetaṁ} \textsanskrit{gahaṇaṁ}}, “Five kinds of people with fixed rebirth: this is just a reference to having results in the next life.” } There are five offenses involving cutting. —\footnote{Sp 5.325: \textit{\textsanskrit{Pañca} \textsanskrit{chedanakā} \textsanskrit{āpattiyo} \textsanskrit{nāma} \textsanskrit{pamāṇātikkante} \textsanskrit{mañcapīṭhe} \textsanskrit{nisīdanakaṇḍuppaṭicchādivassikasāṭikāsu} \textsanskrit{sugatacīvare} ca \textsanskrit{veditabbā}}, “The five offenses involving cutting are to be known as: a bed or bench exceeding the right size; a sitting mat exceeding the right size; an itch-covering cloth exceeding the right size; a rainy-season robe exceeding the right size; a standard robe exceeding the right size.” That is, \href{https://suttacentral.net/pli-tv-bu-vb-pc87/en/brahmali\#1.11.1}{Bu Pc 87:1.11.1} and \href{https://suttacentral.net/pli-tv-bu-vb-pc89/en/brahmali\#2.10.1}{Bu Pc 89:2.10.1}–92. } There are five reasons for committing an offense. —\footnote{Sp 5.325: \textit{\textsanskrit{Pañcahākārehīti} \textsanskrit{alajjitā}, \textsanskrit{aññāṇatā}, \textsanskrit{kukkuccappakatatā}, akappiye \textsanskrit{kappiyasaññitā}, kappiye \textsanskrit{akappiyasaññitāti} imehi \textsanskrit{pañcahi}}, “The five reasons: shamelessness, ignorance, an anxious character, perceiving what is allowable as unallowable, perceiving what is unallowable as allowable—these are the five.” } There are five kinds of offenses because of lying.\footnote{The commentary has no full stop after \textit{\textsanskrit{āpattiyo}}, but instead after the next word, \textit{\textsanskrit{musāvāda}}. Since \textit{\textsanskrit{pañca} \textsanskrit{āpattiyo}} is mentioned as a separate item above, I here follow the commentarial punctuation. Sp 5.325: \textit{\textsanskrit{Pañca} \textsanskrit{āpattiyo} \textsanskrit{musāvādapaccayāti} \textsanskrit{pārājikathullaccayadukkaṭasaṅghādisesapācittiyā}}, “The five kinds of offenses because of lying: offenses entailing expulsion, serious offenses, offenses of wrong conduct, offenses entailing suspension, and offenses entailing confession.” } 

There\marginnote{63.9} are five reasons why a legal procedure is invalid: (1) one does not do the legal procedure oneself; (2) one does not request someone else; (3) one does not give one’s consent or declare one’s purity; (4) one objects while the legal procedure is being carried out; or (5) one has the view that the completed legal procedure is illegitimate. 

There\marginnote{63.11} are five reasons why a legal procedure is valid: (1) one does the legal procedure oneself; (2) one requests someone else; (3) one gives one’s consent or declares one’s purity; (4) one does not object while the legal procedure is being carried out; or (5) one has the view that the completed legal procedure is legitimate. 

There\marginnote{63.13} are five things that are allowable for a monk who only eats almsfood: (1) visiting families before or after a meal invitation, (2) eating in a group, (3) eating a meal before another, (4) non-determination, (5) non-assignment to another.\footnote{Sp 5.325: \textit{\textsanskrit{Anadhiṭṭhānanti} “\textsanskrit{gaṇabhojane} \textsanskrit{aññatra} \textsanskrit{samayā}”ti \textsanskrit{vuttaṁ} \textsanskrit{samayaṁ} \textsanskrit{adhiṭṭhahitvā} \textsanskrit{bhojanaṁ} \textsanskrit{adhiṭṭhānaṁ} \textsanskrit{nāma}; \textsanskrit{tathā} \textsanskrit{akaraṇaṁ} \textsanskrit{anadhiṭṭhānaṁ}. \textsanskrit{Avikappanā} \textsanskrit{nāma} \textsanskrit{yā} paramparabhojane \textsanskrit{vikappanā} \textsanskrit{vuttā}, \textsanskrit{tassā} \textsanskrit{akaraṇaṁ}}, “Non-determination: since it is said, ‘Eating in a group, except on an appropriate occasion’, then, determining the appropriate occasion is called determining a meal. Not doing that is non-determination. Non-assignment to another: the non-doing of what is called transfer in the rule on eating a meal before another.” The former concerns \href{https://suttacentral.net/pli-tv-bu-vb-pc32/en/brahmali\#8.15.1}{Bu Pc 32:8.15.1}, the latter \href{https://suttacentral.net/pli-tv-bu-vb-pc33/en/brahmali\#3.15.1}{Bu Pc 33:3.15.1}. } 

When\marginnote{63.15} a monk has five qualities, whether he is a bad monk or firm in morality, he is suspected and mistrusted: he regularly associates with sex workers, widows, single women, \textit{\textsanskrit{paṇḍakas}}, or nuns. —\footnote{For a discussion of the word \textit{\textsanskrit{paṇḍaka}}, see Appendix of Technical Terms. } There are five kinds of oil: sesame oil, mustard oil, honey-tree oil, castor oil, and oil from fat. — There are five kinds of fat: bear fat, fish fat, alligator fat, pig fat, and donkey fat. — There are five kinds of losses: loss of relatives, loss of property, loss of health, loss of morality, and loss of view. — There are five kinds of successes: success in relatives, success in property, success in health, success in morality, and success in view. 

There\marginnote{63.25} are five reasons why the formal support from a preceptor comes to an end: (1) the preceptor goes away; (2) the preceptor disrobes; (3) the preceptor dies; (4) the preceptor joins another religion or sect; or (5) the preceptor says so.\footnote{For an explanation of the rendering “disrobes” for \textit{vibbhanto}, see \textit{vibbhamati} in Appendix of Technical Terms. } 

There\marginnote{63.27} are five kinds of people who should not be given the full ordination: (1) one lacking in age, (2) one lacking in limbs, (3) one who is deficient as object, (4) one who has acted wrongly, (5) one who is incomplete.\footnote{Sp 5.322: \textit{Vatthuvipanno \textsanskrit{nāma} \textsanskrit{paṇḍako} \textsanskrit{tiracchānagato} \textsanskrit{ubhatobyañjanako} ca. \textsanskrit{Avasesā} \textsanskrit{theyyasaṁvāsakādayo} \textsanskrit{aṭṭha} \textsanskrit{abhabbapuggalā} \textsanskrit{karaṇadukkaṭakā} \textsanskrit{nāma}}, “A \textit{\textsanskrit{paṇḍaka}}, an animal, and a hermaphrodite are called deficient as object. The remaining eight incapable persons, starting with the fake monk, are called those who have acted wrongly.” Sp 5.322: \textit{\textsanskrit{Aparipūro} \textsanskrit{nāma} \textsanskrit{aparipuṇṇapattacīvaro}}, “One who is incomplete: not complete in bowl and robes.” } 

There\marginnote{63.29} are five kinds of rags: (1) those from a charnel ground, (2) those from a shop, (3) those eaten by rats, (4) those eaten by termites, (5) those burned by fire. 

There\marginnote{63.31} are five other kinds of rags: (1) those chewed by cattle, (2) those chewed by goats, (3) those left at a stupa, (4) those discarded from a king’s consecration, (5) those taken to and then brought back from a charnel ground.\footnote{Sp 5.325: \textit{\textsanskrit{Thūpacīvaranti} \textsanskrit{vammikaṁ} \textsanskrit{parikkhipitvā} \textsanskrit{balikammakataṁ}. Ābhisekikanti \textsanskrit{nahānaṭṭhāne} \textsanskrit{vā} \textsanskrit{rañño} \textsanskrit{abhisekaṭṭhāne} \textsanskrit{vā} \textsanskrit{chaḍḍitacīvaraṁ}. \textsanskrit{Bhatapaṭiyābhatanti} \textsanskrit{susānaṁ} \textsanskrit{netvā} puna \textsanskrit{ānītakaṁ}}, “\textit{\textsanskrit{Thūpacīvara}}: having encircled a hill, it is done as an offering. \textit{Ābhisekika}: a cloth discarded at a bathing place or at the place of a king’s consecration. \textit{\textsanskrit{Bhatapaṭiyābhata}}: having taken it to a charnel ground, it is brought back again.” Sp-yoj 5.325: \textit{\textsanskrit{Thūpacīvaranti} ettha \textsanskrit{thūpaṁ} \textsanskrit{parikkhipitvā} \textsanskrit{kataṁ} \textsanskrit{cīvaraṁ} \textsanskrit{thūpacīvaranti} dassento \textsanskrit{āha} “vammika”\textsanskrit{ntiādi}}, “\textit{\textsanskrit{Thūpacīvara}}: here, having encircled a stupa, the expression \textit{\textsanskrit{thūpacīvaraṁ}} referring to a finished robe, ‘hill’ is said, etc.” } 

There\marginnote{63.33} are five kinds of removing: removing by theft, removing by force, removing dependent on conditions, removing by concealing, removing by drawing lots. — There are five notorious gangsters to be found in the world. —\footnote{See \href{https://suttacentral.net/pli-tv-bu-vb-pj4/en/brahmali\#1.3.1}{Bu Pj 4:1.3.1}. } There are five things not to be given away. —\footnote{See \href{https://suttacentral.net/pli-tv-kd16/en/brahmali\#15.2.1}{Kd 16:15.2.1}. } There are five things not to be distributed. —\footnote{See \href{https://suttacentral.net/pli-tv-kd16/en/brahmali\#16.2.13}{Kd 16:16.2.13}. } There are five kinds of offenses that originate from body, not from speech and mind. —\footnote{Sp 5.325: \textit{\textsanskrit{Pañcāpattiyo} \textsanskrit{kāyato} \textsanskrit{samuṭṭhantīti} \textsanskrit{paṭhamena} \textsanskrit{āpattisamuṭṭhānena} \textsanskrit{pañca} \textsanskrit{āpattiyo} \textsanskrit{āpajjati}, “bhikkhu \textsanskrit{kappiyasaññī} \textsanskrit{saññācikāya} \textsanskrit{kuṭiṁ} \textsanskrit{karotī}”ti \textsanskrit{evaṁ} \textsanskrit{antarapeyyāle} \textsanskrit{vuttāpattiyo}}, “The five kinds of offenses that originate from the body: through the first origination of offenses, one commits five offenses. These are the offenses spoken of in the internal repetition in this way: ‘When a monk, perceiving it as allowable and by means of begging, builds a hut.’” The internal repetition is at \href{https://suttacentral.net/pli-tv-pvr4/en/brahmali\#0.1}{Pvr 4}. } There are five kinds of offenses that originate from body and speech, not from mind. —\footnote{Sp 5.325: \textit{\textsanskrit{Pañca} \textsanskrit{āpattiyo} \textsanskrit{kāyato} ca \textsanskrit{vācato} \textsanskrit{cāti} tatiyena \textsanskrit{āpattisamuṭṭhānena} \textsanskrit{pañca} \textsanskrit{āpattiyo} \textsanskrit{āpajjati}, “bhikkhu \textsanskrit{kappiyasaññī} \textsanskrit{saṁvidahitvā} \textsanskrit{kuṭiṁ} \textsanskrit{karotī}”ti \textsanskrit{evaṁ} tattheva \textsanskrit{vuttā} \textsanskrit{āpattiyo}}, “The five kinds of offenses that originate from the body and speech: through the third origination of offenses, one commits five offenses. These are the offenses spoken of just there in this way: ‘When a monk, perceiving it as allowable and having appointed someone, builds a hut.’” Again, this refers to the internal repetition. } There are five kinds of offenses that are confessable. —\footnote{Sp 5.325: \textit{\textsanskrit{Desanāgāminiyoti} \textsanskrit{ṭhapetvā} \textsanskrit{pārājikañca} \textsanskrit{saṅghādisesañca} \textsanskrit{avasesā}}, “Confessable: apart from the offenses entailing expulsion and the offenses entailing suspension, the rest.” } There are five kinds of sangha. —\footnote{See \href{https://suttacentral.net/pli-tv-kd9/en/brahmali\#4.1.1}{Kd 9:4.1.1}. } There are five ways of reciting the Monastic Code. —\footnote{See \href{https://suttacentral.net/pli-tv-kd2/en/brahmali\#15.1.4}{Kd 2:15.1.4}. } Outside the central Ganges plain, the full ordination is to be given by a group of five, including one expert on the Monastic Law. — There are five benefits of performing the robe-making ceremony. —\footnote{See \href{https://suttacentral.net/pli-tv-kd7/en/brahmali\#1.3.4}{Kd 7:1.3.4}. } There are five kinds of legal procedures. —\footnote{Sp 5.325: \textit{\textsanskrit{Pañca} \textsanskrit{kammānīti} \textsanskrit{tajjanīyaniyassapabbājanīyapaṭisāraṇīyāni} \textsanskrit{cattāri} \textsanskrit{ukkhepanīyañca} tividhampi ekanti \textsanskrit{pañca}}, “The five kinds of legal procedures: condemnation, demotion, banishment, and reconciliation are four, and the threefold ejection is one, making it five.” } There are five ‘for the third time’ offenses. —\footnote{Sp 5.325: \textit{\textsanskrit{Yāvatatiyake} \textsanskrit{pañcāti} \textsanskrit{ukkhittānuvattikāya} \textsanskrit{bhikkhuniyā} \textsanskrit{yāvatatiyaṁ} \textsanskrit{samanubhāsanāya} \textsanskrit{appaṭinissajjantiyā} \textsanskrit{pārājikaṁ} \textsanskrit{thullaccayaṁ} \textsanskrit{dukkaṭanti} tisso, \textsanskrit{bhedakānuvattakādisamanubhāsanāsu} \textsanskrit{saṅghādiseso}, \textsanskrit{pāpikāya} \textsanskrit{diṭṭhiyā} \textsanskrit{appaṭinissagge} \textsanskrit{pācittiyaṁ}}, “The five ‘for the third time’: when a nun takes sides with one who has been ejected and does not stop when pressed for the third time, there are three offenses: the offense entailing expulsion, the serious offense, and the offense of wrong conduct. When pressing one who is supporting a schism, etc., there is an offense entailing suspension. For not giving up a bad view, there is an offense entailing confession.” } When five factors are fulfilled, there is an offense entailing expulsion for one who steals. —\footnote{See \href{https://suttacentral.net/pli-tv-bu-vb-pj2/en/brahmali\#6.1.1}{Bu Pj 2:6.1.1}. } When five factors are fulfilled, there is a serious offense for one who steals. —\footnote{See \href{https://suttacentral.net/pli-tv-bu-vb-pj2/en/brahmali\#6.1.9}{Bu Pj 2:6.1.9}. } When five factors are fulfilled, there is an offense of wrong conduct for one who steals. —\footnote{See \href{https://suttacentral.net/pli-tv-bu-vb-pj2/en/brahmali\#6.1.17}{Bu Pj 2:6.1.17}. } There are five unallowable things that should not be used: what has not been given, what is not known about, what is not allowable, what has not been received, what has not been made left over. —\footnote{Sp 5.325: \textit{Aviditanti \textsanskrit{paṭiggaṇhāmīti} \textsanskrit{cetanāya} \textsanskrit{abhāvena} \textsanskrit{aviditaṁ}}, “\textit{Avidita}: not known about because of the absence of the volition ‘I receive’.” The point may simply be that one does not know whether a certain food has been received or not. } There are five allowable things that may be used: what has been given, what is known about, what is allowable, what has been received, what has been made left over. — There are five gifts without merit that are considered meritorious in the world: alcohol, entertainment, a woman, a bull, and a picture. — There are five things that are hard to remove: desire, ill will, confusion, feeling inspired to speak, and the thought of departing.\footnote{See also \href{https://suttacentral.net/an5.160/en/brahmali\#1.1}{AN 5.160:1.1}. } 

There\marginnote{63.58} are five benefits of sweeping: (1) one’s own mind becomes serene, (2) the minds of others become serene, (3) the gods are pleased, (4) one accumulates actions that lead to being inspiring, (5) at the break-up of the body after death one is reborn in heaven. 

There\marginnote{63.60} are five other benefits of sweeping: (1) one’s own mind becomes serene, (2) the minds of others become serene, (3) the gods are pleased, (4) one carries out the Teacher’s instruction, (5) later generations follow one’s example. 

When\marginnote{64.1} an expert on the Monastic Law has five qualities, he is considered ignorant: (1) he does not grasp what is proper for himself to say; (2) he does not grasp what is proper for others to say; (3-5) not grasping either, he makes them act illegitimately, without their admission.\footnote{Sp 5.325: \textit{Attano \textsanskrit{bhāsapariyantaṁ} na \textsanskrit{uggaṇhātīti} “\textsanskrit{imasmiṁ} \textsanskrit{vatthusmiṁ} \textsanskrit{ettakaṁ} \textsanskrit{suttaṁ} upalabbhati, ettako vinicchayo, \textsanskrit{ettakaṁ} \textsanskrit{suttañca} \textsanskrit{vinicchayañca} \textsanskrit{vakkhāmī}”ti \textsanskrit{evaṁ} attano \textsanskrit{bhāsapariyantaṁ} na \textsanskrit{uggaṇhāti}. “\textsanskrit{Ayaṁ} codakassa \textsanskrit{purimakathā}, \textsanskrit{ayaṁ} \textsanskrit{pacchimakathā}, \textsanskrit{ayaṁ} cuditakassa \textsanskrit{purimakathā}, \textsanskrit{ayaṁ} \textsanskrit{pacchimakathā}, \textsanskrit{ettakaṁ} \textsanskrit{gayhūpagaṁ}, \textsanskrit{ettakaṁ} na \textsanskrit{gayhūpaga}”nti \textsanskrit{evaṁ} \textsanskrit{anuggaṇhanto} pana parassa \textsanskrit{bhāsapariyantaṁ} na \textsanskrit{uggaṇhāti} \textsanskrit{nāma}}, “\textit{Attano \textsanskrit{bhāsapariyantaṁ} na \textsanskrit{uggaṇhāti}}: ‘In regard to this action that is a basis for an offense, this is the rule and this is its application. I will speak about the rule and its application.’ He does not grasp, in such a way, what is proper for himself to say. ‘This was said first by the accuser, this was said last. This was said first by the accused, this was said last. This is fit to be taken up, this is not.’ If he does not grasp, in such a way, it is called, ‘He does not grasp what is proper for others to say’.” In the Suttas, \textit{pariyanta}, when used in combination with speech, often refers to conciseness of speech, e.g. in the gradual training at \href{https://suttacentral.net/mn27/en/brahmali\#13.7}{MN 27}. Here the context requires a different meaning. } 

But\marginnote{64.3} when an expert on the Monastic Law has five qualities, he is considered learned: (1) he grasps what is proper for himself to say; (2) he grasps what is proper for others to say; (3-5) grasping both, he makes them act legitimately, in accordance with their admission. 

When\marginnote{64.5} an expert on the Monastic Law has five other qualities, he is considered ignorant: (1) he does not know the offenses; (2) he does not know the roots of the offenses; (3) he does not know the origins of the offenses; (4) he does not know the ending of the offenses; (5) he does not know the path leading to the ending of the offenses. 

But\marginnote{64.7} when an expert on the Monastic Law has five qualities, he is considered learned: (1) he knows the offenses; (2) he knows the roots of the offenses; (3) he knows the origins of the offenses; (4) he knows the ending of the offenses; (5) he knows the path leading to the ending of the offenses. 

When\marginnote{64.9} an expert on the Monastic Law has five other qualities, he is considered ignorant: (1) he does not know the legal issues; (2) he does not know the roots of the legal issues; (3) he does not know the origins of the legal issues; (4) he does not know the ending of the legal issues; (5) he does not know the path leading to the ending of the legal issues. 

But\marginnote{64.11} when an expert on the Monastic Law has five qualities, he is considered learned: (1) he knows the legal issues; (2) he knows the roots of the legal issues; (3) he knows the origins of the legal issues; (4) he knows the ending of the legal issues; (5) he knows the path leading to the ending of the legal issues. 

When\marginnote{64.13} an expert on the Monastic Law has five other qualities, he is considered ignorant: (1) he does not know the actions that are the bases for offenses; (2) he does not know the origin stories; (3) he does not know the rules; (4) he does not know the additions to the rules; (5) he does not know the sequence of statements.\footnote{Sp 5.359: \textit{\textsanskrit{Anusandhivacanapathaṁ} na \textsanskrit{jānātīti} \textsanskrit{kathānusandhivacanaṁ} \textsanskrit{vinicchayānusandhivacanañca} na \textsanskrit{jānāti}}, “‘He does not know the sequence of statements’: he does not know the sequence of statements and the sequence of decisions.” } 

But\marginnote{64.15} when an expert on the Monastic Law has five qualities, he is considered learned: (1) he knows the actions that are the bases for offenses; (2) he knows the origin stories; (3) he knows the rules; (4) he knows the additions to the rules; (5) he knows the sequence of statements. 

When\marginnote{64.17} an expert on the Monastic Law has five other qualities, he is considered ignorant: (1) he does not know the motion; (2) he does not know how the motion is done; (3) he is not skilled in what comes first; (4) he is not skilled in what comes afterwards; (5) he does not know the right time.\footnote{Sp 5.325: \textit{Na pubbakusalo hoti na aparakusaloti pubbe \textsanskrit{kathetabbañca} \textsanskrit{pacchā} \textsanskrit{kathetabbañca} na \textsanskrit{jānāti}, \textsanskrit{ñatti} \textsanskrit{nāma} pubbe \textsanskrit{ṭhapetabbā}, \textsanskrit{pacchā} na \textsanskrit{ṭhapetabbātipi} na \textsanskrit{jānāti}}, “He is not skilled in what comes first; he is not skilled in what comes afterwards: he does not know what should be said first and what should be said afterwards. He does not know that the motion should be put forward first, not afterwards.” } 

But\marginnote{64.19} when an expert on the Monastic Law has five qualities, he is considered learned: (1) he knows the motion; (2) he knows how the motion is done; (3) he is skilled in what comes first; (4) he is skilled in what comes afterwards; (5) he knows the right time. 

When\marginnote{64.21} an expert on the Monastic Law has five other qualities, he is considered ignorant: (1) he does not know the offenses and non-offenses; (2) he does not know the light and heavy offenses; (3) he does not know the curable and incurable offenses; (4) he does not know the grave and minor offenses; (5) he has not well-learned and well-remembered his teachers’ tradition. 

But\marginnote{64.23} when an expert on the Monastic Law has five qualities, he is considered learned: (1) he knows the offenses and non-offenses; (2) he knows the light and heavy offenses; (3) he knows the curable and incurable offenses; (4) he knows the grave and minor offenses; (5) he has well-learned and well-remembered his teachers’ tradition. 

When\marginnote{64.25} an expert on the Monastic Law has five other qualities, he is considered ignorant: (1) he does not know the offenses and non-offenses; (2) he does not know the light and heavy offenses; (3) he does not know the curable and incurable offenses; (4) he does not know the grave and minor offenses; (5) he has not properly learned either Monastic Codes in detail, and he has not analyzed them well, thoroughly mastered them, or investigated them well, either in terms of the rules or their detailed exposition. 

But\marginnote{64.27} when an expert on the Monastic Law has five qualities, he is considered learned: (1) he knows the offenses and non-offenses; (2) he knows the light and heavy offenses; (3) he knows the curable and incurable offenses; (4) he knows the grave and minor offenses; (5) he has properly learned both Monastic Codes in detail, and has analyzed them well, thoroughly mastered them, and investigated them well, both in terms of the rules and their detailed exposition. 

When\marginnote{64.29} an expert on the Monastic Law has five other qualities, he is considered ignorant: (1) he does not know the offenses and non-offenses; (2) he does not know the light and heavy offenses; (3) he does not know the curable and incurable offenses; (4) he does not know the grave and minor offenses; (5) he is not skilled in deciding legal issues. 

But\marginnote{64.31} when an expert on the Monastic Law has five qualities, he is considered learned: (1) he knows the offenses and non-offenses; (2) he knows the light and heavy offenses; (3) he knows the curable and incurable offenses; (4) he knows the grave and minor offenses; (5) he is skilled in deciding legal issues. 

There\marginnote{65.1} are five kinds of wilderness dwellers: one who is a wilderness dweller (1) because of stupidity and folly; (2) because one is overcome by bad desires; (3) because of insanity and derangement; (4) because it is praised by the Buddhas and their disciples; or (5) because of fewness of wishes, contentment, self-effacement, seclusion, and not needing anything else.\footnote{Sp 5.325: \textit{\textsanskrit{Taṁ} \textsanskrit{idamatthitaṁyeva} \textsanskrit{nissāya} na \textsanskrit{aññaṁ} \textsanskrit{kiñci} \textsanskrit{lokāmisanti} attho}, “They depend on ‘not needing anything else’, not on any other material things. This is the meaning.” } 

There\marginnote{65.3} are five kinds of people who only eat almsfood … There are five kinds of rag-robe wearers … There are five kinds of people who live at the foot of a tree … There are five kinds of people who live in charnel grounds … There are five kinds of people who live out in the open …\footnote{For an explanation of the rendering “out in the open” for \textit{\textsanskrit{abbhokāsa}}, see Appendix of Technical Terms. } There are five kinds of people who only have three robes … There are five kinds of people who go on continuous almsround … There are five kinds of people who never lie down … There are five kinds of people who accept any kind of resting place …\footnote{Nidd-a.I.17: \textit{Yadeva \textsanskrit{santhataṁ} \textsanskrit{yathāsanthataṁ}, \textsanskrit{idaṁ} \textsanskrit{tuyhaṁ} \textsanskrit{pāpuṇātīti} \textsanskrit{evaṁ} \textsanskrit{paṭhamaṁ} \textsanskrit{uddiṭṭhasenāsanassetaṁ} \textsanskrit{adhivacanaṁ}}, “\textit{\textsanskrit{Yathāsanthata}}: whatever is a mat. ‘This is for you’: in this way this is a description of the resting place one is designated first.” } There are five kinds of people who eat in one sitting per day … There are five kinds of people who refuse to accept food offered after the meal has begun … There are five kinds of people who eat only from the almsbowl: one who eats only from the almsbowl (1) because of stupidity and folly; (2) because one is overcome by bad desires; (3) because of insanity and derangement; (4) because it is praised by the Buddhas and their disciples; or (5) because of fewness of wishes, contentment, self-effacement, seclusion, and not needing anything else. 

When\marginnote{66.1} a monk has five qualities, he should not live without formal support: (1) he does not know about the observance-day ceremony; (2) he does not know the observance-day procedure; (3) he does not know the Monastic Code; (4) he does not know the recitation of the Monastic Code; (5) he has less than five years of seniority. 

But\marginnote{66.3} when a monk has five qualities, he may live without formal support: (1) he knows about the observance-day ceremony; (2) he knows the observance-day procedure; (3) he knows the Monastic Code; (4) he knows the recitation of the Monastic Code; (5) he has five or more years of seniority. 

When\marginnote{66.5} a monk has five other qualities, he should not live without formal support: (1) he does not know about the invitation ceremony; (2) he does not know the invitation procedure; (3) he does not know the Monastic Code; (4) he does not know the recitation of the Monastic Code; (5) he has less than five years of seniority. 

But\marginnote{66.7} when a monk has five qualities, he may live without formal support: (1) he knows about the invitation ceremony; (2) he knows the invitation procedure; (3) he knows the Monastic Code; (4) he knows the recitation of the Monastic Code; (5) he has five or more years of seniority. 

When\marginnote{66.9} a monk has five other qualities, he should not live without formal support: (1) he does not know the offenses and non-offenses; (2) he does not know the light and heavy offenses; (3) he does not know the curable and incurable offenses; (4) he does not know the grave and minor offenses; (5) he has less than five years of seniority. 

But\marginnote{66.11} when a monk has five qualities, he may live without formal support: (1) he knows the offenses and non-offenses; (2) he knows the light and heavy offenses; (3) he knows the curable and incurable offenses; (4) he knows the grave and minor offenses; (5) he has five or more years of seniority. 

When\marginnote{66.13} a nun has five qualities, she should not live without formal support: (1) she does not know about the observance-day ceremony; (2) she does not know the observance-day procedure; (3) she does not know the Monastic Code; (4) she does not know the recitation of the Monastic Code; (5) she has less than five years of seniority. 

But\marginnote{66.15} when a nun has five qualities, she may live without formal support: (1) she knows about the observance-day ceremony; (2) she knows the observance-day procedure; (3) she knows the Monastic Code; (4) she knows the recitation of the Monastic Code; (5) she has five or more years of seniority. 

When\marginnote{66.17} a nun has five other qualities, she should not live without formal support: (1) she does not know about the invitation ceremony; (2) she does not know the invitation procedure; (3) she does not know the Monastic Code; (4) she does not know the recitation of the Monastic Code; (5) she has less than five years of seniority. 

But\marginnote{66.19} when a nun has five qualities, she may live without formal support: (1) she knows about the invitation ceremony; (2) she knows the invitation procedure; (3) she knows the Monastic Code; (4) she knows the recitation of the Monastic Code; (5) she has five or more years of seniority. 

When\marginnote{66.21} a nun has five other qualities, she should not live without formal support: (1) she does not know the offenses and non-offenses; (2) she does not know the light and heavy offenses; (3) she does not know the curable and incurable offenses; (4) she does not know the grave and minor offenses; (5) she has less than five years of seniority. 

But\marginnote{66.23} when a nun has five qualities, she may live without formal support: (1) she knows the offenses and non-offenses; (2) she knows the light and heavy offenses; (3) she knows the curable and incurable offenses; (4) she knows the grave and minor offenses; (5) she has five or more years of seniority. 

There\marginnote{67.1} are five dangers in being uninspiring:\footnote{Sp 5.325: \textit{\textsanskrit{Apāsādikanti} \textsanskrit{kāyaduccaritādi} \textsanskrit{akusalakammaṁ} vuccati}, “Unwholesome actions of misconduct by body, etc., are called ‘being uninspiring’.” } (1) you criticize yourself; (2) after investigating, wise people condemn you; (3) you get a bad reputation; (4) you die confused; (5) after death, you are reborn in a lower realm. 

There\marginnote{67.3} are five benefits in being inspiring: (1) you do not criticize yourself; (2) after investigating, wise people praise you; (3) you get a good reputation; (4) you die unconfused; (5) after death, you are reborn in heaven. 

There\marginnote{67.5} are five other dangers in being uninspiring: (1) you do not give rise to confidence in those without it; (2) you cause some to lose their confidence; (3) you don’t carry out the Teacher’s instructions; (4) later generations follow your example; (5) your mind doesn’t become serene.\footnote{I read \textit{\textsanskrit{āpajjati}} with SRT, as against \textit{\textsanskrit{nāpajjati}} in MS. } 

And\marginnote{67.7} there are five benefits in being inspiring: (1) you give rise to confidence in those without it; (2) you increase the confidence of those who have it; (3) you carry out the Teacher’s instructions; (4) later generations follow your example; (5) your mind becomes serene. 

There\marginnote{67.9} are five dangers of associating with families: (1) one commits the offense of visiting families before or after a meal invitation; (2) one commits the offense of sitting down in private; (3) one commits the offense of sitting down on a concealed seat; (4) one commits the offense of teaching more than five or six sentences to a woman; (5) one has a lot of worldly thoughts.\footnote{The first four of these refer respectively to: \href{https://suttacentral.net/pli-tv-bu-vb-pc46/en/brahmali\#5.6.1}{Bu Pc 46:5.6.1}/\href{https://suttacentral.net/pli-tv-bi-vb-pc127/en/brahmali}{Bi Pc 127}; \href{https://suttacentral.net/pli-tv-bu-vb-pc45/en/brahmali\#1.14.1}{Bu Pc 45:1.14.1}/\href{https://suttacentral.net/pli-tv-bi-vb-pc126/en/brahmali}{Bi Pc 126}; \href{https://suttacentral.net/pli-tv-bu-vb-pc44/en/brahmali\#1.14.1}{Bu Pc 44:1.14.1}/\href{https://suttacentral.net/pli-tv-bi-vb-pc125/en/brahmali}{Bi Pc 125}; and \href{https://suttacentral.net/pli-tv-bu-vb-pc7/en/brahmali\#3.11.1}{Bu Pc 7:3.11.1}/\href{https://suttacentral.net/pli-tv-bi-vb-pc103/en/brahmali}{Bi Pc 103}. } 

There\marginnote{67.11} are five dangers for a monk who associates with families: (1) when he associates too much with families, he often sees women; (2) because of seeing them, he associates with them; (3) because of associating with them, there is intimacy; (4) because of intimacy, there is lust; (5) because his mind is overcome by lust, it is to be expected that he will be dissatisfied with the spiritual life, that he will commit a certain defiled offense, or that he will renounce the training and return to the lower life.\footnote{See also \href{https://suttacentral.net/an5.55/en/brahmali\#1.1}{AN 5.55:1.1}. } 

There\marginnote{68.1} are five kinds of propagation: (1) propagation from roots, (2) propagation from stems, (3) propagation from joints, (4) propagation from cuttings, (5) propagation from regular seeds as the fifth. 

When\marginnote{68.3} fruit is allowable for monastics for any of five reasons, it may be eaten: (1) it has been damaged by fire, (2) it has been damaged by a knife, (3) it has been damaged by a fingernail, (4) it’s seedless, (5) the seeds have been removed. 

There\marginnote{68.5} are five kinds of purification: (1) After reciting the introduction, the rest is announced as if heard. (2) After reciting the introduction and the four rules entailing expulsion, the rest is announced as if heard. (3) After reciting the introduction, the four rules entailing expulsion, and the thirteen rules entailing suspension, the rest is announced as if heard. (4) After reciting the introduction, the four rules entailing expulsion, the thirteen rules entailing suspension, and the two undetermined rules, the rest is announced as if heard. (5) In full is the fifth. 

There\marginnote{68.11} are five other kinds of purification: (1) the observance-day ceremony which consists of reciting the Monastic Code, (2) the observance-day ceremony which consists of declaring purity, (3) the observance-day ceremony which consists of a determination, (4) the invitation ceremony, (5) the observance-day ceremony for the sake of unity as the fifth. 

There\marginnote{68.13} are five benefits of being an expert on the Monastic Law: (1) your own morality is well guarded; (2) you are a refuge for those who are habitually anxious; (3) you speak with confidence in the midst of the Sangha; (4) you can legitimately and properly refute an opponent; (5) you are practicing for the longevity of the true Teaching. 

There\marginnote{68.15} are five kinds of illegitimate cancellations of the Monastic Code. —\footnote{See \href{https://suttacentral.net/pli-tv-kd19/en/brahmali\#3.3.25}{Kd 19:3.3.25}. } There are five kinds of legitimate cancellations of the Monastic Code.”\footnote{See \href{https://suttacentral.net/pli-tv-kd19/en/brahmali\#3.3.31}{Kd 19:3.3.31}. } 

\scend{The section on fives is finished. }

\scuddanaintro{This is the summary: }

\begin{scuddana}%
“Offense,\marginnote{71.1} classes of offenses, \\
Training, and with the next life; \\
People, and involving cutting, \\
And committing, because of. 

Invalid,\marginnote{72.1} and valid, \\
Allowable, suspected, and oil; \\
Fat, loss, successes, \\
Comes to an end, and with people. 

Charnel\marginnote{73.1} ground, and chewed, \\
Theft, and one called a gangster; \\
Not to be given away, not to be shared out, \\
From body, from body and speech. 

Confessable,\marginnote{74.1} Sangha, reciting, \\
Outside, and with the robe-making ceremony; \\
Legal procedures, for the third time, \\
An offense entailing expulsion, a serious offense, an offense of wrong conduct. 

Unallowable,\marginnote{75.1} and allowable, \\
Without merit, hard to remove; \\
Sweeping, and other, \\
To say, and also offenses. 

Legal\marginnote{76.1} issue, action that is the basis for an offense, motion, \\
And both offenses and non-offenses; \\
These are light and strong, \\
You should understand dark and bright. 

Wilderness,\marginnote{77.1} and almsfood, \\
Rag-robe, tree, people who live in charnel grounds; \\
Out in the open, and robe, \\
Continuous, one who never lies down. 

Resting\marginnote{78.1} place, also after, \\
And one who eats only from the bowl; \\
Observance-day ceremony, invitation ceremony, \\
And also offenses and non-offenses. 

These\marginnote{79.1} verses on dark and bright, \\
They are the same for the nuns; \\
Being uninspiring, being inspiring, \\
And so two others. 

Associating\marginnote{80.1} with families, too much, \\
Propagation, and allowable for monastics; \\
Purification, and another, \\
Monastic Law, and with illegitimate; \\
And so legitimate is spoken of, \\
The basic section on fives is finished.” 

%
\end{scuddana}

\section*{6. The section on sixes }

“There\marginnote{81.1} are six kinds of disrespect. —\footnote{See \href{https://suttacentral.net/pli-tv-pvr4/en/brahmali\#9.2}{Pvr 4:9.2}. } There are six kinds of respect. —\footnote{See \href{https://suttacentral.net/pli-tv-pvr4/en/brahmali\#10.2}{Pvr 4:10.2}. } There are six grounds of training. —\footnote{See \href{https://suttacentral.net/pli-tv-pvr4/en/brahmali\#11.2}{Pvr 4:11.2}. } There are six proper ways. —\footnote{Sp 5.326: \textit{Cha \textsanskrit{sāmīciyoti} “so ca bhikkhu anabbhito, te ca \textsanskrit{bhikkhū} \textsanskrit{gārayhā}, \textsanskrit{ayaṁ} tattha \textsanskrit{sāmīci}”, “\textsanskrit{yuñjantāyasmanto} \textsanskrit{sakaṁ}, \textsanskrit{mā} vo \textsanskrit{sakaṁ} \textsanskrit{vinassāti} \textsanskrit{ayaṁ} tattha \textsanskrit{sāmīci}”, “\textsanskrit{ayaṁ} te bhikkhu patto \textsanskrit{yāva} \textsanskrit{bhedanāya} \textsanskrit{dhāretabboti} \textsanskrit{ayaṁ} tattha \textsanskrit{sāmīci}”, “tato \textsanskrit{nīharitvā} \textsanskrit{bhikkhūhi} \textsanskrit{saddhiṁ} \textsanskrit{saṁvibhajitabbaṁ}, \textsanskrit{ayaṁ} tattha \textsanskrit{sāmīci}”, “\textsanskrit{aññātabbaṁ} \textsanskrit{paripucchitabbaṁ} \textsanskrit{paripañhitabbaṁ}, \textsanskrit{ayaṁ} tattha \textsanskrit{sāmīci}”, “yassa bhavissati so \textsanskrit{harissatīti} \textsanskrit{ayaṁ} tattha \textsanskrit{sāmīcī}”ti \textsanskrit{imā} \textsanskrit{bhikkhupātimokkheyeva} cha \textsanskrit{sāmīciyo}}, “The six proper ways: ‘That monk is not rehabilitated and those monks are at fault. This is the proper procedure;’ ‘“Please recover what’s yours, or it might perish.” This is the proper procedure;’ ‘“Monk, this bowl is yours. Keep it until it breaks.” This is the proper procedure;’ ‘He should take it away and share it with the monks. This is the proper procedure;’ ‘He should understand, should question, should enquire. This is the proper procedure;’ ‘“Whoever owns it will come and get it.” This is the proper procedure.’ These are the six proper ways in the Monastic Code for the monks.” } There are six originations of offenses. — There are six offenses involving cutting. —\footnote{Sp 5.326: \textit{Cha \textsanskrit{chedanakāti} \textsanskrit{pañcake} \textsanskrit{vuttā} \textsanskrit{pañca} \textsanskrit{bhikkhunīnaṁ} \textsanskrit{udakasāṭikāya} \textsanskrit{saddhiṁ} cha}, “The six involving cutting: the five mentioned in the fives together with the bathing robe for nuns, making it six.” } There are six ways of committing an offense. —\footnote{Sp 5.326: \textit{\textsanskrit{Chahākārehīti} \textsanskrit{alajjitā} \textsanskrit{aññāṇatā} \textsanskrit{kukkuccapakatatā} akappiye \textsanskrit{kappiyasaññitā} kappiye \textsanskrit{akappiyasaññitā} \textsanskrit{satisammosāti}}, “The six ways: shamelessness, ignorance, an anxious character, perceiving what is allowable as unallowable, perceiving what is unallowable as allowable, and absentmindedness.” } There are six benefits of being an expert on the Monastic Law. —\footnote{Sp 5.326: \textit{Cha \textsanskrit{ānisaṁsā} vinayadhareti \textsanskrit{pañcake} \textsanskrit{vuttā} \textsanskrit{pañca} \textsanskrit{tassādheyyo} uposathoti \textsanskrit{iminā} \textsanskrit{saddhiṁ} cha}, “The six benefits of being an expert on the Monastic Law: the five mentioned in the fives together with the observance-day ceremony being entrusted to them, making it six.” } There are six rules about ‘at the most’. —\footnote{Sp 5.326: \textit{Cha \textsanskrit{paramānīti} “\textsanskrit{dasāhaparamaṁ} \textsanskrit{atirekacīvaraṁ} \textsanskrit{dhāretabbaṁ}, \textsanskrit{māsaparamaṁ} tena \textsanskrit{bhikkhunā} \textsanskrit{taṁ} \textsanskrit{cīvaraṁ} \textsanskrit{nikkhipitabbaṁ}, \textsanskrit{santaruttaraparamaṁ} tena \textsanskrit{bhikkhunā} tato \textsanskrit{cīvaraṁ} \textsanskrit{sāditabbaṁ}, \textsanskrit{chakkhattuparamaṁ} \textsanskrit{tuṇhībhūtena} uddissa \textsanskrit{ṭhātabbaṁ}, \textsanskrit{navaṁ} pana \textsanskrit{bhikkhunā} \textsanskrit{santhataṁ} \textsanskrit{kārāpetvā} \textsanskrit{chabbassāni} \textsanskrit{dhāretabbaṁ} \textsanskrit{chabbassaparamatā} \textsanskrit{dhāretabbaṁ}, \textsanskrit{tiyojanaparamaṁ} \textsanskrit{sahatthā} \textsanskrit{dhāretabbāni}, \textsanskrit{dasāhaparamaṁ} atirekapatto \textsanskrit{dhāretabbo}, \textsanskrit{sattāhaparamaṁ} \textsanskrit{sannidhikārakaṁ} \textsanskrit{paribhuñjitabbāni}, \textsanskrit{chārattaparamaṁ} tena \textsanskrit{bhikkhunā} tena \textsanskrit{cīvarena} \textsanskrit{vippavasitabbaṁ}, \textsanskrit{catukkaṁsaparamaṁ}, \textsanskrit{aḍḍhateyyakaṁsaparamaṁ}, \textsanskrit{dvaṅgulapabbaparamaṁ} \textsanskrit{ādātabbaṁ}, \textsanskrit{aṭṭhaṅgulaparamaṁ} \textsanskrit{mañcapaṭipādakaṁ}, \textsanskrit{aṭṭhaṅgulaparamaṁ} \textsanskrit{dantakaṭṭha}”nti \textsanskrit{imāni} cuddasa \textsanskrit{paramāni}. Tattha \textsanskrit{paṭhamāni} cha \textsanskrit{ekaṁ} \textsanskrit{chakkaṁ}}, “The six rules on ‘at the most’: ‘Should keep an extra robe for ten days at the most;’ ‘He should keep it at most one month;’ ‘He should accept at most one sarong and one upper robe;’ ‘Should stand in silence for it at most six times;’ ‘If a monk has had a new blanket made, he should keep it for six years. He should keep it for six years at a minimum;’ ‘May carry it himself for at most 40 kilometers;’ ‘Should keep an extra almsbowl for ten days at the most;’ ‘Should be used from storage for at most seven days;’ ‘He should stay apart from that robe for six days at the most;’ ‘It is to be worth at most four \textit{\textsanskrit{kaṁsa}} coins;’ ‘It is to be worth at most two-and-a-half \textit{\textsanskrit{kaṁsa}} coins;’ ‘May insert two finger joints at the most;’ ‘Bed supports that are at the most eight standard fingerbreadths long;’ ‘Tooth cleaners that are at most thirteen centimeters long.’ These are the fourteen rules on ‘at the most’. There, the first six are one set of sixes.” These quotes refer respectively to \href{https://suttacentral.net/pli-tv-bu-vb-np1/en/brahmali\#2.17.1}{Bu NP 1:2.17.1}, \href{https://suttacentral.net/pli-tv-bu-vb-np3/en/brahmali\#1.3.13.1}{Bu NP 3:1.3.13.1}, \href{https://suttacentral.net/pli-tv-bu-vb-np7/en/brahmali\#1.31.1}{Bu NP 7:1.31.1}, \href{https://suttacentral.net/pli-tv-bu-vb-np10/en/brahmali\#1.3.1}{Bu NP 10:1.3.1}, \href{https://suttacentral.net/pli-tv-bu-vb-np14/en/brahmali\#2.38.1}{Bu NP 14:2.38.1}, \href{https://suttacentral.net/pli-tv-bu-vb-np16/en/brahmali\#1.23.1}{Bu NP 16:1.23.1}, \href{https://suttacentral.net/pli-tv-bu-vb-np21/en/brahmali\#2.17.1}{Bu NP 21:2.17.1}, \href{https://suttacentral.net/pli-tv-bu-vb-np23/en/brahmali\#1.3.32.1}{Bu NP 23:1.3.32.1}, \href{https://suttacentral.net/pli-tv-bu-vb-np29/en/brahmali\#1.2.16.1}{Bu NP 29:1.2.16.1}, \href{https://suttacentral.net/pli-tv-bi-vb-np11/en/brahmali\#1.21.1}{Bi NP 11:1.21.1}, \href{https://suttacentral.net/pli-tv-bi-vb-np12/en/brahmali\#1.21.1}{Bi NP 12:1.21.1}, \href{https://suttacentral.net/pli-tv-bi-vb-pc5/en/brahmali\#1.2.12.1}{Bi Pc 5:1.2.12.1}, \href{https://suttacentral.net/pli-tv-kd16/en/brahmali\#2.5.16}{Kd 16:2.5.16}, and \href{https://suttacentral.net/pli-tv-kd15/en/brahmali\#31.2.5}{Kd 15:31.2.5}. } One may be stay apart from one’s three robes for six days. — There are six kinds of robe-cloth. — There are six kinds of dye. — There are six kinds of offenses that originate from body and mind, not from speech. —\footnote{See \href{https://suttacentral.net/pli-tv-pvr4/en/brahmali\#44.1}{Pvr 4:44.1}. } There are six kinds of offenses that originate from speech and mind, not from body. —\footnote{See \href{https://suttacentral.net/pli-tv-pvr4/en/brahmali\#46.1}{Pvr 4:46.1}. } There are six kinds of offenses that originate from body, speech, and mind. —\footnote{See \href{https://suttacentral.net/pli-tv-pvr4/en/brahmali\#48.1}{Pvr 4:48.1}. } There are six kinds of legal procedures. —\footnote{Sp 5.326: \textit{Cha \textsanskrit{kammānīti} \textsanskrit{tajjanīya}-niyassa-\textsanskrit{pabbājanīya}-\textsanskrit{paṭisāraṇīyāni} \textsanskrit{cattāri}, \textsanskrit{āpattiyā} adassane ca \textsanskrit{appaṭikamme} ca vuttadvayampi \textsanskrit{ekaṁ}, \textsanskrit{pāpikāya} \textsanskrit{diṭṭhiyā} \textsanskrit{appaṭinissagge} ekanti cha}, “The six kinds of legal procedures: condemnation, demotion, banishment, and reconciliation are four, the mentioned pair of not recognizing and not making amends for an offense is one, and not letting go of a bad view is one, making it six.” } There are six roots of disputes. —\footnote{See \href{https://suttacentral.net/pli-tv-kd14/en/brahmali\#14.3.2}{Kd 14:14.3.2}. } There are six roots of accusations. —\footnote{See \href{https://suttacentral.net/pli-tv-kd14/en/brahmali\#14.5.2}{Kd 14:14.5.2}. } There are six aspects of friendliness. —\footnote{There is no apparent reason why this set of sixes in included here. Apart from the \textsanskrit{Parivāra}, it is not encountered elsewhere in the Vinaya \textsanskrit{Piṭaka}, but see \href{https://suttacentral.net/mn48/en/brahmali\#6.2}{MN 48:6.2}. } Six standard handspans in length. —\footnote{See \href{https://suttacentral.net/pli-tv-bu-vb-pc91/en/brahmali\#2.5}{Bu Pc 91:2.5}. } Six handspans wide. —\footnote{See \href{https://suttacentral.net/pli-tv-bu-vb-pc92/en/brahmali\#2.1.6}{Bu Pc 92:2.1.6}. } There are six reasons why the formal support from a teacher comes to an end. —\footnote{See \href{https://suttacentral.net/pli-tv-kd1/en/brahmali\#36.1.6}{Kd 1:36.1.6}. } There are six additions to the rule on bathing. —\footnote{Sp 5.326: \textit{\textsanskrit{Nahāneti} \textsanskrit{orenaḍḍhamāsaṁ} \textsanskrit{nahāne}}, “On bathing: on bathing at intervals of less than a half-month.” This refers to \href{https://suttacentral.net/pli-tv-bu-vb-pc57/en/brahmali\#6.7.1}{Bu Pc 57:6.7.1}. } One takes an unfinished robe and leaves the monastery. —\footnote{Sp 5.326: \textit{\textsanskrit{Vippakatacīvarādichakkadvayaṁ} kathinakkhandhake \textsanskrit{niddiṭṭhaṁ}}, “The pair of sixes on an unfinished robe, etc., are specified in The Chapter on the Robe-making Ceremony.” See \href{https://suttacentral.net/pli-tv-kd7/en/brahmali\#4.1.0}{Kd 7:4.1.0}. This segment and the next each constitute a separate set of sixes. } One leaves the monastery with an unfinished robe.\footnote{See \href{https://suttacentral.net/pli-tv-kd7/en/brahmali\#5.1.0}{Kd 7:5.1.0}. } 

When\marginnote{82.1} a monk has six qualities, he may give the full ordination, give formal support, and have a novice monk attend on him: he has (1) the virtue, (2) stillness, (3) wisdom, (4) freedom, and (5) the knowledge and vision of freedom of one who is fully trained, and (6) he has ten or more years of seniority. 

When\marginnote{83.1} a monk has six other qualities, he may give the full ordination, give formal support, and have a novice monk attend on him: (1) he has the virtue of one who is fully trained himself and encourages others in it; (2) he has the stillness of one who is fully trained himself and encourages others in it; (3) he has the wisdom of one who is fully trained himself and encourages others in it; (4) he has the freedom of one who is fully trained himself and encourages others in it; (5) he has the knowledge and vision of freedom of one who is fully trained himself and encourages others in it; (6) he has ten or more years of seniority. 

When\marginnote{84.1} a monk has six other qualities, he may give the full ordination, give formal support, and have a novice monk attend on him: he has (1) faith, (2) conscience, (3) moral prudence, (4) energy, (5) mindfulness, and (6) ten or more years of seniority. 

When\marginnote{85.1} a monk has six other qualities, he may give the full ordination, give formal support, and have a novice monk attend on him: (1) he has not failed in the higher morality; (2) he has not failed in conduct; (3) he has not failed in view; (4) he is learned; (5) he is wise; (6) he has ten or more years of seniority. 

When\marginnote{86.1} a monk has six other qualities, he may give the full ordination, give formal support, and have a novice monk attend on him: he is capable of three things in regard to a student: (1) to nurse him or have him nursed when he is sick; (2) to send him away or have him sent away when he is discontent with the spiritual life; (3) to use the Teaching to dispel anxiety. And (4) he knows the offenses; (5) he knows how offenses are cleared; and (6) he has ten or more years of seniority. 

When\marginnote{87.1} a monk has six other qualities, he may give the full ordination, give formal support, and have a novice monk attend on him: he is capable of five things in regard to a student: (1) to train him in good conduct; (2) to train him in the basics of the spiritual life; (3) to train him in the Teaching; (4) to train him in the Monastic Law; and (5) to use the Teaching to make him give up wrong views. And (6) he has ten or more years of seniority. 

When\marginnote{88.1} a monk has six other qualities, he may give the full ordination, give formal support, and have a novice monk attend on him: (1) he knows the offenses; (2) he knows the non-offenses; (3) he knows the light offenses; (4) he knows the heavy offenses; (5) he has properly learned both Monastic Codes in detail, and he has analyzed them well, thoroughly mastered them, and investigated them well, both in terms of the rules and their detailed exposition; (6) he has ten or more years of seniority. 

There\marginnote{89.1} are six kinds of illegitimate cancellations of the Monastic Code. —\footnote{See \href{https://suttacentral.net/pli-tv-kd19/en/brahmali\#3.3.37}{Kd 19:3.3.37}. } There are six kinds of legitimate cancellations of the Monastic Code.”\footnote{See \href{https://suttacentral.net/pli-tv-kd19/en/brahmali\#3.3.42}{Kd 19:3.3.42}. } 

\scend{The section on sixes is finished. }

\scuddanaintro{This is the summary: }

\begin{scuddana}%
“Disrespect,\marginnote{92.1} and respect, \\
Training, and also proper ways; \\
Originations, and cutting, \\
Ways, and with benefit. 

And\marginnote{93.1} ‘at the most’, six days, \\
Robe-cloth, and kinds of dye; \\
And from body and mind, \\
And from speech and mind. 

And\marginnote{94.1} from body, speech and mind, \\
Legal procedure, and dispute; \\
Accusations, and in length, \\
Wide, and with formal support. 

Additions\marginnote{95.1} to the rule, takes, \\
And so with; \\
Fully trained, one who encourages, \\
Faith, and with higher morality; \\
Sick, good conduct, \\
Offense, illegitimate, legitimate.” 

%
\end{scuddana}

\section*{7. The section on sevens }

“There\marginnote{96.1} are seven kinds of offenses. —\footnote{See \href{https://suttacentral.net/pli-tv-pvr4/en/brahmali\#6.1}{Pvr 4:6.1}. } There are seven classes of offenses. —\footnote{See \href{https://suttacentral.net/pli-tv-pvr4/en/brahmali\#7.1}{Pvr 4:7.1}. } There are seven grounds of training. —\footnote{See \href{https://suttacentral.net/pli-tv-pvr4/en/brahmali\#8.1}{Pvr 4:8.1}. } There are seven proper ways. —\footnote{Sp 5.327: \textit{Satta \textsanskrit{sāmīciyoti} pubbe vuttesu chasu “\textsanskrit{sā} ca \textsanskrit{bhikkhunī} \textsanskrit{anabbhitā}, \textsanskrit{tā} ca bhikkhuniyo \textsanskrit{gārayhā}, \textsanskrit{ayaṁ} tattha \textsanskrit{sāmīcī}”ti \textsanskrit{imaṁ} \textsanskrit{pakkhipitvā} satta \textsanskrit{veditabbā} }, “The seven proper ways: the seven are to be understood as the six previously spoken of and adding this: ‘That nun is not rehabilitated and those nuns are at fault. This is proper procedure.’” } There are seven illegitimate ways of acting according to what has been admitted. —\footnote{Sp 5.327: \textit{Satta \textsanskrit{adhammikā} \textsanskrit{paṭiññātakaraṇāti} “bhikkhu \textsanskrit{pārājikaṁ} \textsanskrit{ajjhāpanno} hoti, \textsanskrit{pārājikena} \textsanskrit{codiyamāno} ‘\textsanskrit{saṅghādisesaṁ} \textsanskrit{ajjhāpannomhī}’ti \textsanskrit{paṭijānāti}, \textsanskrit{taṁ} \textsanskrit{saṅgho} \textsanskrit{saṅghādisesena} \textsanskrit{kāreti}, \textsanskrit{adhammikaṁ} \textsanskrit{paṭiññātakaraṇa}”nti \textsanskrit{evaṁ} samathakkhandhake \textsanskrit{niddiṭṭhā}}, “The seven illegitimate ways of acting according to what has been admitted: ‘A monk has committed an offense entailing expulsion. When he is accused of having committed such an offense, he admits to committing an offense entailing suspension. The Sangha deals with him for an offense entailing suspension. That acting according to what has been admitted is illegitimate.’ In this way, it is specified in The Chapter on the Settling of Legal Issues.” See \href{https://suttacentral.net/pli-tv-kd14/en/brahmali\#8.1.2}{Kd 14:8.1.2}. } There are seven legitimate ways of acting according to what has been admitted. —\footnote{Sp 5.327: \textit{\textsanskrit{Dhammikāpi} tattheva \textsanskrit{niddiṭṭhā}}, “Legitimate: it is specified just there.” See \href{https://suttacentral.net/pli-tv-kd14/en/brahmali\#8.2.2}{Kd 14:8.2.2}. } There is no offense in going for seven days to seven kinds of people. —\footnote{Sp 5.327: \textit{\textsanskrit{Sattannaṁ} \textsanskrit{anāpatti} \textsanskrit{sattāhakaraṇīyena} gantunti \textsanskrit{vassūpanāyikakkhandhake} \textsanskrit{vuttaṁ}}, “There is no offense in going for seven days to seven kinds of people: it is spoken of in The Chapter on Entering the Rainy-season Residence.” See \href{https://suttacentral.net/pli-tv-kd3/en/brahmali\#5.4.2}{Kd 3:5.4.2} and \href{https://suttacentral.net/pli-tv-kd3/en/brahmali\#7.2.1}{Kd 3:7.2.1}. } There are seven benefits of being an expert on the Monastic Law. —\footnote{Sp 5.327: \textit{\textsanskrit{Sattānisaṁsā} vinayadhareti “\textsanskrit{tassādheyyo} uposatho \textsanskrit{pavāraṇā}”ti imehi \textsanskrit{saddhiṁ} \textsanskrit{pañcake} \textsanskrit{vuttā} \textsanskrit{pañca} satta honti}, “The seven benefits of being an expert on the Monastic Law: the five mentioned in the fives together with the observance-day ceremony and the invitation ceremony being entrusted to them, making it seven.” } There are seven rules about ‘at the most’. —\footnote{Sp 5.327: \textit{Satta \textsanskrit{paramānīti} chakke \textsanskrit{vuttāniyeva} sattakavasena \textsanskrit{yojetabbāni}}, “The seven rules on ‘at the most’: what is spoken of in the sixes is to be constructed as a group of seven.” Sp-\textsanskrit{ṭ} 5.327: \textit{Sattakesu chakke \textsanskrit{vuttāniyeva} sattakavasena \textsanskrit{yojetabbānīti} chakke \textsanskrit{vuttacuddasaparamāni} \textsanskrit{dvidhā} \textsanskrit{katvā} \textsanskrit{dvinnaṁ} \textsanskrit{sattakānaṁ} vasena \textsanskrit{yojetabbāni}}, “In the sevens, what is spoken of in the sixes is to be constructed as a group of seven: having divided in two parts the fourteen ‘at the mosts’ spoken of in the sixes, it is to be constructed on account of two groups of seven.” } There is becoming subject to relinquishment at dawn on the seventh day. —\footnote{See \href{https://suttacentral.net/pli-tv-bu-vb-np29/en/brahmali\#2.27}{Bu NP 29:2.27}. } There are seven principles for settling legal issues. — There are seven kinds of legal procedures. —\footnote{The seven are: condemnation, demotion, banishment, reconciliation, not recognizing an offense, not making amends for an offense, and not letting go of a bad view. See \href{https://suttacentral.net/pli-tv-kd11/en/brahmali\#0.3}{Kd 11}. } There are seven kinds of raw grain. —\footnote{See \href{https://suttacentral.net/pli-tv-bi-vb-pc7/en/brahmali\#2.1.6}{Bi Pc 7:2.1.6}. } It is seven wide inside. —\footnote{See \href{https://suttacentral.net/pli-tv-bu-vb-ss6/en/brahmali\#2.1.14}{Bu Ss 6:2.1.14}. } There are seven additions to the rule on eating in a group. —\footnote{See \href{https://suttacentral.net/pli-tv-bu-vb-pc32/en/brahmali\#8.15.1}{Bu Pc 32:8.15.1}. } After being received, the tonics should be used from storage for at most seven days. —\footnote{See \href{https://suttacentral.net/pli-tv-bu-vb-np23/en/brahmali\#2.12}{Bu NP 23:2.12}. } One takes a finished robe and leaves the monastery. —\footnote{Sp 5.327: \textit{\textsanskrit{Katacīvarantiādīni} dve \textsanskrit{sattakāni} kathinakkhandhake \textsanskrit{niddiṭṭhāni}}, “A finished robe, etc.: the two groups of seven that are specified in The Chapter on the Robe-making Ceremony.” See \href{https://suttacentral.net/pli-tv-kd7/en/brahmali\#2.1.0}{Kd 7:2.1.0}. This concerns the present and the next item. } One leaves the monastery with a finished robe. —\footnote{See \href{https://suttacentral.net/pli-tv-kd7/en/brahmali\#3.1.0}{Kd 7:3.1.0}. } A monk doesn’t have any offense he needs to recognize. —\footnote{Sp 5.327: \textit{Bhikkhussa na hoti \textsanskrit{āpatti} \textsanskrit{daṭṭhabbā}, bhikkhussa hoti \textsanskrit{āpatti} \textsanskrit{daṭṭhabbā}, bhikkhussa hoti \textsanskrit{āpatti} \textsanskrit{paṭikātabbāti} \textsanskrit{imāni} \textsanskrit{tīṇi} \textsanskrit{sattakāni}, dve \textsanskrit{adhammikāni}, \textsanskrit{ekaṁ} \textsanskrit{dhammikaṁ}; \textsanskrit{tāni} \textsanskrit{tīṇipi} campeyyake \textsanskrit{niddiṭṭhāni}}, “A monk doesn’t have any offense he needs to recognize; a monk does have an offense he needs to recognize; a monk has an offense he needs to make amends for: for these three groups of seven, two are illegitimate, one is legitimate. These three are specified in The Chapter Connected with \textsanskrit{Campā}.” See \href{https://suttacentral.net/pli-tv-kd9/en/brahmali\#5.1.0}{Kd 9:5.1.0}. This comment refers to the present and the next two items. } A monk does have an offense he needs to recognize. — A monk has an offense he needs to make amends for. —\footnote{MS is faulty. I follow SRT and the commentary which have \textit{bhikkhussa hoti \textsanskrit{āpatti} \textsanskrit{paṭikātabbā}}. } There are seven kinds of illegitimate cancellations of the Monastic Code. —\footnote{See \href{https://suttacentral.net/pli-tv-kd19/en/brahmali\#3.3.48}{Kd 19:3.3.48}. } There are seven kinds of legitimate cancellations of the Monastic Code.\footnote{See \href{https://suttacentral.net/pli-tv-kd19/en/brahmali\#3.3.52}{Kd 19:3.3.52}. } 

When\marginnote{97.1} a monk has seven qualities, he is an expert on the Monastic Law: (1) He knows the offenses. (2) He knows the non-offenses. (3) He knows the light offenses. (4) He knows the heavy offenses. (5) He is virtuous and restrained by the Monastic Code. His conduct is good, he associates with the right people, and he sees danger in minor faults. And he undertakes and trains in the training rules. (6) Whenever he wants, he accesses the four absorptions, those pleasant meditations of the higher mind. (7) And because of the ending of the corruptions, he has realized with his own insight, in this very life, the liberation by mind and the liberation by wisdom. 

When\marginnote{98.1} a monk has seven other qualities, he is an expert on the Monastic Law: (1) He knows the offenses. (2) He knows the non-offenses. (3) He knows the light offenses. (4) He knows the heavy offenses. (5) He has learned much, and he retains and accumulates what he has learned. Those teachings that are good in the beginning, good in the middle, and good in the end, that have a true goal and are well articulated, and that set out the perfectly complete and pure spiritual life—he has learned many such teachings, retained them in mind, recited them verbally, mentally investigated them, and penetrated them well by view. (6) Whenever he wants, he accesses the four absorptions, those pleasant meditations of the higher mind. (7) And because of the ending of the corruptions, he has realized with his own insight, in this very life, the liberation by mind and the liberation by wisdom. 

When\marginnote{99.1} a monk has seven other qualities, he is an expert on the Monastic Law: (1) He knows the offenses. (2) He knows the non-offenses. (3) He knows the light offenses. (4) He knows the heavy offenses. (5) He has properly learned both Monastic Codes in detail; he has analyzed them well, thoroughly mastered them, and investigated them well, both in terms of the rules and their detailed exposition. (6) Whenever he wants, he accesses the four absorptions, those pleasant meditations of the higher mind. (7) And because of the ending of the corruptions, he has realized with his own insight, in this very life, the liberation by mind and the liberation by wisdom. 

When\marginnote{100.1} a monk has seven other qualities, he is an expert on the Monastic Law: (1) He knows the offenses. (2) He knows the non-offenses. (3) He knows the light offenses. (4) He knows the heavy offenses. (5) He recollects many past lives, that is, one birth, two births, three births, four births, five births, ten births, twenty births, thirty births, forty births, fifty births, a hundred births, a thousand births, a hundred thousand births; many eons of world dissolution, many eons of world evolution, many eons of both dissolution and evolution; and he knows: ‘There I had such a name, such a family, such an appearance, such food, such an experience of pleasure and pain, and such a lifespan. Passing away from there, I was reborn elsewhere, and there I had such a name, such a family, such an appearance, such food, such an experience of pleasure and pain, and such a lifespan. Passing away from there, I was reborn here.’ In this way he recollects many past lives with their characteristics and particulars. (6) With superhuman and purified clairvoyance, he sees beings passing away and getting reborn, inferior and superior, beautiful and ugly, gone to good destinations and to bad destinations, and he understands how beings pass on according to their actions: ‘These beings who engaged in misconduct by body, speech, and mind, who abused the noble ones, who had wrong views and acted accordingly, at the breaking up of the body after death, have been reborn in a lower realm, a bad destination, a world of misery, hell. But these beings who engaged in good conduct of body, speech, and mind, who did not abuse the noble ones, who held right view and acted accordingly, at the breaking up of the body after death, have been reborn in a good destination, a heaven world.’ In this way, with superhuman and purified clairvoyance, he sees beings passing away and getting reborn, inferior and superior, beautiful and ugly, gone to good destinations and to bad destinations, and he understands how beings pass on according to their actions. (7) And because of the ending of the corruptions, he has realized with his own insight, in this very life, the liberation by mind and the liberation by wisdom. 

When\marginnote{101.1} an expert on the Monastic Law has seven qualities, he shines: (1) He knows the offenses. (2) He knows the non-offenses. (3) He knows the light offenses. (4) He knows the heavy offenses. (5) He is virtuous and restrained by the Monastic Code. His conduct is good, he associates with the right people, and he sees danger in minor faults. And he undertakes and trains in the training rules. (6) Whenever he wants, he accesses the four absorptions, those pleasant meditations of the higher mind. (7) And because of the ending of the corruptions, he has realized with his own insight, in this very life, the liberation by mind and the liberation by wisdom. 

When\marginnote{102.1} an expert on the Monastic Law has seven qualities, he shines: (1) He knows the offenses. (2) He knows the non-offenses. (3) He knows the light offenses. (4) He knows the heavy offenses. (5) He has learned much, and he retains and accumulates what he has learned. Those teachings that are good in the beginning, good in the middle, and good in the end, that have a true goal and are well articulated, and that set out the perfectly complete and pure spiritual life—he has learned many such teachings, retained them in mind, recited them verbally, mentally investigated them, and penetrated them well by view. (6) Whenever he wants, he accesses the four absorptions, those pleasant meditations of the higher mind. (7) And because of the ending of the corruptions, he has realized with his own insight, in this very life, the liberation by mind and the liberation by wisdom. 

When\marginnote{103.1} an expert on the Monastic Law has seven qualities, he shines: (1) He knows the offenses. (2) He knows the non-offenses. (3) He knows the light offenses. (4) He knows the heavy offenses. (5) He has properly learned both Monastic Codes in detail; he has analyzed them well, thoroughly mastered them, and investigated them well, both in terms of the rules and their detailed exposition. (6) Whenever he wants, he accesses the four absorptions, those pleasant meditations of the higher mind. (7) And because of the ending of the corruptions, he has realized with his own insight, in this very life, the liberation by mind and the liberation by wisdom. 

When\marginnote{104.1} an expert on the Monastic Law has seven qualities, he shines: (1) He knows the offenses. (2) He knows the non-offenses. (3) He knows the light offenses. (4) He knows the heavy offenses. (5) He recollects many past lives, that is, one birth, two births, three births, four births, five births, ten births, twenty births, thirty births, forty births, fifty births, a hundred births, a thousand births, a hundred thousand births; many eons of world dissolution, many eons of world evolution, many eons of both dissolution and evolution; and he knows: ‘There I had such a name, such a family, such an appearance, such food, such an experience of pleasure and pain, and such a lifespan. Passing away from there, I was reborn elsewhere, and there I had such a name, such a family, such an appearance, such food, such an experience of pleasure and pain, and such a lifespan. Passing away from there, I was reborn here.’ In this way he recollects many past lives with their characteristics and particulars. (6) With superhuman and purified clairvoyance, he sees beings passing away and getting reborn, inferior and superior, beautiful and ugly, gone to good destinations and to bad destinations, and he understands how beings pass on according to their actions: ‘These beings who engaged in misconduct by body, speech, and mind, who abused the noble ones, who had wrong views and acted accordingly, at the breaking up of the body after death, have been reborn in a lower realm, a bad destination, a world of misery, hell. But these beings who engaged in good conduct of body, speech, and mind, who did not abuse the noble ones, who held right view and acted accordingly, at the breaking up of the body after death, have been reborn in a good destination, a heaven world.’ In this way, with superhuman and purified clairvoyance, he sees beings passing away and getting reborn, inferior and superior, beautiful and ugly, gone to good destinations and to bad destinations, and he understands how beings pass on according to their actions. (7) And because of the ending of the corruptions, he has realized with his own insight, in this very life, the liberation by mind and the liberation by wisdom. 

There\marginnote{105.1} are seven bad qualities: one has no faith, conscience, or moral prudence; and one is ignorant, lazy, absentminded, and foolish. — 

There\marginnote{106.1} are seven good qualities: one has faith, conscience, and moral prudence; and one is learned, energetic, mindful, and wise.” 

\scend{The section on sevens is finished. }

\scuddanaintro{This is the summary: }

\begin{scuddana}%
“Offense,\marginnote{109.1} classes of offenses, \\
Training, and proper ways; \\
Illegitimate, and legitimate, \\
And seven days is no offense. 

Benefits,\marginnote{110.1} ‘at the most’, \\
Dawn, and with settling; \\
Legal procedures, and kinds of raw grain, \\
Wide, eating in a group. 

At\marginnote{111.1} most seven days, takes, \\
And so with; \\
Doesn’t, does, and does, \\
Illegitimate and legitimate. 

Four\marginnote{112.1} about experts on the Monastic Law, \\
And four about monks who shine; \\
And seven bad qualities, \\
Seven good qualities have been taught.” 

%
\end{scuddana}

\section*{8. The section on eights }

“When\marginnote{113.1} you see eight benefits, you should not eject a monk for not recognizing an offense. —\footnote{See \href{https://suttacentral.net/pli-tv-kd10/en/brahmali\#1.6.4}{Kd 10:1.6.4}–1.7.10. } When you see eight benefits, you should confess an offense even out of confidence in the others. —\footnote{See \href{https://suttacentral.net/pli-tv-kd10/en/brahmali\#1.8.6}{Kd 10:1.8.6}–1.8.16. } There are eight ‘after the third’. —\footnote{Sp 5.328: \textit{\textsanskrit{Aṭṭha} \textsanskrit{yāvatatiyakāti} \textsanskrit{bhikkhūnaṁ} terasake \textsanskrit{cattāro}, \textsanskrit{bhikkhunīnaṁ} sattarasake \textsanskrit{bhikkhūhi} \textsanskrit{asādhāraṇā} \textsanskrit{cattāroti} \textsanskrit{aṭṭha}}, “The eight ‘after the third’: four among the thirteen for the monks and four among the seventeen for the nuns that are not in common with the monks, making eight.” This refers to the offenses entailing suspension. } There are eight ways of corrupting families. —\footnote{Sp 5.328: \textit{\textsanskrit{Aṭṭhahākārehi} \textsanskrit{kulāni} \textsanskrit{dūsetīti} \textsanskrit{kulāni} \textsanskrit{dūseti} pupphena \textsanskrit{vā} phalena \textsanskrit{vā} \textsanskrit{cuṇṇena} \textsanskrit{vā} \textsanskrit{mattikāya} \textsanskrit{vā} \textsanskrit{dantakaṭṭhena} \textsanskrit{vā} \textsanskrit{veḷuyā} \textsanskrit{vā} \textsanskrit{vejjikāya} \textsanskrit{vā} \textsanskrit{jaṅghapesanikena} \textsanskrit{vāti} imehi \textsanskrit{aṭṭhahi}}, “The eight ways of corrupting families: one corrupts families with flowers, fruit, bath powder, soap, tooth cleaners, bamboo, medical treatment, or by taking messages on foot—with these eight.” See \href{https://suttacentral.net/pli-tv-bu-vb-ss13/en/brahmali\#2.8}{Bu Ss 13:2.8}. } There are eight key phrases for the giving of robe-cloth. —\footnote{See \href{https://suttacentral.net/pli-tv-kd8/en/brahmali\#32.1.2}{Kd 8:32.1.2}. } There are eight key phrases for when the robe season comes to an end. —\footnote{See \href{https://suttacentral.net/pli-tv-kd7/en/brahmali\#1.7.3}{Kd 7:1.7.3}. } There are eight kinds of drinks. —\footnote{See \href{https://suttacentral.net/pli-tv-kd6/en/brahmali\#35.6.3}{Kd 6:35.6.3}. } Because he was overcome and consumed by eight bad qualities, Devadatta is irredeemably destined to an eon in hell. —\footnote{See \href{https://suttacentral.net/pli-tv-kd17/en/brahmali\#4.7.1}{Kd 17:4.7.1}–4.7.10. } There are eight worldly phenomena. —\footnote{See \href{https://suttacentral.net/an8.5/en/brahmali\#1.1}{AN 8.5:1.1}. } There are eight important principles. —\footnote{See \href{https://suttacentral.net/pli-tv-kd20/en/brahmali\#1.4.3}{Kd 20:1.4.3}–1.4.20. } There are eight offenses entailing acknowledgment. —\footnote{See \href{https://suttacentral.net/pli-tv-bi-vb-pd1/en/brahmali}{Bi Pd 1}–8. } Lying has eight factors. —\footnote{Sp 5.328: \textit{\textsanskrit{Aṭṭhaṅgiko} \textsanskrit{musāvādoti} “\textsanskrit{vinidhāya} \textsanskrit{sañña}”nti \textsanskrit{iminā} \textsanskrit{saddhiṁ} \textsanskrit{pāḷiyaṁ} \textsanskrit{āgatehi} \textsanskrit{sattahīti} \textsanskrit{aṭṭhahi} \textsanskrit{aṅgehi} \textsanskrit{aṭṭhaṅgiko}}, “Lying has eight factors: the seven found in the Canonical text together with this: ‘He misrepresents his perception’. It is eightfold because of eight factors.” For the first seven factors see \href{https://suttacentral.net/pli-tv-bu-vb-pc1/en/brahmali\#2.2.10}{Bu Pc 1:2.2.10}. } The observance day has eight factors. —\footnote{According to the commentary at Sp 5.328 this refers to the eight precepts. } There are eight qualities of a qualified messenger. —\footnote{See \href{https://suttacentral.net/pli-tv-kd17/en/brahmali\#4.6.3}{Kd 17:4.6.3}. } There are eight proper conducts of monastics of other religions. —\footnote{See \href{https://suttacentral.net/pli-tv-kd1/en/brahmali\#38.8.2}{Kd 1:38.8.2}–38.10.2. According to Sp 3.87, the last two items on this list counts as four separate practices, thus making eight in total. } The great ocean has eight amazing qualities. —\footnote{See \href{https://suttacentral.net/pli-tv-kd19/en/brahmali\#1.3.1}{Kd 19:1.3.1}–1.3.24. } This spiritual path has eight amazing qualities. —\footnote{See \href{https://suttacentral.net/pli-tv-kd19/en/brahmali\#1.4.1}{Kd 19:1.4.1}–1.4.32. } There are eight ‘not left overs’. —\footnote{See \href{https://suttacentral.net/pli-tv-bu-vb-pc35/en/brahmali\#3.1.10}{Bu Pc 35:3.1.10}–3.1.17. } There are eight ‘left overs’. —\footnote{See \href{https://suttacentral.net/pli-tv-bu-vb-pc35/en/brahmali\#3.1.20}{Bu Pc 35:3.1.20}–3.1.27. } There is becoming subject to relinquishment at dawn on the eighth day. —\footnote{See \href{https://suttacentral.net/pli-tv-bu-vb-np23/en/brahmali\#2.15}{Bu NP 23:2.15}. } There are eight offenses entailing expulsion. —\footnote{This probably refers to the four offenses entailing expulsion for the monks, in addition to the four offenses entailing expulsion for the nuns that are not in common with the monks. } When she fulfills the eight parts, she should be expelled. —\footnote{This refers to \href{https://suttacentral.net/pli-tv-bi-vb-pj8/en/brahmali\#2.1.33}{Bi Pj 8:2.1.33}. The Pali actually says “the eighth part”, which in context means that all eight parts have been completed. For an explanation of the rendering “should be expelled” for \textit{\textsanskrit{nāsetabbā}}, see \textit{\textsanskrit{nāseti}} in Appendix of Technical Terms. } When she fulfills the eight parts, even if she confesses, it is not actually confessed. — There is full ordination with eight statements. —\footnote{This refers to the full ordination of a nun, which happens before both Sanghas. The eight statements are the two motions and the six announcements. } One should stand up for eight people. —\footnote{Sp 5.328: \textit{\textsanskrit{Aṭṭhannaṁ} \textsanskrit{paccuṭṭhātabbanti} bhattagge \textsanskrit{vuḍḍhabhikkhunīnaṁ}}, “One should stand up for eight people: for the senior nuns in the dining hall.” } One should offer a seat to eight people. —\footnote{Sp 5.328: \textit{Āsanampi \textsanskrit{tāsaṁyeva} \textsanskrit{dātabbaṁ}}, “A seat should also be given to them.” } The female lay follower who asked for eight favors. —\footnote{See \href{https://suttacentral.net/pli-tv-kd8/en/brahmali\#15.7.1}{Kd 8:15.7.1}. } When a monk has eight qualities, he may be appointed as an instructor of the nuns. —\footnote{See \href{https://suttacentral.net/pli-tv-bu-vb-pc21/en/brahmali\#2.26}{Bu Pc 21:2.26}–2.34. } There are eight benefits of being an expert on the Monastic Law. —\footnote{Sp 5.328: \textit{\textsanskrit{Aṭṭhānisaṁsā} vinayadhareti \textsanskrit{pañcake} vuttesu \textsanskrit{pañcasu} “\textsanskrit{tassādheyyo} uposatho, \textsanskrit{pavāraṇā}, \textsanskrit{saṅghakamma}”nti ime tayo \textsanskrit{pakkhipitvā} \textsanskrit{aṭṭha} \textsanskrit{veditabbā}}, “The eight benefits of being an expert on the Monastic Law: the eight are to be understood as the five mentioned in the fives and adding these three: the observance-day ceremony, the invitation ceremony, and legal procedures being entrusted to them.” } There are eight rules about ‘at the most’. —\footnote{Sp 5.328: \textit{\textsanskrit{Aṭṭha} \textsanskrit{paramānīti} pubbe \textsanskrit{vuttaparamāneva} \textsanskrit{aṭṭhakavasena} \textsanskrit{yojetvā} \textsanskrit{veditabbāni}}, “There are eight rules on ‘at the most’: they are to be understood just as the previously mentioned ‘at the most’, having constructed it as a group of eight.” } A monk who has had a legal procedure of further penalty done against himself should behave properly in eight respects. —\footnote{See \href{https://suttacentral.net/pli-tv-kd14/en/brahmali\#12.5.7.7}{Kd 14:12.5.7.7}–12.5.8. } There are eight kinds of illegitimate cancellations of the Monastic Code. —\footnote{See \href{https://suttacentral.net/pli-tv-kd19/en/brahmali\#3.3.56}{Kd 19:3.3.56}–3.3.60. } There are eight kinds of legitimate cancellations of the Monastic Code.”\footnote{See \href{https://suttacentral.net/pli-tv-kd19/en/brahmali\#3.3.62}{Kd 19:3.3.62}–3.3.66. } 

\scend{The section on eights is finished. }

\scuddanaintro{This is the summary: }

\begin{scuddana}%
“Not\marginnote{116.1} that monk, even in the others, \\
For the third time, corrupting; \\
Key phrases, the robe season coming to an end, \\
Drinks, and with overcome by. 

Worldly\marginnote{117.1} phenomena, important principles, \\
Offenses entailing acknowledgment, lying; \\
And observance days, qualities of a qualified messenger, \\
Monastics of other religions, and also of the ocean. 

Amazing,\marginnote{118.1} not left over, \\
Left over, subject to relinquishment; \\
Offenses entailing expulsion, parts, \\
Not actually confessed, full ordination. 

One\marginnote{119.1} should stand up for, and seat, \\
Favor, and with an instructor; \\
Benefits, ‘at the most’, \\
Behaving in eight respects; \\
Illegitimate, and legitimate, \\
The section on eights has been well proclaimed.” 

%
\end{scuddana}

\section*{9. The section on nines }

“There\marginnote{120.1} are nine grounds for resentment. —\footnote{See \href{https://suttacentral.net/an9.29/en/brahmali\#1.1}{AN 9.29:1.1}. } There are nine ways of getting rid of resentment. —\footnote{See \href{https://suttacentral.net/an9.30/en/brahmali\#1.1}{AN 9.30:1.1}. } There are nine grounds of training. —\footnote{Sp 5.329: \textit{Nava \textsanskrit{vinītavatthūnīti} navahi \textsanskrit{āghātavatthūhi} \textsanskrit{ārati} virati \textsanskrit{paṭivirati} \textsanskrit{setughāto}}, “The nine grounds of training: the refraining from, the keeping away from, the desisting from, the incapability with respect to the nine grounds for resentment.” } There are nine immediate offenses. —\footnote{The first nine offenses entailing suspension for monks. } The Sangha is split by a group of nine. —\footnote{See \href{https://suttacentral.net/pli-tv-kd17/en/brahmali\#5.1.24}{Kd 17:5.1.24}. } There are nine fine foods. —\footnote{See \href{https://suttacentral.net/pli-tv-bu-vb-pc39/en/brahmali\#2.10.1}{Bu Pc 39:2.10.1}. } There is an offense of wrong conduct for eating nine kinds of meat. —\footnote{See \href{https://suttacentral.net/pli-tv-kd6/en/brahmali\#23.10.8}{Kd 6:23.10.8}–23.15.9. } There are nine ways of reciting the Monastic Code. —\footnote{This probably refers to the five ways of reciting for monks, mentioned at \href{https://suttacentral.net/pli-tv-kd2/en/brahmali\#15.1.4}{Kd 2:15.1.4}, together with the four ways of reciting for nuns, referred to at \href{https://suttacentral.net/pli-tv-pvr2.1/en/brahmali\#1.8}{Pvr 2.1:1.8} and in the ensuing discussion. } There are nine rules about ‘at the most’. —\footnote{Sp 5.329: \textit{Nava \textsanskrit{paramānīti} pubbe \textsanskrit{vuttaparamāneva} navakavasena \textsanskrit{yojetvā} \textsanskrit{veditabbāni}}, “There are nine rules on ‘at the most’: they are to be understood just as the previously mentioned ‘at the most’, having constructed it as a group of nine.” } There are nine things rooted in craving. —\footnote{See \href{https://suttacentral.net/an9.23/en/brahmali\#1.1}{AN 9.23:1.1}. } There are nine kinds of conceit. —\footnote{See \href{https://suttacentral.net/vb17/en/brahmali\#310.1}{Vb 17:310.1}. } There are nine kinds of robes that should be determined. —\footnote{See \href{https://suttacentral.net/pli-tv-kd8/en/brahmali\#20.2.4}{Kd 8:20.2.4}–20.2.10. } There are nine kinds of robes that should not be assigned to another. —\footnote{As above. } Nine standard handspans long. —\footnote{See \href{https://suttacentral.net/pli-tv-bu-vb-pc92/en/brahmali\#2.1.6}{Bu Pc 92:2.1.6}. } There are nine illegitimate kinds of gifts. —\footnote{Sp 5.329: \textit{Nava \textsanskrit{adhammikāni} \textsanskrit{dānānīti} \textsanskrit{saṅghassa} \textsanskrit{pariṇataṁ} \textsanskrit{aññasaṅghassa} \textsanskrit{vā} cetiyassa \textsanskrit{vā} puggalassa \textsanskrit{vā} \textsanskrit{pariṇāmeti}, cetiyassa \textsanskrit{pariṇataṁ} \textsanskrit{aññacetiyassa} \textsanskrit{vā} \textsanskrit{saṅghassa} \textsanskrit{vā} puggalassa \textsanskrit{vā} \textsanskrit{pariṇāmeti}, puggalassa \textsanskrit{pariṇataṁ} \textsanskrit{aññapuggalassa} \textsanskrit{vā} \textsanskrit{saṅghassa} \textsanskrit{vā} cetiyassa \textsanskrit{vā} \textsanskrit{pariṇāmetīti}}, “The nine illegitimate kinds of gifts: if it was intended for one sangha and he diverts it to another sangha or to a shrine or to an individual; if it was intended for one shrine and he diverts it to another shrine or to a sangha or to an individual; if it was intended for one individual and he diverts it to another individual or to a sangha or to a shrine.” This refers to \href{https://suttacentral.net/pli-tv-bu-vb-np30/en/brahmali\#2.26}{Bu NP 30:2.26}–2.28 and \href{https://suttacentral.net/pli-tv-bu-vb-pc82/en/brahmali\#2.2.4}{Bu Pc 82:2.2.4}–2.2.6. } There are nine illegitimate kinds of receiving. —\footnote{Sp 5.329: \textit{Nava \textsanskrit{paṭiggahaparibhogāti} \textsanskrit{etesaṁyeva} \textsanskrit{dānānaṁ} \textsanskrit{paṭiggahā} ca \textsanskrit{paribhogā} ca}, “The nine illegitimate kinds of receiving and possessions: the receiving and possession of these kinds of gifts.” This refers to the present and the next item. } There are nine illegitimate kinds of possession. — There are three legitimate kinds of gifts, three legitimate kinds of receiving, and three legitimate kinds of possession. —\footnote{Sp 5.329: \textit{\textsanskrit{Tīṇi} \textsanskrit{dhammikāni} \textsanskrit{dānānīti} \textsanskrit{saṅghassa} \textsanskrit{ninnaṁ} \textsanskrit{saṅghasseva} deti, cetiyassa \textsanskrit{ninnaṁ} cetiyasseva, puggalassa \textsanskrit{ninnaṁ} puggalasseva \textsanskrit{detīti} \textsanskrit{imāni} \textsanskrit{tīṇi}. \textsanskrit{Paṭiggahapaṭibhogāpi} \textsanskrit{tesaṁyeva} \textsanskrit{paṭiggahā} ca \textsanskrit{paribhogā} ca}, “The three legitimate kinds of gift: inclining toward the Sangha, one gives to the Sangha, and inclining toward a shrine to a shrine. Inclining toward an individual, one gives to an individual. These are the three. Also receiving and possessions is the receiving and possession of these.” } There are nine illegitimate ways of winning over. —\footnote{See \href{https://suttacentral.net/pli-tv-kd14/en/brahmali\#2.1.1}{Kd 14:2.1.1}–2.1.27. } There are nine legitimate ways of winning over. —\footnote{See \href{https://suttacentral.net/pli-tv-kd14/en/brahmali\#3.1.1}{Kd 14:3.1.1}–3.1.27. } There are two groups of nine on illegitimate legal procedures. —\footnote{See \href{https://suttacentral.net/pli-tv-bu-vb-pc21/en/brahmali\#3.2.1}{Bu Pc 21:3.2.1}–3.2.18. } There are two groups of nine on legitimate legal procedures. —\footnote{See \href{https://suttacentral.net/pli-tv-bu-vb-pc21/en/brahmali\#3.2.19}{Bu Pc 21:3.2.19}–3.2.36. } There are nine kinds of illegitimate cancellations of the Monastic Code. —\footnote{See \href{https://suttacentral.net/pli-tv-kd19/en/brahmali\#3.3.69}{Kd 19:3.3.69}–3.3.71. } There are nine kinds of legitimate cancellations of the Monastic Code.”\footnote{See \href{https://suttacentral.net/pli-tv-kd19/en/brahmali\#3.3.75}{Kd 19:3.3.75}–3.3.77. } 

\scend{The section on nines is finished. }

\scuddanaintro{This is the summary: }

\begin{scuddana}%
“Grounds\marginnote{123.1} for resentment, getting rid of, \\
Training, and with immediate; \\
And is split, and fine, \\
Meat, reciting, and ‘at the most’. 

Craving,\marginnote{124.1} conceit, determined, \\
And assignment to another, handspans; \\
Gifts, kinds of receiving, kinds of possession, \\
Again threefold legitimate kinds. 

Illegitimate\marginnote{125.1} ways of winning over, and legitimate ways of winning over, \\
And twice two groups of nine; \\
Cancellations of the Monastic Code, \\
Illegitimate, and legitimate.” 

%
\end{scuddana}

\section*{10. The section on tens }

“There\marginnote{126.1} are ten grounds for resentment. —\footnote{Sp 5.330: \textit{Dasa \textsanskrit{āghātavatthūnīti} navakesu \textsanskrit{vuttāni} nava “\textsanskrit{aṭṭhāne} \textsanskrit{vā} pana \textsanskrit{āghāto} \textsanskrit{jāyatī}”ti \textsanskrit{iminā} \textsanskrit{saddhiṁ} dasa honti}, “The ten grounds for resentment: the ten are the nine spoken of in the nines together with this: one gives rise to resentment without reason.” } There are ten ways of getting rid of resentment. —\footnote{Sp 5.330: \textit{\textsanskrit{Āghātapaṭivinayāpi} tattha \textsanskrit{vuttā} nava “\textsanskrit{aṭṭhāne} \textsanskrit{vā} pana \textsanskrit{āghāto} \textsanskrit{jāyati}, \textsanskrit{taṁ} kutettha \textsanskrit{labbhāti} \textsanskrit{āghātaṁ} \textsanskrit{paṭivinetī}”ti \textsanskrit{iminā} \textsanskrit{saddhiṁ} dasa \textsanskrit{veditabbā}}, “The ten are to be understood as the nine ways of getting rid of resentment spoken of there together with this: one gives rise to resentment without reason, and one gets rid of it, thinking, ‘How will this help?’” } There are ten grounds of training. —\footnote{Sp 5.330: \textit{Dasa \textsanskrit{vinītavatthūnīti} dasahi \textsanskrit{āghātavatthūhi} \textsanskrit{viratisaṅkhātāni} dasa}, “The ten grounds of training: the ten are the abstaining from the ten grounds for resentment.” } There are ten subject matters of wrong view. —\footnote{Sp 5.330: \textit{\textsanskrit{Dasavatthukā} \textsanskrit{micchādiṭṭhīti} “natthi dinna”\textsanskrit{ntiādivasena} \textsanskrit{veditabbā}}, “The ten subject matters of wrong view: they are to be understood on account of ‘there is nothing given’ etc.” } There are ten subject matters of right view. —\footnote{Sp 5.330: \textit{“Atthi dinna”\textsanskrit{ntiādivasena} \textsanskrit{sammādiṭṭhi}}, “There is right view on account of ‘there is the given’ etc.” } There are ten extreme views. —\footnote{Sp 5.330: \textit{“Sassato loko”\textsanskrit{tiādinā} vasena pana \textsanskrit{antaggāhikā} \textsanskrit{diṭṭhi} \textsanskrit{veditabbā}}, “The extreme views are to be understood on account of ‘the world is eternal’ etc.” } There are ten kinds of wrongness. —\footnote{Sp 5.330: \textit{Dasa \textsanskrit{micchattāti} \textsanskrit{micchādiṭṭhiādayo} \textsanskrit{micchāvimuttipariyosānā}, \textsanskrit{viparītā} \textsanskrit{sammattā}}, “The ten kinds wrongness: wrong view, etc., ending with wrong liberation.” } There are ten kinds of rightness. —\footnote{Sp 5.330: \textit{\textsanskrit{Viparītā} \textsanskrit{sammattā}}, “The kinds of rightness are the reverse.” } There are ten ways of doing unskillful deeds. —\footnote{For this and the next item see \href{https://suttacentral.net/an10.211/en/brahmali\#1.1}{AN 10.211}. } There are ten ways of doing skillful deeds. — There are ten reasons why a vote is illegitimate. —\footnote{See \href{https://suttacentral.net/pli-tv-kd14/en/brahmali\#10.1.3}{Kd 14:10.1.3}. } There are ten reasons why a vote is legitimate. —\footnote{See \href{https://suttacentral.net/pli-tv-kd14/en/brahmali\#10.2.2}{Kd 14:10.2.2}. } There are ten training rules for novice monks. —\footnote{See \href{https://suttacentral.net/pli-tv-kd1/en/brahmali\#56.1.4}{Kd 1:56.1.4}–56.1.14. } A novice monk who has ten qualities should be expelled.\footnote{See \href{https://suttacentral.net/pli-tv-kd1/en/brahmali\#60.1.5}{Kd 1:60.1.5}–60.1.15. } 

When\marginnote{127.1} an expert on the Monastic Law has ten qualities, he is considered ignorant: (1) he does not grasp what is proper for himself to say; (2) he does not grasp what is proper for others to say; (3-5) not grasping either, he makes them act illegitimately, without their admission; (6) he does not know the offenses; (7) he does not know the roots of the offenses; (8) he does not know the origin of the offenses; (9) he does not know the ending of the offenses; (10) he does not know the path leading to the ending of the offenses.\footnote{Here the punctuation of the Pali is wrong. The comma should come after \textit{\textsanskrit{appaṭiññāya}}, not before it; see \href{https://suttacentral.net/pli-tv-pvr7/en/brahmali\#64.2}{Pvr 7:64.2}. } 

But\marginnote{128.1} when an expert on the Monastic Law has ten qualities, he is considered learned: (1) he grasps what is proper for himself to say; (2) he grasps what is proper for others to say; (3-5) grasping both, he makes them act legitimately, in accordance with their admission; (6) he knows the offenses; (7) he knows the roots of the offenses; (8) he knows the origin of the offenses; (9) he knows the ending of the offenses; (10) he knows the path leading to the ending of the offenses. 

When\marginnote{129.1} an expert on the Monastic Law has ten other qualities, he is considered ignorant: (1) he does not know the legal issues; (2) he does not know the roots of the legal issues; (3) he does not know the origin of the legal issues; (4) he does not know the ending of the legal issues; (5) he does not know the path leading to the ending of the legal issues; (6) he does not know the actions that are the bases for offenses; (7) he does not know the origin stories; (8) he does not know the rules; (9) he does not know the additions to the rules; (10) he does not know the sequence of statements. 

But\marginnote{130.1} when an expert on the Monastic Law has ten qualities, he is considered learned: (1) he knows legal issues; (2) he knows the roots of legal issues; (3) he knows the origin of legal issues; (4) he knows the ending of legal issues; (5) he knows the path leading to the ending of legal issues; (6) he knows the actions that are the bases for offenses; (7) he knows the origin stories; (8) he knows the rules; (9) he knows the additions to the rules; (10) he knows the sequence of statements. 

When\marginnote{131.1} an expert on the Monastic Law has ten other qualities, he is considered ignorant: (1) he does not know the motion; (2) he does not know how the motion is done; (3) he is not skilled in what comes first; (4) he is not skilled in what comes afterwards; (5) he does not know the right time; (6) he does not know the offenses and non-offenses; (7) he does not know the light and heavy offenses; (8) he does not know the curable and incurable offenses; (9) he does not know the grave and minor offenses; (10) he has not well-learned or well-remembered his teachers’ tradition. 

But\marginnote{132.1} when an expert on the Monastic Law has ten qualities, he is considered learned: (1) he knows the motion; (2) he knows how the motion is done; (3) he is skilled in what comes first; (4) he is skilled in what comes afterwards; (5) he knows the right time; (6) he knows the offenses and non-offenses; (7) he knows the light and heavy offenses; (8) he knows the curable and incurable offenses; (9) he knows the grave and minor offenses; (10) he has well-learned and well-remembered his teachers’ tradition. 

When\marginnote{133.1} an expert on the Monastic Law has ten other qualities, he is considered ignorant: (1) he does not know the offenses and non-offenses; (2) he does not know the light and heavy offenses; (3) he does not know the curable and incurable offenses; (4) he does not know the grave and minor offenses; (5) he has not properly learned both Monastic Codes in detail, not having analyzed them well, thoroughly mastered them, and investigated them well, both in terms of the rules and their detailed exposition; (6) he does not know the offenses and non-offenses; (7) he does not know the light and heavy offenses; (8) he does not know the curable and incurable offenses; (9) he does not know the grave and minor offenses; (10) he is not skilled in deciding legal issues.\footnote{The redundancy in this set seems to be a result of mechanically putting together two sets from the group of fives above. And the same below. } 

But\marginnote{134.1} when an expert on the Monastic Law has ten qualities, he is considered learned: (1) he knows the offenses and non-offenses; (2) he knows the light and heavy offenses; (3) he knows the curable and incurable offenses; (4) he knows the grave and minor offenses; (5) he has properly learned both Monastic Codes in detail, having analyzed them well, thoroughly mastered them, and investigated them well, both in terms of the rules and their detailed exposition; (6) he knows the offenses and non-offenses; (7) he knows the light and heavy offenses; (8) he knows the curable and incurable offenses; (9) he knows the grave and minor offenses; (10) he is skilled in deciding legal issues. 

A\marginnote{135.1} monk who has ten qualities may be appointed to a committee. —\footnote{See \href{https://suttacentral.net/pli-tv-kd14/en/brahmali\#14.19.2}{Kd 14:14.19.2}–14.19.12. } The Buddha laid down the training rules for his disciples for ten reasons. —\footnote{See e.g. \href{https://suttacentral.net/pli-tv-bu-vb-pj1/en/brahmali\#5.11.32}{Bu Pj 1:5.11.32}. } There are ten dangers of entering a royal compound. —\footnote{See e.g. \href{https://suttacentral.net/pli-tv-bu-vb-pc83/en/brahmali\#1.3.1}{Bu Pc 83:1.3.1}–1.3.53. For an explanation of the rendering “royal compound” for \textit{\textsanskrit{rājantepura}}, see \textit{antepura} in Appendix of Technical Terms. } There are ten reasons for giving. —\footnote{See \href{https://suttacentral.net/an8.33/en/brahmali\#1.1}{AN 8.33:1.1}. } There are ten precious things. —\footnote{See e.g. \href{https://suttacentral.net/pli-tv-bu-vb-pc84/en/brahmali\#4.1.6}{Bu Pc 84:4.1.6}. } A sangha of monks consisting of a group of ten. — A group of ten may give the full ordination. — There are ten kinds of rags. —\footnote{See above at \href{https://suttacentral.net/pli-tv-pvr7/en/brahmali\#63.30}{Pvr 7:63.30}–63.32. } There are ten kinds of robe wearing. —\footnote{Sp 5.330: \textit{Dasa \textsanskrit{cīvaradhāraṇāti} “\textsanskrit{sabbanīlakāni} \textsanskrit{cīvarāni} \textsanskrit{dhārentī}”ti vuttavasena \textsanskrit{dasāti} \textsanskrit{kurundiyaṁ} \textsanskrit{vuttaṁ}. \textsanskrit{Mahāaṭṭhakathāyaṁ} pana “navasu \textsanskrit{kappiyacīvaresu} \textsanskrit{udakasāṭikaṁ} \textsanskrit{vā} \textsanskrit{saṅkaccikaṁ} \textsanskrit{vā} \textsanskrit{pakkhipitvā} \textsanskrit{dasā}”ti \textsanskrit{vuttaṁ}}, “The ten kinds of robe wearing: it is said in the \textsanskrit{Kurundī} that the ten are on account of what was said: ‘They wore completely blue robes.’ But it is said in the Great Commentary that it is the nine allowable robes, adding the bathing robe or the chest wrap.” The former refers to \href{https://suttacentral.net/pli-tv-kd8/en/brahmali\#29.1.1}{Kd 8:29.1.1}–29.1.11 and the nine allowable robes to \href{https://suttacentral.net/pli-tv-kd8/en/brahmali\#20.2.2}{Kd 8:20.2.2}. } One should keep an extra robe for ten days at the most. —\footnote{See \href{https://suttacentral.net/pli-tv-bu-vb-np1/en/brahmali\#3.1.6}{Bu NP 1:3.1.6}. } There are ten kinds of semen. —\footnote{See \href{https://suttacentral.net/pli-tv-bu-vb-ss1/en/brahmali\#2.2.5}{Bu Ss 1:2.2.5}. } There are ten kinds of women. —\footnote{See \href{https://suttacentral.net/pli-tv-bu-vb-ss5/en/brahmali\#4.1.1}{Bu Ss 5:4.1.1}. } There are ten kinds of wives. —\footnote{See \href{https://suttacentral.net/pli-tv-bu-vb-ss5/en/brahmali\#4.1.2}{Bu Ss 5:4.1.2}. } The ten practices proclaimed as allowable at \textsanskrit{Vesālī}. —\footnote{See \href{https://suttacentral.net/pli-tv-kd22/en/brahmali\#1.1.2}{Kd 22:1.1.2}. } There are ten kinds of people a monk should not pay respect to. —\footnote{See \href{https://suttacentral.net/pli-tv-kd16/en/brahmali\#6.5.2}{Kd 16:6.5.2}. } There are ten kinds of abuse. —\footnote{See \href{https://suttacentral.net/pli-tv-bu-vb-pc2/en/brahmali\#2.1.2}{Bu Pc 2:2.1.2}. } There are ten ways of engaging in malicious talebearing. —\footnote{See \href{https://suttacentral.net/pli-tv-bu-vb-pc3/en/brahmali\#2.1.3}{Bu Pc 3:2.1.3}. } There are ten kinds of furniture. —\footnote{Sp 5.330: \textit{Dasa \textsanskrit{senāsanānīti} \textsanskrit{mañco}, \textsanskrit{pīṭhaṁ}, bhisi, \textsanskrit{bimbohanaṁ}, \textsanskrit{cimilikā}, \textsanskrit{uttarattharaṇaṁ}, \textsanskrit{taṭṭikā}, \textsanskrit{cammakhaṇḍo}, \textsanskrit{nisīdanaṁ}, \textsanskrit{tiṇasanthāro}, \textsanskrit{paṇṇasanthāroti}}, “The ten kinds of furniture: a bed, a bench, a mattress, a pillow, a mat underlay, a bedspread, a straw-mat, a hide, a sitting mat, a spread of grass, a spread of leaves.” See \href{https://suttacentral.net/pli-tv-bu-vb-pc14/en/brahmali\#1.1.9.1}{Bu Pc 14:1.1.9.1} and \href{https://suttacentral.net/pli-tv-bu-vb-pc15/en/brahmali\#2.1.8}{Bu Pc 15:2.1.8}. The ten kinds of furniture have become eleven in the commentary! Sp-yoj 5.330 helpfully adds: \textit{\textsanskrit{Paṇṇasanthāro} \textsanskrit{tiṇasanthārena} \textsanskrit{saṅgahito}}, “The spread of leaves is included in the spread of grass.” For an explanation of the rendering “furniture” for \textit{\textsanskrit{senāsana}}, see Appendix of Technical Terms. } They asked for ten favors. —\footnote{Sp 5.330: \textit{Dasa \textsanskrit{varāni} \textsanskrit{yāciṁsūti} \textsanskrit{visākhā} \textsanskrit{aṭṭha}, \textsanskrit{suddhodanamahārājā} \textsanskrit{ekaṁ}, \textsanskrit{jīvako} \textsanskrit{ekaṁ}}, “‘They asked for ten favors’: \textsanskrit{Visākhā} eight, the great king Suddhodana one, \textsanskrit{Jīvaka} one.” } There are ten kinds of illegitimate cancellations of the Monastic Code. —\footnote{See \href{https://suttacentral.net/pli-tv-kd19/en/brahmali\#3.3.81}{Kd 19:3.3.81}–3.3.84. } There are the ten legitimate cancellations of the Monastic Code. —\footnote{See \href{https://suttacentral.net/pli-tv-kd19/en/brahmali\#3.3.87}{Kd 19:3.3.87}–3.3.90. } These are the ten benefits of congee. —\footnote{See \href{https://suttacentral.net/pli-tv-kd6/en/brahmali\#24.6.3}{Kd 6:24.6.3}. } There are ten kinds of unallowable meat. —\footnote{See \href{https://suttacentral.net/pli-tv-kd6/en/brahmali\#23.10.8}{Kd 6:23.10.8}–23.15.9 plus human meat as the tenth, see \href{https://suttacentral.net/pli-tv-kd6/en/brahmali\#23.9.7}{Kd 6:23.9.7}. } There are ten rules on ‘at the most’. —\footnote{Ten out of the fourteen listed at Sp 5.326, see above at \href{https://suttacentral.net/pli-tv-pvr7/en/brahmali\#81.9}{Pvr 7:81.9}. } A competent and capable monk who has ten years of seniority may give the going forth, the full ordination, and formal support, and may have a novice monk attend on him. — A competent and capable nun who has ten years of seniority may give the going forth, the full ordination, and formal support, and may have a novice nun attend on her. —\footnote{This and the next item contradict \href{https://suttacentral.net/pli-tv-bi-vb-pc74/en/brahmali\#1.14.1}{Bi Pc 74:1.14.1}, which stipulates twelve years of seniority for a nun to give the full ordination. } A competent and capable nun who has ten years of seniority may agree to be approved to give the full admission. — The training may be given to a married girl who is ten years old.” 

\scend{The section on tens is finished. }

\scuddanaintro{This is the summary: }

\begin{scuddana}%
“Resentment,\marginnote{138.1} getting rid of, grounds, \\
Wrong, and right, extreme; \\
And kinds of wrongness, kinds of rightness, \\
Unskillful, and also skillful. 

An\marginnote{139.1} illegitimate vote, legitimate, \\
Novice monks, and expelling; \\
To say, and legal issue, \\
Motion, and light. 

These\marginnote{140.1} light and heavy, \\
Understand the dark and bright; \\
And committee, and training, \\
And compound, reasons. 

Precious\marginnote{141.1} thing, and a group of ten, \\
And so the full ordination; \\
Rag, and wearing, \\
Ten days, semen, women. 

Wives,\marginnote{142.1} ten practices, \\
Should not pay respect to, and with abuse; \\
And malicious talebearing, kinds of furniture, \\
And favors, illegitimate. 

Legitimate,\marginnote{143.1} congee, and meat, \\
‘At the most’, monk, nun; \\
Full admission, married girl, \\
The section on tens have been well proclaimed.” 

%
\end{scuddana}

\section*{11. The section on elevens }

“There\marginnote{144.1} are eleven kinds of people who should not be ordained, and if they have been ordained, they should be expelled. —\footnote{See \href{https://suttacentral.net/pli-tv-kd1/en/brahmali\#61.1.19}{Kd 1:61.1.19}–68.1.4. A convenient list of the eleven is found at \textsanskrit{Khuddasikkhā}-\textsanskrit{purāṇaṭīkā} 16. } There are eleven kinds of unallowable shoes. —\footnote{Sp 5.331: \textit{\textsanskrit{Ekādasa} \textsanskrit{pādukāti} dasa \textsanskrit{ratanamayā}, \textsanskrit{ekā} \textsanskrit{kaṭṭhapādukā}}, “The eleven kinds of shoes: ten made with precious substances and one made of wood.” See the last ten items at \href{https://suttacentral.net/pli-tv-kd5/en/brahmali\#8.3.4}{Kd 5:8.3.4} and \href{https://suttacentral.net/pli-tv-kd5/en/brahmali\#6.4.12}{Kd 5:6.4.12}. } There are eleven kinds of unallowable almsbowls. —\footnote{Sp 5.331: \textit{\textsanskrit{Ekādasa} \textsanskrit{pattāti} tambalohamayena \textsanskrit{vā} \textsanskrit{dārumayena} \textsanskrit{vā} \textsanskrit{saddhiṁ} dasa \textsanskrit{ratanamayā}}, “The eleven kinds of bowls: ten made with precious substances together with one made of copper or one made of wood.” See \href{https://suttacentral.net/pli-tv-kd15/en/brahmali\#8.2.26}{Kd 15:8.2.26}–9.1.14. It is curious that the copper bowl is mentioned here as if separate from the list of ten, yet it also occurs in the list of ten. There seems to be an error either in the Canonical text or the commentary. } There are eleven kinds of unallowable robes. —\footnote{Sp 5.331: \textit{\textsanskrit{Ekādasa} \textsanskrit{cīvarānīti} \textsanskrit{sabbanīlakādīni} \textsanskrit{ekādasa}}, “The eleven kinds of robes: the eleven are those that are entirely blue, etc.” See \href{https://suttacentral.net/pli-tv-kd8/en/brahmali\#29.1.1}{Kd 8:29.1.1}–29.1.11. } There are eleven ‘after the third’. —\footnote{Sp 5.331: \textit{\textsanskrit{Yāvatatiyakāti} \textsanskrit{ukkhittānuvattikā} \textsanskrit{bhikkhunī}, \textsanskrit{saṅghādisesā} \textsanskrit{aṭṭha}, \textsanskrit{ariṭṭho}, \textsanskrit{caṇḍakāḷīti}}, “‘After the third’: a nun takes sides with one who has been ejected, eight offenses entailing suspension, \textsanskrit{Ariṭṭha}, and \textsanskrit{Caṇḍakāḷī}.” This refers to \href{https://suttacentral.net/pli-tv-bi-vb-pj7/en/brahmali\#1.11.1}{Bi Pj 7:1.11.1}, \href{https://suttacentral.net/pli-tv-bu-vb-ss10/en/brahmali\#1.3.16.1}{Bu Ss 10:1.3.16.1}–13, \href{https://suttacentral.net/pli-tv-bi-vb-ss10/en/brahmali\#1.19.1}{Bi Ss 10:1.19.1}–13, \href{https://suttacentral.net/pli-tv-bu-vb-pc68/en/brahmali\#1.49.1}{Bu Pc 68:1.49.1}, and \href{https://suttacentral.net/pli-tv-bi-vb-pc36/en/brahmali\#1.11.1}{Bi Pc 36:1.11.1}. } The nuns should be asked about the eleven obstacles. —\footnote{Sp 5.331: \textit{\textsanskrit{Ekādasa} \textsanskrit{antarāyikā} \textsanskrit{nāma} “nasi \textsanskrit{animittā}”ti \textsanskrit{ādayo}}, “The eleven obstacles: ‘Do you lack genitals? etc.” See \href{https://suttacentral.net/pli-tv-kd20/en/brahmali\#17.1.6}{Kd 20:17.1.6}. } There are eleven kinds of robes that should be determined. —\footnote{Sp 5.331: \textit{\textsanskrit{Ekādasa} \textsanskrit{cīvarāni} \textsanskrit{adhiṭṭhātabbānīti} \textsanskrit{ticīvaraṁ}, \textsanskrit{vassikasāṭikā}, \textsanskrit{nisīdanaṁ}, \textsanskrit{paccattharaṇaṁ}, \textsanskrit{kaṇḍuppaṭicchādi}, \textsanskrit{mukhapuñchanacoḷaṁ}, \textsanskrit{parikkhāracoḷaṁ}, \textsanskrit{udakasāṭikā}, \textsanskrit{saṅkaccikāti}}, “Eleven kinds of robes that should be determined: the three robes, the rainy-season robe, the sitting mat, a sheet, an itch-covering cloth, a washcloth, a cloth for requisites, the bathing robe, and the chest wrap.” } There are eleven kinds of robes that should not be assigned to another. —\footnote{Sp 5.331: \textit{Na \textsanskrit{vikappetabbānīti} \textsanskrit{etāneva} \textsanskrit{adhiṭṭhitakālato} \textsanskrit{paṭṭhāya} na \textsanskrit{vikappetabbāni}}, “Should not be assigned to another: beginning from the time of determination, these should not be assigned to another.” } Becoming subject to relinquishment at dawn on the eleventh day. —\footnote{See \href{https://suttacentral.net/pli-tv-bu-vb-np1/en/brahmali\#3.2.2}{Bu NP 1:3.2.2}. } There are eleven kinds of allowable toggles. —\footnote{Sp 5.331: \textit{\textsanskrit{Gaṇṭhikā} ca \textsanskrit{vidhā} ca suttamayena \textsanskrit{saddhiṁ} \textsanskrit{ekādasa} honti, te sabbe khuddakakkhandhake \textsanskrit{niddiṭṭhā}}, “Together with those made of thread, there are eleven kinds of toggles and buckles. They are all specified in The Chapter on Minor Topics.” For the toggles see \href{https://suttacentral.net/pli-tv-kd15/en/brahmali\#29.3.12}{Kd 15:29.3.12}. } There are eleven kinds of allowable buckles. —\footnote{See \href{https://suttacentral.net/pli-tv-kd15/en/brahmali\#29.2.23}{Kd 15:29.2.23}–29.2.24. } There are eleven kinds of unallowable earth. —\footnote{See \href{https://suttacentral.net/pli-tv-bu-vb-pc10/en/brahmali\#2.1.8}{Bu Pc 10:2.1.8}–2.1.10. } There are eleven kinds of allowable earth. —\footnote{See \href{https://suttacentral.net/pli-tv-bu-vb-pc10/en/brahmali\#2.1.12}{Bu Pc 10:2.1.12}–2.1.14. } There are eleven reasons why formal support comes to an end. —\footnote{Sp 5.331: \textit{\textsanskrit{Nissayapaṭipassaddhiyo} \textsanskrit{upajjhāyamhā} \textsanskrit{pañca}, \textsanskrit{ācariyamhā} cha; \textsanskrit{evaṁ} \textsanskrit{ekādasa}}, “The reasons why formal support comes to an end are five for a preceptor and six for a teacher, thus eleven.” See \href{https://suttacentral.net/pli-tv-kd1/en/brahmali\#36.1.4}{Kd 1:36.1.4} and \href{https://suttacentral.net/pli-tv-kd1/en/brahmali\#36.1.7}{Kd 1:36.1.7}. } There are eleven kinds of people a monk should not pay respect to. —\footnote{Sp 5.331: \textit{\textsanskrit{Avandiyapuggalā} naggena \textsanskrit{saddhiṁ} \textsanskrit{ekādasa}}, “The eleven are the people a monk should not pay respect to together with the naked person.” See respectively \href{https://suttacentral.net/pli-tv-kd16/en/brahmali\#6.5.2}{Kd 16:6.5.2} and \href{https://suttacentral.net/pli-tv-kd15/en/brahmali\#15.1.4}{Kd 15:15.1.4}–15.1.7. } There are eleven rules on ‘at the most’. —\footnote{Sp 5.331: \textit{\textsanskrit{Ekādasa} \textsanskrit{paramāni} pubbe vuttesu cuddasasu \textsanskrit{ekādasakavasena} \textsanskrit{yojetvā} \textsanskrit{veditabbāni}}, “The eleven rules on ‘at the most’: they are to be understood among the previously mentioned fourteen, having constructed them as a group of eleven.” } They asked for eleven favors. —\footnote{Sp 5.331: \textit{\textsanskrit{Ekādasa} \textsanskrit{varānīti} \textsanskrit{mahāpajāpatiyā} \textsanskrit{yācitavarena} \textsanskrit{saddhiṁ} pubbe \textsanskrit{vuttāni} dasa}, “The eleven favors: the ten spoken of previously together with the favor asked for by \textsanskrit{Mahāpajāpatī}.” For the previous ten see above at \href{https://suttacentral.net/pli-tv-pvr7/en/brahmali\#135.19}{Pvr 7:135.19}. For the eleventh favor see \href{https://suttacentral.net/pli-tv-kd20/en/brahmali\#3.1.3}{Kd 20:3.1.3}. } There are eleven kinds of flaws in monastery zones. —\footnote{Sp 5.331: \textit{\textsanskrit{Ekādasa} \textsanskrit{sīmādosāti} “\textsanskrit{atikhuddakaṁ} \textsanskrit{sīmaṁ} \textsanskrit{sammannantī}”\textsanskrit{tiādinā} nayena kammavagge \textsanskrit{āgamissanti}}, “The eleven flaws in monastery zones: ‘They approved a monastery zone that was too small,’ etc. They are handed down in ‘the subchapter on legal procedures’ through this method.” This is not a reference to Kd 11, but to \href{https://suttacentral.net/pli-tv-pvr21/en/brahmali\#5.3}{Pvr 21:5.3}. } There are eleven dangers to be expected for people who abuse and revile.\footnote{See \href{https://suttacentral.net/an11.6/en/brahmali\#1.1}{AN 11.6:1.1}. } 

There\marginnote{144.20} are eleven benefits to be expected for one who practices the liberation of mind through love, who develops, cultivates, and makes it a vehicle and basis, who keeps it up, accumulates, and properly implements it:\footnote{See \href{https://suttacentral.net/an11.15/en/brahmali\#1.1}{AN 11.15:1.1}. } (1) you sleep well, (2) you wake up rested, and (3) you have no nightmares; (4) you are loved by humans and (5) spirits; (6) you are protected by the gods; (7) you cannot be harmed by fire, poison, or weapons; (8) your mind is quickly stilled; (9) your face is serene; (10) you die unconfused; and (11) if you do not go any further, you are reborn in the world of the supreme beings.” 

\scend{The section on elevens is finished. }

\scuddanaintro{This is the summary: }

\begin{scuddana}%
“Should\marginnote{147.1} be expelled, and shoes, \\
And almsbowls, and robes; \\
Thirds, and should be asked about, \\
Determination, assigning to another. 

Dawn,\marginnote{148.1} toggles, buckles, \\
And unallowable, allowable; \\
Formal support, and should not pay respect to, \\
‘At the most’, and favors; \\
And flaws in monastery zones, abuse, \\
With love—the elevens are done.” 

%
\end{scuddana}

\scendsutta{The numerical method is finished. }

\scuddanaintro{This is the summary: }

\begin{scuddana}%
“The\marginnote{151.1} ones, and the twos, \\
And the threes, fours, and fives; \\
And the sixes, sevens, eights, and nines, \\
The tens, and the elevens. 

For\marginnote{152.1} the welfare of all beings, \\
By the Unwavering One who knows the Teaching; \\
The stainless numerical method, \\
Was taught by the Great Hero.” 

%
\end{scuddana}

\scend{The numerical method is finished. }

%
\chapter*{{\suttatitleacronym Pvr 8}{\suttatitletranslation Aspects of the legal procedures }{\suttatitleroot Uposathādipucchāvissajjanā}}
\addcontentsline{toc}{chapter}{\tocacronym{Pvr 8} \toctranslation{Aspects of the legal procedures } \tocroot{Uposathādipucchāvissajjanā}}
\markboth{Aspects of the legal procedures }{Uposathādipucchāvissajjanā}
\extramarks{Pvr 8}{Pvr 8}

\section*{1. Questions on the beginning, the middle, and the end }

What\marginnote{1.1} is the beginning, the middle, and the end of the observance-day procedure? What is the beginning, the middle, and the end of the invitation procedure? What is the beginning, the middle, and the end of a legal procedure of condemnation? What is the beginning, the middle, and the end of a legal procedure of demotion? What is the beginning, the middle, and the end of a legal procedure of banishment? What is the beginning, the middle, and the end of a legal procedure of reconciliation? What is the beginning, the middle, and the end of a legal procedure of ejection? What is the beginning, the middle, and the end of giving probation? What is the beginning, the middle, and the end of sending back to the beginning? What is the beginning, the middle, and the end of giving the trial period? What is the beginning, the middle, and the end of rehabilitation? What is the beginning, the middle, and the end of an ordination procedure? What is the beginning, the middle, and the end of the lifting of a legal procedure of condemnation? What is the beginning, the middle, and the end of the lifting of a legal procedure of demotion? What is the beginning, the middle, and the end of the lifting of a legal procedure of banishment? What is the beginning, the middle, and the end of the lifting of a legal procedure of reconciliation? What is the beginning, the middle, and the end of the lifting of a legal procedure of ejection? What is the beginning, the middle, and the end of resolution through recollection? What is the beginning, the middle, and the end of resolution because of past insanity? What is the beginning, the middle, and the end of a further penalty? What is the beginning, the middle, and the end of covering over as if with grass? What is the beginning, the middle, and the end of appointing an instructor of the nuns? What is the beginning, the middle, and the end of the permission to stay apart from one’s three robes? What is the beginning, the middle, and the end of a permission to make a blanket?\footnote{For an explanation of the rendering “blanket” for \textit{santhata}, see Appendix of Technical Terms. } What is the beginning, the middle, and the end of appointing a money discarder? What is the beginning, the middle, and the end of appointing a distributor of rainy-season bathing cloths? What is the beginning, the middle, and the end of appointing a distributor of almsbowls? What is the beginning, the middle, and the end of the permission to use a staff? What is the beginning, the middle, and the end of the permission to use a carrying net? What is the beginning, the middle, and the end of the permission to use a staff and a carrying net? 

\section*{2. Replies on the beginning, the middle, and the end }

What\marginnote{2.1} is the beginning, the middle, and the end of the observance-day procedure? A complete assembly is the beginning of an observance-day procedure, carrying it out is the middle, and concluding it is the end. 

What\marginnote{3.1} is the beginning, the middle, and the end of the invitation procedure? A complete assembly is the beginning of an invitation procedure, carrying it out is the middle, and concluding it is the end. 

What\marginnote{4.1} is the beginning, the middle, and the end of a legal procedure of condemnation? 

The\marginnote{4.2} action that is the basis for the offense and the person who has done it are the beginning of a procedure of condemnation, the motion is the middle, and the announcement is the end. 

What\marginnote{5.1} is the beginning, the middle, and the end of a legal procedure of demotion? … of a legal procedure of banishment? … of a legal procedure of reconciliation? … of a legal procedure of ejection? … of giving probation? … of sending back to the beginning? … of giving the trial period? … of rehabilitation? The action that is the basis for the offense and the person who has done it are the beginning of rehabilitation, the motion is the middle, and the announcement is the end. 

What\marginnote{6.1} is the beginning, the middle, and the end of an ordination procedure? The person is the beginning of the ordination procedure, the motion is the middle, and the announcement is the end. 

What\marginnote{7.1} is the beginning, the middle, and the end of the lifting of a legal procedure of condemnation? Proper conduct is the beginning of the lifting of a legal procedure of condemnation, the motion is the middle, and the announcement is the end. 

What\marginnote{8.1} is the beginning, the middle, and the end of the lifting of a legal procedure of demotion? … of a legal procedure of banishment? … of a legal procedure of reconciliation? … of a legal procedure of ejection? Proper conduct is the beginning of the lifting of a legal procedure of ejection, the motion is the middle, and the announcement is the end. 

What\marginnote{9.1} is the beginning, the middle, and the end of resolution through recollection? The action that is the basis for the offense and the person who has done it are the beginning of resolution through recollection, the motion is the middle, and the announcement is the end. 

What\marginnote{10.1} is the beginning, the middle, and the end of resolution because of past insanity? … of a further penalty? … of covering over as if with grass? … of appointing an instructor of the nuns? … of the permission to stay apart from one’s three robes? … of a permission to make a blanket? … of appointing a money discarder? … of appointing a distributor of rainy-season bathing cloths? … of appointing a distributor of almsbowls? … of the permission to use a staff? … of the permission to use a carrying net? … of the permission to use a staff and a carrying net? The topic and the person are the beginning of the permission to use a staff and a carrying net, the motion is the middle, and the announcement is the end. 

\scend{The aspects of the legal procedures are finished. }

%
\chapter*{{\suttatitleacronym Pvr 9}{\suttatitletranslation The ten reasons for the training rules }{\suttatitleroot Atthavasapakaraṇa}}
\addcontentsline{toc}{chapter}{\tocacronym{Pvr 9} \toctranslation{The ten reasons for the training rules } \tocroot{Atthavasapakaraṇa}}
\markboth{The ten reasons for the training rules }{Atthavasapakaraṇa}
\extramarks{Pvr 9}{Pvr 9}

“The\marginnote{1.1} Buddha laid down the training rules for his disciples for ten reasons: for the well-being of the Sangha, for the comfort of the Sangha, for the restraint of bad people, for the ease of good monks, for the restraint of the corruptions relating to the present life, for the restraint of the corruptions relating to future lives, to give rise to confidence in those without it, to increase the confidence of those who have it, for the longevity of the true Teaching, and for supporting the training.\footnote{For an explanation of the rendering “training” for \textit{vinaya}, see Appendix of Technical Terms. } 

The\marginnote{2.1} well-being of the Sangha is the comfort of the Sangha. The comfort of the Sangha is for the restraint of bad people. The restraint of bad people is for the ease of good monks. The ease of good monks is for the restraint of the corruptions relating to the present life. The restraint of the corruptions relating to the present life is for the restraint of the corruptions relating to future lives. The restraint of the corruptions relating to future lives is to give rise to confidence in those without it. The giving rise to confidence in those without it is to increase the confidence of those who have it. The increase in confidence of those who have it is for the longevity of the true Teaching. The longevity of the true Teaching is for supporting the training. 

The\marginnote{3.1} well-being of the Sangha is the comfort of the Sangha. The well-being of the Sangha is for the restraint of bad people. The well-being of the Sangha is for the ease of good monks. The well-being of the Sangha is for the restraint of the corruptions relating to the present life. The well-being of the Sangha is for the restraint of the corruptions relating to future lives. The well-being of the Sangha is to give rise to confidence in those without it. The well-being of the Sangha is to increase the confidence of those who have it. The well-being of the Sangha is for the longevity of the true Teaching. The well-being of the Sangha is for supporting the training. 

The\marginnote{4.1} comfort of the Sangha is for the restraint of bad people. The comfort of the Sangha is for the ease of good monks. The comfort of the Sangha is for the restraint of the corruptions relating to the present life. The comfort of the Sangha is for the restraint of the corruptions relating to future lives. The comfort of the Sangha is to give rise to confidence in those without it. The comfort of the Sangha is to increase the confidence of those who have it. The comfort of the Sangha is for the longevity of the true Teaching. The comfort of the Sangha is for supporting the training. The comfort of the Sangha is the well-being of the Sangha. 

The\marginnote{5.1} restraint of bad people … The ease of good monks … The restraint of the corruptions relating to the present life … The restraint of the corruptions relating to future lives … The giving rise to confidence in those without it … The increase in confidence of those who have it … The longevity of the true Teaching … The support of the training is the well-being of the Sangha. The support of the training is the comfort of the Sangha. The support of the training is for the restraint of bad people. The support of the training is for the ease of good monks. The support of the training is for the restraint of the corruptions relating to the present life. The support of the training is for the restraint of the corruptions relating to future lives. The support of the training is to give rise to confidence in those without it. The support of the training is to increase the confidence of those who have it. The support of the training is for the longevity of the true Teaching.” 

\begin{verse}%
“A\marginnote{6.1} hundred purposes, a hundred teachings, \\
And two hundred expressions; \\
Four hundred knowledges, \\
In the exposition of the reasons.” 

%
\end{verse}

\scendsutta{The ten reasons for the training rules are finished. }

\scend{The Great Division is finished. }

\scuddanaintro{This is the summary: }

\begin{scuddana}%
“First\marginnote{9.1} eight on questions, \\
And then eight on ‘a result of’; \\
These sixteen for monks, \\
And sixteen for nuns. 

The\marginnote{10.1} internal repetition, subdivision, \\
And the numerical method; \\
Invitation ceremony, about reasons—\\
This is included in the Great Division.” 

%
\end{scuddana}

\scend{The ten reasons for the training rules are finished. }

%
\chapter*{{\suttatitleacronym Pvr 10}{\suttatitletranslation Verses on the training rules }{\suttatitleroot Gāthāsaṅgaṇika}}
\addcontentsline{toc}{chapter}{\tocacronym{Pvr 10} \toctranslation{Verses on the training rules } \tocroot{Gāthāsaṅgaṇika}}
\markboth{Verses on the training rules }{Gāthāsaṅgaṇika}
\extramarks{Pvr 10}{Pvr 10}

\section*{1. Training rules laid down in seven towns }

\begin{verse}%
“Arranging\marginnote{1.1} your robe over one shoulder, \\
Raising your joined palms—\\
What are you hoping for, \\
That you have come here?” 

“What\marginnote{2.1} has been laid down in the two Monastic Laws, \\
Which come up for recitation on the observance days—\\
How many training rules do they have? \\
In how many towns were they laid down?” 

“Your\marginnote{3.1} approach is excellent, \\
You question sensibly; \\
And so I will tell you, \\
According to your skill.\footnote{The end quote in the Pali seems to be a mistake. } 

What\marginnote{4.1} has been laid down in the two Monastic Laws, \\
Which come up for recitation on the observance days—\\
They are three hundred and fifty,\footnote{This number seems to be the sum of all the rules in both Monastic Codes, with the rules in common between the nuns and the monks only counted once. Also, the seven principles for the settling of legal issues are not counted. } \\
Laid down in seven towns.” 

“In\marginnote{5.1} which seven towns were they laid down? \\
Please tell me this; \\
And after attending carefully to your explanation, \\
We will practice for our own benefit.” 

“They\marginnote{6.1} were laid down at \textsanskrit{Vesālī}, at \textsanskrit{Rājagaha}, \\
At \textsanskrit{Sāvatthī}, at \textsanskrit{Āḷavī}; \\
And at \textsanskrit{Kosambī}, and in the Sakyan country, \\
And also among the Bhaggas.” 

“How\marginnote{7.1} many were laid down at \textsanskrit{Vesālī}? \\
How many were pronounced at \textsanskrit{Rājagaha}? \\
How many were there at \textsanskrit{Sāvatthī}? \\
How many were pronounced at \textsanskrit{Āḷavī}? 

How\marginnote{8.1} many were laid down at \textsanskrit{Kosambī}? \\
How many were spoken in the Sakyan country? \\
How many were laid down among the Bhaggas? \\
Please tell me, the one who has asked.” 

“Ten\marginnote{9.1} were laid down at \textsanskrit{Vesālī}, \\
Twenty-one pronounced at \textsanskrit{Rājagaha}; \\
And two hundred and ninety-four, \\
Pronounced at \textsanskrit{Sāvatthī}. 

Six\marginnote{10.1} were laid down at \textsanskrit{Āḷavī}, \\
Eight pronounced at \textsanskrit{Kosambī}; \\
Eight spoken in the Sakyan country, \\
And three laid down among the Bhaggas. 

Those\marginnote{11.1} laid down in \textsanskrit{Vesālī}, \\
Listen to them as they truly are: \\
Sexual intercourse, person, super,\footnote{These are \href{https://suttacentral.net/pli-tv-bu-vb-pj1/en/brahmali\#7.1.16.1}{Bu Pj 1:7.1.16.1}, \href{https://suttacentral.net/pli-tv-bu-vb-pj3/en/brahmali\#2.49.1}{Bu Pj 3:2.49.1}, and \href{https://suttacentral.net/pli-tv-bu-vb-pj4/en/brahmali\#2.12.1}{Bu Pj 4:2.12.1}. } \\
And extra, black.\footnote{\href{https://suttacentral.net/pli-tv-bu-vb-np1/en/brahmali\#2.17.1}{Bu NP 1:2.17.1} and \href{https://suttacentral.net/pli-tv-bu-vb-np12/en/brahmali\#1.15.1}{Bu NP 12:1.15.1}. } 

True,\marginnote{12.1} a meal before another,\footnote{\href{https://suttacentral.net/pli-tv-bu-vb-pc8/en/brahmali\#1.2.26.1}{Bu Pc 8:1.2.26.1} and \href{https://suttacentral.net/pli-tv-bu-vb-pc33/en/brahmali\#3.15.1}{Bu Pc 33:3.15.1}. } \\
With tooth cleaner, naked ascetic;\footnote{\href{https://suttacentral.net/pli-tv-bu-vb-pc40/en/brahmali\#2.5.1}{Bu Pc 40:2.5.1} and \href{https://suttacentral.net/pli-tv-bu-vb-pc41/en/brahmali\#1.2.15.1}{Bu Pc 41:1.2.15.1}. } \\
And abuse from the nuns—\footnote{\href{https://suttacentral.net/pli-tv-bi-vb-pc52/en/brahmali\#1.29.1}{Bi Pc 52:1.29.1}. } \\
These ten were pronounced at \textsanskrit{Vesālī}. 

Those\marginnote{13.1} laid down at \textsanskrit{Rājagaha}, \\
Listen to them as they truly are: \\
Stealing at \textsanskrit{Rājagaha},\footnote{\href{https://suttacentral.net/pli-tv-bu-vb-pj2/en/brahmali\#2.28.1}{Bu Pj 2:2.28.1}. } \\
Two on charging, and also two on schism.\footnote{\href{https://suttacentral.net/pli-tv-bu-vb-ss8/en/brahmali\#1.9.32.1}{Bu Ss 8:1.9.32.1}, \href{https://suttacentral.net/pli-tv-bu-vb-ss9/en/brahmali\#1.2.14.1}{Bu Ss 9:1.2.14.1}, \href{https://suttacentral.net/pli-tv-bu-vb-ss10/en/brahmali\#1.3.16.1}{Bu Ss 10:1.3.16.1}, and \href{https://suttacentral.net/pli-tv-bu-vb-ss11/en/brahmali\#1.19.1}{Bu Ss 11:1.19.1}. } 

Sarong,\marginnote{14.1} money, thread,\footnote{\href{https://suttacentral.net/pli-tv-bu-vb-np5/en/brahmali\#2.10.1}{Bu NP 5:2.10.1}, \href{https://suttacentral.net/pli-tv-bu-vb-np18/en/brahmali\#1.28.1}{Bu NP 18:1.28.1}, and \href{https://suttacentral.net/pli-tv-bu-vb-np26/en/brahmali\#1.23.1}{Bu NP 26:1.23.1}. } \\
And with complaining, having almsfood prepared;\footnote{\href{https://suttacentral.net/pli-tv-bu-vb-pc13/en/brahmali\#2.12.1}{Bu Pc 13:2.12.1} and \href{https://suttacentral.net/pli-tv-bu-vb-pc29/en/brahmali\#2.13.1}{Bu Pc 29:2.13.1}. } \\
A group meal, and at the wrong time,\footnote{\href{https://suttacentral.net/pli-tv-bu-vb-pc32/en/brahmali\#8.15.1}{Bu Pc 32:8.15.1} and \href{https://suttacentral.net/pli-tv-bu-vb-pc37/en/brahmali\#1.22.1}{Bu Pc 37:1.22.1}. } \\
Visiting, bathing, less than twenty.\footnote{\href{https://suttacentral.net/pli-tv-bu-vb-pc46/en/brahmali\#5.6.1}{Bu Pc 46:5.6.1}, \href{https://suttacentral.net/pli-tv-bu-vb-pc57/en/brahmali\#6.7.1}{Bu Pc 57:6.7.1}, and \href{https://suttacentral.net/pli-tv-bu-vb-pc65/en/brahmali\#1.53.1}{Bu Pc 65:1.53.1}. } 

Gives\marginnote{15.1} out a robe, giving directions—\footnote{\href{https://suttacentral.net/pli-tv-bu-vb-pc81/en/brahmali\#1.16.1}{Bu Pc 81:1.16.1} and \href{https://suttacentral.net/pli-tv-bu-vb-pd2/en/brahmali\#1.15.1}{Bu Pd 2:1.15.1}. } \\
These were pronounced at \textsanskrit{Rājagaha}; \\
Hilltop, wandering, right there,\footnote{\href{https://suttacentral.net/pli-tv-bi-vb-pc10/en/brahmali\#1.15.1}{Bi Pc 10:1.15.1}, \href{https://suttacentral.net/pli-tv-bi-vb-pc39/en/brahmali\#1.14.1}{Bi Pc 39:1.14.1}, and \href{https://suttacentral.net/pli-tv-bi-vb-pc40/en/brahmali\#1.13.1}{Bi Pc 40:1.13.1}. } \\
With given consent it is twenty-one.\footnote{\href{https://suttacentral.net/pli-tv-bi-vb-pc81/en/brahmali\#1.13.1}{Bi Pc 81:1.13.1}. } 

Those\marginnote{16.1} laid down at \textsanskrit{Sāvatthī},\footnote{All rules not laid down in the six other towns were laid down at \textsanskrit{Sāvatthī}. } \\
Listen to them as they truly are: \\
Four offenses entailing expulsion, \\
And sixteen offenses entailing suspension. 

And\marginnote{17.1} the two undetermined offenses, \\
Twenty-four offenses entailing relinquishment; \\
And one hundred and fifty-six, \\
Minor offenses were spoken. 

And\marginnote{18.1} ten blameworthy offenses,\footnote{According to Vmv 5.335 the ten are \href{https://suttacentral.net/pli-tv-bu-vb-pd1/en/brahmali\#1.35.1}{Bu Pd 1:1.35.1} and \href{https://suttacentral.net/pli-tv-bu-vb-pd3/en/brahmali\#3.15.1}{Bu Pd 3:3.15.1}, plus \href{https://suttacentral.net/pli-tv-bi-vb-pd1/en/brahmali\#1.2.9.1}{Bi Pd 1:1.2.9.1}–8. } \\
And seventy-two on training—\\
Two hundred and ninety-four, \\
All pronounced at \textsanskrit{Sāvatthī}. 

Those\marginnote{19.1} laid down at \textsanskrit{Āḷavī}, \\
Listen to them as they truly are: \\
Hut, silk, and sleeping place,\footnote{\href{https://suttacentral.net/pli-tv-bu-vb-ss6/en/brahmali\#1.6.6.1}{Bu Ss 6:1.6.6.1}, \href{https://suttacentral.net/pli-tv-bu-vb-np11/en/brahmali\#1.23.1}{Bu NP 11:1.23.1}, and \href{https://suttacentral.net/pli-tv-bu-vb-pc5/en/brahmali\#2.16.1}{Bu Pc 5:2.16.1}. } \\
On digging, go deity;\footnote{\href{https://suttacentral.net/pli-tv-bu-vb-pc10/en/brahmali\#1.16.1}{Bu Pc 10:1.16.1} and \href{https://suttacentral.net/pli-tv-bu-vb-pc11/en/brahmali\#1.29.1}{Bu Pc 11:1.29.1}. } \\
And they pour water that contains living beings—\footnote{\href{https://suttacentral.net/pli-tv-bu-vb-pc20/en/brahmali\#1.12.1}{Bu Pc 20:1.12.1}. } \\
These six were pronounced at \textsanskrit{Āḷavī}. 

Those\marginnote{20.1} laid down at \textsanskrit{Kosambī}, \\
Listen to them as they truly are: \\
A large dwelling, difficult to correct,\footnote{\href{https://suttacentral.net/pli-tv-bu-vb-ss7/en/brahmali\#1.19.1}{Bu Ss 7:1.19.1} and \href{https://suttacentral.net/pli-tv-bu-vb-ss12/en/brahmali\#1.26.1}{Bu Ss 12:1.26.1}. } \\
Evasive, door, and with alcohol;\footnote{\href{https://suttacentral.net/pli-tv-bu-vb-pc12/en/brahmali\#2.28.1}{Bu Pc 12:2.28.1}, \href{https://suttacentral.net/pli-tv-bu-vb-pc19/en/brahmali\#1.18.1}{Bu Pc 19:1.18.1}, and \href{https://suttacentral.net/pli-tv-bu-vb-pc51/en/brahmali\#1.46.1}{Bu Pc 51:1.46.1}. } \\
Disrespect, legitimately,\footnote{\href{https://suttacentral.net/pli-tv-bu-vb-pc54/en/brahmali\#1.15.1}{Bu Pc 54:1.15.1} and \href{https://suttacentral.net/pli-tv-bu-vb-pc71/en/brahmali\#1.19.1}{Bu Pc 71:1.19.1}. } \\
The eighth is with a milk drink.\footnote{\href{https://suttacentral.net/pli-tv-bu-vb-sk51/en/brahmali\#1.19.1}{Bu Sk 51:1.19.1}. } 

Those\marginnote{21.1} laid down in the Sakyan country, \\
Listen to them as they truly are: \\
Wool, and bowl,\footnote{\href{https://suttacentral.net/pli-tv-bu-vb-np17/en/brahmali\#1.20.1}{Bu NP 17:1.20.1} and \href{https://suttacentral.net/pli-tv-bu-vb-np22/en/brahmali\#1.3.17.1}{Bu NP 22:1.3.17.1}. } \\
And instruction, tonics.\footnote{\href{https://suttacentral.net/pli-tv-bu-vb-pc23/en/brahmali\#2.19.1}{Bu Pc 23:2.19.1} and \href{https://suttacentral.net/pli-tv-bu-vb-pc47/en/brahmali\#1.4.27.1}{Bu Pc 47:1.4.27.1}. } 

Needle,\marginnote{22.1} and wilderness:\footnote{\href{https://suttacentral.net/pli-tv-bu-vb-pc86/en/brahmali\#1.20.1}{Bu Pc 86:1.20.1} and \href{https://suttacentral.net/pli-tv-bu-vb-pd4/en/brahmali\#2.12.1}{Bu Pd 4:2.12.1}. } \\
These six at Kapilavatthu;\footnote{MS reads \textit{\textsanskrit{aṭṭha}}, “eight”, whereas SRT has \textit{cha}, “six”, which fits better. } \\
With cleaning with water, and instruction,\footnote{\href{https://suttacentral.net/pli-tv-bi-vb-pc5/en/brahmali\#1.2.12.1}{Bi Pc 5:1.2.12.1} and \href{https://suttacentral.net/pli-tv-bi-vb-pc58/en/brahmali\#1.14.1}{Bi Pc 58:1.14.1}. } \\
Spoken among the nuns. 

Those\marginnote{23.1} laid down among the Bhaggas, \\
Listen to them as they truly are: \\
They lit a fire to warm themselves,\footnote{\href{https://suttacentral.net/pli-tv-bu-vb-pc56/en/brahmali\#2.5.1}{Bu Pc 56:2.5.1}. } \\
Soiled with food, containing rice.\footnote{\href{https://suttacentral.net/pli-tv-bu-vb-sk55/en/brahmali\#1.14.1}{Bu Sk 55:1.14.1} and \href{https://suttacentral.net/pli-tv-bu-vb-sk56/en/brahmali\#1.14.1}{Bu Sk 56:1.14.1}. } 

The\marginnote{24.1} four offenses entailing expulsion, \\
And seven offenses entailing suspension; \\
Eight offenses entailing relinquishment, \\
And thirty-two minor offenses. 

Two\marginnote{25.1} blameworthy offenses, and three on training—\\
Fifty-six training rules; \\
Laid down at six towns, \\
By the Buddha, the Kinsman of the Sun. 

Two\marginnote{26.1} hundred and ninety-four, \\
All pronounced at \textsanskrit{Sāvatthī}; \\
By the compassionate Buddha, \\
The famous Gotama.” 

%
\end{verse}

\section*{2. The four kinds of failure }

\begin{verse}%
“What\marginnote{27.1} we have asked, you have answered, \\
All is explained, not otherwise; \\
I wish to ask you another question—please tell me this: \\
Serious, and light, curable; \\
Incurable, and grave, minor, \\
And those that are ‘after the third’. 

In\marginnote{28.1} common, not in common, \\
The ways that failures are settled—\footnote{Reading \textit{vipattiyo} with the PTS edition in place of \textit{vibhattiyo}. } \\
Please explain all these too, \\
And we will listen to you.” 

“There\marginnote{29.1} are thirty-one serious ones, \\
And eight here are incurable; \\
Those that are serious are grave, \\
Those that are grave are failures in morality; \\
Offenses entailing expulsion, offenses entailing suspension—\\
They are called ‘failure in morality’. 

Serious\marginnote{30.1} offenses, offenses entailing confession, \\
Offenses entailing acknowledgment, offenses of wrong conduct; \\
Offenses of wrong speech, \\
And whoever calls another names for fun—\\
This is considered failure in conduct. 

Holding\marginnote{31.1} on to distorted views, \\
Preferring what is contrary to the true Dhamma; \\
Misrepresenting the Awakened One, \\
Being foolish, enveloped in delusion—\\
This is considered failure in view. 

%
\end{verse}

When,\marginnote{32.1} to make a living—having bad desires, overcome by desire—one claims to have a non-existent superhuman quality; when, to make a living, one acts as a matchmaker; when, to make a living, one says, ‘The monk who stays in your dwelling is a perfected one;’ when, to make a living, a monk eats fine foods that he has requested for himself; when, to make a living, a nun eats fine foods that she herself has asked for; when, to make a living, one eats bean curry or rice that one has requested for oneself—this is considered failure in livelihood. 

\begin{verse}%
There\marginnote{33.1} are eleven ‘after the thirds’, \\
Listen to them as they truly are: \\
A nun who takes sides with one who has been ejected,\footnote{\href{https://suttacentral.net/pli-tv-bi-vb-pj7/en/brahmali\#1.11.1}{Bi Pj 7:1.11.1}. } \\
Eight ‘after the thirds’;\footnote{\href{https://suttacentral.net/pli-tv-bu-vb-ss10/en/brahmali\#1.3.16.1}{Bu Ss 10:1.3.16.1}–13 and \href{https://suttacentral.net/pli-tv-bi-vb-ss10/en/brahmali\#1.19.1}{Bi Ss 10:1.19.1}–13. } \\
\textsanskrit{Ariṭṭha}, and \textsanskrit{Caṇḍakālī}—\footnote{\href{https://suttacentral.net/pli-tv-bu-vb-pc68/en/brahmali\#1.49.1}{Bu Pc 68:1.49.1} and \href{https://suttacentral.net/pli-tv-bi-vb-pc36/en/brahmali\#1.11.1}{Bi Pc 36:1.11.1}. } \\
These are those ‘after the thirds’.” 

%
\end{verse}

\section*{3. To be cut down, etc. }

\begin{verse}%
“How\marginnote{34.1} many on ‘to be cut down’? \\
How many on ‘to be destroyed’? \\
How many on ‘to be stripped’? \\
How many on ‘no other, he commits an offense entailing confession’? 

How\marginnote{34.5} many on ‘the monks have agreed’? \\
How many on ‘what is proper’? \\
How many on ‘at the most’? 

How many on ‘knowing’? Were laid down by the Buddha, the Kinsman of the Sun?” “There\marginnote{36.1} are six on ‘to be cut down’.\footnote{\href{https://suttacentral.net/pli-tv-bu-vb-pc87/en/brahmali\#1.11.1}{Bu Pc 87:1.11.1}, \href{https://suttacentral.net/pli-tv-bu-vb-pc89/en/brahmali\#2.10.1}{Bu Pc 89:2.10.1}–92, and \href{https://suttacentral.net/pli-tv-bi-vb-pc22/en/brahmali\#1.14.1}{Bi Pc 22:1.14.1}. } \\
There is one on ‘to be destroyed’.\footnote{\href{https://suttacentral.net/pli-tv-bu-vb-pc86/en/brahmali\#1.20.1}{Bu Pc 86:1.20.1}. } \\
There is one on ‘to be stripped’.\footnote{\href{https://suttacentral.net/pli-tv-bu-vb-pc88/en/brahmali\#1.14.1}{Bu Pc 88:1.14.1}. } \\
There are four on ‘no other, he commits an offense entailing confession’.\footnote{\href{https://suttacentral.net/pli-tv-bu-vb-pc16/en/brahmali\#1.16.1}{Bu Pc 16:1.16.1}, \href{https://suttacentral.net/pli-tv-bu-vb-pc42/en/brahmali\#1.16.1}{Bu Pc 42:1.16.1}, \href{https://suttacentral.net/pli-tv-bu-vb-pc77/en/brahmali\#1.19.1}{Bu Pc 77:1.19.1}, and \href{https://suttacentral.net/pli-tv-bu-vb-pc78/en/brahmali\#1.18.1}{Bu Pc 78:1.18.1}. } 

There\marginnote{36.5} are four on ‘the monks have agreed’.\footnote{\href{https://suttacentral.net/pli-tv-bu-vb-np2/en/brahmali\#2.39.1}{Bu NP 2:2.39.1}, \href{https://suttacentral.net/pli-tv-bu-vb-np14/en/brahmali\#2.38.1}{Bu NP 14:2.38.1}, \href{https://suttacentral.net/pli-tv-bu-vb-np29/en/brahmali\#1.2.16.1}{Bu NP 29:1.2.16.1}, and \href{https://suttacentral.net/pli-tv-bu-vb-pc9/en/brahmali\#1.20.1}{Bu Pc 9:1.20.1}. } \\
There are seven on ‘what is proper’.\footnote{\href{https://suttacentral.net/pli-tv-bu-vb-np10/en/brahmali\#1.3.1}{Bu NP 10:1.3.1} and \href{https://suttacentral.net/pli-tv-bu-vb-np22/en/brahmali\#1.3.17.1}{Bu NP 22:1.3.17.1}; \href{https://suttacentral.net/pli-tv-bu-vb-pc34/en/brahmali\#1.2.21.1}{Bu Pc 34:1.2.21.1}, \href{https://suttacentral.net/pli-tv-bu-vb-pc71/en/brahmali\#1.19.1}{Bu Pc 71:1.19.1}, and \href{https://suttacentral.net/pli-tv-bu-vb-pc84/en/brahmali\#3.17.1}{Bu Pc 84:3.17.1}, and the concluding section for each of the monks’ and the nuns’ \textit{\textsanskrit{saṅghādisesa}} offenses at \href{https://suttacentral.net/pli-tv-bu-vb-ss13/en/brahmali\#3.2.11}{Bu Ss 13:3.2.11} and \href{https://suttacentral.net/pli-tv-bi-vb-ss13/en/brahmali\#3.19}{Bi Ss 13:3.19} respectively. } \\
There are fourteen on ‘at the most’.\footnote{See note to \href{https://suttacentral.net/pli-tv-pvr7/en/brahmali\#81.9}{Pvr 7:81.9}. } \\
There are sixteen on ‘knowing’.\footnote{\href{https://suttacentral.net/pli-tv-bu-vb-np30/en/brahmali\#1.27.1}{Bu NP 30:1.27.1}; \href{https://suttacentral.net/pli-tv-bu-vb-pc16/en/brahmali\#1.16.1}{Bu Pc 16:1.16.1}, \href{https://suttacentral.net/pli-tv-bu-vb-pc20/en/brahmali\#1.12.1}{Bu Pc 20:1.12.1}, \href{https://suttacentral.net/pli-tv-bu-vb-pc29/en/brahmali\#2.13.1}{Bu Pc 29:2.13.1}, \href{https://suttacentral.net/pli-tv-bu-vb-pc36/en/brahmali\#1.28.1}{Bu Pc 36:1.28.1}, \href{https://suttacentral.net/pli-tv-bu-vb-pc62/en/brahmali\#1.11.1}{Bu Pc 62:1.11.1}–66, \href{https://suttacentral.net/pli-tv-bu-vb-pc69/en/brahmali\#1.11.1}{Bu Pc 69:1.11.1}, \href{https://suttacentral.net/pli-tv-bu-vb-pc70/en/brahmali\#1.46.1}{Bu Pc 70:1.46.1}, and \href{https://suttacentral.net/pli-tv-bu-vb-pc82/en/brahmali\#1.26.1}{Bu Pc 82:1.26.1}; \href{https://suttacentral.net/pli-tv-bi-vb-pj6/en/brahmali\#1.23.1}{Bi Pj 6:1.23.1}, \href{https://suttacentral.net/pli-tv-bi-vb-ss2/en/brahmali\#1.40.1}{Bi Ss 2:1.40.1}; and \href{https://suttacentral.net/pli-tv-bi-vb-pc51/en/brahmali\#3.9.1}{Bi Pc 51:3.9.1}. } \\
They were laid down by the Buddha, the Kinsman of the Sun.” 

%
\end{verse}

\section*{4. Not in common, etc. }

\begin{verse}%
“Two\marginnote{38.1} hundred and twenty,\footnote{Leaving out the seven principles for settling legal issues. } \\
Training rules for the monks; \\
Come up for recitation on the observance days, \\
Three hundred and four;\footnote{Again, leaving out the seven principles for settling legal issues. } \\
Training rules for the nuns, \\
Come up for recitation on the observance days. 

Forty-six\marginnote{39.1} of the monks’ rules, \\
Are not in common with the nuns; \\
One hundred and thirty of the nuns’ rules, \\
Are not in common with the monks. 

So\marginnote{40.1} one hundred and seventy-six in total, \\
Are not in common; \\
One hundred and seventy-four,\footnote{Again, leaving out the seven principles for settling legal issues. } \\
Are trained in equally by both. 

Two\marginnote{41.1} hundred and twenty, \\
Training rules for the monks; \\
Come up for recitation on the observance days, \\
Listen to them as they truly are: 

The\marginnote{42.1} four offenses entailing expulsion, \\
There are thirteen offenses entailing suspension; \\
There are two undetermined offenses. 

The\marginnote{43.1} thirty offenses entailing relinquishment, \\
And ninety-two minor offenses; \\
The four offenses entailing acknowledgment, \\
The seventy-five rules to be trained in. 

These\marginnote{44.1} are the two hundred and twenty, \\
Training rules for the monks; \\
That come up for recitation on the observance days. 

Three\marginnote{45.1} hundred and four, \\
Training rules for the nuns; \\
Come up for recitation on the observance days, \\
Listen to them as they truly are: 

The\marginnote{46.1} eight offenses entailing expulsion, \\
There are seventeen offenses entailing suspension; \\
The thirty offenses entailing relinquishment, \\
And one hundred and sixty-six; \\
Are called minor offenses. 

The\marginnote{47.1} eight offenses entailing acknowledgment, \\
The seventy-five rules to be trained in; \\
These are the three hundred and four, \\
Training rules for the nuns; \\
That come up for recitation on the observance days. 

Forty-six\marginnote{48.1} of the monks’ rules, \\
Are not in common with the nuns; \\
Listen to them as they truly are: 

Six\marginnote{49.1} offenses entailing suspension, \\
With the two undetermined offenses are eight; \\
Twelve offenses entailing relinquishment, \\
With these there are twenty. 

Twenty-two\marginnote{50.1} minor offenses, \\
Four offenses entailing acknowledgment; \\
These are the forty-six, \\
That the monks do not have in common with the nuns. 

One\marginnote{51.1} hundred and thirty of the nuns’ rules, \\
Are not in common with the monks; \\
Listen to them as they truly are: 

Four\marginnote{52.1} offenses entailing expulsion, \\
Ten offenses where one is sent away from the Sangha; \\
Twelve offenses entailing relinquishment, \\
And ninety-six minor offenses; \\
Eight offenses entailing acknowledgment. 

These\marginnote{53.1} are the one hundred and thirty, \\
That the nuns do not have in common with the monks. \\
The one hundred and seventy-six, \\
That are not in common; \\
Listen to them as they truly are: 

Four\marginnote{54.1} offenses entailing expulsion, \\
There are sixteen offenses entailing suspension; \\
There are the two undetermined offenses, \\
The twenty-four offenses entailing relinquishment; \\
And one hundred and eighteen, \\
Are called minor offenses; \\
The twelve offenses entailing acknowledgment. 

These\marginnote{55.1} are the hundred and seventy-six, \\
That the two do not have in common. \\
One hundred and seventy-four, \\
Are trained in equally by both; \\
Listen to them as they truly are: 

Four\marginnote{56.1} offenses entailing expulsion, \\
There are seven offenses entailing suspension; \\
Eighteen offenses entailing relinquishment, \\
Seventy minor offenses; \\
Seventy-five rules to be trained in. 

These\marginnote{57.1} are the hundred and seventy-four, \\
That are trained in equally by both. \\
The eight offenses entailing expulsion are dangerous to meet with:\footnote{\href{https://suttacentral.net/Sp 5.338/en/brahmali}{SP 5.338}: \textit{Tattha \textsanskrit{durāsadāti} \textsanskrit{iminā} \textsanskrit{tesaṁ} \textsanskrit{sappaṭibhayataṁ} dasseti}, “There \textit{\textsanskrit{durāsadā}}: by this is shown their danger.” } \\
That person is like the simile of the palm stump, 

Like\marginnote{58.1} a withered leaf, like an ordinary stone that has broken in half, \\
Like someone with their head cut off; \\
Like a palm-tree with its top cut off, \\
They are incapable of growth. 

Twenty-three\marginnote{59.1} offenses entailing suspension, \\
Two undetermined offenses; \\
Forty-two offenses entailing relinquishment, \\
One hundred and eighty-eight offenses entailing confession; \\
Twelve offenses entailing acknowledgment, 

Seventy-five\marginnote{60.1} rules to be trained in. \\
They are settled through three principles for settling legal issues: \\
In the presence of, and with the admission of, \\
And through covering over as if with grass. 

There\marginnote{61.1} are two observance days, two invitation days, \\
And four legal procedures taught by the Victor. \\
There are five recitations, and four, not otherwise;\footnote{This probably refers to the monks’ five ways of reciting the Monastic Code, mentioned at \href{https://suttacentral.net/pli-tv-kd2/en/brahmali\#15.1.4}{Kd 2:15.1.4}, and then the four ways for nuns, referred to at \href{https://suttacentral.net/pli-tv-pvr2.1/en/brahmali\#1.8}{Pvr 2.1:1.8} and in the ensuing discussion. } \\
And there are seven classes of offenses. 

The\marginnote{62.1} four kinds of legal issues, \\
Are settled through seven principles: \\
Through two of them, through four, through three, \\
And business is settled through one.” 

%
\end{verse}

\section*{5. The offenses entailing expulsion, etc. }

\begin{verse}%
“It\marginnote{63.1} is said, ‘An offense entailing expulsion’. \\
Listen to it as it truly is: \\
When one has fallen away, offended, and fallen down, \\
Removed from the true Teaching, \\
And excluded from the community there—\\
This is why it is called that. 

It\marginnote{64.1} is said, ‘An offense entailing suspension’. \\
Listen to it as it truly is: \\
Only the Sangha gives probation, \\
Sends back to the beginning; \\
Gives the trial period, and rehabilitates—\\
This is why it is called that. 

It\marginnote{65.1} is said, ‘An undetermined offense’. \\
Listen to it as it truly is: \\
Undetermined, not determined, \\
An undecided rule; \\
It is one of three cases—\\
It is called, ‘An undetermined offense’. 

It\marginnote{66.1} is said, ‘A serious offense’. \\
Listen to it as it truly is: \\
The one who confesses to instigating another, \\
And the one who agrees to it—\\
There is no offense like that.\footnote{Sp 5.339: \textit{\textsanskrit{Catutthagāthāya} accayo tena samo \textsanskrit{natthīti} \textsanskrit{desanāgāminīsu} accayesu tena samo \textsanskrit{thūlo} accayo natthi}, “In the fourth verse, ‘There is no offense like that,’ means that in regard to the offenses that are to be confessed, there is no gross offense that is equivalent to that.” } \\
This is why it is called that. 

It\marginnote{67.1} is said, ‘An offense entailing relinquishment’. \\
Listen to it as it truly is: \\
In the midst of the Sangha, in the midst of a group, \\
Or just one with one; \\
One relinquishes and then confesses—\\
This is why it is called that. 

It\marginnote{68.1} is said, ‘An offense entailing confession’. \\
Listen to it as it truly is: \\
One drops the wholesome, \\
Misses the noble path; \\
Having a deluded mind—\\
This is why it is called that. 

It\marginnote{69.1} is said, ‘An offense entailing acknowledgment’. \\
Listen to it as it truly is: \\
An unrelated monk, \\
Whatever food she has obtained with difficulty; \\
Should he receive it himself and eat it, \\
It is called blameworthy. 

When\marginnote{70.1} eating at an invitation, \\
And a nun there gives directions based on favoritism; \\
If they eat there without having stopped her, \\
It is called blameworthy. 

Going\marginnote{71.1} to a family that has faith, \\
But is poor with little wealth; \\
If one eats there without being sick, \\
It is called blameworthy. 

If\marginnote{72.1} anyone stays in a wilderness, \\
That is risky and dangerous; \\
And they eat there without making it known,\footnote{I take \textit{\textsanskrit{aviditaṁ}} to be a contracted version of \textit{\textsanskrit{appaṭisaṁviditaṁ}}, see \href{https://suttacentral.net/pli-tv-bu-vb-pd4/en/brahmali\#2.12.1}{Bu Pd 4:2.12.1}. } \\
It is called blameworthy. 

An\marginnote{73.1} unrelated nun, \\
Whatever belongs to others—\\
Ghee, oil, honey, syrup, \\
Fish, meat, milk, and curd—\\
If she herself asks for them, \\
She has committed a blameworthy act in the instruction of the Accomplished One. 

It\marginnote{74.1} is said, ‘An offense of wrong conduct’. \\
Listen to it as it truly is: \\
Offended, and failed, \\
Faltered, and what is badly done. 

Whatever\marginnote{75.1} people do that is bad, \\
Whether in public or in private; \\
They declare to be wrong conduct. \\
This is why it is called that. 

It\marginnote{76.1} is said, ‘An offense of wrong speech’. \\
Listen to it as it truly is: \\
Wrong speech, wrong utterance, \\
Whatever sentence is defiled; \\
And condemned by the wise—\\
This is why it is called that. 

It\marginnote{77.1} is said, ‘To be trained in’. \\
Listen to it as it truly is: \\
For a trainee who is training, \\
Following the straight path, 

This\marginnote{78.1} is the beginning and the right conduct, \\
With a controlled and restrained mouth—\\
There is no training like this. \\
This is why it is called that. 

It\marginnote{79.1} rains on what’s concealed, \\
Not on what’s revealed; \\
Therefore, reveal the concealed, \\
And it won’t be rained upon. 

The\marginnote{80.1} forest is the destination of deer, \\
The air is the destination of birds; \\
Non-existence is the destination of phenomena, \\
Extinguishment is the destination of a Perfected One.” 

%
\end{verse}

\scendsutta{The verses on the training rules  are finished. }

\scuddanaintro{This is the summary: }

\begin{scuddana}%
“Laid\marginnote{83.1} down in seven towns, \\
And also the four kinds of failure; \\
In common, not in common, \\
Between the monks and the nuns; \\
To support Buddhism, \\
There are these verses on the training rules.” 

%
\end{scuddana}

\scend{The verses on the training rules  are finished. }

%
\chapter*{{\suttatitleacronym Pvr 11}{\suttatitletranslation The four legal issues and their resolution }{\suttatitleroot Adhikaraṇabheda}}
\addcontentsline{toc}{chapter}{\tocacronym{Pvr 11} \toctranslation{The four legal issues and their resolution } \tocroot{Adhikaraṇabheda}}
\markboth{The four legal issues and their resolution }{Adhikaraṇabheda}
\extramarks{Pvr 11}{Pvr 11}

\section*{1. The subdivision on reopening, etc. }

There\marginnote{1.1} are four kinds of legal issues: legal issues arising from disputes, legal issues arising from accusations, legal issues arising from offenses, legal issues arising from business. 

How\marginnote{2.1} many kinds of reopening are there of these four legal issues? Ten. There are two kinds of reopening of legal issues arising from disputes, four of legal issues arising from accusations, three of legal issues arising from offenses, and one of legal issues arising from business. 

When\marginnote{3.1} reopening a legal issue arising from a dispute, how many of the principles for settling it does one reopen? When reopening a legal issue arising from an accusation, how many of the principles for settling it does one reopen? When reopening a legal issue arising from an offense, how many of the principles for settling it does one reopen? When reopening a legal issue arising from business, how many of the principles for settling it does one reopen? 

When\marginnote{4.1} reopening a legal issue arising from a dispute, one reopens two principles for settling it. When reopening a legal issue arising from an accusation, one reopens four principles for settling it. When reopening a legal issue arising from an offense, one reopens three principles for settling it. When reopening a legal issue arising from business, one reopens one principle for settling it. 

How\marginnote{5.1} many kinds of reopening are there? In how many ways does a reopening come about? How many attributes do people who reopen legal issues have? How many kinds of people commit an offense when they reopen a legal issue? 

There\marginnote{6.1} are twelve kinds of reopening. A reopening comes about in ten ways. People who reopen legal issues have four attributes. There are four kinds of people who commit an offense when they reopen legal issues. 

What\marginnote{7.1} are the twelve kinds of reopening? “The legal procedure hasn’t been done”; “it’s been done badly”; “it should be done again”; “it’s not been settled”; “it’s been badly settled”; “it should be settled again”; “it’s not been decided”; “it’s been badly decided”; “it should be decided again”; “it’s not been disposed of”; “it’s been badly disposed of”; “it should be disposed of again”. 

What\marginnote{8.1} are the ten ways that a reopening comes about? One reopens a legal issue where it arose; one reopens a legal issue where it arose and was resolved; one reopens a legal issue while traveling; one reopens a legal issue that was resolved while traveling; one reopens a legal issue after going there; one reopens a legal issue that was resolved after going there; one reopens a resolution through recollection; one reopens a resolution because of past insanity; one reopens a decision on giving a further penalty; one reopens a decision on covering over as if with grass. 

What\marginnote{9.1} are the four attributes of people who reopen legal issues? They reopen legal issues biased by desire, ill will, confusion, or fear. 

Who\marginnote{10.1} are the four kinds of people who commit an offense when they reopen a legal issue? If one who was ordained on that very day does the reopening, they commit an offense entailing confession; if one who has newly arrived does the reopening, they commit an offense entailing confession; if the original doer does the reopening, they commit an offense entailing confession; if one who had given their consent does the reopening, they commit an offense entailing confession.\footnote{Sp 5.341: \textit{\textsanskrit{Kārakoti} \textsanskrit{ekaṁ} \textsanskrit{saṅghena} \textsanskrit{saddhiṁ} \textsanskrit{adhikaraṇaṁ} \textsanskrit{vinicchinitvā} \textsanskrit{pariveṇagataṁ} \textsanskrit{parājitā} \textsanskrit{bhikkhū} vadanti “kissa, bhante, tumhehi \textsanskrit{evaṁ} \textsanskrit{vinicchitaṁ} \textsanskrit{adhikaraṇaṁ}, nanu \textsanskrit{evaṁ} vinicchinitabba”nti. So “\textsanskrit{kasmā} \textsanskrit{paṭhamaṁyeva} \textsanskrit{evaṁ} na \textsanskrit{vaditthā}”ti \textsanskrit{taṁ} \textsanskrit{adhikaraṇaṁ} \textsanskrit{ukkoṭeti}. \textsanskrit{Evaṁ} yo \textsanskrit{kārako} \textsanskrit{ukkoṭeti}}, “The original doer: having decided a legal issue together with the Sangha, he goes to the yard where the defeated monks say, ‘Venerable, why did you decide the legal issue in that way? Should it not be decided in this way?’ Saying, ‘Why did you not say this straightaway?’ he reopens that legal issue. In this way, the original doer reopens it.” } 

\section*{2. The sources of the legal issues, etc. }

What\marginnote{11.1} is the source, the origin, the birth, the arising, the production, the origination of legal issues arising from disputes? What is the source, the origin, the birth, the arising, the production, the origination of legal issues arising from accusations? What is the source, the origin, the birth, the arising, the production, the origination of legal issues arising from offenses? What is the source, the origin, the birth, the arising, the production, the origination of legal issues arising from business? 

Legal\marginnote{12.1} issues arising from disputes have disputes as their source, their origin, their birth, their arising, their production, their origination. Legal issues arising from accusations have accusations as their source, their origin, their birth, their arising, their production, their origination. Legal issues arising from offenses have offenses as their source, their origin, their birth, their arising, their production, their origination. Legal issues arising from business have business as their source, their origin, their birth, their arising, their production, their origination. 

What\marginnote{13.1} is the source, the origin, the birth, the arising, the production, the origination of legal issues arising from disputes? … of legal issues arising from accusations? … of legal issues arising from offenses? … of legal issues arising from business? 

Legal\marginnote{14.1} issues arising from disputes have causes as their source, their origin, their birth, their arising, their production, their origination.\footnote{Sp 5.342: \textit{\textsanskrit{Dutiyapucchāya} \textsanskrit{hetunidānantiādimhi} vissajjane \textsanskrit{navannaṁ} \textsanskrit{kusalākusalābyākatahetūnaṁ} vasena \textsanskrit{hetunidānāditā} \textsanskrit{veditabbā}}, “In the second question, in ‘causes as their source, etc.’: in the reply, causes as their source, etc., is to be understood on account of the nine wholesome, unwholesome, and indeterminate causes.” } Legal issues arising from accusations … Legal issues arising from offenses … Legal issues arising from business have causes as their source, their origin, their birth, their arising, their production, their origination. 

What\marginnote{15.1} is the source, the origin, the birth, the arising, the production, the origination of legal issues arising from disputes? … of legal issues arising from accusations? … of legal issues arising from offenses? … of legal issues arising from business? 

Legal\marginnote{16.1} issues arising from disputes have conditions as their source, their origin, their birth, their arising, their production, their origination. Legal issues arising from accusations … Legal issues arising from offenses … Legal issues arising from business have conditions as their source, their origin, their birth, their arising, their production, their origination. 

\section*{3. The roots of legal issues, etc. }

How\marginnote{17.1} many roots do the four kinds of legal issues have, and how many originations? They have thirty-three roots, and thirty-three originations. 

What\marginnote{18.1} are the thirty-three roots? Legal issues arising from disputes have twelve roots; legal issues arising from accusations have fourteen roots; legal issues arising from offenses have six roots; legal issues arising from business have one root, the Sangha. 

What\marginnote{19.1} are the thirty-three originations? Legal issues arising from disputes originate from the eighteen grounds for schism; legal issues arising from accusations originate from the four kinds of failure; legal issues arising from offenses originate from the seven classes of offenses; legal issues arising from business originate from the four kinds of legal procedures. 

\section*{4. Offenses because of legal issues }

“Is\marginnote{20.1} a legal issue arising from a dispute an offense or not an offense?” It is not an offense. “Is it possible to commit an offense because of a legal issue arising from a dispute?” Yes. How many offenses does one commit because of legal issues arising from disputes? Two: there is an offense entailing confession for abusing one who is fully ordained; there is an offense of wrong conduct for abusing one who is not fully ordained. 

When\marginnote{21.1} it comes to these offenses, to how many of the four kinds of failure do they belong? To which of the four kinds of legal issues do they belong? In how many of the seven classes of offenses are they found? Through how many of the six kinds of originations of offenses do they originate? Through how many kinds of legal issues, in how many places, and through how many of the principles for settling legal issues are they settled? 

They\marginnote{22.1} belong to one kind of failure: failure in conduct. They belong to legal issues arising from an offense. They are found in two classes of offenses: they may be in the class of offenses entailing confession; they may be in the class of offenses of wrong conduct. They originate in three ways. They are settled through one kind of legal issue: a legal issue arising from business. They are settled in three places: in the midst of the Sangha, in the midst of a group, or in the presence of an individual. They are settled through three principles: they may be settled by resolution face-to-face and by acting according to what has been admitted; or they may be settled by resolution face-to-face and by covering over as if with grass. 

“Is\marginnote{23.1} a legal issue arising from an accusation an offense or not an offense?” It is not an offense. “Is it possible to commit an offense because of a legal issue arising from an accusation?” Yes. How many offenses does one commit because of legal issues arising from accusations? Three: there is an offense entailing suspension for groundlessly charging a monk with an offense entailing expulsion; there is an offense entailing confession for groundlessly charging someone with an offense entailing suspension; there is an offense of wrong conduct for groundlessly charging someone with failure in conduct. 

When\marginnote{24.1} it comes to these offenses, to how many of the four kinds of failure do they belong? To which of the four kinds of legal issues do they belong? In how many of the seven classes of offenses are they found? Through how many of the six kinds of originations of offenses do they originate? Through how many kinds of legal issues, in how many places, and through how many of the principles for settling legal issues are they settled? 

They\marginnote{25.1} belong to two kinds of failure: they may be failure in morality; they may be failure in conduct. They belong to legal issues arising from an offense. They are found in three classes of offenses: they may be in the class of offenses entailing suspension; they may be in the class of offenses entailing confession; they may be in the class of offenses of wrong conduct. They originate in three ways. The heavy offenses are settled through one kind of legal issue: a legal issue arising from business. They are settled in one place: in the midst of the Sangha. They are settled through two principles: by resolution face-to-face and by acting according to what has been admitted. The light offenses are settled through one kind of legal issue: a legal issue arising from business. They are settled in three places: in the midst of the Sangha, in the midst of a group, or in the presence of an individual. They are settled through three principles: they may be settled by resolution face-to-face and by acting according to what has been admitted; or they may be settled by resolution face-to-face and by covering over as if with grass. 

“Is\marginnote{26.1} a legal issue arising from an offense an offense or not an offense?” It is an offense. “Is it possible to commit an offense because of a legal issue arising from an offense?” Yes. How many offenses does one commit because of legal issues arising from offenses? Four: there is an offense entailing expulsion for a nun who knowingly conceals an offense entailing expulsion; there is a serious offense for concealing it if she is unsure; there is an offense entailing confession for a monk who conceals an offense entailing suspension; there is an offense of wrong conduct for concealing a failure in conduct. 

When\marginnote{27.1} it comes to these offenses, to how many of the four kinds of failure do they belong? To which of the four kinds of legal issues do they belong? In how many of the seven classes of offenses are they found? Through how many of the six kinds of originations of offenses do they originate? Through how many kinds of legal issues, in how many places, and through how many of the principles for settling legal issues are they settled? 

They\marginnote{28.1} belong to two kinds of failure: they may be failure in morality; they may be failure in conduct. They belong to legal issues arising from an offense. They are found in four classes of offenses: they may be in the class of offenses entailing expulsion; they may be in the class of serious offenses; they may be in the class of offenses entailing confession; they may be in the class of offenses of wrong conduct. They originate in one way: from body, speech, and mind. The incurable offense is not settled through any kind of legal issue, in any place, or through any of the principles for settling legal issues.\footnote{That is, the offense entailing expulsion. } The light offenses are settled through one kind of legal issue: a legal issue arising from business. They are settled in three places: in the midst of the Sangha, in the midst of a group, or in the presence of an individual. They are settled through three principles: they may be settled by resolution face-to-face and by acting according to what has been admitted; or they may be settled by resolution face-to-face and by covering over as if with grass. 

“Is\marginnote{29.1} a legal issue arising from business an offense or not an offense?” It is not an offense. “Is it possible to commit an offense because of a legal issue arising from business?” Yes. How many offenses does one commit because of legal issues arising from business? Five: there is an offense of wrong conduct after the motion when a nun takes sides with one who has been ejected and does not stop when pressed up to three times; there is a serious offense after each of the first two announcements; there is an offense entailing expulsion when the last announcement is finished; there is an offense entailing suspension when monks who side with a monk who is pursuing schism do not stop when pressed for the third time; there is an offense entailing confession when not giving up a bad view after being pressed for the third time. 

When\marginnote{30.1} it comes to these offenses, to how many of the four kinds of failure do they belong? To which of the four kinds of legal issues do they belong? In how many of the seven classes of offenses are they found? Through how many of the six kinds of originations of offenses do they originate? Through how many kinds of legal issues, in how many places, and through how many of the principles for settling legal issues are they settled? 

They\marginnote{31.1} belong to two kinds of failure: they may be failure in morality; they may be failure in conduct. They belong to legal issues arising from an offense. They are found in five classes of offenses: they may be in the class of offenses entailing expulsion; they may be in the class of offenses entailing suspension; they may be in the class of serious offenses; they may be in the class of offenses entailing confession; they may be in the class of offenses of wrong conduct. They originate in one way: from body, speech, and mind. The incurable offense is not settled through any kind of legal issue, in any place, or through any of the principles for settling legal issues. The heavy offense is settled through one kind of legal issue:\footnote{That is, the offense entailing suspension. } a legal issue arising from business. It is settled in one place: in the midst of the Sangha. It is settled through two principles: by resolution face-to-face and by acting according to what has been admitted. The light offenses are settled through one kind of legal issue: a legal issue arising from business. They are settled in three places: in the midst of the Sangha, in the midst of a group, or in the presence of an individual. They are settled through three principles: they may be settled by resolution face-to-face and by acting according to what has been admitted; or they may be settled by resolution face-to-face and by covering over as if with grass. 

\section*{5. The difference between legal issues }

Is\marginnote{32.1} a legal issue arising from a dispute a legal issue arising from an accusation, a legal issue arising from an offense, or a legal issue arising from business?\footnote{At first sight the Pali appears to be a straightforward declarative statement (“a legal issue arising from a dispute is a legal issue arising from an accusation, a legal issue arising from an offense, and a legal issue arising from business”), not a question. Yet this would mean that the next sentence is a direct contradiction to present one. The sub-commentary clears this is up by saying it is meant as a question. Vmv 5.348: \textit{\textsanskrit{Vivādādhikaraṇaṁ} hoti \textsanskrit{anuvādādhikaraṇantiādīsu} \textsanskrit{vivādādhikaraṇameva} \textsanskrit{anuvādādhikaraṇādipi} \textsanskrit{hotīti} \textsanskrit{pucchāya} \textsanskrit{vivādādhikaraṇaṁ} \textsanskrit{vivādādhikaraṇameva} hoti, \textsanskrit{anuvādādayo} na \textsanskrit{hotīti} \textsanskrit{vissajjanaṁ}}, “About the phrase \textit{\textsanskrit{vivādādhikaraṇaṁ} hoti \textsanskrit{anuvādādhikaraṇan}}, etc.: to the question whether a legal issue arising from a dispute is also a legal issue arising from an accusation, etc., the reply is that a legal issue arising from a dispute is a legal issue arising from a dispute, not a legal issue arising from an accusation, etc.” } A legal issue arising from a dispute is neither a legal issue arising from an accusation, nor a legal issue arising from an offense, nor a legal issue arising from business. Nevertheless, because of a legal issue arising from a dispute, there are legal issues arising from accusations, legal issues arising from offenses, and legal issues arising from business. How is this? It may be that monks are disputing, saying, “This is the Teaching”, “This is contrary to the Teaching” … “This is a grave offense”, or “This is a minor offense.” In regard to this, whatever there is of quarreling, arguing, conflict, disputing, variety in opinion, difference in opinion, heated speech, or strife—this is called a legal issue arising from a dispute. When, during a legal issue arising from a dispute, the Sangha disputes, there is a legal issue arising from a dispute. When one who is disputing makes an accusation, there is a legal issue arising from an accusation. When one who is accusing commits an offense, there is a legal issue arising from an offense. When the Sangha does a legal procedure because of that offense, there is a legal issue arising from business. In this way, because of a legal issue arising from a dispute, there are legal issues arising from accusations, legal issues arising from offenses, and legal issues arising from business. 

Is\marginnote{33.1} a legal issue arising from an accusation a legal issue arising from an offense, a legal issue arising from business, or legal issue arising from a dispute? A legal issue arising from an accusation is neither a legal issue arising from an offense, nor a legal issue arising from business, nor a legal issue arising from a dispute. Nevertheless, because of a legal issue arising from an accusation, there are legal issues arising from offenses, legal issues arising from business, and legal issues arising from disputes. How is this? It may be that the monks accuse a monk of failure in morality, failure in conduct, failure in view, or failure in livelihood. In regard to this, whatever there is of accusations, accusing, allegations, blame, taking sides because of friendship, taking part in the accusation, or supporting the accusation—this is called a legal issue arising from an accusation. When, during a legal issue arising from an accusation, the Sangha disputes, there is a legal issue arising from a dispute. When one who is disputing makes an accusation, there is a legal issue arising from an accusation. When one who is accusing commits an offense, there is a legal issue arising from an offense. When the Sangha does a legal procedure because of that offense, there is a legal issue arising from business. In this way, because of a legal issue arising from an accusation, there are legal issues arising from offenses, legal issues arising from business, and legal issues arising from disputes. 

Is\marginnote{34.1} a legal issue arising from an offense a legal issue arising from business, a legal issue arising from a dispute, or a legal issue arising from an accusation? A legal issue arising from an offense is neither a legal issue arising from business, nor a legal issue arising from a dispute, nor a legal issue arising from an accusation. Nevertheless, because of a legal issue arising from an offense, there are legal issues arising from business, legal issues arising from disputes, and legal issues arising from accusations. How is this? There are legal issues arising from offenses because of the five classes of offenses, and there are legal issues arising from offenses because of the seven classes of offenses—these are called legal issues arising from offenses. When, during a legal issue arising from an offense, the Sangha disputes, there is a legal issue arising from a dispute. When one who is disputing makes an accusation, there is a legal issue arising from an accusation. When one who is accusing commits an offense, there is a legal issue arising from an offense. When the Sangha does a legal procedure because of that offense, there is a legal issue arising from business. In this way, because of a legal issue arising from an offense, there are legal issues arising from business, legal issues arising from disputes, and legal issues arising from accusations. 

Is\marginnote{35.1} a legal issue arising from business a legal issue arising from a dispute, a legal issue arising from an accusation, or a legal issue arising from an offense? A legal issue arising from business is neither a legal issue arising from a dispute, nor a legal issue arising from an accusation, nor a legal issue arising from an offense. Nevertheless, because of a legal issue arising from business, there are legal issues arising from disputes, legal issues arising from accusations, and legal issues arising from offenses. How is this? Whatever is the duty or the business of the Sangha—a legal procedure consisting of getting permission, a legal procedure consisting of one motion, a legal procedure consisting of one motion and one announcement, a legal procedure consisting of one motion and three announcements—this is called a legal issue arising from business. When, during a legal issue arising from business, the Sangha disputes, there is a legal issue arising from a dispute. When one who is disputing makes an accusation, there is a legal issue arising from an accusation. When one who is accusing commits an offense, there is a legal issue arising from an offense. When the Sangha does a legal procedure because of that offense, there is a legal issue arising from business. In this way, because of a legal issue arising from business, there are legal issues arising from disputes, legal issues arising from accusations, and legal issues arising from offenses. 

\section*{6. The section on questioning }

When\marginnote{36.1} there is resolution through recollection, is there also resolution face-to-face? When there is resolution face-to-face, is there also resolution through recollection? When there is resolution because of past insanity, is there also resolution face-to-face? When there is resolution face-to-face, is there also resolution because of past insanity? When there is acting according to what has been admitted, is there also resolution face-to-face? When there is resolution face-to-face, is there also acting according to what has been admitted? When there is a majority decision, is there also resolution face-to-face? When there is resolution face-to-face, is there also a majority decision? When there is a further penalty, is there also resolution face-to-face? When there is resolution face-to-face, is there also a further penalty? When there is covering over as if with grass, is there also resolution face-to-face? When there is resolution face-to-face, is there also covering over as if with grass? 

\section*{7. The section on responding }

There\marginnote{37.1} may be an occasion when a legal issue is resolved by resolution face-to-face and resolution through recollection. Then, when there is resolution through recollection there is also resolution face-to-face, and when there is resolution face-to-face there is also resolution through recollection; but there is not resolution because of past insanity, nor acting according to what has been admitted, nor a majority decision, nor a further penalty, nor covering over as if with grass. There may be an occasion when a legal issue is resolved by resolution face-to-face and resolution because of past insanity. … by resolution face-to-face and by acting according to what has been admitted … by resolution face-to-face and by a majority decision … by resolution face-to-face and by a further penalty … There may be an occasion when a legal issue is resolved by resolution face-to-face and by covering over as if with grass. Then, when there is covering over as if with grass there is also resolution face-to-face, and when there is resolution face-to-face there is also covering over as if with grass; but there is not resolution through recollection, nor resolution because of past insanity, nor acting according to what has been admitted, nor a majority decision, nor a further penalty. 

\section*{8. The section on connected }

Is\marginnote{38.1} a resolution face-to-face and a resolution through recollection connected or disconnected? Is it possible to completely separate them and point to their difference? Is a resolution face-to-face and resolution because of past insanity … Is a resolution face-to-face and acting according to what has been admitted … Is a resolution face-to-face and a majority decision … Is a resolution face-to-face and a further penalty … Is a resolution face-to-face and covering over as if with grass connected or disconnected? Is it possible to completely separate them and point to their difference? 

Resolution\marginnote{39.1} face-to-face and resolution through recollection are connected, not disconnected, and it is not possible to completely separate them and point to their difference. Resolution face-to-face and resolution because of past insanity … Resolution face-to-face and acting according to what has been admitted … Resolution face-to-face and a majority decision … Resolution face-to-face and a further penalty … Resolution face-to-face and covering over as if with grass are connected, not disconnected, and it is not possible to completely separate them and point to their difference. 

\section*{9. The sources of the seven principles for settling legal issues }

What\marginnote{40.1} is the source, the origin, the birth, the arising, the production, the origination of resolution face-to-face? What is the source, the origin, the birth, the arising, the production, the origination of resolution through recollection? What is the source, the origin, the birth, the arising, the production, the origination of resolution because of past insanity? What is the source, the origin, the birth, the arising, the production, the origination of acting according to what has been admitted? What is the source, the origin, the birth, the arising, the production, the origination of a majority decision? What is the source, the origin, the birth, the arising, the production, the origination of a further penalty? What is the source, the origin, the birth, the arising, the production, the origination of covering over as if with grass? 

Resolution\marginnote{41.1} face-to-face has sources as its source, its origin, its birth, its arising, its production, its origination. Resolution through recollection … Resolution because of past insanity … Acting according to what has been admitted … A majority decision … A further penalty … Covering over as if with grass has sources as its source, its origin, its birth, its arising, its production, its origination. 

What\marginnote{42.1} is the source, the origin, the birth, the arising, the production, the origination of resolution face-to-face? … of resolution through recollection? … of resolution because of past insanity? … of acting according to what has been admitted? … of a majority decision? … of a further penalty? What is the source, the origin, the birth, the arising, the production, the origination of covering over as if with grass? 

Resolution\marginnote{43.1} face-to-face has causes as its source, its origin, its birth, its arising, its production, its origination. Resolution through recollection … Resolution because of past insanity … Acting according to what has been admitted … A majority decision … A further penalty … Covering over as if with grass has causes as its source, its origin, its birth, its arising, its production, its origination. 

What\marginnote{44.1} is the source, the origin, the birth, the arising, the production, the origination of resolution face-to-face? … of resolution through recollection? … of resolution because of past insanity? … of acting according to what has been admitted? … of a majority decision? … of a further penalty? What is the source, the origin, the birth, the arising, the production, the origination of covering over as if with grass? 

Resolution\marginnote{44.8} face-to-face has conditions as its source, its origin, its birth, its arising, its production, its origination. Resolution through recollection … Resolution because of past insanity … Acting according to what has been admitted … A majority decision … A further penalty … Covering over as if with grass has conditions as its source, its origin, its birth, its arising, its production, its origination. 

How\marginnote{45.1} many roots do the seven principles for settling legal issues have, and how many originations? They have twenty-six roots and thirty-six originations. What are those twenty-six roots? Resolution face-to-face has four roots: face-to-face with the Sangha, face-to-face with the Teaching, face-to-face with the Monastic Law, face-to-face with the persons concerned. Resolution through recollection has four roots. Resolution because of past insanity has four roots. Acting according to what has been admitted has two roots: The one who confesses and the one he confesses to. A majority decision has four roots. A further penalty has four roots. Covering over as if with grass has four roots: face-to-face with the Sangha, face-to-face with the Teaching, face-to-face with the Monastic Law, face-to-face with the persons concerned. 

What\marginnote{46.1} are those thirty-six originations? The doing of, the performing of, the participation in, the consent to, the agreement to, the non-objection to a legal procedure of resolution through recollection. … a legal procedure of resolution because of past insanity. … a legal procedure of acting according to what has been admitted. … a legal procedure of majority decision. … a legal procedure of further penalty. The doing of, the performing of, the participation in, the consent to, the agreement to, the non-objection to a legal procedure of covering over as if with grass. 

\section*{10. The variety in meaning, etc., between the seven principles for settling legal issues }

“Are\marginnote{47.1} resolution face-to-face and resolution through recollection different in meaning and different in wording, or the same in meaning and just different in wording?\footnote{There is a \textit{ti} at the end of the text, signifying the end of a quote, but it is not all clear where the quote is supposed to begin. It may well be that at some point the whole text was supposed to be quoted. Yet, the way the text is presented now, with interspersed headings, it is natural to start the quote only after the final heading, which is here. In other words, it does not make sense to include the headings within the quotes. } Are resolution face-to-face and resolution because of past insanity … Are resolution face-to-face and acting according to what has been admitted … Are resolution face-to-face and a majority decision … Are resolution face-to-face and a further penalty … Are resolution face-to-face and covering over as if with grass different in meaning and different in wording, or the same in meaning and just different in wording? Resolution face-to-face and resolution through recollection are different in meaning and different in wording. Resolution face-to-face and resolution because of past insanity … Resolution face-to-face and acting according to what has been admitted … Resolution face-to-face and a majority decision … Resolution face-to-face and a further penalty … Resolution face-to-face and covering over as if with grass are different in meaning and different in wording. 

Are\marginnote{48.1} there disputes that are also legal issues arising from a dispute? Are there disputes that are not legal issues? Are there legal issues that are not disputes? Are there legal issues that are also disputes? 

There\marginnote{48.2} may be disputes that are also legal issues arising from a dispute. There may be disputes that are not legal issues. There may legal issues that are not disputes. There may be legal issues that are also disputes. 

How\marginnote{49.1} is there a dispute that is also a legal issue arising from a dispute? It may be that the monks are disputing, saying, ‘This is the Teaching’, ‘This is contrary to the Teaching’ … ‘This is a grave offense’, or ‘This is a minor offense.’ In regard to this, whatever there is of quarreling, arguing, conflict, disputing, variety in opinion, difference in opinion, heated speech, or strife—this is a dispute that is also a legal issue arising from a dispute. 

How\marginnote{50.1} is there a dispute that is not a legal issue? A mother disputes with her offspring; an offspring with their mother; a father with his offspring; an offspring with their father; a brother with his brother; a brother with his sister; a sister with her brother; a friend with their friend—these are disputes that are not legal issues. 

How\marginnote{51.1} is there a legal issue that is not a dispute? A legal issue arising from an accusation, a legal issue arising from an offense, a legal issue arising from business—these are legal issues that are not disputes. 

How\marginnote{52.1} is there a legal issue that is also a dispute? A legal issue arising from a dispute is a legal issue and also a dispute. 

Are\marginnote{53.1} there accusations that are also legal issues arising from accusations? Are there accusations that are not legal issues? Are there legal issues that are not accusations? Are there legal issues that are also accusations? 

There\marginnote{53.2} may be accusations that are also legal issues arising from accusations. There may be accusations that are not legal issues. There may be legal issues that are not accusations. There may legal issues that are also accusations. 

How\marginnote{54.1} is there an accusation that is also a legal issue arising from an accusation? It may be that the monks accuse a monk of failure in morality, failure in conduct, failure in view, or failure in livelihood. In regard to this, whatever there are of accusations, accusing, allegations, blame, taking sides because of friendship, taking part in the accusation, or supporting the accusation—this is an accusation that is a legal issue arising from an accusation. 

How\marginnote{55.1} is there an accusation that is not a legal issue? A mother accuses her offspring; an offspring their mother; a father his offspring; an offspring their father; a brother his brother; a brother his sister; a sister her brother; a friend their friend—these are accusations that are not legal issues. 

How\marginnote{56.1} is there a legal issue that is not an accusation? A legal issue arising from an offense, a legal issue arising from business, a legal issue arising from a dispute—these are legal issues that are not accusations. 

How\marginnote{57.1} is there a legal issue that is also an accusation? A legal issue arising from an accusation is a legal issue and also an accusation. 

Are\marginnote{58.1} there offenses that are also legal issues arising from an offense? Are there offenses that are not legal issues? Are there legal issues that are not offenses? Are there legal issues that are also offenses? 

There\marginnote{58.2} may be offenses that are also legal issues arising from an offense. There may be offenses that are not legal issues. There may be legal issues that are not offenses. There may be legal issues that are also offenses. 

How\marginnote{59.1} is there an offense that is also a legal issue arising from an offense? There are legal issues arising from offenses because of the five classes of offenses; there are legal issues arising from offenses because of the seven classes of offenses—these are offenses that are also legal issues arising from an offense. 

How\marginnote{60.1} is there an offense that is not a legal issue? The attainment of stream-entry—this is an offense/attainment that is not a legal issue.\footnote{In the Pali, the word \textit{\textsanskrit{āpatti}}/\textit{\textsanskrit{samāpatti}} means both offense and attainment. The Pali is a play on words, which is impossible to replicate in English. } 

How\marginnote{61.1} is there a legal issue that is not an offense? A legal issue arising from business, a legal issue arising from a dispute, a legal issue arising from an accusation—these are legal issues that are not offenses. 

How\marginnote{62.1} is there a legal issue that is also an offense? A legal issue arising from an offense is a legal issue and also an offense. 

Is\marginnote{63.1} there business that is also a legal issue arising from business? Is there business that is not a legal issue? Are there legal issues that are not business? Are there legal issues that are also business? 

There\marginnote{63.2} may be business that is also a legal issue arising from business. There may be business that is not a legal issue. There may be legal issues that are not business. There may be legal issues that are also business. 

How\marginnote{64.1} is there business that is also a legal issue arising from business? Whatever is the duty or the business of the Sangha—a legal procedure consisting of getting permission, a legal procedure consisting of one motion, a legal procedure consisting of one motion and one announcement, a legal procedure consisting of one motion and three announcements—these are business that are also legal issues arising from business. 

How\marginnote{65.1} is there business that is not a legal issue? The duty to a teacher, the duty to a preceptor, the duty to a co-student, the duty to a co-pupil—these are business that are not legal issues. 

How\marginnote{66.1} is there a legal issue that is not business? A legal issue arising from a dispute, a legal issue arising from an accusation, a legal issue arising from an offense—these are legal issues that are not business. 

How\marginnote{67.1} is there a legal issue that is also business? A legal issue arising from business is a legal issue and also business.” 

\scend{The four legal issues and their resolution are finished. }

\scuddanaintro{This is the summary: }

\begin{scuddana}%
“Legal\marginnote{70.1} issue, kinds of reopening, \\
Ways, and with people; \\
Sources, causes, conditions, \\
Roots, and with origination. 

Offense,\marginnote{71.1} is, and when, \\
Connected, and with source; \\
Causes, conditions, roots, \\
With origination, wording; \\
Dispute, legal issue.” \\
“This is in the subdivision on legal issues.” 

%
\end{scuddana}

%
\chapter*{{\suttatitleacronym Pvr 12}{\suttatitletranslation Questions and answers on accusing, etc. }{\suttatitleroot Codanādipucchāvissajjanā}}
\addcontentsline{toc}{chapter}{\tocacronym{Pvr 12} \toctranslation{Questions and answers on accusing, etc. } \tocroot{Codanādipucchāvissajjanā}}
\markboth{Questions and answers on accusing, etc. }{Codanādipucchāvissajjanā}
\extramarks{Pvr 12}{Pvr 12}

\begin{verse}%
“What\marginnote{1.1} is the purpose of accusing? \\
Why is there reminding? \\
What is the purpose of the Sangha? \\
Why is there the taking of advice?\footnote{Sp 5.359: \textit{\textsanskrit{Matikammaṁ} vuccati \textsanskrit{mantaggahaṇaṁ}}, “Taking advice is called \textit{\textsanskrit{matikammaṁ}}.” Vmv 5.359: \textit{\textsanskrit{Mantaggahaṇanti} \textsanskrit{tesaṁ} \textsanskrit{vicāraṇāgahaṇaṁ}, \textsanskrit{suttantikattherānaṁ}, \textsanskrit{vinayadharattherānañca} \textsanskrit{adhippāyagahaṇanti} attho}, “\textit{\textsanskrit{Mantaggahaṇa}}: learning from them by investigation; the meaning is learning the meaning from the experts on the discourse and the experts on the Monastic Law.” } 

The\marginnote{2.1} purpose of accusing is reminding, \\
The purpose of reminding is restraint; \\
The purpose of the Sangha is scrutiny,\footnote{Sp 5.359: \textit{\textsanskrit{Saṅgho} \textsanskrit{pariggahatthāyāti} tattha sannipatito \textsanskrit{saṅgho} \textsanskrit{vinicchayapariggahaṇatthāya}; \textsanskrit{dhammādhammaṁ} \textsanskrit{tulanatthāya} \textsanskrit{suvinicchitadubbinicchitaṁ} \textsanskrit{jānanatthāyāti} attho}, “\textit{\textsanskrit{Saṅgho} \textsanskrit{pariggahatthāya}}: the Sangha is assembled there for the purpose of investigating to make a decision. The meaning is that it is for the purpose of weighing up what is the Teaching and what is contrary to the Teaching, and for the purpose of knowing what is well and badly decided.” } \\
But taking advice is individual. 

Don’t\marginnote{3.1} speak hastily, \\
Don’t speak fiercely; \\
Don’t be hostile—\\
If you are an investigator. 

Don’t\marginnote{4.1} speak fast, \\
Argumentative speech is not beneficial. \\
In line with the discourses and the Monastic Law, \\
In line with what has been laid down, 

Consider\marginnote{5.1} carefully the proper procedure of examination, \\
As formulated by the skilled Awakened One, \\
Well-spoken in line with the training rules—\\
Do not ruin your future rebirth. \\
You should seek what is beneficial, \\
At the right time, what is connected with the goal. 

The\marginnote{6.1} statements of the accuser and the accused, \\
Do not consider them hastily: \\
If the accuser says the accused has offended, \\
But the accused says he has not, 

Then,\marginnote{7.1} proceeding, \\
Both should be dealt with in line with their admission. \\
The conscientious admit their faults,\footnote{Sp 5.359: \textit{\textsanskrit{Lajjiṁ} \textsanskrit{paṭiññāya} \textsanskrit{kāraye}; \textsanskrit{alajjiṁ} \textsanskrit{vattānusandhināti} attho. \textsanskrit{Tasmā} eva \textsanskrit{paṭiññā} \textsanskrit{lajjīsūti} \textsanskrit{gāthamāha}}, “The meaning is that the conscientious should be dealt with according to their admission, but the shameless according to their conduct. This is why the verse says \textit{\textsanskrit{paṭiññā} \textsanskrit{lajjīsu}}.” } \\
But there is no such thing among the shameless; \\
For even if the shameless should speak a lot, \\
They should be dealt with in line with their conduct. 

What\marginnote{8.1} are the shameless like, \\
In that an admission is not effective? \\
I ask you this: \\
What are they like, the people called shameless? 

They\marginnote{9.1} intentionally commit offenses, \\
Hide their offenses, \\
And follow a wrong course—\\
Such people are called shameless. 

I\marginnote{10.1} too know the truth—\\
Such people are called shameless. \\
And may I ask you something else: \\
What are they like, the people called conscientious? 

They\marginnote{11.1} do not intentionally commit offenses, \\
Nor do they hide them, \\
Nor do they follow a wrong course—\\
Such people are called conscientious. 

I\marginnote{12.1} too know the truth—\\
Such people are called conscientious. \\
And may I ask you something else: \\
What are they like, those who accuse illegitimately? 

They\marginnote{13.1} accuse at the wrong time, untruthfully, \\
Harshly, without benefit; \\
They accuse with a mind of ill will, not a mind of good will—\\
Such a person is called one who accuses illegitimately. 

I\marginnote{14.1} too know the truth—\\
Such a person is called one who accuses illegitimately. \\
And may I ask you something else: \\
What are they like, those who accuse legitimately? 

They\marginnote{15.1} accuse at the right time, truthfully, \\
Gently, beneficially; \\
They accuse with a mind of good will, not a mind of ill will—\\
Such a person is called one who accuses legitimately. 

I\marginnote{16.1} too know the truth—\\
Such a person is called one who accuses legitimately. \\
And may I ask you something else: \\
What are they like, the people called ignorant accusers? 

They\marginnote{17.1} do not know the right order;\footnote{Sp 5.359: \textit{\textsanskrit{Pubbāparaṁ} na \textsanskrit{jānātīti} \textsanskrit{purekathitañca} \textsanskrit{pacchākathitañca} na \textsanskrit{jānāti}}, “They do not know the right order: they do not know what was said first and what afterwards.” } \\
They are ignorant about it. \\
They do not know the sequence of statements; \\
They are ignorant about it—\\
Such people are called ignorant accusers. 

I\marginnote{18.1} too know the truth—\\
Such people are called ignorant accusers. \\
And may I ask you something else: \\
What are they like, the people called learned accusers? 

They\marginnote{19.1} know the right order; \\
They are knowledgeable about it. \\
They know the sequence of statements; \\
They are knowledgeable about it—\\
Such people are called learned accusers. 

I\marginnote{20.1} too know the truth—\\
Such people are called learned accusers. \\
And may I ask you something else: \\
What is meant by accusing? 

One\marginnote{21.1} accuses because of failure in morality, \\
And for failure in conduct and view; \\
Also one accuses for failure in livelihood—\\
Because of this it is called accusing.” 

%
\end{verse}

\scend{The verses on how to accuse properly are finished. }

%
\chapter*{{\suttatitleacronym Pvr 13}{\suttatitletranslation The process of investigation }{\suttatitleroot Codanākaṇḍa}}
\addcontentsline{toc}{chapter}{\tocacronym{Pvr 13} \toctranslation{The process of investigation } \tocroot{Codanākaṇḍa}}
\markboth{The process of investigation }{Codanākaṇḍa}
\extramarks{Pvr 13}{Pvr 13}

\section*{1. Investigation }

The\marginnote{1.1} investigator should ask the accuser: “Do you accuse this monk of failure in morality, conduct, or view?” 

If\marginnote{1.3} he says, “I am accusing him of failure in morality,” “I am accusing him of failure in conduct,” or “I am accusing him of failure in view,” he should be asked, “Do you know what failure in morality is?” “Do you know what failure in conduct is?” or “Do you know what failure in view is?” 

If\marginnote{1.6} he says, “I do,” he should be asked what they are. 

If\marginnote{1.8} he says, “The four offenses entailing expulsion and the thirteen entailing suspension are failure in morality,” “The serious offenses, the offenses entailing confession, the offenses entailing acknowledgment, the offenses of wrong conduct, and the offenses of wrong speech are failure in conduct,” “Wrong views and extreme views are failure in view,” he should be asked, “Are you accusing this monk because of what you’ve seen, what you’ve heard, or what you suspect?” 

If\marginnote{1.16} he says, “I’m accusing him because of what I’ve seen,” “I’m accusing him because of what I’ve heard,” or “I’m accusing him because of what I suspect,” he should be asked, “Since you’re accusing this monk because of what you’ve seen, what have you seen? How did you see it? When did you see it? Where did you see it? Did you see him commit an offense entailing expulsion? Did you see him commit an offense entailing suspension? Did you see him commit a serious offense? … an offense entailing confession? … an offense entailing acknowledgment? … an offense of wrong conduct? … an offense of wrong speech? Where were you? Where was this monk? What were you doing? What was this monk doing?” 

If\marginnote{1.23} he says, “I didn’t accuse this monk because of what I’ve seen, but because of what I’ve heard,” he should be asked, “Since you’re accusing this monk because of what you’ve heard, what have you heard? How did you hear it? When did you hear it? Where did you hear it? Did you hear that he has committed an offense entailing expulsion? Did you hear that he has committed an offense entailing suspension? … a serious offense? … an offense entailing confession? … an offense entailing acknowledgment? … an offense of wrong conduct? Did you hear that he has committed an offense of wrong speech? Did you hear it from a monk, a nun, a trainee nun, a novice monk, a novice nun, a male lay follower, or a female lay follower? Or did you hear it from kings, a king’s officials, the monastics of another religion, or the lay followers of another religion?” 

If\marginnote{1.31} he says, “I didn’t accuse this monk because of what I’ve heard, but because of what I suspect,” he should be asked, “Since you’re accusing this monk because of suspicion, what do you suspect? How do you suspect it? When did you suspect it? Where did you suspect it? Do you suspect that he has committed an offense entailing expulsion? Do you suspect that he has committed an offense entailing suspension? Do you suspect that he has committed a serious offense? … an offense entailing confession? … an offense entailing acknowledgment? … an offense of wrong conduct? Do you suspect that he has committed an offense of wrong speech? Do you suspect it after hearing about it from a monk, a nun, a trainee nun, a novice monk, a novice nun, a male lay follower, or a female lay follower? Or do you suspect it after hearing about it from kings, a king’s officials, the monastics of another religion, or the lay followers of another religion?” 

\begin{verse}%
If\marginnote{2.1} what he saw agrees with what he says he saw,\footnote{Sp 5.361: \textit{Tattha \textsanskrit{diṭṭhaṁ} \textsanskrit{diṭṭhenāti} \textsanskrit{gāthāya} ayamattho – ekeneko \textsanskrit{mātugāmena} \textsanskrit{saddhiṁ} \textsanskrit{ekaṭṭhānato} nikkhamanto \textsanskrit{vā} pavisanto \textsanskrit{vā} \textsanskrit{diṭṭho}, so \textsanskrit{taṁ} \textsanskrit{pārājikena} codeti, itaro tassa \textsanskrit{dassanaṁ} \textsanskrit{anujānāti}. \textsanskrit{Taṁ} pana \textsanskrit{dassanaṁ} \textsanskrit{paṭicca} \textsanskrit{pārājikaṁ} na upeti, na \textsanskrit{paṭijānāti}. Evamettha \textsanskrit{yaṁ} tena \textsanskrit{diṭṭhaṁ}, \textsanskrit{taṁ} tassa “\textsanskrit{diṭṭho} \textsanskrit{mayā}”ti \textsanskrit{iminā} \textsanskrit{diṭṭhavacanena} sameti}, “The meaning of \textit{\textsanskrit{diṭṭhaṁ} \textsanskrit{diṭṭhena}} in the verse is this: a monk is seen emerging or entering a place alone with a woman, and one accuses him of an offense entailing expulsion, and the monk admits to what the other has seen. What was seen does not amount to an offense entailing expulsion, and he does not admit to it. So in this case, what was seen by him agrees with his statement, ‘It was seen by me.’” } \\
If they correspond with each other, \\
But what was seen is not adequate to prove the accusation, \\
Then the one suspecting impurity \\
Should admit it, \\
And they should then do the observance-day ceremony with him. 

If\marginnote{3.1} what he heard agrees with what he says he heard, \\
If they correspond with each other, \\
But what was heard is not adequate to prove the accusation, \\
Then the one suspecting impurity \\
Should admit it, \\
And they should then do the observance-day ceremony with him. 

If\marginnote{4.1} what he sensed agrees with what he says he sensed, \\
If they correspond with each other, \\
But what was sensed is not adequate to prove the accusation, \\
Then the one suspecting impurity \\
Should admit it, \\
And they should then do the observance-day ceremony with him. 

%
\end{verse}

What\marginnote{5.1} is the beginning, the middle, and the end of an accusation? Getting permission for the accusation is the beginning, doing it is the middle, settling it is the end. 

How\marginnote{5.3} many roots does accusing have, how many bases, and how many grounds? And in how many ways does one accuse? It has two roots, three bases, and five grounds. And one accuses in two ways. 

What\marginnote{5.5} are the two roots? With a root and without a root. What are the three bases? The seen, the heard, and the suspected. What are the five grounds? “I’ll speak at an appropriate time, not at an inappropriate one; I’ll speak the truth, not falsehood; I’ll speak gently, not harshly; I’ll speak what’s beneficial, not what’s unbeneficial; I’ll speak with a mind of good will, not with ill will.” What are the two ways of accusing? One accuses by body or by speech. 

\section*{2. The proceeding of an accuser, etc. }

How\marginnote{7.1} should the accuser proceed? How should the accused proceed? How should the Sangha proceed? How should the investigator proceed? 

“How\marginnote{7.5} should the accuser proceed?” The accuser should set up five qualities before accusing another: “I’ll speak at an appropriate time, not at an inappropriate one; I’ll speak the truth, not falsehood; I’ll speak gently, not harshly; I’ll speak what’s beneficial, not what’s unbeneficial; I’ll speak with a mind of good will, not with ill will.” 

“How\marginnote{7.8} should the accused proceed?” The accused should set up two qualities: truth and composure. 

“How\marginnote{7.12} should the Sangha proceed?” The Sangha should know what has been brought before it and what has not.\footnote{Sp 5.363: \textit{\textsanskrit{Otiṇṇānotiṇṇaṁ} \textsanskrit{jānitabbanti} \textsanskrit{otiṇṇañca} \textsanskrit{anotiṇṇañca} \textsanskrit{vacanaṁ} \textsanskrit{jānitabbaṁ}. \textsanskrit{Tatrāyaṁ} \textsanskrit{jānanavidhi} – \textsanskrit{ettakā} codakassa \textsanskrit{pubbakathā}, \textsanskrit{ettakā} \textsanskrit{pacchimakathā}, \textsanskrit{ettakā} cuditakassa \textsanskrit{pubbakathā}, \textsanskrit{ettakā} \textsanskrit{pacchimakathāti} \textsanskrit{jānitabbā}. Codakassa \textsanskrit{pamāṇaṁ} \textsanskrit{gaṇhitabbaṁ}, cuditakassa \textsanskrit{pamāṇaṁ} \textsanskrit{gaṇhitabbaṁ}, anuvijjakassa \textsanskrit{pamāṇaṁ} \textsanskrit{gaṇhitabbaṁ}, anuvijjako appamattakampi \textsanskrit{ahāpento} “\textsanskrit{āvuso} \textsanskrit{samannāharitvā} \textsanskrit{ujuṁ} \textsanskrit{katvā} \textsanskrit{āharā}”ti vattabbo, \textsanskrit{saṅghena} \textsanskrit{evaṁ} \textsanskrit{paṭipajjitabbaṁ}}, “\textit{\textsanskrit{Otiṇṇānotiṇṇaṁ} \textsanskrit{jānitabbaṁ}}: a statement is to be known as \textit{\textsanskrit{otiṇṇa}} or \textit{\textsanskrit{anotiṇṇa}}. This is the way of knowing: ‘The accuser should say this first and that afterwards. The accused should say this first and that afterwards.’ This is to be known. The accuser is to be assessed, so is the accused and the investigator. If the investigator mangles (his speech) even a little, he should be told, ‘Reflect, fix it up, and then speak.’ This how the Sangha should proceed.” } 

“How\marginnote{7.15} should the investigator proceed?” The investigator resolves that legal issue in accordance with the Teaching, the Monastic Law, and the Teacher’s instruction. 

\begin{verse}%
What\marginnote{8.1} is the purpose of the observance-day ceremony? \\
Why is there an invitation ceremony? \\
What is the purpose of probation? \\
Why is there a sending back to the beginning? \\
What is the purpose of the trial period? \\
Why is there rehabilitation? 

The\marginnote{9.1} purpose of the observance-day ceremony is unity. \\
The purpose of the invitation ceremony is purity. \\
The purpose of probation is the trial period. \\
The purpose of sending back to the beginning is restraint. \\
The purpose of the trial period is rehabilitation. \\
The purpose of rehabilitation is purity. 

If,\marginnote{10.1} because of desire, ill will, fear, or confusion, \\
One reviles senior monks, \\
Then, when the body breaks up, that foolish person, \\
Damaged, with impaired faculties, \\
Being stupid, goes to hell, \\
Without respect for the training. 

But\marginnote{11.1} not depending on worldly gain, \\
Not depending on individuals,\footnote{Sp 5.364: \textit{Na ca \textsanskrit{nissāya} puggalanti “\textsanskrit{ayaṁ} me \textsanskrit{upajjhāyo} \textsanskrit{vā} \textsanskrit{ācariyo} \textsanskrit{vā}”\textsanskrit{tiādinā} nayena \textsanskrit{chandādīhi} gacchanto \textsanskrit{puggalaṁ} \textsanskrit{nissāya} karoti, \textsanskrit{evaṁ} na kareyya}, “Not depending on individuals: through the method that begins with thinking, ‘This is my preceptor or teacher,’ one does not create dependence on an individual through desire, etc.” } \\
Giving up both of these, \\
One should act according to the Teaching. 

%
\end{verse}

\section*{3. The accuser burning himself }

\begin{verse}%
“Angry\marginnote{12.1} and resentful, \\
And fierce, reviling, \\
He charges a non-offender with an offense—\\
Such an accuser burns himself. 

Whispering\marginnote{13.1} in the ear, looking for flaws, \\
Vindictive, following the wrong path,\footnote{SN-a 1.188: \textit{\textsanskrit{Vītiharatīti} katassa \textsanskrit{paṭikāraṁ} karoti}, “\textit{\textsanskrit{Vītiharati}}: he acts with a counter-action toward the one who has acted.” } \\
He charges a non-offender with an offense—\\
Such an accuser burns himself. 

Accusing\marginnote{14.1} at the wrong time, untruthfully, \\
Harshly, without benefit, \\
With a mind of ill will, not a mind of good will, \\
He charges a non-offender with an offense—\\
Such an accuser burns himself. 

Not\marginnote{15.1} knowing the Teaching and what is contrary to it, \\
Ignorant about both, \\
He charges a non-offender with an offense—\\
Such an accuser burns himself. 

Not\marginnote{16.1} knowing the Monastic Law and what is contrary to it, \\
Ignorant about both, \\
He charges a non-offender with an offense—\\
Such an accuser burns himself. 

Not\marginnote{17.1} knowing what was spoken and what was not, \\
Ignorant about both, \\
He charges a non-offender with an offense—\\
Such an accuser burns himself. 

Not\marginnote{18.1} knowing what was practiced and what was not, \\
Ignorant about both, \\
He charges a non-offender with an offense—\\
Such an accuser burns himself. 

Not\marginnote{19.1} knowing what was laid down and what was not, \\
Ignorant about both, \\
He charges a non-offender with an offense—\\
Such an accuser burns himself. 

Not\marginnote{20.1} knowing the offenses and the non-offenses, \\
Ignorant about both, \\
He charges a non-offender with an offense—\\
Such an accuser burns himself. 

Not\marginnote{21.1} knowing light and heavy offenses, \\
Ignorant about both, \\
He charges a non-offender with an offense—\\
Such an accuser burns himself. 

Not\marginnote{22.1} knowing curable and incurable offenses, \\
Ignorant about both, \\
He charges a non-offender with an offense—\\
Such an accuser burns himself. 

Not\marginnote{23.1} knowing grave and minor offenses, \\
Ignorant about both, \\
He charges a non-offender with an offense—\\
Such an accuser burns himself. 

Not\marginnote{24.1} knowing the right order, \\
Ignorant about it, \\
He charges a non-offender with an offense—\\
Such an accuser burns himself. 

Not\marginnote{25.1} knowing the sequence of statements, \\
Ignorant about it, \\
He charges a non-offender with an offense—\\
Such an accuser burns himself.” 

%
\end{verse}

\scendkanda{The process of investigation  is finished. }

\scuddanaintro{This is the summary: }

\begin{scuddana}%
“Accusing,\marginnote{28.1} and investigating, \\
Beginning, at the root, observance day, \\
Destiny—in the process of investigation, \\
They firmly establish Buddhism.” 

%
\end{scuddana}

%
\chapter*{{\suttatitleacronym Pvr 14}{\suttatitletranslation The procedure for an investigator }{\suttatitleroot Anuvijjakassapaṭipatti}}
\addcontentsline{toc}{chapter}{\tocacronym{Pvr 14} \toctranslation{The procedure for an investigator } \tocroot{Anuvijjakassapaṭipatti}}
\markboth{The procedure for an investigator }{Anuvijjakassapaṭipatti}
\extramarks{Pvr 14}{Pvr 14}

“When\marginnote{1.1} a monk who is involved in a conflict is about to approach the Sangha, he should: be humble; be intent on removing defilements;\footnote{Sp 5.365: \textit{\textsanskrit{Rajoharaṇasamenāti} \textsanskrit{pādapuñchanasamena}; \textsanskrit{yathā} \textsanskrit{rajoharaṇassa} \textsanskrit{saṁkiliṭṭhe} \textsanskrit{vā} \textsanskrit{asaṁkiliṭṭhe} \textsanskrit{vā} \textsanskrit{pāde} \textsanskrit{puñchiyamāne} neva \textsanskrit{rāgo} na doso}, “\textit{\textsanskrit{Rajoharaṇasamena}}: like a doormat. As when wiping dirty or clean feet with a floor cloth, there is neither desire nor ill will.” } be skilled in appropriate seating and where to sit down, taking a seat without encroaching on the senior monks and without blocking the junior monks; not ramble or engage in worldly talk, but speak according to the Teaching or invite others to speak or value noble silence.\footnote{Sp 5.365: \textit{\textsanskrit{Anānākathikenāti} \textsanskrit{nānāvidhaṁ} \textsanskrit{taṁ} \textsanskrit{taṁ} \textsanskrit{anatthakathaṁ} akathentena}, “\textit{\textsanskrit{Anānākathika}}: not speaking a variety of unbeneficial talk.” } 

An\marginnote{2.1} investigator—who has been approved by the Sangha and who wishes to investigate—should not ask about preceptor, teacher, student, pupil, co-student, co-pupil, caste, name, family, reciter tradition, home address, or nationality.\footnote{Sp 5.365: \textit{Na \textsanskrit{upajjhāyo} pucchitabboti “ko \textsanskrit{nāmo} \textsanskrit{tuyhaṁ} \textsanskrit{upajjhāyo}”ti na pucchitabbo}, “\textit{Na \textsanskrit{upajjhāyo} pucchitabbo}: he is not to be asked: ‘What is the name of your preceptor?’” } What is the reason for that? In these cases there is affection or ill will. When there is affection or ill will, one might be biased by favoritism, ill will, confusion, or fear. 

An\marginnote{3.1} investigator—who has been approved by the Sangha and who wishes to investigate—should respect the Sangha, not individuals; should value the true Dhamma, not worldly things; should value the goal, not conforming to the gathering; should investigate at an appropriate time, not at an inappropriate one; should investigate truthfully, not falsely; should investigate gently, not harshly; should investigate beneficially, not unbeneficially; should investigate with a mind of good will, not with ill will; should not whisper in the ear; should not look for flaws; should not wink; should not raise an eyebrow; should not raise the head; should not signal with the hand; should not gesture with the hand. 

He\marginnote{4.1} should be skilled in appropriate seating and where to sit down. Looking a plow’s length ahead, acting in line with his aim, he should sit down on his own seat. He should not get up from his seat, bungle the investigation, take a wrong path, or gesticulate. He should proceed without haste or force, not be fierce, and be patient with others’ speech. He should have mind of loving kindness with compassion and empathy, and strive for what is beneficial. He should not speak idly but to the point, without being angry or argumentative. He should assess himself, the others, the accuser, the accused, one who accuses illegitimately, one who is accused illegitimately, one who accuses legitimately, and one who is accused legitimately. Not omitting what has been said, nor announcing what has not been said, he should carefully scrutinize the sentences and words under consideration, question the others, and deal with them according to what they have admitted. He should gladden those who are confused, comfort those who are frightened, restrain those who are fierce, and expose those who are impure. Being upright and gentle, he should not be biased by favoritism, anger, confusion, or fear. He should be impartial in regard to the Teaching and the people involved.\footnote{Sp 5.365: \textit{Na \textsanskrit{vītihātabbanti} na vinicchayo \textsanskrit{hāpetabbo}}, “\textit{Na \textsanskrit{vītihātabba}}: the investigation should not be bungled.” | Sp 5.365: \textit{\textsanskrit{Anasuruttenāti} na asuruttena. \textsanskrit{Asuruttaṁ} vuccati \textsanskrit{viggāhikakathāsaṅkhātaṁ} \textsanskrit{asundaravacanaṁ}; \textsanskrit{taṁ} na kathetabbanti attho}, “\textit{Anasuruttena}: not \textit{asuruttena}. What is considered argumentative speech, speech that is displeasing, is called \textit{asurutta}. The meaning is that that should not be spoken.” | Sp 5.365: \textit{\textsanskrit{Attā} pariggahetabboti “\textsanskrit{vinicchinituṁ} \textsanskrit{vūpasametuṁ} \textsanskrit{sakkhissāmi} nu kho no”ti \textsanskrit{evaṁ} \textsanskrit{attā} pariggahetabbo; attano \textsanskrit{pamāṇaṁ} \textsanskrit{jānitabbanti} attho}, “\textit{\textsanskrit{Attā} pariggahetabbo}: ‘Am I able to decide and resolve this matter?’ In this way, he should examine himself. The meaning is that he should assess himself.” | Sp 5.365: \textit{Asuci \textsanskrit{vibhāvetabboti} \textsanskrit{alajjiṁ} \textsanskrit{pakāsetvā} \textsanskrit{āpattiṁ} \textsanskrit{desāpetabbo}}, “\textit{Asuci \textsanskrit{vibhāvetabbo}}: revealing those who are shameless, he has them confess their offenses.” } In this way an investigator is acting in accordance with the instruction of the Teacher. And they are dear, agreeable, respected, and esteemed by their discerning fellow monastics. 

The\marginnote{5.1} Monastic Code is for the sake of concluding, a simile for the sake of illustration, the goal is to be made known, and questioning is for sake of establishing. Asking for permission is for the sake of accusing, accusing for reminding, reminding for directing, directing for obstructing, obstructing for investigating, investigating for weighing up, weighing up for deciding what is and is not the case, and deciding what is and is not the case is for the sake of restraining bad people and to support the good monks. The Sangha has the purpose of examining and accepting the decision. The Sangha should appoint people who are trustworthy to positions of authority.\footnote{Sp 5.366: \textit{\textsanskrit{Suttaṁ} \textsanskrit{saṁsandanatthāyātiādīsu} tena ca pana \textsanskrit{evaṁ} \textsanskrit{sabrahmacārīnaṁ} \textsanskrit{piyamanāpagarubhāvanīyena} anuvijjakena \textsanskrit{samudāhaṭesu} \textsanskrit{suttādīsu} \textsanskrit{suttaṁ} \textsanskrit{saṁsandanatthāya}; \textsanskrit{āpattānāpattīnaṁ} \textsanskrit{saṁsandanatthanti} \textsanskrit{veditabbaṁ}}, “\textit{\textsanskrit{Suttaṁ} \textsanskrit{saṁsandanatthāya}}: it is to be understood that because that investigator is thus dear, agreeable, respected, and esteemed by his fellow monastics, then, in regard to citing from the Monastic Code, etc., it is for the purpose of concluding about the Monastic Law. The purpose is to conclude about offenses and non-offenses.” Vmv 5.366: \textit{\textsanskrit{Saṁsandanatthanti} \textsanskrit{āpatti} \textsanskrit{vā} \textsanskrit{anāpatti} \textsanskrit{vāti} \textsanskrit{saṁsaye} \textsanskrit{jāte} \textsanskrit{saṁsanditvā} \textsanskrit{nicchayakaraṇatthaṁ} vuttanti \textsanskrit{adhippāyo}}, “The meaning is: when doubt has arisen about an offense or non-offense, then, having concluded, what is said has the purpose of creating certainty.” | Sp 5.366: \textit{Attho \textsanskrit{viññāpanatthāyāti} attho \textsanskrit{jānāpanatthāya}}, “\textit{Attho \textsanskrit{viññāpanatthāya}}: the goal is for the purpose of making known.” Vmv 5.366: \textit{Attho \textsanskrit{jānāpanatthāyāti} \textsanskrit{evaṁ} \textsanskrit{vibhāvito} attho \textsanskrit{codakacuditakasaṅghānaṁ} \textsanskrit{ñāpanatthāyanijjhāpanatthāya}, \textsanskrit{sampaṭicchāpanatthāyāti} attho}, “The goal is for the purpose of making known: for the purpose of making the accuser, the accused, and the Sangha know, understand, and accept that in this way the meaning is destroyed. This is the meaning.” | Sp 5.366: \textit{\textsanskrit{Paṭipucchā} \textsanskrit{ṭhapanatthāyāti} \textsanskrit{pucchā} puggalassa \textsanskrit{ṭhapanatthāya}}, “\textit{\textsanskrit{Paṭipucchā} \textsanskrit{ṭhapanatthāya}}: asking is for the sake of establishing a person.” Vmv 5.365: \textit{Puggalassa \textsanskrit{ṭhapanatthāyāti} codakacuditake attano \textsanskrit{paṭiññāya} eva \textsanskrit{āpattiyaṁ}, \textsanskrit{anāpattiyaṁ} \textsanskrit{vā} \textsanskrit{patiṭṭhāpanatthāya}}, “For the sake of establishing of a person: for the sake establishing the offenses or non-offenses through the own admission of the accuser and the accused.” | Sp 5.366: \textit{\textsanskrit{Savacanīyaṁ} \textsanskrit{palibodhatthāyāti} \textsanskrit{savacanīyaṁ} “\textsanskrit{imamhā} \textsanskrit{āvāsā} \textsanskrit{paraṁ} \textsanskrit{mā} \textsanskrit{pakkamī}”ti}, “\textit{\textsanskrit{Savacanīyaṁ} \textsanskrit{palibodhatthāya}}: one initiates a legal process, thinking, ‘The other should not leave this monastery.’” | Sp 5.366: \textit{Vinicchayo \textsanskrit{santīraṇatthāyāti} \textsanskrit{dosādosaṁ} \textsanskrit{santīraṇatthāya} \textsanskrit{tulanatthāya}}, “\textit{Vinicchayo \textsanskrit{santīraṇatthāya}}: for the purpose of \textit{\textsanskrit{santīraṇa}} of faults and non-faults, for the purpose of weighing them up.” | Sp 5.366: \textit{\textsanskrit{Saṅgho} \textsanskrit{sampariggahasampaṭicchanatthāyāti} \textsanskrit{vinicchayasampaṭiggahaṇatthāya} ca; \textsanskrit{suvinicchitadubbinicchitabhāvajānanatthāya} \textsanskrit{cāti}}, “\textit{\textsanskrit{Saṅgho} \textsanskrit{sampariggahasampaṭicchanatthāya}}: to examine the decision and to know whether it is well or badly decided.” | Sp 5.366: \textit{\textsanskrit{Paccekaṭṭhāyino} \textsanskrit{avisaṁvādakaṭṭhāyinoti} \textsanskrit{issariyādhipaccajeṭṭhakaṭṭhāne} ca \textsanskrit{avisaṁvādakaṭṭhāne} ca \textsanskrit{ṭhitā}; na te \textsanskrit{apasādetabbāti}}, “\textit{\textsanskrit{Paccekaṭṭhāyino} \textsanskrit{avisaṁvādakaṭṭhāyino}}: they are in a position of authority, power, and seniority, and also established in truthfulness. They cannot be dismissed.” } 

The\marginnote{6.1} Monastic Law is for the sake of restraint, restraint for non-regret, non-regret for joy, joy for rapture, rapture for tranquility, tranquility for bliss, bliss for stillness, stillness for seeing things according to reality, seeing things according to reality for repulsion, repulsion for dispassion, dispassion for liberation, liberation for knowledge and vision of liberation, and knowledge and vision of liberation is for the sake of extinguishment without grasping. This is the purpose of discussion, this is the purpose of consultation, this is the purpose of vital conditions, this is the purpose of listening,\footnote{Sp 5.366: \textit{\textsanskrit{Upanisāti} \textsanskrit{ayaṁ} “vinayo \textsanskrit{saṁvaratthāyā}”\textsanskrit{tiādikā} \textsanskrit{paramparapaccayatāpi} \textsanskrit{etadatthāya}}, “\textit{\textsanskrit{Upanisā}}: here, the conditionality of the sequence: ‘The Monastic Law is for the sake of restraint,’ etc. For this purpose.” } that is, the release of mind without grasping.” 

\begin{verse}%
“Consider\marginnote{7.1} carefully the proper procedure of examination, \\
As formulated by the skilled Awakened One, \\
Well-spoken in line with the training rules—\\
Do not ruin your future rebirth. 

Ignorant\marginnote{8.1} about basis, failure, and offense, \\
As well as source and ways; \\
Not knowing the right order, \\
Nor what has and has not been done. 

Ignorant\marginnote{9.1} about legal procedures and legal issues, \\
As well as their settling. \\
Greedy, angry, and confused, \\
Biased by fear and confusion, 

Not\marginnote{10.1} skilled in persuasion, \\
Nor in making others understand; \\
A shameless one who has obtained supporters, \\
Disrespectful and doing dark deeds—\\
A monk such as this \\
Is called ‘not worthy of attention’.\footnote{Sp 5.367: \textit{Sa ve \textsanskrit{tādisako} bhikkhu \textsanskrit{apaṭikkhoti} vuccati, na \textsanskrit{paṭikkhitabbo} na oloketabbo, na \textsanskrit{sammannitvā} \textsanskrit{issariyādhipaccajeṭṭhakaṭṭhāne} \textsanskrit{ṭhapetabboti} attho}, “\textit{Sa ve \textsanskrit{tādisako} bhikkhu \textsanskrit{apaṭikkhoti} vuccati}: the meaning is he should not be looked to or relied on, and should not be approved to a position of authority, power, and seniority.” } 

Understanding\marginnote{11.1} basis, failure, and offense, \\
As well as source and ways; \\
Knowing the right order, \\
And also what has and has not been done. 

Understanding\marginnote{12.1} legal procedures and legal issues, \\
As well as their settling. \\
Not greedy, angry, or confused, \\
Biased neither by fear nor confusion, 

Skilled\marginnote{13.1} in persuasion, \\
And in making others understand; \\
A conscientious one who has obtained supporters, \\
Respectful and doing bright deeds—\\
A monk such as this \\
Is called ‘worthy of attention’.” 

%
\end{verse}

\scend{The short section on conflict is finished }

\scuddanaintro{This is the summary: }

\begin{scuddana}%
“Humble,\marginnote{16.1} one may ask, \\
Respect for the Sangha, not individuals; \\
The Monastic Code is for the sake of concluding, \\
And through supporting the training—\\
The summary of the short section on conflict, \\
Made into one recitation.” 

%
\end{scuddana}

%
\chapter*{{\suttatitleacronym Pvr 15}{\suttatitletranslation The great section on conflict }{\suttatitleroot Mahāsaṅgāma}}
\addcontentsline{toc}{chapter}{\tocacronym{Pvr 15} \toctranslation{The great section on conflict } \tocroot{Mahāsaṅgāma}}
\markboth{The great section on conflict }{Mahāsaṅgāma}
\extramarks{Pvr 15}{Pvr 15}

\section*{1. What is to be known by one who is speaking, etc. }

When\marginnote{1.1} a monk who is involved in a conflict is speaking in the Sangha, he should know the actions that are the bases for offenses, as well as the failures, the offenses, the origin stories, the attributes, the right order, what has and has not been done, the legal procedures, the legal issues, and their settling. He should not be biased by favoritism, ill will, confusion, or fear. He should persuade when persuasion is appropriate, should make others understand when making understand is appropriate, should look on when looking on is appropriate, and should inspire when inspiration is appropriate. Thinking, “I’ve obtained supporters,” he should not despise the supporters of others. Thinking, “I’m learned,” he should not despise those who are ignorant. Thinking, “I’m more senior,” he should not despise those who are junior. He should not speak about what has not been reached, and he should not use the Teaching or the Monastic Law to neglect what has been reached. He should resolve that legal issue in accordance with the Teaching, the Monastic Law, and the Teacher’s instruction. 

\begin{description}%
\item[He should know the actions that are the bases for offenses: ] he should know the actions that are the bases for the eight offenses entailing expulsion, for the twenty-three offenses entailing suspension, for the two undetermined offenses, for the forty-two offenses entailing relinquishment, for the one hundred and eighty-eight offenses entailing confession, for the twelve offenses entailing acknowledgment, for the offenses of wrong conduct, and for the offenses of wrong speech. %
\item[He should know the failures: ] he should know failure in morality, failure in conduct, failure in view, and failure in livelihood. %
\item[He should know the offenses: ] he should know the offenses entailing expulsion, the offenses entailing suspension, the serious offenses, the offenses entailing confession, the offenses entailing acknowledgement, the offenses of wrong conduct, and the offenses of wrong speech. %
\item[He should know the origin stories: ] he should know the origin stories to the eight offenses entailing expulsion, to the twenty-three offenses entailing suspension, to the two undetermined offenses, to the forty-two offenses entailing relinquishment, to the one hundred and eighty-eight offenses entailing confession, to the twelve offenses entailing acknowledgment, to the offenses of wrong conduct, and to the offenses of wrong speech. %
\item[He should know the attributes: ] he should know the attributes of a sangha, of a group, of an individual, of an accuser, of an accused. %
\item[He should know the attributes of a sangha: ] is this sangha capable of resolving this legal issue according to the Teaching, the Monastic Law, and the Teacher’s instruction, or is it not? %
\item[He should know the attributes of a group: ] is this group capable of resolving this legal issue according to the Teaching, the Monastic Law, and the Teacher’s instruction, or is it not? %
\item[He should know the attributes of an individual: ] is this individual capable of resolving this legal issue according to the Teaching, the Monastic Law, and the Teacher’s instruction, or is he not? %
\item[He should know the attributes of an accuser: ] is this venerable established in the five qualities before accusing another, or is he not? %
\item[He should know the attributes of an accused: ] is this venerable established in the two qualities of truth and composure, or is he not? %
\item[He should know the right order: ] does this venerable go from one action that is the basis for an offense to another action, from one failure to another, from one offense to another? Does he assert things after denying them, deny things after asserting them, or evade the issue? Or does he not? %
\item[He should know what has and has not been done: ] he should know sexual intercourse, he should know what amounts to sexual intercourse, he should know the preliminaries of sexual intercourse. %
\item[He should know sexual intercourse: ] he should know that which is done wherever there are couples. %
\item[He should know what amounts to sexual intercourse: ] a monk taking the genitals of another in his mouth. %
\item[He should know the preliminaries to sexual intercourse: ] various colors, physical contact, indecent speech, satisfying one’s own desires, encouraging through speech.\footnote{Sp 5.375: \textit{\textsanskrit{Vaṇṇāvaṇṇoti} \textsanskrit{nīlādivaṇṇāvaṇṇavasena} \textsanskrit{sukkavissaṭṭhisikkhāpadaṁ} \textsanskrit{vuttaṁ}. \textsanskrit{Vacanamanuppadānanti} \textsanskrit{sañcarittaṁ} \textsanskrit{vuttaṁ}}, “‘Various colors’ is said on account of the various colors of blue, etc., in the training rule on emission of semen. ‘Encouraging through speech’: matchmaking is spoken of.” In other words, this refers to the first and fifth offenses entailing suspension for monks. } %
\item[He should know the legal procedures: ] he should know the sixteen legal procedures: he should know the four kinds of legal procedures consisting of getting permission, the four kinds of legal procedures consisting of one motion, the four kinds of legal procedures consisting one motion and one announcement, and the four kinds of legal procedures consisting of one motion and three announcements.\footnote{Sp 5.376: \textit{\textsanskrit{Cattāri} \textsanskrit{apalokanakammānīti} \textsanskrit{adhammenavaggādīni}}, “‘The four kinds of legal procedures consisting of getting permission’: illegitimate, an incomplete assembly, etc.” } %
\item[He should know the legal issues: ] he should know the four kinds of legal issues: he should know the legal issues arising from disputes, the legal issues arising from accusations, the legal issues arising from offenses, and the legal issues arising from business. %
\item[He should know settling: ] he should know the seven principles for settling legal issues: he should know resolution face-to-face, resolution through recollection, resolution because of past insanity, acting according to what has been admitted, majority decision, further penalty, and covering over as if with grass. %
\end{description}

\section*{2. Not to be biased }

\begin{description}%
\item[He should not be biased by favoritism: ] How\marginnote{12.2} is one biased by favoritism? 

It\marginnote{12.3} may be that someone thinks, “This is my preceptor, teacher, student, pupil, co-student, co-pupil, friend, companion, or relative.” To be compassionate toward and protect this person, he proclaims what is contrary to the Teaching as being in accordance with it and what is in accordance with the Teaching as contrary to it. He proclaims what is contrary to the Monastic Law as being in accordance with it, and what is in accordance with the Monastic Law as contrary to it. He proclaims what hasn’t been spoken by the Buddha as spoken by him, and what has been spoken by the Buddha as not spoken by him. He proclaims what was not practiced by the Buddha as practiced by him, and what was practiced by the Buddha as not practiced by him. He proclaims what was not laid down by the Buddha as laid down by him, and what was laid down by the Buddha as not laid down by him. He proclaims a non-offense as an offense, and an offense as a non-offense. He proclaims a light offense as heavy, and a heavy offense as light. He proclaims a curable offense as incurable, and an incurable offense as curable. He proclaims a grave offense as minor, and a minor offense as grave. 

If\marginnote{12.4} he is biased by favoritism by way of these eighteen grounds, then his behavior is unbeneficial and a cause of unhappiness for humanity; it is harmful, detrimental, and a cause of suffering for gods and humans. If he is biased by favoritism by way of these eighteen grounds, then he is damaged and impaired, blamed and criticized by sensible people, and makes much demerit. 

%
\item[He should not be biased by ill will: ] How\marginnote{13.2} is one biased by ill will? 

It\marginnote{13.3} may be that someone thinks, “They’ve harmed me,” and he feels resentful. Or he thinks, “They’re harming me,” and he feels resentful. Or he thinks, “They’ll harm me,” and he feels resentful. Or he thinks, “They’ve harmed someone who’s dear to me” … “They’re harming someone who’s dear to me” … “They’ll harm someone who’s dear to me” … “They’ve benefited someone I dislike” … “They’re benefiting someone I dislike” … “They’ll benefit someone I dislike,” and he feels resentful. Because of these nine grounds for resentment, he is resentful, hostile, angry, and overcome by anger, and then proclaims what is contrary to the Teaching as being in accordance with it and what is in accordance with the Teaching as contrary to it … He proclaims a grave offense as minor, and a minor offense as grave. 

If\marginnote{13.6} he is biased by ill will by way of these eighteen grounds, then his behavior is unbeneficial and a cause of unhappiness for humanity; it is harmful, detrimental, and a cause of suffering for gods and humans. If he is biased by ill will by way of these eighteen grounds, then he is damaged and impaired, blamed and criticized by sensible people, and makes much demerit. 

%
\item[He should not be biased by confusion: ] How\marginnote{14.2} is one biased by confusion? 

Biased\marginnote{14.3} by favoritism, ill will, or confusion, or by a grasped view, he is confused, deluded, and overcome by confusion, and then proclaims what is contrary to the Teaching as being in accordance with it and what is in accordance with the Teaching as contrary to it … He proclaims a grave offense as minor, and a minor offense as grave. 

If\marginnote{14.5} he is biased by confusion by way of these eighteen grounds, then his behavior is unbeneficial and a cause of unhappiness for humanity; it is harmful, detrimental, and a cause of suffering for gods and humans. If he is biased by confusion by way of these eighteen grounds, then he is damaged and impaired, blamed and criticized by sensible people, and makes much demerit. 

%
\item[He should not be biased by fear: ] How\marginnote{15.2} is one biased by fear? 

It\marginnote{15.3} may be that someone thinks, “This one relies on the uneven, on thick covers, and on powerful individuals; he’s cruel and harsh, and might be a threat to life or the monastic life.” Fearful or frightened of him, he proclaims what is contrary to the Teaching as being in accordance with it and what is in accordance with the Teaching as contrary to it. He proclaims what is contrary to the Monastic Law as being in accordance with it, and what is in accordance with the Monastic Law as contrary to it. He proclaims what hasn’t been spoken by the Buddha as spoken by him, and what has been spoken by the Buddha as not spoken by him. He proclaims what was not practiced by the Buddha as practiced by him, and what was practiced by the Buddha as not practiced by him. He proclaims what was not laid down by the Buddha as laid down by him, and what was laid down by the Buddha as not laid down by him. He proclaims a non-offense as an offense, and an offense as a non-offense. He proclaims a light offense as heavy, and a heavy offense as light. He proclaims a curable offense as incurable, and an incurable offense as curable. He proclaims a grave offense as minor, and a minor offense as grave.\footnote{Sp 5.382: \textit{Visamanissitoti \textsanskrit{visamāni} \textsanskrit{kāyakammādīni} nissito. Gahananissitoti \textsanskrit{micchādiṭṭhiantaggāhikadiṭṭhisaṅkhātaṁ} \textsanskrit{gahanaṁ} nissito. Balavanissitoti balavante \textsanskrit{abhiññāte} \textsanskrit{bhikkhū} nissito}, “‘Relies on the uneven’: relies on uneven bodily conduct, etc. ‘Relies on thick covers’: relies on thick covers known as wrong views and extreme views. ‘Relies on the powerful’: relies on powerful and well-known monks.” } 

If\marginnote{15.4} he is biased by fear by way of these eighteen grounds, then his behavior is unbeneficial and a cause of unhappiness for humanity; it is harmful, detrimental, and a cause of suffering for gods and humans. If he is biased by fear by way of these eighteen grounds, then he is damaged and impaired, blamed and criticized by sensible people, and makes much demerit. 

%
\end{description}

\begin{verse}%
“If,\marginnote{16.1} because of favoritism, ill will, fear, or confusion, \\
He goes beyond the Teaching, \\
Then his reputation is harmed, \\
Like the moon during the waning half-month.” 

%
\end{verse}

\section*{3. Not being biased }

\begin{description}%
\item[How is one not biased by favoritism? ] One\marginnote{17.2} is not biased by favoritism if one proclaims what is contrary to the Teaching as such, and what is in accordance with the Teaching as such; if one proclaims what is contrary to the Monastic Law as such, and what is in accordance with the Monastic Law as such; if one proclaims what hasn’t been spoken by the Buddha as such, and what has been spoken by the Buddha as such; if one proclaims what was not practiced by the Buddha as such, and what was practiced by the Buddha as such; if one proclaims what was not laid down by the Buddha as such, and what was laid down by the Buddha as such; if one proclaims a non-offense as such, and an offense as such; if one proclaims a light offense as light, and a heavy offense as heavy; if one proclaims a curable offense as curable, and an incurable offense as incurable; if one proclaims a grave offense as grave, and a minor offense as minor. 

%
\item[How is one not biased by ill will? ] One\marginnote{18.2} is not biased by ill will if one proclaims what is contrary to the Teaching as such, and what is in accordance with the Teaching as such … if one proclaims a grave offense as grave, and a minor offense as minor. 

%
\item[How is one not biased by confusion? ] One\marginnote{19.2} is not biased by confusion if one proclaims what is contrary to the Teaching as such, and what is in accordance with the Teaching as such … if one proclaims a grave offense as grave, and a minor offense as minor. 

%
\item[How is one not biased by fear? ] One\marginnote{20.2} is not biased by fear if one proclaims what is contrary to the Teaching as such, and what is in accordance with the Teaching as such; if one proclaims what is contrary to the Monastic Law as such, and what is in accordance with the Monastic Law as such; if one proclaims what hasn’t been spoken by the Buddha as such, and what has been spoken by the Buddha as such; if one proclaims what was not practiced by the Buddha as such, and what was practiced by the Buddha as such; if one proclaims what was not laid down by the Buddha as such, and what was laid down by the Buddha as such; if one proclaims a non-offense as such, and an offense as such; if one proclaims a light offense as light, and a heavy offense as heavy; if one proclaims a curable offense as curable, and an incurable offense as incurable; if one proclaims a grave offense as grave, and a minor offense as minor. 

%
\end{description}

\begin{verse}%
“If,\marginnote{21.1} because of favoritism, ill will, fear, or confusion, \\
He does not go beyond the Teaching, \\
Then his reputation grows, \\
Like the moon during the waxing fortnight.” 

%
\end{verse}

\section*{4. To be persuaded, etc. }

\begin{description}%
\item[How does one persuade when persuasion is appropriate? ] Proclaiming what is contrary to the Teaching as such, and what is in accordance with the Teaching as such, he persuades others when persuasion is appropriate. … Proclaiming a grave offense as grave and a minor offense as minor, he persuades others when persuasion is appropriate. %
\item[How does one make others understand when making understand is appropriate? ] Proclaiming what is contrary to the Teaching as such, and what is in accordance with the Teaching as such, he makes others understand when making understand is appropriate. … Proclaiming a grave offense as grave and a minor offense as minor, he makes others understand when making understand is appropriate. %
\item[How does one look on when looking on is appropriate? ] Proclaiming what is contrary to the Teaching as such, and what is in accordance with the Teaching as such, he looks on when looking on is appropriate. … Proclaiming a grave offense as grave and a minor offense as minor, he looks on when looking on is appropriate. %
\item[How does one inspire when inspiration is appropriate? ] Proclaiming what is contrary to the Teaching as such, and what is in accordance with the Teaching as such, he inspires when inspiration is appropriate. … Proclaiming a grave offense as grave and a minor offense as minor, he inspires when inspiration is appropriate. %
\end{description}

\section*{5. Despising the supporters of others, etc. }

\begin{description}%
\item[Thinking, “I’ve obtained supporters,” how does one despise the supporters of others? ] It may be that someone has obtained supporters and a group of followers, and has relatives. He thinks, “This one doesn’t have supporters or a group of followers, and doesn’t have relatives,” and despising him, he proclaims what is contrary to the Teaching as being in accordance with it and what is in accordance with the Teaching as contrary to it. … He proclaims a grave offense as minor, and a minor offense as grave. %
\item[Thinking, “I’m learned,” how does one despise those who are ignorant? ] It may be that someone is learned, one who has retained and accumulated what he has learned. He thinks, “This one is ignorant; he has learned little and remembers little,” and despising him, he proclaims what is contrary to the Teaching as being in accordance with it and what is in accordance with the Teaching as contrary to it. … He proclaims a grave offense as minor, and a minor offense as grave. %
\item[Thinking, “I’m more senior,” how does one despise those who are more junior? ] It may be that someone is a senior monk of long standing. He thinks, “This is an unknown and ignorant junior monk; one shouldn’t do as he asks,” and despising him, he proclaims what is contrary to the Teaching as being in accordance with it and what is in accordance with the Teaching as contrary to it. … He proclaims a grave offense as minor, and a minor offense as grave. %
\item[He should not speak about what has not been reached: ] he should not bring up an issue not under consideration. %
\item[He should not use the Teaching or the Monastic Law to neglect what has been reached: ] the purpose for which the Sangha has gathered should not be neglected using the Teaching or the Monastic Law. %
\item[In accordance with the Teaching: ] in according with truth, in accordance with the action that was the basis for the offense. %
\item[In accordance with the Monastic Law: ] having accused and having reminded. %
\item[In accordance with the Teacher’s instruction: ] he resolves that legal issue, complete in motion and complete in announcement, in accordance with the Teaching, the Monastic Law, and the Teacher’s instruction. %
\end{description}

\section*{6. Questioning by the investigator }

The\marginnote{31.1} investigator should ask the accuser, “Are you canceling this monk’s invitation because he has failed in morality, in conduct, or in view?” 

If\marginnote{31.3} he says, “I’m canceling it because he has failed in morality,” “I’m canceling it because he has failed in conduct,” or “I’m canceling it because he has failed in view,” he should be asked, “Do you know what failure in morality is?” “Do you know what failure in conduct is?” or “Do you know what failure in view is?” 

If\marginnote{31.6} he says, “I do,” he should be asked what they are. 

If\marginnote{31.8} he says, “The four offenses entailing expulsion and the thirteen entailing suspension are failure in morality,” “The serious offenses, the offenses entailing confession, the offenses entailing acknowledgment, the offenses of wrong conduct, and the offenses of wrong speech are failure in conduct,” “Wrong views and extreme views are failure in view,” he should be asked, “Are you canceling this monk’s invitation because of what you’ve seen, what you’ve heard, or what you suspect?” 

If\marginnote{31.14} he says, “I’m canceling it because of what I’ve seen,” “I’m canceling it because of what I’ve heard,” or “I’m canceling it because of what I suspect,” he should be asked, “Since you’re canceling this monk’s invitation because of what you’ve seen, what have you seen? How did you see it? When did you see it? Where did you see it? Did you see him commit an offense entailing expulsion? Did you see him commit an offense entailing suspension? Did you see him commit a serious offense, an offense entailing confession, an offense entailing acknowledgment, an offense of wrong conduct, or an offense of wrong speech? Where were you? Where was this monk? What were you doing? What was this monk doing?” 

If\marginnote{31.21} he says, “I didn’t cancel this monk’s invitation because of what I’ve seen, but because of what I’ve heard,” he should be asked, “Since you’re canceling this monk’s invitation because of what you’ve heard, what have you heard? How did you hear it? When did you hear it? Where did you hear it? Did you hear that he has committed an offense entailing expulsion? Did you hear that he has committed an offense entailing suspension? Did you hear that he has committed a serious offense, an offense entailing confession, an offense entailing acknowledgment, an offense of wrong conduct, or an offense of wrong speech? Did you hear it from a monk, a nun, a trainee nun, a novice monk, a novice nun, a male lay follower, or a female lay follower? Or did you hear it from kings, a king’s officials, the monastics of another religion, or the lay followers of another religion?” 

If\marginnote{31.28} he says, “I didn’t cancel this monk’s invitation because of what I’ve heard, but because of what I suspect,” he should be asked, “Since you’re canceling this monk’s invitation because of suspicion, what do you suspect? How do you suspect it? When did you suspect it? Where did you suspect it? Do you suspect that he has committed an offense entailing expulsion? Do you suspect that he has committed an offense entailing suspension? Do you suspect that he has committed a serious offense, an offense entailing confession, an offense entailing acknowledgment, an offense of wrong conduct, or an offense of wrong speech? Do you suspect it after hearing about it from a monk, a nun, a trainee nun, a novice monk, a novice nun, a male lay follower, or a female lay follower? Or do you suspect it after hearing about it from kings, a king’s officials, the monastics of another religion, or the lay followers of another religion?” 

\begin{verse}%
“If\marginnote{32.1} what he saw agrees with what he says he saw, \\
If they correspond with each other, \\
But what was seen isn’t adequate to prove the accusation, \\
Then the one suspecting impurity \\
Should admit it, \\
And they should then do the invitation ceremony with him. 

If\marginnote{33.1} what he heard agrees with what he says he heard, \\
If they correspond with each other, \\
But what was heard isn’t adequate to prove the accusation, \\
Then the one suspecting impurity \\
Should admit it, \\
And they should then do the invitation ceremony with him. 

If\marginnote{34.1} what he sensed agrees with what he says he sensed, \\
If they correspond with each other, \\
But what was sensed isn’t adequate to prove the accusation, \\
Then the one suspecting impurity \\
Should admit it, \\
And they should then do the invitation ceremony with him.” 

%
\end{verse}

\section*{7. The details of asking }

“What\marginnote{35.1} are the questions in regard to ‘What have you seen?’ What are the questions in regard to ‘How did you see it?’ What are the questions in regard to ‘When did you see it?’ What are the questions in regard to ‘Where did you see it?’ 

\begin{description}%
\item[In regard to ‘What have you seen?’ ] there are questions on the actions that are the bases for offenses, there are questions on failure, there are questions on offenses, and there are questions on misconduct. ‘Questions on the actions that are the bases for offenses’: there are questions on the actions that are the bases for the eight offenses entailing expulsion, on the actions that are the bases for the twenty-three offenses entailing suspension, for the two undetermined offenses, for the forty-two offenses entailing relinquishment, for the one hundred and eighty-eight offenses entailing confession, for the twelve offenses entailing acknowledgment, for the offenses of wrong conduct, and for the offenses of wrong speech. ‘Questions on failure’: there are questions on failure in morality, on failure in conduct, on failure in view, and on failure in livelihood. ‘Questions on the offenses’: there are questions on the offenses entailing expulsion, on the offenses entailing suspension, on the serious offenses, on the offenses entailing confession, on the offenses entailing acknowledgement, on the offenses of wrong conduct, and on the offenses of wrong speech. ‘Questions on misconduct’: there are questions on that which is done wherever there are couples. %
\item[In regard to ‘How did you see it?’ ] there are questions on characteristics, postures, attributes, and modes. ‘Questions on characteristics’: tall, short, dark-skinned, or light-skinned. ‘Questions on postures’: walking, standing, sitting, or lying down. ‘Questions on attributes’: the characteristics of a householder, of a monastic of another religion, or of one gone forth. ‘Questions on modes’: walking, standing, sitting, or lying down. %
\item[In regard to ‘When did you see it?’ ] there are questions on time, on occasion, on day, and on season. ‘Questions on time’: in the morning, at midday, or in the evening. ‘Questions on occasion’: in the morning, at midday, or in the evening. ‘Questions on day’: before the meal, after the meal, at night, by day, during the waning moon, or during the waxing moon. ‘Questions on season’: in winter, in summer, or during the rainy season. %
\item[In regard to ‘Where did you see it?’ ] there are questions about place, about elevation, about location, and about region. ‘Questions about place’: in the ground, on the earth, on the planet, or in the world.\footnote{I follow the usage at \href{https://suttacentral.net/pli-tv-bu-vb-pj2/en/brahmali\#4.2.2}{Bu Pj 2:4.2.2} where \textit{\textsanskrit{bhūmiyā}} means “in the ground”. } ‘Questions about elevation’: in the ground, on the the earth, on a mountain, on a rock, or in a stilt house.\footnote{For an explanation of the rendering “stilt house” for \textit{\textsanskrit{pāsāda}}, see Appendix of Technical Terms. } ‘Questions about location’: to the east, to the west, to the north, or to the south. ‘Questions about region’: to the east, to the west, to the north, or to the south.” %
\end{description}

\scend{The great section on conflict is finished. }

\scuddanaintro{This is the summary: }

\begin{scuddana}%
“Basis\marginnote{42.1} for an offense, origin story, attribute, \\
The right order, what has and has not been done; \\
Legal procedure and legal issue, \\
Settling, and biased by favoritism. 

By\marginnote{43.1} ill will, confusion, and fear, \\
Persuasion, and by making understand; \\
Looking on, inspiration, I have supporters, \\
Learned, and with more senior. 

And\marginnote{44.1} not reached, reached, \\
According to the Teaching, and the Monastic Law; \\
Also according to the Teacher’s instruction—\\
The explanation in the great section on conflict.” 

%
\end{scuddana}

%
\chapter*{{\suttatitleacronym Pvr 16}{\suttatitletranslation The robe-making ceremony }{\suttatitleroot Kathinabheda}}
\addcontentsline{toc}{chapter}{\tocacronym{Pvr 16} \toctranslation{The robe-making ceremony } \tocroot{Kathinabheda}}
\markboth{The robe-making ceremony }{Kathinabheda}
\extramarks{Pvr 16}{Pvr 16}

\section*{1. Participated in the robe-making ceremony, etc.\footnote{For a discussion of the word \textit{kathina}, see Appendix of Technical Terms. } }

Who\marginnote{1.1} has not participated in the robe-making ceremony? Who has participated in the robe-making ceremony? How has the robe-making ceremony not been performed? How has the robe-making ceremony been performed? 

\begin{description}%
\item[Who has not participated in the robe-making ceremony? ] Two kinds of people: those who have not performed the robe-making ceremony and those who have not expressed their appreciation.\footnote{The performer of the ceremony is the person who receives the specially made \textit{kathina} robe. The other participants are those who take part in the legal procedure and then express their appreciation. } %
\item[Who has participated the robe-making ceremony? ] Two kinds of people: those who have performed the robe-making ceremony and those who have expressed their appreciation. %
\item[How has the robe-making ceremony not been performed? ] There are twenty-four ways in which the robe-making ceremony has not been performed:\footnote{For a further explanation of these twenty-four, see \href{https://suttacentral.net/pli-tv-kd7/en/brahmali\#1.5.3}{Kd 7:1.5.3}–1.5.26. } The\marginnote{4.2} robe-making ceremony has not been performed merely by marking the cloth, merely by washing the cloth, merely by planning the robe, merely by cutting the cloth, merely by tacking the cloth, merely by sewing a hem, merely by marking with a strip of cloth, merely by strengthening, merely by adding a border lengthwise, merely by adding a border crosswise, merely by patching, merely by partial dyeing; nor has it been performed if a monk has made an indication, if a monk has given a hint, if the robe-cloth has been borrowed,\footnote{For an explanation of the rendering “robe-cloth” for \textit{\textsanskrit{cīvara}}, see Appendix of Technical Terms. } if it has been stored, if it is to be relinquished, if it has not been marked, if it is not an outer robe or an upper robe or a sarong; nor has it been performed if the robe has not been made on that very day with five or more cut sections with panels, if the robe-making ceremony was not performed by an individual, or if the robe-making ceremony has been performed correctly but the appreciation for the ceremony was expressed outside the monastery zone.\footnote{For an explanation of the rendering “monastery zone” for \textit{\textsanskrit{sīmā}}, see Appendix of Technical Terms. } 

Making\marginnote{5.1} an indication: one makes an indication, thinking, “I’ll perform the robe-making ceremony with this cloth.” Hinting: one gives a hint, thinking, “With this hint, I’ll make a cloth for the robe-making ceremony appear.” Borrowed: a gift not to be taken as one’s own is so called. Stored: there are two kinds of storing: storing for the purpose of making and storing for the purpose of accumulation. To be relinquished: dawn arrives while it is being made. 

%
\item[How has the robe-making ceremony been performed? ] The robe-making ceremony has been performed through seventeen aspects: The robe-making ceremony has been performed if the cloth is brand new, if it is nearly new, if it is old, if it is a rag, if it is from a shop; it has been performed if a monk has not made an indication, if a monk has not given a hint, if the robe-cloth has not been borrowed, if it has not been stored, if it is not to be relinquished, if it has been marked, if it is an outer robe or an upper robe or a sarong; it has been performed if the robe has been made on that very day with five or more cut sections with panels, if the robe-making ceremony was performed by an individual, if the robe-making ceremony has been performed correctly and if the appreciation for the ceremony was expressed inside the monastery zone. %
\end{description}

How\marginnote{7.1} many things are produced together with the participation in the robe-making ceremony? Fifteen things: eight key phrases, two obstacles, and five benefits. 

\section*{2. The proximity condition for the robe-making ceremony, etc. }

Which\marginnote{8.1} things are a condition for effort by being a proximity condition, an immediacy condition, a support condition, a decisive support condition, a pre-arising condition, a post-arising condition, and a co-arising condition?\footnote{Sp 5.404: \textit{\textsanskrit{Payogassāti} \textsanskrit{cīvaradhovanādino} sattavidhassa \textsanskrit{pubbakaraṇassatthāya} yo \textsanskrit{udakāharaṇādiko} payogo kayirati, tassa payogassa}, “For effort: the effort of fetching water, etc., is done for the sake of the seven-fold prior action, beginning with washing the robe—for that effort.” Here we see the introduction of the Abhidhamma terminology of conditionality. } 

Which\marginnote{8.2} things are a condition for the prior duties by being a proximity condition … Which things are a condition for relinquishment …\footnote{Sp 5.404: \textit{\textsanskrit{Paccuddhārassāti} \textsanskrit{purāṇasaṅghāṭiādīnaṁ} \textsanskrit{paccuddharaṇassa}}, “Relinquishment: relinquishment of the old outer robe, etc.” } Which things are a condition for determining …\footnote{Sp 5.404: \textit{\textsanskrit{Adhiṭṭhānassāti} \textsanskrit{kathinacīvarādhiṭṭhānassa}}, “Determining: determining of the robe of the robe-making ceremony.” } Which things are a condition for participating in the robe-making ceremony … Which things are a condition for the key phrases and the obstacles … Which things are a condition for the object by being a proximity condition, an immediacy condition, a support condition, a decisive support condition, a pre-arising condition, a post-arising condition, and a co-arising condition?\footnote{Sp 5.404: \textit{\textsanskrit{Vatthussāti} \textsanskrit{saṅghāṭiādino} kathinavatthussa}, “Object: the object of the robe-making ceremony, which is the outer robe, etc.” } 

The\marginnote{9.1} prior duties are a condition for effort by being a proximity condition, an immediacy condition, a support condition, and a decisive support condition.\footnote{Sp 5.404: \textit{\textsanskrit{Yasmā} tena payogena \textsanskrit{nipphādetabbassa} \textsanskrit{pubbakaraṇassatthāya} so payogo kayirati, \textsanskrit{tasmā} imehi \textsanskrit{catūhi} paccayehi paccayo hoti}, “Because that effort is done for the purpose of the preliminary actions, which are to be produced by that effort, therefore it is the condition by way of these four conditions.” It is curious that the prior duties are said to be conditions for effort when in fact the effort is done first and the outcome is the prior duties. The commentary explains this strange state of affairs as follows. Sp 5.404: \textit{Katame \textsanskrit{dhammā} anantarapaccayena paccayoti \textsanskrit{anāgatavasena} \textsanskrit{anantarā} \textsanskrit{hutvā} katame \textsanskrit{dhammā} \textsanskrit{paccayā} \textsanskrit{hontīti} attho}, “Which things are a condition as a proximity condition: which things are conditions of proximity by way of the future? This is the meaning.” In this way the ordinary conditioning relationship is inverted, that is, the condition is temporarily later than the outcome. We see the same pattern below. My sincere thanks go to Ven. Bhikkhu Bodhi for helping me untangle the ideas in this section. } Effort is a condition for the prior duties by being a pre-arising condition. The prior duties are a condition for effort by being a post-arising condition. The fifteen things are a condition by being a co-arising condition. 

Relinquishment\marginnote{9.5} is a condition for the prior duties by being a proximity condition, an immediacy condition, a support condition, and a decisive support condition. The prior duties are a condition for relinquishment by being a pre-arising condition. Relinquishment is a condition for the prior duties by being a post-arising condition. The fifteen things are a condition by being a co-arising condition. 

Determining\marginnote{9.9} is a condition for relinquishment by being a proximity condition, an immediacy condition, a support condition, and a decisive support condition. Relinquishment is a condition for determining by being a pre-arising condition. Determining is a condition for relinquishment by being a post-arising condition. The fifteen things are a condition by being a co-arising condition. 

Performing\marginnote{9.13} the robe-making ceremony is a condition for determining by being a proximity condition, an immediacy condition, a support condition, and a decisive support condition. Determining is a condition for performing the robe-making ceremony by being a pre-arising condition. Performing the robe-making ceremony is a condition for determining by being a post-arising condition. The fifteen things are a condition by being a co-arising condition. 

The\marginnote{9.17} key phrases and the obstacles are a condition for performing the robe-making ceremony by being a proximity condition, an immediacy condition, a support condition, and a decisive support condition. Performing the robe-making ceremony is a condition for the key phrases and the obstacles by being a pre-arising condition. The key phrases and the obstacles are a condition for performing the robe-making ceremony by being a post-arising condition. The fifteen things are a condition by being a co-arising condition. 

Expectation\marginnote{9.21} and non-expectation are a condition for the object by being a proximity condition, an immediacy condition, a support condition, and a decisive support condition. The object is a condition for expectation and non-expectation by being a pre-arising condition. Expectation and non-expectation are a condition for the object by being a post-arising condition. The fifteen things are a condition by being a co-arising condition. 

\section*{3. The details on the source of the prior duties, etc. }

What\marginnote{10.1} is the source, the origin, the birth, the arising, the production, the origination of the prior duties? What is the source, the origin, the birth, the arising, the production, the origination of relinquishment? What is the source, the origin, the birth, the arising, the production, the origination of determining? What is the source, the origin, the birth, the arising, the production, the origination of performing the robe-making ceremony? What is the source, the origin, the birth, the arising, the production, the origination of the key phrases and the obstacles? What is the source, the origin, the birth, the arising, the production, the origination of expectation and non-expectation? 

The\marginnote{11.1} prior duties have effort as their source, origin, birth, arising, production, and origination. Relinquishment has the prior duties as its source, origin, birth, arising, production, and origination. Determining has relinquishment as its source, origin, birth, arising, production, and origination. Performing the robe-making ceremony has determining as its source, origin, birth, arising, production, and origination. The Key Terms and the obstacles have participating in the robe-making ceremony as their source, origin, birth, arising, production, and origination. Expectation and non-expectation have the key phrases and the obstacles as their source, origin, birth, arising, production, and origination. 

What\marginnote{12.1} is the source, the origin, the birth, the arising, the production, the origination of effort? … of the prior duties? … of relinquishment? … of determining? … of performing the robe-making ceremony? … of the key phrases and the obstacles? … of the object? What is the source, the origin, the birth, the arising, the production, the origination of expectation and non-expectation? 

Effort\marginnote{13.1} has causes as its source, origin, birth, arising, production, and origination. The prior duties … Relinquishment … Determining … Performing the robe-making ceremony … The key phrases and the obstacles … The object … Expectation and non-expectation have causes as their source, origin, birth, arising, production, and origination. 

What\marginnote{14.1} is the source, the origin, the birth, the arising, the production, the origination of effort? … of the prior duties? … of relinquishment? … of determining? … of performing the robe-making ceremony? … of the key phrases and the obstacles? … of the object? What is the source, the origin, the birth, the arising, the production, the origination of expectation and non-expectation? 

Effort\marginnote{15.1} has conditions as its source, origin, birth, arising, production, and origination. The prior duties … Relinquishment … Determining … Performing the robe-making ceremony … The key phrases and the obstacles … The object … Expectation and non-expectation have conditions as their source, origin, birth, arising, production, and origination. 

How\marginnote{16.1} many things are grouped with the prior duties? Seven things: washing, planning, cutting, tacking, sewing, dyeing, and marking. 

How\marginnote{17.1} many things are grouped with relinquishment? Three things: the outer robe, the upper robe, and the sarong. 

How\marginnote{18.1} many things are grouped with determining? Three things: the outer robe, the upper robe, and the sarong. 

How\marginnote{19.1} many things are grouped with performing the robe-making ceremony? One thing: breaking into speech. 

How\marginnote{20.1} many roots does the robe-making ceremony have, how many objects, and how many grounds? The robe-making ceremony has one root: the Sangha. It has three objects: the outer robe, the upper robe, and the sarong. It has six grounds: linen, cotton, silk, wool, sunn hemp, and hemp. 

What\marginnote{21.1} is the beginning, the middle, and the end of the robe-making ceremony? The prior duties are the beginning, the performing is the middle, the robe-making ceremony is the end.\footnote{Sp 5.408: \textit{\textsanskrit{Kiriyā} majjheti \textsanskrit{paccuddhāro} ceva \textsanskrit{adhiṭṭhānañca}}, “\textit{\textsanskrit{Kiriyā} majjhe}: just relinquishing and determining.” } 

What\marginnote{22.1} sort of person is unable to perform the robe-making ceremony? What sort of person is able to perform the robe-making ceremony? A person who has eight qualities is unable to perform the robe-making ceremony. A person who has eight qualities is able to perform the robe-making ceremony. 

What\marginnote{22.5} are the eight qualities of a person who is unable to perform the robe-making ceremony? They do not know the prior duties, relinquishment, determining, the robe-making ceremony, the key phrases, the obstacles, the ending of the robe season, or the benefits. 

What\marginnote{22.7} are the eight qualities of a person who is able to perform the robe-making ceremony? They know the prior duties, relinquishment, determining, the robe-making ceremony, the key phrases, the obstacles, the ending of the robe season, and the benefits. 

For\marginnote{23.1} how many kinds of people is the robe-making ceremony not effective? For how many kinds of people is the robe-making ceremony effective? It is not effective for three kinds of people. It is effective for three kinds of people. 

For\marginnote{23.5} which three kinds of people is it not effective? For one who expresses their appreciation outside the monastery zone; for one who does not express their appreciation verbally; for one who expresses it verbally, but does not inform anyone. 

For\marginnote{23.7} which three kinds of people is it effective? For one who expresses their appreciation inside the monastery zone; for one who expresses their appreciation verbally; for one who expresses it verbally and informs someone. 

How\marginnote{24.1} many kinds of robe-making ceremonies are invalid? How many kinds of robe-making ceremonies are valid? Three kinds of robe-making ceremonies are invalid. Three kinds of robe-making ceremonies are valid. What are the three kinds of robe-making ceremonies that are invalid? The object fails; the timing fails; the making fails.\footnote{Sp 5.411: \textit{\textsanskrit{Vatthuvipannaṁ} \textsanskrit{hotīti} \textsanskrit{akappiyadussaṁ} hoti. \textsanskrit{Kālavipannaṁ} \textsanskrit{nāma} ajja \textsanskrit{dāyakehi} \textsanskrit{dinnaṁ} sve \textsanskrit{saṅgho} \textsanskrit{kathinatthārakassa} deti. \textsanskrit{Karaṇavipannaṁ} \textsanskrit{nāma} tadaheva \textsanskrit{chinditvā} \textsanskrit{akataṁ}}, “‘The object fails’: the cloth is unallowable. ‘The timing fails’: today it is given by the donors; tomorrow the Sangha gives to the one who is doing the robe-making ceremony. ‘The making fails’: they cut it, but do not finish it on the same day.” } What are the three kinds of robe-making ceremonies that are valid? The object succeeds; the timing succeeds; the making succeeds. 

\section*{4. The details on what is to be known about the robe-making ceremony, etc. }

The\marginnote{25.1} robe-making ceremony is to be known. Participation in the robe-making ceremony is to be known. The month of the robe-making ceremony is to be known. Failure of the robe-making ceremony is to be known. Success of the robe-making ceremony is to be known. The giving of an indication is to be known. A hint is to be known. Borrowing is to be known. Storing is to be known. Relinquishment is to be known. 

\begin{description}%
\item[The robe-making ceremony is to be known: ] “the robe-making ceremony” is the grouping and coming together of just those things—their name, appellation, label, terminology, wording, designation.\footnote{Sp 5.412: \textit{\textsanskrit{Tesaññeva} \textsanskrit{dhammānanti} yesu \textsanskrit{rūpādidhammesu} sati \textsanskrit{kathinaṁ} \textsanskrit{nāma} hoti, \textsanskrit{tesaṁ} \textsanskrit{samodhānaṁ} \textsanskrit{missībhāvo}}, “‘Of just those things’: the combination and mixing together of those things, such as form, etc., by which there is the name \textit{kathina}.” Vmv 5.412: \textit{Yesu \textsanskrit{rūpādidhammesūti} “\textsanskrit{purimavassaṁvutthā} \textsanskrit{bhikkhū}, \textsanskrit{pañcahi} \textsanskrit{anūno} \textsanskrit{saṅgho}, \textsanskrit{cīvaramāso}, dhammena samena \textsanskrit{samuppannaṁ} \textsanskrit{cīvara}”nti \textsanskrit{evamādīsu} yesu \textsanskrit{rūpārūpadhammesu}}, “‘Of those things, such as form, etc.’: monks who have completed the first rains residence, a sangha of no less than five, the robe month, a robe that has arisen legitimately and validly, etc., are those things such as form, etc.” } %
\item[The month of the robe-making ceremony is to be known: ] the last month of the rainy season. %
\item[Failure of the robe-making ceremony is to be known: ] the robe-making ceremony fails in twenty-four ways.\footnote{For the twenty-four, see above and \href{https://suttacentral.net/pli-tv-kd7/en/brahmali\#1.5.3}{Kd 7:1.5.3}–1.5.26. } %
\item[Success of the robe-making ceremony is to be known: ] The robe-making ceremony is successful through seventeen aspects.\footnote{For the seventeen, see above and \href{https://suttacentral.net/pli-tv-kd7/en/brahmali\#1.6.2}{Kd 7:1.6.2}. } %
\item[The giving of an indication is to be known: ] one makes an indication, thinking, “I’ll perform the robe-making ceremony with this cloth.” %
\item[A hint is to be known: ] one gives a hint, thinking, “With this hint, I’ll make a cloth for the robe-making ceremony appear.” %
\item[Borrowing is to be known: ] a gift not to be taken as one’s own. %
\item[Storing is to be known: ] there are two kinds of storing: for the purpose of making and for the purpose of accumulation. %
\item[Relinquishment is to be known: ] dawn arrives while it is being made. %
\item[Participation in the robe-making ceremony is to be known: ] If\marginnote{35.2} cloth has been given to the Sangha for the robe-making ceremony, how should the Sangha proceed? How should the one who performs the robe-making ceremony proceed? How should one who expresses their appreciation proceed? 

The\marginnote{36.1} Sangha should give the cloth, through a legal procedure consisting of one motion and one announcement, to the monk who is performing the robe-making ceremony. The monk who is performing the robe-making ceremony should wash the cloth, iron it, plan it, cut it, sew it, dye it, mark it, and then perform the robe-making ceremony. If he wishes to perform the robe-making ceremony with an outer robe, he should relinquish his old outer robe and then determine the new one. He should then say, “I perform the robe-making ceremony with this outer robe.” If he wishes to perform the robe-making ceremony with an upper robe, he should relinquish his old upper robe and then determine the new one. He should then say, “I perform the robe-making ceremony with this upper robe.” If he wishes to perform the robe-making ceremony with a sarong, he should relinquish his old sarong and then determine the new one. He should then say, “I perform the robe-making ceremony with this sarong.” 

After\marginnote{36.8} approaching the Sangha, that monk who is performing the robe-making ceremony should arrange his upper robe over one shoulder, raise his joined palms, and say: “Venerable sirs, the Sangha’s robe-making ceremony has been performed. The robe-making ceremony is legitimate. Please express your appreciation.” The monks who are expressing their appreciation should arrange their upper robes over one shoulder, raise their joined palms, and say: “The Sangha’s robe-making ceremony has been performed. The robe-making ceremony is legitimate. We express our appreciation.” 

Or:\marginnote{36.12} after approaching several monks, that monk who is performing the robe-making ceremony should arrange his upper robe over one shoulder, raise his joined palms, and say:\footnote{For an explanation of the rendering “several” for \textit{sambahula}, see Appendix of Technical Terms. } “Venerable sirs, the Sangha’s robe-making ceremony has been performed. The robe-making ceremony is legitimate. Please express your appreciation.” The monks who are expressing their appreciation should arrange their upper robes over one shoulder, raise their joined palms, and say: “The Sangha’s robe-making ceremony has been performed. The robe-making ceremony is legitimate. We express our appreciation.” 

Or:\marginnote{36.16} after approaching a single monk, that monk who is performing the robe-making ceremony should arrange his upper robe over one shoulder, raise his joined palms, and say: “The Sangha’s robe-making ceremony has been performed. The robe-making ceremony is legitimate. Please express your appreciation.” The monk who is expressing his appreciation should arrange his upper robe over one shoulder, raise his joined palms, and say: “The Sangha’s robe-making ceremony has been performed. The robe-making ceremony is legitimate. I express my appreciation.” 

%
\end{description}

\section*{5. The robe-making ceremony for an individual }

“Does\marginnote{37.1} the Sangha perform the robe-making ceremony? Does a group perform the robe-making ceremony? Does an individual perform the robe-making ceremony?” “The Sangha does not perform the robe-making ceremony, nor does a group, but an individual does.” If the Sangha does not perform the robe-making ceremony, nor a group, but an individual does, then the Sangha has not performed the robe-making ceremony, nor has a group, but an individual has. 

“Does\marginnote{37.5} the Sangha recite the Monastic Code? Does a group recite the Monastic Code? Does an individual recite the Monastic Code?” “The Sangha does not recite the Monastic Code, nor does a group, but an individual does.” If the Sangha does not recite the Monastic Code, nor a group, but an individual does, then the Sangha has not recited the Monastic Code, nor has a group, but an individual has. 

“Yet\marginnote{37.9} when an individual recites to unite the Sangha, to unite the group, then the Monastic Code has been recited by the Sangha, by the group, and by the individual. It is in this way that the Sangha does not perform the robe-making ceremony, nor does a group, but an individual does. Yet when an individual performs the robe-making ceremony, and the Sangha expresses its appreciation, the group expresses its appreciation, then the robe-making ceremony has been performed by the Sangha, by the group, and by the individual.” 

\section*{6. Questions and answers regarding the obstacles }

\begin{verse}%
“The\marginnote{38.1} robe season ends when one departs from the monastery. \\
So said the Kinsman of the Sun. \\
And about this I ask you: \\
Which obstacle is removed first? 

The\marginnote{39.1} robe season ends when one departs from the monastery. \\
So said the Kinsman of the Sun. \\
And about this I answer you: \\
The robe obstacle is removed first. \\
The monastery obstacle is removed when one goes outside the monastery zone. 

The\marginnote{40.1} robe season ends when the robe is finished. \\
So said the Kinsman of the Sun. \\
And about this I ask you: \\
Which obstacle is removed first? 

The\marginnote{41.1} robe season ends when the robe is finished. \\
So said the Kinsman of the Sun. \\
And about this I answer you: \\
The monastery obstacle is removed first. \\
The robe obstacle is removed when the robe is finished. 

The\marginnote{42.1} robe season ends when he makes that decision. \\
So said the Kinsman of the Sun. \\
And about this I ask you: \\
Which obstacle is removed first? 

The\marginnote{43.1} robe season ends when he makes that decision. \\
So said the Kinsman of the Sun. \\
And about this I answer you: \\
The two obstacles are removed simultaneously. 

The\marginnote{44.1} robe season ends when the robe-cloth is lost. \\
So said the Kinsman of the Sun. \\
And about this I ask you: \\
Which obstacle is removed first? 

The\marginnote{45.1} robe season ends when the robe-cloth is lost. \\
So said the Kinsman of the Sun. \\
And about this I answer you: \\
The monastery obstacle is removed first. \\
The robe obstacle is removed when the robe-cloth is lost. 

The\marginnote{46.1} robe season ends when he hears about the end of the robe season. \\
So said the Kinsman of the Sun. \\
And about this I ask you: \\
Which obstacle is removed first? 

The\marginnote{47.1} robe season ends when he hears about the end of the robe season. \\
So said the Kinsman of the Sun. \\
And about this I answer you: \\
The robe obstacle is removed first. \\
The monastery obstacle is removed when one hears about the end of the robe season. 

The\marginnote{48.1} robe season ends when the expectation is disappointed. \\
So said the Kinsman of the Sun. \\
And about this I ask you: \\
Which obstacle is removed first? 

The\marginnote{49.1} robe season ends when the expectation is disappointed. \\
So said the Kinsman of the Sun. \\
And about this I answer you: \\
The monastery obstacle is removed first. \\
The robe obstacle is removed when the expectation of more robe-cloth is disappointed. 

The\marginnote{50.1} robe season ends while he is outside the monastery zone. \\
So said the Kinsman of the Sun. \\
And about this I ask you: \\
Which obstacle is removed first? 

The\marginnote{51.1} robe season ends while he is outside the monastery zone. \\
So said the Kinsman of the Sun. \\
And about this I answer you: \\
The robe obstacle is removed first. \\
The monastery obstacle is removed when one is outside the monastery zone. 

The\marginnote{52.1} robe season ends together. \\
So said the Kinsman of the Sun. \\
And about this I ask you: \\
Which obstacle is removed first? 

The\marginnote{53.1} robe season ends together. \\
So said the Kinsman of the Sun. \\
And about this I answer you: \\
The two obstacles are removed simultaneously.” 

%
\end{verse}

“How\marginnote{54.1} many kinds of endings of the robe season depend on the Sangha? How many kinds of endings of the robe season depend on an individual? How many kinds of endings of the robe season depend neither on the Sangha nor on an individual? 

One\marginnote{54.4} kind of ending of the robe season depends on the Sangha: when the robe season ends midway.\footnote{According to Sp 1.462, this refers to the legal procedure done by the Sangha to end the robe season before its natural end, for which see \href{https://suttacentral.net/pli-tv-bi-vb-pc30/en/brahmali\#1.1.10}{Bi Pc 30:1.1.10}. } Four kinds of endings of the robe season depend on an individual: when one departs from the monastery, when the robe is finished, when one makes a decision, and when one is outside the monastery zone. Four kinds of endings of the robe season depend neither on the Sangha nor on an individual: when the robe-cloth is lost, when one hears about the end of the robe season, when an expectation of more robe-cloth is disappointed, and when the robe season ends together. 

How\marginnote{54.10} many kinds of endings of the robe season happen inside the monastery zone? How many kinds of endings of the robe season happen outside the monastery zone? How many kinds of endings of the robe season may happen either inside or outside the monastery zone? 

Two\marginnote{54.13} kinds of endings of the robe season happen inside the monastery zone: when the robe season ends midway, and when the robe season ends together. Three kinds of endings of the robe season happen outside the monastery zone: when one departs from the monastery, when one hears about the end of the robe season, and when one is outside the monastery zone. Four kinds of endings of the robe season may happen inside or outside the monastery zone: when the robe is finished, when one makes a decision, when the robe-cloth is lost, and when an expectation of more robe-cloth is disappointed. 

How\marginnote{55.1} many kinds of endings of the robe season arise together and end together? How many kinds of endings of the robe season arise together and end separately? 

Two\marginnote{55.3} kinds of endings of the robe season arise together and end together: when the robe season ends midway, and when the robe season ends together. The rest of the endings of the robe season arise together but end separately.” 

\scend{The robe-making ceremony  is finished. }

\scuddanaintro{This is the summary: }

\begin{scuddana}%
“Who\marginnote{58.1} has, how, fifteen, \\
Things, source, and cause; \\
Condition, grouped with, roots, \\
And beginning, persons who perform the robe-making ceremony. 

For\marginnote{59.1} three, three, to be known, \\
The robe-making ceremony, and with recitation; \\
Obstacles, depend on, in a monastery zone, \\
And with arise and end.” 

%
\end{scuddana}

%
\chapter*{{\suttatitleacronym Pvr 17}{\suttatitletranslation Ven. Upāli questions the Buddha }{\suttatitleroot Upālipañcaka}}
\addcontentsline{toc}{chapter}{\tocacronym{Pvr 17} \toctranslation{Ven. Upāli questions the Buddha } \tocroot{Upālipañcaka}}
\markboth{Ven. Upāli questions the Buddha }{Upālipañcaka}
\extramarks{Pvr 17}{Pvr 17}

\section*{1. The subchapter on “without formal support” }

At\marginnote{1.1} one time the Buddha was staying at \textsanskrit{Sāvatthī} in the Jeta Grove, \textsanskrit{Anāthapiṇḍika}’s Monastery. Venerable \textsanskrit{Upāli} went to the Buddha, bowed, sat down, and said, “Venerable sir, what sort of monk should live with formal support for life?” 

“One\marginnote{2.1} who has five qualities, \textsanskrit{Upāli}: (1) he doesn’t know about the observance-day ceremony; (2) he doesn’t know the observance-day procedure; (3) he doesn’t know the Monastic Code; (4) he doesn’t know the recitation of the Monastic Code; (5) he has less than five years of seniority.\footnote{For an explanation of the rendering “observance-day ceremony” for \textit{uposatha}, see Appendix of Technical Terms. } But a monk who has five qualities may live without formal support for life: (1) he knows about the observance-day ceremony; (2) he knows the observance-day procedure; (3) he knows the Monastic Code; (4) he knows the recitation of the Monastic Code; (5) he has five or more years of seniority. 

A\marginnote{3.1} monk who has five other qualities should also live with formal support for life: (1) he doesn’t know about the invitation ceremony; (2) he doesn’t know the invitation procedure; (3) he doesn’t know the Monastic Code; (4) he doesn’t know the recitation of the Monastic Code; (5) he has less than five years of seniority.\footnote{For an explanation of the rendering “invitation ceremony” for \textit{\textsanskrit{pavāraṇā}}, see Appendix of Technical Terms. } But a monk who has five qualities may live without formal support for life: (1) he knows about the invitation ceremony; (2) he knows the invitation procedure; (3) he knows the Monastic Code; (4) he knows the recitation of the Monastic Code; (5) he has five or more years of seniority. 

A\marginnote{4.1} monk who has five other qualities should also live with formal support for life: (1) he doesn’t know the offenses and non-offenses; (2) he doesn’t know the light and heavy offenses; (3) he doesn’t know the curable and incurable offenses; (4) he doesn’t know the grave and minor offenses; (5) he has less than five years of seniority. But a monk who has five qualities may live without formal support for life: (1) he knows the offenses and non-offenses; (2) he knows the light and heavy offenses; (3) he knows the curable and incurable offenses; (4) he knows the grave and minor offenses; (5) he has five or more years of seniority. 

“Sir,\marginnote{5.1} what sort of monk shouldn’t give the full ordination or formal support, nor have a novice monk attend on him?”\footnote{For an explanation of the rendering “formal support” for \textit{nissaya}, see Appendix of Technical Terms. } 

“One\marginnote{6.1} who has five qualities: He’s incapable of three things in regard to a student: (1) of nursing him or having him nursed when he’s sick; (2) of sending him away or having him sent away when he’s discontent with the spiritual life; and (3) of using the Teaching to dispel anxiety. And (4) he is incapable of training him in the Teaching; and (5) he is incapable of training him in the Monastic Law. But a monk who has five qualities may give the full ordination and formal support, and he may have a novice monk attend on him: He’s capable of three things in regard to a student: (1) of nursing him or having him nursed when he’s sick; (2) of sending him away or having him sent away when he’s discontent with the spiritual life; and (3) of using the Teaching to dispel anxiety. And (4) he is capable of training him in the Teaching; and (5) he is capable of training him in the Monastic Law. 

A\marginnote{7.1} monk who has five other qualities also shouldn’t give the full ordination or formal support, nor have a novice monk attend on him: He’s incapable of five things in regard to a student: (1) of training him in good conduct; (2) of training him in the basics of the spiritual life; (3) of training him in the higher morality; (4) of training him in the higher mind; (5) of training him in the higher wisdom. But a monk who has five qualities may give the full ordination and formal support, and he may have a novice monk attend on him: He’s capable of five things in regard to a student: (1) of training him in good conduct; (2) of training him in the basics of the spiritual life; (3) of training him in the higher morality; (4) of training him in the higher mind; (5) of training him in the higher wisdom.” 

“Sir,\marginnote{8.1} against what sort of monk should a legal procedure be done?” 

“Against\marginnote{9.1} one who has five qualities: he is shameless, ignorant, and not a regular monk, and he has wrong view, and he has failed in livelihood. 

A\marginnote{10.1} legal procedure should also be done against a monk who has five other qualities: he has failed in the higher morality; he has failed in conduct; he has failed in view; he has wrong view; and he has failed in livelihood. 

A\marginnote{11.1} legal procedure should also be done against a monk who has five other qualities: his bodily conduct is frivolous; his verbal conduct is frivolous; his bodily and verbal conduct are frivolous; he has wrong view; and he has failed in livelihood. 

A\marginnote{12.1} legal procedure should also be done against a monk who has five other qualities: he is improperly behaved by body; he is improperly behaved by speech; he is improperly behaved by body and speech; he has wrong view; and he has failed in livelihood. 

A\marginnote{13.1} legal procedure should also be done against a monk who has five other qualities: his bodily conduct is harmful; his verbal conduct is harmful; his bodily and verbal conduct are harmful; he has wrong view; and he has failed in livelihood. 

A\marginnote{14.1} legal procedure should also be done against a monk who has five other qualities: he has wrong livelihood by body; he has wrong livelihood by speech; he has wrong livelihood by body and speech; he has wrong view; and he has failed in livelihood. 

A\marginnote{15.1} legal procedure should also be done against a monk who has five other qualities: if, after committing an offense and having had a legal procedure done against him, he: (1) gives the full ordination, (2) gives formal support, (3) has a novice monk attend on him, (4) accepts being appointed as an instructor of the nuns, (5) instructs the nuns, whether appointed or not. 

A\marginnote{16.1} legal procedure should also be done against a monk who has five other qualities: (1) he commits the same offense for which the Sangha did the legal procedure against him; (2) he commits an offense similar to the one for which the Sangha did the legal procedure against him; (3) he commits an offense worse than the one for which the Sangha did the legal procedure against him; (4) he criticizes the procedure; (5) he criticizes those who did the procedure. 

A\marginnote{17.1} legal procedure should also be done against a monk who has five other qualities: he disparages the Buddha; he disparages the Teaching; he disparages the Sangha; he has wrong view; and he has failed in livelihood.” 

\scendvagga{The first subchapter on “without formal support” is finished. }

\scuddanaintro{This is the summary: }

\begin{scuddana}%
“Observance\marginnote{20.1} day, invitation ceremony, \\
And offense, one who is sick; \\
Good conduct, and shameless, \\
Higher morality, and with frivolity. 

Improperly\marginnote{21.1} behaved, harmful, \\
Wrong, and offense; \\
Offense for which, of the Buddha—\\
The compilation of the first subchapter is finished.” 

%
\end{scuddana}

\section*{2. The subchapter on not lifting }

“Sir,\marginnote{22.1} for what sort of monk should a legal procedure not be lifted?” 

“For\marginnote{23.1} one who has five qualities, \textsanskrit{Upāli}: if, after committing an offense and having had a legal procedure done against him, he: (1) gives the full ordination, (2) gives formal support, (3) has a novice monk attend on him, (4) accepts being appointed as an instructor of the nuns, (5) instructs the nuns, whether appointed or not. 

A\marginnote{24.1} legal procedure should also not be lifted for a monk who has five other qualities: (1) he commits the same offense for which the Sangha did the legal procedure against him; (2) he commits an offense similar to the one for which the Sangha did the legal procedure against him; (3) he commits an offense worse than the one for which the Sangha did the legal procedure against him; (4) he criticizes the procedure; (5) he criticizes those who did the procedure. 

A\marginnote{25.1} legal procedure should also not be lifted for a monk who has five other qualities: he disparages the Buddha; he disparages the Teaching; he disparages the Sangha; he has wrong view; and he has failed in livelihood. 

A\marginnote{26.1} legal procedure should also not be lifted for a monk who has five other qualities: he is shameless, ignorant, and not a regular monk, and he is a bully, and he doesn’t fulfill the training in proper conduct.” 

“Sir,\marginnote{27.1} when a monk is involved in a conflict and is about to approach the Sangha, how many qualities should he first set up in himself?” 

“He\marginnote{28.1} should set up five qualities in himself: (1) he should be humble; (2) he should be intent on removing defilements; (3) he should be skilled in appropriate seating and where to sit down, taking a seat without encroaching on the senior monks and without blocking the junior monks; (4) he shouldn’t ramble or engage in worldly talk, but should speak according to the Teaching or invite others to speak or value noble silence; (5) if the Sangha is doing legal procedures requiring unity, but the monk doesn’t approve, then he should reveal his view but think, ‘I shouldn’t be at variance with the Sangha,’ and unity can then be announced.” 

“What\marginnote{29.1} sort of monk does the majority dislike and disapprove of when he speaks in the Sangha?” 

“One\marginnote{30.1} who has five qualities: (1) he’s arrogant; (2) he repeats what others say; (3) he doesn’t keep to the topic; (4) he doesn’t accuse others according to the Teaching, the Monastic Law, or their offense; (5) he doesn’t act according to the Teaching, the Monastic Law, or his offenses. But when a monk has five qualities, the majority likes and approves of him when he speaks in the Sangha: (1) he’s not arrogant; (2) he doesn’t repeat what others say; (3) he keeps to the topic; (4) he accuses others according to the Teaching, the Monastic Law, and their offense; (5) he acts according to the Teaching, the Monastic Law, and his offenses. 

When\marginnote{31.1} a monk has five other qualities, the majority dislikes and disapproves of him when he speaks in the Sangha: (1) he praises; and (2) he blames; (3) he maintains what is contrary to the Teaching; (4) he obstructs what is in accordance with the Teaching; and (5) he often speaks frivolously. But when a monk has five qualities, the majority likes and approves of him when he speaks in the Sangha: (1) he doesn’t praise; and (2) he doesn’t blame; (3) he maintains what is in accordance with the Teaching; (4) he obstructs what is contrary to the Teaching; and (5) he rarely speaks frivolously. 

When\marginnote{32.1} a monk has five other qualities, the majority dislikes and disapproves of him when he speaks in the Sangha: (1) he speaks forcefully; (2) he speaks without having gotten permission; (3) he doesn’t accuse others according to the Teaching, the Monastic Law, or their offense; (4) he doesn’t act according to the Teaching, the Monastic Law, or his offenses; (5) he doesn’t explain things according to his own view. But when a monk has five qualities, the majority likes and approves of him when he speaks in the Sangha: (1) he doesn’t speak forcefully; (2) he doesn’t speak without having gotten permission; (3) he accuses others according to the Teaching, the Monastic Law, and their offense; (4) he acts according to the Teaching, the Monastic Law, and his offenses; (5) he explains things according to his own view.” 

“Sir,\marginnote{33.1} how many benefits are there of studying the Monastic Law?” 

“There\marginnote{34.1} are these five benefits: (1) your own morality is well guarded; (2) you’re a refuge for those who are habitually anxious; (3) you speak with confidence in the midst of the Sangha; (4) you can legitimately and properly refute an opponent; (5) you’re practicing for the longevity of the true Teaching.” 

\scendvagga{The second subchapter on not lifting is finished. }

\scuddanaintro{This is the summary: }

\begin{scuddana}%
“Committing,\marginnote{37.1} for which, and praise, \\
Shameless, and with conflict; \\
Arrogant, and praises, \\
Forcefully, studying.” 

%
\end{scuddana}

\scendsutta{The first pairs have been laid down. }

\section*{3. The subchapter on speech }

“Sir,\marginnote{39.1} what sort of monk shouldn’t speak in the Sangha?” 

“One\marginnote{40.1} who has five qualities, \textsanskrit{Upāli}: (1) he doesn’t know the offenses; (2) he doesn’t know the origination of the offenses; (3) he doesn’t know the kind of effort required to commit the offenses; (4) he doesn’t know the settling of offenses; (5) he’s not skilled in deciding on offenses.\footnote{Sp 5.424: \textit{\textsanskrit{Āpattiyā} \textsanskrit{payogaṁ} na \textsanskrit{jānātīti} “\textsanskrit{ayaṁ} \textsanskrit{āpatti} \textsanskrit{kāyappayogā}, \textsanskrit{ayaṁ} \textsanskrit{vacīpayogā}”ti na \textsanskrit{jānāti} … \textsanskrit{Āpattiyā} na vinicchayakusalo \textsanskrit{hotīti} “\textsanskrit{imasmiṁ} \textsanskrit{vatthusmiṁ} \textsanskrit{ayaṁ} \textsanskrit{āpattī}”ti na \textsanskrit{jānāti}}, “\textit{\textsanskrit{Āpattiyā} \textsanskrit{payogaṁ} na \textsanskrit{jānāti}} means he does not know: ‘This is an offense because of bodily effort; this is an offense because of verbal effort.’ … \textit{\textsanskrit{Āpattiyā} na vinicchayakusalo hoti} means he does not know: ‘When there is this action, there is this offense.’” } But a monk who has five qualities may speak in the Sangha: (1) he knows the offenses; (2) he knows the origination of the offenses; (3) he knows the kind of effort required to commit the offenses; (4) he knows the settling of offenses; (5) he’s skilled in deciding on offenses. 

A\marginnote{41.1} monk who has five other qualities also shouldn’t speak in the Sangha: (1) he doesn’t know the legal issues; (2) he doesn’t know the origination of the legal issues; (3) he doesn’t know the kind of effort that is the source of the legal issues; (4) he doesn’t know the settling of legal issues; (5) he’s not skilled in deciding legal issues.\footnote{Sp 5.424: \textit{\textsanskrit{Adhikaraṇānañhi} \textsanskrit{yathāsakaṁmūlameva} \textsanskrit{payogā} \textsanskrit{nāma} honti}, “For it is just the root of the legal issues that is called effort.” } But a monk who has five qualities may speak in the Sangha: (1) he knows the legal issues; (2) he knows the origination of the legal issues; (3) he knows the kind of effort that is the source of the legal issues; (4) he knows the settling of legal issues; (5) he’s skilled in deciding legal issues. 

A\marginnote{42.1} monk who has five other qualities also shouldn’t speak in the Sangha: (1) he speaks forcefully; (2) he speaks without having gotten permission; (3) he doesn’t accuse others according to the Teaching, the Monastic Law, or their offense; (4) he doesn’t act according to the Teaching, the Monastic Law, or his offenses; (5) he doesn’t explain things according to his own view. But a monk who has five qualities may speak in the Sangha: (1) he doesn’t speak forcefully; (2) he doesn’t speak without having gotten permission; (3) he accuses others according to the Teaching, the Monastic Law, and their offense; (4) he acts according to the Teaching, the Monastic Law, and his offenses; (5) he explains things according to his own view. 

A\marginnote{43.1} monk who has five other qualities also shouldn’t speak in the Sangha: (1) he doesn’t know the offenses and non-offenses; (2) he doesn’t know the light and heavy offenses; (3) he doesn’t know the curable and incurable offenses; (4) he doesn’t know the grave and minor offenses; (5) he doesn’t know the offenses that are clearable by making amends and the offenses that are not clearable by making amends. But a monk who has five qualities may speak in the Sangha: (1) he knows the offenses and non-offenses; (2) he knows the light and heavy offenses; (3) he knows the curable and incurable offenses; (4) he knows the grave and minor offenses; (5) he knows the offenses that are clearable by making amends and the offenses that are not clearable by making amends. 

A\marginnote{44.1} monk who has five other qualities also shouldn’t speak in the Sangha: (1) he doesn’t know the legal procedures; (2) he doesn’t know how the legal procedures are done; (3) he doesn’t know the actions that are the bases for the legal procedures; (4) he doesn’t know the proper conduct in relation to the legal procedures; (5) he doesn’t know the settling of the legal procedures.\footnote{Sp 5.424: \textit{Vattanti sattasu kammesu \textsanskrit{heṭṭhā} \textsanskrit{catunnaṁ} \textsanskrit{kammānaṁ} \textsanskrit{aṭṭhārasavidhaṁ} tividhassa ca \textsanskrit{ukkhepanīyakammassa} \textsanskrit{tecattālīsavidhaṁ} \textsanskrit{vattaṁ} na \textsanskrit{jānāti}}, “\textit{Vatta}: among the seven legal procedures found below, he does not know the eighteen kinds of proper conduct of the four legal procedures or the forty-three kinds of proper conduct of the three kinds of legal procedures of ejection.” } But a monk who has five qualities may speak in the Sangha: (1) he knows the legal procedures; (2) he knows how the legal procedures are done; (3) he knows the actions that are the bases for the legal procedures; (4) he knows the proper conduct in relation to the legal procedures; (5) he knows the settling of the legal procedures. 

A\marginnote{45.1} monk who has five other qualities also shouldn’t speak in the Sangha: (1) he doesn’t know the actions that are the bases for offenses; (2) he doesn’t know the origin stories; (3) he doesn’t know the rules; (4) he doesn’t know the right order of words; (5) he doesn’t know the sequence of statements.\footnote{Sp 5.424: \textit{\textsanskrit{Padapaccābhaṭṭhaṁ} na \textsanskrit{jānātīti} \textsanskrit{sammukhā} \textsanskrit{kātabbaṁ} \textsanskrit{padaṁ} na \textsanskrit{jānāti}. “Buddho \textsanskrit{bhagavā}”ti vattabbe “\textsanskrit{bhagavā} buddho”ti \textsanskrit{heṭṭhupariyaṁ} \textsanskrit{katvā} \textsanskrit{padaṁ} yojeti}, “\textit{\textsanskrit{Padapaccābhaṭṭhaṁ} na \textsanskrit{jānāti}}: he does not know the words that are to be done in the presence of. When ‘Buddho \textsanskrit{bhagavā}’ is to be said, having reversed the order, he makes it ‘\textsanskrit{bhagavā} Buddho’.” | Sp 5.325: \textit{Anusandhivacanapathanti \textsanskrit{kathānusandhi}-\textsanskrit{vinicchayānusandhivasena} \textsanskrit{vatthuṁ} na \textsanskrit{jānāti}}, “\textit{Anusandhivacanapatha}: he does not understand the basis for the sequence of statements and the sequence of decisions.” } But a monk who has five qualities may speak in the Sangha: (1) he knows the actions that are the bases for offenses; (2) he knows the origin stories; (3) he knows the rules; (4) he knows the right order of words; (5) he knows the sequence of statements. 

A\marginnote{46.1} monk who has five other qualities also shouldn’t speak in the Sangha: he’s biased by desire, ill will, confusion, or fear, and he’s shameless. But a monk who has five qualities may speak in the Sangha: he’s not biased by desire, ill will, confusion, or fear, and he has a sense of conscience. 

A\marginnote{47.1} monk who has five other qualities also shouldn’t speak in the Sangha: he’s biased by desire, ill will, confusion, or fear, and he’s unskilled in the Monastic Law. But a monk who has five qualities may speak in the Sangha: he’s not biased by desire, ill will, confusion, or fear, and he’s skilled in the Monastic Law. 

A\marginnote{48.1} monk who has five other qualities also shouldn’t speak in the Sangha: (1) he doesn’t know the motion; (2) he doesn’t know how the motion is done; (3) he doesn’t know the proclamation of the motion; (4) he doesn’t know settling through a motion; (5) he doesn’t know resolution through a motion.\footnote{Sp 5.424: \textit{\textsanskrit{Ñattiyā} \textsanskrit{samathaṁ} na \textsanskrit{jānātīti} \textsanskrit{yvāyaṁ} sativinayo, \textsanskrit{amūḷhavinayo}, \textsanskrit{tassapāpiyasikā}, \textsanskrit{tiṇavatthārakoti} catubbidho samatho \textsanskrit{ñattiyā} \textsanskrit{vinā} na hoti, \textsanskrit{taṁ} \textsanskrit{ñattiyā} samathoti na \textsanskrit{jānāti}. \textsanskrit{Ñattiyā} \textsanskrit{vūpasamaṁ} na \textsanskrit{jānātīti} \textsanskrit{yaṁ} \textsanskrit{adhikaraṇaṁ} \textsanskrit{iminā} catubbidhena \textsanskrit{ñattisamathena} \textsanskrit{vūpasamati}, tassa \textsanskrit{taṁ} \textsanskrit{vūpasamaṁ} “\textsanskrit{ayaṁ} \textsanskrit{ñattiyā} \textsanskrit{vūpasamo} kato”ti na \textsanskrit{jānāti}}, “‘He doesn’t know settling through a motion’: without a motion, there is no fourfold settling of resolution through recollection, of resolution because of past insanity, of a further penalty, or of covering over as if with grass. He does not know that settling through a motion. ‘He doesn’t know resolution through a motion’: regarding a legal issue that is resolved through this fourfold settling through a motion, he does not know the resolution of it: ‘This resolution is done through a motion’.” } But a monk who has five qualities may speak in the Sangha: (1) he knows the motion; (2) he knows how the motion is done; (3) he knows the proclamation of the motion; (4) he knows settling by way of a motion; (5) he knows resolution by way of a motion. 

A\marginnote{49.1} monk who has five other qualities also shouldn’t speak in the Sangha: (1) he doesn’t know the Monastic Code; (2) he doesn’t know what’s in accordance with the Monastic Code; (3) he doesn’t know the Monastic Law; (4) he doesn’t know what’s in accordance with the Monastic Law; (5) he’s not skilled in what is and is not possible.\footnote{Sp 5.424: \textit{\textsanskrit{Suttaṁ} na \textsanskrit{jānātīti} \textsanskrit{ubhatovibhaṅgaṁ} na \textsanskrit{jānāti}. \textsanskrit{Suttānulomaṁ} na \textsanskrit{jānātīti} \textsanskrit{cattāro} \textsanskrit{mahāpadese} na \textsanskrit{jānāti}. \textsanskrit{Vinayaṁ} na \textsanskrit{jānātīti} \textsanskrit{khandhakaparivāraṁ} na \textsanskrit{jānāti}. \textsanskrit{Vinayānulomaṁ} na \textsanskrit{jānātīti} \textsanskrit{cattāro} \textsanskrit{mahāpadeseyeva} na \textsanskrit{jānāti}}, “‘He doesn’t know the Monastic Code’: he does not know the analyses of both Monastic Codes. ‘He doesn’t know what’s in accordance with the Monastic Code’: he does not know the four great standards. ‘He doesn’t know the Monastic Law’: he does not know the Chapters and the Compendium. ‘He doesn’t know what’s in accordance with the Monastic Law’: he does not know the four great standards.” Instead of following the commentary, I prefer to understand the word \textit{sutta} in its early usage of Monastic Code, equivalent to the \textsanskrit{Pātimokkha}, and Vinaya as the entire Vinaya corpus, the Monastic Law. } But a monk who has five qualities may speak in the Sangha: (1) he knows the Monastic Code; (2) he knows what’s in accordance with the Monastic Code; (3) he knows the Monastic Law; (4) he knows what’s in accordance with the Monastic Law; (5) he’s skilled in what is and is not possible. 

A\marginnote{50.1} monk who has five other qualities also shouldn’t speak in the Sangha: (1) he doesn’t know the Teaching; (2) he doesn’t know what’s in accordance with the Teaching; (3) he doesn’t know the Monastic Law; (4) he doesn’t know what’s in accordance with the Monastic Law; (5) he’s not skilled in the right order.\footnote{Sp 5.424: \textit{Na ca \textsanskrit{pubbāparakusalo} \textsanskrit{hotīti} \textsanskrit{purekathāya} ca \textsanskrit{pacchākathāya} ca akusalo hoti}, “‘He’s not skilled in the right order’: he is unskilled in what should be said first and what should be said afterwards.” } But a monk who has five qualities may speak in the Sangha: (1) he knows the Teaching; (2) he knows what’s in accordance with the Teaching; (3) he knows the Monastic Law; (4) he knows what’s in accordance with the Monastic Law; (5) he’s skilled in the right order.” 

\scendvagga{The third subchapter on speech is finished. }

\scuddanaintro{This is the summary: }

\begin{scuddana}%
“Offenses,\marginnote{53.1} legal issues, \\
Forcefully, knowing offenses; \\
Legal procedures, the actions that are the bases, and shameless, \\
And unskilled, of the motion; \\
He does not know the Monastic Code, or the Teaching—\\
The compilation of the third subchapter is finished.” 

%
\end{scuddana}

\section*{4. The subchapter on revealing one’s view }

“Sir,\marginnote{54.1} how many illegitimate kinds of revealing one’s view are there?” 

“There\marginnote{55.1} are five, \textsanskrit{Upāli}: (1) one reveals a view about a non-offense; (2) one reveals a view about an offense that isn’t clearable by confession; (3) one reveals a view about an offense that has been confessed; (4) one reveals a view to four or five people; (5) one reveals a view by mind.\footnote{Sp 5.425: \textit{\textsanskrit{Anāpattiyā} \textsanskrit{diṭṭhiṁ} \textsanskrit{āvi} \textsanskrit{karotīti} \textsanskrit{anāpattimeva} \textsanskrit{āpattīti} \textsanskrit{desetīti} attho}, “‘One reveals a view about a non-offense’: the meaning is that one confesses a non-offense as an offense.” Sp 5.425: \textit{\textsanskrit{Catūhi} \textsanskrit{pañcahi} \textsanskrit{diṭṭhinti} … \textsanskrit{cattāro} \textsanskrit{pañca} \textsanskrit{janā} ekato \textsanskrit{āpattiṁ} \textsanskrit{desentīti} attho}, “‘A view to four or five people’: … The meaning is that he confesses the offense to four or five people together.” Sp 5.425: \textit{\textsanskrit{Manomānasena} … \textsanskrit{vacībhedaṁ} \textsanskrit{akatvā} citteneva \textsanskrit{āpattiṁ} \textsanskrit{desetīti} attho}, “‘By mind’: the meaning is that he confesses the offense by mind, without breaking into speech.” } 

But\marginnote{56.1} there are five legitimate kinds of revealing one’s view: (1) one reveals a view about an offense; (2) one reveals a view about an offense that’s clearable by confession; (3) one reveals a view about an offense that hasn’t been confessed; (4) one doesn’t reveal a view to four or five people; (5) one doesn’t reveal a view by mind. 

There\marginnote{57.1} are five other illegitimate kinds of revealing one’s view: (1) one reveals a view to someone who belongs to a different Buddhist sect; (2) one reveals a view to someone in a different monastery zone; (3) one reveals a view to someone who’s not a regular monk; (4) one reveals a view to four or five people; (5) one reveals a view by mind. 

But\marginnote{58.1} there are five legitimate kinds of revealing one’s view: (1) one reveals a view to someone who belongs to the same Buddhist sect; (2) one reveals a view to someone in the same monastery zone; (3) one reveals a view to a regular monk; (4) one doesn’t reveal a view to four or five people; (5) one doesn’t reveal a view by mind.” 

“How\marginnote{59.1} many illegitimate kinds of receiving are there?” 

“There\marginnote{60.1} are five: (1) when someone gives by body and one doesn’t receive by body; (2) when someone gives by body and one doesn’t receive with something connected to the body; (3) when someone gives with something connected to the body and one doesn’t receive by body; (4) when someone gives with something connected to the body and one doesn’t receive with something connected to the body; (5) when someone gives by releasing and one doesn’t receive by body or with something connected to the body. 

But\marginnote{61.1} there are five legitimate kinds of receiving: (1) when someone gives by body and one receives by body; (2) when someone gives by body and one receives with something connected to the body; (3) when someone gives with something connected to the body and one receives by body; (4) when someone gives with something connected to the body and one receives with something connected to the body; (5) when someone gives by releasing and one receives by body or with something connected to the body.” 

“In\marginnote{62.1} how many ways is something considered ‘not left over’?” 

“In\marginnote{63.1} five ways: (1) the making it left over is done with food that’s unallowable; (2) it’s done with food that hasn’t been received; (3) it’s done with food that’s not held in hand; (4) it’s done by one who’s not within arm’s reach; (5) ‘I’ve had enough,’ hasn’t been said. 

And\marginnote{64.1} there are five aspects for something to be considered ‘left over’: (1) the making it left over is done with food that’s allowable; (2) it’s done with food that has been received; (3) it’s done with food that’s held in hand; (4) it’s done by one who’s within arm’s reach; (5) ‘I’ve had enough,’ has been said.” 

“How\marginnote{65.1} many aspects are there of refusing an invitation to eat more?” 

“There\marginnote{66.1} are five aspects: there is eating; there is cooked food; they stand within arm’s reach; there is an offering; there is a refusal.” 

“How\marginnote{67.1} many illegitimate ways are there of acting according to what has been admitted?” 

“There\marginnote{68.1} are five ways: (1) A monk has committed an offense entailing expulsion. When he’s accused of having committed such an offense, he admits to committing an offense entailing suspension. The Sangha deals with him for an offense entailing suspension. That acting according to what has been admitted is illegitimate. Again, a monk has committed an offense entailing expulsion. When he’s accused of having committed such an offense, he admits to committing an offense entailing confession … an offense entailing acknowledgment … an offense of wrong conduct. The Sangha deals with him for an offense of wrong conduct. That acting according to what has been admitted is illegitimate. 

(2)\marginnote{68.7} A monk has committed an offense entailing suspension … 

(3)\marginnote{68.8} A monk has committed an offense entailing confession … 

(4)\marginnote{68.9} A monk has committed an offense entailing acknowledgment … 

(5)\marginnote{68.10} A monk has committed an offense of wrong conduct. When he’s accused of having committed such an offense, he admits to committing an offense entailing expulsion. The Sangha deals with him for an offense entailing expulsion. That acting according to what has been admitted is illegitimate. Again, a monk has committed an offense of wrong conduct. When he’s accused of having committed such an offense, he admits to committing an offense entailing suspension … an offense entailing confession … an offense entailing acknowledgment. The Sangha deals with him for an offense entailing acknowledgment. That acting according to what has been admitted is illegitimate. 

There\marginnote{69.1} are five ways of legitimately acting according to what has been admitted: (1) A monk has committed an offense entailing expulsion. When he’s accused of having committed such an offense, he admits it. The Sangha deals with him for an offense entailing expulsion. That acting according to what has been admitted is legitimate. 

(2)\marginnote{69.4} A monk has committed an offense entailing suspension … 

(3)\marginnote{69.5} A monk has committed an offense entailing confession … 

(4)\marginnote{69.6} A monk has committed an offense entailing acknowledgment … 

(5)\marginnote{69.7} A monk has committed an offense of wrong conduct. When he’s accused of having committed such an offense, he admits it. The Sangha deals with him for an offense of wrong conduct. That acting according to what has been admitted is legitimate.” 

“Sir,\marginnote{70.1} what sort of monk is unqualified to get permission to correct someone?” 

“One\marginnote{71.1} who has five qualities: (1) he’s shameless; (2) he’s ignorant; (3) he’s not a regular monk; (4) he speaks to make someone disrobe, (5) not with the aim of clearing their offenses. 

But\marginnote{72.1} a monk who has five qualities is qualified to get permission to correct someone: (1) he has a sense of conscience; (2) he’s knowledgeable; (3) he’s a regular monk; (4) he speaks with the aim of clearing someone’s offense, (5) not to make them disrobe.” 

“What\marginnote{73.1} sort of monk should one not discuss the Monastic Law with?” 

“One\marginnote{74.1} who has five qualities: (1) he doesn’t know the actions that are the bases for offenses; (2) he doesn’t know the origin stories; (3) he doesn’t know the rules; (4) he doesn’t know the right order of words; (5) he doesn’t know the sequence of statements. 

But\marginnote{75.1} the Monastic Law may be discussed with a monk who has five qualities: (1) he knows the actions that are the bases for offenses; (2) he knows the origin stories; (3) he knows the rules; (4) he knows the right order of words; (5) he knows the sequence of statements.” 

“How\marginnote{76.1} many kinds of questions and enquiries are there?” 

“There\marginnote{77.1} are five: one asks (1) because of stupidity and folly; (2) because one is overcome by bad desires; (3) because of contempt; (4) because one desires to know; (5) because of the thought, ‘If he explains correctly when I ask him, all is well, but if he doesn’t, I’ll explain it correctly to him.’” 

“How\marginnote{78.1} many kinds of declaration of perfect insight are there?” 

“There\marginnote{79.1} are five: one declares perfect insight (1) because of stupidity and folly; (2) because one is overcome by bad desires; (3) because of insanity and derangement; (4) because of overestimation; (5) because it’s true.” 

“How\marginnote{80.1} many kinds of purification are there?” 

“There\marginnote{81.1} are five: (1) After reciting the introduction, the rest is announced as if heard. (2) After reciting the introduction and the four rules entailing expulsion, the rest is announced as if heard. (3) After reciting the introduction, the four rules entailing expulsion, and the thirteen rules entailing suspension, the rest is announced as if heard. (4) After reciting the introduction, the four rules entailing expulsion, the thirteen rules entailing suspension, and the two undetermined rules, the rest is announced as if heard. (5) In full is the fifth.”\footnote{This refers to the recitation of the Monastic Code, the \textsanskrit{Pātimokkha}. } 

“How\marginnote{82.1} many kinds of cooked food are there?” 

“There\marginnote{83.1} are five: cooked grain, porridge, flour products, fish, and meat.”\footnote{For an explanation of the rendering “flour products” for \textit{sattu}, see Appendix of Technical Terms. } 

\scendvagga{The fourth subchapter on revealing one’s view is finished. }

\scuddanaintro{This is the summary: }

\begin{scuddana}%
“Revealing\marginnote{86.1} one’s view, other, \\
Receiving, not left over; \\
Refusing an invitation to eat more, according to what has been admitted, \\
Permission, and with discussion; \\
Question, declarations of perfect insight, \\
And also purification, cooked food.” 

%
\end{scuddana}

\section*{5. The subchapter on raising an issue }

“Sir,\marginnote{87.1} how many qualities should a monk see in himself before accusing another?” 

“He\marginnote{88.1} should see five qualities in himself: (1) He should reflect: ‘Is my bodily conduct pure and flawless? Is this quality found in me or not?’ If it’s not, there will be those who say, ‘Please train your own bodily conduct first.’ 

(2)\marginnote{90.1} He should reflect: ‘Is my verbal conduct pure and flawless? Is this quality found in me or not?’ If it’s not, there will be those who say, ‘Please train your own verbal conduct first.’ 

(3)\marginnote{91.1} He should reflect: ‘Do I have a mind of good will toward my fellow monastics, a mind free from anger? Is this quality found in me or not?’ If it’s not, there will be those who say, ‘Please set up a mind of good will toward your fellow monastics first.’ 

(4)\marginnote{92.1} He should reflect: ‘Have I learned much and do I retain and accumulate what I’ve learned? Those teachings that are good in the beginning, good in the middle, and good in the end, that have a true goal and are well articulated, and that set out the perfectly complete and pure spiritual life—have I learned many such teachings, retained them in mind, recited them verbally, mentally investigated them, and penetrated them well by view? Is this quality found in me or not?’ If it’s not, there will be those who say, ‘Please learn the tradition first.’ 

(5)\marginnote{93.1} He should reflect: ‘Have I properly learned both Monastic Codes in detail; have I analyzed them well, thoroughly mastered them, and investigated them well, both in terms of the rules and their detailed exposition? Is this quality found in me or not?’ If it’s not, then when he’s asked, ‘Where was this said by the Buddha?’ he won’t be able to reply. And there will be those who say, ‘Please learn the Monastic Law first.’” 

“Sir,\marginnote{94.1} how many qualities should a monk set up in himself before accusing another?” 

“He\marginnote{95.1} should set up five qualities in himself: (1) ‘I’ll speak at an appropriate time, not at an inappropriate one; (2) I’ll speak the truth, not falsehood; (3) I’ll speak gently, not harshly; (4) I’ll speak what’s beneficial, not what’s unbeneficial; (5) I’ll speak with a mind of good will, not with ill will.’” 

“How\marginnote{96.1} many qualities should a monk attend to in himself before accusing another?” 

“He\marginnote{97.1} should attend to five qualities in himself: compassion, being of benefit, sympathy, the idea of clearing offenses, and the idea of prioritizing the training.” 

“What\marginnote{98.1} sort of monk is unqualified to get permission to correct someone?” 

“One\marginnote{99.1} who has five qualities: (1) he’s impure in bodily conduct; (2) he’s impure in verbal conduct; (3) he’s impure in livelihood; (4) he’s ignorant and incompetent; (5) he’s incapable of answering properly when questioned.” 

But\marginnote{100.1} a monk who has five qualities is qualified to get permission to correct someone: (1) he’s pure in bodily conduct; (2) he’s pure in verbal conduct; (3) he’s pure in livelihood; (4) he’s knowledgeable and competent; (5) he’s capable of answering properly when questioned.” 

“Sir,\marginnote{101.1} if a monk wishes to raise an issue, what factors should be fulfilled?” 

“Five\marginnote{102.1} factors should be fulfilled: He should reflect whether it’s the right time to raise it. If he knows it’s the wrong time, he shouldn’t raise it. 

(1)\marginnote{103.1} But if he knows it’s the right time, he should reflect further whether it’s a real issue. If he knows it’s not, he shouldn’t raise it. 

(2)\marginnote{104.1} But if he knows it is, he should reflect further whether raising the issue will be beneficial. If he knows it won’t, he shouldn’t raise it. 

(3)\marginnote{105.1} But if he knows it will, he should reflect further whether the monks who are on the side of the Teaching and the Monastic Law will support him. If he knows that they won’t, he shouldn’t raise it. 

(4)\marginnote{106.1} But if he knows that they will, he should reflect further whether raising the issue will lead to arguments and disputes, to fracture and schism in the Sangha. If he knows it will, he shouldn’t raise it. 

(5)\marginnote{107.1} But if he knows it won’t, he may raise it. In this way, when five factors are fulfilled, he won’t regret raising that issue.” 

“What\marginnote{108.1} sort of monk is of great help to monks involved in a legal issue?” 

“One\marginnote{109.1} who has five qualities: (1) He’s virtuous and restrained by the Monastic Code. His conduct is good, he associates with the right people, and he sees danger in minor faults. He undertakes and trains in the training rules. (2) He has learned much, and he retains and accumulates what he has learned. (3) Those teachings that are good in the beginning, good in the middle, and good in the end, that have a true goal and are well articulated, and that set out the perfectly complete and pure spiritual life—he has learned many such teachings, retained them in mind, recited them verbally, mentally investigated them, and penetrated them well by view. (4) He’s firmly committed to the Monastic Law. (5) He’s capable of making both sides relax, of persuading them, of convincing them, of making them see, of reconciling them. 

A\marginnote{110.1} monk who has five other qualities is also of great help to monks involved in a legal issue: he’s pure in bodily conduct; he’s pure in verbal conduct; he’s pure in livelihood; he’s knowledgeable and competent; he’s capable of answering properly when questioned. 

A\marginnote{111.1} monk who has five other qualities is also of great help to monks involved in a legal issue: he knows the actions that are the bases for offenses; he knows the origin stories; he knows the rules; he knows the right order of words; he knows the sequence of statements.” 

“Sir,\marginnote{112.1} what sort of monk shouldn’t be examined?” 

“One\marginnote{113.1} who has five qualities: (1) he doesn’t know the Monastic Code; (2) he doesn’t know what is in accordance with the Monastic Code; (3) he doesn’t know the Monastic Law; (4) he doesn’t know what’s in accordance with the Monastic Law; (5) he’s not skilled in what is and is not possible. 

But\marginnote{114.1} a monk who has five qualities may be examined: (1) he knows the Monastic Code; (2) he knows what’s in accordance with the Monastic Code; (3) he knows the Monastic Law; (4) he knows what’s in accordance with the Monastic Law; (5) he’s skilled in what is and is not possible. 

A\marginnote{115.1} monk who has five other qualities also shouldn’t be examined: (1) he doesn’t know the Teaching; (2) he doesn’t know what’s in accordance with the Teaching; (3) he doesn’t know the Monastic Law; (4) he doesn’t know what’s in accordance with the Monastic Law; (5) he’s not skilled in the right order. 

But\marginnote{116.1} a monk who has five qualities may be examined: (1) he knows the Teaching; (2) he knows what’s in accordance with the Teaching; (3) he knows the Monastic Law; (4) he knows what’s in accordance with the Monastic Law; (5) he’s skilled in the right order. 

A\marginnote{117.1} monk who has five other qualities also shouldn’t be examined: (1) he doesn’t know the actions that are the bases for offenses; (2) he doesn’t know the origin stories; (3) he doesn’t know the rules; (4) he doesn’t know the right order of words; (5) he doesn’t know the sequence of statements. 

But\marginnote{118.1} a monk who has five qualities may be examined: (1) he knows the actions that are the bases for offenses; (2) he knows the origin stories; (3) he knows the rules; (4) he knows the right order of words; (5) he knows the sequence of statements. 

A\marginnote{119.1} monk who has five other qualities also shouldn’t be examined: (1) he doesn’t know the offenses; (2) he doesn’t know the origination of the offenses; (3) he doesn’t know the kind of effort required to commit the offenses; (4) he doesn’t know the settling of offenses; (5) he’s not skilled in deciding on offenses. 

But\marginnote{120.1} a monk who has five qualities may be examined: (1) he knows the offenses; (2) he knows the origination of the offenses; (3) he knows the kind of effort required to commit the offenses; (4) he knows the settling of offenses; (5) he’s skilled in deciding on offenses. 

A\marginnote{121.1} monk who has five other qualities also shouldn’t be examined: (1) he doesn’t know the legal issues; (2) he doesn’t know the origination of the legal issues; (3) he doesn’t know the kind of effort that’s the source of the legal issues; (4) he doesn’t know the settling of legal issues; (5) he’s not skilled in deciding legal issues. 

But\marginnote{122.1} a monk who has five qualities may be examined: (1) he knows the legal issues; (2) he knows the origination of the legal issues; (3) he knows the kind of effort that’s the source of the legal issues; (4) he knows the settling of legal issues; (5) he’s skilled in deciding legal issues.” 

\scendvagga{The fifth subchapter on raising an issue is finished. }

\scuddanaintro{This is the summary: }

\begin{scuddana}%
“And\marginnote{125.1} pure, at an appropriate time, \\
Compassion, and with permission; \\
Raising an issue, legal issue, \\
And also other, and the actions that are the bases; \\
The Monastic Code, the Teaching, and the actions that are the bases, \\
Offense, and with legal issue.” 

%
\end{scuddana}

\section*{6. The subchapter on ascetic practices }

“Sir,\marginnote{126.1} how many kinds of wilderness dwellers are there?” 

“There\marginnote{126.2} are five kinds: those who are wilderness dwellers: (1) because of stupidity and folly; (2) because they are overcome by bad desires; (3) because of insanity and derangement; (4) because it is praised by the Buddhas and their disciples; (5) because of fewness of wishes, contentment, self-effacement, seclusion, and not needing anything else.”\footnote{Sp 5.325: \textit{\textsanskrit{Taṁ} \textsanskrit{idamatthitaṁyeva} \textsanskrit{nissāya} na \textsanskrit{aññaṁ} \textsanskrit{kiñci} \textsanskrit{lokāmisanti} attho}, “They depend on ‘not needing anything else’, not on any other material things. This is the meaning.” } 

“How\marginnote{127.1} many kinds of people are there who only eat almsfood?” … “How many kinds of rag-robe wearers are there?” … “How many kinds of people are there who live at the foot of a tree?” … “How many kinds of people are there who live in charnel grounds?” … “How many kinds of people are there who live out in the open?” … “How many kinds of people are there who only have three robes?” … “How many kinds of people are there who go on continuous almsround?” … “How many kinds of people are there who never lie down?” … “How many kinds of people are there who accept any kind of resting place?” … “How many kinds of people are there who eat in one sitting per day?” … “How many kinds of people are there who refuse to accept food offered after the meal has begun?” … “How many kinds of people are there who eat only from the almsbowl?” 

“There\marginnote{127.13} are five kinds: those who eat only from the almsbowl: (1) because of stupidity and folly; (2) because they are overcome by bad desires; (3) because of insanity and derangement; (4) because it is praised by the Buddhas and their disciples; (5) because of fewness of wishes, contentment, self-effacement, seclusion, and not needing anything else.” 

\scendvagga{The sixth subchapter on ascetic practices is finished. }

\scuddanaintro{This is the summary: }

\begin{scuddana}%
“Wilderness\marginnote{130.1} dweller, almsfood, rag-robe, \\
Tree, charnel ground is the fifth; \\
Out in the open, and the three robes, \\
Continuous, those who never lie down; \\
Resting place, and one sitting, \\
After, those who eat only from the bowl.” 

%
\end{scuddana}

\section*{7. The subchapter on lying }

“Sir,\marginnote{131.1} how many kinds of lying are there?” 

“There\marginnote{131.2} are five kinds: (1) there’s lying that leads to an offense entailing expulsion; (2) there’s lying that leads to an offense entailing suspension; (3) there’s lying that leads to a serious offense; (4) there’s lying that leads to an offense entailing confession; (5) there’s lying that leads to an offense of wrong conduct.” 

“A\marginnote{132.1} monk may be canceling someone’s observance day or invitation in the midst of the Sangha. Among such monks, what sort should be pressed: ‘Enough. No more arguing and disputing,’ with the Sangha then doing the observance-day ceremony or the invitation ceremony?” 

“A\marginnote{133.1} monk who has five qualities: he’s shameless; he’s ignorant; he’s not a regular monk; he speaks to make someone disrobe, not with the aim of clearing their offenses. 

The\marginnote{134.1} same procedure should be followed also for a monk who has five other qualities: he’s impure in bodily conduct; he’s impure in verbal conduct; he’s impure in livelihood; he’s ignorant and incompetent; he’s quarrelsome and argumentative.” 

“What\marginnote{135.1} sort of monk shouldn’t be allowed to question?” 

“One\marginnote{136.1} who has five qualities: (1) he doesn’t know the offenses and non-offenses; (2) he doesn’t know the light and heavy offenses; (3) he doesn’t know the curable and incurable offenses; (4) he doesn’t know the grave and minor offenses; (5) he doesn’t know the offenses that are clearable by making amends and the offenses that are not clearable by making amends. 

But\marginnote{137.1} a monk who has five other qualities may question: (1) he knows the offenses and non-offenses; (2) he knows the light and heavy offenses; (3) he knows the curable and incurable offenses; (4) he knows the grave and minor offenses; (5) he knows the offenses that are clearable by making amends and the offenses that are not clearable by making amends.” 

“For\marginnote{138.1} how many reasons does a monk commit an offense?” 

“For\marginnote{138.2} five reasons: because of shamelessness; because of ignorance; because of being overcome by anxiety; because of perceiving what’s unallowable as allowable; because of perceiving what’s allowable as unallowable. 

A\marginnote{139.1} monk also commits an offense for five other reasons: because of not seeing; because of not hearing; because of sleeping; because of perceiving it as allowable; because of absentmindedness.”\footnote{Sp 5.447: \textit{\textsanskrit{Adassanenāti} vinayadharassa adassanena. … \textsanskrit{Assavanenāti} \textsanskrit{ekavihārepi} vasanto pana vinayadharassa \textsanskrit{upaṭṭhānaṁ} \textsanskrit{gantvā} \textsanskrit{kappiyākappiyaṁ} \textsanskrit{apucchitvā} \textsanskrit{vā} \textsanskrit{aññesañca} \textsanskrit{vuccamānaṁ} \textsanskrit{asuṇanto} \textsanskrit{āpajjatiyeva}, tena \textsanskrit{vuttaṁ} “\textsanskrit{assavanenā}”ti. \textsanskrit{Pasuttakatāti} \textsanskrit{pasuttakatāya}. \textsanskrit{Sahagāraseyyañhi} \textsanskrit{pasuttakabhāvenapi} \textsanskrit{āpajjati}}, “‘Because of not seeing’: because of not seeing an expert on the Monastic Law. … ‘Because of not hearing’: when living in a monastery, one goes to attend on an expert in the Monastic Law, then, having asked and being spoken to about something else, one ends up not hearing about it. It is because of that that it is said ‘because of not hearing’. ‘Because of sleeping’: due to sleeping. One commits the offense while sleeping in the same sleeping place in a house.” The offenses committed ‘because of sleeping’ refers to \href{https://suttacentral.net/pli-tv-bu-vb-pc5/en/brahmali\#2.16.1}{Bu Pc 5:2.16.1} and \href{https://suttacentral.net/pli-tv-bu-vb-pc6/en/brahmali\#1.51.1}{Bu Pc 6:1.51.1}, and the corresponding rules for \textit{\textsanskrit{bhikkhunīs}}. } 

“How\marginnote{140.1} many kinds of hostility are there?” 

“There\marginnote{140.2} are five: killing living beings; stealing; sexual misconduct; lying; alcohol, which causes heedlessness.” 

“How\marginnote{141.1} many kinds of abstention are there?” 

“There\marginnote{141.2} are five: abstention from killing living beings; from stealing; from sexual misconduct; from lying; from alcohol, which causes heedlessness.” 

“How\marginnote{142.1} many kinds of loss are there?” 

“There\marginnote{142.2} are five: loss of relatives, property, health, morality, and view.” 

“How\marginnote{143.1} many kinds of success are there?” 

“There\marginnote{143.2} are five: success in relatives, property, health, morality, and view.” 

\scendvagga{The seventh subchapter on lying is finished. }

\scuddanaintro{This is the summary: }

\begin{scuddana}%
“And\marginnote{146.1} lying, pressed, \\
Other, question; \\
And offense, other, \\
Hostility, and abstention; \\
Loss, and success—\\
The compilation of the seventh subchapter is finished.” 

%
\end{scuddana}

\section*{8. The subchapter on instructing the nuns }

“Sir,\marginnote{147.1} against what sort of monk should the Sangha of nuns do a legal procedure?” 

“They\marginnote{148.1} should do a legal procedure, prohibiting the Sangha of nuns from paying respect to him, against a monk who has five qualities: (1) he exposes his body to the nuns; (2) he exposes his thighs to the nuns; (3) he exposes his genitals to the nuns; (4) he exposes both shoulders to the nuns; (5) he speaks indecently to the nuns; he associates inappropriately with householders.\footnote{It is not clear why this is considered five rather than six items. Perhaps the last item on householders is not to be counted. The same list recurs at \href{https://suttacentral.net/pli-tv-kd20/en/brahmali\#9.1.18}{Kd 20:9.1.18}, but without “he exposes both shoulders to the nuns” and without any reference to householders. The exposition at Kd 20 seems more plausible. } 

They\marginnote{149.1} should do the same legal procedure also against a monk who has five other qualities: (1) he’s trying to stop nuns from getting material support; (2) he’s trying to harm nuns; (3) he’s trying to get nuns to lose their place of residence; (4) he abuses and reviles nuns; (5) he causes division between the monks and the nuns. 

They\marginnote{150.1} should do the same legal procedure also against a monk who has five other qualities: (1) he’s trying to stop nuns from getting material support; (2) he’s trying to harm nuns; (3) he’s trying to get nuns to lose their place of residence; (4) he abuses and reviles nuns; (5) he causes the monks to associate inappropriately with the nuns.” 

“Against\marginnote{151.1} what sort of nun should a legal procedure be done?” 

“One\marginnote{152.1} who has five qualities: (1) she exposes her body to the monks; (2) she exposes her thighs to the monks; (3) she exposes her genitals to the monks; (4) she exposes both shoulders to the monks; (5) she speaks indecently to the monks; she associates inappropriately with householders.\footnote{Again, it is not clear why this is counted as five rather than six items. See above. } 

A\marginnote{153.1} legal procedure should be done also against a nun who has five other qualities: (1) she’s trying to stop monks from getting material support; (2) she’s trying to harm monks; (3) she’s trying to get monks to lose their place of residence; (4) she abuses and reviles monks; (5) she causes division between the nuns and the monks. 

A\marginnote{154.1} legal procedure should be done also against a nun who has five other qualities: (1) she’s trying to stop monks from getting material support; (2) she’s trying to harm monks; (3) she’s trying to get monks to lose their place of residence; (4) she abuses and reviles monks; (5) she causes the nuns to associate inappropriately with the monks.” 

“What\marginnote{155.1} sort of monk shouldn’t cancel the nuns’ instruction?” 

“One\marginnote{156.1} who has five qualities: he’s shameless; he’s ignorant; he’s not a regular monk; he speaks to make someone disrobe, not with the aim of clearing their offenses. 

A\marginnote{157.1} monk who has five other qualities also shouldn’t cancel the nuns’ instruction: he’s impure in bodily conduct; he’s impure in verbal conduct; he’s impure in livelihood; he’s ignorant and incompetent; he’s incapable of answering properly when questioned. 

A\marginnote{158.1} monk who has five other qualities also shouldn’t cancel the nuns’ instruction: he’s improperly behaved by body; he’s improperly behaved by speech; he’s improperly behaved by body and speech; he abuses and reviles nuns; he socializes improperly with the nuns. 

A\marginnote{159.1} monk who has five other qualities also shouldn’t cancel the nuns’ instruction: he’s shameless, ignorant, and not a regular monk, and he’s quarrelsome and argumentative, and he doesn’t fulfill the training.” 

“What\marginnote{160.1} sort of monk shouldn’t agree to instruct the nuns?” 

“One\marginnote{161.1} who has five qualities: he’s improperly behaved by body; he’s improperly behaved by speech; he’s improperly behaved by body and speech; he abuses and reviles nuns; he socializes improperly with the nuns. 

A\marginnote{162.1} monk who has five other qualities also shouldn’t agree to instruct the nuns: he’s shameless, ignorant, and not a regular monk, or he’s about to depart, or he’s sick.” 

“What\marginnote{163.1} sort of monk should one not have a discussion with?” 

“One\marginnote{164.1} who has five qualities: he doesn’t have the virtue, stillness, wisdom, freedom, or knowledge and vision of freedom of one who’s fully trained. But one may have a discussion with a monk who has five qualities: he has the virtue, stillness, wisdom, freedom, and knowledge and vision of freedom of one who’s fully trained. 

One\marginnote{165.1} also shouldn’t have a discussion with a monk who has five other qualities: he hasn’t achieved the analysis of meaning, the analysis of text, the analysis of terminology, and the analysis of articulation, and he hasn’t reviewed the extent of his mind’s freedom. But one may have a discussion with a monk who has five qualities: he has achieved the analysis of meaning, the analysis of text, the analysis of terminology, and the analysis of articulation, and he reviews the extent of his mind’s freedom.” 

\scendvagga{The eighth subchapter on instructing the nuns is finished. }

\scuddanaintro{This is the summary: }

\begin{scuddana}%
“The\marginnote{168.1} nuns should do, \\
And another two of the same; \\
Three on legal procedures against nuns, \\
Twice two on shouldn’t cancel; \\
Two were spoken on shouldn’t agree, \\
And twice two on discussions.” 

%
\end{scuddana}

\section*{9. The subchapter on committees }

“What\marginnote{169.1} sort of monk shouldn’t be appointed to a committee?” 

“One\marginnote{170.1} who has five qualities: he’s not skilled in the meaning, the Teaching, the terminology, the wording, or the right order. But a monk who has five qualities may be appointed to a committee: he’s skilled in the meaning, the Teaching, the terminology, the wording, and the right order. 

A\marginnote{171.1} monk who has five other qualities also shouldn’t be appointed to a committee: (1) he’s angry, overcome by anger; (2) he’s denigrating, overcome by denigration; (3) he’s domineering, overcome by being domineering; (4) he’s envious, overcome by envy; (5) he obstinately grasps his own views and only gives them up with difficulty. But a monk who has five qualities may be appointed to a committee: (1) he’s not angry or overcome by anger; (2) he’s not denigrating or overcome by denigration; (3) he’s not domineering or overcome by being domineering; (4) he’s not envious or overcome by envy; (5) he doesn’t obstinately grasp his own views and gives them up with ease. 

A\marginnote{172.1} monk who has five other qualities also shouldn’t be appointed to a committee: he (1) gets agitated, (2) has ill will, (3) becomes hardhearted, (4) gives rise to anger, and (5) is resistant and doesn’t receive instructions respectfully. But a monk who has five qualities may be appointed to a committee: he (1) doesn’t get agitated, (2) doesn’t have ill will, (3) doesn’t become hardhearted, (4) doesn’t give rise to anger, and (5) isn’t resistant but receives instructions respectfully. 

A\marginnote{173.1} monk who has five other qualities also shouldn’t be appointed to a committee: (1) he causes confusion, not recollection; (2) he speaks without having gotten permission; (3) he doesn’t accuse others according to the Teaching, the Monastic Law, or their offense; (4) he doesn’t act according to the Teaching, the Monastic Law, or his offenses; (5) he doesn’t explain things according to his own view.\footnote{Sp 5.455: \textit{\textsanskrit{Pasāretā} hoti no \textsanskrit{sāretāti} \textsanskrit{mohetā} hoti, na \textsanskrit{satiuppādetā}; \textsanskrit{codakacuditakānaṁ} \textsanskrit{kathaṁ} moheti pidahati na \textsanskrit{sāretīti} attho}, “‘He causes confusion, not recollection’: he confuses, does not cause others to remember; the meaning is that, in regard to the speech of the accuser and the one who is accused, he confuses, conceals, and does not cause them to remember.” } But a monk who has five qualities may be appointed to a committee: (1) he causes recollection, not confusion; (2) he speaks after getting permission; (3) he accuses others according to the Teaching, the Monastic Law, and their offense; (4) he acts according to the Teaching, the Monastic Law, and his offenses; (5) he explains things according to his own view. 

A\marginnote{174.1} monk who has five other qualities also shouldn’t be appointed to a committee: he’s biased by desire, ill will, confusion, or fear, and he’s shameless. But a monk who has five qualities may be appointed to a committee: he’s not biased by desire, ill will, confusion, or fear, and he has a sense of conscience. 

A\marginnote{175.1} monk who has five other qualities also shouldn’t be appointed to a committee: he’s biased by desire, ill will, confusion, or fear, and he’s unskilled in the Monastic Law. But a monk who has five qualities may be appointed to a committee: he’s not biased by desire, ill will, confusion, or fear, and he’s skilled in the Monastic Law.” 

“What\marginnote{176.1} sort of monk is considered ignorant?” 

“One\marginnote{177.1} who has five qualities: (1) he doesn’t know the Monastic Code; (2) he doesn’t know what’s in accordance with the Monastic Code; (3) he doesn’t know the Monastic Law; (4) he doesn’t know what’s in accordance with the Monastic Law; (5) he’s not skilled in what is and is not possible. But a monk who has five qualities is considered learned: (1) he knows the Monastic Code; (2) he knows what’s in accordance with the Monastic Code; (3) he knows the Monastic Law; (4) he knows what’s in accordance with the Monastic Law; (5) he’s skilled in what is and isn’t possible. 

A\marginnote{178.1} monk who has five other qualities is also considered ignorant: (1) he doesn’t know the Teaching; (2) he doesn’t know what’s in accordance with the Teaching; (3) he doesn’t know the Monastic Law; (4) he doesn’t know what’s in accordance with the Monastic Law; (5) he’s not skilled in the right order. But a monk who has five qualities is considered learned: (1) he knows the Teaching; (2) he knows what’s in accordance with the Teaching; (3) he knows the Monastic Law; (4) he knows what’s in accordance with the Monastic Law; (5) he’s skilled in the right order. 

A\marginnote{179.1} monk who has five other qualities is also considered ignorant: (1) he doesn’t know the actions that are the bases for offenses; (2) he doesn’t know the origin stories; (3) he doesn’t know the rules; (4) he doesn’t know the right order of words; (5) he doesn’t know the sequence of statements. But a monk who has five qualities is considered learned: (1) he knows the actions that are the bases for offenses; (2) he knows the origin stories; (3) he knows the rules; (4) he knows the right order of words; (5) he knows the sequence of statements. 

A\marginnote{180.1} monk who has five other qualities is also considered ignorant: (1) he doesn’t know the offenses; (2) he doesn’t know the origination of the offenses; (3) he doesn’t know the kind of effort required to commit the offenses; (4) he doesn’t know the settling of offenses; (5) he’s not skilled in deciding on offenses. But a monk who has five qualities is considered learned: (1) he knows the offenses; (2) he knows the origination of the offenses; (3) he knows the kind of effort required to commit the offenses; (4) he knows the settling of offenses; (5) he’s skilled in deciding on offenses. 

A\marginnote{181.1} monk who has five other qualities is also considered ignorant: (1) he doesn’t know the legal issues; (2) he doesn’t know the origination of the legal issues; (3) he doesn’t know the kind of effort that’s the source of the legal issues; (4) he doesn’t know the settling of legal issues; (5) he’s not skilled in deciding legal issues. But a monk who has five qualities is considered learned: (1) he knows the legal issues; (2) he knows the origination of the legal issues; (3) he knows the kind of effort that is the source of the legal issues; (4) he knows the settling of legal issues; (5) he’s skilled in deciding legal issues.” 

\scendvagga{The ninth subchapter on committees is finished. }

\scuddanaintro{This is the summary: }

\begin{scuddana}%
“And\marginnote{184.1} not skilled in the meaning, \\
Angry, and one who is agitated; \\
One who confuses, biased by desire, \\
And so unskilled. 

The\marginnote{185.1} Monastic Code, and the Teaching, and the actions that are the bases, \\
Offense, legal issue—\\
All proclaimed in groups of two: \\
You should understand the dark and the bright.” 

%
\end{scuddana}

\section*{10. The subchapter on the resolving of legal issues }

“What\marginnote{186.1} sort of monk is unqualified to resolve a legal issue?” 

“One\marginnote{187.1} who has five qualities: (1) he doesn’t know the offenses; (2) he doesn’t know the origination of the offenses; (3) he doesn’t know the kind of effort required to commit the offenses; (4) he doesn’t know the settling of offenses; (5) he’s not skilled in deciding on offenses. But monk who has five qualities is qualified to resolve a legal issue: (1) he knows the offenses; (2) he knows the origination of the offenses; (3) he knows the kind of effort required to commit the offenses; (4) he knows the settling of offenses; (5) he’s skilled in deciding on offenses. 

A\marginnote{188.1} monk who has five other qualities is also unqualified to resolve a legal issue: (1) he doesn’t know the legal issues; (2) he doesn’t know the origination of the legal issues; (3) he doesn’t know the kind of effort that’s the source of the legal issues; (4) he doesn’t know the settling of legal issues; (5) he’s not skilled in deciding legal issues. 

But\marginnote{189.1} a monk who has five qualities is qualified to resolve a legal issue: (1) he knows the legal issues; (2) he knows the origination of the legal issues; (3) he knows the kind of effort that’s the source of the legal issues; (4) he knows the settling of legal issues; (5) he’s skilled in deciding legal issues. 

A\marginnote{190.1} monk who has five other qualities is also unqualified to resolve a legal issue: he’s biased by desire, ill will, confusion, or fear, and he’s shameless. But a monk who has five qualities is qualified to resolve a legal issue: he’s not biased by desire, ill will, confusion, or fear, and he has a sense of conscience. 

A\marginnote{191.1} monk who has five other qualities is also unqualified to resolve a legal issue: he’s biased by desire, ill will, confusion, or fear, and he’s ignorant. But a monk who has five qualities is qualified to resolve a legal issue: he’s not biased by desire, ill will, confusion, or fear, and he’s learned. 

A\marginnote{192.1} monk who has five other qualities is also unqualified to resolve a legal issue: (1) he doesn’t know the actions that are the bases for offenses; (2) he doesn’t know the origin stories; (3) he doesn’t know the rules; (4) he doesn’t know the right order of words; (5) he doesn’t know the sequence of statements. But a monk who has five qualities is qualified to resolve a legal issue: (1) he knows the actions that are the bases for offenses; (2) he knows the origin stories; (3) he knows the rules; (4) he knows the right order of words; (5) he knows the sequence of statements. 

A\marginnote{193.1} monk who has five other qualities is also unqualified to resolve a legal issue: he’s biased by desire, ill will, confusion, or fear, and he’s unskilled in the Monastic Law. But a monk who has five qualities is qualified to resolve a legal issue: he’s not biased by desire, ill will, confusion, or fear, and he’s skilled in the Monastic Law. 

A\marginnote{194.1} monk who has five other qualities is also unqualified to resolve a legal issue: he’s biased by desire, ill will, confusion, or fear, and he respects individuals, not the Sangha. But a monk who has five qualities is qualified to resolve a legal issue: he’s not biased by desire, ill will, confusion, or fear, and he respects the Sangha, not individuals.\footnote{Sp 5.457: \textit{Puggalagaru \textsanskrit{hotīti} “\textsanskrit{ayaṁ} me \textsanskrit{upajjhāyo}, \textsanskrit{ayaṁ} me \textsanskrit{ācariyo}”\textsanskrit{tiādīni} \textsanskrit{cintetvā} tassa \textsanskrit{jayaṁ} \textsanskrit{ākaṅkhamāno} “\textsanskrit{adhammaṁ} dhammo”ti \textsanskrit{dīpeti}. \textsanskrit{Saṅghagaru} \textsanskrit{hotīti} \textsanskrit{dhammañca} \textsanskrit{vinayañca} \textsanskrit{amuñcitvā} vinicchinanto \textsanskrit{saṅghagaruko} \textsanskrit{nāma} hoti}, “‘He respects individuals’: thinking, ‘This is my preceptor’ or ‘This is my teacher’, etc., and desiring him to win, he proclaims what is contrary to the Teaching as being in accordance with it. ‘He respects the Sangha’: making decisions without letting go of the Teaching or the Monastic Law, he is called one who respects the Sangha.” } 

A\marginnote{195.1} monk who has five other qualities is also unqualified to resolve a legal issue: he’s biased by desire, ill will, confusion, or fear; and he values worldly things, not the true Teaching. But a monk who has five qualities is qualified to resolve a legal issue: he’s not biased by desire, ill will, confusion, or fear; and he values the true Teaching, not worldly things.” 

“Sir,\marginnote{196.1} in how many ways is there schism in the Sangha?” 

“In\marginnote{196.2} five ways, \textsanskrit{Upāli}: through a legal procedure, through recitation, through speaking, through a proclamation, and through voting.”\footnote{Sp 5.458: \textit{\textsanskrit{Uddesenāti} \textsanskrit{pañcasu} \textsanskrit{pātimokkhuddesesu} \textsanskrit{aññatarena} uddesena. Voharantoti kathayanto; \textsanskrit{tāhi} \textsanskrit{tāhi} \textsanskrit{upapattīhi} “\textsanskrit{adhammaṁ} dhammo”\textsanskrit{tiādīni} \textsanskrit{aṭṭhārasa} \textsanskrit{bhedakaravatthūni} \textsanskrit{dīpento}}, “‘Through recitation’: through reciting any of the five recitations of the Monastic Code. ‘Through speaking’: talking; proclaiming the eighteen bases for schism, that is, what is contrary to the Teaching as the Teaching, etc., however it arises.” } 

“Sir,\marginnote{197.1} we speak of ‘fracture in the Sangha’. But how is there fracture in the Sangha, yet not schism? And how is there both fracture and schism in the Sangha?” 

“(1)\marginnote{197.4} \textsanskrit{Upāli}, I’ve laid down the proper conduct for newly-arrived monks. Even though I’ve carefully laid down the training rules, the newly-arrived monks don’t practice that proper conduct. In this way, there’s fracture in the Sangha, but not schism. 

(2)\marginnote{197.7} I’ve laid down the proper conduct for resident monks. Even though I’ve carefully laid down the training rules, the resident monks don’t practice that proper conduct. In this way, there’s fracture in the Sangha, but not schism. 

(3)\marginnote{197.10} I’ve laid down the proper conduct for monks in the dining hall: the best seat, the best water, and the best almsfood is to be given out according to seniority and according to what’s proper. Even though I’ve carefully laid down the training rules, the junior monks block the senior monks from seats. In this way, there’s fracture in the Sangha, but not schism. 

(4)\marginnote{197.14} I’ve laid down the proper conduct for the monks in regard to dwellings: they’re to be given out according to seniority and according to what’s proper. Even though I’ve carefully laid down the training rules, the junior monks block the senior monks from dwellings. In this way, there’s fracture in the Sangha, but not schism. 

(5)\marginnote{197.18} For monks within the same monastery zone, I’ve laid down this: a joint observance-day ceremony; a joint invitation ceremony; joint legal procedures of the Sangha; joint legal procedures of whatever kind. Even though I’ve carefully laid down the training rules, they form a faction, a subgroup, right there within the monastery zone. They then do a separate observance-day ceremony, a separate invitation ceremony, separate legal procedures of the Sangha, or separate legal procedures of whatever kind. In this way, there’s both fracture and schism in the Sangha.” 

\scendvagga{The tenth subchapter on the resolving of legal issues is finished. }

\scuddanaintro{This is the summary: }

\begin{scuddana}%
“Offenses,\marginnote{200.1} legal issues, \\
Desire, and with ignorant; \\
And the actions that are the bases, and unskilled, \\
Individual, and with worldly things; \\
Schism, and fracture in the Sangha, \\
And so too schism in the Sangha.” 

%
\end{scuddana}

\section*{11. The subchapter on schism in the Sangha }

“Sir,\marginnote{201.1} what sort of monk who has caused a schism in the Sangha is irredeemably destined to an eon in hell?” 

“One\marginnote{202.1} who has five qualities, \textsanskrit{Upāli}: (1) a monk proclaims what’s contrary to the Teaching as being in accordance with it, (2) what’s in accordance with the Teaching as contrary to it, (3) what’s contrary to the Monastic Law as being in accordance with it, (4) what’s in accordance with the Monastic Law as contrary to it, and (5) he misrepresents his view of what’s true during the legal procedure.\footnote{Sp 5.458: \textit{\textsanskrit{Vinidhāya} \textsanskrit{diṭṭhiṁ} \textsanskrit{kammenāti} tesu \textsanskrit{adhammādīsu} \textsanskrit{adhammādayo} eteti \textsanskrit{evaṁdiṭṭhikova} \textsanskrit{hutvā} \textsanskrit{taṁ} \textsanskrit{diṭṭhiṁ} \textsanskrit{vinidhāya} te \textsanskrit{dhammādivasena} \textsanskrit{dīpetvā} \textsanskrit{visuṁ} \textsanskrit{kammaṁ} karoti}, “‘He misrepresents his view of what is true during the legal procedure’: in regard to what is not the Teaching, etc., he has the view that it is not the Teaching, etc., but he misrepresents that view, proclaiming it is the Teaching, etc., and he then does a separate legal procedure.” The separate legal procedure is what finalizes the schism. } 

When\marginnote{203.1} one who has caused a schism in the Sangha has five other qualities, he’s also irredeemably destined to an eon in hell: (1) a monk proclaims what’s contrary to the Teaching as being in accordance with it, (2) what’s in accordance with the Teaching as contrary to it, (3) what’s contrary to the Monastic Law as being in accordance with it, (4) what’s in accordance with the Monastic Law as contrary to it, and (5) he misrepresents his view of what’s true during the recitation.\footnote{That is, the recitation that finalizes the schism. And the same below. } 

When\marginnote{204.1} one who has caused a schism in the Sangha has five other qualities, he’s also irredeemably destined to an eon in hell: (1) a monk proclaims what’s contrary to the Teaching as being in accordance with it, (2) what’s in accordance with the Teaching as contrary to it, (3) what’s contrary to the Monastic Law as being in accordance with it, (4) what’s in accordance with the Monastic Law as contrary to it, and (5) he misrepresents his view of what’s true while he speaks. 

When\marginnote{205.1} one who has caused a schism in the Sangha has five other qualities, he’s also irredeemably destined to an eon in hell: (1) a monk proclaims what’s contrary to the Teaching as being in accordance with it, (2) what’s in accordance with the Teaching as contrary to it, (3) what’s contrary to the Monastic Law as being in accordance with it, (4) what’s in accordance with the Monastic Law as contrary to it, and (5) he misrepresents his view of what’s true during the proclamation. 

When\marginnote{206.1} one who has caused a schism in the Sangha has five other qualities, he’s also irredeemably destined to an eon in hell: (1) a monk proclaims what’s contrary to the Teaching as being in accordance with it, (2) what’s in accordance with the Teaching as contrary to it, (3) what’s contrary to the Monastic Law as being in accordance with it, (4) what’s in accordance with the Monastic Law as contrary to it, and (5) he misrepresents his view of what’s true during the voting. 

When\marginnote{207.1} one who has caused a schism in the Sangha has five other qualities, he’s also irredeemably destined to an eon in hell: (1) a monk proclaims what’s contrary to the Teaching as being in accordance with it, (2) what’s in accordance with the Teaching as contrary to it, (3) what’s contrary to the Monastic Law as being in accordance with it, (4) what’s in accordance with the Monastic Law as contrary to it, and (5) he misrepresents his belief of what’s true during the legal procedure. 

…\marginnote{207.4} (5) he misrepresents his belief of what’s true during the recitation. 

…\marginnote{207.5} (5) he misrepresents his belief of what’s true while he speaks. 

…\marginnote{207.6} (5) he misrepresents his belief of what’s true during the proclamation. 

…\marginnote{207.7} (5) he misrepresents his belief of what’s true during the voting. 

When\marginnote{208.1} one who has caused a schism in the Sangha has five other qualities, he’s also irredeemably destined to an eon in hell: (1) a monk proclaims what’s contrary to the Teaching as being in accordance with it, (2) what’s in accordance with the Teaching as contrary to it, (3) what’s contrary to the Monastic Law as being in accordance with it, (4) what’s in accordance with the Monastic Law as contrary to it, and (5) he misrepresents his acceptance of what’s true during the legal procedure. 

…\marginnote{208.4} (5) he misrepresents his acceptance of what’s true during the recitation. 

…\marginnote{208.5} (5) he misrepresents his acceptance of what’s true while he speaks. 

…\marginnote{208.6} (5) he misrepresents his acceptance of what’s true during the proclamation. 

…\marginnote{208.7} (5) he misrepresents his acceptance of what’s true during the voting. 

When\marginnote{209.1} one who has caused a schism in the Sangha has five other qualities, he’s also irredeemably destined to an eon in hell: (1) a monk proclaims what’s contrary to the Teaching as being in accordance with it, (2) what’s in accordance with the Teaching as contrary to it, (3) what’s contrary to the Monastic Law as being in accordance with it, (4) what’s in accordance with the Monastic Law as contrary to it, and (5) he misrepresents his perception of what’s true during the legal procedure. 

…\marginnote{209.4} (5) he misrepresents his perception of what’s true during the recitation. 

…\marginnote{209.5} (5) he misrepresents his perception of what’s true while he speaks. 

…\marginnote{209.6} (5) he misrepresents his perception of what’s true during the proclamation. 

…\marginnote{209.7} (5) he misrepresents his perception of what’s true during the voting.” 

\scendvagga{The eleventh subchapter on schism in the Sangha is finished. }

\scuddanaintro{This is the summary: }

\begin{scuddana}%
“He\marginnote{212.1} misrepresents his view during the legal procedure, \\
During the recitation, and while he speaks; \\
During the proclamation, during the voting—\\
These five are dependent on view; \\
Belief, and acceptance, and perception—\\
These three by the fivefold method.” 

%
\end{scuddana}

\section*{12. The second subchapter on schism in the Sangha }

“Sir,\marginnote{213.1} what sort of monk who has caused a schism in the Sangha is redeemable, not destined to an eon in hell?” 

“One\marginnote{214.1} who has five qualities, \textsanskrit{Upāli}: (1) a monk proclaims what’s contrary to the Teaching as being in accordance with it, (2) what’s in accordance with the Teaching as contrary to it, (3) what’s contrary to the Monastic Law as being in accordance with it, (4) what’s in accordance with the Monastic Law as contrary to it, but (5) he doesn’t misrepresent his view of what’s true during the legal procedure. 

When\marginnote{215.1} one who has caused a schism in the Sangha has five other qualities, he’s also redeemable, not destined to an eon in hell: (1) a monk proclaims what’s contrary to the Teaching as being in accordance with it, (2) what’s in accordance with the Teaching as contrary to it, (3) what’s contrary to the Monastic Law as being in accordance with it, (4) what’s in accordance with the Monastic Law as contrary to it, but (5) he doesn’t misrepresent his view of what’s true during the recitation. 

When\marginnote{216.1} one who has caused a schism in the Sangha has five other qualities, he’s also redeemable, not destined to an eon in hell: (1) a monk proclaims what’s contrary to the Teaching as being in accordance with it, (2) what’s in accordance with the Teaching as contrary to it, (3) what’s contrary to the Monastic Law as being in accordance with it, (4) what’s in accordance with the Monastic Law as contrary to it, but (5) he doesn’t misrepresent his view of what’s true while he speaks. 

When\marginnote{217.1} one who has caused a schism in the Sangha has five other qualities, he’s also redeemable, not destined to an eon in hell: (1) a monk proclaims what’s contrary to the Teaching as being in accordance with it, (2) what’s in accordance with the Teaching as contrary to it, (3) what’s contrary to the Monastic Law as being in accordance with it, (4) what’s in accordance with the Monastic Law as contrary to it, but (5) he doesn’t misrepresent his view of what’s true during the proclamation. 

When\marginnote{218.1} one who has caused a schism in the Sangha has five other qualities, he’s also redeemable, not destined to an eon in hell: (1) a monk proclaims what’s contrary to the Teaching as being in accordance with it, (2) what’s in accordance with the Teaching as contrary to it, (3) what’s contrary to the Monastic Law as being in accordance with it, (4) what’s in accordance with the Monastic Law as contrary to it, but (5) he doesn’t misrepresent his view of what’s true during the voting. 

When\marginnote{219.1} one who has caused a schism in the Sangha has five other qualities, he’s also redeemable, not destined to an eon in hell: (1) a monk proclaims what’s contrary to the Teaching as being in accordance with it, (2) what’s in accordance with the Teaching as contrary to it, (3) what’s contrary to the Monastic Law as being in accordance with it, (4) what’s in accordance with the Monastic Law as contrary to it, but (5) he doesn’t misrepresent his belief of what’s true during the legal procedure. 

…\marginnote{219.4} (5) he doesn’t misrepresent his belief of what’s true during the recitation. 

…\marginnote{219.5} (5) he doesn’t misrepresent his belief of what’s true while he speaks. 

…\marginnote{219.6} (5) he doesn’t misrepresent his belief of what’s true during the proclamation. 

…\marginnote{219.7} (5) he doesn’t misrepresent his belief of what’s true during the voting. 

When\marginnote{220.1} one who has caused a schism in the Sangha has five other qualities, he’s also redeemable, not destined to an eon in hell: (1) a monk proclaims what’s contrary to the Teaching as being in accordance with it, (2) what’s in accordance with the Teaching as contrary to it, (3) what’s contrary to the Monastic Law as being in accordance with it, (4) what’s in accordance with the Monastic Law as contrary to it, but (5) he doesn’t misrepresent his acceptance of what’s true during the legal procedure. 

…\marginnote{220.4} (5) he doesn’t misrepresent his acceptance of what’s true during the recitation. 

…\marginnote{220.5} (5) he doesn’t misrepresent his acceptance of what’s true while he speaks. 

…\marginnote{220.6} (5) he doesn’t misrepresent his acceptance of what’s true during the proclamation. 

…\marginnote{220.7} (5) he doesn’t misrepresent his acceptance of what’s true during the voting. 

When\marginnote{221.1} one who has caused a schism in the Sangha has five other qualities, he’s also redeemable, not destined to an eon in hell: (1) a monk proclaims what’s contrary to the Teaching as being in accordance with it, (2) what’s in accordance with the Teaching as contrary to it, (3) what’s contrary to the Monastic Law as being in accordance with it, (4) what’s in accordance with the Monastic Law as contrary to it, but (5) he doesn’t misrepresent his perception of what’s true during the legal procedure. 

…\marginnote{221.4} (5) he doesn’t misrepresent his perception of what’s true during the recitation. 

…\marginnote{221.5} (5) he doesn’t misrepresent his perception of what’s true while he speaks. 

…\marginnote{221.6} (5) he doesn’t misrepresent his perception of what’s true during the proclamation. 

…\marginnote{221.7} (5) he doesn’t misrepresent his perception of what’s true during the voting.” 

\scendvagga{The twelfth subchapter, the second on schism in the Sangha, is finished. }

\scuddanaintro{This is the summary: }

\begin{scuddana}%
“He\marginnote{224.1} does not misrepresent his view during the legal procedure, \\
During the recitation, and while he speaks; \\
During the proclamation, during the voting—\\
These five are dependent on view; \\
Belief, and acceptance, and perception—\\
These three by the fivefold method, as above. 

As\marginnote{225.1} these on the dark side, \\
Have twenty ways; \\
So are there twenty on the bright side, \\
You should understand.” 

%
\end{scuddana}

\section*{13. The subchapter on resident monks }

“Sir,\marginnote{226.1} what sort of resident monk is dumped in hell?” 

“One\marginnote{227.1} who has five qualities: he’s biased by desire, ill will, confusion, or fear, and he uses what belongs to the Sangha as if belonging to an individual. 

But\marginnote{228.1} a resident monk who has five qualities is deposited in heaven: he’s not biased by desire, ill will, confusion, or fear, and he doesn’t use what belongs to the Sangha as if belonging to an individual.” 

“How\marginnote{229.1} many illegitimate explanations of the Monastic Law are there?” 

“There\marginnote{230.1} are five: (1) a monk develops what’s contrary to the Teaching as being in accordance with it, (2) what’s in accordance with the Teaching as contrary to it, (3) what’s contrary to the Monastic Law as being in accordance with it, (4) what’s in accordance with the Monastic Law as contrary to it, and (5) he lays down new rules and gets rid of the existing ones.\footnote{Sp 5.462: \textit{\textsanskrit{Pariṇāmetīti} \textsanskrit{niyāmeti} \textsanskrit{dīpeti} katheti}, “Develops: he specifies, he proclaims, he speaks.” The basic meaning of \textit{\textsanskrit{pariṇāmeti}} is to change, often a change towards ripening or maturity. The meaning here is presumably that he changes or transforms the Teaching and the Monastic Law by distorting them. } And there are five legitimate explanations of the Monastic Law: (1) a monk develops what’s contrary to the Teaching to what’s contrary to the Teaching, (2) what’s in accordance with the Teaching to what’s in accordance with the Teaching, (3) what’s contrary to the Monastic Law to what’s contrary to the Monastic Law, (4) what’s in accordance with the Monastic Law to what’s in accordance with the Monastic Law, and (5) he doesn’t lay down new rules or get rid of the existing ones.” 

“What\marginnote{231.1} sort of designator of meals is dumped in hell?” 

“One\marginnote{232.1} who has five qualities: he’s biased by favoritism, ill will, confusion, or fear, and he doesn’t know what has and hasn’t been designated. 

But\marginnote{233.1} a designator of meals who has five qualities is deposited in heaven: he’s not biased by favoritism, ill will, confusion, or fear, and he knows what has and hasn’t been designated.” 

“What\marginnote{234.1} sort of assigner of dwellings … storeman … receiver of robe-cloth … distributor of robe-cloth … distributor of congee … distributor of fruit … distributor of fresh foods … distributor of minor requisites … distributor of rainy-season bathing cloths … distributor of almsbowls … supervisor of monastery workers … supervisor of novice monks is dumped in hell?” 

“One\marginnote{235.1} who has five qualities: he’s biased by favoritism, ill will, confusion, or fear, and he doesn’t know who is and is not supervised. But a supervisor of novice monks who has five qualities is deposited in heaven: he’s not biased by favoritism, ill will, confusion, or fear, and he knows who is and is not supervised.” 

\scendvagga{The thirteenth subchapter on resident monks is finished. }

\scuddanaintro{This is the summary: }

\begin{scuddana}%
“Resident,\marginnote{238.1} explanations, \\
Designator of meals, and dwellings; \\
Storeman, and receiver of robe-cloth, \\
And distributor of robe-cloth. 

Congee,\marginnote{239.1} fruit, and fresh foods, \\
Minor requisites, distributor of rainy-season bathing cloths; \\
Bowl, and monastery worker, \\
Supervisor of novice monks.” 

%
\end{scuddana}

\section*{14. The subchapter on the robe-making ceremony }

“Sir,\marginnote{240.1} how many benefits are there in participating in the robe-making ceremony?” 

“There\marginnote{241.1} are five benefits, \textsanskrit{Upāli}: (1) going without informing; (2) going without taking; (3) eating in a group; (4) as much extra robe-cloth as you need; and (5) whatever robe-cloth is given there is for you.”\footnote{For the first four of these five see \href{https://suttacentral.net/pli-tv-bu-vb-pc46/en/brahmali\#5.6.1}{Bu Pc 46:5.6.1}, \href{https://suttacentral.net/pli-tv-bu-vb-np2/en/brahmali\#2.39.1}{Bu NP 2:2.39.1}, \href{https://suttacentral.net/pli-tv-bu-vb-pc32/en/brahmali\#8.15.1}{Bu Pc 32:8.15.1}, and \href{https://suttacentral.net/pli-tv-bu-vb-np1/en/brahmali\#2.17.1}{Bu NP 1:2.17.1} respectively. } 

“How\marginnote{242.1} many drawbacks are there in falling asleep absentminded and heedless?” 

“There\marginnote{243.1} are these five drawbacks: you don’t sleep well; you wake up feeling miserable; you have nightmares; the gods don’t guard you; you emit semen. 

But\marginnote{244.1} there are five benefits in falling asleep mindfully, with clear awareness: you sleep well; you wake up feeling good; you don’t have nightmares; the gods guard you; you don’t emit semen.” 

“How\marginnote{245.1} many kinds of people should one not pay respect to?” 

“These\marginnote{246.1} five: one who has entered an inhabited area; one who’s walking along a street; one who’s in the dark; one who’s not paying attention; one who’s asleep.\footnote{Sp 5.467: \textit{\textsanskrit{Asamannāharantoti} \textsanskrit{kiccayapasutattā} \textsanskrit{vandanaṁ} \textsanskrit{asamannāharanto}}, “‘One who’s not paying attention’: not paying attention to the paying respect because of being occupied with a task.” } 

There\marginnote{247.1} are five other kinds of people you also shouldn’t pay respect to: one who’s drinking congee; one who’s in the dining hall; one who has turned away; one who’s preoccupied with something else; one who’s naked. 

There\marginnote{248.1} are five other kinds of people you also shouldn’t pay respect to: one who’s eating fresh food; one who’s eating cooked food; one who’s defecating; one who’s urinating; one who has been ejected. 

There\marginnote{249.1} are five other kinds of people you also shouldn’t pay respect to: (1) one who has been given the full ordination after you; (2) one who isn’t fully ordained; (3) one who belongs to a different Buddhist sect who’s senior to you, but who speaks contrary to the Teaching; (4) a woman; (5) a \textit{\textsanskrit{paṇḍaka}}. 

There\marginnote{250.1} are five other kinds of people you also shouldn’t pay respect to: (1) one who’s on probation; (2) one who deserves to be sent back to the beginning; (3) one who deserves the trial period; (4) one who’s undertaking the trial period; (5) one who deserves rehabilitation.” 

“But\marginnote{251.1} how many kinds of people should one pay respect to?” 

“These\marginnote{252.1} five: (1) one who has been given the full ordination before you; (2) one who belongs to a different Buddhist sect who’s senior to you and who speaks in accordance with the Teaching; (3) your teacher; (4) your preceptor; and (5) in this world with its gods, lords of death, and supreme beings, in this society with its monastics and brahmins, its gods and humans, you should pay respect to the Buddha, perfected and fully awakened.” 

“Sir,\marginnote{253.1} when a monk is paying respect to a more senior monk, how many qualities should he first set up in himself?” 

“He\marginnote{254.1} should set up five qualities in himself: (1) he should arrange his upper robe over one shoulder; (2) he should raise his joined palms; (3) he should stroke the feet with the palms of both hands; (4) he should set up a sense of affection; (5) he should set up a sense of respect.” 

\scendvagga{The fourteenth subchapter on the robe-making ceremony is finished. }

\scuddanaintro{This is the summary: }

\begin{scuddana}%
“The\marginnote{257.1} robe-making ceremony, and asleep, \\
Inhabited, congee, when eating fresh food; \\
And before, and one on probation, \\
Should pay respect to, to be paid respect.” 

%
\end{scuddana}

\scend{Ven. \textsanskrit{Upāli} questions the Buddha is finished. }

\scuddanaintro{This is the summary of these subchapters: }

\begin{scuddana}%
“Without\marginnote{260.1} formal support, and legal procedure, \\
Speech, and with revealing; \\
And accusing, and ascetic practices, \\
Lying, and nun. 

Committee,\marginnote{261.1} legal issue, \\
Schism, the fifth before; \\
Resident monks, and robe-making ceremony—\\
The fourteen have been well proclaimed.” 

%
\end{scuddana}

%
\chapter*{{\suttatitleacronym Pvr 18}{\suttatitletranslation The origination of offenses }{\suttatitleroot Atthāpattisamuṭṭhāna}}
\addcontentsline{toc}{chapter}{\tocacronym{Pvr 18} \toctranslation{The origination of offenses } \tocroot{Atthāpattisamuṭṭhāna}}
\markboth{The origination of offenses }{Atthāpattisamuṭṭhāna}
\extramarks{Pvr 18}{Pvr 18}

\section*{1. The offenses entailing expulsion }

There\marginnote{1.1} are offenses that are committed unintentionally, but cleared intentionally.\footnote{For this and the following three items, see \href{https://suttacentral.net/pli-tv-pvr7/en/brahmali\#46.13}{Pvr 7:46.13}–46.16. } There are offenses that are committed intentionally, but cleared unintentionally. There are offenses that are committed unintentionally and cleared unintentionally. There are offenses that are committed intentionally and cleared intentionally. 

There\marginnote{1.5} are offenses that are committed with a wholesome mind and cleared with a wholesome mind. There are offenses that are committed with a wholesome mind, but cleared with an unwholesome mind. There are offenses that are committed with a wholesome mind, but cleared with an indeterminate mind. 

There\marginnote{1.8} are offenses that are committed with an unwholesome mind, but cleared with a wholesome mind. There are offenses that are committed with an unwholesome mind and cleared with an unwholesome mind. There are offenses that are committed with an unwholesome mind, but cleared with an indeterminate mind. 

There\marginnote{1.11} are offenses that are committed with an indeterminate mind, but cleared with a wholesome mind. There are offenses that are committed with an indeterminate mind, but cleared with an unwholesome mind. There are offenses that are committed with an indeterminate mind and cleared with an indeterminate mind. 

How\marginnote{2.1} many originations does the first offense entailing expulsion have? It has one: body and mind, not speech. 

How\marginnote{3.1} many originations does the second offense entailing expulsion have? It has three: body and mind, not speech; or speech and mind, not body; or body, speech, and mind. 

How\marginnote{4.1} many originations does the third offense entailing expulsion have? It has three: body and mind, not speech; or speech and mind, not body; or body, speech, and mind. 

How\marginnote{5.1} many originations does the fourth offense entailing expulsion have? It has three: body and mind, not speech; or speech and mind, not body; or body, speech, and mind. 

\scend{The four offenses entailing expulsion are finished. }

\section*{2. The offenses entailing suspension }

There\marginnote{7.1} is an offense entailing suspension for emitting semen by means of effort. How many originations does it have? It has one: body and mind, not speech. 

There\marginnote{8.1} is an offense entailing suspension for making physical contact with a woman. How many originations does it have? It has one: body and mind, not speech. 

There\marginnote{9.1} is an offense entailing suspension for speaking indecently to a woman. How many originations does it have? It has three: body and mind, not speech; or speech and mind, not body; or body, speech, and mind. 

There\marginnote{10.1} is an offense entailing suspension for encouraging a woman to satisfy one’s own desires. How many originations does it have? It has three: … 

There\marginnote{11.1} is an offense entailing suspension for acting as a matchmaker. How many originations does it have? It has six: body, not speech or mind; or speech, not body or mind; or body and speech, not mind; or body and mind, not speech; or speech and mind, not body; or body, speech, and mind. 

There\marginnote{12.1} is an offense entailing suspension for having a hut built by means of begging. How many originations does it have? It has six: … 

There\marginnote{13.1} is an offense entailing suspension for having a large dwelling built. How many originations does it have? It has six: … 

There\marginnote{14.1} is an offense entailing suspension for groundlessly charging a monk with an offense entailing expulsion. How many originations does it have? It has three: … 

There\marginnote{15.1} is an offense entailing suspension for charging a monk with an offense entailing expulsion, using an unrelated legal issue as a pretext. How many originations does it have? It has three: … 

There\marginnote{16.1} is an offense entailing suspension for a monk who does not stop pursuing schism in the Sangha when pressed for the third time. How many originations does it have? It has one: body, speech, and mind. 

There\marginnote{17.1} is an offense entailing suspension for monks who do not stop siding with one who is pursuing schism in the Sangha when pressed for the third time. How many originations does it have? It has one: body, speech, and mind. 

There\marginnote{18.1} is an offense entailing suspension for a monk who does not stop being difficult to correct when pressed for the third time. How many originations does it have? It has one: body, speech, and mind. 

There\marginnote{19.1} is an offense entailing suspension for a monk who does not stop being a corrupter of families when pressed for the third time. How many originations does it have? It has one: body, speech, and mind. 

\scend{The thirteen offenses entailing suspension are finished }

\section*{3. The rules to be trained in }

…\marginnote{21.1} There is an offense of wrong conduct for, out of disrespect, defecating, urinating, or spitting in water. How many originations does it have? It has one: body and mind, not speech. 

\scend{The rules to be trained in are finished. }

\section*{4. The offenses entailing expulsion, etc. }

How\marginnote{23.1} many originations do the four offenses entailing expulsion have? They have three: body and mind, not speech; or speech and mind, not body; or body, speech, and mind. 

How\marginnote{24.1} many originations do the thirteen offenses entailing suspension have? They have six: body, not speech or mind; or speech, not body or mind; or body and speech, not mind; or body and mind, not speech; or speech and mind, not body; or body, speech, and mind. 

How\marginnote{25.1} many originations do the two undetermined offenses have? They have three: body and mind, not speech; or speech and mind, not body; or body, speech, and mind. 

How\marginnote{26.1} many originations do the thirty offenses entailing relinquishment have? They have six: body, not speech or mind; or speech, not body or mind; or body and speech, not mind; or body and mind, not speech; or speech and mind, not body; or body, speech, and mind. 

How\marginnote{27.1} many originations do the ninety-two offenses entailing confession have? They have six: body, not speech or mind; or speech, not body or mind; or body and speech, not mind; or body and mind, not speech; or speech and mind, not body; or body, speech, and mind. 

How\marginnote{28.1} many originations do the four offenses entailing acknowledgment have? They have four: body, not speech or mind; or body and speech, not mind; or body and mind, not speech; or body, speech, and mind. 

How\marginnote{29.1} many originations do the seventy-five rules to be trained in have? They have three: body and mind, not speech; or speech and mind, not body; or body, speech, and mind. 

\scendsutta{The originations are finished. }

\scuddanaintro{This is the summary: }

\begin{scuddana}%
“Unintentionally,\marginnote{32.1} and wholesome, \\
And originations in every way; \\
By the method that accords with the Teaching, \\
You should know origination.” 

%
\end{scuddana}

%
\chapter*{{\suttatitleacronym Pvr 19}{\suttatitletranslation Verses on offenses, training rules, and legal procedures }{\suttatitleroot Dutiyagāthāsaṅgaṇika}}
\addcontentsline{toc}{chapter}{\tocacronym{Pvr 19} \toctranslation{Verses on offenses, training rules, and legal procedures } \tocroot{Dutiyagāthāsaṅgaṇika}}
\markboth{Verses on offenses, training rules, and legal procedures }{Dutiyagāthāsaṅgaṇika}
\extramarks{Pvr 19}{Pvr 19}

\section*{1. Offenses by body, etc. }

\begin{verse}%
“How\marginnote{1.1} many kinds of offenses are there by body? \\
How many are done by speech? \\
How many kinds of offenses are there for concealing? \\
How many because of contact? 

There\marginnote{2.1} are six kinds of offenses by body,\footnote{Sp 5.474: \textit{Cha \textsanskrit{āpattiyo} \textsanskrit{kāyikāti} \textsanskrit{antarapeyyāle} catutthena \textsanskrit{āpattisamuṭṭhānena} cha \textsanskrit{āpattiyo} \textsanskrit{āpajjati}}, “‘There are six kinds of offenses by body’: in the Internal Repetition, six offenses are committed through the fourth origination of offenses.” See \href{https://suttacentral.net/pli-tv-pvr4/en/brahmali\#34.1}{Pvr 4:34.1}–34.14. It is not clear why the commentary points to the fourth origination which concerns body and mind, not just body. } \\
Six are done by speech;\footnote{Sp 5.474: \textit{Cha \textsanskrit{vācasikāti} \textsanskrit{tasmiṁyeva} \textsanskrit{antarapeyyāle} \textsanskrit{pañcamena} \textsanskrit{āpattisamuṭṭhānena} cha \textsanskrit{āpattiyo} \textsanskrit{āpajjati}}, “‘Six by speech’: in the Internal Repetition, six offenses are committed through the fifth origination of offenses.” See \href{https://suttacentral.net/pli-tv-pvr4/en/brahmali\#35.1}{Pvr 4:35.1}–35.14. Again, it is not clear why the commentary points to the fifth origination which concerns speech and mind, not just speech. } \\
There are three kinds of offenses for concealing,\footnote{Sp 5.474: \textit{\textsanskrit{Chādentassa} tissoti \textsanskrit{vajjapaṭicchādikāya} \textsanskrit{bhikkhuniyā} \textsanskrit{pārājikaṁ}, bhikkhussa \textsanskrit{saṅghādisesapaṭicchādane} \textsanskrit{pācittiyaṁ}, attano \textsanskrit{duṭṭhullāpattipaṭicchādane} \textsanskrit{dukkaṭaṁ}}, “‘Three for concealing’: an offense entailing expulsion for a nun who hides offenses, an offense entailing confession for a monk who hides an offense entailing suspension, an offense of wrong conduct for hiding a coarse offense.” See respectively \href{https://suttacentral.net/pli-tv-bi-vb-pj6/en/brahmali\#1.23.1}{Bi Pj 6:1.23.1}, \href{https://suttacentral.net/pli-tv-bu-vb-pc64/en/brahmali\#1.23.1}{Bu Pc 64:1.23.1}, and \href{https://suttacentral.net/pli-tv-kd13/en/brahmali\#34.1.4}{Kd 13:34.1.4}. } \\
Five because of contact.\footnote{Sp 5.474: \textit{\textsanskrit{Pañca} \textsanskrit{saṁsaggapaccayāti} \textsanskrit{bhikkhuniyā} \textsanskrit{kāyasaṁsagge} \textsanskrit{pārājikaṁ}, bhikkhuno \textsanskrit{saṅghādiseso}, \textsanskrit{kāyena} \textsanskrit{kāyapaṭibaddhe} \textsanskrit{thullaccayaṁ}, nissaggiyena \textsanskrit{kāyapaṭibaddhe} \textsanskrit{dukkaṭaṁ}, \textsanskrit{aṅgulipatodake} \textsanskrit{pācittiyanti} \textsanskrit{imā} \textsanskrit{kāyasaṁsaggapaccayā} \textsanskrit{pañcāpattiyo}}, “‘Five because of contact’: physical contact is an offense entailing expulsion for a nun. For a monk it’s an offense entailing suspension. Contacting with the body what is connected to the body is a serious offense. Contacting by releasing what is connected to the body is an offense of wrong conduct. Tickling is an offense entailing confession.” The first of these refers to \href{https://suttacentral.net/pli-tv-bi-vb-pj5/en/brahmali\#1.54.1}{Bi Pj 5:1.54.1}; the second, third, and fourth all refer to \href{https://suttacentral.net/pli-tv-bu-vb-ss2/en/brahmali\#1.2.15.1}{Bu Ss 2:1.2.15.1}; and the last refers to \href{https://suttacentral.net/pli-tv-bu-vb-pc52/en/brahmali\#1.12.1}{Bu Pc 52:1.12.1}. } 

How\marginnote{3.1} many kinds of offenses are there at dawn? \\
How many after the third announcement? \\
How many here have eight parts? \\
Through how many are they all included? 

There\marginnote{4.1} are three kinds of offenses at dawn,\footnote{Sp 5.474: \textit{\textsanskrit{Aruṇugge} tissoti \textsanskrit{ekarattachārattasattāhadasāhamāsātikkamavasena} \textsanskrit{nissaggiyaṁ} \textsanskrit{pācittiyaṁ}, \textsanskrit{bhikkhuniyā} \textsanskrit{rattivippavāse} \textsanskrit{saṅghādiseso}, “\textsanskrit{paṭhamampi} \textsanskrit{yāmaṁ} \textsanskrit{chādeti}, dutiyampi tatiyampi \textsanskrit{yāmaṁ} \textsanskrit{chādeti}, uddhaste \textsanskrit{aruṇe} \textsanskrit{channā} hoti \textsanskrit{āpatti}, yo \textsanskrit{chādeti} so \textsanskrit{dukkaṭaṁ} \textsanskrit{desāpetabbo}”ti \textsanskrit{imā} \textsanskrit{aruṇugge} tisso \textsanskrit{āpattiyo} \textsanskrit{āpajjati}}, “‘Three at dawn’: there is an offense entailing relinquishment and confession on account of going beyond one day, six days, seven days, ten days, and one month. There is an offense entailing suspension for a nun who stays apart for one night. ‘He conceals it during the first part of the night, during the second part of the night, and during the third part of the night. If he’s still concealing it at dawn, he has committed an offense and is to confess an offense of wrong conduct.’ These three offenses are committed at dawn.” The first category refers respectively to \href{https://suttacentral.net/pli-tv-bu-vb-np2/en/brahmali\#2.39.1}{Bu NP 2:2.39.1}, \href{https://suttacentral.net/pli-tv-bu-vb-np29/en/brahmali\#1.2.16.1}{Bu NP 29:1.2.16.1}, \href{https://suttacentral.net/pli-tv-bu-vb-np23/en/brahmali\#1.3.32.1}{Bu NP 23:1.3.32.1}, \href{https://suttacentral.net/pli-tv-bu-vb-np1/en/brahmali\#2.17.1}{Bu NP 1:2.17.1}, and \href{https://suttacentral.net/pli-tv-bu-vb-np3/en/brahmali\#1.3.13.1}{Bu NP 3:1.3.13.1}; the second category to \href{https://suttacentral.net/pli-tv-bi-vb-ss3/en/brahmali\#4.14.1}{Bi Ss 3:4.14.1}; and the last to the monks’ offenses entailing suspension at \href{https://suttacentral.net/pli-tv-kd13/en/brahmali\#34.2.4}{Kd 13:34.2.4}–34.2.6. } \\
Two after the third announcement;\footnote{Sp 5.474: \textit{Dve \textsanskrit{yāvatatiyakāti} \textsanskrit{ekādasa} \textsanskrit{yāvatatiyakā} \textsanskrit{nāma}, \textsanskrit{paññattivasena} pana dve honti \textsanskrit{bhikkhūnaṁ} \textsanskrit{yāvatatiyakā} \textsanskrit{bhikkhunīnaṁ} \textsanskrit{yāvatatiyakāti}}, “‘Two after the third announcement’: eleven (offenses) are called ‘after the third announcement’. But on account of their designation, there are two: after the third announcement for the monks and after the third announcement for nuns.” } \\
One here has eight parts,\footnote{Sp 5.474: \textit{Ekettha \textsanskrit{aṭṭhavatthukāti} \textsanskrit{bhikkhunīnaṁyeva} \textsanskrit{ekā} ettha \textsanskrit{imasmiṁ} \textsanskrit{sāsane} \textsanskrit{aṭṭhavatthukā} \textsanskrit{nāma}}, “‘One here has eight parts’: here in Buddhism one of the nuns’ (rules) is called ‘having eight parts’.” This refers to \href{https://suttacentral.net/pli-tv-bi-vb-pj8/en/brahmali\#1.11.1}{Bi Pj 8:1.11.1}. “Rules” is supplied from Vmv 5.474. } \\
They are all included through one.\footnote{Sp 5.474: \textit{Ekena \textsanskrit{sabbasaṅgahoti} “yassa \textsanskrit{siyā} \textsanskrit{āpatti}, so \textsanskrit{āvikareyyā}”ti \textsanskrit{iminā} ekena \textsanskrit{nidānuddesena} \textsanskrit{sabbasikkhāpadānañca} \textsanskrit{sabbapātimokkhuddesānañca} \textsanskrit{saṅgaho} hoti}, “‘They are all included through one’: all training rules and the whole recitation of the Monastic Code are included through this one reference in the Introduction that ‘whoever might have an offense should reveal it’.” See \href{https://suttacentral.net/pli-tv-kd2/en/brahmali\#3.3.9}{Kd 2:3.3.9}. } 

How\marginnote{5.1} many roots are there of the Monastic Law, \\
Which were laid down by the Buddha? \\
How many are said to be heavy in the Monastic Law? \\
How many kinds of concealing are there of grave offenses? 

There\marginnote{6.1} are two roots of the Monastic Law,\footnote{Sp 5.474: \textit{Vinayassa dve \textsanskrit{mūlānīti} \textsanskrit{kāyo} ceva \textsanskrit{vācā} ca}, “‘There are two roots of the Monastic Law’: body and speech.” } \\
Which were laid down by the Buddha; \\
Two are said to be heavy in the Monastic Law,\footnote{Sp 5.474: \textit{\textsanskrit{Garukā} dve \textsanskrit{vuttāti} \textsanskrit{pārājikasaṅghādisesā}}, “‘Two are said to be heavy’: offenses entailing expulsion and offenses entailing suspension.” } \\
There are two kinds of concealing of grave offenses.\footnote{Sp 5.474: \textit{Dve \textsanskrit{duṭṭhullacchādanāti} \textsanskrit{vajjapaṭicchādikāya} \textsanskrit{pārājikaṁ} \textsanskrit{saṅghādisesaṁ} \textsanskrit{paṭicchādakassa} \textsanskrit{pācittiyanti} \textsanskrit{imā} dve \textsanskrit{duṭṭhullacchādanāpattiyo} \textsanskrit{nāma}}, “‘There are two kinds of concealing of grave offenses’: there is an offense entailing expulsion for a nun who conceals an offense and an offense entailing confession for a monk who conceals an offense entailing suspension. These two are called offenses for concealing grave offenses.” The first refers to \href{https://suttacentral.net/pli-tv-bi-vb-pj6/en/brahmali\#1.23.1}{Bi Pj 6:1.23.1} and the second to \href{https://suttacentral.net/pli-tv-bu-vb-pc64/en/brahmali\#1.23.1}{Bu Pc 64:1.23.1}. } 

How\marginnote{7.1} many kinds of offenses are there concerning the next inhabited area? \\
How many because of crossing a river? \\
How many serious offenses are there in relation to meat? \\
How many offenses of wrong conduct are there in relation to meat? 

There\marginnote{8.1} are four kinds of offenses concerning the next inhabited area,\footnote{Sp 5.474: \textit{\textsanskrit{Gāmantare} catassoti “bhikkhu \textsanskrit{bhikkhuniyā} \textsanskrit{saddhiṁ} \textsanskrit{saṁvidahati}, \textsanskrit{dukkaṭaṁ}; \textsanskrit{aññassa} \textsanskrit{gāmassa} \textsanskrit{upacāraṁ} okkamati, \textsanskrit{pācittiyaṁ}; \textsanskrit{bhikkhuniyā} \textsanskrit{gāmantaraṁ} \textsanskrit{gacchantiyā} parikkhitte \textsanskrit{gāme} \textsanskrit{paṭhamapāde} \textsanskrit{thullaccayaṁ}, \textsanskrit{dutiyapāde} \textsanskrit{saṅghādiseso} …”ti}, “‘Four concerning the next village’: there is an offense of wrong conduct when a monk makes an arrangement with a nun, and an offense entailing confession when entering the vicinity of the next village. When a nun goes to another village, there is a serious offense for entering an enclosed village with the first foot, and an offense entailing suspension for the second foot. …” The first two of these refer to \href{https://suttacentral.net/pli-tv-bu-vb-pc27/en/brahmali\#2.15.1}{Bu Pc 27:2.15.1}, and the last two to \href{https://suttacentral.net/pli-tv-bi-vb-ss3/en/brahmali\#4.14.1}{Bi Ss 3:4.14.1}. The idea of “vicinity” of the next village is taken from the commentary to Bu Pc 27. } \\
Four because of crossing a river;\footnote{Sp 5.474: \textit{Catasso \textsanskrit{nadipārapaccayāti} “bhikkhu \textsanskrit{bhikkhuniyā} \textsanskrit{saddhiṁ} \textsanskrit{saṁvidahati}, \textsanskrit{dukkaṭaṁ}; \textsanskrit{nāvaṁ} abhiruhati, \textsanskrit{pācittiyaṁ}; \textsanskrit{bhikkhuniyā} \textsanskrit{nadipāraṁ} \textsanskrit{gacchantiyā} \textsanskrit{uttaraṇakāle} \textsanskrit{paṭhamapāde} \textsanskrit{thullaccayaṁ}, \textsanskrit{dutiyapāde} \textsanskrit{saṅghādiseso}”ti \textsanskrit{imā} catasso}, “Four because of crossing a river’: there is an offense of wrong conduct when a monk makes an arrangement with a nun, and an offense entailing confession when boarding the boat. At the time of crossing with the first foot, there is a serious offense for a nun crossing a river, and for the second foot, there is an offense entailing suspension. These are the four.” The first two of these refer to \href{https://suttacentral.net/pli-tv-bu-vb-pc28/en/brahmali\#2.16.1}{Bu Pc 28:2.16.1}, and the last two to \href{https://suttacentral.net/pli-tv-bi-vb-ss3/en/brahmali\#4.14.1}{Bi Ss 3:4.14.1}. } \\
There is one serious offense in relation to meat,\footnote{Sp 5.474: \textit{\textsanskrit{Ekamaṁse} thullaccayanti \textsanskrit{manussamaṁse}}, “‘There is one serious offense in relation to meat’: in relation to human flesh.” See \href{https://suttacentral.net/pli-tv-kd6/en/brahmali\#23.9.7}{Kd 6:23.9.7}. } \\
And nine offenses of wrong conduct.\footnote{Sp 5.474: \textit{\textsanskrit{Navamaṁsesu} \textsanskrit{dukkaṭanti} \textsanskrit{sesaakappiyamaṁsesu}}, “‘And nine offenses of wrong conduct’: the rest of the unallowable meats.” See \href{https://suttacentral.net/pli-tv-kd6/en/brahmali\#23.10.8}{Kd 6:23.10.8}–23.15.9. } 

How\marginnote{9.1} many kinds of offenses are there by speech at night? \\
How many are there by speech by day? \\
How many kinds of offenses are there for one who’s giving? \\
And how many for one who’s taking? 

There\marginnote{10.1} are two kinds of offenses by speech at night,\footnote{Sp 5.474: \textit{Dve \textsanskrit{vācasikā} rattinti \textsanskrit{bhikkhunī} \textsanskrit{rattandhakāre} \textsanskrit{appadīpe} purisena \textsanskrit{saddhiṁ} \textsanskrit{hatthapāse} \textsanskrit{ṭhitā} sallapati, \textsanskrit{pācittiyaṁ}; \textsanskrit{hatthapāsaṁ} \textsanskrit{vijahitvā} \textsanskrit{ṭhitā} sallapati, \textsanskrit{dukkaṭaṁ}}, “‘There are two kinds of offenses by speech at night’: for a nun who stands talking within arm’s reach of a man in the dark of the night without a lamp there is an offense entailing confession. For standing and talking outside of arm’s reach, there is an offense of wrong conduct.” See \href{https://suttacentral.net/pli-tv-bi-vb-pc11/en/brahmali\#1.12.1}{Bi Pc 11:1.12.1}. } \\
Two by speech by day;\footnote{Sp 5.474: \textit{Dve \textsanskrit{vācasikā} \textsanskrit{divāti} \textsanskrit{bhikkhunī} \textsanskrit{divā} \textsanskrit{paṭicchanne} \textsanskrit{okāse} purisena \textsanskrit{saddhiṁ} \textsanskrit{hatthapāse} \textsanskrit{ṭhitā} sallapati, \textsanskrit{pācittiyaṁ}; \textsanskrit{hatthapāsaṁ} \textsanskrit{vijahitvā} sallapati, \textsanskrit{dukkaṭaṁ}}, “‘Two by speech by day’: for a nun who stands talking within arm’s reach of a man during the day in a concealed place there is an offense entailing confession. For standing and talking outside of arm’s reach, there is an offense of wrong conduct.” See \href{https://suttacentral.net/pli-tv-bi-vb-pc12/en/brahmali\#1.13.1}{Bi Pc 12:1.13.1}. } \\
There are three kinds of offenses for one who is giving,\footnote{Sp 5.474: \textit{\textsanskrit{Dadamānassa} tissoti \textsanskrit{maraṇādhippāyo} manussassa \textsanskrit{visaṁ} deti, so ce tena marati, \textsanskrit{pārājikaṁ}; \textsanskrit{yakkhapetānaṁ} deti, te ce maranti, \textsanskrit{thullaccayaṁ}; \textsanskrit{tiracchānagatassa} deti, so ce marati, \textsanskrit{pācittiyaṁ}; \textsanskrit{aññātikāya} \textsanskrit{bhikkhuniyā} \textsanskrit{cīvaradāne} \textsanskrit{pācittiyanti} \textsanskrit{evaṁ} \textsanskrit{dadamānassa} tisso \textsanskrit{āpattiyo}}, “‘Three for one who is giving’: if, aiming at death, one gives poison to a human being, then, if they die from that, there is an offense entailing expulsion. If one gives it to a spirit or a ghost, and it dies from that, there is a serious offense. If one gives it to an animal, and it dies from that, there is an offense entailing confession. In giving robe-cloth to an unrelated nun, there is an offense entailing confession. In this way there are three kinds of offenses for one who is giving.” The first three refer to \href{https://suttacentral.net/pli-tv-bu-vb-pj3/en/brahmali\#2.49.1}{Bu Pj 3:2.49.1}, and the fourth to \href{https://suttacentral.net/pli-tv-bu-vb-pc25/en/brahmali\#2.11.1}{Bu Pc 25:2.11.1}. } \\
And four for taking.”\footnote{Sp 5.474: \textit{\textsanskrit{Cattāro} ca \textsanskrit{paṭiggaheti} \textsanskrit{hatthaggāha}-\textsanskrit{veṇiggāhesu} \textsanskrit{saṅghādiseso}, mukhena \textsanskrit{aṅgajātaggahaṇe} \textsanskrit{pārājikaṁ}, \textsanskrit{aññātikāya} \textsanskrit{bhikkhuniyā} \textsanskrit{cīvarapaṭiggahaṇe} \textsanskrit{nissaggiyaṁ} \textsanskrit{pācittiyaṁ}, \textsanskrit{avassutāya} avassutassa hatthato \textsanskrit{khādanīyaṁ} \textsanskrit{bhojanīyaṁ} \textsanskrit{paṭiggaṇhantiyā} \textsanskrit{thullaccayaṁ}}, “‘And four for taking’: there is an offense entailing suspension for taking hold of a braid of hair; an offense entailing expulsion for taking a penis in the mouth; an offense entailing relinquishment and confession for receiving a robe from an unrelated nun; a serious offense for a lustful nun to take fresh or cooked food directly from the hands of a lustful man.” The first of these refer to \href{https://suttacentral.net/pli-tv-bu-vb-ss2/en/brahmali\#1.2.15.1}{Bu Ss 2:1.2.15.1}, the second to \href{https://suttacentral.net/pli-tv-bu-vb-pj1/en/brahmali\#7.1.16.1}{Bu Pj 1:7.1.16.1}, the third to \href{https://suttacentral.net/pli-tv-bu-vb-np5/en/brahmali\#2.10.1}{Bu NP 5:2.10.1}, and the fourth to \href{https://suttacentral.net/pli-tv-bi-vb-ss5/en/brahmali\#2.1.15}{Bi Ss 5:2.1.15}. } 

%
\end{verse}

\section*{2. Offenses that are clearable by confession, etc. }

\begin{verse}%
“How\marginnote{11.1} many kinds of offenses are clearable by confession? \\
How many require amends to be made? \\
How many here do not to require amends, \\
According to the Buddha, the Kinsman of the Sun? 

Five\marginnote{12.1} kinds are clearable by confession,\footnote{Sp 5.475: \textit{\textsanskrit{Pañca} \textsanskrit{desanāgāminiyoti} \textsanskrit{lahukā} \textsanskrit{pañca}}, “‘Five kinds are clearable by confession’: the five kinds of light offenses.” } \\
Six require amends to be made;\footnote{Sp 5.475: \textit{Cha \textsanskrit{sappaṭikammāti} \textsanskrit{pārājikaṁ} \textsanskrit{ṭhapetvā} \textsanskrit{avasesā}}, “‘Six require amends’: all apart from the offenses entailing expulsion.” } \\
One here does not require amends,\footnote{Sp 5.475: \textit{Ekettha \textsanskrit{appaṭikammāti} \textsanskrit{ekā} \textsanskrit{pārājikāpatti}}, “‘One does not require amends’: the one is the offenses entailing expulsion.” } \\
According to the Buddha, the Kinsman of the Sun. 

How\marginnote{13.1} many are said to be heavy in the Monastic Law, \\
And done by body or speech? \\
How many grain juices are allowable at the wrong time? \\
How many appointments are made through one motion and three announcements? 

Two\marginnote{14.1} are said to be heavy in the Monastic Law,\footnote{See above. } \\
And are done by body or speech;\footnote{Sp 5.475: \textit{\textsanskrit{Kāyavācasikāni} \textsanskrit{cāti} \textsanskrit{sabbāneva} \textsanskrit{sikkhāpadāni} \textsanskrit{kāyavācasikāni}, \textsanskrit{manodvāre} \textsanskrit{paññattaṁ} \textsanskrit{ekasikkhāpadampi} natthi}, “‘And are done by body or speech’: all training rules are done by body or speech. There is not even a single training rule that was laid down in regard to the mind door.” In the Canonical text, however, this line seems to be connected to the previous one, that is, only to the heavy offenses. } \\
One grain juice is allowable at the wrong time,\footnote{Sp 5.475: \textit{Eko \textsanskrit{vikāle} \textsanskrit{dhaññarasoti} \textsanskrit{loṇasovīrakaṁ}}, “‘One grain juice is allowable at the wrong time’: the salty purgative.” } \\
One appointment is made through one motion and three announcements.\footnote{Sp 5.475: \textit{\textsanskrit{Ekā} \textsanskrit{ñatticatutthena} \textsanskrit{sammutīti} \textsanskrit{bhikkhunovādakasammuti}}, “‘One appointment is made through one motion and three announcements’: the appointment of an instructor of the nuns.” } 

How\marginnote{15.1} many offenses entailing expulsion are done by body? \\
How many grounds are there for belonging to the same Buddhist sect? \\
For how many kinds of people are there non-countable days? \\
How many rules concern 3.5 centimeters?\footnote{For a discussion of various measures, see \textit{sugata} in Appendix of Technical Terms. } 

Two\marginnote{16.1} offenses entailing expulsion are done by body,\footnote{Sp 5.475: \textit{\textsanskrit{Pārājikā} \textsanskrit{kāyikā} dveti \textsanskrit{bhikkhūnaṁ} \textsanskrit{methunapārājikaṁ} \textsanskrit{bhikkhunīnañca} \textsanskrit{kāyasaṁsaggapārājikaṁ}}, “‘Two offenses entailing expulsion are done by body’: the monks’ offense entailing expulsion for sexual intercourse and the nuns’ offense entailing expulsion for physical contact.” } \\
There are two grounds for belonging to the same Buddhist sect;\footnote{Sp 5.475: \textit{Dve \textsanskrit{saṁvāsabhūmiyoti} \textsanskrit{attanā} \textsanskrit{vā} \textsanskrit{attānaṁ} \textsanskrit{samānasaṁvāsakaṁ} karoti, samaggo \textsanskrit{vā} \textsanskrit{saṅgho} \textsanskrit{ukkhittaṁ} \textsanskrit{osāreti}}, “‘There are two grounds for belonging to the same Buddhist sect’: either one makes oneself belong to the same Buddhist sect, or a unanimous assembly readmits one who has been ejected.” See \href{https://suttacentral.net/pli-tv-kd10/en/brahmali\#1.10.6}{Kd 10:1.10.6}. } \\
There are non-countable days for two kinds of people,\footnote{Sp 5.475: \textit{\textsanskrit{Dvinnaṁ} ratticchedoti \textsanskrit{pārivāsikassa} ca \textsanskrit{mānattacārikassa} ca \textsanskrit{paññattā}}, “‘There are non-countable days for two kinds of people’: it is laid down for those on probation and for those undertaking the trial period.” } \\
There are two rules that concern 3.5 centimeters.\footnote{This concerns \href{https://suttacentral.net/pli-tv-bi-vb-pc5/en/brahmali\#1.2.12.1}{Bi Pc 5:1.2.12.1} and \href{https://suttacentral.net/pli-tv-kd15/en/brahmali\#2.2.7}{Kd 15:2.2.7}. } 

How\marginnote{17.1} many are there on having beaten oneself? \\
How many kinds of people cause a schism in the Sangha? \\
How many here are immediate offenses? \\
How many ways are there of doing a motion? 

There\marginnote{18.1} are two on having beaten oneself,\footnote{Sp 5.475: \textit{Dve \textsanskrit{attānaṁ} \textsanskrit{vadhitvānāti} \textsanskrit{bhikkhunī} \textsanskrit{attānaṁ} \textsanskrit{vadhitvā} dve \textsanskrit{āpattiyo} \textsanskrit{āpajjati}; vadhati rodati, \textsanskrit{āpatti} \textsanskrit{pācittiyassa}; vadhati na rodati, \textsanskrit{āpatti} \textsanskrit{dukkaṭassa}}, “‘There are two on having beaten oneself’: a nun who beats herself commits two offenses: if she beats and cries, an offense entailing confession; if she beats but does not cry, an offense of wrong conduct.” See \href{https://suttacentral.net/pli-tv-bi-vb-pc20/en/brahmali\#1.11.1}{Bi Pc 20:1.11.1}. } \\
Two kinds cause a schism in the Sangha;\footnote{Sp 5.475: \textit{\textsanskrit{Dvīhi} \textsanskrit{saṅgho} \textsanskrit{bhijjatīti} kammena ca \textsanskrit{salākaggāhena} ca}, “‘Two kinds cause a schism in the Sangha’: through a legal procedure and through voting.” } \\
Two here are immediate offenses,\footnote{Sp 5.475: \textit{Dvettha \textsanskrit{paṭhamāpattikāti} ettha sakalepi vinaye dve \textsanskrit{paṭhamāpattikā} \textsanskrit{ubhinnaṁ} \textsanskrit{paññattivasena}. \textsanskrit{Itarathā} pana nava \textsanskrit{bhikkhūnaṁ} nava \textsanskrit{bhikkhunīnanti} \textsanskrit{aṭṭhārasa} honti}, “‘Two here are immediate offenses’: here, in the whole Monastic Law, there are two kinds immediate offenses on account of their designation to both.” Sp-yoj 5.475: \textit{Ubhinnanti \textsanskrit{bhikkhubhikkhunīnaṁ}}, “‘To both’: to the monks and to the nuns.” } \\
There are two ways of doing a motion.\footnote{Sp 5.475: \textit{\textsanskrit{Ñattiyā} \textsanskrit{karaṇā} duveti dve \textsanskrit{ñattikiccāni} – \textsanskrit{kammañca} \textsanskrit{kammapādakā} ca}, “‘There are two ways of doing a motion’: there are two kinds of motions to be done: the legal procedure and the legal procedure with supports.” Sp 5.475: \textit{\textsanskrit{Kammañca} \textsanskrit{kammapādakā} \textsanskrit{cāti} ettha \textsanskrit{yasmā} \textsanskrit{ñattikammesu} \textsanskrit{ñatti} sayameva \textsanskrit{kammaṁ} hoti, \textsanskrit{ñattidutiyañatticatutthesu} kammesu \textsanskrit{anussāvanasaṅkhātassa} kammassa \textsanskrit{ñattipādakabhāvena} \textsanskrit{tiṭṭhati}, \textsanskrit{tasmā} \textsanskrit{imāni} dve “\textsanskrit{ñattikiccānī}”ti \textsanskrit{vuttāni}}, “‘The legal procedure and the legal procedure with supports’: here, for the legal procedure consisting of one motion, the motion itself is the legal procedure. For the legal procedure consisting of one motion and one announcement and for the legal procedure consisting of one motion and three announcements, there remains for the procedure that which is called the proclamation by way of supporting the motion. Therefore it is said that there are these two kinds of motions to be done.” } 

How\marginnote{19.1} many kinds offenses are there for killing living beings? \\
How many offenses entailing expulsion are there because of speech? \\
How many kinds were spoken because of indecent speech? \\
How many kinds because of matchmaking? 

There\marginnote{20.1} are three kinds of offenses for killing living beings,\footnote{Sp 5.475: \textit{\textsanskrit{Pāṇātipāte} tissoti “anodissa \textsanskrit{opātaṁ} \textsanskrit{khaṇati}, sace manusso marati, \textsanskrit{pārājikaṁ}; \textsanskrit{yakkhapetānaṁ} \textsanskrit{maraṇe} \textsanskrit{thullaccayaṁ}; \textsanskrit{tiracchānagatassa} \textsanskrit{maraṇe} \textsanskrit{pācittiya}”nti}, “‘Three for killing living beings’: if one digs a non-specific pit, and if a human being dies, there is an offense entailing expulsion. If a spirit or ghost dies, there is a serious offense. If an animal dies, there is an offense of wrong conduct.” } \\
There are three offenses entailing expulsion because of speech;\footnote{This refers to \href{https://suttacentral.net/pli-tv-bi-vb-pj6/en/brahmali\#1.23.1}{Bi Pj 6:1.23.1}, \href{https://suttacentral.net/pli-tv-bi-vb-pj7/en/brahmali\#1.11.1}{Bi Pj 7:1.11.1}, and \href{https://suttacentral.net/pli-tv-bi-vb-pj8/en/brahmali\#1.11.1}{Bi Pj 8:1.11.1}. } \\
Three kinds were spoken because of indecent speech,\footnote{Sp 5.475: \textit{\textsanskrit{Obhāsanā} tayoti \textsanskrit{vaccamaggaṁ} \textsanskrit{passāvamaggaṁ} \textsanskrit{ādissa} \textsanskrit{vaṇṇāvaṇṇabhāsane} \textsanskrit{saṅghādiseso}, \textsanskrit{vaccamaggaṁ} \textsanskrit{passāvamaggaṁ} \textsanskrit{ṭhapetvā} \textsanskrit{adhakkhakaṁ} \textsanskrit{ubbhajāṇumaṇḍalaṁ} \textsanskrit{ādissa} \textsanskrit{vaṇṇāvaṇṇabhaṇane} \textsanskrit{thullaccayaṁ}, \textsanskrit{ubbhakkhakaṁ} \textsanskrit{adhojāṇumaṇḍalaṁ} \textsanskrit{ādissa} \textsanskrit{vaṇṇāvaṇṇabhaṇane} \textsanskrit{dukkaṭaṁ}}, “‘Three kinds because of indecent speech’: there is an offense entailing suspension for praising and disparaging the private parts. Apart from the private parts, there is a serious offense for praising and disparaging what is below the collar bone but above the knees. There is an offense of wrong conduct for praising and disparaging what is below the knees or above the collar bone.” All refer to \href{https://suttacentral.net/pli-tv-bu-vb-ss3/en/brahmali}{Bu Ss 3}. } \\
And three kinds because of matchmaking.\footnote{Sp 5.475: \textit{\textsanskrit{Sañcarittena} \textsanskrit{vā} tayoti \textsanskrit{paṭiggaṇhāti} \textsanskrit{vīmaṁsati} \textsanskrit{paccāharati}, \textsanskrit{āpatti} \textsanskrit{saṅghādisesassa}; \textsanskrit{paṭiggaṇhāti} \textsanskrit{vīmaṁsati} na \textsanskrit{paccāharati}, \textsanskrit{āpatti} thullaccayassa; \textsanskrit{paṭiggaṇhāti} na \textsanskrit{vīmaṁsati} na \textsanskrit{paccāharati}, \textsanskrit{āpatti} \textsanskrit{dukkaṭassāti}}, “‘And three kinds because of matchmaking’: there is an offense entailing suspension if one accepts the mission, finds out the response, and reports back. There is a serious offense if one accepts the mission, finds out the response, but does not report back. There is an offense of wrong conduct if one accepts the mission, but does not find out the response, nor report back.” All three refer to \href{https://suttacentral.net/pli-tv-bu-vb-ss5/en/brahmali\#2.2.13.1}{Bu Ss 5:2.2.13.1}. } 

How\marginnote{21.1} many kinds of people should not be ordained? \\
How many things bring the legal procedures together? \\
How many kinds are said to be expelled? \\
How many are included in a single proclamation? 

Three\marginnote{22.1} kinds of people should not be ordained,\footnote{Sp 5.475: \textit{Tayo \textsanskrit{puggalā} na \textsanskrit{upasampādetabbāti} \textsanskrit{addhānahīno} \textsanskrit{aṅgahīno} vatthuvipanno ca \textsanskrit{tesaṁ} \textsanskrit{nānākaraṇaṁ} vuttameva}, “‘Three kinds of people should not be ordained’: one lacking in age, one lacking in limbs, and one deficient as object. The difference between them has been spoken of.” See \href{https://suttacentral.net/pli-tv-pvr7/en/brahmali\#9.2}{Pvr 7:9.2} and \href{https://suttacentral.net/pli-tv-pvr7/en/brahmali\#63.28}{Pvr 7:63.28}. } \\
Three things bring the legal procedures together;\footnote{Sp 5.475: \textit{Tayo \textsanskrit{kammānaṁ} \textsanskrit{saṅgahāti} … Aparehipi \textsanskrit{tīhi} \textsanskrit{kammāni} \textsanskrit{saṅgayhanti} – \textsanskrit{vatthunā}, \textsanskrit{ñattiyā}, \textsanskrit{anussāvanāyāti}. \textsanskrit{Vatthusampannañhi} \textsanskrit{ñattisampannaṁ} \textsanskrit{anussāvanasampannañca} \textsanskrit{kammaṁ} \textsanskrit{nāma} hoti, tena \textsanskrit{vuttaṁ} “tayo \textsanskrit{kammānaṁ} \textsanskrit{saṅgahā}”ti}, “‘Three things bring the legal procedures together’: … The legal procedures are also brought together through three other things: the object, the motion, and the announcement. For it is called a legal procedure when endowed with object, motion, and announcement. Because of that it is said, ‘Three things bring the legal procedures together.’” I have elided the first explanation in the commentary because of obscurity. } \\
Three kinds are said to be expelled,\footnote{Sp 5.475: \textit{\textsanskrit{Nāsitakā} tayo \textsanskrit{nāma} \textsanskrit{mettiyaṁ} \textsanskrit{bhikkhuniṁ} \textsanskrit{nāsetha}, \textsanskrit{dūsako} \textsanskrit{nāsetabbo}, \textsanskrit{dasahaṅgehi} \textsanskrit{samannāgato} \textsanskrit{sāmaṇero} \textsanskrit{nāsetabbo}, \textsanskrit{kaṇṭakaṁ} \textsanskrit{samaṇuddesaṁ} \textsanskrit{nāsethāti} \textsanskrit{evaṁ} \textsanskrit{liṅgasaṁvāsadaṇḍakammanāsanāvasena} tayo \textsanskrit{nāsitakā} \textsanskrit{veditabbā}}, “‘Three kinds are expelled’: ‘Expel the nun \textsanskrit{Mettiyā}’; a rapist is to be expelled; a novice monk who has ten qualities is to be expelled, ‘Expel the novice monk \textsanskrit{Kaṇṭaka}.’ In this way, it is to be understood that three kinds of persons are to be expelled: expulsion because of their characteristics, from community, and as a penalty.” Sp 1.384: \textit{Tattha tisso \textsanskrit{nāsanā} – \textsanskrit{liṅganāsanā}, \textsanskrit{saṁvāsanāsanā}, \textsanskrit{daṇḍakammanāsanāti}. \textsanskrit{Tāsu} “\textsanskrit{dūsako} \textsanskrit{nāsetabbo}”ti \textsanskrit{ayaṁ} “\textsanskrit{liṅganāsanā}”. \textsanskrit{Āpattiyā} adassane \textsanskrit{vā} \textsanskrit{appaṭikamme} \textsanskrit{vā} \textsanskrit{pāpikāya} \textsanskrit{diṭṭhiyā} \textsanskrit{appaṭinissagge} \textsanskrit{vā} \textsanskrit{ukkhepanīyakammaṁ} karonti, \textsanskrit{ayaṁ} “\textsanskrit{saṁvāsanāsanā}”. “Cara pire \textsanskrit{vinassā}”ti \textsanskrit{daṇḍakammaṁ} karonti, \textsanskrit{ayaṁ} “\textsanskrit{daṇḍakammanāsanā}”. Idha pana \textsanskrit{liṅganāsanaṁ} \textsanskrit{sandhāyāha} – “\textsanskrit{mettiyaṁ} \textsanskrit{bhikkhuniṁ} \textsanskrit{nāsethā}”ti}, “There are three kinds of expulsion: expulsion because of characteristic, expulsion from the community, and expulsion as penalty. Among these, ‘A rapist is to be expelled’—this is expulsion because of characteristic. Doing a legal procedure of ejection for not recognizing an offense, for not making amends, or for not giving up a bad view—this is expulsion from the community. Giving a penalty, saying, ‘Go! Away with you!’—this is expulsion as penalty. But in this case, ‘Expel the nun \textsanskrit{Mettiyā},’ was said on account of expulsion because of characteristic.” For the expulsion of the nun \textsanskrit{Mettiyā}, see \href{https://suttacentral.net/pli-tv-bu-vb-ss8/en/brahmali\#1.9.13}{Bu Ss 8:1.9.13}. For the rapist, see \href{https://suttacentral.net/pli-tv-kd1/en/brahmali\#67.1.12}{Kd 1:67.1.12}. For the ten qualities of a novice monk to be expelled, see \href{https://suttacentral.net/pli-tv-kd1/en/brahmali\#60.1.5}{Kd 1:60.1.5}–60.1.15. } \\
Three are included in a single proclamation.\footnote{Sp 5.475: \textit{\textsanskrit{Tiṇṇannaṁ} \textsanskrit{ekavācikāti} “\textsanskrit{anujānāmi} bhikkhave dve tayo \textsanskrit{ekānussāvane} \textsanskrit{kātu}”nti vacanato \textsanskrit{tiṇṇaṁ} \textsanskrit{janānaṁ} \textsanskrit{ekupajjhāyena} \textsanskrit{nānācariyena} \textsanskrit{ekānussāvanā} \textsanskrit{vaṭṭati}}, “‘Three are included in a single proclamation’: a single proclamation for three people with a single preceptor and many teachers is allowed because of the saying, ‘Monks, I allow you to give the full ordination to two or three with a single proclamation.’” For the context of the quoted part, see \href{https://suttacentral.net/pli-tv-kd1/en/brahmali\#74.3.7}{Kd 1:74.3.7}. } 

How\marginnote{23.1} many kinds of offenses are there for stealing? \\
How many because of sexual intercourse? \\
How many kinds of offenses for cutting? \\
How many because of discarding? 

There\marginnote{24.1} are three kinds of offenses for stealing,\footnote{Sp 5.475: \textit{\textsanskrit{Adinnādāne} tissoti \textsanskrit{pāde} \textsanskrit{vā} \textsanskrit{atirekapāde} \textsanskrit{vā} \textsanskrit{pārājikaṁ}, \textsanskrit{atirekamāsake} \textsanskrit{thullaccayaṁ}, \textsanskrit{māsake} \textsanskrit{vā} \textsanskrit{ūnamāsake} \textsanskrit{vā} \textsanskrit{dukkaṭaṁ}}, “‘Three for stealing’: for a \textit{\textsanskrit{pāda}} coin or more than a \textit{\textsanskrit{pāda}}, there is an offense entailing expulsion. For more than a \textit{\textsanskrit{māsaka}}, there is a serious offense. For a \textit{\textsanskrit{māsaka}} or less than a \textit{\textsanskrit{māsaka}}, there is an offense of wrong conduct.” This all refers to Bu Pj 2, respectively at \href{https://suttacentral.net/pli-tv-bu-vb-pj2/en/brahmali\#3.19}{Bu Pj 2:3.19}, \href{https://suttacentral.net/pli-tv-bu-vb-pj2/en/brahmali\#6.1.12}{Bu Pj 2:6.1.12}, and \href{https://suttacentral.net/pli-tv-bu-vb-pj2/en/brahmali\#6.1.20}{Bu Pj 2:6.1.20}. } \\
Four because of sexual intercourse;\footnote{Sp 5.475: \textit{Catasso \textsanskrit{methunapaccayāti} akkhayite \textsanskrit{pārājikaṁ}, yebhuyyena khayite \textsanskrit{thullaccayaṁ}, \textsanskrit{vivaṭakate} mukhe \textsanskrit{dukkaṭaṁ}, \textsanskrit{jatumaṭṭhake} \textsanskrit{pācittiyaṁ}}, “‘Four because of sexual intercourse’: there is an offense entailing expulsion for an undecomposed corpse. There is a serious offense for a mostly decomposed corpse. There is an offense of wrong conduct for an open mouth. There is an offense entailing confession for using a dildo.” The first three of these refer to \href{https://suttacentral.net/pli-tv-bu-vb-pj1/en/brahmali\#9.3.20}{Bu Pj 1:9.3.20}, \href{https://suttacentral.net/pli-tv-bu-vb-pj1/en/brahmali\#9.3.23}{Bu Pj 1:9.3.23}, and \href{https://suttacentral.net/pli-tv-bu-vb-pj1/en/brahmali\#10.13.15}{Bu Pj 1:10.13.15}, respectively. The fourth refers to \href{https://suttacentral.net/pli-tv-bi-vb-pc4/en/brahmali\#1.21.1}{Bi Pc 4:1.21.1}. } \\
There are three kinds of offenses for cutting,\footnote{Sp 5.475: \textit{Chindantassa tissoti \textsanskrit{vanappatiṁ} chindantassa \textsanskrit{pārājikaṁ}, \textsanskrit{bhūtagāme} \textsanskrit{pācittiyaṁ}, \textsanskrit{aṅgajāte} \textsanskrit{thullaccayaṁ}}, “‘Three for cutting’: there is an offense entailing expulsion for cutting a forest tree; an offense entailing confession for cutting vegetation; a serious offense for cutting off one’s penis.” The first of these refers to \href{https://suttacentral.net/pli-tv-bu-vb-pj2/en/brahmali\#4.18.1}{Bu Pj 2:4.18.1}; the second to \href{https://suttacentral.net/pli-tv-bu-vb-pc11/en/brahmali\#1.29.1}{Bu Pc 11:1.29.1}; and the third to \href{https://suttacentral.net/pli-tv-kd15/en/brahmali\#7.1.4}{Kd 15:7.1.4}. } \\
Five because of discarding.\footnote{Sp 5.475: \textit{\textsanskrit{Pañca} \textsanskrit{chaḍḍitapaccayāti} anodissa \textsanskrit{visaṁ} \textsanskrit{chaḍḍeti}, sace tena manusso marati, \textsanskrit{pārājikaṁ}; yakkhapetesu \textsanskrit{thullaccayaṁ}; \textsanskrit{tiracchānagate} \textsanskrit{pācittiyaṁ}; \textsanskrit{vissaṭṭhichaḍḍane} \textsanskrit{saṅghādiseso}; sekhiyesu harite \textsanskrit{uccārapassāvachaḍḍane} \textsanskrit{dukkaṭaṁ} – \textsanskrit{imā} \textsanskrit{chaḍḍitapaccayā} \textsanskrit{pañcāpattiyo} honti}, “‘Five because of discarding’: if one puts out poison without specific reference, and if a human being dies because of that, one commits an offense entailing expulsion. For a spirit or ghost, there is a serious offense. For an animal, there is an offense entailing confession. In discarding through emission, there is an offense entailing suspension. In the rules to be trained in, if one discards feces or urine on crops, there is an offense of wrong conduct.” The first three of these refers to \href{https://suttacentral.net/pli-tv-bu-vb-pj3/en/brahmali\#4.7.2}{Bu Pj 3:4.7.2}, \href{https://suttacentral.net/pli-tv-bu-vb-pj3/en/brahmali\#4.5.11}{Bu Pj 3:4.5.11}, and \href{https://suttacentral.net/pli-tv-bu-vb-pj3/en/brahmali\#4.5.14}{Bu Pj 3:4.5.14}, respectively. The fourth concerns \href{https://suttacentral.net/pli-tv-bu-vb-ss1/en/brahmali\#2.1.13.1}{Bu Ss 1:2.1.13.1}, and the fifth refers to \href{https://suttacentral.net/pli-tv-bu-vb-sk74/en/brahmali\#1.3.1}{Bu Sk 74:1.3.1}. } 

In\marginnote{25.1} the subchapter on the instructor of nuns, \\
Is there an offense of wrong conduct together with an offense entailing confession? \\
How many groups of nine are mentioned there? \\
And how many about robes? 

In\marginnote{26.1} the subchapter on the instructor of nuns, \\
There is one offense of wrong conduct together with one offense entailing confession;\footnote{Sp 5.475: \textit{\textsanskrit{Pācittiyena} \textsanskrit{dukkaṭā} \textsanskrit{katāti} \textsanskrit{bhikkhunovādakavaggasmiṁ} dasasu \textsanskrit{sikkhāpadesu} \textsanskrit{pācittiyena} \textsanskrit{saddhiṁ} \textsanskrit{dukkaṭā} \textsanskrit{katā} \textsanskrit{evāti} attho}, “‘There is one offense of wrong conduct together with one offense entailing confession’: in the ten training rules in the subchapter on the instructor of the nuns, there is an offense of wrong conduct together with an offense entailing confession. This is the meaning.” \textit{\textsanskrit{Dukkaṭā} \textsanskrit{katā}} would seem to be plural, but according to the sub-commentary, it should be read as singular. Vjb 5.475: \textit{\textsanskrit{Dukkaṭā} \textsanskrit{katāti} \textsanskrit{dukkaṭaṁ} \textsanskrit{vuttaṁ}}, “\textit{\textsanskrit{Dukkaṭā} \textsanskrit{katā}}: an offense of wrong conduct is said.” This refers to \href{https://suttacentral.net/pli-tv-bu-vb-pc21/en/brahmali\#1.44.1}{Bu Pc 21:1.44.1}–30. } \\
Four groups of nine are mentioned there,\footnote{Sp 5.475: \textit{Caturettha \textsanskrit{navakā} \textsanskrit{vuttāti} \textsanskrit{paṭhamasikkhāpadamhiyeva} adhammakamme dve, dhammakamme dveti \textsanskrit{evaṁ} \textsanskrit{cattāro} \textsanskrit{navakā} \textsanskrit{vuttāti} attho}, “‘Four groups of nine are mentioned there’: in the first training rule, there are two illegitimate and two legitimate legal procedures. In this way, it said that there are four groups of nine. This is the meaning.” This refers to \href{https://suttacentral.net/pli-tv-bu-vb-pc21/en/brahmali\#3.2.1}{Bu Pc 21:3.2.1}. } \\
And two about robes.\footnote{Sp 5.475: \textit{\textsanskrit{Dvinnaṁ} \textsanskrit{cīvarena} \textsanskrit{cāti} \textsanskrit{bhikkhūnaṁ} santike \textsanskrit{upasampannāya} \textsanskrit{cīvaraṁ} dentassa \textsanskrit{pācittiyaṁ}, \textsanskrit{bhikkhunīnaṁ} santike \textsanskrit{upasampannāya} dentassa \textsanskrit{dukkaṭanti} \textsanskrit{evaṁ} \textsanskrit{dvinnaṁ} \textsanskrit{bhikkhunīnaṁ} \textsanskrit{cīvaraṁ} dentassa \textsanskrit{cīvarena} \textsanskrit{kāraṇabhūtena} \textsanskrit{āpatti} \textsanskrit{hotīti} attho}, “‘And two about robes’: for one giving a robe to a nun who is fully ordained in the presence of the monks, there is an offense entailing confession. For one giving to a nun who is fully ordained in the presence of the nuns, there is an offense of wrong conduct.” This refers to \href{https://suttacentral.net/pli-tv-bu-vb-pc25/en/brahmali\#2.11.1}{Bu Pc 25:2.11.1}. “In the presence of the nuns” presumably means that the nun is only ordained on one side, whereas “in the presence of the monks” must refer to one ordained on both sides. } 

How\marginnote{27.1} many offenses entailing acknowledgment \\
Have been declared to the nuns? \\
For one eating raw grain, \\
How many are the offenses of wrong conduct, together with the offense entailing confession? 

Eight\marginnote{28.1} offenses entailing acknowledgment \\
Have been declared to the nuns; \\
For one eating raw grain,\footnote{Sp 5.475: \textit{\textsanskrit{Bhuñjantāmakadhaññena} \textsanskrit{pācittiyena} \textsanskrit{dukkaṭā} \textsanskrit{katāti} \textsanskrit{āmakadhaññaṁ} \textsanskrit{viññāpetvā} \textsanskrit{bhuñjantiyā} \textsanskrit{pācittiyena} \textsanskrit{saddhiṁ} \textsanskrit{dukkaṭā} \textsanskrit{katāyeva}}, “‘For one eating raw grain, there is an offense of wrong conduct together with an offense entailing confession’: for a nun who asks for raw grain and then eats it, there is an offense of wrong conduct together with an offense entailing confession.” See \href{https://suttacentral.net/pli-tv-bi-vb-pc7/en/brahmali\#1.14.1}{Bi Pc 7:1.14.1}. } \\
There is one offense of wrong conduct together with one offense entailing confession. 

How\marginnote{29.1} many kinds of offenses are there for one who is traveling? \\
And how many are there for standing? \\
How many kinds of offenses are there for sitting? \\
And how many are there for lying down? 

There\marginnote{30.1} are four kinds of offenses for one who is traveling,\footnote{Sp 5.475: \textit{Gacchantassa catassoti \textsanskrit{bhikkhuniyā} \textsanskrit{vā} \textsanskrit{mātugāmena} \textsanskrit{vā} \textsanskrit{saddhiṁ} \textsanskrit{saṁvidhāya} gacchantassa \textsanskrit{dukkaṭaṁ}, \textsanskrit{gāmūpacārokkamane} \textsanskrit{pācittiyaṁ}, \textsanskrit{yā} \textsanskrit{bhikkhunī} \textsanskrit{ekā} \textsanskrit{gāmantaraṁ} gacchati, \textsanskrit{tassā} \textsanskrit{gāmūpacāraṁ} \textsanskrit{okkamantiyā} \textsanskrit{paṭhamapāde} \textsanskrit{thullaccayaṁ}, \textsanskrit{dutiyapāde} \textsanskrit{saṅghādisesoti}}, “‘There are four for one who is traveling’: for one going by arrangement with a nun or a woman, there is an offense of wrong conduct. When entering the vicinity of a village, there is an offense entailing confession. If a nun walks to the next inhabited area by herself, there is a serious offense when she enters the vicinity of the inhabited area with her first foot. There is an offense entailing suspension when she enters with the second foot.” For the first two of these, see \href{https://suttacentral.net/pli-tv-bu-vb-pc27/en/brahmali\#2.15.1}{Bu Pc 27:2.15.1} and \href{https://suttacentral.net/pli-tv-bu-vb-pc67/en/brahmali\#1.28.1}{Bu Pc 67:1.28.1}. For the remaining two, see \href{https://suttacentral.net/pli-tv-bi-vb-ss3/en/brahmali\#4.14.1}{Bi Ss 3:4.14.1}. } \\
And the same number for standing;\footnote{Sp 5.475: \textit{Ṭhitassa \textsanskrit{cāpi} \textsanskrit{tattakāti} \textsanskrit{ṭhitassapi} catasso \textsanskrit{evāti} attho. \textsanskrit{Kathaṁ}? \textsanskrit{Bhikkhunī} \textsanskrit{andhakāre} \textsanskrit{vā} \textsanskrit{paṭicchanne} \textsanskrit{vā} \textsanskrit{okāse} mittasanthavavasena purisassa \textsanskrit{hatthapāse} \textsanskrit{tiṭṭhati}, \textsanskrit{pācittiyaṁ}; \textsanskrit{hatthapāsaṁ} \textsanskrit{vijahitvā} \textsanskrit{tiṭṭhati}, \textsanskrit{dukkaṭaṁ}; \textsanskrit{aruṇuggamanakāle} \textsanskrit{dutiyikāya} \textsanskrit{hatthapāsaṁ} \textsanskrit{vijahantī} \textsanskrit{tiṭṭhati}, \textsanskrit{thullaccayaṁ}; \textsanskrit{vijahitvā} \textsanskrit{tiṭṭhati}, \textsanskrit{saṅghādisesoti}}, “‘And the same number for standing’: the meaning is that also for standing there are just four. How? If a nun, on account of close friendship, stands within arm’s reach of a man in a dark or secluded place, there is an offense entailing confession. If she stands outside of arm’s reach, there is an offense of wrong conduct. If, at the time of dawn, she is in the process of going beyond arm’s reach of her companion, there is a serious offense. If she stands outside of arm’s reach, there is an offense entailing suspension.” For the first two of these, see \href{https://suttacentral.net/pli-tv-bi-vb-pc11/en/brahmali\#1.12.1}{Bi Pc 11:1.12.1} and \href{https://suttacentral.net/pli-tv-bi-vb-pc12/en/brahmali\#1.13.1}{Bi Pc 12:1.13.1}. For the remaining two, see \href{https://suttacentral.net/pli-tv-bi-vb-ss3/en/brahmali\#4.14.1}{Bi Ss 3:4.14.1}. } \\
There are four kinds of offenses for sitting down,\footnote{Sp 5.475: \textit{Nisinnassa catasso \textsanskrit{āpattiyo} \textsanskrit{nipannassāpi} \textsanskrit{tattakāti} sacepi hi \textsanskrit{sā} \textsanskrit{nisīdati} \textsanskrit{vā} nipajjati \textsanskrit{vā}, \textsanskrit{etāyeva} catasso \textsanskrit{āpattiyo} \textsanskrit{āpajjati}}, “‘There are four kinds of offenses for sitting, and the same number for lying down’: for even if she sits down or lies down, she commits these four offenses.” } \\
And the same number for lying down.” 

%
\end{verse}

\section*{3. Offenses entailing confession }

\begin{verse}%
“How\marginnote{31.1} many offenses entailing confession, \\
All with different bases, \\
Might one commit together, \\
All at the same time? 

There\marginnote{32.1} are five offenses entailing confession,\footnote{Sp 5.476: \textit{\textsanskrit{Pañca} \textsanskrit{pācittiyānīti} \textsanskrit{pañca} \textsanskrit{bhesajjāni} \textsanskrit{paṭiggahetvā} \textsanskrit{nānābhājanesu} \textsanskrit{vā} \textsanskrit{ekabhājane} \textsanskrit{vā} \textsanskrit{amissetvā} \textsanskrit{ṭhapitāni} honti, \textsanskrit{sattāhātikkame} so bhikkhu \textsanskrit{pañca} \textsanskrit{pācittiyāni} \textsanskrit{sabbāni} \textsanskrit{nānāvatthukāni} \textsanskrit{ekakkhaṇe} \textsanskrit{āpajjati}}, “‘There are five offenses entailing confession’: having received the five tonics, not mixing them in one or many vessels, and setting them aside, then, when seven days have passed, that monk commits five offenses entailing confession, all with different bases, in one instant.” } \\
All with different bases; \\
That one might commit together, \\
All at the same time. 

How\marginnote{33.1} many offenses entailing confession, \\
All with different bases, \\
Might one commit together, \\
All at the same time? 

There\marginnote{34.1} are nine offenses entailing confession,\footnote{Sp 5.476: \textit{Nava \textsanskrit{pācittiyānīti} yo bhikkhu nava \textsanskrit{paṇītabhojanāni} \textsanskrit{viññāpetvā} tehi \textsanskrit{saddhiṁ} ekato \textsanskrit{ekaṁ} \textsanskrit{kabaḷaṁ} \textsanskrit{omadditvā} mukhe \textsanskrit{pakkhipitvā} \textsanskrit{paragaḷaṁ} \textsanskrit{atikkāmeti}, \textsanskrit{ayaṁ} nava \textsanskrit{pācittiyāni} \textsanskrit{sabbāni} \textsanskrit{nānāvatthukāni} \textsanskrit{ekakkhaṇe} \textsanskrit{āpajjati}}, “‘There are nine offenses entailing confession’: if a monk has asked for the nine fine foods, and then, having pressed them together into a single mouthful, having placed them in his mouth, he lets it pass beyond the throat. He then commits nine offenses entailing confession, all with different bases, in one instant.” } \\
All with different bases; \\
That one might commit together, \\
All at the same time. 

How\marginnote{35.1} many offenses entailing confession, \\
All with different bases, \\
Should be confessed through how many statements, \\
As spoken by the Kinsman of the Sun? 

There\marginnote{36.1} are five offenses entailing confession, \\
All with different bases; \\
That should be confessed through a single statement,\footnote{Sp 5.476: \textit{\textsanskrit{Ekavācāya} \textsanskrit{deseyyāti} “\textsanskrit{ahaṁ}, bhante, \textsanskrit{pañca} \textsanskrit{bhesajjāni} \textsanskrit{paṭiggahetvā} \textsanskrit{sattāhaṁ} \textsanskrit{atikkāmetvā} \textsanskrit{pañca} \textsanskrit{āpattiyo} \textsanskrit{āpanno}, \textsanskrit{tā} \textsanskrit{tumhamūle} \textsanskrit{paṭidesemī}”ti \textsanskrit{evaṁ} \textsanskrit{ekavācāya} deseyya}, “‘That should be confessed through a single statement’: ‘Venerable, having received the five tonics, having gone beyond seven days, I have committed five offenses. I confess them to you.’ In this way, one should confess through a single statement.” } \\
As spoken by the Kinsman of the Sun. 

How\marginnote{37.1} many offenses entailing confession, \\
All with different bases, \\
Should be confessed by how many statements, \\
As spoken by the Kinsman of the Sun? 

There\marginnote{38.1} are nine offenses entailing confession, \\
All with different bases; \\
That should be confessed by one statement,\footnote{This follows the same pattern as above, but substituting the nine fine foods for the five tonics. } \\
As spoken by the Kinsman of the Sun. 

How\marginnote{39.1} many offenses entailing confession, \\
All with different bases, \\
Should one describe and then confess, \\
As spoken by the Kinsman of the Sun? 

There\marginnote{40.1} are five offenses entailing confession, \\
All with different bases; \\
Where one should describe the basis and then confess,\footnote{Sp 5.476: \textit{\textsanskrit{Vatthuṁ} \textsanskrit{kittetvā} \textsanskrit{deseyyāti} “\textsanskrit{ahaṁ}, bhante, \textsanskrit{pañca} \textsanskrit{bhesajjāni} \textsanskrit{paṭiggahetvā} \textsanskrit{sattāhaṁ} \textsanskrit{atikkāmesiṁ}, \textsanskrit{yathāvatthukaṁ} \textsanskrit{taṁ} \textsanskrit{tumhamūle} \textsanskrit{paṭidesemī}”ti \textsanskrit{evaṁ} \textsanskrit{vatthuṁ} \textsanskrit{kittetvā} deseyya, \textsanskrit{desitāva} honti \textsanskrit{āpattiyo}, \textsanskrit{āpattiyā} \textsanskrit{nāmaggahaṇena} \textsanskrit{kiccaṁ} natthi. Dutiyavissajjanepi “‘\textsanskrit{ahaṁ}, bhante, nava \textsanskrit{paṇītabhojanāni} \textsanskrit{viññāpetvā} bhutto, \textsanskrit{yathāvatthukaṁ} \textsanskrit{taṁ} \textsanskrit{tumhamūle} \textsanskrit{paṭidesemī}”ti \textsanskrit{vattabbaṁ}}, “‘Where one should describe the basis and then confess’: ‘Venerable, having received the five tonics, I went beyond seven days. I confess it to you according to basis.’ In this way, one should describe the basis and then confess.” } \\
As spoken by the Kinsman of the Sun. 

How\marginnote{41.1} many offenses entailing confession, \\
All with different bases, \\
Should one describe and then confess, \\
As spoken by the Kinsman of the Sun? 

There\marginnote{42.1} are nine offenses entailing confession, \\
All with different bases; \\
Where one should describe the basis and then confess,\footnote{As above, but substituting the nine fine foods for the five tonics. } \\
As spoken by the Kinsman of the Sun. 

How\marginnote{43.1} many offenses are there after the third announcement? \\
How many are there because of speech? \\
How many are there for one who is eating? \\
And how many because of cooked food? 

There\marginnote{44.1} are three offenses after the third announcement,\footnote{Sp 5.476: \textit{\textsanskrit{Yāvatatiyake} tissoti \textsanskrit{ukkhittānuvattikāya} \textsanskrit{pārājikaṁ} \textsanskrit{bhedakānuvattakānaṁ} \textsanskrit{kokālikādīnaṁ} \textsanskrit{saṅghādisesaṁ}, \textsanskrit{pāpikāya} \textsanskrit{diṭṭhiyā} \textsanskrit{appaṭinissagge} \textsanskrit{caṇḍakāḷikāya} ca \textsanskrit{bhikkhuniyā} \textsanskrit{pācittiyanti}}, “‘Three after the third announcement’: there is an offense entailing expulsion for a nun who sides with one who has been ejected. There is an offense entailing suspension for supporting a schismatic, as in the case of \textsanskrit{Kokālika}, etc. There is an offense entailing confession for a nun, such as \textsanskrit{Caṇḍakāḷikā}, not to give up a bad view.” } \\
Six because of speech;\footnote{Sp 5.476: \textit{Cha \textsanskrit{vohārapaccayāti} \textsanskrit{payuttavācāpaccayā} cha \textsanskrit{āpattiyo} \textsanskrit{āpajjatīti} attho. \textsanskrit{Kathaṁ}? \textsanskrit{Ājīvahetu} \textsanskrit{ājīvakāraṇā} \textsanskrit{pāpiccho} \textsanskrit{icchāpakato} \textsanskrit{asantaṁ} \textsanskrit{abhūtaṁ} \textsanskrit{uttarimanussadhammaṁ} ullapati, \textsanskrit{āpatti} \textsanskrit{pārājikassa}. \textsanskrit{Ājīvahetu} \textsanskrit{ājīvakāraṇā} \textsanskrit{sañcarittaṁ} \textsanskrit{samāpajjati}, \textsanskrit{āpatti} \textsanskrit{saṅghādisesassa}. \textsanskrit{Ājīvahetu} \textsanskrit{ājīvakāraṇā} yo te \textsanskrit{vihāre} vasati so \textsanskrit{arahāti} vadati, \textsanskrit{āpatti} thullaccayassa. \textsanskrit{Ājīvahetu} \textsanskrit{ājīvakāraṇā} bhikkhu \textsanskrit{paṇītabhojanāni} attano \textsanskrit{atthāya} \textsanskrit{viññāpetvā} \textsanskrit{bhuñjati}, \textsanskrit{āpatti} \textsanskrit{pācittiyassa}. \textsanskrit{Ājīvahetu} \textsanskrit{ājīvakāraṇā} \textsanskrit{bhikkhunī} \textsanskrit{paṇītabhojanāni} attano \textsanskrit{atthāya} \textsanskrit{viññāpetvā} \textsanskrit{bhuñjati}, \textsanskrit{āpatti} \textsanskrit{pāṭidesanīyassa}. \textsanskrit{Ājīvahetu} \textsanskrit{ājīvakāraṇā} \textsanskrit{sūpaṁ} \textsanskrit{vā} \textsanskrit{odanaṁ} \textsanskrit{vā} \textsanskrit{agilāno} attano \textsanskrit{atthāya} \textsanskrit{viññāpetvā} \textsanskrit{bhuñjati}, \textsanskrit{āpatti} \textsanskrit{dukkaṭassāti}}, “‘Six because of speech’: if, because of livelihood, bad desires, and being overcome by desire, one claims a non-existent superhuman quality, there is an offense entailing expulsion. If, because of livelihood, one acts as a matchmaker, there is an offense entailing suspension. If, because of livelihood, one says, ‘The person staying in your dwelling is a perfected one’, there is a serious offense. If, because of livelihood, a monk eats fine foods that he has requested for himself, there is an offense entailing confession. If, because of livelihood, a nun eats fine foods that she has requested for herself, there is an offense entailing acknowledgment. If, because of livelihood, one who is not sick requests bean curry or rice for themselves and then eats it, there is an offense of wrong conduct.” The first and the third of these refer to \href{https://suttacentral.net/pli-tv-bu-vb-pj4/en/brahmali\#2.12.1}{Bu Pj 4:2.12.1} and \href{https://suttacentral.net/pli-tv-bu-vb-pj4/en/brahmali\#6.1.53}{Bu Pj 4:6.1.53}, the second to \href{https://suttacentral.net/pli-tv-bu-vb-ss5/en/brahmali\#2.2.13.1}{Bu Ss 5:2.2.13.1}, the fourth to \href{https://suttacentral.net/pli-tv-bu-vb-pc39/en/brahmali\#2.10.1}{Bu Pc 39:2.10.1}, the fifth to \href{https://suttacentral.net/pli-tv-bi-vb-pd1/en/brahmali\#1.2.9.1}{Bi Pd 1:1.2.9.1}–8, and the sixth to \href{https://suttacentral.net/pli-tv-bu-vb-sk37/en/brahmali\#2.10.1}{Bu Sk 37:2.10.1}. } \\
There are three offenses for one who is eating,\footnote{Sp 5.476: \textit{\textsanskrit{Khādantassa} tissoti \textsanskrit{manussamaṁse} \textsanskrit{thullaccayaṁ}, avasesesu \textsanskrit{akappiyamaṁsesu} \textsanskrit{dukkaṭaṁ}, \textsanskrit{bhikkhuniyā} \textsanskrit{lasuṇe} \textsanskrit{pācittiyaṁ}}, “‘Three for one who is eating’: there is a serious offense for eating human flesh. There is an offense of wrong conduct for the remaining unallowable meats. There is an offense entailing confession for a nun to eat garlic.” See respectively \href{https://suttacentral.net/pli-tv-kd6/en/brahmali\#23.9.7}{Kd 6:23.9.7}, \href{https://suttacentral.net/pli-tv-kd6/en/brahmali\#23.10.8}{Kd 6:23.10.8}–23.15.9, and \href{https://suttacentral.net/pli-tv-bi-vb-pc1/en/brahmali\#1.41.1}{Bi Pc 1:1.41.1}. } \\
Five because of cooked food.\footnote{Sp 5.476: \textit{\textsanskrit{Pañca} \textsanskrit{bhojanapaccayāti} \textsanskrit{avassutā} avassutassa purisassa hatthato \textsanskrit{bhojanaṁ} \textsanskrit{gahetvā} tattheva \textsanskrit{manussamaṁsaṁ} \textsanskrit{lasuṇaṁ} attano \textsanskrit{atthāya} \textsanskrit{viññāpetvā} \textsanskrit{gahitapaṇītabhojanāni} \textsanskrit{avasesañca} \textsanskrit{akappiyamaṁsaṁ} \textsanskrit{pakkhipitvā} \textsanskrit{vomissakaṁ} \textsanskrit{omadditvā} \textsanskrit{ajjhoharamānā} \textsanskrit{saṅghādisesaṁ}, \textsanskrit{thullaccayaṁ}, \textsanskrit{pācittiyaṁ}, \textsanskrit{pāṭidesanīyaṁ}, \textsanskrit{dukkaṭanti}}, “‘Five because of cooked food’: a lustful nun receives cooked food directly from a lustful man; just so human flesh; garlic; fine foods that one has requested for oneself; and the remaining unallowable meats—in putting it down, pressing it together, and swallowing there are respectively an offense entailing suspension, a serious offense, an offense entailing confession, an offense entailing acknowledgment, and an offense of wrong conduct.” The first of these refers to \href{https://suttacentral.net/pli-tv-bi-vb-ss5/en/brahmali\#1.14.1}{Bi Ss 5:1.14.1}, the second to \href{https://suttacentral.net/pli-tv-kd6/en/brahmali\#23.9.7}{Kd 6:23.9.7}, the third to \href{https://suttacentral.net/pli-tv-bi-vb-pc1/en/brahmali\#1.41.1}{Bi Pc 1:1.41.1}, the fourth to \href{https://suttacentral.net/pli-tv-bi-vb-pd1/en/brahmali\#1.2.9.1}{Bi Pd 1:1.2.9.1}–8, and the fifth to \href{https://suttacentral.net/pli-tv-kd6/en/brahmali\#23.10.8}{Kd 6:23.10.8}–23.15.9. I read \textit{tatheva} with Sp-yoj 5.476 instead of \textit{tattheva}. } 

Of\marginnote{45.1} all offenses after the third announcement, \\
How many cases are there? \\
And for how many are there offenses? \\
And for how many is there a legal issue? 

Of\marginnote{46.1} all offenses after the third announcement, \\
There are five cases;\footnote{Sp 5.476: \textit{\textsanskrit{Pañca} \textsanskrit{ṭhānānīti} “\textsanskrit{ukkhittānuvattikāya} \textsanskrit{bhikkhuniyā} \textsanskrit{yāvatatiyaṁ} \textsanskrit{samanubhāsanāya} \textsanskrit{appaṭinissajjantiyā} \textsanskrit{ñattiyā} \textsanskrit{dukkaṭaṁ}, \textsanskrit{dvīhi} \textsanskrit{kammavācāhi} \textsanskrit{thullaccayaṁ}, \textsanskrit{kammavācāpariyosāne} \textsanskrit{āpatti} \textsanskrit{pārājikassa}, \textsanskrit{saṅghabhedāya} \textsanskrit{parakkamanādīsu} \textsanskrit{saṅghādiseso}, \textsanskrit{pāpikāya} \textsanskrit{diṭṭhiyā} \textsanskrit{appaṭinissagge} \textsanskrit{pācittiya}”nti}, “‘There are five cases’: if a nun who sides with one who has been ejected does not stop when spoken to up to the third time, there is an offense of wrong conduct after the motion. After each of the first two announcements, there is a serious offense. When the last announcement is finished, there is an offense entailing expulsion. In pursuing schism in the Sangha, there is an offense entailing suspension. In not giving up a bad view, there is an offense entailing confession.” The first three of these refer to \href{https://suttacentral.net/pli-tv-bi-vb-pj7/en/brahmali\#1.11.1}{Bi Pj 7:1.11.1}, the fourth to \href{https://suttacentral.net/pli-tv-bu-vb-ss10/en/brahmali\#1.3.16.1}{Bu Ss 10:1.3.16.1}, and the fifth to \href{https://suttacentral.net/pli-tv-bu-vb-pc68/en/brahmali\#1.49.1}{Bu Pc 68:1.49.1}. } \\
And there are offenses for five,\footnote{Sp 5.476: \textit{\textsanskrit{Pañcannañceva} \textsanskrit{āpattīti} \textsanskrit{āpatti} \textsanskrit{nāma} \textsanskrit{pañcannaṁ} \textsanskrit{sahadhammikānaṁ} hoti, tattha \textsanskrit{dvinnaṁ} \textsanskrit{nippariyāyena} \textsanskrit{āpattiyeva}, \textsanskrit{sikkhāmānasāmaṇerisāmaṇerānaṁ} pana \textsanskrit{akappiyattā} na \textsanskrit{vaṭṭati}. \textsanskrit{Iminā} \textsanskrit{pariyāyena} \textsanskrit{tesaṁ} \textsanskrit{āpatti} na \textsanskrit{desāpetabbā}, \textsanskrit{daṇḍakammaṁ} pana \textsanskrit{tesaṁ} \textsanskrit{kātabbaṁ}}, “‘And there are offenses for five’: for the five co-monastics there are offenses. Therein, for two there are offenses with no ambiguity. But for trainee nuns, novice monks, and novice nuns the unallowable is not allowed. In this interpretation, there is no offense to be confessed for them, but a penalty may be imposed on them.” } \\
And legal issues for five.\footnote{Sp 5.476: \textit{\textsanskrit{Pañcannaṁ} \textsanskrit{adhikaraṇena} \textsanskrit{cāti} \textsanskrit{adhikaraṇañca} \textsanskrit{pañcannamevāti} attho. \textsanskrit{Etesaṁyeva} hi \textsanskrit{pañcannaṁ} \textsanskrit{pattacīvarādīnaṁ} \textsanskrit{atthāya} \textsanskrit{vinicchayavohāro} \textsanskrit{adhikaraṇanti} vuccati}, “‘And legal issues for five’: the meaning is a legal issue for five. For the speaking of a decision for the sake of a bowl or a robe, etc., to these five is called a legal issue.” The first sentence merely shows that the Canonical text should be read as an accusative rather than an instrumental. Sp-yoj 5.476: \textit{\textsanskrit{Pañcannamevāti} \textsanskrit{sahadhammikānaṁyeva}}, “For five: for one’s co-monastics.” The commentarial explanations seem to point to the procedures by which requisites are given out to monastics. } 

For\marginnote{47.1} how many are there decisions? \\
For how many are there resolutions? \\
For how many are there non-offenses? \\
For how many reasons does one shine? 

There\marginnote{48.1} are decisions for five,\footnote{Sp 5.476: \textit{\textsanskrit{Pañcannaṁ} vinicchayo \textsanskrit{hotīti} \textsanskrit{pañcannaṁ} \textsanskrit{sahadhammikānaṁyeva} vinicchayo \textsanskrit{nāma} hoti}, “‘There are decisions for five’: there are decisions for one’s five co-monastics.” } \\
And resolutions for five.\footnote{Sp 5.476: \textit{\textsanskrit{Pañcannaṁ} \textsanskrit{vūpasamena} \textsanskrit{cāti} \textsanskrit{etesaṁyeva} \textsanskrit{pañcannaṁ} \textsanskrit{adhikaraṇaṁ} \textsanskrit{vinicchitaṁ} \textsanskrit{vūpasantaṁ} \textsanskrit{nāma} \textsanskrit{hotīti} attho}, “‘And resolutions for five’: for these five there are legal issues, decisions, and resolutions. This is the meaning.” } \\
There are non-offenses for five,\footnote{Sp 5.476: \textit{\textsanskrit{Pañcannañceva} \textsanskrit{anāpattīti} \textsanskrit{etesaṁyeva} \textsanskrit{pañcannaṁ} \textsanskrit{anāpatti} \textsanskrit{nāma} \textsanskrit{hotīti} attho}, “‘There are non-offenses for five’: for these five there are non-offenses. This is the meaning.” } \\
And one shines for three reasons.\footnote{Sp 5.476: \textit{\textsanskrit{Tīhi} \textsanskrit{ṭhānehi} \textsanskrit{sobhatīti} \textsanskrit{saṅghādīhi} \textsanskrit{tīhi} \textsanskrit{kāraṇehi} sobhati. \textsanskrit{Katavītikkamo} hi puggalo \textsanskrit{sappaṭikammaṁ} \textsanskrit{āpattiṁ} \textsanskrit{saṅghamajjhe} \textsanskrit{gaṇamajjhe} puggalasantike \textsanskrit{vā} \textsanskrit{paṭikaritvā} \textsanskrit{abbhuṇhasīlo} \textsanskrit{pākatiko} hoti, \textsanskrit{tasmā} \textsanskrit{tīhi} \textsanskrit{ṭhānehi} \textsanskrit{sobhatīti} vuccati}, “‘And one shines for three reasons’: one shines for three reasons, that is, through the Sangha, etc. A person who has committed an offense makes amends for a curable offense in the middle of the Sangha, in the middle of a group, or to an individual. One has then refreshed one’s virtue and is restored to one’s natural state. For this reason one shines for three reasons.” Sp-\textsanskrit{ṭ} 5.476: \textit{\textsanskrit{Abbhuṇhasīloti} \textsanskrit{abhinavasīlo}}: “\textit{\textsanskrit{Abbhuṇhasīlo}}: fresh virtue.” } 

How\marginnote{49.1} many kinds of offenses are there by body at night? \\
How many are there by body by day? \\
How many kinds of offenses are there for staring? \\
How many because of almsfood? 

There\marginnote{50.1} are two kinds of offenses by body at night,\footnote{Sp 5.476: \textit{Dve \textsanskrit{kāyikā} rattinti \textsanskrit{bhikkhunī} \textsanskrit{rattandhakāre} purisassa \textsanskrit{hatthapāse} \textsanskrit{ṭhānanisajjasayanāni} \textsanskrit{kappayamānā} \textsanskrit{pācittiyaṁ}, \textsanskrit{hatthapāsaṁ} \textsanskrit{vijahitvā} \textsanskrit{ṭhānādīni} \textsanskrit{kappayamānā} \textsanskrit{dukkaṭanti}}, “‘There are two kinds of offenses by body at night’: there is an offense entailing confession for a nun who, in the dark of the night, stands, sits down, or lies down within arm’s reach a man. There is an offense of wrong conduct if she stands, etc., outside of arm’s reach.” See \href{https://suttacentral.net/pli-tv-bi-vb-pc11/en/brahmali\#1.12.1}{Bi Pc 11:1.12.1}. } \\
Two by body by day;\footnote{Sp 5.476: \textit{Dve \textsanskrit{kāyikā} \textsanskrit{divāti} eteneva \textsanskrit{upāyena} \textsanskrit{divā} \textsanskrit{paṭicchanne} \textsanskrit{okāse} dve \textsanskrit{āpattiyo} \textsanskrit{āpajjati}}, “‘Two by body by day’: by the same method, one commits two offenses in a concealed place by day.” See \href{https://suttacentral.net/pli-tv-bi-vb-pc12/en/brahmali\#1.13.1}{Bi Pc 12:1.13.1}. } \\
There is one kind of offense for staring,\footnote{Sp 5.476: \textit{\textsanskrit{Nijjhāyantassa} \textsanskrit{ekā} \textsanskrit{āpattīti} “na ca, bhikkhave, \textsanskrit{sārattena} \textsanskrit{mātugāmassa} \textsanskrit{aṅgajātaṁ} \textsanskrit{upanijjhāyitabbaṁ}. Yo \textsanskrit{upanijjhāyeyya}, \textsanskrit{āpatti} \textsanskrit{dukkaṭassā}”ti}, “‘There is one offense for staring’: ‘But you should not stare at a woman’s genitals motivated by lust. If you do, you commit an offense of wrong conduct.’” See \href{https://suttacentral.net/pli-tv-bu-vb-ss1/en/brahmali\#5.12.5}{Bu Ss 1:5.12.5}. } \\
One because of almsfood.\footnote{Sp 5.476: \textit{\textsanskrit{Ekā} \textsanskrit{piṇḍapātapaccayāti} “na ca, bhikkhave, \textsanskrit{bhikkhādāyikāya} \textsanskrit{mukhaṁ} oloketabba”nti ettha \textsanskrit{dukkaṭāpatti}}, “‘One because of almsfood’: ‘Monks, one shouldn’t look the donor in the face.’ Here there is an offense of wrong conduct.” See \href{https://suttacentral.net/pli-tv-kd18/en/brahmali\#5.2.27}{Kd 18:5.2.27}. } 

Seeing\marginnote{51.1} how many benefits \\
Should one confess out of confidence in others? \\
How many kinds are said to be ejected? \\
And how many are the proper conducts? 

Seeing\marginnote{52.1} eight benefits,\footnote{Sp 5.476: \textit{\textsanskrit{Aṭṭhānisaṁse} sampassanti kosambakakkhandhake \textsanskrit{vuttānisaṁse}}, “‘Seeing eight benefits’: the benefits mentioned in The Chapter Connected with \textsanskrit{Kosambī}.” According to Vmv 5.476 this refers to \href{https://suttacentral.net/pli-tv-kd10/en/brahmali\#1.8.6}{Kd 10:1.8.6}–1.8.16. } \\
One should confess out of confidence in others; \\
Three kinds are said to be ejected,\footnote{Sp 5.476: \textit{\textsanskrit{Ukkhittakā} tayo \textsanskrit{vuttāti} \textsanskrit{āpattiyā} adassane \textsanskrit{appaṭikamme} \textsanskrit{pāpikāya} ca \textsanskrit{diṭṭhiyā} \textsanskrit{appaṭinissaggeti}}, “‘Three kinds are said to be ejected’: one not recognizing an offense, one not making amends, one not giving up a bad view.” } \\
There are forty-three proper conducts.\footnote{Sp 5.476: \textit{\textsanskrit{Tecattālīsa} \textsanskrit{sammāvattanāti} \textsanskrit{tesaṁyeva} \textsanskrit{ukkhittakānaṁ} ettakesu vattesu \textsanskrit{vattanā}}, “‘There are forty-three proper conducts’: the conduct of those who have been ejected.” See \href{https://suttacentral.net/pli-tv-kd11/en/brahmali\#27.1.3}{Kd 11:27.1.3}–27.1.45. } 

How\marginnote{53.1} many cases of lying are there? \\
How many are called ‘at most’? \\
How many offenses entailing acknowledgment are there? \\
And for how many is there confession? 

There\marginnote{54.1} are five cases of lying,\footnote{Sp 5.476: \textit{\textsanskrit{Pañcaṭhāne} \textsanskrit{musāvādoti} \textsanskrit{pārājikasaṅghādisesathullaccayapācittiyadukkaṭasaṅkhāte} \textsanskrit{pañcaṭṭhāne} \textsanskrit{musāvādo} gacchati}, “‘There are five cases of lying’: lying happens in five cases, that is, in regard to offenses entailing expulsion, offenses entailing suspension, serious offenses, offenses entailing confession, and offenses of wrong conduct.” Examples of these, are as follows: the first and the third can refer to \href{https://suttacentral.net/pli-tv-bu-vb-pj4/en/brahmali\#2.12.1}{Bu Pj 4:2.12.1}, the second to \href{https://suttacentral.net/pli-tv-bu-vb-ss8/en/brahmali\#1.9.32.1}{Bu Ss 8:1.9.32.1}, the fourth to \href{https://suttacentral.net/pli-tv-bu-vb-pc1/en/brahmali\#1.20.1}{Bu Pc 1:1.20.1}, and the fifth to all of the above. } \\
Fourteen are called ‘at most’;\footnote{Sp 5.476: \textit{Cuddasa paramanti \textsanskrit{vuccatīti} \textsanskrit{dasāhaparamādinayena} \textsanskrit{heṭṭhā} \textsanskrit{vuttaṁ}}, “‘Fourteen are called “at most”’: by the method of ten days at the most, etc., mentioned earlier.” } \\
There are twelve offenses entailing acknowledgment,\footnote{Sp 5.476: \textit{\textsanskrit{Dvādasa} \textsanskrit{pāṭidesanīyāti} \textsanskrit{bhikkhūnaṁ} \textsanskrit{cattāri} \textsanskrit{bhikkhunīnaṁ} \textsanskrit{aṭṭha}}, “‘There are twelve offenses entailing acknowledgment’: four for the monks and eight for the nuns.” } \\
And there is confession for four.\footnote{Sp 5.476: \textit{\textsanskrit{Catunnaṁ} \textsanskrit{desanāya} \textsanskrit{cāti} \textsanskrit{catunnaṁ} \textsanskrit{accayadesanāyāti} attho. \textsanskrit{Katamā} pana \textsanskrit{sāti}? Devadattena \textsanskrit{payojitānaṁ} \textsanskrit{abhimārānaṁ} \textsanskrit{accayadesanā}, anuruddhattherassa \textsanskrit{upaṭṭhāyikāya} \textsanskrit{accayadesanā}, \textsanskrit{vaḍḍhassa} licchavino \textsanskrit{accayadesanā}, \textsanskrit{vāsabhagāmiyattherassa} \textsanskrit{ukkhepanīyakammaṁ} \textsanskrit{katvā} \textsanskrit{āgatānaṁ} \textsanskrit{bhikkhūnaṁ} \textsanskrit{accayadesanāti}}, “‘And there is confession for four’: the meaning is the confession of offenses for four. But what are they? For the assassins who were sent by Devadatta, there was a confession of offenses. For the female who attended on the senior monk Anuruddha, there was a confession of offenses. For \textsanskrit{Vaḍḍha} the \textsanskrit{Licchavī}, there was a confession of offenses. Having done a legal act of ejection against the senior monk in the village \textsanskrit{Vāsabha}, there was a confession of offenses for the arrived monks.” The first of these refers to \href{https://suttacentral.net/pli-tv-kd17/en/brahmali\#3.7.7}{Kd 17:3.7.7}–3.8.21, the second to \href{https://suttacentral.net/pli-tv-bu-vb-pc6/en/brahmali\#1.51.1}{Bu Pc 6:1.51.1}, the third to \href{https://suttacentral.net/pli-tv-kd15/en/brahmali\#20.5.12}{Kd 15:20.5.12}, and the fourth to \href{https://suttacentral.net/pli-tv-kd9/en/brahmali\#1.9.3}{Kd 9:1.9.3}. } 

How\marginnote{55.1} many factors does lying have? \\
And how many factors does the observance day have? \\
How many qualities does a qualified messenger have? \\
How many kinds of proper conduct are there for the monastics of other religions? 

Lying\marginnote{56.1} has eight factors,\footnote{Sp 5.476: \textit{\textsanskrit{Aṭṭhaṅgiko} \textsanskrit{musāvādoti} “pubbevassa hoti \textsanskrit{musā} \textsanskrit{bhaṇissa}”nti \textsanskrit{ādiṁ} \textsanskrit{katvā} “\textsanskrit{vinidhāya} \textsanskrit{sañña}”nti \textsanskrit{pariyosānehi}}, “‘Lying has eight factors’: starting with ‘before he has lied, he knows he is going to lie,’ and ending with ‘he misrepresents his perception of what’s true.’” See \href{https://suttacentral.net/pli-tv-bu-vb-pc1/en/brahmali\#2.2.10}{Bu Pc 1:2.2.10} and \href{https://suttacentral.net/pli-tv-pvr17/en/brahmali\#209.3}{Pvr 17:209.3}. } \\
The observance day has eight factors;\footnote{This refers to the eight precepts, see e.g. \href{https://suttacentral.net/an3.70/en/brahmali\#19.2}{AN 3.70:19.2}–26.2. } \\
A qualified messenger has eight qualities,\footnote{See \href{https://suttacentral.net/pli-tv-kd17/en/brahmali\#4.6.3}{Kd 17:4.6.3}. } \\
There are eight kinds of proper conduct for monastics of other religions.\footnote{See \href{https://suttacentral.net/pli-tv-kd1/en/brahmali\#38.8.2}{Kd 1:38.8.2}–38.10.2. According to Sp 3.87, the last item on this list counts as four separate practices, thus making eight in total. } 

How\marginnote{57.1} many statements are there for an ordination? \\
For how many should one get up? \\
To how many should one give a seat? \\
Through how many qualities is one an instructor of the nuns? 

There\marginnote{58.1} are eight statements for an ordination,\footnote{Sp 5.476: \textit{\textsanskrit{Aṭṭhavācikā} \textsanskrit{upasampadāti} \textsanskrit{bhikkhunīnaṁ} \textsanskrit{upasampadaṁ} \textsanskrit{sandhāya} \textsanskrit{vuttaṁ}}, “‘There are eight statements for an ordination’: this is said with reference to the ordination of nuns.” That is, one motion and three announcements in front of each Sangha. } \\
One should get up for eight;\footnote{Sp 5.476: \textit{\textsanskrit{Aṭṭhannaṁ} \textsanskrit{paccuṭṭhātabbanti} bhattagge \textsanskrit{aṭṭhannaṁ} \textsanskrit{bhikkhunīnaṁ} \textsanskrit{itarāhi} \textsanskrit{paccuṭṭhāya} \textsanskrit{āsanaṁ} \textsanskrit{dātabbaṁ}}, “‘One should get up for eight’: in the dining hall, the rest should get up for eight nuns to give them a seat.” See \href{https://suttacentral.net/pli-tv-kd20/en/brahmali\#18.1.3}{Kd 20:18.1.3}. } \\
Likewise give a seat to eight, \\
One is an instructor of the nuns through eight qualities.\footnote{See \href{https://suttacentral.net/pli-tv-bu-vb-pc21/en/brahmali\#2.26}{Bu Pc 21:2.26}–2.34. } 

For\marginnote{59.1} how many is there a ‘cutting off’? \\
For how many is there a serious offense? \\
And for how many is there no offense—\\
Yet all of them did the same act as basis? 

There\marginnote{60.1} is a ‘cutting off’ for one,\footnote{Sp 5.476: \textit{Ekassa chejjanti \textsanskrit{gāthāya} navasu janesu yo \textsanskrit{salākaṁ} \textsanskrit{gāhetvā} \textsanskrit{saṅghaṁ} bhindati, tasseva \textsanskrit{chejjaṁ} hoti, devadatto viya \textsanskrit{pārājikaṁ} \textsanskrit{āpajjati}}, “‘A “cutting off” for one’: among the nine people in the verse, the one who distributes the ballots causes a schism in the Sangha. There is a ‘cutting off’ for him, like Devadatta who commits an offense entailing expulsion.” See \href{https://suttacentral.net/pli-tv-kd17/en/brahmali\#5.1.21}{Kd 17:5.1.21}. } \\
A serious offense for four;\footnote{Sp 5.476: \textit{\textsanskrit{Bhedakānuvattakānaṁ} \textsanskrit{catunnaṁ} \textsanskrit{thullaccayaṁ} \textsanskrit{kokālikādīnaṁ} viya}, “There is a serious offense for the four who support the schism, like \textsanskrit{Kokālika}.” See \href{https://suttacentral.net/pli-tv-kd17/en/brahmali\#4.4.6}{Kd 17:4.4.6}. } \\
And there is no offense for four—\footnote{Sp 5.476: \textit{\textsanskrit{Dhammavādīnaṁ} \textsanskrit{catunnaṁ} \textsanskrit{anāpatti}}, “For the four who speak according to the Teaching, there is no offense.” } \\
Yet all of them did the same act as basis.\footnote{Sp 5.476: \textit{\textsanskrit{Imā} pana \textsanskrit{āpattiyo} ca \textsanskrit{anāpattiyo} ca \textsanskrit{sabbesaṁ} \textsanskrit{ekavatthukā} \textsanskrit{saṅghabhedavatthukā} eva}, “These offenses and non-offenses have one act as basis for all, that is, the act of causing a schism in the Sangha.” } 

How\marginnote{61.1} many are the grounds for resentment? \\
How many cause a schism in the Sangha? \\
How many here are immediate offenses? \\
How many acts are there through a motion? 

There\marginnote{62.1} are nine grounds for resentment,\footnote{See \href{https://suttacentral.net/an9.29/en/brahmali\#1.1}{AN 9.29}. } \\
Nine cause a schism in the Sangha;\footnote{Sp 5.476: \textit{\textsanskrit{Navahīti} navahi \textsanskrit{bhikkhūhi} \textsanskrit{saṅgho} bhijjati}, “‘Nine’: the Sangha is split by nine monks.” } \\
There are nine immediate offenses here,\footnote{This refers to the first nine offenses entailing suspension for monks. } \\
There are nine acts through a motion.”\footnote{Sp 5.476: \textit{\textsanskrit{Ñattiyā} \textsanskrit{karaṇā} \textsanskrit{navāti} \textsanskrit{ñattiyā} \textsanskrit{kātabbāni} \textsanskrit{kammāni} \textsanskrit{navāti} attho}, “‘There are nine acts through a motion’: the meaning is that there are nine legal procedures to be done through a motion.” That is, the legal procedures that have a motion but no announcement. } 

%
\end{verse}

\section*{4. Persons one should not pay respect to, etc. }

\begin{verse}%
“How\marginnote{63.1} many kinds of people should one not bow down to, \\
Nor raise one’s joined palms to, nor do acts of respect toward? \\
For how many is there an offense of wrong conduct? \\
How many keepings of a robe are there? 

There\marginnote{64.1} are ten kinds of people one should not bow down to,\footnote{For this and the next item, see \href{https://suttacentral.net/pli-tv-kd16/en/brahmali\#6.5.2}{Kd 16:6.5.2}. } \\
Nor raise one’s joined palms to, nor do acts of respect toward; \\
There is an offense of wrong conduct for ten,\footnote{Sp 5.477: \textit{\textsanskrit{Dasannaṁ} \textsanskrit{dukkaṭanti} \textsanskrit{tesaṁyeva} \textsanskrit{dasannaṁ} \textsanskrit{evaṁ} karontassa \textsanskrit{dukkaṭaṁ} hoti}, “‘An offense of wrong conduct for ten’: there is an offense of wrong conduct for one who does these things to those ten.” } \\
There are ten keepings of a robe.\footnote{Sp 5.477: \textit{Dasa \textsanskrit{cīvaradhāraṇāti} dasa \textsanskrit{divasāni} \textsanskrit{atirekacīvarassa} \textsanskrit{dhāraṇā} \textsanskrit{anuññātāti} attho}, “‘There are ten keepings of a robe’: the meaning is that it is allowed to keep an extra robe for ten days.” } 

To\marginnote{65.1} how many kinds who have completed the rainy-season residence \\
Should robe-cloth be given? \\
When one exists, to how many kinds should it be given? \\
And to how many kinds should it not be given? 

Robe-cloth\marginnote{66.1} should be given to five kinds\footnote{Sp 5.477: \textit{\textsanskrit{Pañcannaṁ} \textsanskrit{vassaṁvuṭṭhānaṁ}, \textsanskrit{dātabbaṁ} idha \textsanskrit{cīvaranti} \textsanskrit{pañcannaṁ} \textsanskrit{sahadhammikānaṁ} \textsanskrit{sammukhāva} \textsanskrit{dātabbaṁ}}, “‘Robe-cloth should be given to five kinds who have completed the rainy-season residence’: it should be given, in their presence, to the five kinds of co-monastics.” } \\
Who have completed the rainy-season residence; \\
When one exists, it should be given to seven kinds,\footnote{Sp 5.477: \textit{\textsanskrit{Sattannaṁ} santeti \textsanskrit{disāpakkantaummattakakhittacittavedanāṭṭānaṁ} \textsanskrit{tiṇṇañca} \textsanskrit{ukkhittakānanti} \textsanskrit{imesaṁ} \textsanskrit{sattannaṁ} sante \textsanskrit{patirūpe} \textsanskrit{gāhake} \textsanskrit{parammukhāpi} \textsanskrit{dātabbaṁ}}, “‘When one exists, to seven kinds’: when a suitable recipient exists, it should be given to these seven kinds in their absence, that is, those who have left for the districts, those who are insane, those who are deranged, those who are overwhelmed by pain, and the three kinds who have been ejected.” } \\
It should not be given to sixteen kinds.\footnote{Sp 5.477: \textit{\textsanskrit{Soḷasannaṁ} na \textsanskrit{dātabbanti} \textsanskrit{sesānaṁ} \textsanskrit{cīvarakkhandhake} \textsanskrit{vuttānaṁ} \textsanskrit{paṇḍakādīnaṁ} \textsanskrit{soḷasannaṁ} na \textsanskrit{dātabbaṁ}}, “‘It should not be given to sixteen kinds’: it should not be given to the remaining sixteen, that is, the \textit{\textsanskrit{paṇḍakas}}, etc., mentioned in The Chapter on Robes.” See \href{https://suttacentral.net/pli-tv-kd8/en/brahmali\#30.1.1}{Kd 8:30.1.1}. } 

Having\marginnote{67.1} concealed how many hundreds of offenses \\
For a hundred days? \\
After spending how many days, \\
Should he be released from probation? 

Having\marginnote{68.1} concealed ten hundred offenses\footnote{Sp 5.477: \textit{\textsanskrit{Dasasataṁ} \textsanskrit{rattisataṁ}, \textsanskrit{āpattiyo} \textsanskrit{chādayitvānāti} \textsanskrit{dasasataṁ} \textsanskrit{āpattiyo} \textsanskrit{rattisataṁ} \textsanskrit{chādayitvāna}. \textsanskrit{Ayañhettha} \textsanskrit{saṅkhepattho} – yo divase \textsanskrit{sataṁ} \textsanskrit{sataṁ} \textsanskrit{saṅghādisesāpattiyo} \textsanskrit{āpajjitvā} dasa dasa divase \textsanskrit{paṭicchādeti}, tena \textsanskrit{rattisataṁ} \textsanskrit{āpattisahassaṁ} \textsanskrit{paṭicchāditaṁ} hoti}, “‘Having concealed ten hundred offenses for a hundred days’: having concealed for a hundred days ten hundred offenses. This, below, is the meaning in brief: whoever commits one hundred offenses entailing suspension in one hundred days and conceals each one for ten days, he has concealed a thousand offenses for a hundred days.” } \\
For a hundred days; \\
Then, after spending ten days,\footnote{Sp 5.477: \textit{So \textsanskrit{sabbāva} \textsanskrit{tā} \textsanskrit{āpattiyo} \textsanskrit{dasāhapaṭicchannāti} \textsanskrit{parivāsaṁ} \textsanskrit{yācitvā} dasa rattiyo \textsanskrit{vasitvāna} mucceyya \textsanskrit{pārivāsikoti}}, “Having asked for probation for all those offenses concealed for ten days, ‘then, after spending ten days, he should be released from probation.’” } \\
He should be released from probation. 

How\marginnote{69.1} many kinds of flaws in legal procedures \\
Were mentioned by the Buddha, the Kinsman of the Sun? \\
In the Monastic Law, in the account of \textsanskrit{Campā}, \\
How many were illegitimate? 

Twelve\marginnote{70.1} kinds of flaws in legal procedures\footnote{Sp 5.477: \textit{\textsanskrit{Dvādasa} \textsanskrit{kammadosā} \textsanskrit{vuttāti} \textsanskrit{apalokanakammaṁ} \textsanskrit{adhammenavaggaṁ}, \textsanskrit{adhammenasamaggaṁ}, \textsanskrit{dhammenavaggaṁ}, \textsanskrit{tathā} \textsanskrit{ñattikammañattidutiyakammañatticatutthakammānipīti} \textsanskrit{evaṁ} \textsanskrit{ekekasmiṁ} kamme tayo tayo \textsanskrit{katvā} \textsanskrit{dvādasa} \textsanskrit{kammadosā} \textsanskrit{vuttā}}, “‘Twelve kinds of flaws in legal procedures were mentioned’: there is the legal procedure consisting of getting permission that is illegitimate with an incomplete assembly, that is illegitimate with a unanimous assembly, and that is legitimate with an incomplete assembly, and likewise a legal procedure consisting of one motion, a legal procedure consisting of one motion and one announcement, and a legal procedure consisting of one motion and three announcements. Thus, having produced three for each legal procedure, twelve flaws in legal procedures are mentioned.” } \\
Were mentioned by the Buddha, the Kinsman of the Sun; \\
In the Monastic Law, in the account of \textsanskrit{Campā}, \\
All were done illegitimately. 

How\marginnote{71.1} many kinds of accomplishments of legal procedures \\
Were mentioned by the Buddha, the Kinsman of the Sun? \\
In the Monastic Law, in the account of \textsanskrit{Campā}, \\
How many were legitimate? 

Four\marginnote{72.1} kinds of accomplishments of legal procedures\footnote{Sp 5.477: \textit{Catasso kammasampattiyoti \textsanskrit{apalokanakammaṁ} \textsanskrit{dhammenasamaggaṁ}, \textsanskrit{tathā} \textsanskrit{sesānipīti}}, “‘Four kinds of accomplishments of legal procedures’: there is the legal procedure consisting of getting permission that is legitimate with a unanimous assembly, and likewise the rest.” } \\
Were mentioned by the Buddha, the Kinsman of the Sun; \\
In the Monastic Law, in the account of \textsanskrit{Campā}, \\
All were done legitimately. 

How\marginnote{73.1} many kinds of legal procedures \\
Were mentioned by the Buddha, the Kinsman of the Sun? \\
In the Monastic Law, in the account of \textsanskrit{Campā}, \\
How many were legitimate and how many illegitimate? 

Six\marginnote{74.1} kinds of legal procedures\footnote{Sp 5.477: \textit{Cha \textsanskrit{kammānīti} \textsanskrit{adhammenavaggakammaṁ}, \textsanskrit{adhammenasamaggakammaṁ}, \textsanskrit{dhammapatirūpakenavaggakammaṁ}, \textsanskrit{dhammapatirūpakenasamaggakammaṁ}, \textsanskrit{dhammenavaggakammaṁ}, dhammenasamaggakammanti}, “‘Six kinds of legal procedures’: the legal procedure that is illegitimate with an incomplete assembly, the one that is illegitimate with a unanimous assembly, the one that is legitimate-like with an incomplete assembly, the one that is legitimate-like with a unanimous assembly, the one that is legitimate with an incomplete assembly, the one that is legitimate with a unanimous assembly.” } \\
Were mentioned by the Buddha, the Kinsman of the Sun; \\
In the Monastic Law, in the account of \textsanskrit{Campā}, \\
One done legitimately, \\
And five illegitimately, \\
Were mentioned of by the Buddha, the Kinsman of the Sun. 

How\marginnote{75.1} many kinds of legal procedures \\
Were mentioned by the Buddha, the Kinsman of the Sun? \\
In the Monastic Law, in the account of \textsanskrit{Campā}, \\
How many were legitimate and how many illegitimate? 

Four\marginnote{76.1} kinds of legal procedures\footnote{Presumably this refers to \href{https://suttacentral.net/pli-tv-kd9/en/brahmali\#2.4.2}{Kd 9:2.4.2}. } \\
Were mentioned by the Buddha, the Kinsman of the Sun; \\
In the Monastic Law, in the account of \textsanskrit{Campā}, \\
One done legitimately, \\
And three illegitimately, \\
Were mentioned by the Buddha, the Kinsman of the Sun. 

In\marginnote{77.1} regard to the classes of offenses taught by the unbounded Victor, the Unshakable One, \\
The Knower of seclusion; \\
How many are settled without settling? \\
I ask you who are skilled in analysis, please say. 

In\marginnote{78.1} regard to the classes of offenses taught by the unbounded Victor, the Unshakable One, \\
The Knower of seclusion; \\
One is settled without settling,\footnote{This would seem to refer to the offenses entailing expulsion, the \textit{\textsanskrit{pārājikas}}, which cannot be “settled”, that is, cleared. } \\
Skilled in analysis, I declare this to you. 

How\marginnote{79.1} many who are destined to misery \\
Were mentioned by the Buddha, the Kinsman of the Sun? \\
We will listen to the Monastic Law from you, \\
One who understands it. 

One\marginnote{80.1} hundred and forty-four\footnote{Sp 5.477: \textit{\textsanskrit{Chaūnadiyaḍḍhasatāti} “idha, \textsanskrit{upāli}, bhikkhu \textsanskrit{adhammaṁ} dhammoti \textsanskrit{dīpeti}, \textsanskrit{tasmiṁ} \textsanskrit{adhammadiṭṭhi} bhede \textsanskrit{adhammadiṭṭhi}, \textsanskrit{tasmiṁ} \textsanskrit{adhammadiṭṭhi} bhede \textsanskrit{dhammadiṭṭhi}, \textsanskrit{tasmiṁ} \textsanskrit{adhammadiṭṭhi} bhede vematiko, \textsanskrit{tasmiṁ} \textsanskrit{dhammadiṭṭhi} bhede \textsanskrit{adhammadiṭṭhi}, \textsanskrit{tasmiṁ} \textsanskrit{dhammadiṭṭhi} bhede vematiko, \textsanskrit{tasmiṁ} vematiko bhede \textsanskrit{adhammadiṭṭhi}, \textsanskrit{tasmiṁ} vematiko bhede \textsanskrit{dhammadiṭṭhi}, \textsanskrit{tasmiṁ} vematiko bhede vematiko”ti \textsanskrit{evaṁ} \textsanskrit{yāni} \textsanskrit{aṭṭhārasannaṁ} \textsanskrit{bhedakaravatthūnaṁ} vasena \textsanskrit{aṭṭhārasa} \textsanskrit{aṭṭhakāni} \textsanskrit{saṅghabhedakakkhandhake} \textsanskrit{vuttāni}, \textsanskrit{tesaṁ} vasena \textsanskrit{chaūnadiyaḍḍhasataṁ} \textsanskrit{āpāyikā} \textsanskrit{veditabbā}}, “‘One hundred and forty-four’: \textsanskrit{Upāli}, a monk may proclaim what is contrary to the Teaching as being in accordance with it. In regard to that, he has the view that what he says is illegitimate and the view that the schism is illegitimate; he has the view that what he says is illegitimate, but the view that the schism is legitimate; he has the view that what he says is illegitimate, but is unsure about the schism; he has the view that what he says is legitimate, but the view that the schism is illegitimate; he has the view that what he says is legitimate, but is unsure about the schism; he is unsure about what he says, but has the view that the schism is illegitimate; he is unsure about what he says, but has the view that the schism is legitimate; or he is unsure about what he says and is unsure about the schism. Thus, on account of the eighteen grounds for schism, eighteen groups of eight are mentioned in The Chapter on Schism in the Sangha. It is on account of these that the one hundred and forty-four that are destined to misery are to be understood.” See \href{https://suttacentral.net/pli-tv-kd17/en/brahmali\#5.5.5.1}{Kd 17:5.5.5.1}–5.5.54. } \\
Were mentioned by the Buddha, the Kinsman of the Sun; \\
Destined to misery, bound for hell, \\
He who causes a schism in the Sangha remains there for an eon; \\
Listen to the Monastic Law from me, \\
One who understands it. 

How\marginnote{81.1} many who are not destined to misery \\
Were mentioned by the Buddha, the Kinsman of the Sun? \\
We will listen to the Monastic Law from you, \\
One who understands it. 

Eighteen\marginnote{82.1} not destined to misery\footnote{Sp 5.477: \textit{\textsanskrit{Aṭṭhārasa} \textsanskrit{anāpāyikāti} “idha, \textsanskrit{upāli}, bhikkhu \textsanskrit{adhammaṁ} dhammoti \textsanskrit{dīpeti}, \textsanskrit{tasmiṁ} \textsanskrit{dhammadiṭṭhi} bhede \textsanskrit{dhammadiṭṭhi} \textsanskrit{avinidhāya} \textsanskrit{diṭṭhiṁ} \textsanskrit{avinidhāya} \textsanskrit{khantiṁ} \textsanskrit{avinidhāya} \textsanskrit{ruciṁ} \textsanskrit{avinidhāya} \textsanskrit{bhāvaṁ} \textsanskrit{anussāveti}, \textsanskrit{salākaṁ} \textsanskrit{gāheti} ‘\textsanskrit{ayaṁ} dhammo, \textsanskrit{ayaṁ} vinayo, \textsanskrit{idaṁ} \textsanskrit{satthusāsanaṁ}, \textsanskrit{imaṁ} \textsanskrit{gaṇhatha}, \textsanskrit{imaṁ} \textsanskrit{rocethā}’ti, ayampi kho, \textsanskrit{upāli}, \textsanskrit{saṅghabhedako} na \textsanskrit{āpāyiko} na nerayiko na \textsanskrit{kappaṭṭho} na atekiccho”ti \textsanskrit{evaṁ} \textsanskrit{ekekasmiṁ} \textsanskrit{vatthusmiṁ} \textsanskrit{ekekaṁ} \textsanskrit{katvā} \textsanskrit{saṅghabhedakakkhandhakāvasāne} \textsanskrit{vuttā} \textsanskrit{aṭṭhārasa} \textsanskrit{janā}}, “‘Eighteen not destined to misery’: ‘\textsanskrit{Upāli}, a monk may proclaim what is contrary to the Teaching as being in accordance with it. In regard to that, he has the view that what he says is legitimate and the view that the schism is legitimate. He doesn’t misrepresent his view of what’s true, his belief of what’s true, his acceptance of what’s true, or his sentiment of what’s true. He makes a proclamation and distributes ballots, saying, “This is the Teaching, this is the Monastic Law, this is the Teacher’s instruction; take this, approve of this.” When such a person causes a schism in the Sangha, he’s not irredeemably destined to an eon in hell.’ Thus, taking the grounds one by one, eighteen people are mentioned in The Chapter on Schism in the Sangha.” See \href{https://suttacentral.net/pli-tv-kd17/en/brahmali\#5.6.1}{Kd 17:5.6.1}–5.6.10. } \\
Were mentioned by the Buddha, the Kinsman of the Sun; \\
Listen to the Monastic Law from me, \\
One who understands it. 

How\marginnote{83.1} many groups of eight \\
Were mentioned by the Buddha, the Kinsman of the Sun? \\
We will listen to the Monastic Law from you, \\
One who understands it. 

Eighteen\marginnote{84.1} groups of eight\footnote{Sp 5.477: \textit{\textsanskrit{Aṭṭhārasa} \textsanskrit{aṭṭhakā} \textsanskrit{chaūnadiyaḍḍhasatavissajjane} \textsanskrit{vuttāyeva}}, “Eighteen groups of eight are mentioned in the explanation of the one hundred and forty-four.” } \\
Were mentioned by the Buddha, the Kinsman of the Sun; \\
Listen to the Monastic Law from me, \\
One who understands it.” 

%
\end{verse}

\section*{5. The sixteen legal procedures, etc. }

\begin{verse}%
“How\marginnote{85.1} many kinds of legal procedures \\
Were mentioned by the Buddha, the Kinsman of the Sun? \\
We will listen to the Monastic Law from you, \\
One who understands it. 

Sixteen\marginnote{86.1} kinds of legal procedures\footnote{See \href{https://suttacentral.net/pli-tv-pvr15/en/brahmali\#9.3}{Pvr 15:9.3}. } \\
Were mentioned by the Buddha, the Kinsman of the Sun; \\
Listen to the Monastic Law from me, \\
One who understands it. 

How\marginnote{87.1} many kinds of flaws in legal procedures \\
Were mentioned by the Buddha, the Kinsman of the Sun? \\
We will listen to the Monastic Law from you, \\
One who understands it. 

Twelve\marginnote{88.1} kinds of flaws\footnote{As above in the main text. } \\
Were mentioned by the Buddha, the Kinsman of the Sun; \\
Listen to the Monastic Law from me, \\
One who understands it. 

How\marginnote{89.1} many kinds of accomplishments of legal procedures \\
Were mentioned by the Buddha, the Kinsman of the Sun? \\
We will listen to the Monastic Law from you, \\
One who understands it. 

Four\marginnote{90.1} kinds of accomplishments\footnote{As above. } \\
Were mentioned by the Buddha, the Kinsman of the Sun; \\
Listen to the Monastic Law from me, \\
One who understands it. 

How\marginnote{91.1} many kinds of legal procedures \\
Were mentioned by the Buddha, the Kinsman of the Sun? \\
We will listen to the Monastic Law from you, \\
One who understands it. 

Six\marginnote{92.1} kinds of legal procedures\footnote{As above. } \\
Were mentioned by the Buddha, the Kinsman of the Sun; \\
Listen to the Monastic Law from me, \\
One who understands it. 

How\marginnote{93.1} many kinds of legal procedures \\
Were mentioned by the Buddha, the Kinsman of the Sun? \\
We will listen to the Monastic Law from you, \\
One who understands it. 

Four\marginnote{94.1} kinds of legal procedures\footnote{As above. } \\
Were mentioned by the Buddha, the Kinsman of the Sun; \\
Listen to the Monastic Law from me, \\
One who understands it. 

How\marginnote{95.1} many offenses entailing expulsion \\
Were mentioned by the Buddha, the Kinsman of the Sun? \\
We will listen to the Monastic Law from you, \\
One who understands it. 

Eight\marginnote{96.1} offenses entailing entailing expulsion \\
Were mentioned by the Buddha, the Kinsman of the Sun; \\
Listen to the Monastic Law from me, \\
One who understands it. 

How\marginnote{97.1} many offenses entailing suspension \\
Were mentioned by the Buddha, the Kinsman of the Sun? \\
We will listen to the Monastic Law from you, \\
One who understands it. 

Twenty-three\marginnote{98.1} offenses entailing suspension \\
Were mentioned by the Buddha, the Kinsman of the Sun; \\
Listen to the Monastic Law from me, \\
One who understands it. 

How\marginnote{99.1} many undetermined offenses \\
Were mentioned by the Buddha, the Kinsman of the Sun? \\
We will listen to the Monastic Law from you, \\
One who understands it. 

Two\marginnote{100.1} undetermined offenses \\
Were mentioned by the Buddha, the Kinsman of the Sun; \\
Listen to the Monastic Law from me, \\
One who understands it. 

How\marginnote{101.1} many offenses entailing relinquishment \\
Were mentioned by the Buddha, the Kinsman of the Sun? \\
We will listen to the Monastic Law from you, \\
One who understands it. 

Forty-two\marginnote{102.1} offenses entailing relinquishment \\
Were mentioned by the Buddha, the Kinsman of the Sun; \\
Listen to the Monastic Law from me, \\
One who understands it. 

How\marginnote{103.1} many offenses entailing confession \\
Were mentioned by the Buddha, the Kinsman of the Sun? \\
We will listen to the Monastic Law from you, \\
One who understands it. 

One\marginnote{104.1} hundred and eighty-eight offenses entailing confession \\
Were mentioned by the Buddha, the Kinsman of the Sun; \\
Listen to the Monastic Law from me, \\
One who understands it. 

How\marginnote{105.1} many offenses entailing acknowledgment \\
Were mentioned by the Buddha, the Kinsman of the Sun? \\
We will listen to the Monastic Law from you, \\
One who understands it. 

Twelve\marginnote{106.1} offenses entailing acknowledgment \\
Were mentioned by the Buddha, the Kinsman of the Sun; \\
Listen to the Monastic Law from me, \\
One who understands it. 

How\marginnote{107.1} many rules to be trained in \\
Were mentioned by the Buddha, the Kinsman of the Sun? \\
We will listen to the Monastic Law from you, \\
One who understands it. 

Seventy-five\marginnote{108.1} rules to be trained in \\
Were mentioned by the Buddha, the Kinsman of the Sun; \\
Listen to the Monastic Law from me, \\
One who understands it. 

Thus\marginnote{109.1} far you have asked well, \\
Thus far I have answered well; \\
In either the questions or the answers, \\
There is nothing other than the Teaching.” 

%
\end{verse}

\scendsection{The verses on offenses, training rules, and legal procedures are finished. }

%
\chapter*{{\suttatitleacronym Pvr 20}{\suttatitletranslation The sudorific verses }{\suttatitleroot Sedamocanagāthā}}
\addcontentsline{toc}{chapter}{\tocacronym{Pvr 20} \toctranslation{The sudorific verses } \tocroot{Sedamocanagāthā}}
\markboth{The sudorific verses }{Sedamocanagāthā}
\extramarks{Pvr 20}{Pvr 20}

\section*{1. Question on staying apart }

\begin{verse}%
For\marginnote{1.1} one excluded from the community of the monks and the nuns,\footnote{This refers to people who are not fully ordained, including those who have committed an offense entailing expulsion. } \\
Some interactions are unallowable—\\
How, then, is there no offense for one who is not staying apart?\footnote{Sp 5.479: \textit{\textsanskrit{Avippavāsena} \textsanskrit{anāpattīti} \textsanskrit{sahagāraseyyāya} \textsanskrit{anāpatti}}, “‘There is no offense for one who is not staying apart’: there is no offense for sharing a sleeping place in a house.” } \\
This question was thought out by those with skill.\footnote{Vmv 5.479: \textit{\textsanskrit{Pañhā} \textsanskrit{mesāti} ettha ma-\textsanskrit{kāro} padasandhikaro}, “\textit{\textsanskrit{Pañhā} \textsanskrit{mesā}}: here the syllable ‘\textit{ma}’ creates a junction between the words.” } 

Of\marginnote{2.1} things not to be given away, nor to be shared out, \\
Five are mentioned by the Great Sage—\\
How, then, is there no offense for using what has been given away?\footnote{Vjb 5.479: \textit{\textsanskrit{Varasenāsanarakkhaṇatthāya} \textsanskrit{vissajjetvāparibhuñjituṁ} \textsanskrit{vaṭṭatī}”ti \textsanskrit{garubhaṇḍavinicchaye} vutto}, “It is said in the investigation of valuable goods that it is allowable to give away and use a good dwelling for the purpose of protecting it.” } \\
This question was thought out by those with skill. 

I\marginnote{3.1} do not say the ten people,\footnote{Sp 5.479: \textit{Dasa puggale na \textsanskrit{vadāmīti} \textsanskrit{senāsanakkhandhake} vutte dasa puggale na \textsanskrit{vadāmi}}, “‘I do not say the ten people’: I do not say the ten people mentioned in The Chapter on Resting Places.” See \href{https://suttacentral.net/pli-tv-kd16/en/brahmali\#6.5.2}{Kd 16:6.5.2}. } \\
Or the eleven to be avoided—\footnote{Sp 5.479: \textit{\textsanskrit{Ekādasa} \textsanskrit{vivajjiyāti} ye \textsanskrit{mahākhandhake} \textsanskrit{ekādasa} \textsanskrit{vivajjanīyapuggalā} \textsanskrit{vuttā}, tepi na \textsanskrit{vadāmi}}, “‘The eleven to be avoided’: I do not say the eleven people to be avoided mentioned in The Great Chapter.” See \href{https://suttacentral.net/pli-tv-kd1/en/brahmali\#61.1.19}{Kd 1:61.1.19}–68.1.4. } \\
How, then, is there an offense for paying respect to one who is senior?\footnote{Sp 5.479: \textit{\textsanskrit{Ayaṁ} \textsanskrit{pañhā} \textsanskrit{naggaṁ} \textsanskrit{bhikkhuṁ} \textsanskrit{sandhāya} \textsanskrit{vuttā}}, “This question was asked with reference to a naked monk.” See \href{https://suttacentral.net/pli-tv-kd15/en/brahmali\#15.1.6}{Kd 15:15.1.6}. } \\
This question was thought out by those with skill. 

One\marginnote{4.1} who has not been ejected, nor is on probation, \\
Who has not caused a schism in the Sangha, or joined another religion or sect; \\
Who belongs to the same Buddhist sect—\\
How, then, could they not share in the training?\footnote{Sp 5.479: \textit{\textsanskrit{Kathaṁ} nu \textsanskrit{sikkhāya} \textsanskrit{asādhāraṇoti} \textsanskrit{pañhā} \textsanskrit{nahāpitapubbakaṁ} \textsanskrit{bhikkhuṁ} \textsanskrit{sandhāya} \textsanskrit{vuttā}. \textsanskrit{Ayañhi} \textsanskrit{khurabhaṇḍaṁ} \textsanskrit{pariharituṁ} na labhati, \textsanskrit{aññe} labhanti; \textsanskrit{tasmā} \textsanskrit{sikkhāya} \textsanskrit{asādhāraṇo}}, “‘How could they not share in the training?’ This question was asked with reference to the monk who was previously a barber. For him it was not allowable to carry barber equipment around, but for others it was. Therefore the training was not shared.” See \href{https://suttacentral.net/pli-tv-kd6/en/brahmali\#37.5.7}{Kd 6:37.5.7}. } \\
This question was thought out by those with skill. 

Questioning,\marginnote{5.1} one arrives at the Teaching, \\
At what is wholesome and beneficial; \\
One who is not alive, nor dead or extinguished—\\
What kind of person is that, say the Buddhas?\footnote{Sp 5.479: \textit{\textsanskrit{Taṁ} \textsanskrit{puggalaṁ} \textsanskrit{katamaṁ} vadanti \textsanskrit{buddhāti} \textsanskrit{ayaṁ} \textsanskrit{pañhā} \textsanskrit{nimmitabuddhaṁ} \textsanskrit{sandhāya} \textsanskrit{vuttā}}, “‘What kind of person is that, say the Buddhas’: this question was asked with reference to a created Buddha.” Sp-yoj 5.479: \textit{Nimmitabuddhanti buddhena \textsanskrit{nimmitaṁ} \textsanskrit{buddharūpaṁ}}, “‘A created Buddha’: a Buddha form created by the Buddha.” It seems this refers to creating an image of the Buddha through supernormal powers. } \\
This question was thought out by those with skill. 

I\marginnote{6.1} do not say above the collar bone, \\
Having abandoned what is below the navel; \\
Because of sexual intercourse, \\
How might there be an offense entailing expulsion?\footnote{Sp 5.479: \textit{\textsanskrit{Ayaṁ} \textsanskrit{pañhā} \textsanskrit{yaṁ} \textsanskrit{taṁ} \textsanskrit{asīsakaṁ} \textsanskrit{kabandhaṁ}, yassa ure \textsanskrit{akkhīni} ceva \textsanskrit{mukhañca} hoti, \textsanskrit{taṁ} \textsanskrit{sandhāya} \textsanskrit{vuttā}}, “This question was asked with reference to a headless body with eyes and mouth on the chest.” } \\
This question was thought out by those with skill. 

A\marginnote{7.1} monk, by means of begging, builds a hut, \\
Whose site has not been approved, which exceeds the right size, where harm will be done, and which lacks a space on all sides; \\
How, then, does he not commit an offense?\footnote{Sp 5.479: \textit{Bhikkhu \textsanskrit{saññācikāya} \textsanskrit{kuṭinti} \textsanskrit{ayaṁ} \textsanskrit{pañhā} \textsanskrit{tiṇacchādanaṁ} \textsanskrit{kuṭiṁ} \textsanskrit{sandhāya} \textsanskrit{vuttā}}, “‘A monk, by means of begging, a hut’: this question was asked with reference to a hut with a grass roof.” See \href{https://suttacentral.net/pli-tv-bu-vb-ss6/en/brahmali\#3.16.4}{Bu Ss 6:3.16.4}. } \\
This question was thought out by those with skill. 

A\marginnote{8.1} monk, by means of begging, builds a hut, \\
Whose site has been approved, which is the right size, where no harm will be done, and which has a space on all sides. \\
How, then, does he commit an offense?\footnote{Sp 5.479: \textit{\textsanskrit{Dutiyapañhā} \textsanskrit{sabbamattikāmayaṁ} \textsanskrit{kuṭiṁ} \textsanskrit{sandhāya} \textsanskrit{vuttā}}, “The second question was asked with reference to a hut made entirely of clay.” See \href{https://suttacentral.net/pli-tv-bu-vb-pj2/en/brahmali\#1.2.11}{Bu Pj 2:1.2.11}. } \\
This question was thought out by those with skill. 

One\marginnote{9.1} does not do anything by body, \\
Nor does one say anything to another; \\
How, then, does one commit a heavy offense, a ground for cutting off?\footnote{Sp 5.479: \textit{Āpajjeyya \textsanskrit{garukaṁ} chejjavatthunti \textsanskrit{ayaṁ} \textsanskrit{pañhā} \textsanskrit{vajjapaṭicchādikaṁ} \textsanskrit{bhikkhuniṁ} \textsanskrit{sandhāya} \textsanskrit{vuttā}}, “‘How, then, does one commit a heavy offense, a ground for cutting off’: this question was asked with reference to a nun who hides offenses.” See \href{https://suttacentral.net/pli-tv-bi-vb-pj6/en/brahmali\#1.23.1}{Bi Pj 6:1.23.1}. } \\
This question was thought out by those with skill. 

Nothing\marginnote{10.1} bad by body, speech, or mind \\
Would a good person do; \\
How, then, when he is expelled, would it be right?\footnote{Sp 5.479: \textit{\textsanskrit{Dutiyapañhā} \textsanskrit{paṇḍakādayo} abhabbapuggale \textsanskrit{sandhāya} \textsanskrit{vuttā}. \textsanskrit{Ekādasapi} hi te \textsanskrit{gihibhāveyeva} \textsanskrit{pārājikaṁ} \textsanskrit{pattā}}, “‘The second question was asked with reference to the incapable people, starting with the \textit{\textsanskrit{paṇḍakas}}. ” See \href{https://suttacentral.net/pli-tv-kd1/en/brahmali\#61.1.19}{Kd 1:61.1.19}–68.1.4. } \\
This question was thought out by those with skill. 

Not\marginnote{11.1} speaking with any human, \\
Nor saying anything to others; \\
How, then, does one commit an offense of speech, not one of body?\footnote{Sp 5.479: \textit{\textsanskrit{Ayaṁ} \textsanskrit{pañhā} “\textsanskrit{santiṁ} \textsanskrit{āpattiṁ} \textsanskrit{nāvikareyya}, \textsanskrit{sampajānamusāvādassa} \textsanskrit{hotī}”ti \textsanskrit{imaṁ} \textsanskrit{musāvādaṁ} \textsanskrit{sandhāya} \textsanskrit{vuttā}}, “This question was asked with reference to this kind of lying: ‘Should he not reveal an existing offense, he has lied in full awareness.’” See \href{https://suttacentral.net/pli-tv-kd2/en/brahmali\#3.3.13}{Kd 2:3.3.13}. } \\
This question was thought out by those with skill. 

The\marginnote{12.1} training rules praised by the splendid Buddha \\
Include which four offenses entailing suspension; \\
All committed through a single effort?\footnote{Sp 5.479: \textit{\textsanskrit{Saṅghādisesā} caturoti \textsanskrit{ayaṁ} \textsanskrit{pañhā} \textsanskrit{aruṇugge} \textsanskrit{gāmantarapariyāpannaṁ} \textsanskrit{nadipāraṁ} \textsanskrit{okkantabhikkhuniṁ} \textsanskrit{sandhāya} \textsanskrit{vuttā}, \textsanskrit{sā} hi \textsanskrit{sakagāmato} \textsanskrit{paccūsasamaye} \textsanskrit{nikkhamitvā} \textsanskrit{aruṇuggamanakāle} \textsanskrit{vuttappakāraṁ} \textsanskrit{nadipāraṁ} \textsanskrit{okkantamattāva} \textsanskrit{rattivippavāsagāmantaranadipāragaṇamhāohīyanalakkhaṇena} \textsanskrit{ekappahāreneva} caturo \textsanskrit{saṅghādisese} \textsanskrit{āpajjati}}, “‘Four offenses entailing suspension’: this question was asked with reference to a nun who has gone away and, at dawn, has entered another village and crossed a river. Having departed from her own village early in the morning, and in the said manner, just having crossed a river at the time of dawn, then, by the characteristics of staying apart for a night, of going to the next village, of crossing a river, and of lagging behind her group, she commits four offenses entailing suspension in one go.” See \href{https://suttacentral.net/pli-tv-bi-vb-ss3/en/brahmali\#4.14.1}{Bi Ss 3:4.14.1}. } \\
This question was thought out by those with skill. 

Two\marginnote{13.1} nuns were ordained together, \\
And one receives a robe directly from both; \\
Might the offenses be different?\footnote{Sp 5.479: \textit{\textsanskrit{Siyā} \textsanskrit{āpattiyo} \textsanskrit{nānāti} \textsanskrit{ayaṁ} \textsanskrit{pañhā} \textsanskrit{ekatoupasampannā} dve bhikkhuniyo \textsanskrit{sandhāya} \textsanskrit{vuttā}. \textsanskrit{Tāsu} hi \textsanskrit{bhikkhūnaṁ} santike \textsanskrit{ekatoupasampannāya} hatthato \textsanskrit{gaṇhantassa} \textsanskrit{pācittiyaṁ}, \textsanskrit{bhikkhunīnaṁ} santike \textsanskrit{ekatoupasampannāya} hatthato \textsanskrit{gaṇhantassa} \textsanskrit{dukkaṭaṁ}}, “‘Might the offenses be different’: this question was asked with reference to two nuns fully ordained on one side only. When they are in the presence of the monks, there is an offense entailing confession for one who receives directly from a nun who is fully ordained only on one side. In the presence of the nuns, there is an offense of wrong conduct for one who receives directly from a nun who is fully ordained only on one side.” See \href{https://suttacentral.net/pli-tv-bu-vb-np5/en/brahmali\#3.2.4}{Bu NP 5:3.2.4}. } \\
This question was thought out by those with skill. 

Four\marginnote{14.1} people having made an arrangement, \\
Took valuable goods; \\
How, then, did three commit an offense entailing expulsion, but one did not?\footnote{Sp 5.479: \textit{Caturo \textsanskrit{janā} \textsanskrit{saṁvidhāyāti} \textsanskrit{ācariyo} ca tayo ca \textsanskrit{antevāsikā} \textsanskrit{chamāsakaṁ} \textsanskrit{bhaṇḍaṁ} \textsanskrit{avahariṁsu}, \textsanskrit{ācariyassa} \textsanskrit{sāhatthikā} tayo \textsanskrit{māsakā}, \textsanskrit{āṇattiyāpi} tayova \textsanskrit{tasmā} \textsanskrit{thullaccayaṁ} \textsanskrit{āpajjati}, \textsanskrit{itaresaṁ} \textsanskrit{sāhatthiko} ekeko, \textsanskrit{āṇattikā} \textsanskrit{pañcāti} \textsanskrit{tasmā} \textsanskrit{pārājikaṁ} \textsanskrit{āpajjiṁsu}}, “‘Four people having made an arrangement’: a teacher and three pupils took goods worth six \textit{\textsanskrit{māsaka}} coins. The teacher took three \textit{\textsanskrit{māsakas}’} worth with his own hands and three by command. Because of that, he committed a serious offense. Among the others, they took a single one by hand and five through command. Because of that, they committed offenses entailing expulsion.” The point here is that one incurs a \textit{\textsanskrit{pārājika}} if one takes five or more \textit{\textsanskrit{māsakas}} with a single intention. See \href{https://suttacentral.net/pli-tv-bu-vb-pj2/en/brahmali\#6.1.1}{Bu Pj 2:6.1.1}–6.1.16. } \\
This question was thought out by those with skill. 

%
\end{verse}

\section*{2. Questions on the offenses entailing expulsion, etc. }

\begin{verse}%
The\marginnote{15.1} woman is within, \\
And the monk is outside; \\
In that house there is no hole—\\
Then, because of sexual intercourse; \\
How could there be an offense entailing expulsion?\footnote{Sp 5.480: \textit{\textsanskrit{Chiddaṁ} \textsanskrit{tasmiṁ} ghare \textsanskrit{natthīti} \textsanskrit{ayaṁ} \textsanskrit{pañhā} \textsanskrit{dussakuṭiādīni} \textsanskrit{santhatapeyyālañca} \textsanskrit{sandhāya} \textsanskrit{vuttā}}, “‘In that house there is no hole’: this question was asked with reference to a hut made of cloth, etc., and the successive series on ‘covered’.” \href{https://suttacentral.net/pli-tv-bu-vb-pj1/en/brahmali\#9.4.0}{Bu Pj 1:9.4.0}–9.6.60. } \\
This question was thought out by those with skill. 

Oil,\marginnote{16.1} honey, syrup, and ghee, \\
Having received it oneself, one stores it; \\
Not exceeding seven days. \\
Then if one uses it, even when there is a reason, how is there an offense?\footnote{Sp 5.480: \textit{\textsanskrit{Telaṁ} \textsanskrit{madhuṁ} \textsanskrit{phāṇitanti} \textsanskrit{gāthā} \textsanskrit{liṅgaparivattaṁ} \textsanskrit{sandhāya} \textsanskrit{vuttā}}, “‘Oil, honey, syrup’: the verse was spoken with reference to change of sex.” Vmv 5.480: \textit{\textsanskrit{Liṅgaparivattaṁ} \textsanskrit{sandhāya} \textsanskrit{vuttāti} \textsanskrit{liṅge} parivatte \textsanskrit{paṭiggahaṇassa} vijahanato puna \textsanskrit{appaṭiggahetvā} \textsanskrit{paribhuñjanāpattiṁ} \textsanskrit{sandhāya} \textsanskrit{vuttaṁ}}, “‘Was spoken with reference to change of sex’: when there is a change of sex, then, for the one receiving, it is given up. It was spoken with reference to the offense for using it without again having received it.” } \\
This question was thought out by those with skill. 

How\marginnote{17.1} is there an offense entailing relinquishment, \\
And an ordinary offense entailing confession; \\
Both together, for one who commits it?\footnote{Sp 5.480: \textit{\textsanskrit{Nissaggiyenāti} \textsanskrit{gāthā} \textsanskrit{pariṇāmanaṁ} \textsanskrit{sandhāya} \textsanskrit{vuttā}. Yo hi \textsanskrit{saṅghassa} \textsanskrit{pariṇatalābhato} \textsanskrit{ekaṁ} \textsanskrit{cīvaraṁ} attano, \textsanskrit{ekaṁ} \textsanskrit{aññassāti} dve \textsanskrit{cīvarāni} “\textsanskrit{ekaṁ} \textsanskrit{mayhaṁ}, \textsanskrit{ekaṁ} tassa \textsanskrit{dehī}”ti ekapayaogena \textsanskrit{pariṇāmeti}, so \textsanskrit{nissaggiyapācittiyañceva} \textsanskrit{suddhikapācittiyañca} ekato \textsanskrit{āpajjati}}, “‘Entailing relinquishment’: the verse was spoken with reference to diverting. One who diverts the gain of two robes directed to the Sangha, one to himself and one to someone else, with a single effort, he commits an offense entailing relinquishment and confession and a regular offense entailing confession together.” This refers to \href{https://suttacentral.net/pli-tv-bu-vb-np30/en/brahmali\#1.27.1}{Bu NP 30:1.27.1} and \href{https://suttacentral.net/pli-tv-bu-vb-pc82/en/brahmali\#1.26.1}{Bu Pc 82:1.26.1}. } \\
This question was thought out by those with skill. 

Twenty\marginnote{18.1} monks have come together, \\
Perceiving unity, they do a legal procedure; \\
If a monk is 150 km away,\footnote{For a discussion of the \textit{yojana}, see \textit{sugata} in Appendix of Technical Terms. } \\
How is that procedure reversible because the assembly is incomplete?\footnote{Sp 5.480: \textit{\textsanskrit{Kammañca} \textsanskrit{taṁ} kuppeyya \textsanskrit{vaggapaccayāti} \textsanskrit{ayaṁ} \textsanskrit{pañhā} \textsanskrit{dvādasayojanapamāṇesu} \textsanskrit{bārāṇasiādīsu} nagaresu \textsanskrit{gāmasīmaṁ} \textsanskrit{sandhāya} \textsanskrit{vuttā}}, “‘How is that procedure invalid because the assembly is incomplete’: this question was asked with reference to the zones of inhabited areas of towns like Benares that are 150 km in size.” See \href{https://suttacentral.net/pli-tv-kd2/en/brahmali\#12.7.1}{Kd 2:12.7.1}. For the rendering “reversible” for \textit{kuppeyya}, see \textit{kuppa} in Appendix of Technical Terms. } \\
This question was thought out by those with skill. 

When,\marginnote{19.1} after speaking, one merely takes a single step, \\
How would one, all at once, commit sixty-four heavy offenses; \\
All to be made amends for?\footnote{Sp 5.480: \textit{\textsanskrit{Padavītihāramattenāti} \textsanskrit{gāthā} \textsanskrit{sañcarittaṁ} \textsanskrit{sandhāya} \textsanskrit{vuttā}, atthopi \textsanskrit{cassā} \textsanskrit{sañcarittavaṇṇanāyameva} vutto}, “‘One merely takes a single step’: this line was spoken with reference to matchmaking. The meaning of it is spoken of in the commentary on matchmaking.” See \href{https://suttacentral.net/pli-tv-bu-vb-ss5/en/brahmali\#2.2.13.1}{Bu Ss 5:2.2.13.1}. } \\
This question was thought out by those with skill. 

Dressed\marginnote{20.1} in a sarong, \\
And a double-layered upper robe—\footnote{For an explanation of the rendering “upper robe” for \textit{\textsanskrit{saṅghāṭi}}, see Appendix of Technical Terms. } \\
How could they all be subject to relinquishment?\footnote{Sp 5.480: \textit{\textsanskrit{Sabbāni} \textsanskrit{tāni} \textsanskrit{nissaggiyānīti} \textsanskrit{ayaṁ} \textsanskrit{pañhā} \textsanskrit{aññātikāya} \textsanskrit{bhikkhuniyā} \textsanskrit{dhovāpanaṁ} \textsanskrit{sandhāya} \textsanskrit{vuttā}. Sace hi \textsanskrit{tiṇṇampi} \textsanskrit{cīvarānaṁ} \textsanskrit{kākaūhadanaṁ} \textsanskrit{vā} \textsanskrit{kaddamamakkhitaṁ} \textsanskrit{vā} \textsanskrit{kaṇṇaṁ} \textsanskrit{gahetvā} \textsanskrit{bhikkhunī} udakena dhovati, bhikkhussa \textsanskrit{kāyagatāneva} \textsanskrit{nissaggiyāni} honti}, “‘How could they all be subject to relinquishment’: this question was asked with reference to having an unrelated nun wash a robe. If, in regard to the three robes, a nun takes a corner that is soiled with crow excretions or mud, and she washes it with water, there is an offense entailing relinquishment if they had been worn by the monk.” See \href{https://suttacentral.net/pli-tv-bu-vb-np4/en/brahmali\#1.31.1}{Bu NP 4:1.31.1}. } \\
This question was thought out by those with skill. 

There\marginnote{21.1} was no motion, nor announcement, \\
Nor had the Victor said, “Come, monk”; \\
Nor had he gone for refuge—\\
How, then, was the ordination irreversible?\footnote{Sp 5.480: \textit{\textsanskrit{Ayaṁ} pana \textsanskrit{pañhā} \textsanskrit{mahāpajāpatiyā} \textsanskrit{upasampadaṁ} \textsanskrit{sandhāya} \textsanskrit{vuttā}}, “This question was asked with reference to the full ordination of \textsanskrit{Mahāpajāpati}.” See \href{https://suttacentral.net/pli-tv-kd20/en/brahmali\#2.2.15}{Kd 20:2.2.15}. } \\
This question was thought out by those with skill. 

If\marginnote{22.1} one kills a woman who is not one’s mother, \\
If one kills a man who is not one’s father; \\
If a fool kills one who is not noble, \\
How, because of that, would one experience the result in the next life?\footnote{Sp 5.480: \textit{\textsanskrit{Ayaṁ} \textsanskrit{pañhā} \textsanskrit{liṅgaparivattena} \textsanskrit{itthibhūtaṁ} \textsanskrit{pitaraṁ} \textsanskrit{purisabhūtañca} \textsanskrit{mātaraṁ} \textsanskrit{sandhāya} \textsanskrit{vuttā}}, “This question was asked with reference to the father becoming a woman or the mother becoming a man on account of change in sex.” } \\
This question was thought out by those with skill. 

If\marginnote{23.1} one kills a woman who is one’s mother, \\
If one kills a man who is one’s father; \\
How, having done this, \\
Does one not experience the result in the next life?\footnote{Sp 5.480: \textit{Na \textsanskrit{tenānantaraṁ} phuseti \textsanskrit{ayaṁ} \textsanskrit{pañhā} \textsanskrit{migasiṅgatāpasasīhakumārādīnaṁ} viya \textsanskrit{tiracchānamātāpitaro} \textsanskrit{sandhāya} \textsanskrit{vuttā}}, “‘Does one not experience the result in the next life’: this question was asked with reference to an animal father and mother of a youthful deer or a young lion, etc.” Sp-yoj 5.480: \textit{\textsanskrit{Migasiṅgatāpasoti} \textsanskrit{migasiṅganāmako}}, “\textit{\textsanskrit{Migasiṅgatāpasa}}: one named a youthful dear.” } \\
This question was thought out by those with skill. 

If,\marginnote{24.1} without accusing, without reminding,\footnote{Sp 5.480: \textit{\textsanskrit{Acodayitvāti} \textsanskrit{gāthā} \textsanskrit{dūtenupasampadaṁ} \textsanskrit{sandhāya} \textsanskrit{vuttā}}, “‘Without accusing’: this verse was spoken with reference to ordination by messenger.” See \href{https://suttacentral.net/pli-tv-kd20/en/brahmali\#22.1.13}{Kd 20:22.1.13}–22.3.46. } \\
They do a legal procedure, but not face-to-face—\\
How could the procedure be valid, \\
And the Sangha not have committed an offense? \\
This question was thought out by those with skill. 

If,\marginnote{25.1} after accusing and reminding,\footnote{Sp 5.480: \textit{\textsanskrit{Codayitvāti} \textsanskrit{gāthā} \textsanskrit{paṇḍakādīnaṁ} \textsanskrit{upasampadaṁ} \textsanskrit{sandhāya} \textsanskrit{vuttā}}, “‘After accusing’: this verse was spoken with reference to the ordination of \textit{\textsanskrit{paṇḍakas}}, etc.” See \href{https://suttacentral.net/pli-tv-kd1/en/brahmali\#61.1.19}{Kd 1:61.1.19}–68.1.4. } \\
They do a legal procedure face-to-face—\\
How could the procedure be invalid, \\
And the Sangha have committed an offense? \\
This question was thought out by those with skill. 

How\marginnote{26.1} is there an offense for cutting?\footnote{Sp 5.480: \textit{Chindantassa \textsanskrit{āpattīti} \textsanskrit{vanappatiṁ} chindantassa \textsanskrit{pārājikaṁ}, \textsanskrit{tiṇalatādiṁ} chindantassa \textsanskrit{pācittiyaṁ}, \textsanskrit{aṅgajātaṁ} chindantassa \textsanskrit{thullaccayaṁ}}, “‘How is there an offense for cutting’: there is an offense entailing expulsion for cutting down a forest tree; an offense entailing confession for cutting grass, creepers, etc.; and a serious offense for cutting off the penis.” The first of these refers to \href{https://suttacentral.net/pli-tv-bu-vb-pj2/en/brahmali\#4.18.1}{Bu Pj 2:4.18.1}, the second to \href{https://suttacentral.net/pli-tv-bu-vb-pc11/en/brahmali\#1.29.1}{Bu Pc 11:1.29.1}, and the last to \href{https://suttacentral.net/pli-tv-kd15/en/brahmali\#7.1.1}{Kd 15:7.1.1}. } \\
Yet no offense for cutting?\footnote{Sp 5.480: \textit{Chindantassa \textsanskrit{anāpattīti} kese ca nakhe ca chindantassa \textsanskrit{anāpatti}}, “‘Yet no offense for cutting’: there is no offense for cutting the hair and the nails.” } \\
How is there an offense for concealing?\footnote{Sp 5.480: \textit{\textsanskrit{Chādentassa} \textsanskrit{āpattīti} attano \textsanskrit{āpattiṁ} \textsanskrit{chādentassa} \textsanskrit{aññesaṁ} \textsanskrit{vā} \textsanskrit{āpattiṁ}}, “‘How is there an offense for concealing’: there is an offense for concealing one’s own or another’s offense.” See especially \href{https://suttacentral.net/pli-tv-bu-vb-pc64/en/brahmali\#1.23.1}{Bu Pc 64:1.23.1} and \href{https://suttacentral.net/pli-tv-bi-vb-pj6/en/brahmali\#1.23.1}{Bi Pj 6:1.23.1}. } \\
Yet no offense for concealing?\footnote{Sp 5.480: \textit{\textsanskrit{Chādentassa} \textsanskrit{anāpattīti} \textsanskrit{gehādīni} \textsanskrit{chādentassa} \textsanskrit{anāpatti}}, “‘Yet no offense for concealing’: there is no offense for concealing a house, etc.” Here concealing means covering, that is, roofing. } \\
This question was thought out by those with skill. 

How\marginnote{27.1} is speaking the truth a heavy offense,\footnote{Sp 5.480: \textit{\textsanskrit{Saccaṁ} \textsanskrit{bhaṇantoti} \textsanskrit{gāthāya} “\textsanskrit{sikharaṇīsi} \textsanskrit{ubhatobyañjanāsī}”ti \textsanskrit{saccaṁ} \textsanskrit{bhaṇanto} \textsanskrit{garukaṁ} \textsanskrit{āpajjati}, \textsanskrit{sampajānamusāvāde} pana \textsanskrit{musā} \textsanskrit{bhāsato} \textsanskrit{lahukāpatti} hoti, \textsanskrit{abhūtārocane} \textsanskrit{musā} \textsanskrit{bhaṇanto} \textsanskrit{garukaṁ} \textsanskrit{āpajjati}, \textsanskrit{bhūtārocane} \textsanskrit{saccaṁ} \textsanskrit{bhāsato} \textsanskrit{lahukāpatti} \textsanskrit{hotīti}}, “‘How is speaking the truth’: in the verse, saying ‘You have genital prolapse, you’re a hermaphrodite,’ one commits a heavy offense for speaking the truth. But for lying in full awareness, one commits a light offense for lying. For telling what is not true, one commits a heavy offense for lying. For telling what is true, one commits a light offense for speaking the truth.” For the first of these see \href{https://suttacentral.net/pli-tv-bu-vb-ss3/en/brahmali\#3.1.31}{Bu Ss 3:3.1.31}; for the second \href{https://suttacentral.net/pli-tv-bu-vb-pc1/en/brahmali\#1.20.1}{Bu Pc 1:1.20.1}; for the third \href{https://suttacentral.net/pli-tv-bu-vb-pj4/en/brahmali\#3.32}{Bu Pj 4:3.32}; and for the last \href{https://suttacentral.net/pli-tv-bu-vb-pc8/en/brahmali\#1.2.26.1}{Bu Pc 8:1.2.26.1}. } \\
While lying is a light one? \\
And how is lying a heavy offense, \\
While speaking the truth is a light one? \\
This question was thought out by those with skill. 

%
\end{verse}

\section*{3. Questions on the offenses entailing confession, etc. }

\begin{verse}%
It\marginnote{28.1} is determined and dyed, \\
And also marked; \\
How is there an offense for using it?\footnote{Sp 5.481: \textit{\textsanskrit{Adhiṭṭhitanti} \textsanskrit{gāthā} \textsanskrit{nissaggiyacīvaraṁ} \textsanskrit{anissajjitvā} \textsanskrit{paribhuñjantaṁ} \textsanskrit{sandhāya} \textsanskrit{vuttā}}, “‘It is determined’: the verse was spoken with reference to using a robe to be relinquished without first relinquishing it.” See e.g. \href{https://suttacentral.net/pli-tv-bu-vb-np1/en/brahmali\#4.11}{Bu NP 1:4.11}. } \\
This question was thought out by those with skill. 

A\marginnote{29.1} monk eats meat after sunset,\footnote{Sp 5.481: \textit{\textsanskrit{Atthaṅgate} \textsanskrit{sūriyeti} \textsanskrit{gāthā} \textsanskrit{romanthakaṁ} \textsanskrit{sandhāya} \textsanskrit{vuttā}}, “‘After sunset’: the verse was spoken with reference to a regurgitator.” See \href{https://suttacentral.net/pli-tv-kd15/en/brahmali\#25.1.7}{Kd 15:25.1.7}. } \\
And is neither insane nor deranged; \\
Nor overwhelmed by pain—\\
How, then, is there no offense for him, \\
Yet a rule was taught by the Accomplished One? \\
This question was thought out by those with skill. 

One\marginnote{30.1} is neither lustful nor intent on stealing,\footnote{Sp 5.481: \textit{Na rattacittoti \textsanskrit{gāthāya} ayamattho – rattacitto \textsanskrit{methunadhammapārājikaṁ} \textsanskrit{āpajjati}. Theyyacitto \textsanskrit{adinnādānapārājikaṁ}, \textsanskrit{paraṁ} \textsanskrit{maraṇāya} cetento \textsanskrit{manussaviggahapārājikaṁ}, \textsanskrit{saṅghabhedako} pana na rattacitto na ca pana theyyacitto na \textsanskrit{cāpi} so \textsanskrit{paraṁ} \textsanskrit{maraṇāya} cetayi, \textsanskrit{salākaṁ} panassa dentassa hoti \textsanskrit{chejjaṁ}, \textsanskrit{pārājikaṁ} hoti, \textsanskrit{salākaṁ} \textsanskrit{paṭiggaṇhantassa} \textsanskrit{bhedakānuvattakassa} \textsanskrit{thullaccayaṁ}}, “‘One is neither lustful’: this is the meaning of the verse. One who is lustful commits an offense entailing expulsion in regard to sexual intercourse. One intent on stealing commits an offense entailing expulsion in regard to taking what is not given. One intent on killing another commits an offense entailing expulsion in regard to a human being. A schismatic is neither lustful, not intent on stealing, nor intent on killing another, yet he is cut off with an offense entailing expulsion by giving out ballots. For the receiver of a ballot, there is a serious offense.” See \href{https://suttacentral.net/pli-tv-kd1/en/brahmali\#67.1.13}{Kd 1:67.1.13} and \href{https://suttacentral.net/pli-tv-kd17/en/brahmali\#4.4.6}{Kd 17:4.4.6}. } \\
Nor intent on killing another; \\
How, then, in giving out a ballot is one cut off? \\
And how does the receiver commit a serious offense? \\
This question was thought out by those with skill. 

It\marginnote{31.1} is not a risky wilderness dwelling, \\
Nor was his robe given by the Sangha; \\
Nor did he participate in the robe-making ceremony there—\\
How, then, if he stores the robe and then travels 6 kilometers,\footnote{Sp 5.481: \textit{Gaccheyya \textsanskrit{aḍḍhayojananti} \textsanskrit{ayaṁ} \textsanskrit{pañhā} \textsanskrit{suppatiṭṭhitanigrodhasadisaṁ} ekakulassa \textsanskrit{rukkhamūlaṁ} \textsanskrit{sandhāya} \textsanskrit{vuttā}}, “‘Travels 6 kilometers’: this question was asked with reference to the foot of a tree belonging to one clan, like the banyan tree ‘Well-planted’.” This question refers to Bu Np 2, which states that one must remain “within the area of the midday shadow of the tree”, see \href{https://suttacentral.net/pli-tv-bu-vb-np2/en/brahmali\#3.16.1}{Bu NP 2:3.16.1}. According to the story at \href{https://suttacentral.net/an6.54/en/brahmali\#9.2}{AN 6.54}, the canopy of this banyan tree spread for twelve \textit{yojanas}, that is, almost 150 km according to the estimate I use here. } \\
Is there no offense at dawn? \\
This question was thought out by those with skill. 

Done\marginnote{32.1} by body, not by speech,\footnote{Sp 5.481: \textit{\textsanskrit{Kāyikānīti} \textsanskrit{ayaṁ} \textsanskrit{gāthā} \textsanskrit{sambahulānaṁ} \textsanskrit{itthīnaṁ} kese \textsanskrit{vā} \textsanskrit{aṅguliyo} \textsanskrit{vā} ekato \textsanskrit{gaṇhantaṁ} \textsanskrit{sandhāya} \textsanskrit{vuttā}}, “‘Done by body’: this verse was spoken with reference to grasping the hair or the fingers of many women at once.” This refers to \href{https://suttacentral.net/pli-tv-bu-vb-ss2/en/brahmali\#1.2.15.1}{Bu Ss 2:1.2.15.1}. } \\
All with different bases for the offense; \\
How, then, does one commit them all together at the same time? \\
This question was thought out by those with skill. 

Done\marginnote{33.1} by speech, not by body,\footnote{Sp 5.481: \textit{\textsanskrit{Vācasikānīti} \textsanskrit{ayaṁ} \textsanskrit{gāthā} “\textsanskrit{sabbā} tumhe \textsanskrit{sikharaṇiyo}”\textsanskrit{tiādinā} nayena \textsanskrit{duṭṭhullabhāṇiṁ} \textsanskrit{sandhāya} \textsanskrit{vuttā}}, “‘Done by speech’: this verse was spoken with reference to indecent speech according to the way of ‘You all have genital prolapse,’ etc.” This refers to \href{https://suttacentral.net/pli-tv-bu-vb-ss3/en/brahmali\#1.2.14.1}{Bu Ss 3:1.2.14.1}. } \\
All with different bases for the offense; \\
How, then, does one commit them all together at the same time? \\
This question was thought out by those with skill. 

One\marginnote{34.1} does not have sex with three kinds of women, \\
Nor with three kinds of men, three kinds of ignoble ones, or \textit{\textsanskrit{paṇḍakas}},\footnote{Sp 5.481: \textit{Tayo \textsanskrit{anariyapaṇḍaketi} \textsanskrit{ubhatobyañjanasaṅkhāte} tayo anariye}, “Three kinds of ignoble ones or \textit{\textsanskrit{paṇḍakas}}’: the three kinds of hermaphrodites are called ignoble ones.” See \href{https://suttacentral.net/pli-tv-bu-vb-pj1/en/brahmali\#9.1.1}{Bu Pj 1:9.1.1}–9.1.8. } \\
As stated in the rule—\footnote{Sp 5.481: \textit{Na \textsanskrit{cācare} \textsanskrit{methunaṁ} \textsanskrit{byañjanasminti} \textsanskrit{anulomapārājikavasenapi} \textsanskrit{methunaṁ} \textsanskrit{nācarati}}, “‘As stated in the rule’: one does not have sex that is in conformity with an offense entailing expulsion.” See \href{https://suttacentral.net/pli-tv-bu-vb-pj1/en/brahmali\#9.1.9.1}{Bu Pj 1:9.1.9.1}–9.1.25. } \\
How, then, might one still be cut off conditioned by sexual intercourse?\footnote{Sp 5.481: \textit{\textsanskrit{Chejjaṁ} \textsanskrit{siyā} \textsanskrit{methunadhammapaccayāti} \textsanskrit{siyā} \textsanskrit{methunadhammapaccayā} \textsanskrit{pārājikanti}. \textsanskrit{Ayaṁ} \textsanskrit{pañhā} \textsanskrit{aṭṭhavatthukaṁ} \textsanskrit{sandhāya} \textsanskrit{vuttā}, \textsanskrit{tassā} hi methunadhammassa \textsanskrit{pubbabhāgaṁ} \textsanskrit{kāyasaṁsaggaṁ} \textsanskrit{āpajjituṁ} \textsanskrit{vāyamantiyā} \textsanskrit{methunadhammapaccayā} \textsanskrit{chejjaṁ} hoti}, “‘How, then, might one still be cut off conditioned by sexual intercourse’: one would commit an offense entailing expulsion conditioned by sexual intercourse. This question was asked with reference to the training rule having eight parts. If she commits physical contact as a precursor to sexual intercourse, then, through the effort conditioned by sexual intercourse, she is cut off.” This concerns \href{https://suttacentral.net/pli-tv-bi-vb-pj8/en/brahmali\#1.11.1}{Bi Pj 8:1.11.1}. The point seems to be that the effort described fulfills the last and eighth part needed to commit this offense. Actual sexual intercourse is not required. } \\
This question was thought out by those with skill. 

One\marginnote{35.1} might ask one’s mother for robe-cloth,\footnote{Sp 5.481: \textit{\textsanskrit{Mātaraṁ} \textsanskrit{cīvaranti} \textsanskrit{ayaṁ} \textsanskrit{gāthā} \textsanskrit{piṭṭhisamaye} \textsanskrit{vassikasāṭikatthaṁ} \textsanskrit{satuppādakaraṇaṁ} \textsanskrit{sandhāya} \textsanskrit{vuttā}}, “‘One’s mother for robe-cloth’: this verse was spoken with reference to reminding someone for the sake of a rainy-season robe outside of the robe season.” This refers to \href{https://suttacentral.net/pli-tv-bu-vb-np24/en/brahmali\#1.18.1}{Bu NP 24:1.18.1}, for which there is an offense even in asking one’s own mother. } \\
And it was not intended for the Sangha; \\
How, then, is there an offense for one, \\
If there is no offense in regard to one’s relatives?\footnote{The point seems to be that there is normally no offense in asking from one’s relatives, but \href{https://suttacentral.net/pli-tv-bu-vb-np24/en/brahmali\#1.18.1}{Bu NP 24:1.18.1} is an exception to this pattern. See Sp-\textsanskrit{ṭ} 1.628. } \\
This question was thought out by those with skill. 

One\marginnote{36.1} angry person is approved of,\footnote{Sp 5.481: \textit{Kuddho \textsanskrit{ārādhako} \textsanskrit{hotīti} \textsanskrit{gāthā} \textsanskrit{titthiyavattaṁ} \textsanskrit{sandhāya} \textsanskrit{vuttā}. Titthiyo hi \textsanskrit{vattaṁ} \textsanskrit{pūrayamāno} \textsanskrit{titthiyānaṁ} \textsanskrit{vaṇṇe} \textsanskrit{bhaññamāne} kuddho \textsanskrit{ārādhako} hoti}, “‘One angry person is approved of’: this verse was spoken with reference to the proper conduct of the monastics of other religions. Monastics of other religions fulfilling the proper conduct are angry when the monastics of other religions are praised.” See \href{https://suttacentral.net/pli-tv-kd1/en/brahmali\#38.10.2}{Kd 1:38.10.2}. } \\
One angry person is censured; \\
What is the name of that rule, \\
In which an angry person is praised? \\
This question was thought out by those with skill. 

One\marginnote{37.1} contented person is approved of,\footnote{Sp 5.481: \textit{\textsanskrit{Dutiyagāthāpi} tameva \textsanskrit{sandhāya} \textsanskrit{vuttā}}, “Also the second verse was spoken with reference to that.” See \href{https://suttacentral.net/pli-tv-kd1/en/brahmali\#38.7.2}{Kd 1:38.7.2}. } \\
One contented person is censured; \\
What is the name of that rule, \\
In which a contented person is censured? \\
This question was thought out by those with skill. 

An\marginnote{38.1} offense entailing suspension, a serious offense,\footnote{Sp 5.481: \textit{\textsanskrit{Saṅghādisesantiādi} \textsanskrit{gāthā} \textsanskrit{yā} \textsanskrit{bhikkhunī} \textsanskrit{avassutāva} avassutassa purisassa hatthato \textsanskrit{piṇḍapātaṁ} \textsanskrit{gahetvā} \textsanskrit{manussamaṁsalasuṇapaṇītabhojanasesaakappiyamaṁsehi} \textsanskrit{saddhiṁ} \textsanskrit{omadditvā} ajjhoharati, \textsanskrit{taṁ} \textsanskrit{sandhāya} \textsanskrit{vuttā}}, “‘An offense entailing suspension’: this verse was spoken with reference to a lustful nun receiving almsfood directly from a lustful man. She then presses it into a mouthful containing human flesh, garlic, fine foods, and the remaining unallowable meats, finally swallowing it.” See respectively \href{https://suttacentral.net/pli-tv-bi-vb-ss5/en/brahmali\#1.14.1}{Bi Ss 5:1.14.1}, \href{https://suttacentral.net/pli-tv-kd6/en/brahmali\#23.9.7}{Kd 6:23.9.7}, \href{https://suttacentral.net/pli-tv-bi-vb-pc1/en/brahmali\#1.41.1}{Bi Pc 1:1.41.1}, \href{https://suttacentral.net/pli-tv-bi-vb-pd1/en/brahmali\#1.2.9.1}{Bi Pd 1:1.2.9.1}–8, and \href{https://suttacentral.net/pli-tv-kd6/en/brahmali\#23.10.8}{Kd 6:23.10.8}–23.15.9. } \\
An offense entailing confession, one entailing acknowledgment, and one of wrong conduct—\\
How does one commit all together? \\
This question was thought out by those with skill. 

Both\marginnote{39.1} are over twenty years old, \\
Both have the same preceptor, \\
The same teacher, the same ordination procedure—\\
How, then, is it that one is ordained, but not the other?\footnote{Sp 5.481: \textit{Eko upasampanno eko anupasampannoti \textsanskrit{gāthā} \textsanskrit{ākāsagataṁ} \textsanskrit{sandhāya} \textsanskrit{vuttā}. Sace hi \textsanskrit{dvīsu} \textsanskrit{sāmaṇeresu} eko \textsanskrit{iddhiyā} kesaggamattampi \textsanskrit{pathaviṁ} \textsanskrit{muñcitvā} nisinno hoti, so anupasampanno \textsanskrit{nāma} hoti}, “‘How, then, is it that one is ordained, but not the other’: this verse was spoken with reference to one floating in the air. If one of the two novices, through supernormal power, is seated unconnected to the ground even by a hair’s breadth, he is not ordained.” This ruling is not found in any Canonical text. What is found there is that a person floating in the air cannot make up the quorum for a legal procedure, see \href{https://suttacentral.net/pli-tv-kd9/en/brahmali\#4.5.25}{Kd 9:4.5.25}, nor can such a person object to a legal procedure, see \href{https://suttacentral.net/pli-tv-kd9/en/brahmali\#4.7.28}{Kd 9:4.7.28}. } \\
This question was thought out by those with skill. 

It\marginnote{40.1} is neither marked nor dyed,\footnote{Sp 5.481: \textit{Akappakatanti \textsanskrit{gāthā} \textsanskrit{acchinnacīvarakaṁ} \textsanskrit{bhikkhuṁ} \textsanskrit{sandhāya} \textsanskrit{vuttā}}, “‘It is neither marked’: this verse was spoken with reference to a monk whose robe had been stolen.” See \href{https://suttacentral.net/pli-tv-bu-vb-np6/en/brahmali\#2.18.1}{Bu NP 6:2.18.1}. } \\
Yet wearing that sarong one may go where one likes—\\
How is there no offense for that person, \\
Yet a rule was taught by the Accomplished One? \\
This question was thought out by those with skill. 

She\marginnote{41.1} neither gives nor receives,\footnote{Sp 5.481: \textit{Na deti na \textsanskrit{paṭiggaṇhātīti} \textsanskrit{nāpi} \textsanskrit{uyyojikā} deti, na \textsanskrit{uyyojitā} \textsanskrit{tassā} hatthato \textsanskrit{gaṇhāti}}, “‘She neither gives nor receives’: the inciter does not give; the one who is incited does not receive directly from her.” } \\
Because of that there is no recipient;\footnote{Sp 5.481: \textit{\textsanskrit{Paṭiggaho} tena na \textsanskrit{vijjatīti} teneva \textsanskrit{kāraṇena} \textsanskrit{uyyojikāya} hatthato \textsanskrit{uyyojitāya} \textsanskrit{paṭiggaho} na vijjati}, “‘Because of that there is no recipient’: for that reason, the one who is incited does not receive directly from the inciter.” } \\
How, then, does she commit a heavy offense,\footnote{Sp 5.481: \textit{Āpajjati garukanti \textsanskrit{evaṁ} santepi avassutassa hatthato \textsanskrit{piṇḍapātaggahaṇe} \textsanskrit{uyyojentī} \textsanskrit{saṅghādisesāpattiṁ} \textsanskrit{āpajjati}}, “‘How, then, does she commit a heavy offense’: when it is like this, she who is inciting her to take almsfood directly from a lustful man, she commits an offense entailing suspension.” } \\
Not a light one, because of using?\footnote{Sp 5.481: \textit{\textsanskrit{Tañca} \textsanskrit{paribhogapaccayāti} \textsanskrit{tañca} pana \textsanskrit{āpattiṁ} \textsanskrit{āpajjamānā} \textsanskrit{tassā} \textsanskrit{uyyojitāya} \textsanskrit{paribhogapaccayā} \textsanskrit{āpajjati}. \textsanskrit{Tassā} hi \textsanskrit{bhojanapariyosāne} \textsanskrit{uyyojikāya} \textsanskrit{saṅghādiseso} \textsanskrit{hotīti}}, “‘Because of using’: in committing that offense, she commits it because of the using by the one who was incited by her. For at the end of her meal, the inciter commits an offense entailing suspension.” These four lines refer to \href{https://suttacentral.net/pli-tv-bi-vb-ss6/en/brahmali\#1.26.1}{Bi Ss 6:1.26.1}. To sum up, there is an offense entailing suspension for successfully inciting another nun to receive and eat almsfood from a lustful man. } \\
This question was thought out by those with skill. 

She\marginnote{42.1} neither gives nor receives,\footnote{Sp 5.481: \textit{\textsanskrit{Dutiyagāthā} \textsanskrit{tassāyeva} \textsanskrit{udakadantaponaggahaṇe} \textsanskrit{uyyojanaṁ} \textsanskrit{sandhāya} \textsanskrit{vuttā}}, “The second verse was spoken with reference to inciting her to take water or a tooth cleaner.” See \href{https://suttacentral.net/pli-tv-bi-vb-ss6/en/brahmali\#2.2.1}{Bi Ss 6:2.2.1}. } \\
Because of that there is no recipient; \\
How, then, does she commit a light offense, \\
Not a heavy one, because of using? \\
This question was thought out by those with skill. 

One\marginnote{43.1} commits a curable heavy offense, \\
One conceals it out of disrespect; \\
If it was not a nun, how is it that one is not affected by the fault? \\
This question was thought out by those with skill.\footnote{Sp 5.481: \textit{\textsanskrit{Pañhā} \textsanskrit{mesā} kusalehi \textsanskrit{cintitāti} \textsanskrit{ayaṁ} kira \textsanskrit{pañhā} \textsanskrit{ukkhittakabhikkhuṁ} \textsanskrit{sandhāya} \textsanskrit{vuttā}. Tena hi \textsanskrit{saddhiṁ} \textsanskrit{vinayakammaṁ} natthi, \textsanskrit{tasmā} so \textsanskrit{saṅghādisesaṁ} \textsanskrit{āpajjitvā} \textsanskrit{chādento} \textsanskrit{vajjaṁ} na \textsanskrit{phusatīti}}, “‘This question was thought out by those with skill’: this question was asked with reference to an ejected monk. The legal procedures of the Monastic Law are not to be done with him. Therefore, having committed an offense entailing suspension, then, in concealing it, one is not affected by the fault.” } 

%
\end{verse}

\scendsection{The sudorific verses are finished. }

\scuddanaintro{This is the summary: }

\begin{scuddana}%
“Excluded\marginnote{46.1} from the community, not to be given away, \\
And ten, one who has not been ejected; \\
One arrives at the Teaching, above the collar bone, \\
And then two on begging. 

And\marginnote{47.1} heavy offense by body, \\
Not by body or speech; \\
Not speaking, and training, \\
And two, four people. 

Woman,\marginnote{48.1} and oil, relinquishment, \\
And monks, a single step; \\
And dressed in a sarong, and no motion, \\
If one kills not one’s mother, if one kills one’s father. 

Without\marginnote{49.1} accusing, after accusing, \\
Cutting, and the truth; \\
And determined, after sunset, \\
Neither lustful, and not the wilderness. 

By\marginnote{50.1} body, and by speech, \\
And three kinds of women, mother; \\
An angry person who is approved of, contented, \\
And an offense entailing suspension, both. 

Not\marginnote{51.1} marked, she neither gives, \\
She neither gives, one commits a heavy offense—\\
The sudorific verses: \\
Questions explained by the wise.” 

%
\end{scuddana}

%
\chapter*{{\suttatitleacronym Pvr 21}{\suttatitletranslation Legal procedures, why a Monastic Law, resolution of legal issues }{\suttatitleroot Pañcavagga}}
\addcontentsline{toc}{chapter}{\tocacronym{Pvr 21} \toctranslation{Legal procedures, why a Monastic Law, resolution of legal issues } \tocroot{Pañcavagga}}
\markboth{Legal procedures, why a Monastic Law, resolution of legal issues }{Pañcavagga}
\extramarks{Pvr 21}{Pvr 21}

\section*{1. The subchapter on legal procedures }

There\marginnote{1.1} are four kinds of legal procedures: the legal procedure consisting of getting permission, the legal procedure consisting of one motion, the legal procedure consisting of one motion and one announcement, and the legal procedure consisting of one motion and three announcements. In how many ways do they fail? In five ways: with respect to object, motion, announcement, monastery zone, or gathering. 

\begin{description}%
\item[How do legal procedures fail with respect to object? ] If\marginnote{2.2} they don’t do a legal procedure face-to-face that should be done face-to-face, that procedure has failed with respect to object and is illegitimate. If they do a legal procedure without questioning that should be done with questioning, that procedure has failed with respect to object and is illegitimate. If they do a legal procedure without admission that should be done by admission, that procedure has failed with respect to object and is illegitimate. If they apply resolution because of past insanity to one deserving resolution through recollection, that procedure has failed with respect to object and is illegitimate. If they do a legal procedure of further penalty against one deserving resolution because of past insanity, that procedure has failed with respect to object and is illegitimate. 

If\marginnote{2.7} they do a legal procedure of condemnation against one deserving a procedure of further penalty, that procedure has failed with respect to object and is illegitimate. If they do a legal procedure of demotion against one deserving a procedure of condemnation, that procedure has failed with respect to object and is illegitimate.\footnote{For an explanation of the rendering “demotion” for \textit{niyassa}, see Appendix of Technical Terms. } If they do a procedure of banishment against one deserving a procedure of demotion, that procedure has failed with respect to object and is illegitimate. If they do a procedure of reconciliation against one deserving a procedure of banishment, that procedure has failed with respect to object and is illegitimate. If they do a legal procedure of ejection against one deserving a procedure of reconciliation, that procedure has failed with respect to object and is illegitimate. 

If\marginnote{2.12} they give probation to one deserving a procedure of ejection, that procedure has failed with respect to object and is illegitimate. If they send back to the beginning one deserving probation, that procedure has failed with respect to object and is illegitimate. If they give the trial period to one deserving to be sent back to the beginning, that procedure has failed with respect to object and is illegitimate. If they rehabilitate one deserving the trial period, that procedure has failed with respect to object and is illegitimate. 

If\marginnote{2.16} they give full ordination to one deserving rehabilitation, that procedure has failed with respect to object and is illegitimate. If they do the observance-day ceremony on a non-observance day, that procedure has failed with respect to object and is illegitimate. If they do the invitation ceremony on a non-invitation day, that procedure has failed with respect to object and is illegitimate. 

%
\item[How do legal procedures fail with respect to motion? ] In five ways: they do not touch on the object, the Sangha, the person, or the motion, or the motion is put forward after the announcement. %
\item[How do legal procedures fail with respect to announcement? ] In five ways: they do not touch on the object, the Sangha, or the person, or an announcement is omitted, or the announcement is made at the wrong time.\footnote{Sp 5.485: \textit{\textsanskrit{Sāvanaṁ} \textsanskrit{hāpetīti} sabbena \textsanskrit{sabbaṁ} \textsanskrit{kammavācāya} \textsanskrit{anussāvanaṁ} na karoti, \textsanskrit{ñattidutiyakamme} \textsanskrit{dvikkhattuṁ} \textsanskrit{ñattimeva} \textsanskrit{ṭhapeti}, \textsanskrit{ñatticatutthakamme} \textsanskrit{catukkhattuṁ} \textsanskrit{ñattimeva} \textsanskrit{ṭhapeti}; \textsanskrit{evaṁ} \textsanskrit{anussāvanaṁ} \textsanskrit{hāpeti}}, “‘The announcement is omitted’: they either do not do the announcement at all; or, in a legal procedure with one motion and one announcement, they just put forward the motion twice; or, in a legal procedure with one motion and three announcements, they just put forward the motion four times. In this way they omit the announcement.” The commentary then goes on to say that even omitting syllables or mispronunciation counts as \textit{\textsanskrit{hāpeti}}. This, however, goes further than the rules for legal procedures laid down in The Chapter Connected with \textsanskrit{Campā} at \href{https://suttacentral.net/pli-tv-kd9/en/brahmali\#3.3.3}{Kd 9:3.3.3}–3.4.9. | Sp 5.485: \textit{\textsanskrit{Akāle} \textsanskrit{vā} \textsanskrit{sāvetīti} \textsanskrit{sāvanāya} \textsanskrit{akāle} \textsanskrit{anokāse} \textsanskrit{ñattiṁ} \textsanskrit{aṭṭhapetvā} \textsanskrit{paṭhamaṁyeva} \textsanskrit{anussāvanakammaṁ} \textsanskrit{katvā} \textsanskrit{pacchā} \textsanskrit{ñattiṁ} \textsanskrit{ṭhapeti}}, “‘Or the announcement is made at the wrong time’: it is the wrong time, the wrong occasion, for the announcement. Without putting forward the motion, they first make the announcement and then put forward the motion.” } %
\item[How do legal procedures fail with respect to the monastery zone? ] In eleven ways: (1) they establish a zone that is too small; (2) they establish a zone that is too large; (3) they establish an incomplete zone; (4) they establish a zone with a shadow as a zone marker; (5) they establish a zone without zone markers; (6) they establish a zone while standing outside it; (7) they establish a zone in a river; (8) they establish a zone in an ocean; (9) they establish a zone in a lake; (10) they establish a zone that merges with an existing zone; (11) they establish a zone that encloses an existing zone.\footnote{See \href{https://suttacentral.net/pli-tv-kd2/en/brahmali\#7.1.6}{Kd 2:7.1.6} etc. } %
\item[How do legal procedures fail with respect to gathering? ] In\marginnote{6.2} twelve ways: In regard to legal procedures that require a group of four: (1) the monks who should take part haven’t all arrived, consent has not been brought for those who are eligible to give their consent, or someone present objects to the decision; (2) the monks who should take part have arrived, but consent has not been brought for those who are eligible to give their consent, or someone present objects to the decision; (3) the monks who should take part have arrived, and consent has been brought for those who are eligible to give their consent, but someone present objects to the decision.\footnote{Sp 3.388: \textit{Kammappattoti \textsanskrit{kammaṁ} patto, kammayutto \textsanskrit{kammāraho}; na \textsanskrit{kiñci} \textsanskrit{kammaṁ} \textsanskrit{kātuṁ} \textsanskrit{nārahatīti} attho}, “‘Who should take part’: who are able in regard to the legal procedure, suitable for the legal procedure, fit for the legal procedure. The meaning is that one should not not do any kind of legal procedure.” The last line means one should or must take part in the legal procedure. } 

In\marginnote{6.4} regard to legal procedures that require a group of five: … (4-6) … 

In\marginnote{6.5} regard to legal procedures that require a group of ten: … (7-9) … 

In\marginnote{6.6} regard to legal procedures that require a group of twenty: (10) the monks who should take part haven’t all arrived, consent has not been brought for those who are eligible to give their consent, or someone present objects to the decision; (11) the monks who should take part have arrived, but consent has not been brought for those who are eligible to give their consent, or someone present objects to the decision; (12) the monks who should take part have arrived, and consent has been brought for those who are eligible to give their consent, but someone present objects to the decision. 

%
\end{description}

In\marginnote{7.1} regard to legal procedures that require a group of four, four regular monks should take part, while the remainder of regular monks are entitled to give their consent. The one who is subject to the legal procedure should neither take part in the decision nor give his consent, but is deserving of the legal procedure.\footnote{The one who is subject to the legal procedure cannot take part in the legal procedure in the sense of being part of the decision making. However, he needs to be present \textit{at} the legal procedure when the decision is made. } In regard to legal procedures that require a group of five, five regular monks should take part, while the remainder of regular monks are entitled to give their consent. The one who is subject to the legal procedure should neither take part in the decision nor give his consent, but is deserving of the legal procedure. In regard to legal procedures that require a group of ten, ten regular monks should take part, while the remainder of regular monks are entitled to give their consent. The one who is subject to the legal procedure should neither take part in the decision nor give his consent, but is deserving of the legal procedure. In regard to legal procedures that require a group of twenty, twenty regular monks should take part, while the remainder of regular monks are entitled to give their consent. The one who is subject to the legal procedure should neither take part in the decision nor give his consent, but is deserving of the legal procedure. 

There\marginnote{8.1} are four kinds of legal procedures: the procedure consisting of getting permission, the procedure consisting of one motion, the procedure consisting of one motion and one announcement, and the procedure consisting of one motion and three announcements. In how many ways do they fail? In five ways: with respect to object, motion, announcement, monastery zone, or gathering. 

\begin{description}%
\item[How do legal procedures fail with respect to object? ] If they give the full ordination to a \textit{\textsanskrit{paṇḍaka}}, that procedure has failed in object and is illegitimate.\footnote{For a discussion of the word \textit{\textsanskrit{paṇḍaka}}, see Appendix of Technical Terms. } If they give the full ordination to a fake monk, that procedure has failed in object and is illegitimate. If they give the full ordination to one who has previously left to join the monastics of another religion, that procedure has failed in object and is illegitimate. If they give the full ordination to an animal, that procedure has failed in object and is illegitimate. If they give the full ordination to a matricide, that procedure has failed in object and is illegitimate. If they give the full ordination to a patricide, that procedure has failed in object and is illegitimate. If they give the full ordination to a murderer of a perfected one, that procedure has failed in object and is illegitimate. If they give the full ordination to a rapist of a nun, that procedure has failed in object and is illegitimate.\footnote{For an explanation of the rendering “rapist” for \textit{\textsanskrit{dūsaka}}, see Appendix of Technical Terms. } If they give the full ordination to one who has caused a schism in the Sangha, that procedure has failed in object and is illegitimate. If they give the full ordination to one who has caused the Buddha to bleed, that procedure has failed in object and is illegitimate. If they give the full ordination to a hermaphrodite, that procedure has failed in object and is illegitimate.\footnote{For an explanation of the rendering “hermaphrodite” for \textit{\textsanskrit{ubhatobyañjana}}, see \textit{\textsanskrit{ubhatobyañjanaka}} in Appendix of Technical Terms. } If they give the full ordination to a person less then twenty years old, that procedure has failed in object and is illegitimate. %
\item[How do legal procedures fail with respect to motion? ] In five ways: they do not touch on the object, the Sangha, the person, or the motion, or the motion is put forward after the announcement. %
\item[How do legal procedures fail with respect to announcement? ] In five ways: they do not touch on the object, the Sangha, or the person, or an announcement is omitted, or the announcement is made at the wrong time. %
\item[How do legal procedures fail with respect to the monastery zone? ] In eleven ways: (1) they establish a zone that is too small; (2) they establish a zone that is too large; (3) they establish an incomplete zone; (4) they establish a zone with a shadow as a zone marker; (5) they establish a zone without zone markers; (6) they establish a zone while standing outside it; (7) they establish a zone in a river; (8) they establish a zone in an ocean; (9) they establish a zone in a lake; (10) they establish a zone that merges with an existing zone; (11) they establish a zone that encloses an existing zone. %
\item[How do legal procedures fail with respect to gathering? ] In\marginnote{13.2} twelve ways: In regard to legal procedures that require a group of four: (1) the monks who should take part haven’t all arrived, consent has not been brought for those who are eligible to give their consent, or someone present objects to the decision; (2) the monks who should take part have arrived, but consent has not been brought for those who are eligible to give their consent, or someone present objects to the decision; (3) the monks who should take part have arrived, and consent has been brought for those who are eligible to give their consent, but someone present objects to the decision. 

In\marginnote{13.4} regard to legal procedures that require a group of five: … (4-6) … 

In\marginnote{13.5} regard to legal procedures that require a group of ten: … (7-9) … 

In\marginnote{13.6} regard to legal procedures that require a group of twenty: (10) the monks who should take part haven’t all arrived, consent has not been brought for those who are eligible to give their consent, or someone present objects to the decision; (11) the monks who should take part have arrived, but consent has not been brought for those who are eligible to give their consent, or someone present objects to the decision; (12) the monks who should take part have arrived, and consent has been brought for those who are eligible to give their consent, but someone present objects to the decision. 

%
\end{description}

In\marginnote{14.1} how many situations does the legal procedure consisting of getting permission apply? In how many situations does the legal procedure consisting of one motion apply? In how many situations does the legal procedure consisting of one motion and one announcement apply? In how many situations does the legal procedure consisting of one motion and three announcements apply? 

A\marginnote{14.5} legal procedure consisting of getting permission applies in five situations. A legal procedure consisting of one motion applies in nine situations. A legal procedure consisting of one motion and one announcement applies in seven situations. A legal procedure consisting of one motion and three announcements applies in seven situations. 

In\marginnote{15.1} which five situations does the legal procedure consisting of getting permission apply? Admittance, sending away, shaving, the supreme penalty, and any legal procedure with similar characteristics.\footnote{In regard to admittance and sending away, the commentary at Sp 5.496 refers to the expulsion and readmittance of a novice monk on account of bad behavior and the subsequent giving up of the same behavior, see \href{https://suttacentral.net/pli-tv-kd1/en/brahmali\#60.1.5}{Kd 1:60.1.5}–60.1.15. As for “shaving”, the commentary refers to the Sangha giving permission to shave someone’s head prior to ordination, see \href{https://suttacentral.net/pli-tv-kd1/en/brahmali\#48.2.7}{Kd 1:48.2.7}. For the “supreme penalty”, see \href{https://suttacentral.net/pli-tv-kd21/en/brahmali\#1.12.1}{Kd 21:1.12.1}–1.12.9. Regarding “any legal procedure with similar characteristics” the commentary points to the allowance for nuns to penalize a monk who is misbehaving toward them, see \href{https://suttacentral.net/pli-tv-kd20/en/brahmali\#9.1.4}{Kd 20:9.1.4}–9.1.24. The commentary at Sp 5.496 then says: \textit{Tassa hi \textsanskrit{kammaññeva} \textsanskrit{lakkhaṇaṁ}, na \textsanskrit{osāraṇādīni}; \textsanskrit{tasmā} “\textsanskrit{kammalakkhaṇa}”nti vuccati}, “It has the characteristics of a legal procedure, but it is not admittance, etc. Therefore it is called a legal procedure with similar characteristics.” } 

In\marginnote{15.4} which nine situations does the legal procedure consisting of one motion apply? Admittance, sending away, the observance day, the invitation ceremony, agreements, giving, receiving, postponement, and any legal procedure with similar characteristics.\footnote{In regard to admittance, the commentary gives the example of \href{https://suttacentral.net/pli-tv-kd1/en/brahmali\#76.8.4}{Kd 1:76.8.4}–76.8.8. In regard to sending away, it quotes \href{https://suttacentral.net/pli-tv-kd14/en/brahmali\#14.22.3}{Kd 14:14.22.3}–14.22.6. For the observance day procedure, see \href{https://suttacentral.net/pli-tv-kd2/en/brahmali\#3.3.3}{Kd 2:3.3.3}–3.3.4. For the invitation procedure, see \href{https://suttacentral.net/pli-tv-kd4/en/brahmali\#1.14.3}{Kd 4:1.14.3}–1.14.5. In regard to agreements, the commentary gives the following examples: \href{https://suttacentral.net/pli-tv-kd1/en/brahmali\#76.5.10}{Kd 1:76.5.10}–76.5.12, \href{https://suttacentral.net/pli-tv-kd1/en/brahmali\#76.6.5}{Kd 1:76.6.5}, \href{https://suttacentral.net/pli-tv-kd1/en/brahmali\#76.9.4}{Kd 1:76.9.4}, \href{https://suttacentral.net/pli-tv-kd2/en/brahmali\#15.7.4}{Kd 2:15.7.4}, \href{https://suttacentral.net/pli-tv-kd2/en/brahmali\#15.7.9}{Kd 2:15.7.9}, \href{https://suttacentral.net/pli-tv-kd2/en/brahmali\#15.10.4}{Kd 2:15.10.4}, and \href{https://suttacentral.net/pli-tv-kd2/en/brahmali\#15.10.9}{Kd 2:15.10.9}. In regard to  giving, the commentary refers to the returning of a relinquished robe at \href{https://suttacentral.net/pli-tv-bu-vb-np1/en/brahmali\#3.2.11}{Bu NP 1:3.2.11}–3.2.13 and \href{https://suttacentral.net/pli-tv-bu-vb-np1/en/brahmali\#3.2.20}{Bu NP 1:3.2.20}–3.2.22. In regard to receiving, the commentary refers to the receiving of confession at \href{https://suttacentral.net/pli-tv-kd14/en/brahmali\#14.32.6}{Kd 14:14.32.6}–14.32.8 and \href{https://suttacentral.net/pli-tv-kd14/en/brahmali\#14.31.6}{Kd 14:14.31.6}–14.31.12. In regard to postponement, the commentary gives the following examples: \href{https://suttacentral.net/pli-tv-kd4/en/brahmali\#17.4.3}{Kd 4:17.4.3}–17.4.4 and \href{https://suttacentral.net/pli-tv-kd4/en/brahmali\#17.5.1}{Kd 4:17.5.1}–17.5.3. } 

In\marginnote{15.7} which seven situations does the legal procedure consisting of one motion and one announcement apply? Admittance, sending away, agreements, giving, ending of the robe season, approval, and any legal procedure with similar characteristics.\footnote{In regard to admittance and sending away, the commentary gives the example of \textsanskrit{Vaḍḍha} the \textsanskrit{Licchavī} at \href{https://suttacentral.net/pli-tv-kd15/en/brahmali\#20.4.3}{Kd 15:20.4.3}–20.4.13 and \href{https://suttacentral.net/pli-tv-kd15/en/brahmali\#20.7.8}{Kd 15:20.7.8}–20.7.20. In regard to agreements, the commentary gives the examples of agreements about monastery zones, e.g. at \href{https://suttacentral.net/pli-tv-kd2/en/brahmali\#6.1.10}{Kd 2:6.1.10}–6.2.7 and \href{https://suttacentral.net/pli-tv-kd2/en/brahmali\#12.2.3}{Kd 2:12.2.3}–12.2.11, about blankets at \href{https://suttacentral.net/pli-tv-bu-vb-np14/en/brahmali\#2.22}{Bu NP 14:2.22}–2.36, and about community officials at \href{https://suttacentral.net/pli-tv-kd16/en/brahmali\#21.1.9.0}{Kd 16:21.1.9.0}–21.3.37. In regard to giving, the commentary refers to the giving of the robe of the robe-making ceremony at \href{https://suttacentral.net/pli-tv-kd7/en/brahmali\#1.4.2}{Kd 7:1.4.2}–1.4.12. In regard to ending of the robe season, the commentary refers to \href{https://suttacentral.net/pli-tv-bi-vb-pc30/en/brahmali\#1.1.10}{Bi Pc 30:1.1.10}–1.1.18. In regard to approval, the commentary mentions the approval of a site for the building of a hut at \href{https://suttacentral.net/pli-tv-bu-vb-ss6/en/brahmali\#2.2.35}{Bu Ss 6:2.2.35}–2.2.48. Finally, in regard to a legal procedure with similar characteristics, the commentary suggest the procedure of covering over as if with grass at \href{https://suttacentral.net/pli-tv-kd14/en/brahmali\#13.3.2}{Kd 14:13.3.2}–13.3.14. } 

In\marginnote{15.10} which seven situations does the legal procedure consisting of one motion and three announcements apply? Admittance, sending away, agreements, giving, restraining, pressing, and any legal procedure with similar characteristics.\footnote{In regard to admittance and sending away, the commentary gives the example of the seven legal procedures that impose penalties in \href{https://suttacentral.net/pli-tv-kd11/en/brahmali}{Kd 11}. In regard to agreements, the commentary gives the example of appointing an instructor of the nuns at \href{https://suttacentral.net/pli-tv-bu-vb-pc21/en/brahmali\#1.27}{Bu Pc 21:1.27}–1.41. In regard to giving and restraining, the commentary refers to the giving of probation and the trial period, and to restraining by sending back to the beginning, all in connection with offenses entailing suspension at Kd 13. In regard to pressing, the commentary refers to the eleven offenses entailing suspension, four for the monks and seven unique ones for the nuns, that are committed when pressed for the third time. Finally, in regard to a legal procedure with similar characteristics, the commentary points to the ordination procedure at \href{https://suttacentral.net/pli-tv-kd1/en/brahmali\#76.10.2}{Kd 1:76.10.2}–76.12.16, and the rehabilitation procedure, starting at \href{https://suttacentral.net/pli-tv-kd13/en/brahmali\#2.3.2}{Kd 13:2.3.2}–2.3.36. } 

In\marginnote{16.1} regard to legal procedures that require a group of four, four regular monks should take part, while the remainder of regular monks are entitled to give their consent. The one who is subject to the legal procedure should neither take part in the decision nor give his consent, but is deserving of the legal procedure. In regard to legal procedures that require a group of five, five regular monks should take part, while the remainder of regular monks are entitled to give their consent. The one who is subject to the legal procedure should neither take part in the decision nor give his consent, but is deserving of the legal procedure. In regard to legal procedures that require a group of ten, ten regular monks should take part, while the remainder of regular monks are entitled to give their consent. The one who is subject to the legal procedure should neither take part in the decision nor give his consent, but is deserving of the legal procedure. In regard to legal procedures that require a group of twenty, twenty regular monks should take part, while the remainder of regular monks are entitled to give their consent. The one who is subject to the legal procedure should neither take part in the decision nor give his consent, but is deserving of the legal procedure. 

\scendvagga{The first subchapter on legal procedures is finished. }

\section*{2. The subchapter on reasons }

The\marginnote{18.1} Buddha laid down the training rules for his disciples for two reasons: for the well-being of the Sangha and for the comfort of the Sangha. 

The\marginnote{19.1} Buddha laid down the training rules for his disciples for two reasons: for the restraint of bad people and for the ease of good monks. 

The\marginnote{20.1} Buddha laid down the training rules for his disciples for two reasons: for the restraint of the corruptions relating to the present life and for the restraint of the corruptions relating to future lives. 

The\marginnote{21.1} Buddha laid down the training rules for his disciples for two reasons: for the restraint of threats relating to the present life and for avoiding threats relating to future lives. 

The\marginnote{22.1} Buddha laid down the training rules for his disciples for two reasons: for the restraint of faults relating to the present life and for avoiding faults relating to future lives. 

The\marginnote{23.1} Buddha laid down the training rules for his disciples for two reasons: for the restraint of dangers relating to the present life and for avoiding dangers relating to future lives. 

The\marginnote{24.1} Buddha laid down the training rules for his disciples for two reasons: for the restraint of unwholesome qualities relating to the present life and for avoiding unwholesome qualities relating to future lives. 

The\marginnote{25.1} Buddha laid down the training rules for his disciples for two reasons: out of compassion for householders and for breaking up the factions of those with bad desires. 

The\marginnote{26.1} Buddha laid down the training rules for his disciples for two reasons: to give rise to confidence in those without it and to increase the confidence of those who have it. 

The\marginnote{27.1} Buddha laid down the training rules for his disciples for two reasons: for the longevity of the true Teaching and for supporting the training.\footnote{For an explanation of the rendering “training” for \textit{vinaya}, see Appendix of Technical Terms. } 

\scendvagga{The second subchapter on reasons is finished. }

\section*{3. The subchapter on laying down }

The\marginnote{29.1} Buddha laid down the Monastic Code for his disciples for two reasons: … laid down the recitation of the Monastic Code … laid down the canceling of the Monastic Code … laid down the invitation ceremony … laid down the canceling of the invitation ceremony … laid down the legal procedure of condemnation … laid down the legal procedure of demotion … laid down the legal procedure of banishment … laid down the legal procedure of reconciliation … laid down the legal procedure of ejection … laid down the giving of probation … laid down the sending back to the beginning … laid down the giving of the trial period … laid down the rehabilitation … laid down the admittance … laid down the sending away … laid down the full ordination … laid down the legal procedure consisting of getting permission … laid down the legal procedure consisting of one motion … laid down the legal procedure consisting of one motion and one announcement … laid down the legal procedure consisting of one motion and three announcements … 

\scendvagga{The third subchapter on laying down is finished. }

\section*{4. The subchapter on “laid down a rule when there was no existing rule” }

…\marginnote{31.1} laid down a rule when there was no existing rule, and laid down an addition to an existing rule …\footnote{Sp 5.500: \textit{\textsanskrit{Apaññatte} \textsanskrit{paññattanti} \textsanskrit{sattāpattikkhandhā} \textsanskrit{kakusandhañca} \textsanskrit{sammāsambuddhaṁ} \textsanskrit{koṇāgamanañca} \textsanskrit{kassapañca} \textsanskrit{sammāsambuddhaṁ} \textsanskrit{ṭhapetvā} \textsanskrit{antarā} kenaci \textsanskrit{apaññatte} \textsanskrit{sikkhāpade} \textsanskrit{paññattaṁ} \textsanskrit{nāma}. \textsanskrit{Makkaṭivatthuādivinītakathā} \textsanskrit{sikkhāpade} \textsanskrit{paññatte} \textsanskrit{anupaññattaṁ} \textsanskrit{nāma}}, “‘Laid down a rule when there was no existing rule’: apart from the fully Awakened Buddhas Kakusandha, \textsanskrit{Koṇāgamana}, and Kassapa, by whomever, when there was no existing rule, something was laid down among the seven classes of offenses. This is called laid down. When, in regard to a training rule, there is a subsidiary case story, such as the story of the monkey, etc., it is called ‘laid down an addition to an existing rule’.” } laid down resolution face-to-face … laid down resolution by recollection … laid down resolution because of past insanity … laid down acting according to what has been admitted … laid down the majority decision … laid down the further penalty … laid down the covering over as if with grass for the well-being of the Sangha and for the comfort of the Sangha. 

The\marginnote{32.1} Buddha laid down the covering over as if with grass for his disciples for two reasons: for the restraint of bad people and for the ease of good monks. 

The\marginnote{33.1} Buddha laid down the covering over as if with grass for his disciples for two reasons: for the restraint of the corruptions relating to the present life and for the restraint of the corruptions relating to future lives. 

The\marginnote{34.1} Buddha laid down the covering over as if with grass for his disciples for two reasons: for the restraint of threats relating to the present life and for avoiding threats relating to future lives. 

The\marginnote{35.1} Buddha laid down the covering over as if with grass for his disciples for two reasons: for the restraint of faults relating to the present life and for avoiding faults relating to future lives. 

The\marginnote{36.1} Buddha laid down the covering over as if with grass for his disciples for two reasons: for the restraint of dangers relating to the present life and for avoiding dangers relating to future lives. 

The\marginnote{37.1} Buddha laid down the covering over as if with grass for his disciples for two reasons: for the restraint of unwholesome qualities relating to the present life and for avoiding unwholesome qualities relating to future lives. 

The\marginnote{38.1} Buddha laid down the covering over as if with grass for his disciples for two reasons: out of compassion for householders and for breaking up the factions of those with bad desires. 

The\marginnote{39.1} Buddha laid down the covering over as if with grass for his disciples for two reasons: to give rise to confidence in those without it and to increase the confidence of those who have it. 

The\marginnote{40.1} Buddha laid down the covering over as if with grass for his disciples for two reasons: for the longevity of the true Teaching and for supporting the training. 

\scendvagga{The fourth subchapter on “laid down a rule when there was no existing rule” is finished. }

\section*{5. The subchapter on nine kinds of “being found among” }

“There\marginnote{42.1} are nine kinds of ‘being found among’: being found among the actions that are the bases for offenses, being found among the failures, being found among the offenses, being found among the origin stories, being found among persons, being found among the classes, being found among the originations, being found among the legal issues, and being found among the settling of legal issues. 

When\marginnote{43.1} a legal issue has arisen, if the two opponents come, they should be told to inform about the action that was the basis for the disagreement. After hearing the testimony of both, they should be told, ‘When we’ve resolved this legal issue, you should both be satisfied.’ If they say, ‘We’ll both be satisfied,’ then the Sangha should take on that legal issue. If there are many shameless people in the gathering, then the issue should be resolved by means of a committee. If there are many ignorant people in the gathering, they should search for an expert on the Monastic Law to resolve that legal issue in accordance with the Teaching, the Monastic Law, and the Teacher’s instruction. That legal issue should be resolved in this way. 

One\marginnote{44.1} should know the basis for an offense, one should know the category, one should know the name, and one should know the offense. 

‘Sexual\marginnote{45.1} intercourse’ is the basis for an offense, as well as a category. ‘Offense entailing expulsion’ is the name, as well as the offense. 

‘Stealing’\marginnote{46.1} is the basis for an offense, as well as a category. ‘Offense entailing expulsion’ is the name, as well as the offense. 

‘Human\marginnote{47.1} being’ is the basis for an offense, as well as a category. ‘Offense entailing expulsion’ is the name, as well as the offense. 

‘Superhuman\marginnote{48.1} quality’ is the basis for an offense, as well as a category. ‘Offense entailing expulsion’ is the name, as well as the offense. 

‘Emission\marginnote{49.1} of semen’ is the basis for an offense, as well as a category. ‘Offense entailing suspension’ is the name, as well as the offense. 

‘Physical\marginnote{50.1} contact’ is the basis for an offense, as well as a category. ‘Offense entailing suspension’ is the name, as well as the offense. 

‘Indecent\marginnote{51.1} speech’ is the basis for an offense, as well as a category. ‘Offense entailing suspension’ is the name, as well as the offense. 

‘One’s\marginnote{52.1} own desires’ is the basis for an offense, as well as a category. ‘Offense entailing suspension’ is the name, as well as the offense. 

‘Matchmaking’\marginnote{53.1} is the basis for an offense, as well as a category. ‘Offense entailing suspension’ is the name, as well as the offense. 

‘Building\marginnote{54.1} a hut by means of begging’ is the basis for an offense, as well as a category. ‘Offense entailing suspension’ is the name, as well as the offense. 

‘Building\marginnote{55.1} a large dwelling’ is the basis for an offense, as well as a category. ‘Offense entailing suspension’ is the name, as well as the offense. 

‘Groundlessly\marginnote{56.1} charging a monk with an offense entailing expulsion’ is the basis for an offense, as well as a category. ‘Offense entailing suspension’ is the name, as well as the offense. 

‘Charging\marginnote{57.1} a monk with an offense entailing expulsion, using an unrelated legal issue as a pretext’ is the basis for an offense, as well as a category. ‘Offense entailing suspension’ is the name, as well as the offense. 

‘A\marginnote{58.1} monk not stopping with pursuing schism in the Sangha when pressed for the third time’ is the basis for an offense, as well as a category. ‘Offense entailing suspension’ is the name, as well as the offense. 

‘Monks\marginnote{59.1} not stopping siding with one who is pursuing schism in the Sangha when pressed for the third time’ is the basis for an offense, as well as a category. ‘Offense entailing suspension’ is the name, as well as the offense. 

‘A\marginnote{60.1} monk not stopping with being difficult to correct when pressed for the third time’ is the basis for an offense, as well as a category. ‘Offense entailing suspension’ is the name, as well as the offense. 

‘A\marginnote{61.1} monk not stopping with being a corrupter of families when pressed for the third time’ is the basis for an offense, as well as a category. ‘Offense entailing suspension’ is the name, as well as the offense. … 

‘Out\marginnote{62.1} of disrespect, defecating, urinating, or spitting in water’ is the basis for an offense, as well as a category. ‘Offense of wrong conduct’ is the name, as well as the offense.” 

\scendvagga{The fifth subchapter on nine kinds of “being found among” is finished. }

\scuddanaintro{This is the summary: }

\begin{scuddana}%
“Getting\marginnote{65.1} permission, and motion, \\
One motion and one announcement, and with one motion and three announcements; \\
Object, motion, announcement, \\
Monastery zone, and gathering. 

Face-to-face,\marginnote{66.1} and questioning, \\
Admission, resolution, deserving; \\
Object, Sangha, and person, \\
Motions, and not the motion afterwards. 

Object,\marginnote{67.1} Sangha, and person, \\
Announcement, and at the wrong time; \\
Too small, and large, \\
Incomplete, shadow, without zone markers.\footnote{Reading this as a single compound \textit{\textsanskrit{khaṇḍacchāyānimittakā}} with SRT, which allows the final word to be read as \textit{\textsanskrit{animittakā}}. } 

Outside,\marginnote{68.1} river, and in an ocean, \\
And in a lake, merges; \\
It encloses a zone, \\
A group of four, and of five. 

A\marginnote{69.1} group of ten, and of twenty, \\
Not brought and brought; \\
Who should take part, eligible to give their consent, \\
And person deserving of the legal procedure. 

Five\marginnote{70.1} situations for getting permission, \\
And nine situations for one motion; \\
Seven situations for one motion and one announcement, \\
Seven situations for one motion and three announcements. 

Well-being,\marginnote{71.1} and comfort, bad, \\
And good, corruptions; \\
Threats, faults, and dangers, \\
Unwholesome, and for householders. 

Those\marginnote{72.1} with bad desires, those without confidence, \\
Confidence, the longevity of the Teaching; \\
And supporting the training, \\
The Monastic Code, and with the recitation. 

And\marginnote{73.1} the canceling of the Monastic Code, \\
And the invitation ceremony, its suspension; \\
Condemnation, and demotion, \\
Banishment, reconciliation. 

Ejection,\marginnote{74.1} probation, \\
Beginning, trial period, rehabilitation; \\
Admittance, sending away, \\
And so the full ordination. 

Getting\marginnote{75.1} permission, and motion, \\
One motion and one announcement, one motion and three announcements; \\
When there was no existing rule, laid down an addition, \\
Resolution face-to-face, recollection. 

Past\marginnote{76.1} insanity, admitted, majority, \\
Further penalty, covering over as if with grass; \\
Basis, failure, offense, \\
Origin story, and with person. 

And\marginnote{77.1} classes, originations, \\
And legal issue; \\
And found among the settling, \\
Name, and so offense.” 

%
\end{scuddana}

\scend{The Compendium is finished. }

\scendbook{The canonical text of the Compendium is finished. }

\scuddanaintro{Concluding verses: }

\begin{scuddana}%
“Having\marginnote{81.1} asked this and that\footnote{Reading \textit{\textsanskrit{pubbācariyamaggañca}} with SRT. } \\
About the practice of past teachers—\\
\textsanskrit{Dīpanāma}, who had great wisdom, \\
Superb memory, clarity of sight. 

This\marginnote{82.1} contraction of the detail, \\
With a path for study, in the middle; \\
Having thought it out, he had it written down, \\
Bringing happiness to disciples. 

That\marginnote{83.1} which is called the ‘Compendium’, \\
With all bases for offenses with their characteristics, \\
Meaning in accord with the meaning in the true Teaching, \\
Rule in accord with the rule in what is laid down. 

It\marginnote{84.1} encompasses the instruction, \\
Like the ocean encompasses India; \\
Not knowing the Compendium, \\
How does one decide on the rules? 

How\marginnote{85.1} does one decide on failures, bases, rules, \\
Additions to the rules, persons; \\
Whether on each side or on both sides, \\
A rule by convention from a moral fault? 

When\marginnote{86.1} anyone gives rise to doubt, \\
It is cut off by the Compendium; \\
Like a universal monarch in the midst of his great army, \\
Like a lion in a herd of deer; 

Like\marginnote{87.1} the sun surrounded by its rays, \\
Like the moon in the starry sky; \\
Like the Supreme Being in his assembly, \\
Like a leader with his retinue—\\
In this way, the true Teaching and the Monastic Law \\
Shine through the Compendium.” 

%
\end{scuddana}

%
\backmatter%
%
\chapter*{Appendices}
\addcontentsline{toc}{chapter}{Appendices}
\markboth{Appendices}{Appendices}

\emph{Appendices for all volumes may be found at the end of the first volume, The Great Analysis, part I.}

%
\chapter*{Colophon}
\addcontentsline{toc}{chapter}{Colophon}
\markboth{Colophon}{Colophon}

\section*{The Translator}

Bhikkhu Brahmali was born Norway in 1964. He first became interested in Buddhism and meditation in his early 20s after a visit to Japan. Having completed degrees in engineering and finance, he began his monastic training as an anagarika (keeping the eight precepts) in England at Amaravati and Chithurst Buddhist Monastery.

After hearing teachings from Ajahn Brahm he decided to travel to Australia to train at Bodhinyana Monastery. Bhikkhu Brahmali has lived at Bodhinyana Monastery since 1994, and was ordained as a Bhikkhu, with Ajahn Brahm as his preceptor, in 1996. In 2015 he entered his 20th Rains Retreat as a fully ordained monastic and received the title Maha Thera (Great Elder).

Bhikkhu Brahmali’s knowledge of the Pali language and of the Suttas is excellent. Bhikkhu Bodhi, who translated most of the Pali Canon into English for Wisdom Publications, called him one of his major helpers for the 2012 translation of \emph{The Numerical Discourses of the Buddha}. He has also published two essays on Dependent Origination and a book called \emph{The Authenticity of the Early Buddhist Texts} with the Buddhist Publication Society in collaboration with Bhante Sujato.

The monastics of the Buddhist Society of WA (BSWA) often turn to him to clarify Vinaya (monastic discipline) or Sutta questions. They also greatly appreciate his Sutta and Pali classes. Furthermore he has been instrumental in most of the building and maintenance projects at Bodhinyana Monastery and at the emerging Hermit Hill property in Serpentine.

\section*{Creation Process}

Translated from the Pali. The primary source was the \textsanskrit{Mahāsaṅgīti} edition, with occasional reference to other Pali editions, especially the \textsanskrit{Chaṭṭha} \textsanskrit{Saṅgāyana} edition and the Pali Text Society edition. I cross-checked with I.B. Horner’s English translation, “The Book of the Discipline”, as well as Bhikkhu \textsanskrit{Ñāṇatusita}’s “A Translation and Analysis of the \textsanskrit{Pātimokkha}” and Ajahn \textsanskrit{Ṭhānissaro}’s “Buddhist Monastic Code”.

\section*{The Translation}

This is the first complete translation of the Vinaya \textsanskrit{Piṭaka} in English. The aim has been to produce a translation that is easy to read, clear, and accurate, and also modern in vocabulary and style.

\section*{About SuttaCentral}

SuttaCentral publishes early Buddhist texts. Since 2005 we have provided root texts in Pali, Chinese, Sanskrit, Tibetan, and other languages, parallels between these texts, and translations in many modern languages. Building on the work of generations of scholars, we offer our contribution freely.

SuttaCentral is driven by volunteer contributions, and in addition we employ professional developers. We offer a sponsorship program for high quality translations from the original languages. Financial support for SuttaCentral is handled by the SuttaCentral Development Trust, a charitable trust registered in Australia.

\section*{About Bilara}

“Bilara” means “cat” in Pali, and it is the name of our Computer Assisted Translation (CAT) software. Bilara is a web app that enables translators to translate early Buddhist texts into their own language. These translations are published on SuttaCentral with the root text and translation side by side.

\section*{About SuttaCentral Editions}

The SuttaCentral Editions project makes high quality books from selected Bilara translations. These are published in formats including HTML, EPUB, PDF, and print.

You are welcome to print any of our Editions.

%
\end{document}